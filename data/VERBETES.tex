\begin{letra}{a}	

\verb{a}{}{[a]}{(pron.)}{1}{}{Pronome anafórico de terceira pessoa de plural.}{\textbf{A xê.} \textit{Saíram}.}{}{}
\verb{a}{}{[a]}{(pron.)}{2}{}{Pronome indefinido singular ou plural.}{\textbf{A ka fla.} \textit{Diz-se.} \textit{Dizem}.}{}{}
\verb{a}{}{[a]}{(pron.)}{3}{}{Pronome utilizado em frases imperativas referentes à primeira ou segunda pessoa do plural.}{\textbf{A bomu fla
santome!} \textit{Vamos falar santome!}} \textbf{A xa saku!} \textit{Encham
os sacos!}{}{}
\verb{a}{}{[a]}{(v.)}{1}{}{Há.}{\textbf{A tela ku ka tê vintxi kôlô koblo.}
\textit{Há terras com vinte espécies de cobras.}}{}
\verb{aba}{}{[ˈaba]}{(n.)}{1}{}{Aba.}{}{}%
\verb{aba}{}{[ˈaba]}{(n.)}{2}{}{Tronco de árvore.}{}{}
\verb{ababa ê}{}{[abaˈba ˈe]}{(adv.)}{1}{}{Exatamente.}{}{}
\verb{ababa ê}{}{[abaˈba ˈe]}{(adv.)}{2}{}{Precisamente.}{}{}
\verb{ababa ê}{}{[abaˈba ˈe]}{(adv.)}{2}{}{Tal e qual.}{}{}%
\verb{aba-d'ope}{}{[ˈaba dᴐˈpɛ]}{(n.)}{1}{}{Pezão.}{}{}
\verb{abensa}{}{[aˈbẽsa]}{(n.)}{1}{}{Cumprimento respeitoso.}{}{}
\verb{abli}{}{[aˈbli]}{(n.)}{1}{}{Abril.}{}{}
\verb{ablidaji}{}{[aˈblidaʒi]}{(n.)}{1}{}{Habilidade.}{}{}
\verb{ablidaji}{}{[aˈblidaʒi]}{(n.)}{2}{}{Subterfúgio.}{}{}
\verb{ablidaji}{}{[aˈblidaʒi]}{(n.)}{3}{}{Truque.}{}{}
\verb{ablunusu}{}{[abluˈnusu]}{(interj.)}{1}{}{Abrenúncio!}{}{}
\verb{abube}{}{[aˈbubɛ]}{(n.)}{1}{}{Analfabeto.}{}{}
\verb{abube}{}{[aˈbubɛ]}{(n.)}{2}{}{Ignorante.}{}{}
\verb{abutu}{}{[ˈabutu]}{(n.)}{2}{}{Hábito.}{}{}
\verb{abutu}{}{[ˈabutu]}{(n.)}{3}{}{Mortalha.}{}{}
\verb{abutu}{}{[ˈabutu]}{(n.)}{1}{}{Traje do falecido.}{}{}%
\verb{abutu}{}{[ˈabutu]}{(n.)}{4}{}{Vestes sacerdotais.}{}{}%
\verb{abuzu}{}{[aˈbuzu]}{(n.)}{1}{}{Atrevimento.}{}{}%
\verb{adêwa-kongô}{}{[aˈdewa ˈkõgo]}{(n.)}{1}{}{Adeus para sempre.}{}{}
\verb{adêzu}{}{[aˈdezu]}{(n.)}{1}{}{Adeus.}{}{}
\verb{afe}{}{[aˈfɛ]}{(n.)}{1}{}{Fé.}{}{}
\verb{Aflika}{}{[ˈaflika]}{(top.)}{1}{}{África.}{}{}
\verb{aflikanu}{}{[afliˈkanu]}{(n.)}{1}{}{Africano.}{}{}
\verb{aflison}{}{[afliˈsõ]}{(n.)}{1}{}{Aflição.}{}{}
\verb{aglasa}{}{[aˈglasa]}{(n.)}{2}{}{Nome.}{}{}
\verb{aglasa}{}{[aˈglasa]}{(n.)}{1}{}{Simpatia.}{}{}
\verb{agola}{}{[aˈgɔla]}{(adv.)}{1}{}{E agora.}{}{}
\verb{agola}{}{[aˈgɔla]}{(adv.)}{1}{}{Então.}{}{}
\verb{agola}{}{[aˈgɔla]}{(adv.)}{2}{}{Nesse caso.}{}{}
\verb{agola}{}{[aˈgɔla]}{(conj.)}{1}{}{Contudo.}{}{}
\verb{agola}{}{[aˈgɔla]}{(conj.)}{2}{}{Ora.}{}{}
\verb{agola}{}{[aˈgɔla]}{(conj.)}{3}{}{Porém.}{}{}
\verb{agô-matu}{}{[aˈgo
ˈmatu]}{(n.)}{1}{}{Anileiro.}{\textit{\textbf{Indigofera tinctoria}.}}{}
\verb{agôxtô}{}{[aˈgoʃto]}{(n.)}{1}{}{Agosto.}{}{}
\verb{agwêdê}{}{[agweˈde]}{(n.)}{1}{}{Adivinha.}{}{}
\verb{agwêdê}{}{[agweˈde]}{(n.)}{2}{}{\textit{Agwêdê}.}{Fórmula de introdução
de uma adivinha utilizada pelo emissor, ao que o receptor responde
\textbf{alê}.}{}{}{}
\verb{ai}{}{[aˈi]}{(adv.)}{1}{}{Cá.}{}{}
\verb{aja vida ku saôdji}{}{[ˈaʒa ˈvida ku saˈodʒi]}{(expr.)}{1}{}{Haja vida
com saúde!}{}
\verb{aji}{}{[ˈaʒi]}{(n.)}{1}{}{Ás.}{}{}
\verb{akaka}{}{[akaˈka]}{(interj.)}{1}{}{Atenção!}{}{}
\verb{akaka}{}{[akaˈka]}{(interj.)}{1}{}{Caramba!}{}{}
\verb{akaka}{}{[akaˈka]}{(interj.)}{1}{}{Cuidado!}{}{}
\verb{akansa}{}{[aˈk\~{\textturna}sa]}{(n.)}{1}{}{Feijão-de-praia.}{\textit{\textbf{Canavalia
maritima}.}}{}
\verb{akêlê}{}{[aˈkele]}{(n.)}{1}{}{Rã.}{}{}
\verb{akêlê}{}{[aˈkele]}{(n.)}{2}{}{Sapo.}{}{}
\verb{akondô}{}{[aˈkõdo]}{(adj.)}{1}{}{Exibicionista.}{}{}
\verb{akondô}{}{[aˈkõdo]}{(adj.)}{2}{}{Vaidoso.}{}{}
\verb{ala}{}{[aˈla]}{(adv.)}{1}{}{Acolá.}{}{}
\verb{ala}{}{[aˈla]}{(adv.)}{2}{}{Lá.}{}{}
\verb{alada}{}{[ˈalada]}{(n.)}{1}{}{Peixe
arco-íris.}{\textit{\textbf{Elegatis bipinnulata}.}}{}
\verb{alada}{}{[ˈalada]}{(n.)}{1}{}{\textit{Alada}.}{Dança típica.}{}
\verb{alamanda}{}{[alaˈm\~{\textturna}da]}{(n.)}{1}{}{Alamanda-de-flor-grande.}{\textit{\textbf{Orelia
grandiflora}.}}{}
\verb{alamanda}{}{[alaˈm\~{\textturna}da]}{(n.)}{2}{}{Orelha-de-onça.}{\textit{\textbf{Orelia
grandiflora}.}}{}
\verb{alami}{}{[aˈlami]}{(n.)}{1}{}{Arame.}{}{}
\verb{alan}{}{[aˈl\~{\textturna}]}{(n.)}{1}{}{Aranha.}{}{}
\verb{alê}{}{[aˈle]}{(adv.)}{1}{}{Pronto!}{Fórmula de resposta utilizada pelo
receptor à introdução de uma adivinha.}{}
\verb{alê}{}{[aˈle]}{(n.)}{1}{}{Lei.}{}{}%
\verb{alê}{}{[aˈle]}{(n.)}{2}{}{Rei.}{}{}%
\verb{alêglia}{}{[aleˈglia]}{(n.)}{1}{}{Alegria.}{}{}
\verb{alêdunha}{}{[aleˈduɲa]}{(n.)}{1}{}{Doninha.}{\textbf{\textit{Mustela
nivalis}}.}{}
\verb{alfabaka}{}{[alfaˈbaka]}{(n.)}{1}{}{Alfavaca.}{\textit{\textbf{Peperomia pellucida}.}}{}
\verb{alfasi}{}{[alˈfasi]}{(n.)}{1}{}{Alface.}{}{}
\verb{aliadu}{}{[aliˈadu]}{(n.)}{1}{}{Aliado.}{}{}
\verb{aliba}{}{[ˈaliba]}{(n.)}{1}{}{Capim.}{}{}%
\verb{aliba}{}{[ˈaliba]}{(n.)}{2}{}{Erva.}{}{}
\verb{aliba}{}{[ˈaliba]}{(n.)}{3}{}{Grama.}{}{}
\verb{aliba}{}{[ˈaliba]}{(n.)}{4}{}{Relva.}{}{}
\verb{aliba-blaboza}{}{[ˈaliba blaˈbɔza]}{(n.)}{1}{}{Aloés.}{Cf. \textbf{blaboza}.}{}{}
\verb{aliba-guya}{}{[ˈaliba ˈguja]}{(n.)}{1}{}{Carrapicho-de-agulha.}{\textit{\textbf{Bidens pilosa}.}}{}
\verb{aliba-guya}{}{[ˈaliba
ˈguja]}{(n.)}{3}{}{Folha-agulha.}{\textit{\textbf{Ludwigia erecta}}.}{}
\verb{aliba-kasô}{}{[ˈaliba
kaˈso]}{(n.)}{1}{}{Capim-de-burro.}{\textit{\textbf{Eleusine indica}.}}{}{}
\verb{aliba-kasô}{}{[ˈaliba
kaˈso]}{(n.)}{2}{}{Capim-vassoura.}{\textit{\textbf{Eleusine indica}.}}{}{}
\verb{aliba-kasô}{}{[ˈaliba
kaˈso]}{(n.)}{3}{}{Erva-cão.}{\textit{\textbf{Eleusine indica}.}}{}{}
\verb{aliba-manswenswe}{}{[ˈaliba
m\~{\textturna}sw\~ɛˈswɛ]}{(n.)}{1}{Sorgo.}{\textit{\textbf{Sorghum
bicolor}.}}{}{}
\verb{alika}{}{[ˈalika]}{(n.)}{1}{}{Arca.}{}{}%
\verb{alika}{}{[ˈalika]}{(n.)}{2}{}{Baú.}{}{}
\verb{alima}{}{[ˈalima]}{(n.)}{1}{}{Alma.}{}{}%
\verb{alima}{}{[ˈalima]}{(n.)}{2}{}{Espírito.}{}{}%
\verb{alima}{}{[ˈalima]}{(n.)}{3}{}{Fantasma.}{}{}
\verb{alimandadji}{}{[alim\~{\textturna}ˈdadʒi]}{(n.)}{3}{}{\textit{Alimandadji}.}{Dança
comumente executada no fim das empreitadas, sobretudo nas roças.}{}{}
\verb{alimandadji}{}{[alim\~{\textturna}ˈdadʒi]}{(n.)}{1}{}{Coletivo
constituído por vinte a trinta membros que, antigamente, se atarefavam na
colheita do café, cacau e nos demais serviços agrícolas.}{}{}{}%
\verb{alimandadji}{}{[alim\~{\textturna}ˈdadʒi]}{(n.)}{3}{}{Irmandade.}{}{}{}
\verb{alimanze}{}{[alim\~{\textturna}ˈzɛ]}{(n.)}{1}{}{Armazém.}{Cf. \textbf{alumanze}.}{}{}
\verb{alima-plêdidu}{}{[ˈalima pleˈdidu]}{(n.)}{1}{}{Alma penada.}{}{}%
\verb{alkase}{}{[alkaˈsɛ]}{(n.)}{1}{}{Acácia.}{\textit{\textbf{Cassia
spectabilis}.}}{}
\verb{almidi}{}{[alˈmidi]}{(n.)}{1}{}{Almude.}{Cf. \textbf{almudi}.}{}
\verb{almilantxi}{}{[almiˈl\~{\textturna}tʃi]}{(n.)}{1}{}{Almirante.}{}{}
\verb{almilantxi}{}{[almiˈl\~{\textturna}tʃi]}{(n.)}{2}{}{Figurante do
\textbf{sokope}.}{}{}
\verb{almudi}{}{[alˈmudi]}{(n.)}{1}{}{Almude.}{Medida para líquidos.}{}{}
\verb{aloko}{}{[alɔˈkɔ]}{(n.)}{1}{}{Álcool.}{}{}
\verb{alugudon}{}{[aluguˈdõ]}{(n.)}{1}{}{Algodão.}{}{}
\verb{aluku}{}{[ˈaluku]}{(n.)}{1}{}{Arco.}{}{}
\verb{aluku}{}{[ˈaluku]}{(n.)}{1}{}{Arco-íris.}{}{}
\verb{alumanze}{}{[alum\~{\textturna}ˈzɛ]}{(n.)}{}{Armazém.}{}{}
\verb{alumayu}{}{[aluˈmaju]}{(n.)}{1}{}{Armário.}{}{}
\verb{alumidu}{}{[aluˈmidu]}{(n.)}{1}{}{Alumínio.}{}{}
\verb{alumidu}{}{[aluˈmidu]}{(n.)}{1}{}{Recipiente de alumínio.}{}{}
\verb{aluminu}{}{[aluˈminu]}{(n.)}{1}{}{Alumínio.}{Cf. \textbf{alumidu}.}{}
\verb{alunu}{}{[aˈlunu]}{(n.)}{1}{}{Aluno.}{}{}
\verb{aluvu}{}{[ˈaluvu]}{(n.)}{1}{}{Árvore.}{}{}
\verb{aluvu-sê-fya}{}{[ˈaluvu ˈse
ˈfja]}{(n.)}{1}{}{Árvore-sem-folha.}{\textbf{\textit{Euphorbia
tirucalli}}.}{}
\verb{alya}{}{[aˈlja]}{(n.)}{1}{}{Areia.}{}{}
\verb{am}{}{[\~{\textturna}]}{(pron.)}{1}{}{Eu.}{Cf. \textbf{ami.}}{}
\verb{ama}{}{[ˈ\~{\textturna}ma]}{(n.)}{1}{}{Gita.}{\textit{\textbf{Boaedon
lineatus bedriage.}}}{}
\verb{amanha}{}{[am\~{\textturna}ˈɲa]}{(adv.)}{1}{}{Amanhã.}{Cf.
\textbf{amanhan}.}{}
\verb{amanhan}{}{[am\~{\textturna}ˈɲ\~{\textturna}]}{(adv.)}{1}{}{Amanhã.}{}{}
\verb{aman-pasa}{}{[aˈm\~{\textturna} paˈsa]}{(adv.)}{1}{}{Depois de
amanhã.}{}{}
\verb{ama-seka}{}{[ˈ\~{\textturna}ma ˈsɛka]}{(n.)}{1}{}{Ama de leite.}{}{}%
\verb{ama-seka}{}{[ˈ\~{\textturna}ma ˈsɛka]}{(n.)}{1}{}{Ama-seca.}{}{}%
\verb{amblôyô}{}{[\~{\textturna}bloˈjo]}{(adv.)}{1}{}{Abundantemente.}{}{}
\verb{amblôyô}{}{[\~{\textturna}bloˈjo]}{(adv.)}{2}{}{Em grande
quantidade.}{}{}
\verb{amblôyô}{}{[\~{\textturna}bloˈjo]}{(adv.)}{3}{}{Muito.}{}{}
\verb{amblu}{}{[ˈ\~{\textturna}blu]}{(n.)}{1}{}{Ombro.}{}{}
\verb{ambo}{}{[\~{\textturna}ˈbɔ]}{(n.)}{1}{}{Tartaruga-verde.}{\textit{\textbf{Chelonia
midas}.}}{}
\verb{amêjua}{}{[aˈmeʒua]}{(n.)}{1}{}{Amêijoa.}{\textbf{\textit{Cerastoderma
edule}}.}{}
\verb{amen}{}{[ˈamẽ]}{(interj.)}{1}{}{Amén.}{}{}
\verb{amenu}{}{[aˈmɛnu]}{(adv.)}{1}{}{Ao menos.}{}{}
\verb{ami}{}{[aˈmi]}{(pron.)}{1}{}{Eu.}{\textbf{Ami so be}. \textit{Eu é que fui}.}{}{}
\verb{amixtlason}{}{[amiʃtlaˈsõ]}{(n.)}{1}{}{Administração.}{}{}
\verb{amôlê}{}{[aˈmole]}{(n.)}{1}{}{Amor.}{}{}
\verb{amôlê-pedasu}{}{[aˈmole peˈdasu]}{(n.)}{1}{}{Manta ou peça de vestuário
à base de retalhos.}{}{}
\verb{amwêlê}{}{[aˈmwele]}{(n.)}{1}{}{Amor.}{Cf. \textbf{amôlê}.}{}
\verb{an}{}{[\~{\textturna}]}{(int.)}{1}{Onde.}{\textbf{An bô xka be}?
\textit{Onde estás a ir}?}{}{}
\verb{an}{}{[\~{\textturna}]}{(int.)}{2}{Onde.}{Cf. \textbf{andji}.}{}{}
\verb{an}{}{[\~{\textturna}]}{(part.)}{1}{Partícula
interrogativa.}{}{\textbf{Bô tendê an?} \textit{Compreendeste?}}{}%
\verb{an}{}{[\~{\textturna}]}{(part.)}{2}{Partícula enfática que ocorre
tipicamente com os verbos declarativos \textbf{fla} e
\textbf{fada}.}{}{\textbf{Santome fla an\ldots{}} \textit{Em santome
diz-se\ldots{}}}{}
\verb{anan}{}{[aˈn\~{\textturna}]}{(n.)}{1}{}{Anã.}{}{}{}
\verb{anan}{}{[aˈn\~{\textturna}]}{(n.)}{2}{}{Anão.}{}{}{}%
\verb{anda}{}{[ˈ\~{\textturna}da]}{(adv.)}{1}{}{Não é que.}{\textbf{Anda ê ta ka fla}. \textit{Não é que ele estava a dizer.}}{}{}
\verb{anda}{}{[ˈ\~{\textturna}da]}{(adv.)}{2}{}{Afinal.}{}{}{}
\verb{andali}{}{[\~{\textturna}ˈdali]}{(n.)}{1}{}{Andar.}{}{}%
\verb{andali}{}{[\~{\textturna}ˈdali]}{(n.)}{2}{}{Piso.}{}{}%
\verb{andami}{}{[\~{\textturna}ˈdami]}{(n.)}{1}{}{Andaime.}{}{}
\verb{andeji}{}{[\~{\textturna}ˈdɛʒi]}{(adv.)}{1}{}{Ademais.}{}{}%
\verb{andeji}{}{[\~{\textturna}ˈdɛʒi]}{(adv.)}{2}{}{Além disso.}{}{}%
\verb{andeji}{}{[\~{\textturna}ˈdɛʒi]}{(adv.)}{3}{}{Aliás.}{}{}
\verb{andeji me}{}{[\~{\textturna}ˈdɛʒi ˈmɛ]}{(adv.)}{1}{}{Além disso.}{}{}
\verb{andeji me}{}{[\~{\textturna}ˈdɛʒi ˈmɛ]}{(adv.)}{2}{}{Tanto é que.}{}{}
\verb{andji}{}{[ˈ\~{\textturna}dʒi]}{(int.)}{1}{}{Onde.}{}{}
\verb{andji}{}{[\~{\textturna}ˈdʒi]}{(n.)}{1}{}{Andim.}{\textit{\textbf{Elaeis
guinensis}.}}{}{}
\verb{andôlin}{}{[\~{\textturna}doˈlĩ]}{(n.)}{1}{}{Andorinha.}{}{}
\verb{andôlin}{}{[\~{\textturna}doˈlĩ]}{(n.)}{1}{}{Rabo-espinhoso de São
Tomé.}{\textit{\textbf{Zoonavena thomensis}.}}{}
\verb{andôlô}{}{[\~{\textturna}ˈdolo]}{(n.)}{1}{}{Andar da procissão.}{}{}%
\verb{andôlô}{}{[\~{\textturna}ˈdolo]}{(n.)}{2}{}{Andor.}{}{}
\verb{angu}{}{[\~{\textturna}ˈgu]}{(n.)}{1}{}{\textit{Angu}.}{}{}
\verb{angu}{}{[\~{\textturna}ˈgu]}{(n.)}{2}{}{Massa de banana.}{}{}%
\verb{anha}{}{[\~{\textturna}ˈɲa]}{(v.)}{1}{}{Agarrar.}{}{}
\verb{anha}{}{[\~{\textturna}ˈɲa]}{(v.)}{2}{}{Apanhar.}{}{}%
\verb{anili}{}{[aˈnili]}{(n.)}{1}{}{Anil.}{}{}{}
\verb{anji}{}{[ˈ\~{\textturna}ʒi]}{(int.)}{1}{}{Onde.}{Cf. \textbf{andji}.}{}\verb{anjogo}{}{[\~{\textturna}ʒɔˈgɔ]}{(n.)}{1}{}{\textit{Anjogo}.}{\textit{\textbf{Tristemma
mauritianum}}.}{}
\verb{anka}{}{[ˈ\~{\textturna}ka]}{(n.)}{1}{}{Caranguejo.}{}{}
\verb{ankêlê}{}{[ˈ\~{\textturna}kele]}{(n.)}{1}{}{Rã.}{}{}
\verb{ankêlê}{}{[ˈ\~{\textturna}kele]}{(n.)}{1}{}{Sapo.}{Cf. \textbf{akêlê}.}{}
\verb{ankôla}{}{[ˈ\~{\textturna}kola]}{(n.)}{1}{}{Âncora.}{}{}
\verb{antan}{}{[\~{\textturna}ˈt\~{\textturna}]}{(adv.)}{1}{}{Então.}{Cf.
\textbf{anton}.}{}
\verb{antawo}{}{[\~{\textturna}taˈwɔ]}{(adv.)}{1}{}{Ainda.}{\textbf{Bô sa
mina-pikina antawo}. \textit{Ainda és uma criança}.}{}
\verb{antawo}{}{[\~{\textturna}taˈwɔ]}{(adv.)}{2}{}{Ainda não.}{\textbf{N na
tê fa. Antawo.} \textit{Não tenho. Ainda não}.}{}%
\verb{antê}{}{[\~{\textturna}ˈte]}{(adv.)}{1}{}{Até.}{}{}{}
\verb{antê}{}{[\~{\textturna}ˈte]}{(adv.)}{1}{}{Inclusive.}{}{}{}
\verb{antê}{}{[\~{\textturna}ˈte]}{(prep.)}{1}{}{Até.}{}{}{}
\verb{antê}{}{[\~{\textturna}ˈte]}{(prep.)}{2}{}{Até que.}{}{}{}
\verb{antê ku}{}{[\~{\textturna}ˈte ˈku]}{(conj.)}{1}{}{Até que.}{}{}
\verb{antix}{}{[\~{\textturna}ˈtiʃ]}{(adv.)}{1}{}{Antes.}{}{}%
\verb{antix}{}{[\~{\textturna}ˈtiʃ]}{(adv.)}{2}{}{Outrora.}{}{}
\verb{antix ku}{}{[ˈ\~{\textturna}tiʃ ˈku]}{(conj.)}{1}{}{Antes.}{}{}
\verb{antix pa}{}{[ˈ\~{\textturna}tiʃ ˈpa]}{(conj.)}{1}{}{Antes que.}{}{}
\verb{anton}{}{[\~{\textturna}ˈtõ]}{(adv.)}{1}{}{Então.}{}{}
\verb{antonte}{}{[\~{\textturna}ˈt\~ɔtɛ]}{(adv.)}{1}{}{Anteontem.}{Cf. \textbf{antonten}.}{}{}
\verb{antonten}{}{[\~{\textturna}ˈt\~ɔt\~ɛ]}{(adv.)}{2}{}{Anteontem.}{}{}
\verb{anu}{}{[ˈ\~{\textturna}nu]}{(n.)}{1}{}{Ano.}{}{}
\verb{anumu}{}{[ˈanumu]}{(n.)}{1}{}{Alento.}{}{}{}
\verb{anumu}{}{[ˈanumu]}{(n.)}{1}{}{Ânimo.}{}{}{}
\verb{anxa}{}{[ˈ\~{\textturna}ʃa]}{(n.)}{1}{}{Angústia.}{}{}
\verb{anxa}{}{[ˈ\~{\textturna}ʃa]}{(n.)}{2}{}{Ânsia.}{}{}%
\verb{anxa}{}{[ˈ\~{\textturna}ʃa]}{(n.)}{3}{}{Suspiro.}{}{}%
\verb{anzu}{}{[ˈ\~{\textturna}zu]}{(n.)}{1}{}{Anjo.}{}{}%
\verb{anzu}{}{[ˈ\~{\textturna}zu]}{(n.)}{2}{}{Bebê.}{}{}
\verb{anzu}{}{[ˈ\~{\textturna}zu]}{(n.)}{3}{}{Criança.}{}{}
\verb{anzu-dêsu}{}{[ˈ\~{\textturna}zu ˈdesu]}{(n.)}{1}{}{Bebê.}{}{}
\verb{anzu-mama}{}{[ˈ\~{\textturna}zuˈm\~{\textturna}ma]}{(n.)}{1}{}{Recém-nascido.}{}{}
\verb{apa}{}{[aˈpa]}{(n.)}{1}{}{Pá.}{}{}
\verb{apilidu}{}{[apiˈlidu]}{(n.)}{1}{}{Apelido.}{}{}
\verb{apoxtulu}{}{[aˈpɔʃtulu]}{(n.)}{1}{}{Apóstolo.}{}{}%
\verb{apoxtulu}{}{[aˈpɔʃtulu]}{(n.)}{2}{}{Discípulo.}{}{}
\verb{apoyadu}{}{[apɔˈjadu]}{(adj.)}{1}{}{Apoiado.}{}{}
\verb{ase}{}{[aˈsɛ]}{(n.)}{1}{}{Sé-catedral.}{}{}
\verb{asoxiason}{}{[asɔʃiaˈs\~ɔ]}{(n.)}{1}{}{Associação.}{}{}
\verb{asu}{}{[ˈasu]}{(n.)}{1}{}{Aço.}{}{}
\verb{asuntu}{}{[aˈsũtu]}{(n.)}{1}{}{Assunto.}{}{}%
\verb{ata}{}{[aˈta]}{(n.)}{1}{}{Mandioca.}{\textbf{\textit{Manihot
esculenta}}.}{}
\verb{atali}{}{[aˈtali]}{(n.)}{1}{}{Altar.}{}{}
\verb{atisô}{}{[atiˈso]}{(n.)}{1}{}{\textit{Atisô}.}{Tipo de tecido.}{}
\verb{atlimija}{}{[atliˈmiʒa]}{(n.)}{1}{}{Natruja.}{\textit{\textbf{Artemisia
vulgaris}.}}{}
\verb{atukaru}{}{[atuˈkaru]}{(n.)}{1}{}{Autocarro.}{}{}
\verb{atukaru}{}{[atuˈkaru]}{(n.)}{2}{}{Ônibus.}{}{}
\verb{atxi}{}{[ˈatʃi]}{(n.)}{1}{}{Arte.}{}{}
\verb{avia}{}{[aˈvia]}{(v.)}{1}{}{Havia.}{Fórmula discursiva com que se
iniciam histórias tradicionais, como em: {\textbf{Avia ũa alê.} \textit{Era
uma vez um rei.}}}{}
\verb{avlê}{}{[ˈavle]}{(interj.)}{1}{}{Ave!}{}{}
\verb{avlê}{}{[ˈavle]}{(interj.)}{2}{}{Salve!}{}{}
\verb{avlê-maya}{}{[ˈavle maˈja]}{(n.)}{1}{}{Ave-Maria.}{}
\verb{avogadu}{}{[avɔˈgadu]}{(n.)}{1}{}{Advogado.}{}{}
\verb{avyon}{}{[aˈvjõ]}{(n.)}{1}{}{Avião.}{}{}
\verb{awa}{}{[ˈawa]}{(n.)}{1}{}{Água.}{}{}%
\verb{awa}{}{[ˈawa]}{(n.)}{2}{}{Ribeira.}{}{}
\verb{awa}{}{[ˈawa]}{(n.)}{3}{}{Rio.}{}{}
\verb{awa-bôbô}{}{[ˈawa boˈbo]}{(n.)}{2}{}{Nascente de água.}{}{}
\verb{awa-boka}{}{[ˈawa ˈbɔka]}{(n.)}{1}{}{Baba.}{}{}%
\verb{awa-boka}{}{[ˈawa ˈbɔka]}{(n.)}{2}{}{Restos.}{}{}
\verb{awa-boka}{}{[ˈawa ˈbɔka]}{(n.)}{3}{}{Saliva.}{}{}
\verb{awa-boka}{}{[ˈawa ˈbɔka]}{(n.)}{4}{}{Sobejos.}{}{}
\verb{awadentxi}{}{[ˈawaˈdẽtʃi]}{(n.)}{1}{}{Aguardente.}{}{}%
\verb{awadentxi}{}{[ˈawaˈdẽtʃi]}{(n.)}{2}{}{Cachaça.}{}{}
\verb{awa-flêbê}{}{[ˈawa fleˈbe]}{(n.)}{1}{}{Água mineral gasosa.}{}{}
\verb{awa-fumadu}{}{[ˈawa fuˈmadu]}{(n.)}{1}{}{}{Dique.}{}
\verb{awa-kalabana}{}{[ˈawa ˈkalaˈbana]}{(n.)}{1}{}{Xarope Carabana.}{}{}
\verb{awa-kobo}{}{[ˈawa kɔˈbɔ]}{(n.)}{1}{}{Poço escavado.}{}{}
\verb{awa-lôdô}{}{[ˈawa ˈlodo]}{(n.)}{1}{}{Charco.}{}{}
\verb{awa-lôdô}{}{[ˈawa ˈlodo]}{(n.)}{1}{}{Lagoa.}{}{}
\verb{awa-lôdô}{}{[ˈawa ˈlodo]}{(n.)}{2}{}{Laguna.}{}{}%
\verb{awa-matu}{}{[ˈawa ˈmatu]}{(n.)}{1}{}{Cheia.}{}{}
\verb{awa-matu}{}{[ˈawa ˈmatu]}{(n.)}{2}{}{Enchente.}{}{}
\verb{awa-matu}{}{[ˈawa ˈmatu]}{(n.)}{3}{}{Enxurrada.}{}{}%
\verb{awa-po}{}{[ˈawa ˈpɔ]}{(n.)}{1}{}{Esperma.}{}{}
\verb{awa toma}{}{[tɔˈma]}{(expr.)}{1}{}{Alagar.}{\textbf{Awa toma poson}. \textit{A cidade ficou alagada.}}{}
\verb{awa-txotxo}{}{[ˈawa tʃɔˈtʃɔ]}{(n.)}{1}{}{Certos pequenos cursos
d'água.}{}{}%
\verb{awa-wê}{}{[ˈawa ˈwe]}{(n.)}{1}{}{Lágrima.}{}{}
\verb{awa-xelele}{}{[ˈawa ʃɛlɛˈlɛ]}{(n.)}{1}{}{Riacho de águas calmas
(utilizado para a plantação de agrião).}{}{}
\verb{awa-xelele}{}{[ˈawa ʃɛlɛˈlɛ]}{(n.)}{1}{}{Água potável.}{}{}
\verb{awa-xelele}{}{[ˈawa ʃɛlɛˈlɛ]}{(n.)}{3}{}{Ribeiro.}{}{}
\verb{awo}{}{[aˈwɔ]}{(interj.)}{1}{}{Haja Paciência!}{}{}{}
\verb{axa}{}{[ˈaʃa]}{(n.)}{1}{}{Acha.}{}{}
\verb{axa}{}{[ˈaʃa]}{(n.)}{2}{}{Bordão.}{}{}
\verb{axa}{}{[ˈaʃa]}{(n.)}{3}{}{Cacete empregado na \textbf{bliga}}{}{}%
\verb{axa}{}{[ˈaʃa]}{(n.)}{4}{}{Pau.}{}{}%
\verb{axa-mon}{}{[ˈaʃa ˈmõ]}{(n.)}{1}{}{Antebraço.}{}{}
\verb{axa-mon}{}{[ˈaʃa ˈmõ]}{(n.)}{1}{}{Braço.}{}{}
\verb{axen}{}{[aˈʃ\~ɛ]}{(adv.)}{1.}{}{Assim.}{}{}{}
\verb{axen}{}{[aˈʃ\~ɛ]}{(adv.)}{1.}{}{Desta forma.}{}{}{}
\verb{axen me}{}{[aˈʃ\~ɛ ˈmɛ]}{(conj.)}{1}{}{Apesar disso.}{}{}
\verb{axen me}{}{[aˈʃ\~ɛ ˈmɛ]}{(conj.)}{2}{}{Mesmo assim.}{}{}
\verb{axen me}{}{[aˈʃ\~ɛ ˈmɛ]}{(interj.)}{1}{}{Bem feito!}{}{}
\verb{axen so}{}{[aˈʃ\~ɛ ˈsɔ]}{(interj.)}{1}{}{A mesma história de
sempre!}{}{}{}
\verb{axen so}{}{[aˈʃ\~ɛ ˈsɔ]}{(interj.)}{2}{}{Sempre o mesmo!}{}{}{}
\verb{axi}{}{[aˈʃi]}{(adv.)}{1}{}{Assim.}{}{}
\verb{axi bonja}{}{[ˈaʃi bõˈʒa]}{(adv.)}{1}{}{Ainda bem.}{}{}
\verb{axi fa}{}{[aˈʃi ˈfa]}{(conj.)}{1}{}{Caso contrário.}{}{}
\verb{axi fa}{}{[aˈʃi ˈfa]}{(conj.)}{2}{}{Senão.}{}{}%
\verb{axitentxi}{}{[aʃiˈt\~etʃi]}{(adj.)}{1}{}{Assistente.}{}{}
\verb{ayen}{}{[aˈjẽ]}{(adv.)}{1}{}{Aqui.}{}{}
\verb{ayu}{}{[ˈaju]}{(n.)}{1}{}{Alho.}{}{}
\verb{ayu-d'ôbô}{}{[ˈaju
doˈbo]}{(n.)}{1}{}{\textit{Alho-d'ôbô}.}{\textit{\textbf{Psychotria
peduncularis}.}}{}
\verb{aza}{}{[ˈaza]}{(n.)}{1}{}{Asa.}{}{}
\verb{aza}{}{[aˈza]}{(n.)}{1}{}{Azar.}{}{}
\verb{azagwa}{}{[aˈzagwa]}{(n.)}{1}{}{\textit{Azagwa}.}{Prato típico da Ilha do Príncipe, à base de folhas e feijão, acompanhado de farinha de mandioca.}{}{}{}
\verb{aza-pixi}{}{[ˈaza ˈpiʃi]}{(n.)}{1}{}{Barbatana.}{}{}
\verb{aze}{}{[aˈzɛ]}{(id.)}{1}{}{Cf. \textbf{fitxisêlu aze}.}{}{}
\verb{azêtô}{}{[azeˈto]}{(adj.)}{1}{}{Abandonado.}{}{}{}
\verb{azêtô}{}{[azeˈto]}{(adj.)}{2}{}{Desertado.}{}{}{}
\verb{azunu}{}{[ˈazunu]}{(n.)}{1}{}{Asno.}{\textit{\textbf{Balistes
punctatus}.}}{}
\end{letra}

\begin{letra}{b}

\verb{ba}{}{[ˈba]}{(onom.)}{1}{Exprime o som de um embate.}{}{}{}
\verb{ba}{}{[ˈba]}{(prep. v.)}{1}{}{Para.}{\textbf{Non kôlê ba ke}.
\textit{Corremos para casa.}}{}
\verb{ba}{}{[ˈba]}{(v.)}{1}{}{Ir (quando seguido de complemento
locativo).}{\textbf{Non ba ple}. \textit{Fomos à praia}.}{}
\verb{ba awa la kwa}{}{[ˈba ˈawa ˈla ˈkwa]}{(expr.)}{1}{}{Lavar loiça ou
roupa no rio.}{}{}
\verb{baba}{}{[baˈba]}{(n.)}{1}{}{Laço.}{}{}
\verb{bababa}{}{[babaˈba]}{(adv.)}{1}{}{Exatamente.}{}{}
\verb{bababa}{}{[babaˈba]}{(adv.)}{2}{}{Precisamente.}{}{}
\verb{bababa}{}{[babaˈba]}{(id.)}{1}{}{Cf. \textbf{vlêmê bababa.}}{}
\verb{babaka}{}{[babaˈka]}{(id.)}{1}{}{Cf. \textbf{pya babaka.}}{}{}
\verb{babaka}{}{[babaˈka]}{(v.)}{1}{}{Embasbacar.}{}{}
\verb{babaka}{}{[babaˈka]}{(v.)}{2}{}{Ficar admirado.}{}{}
\verb{babakadu}{}{[babaˈkadu]}{(adj.)}{1}{}{Boquiaberto.}{}{}
\verb{babakadu}{}{[babaˈkadu]}{(adj.)}{2}{}{Embasbacado.}{}{}
\verb{babakadu}{}{[babaˈkadu]}{(adj.)}{3}{}{Pasmado.}{}{}
\verb{ba bligidi}{}{[ˈba bligiˈdi]}{(expr.)}{1}{}{Desmoronar.}{}{}
\verb{ba bligidi}{}{[ˈba bligiˈdi]}{(expr.)}{1}{}{Despencar.}{}{}
%\verb{baboza}{}{[baˈbɔza]}{(n.)}{1}{}{Aloés.}{Cf. \textbf{aliba-blaboza}.}{}{}
\verb{badêzu}{}{[baˈdezu]}{(n.)}{1}{}{Badejo.}{\textbf{\textit{Epinephelus
goreensis}}.}{}
\verb{badja}{}{[ˈbadʒa]}{(n.)}{1}{}{Banho tradicional à base de folhas
medicinais.}{}{}
\verb{badja-badja}{}{[ˈbadʒa ˈbadʒa]}{(v.)}{1}{}{Ato de dar o \textbf{badja},
massageando suavemente o corpo.}{}{}
\verb{badji}{}{[ˈbadʒi]}{(n.)}{1}{}{Abade.}{}{}
\verb{badji}{}{[ˈbadʒi]}{(n.)}{2}{}{Balde.}{}{}
\verb{badji}{}{[ˈbadʒi]}{(n.)}{2}{}{Fruta-pão muito verde, usada para cozer
apenas.}{}{}
\verb{badô}{}{[baˈdo]}{(n.)}{1}{}{Aquele que vai.}{}{}
\verb{badu}{}{[ˈbadu]}{(adj.)}{1}{}{Ido.}{}{}
\verb{baêta}{}{[baˈeta]}{(n.)}{1}{}{Baeta.}{}{}
\verb{bafa}{}{[baˈfa]}{(v.)}{1}{}{Abafar.}{}{}
\verb{bafa}{}{[baˈfa]}{(v.)}{2}{}{Ocultar.}{}{}
\verb{bafa}{}{[baˈfa]}{(v.)}{3}{}{Tapar.}{}{}
\verb{bafadu}{}{[baˈfadu]}{(adj.)}{1}{}{Abafado.}{}{}
\verb{bafadu}{}{[baˈfadu]}{(adj.)}{2}{}{Oculto.}{}{}
\verb{bafadu}{}{[baˈfadu]}{(adj.)}{3}{}{Tapado.}{}{}
\verb{bafama}{}{[bafaˈma]}{(v.)}{1}{}{Afamar.}{}{}
\verb{bafama}{}{[bafaˈma]}{(v.)}{1}{}{Difamar.}{}{}
\verb{bafama}{}{[bafaˈma]}{(v.)}{1}{}{Ter fama.}{}{}
\verb{bafamadu}{}{[bafaˈmadu]}{(adj.)}{1}{}{Famoso.}{}{}
\verb{baga}{}{[ˈbaga]}{(n.)}{1}{}{Panela.}{Cf. \textbf{ubaga}.}{}{}
\verb{bagadji}{}{[baˈgadʒi]}{(n.)}{1}{}{Bagagem.}{}{}
\verb{bagaji}{}{[baˈgaʒi]}{(n.)}{1}{}{Bagagem.}{Cf. \textbf{bagadji}.}{}{}
\verb{bagasa}{}{[bagaˈsa]}{(v.)}{1}{}{Despedaçar.}{}{}
\verb{bagasa}{}{[bagaˈsa]}{(v.)}{1}{}{Esboroar.}{}{}
\verb{bagasa}{}{[bagaˈsa]}{(v.)}{1}{}{Retalhar.}{}{}
\verb{bagasadu}{}{[bagaˈsadu]}{(adj.)}{1}{}{Retalhado.}{}{}
\verb{bagasadu}{}{[bagaˈsadu]}{(adj.)}{1}{}{Com vários cortes.}{}{}
\verb{bagasu}{}{[baˈgasu]}{(n.)}{1}{}{Aguardente.}{}{}
\verb{bagasu}{}{[baˈgasu]}{(n.)}{2}{}{Bagaço.}{}{}
\verb{bagon}{}{[baˈgõ]}{(n.)}{1}{}{Vagão.}{}{}
\verb{bagon}{}{[baˈgõ]}{(n.)}{2}{}{Vagoneta.}{}{}
\verb{bagu-d'ovu}{}{[ˈbagu ˈdɔvu]}{(n.)}{1}{}{Testículos.}{}{}
\verb{baji}{}{[ˈbaʒi]}{(n.)}{1}{}{fruta-pão muito verde, usada para cozer apenas.}{}{}
\verb{baji}{}{[ˈbaʒi]}{(n.)}{2}{}{Balde.}{Cf. \textbf{badji}.}{}{}
\verb{bakatxi}{}{[baˈkatʃi]}{(n.)}{1}{}{Abacate.}{}{}{}
\verb{bakatxi}{}{[baˈkatʃi]}{(n.)}{2}{}{Abacateiro.}{\textit{\textbf{Persea
americana}.}}{}{}
\verb{bakaya}{}{[bakaˈja]}{(n.)}{1}{Bacalhau.}{}{}{}%
\verb{bakaya}{}{[bakaˈja]}{(n.)}{2}{}{Bacalhau seco.}{}{}
\verb{bakê}{}{[baˈke]}{(n.)}{1}{}{Borda da panela.}{}{}
\verb{bakê}{}{[baˈke]}{(n.)}{2}{}{Prisão de ventre.}{}{}	
\verb{bakê}{}{[baˈke]}{(n.)}{3}{}{Recipiente para cozinhar.}{}{}
\verb{bakôwa}{}{[bakoˈwa]}{(n.)}{1}{}{Anã.}{}{}
\verb{bakôwa}{}{[bakoˈwa]}{(n.)}{2}{}{Anão.}{}{}
\verb{bakuda}{}{[baˈkuda]}{(n.)}{1}{}{Barracuda.}{\textit{\textbf{Sphyraena
guachancho}.}}{}
\verb{bala}{}{[ˈbala]}{(n.)}{2}{}{Barra.}{}{}
\verb{bala}{}{[ˈbala]}{(n.)}{3}{}{Bola de ferro empregada para descascar as
sementes do \textit{izaquente}.}{}{}
\verb{balabala}{}{[ˈbalaˈbala]}{(n.)}{1}{}{Brotos.}{}{}{}
\verb{balabala}{}{[ˈbalaˈbala]}{(n.)}{2}{}{Mulato.}{\textit{\textbf{Paranthias
furcifer}.}}{}
\verb{balakon}{}{[balaˈkõ]}{(n.)}{1}{}{Balcão.}{}{}
\verb{balansa}{}{[baˈl\~{\textturna}sa]}{(n.)}{1}{}{Balança.}{}{}
\verb{balansa}{}{[bal\~{\textturna}ˈsa]}{(v.)}{1}{}{Balançar.}{}{}%
\verb{balansa}{}{[bal\~{\textturna}ˈsa]}{(v.)}{2}{}{Movimentar.}{}{}
\verb{balela}{}{[balɛˈla]}{(v.)}{1}{}{Melhorar.}{}{}%
\verb{balela}{}{[balɛˈla]}{(v.)}{2}{}{Sarar.}{}{}
\verb{balela}{}{[balɛˈla]}{(v.)}{3}{}{Sentir melhoras.}{}{}
\verb{balela}{}{[balɛˈla]}{(v.)}{4}{}{Ventilar a roupa no estendal.}{}{}
\verb{balela}{}{[baˈlɛla]}{(n.)}{2}{}{Tribuna tradicional.}{}{}
\verb{baleladu}{}{[balɛˈladu]}{(adj.)}{1}{}{Assim-assim.}{}{}%
\verb{baleladu}{}{[balɛˈladu]}{(adj.)}{2}{}{Mais ou menos.}{}{}%
\verb{baleladu}{}{[balɛˈladu]}{(adj.)}{3}{}{Melhorado.}{}{}%
\verb{baleladu}{}{[balɛˈladu]}{(adj.)}{4}{}{Remediado.}{}{}
\verb{baleladu}{}{[balɛˈladu]}{(adj.)}{5}{}{Sarado.}{}{}
\verb{balele}{}{[ˈbalɛlɛ]}{(adj.)}{1}{}{De baixa estatura.}{}{}
\verb{balele}{}{[ˈbalɛlɛ]}{(n.)}{1}{}{Galinha-garnisé.}{Cf.
\textbf{nganha-balele}.}{}{}
\verb{balêtê}{}{[baˈlete]}{(n.)}{1}{}{Barrete.}{}{}
\verb{balêtê}{}{[baˈlete]}{(n.)}{1}{}{Chapéu.}{}{}
\verb{balha}{}{[baˈʎa]}{(n.)}{1}{}{Baleia.}{}{}
\verb{balha}{}{[baˈʎa]}{(v.)}{1}{}{Baralhar.}{}{}
\verb{balha}{}{[baˈʎa]}{(v.)}{2}{}{Embaralhar.}{}{}
\verb{bali}{}{[baˈli]}{(v.)}{1}{}{Varrer.}{}{}
\verb{balidô}{}{[baliˈdo]}{(n.)}{1}{}{Varredor.}{}{}
\verb{balidu}{}{[baˈlidu]}{(adj.)}{1}{}{Varrido.}{}{}
\verb{balili}{}{[baˈlili]}{(n.)}{1}{}{Barril.}{}{}
\verb{balingwa}{}{[baˈlĩgwa]}{(v.)}{1}{}{Emboscar.}{Cf.
\textbf{balungwa}.}{}{}
\verb{balingwadu}{}{[balĩˈgwadu]}{(adj.)}{1}{}{Emboscado.}{Cf.
\textbf{balungwadu}.}{}{}
\verb{balon}{}{[baˈlõ]}{(n.)}{1}{}{Balão.}{}{}
\verb{balon}{}{[baˈlõ]}{(n.)}{2}{}{Varrão.}{Cf. \textbf{plôkô-balon}.}{}{}
\verb{balu}{}{[ˈbalu]}{(n.)}{1}{}{Argila.}{}{}%
\verb{balu}{}{[ˈbalu]}{(n.)}{2}{}{Barro.}{}{}%
\verb{balu}{}{[ˈbalu]}{(n.)}{3}{}{Terra.}{}{}
\verb{balungwa}{}{[balũˈgwa]}{(v.)}{1}{}{Emboscar.}{}{}
\verb{balungwa}{}{[balũˈgwa]}{(v.)}{2}{}{Esconder(-se).}{}{}%
\verb{balungwa}{}{[balũˈgwa]}{(v.)}{3}{}{Tocaiar.}{}{}
\verb{balungwadu}{}{[balũˈgwadu]}{(adj.)}{1}{}{Emboscado.}{}{}
\verb{balungwadu}{}{[balũˈgwadu]}{(adj.)}{2}{}{Escondido.}{}{}
\verb{balya}{}{[baˈlja]}{(n.)}{1}{}{Baleia.}{Cf. \textbf{balha}.}{}{}
\verb{bambakwere}{}{[b\~{\textturna}bakwɛˈrɛ]}{(n.)}{1}{}{Pobre diabo.}{}{}
\verb{bambi}{}{[b\~{\textturna}ˈbi]}{(n.)}{1}{}{Depressão.}{}{}
\verb{bambi}{}{[b\~{\textturna}ˈbi]}{(n.)}{2}{}{Pneumonia.}{}{}
\verb{bambi}{}{[b\~{\textturna}ˈbi]}{(n.)}{3}{}{Tristeza causada por decepção, a qual pode ser curada através do ritual \textbf{kota-bambi}.}{}{}{}
\verb{bambu}{}{[b\~{\textturna}ˈbu]}{(n.)}{1}{}{Bambu.}{\textit{\textbf{Bambusa vulgaris}}.}{}{}
\verb{bamu}{}{[ˈb\~{\textturna}mu]}{(v.)}{1}{}{Vamos.}{}{}
\verb{bana}{}{[ˈbana]}{(n.)}{1}{}{Banana-pão.}{\textbf{\textit{Musa
sapientum.}}}{}
\verb{bana}{}{[ˈbana]}{(n.)}{2}{}{Banana-da-terra.}{\textbf{\textit{Musa
sapientum.}}}{}
\verb{bana}{}{[baˈna]}{(v.)}{1}{}{Abanar.}{}{}
\verb{bana-aga}{}{[ˈbana ˈaga]}{(n.)}{1}{}{Banana-pão.}{\textbf{\textit{Musa
sapientum.}}}{}%
\verb{bana-aga}{}{[ˈbana
ˈaga]}{(n.)}{2}{}{Banana-da-terra.}{\textbf{\textit{Musa sapientum.}}}{}
\verb{banadô}{}{[banaˈdo]}{(n.)}{1}{}{Abanador.}{}{}
\verb{banadô}{}{[banaˈdo]}{(n.)}{2}{}{Leque.}{}{}
\verb{bana-gabon}{}{[baˈna gaˈb\~ɔ]}{(n.)}{1}{}{Banana-pão
roxa.}{\textit{\textbf{Musa velutina}}.}{}
\verb{bana-manson}{}{[ˈbana m\~{\textturna}ˈsõ]}{(n.)}{1}{}{Banana-maçã.}{}{}\verb{bana-manson}{}{[ˈbana
m\~{\textturna}ˈsõ]}{(n.)}{2}{}{Bananeira-maçã.}{\textbf{\textit{Musa
acuminata}}.}{}%
\verb{bana-mpon}{}{[ˈbana ˈmpõ]}{(n.)}{1}{}{Banana-pão.}{}{}
\verb{bana-mpon}{}{[ˈbana
ˈmpõ]}{(n.)}{2}{}{Bananeira-pão.}{\textbf{\textit{Musa paradisiaca}}.}{}
\verb{bana-mwala}{}{[ˈbana
ˈmwala]}{(n.)}{1}{}{Esponjeira.}{\textit{\textbf{Acacia farnesiana}}.}{}
\verb{bana-mwala}{}{[ˈbana ˈmwala]}{(n.)}{1}{}{Banana-prata de casca muito
fina.}{}{}
\verb{bana-ôlô}{}{[ˈbana ˈolo]}{(n.)}{1}{}{Banana-ouro.}{}{}
\verb{bana-ôlô}{}{[ˈbana
ˈolo]}{(n.)}{1}{}{Bananeira-ouro.}{\textbf{\textit{Musa x paradisiaca}.}}{}
\verb{bana-plata}{}{[ˈbana ˈplata]}{(n.)}{1}{}{Banana-prata.}{}{}
\verb{bana-plata}{}{[ˈbana
ˈplata]}{(n.)}{2}{}{Bananeira-prata.}{\textit{\textbf{Musa balbisiana}}.}{}%
\verb{bana-plata}{}{[ˈbana ˈplata]}{(n.)}{3}{}{Banana-prata.}{Cf.
\textbf{kitxiba}.}{}{}
\verb{banda}{}{[ˈb\~{\textturna}da]}{(n.)}{1}{}{Lado.}{}{}%
\verb{banda}{}{[ˈb\~{\textturna}da]}{(n.)}{2}{}{Parte.}{}{}
\verb{banda}{}{[ˈb\~{\textturna}da]}{(n.)}{3}{}{Região.}{}{}%
\verb{banda-basu}{}{[ˈb\~{\textturna}da ˈbasu]}{(n.)}{3}{}{Cidade de São Tomé e imediações.}{}{}
\verb{bandela}{}{[b\~{\textturna}ˈdɛla]}{(n.)}{1}{}{Bandeira.}{}{}
\verb{bandela}{}{[b\~{\textturna}dɛˈla]}{(v.)}{1}{}{Embandeirar.}{}{}
\verb{bandela}{}{[b\~{\textturna}dɛˈla]}{(v.)}{2}{}{Passear.}{}{}
\verb{bandela}{}{[b\~{\textturna}dɛˈla]}{(v.)}{3}{}{Vaguear.}{}{}
\verb{bandêza}{}{[b\~{\textturna}ˈdeza]}{(n.)}{1}{}{Bandeja.}{}{}
\verb{bandêza}{}{[b\~{\textturna}ˈdeza]}{(n.)}{2}{}{Tabuleiro.}{}{}
\verb{bandêza}{}{[b\~{\textturna}ˈdeza]}{(n.)}{3}{}{Travessa.}{}{}
\verb{bandilha}{}{[b\~{\textturna}ˈdiʎa]}{(n.)}{1}{}{Baunilha.}{\textit{\textbf{Vanilla planifolia}}.}{}
\verb{bandôlin}{}{[b\~{\textturna}doˈl\~i]}{(n.)}{1}{}{Bandolim.}{}{}
\verb{bandu}{}{[ˈb\~{\textturna}du]}{(n.)}{1}{}{Bando.}{}{}
\verb{banha}{}{[ˈb\~{\textturna}ɲa]}{(n.)}{1}{}{Banha.}{}{}
\verb{banha}{}{[ˈb\~{\textturna}ɲa]}{(n.)}{2}{}{Gordura.}{}{}
\verb{banha}{}{[b\~{\textturna}ˈɲa]}{(v.)}{1}{}{Embainhar.}{}{}
\verb{banha}{}{[b\~{\textturna}ˈɲa]}{(v.)}{1}{}{Fazer a bainha.}{}{}
\verb{banhu}{}{[ˈb\~{\textturna}ɲu]}{(n.)}{1}{}{Banho.}{}{}
\verb{banka}{}{[b\~{\textturna}ˈka]}{(n.)}{1}{}{\textit{Banka}.}{\textbf{\textit{Tephrosia vogelii}.}}{}
\verb{banka}{}{[b\~{\textturna}ˈka]}{(v.)}{1}{}{Abancar.}{}{}
\verb{banka}{}{[b\~{\textturna}ˈka]}{(v.)}{2}{}{Instalar-se.}{}{}
\verb{bankêtê}{}{[b\~{\textturna}ˈkete]}{(n.)}{1}{}{Banquete.}{}{}
\verb{banku}{}{[ˈb\~{\textturna}ku]}{(n.)}{1}{}{Assento.}{}{}
\verb{banku}{}{[ˈb\~{\textturna}ku]}{(n.)}{2}{}{Banco.}{(Instituição financeira.)}{}
\verb{banku-tason}{}{[ˈb\~{\textturna}ku taˈsõ]}{(n.)}{1}{}{Banquinho.}{}{}
\verb{bansa}{}{[ˈb\~{\textturna}sa]}{(n.)}{1}{}{Haste feita com o ramo da palmeira}{}{}%
\verb{bansa}{}{[ˈb\~{\textturna}sa]}{(n.)}{2}{Vara feita com a haste do ramo da palmeira.}{}{}
\verb{bansa}{}{[b\~{\textturna}ˈsa]}{(n.)}{1}{}{Costela.}{}{}
\verb{banza}{}{[ˈb\~{\textturna}za]}{(n.)}{1}{}{Chocolate.}{}{}
\verb{banzu}{}{[ˈb\~{\textturna}zu]}{(n.)}{1}{}{Boquiaberto.}{}{}
\verb{banzu}{}{[ˈb\~{\textturna}zu]}{(n.)}{2}{}{Surpreendido.}{}{}
\verb{basa}{}{[ˈbasa]}{(n.)}{1}{}{Onda.}{}{}
\verb{basô}{}{[baˈso]}{(n.)}{1}{}{Flatulência.}{}{}
\verb{basola}{}{[baˈsɔla]}{(n.)}{1}{}{Vassoura.}{}{}
\verb{basu}{}{[ˈbasu]}{(n.)}{1}{}{Baço.}{}{}
\verb{basu}{}{[ˈbasu]}{(n.)}{2}{}{Vagina.}{}{}
\verb{basu}{}{[ˈbasu]}{(prep. n.)}{1}{}{Baixo.}{}{}%
\verb{basu}{}{[ˈbasu]}{(prep. n.)}{2}{}{Debaixo.}{}{}%
\verb{basu}{}{[ˈbasu]}{(prep. n.)}{4}{}{Embaixo.}{}{}%
\verb{basu-d'ite}{}{[ˈbasu ˈditɛ]}{(n.)}{1}{}{Baixo-ventre.}{}{}
\verb{basu-kadjan}{}{[ˈbasu kaˈdʒ\~{\textturna}]}{(n.)}{1}{}{Parte de baixo da casa tradicional sobre estacas, utilizada para armazenar
materiais.}{}{}
\verb{basu-mon}{}{[ˈbasu ˈmõ]}{(n.)}{1}{}{Axila.}{}{}%
\verb{basu-mon}{}{[ˈbasu ˈmõ]}{(n.)}{2}{}{Sovaco.}{}{}
\verb{basu-son}{}{[ˈbasu ˈsõ]}{(n.)}{1}{}{Cave.}{}{}
\verb{basu-son}{}{[ˈbasu ˈsõ]}{(n.)}{1}{}{Subterrâneo.}{}{}
\verb{basu-wê}{}{[ˈbasu ˈwe]}{(n.)}{1}{}{Pálpebra.}{}{}%
\verb{batata}{}{[baˈtata]}{(n.)}{1}{}{Batata.}{}{}
\verb{batata-doxi}{}{[baˈtata ˈdɔʃi]}{(n.)}{1}{}{Batata-doce.}{\textit{\textbf{Ipomoea batatas}}.}{}
\verb{batata-pimpin}{}{[baˈtata pĩˈpĩ]}{(n.)}{1}{}{Batata pim-pim.}{\textit{\textbf{Peponium vogelii}.}}{}
\verb{batê}{}{[baˈte]}{(v.)}{1}{}{Abater.}{}{}
\verb{batê}{}{[baˈte]}{(v.)}{2}{}{Bater.}{}{}
\verb{batêmentu}{}{[bateˈmẽtu]}{(n.)}{1}{}{Abatimento.}{}{}
\verb{batêmentu}{}{[bateˈmẽtu]}{(n.)}{2}{}{Desconto.}{}{}
\verb{batêmentu}{}{[bateˈmẽtu]}{(n.)}{3}{}{Saldos.}{}{}
\verb{batê-mon}{}{[baˈte ˈmõ]}{(n.)}{1}{}{\textit{Batê-mon}, uma festa tradicional.}{}{}{}
\verb{batê-mon}{}{[baˈte ˈmõ]}{(n.)}{2}{}{Convívio.}{}{}
\verb{batu}{}{[baˈtu]}{(n.)}{1}{}{Jiló.}{Cf. \textbf{batwitwi}.}{}{}
\verb{batuti}{}{[batuˈti]}{(n.)}{1}{}{Jiló.}{Cf. \textbf{batwitwi}.}{}{}
\verb{batwitwi}{}{[batwiˈtwi]}{(n.)}{1}{}{Jiló.}{\textit{\textbf{Solanum aethiopicum}.}}{}
\verb{bau}{}{[baˈu]}{(n.)}{1}{}{Baú.}{}{}
\verb{baxa}{}{[baˈʃa]}{(n.)}{1}{}{Bacia.}{}{}%
\verb{baxa}{}{[baˈʃa]}{(n.)}{2}{}{Vasilha.}{}{}
\verb{baxa}{}{[baˈʃa]}{(v.)}{1}{}{Internar(-se) no hospital.}{}{}
\verb{baxale}{}{[baʃaˈlɛ]}{(n.)}{1}{}{Bacharel.}{}{}
\verb{baxa-xinadu}{}{[ˈbaʃa ʃiˈnadu]}{(n.)}{1}{}{Abaixo-assinado.}{}{}
\verb{baxa-zawa}{}{[baˈʃa ˈzawa]}{(n.)}{1}{}{Bacio.}{}{}
\verb{baxa-zawa}{}{[baˈʃa ˈzawa]}{(n.)}{2}{}{Urinol.}{}{}
\verb{baxin}{}{[baˈʃĩ]}{(n.)}{1}{}{Bacio.}{}{}%
\verb{baxin}{}{[baˈʃĩ]}{(n.)}{2}{}{Penico.}{}{}%
\verb{baxin}{}{[baˈʃĩ]}{(n.)}{3}{}{Prato tradicional de madeira.}{}{}
\verb{baxina}{}{[baˈʃina]}{(n.)}{1}{}{Vacina.}{}{}
\verb{baxina}{}{[baʃiˈna]}{(v.)}{1}{}{Vacinar.}{}{}
\verb{baxinadu}{}{[ˈbaʃiˈnadu]}{(adj.)}{1}{}{Vacinado.}{}{}
\verb{baxta}{}{[baʃˈta]}{(v.)}{1}{}{Bastar.}{}{}%
\verb{baxta}{}{[baʃˈta]}{(v.)}{2}{}{Importar.}{\textbf{Baxta n bê wê dê}.
\textit{O que importa é que vi os seus olhos}.}{}%
\verb{baxta pa}{}{[baʃˈta ˈpa]}{(conj.)}{1}{}{Conquanto que.}{}{}
\verb{baxta pa}{}{[baʃˈta ˈpa]}{(conj.)}{2}{}{Desde que.}{\textbf{Baxta pa n
bê wê dê}. \textit{Desde que eu veja os seus olhos}.}{}
\verb{baxtava}{}{[baʃˈtava]}{(v.)}{1}{}{Bastar.}{Cf. \textbf{baxta}.}{}{}
\verb{baya}{}{[baˈja]}{(n.)}{1}{}{Feitiçaria.}{}{}
\verb{baya}{}{[baˈja]}{(n.)}{2}{}{Feitiço.}{}{}%
\verb{baya}{}{[baˈja]}{(v.)}{1}{}{Enfeitiçar.}{}{}%
\verb{baya}{}{[baˈja]}{(v.)}{2}{}{Fazer feitiço(s) contra alguém.}{}{}
\verb{bayadu}{}{[baˈjadu]}{(adj.)}{1}{}{Enfeitiçado.}{}{}
\verb{bayle}{}{[ˈbajlɛ]}{(n.)}{1}{}{Baile.}{}{}%
\verb{bayle}{}{[ˈbajlɛ]}{(n.)}{2}{}{Dança.}{}{}
\verb{baza}{}{[baˈza]}{(v.)}{1}{}{Abaixar(-se).}{}{}{}
\verb{baza}{}{[baˈza]}{(v.)}{2}{}{Baixar.}{}{}{}
\verb{baza}{}{[baˈza]}{(v.)}{3}{}{Despejar.}{}{}%
\verb{baza}{}{[baˈza]}{(v.)}{4}{}{Diminuir.}{}{}
\verb{bazadu}{}{[baˈzadu]}{(adj.)}{1}{}{Abaixado.}{}{}
\verb{bazadu}{}{[baˈzadu]}{(adj.)}{1}{}{Baixado.}{}{}
\verb{bazadu}{}{[baˈzadu]}{(adj.)}{1}{}{Baixo.}{}{}
\verb{bazadu}{}{[baˈzadu]}{(adj.)}{2}{}{Curvado.}{}{}
\verb{bazadu}{}{[baˈzadu]}{(adj.)}{1}{}{De cócoras.}{}{}
\verb{bazadu}{}{[baˈzadu]}{(adj.)}{3}{}{Vergado.}{}{}
\verb{be}{}{[ˈbɛ]}{(prep. v.)}{1}{}{Para lá.}{\textbf{Bisu vwa be}. \textit{O
pássaro voou para lá}.}{}
\verb{be}{}{[ˈbɛ]}{(v.)}{1}{}{Ir (quando não é seguido do complemento
locativo).} {\textbf{Non be djandjan}. \textit{Fomos depressa}.}{}{}
\verb{bê}{}{[ˈbe]}{(v.)}{1}{}{Encontrar.}{}{}
\verb{bê}{}{[ˈbe]}{(v.)}{2}{}{Ver.}{}{}%
\verb{beba}{}{[ˈbɛba]}{(n.)}{1}{}{Barba.}{}{}%
\verb{beba}{}{[ˈbɛba]}{(n.)}{2}{}{Bigode.}{}{}
\verb{beba-blata}{}{[ˈbɛba
ˈblata]}{(n.)}{1}{}{Barba-de-barata.}{\textit{\textbf{Acacia
kamerunensis}}.}{}
\verb{beba-min}{}{[ˈbɛba ˈmĩ]}{(n.)}{1}{}{Barba do milho.}{}{}
\verb{beba-min}{}{[ˈbɛba ˈmĩ]}{(n.)}{2}{}{Cabelo do milho.}{}{}%
\verb{beba-min}{}{[ˈbɛba ˈmĩ]}{(n.)}{3}{}{Pelos púbicos da puberdade.}{}{}
\verb{bebason}{}{[ˈbɛbaˈsõ]}{(n.)}{1}{}{Menstruação.}{}{}%
\verb{bebason}{}{[ˈbɛbaˈsõ]}{(n.)}{2}{}{Período.}{}{}
\verb{bebe}{}{[bɛˈbɛ]}{(n.)}{1}{}{Bebê.}{}{}{}
\verb{bebe}{}{[bɛˈbɛ]}{(n.)}{2}{}{Criança.}{}{}{}
\verb{bebe}{}{[bɛˈbɛ]}{(n.)}{3}{}{Mudo.}{}{}{}
\verb{bêbê}{}{[beˈbe]}{(v.)}{1}{}{Beber.}{}{}
\verb{bêbê}{}{[ˈbebe]}{(n.)}{1}{}{Bebida alcoólica.}{}{}
\verb{bebeda}{}{[bɛbɛˈda]}{(n.)}{1}{}{Bêbado.}{}{}
\verb{bebeda}{}{[bɛbɛˈda]}{(n.)}{1}{}{Bebedeira.}{}{}
\verb{bebeda}{}{[bɛbɛˈda]}{(n.)}{2}{}{Bebida alcoólica.}{}{}
\verb{bebeda}{}{[bɛbɛˈda]}{(v.)}{1}{}{Embebedar.}{}{}
\verb{bebeda}{}{[bɛbɛˈda]}{(v.)}{1}{}{Estar bêbado.}{}{}
\verb{bebedadu}{}{[bɛbɛˈdadu]}{(adj.)}{1}{}{Bêbado.}{}{}%
\verb{bebedadu}{}{[bɛbɛˈdadu]}{(adj.)}{2}{}{Embriagado.}{}{}
\verb{bebedadu}{}{[bɛbɛˈdadu]}{(adj.)}{2}{}{Imbecil.}{}{}
\verb{bebedadu sete fôlô}{}{[bɛbɛˈdadu ˈsɛtɛ ˈfolo]}{(expr.)}{1}{}{Bebedíssimo.}{}{}
\verb{bebedadu sete fôlô}{}{[bɛbɛˈdadu ˈsɛtɛ ˈfolo]}{(expr.)}{2}{}{Embriagadíssimo.}{}{}
\verb{bêbêdô}{}{[bebeˈdo]}{(n.)}{1}{}{Bêbado.}{}{}
\verb{bêbêdô}{}{[bebeˈdo]}{(n.)}{2}{}{Bebedor.}{}{}%
\verb{bêbêka}{}{[bebeˈka]}{(n.)}{1}{}{\textit{Bebeka}.}{\textit{\textbf{Trachinotus ovatus}.}}{}
\verb{bêbidu}{}{[beˈbidu]}{(adj.)}{1}{}{Bêbado.}{}{}%
\verb{bêbidu}{}{[beˈbidu]}{(adj.)}{2}{}{Bebido.}{}{}%
\verb{bêbidu}{}{[beˈbidu]}{(adj.)}{3}{}{Molhado.}{}{}
\verb{bêbidu-awa}{}{[beˈbidu ˈawa]}{(adj.)}{1}{}{Envelhecido.}{}{}
\verb{bêdêsê}{}{[bedeˈse]}{(n.)}{1}{}{Obediência.}{}{}
\verb{bêdêsê}{}{[bedeˈse]}{(v.)}{1}{}{Obedecer.}{}{}
\verb{bêdêxidu}{}{[bedeˈʃidu]}{(adj.)}{1}{}{Obedecido.}{}{}
\verb{bêdêxidu}{}{[bedeˈʃidu]}{(adj.)}{1}{}{Obediente.}{}{}
\verb{bêdô}{}{[beˈdo]}{(n.)}{1}{}{Aquele que vê.}{}{}
\verb{bêdô}{}{[beˈdo]}{(n.)}{2}{}{\textit{Bêdô}.}{Almofadinha com remédio utilizada, depois de aquecida com uma pedra quente, para aperfeiçoar o formato da cabeça ou do nariz de um recém-nascido}{}{}{}
\verb{bêdô}{}{[beˈdo]}{(n.)}{3}{}{\textit{Bêdô}.}{Pedra utilizada no banho tradicional do recém-nascido para fortalecer o seu corpo ou cicatrizar o umbigo.}{}{}{}
\verb{bega}{}{[ˈbɛga]}{(n.)}{1}{}{Barriga.}{}{}%
\verb{bega}{}{[ˈbɛga]}{(n.)}{2}{}{Estômago.}{}{}%
\verb{bega}{}{[ˈbɛga]}{(n.)}{3}{}{Órgãos internos.}{}{}
\verb{begabega}{}{[ˈbɛgaˈbɛga]}{(adv.)}{1}{}{À beira de.}{}{}
\verb{begabega}{}{[ˈbɛgaˈbɛga]}{(adv.)}{1}{}{Nas imediações de.}{}{}
\verb{begabega}{}{[ˈbɛgaˈbɛga]}{(adv.)}{1}{}{Perto de.}{}{}
\verb{begabega}{}{[ˈbɛgaˈbɛga]}{(adv.)}{1}{}{Próximo de.}{}{}
\verb{begabega}{}{[ˈbɛgaˈbɛga]}{(adv.)}{2}{}{Quase.}{}{}
\verb{bega-d'ope}{}{[ˈbɛga dɔˈpɛ]}{(n.)}{1}{}{Barriga da perna.}{}{}%
\verb{bega-d'ope}{}{[ˈbɛga dɔˈpɛ]}{(n.)}{2}{}{Batata da perna.}{}{}
\verb{bega-d'ope}{}{[ˈbɛga dɔˈpɛ]}{(n.)}{3}{}{Panturrilha.}{}{}%
\verb{bega-kôlê}{}{[ˈbɛga koˈle]}{(n.)}{1}{}{Diarréia.}{}{}
\verb{bega-lanka}{}{[ˈbɛga l\~{\textturna}ˈka]}{(n.)}{1}{}{Aborto.}{}{}
\verb{bega-môvê}{}{[ˈbɛga moˈve]}{(n.)}{1}{}{Aborto.}{}{}
\verb{bega-ni-son}{}{[ˈbɛg\~{\textturna}ˈs\~ɔ]}{(n.)}{1}{}{Barriga-no-chão.}{\textbf{\textit{Achyrospermum oblongifolium}}.}{}
\verb{bega-pampôlê}{}{[bɛˈga p\~{\textturna}ˈpole]}{(n.)}{1}{}{Esquistossomose.}{}{}
\verb{bega-txintxin}{}{[bɛˈga tʃ\~iˈtʃ\~i]}{(n.)}{1}{}{Barrigão.}{}{}
\verb{beku}{}{[ˈbɛku]}{(n.)}{1}{}{Beco.}{}{}
\verb{bê ku pena d'ubwê}{}{[ˈbe ku ˈpɛna duˈbwe]}{(expr.)}{1}{}{Antever.}{}{}\verb{bê ku pena d'ubwê}{}{[ˈbe ku ˈpɛna duˈbwe]}{(expr.)}{2}{}{Intuir.}{}{}
\verb{bê ku pena d'ubwê}{}{[ˈbe ku ˈpɛna duˈbwe]}{(expr.)}{3}{}{Pressentir.}{}{}
\verb{bela}{}{[ˈbɛla]}{(n.)}{1}{}{Beira.}{}{}
\verb{belo-vlêmê}{}{[ˈbɛlɔ vleˈme]}{(n.)}{1}{}{Belo-vermelho.}{\textbf{\textit{Amaranthus graecizans.}}}{}
\verb{belo-vlêmê}{}{[ˈbɛlɔ vleˈme]}{(n.)}{2}{}{Caruru.}{\textbf{\textit{Amaranthus graecizans.}}}{}
\verb{bembêlumbê}{}{[b\~eˈbel\~uˈbe]}{(n.)}{1}{}{Papeira.}{}{}
\verb{ben}{}{[ˈbẽ]}{(adv.)}{1}{}{Bem.}{}{}
\verb{ben}{}{[ˈbẽ]}{(n.)}{1}{}{Bem.}{}{}%
\verb{ben}{}{[ˈbẽ]}{(n.)}{2}{}{Bens.}{}{}
\verb{bendê}{}{[bẽˈde]}{(v.)}{1}{}{Vender.}{}{}
\verb{bendêdô}{}{[bẽdeˈdo]}{(n.)}{1}{}{Vendedor.}{}{}
\verb{bendenxa}{}{[bẽˈdẽʃa]}{(n.)}{1}{}{Cavaqueira.}{}{}%
\verb{bendenxa}{}{[bẽˈdẽʃa]}{(n.)}{2}{}{Conversa sedutora.}{}{}%
\verb{bendenxa}{}{[bẽˈdẽʃa]}{(n.)}{3}{}{Sedução.}{}{}
\verb{bendê-panu}{}{[ˈbẽdeˈpanu]}{(n.)}{1}{}{Borboleta.}{}{}
\verb{benditu}{}{[bẽˈditu]}{(adj.)}{1}{}{Bendito.}{}{}
\verb{bene}{}{[bɛˈnɛ]}{(v.)}{1}{}{Pestanejar.}{}{}{}
\verb{benfebenfe}{}{[b\~ɛˈfɛb\~ɛˈfɛ]}{(id.)}{1}{}{Cf. \textbf{mlagu
benfebenfe.}}{}
\verb{benfetu}{}{[b\~ɛˈfɛtu]}{(adj.)}{1}{}{Bonito.}{}{}
\verb{benfetu}{}{[b\~ɛˈfɛtu]}{(adj.)}{2}{}{Estreito.}{}{}
\verb{benfetu}{}{[b\~ɛˈfɛtu]}{(adj.)}{3}{}{Magricela.}{}{}
\verb{benfetu}{}{[b\~ɛˈfɛtu]}{(adj.)}{4}{}{Magro.}{}{}
\verb{bengala}{}{[bẽˈgala]}{(n.)}{1}{}{Bengala.}{}{}
\verb{bengi}{}{[ˈbẽgi]}{(n.)}{1}{}{Bengue.}{\textit{\textbf{Alchornea cordifolia}.}}{}{}
\verb{bengi-doxi}{}{[ˈbẽgi ˈdɔʃi]}{(n.)}{1}{}{Bengue-doce.}{\textbf{Alchornea cordifolia.}}{}%
\verb{bengi-doxi}{}{[ˈbẽgi ˈdɔʃi]}{(n.)}{1}{}{Fruto do
\textbf{bengi-doxi}.}{}{}%
\verb{bengi-doxi}{}{[ˈbẽgi ˈdɔʃi]}{(n.)}{2}{}{Rola.}{}{}{}
%\verb{bengula}{}{[bẽˈgula]}{(n.)}{1}{}{Cf. \textbf{fe bengula}.}{}{}
\verb{benha}{}{[beˈɲa]}{(v.)}{2}{}{Embainhar.}{}{}
\verb{benjidu}{}{[bẽˈʒidu]}{(adj.)}{1}{}{Benzido.}{}{}
\verb{benku}{}{[bẽˈku]}{(n.)}{1}{}{Tartaruga do fango africana.}{\textit{\textbf{Pelusios castaneus}.}}{}
\verb{bensa}{}{[ˈbẽsa]}{(n.)}{1}{}{Benção.}{}{}
\verb{benson}{}{[b\~ɛˈs\~ɔ]}{(n.)}{1}{}{Benção.}{}{}{}
\verb{benvindu}{}{[b\~ɛˈv\~idu]}{(adj.)}{1}{}{Bem-vindo.}{}{}
\verb{bê nwa}{}{[beˈnwa]}{(expr.)}{1}{}{Estar menstruada.}{}
\verb{benza}{}{[b\~ɛˈza]}{(v.)}{1}{}{Beijar.}{}{}
\verb{benzê}{}{[bẽˈze]}{(v.)}{2}{}{Abençoar.}{}{}
\verb{benzê}{}{[bẽˈze]}{(v.)}{1}{}{Benzer.}{}{}
\verb{benzementu}{}{[bẽzeˈm\~etu]}{(n.)}{1}{}{Inauguração.}{}{}
\verb{besupla}{}{[ˈbɛsupla]}{(n.)}{1}{}{Véspera.}{}{}
\verb{bêtôdô}{}{[beˈtodo]}{(adj.)}{1}{}{Astucioso.}{}{}
\verb{bêtôdô}{}{[beˈtodo]}{(adj.)}{2}{}{Douto.}{}
\verb{bêtôdô}{}{[beˈtodo]}{(adj.)}{3}{}{Inteligente.}{}{}
\verb{bêtôdô}{}{[beˈtodo]}{(adj.)}{4}{}{Sábio.}{}{}
\verb{bêtôdô}{}{[beˈtodo]}{(n.)}{2}{}{\emph{Bêtôdô}.}{Personagem sábia de narrativas tradicionais.}{}
\verb{betu}{}{[ˈbɛtu]}{(adj.)}{1}{}{Aberto.}{}{}
%\verb{betu}{}{[ˈbɛtu]}{(n.)}{1}{}{Alberto.}{}{}
\verb{betu blalala}{}{[ˈbɛtu blalaˈla]}{(expr.)}{1}{}{Escancarado.}{}
\verb{betumi}{}{[bɛˈtumi]}{(n.)}{1}{}{Betume.}{}{}
\verb{betu wan}{}{[ˈbɛtu ˈ\~w\~{\textturna}]}{(expr.)}{1}{}{Abertíssimo.}{}{}\verb{betu wan}{}{[ˈbɛtu ˈ\~w\~{\textturna}]}{(expr.)}{1}{}{Escancarado.}{}{}\verb{bêtwa}{}{[beˈtwa]}{(v.)}{1}{}{Arriscar.}{}{}
\verb{bêtwa}{}{[beˈtwa]}{(v.)}{2}{}{Melindrar.}{}{}
\verb{bêtwa}{}{[beˈtwa]}{(v.)}{3}{}{Provocar.}{}{}
\verb{bêtwadu}{}{[beˈtwadu]}{(adj.)}{1}{}{Atrevido.}{}{}%
\verb{bêtwadu}{}{[beˈtwadu]}{(adj.)}{2}{}{Impertinente.}{}{}%
\verb{bexpla}{}{[ˈbɛʃpla]}{(n.)}{1}{}{Véspera.}{Cf. \textbf{besupla}.}{}{}
\verb{bêzubêzu}{}{[ˈbezu ˈbezu]}{(n.)}{1}{}{Queixo.}{}{}
\verb{bi}{}{[ˈbi]}{(prep. v.)}{1}{}{Para cá.}{\textbf{Sangê kôlê bi}. \textit{A senhora correu para cá}.}{}{}
\verb{bi}{}{[ˈbi]}{(v.)}{1}{}{Vir.}{}{}{}
\verb{bibilha}{}{[ˈbibiʎa]}{(n.)}{1}{}{Bíblia.}{}{}
\verb{bibilhoteka}{}{[bibiʎɔˈtɛka]}{(n.)}{1}{}{Biblioteca.}{}{}
\verb{bifi}{}{[ˈbifi]}{(n.)}{1}{}{Bife.}{}{}
\verb{bi fô}{}{[ˈbi fo]}{(expr.)}{1}{}{Sair de.}{}{}{}
\verb{bi fô}{}{[ˈbi fo]}{(expr.)}{2}{}{Vir de.}{}{}{}
\verb{bigodji}{}{[biˈgɔdʒi]}{(n.)}{1}{}{Bigode.}{}{}
\verb{bika}{}{[ˈbika]}{(n.)}{2}{}{Bica.}{\textbf{\textit{Lethrinus atlanticus}}.}{}
\verb{bika}{}{[ˈbika]}{(n.)}{1}{}{Bica d'água.}{}{}%
\verb{bila}{}{[biˈla]}{(n.)}{1}{}{Túmulo pagão.}{Cf. \textbf{mbila}.}{}{}
\verb{bila}{}{[biˈla]}{(v.)}{1}{}{De novo.}{}{}
\verb{bila}{}{[biˈla]}{(v.)}{2}{}{Novamente.}{}{}
\verb{bila}{}{[biˈla]}{(v.)}{2}{}{Regressar.}{}{}
\verb{bila}{}{[biˈla]}{(v.)}{1}{}{Tornar a.}{}
\verb{bila}{}{[biˈla]}{(v.)}{2}{}{Virar.}{}{}
\verb{bila}{}{[biˈla]}{(v.)}{3}{}{Voltar.}{}{}
\verb{bila bi}{}{[biˈla ˈbi]}{(expr.)}{1}{}{Regressar.}{}{}
\verb{bilabila}{}{[biˈlabiˈla]}{(v.)}{1}{}{Voltar para trás.}{}{}
\verb{biladu}{}{[biˈladu]}{(adj.)}{2}{}{Tornado.}{}{}
\verb{biladu}{}{[biˈladu]}{(adj.)}{1}{}{Transformado.}{}{}
\verb{biladu}{}{[biˈladu]}{(adj.)}{3}{}{Virado.}{}{}
\verb{bila kanfini}{}{[biˈla k\~{\textturna}fiˈni]}{(expr.)}{1}{}{Dar cambalhotas.}{}{}
\verb{bila kanfini}{}{[biˈla k\~{\textturna}fiˈni]}{(expr.)}{1}{}{Fazer a ponta-cabeça.}{}{}
\verb{bila kanfini}{}{[biˈla k\~{\textturna}fiˈni]}{(expr.)}{1}{}{Fazer o pino.}{}{}
\verb{bila kanfini}{}{[biˈla k\~{\textturna}fiˈni]}{(expr.)}{1}{}{Ficar de pernas para o ar.}{}{}
\verb{bila kanfini}{}{[biˈla k\~{\textturna}fiˈni]}{(expr.)}{1}{}{Virar de cabeça para baixo.}{}{}
\verb{bila klongondo}{}{[biˈla kl\~ɔg\~ɔˈdɔ]}{(expr.)}{1}{}{Ficar atrofiado.}{}{}
\verb{bila ngombe}{}{[biˈla ngõˈbɛ]}{(expr.)}{1}{}{Dar cambalhotas.}{}{}
\verb{bila ngombe}{}{[biˈla ngõˈbɛ]}{(expr.)}{1}{}{Rebolar.}{}{}
\verb{bilangwa}{}{[bil\~{\textturna}ˈgwa]}{(n.)}{1}{Socopé.}{Cf. \textbf{sokope}.}{}{}{}
\verb{bilêtê}{}{[biˈlete]}{(n.)}{1}{}{Bilhete.}{}{}
\verb{bili}{}{[biˈli]}{(v.)}{1}{}{Abrir.}{}{}%
\verb{bili}{}{[biˈli]}{(v.)}{2}{}{Abrir-se com.}{}{}%
\verb{bili}{}{[biˈli]}{(v.)}{5}{}{Começar.}{}{}
\verb{bili}{}{[biˈli]}{(v.)}{3}{}{Comunicar.}{}{}%
\verb{bili}{}{[biˈli]}{(v.)}{3}{}{Divulgar.}{}{}%
\verb{bilibili}{}{[biˈlibiˈli]}{(n.)}{1}{}{Pé-de-atleta.}{}{}
\verb{bilidu}{}{[biˈlidu]}{(adj.)}{1}{}{Aberto.}{}{}
\verb{bili funda}{}{[biˈli ˈf\~uda]}{(expr.)}{1}{}{Denunciar.}{}{}
\verb{bili funda}{}{[biˈli ˈf\~uda]}{(expr.)}{1}{}{Desvendar um segredo.}{}{}\verb{bili funda}{}{[biˈli ˈf\~uda]}{(expr.)}{1}{}{Expor.}{}{}
\verb{bili klan}{}{[biˈli ˈkl\~{\textturna}]}{(expr.)}{1}{}{Escancarar portas ou janelas.}{}{}
\verb{bili mina}{}{[biˈli ˈmina]}{(expr.)}{1}{}{Verificar a virgindade.}{}{}
\verb{bili mon}{}{[biˈli ˈm\~ɔ]}{(expr.)}{1}{}{Começar.}{}{}
\verb{bili mon}{}{[biˈli ˈm\~ɔ]}{(expr.)}{1}{}{Permitir.}{}{}
\verb{bimba}{}{[bĩˈba]}{(n.)}{1}{}{\textit{Bimba}.}{Instrumento musical tradicional.}{}{}
\verb{binku}{}{[ˈbĩku]}{(n.)}{1}{}{Umbigo.}{}{}
\verb{binzela}{}{[bĩˈzɛla]}{(n.)}{1}{}{Beringela
branca.}{\textbf{\textit{Solanum ovigerum.}}}{}%
\verb{binzela}{}{[bĩˈzɛla]}{(n.)}{2}{}{Beringela
roxa.}{\textbf{\textit{Solanum melongena}}.}{}
\verb{bisku}{}{[ˈbisku]}{(n.)}{1}{}{Visgo.}{}{}{}
\verb{bisu}{}{[ˈbisu]}{(n.)}{2}{}{Animal.}{}{}%
\verb{bisu}{}{[ˈbisu]}{(n.)}{3}{}{Ave.}{}{}
\verb{bisu}{}{[ˈbisu]}{(n.)}{1}{}{Bicho.}{}{}%
\verb{bisu}{}{[ˈbisu]}{(n.)}{4}{}{Pássaro.}{}{}
\verb{bisu-d'aza}{}{[ˈbisu ˈdaza]}{(n.)}{1}{}{Pássaro.}{}{}
\verb{bisu-witxi}{}{[ˈbisu ˈwitʃi]}{(n.)}{1}{}{Infestação de ácaros que ataca
determinadas plantas, como o tomateiro, o limoeiro e o mamoeiro, deixando-os
brancos.}{}{}
\verb{bisu-witxi}{}{[ˈbisu ˈwitʃi]}{(n.)}{2}{}{Intriguista.}{}{}
\verb{bisu-witxi}{}{[ˈbisu ˈwitʃi]}{(n.)}{3}{}{Pérfido.}{}{}
\verb{bisu-witxi}{}{[ˈbisu ˈwitʃi]}{(n.)}{4}{}{Prejudicador.}{}{}
\verb{bisu-witxi}{}{[ˈbisu ˈwitʃi]}{(n.)}{5}{}{Traidor.}{}{}
\verb{bixa}{}{[ˈbiʃa]}{(n.)}{1}{}{Fila.}{}{}%
\verb{bixbile}{}{[biʃbiˈlɛ]}{(n.)}{1}{}{Confusão.}{}{}%
\verb{bixbile}{}{[biʃbiˈlɛ]}{(n.)}{2}{}{Desentendimento.}{}{}
\verb{bixdona}{}{[biʃˈdɔna]}{(n.)}{1}{}{Bisavó.}{}{}
\verb{bixdonu}{}{[biʃˈdɔnu]}{(n.)}{1}{}{Bisavô.}{}{}
\verb{bixi}{}{[biˈʃi]}{(v.)}{1}{}{Vestir(-se).}{}{}
\verb{bixi fyefyefye}{}{[biˈʃi fjɛfjɛˈfjɛ]}{(expr.)}{1}{}{Vestir
elegantemente.}{}{}
\verb{bixiga}{}{[biˈʃiga]}{(n.)}{1}{}{Varíola.}{}{}
\verb{bixidu}{}{[biˈʃidu]}{n.)}{1}{}{Vestido.}{}{}
\verb{bixka}{}{[ˈbiʃka]}{(n.)}{1}{}{Bisca.}{}{}{}
\verb{bixketa}{}{[biʃˈkɛta]}{(n.)}{1}{}{Bicicleta.}{}{}
\verb{bixkôkô}{}{[biʃˈkoko]}{(id.)}{1}{}{Cf. \textbf{djina bixkôkô}.}{}{}
\verb{bixku}{}{[ˈbiʃku]}{(n.)}{1}{}{Visgo.}{Cf. \textbf{bisku}.}{}{}{}
\verb{bixneta}{}{[biʃˈnɛta]}{(n.)}{1}{}{Bisneta.}{}{}
\verb{bixnetu}{}{[biʃˈnɛtu]}{(n.)}{1}{}{Bisneto.}{}{}
\verb{bixô}{}{[biˈʃo]}{(n.)}{1}{}{Bicho-do-pé.}{\textit{\textbf{Tunga
penetrans}}.}{}%
\verb{bixô}{}{[biˈʃo]}{(n.)}{2}{}{Nígua.}{\textit{\textbf{Tunga
penetrans}}.}{}%
\verb{bixpu}{}{[ˈbiʃpu]}{(n.)}{1}{}{Bispo.}{}{}
\verb{bizôli}{}{[bizoˈli]}{(n.)}{2}{}{Criança.}{}{}
\verb{bizôli}{}{[bizoˈli]}{(n.)}{1}{}{Miúdo.}{}{}%
\verb{blabadu}{}{[blaˈbadu]}{(n.)}{1}{}{Barbudo.}{}{}%
\verb{blabadu}{}{[blaˈbadu]}{(n.)}{2}{}{Embarbado.}{}{}
\verb{blabadu}{}{[blaˈbadu]}{(n.)}{2}{}{Patilhas.}{}{}
\verb{blabadu}{}{[blaˈbadu]}{(n.)}{2}{}{Suiças.}{}{}
\verb{blabêlu}{}{[blaˈbelu]}{(n.)}{1}{}{Barbeiro.}{}{}
\verb{blabi}{}{[blaˈbi]}{(n.)}{1}{}{Varizes.}{}{}
\verb{blabla}{}{[blaˈbla]}{(n.)}{1}{}{\textit{Blabla}.}{Prato típico de São
Tomé parecido com \textbf{kalu}.}{}{}
\verb{blaboza}{}{[blaˈbɔza]}{(n.)}{1}{}{Aloés.}{\textit{\textbf{Aloe humilis}.}}{}
\verb{blaboza-motxi}{}{[blaˈbɔza ˈmɔtʃi]}{(n.)}{1}{}{Baba derradeira do
moribundo.}{}{}
\verb{blabu}{}{[ˈblabu]}{(adj.)}{1}{}{Astuto.}{}{}%
\verb{blabu}{}{[ˈblabu]}{(adj.)}{2}{}{Arguto.}{}{}%
\verb{blabu}{}{[ˈblabu]}{(adj.)}{2}{}{Esperto.}{}{}%
\verb{blabu}{}{[ˈblabu]}{(adj.)}{3}{}{Perspicaz.}{}{}%
\verb{blabu}{}{[ˈblabu]}{(adj.)}{4}{}{Precavido.}{}{}
\verb{blabudu}{}{[blaˈbudu]}{(n.)}{1}{}{Barbudo.}{\textit{\textbf{Galeoides
polydactylus}.}}{}
\verb{blada}{}{[blaˈda]}{(v.)}{1}{}{Bradar.}{}{}%
\verb{blada}{}{[blaˈda]}{(v.)}{2}{}{Gritar.}{}{}
\verb{bladu}{}{[ˈbladu]}{(n.)}{1}{}{Anúncio.}{Cf. \textbf{blandu}.}{}{}
\verb{blaga}{}{[blaˈga]}{(v.)}{1}{}{Arregalar.}{}{}
\verb{blaga}{}{[blaˈga]}{(v.)}{2}{}{Demolir.}{}{}%
\verb{blaga}{}{[blaˈga]}{(v.)}{3}{}{Desfazer.}{}{}%
\verb{blaga}{}{[blaˈga]}{(v.)}{4}{}{Desmanchar.}{}{}%
\verb{blaga}{}{[blaˈga]}{(v.)}{5}{}{Desprender.}{}{}%
\verb{blaga}{}{[blaˈga]}{(v.)}{6}{}{Destruir.}{}{}%
\verb{blaga}{}{[blaˈga]}{(v.)}{7}{}{Resolver.}{}{}%
\verb{blaga}{}{[blaˈga]}{(v.)}{8}{}{Terminar.}{}{}%
\verb{blaga awa-wê}{}{[blaˈga ˈawa ˈwê]}{(expr.)}{1}{}{Derramar lágrimas.}{}{}
\verb{blaga awa-wê plaplapla}{}{[blaˈga ˈawa ˈwê plaplaˈpla]}{(expr.)}{1}{}{Chorar desalmadamente.}{}{}
\verb{blagadu}{}{[blaˈgadu]}{(adj.)}{3}{}{Arregalado.}{}{}
\verb{blagadu}{}{[blaˈgadu]}{(adj.)}{1}{}{Desfeito.}{}{}%
\verb{blagadu}{}{[blaˈgadu]}{(adj.)}{2}{}{Terminado.}{}{}%
\verb{blaga mpenampena}{}{[blaˈga mpɛˈn\~apɛˈna]}{(expr.)}{1}{}{Demolir por completo.}{}{}
\verb{blaga mpenampena}{}{[blaˈga mpɛˈn\~apɛˈna]}{(expr.)}{2}{}{Desfazer por completo.}{}{}
\verb{blaga mpenampena}{}{[blaˈga mpɛˈn\~apɛˈna]}{(expr.)}{3}{}{Desmanchar por completo.}{}{}
\verb{blaga mpenampena}{}{[blaˈga mpɛˈn\~apɛˈna]}{(expr.)}{4}{}{Desmontar por completo.}{}{}
\verb{blaga-ubwa}{}{[blaˈgubwa]}{(n.)}{1}{}{Insignificante.}{}{}
\verb{blaga-ubwa}{}{[blaˈgubwa]}{(n.)}{1}{}{Pessoa espalhafatosa.}{}{}
\verb{blaga-ubwa}{}{[blaˈgubwa]}{(n.)}{1}{}{Pobre.}{}{}
\verb{blaga-ubwa}{}{[blaˈgubwa]}{(n.)}{1}{}{Pobretanas.}{}{}
\verb{blaga xtlôlô}{}{[blaˈga ˈʃtlolo]}{(expr.)}{1}{}{Transpirar.}{}
\verb{blagiya}{}{[blaˈgija]}{(n.)}{1}{}{Braguilha.}{}{}
\verb{blagiya}{}{[blaˈgija]}{(n.)}{2}{}{Vagina.}{}{}
\verb{blagiya}{}{[blaˈgija]}{(n.)}{3}{}{Virilha.}{}{}%
\verb{blaji}{}{[blaˈʒi]}{(top.)}{1}{}{Brasil.}{}{}
\verb{blaji}{}{[blaˈʒi]}{(n.)}{1}{}{Brasileiro.}{}{}
\verb{blakin}{}{[blaˈkĩ]}{(n.)}{1}{}{Canoa grande.}{}{}
\verb{blaku}{}{[ˈblaku]}{(n.)}{1}{}{Buraco.}{}{}%
\verb{blaku}{}{[ˈblaku]}{(n.)}{2}{}{Poço.}{}{}
\verb{blalala}{}{[blalaˈla]}{(id.)}{1}{}{Cf. \textbf{betu blalala.}}{}
\verb{blalala}{}{[ˈblalala]}{(n.)}{1}{}{Trovão.}{}{}
\verb{blanda}{}{[bl\~{\textturna}ˈda]}{(v.)}{1}{}{Abrandar.}{}{}
\verb{blanda}{}{[bl\~{\textturna}ˈda]}{(v.)}{2}{}{Afrouxar.}{}{}
\verb{blandu}{}{[ˈbl\~{\textturna}du]}{(n.)}{1}{}{Anúncio.}{}{}
\verb{blandu}{}{[ˈbl\~{\textturna}du]}{(n.)}{2}{}{Notícia.}{}{}
\verb{blandu}{}{[ˈbl\~{\textturna}du]}{(n.)}{3}{}{Sinal.}{}{}
\verb{blanku}{}{[ˈbl\~{\textturna}ku]}{(adj.)}{1}{}{Branco.}{}{}
\verb{blanku}{}{[ˈbl\~{\textturna}ku]}{(n.)}{1}{}{Branco.}{}{}
\verb{blanku fenene}{}{[ˈbl\~{\textturna}ku
fɛnɛˈnɛ]}{(expr.)}{1}{}{Branquíssimo.}{}{}
\verb{blasa}{}{[blaˈsa]}{(v.)}{1}{}{Abraçar.}{}{}
\verb{blasa}{}{[ˈblasa]}{(n.)}{1}{Braça.}{}{}{}
\verb{blasu}{}{[ˈblasu]}{(n.)}{1}{}{Abraço.}{}{}
\verb{blasu}{}{[ˈblasu]}{(n.)}{2}{}{Braço.}{}{}%
\verb{blasu-d'alê}{}{[ˈblasu daˈle]}{(n.)}{1}{}{Agente da autoridade.}{}{}
\verb{blasu-d'alê}{}{[ˈblasu daˈle]}{(n.)}{1}{}{Policial.}{}{}
\verb{blata}{}{[ˈblata]}{(n.)}{1}{}{Barata.}{}{}
\verb{blatu}{}{[ˈblatu]}{(adj.)}{1}{}{Barato.}{}{}
\verb{blaza}{}{[ˈblaza]}{(n.)}{1}{}{Brasa.}{}{}
\verb{ble}{}{[ˈblɛ]}{(n.)}{1}{}{Lugar remoto.}{}{}
\verb{ble-d’omali}{}{[ˈblɛ dɔˈmali]}{(n.)}{1}{}{Alto mar.}{}{}
\verb{blêgê}{}{[bleˈge]}{(v.)}{3}{}{Moer.}{}{}%
\verb{blêgê}{}{[bleˈge]}{(v.)}{4}{}{Pilar \textbf{izakentxi}.}{}{}
\verb{blêgidu}{}{[bleˈgidu]}{(adj.)}{1}{}{Moído.}{}{}
\verb{blêgidu}{}{[bleˈgidu]}{(adj.)}{1}{}{Pilado.}{}{}
\verb{bliga}{}{[bliˈga]}{(n.)}{1}{}{Jogo de cacete.}{}{}%
\verb{bliga}{}{[bliˈga]}{(n.)}{2}{}{Luta.}{}{}
\verb{bligadô}{}{[bligaˈdo]}{(n.)}{1}{}{Praticante de \textbf{bliga}.}{}{}
\verb{bligason}{}{[bligaˈsõ]}{(n.)}{1}{}{Obrigação.}{}{}
\verb{bligi}{}{[ˈbligi]}{(n.)}{1}{}{Brigue.}{}{}
\verb{bligidi}{}{[bligiˈdi]}{(id.)}{1}{}{Cf. \textbf{ba bligidi.}}{}
\verb{blôa}{}{[ˈbloa]}{(n.)}{1}{}{Broa.}{}{}
\verb{blôa}{}{[ˈbloa]}{(n.)}{2}{}{Pão-de-milho.}{}{}
\verb{blôgôdô}{}{[blogoˈdo]}{(n.)}{1}{}{Precipício.}{}{}%
\verb{bloka}{}{[blɔˈka]}{(v.)}{1}{}{Despejar.}{}{}
\verb{bloka}{}{[blɔˈka]}{(v.)}{1}{}{Entornar.}{}{}
\verb{bloka}{}{[blɔˈka]}{(v.)}{1}{}{Ir-se embora.}{}{}
\verb{bloka}{}{[blɔˈka]}{(v.)}{1}{}{Transbordar.}{}{}
\verb{bloka}{}{[blɔˈka]}{(v.)}{1}{}{Verter.}{}{}
\verb{bloka}{}{[blɔˈka]}{(v.)}{1}{}{Virar (algo).}{}{}
\verb{bloka vungu}{}{[blɔˈka ˈv\~ugu]}{(expr.)}{1}{}{Soltar a voz.}{}{}
\verb{bloki}{}{[ˈblɔki]}{(n.)}{1}{}{Bloco.}{}{}
\verb{blokon}{}{[blɔˈkõ]}{(n.)}{2}{}{Escuridão.}{}{}
\verb{blokon}{}{[blɔˈkõ]}{(n.)}{1}{}{Penumbra.}{}{}
\verb{blonji}{}{[ˈblõʒi]}{(adj.)}{1}{}{Feio.}{}{}
\verb{blonji}{}{[ˈblõʒi]}{(n.)}{1}{}{Bronze.}{}{}
\verb{blonji}{}{[ˈblõʒi]}{(n.)}{2}{}{Roupa feia.}{}{}
\verb{blonji}{}{[blõˈʒi]}{(v.)}{1}{}{Ficar peludo.}{}{}
\verb{blôsê}{}{[bloˈse]}{(v.)}{1}{}{Aborrecer(-se).}{}{}
\verb{blôxidu}{}{[bloˈʃidu]}{(adj.)}{1}{}{Aborrecido.}{}{}
\verb{blu}{}{[ˈblu]}{(conj.)}{2}{}{Ora\ldots{}ora.}{\textbf{Blu ê sa ben, blu
ê sa mali!} \textit{Ora ele está bem, ora está mal!}.}{}
\verb{blu}{}{[ˈblu]}{(conj.)}{1}{}{Quer\ldots{}quer.}{\textbf{Blu \ldots{}
blu} \textit{quer\ldots{}quer}.}{}%
\verb{blublu}{}{[bluˈblu]}{(n.)}{2}{}{Precipitação.}{}{}%
\verb{blublu}{}{[bluˈblu]}{(n.)}{3}{}{Pressa.}{}{}
\verb{blublublu}{}{[blubluˈblu]}{(n.)}{1}{Pressa.}{}{}
\verb{blublublu}{}{[blubluˈblu]}{(n.)}{2}{Precipitação.}{}{}
\verb{blublublu}{}{[blubluˈblu]}{(id.)}{1}{}{Cf. \textbf{flêbê blublublu.}}{}\verb{bluga}{}{[bluˈga]}{(v.)}{1}{}{Descascar.}{}{}%
\verb{bluga}{}{[bluˈga]}{(v.)}{2}{}{Esburgar.}{}{}%
\verb{bluga}{}{[bluˈga]}{(v.)}{3}{}{Esfolar.}{}{}%
\verb{bluga}{}{[bluˈga]}{(v.)}{4}{}{Repuxar o prepúcio.}{}{}
\verb{blugadu}{}{[bluˈgadu]}{(adj.)}{1}{}{Descascado.}{}{}
\verb{blugu}{}{[bluˈgu]}{(v.)}{1}{}{Resvalar.}{}{}%
\verb{blugu}{}{[bluˈgu]}{(v.)}{2}{}{Tropeçar.}{}{}
\verb{bluguna}{}{[bluguˈna]}{(v.)}{1}{}{Cair.}{}{}%
\verb{bluguna}{}{[bluguˈna]}{(v.)}{4}{}{Demolir.}{}{}
\verb{bluguna}{}{[bluguˈna]}{(v.)}{4}{}{Desmoronar.}{}{}
\verb{bluguna}{}{[bluguˈna]}{(v.)}{3}{}{Fazer cair.}{}{}%
\verb{bluguna}{}{[bluguˈna]}{(v.)}{2}{}{Tombar.}{}{}%
\verb{bluku}{}{[ˈbluku]}{(adj.)}{1}{}{Cruel.}{}{}
\verb{bluku}{}{[ˈbluku]}{(adj.)}{2}{}{Feroz.}{}{}
\verb{bluku}{}{[ˈbluku]}{(adj.)}{3}{}{Inclemente.}{}{}
\verb{bluku}{}{[ˈbluku]}{(adj.)}{4}{}{Malvado.}{}{}
\verb{bluku}{}{[ˈbluku]}{(adj.)}{5}{}{Mau.}{}{}
\verb{bluku}{}{[ˈbluku]}{(adj.)}{6}{}{Terrível.}{}{}
\verb{bluma}{}{[ˈbluma]}{(n.)}{1}{}{Crosta.}{}{}
\verb{bluma}{}{[ˈbluma]}{(n.)}{2}{}{Sujidade.}{}{}
\verb{blutu}{}{[ˈblutu]}{(adj.)}{1}{}{Bruto.}{}{}
\verb{blutu}{}{[ˈblutu]}{(adj.)}{2}{}{Ignorante.}{}{}
\verb{bluxa}{}{[ˈbluʃa]}{(n.)}{1}{}{Bruxa.}{}{}%
\verb{bluxa}{}{[ˈbluʃa]}{(n.)}{2}{}{Feiticeira.}{}{}
\verb{bluxa}{}{[bluˈʃa]}{(v.)}{1}{}{Aborrecer}{Cf. \textbf{bluxa kala}.}{}
\verb{bluxadu}{}{[bluˈʃadu]}{(adj.)}{1}{}{Aborrecido.}{}{}
\verb{bluxa kala}{}{[bluˈʃa kaˈla]}{(expr.)}{1}{}{Estar de mau humor.}{}{}
\verb{bluxa kala}{}{[bluˈʃa kaˈla]}{(expr.)}{1}{}{Ficar carrancudo.}{}{}
\verb{bluxa kala}{}{[bluˈʃa kaˈla]}{(expr.)}{2}{}{Ficar de cara fechada.}{}{}\verb{bluxa kala}{}{[bluˈʃa kaˈla]}{(expr.)}{3}{}{Perder a graça.}{}{}
\verb{bô}{}{[ˈbo]}{(int.)}{1}{}{Onde?}{\textbf{Êlê bô}? \textit{Onde está
ele}?}{}
\verb{bô}{}{[ˈbo]}{(poss.)}{1}{}{Teu.}{}{}
\verb{bô}{}{[ˈbo]}{(poss.)}{2}{}{Teus.}{}{}
\verb{bô}{}{[ˈbo]}{(poss.)}{3}{}{Tua.}{}{}
\verb{bô}{}{[ˈbo]}{(poss.)}{4}{}{Tuas.}{}{}
\verb{bô}{}{[ˈbo]}{(pron.)}{1}{}{-te.}{}{}
\verb{bô}{}{[ˈbo]}{(pron.)}{2}{}{Tu.}{}{}
\verb{bô}{}{[ˈbo]}{(pron.)}{3}{}{Você.}{}{}
\verb{bôba}{}{[ˈboba]}{(n.)}{1}{}{Begônia.}{Cf.
\textbf{fya-bôba-blanku}.}{}{}
\verb{bôba}{}{[ˈboba]}{(n.)}{1}{Doença de pele.}{}{}{}
\verb{bobla}{}{[ˈbɔbla]}{(n.)}{1}{}{Abóbora.}{}{}
\verb{bobla}{}{[ˈbɔbla]}{(n.)}{2}{}{Aboboreira.}{}{}
\verb{bobla}{}{[bɔˈbla]}{(v.)}{1}{}{Inflamar.}{}{}%
\verb{bôbô}{}{[ˈbobo]}{(n.)}{1}{}{Bobo.}{}{}
\verb{bôbô}{}{[ˈbobo]}{(n.)}{2}{}{Estúpido.}{}{}
\verb{bôbô}{}{[ˈbobo]}{(n.)}{3}{}{Idiota.}{}{}%
\verb{bôbô}{}{[ˈbobo]}{(n.)}{4}{}{Parvo.}{}{}%
\verb{bôbô}{}{[boˈbo]}{(adj.)}{1}{}{Amarelo.}{}{}%
\verb{bôbô}{}{[boˈbo]}{(adj.)}{2}{}{Castanho-claro.}{}{}%
\verb{bôbô}{}{[boˈbo]}{(adj.)}{3}{}{Claro.}{}{}
\verb{bôbô}{}{[boˈbo]}{(adj.)}{4}{}{Maduro.}{}{}
\verb{bôbô}{}{[boˈbo]}{(n.)}{1}{}{Banana madura.}{}{}%
\verb{bôbô}{}{[boˈbo]}{(n.)}{2}{}{Mulato.}{}{}
\verb{bôbôbôbô}{}{[ˈboboˈbobo]}{(n.)}{1}{}{\textit{Bôbôbôbô}.}{\textit{\textbf{Casearia barteri}}.}{}
\verb{bôbô-dansu}{}{[boˈbo ˈd\~{\textturna}su]}{(n.)}{1}{}{\textit{Bôbô-dansu}.}{Personagem ridículo do \textbf{dansu-kongô}.}{}{}
\verb{bôbôdu}{}{[boˈbodu]}{(adj.)}{2}{}{Amadurecido.}{}{}
\verb{bôbôdu}{}{[boˈbodu]}{(adj.)}{1}{}{Maduro.}{}{}%
\verb{bôbô-fitu}{}{[boˈbo ˈfitu]}{(n.)}{1}{}{Doce de banana da Ilha do Príncipe.}{}{}
\verb{bôbô-kema}{}{[boˈbo ˈkɛma]}{(n.)}{1}{}{Garoupa de pintas.}{\textit{\textbf{Cephalopholis taeniops}}.}{}
\verb{bôbô lalala}{}{[boˈbo lalaˈla]}{(expr.)}{1}{}{Pessoa negra de cor demasiadamente clara.}{}{}
\verb{bôbô mela}{}{[boˈbo mɛˈla]}{(expr.)}{1}{}{Maduro demais.}{}{}
\verb{bôbô mina}{}{[boˈbo ˈmina]}{(expr.)}{1}{}{Transportar uma criança às costas.}{}{}
\verb{bôbô-mina}{}{[boˈbo ˈmina]}{(n.)}{1}{}{Trempe do \textbf{makuku}.}{}
\verb{boboyoko}{}{[bɔbɔˈjɔkɔ]}{(n.)}{1}{}{Palerma.}{}{}
\verb{boboyoko}{}{[bɔbɔˈjɔkɔ]}{(n.)}{2}{}{Parvo.}{}{}%
\verb{boboyoko}{}{[bɔbɔˈjɔkɔ]}{(n.)}{3}{}{Pateta.}{}{}
\verb{boboyoko}{}{[bɔbɔˈjɔkɔ]}{(n.)}{4}{}{Tolo.}{}{}%
\verb{bodji}{}{[ˈbɔdʒi]}{(n.)}{1}{}{Bode.}{}{}
\verb{bodla}{}{[bɔˈdla]}{(v.)}{1}{}{Bordar.}{}{}
\verb{bodo}{}{[bɔˈdɔ]}{(n.)}{1}{}{Cais.}{}{}
\verb{bodo}{}{[ˈbɔdɔ]}{(n.)}{1}{}{Beira.}{}{}
\verb{bodo}{}{[ˈbɔdɔ]}{(n.)}{2}{}{Borda.}{}{}%
\verb{bodo}{}{[ˈbɔdɔ]}{(n.)}{3}{}{Bordo.}{}{}%
\verb{bodo}{}{[ˈbɔdɔ]}{(n.)}{4}{}{Canto.}{}{}%
\verb{bodo}{}{[ˈbɔdɔ]}{(n.)}{5}{}{Lado.}{}{}
\verb{bodo}{}{[ˈbɔdɔ]}{(prep. n.)}{1}{}{Ao lado de.}{}{}
\verb{bodobodo}{}{[bɔˈdɔbɔˈdɔ]}{(adj.)}{1}{}{Roliça.}{}
\verb{bodobodo}{}{[bɔˈdɔbɔˈdɔ]}{(adj.)}{1}{}{Robusta.}{}
\verb{bodobodo}{}{[bɔˈdɔbɔˈdɔ]}{(id.)}{1}{}{Cf. \textbf{mina bodobodo.}}{}
\verb{bodobodo}{}{[bɔˈdɔbɔˈdɔ]}{(n.)}{1}{}{Folha-porco.}{\textbf{\textit{Commelina
congesta}.}}{}
\verb{bodo-boka}{}{[ˈbɔdɔˈbɔka]}{(n.)}{1}{}{Lábios.}{}{}
\verb{bôdôja}{}{[bodoˈʒa]}{(v.)}{1}{}{Bordejar.}{}{}%
\verb{bôdôja}{}{[bodoˈʒa]}{(v.)}{2}{}{Rodear.}{}{}
\verb{bodo-matu}{}{[ˈbɔdɔˈmatu]}{(n.)}{1}{}{Excrementos.}{}{}
\verb{bodon}{}{[bɔˈdõ]}{(n.)}{3}{}{Bordão.}{}{}
\verb{bodon}{}{[bɔˈdõ]}{(n.)}{2}{}{Cacete.}{}{}
\verb{bodon}{}{[bɔˈdõ]}{(n.)}{1}{}{Porrete.}{}{}%
\verb{bofeton}{}{[bɔfɛˈt\~ɔ]}{(n.)}{1}{}{Bofetada.}{}{}
\verb{bofeton}{}{[bɔfɛˈt\~ɔ]}{(n.)}{2}{}{Tapa.}{}{}
\verb{bofyo}{}{[bɔˈfjɔ]}{(n.)}{1}{}{Fruta-pão grande.}{}{}
\verb{bogoso}{}{[bɔgɔˈsɔ]}{(v.)}{1}{}{Comer.}{}{}
\verb{bogoto}{}{[bɔgɔˈtɔ]}{(n.)}{1}{}{\textit{Bogoto}.}{\textit{\textbf{Pollia
condensata}}.}{}
\verb{boka}{}{[ˈbɔka]}{(n.)}{1}{}{Boca.}{}{}
\verb{boka}{}{[ˈbɔka]}{(n.)}{2}{}{Entrada.}{}{}
\verb{boka}{}{[ˈbɔka]}{(n.)}{3}{}{Lábios.}{}{}
\verb{boka-bela}{}{[ˈbɔka ˈbɛla]}{(n.)}{1}{}{Foz de rio.}{}{}
\verb{boka bili}{}{[ˈbɔka biˈli]}{(n.)}{1}{}{Bocejar.}{}{}
\verb{boka-binku}{}{[ˈbɔka ˈbĩku]}{(n.)}{1}{}{Umbigo.}{}{}
\verb{boka-doxi}{}{[ˈbɔka ˈdɔʃi]}{(adj.)}{1}{}{Com apetite.}{}{}
\verb{boka-doxi}{}{[ˈbɔka ˈdɔʃi]}{(adj.)}{1}{}{Simpático.}{}{}
\verb{boka-doxi}{}{[ˈbɔka ˈdɔʃi]}{(n.)}{1}{}{Conversa sedutora.}{}{}
\verb{boka-doxi}{}{[ˈbɔka ˈdɔʃi]}{(n.)}{1}{}{Falinhas mansas.}{}{}
\verb{boka-doxi}{}{[ˈbɔka ˈdɔʃi]}{(n.)}{1}{}{Lisonjeador.}{}{}
\verb{boka-doxi}{}{[ˈbɔka ˈdɔʃi]}{(n.)}{1}{}{Sedutor.}{}{}
\verb{bokadu}{}{[bɔˈkadu]}{(n.)}{1}{}{Bocado.}{}{}{}
\verb{bokadu}{}{[bɔˈkadu]}{(n.)}{2}{}{\textit{Bocado}.}{Ritual tradicional e
familiar que ocorre na Quarta-Feira de Cinzas durante o qual o membro
feminino mais idoso introduz na boca de cada participante uma colher com
alimento tradicional.}{}{}
\verb{boka-fede}{}{[ˈbɔka fɛˈdɛ]}{(n.)}{1}{}{Mau hálito.}{}{}
\verb{boka-ôkô}{}{[ˈbɔka ˈoko]}{(adj.)}{1}{}{Fofoqueiro.}{}{}
\verb{boka-ôkô}{}{[ˈbɔka ˈoko]}{(adj.)}{1}{}{Tagarela.}{}{}
\verb{boka-ôkô}{}{[ˈbɔka ˈoko]}{(n.)}{1}{}{Espingarda.}{}{}
\verb{boka-ple}{}{[ˈbɔka ˈplɛ]}{(n.)}{1}{}{Boca da praia.}{}{}
\verb{boka-pligitu}{}{[ˈbɔka pliˈgitu]}{(n.)}{1}{}{Restos.}{}{}
\verb{boka-pligitu}{}{[ˈbɔka pliˈgitu]}{(n.)}{2}{}{Sobejos.}{}{}%
\verb{boka-suzu}{}{[ˈbɔka ˈsuzu]}{(adj.)}{1}{}{Boçal.}{}{}
\verb{boka-suzu}{}{[ˈbɔka ˈsuzu]}{(adj.)}{1}{}{Insolente.}{}{}
\verb{boka-suzu}{}{[ˈbɔka ˈsuzu]}{(adj.)}{1}{}{Menosprezador.}{}{}
\verb{boka-xinu}{}{[ˈbɔka ˈʃinu]}{(n.)}{1}{}{Calças boca-de-sino.}{}{}
\verb{bola}{}{[ˈbɔla]}{(n.)}{1}{}{Bocado.}{}{}
\verb{bola}{}{[ˈbɔla]}{(n.)}{2}{}{Bola.}{}{}
\verb{bola}{}{[ˈbɔla]}{(n.)}{3}{}{Borra.}{}{}%
\verb{bolila}{}{[ˈbɔlila]}{(n.)}{1}{}{Algo de borla.}{}{}
\verb{bolila}{}{[ˈbɔlila]}{(n.)}{1}{}{Boleia.}{}{}
\verb{bolilo}{}{[bɔˈlilɔ]}{(n.)}{1}{}{Impotente sexual.}{}{}
\verb{bôlô}{}{[ˈbolo]}{(n.)}{1}{}{Bolacha.}{}{}
\verb{bôlô}{}{[ˈbolo]}{(n.)}{2}{}{Bolo.}{}{}
\verb{bolo}{}{[bɔˈlɔ]}{(v.)}{1}{}{Barrar.}{}{}
\verb{bolo}{}{[bɔˈlɔ]}{(v.)}{2}{}{Besuntar.}{}{}
\verb{bolo}{}{[bɔˈlɔ]}{(v.)}{3}{}{Friccionar suavemente.}{}{}
\verb{bolo}{}{[bɔˈlɔ]}{(v.)}{4}{}{Untar.}{}{}
\verb{bolodô-mindjan}{}{[bɔlɔˈdo m\~iˈdʒ\~{\textturna}]}{(n.)}{1}{}{Massagista.}{}{}
\verb{boloja}{}{[bɔlɔˈʒa]}{(v.)}{1}{}{Imiscuir-se.}{}{}
\verb{boloja}{}{[bɔlɔˈʒa]}{(v.)}{1}{}{Tagarelar.}{}{}
\verb{boloja}{}{[bɔlɔˈʒa]}{(v.)}{1}{}{Tentar cativar para obter algo em troca.}{}{}
\verb{bolo mindjan}{}{[bɔˈlɔ m\~iˈdʒ\~{\textturna}]}{(expr.)}{1}{}{Massagear.}{}{}
\verb{bolo mindjan}{}{[bɔˈlɔ m\~iˈdʒ\~{\textturna}]}{(expr.)}{2}{}{Massajar.}{}{}
\verb{bômbôlimbô}{}{[ˈbõboˈlĩbo]}{(adj.)}{1}{}{Deteriorado.}{}{}
\verb{bômbôlimbô}{}{[ˈbõboˈlĩbo]}{(adj.)}{2}{}{Estragado.}{}{}%
\verb{bomu}{}{[ˈbɔmu]}{(v.)}{1}{}{Vamos.}{Cf. \textbf{bamu}.}{}{}
\verb{bon}{}{[ˈbõ]}{(adj.)}{1}{}{Boa.}{}
\verb{bon}{}{[ˈbõ]}{(adj.)}{2}{}{Bom.}{}
\verb{bon}{}{[ˈbõ]}{(adv.)}{1}{}{Em suma.}{Cf. \textbf{mbon}.}{}{}
\verb{bon-afe}{}{[ˈbõ aˈfɛ]}{(n.)}{1}{}{Boa-fé.}{}{}
\verb{bondadji}{}{[bõˈdadʒi]}{(n.)}{1}{}{Bondade.}{}{}
\verb{bondja}{}{[bõˈdʒa]}{(adv.)}{1}{}{Ainda bem.}{\textbf{Bondja bô kuxtuma ku fôlô za}. \textit{Ainda bem que já te acostumaste aos forros}.}{}
\verb{bondja}{}{[bõˈdʒa]}{(adv.)}{2}{}{Em boa hora.}{}{}
\verb{bondja}{}{[bõˈdʒa]}{(adv.)}{3}{}{Felizmente.}{}{}
\verb{bon-dja}{}{[bõˈdʒa]}{(n.)}{1}{}{Bom dia.}{}{}
\verb{bondlega-nglandji}{}{[b\~ɔˈdlɛga ˈŋgl\~{\textturna}dʒi]}{(n.)}{1}{}{Beldroega-grande.}{\textit{\textbf{Talinum triangulare}.}}{}
%\verb{bondlega-mwala}{}{[b\~ɔˈdlɛga ˈmwala]}{(n.)}{1}{}{Beldroega-grande.}{Cf. \textbf{bodlega-nglandji}.}{}{}
%\verb{bondlega-ome}{}{[b\~ɔˈdlɛga ˈɔmɛ]}{(n.)}{1}{}{\emph{Bondlega-ome}.}{}{}{}
\verb{bondlega-pikina}{}{[b\~ɔˈdlɛga piˈkina]}{(n.)}{1}{}{Beldroega-pequena.}{\textit{\textbf{Portulaca oleracea}.}}{}
\verb{bondôzô}{}{[bõˈdozo]}{(adj.)}{1}{}{Generoso.}{}{}
\verb{bongamon}{}{[bõgaˈmõ]}{(n.)}{1}{}{Boga.}{\textit{\textbf{Boops boops}.}}{}
\verb{bon-kloson}{}{[ˈbõ klɔˈs\~ɔ]}{(adj.)}{1}{}{Caridoso.}{}{}
\verb{bosali}{}{[bɔˈsali]}{(adj.)}{1}{}{Boçal.}{}{}
\verb{bosali}{}{[bɔˈsali]}{(adj.)}{2}{}{Grosseiro.}{}{}
\verb{bosali}{}{[bɔˈsali]}{(adj.)}{3}{}{Rude.}{}{}%
\verb{bota}{}{[ˈbɔta]}{(n.)}{1}{}{Bota.}{}{}
\verb{botandji}{}{[bɔt\~{\textturna}ˈdʒi]}{(n.)}{1}{}{Bretangil.}{}{}{}
\verb{bôtê}{}{[boˈte]}{(n.)}{2}{}{Botelha.}{}{}%
\verb{bôtê}{}{[boˈte]}{(n.)}{1}{}{Botija.}{}{}%
\verb{bôtê}{}{[boˈte]}{(n.)}{3}{}{Garrafa.}{}{}%
\verb{bôtê}{}{[boˈte]}{(n.)}{4}{}{Garrafa de barro grande.}{}{}%
\verb{boto}{}{[bɔˈtɔ]}{(n.)}{1}{}{Botão.}{Cf. \textbf{boton}.}{}{}
\verb{boton}{}{[bɔˈt\~ɔ]}{(n.)}{1}{}{Botão.}{}{}
\verb{botono}{}{[bɔtɔˈnɔ]}{(adj.)}{1}{}{Glabro.}{}{}
\verb{boya}{}{[ˈbɔja]}{(n.)}{1}{}{Bóia.}{}{}
\verb{boya}{}{[ˈbɔja]}{(n.)}{1}{}{Fruta-pão.}{}{}
\verb{boya}{}{[bɔˈja]}{(v.)}{1}{}{Flutuar.}{}{}%
\verb{boya}{}{[bɔˈja]}{(v.)}{2}{}{Levantar.}{}{}%
\verb{boya}{}{[bɔˈja]}{(v.)}{3}{}{Suspender.}{}{}
\verb{boyadu}{}{[bɔˈjadu]}{(adj.)}{1}{}{Emerso.}{}{}
\verb{boyadu}{}{[bɔˈjadu]}{(adj.)}{1}{}{Içado.}{}{}
\verb{bonzwanu}{}{[b\~ɔˈzwanu]}{(n.)}{1}{}{feliz ano novo.}{}{}
\verb{bu blugidi}{}{[ˈbu blugiˈdi]}{(expr.)}{1}{}{Desmoronar.}{}{}
\verb{bu blugidi}{}{[ˈbu blugiˈdi]}{(expr.)}{1}{}{Despencar.}{Cf. \textbf{ba bligidi}.}{}
\verb{bubu}{}{[buˈbu]}{(n.)}{1}{}{Baiacu.}{}{}
\verb{budu}{}{[ˈbudu]}{(n.)}{1}{}{Pedra.}{}{}%
\verb{budu-lolodu}{}{[ˈbudu lɔˈlɔdu]}{(n.)}{1}{}{Pedra para pisar.}{}{}%
\verb{budu-magita}{}{[ˈbudu maˈgita]}{(n.)}{1}{}{Pedra para moer a
malagueta.}{}{}%
\verb{budu-pali}{}{[ˈbudu paˈli]}{(n.)}{1}{}{Pedra para a parturiente
repousar quando vai dar à luz.}{}%
\verb{bufa}{}{[buˈfa]}{(v.)}{1}{}{Cobrir(-se).}{}{}{}
\verb{bufadu}{}{[buˈfadu]}{(adj.)}{1}{}{Agasalhado.}{}{}{}
\verb{bufadu}{}{[buˈfadu]}{(adj.)}{1}{}{Disfarçado.}{}{}{}
\verb{bufadu}{}{[buˈfadu]}{(n.)}{1}{}{Assaltante encapuçado.}{}{}{}
\verb{bujibuji}{}{[buˈʒibuˈʒi]}{(n.)}{1}{}{Bengue.}{Cf. \textbf{bengi}.}{}{}%\verb{bujibuji}{}{[buˈʒibuˈʒi]}{(n.)}{2}{}{Capinzal.}{}{}%
\verb{bujibuji}{}{[buˈʒibuˈʒi]}{(n.)}{3}{}{Terreno baldio.}{}{}
\verb{bujigu}{}{[ˈbuʒigu]}{(n.)}{1}{}{Besugo.}{\textit{\textbf{Pomadasys
incisus}.}}{}
\verb{bujina}{}{[buˈʒina]}{(n.)}{1}{}{Buzina.}{}{}
\verb{bujinga}{}{[buˈʒ\~iga]}{(n.)}{1}{}{Poça de água.}{}{}
\verb{bujinganga}{}{[buʒ\~iˈg\~{\textturna}ga]}{(n.)}{1}{}{Bugiganga.}{}{}
\verb{buka}{}{[buˈka]}{(v.)}{1}{}{Buscar.}{}{}
\verb{buka}{}{[buˈka]}{(v.)}{2}{}{Tentar.}{}{}
\verb{buka ledu}{}{[buˈka ˈlɛdu]}{(expr.)}{2}{}{Provocar.}{}{}
\verb{bula}{}{[ˈbula]}{(n.)}{1}{}{Bolo de farinha de milho.}{}{}
\verb{bula}{}{[ˈbula]}{(n.)}{2}{}{Bolo de fubá.}{}{}
\verb{bulawê}{}{[bulaˈwe]}{(n.)}{1}{\textit{Bulawê}.}{Ritmo musical tradicional.}{}{}
\verb{bulhon}{}{[buˈʎ\~ɔ]}{(n.)}{1}{}{Bulhão.}{\textbf{\textit{Bodianus speciosus}}.}{}
\verb{buli}{}{[ˈbuli]}{(n.)}{1}{}{Bule.}{}{}
\verb{buli}{}{[ˈbuli]}{(n.)}{2}{}{\emph{Buli}.}{\textit{\textbf{Voacanga africana}}.}{}
\verb{buli}{}{[buˈli]}{(v.)}{1}{}{Agitar-se.}{}{}
\verb{buli}{}{[buˈli]}{(v.)}{2}{}{Bulir.}{}{}%
\verb{buli}{}{[buˈli]}{(v.)}{3}{}{Mexer-se.}{}{}%
\verb{bulidu}{}{[buˈli]}{(adj.)}{1}{}{Agitado.}{}{}%
\verb{bulidu}{}{[buˈli]}{(adj.)}{1}{}{Mexido.}{}{}
\verb{bulitxin}{}{[buliˈtʃĩ]}{(n.)}{1}{}{Boletim.}{}{}
\verb{bumabuma}{}{[buˈmabuˈma]}{(v.)}{1}{}{Fazer algo atabalhoadamente.}{}{}
\verb{bumbu}{}{[ˈbũbu]}{(n.)}{1}{}{Bombo.}{}{}
\verb{buneku}{}{[buˈnɛku]}{(n.)}{1}{}{Boneco.}{}{}
\verb{bunga}{}{[bũˈga]}{(n.)}{1}{}{Pau-candeia.}{\textit{\textbf{Hernandia
beninensis}.}}{}
\verb{bunitu}{}{[buˈnitu]}{(n.)}{1}{}{Bonito.}{\textit{\textbf{Caranx
crysos}.}}{}
\verb{bunzu}{}{[ˈbũzu]}{(n.)}{1}{}{Búzio.}{}{}
\verb{bunzu-d'ôbô}{}{[ˈbũzu doˈbo]}{(n.)}{1}{}{Búzio do mato.}{}{}
\verb{bunzu-d'omali}{}{[ˈbũzu dɔˈmali]}{(n.)}{1}{}{Búzio do mar.}{}{}
\verb{buru}{}{[ˈbu{\textfishhookr}u]}{(n.)}{1}{}{Burro.}{}{}
\verb{buseta}{}{[buˈsɛta]}{(n.)}{1}{}{Boceta onde os pescadores guardam os seus acessórios de pesca.}{}{}
\verb{buseta}{}{[buˈsɛta]}{(n.)}{1}{}{Receita.}{}{}
\verb{busu}{}{[ˈbusu]}{(n.)}{1}{}{Bucho.}{}{}
\verb{busu}{}{[ˈbusu]}{(n.)}{2}{}{Estômago.}{}{}
\verb{buta}{}{[buˈta]}{(adv.)}{2}{}{Completamente.}{}{}
\verb{buta}{}{[buˈta]}{(adv.)}{1}{}{Fora.}{}{}%
\verb{buta}{}{[buˈta]}{(v.)}{3}{}{Colocar.}{}{}
\verb{buta}{}{[buˈta]}{(v.)}{2}{}{Dizer.}{}{}%
\verb{buta}{}{[buˈta]}{(v.)}{1}{}{Pôr.}{}{}%
\verb{butadô-vungu}{}{[butaˈdo ˈvũgu]}{(n.)}{1}{}{Puxador, aquele que introduz a música, sobretudo no \textbf{sokope}.}{}{}{}
\verb{buta kloson ba lonji}{}{[buˈta klɔˈs\~ɔ ba ˈl\~ɔʒi]}{(expr.)}{1}{}{Divertir(-se).}{}{}
\verb{buta kupi}{}{[buˈta kuˈpi]}{(expr.)}{1}{}{Cuspir.}{}{}
\verb{buta pedasu}{}{[buˈta pɛˈdasu]}{(expr.)}{1}{}{Remendar.}{}{}
\verb{butin}{}{[buˈtĩ]}{(n.)}{1}{}{Bota alta.}{}{}%
\verb{butin}{}{[buˈtĩ]}{(n.)}{2}{}{Butina.}{}{}
\verb{butubutu}{}{[ˈbutuˈbutu]}{(adj.)}{1}{}{Robusto.}{}{}
\verb{butxika}{}{[buˈtʃika]}{(n.)}{1}{}{Botica.}{}{}
\verb{butxika}{}{[buˈtʃika]}{(n.)}{2}{}{Farmácia.}{}{}
\verb{butxiza}{}{[butʃiˈza]}{(v.)}{1}{}{Batizar.}{}{}
\verb{butxizadu}{}{[butʃiˈza]}{(adj.)}{1}{}{Batizado.}{}{}
\verb{butxizumu}{}{[butʃiˈzumu]}{(n.)}{1}{}{Batismo.}{}{}
\verb{butxizumu}{}{[butʃiˈzumu]}{(n.)}{1}{}{Nome próprio.}{}{}
\verb{buya}{}{[buˈja]}{(v.)}{1}{}{Abraçar.}{}{}
\verb{buya}{}{[buˈja]}{(v.)}{2}{}{Complicar.}{}{}
\verb{buya}{}{[buˈja]}{(v.)}{3}{}{Embrulhar.}{}{}%
\verb{buya}{}{[buˈja]}{(v.)}{4}{}{Enredar.}{}{}
\verb{buya}{}{[buˈja]}{(v.)}{5}{}{Ensarilhar.}{}{}
\verb{buyada}{}{[buˈjada]}{(n.)}{1}{}{Alergia.}{}{}
\verb{buyada}{}{[buˈjada]}{(n.)}{2}{}{Mau-olhado.}{}{}
\verb{buyadu}{}{[buˈjadu]}{(adj.)}{1}{}{Abraçado.}{}{}
\verb{buyadu}{}{[buˈjadu]}{(adj.)}{2}{}{Embrulhado.}{}{}
\verb{buyadu}{}{[buˈjadu]}{(adj.)}{1}{}{Em maus lençóis.}{}{}
\verb{buyadu}{}{[buˈjadu]}{(adj.)}{1}{}{Engatado.}{}{}
\verb{buyadu}{}{[buˈjadu]}{(adj.)}{1}{}{Ensarilhado.}{}{}
\verb{buza}{}{[buˈza]}{(v.)}{1}{}{Abusar.}{}{}%
\verb{bwa}{}{[ˈbwa]}{(adv.)}{1}{}{Bem.}{}{}
\verb{bwa}{}{[ˈbwa]}{(adv.)}{1}{}{Bom.}{}{}
\verb{bwa}{}{[ˈbwa]}{(v.)}{1}{}{Ser bom.}{}{}
\verb{bwabwa}{}{[ˈbwaˈbwa]}{(n.)}{1}{}{Cascata.}{}{}
\verb{bwada}{}{[bwaˈda]}{(v.)}{1}{}{Agradar.}{}{}
\verb{bwadu}{}{[ˈbwadu]}{(adj.)}{1}{}{Bom.}{}{}
\verb{bwa-nôtxi}{}{[ˈbwa ˈnotʃi]}{(n.)}{1}{}{Boa noite.}{}{}
\verb{bwa so}{}{[ˈbwa ˈsɔ]}{(expr.)}{1}{}{Excelente.}{}{}
\verb{bwa so}{}{[ˈbwa ˈsɔ]}{(expr.)}{2}{}{Ótimo.}{}{}
\verb{bwatu}{}{[ˈbwatu]}{(n.)}{1}{}{Boato.}{}{}
\verb{bwax-tadji}{}{[ˈbwaʃ ˈtadʒi]}{(n.)}{1}{}{Boa tarde.}{}{}
\verb{bwê}{}{[ˈbwe]}{(n.)}{1}{}{Bovino.}{}{}{}
\verb{bwê-mwala}{}{[ˈbwe ˈmwala]}{(n.)}{1}{}{Vaca.}{}{}{}
\verb{bwê-ome}{}{[ˈbwe ˈɔmɛ]}{(n.)}{2}{}{Boi.}{}{}{}
\verb{byê}{}{[ˈbje]}{(v.)}{1}{}{Cozer.}{}{}
\verb{byê}{}{[ˈbje]}{(v.)}{2}{}{Estar cozido.}{}{}
\verb{byebyebye}{}{[bjɛbjɛˈbjɛ]}{(n.)}{1}{}{Claridade.}{}{}
\verb{byêdu}{}{[ˈbjedu]}{(adj.)}{1}{}{Cozido.}{}{}
\verb{byôkô}{}{[ˈbjoko]}{(n.)}{1}{}{Careta.}{}{}
\verb{byololo}{}{[bjɔlɔˈlɔ]}{(adj.)}{2}{}{Amolecido.}{}{}{}
\verb{byololo}{}{[bjɔlɔˈlɔ]}{(adj.)}{1}{}{Amolejado.}{}{}{}
\verb{byololo}{}{[bjɔlɔˈlɔ]}{(id.)}{1}{}{Cf. \textbf{Sendê byololo}.}{}{}{}
\end{letra}

\begin{letra}{d}
\verb{da}{}{[ˈda]}{(conj.)}{2}{}{Devido a.}{}{}
\verb{da}{}{[ˈda]}{(conj.)}{1}{}{Por causa de.}{}{}%
\verb{da}{}{[ˈda]}{(conj.)}{3}{}{Porque.}{Cf. \textbf{punda}.}{}{}
\verb{da}{}{[ˈda]}{(prep.)}{1}{}{Para.}{}{}
\verb{da}{}{[ˈda]}{(v.)}{3}{}{Bater.}{}{}
\verb{da}{}{[ˈda]}{(v.)}{2}{}{Contar.}{}{}%
\verb{da}{}{[ˈda]}{(v.)}{1}{}{Dar.}{}{}%
\verb{da awa taba}{}{[ˈda ˈawa ˈtaba]}{(expr.)}{1}{}{Turvar a água.}{Com o
objetivo de apanhar camarão de rio.}{}
\verb{da baki}{}{[ˈda ˈbaki]}{(expr.)}{1}{}{Admoestar.}{}{}
\verb{da baki}{}{[ˈda ˈbaki]}{(expr.)}{1}{}{Berrar.}{}{}
\verb{da balansu}{}{[ˈda baˈl\~{\textturna}su]}{(expr.)}{1}{}{Girar.}{}{}
\verb{da balansu}{}{[ˈda baˈl\~{\textturna}su]}{(expr.)}{2}{}{Rodar.}{}{}
\verb{da banka}{}{[ˈda ˈb\~{\textturna}ka]}{(expr.)}{1}{}{Instalar-se.}{}{}
\verb{da bega}{}{[ˈda ˈbɛga]}{(expr.)}{1}{}{Engravidar.}{}{}
\verb{da bendenxa}{}{[ˈda b\~ɛˈd\~ɛʃa]}{(expr.)}{1}{}{Aconchegar.}{}{}
\verb{da bendenxa}{}{[ˈda b\~ɛˈd\~ɛʃa]}{(expr.)}{2}{}{Cavaquear.}{}{}
\verb{da bendenxa}{}{[ˈda b\~ɛˈd\~ɛʃa]}{(expr.)}{3}{}{Seduzir.}{}{}
\verb{da bendenxa}{}{[ˈda b\~ɛˈd\~ɛʃa]}{(expr.)}{4}{}{Tagarelar.}{}{}
\verb{da bensa}{}{[ˈda ˈb\~ɛsa]}{(expr.)}{1}{}{Abençoar.}{}{}
\verb{da benson}{}{[ˈda b\~ɛˈs\~ɔ]}{(expr.)}{1}{}{Abençoar.}{}{}
\verb{da blandu}{}{[ˈda ˈbl\~{\textturna}du]}{(expr.)}{1}{}{Anunciar.}{}{}
\verb{da bodon}{}{[ˈda bɔˈd\~ɔ]}{(expr.)}{1}{}{Abordoar.}{}{}
\verb{da buya}{}{[ˈda ˈbuja]}{(expr.)}{1}{}{Abraçar.}{}{}
\verb{da buya}{}{[ˈda ˈbuja]}{(expr.)}{2}{}{Enrolar.}{}{}
\verb{da dêsu paga}{}{[ˈda ˈdesu paˈga]}{(expr.)}{1}{}{Agradecer.}{}{}
\verb{da dizê}{}{[ˈda diˈze]}{(expr.)}{1}{}{Ajoelhar.}{}{}
\verb{dadji}{}{[ˈdadʒi]}{(n.)}{1}{}{Idade.}{}{}
\verb{dadô}{}{[daˈdo]}{(n.)}{1}{}{Contador.}{}{}{}
\verb{dadô}{}{[daˈdo]}{(n.)}{2}{}{Doador.}{}{}{}
\verb{dadô-patxi}{}{[daˈdo ˈpatʃi]}{(n.)}{1}{}{Denunciante.}{}{}{}
\verb{dadô-soya}{}{[daˈdo ˈsɔja]}{(n.)}{1}{}{Contador de histórias.}{}{}{}
\verb{dadu}{}{[ˈdadu]}{(adj.)}{1}{}{Simpático.}{}{}{}
\verb{dadu}{}{[ˈdadu]}{(n.)}{1}{}{Pessoa com quem se mantém uma relação
amorosa.}{}{}{}
\verb{da faka}{}{[ˈda ˈfaka]}{(expr.)}{1}{}{Esfaquear.}{}{}
\verb{da faka}{}{[ˈda ˈfaka]}{(expr.)}{2}{}{Fazer um corte.}{}{}
\verb{da flokadu}{}{[ˈda flɔˈkadu]}{(expr.)}{1}{}{Empurrar.}{}{}
\verb{dadu}{}{[ˈdadu]}{(n.)}{1}{}{Dardo.}{}{}%
\verb{dadu}{}{[ˈdadu]}{(n.)}{2}{}{Lança.}{}{}
\verb{daga}{}{[daˈga]}{(expr.)}{1}{}{Indagar.}{}{}
\verb{daga}{}{[daˈga]}{(expr.)}{1}{}{Sondar.}{}{}
\verb{daga}{}{[ˈdaga]}{(n.)}{1}{}{Vela de canoa.}{}{}
\verb{dain}{}{[daˈ\~i]}{(n.)}{1}{}{\textit{Dain}.}{\textit{\textbf{Lomariopsis
guineensis}}.}{}
\verb{da jatu}{}{[ˈda ˈʒatu]}{(expr.)}{1}{}{Arrotar.}{}{}
\verb{da jêtu}{}{[ˈda ˈʒetu]}{(expr.)}{1}{}{Remediar.}{}{}
\verb{da jêtu}{}{[ˈda ˈʒetu]}{(expr.)}{1}{}{Ajudar de forma enviesada.}{}{}
\verb{daji}{}{[ˈdaʒi]}{(n.)}{1}{}{Idade.}{Cf. \textbf{dadji}.}{}
\verb{da jinga}{}{[ˈda ˈʒ\~iga]}{(expr.)}{1}{}{Remendar a corda de
trepar.}{}{}
\verb{da kabêsa fundu}{}{[ˈda kaˈbesa ˈf\~udu]}{(expr.)}{1}{}{Mergulhar.}{}{}\verb{da kabêsa fundu}{}{[ˈda kaˈbesa ˈf\~udu]}{(expr.)}{2}{}{Ter relações
sexuais.}{}{}
\verb{da kapotxi}{}{[ˈda kaˈpɔtʃi]}{(expr.)}{1}{}{Derrotar alguém no jogo da bisca.}{}{}
\verb{da kasa}{}{[ˈda ˈkasa]}{(expr.)}{1}{}{Acasalar.}{}{}
\verb{da kasa}{}{[ˈda ˈkasa]}{(expr.)}{1}{}{Fecundar.}{}{}
\verb{da kasa}{}{[ˈda ˈkasa]}{(expr.)}{1}{}{Gerar.}{}{}
\verb{da kasa}{}{[ˈda ˈkasa]}{(expr.)}{2}{}{Procriar.}{}{}
\verb{da kaxtigu}{}{[ˈda kaʃˈtigu]}{(expr.)}{1}{}{Castigar.}{}{}
\verb{da kaxtigu}{}{[ˈda kaʃˈtigu]}{(expr.)}{1}{}{Punir.}{}{}
\verb{da kebla}{}{[ˈda ˈkɛbla]}{(expr.)}{1}{}{Gargalhar.}{}{}
\verb{da kebla kwakwakwa}{}{[ˈda ˈkɛbla kwakwaˈkwa]}{(expr.)}{1}{}{Rir às
gargalhadas.}{}{}
\verb{da kôlô dixi}{}{[ˈda koˈlo ˈdiʃi]}{(expr.)}{1}{}{Despertar.}{}{}
\verb{da kôlô dixi}{}{[ˈda koˈlo ˈdiʃi]}{(expr.)}{2}{}{Ressuscitar.}{}{}
\verb{da ku}{}{[ˈda ˈku]}{(expr.)}{1}{}{Acontecer.}{}{}%
\verb{da ku}{}{[ˈda ˈku]}{(expr.)}{2}{}{Encontrar.}{}{}%
\verb{da ku}{}{[ˈda ˈku]}{(expr.)}{3}{}{Passar-se com.}{}{}
\verb{da kubu}{}{[ˈda ˈkubu]}{(expr.)}{1}{}{Arremeter.}{}{}
\verb{da kubu}{}{[ˈda ˈkubu]}{(expr.)}{1}{}{Fazer investidas.}{}{}
\verb{da ku po}{}{[ˈda ˈku ˈpɔ]}{(expr.)}{1}{}{Bater.}{}{}
\verb{da kusu}{}{[ˈda ˈkusu]}{(expr.)}{1}{}{Ser acometido de diarreia.}{}{}
\verb{da lentla}{}{[ˈda l\~ɛˈtla]}{(expr.)}{1}{}{Entrar de repente.}{}{}
\verb{da lepalu}{}{[ˈda lɛˈpalu]}{(expr.)}{1}{}{Reparar.}{}{}
\verb{da loda}{}{[ˈda ˈlɔda]}{(expr.)}{1}{}{Dar voltas.}{}{}
\verb{da mali}{}{[ˈda ˈmali]}{(expr.)}{1}{}{Falar mal.}{}{}
\verb{da mangingi}{}{[ˈda
m\~{\textturna}gĩˈgi]}{(expr.)}{1}{}{Empoleirar(-se).}{}{}
\verb{da mangingi}{}{[ˈda
m\~{\textturna}gĩˈgi]}{(expr.)}{2}{}{Pendurar(-se).}{}{}
\verb{da maxtôlô}{}{[ˈda maʃˈtolo]}{(expr.)}{1}{}{Esconder(-se).}{}{}
\verb{da misa}{}{[ˈda ˈmisa]}{(expr.)}{1}{}{Rezar missa.}{}{}
\verb{damon}{}{[daˈm\~ɔ]}{(n.)}{1}{}{Colega.}{}{}
\verb{damon}{}{[daˈm\~ɔ]}{(n.)}{2}{}{Companheiro.}{}{}
\verb{damon}{}{[daˈm\~ɔ]}{(n.)}{2}{}{Aliado.}{}{}
\verb{damon}{}{[daˈm\~ɔ]}{(n.)}{3}{}{Parceiro.}{}{}
\verb{dana}{}{[daˈna]}{(v.)}{1}{}{Apodrecer.}{}{}
\verb{dana}{}{[daˈna]}{(v.)}{2}{}{Danificar.}{}{}%
\verb{dana}{}{[daˈna]}{(v.)}{3}{}{Degradar.}{}{}%
\verb{dana}{}{[daˈna]}{(v.)}{4}{}{Estragar(-se).}{}{}%
\verb{danadu}{}{[daˈnadu]}{(adj.)}{1}{}{Danificado.}{}{}%
\verb{danadu}{}{[daˈnadu]}{(adj.)}{2}{}{Estragado.}{}{}%
\verb{danadu}{}{[daˈnadu]}{(adj.)}{3}{}{Furioso.}{}{}%
\verb{danadu}{}{[daˈnadu]}{(adj.)}{4}{}{Malparado.}{}{}
\verb{danadu}{}{[daˈnadu]}{(adj.)}{5}{}{Periclitante.}{}{}
\verb{danadu}{}{[daˈnadu]}{(adj.)}{6}{}{Zangado.}{}{}
\verb{dana-kaxta}{}{[daˈna ˈkaʃta]}{(n.)}{1}{}{Má companhia.}{}{}
\verb{dana kotokoto}{}{[daˈna kɔˈtɔkɔˈtɔ]}{(expr.)}{1}{}{Estragado por
completo.}{}{}
\verb{dangula}{}{[d\~{\textturna}guˈla]}{(n.)}{1}{}{Remendo.}{}{}
\verb{da nomi}{}{[ˈda ˈnɔmi]}{(expr.)}{1}{}{Nomear.}{}{}
\verb{dansa}{}{[d\~{\textturna}ˈsa]}{(v.)}{1}{}{Dançar.}{}{}
\verb{dansadô}{}{[d\~{\textturna}saˈdo]}{(n.)}{1}{}{Dançarino.}{}{}
\verb{dansadô-ome}{}{[d\~{\textturna}saˈdo ˈɔmɛ]}{(n.)}{1}{}{Dançarino.}{}{}
\verb{dansadô-mwala}{}{[d\~{\textturna}saˈdo
ˈmwala]}{(n.)}{1}{}{Dançarina.}{}{}
\verb{dansu-kongô}{}{[ˈd\~{\textturna}su
ˈkõgo]}{(n.)}{1}{}{Danço-Congo.}{Representação com cerca de trinta
figurantes, executada ao som de tambores, ferros e \textbf{kanza}.}{}{}{}
\verb{da odji}{}{[ˈda ˈɔdʒi]}{(expr.)}{1}{}{Ordenar.}{}{}
\verb{da osa}{}{[ˈda ˈɔsa]}{(expr.)}{1}{}{Dar porradas.}{}{}
\verb{da pankada}{}{[ˈda p\~{\textturna}ˈkada]}{(expr.)}{1}{}{Bater.}{}{}
\verb{da pankada}{}{[ˈda p\~{\textturna}ˈkada]}{(expr.)}{1}{}{Espancar.}{}{}
\verb{da patxi}{}{[ˈda ˈpatʃi]}{(expr.)}{1}{}{Anunciar.}{}{}
\verb{da patxi}{}{[ˈda ˈpatʃi]}{(expr.)}{2}{}{Dar parte.}{}{}
\verb{da patxi}{}{[ˈda ˈpatʃi]}{(expr.)}{3}{}{Denunciar.}{}{}
\verb{da pedon}{}{[ˈda pɛˈd\~ɔ]}{(expr.)}{1}{}{Perdoar.}{}{}
\verb{da pena}{}{[ˈda ˈp\~ɛna]}{(expr.)}{1}{}{Causar pena.}{}{}
\verb{da pena}{}{[ˈda ˈp\~ɛna]}{(expr.)}{2}{}{Meter dó.}{}{}
\verb{da pinsu}{}{[ˈda ˈp\~isu]}{(expr.)}{1}{}{Empurrar.}{}{}
\verb{da pontope}{}{[ˈda p\~ɔtɔˈpɛ]}{(expr.)}{1}{}{Pontapear.}{}{}
\verb{da saklixtu}{}{[ˈda saˈkliʃtu]}{(expr.)}{1}{}{Aparecer.}{}{}
\verb{da saya loda}{}{[ˈda ˈsaja ˈlɔda]}{(expr.)}{1}{}{Passear.}{}{}
\verb{da saya loda}{}{[ˈda ˈsaja ˈlɔda]}{(expr.)}{1}{}{Vaguear.}{}{}
\verb{da son}{}{[ˈda ˈs\~ɔ]}{(expr.)}{1}{}{Cair.}{}{}
\verb{da son}{}{[ˈda ˈs\~ɔ]}{(expr.)}{1}{}{Tombar.}{}{}
\verb{da son dĩĩĩ}{}{[ˈda ˈs\~ɔ ˈdĩĩĩ]}{(expr.)}{1}{}{Cair estatelado.}{}{}
\verb{da sôtxi}{}{[ˈda ˈsotʃi]}{(expr.)}{1}{}{Açoitar.}{}{}
\verb{da sôtxi}{}{[ˈda ˈsotʃi]}{(expr.)}{2}{}{Bater.}{}{}
\verb{da topi}{}{[ˈda ˈtɔpi]}{(expr.)}{1}{}{Tropeçar.}{}{}
\verb{da txifina}{}{[ˈda ˈtʃifina]}{(expr.)}{1}{}{Acicatar.}{}{}
\verb{da txifina}{}{[ˈda ˈtʃifina]}{(expr.)}{1}{}{Atiçar.}{}{}
\verb{da txifina}{}{[ˈda ˈtʃifina]}{(expr.)}{1}{}{Espicaçar.}{}{}
\verb{da txinta}{}{[ˈda ˈtʃ\~ita]}{(expr.)}{1}{}{Colorir.}{}{}
\verb{da txinta}{}{[ˈda ˈtʃ\~ita]}{(expr.)}{2}{}{Pintar.}{}{}
\verb{da vala}{}{[ˈda ˈvala]}{(expr.)}{1}{}{Açoitar.}{}{}
\verb{da vala}{}{[ˈda ˈvala]}{(expr.)}{1}{}{Bater.}{}{}
\verb{da venha}{}{[ˈda ˈv\~eɲa]}{(expr.)}{1}{}{Dar margem a.}{}{}
\verb{dawa}{}{[ˈdawa]}{(n.)}{1}{}{Coco tenro.}{}{}
\verb{da wê}{}{[ˈda ˈwe]}{(expr.)}{1}{}{Germinar.}{}{}
\verb{da wê}{}{[ˈda ˈwe]}{(expr.)}{2}{}{Olhar.}{}{}
\verb{da wê}{}{[ˈda ˈwe]}{(expr.)}{3}{}{Vigiar.}{}{}
\verb{da xikotxi}{}{[ˈda ʃiˈkɔtʃi]}{(expr.)}{1}{}{Chicotear.}{}{}
\verb{dê}{}{[ˈde]}{(poss.)}{1}{}{Dela.}{}{}
\verb{dê}{}{[ˈde]}{(poss.)}{2}{}{Delas.}{}{}
\verb{dê}{}{[ˈde]}{(poss.)}{3}{}{Dele.}{}{}
\verb{dê}{}{[ˈde]}{(poss.)}{4}{}{Deles.}{}{}
\verb{dê}{}{[ˈde]}{(poss.)}{5}{}{Seu.}{}{}%
\verb{dê}{}{[ˈde]}{(poss.)}{6}{}{Seus.}{}{}%
\verb{dê}{}{[ˈde]}{(poss.)}{7}{}{Sua.}{}{}%
\verb{dê}{}{[ˈde]}{(poss.)}{8}{}{Sua.}{}{}%
\verb{debota}{}{[dɛbɔˈta]}{(v.)}{3}{}{Desbotar.}{}{}
\verb{debota}{}{[dɛbɔˈta]}{(v.)}{1}{}{Discutir.}{}{}
\verb{debota}{}{[dɛbɔˈta]}{(v.)}{2}{}{Tagarelar.}{}{}
\verb{dêdê}{}{[deˈde]}{(n.)}{1}{}{Querida.}{}{}
\verb{dêdê}{}{[deˈde]}{(n.)}{1}{}{Querido.}{}{}
\verb{dêdê}{}{[deˈde]}{(v.)}{1}{}{Abraçar.}{}{}
\verb{dêdê}{}{[deˈde]}{(v.)}{2}{}{Afagar.}{}{}
\verb{dêdê}{}{[deˈde]}{(v.)}{3}{}{Embalar.}{}{}
\verb{dedu}{}{[ˈdɛdu]}{(n.)}{1}{}{Dedo.}{}{}
\verb{dedu-d'ope}{}{[ˈdɛdu dɔˈpɛ]}{(n.)}{1}{}{Dedo do pé.}{}{}
\verb{dedu-longô}{}{[ˈdɛdu ˈl\~ogo]}{(n.)}{1}{}{Dedo médio.}{}{}
\verb{dedu-neni}{}{[ˈdɛdu ˈnɛni]}{(n.)}{1}{}{Dedo anelar.}{}{}
\verb{dedu-nglandji}{}{[ˈdɛdu ˈŋgl\~{\textturna}dʒi]}{(n.)}{1}{}{Dedo
polegar.}{}{}
\verb{dedu-ponta}{}{[ˈdɛdu p\~ɔˈta]}{(n.)}{1}{}{Dedo indicador.}{}{}
\verb{dedu-txoko}{}{[ˈdɛdu ˈtʃɔkɔ]}{(n.)}{1}{}{Dedo mindinho.}{}{}
\verb{dêfendê}{}{[defẽˈde]}{(v.)}{1}{}{Aguentar.}{}{}
\verb{dêfendê}{}{[defẽˈde]}{(v.)}{2}{}{Defender.}{}{}
\verb{dêfêtu}{}{[deˈfetu]}{(n.)}{1}{}{Defeito.}{}{}
\verb{dêfêza}{}{[deˈfeza]}{(n.)}{1}{}{Defesa.}{}{}
\verb{dêfuntu}{}{[deˈfũtu]}{(n.)}{2}{}{Cadáver.}{}{}%
\verb{dêfuntu}{}{[deˈfũtu]}{(n.)}{1}{}{Defunto.}{}{}%
\verb{dêfuntu}{}{[deˈfũtu]}{(n.)}{3}{}{Fantasma.}{}{}
\verb{deja}{}{[dɛˈʒa]}{(v.)}{1}{}{Atrever-se a.}{}{}
\verb{delegadu}{}{[dɛlɛˈgadu]}{(n.)}{1}{}{Delegado.}{}{}
\verb{demanda}{}{[dɛˈm\~{\textturna}da]}{(n.)}{1}{}{Contenda.}{}{}
\verb{demanda}{}{[dɛˈm\~{\textturna}da]}{(n.)}{2}{}{Demanda.}{}{}
\verb{demanda}{}{[dɛˈm\~{\textturna}da]}{(n.)}{3}{}{Discussão.}{}{}
\verb{demoklaxia}{}{[dɛmɔklaˈʃia]}{(n.)}{1}{}{Democracia.}{}{}
\verb{demono}{}{[dɛˈmɔnɔ]}{(n.)}{1}{}{Demônio.}{}{}
\verb{denge}{}{[d\~ɛˈgɛ]}{(n.)}{1}{}{Denguice.}{}{}
\verb{denge}{}{[d\~ɛˈgɛ]}{(v.)}{1}{}{Agitar-se.}{}{}
\verb{denge}{}{[d\~ɛˈgɛ]}{(v.)}{2}{}{Requebrar-se.}{}{}%
\verb{dentxi}{}{[ˈdẽtʃi]}{(n.)}{1}{}{Dente.}{}{}
\verb{dentxi upa}{}{[ˈdẽtʃi ˈupa]}{(expr.)}{1}{}{Dentes espaçados, sinônimo de
beleza.}{}{}
\verb{dêpendê}{}{[dep\~eˈde]}{(v.)}{1}{}{Depender.}{}{}
\verb{dependenxa}{}{[dɛp\~ɛˈd\~ɛʃa]}{(n.)}{1}{}{Dependência.}{}{}%
\verb{dependenxa}{}{[dɛp\~ɛˈd\~ɛʃa]}{(n.)}{2}{}{Independência.}{}{}
\verb{dêpôji}{}{[deˈpoʒi]}{(adv.)}{1}{}{Depois.}{}{}
\verb{dêpôji ku}{}{[deˈpoʒi ˈku]}{(conj.)}{1}{}{Depois que.}{}{}
\verb{dêpôx}{}{[deˈpoʃ]}{(adv.)}{1}{}{Depois.}{Cf. \textbf{dêpôji}.}{}
\verb{deputadu}{}{[dɛpuˈtadu]}{(n.)}{1}{}{Deputado.}{}{}
\verb{desa}{}{[dɛˈsa]}{(v.)}{1}{}{Deixar.}{}{}
\verb{dêsê}{}{[deˈse]}{(prep. v.)}{1}{}{Para baixo.}{\textbf{A kôlê dêsê}.
\textit{Correram para baixo}.}{}%
\verb{dêsê}{}{[deˈse]}{(v.)}{1}{}{Descer.}{}{}%
\verb{dêsê}{}{[deˈse]}{(v.)}{2}{}{Diminuir.}{}{}
\verb{dêsê awa}{}{[deˈse ˈawa]}{(expr.)}{1}{}{Atingir o orgasmo.}{}{}
\verb{dêsê awa}{}{[deˈse ˈawa]}{(expr.)}{1}{}{Ejacular.}{}{}
\verb{desimu}{}{[ˈdɛsimu]}{(num.)}{1}{}{Décimo.}{}{}
\verb{dêsu}{}{[ˈdesu]}{(n.)}{1}{}{Deus.}{}{}
\verb{dêsu paga}{}{[ˈdesu paˈga]}{(expr.)}{1}{}{Obrigado.}{}{}
\verb{dêsu-paga}{}{[ˈdesu paˈga]}{(interj.)}{1}{}{Bem feito!}{(Ironia.)}{}
\verb{deta}{}{[dɛˈta]}{(v.)}{1}{}{Deitar(-se).}{}{}
\verb{detadu}{}{[dɛˈtadu]}{(adj.)}{1}{}{Deitado.}{}{}
\verb{deva}{}{[ˈdɛva]}{(n.)}{1}{}{Estrela-d'alva.}{}{}
\verb{deva}{}{[ˈdɛva]}{(n.)}{2}{}{Estrela-Polar.}{}{}
\verb{devasa}{}{[dɛˈvasa]}{(n.)}{1}{}{Devassa.}{}{}
\verb{dêvason}{}{[devaˈsõ]}{(n.)}{1}{}{\textit{Dêvason}.}{Amuleto que é
colocado ao pescoço dos recém-nascidos para os proteger dos
feiticeiros ou ao pescoço de um gémeo sobrevivente.}{}{}%
\verb{dêvê}{}{[deˈve]}{(n.)}{1}{}{Dívida espiritual.}{}{}
\verb{dêvê}{}{[deˈve]}{(v.)}{1}{}{Dever.}{}{}
\verb{dêvêsa}{}{[deveˈsa]}{(v.)}{1}{}{Atravessar.}{}{}
\verb{dêvidu}{}{[deˈvidu]}{(n.)}{1}{}{Pessoa nascida com dívida
espiritual.}{}{}
\verb{dexi}{}{[ˈdɛʃi]}{(num.)}{1}{}{Dez.}{}{}
\verb{dêya}{}{[ˈdeja]}{(n.)}{1}{}{Amiga.}{}{}%
\verb{dêya}{}{[ˈdeja]}{(n.)}{1}{}{Amigo.}{}{}%
\verb{dêya}{}{[ˈdeja]}{(n.)}{1}{}{Namorada.}{}{}%
\verb{dêya}{}{[ˈdeja]}{(n.)}{1}{}{Namorado.}{}{}%
\verb{dêya}{}{[deˈja]}{(v.)}{1}{}{Cobiçar.}{}{}%
\verb{dêya}{}{[deˈja]}{(v.)}{2}{}{Desejar.}{}{}
\verb{dêya}{}{[deˈja]}{(v.)}{3}{}{Flertar.}{}{}%
\verb{dêya}{}{[deˈja]}{(v.)}{3}{}{Ludibriar.}{}{}%
\verb{dêya}{}{[deˈja]}{(v.)}{4}{}{Namorar.}{}{}%
\verb{dêya}{}{[deˈja]}{(v.)}{5}{}{Seduzir.}{}{}%
\verb{dêyason}{}{[dejaˈsõ]}{(n.)}{1}{}{Enamoramento.}{}{}
\verb{dezanovi}{}{[dɛzaˈnɔvi]}{(num.)}{1}{}{Dezanove.}{}{}
\verb{dezasete}{}{[dɛzaˈsɛtɛ]}{(num.)}{1}{}{Dezassete.}{}{}
\verb{dezasêxi}{}{[dɛzaˈseʃi]}{(num.)}{1}{}{Dezasseis.}{}{}
\verb{dêzêja}{}{[dezeˈʒa]}{(v.)}{1}{}{Desejar.}{}{}
\verb{dezemblu}{}{[dɛˈz\~ɛblu]}{(n.)}{1}{}{Dezembro.}{}{}
\verb{dêzenhu}{}{[deˈz\~eɲu]}{(n.)}{1}{}{Desenho.}{}{}
\verb{di}{}{[ˈdi]}{(prep.)}{1}{}{De.}{}{}
\verb{dĩĩĩ}{}{[ˈdĩĩĩ]}{(id.)}{1}{}{Cf. \textbf{da son dĩĩĩ.}}{}
\verb{dĩĩĩ}{}{[ˈdĩĩĩ]}{(id.)}{2}{}{Cf. \textbf{kulu dĩĩĩ.}}{}
\verb{dijigôxtô}{}{[diʒiˈgoʃto]}{(n.)}{1}{}{Desgosto.}{}{}
\verb{dika}{}{[diˈka]}{(v.)}{1}{}{Apontar.}{}{}
\verb{dika}{}{[diˈka]}{(v.)}{1}{}{Indicar.}{Cf. \textbf{ndika}.}{}
\verb{dimila}{}{[dimiˈla]}{(v.)}{1}{}{Admirar.}{}{}
\verb{dimiladu}{}{[dimiˈladu]}{(adj.)}{1}{}{Admirado.}{}{}
\verb{dinansê}{}{[diˈn\~ase]}{(poss.)}{1}{}{Vossa.}{}{}
\verb{dinansê}{}{[diˈn\~ase]}{(poss.)}{2}{}{Vossas.}{}{}
\verb{dinansê}{}{[diˈn\~ase]}{(poss.)}{3}{}{Vosso.}{}{}
\verb{dinansê}{}{[diˈn\~ase]}{(poss.)}{4}{}{Vossos.}{}{}
\verb{dinen}{}{[ˈdin\~e]}{(poss.)}{1}{}{Delas.}{}{}
\verb{dinen}{}{[ˈdin\~e]}{(poss.)}{2}{}{Deles.}{}{}
\verb{dinen}{}{[ˈdin\~e]}{(poss.)}{3}{}{Seus.}{}{}%
\verb{dinen}{}{[ˈdin\~e]}{(poss.)}{4}{}{Suas.}{}{}%
\verb{dinvya}{}{[dĩˈvja]}{(v.)}{1}{}{Adivinhar.}{}{}
\verb{dinvyadô}{}{[dĩvjaˈdo]}{(n.)}{1}{}{Adivinhador.}{}{}
\verb{diplomatiku}{}{[diplɔˈmatiku]}{(n.)}{1}{}{Diplomático.}{}{}
\verb{dipôji}{}{[diˈpoʒi]}{(adv.)}{1}{}{Depois.}{Cf. \textbf{dêpôji}.}{}
\verb{diretôru}{}{[di{\textfishhookr}ɛˈto{\textfishhookr}u]}{(n.)}{1}{}{Diretor.}{}{}
\verb{disidi}{}{[disiˈdi]}{(v.)}{1}{}{Decidir.}{}{}%
\verb{disidi}{}{[disiˈdi]}{(v.)}{2}{}{Resolver.}{}{}
\verb{divida}{}{[ˈdivida]}{(n.)}{1}{}{Dívida.}{}{}
\verb{dividi}{}{[diviˈdi]}{(v.)}{1}{}{Dividir.}{}{}%
\verb{dividi}{}{[diviˈdi]}{(v.)}{2}{}{Repartir.}{}{}
\verb{divinu}{}{[diˈvinu]}{(n.)}{1}{}{Divino.}{}{}
\verb{divorsyu}{}{[diˈvɔ{\textfishhookr}sju]}{(n.)}{1}{}{Divórcio.}{}{}
\verb{dixglasa}{}{[diʃˈglasa]}{(n.)}{1}{}{Desgraça.}{}{}
\verb{dixglasadu}{}{[diʃglaˈsadu]}{(n.)}{1}{}{Desgraçado.}{}{}
\verb{dixgôxtô}{}{[diʃˈgoʃto]}{(n.)}{1}{}{Desgosto.}{}{}%
\verb{dixgôxtô}{}{[diʃˈgoʃto]}{(n.)}{2}{}{Mágoa.}{}{}%
\verb{dixgôxtô}{}{[diʃˈgoʃto]}{(n.)}{3}{}{Tristeza.}{}{}
\verb{dixi}{}{[ˈdiʃi]}{(n.)}{1}{}{Esticão.}{}{}%
\verb{dixiplina}{}{[diʃiˈplina]}{(n.)}{1}{}{Disciplina.}{}{}
\verb{dixku}{}{[ˈdiʃku]}{(n.)}{1}{}{Disco.}{}{}
\verb{dixkubli}{}{[diʃkuˈbli]}{(v.)}{1}{}{Descobrir.}{}{}
\verb{dixkunfya}{}{[diʃk\~uˈfja]}{(v.)}{1}{}{Desconfiar.}{}{}
\verb{dixkunfyadu}{}{[diʃk\~uˈfjadu]}{(adj.)}{1}{}{Desconfiado.}{}{}
\verb{dixkuti}{}{[diʃkuˈti]}{(v.)}{1}{}{Discutir.}{}{}
\verb{dixpantu}{}{[ˈdiʃp\~{\textturna}tu]}{(adv.)}{1}{}{De repente.}{}{}
\verb{dixpaxu}{}{[diʃˈpaʃu]}{(n.)}{1}{}{Decisão.}{}{}
\verb{dixpaxu}{}{[diʃˈpaʃu]}{(n.)}{1}{}{Despacho.}{}{}
\verb{dixpaxu}{}{[diʃˈpaʃu]}{(n.)}{1}{}{Ordem.}{}{}
\verb{dixpeza}{}{[diʃˈpeza]}{(n.)}{1}{}{Despesa.}{}{}
\verb{dixpidji}{}{[diʃpiˈdʒi]}{(n.)}{1}{}{Despedida.}{}{}
\verb{dixpidji}{}{[diʃpiˈdʒi]}{(v.)}{1}{}{Despedir(-se).}{}{}
\verb{dixtinu}{}{[diʃˈtinu]}{(n.)}{1}{}{Destino.}{}{}%
\verb{dixtinu}{}{[diʃˈtinu]}{(n.)}{2}{}{Sina.}{}{}
\verb{dizanda}{}{[diz\~{\textturna}ˈda]}{(v.)}{1}{}{Desencaminhar.}{}{}
\verb{dizanda}{}{[diz\~{\textturna}ˈda]}{(v.)}{2}{}{Desequilibrar.}{}{}
\verb{dizandadu}{}{[diz\~{\textturna}ˈdadu]}{(adj.)}{1}{}{Dolorido.}{}{}
\verb{dizandadu}{}{[diz\~{\textturna}ˈdadu]}{(adj.)}{1}{}{Esmagado.}{}{}
\verb{dizandadu}{}{[diz\~{\textturna}ˈdadu]}{(adj.)}{1}{}{Espancado.}{}{}
\verb{dizandadu}{}{[diz\~{\textturna}ˈdadu]}{(adj.)}{3}{}{Maltratado.}{}{}
\verb{dizaxtli}{}{[diˈzaʃtli]}{(n.)}{1}{}{Desastre.}{}{}
\verb{dizê misa}{}{[diˈze]}{(expr.)}{1}{}{Rezar missa.}{}{}
\verb{dizimu}{}{[ˈdizimu]}{(n.)}{1}{}{Dízimo.}{}{}
\verb{dizimu}{}{[ˈdizimu]}{(n.)}{1}{}{Pagamento de uma dívida.}{}{}
\verb{dizimu}{}{[ˈdizimu]}{(n.)}{1}{}{Pagamento de uma promessa.}{}{}
\verb{dizodji}{}{[diˈzɔdʒi]}{(n.)}{1}{}{Confusão.}{}{}
\verb{dizodji}{}{[diˈzɔdʒi]}{(n.)}{2}{}{Demanda.}{}{}
\verb{dizodji}{}{[diˈzɔdʒi]}{(n.)}{3}{}{Desordem.}{}{}
\verb{dizôytô}{}{[diˈzojto]}{(num.)}{1}{}{Dezoito.}{}{}
\verb{dja}{}{[ˈdʒa]}{(n.)}{1}{}{Dia.}{}{}
\verb{djablin}{}{[dʒaˈbl\~i]}{(adj.)}{1}{}{Endiabrado.}{}{}
\verb{djablin}{}{[dʒaˈbl\~i]}{(n.)}{1}{}{Irrequieto.}{}{}
\verb{djabu}{}{[ˈdʒabu]}{(interj.)}{1}{}{Raios!}{}{}
\verb{djabu}{}{[ˈdʒabu]}{(interj.)}{1}{}{Raios partam!}{}{}
\verb{djabu}{}{[ˈdʒabu]}{(n.)}{1}{}{Diabo.}{}{}
\verb{dja-d'anu}{}{[ˈdʒa ˈdanu]}{(n.)}{1}{}{Aniversário.}{}{}
\verb{dja-dja}{}{[ˈdʒa ˈdʒa]}{(adv.)}{1}{}{Qualquer dia.}{}{}
\verb{dja-dja}{}{[ˈdʒa ˈdʒa]}{(adv.)}{1}{}{Um dia.}{}{}
\verb{dja-djingu}{}{[ˈdʒa ˈdʒĩgu]}{(n.)}{1}{}{Domingo.}{}{}
\verb{djagu}{}{[ˈdʒagu]}{(adj.)}{1}{}{Aziago.}{}{}
\verb{djagu}{}{[ˈdʒagu]}{(n.)}{1}{}{Dia aziago.}{}{}
\verb{djambi}{}{[dʒ\~{\textturna}ˈbi]}{(n.)}{1}{}{\textit{Djambi}.}{Ritual
realizado durante a noite em torno de uma fogueira e que, através da música e
da dança num ritmo frenético, pretende levar ao transe os participantes,
principalmente os doentes, com o objetivo de os aliviar ou curar.}{}{}{}
\verb{djandja}{}{[dʒ\~{\textturna}ˈdʒa]}{(adv.)}{1}{}{Apressado.}{}{}{}
\verb{djandja}{}{[dʒ\~{\textturna}ˈdʒa]}{(adv.)}{1}{}{Depressa.}{}{}{}
\verb{djandja}{}{[dʒ\~{\textturna}ˈdʒa]}{(adv.)}{1}{}{Rapidamente.}{Cf. \textbf{djandjan}.}{}{}
\verb{djandjan}{}{[dʒ\~{\textturna}ˈdʒ\~{\textturna}]}{(adv.)}{1}{}{Apressado.}{}{}%
\verb{djandjan}{}{[dʒ\~{\textturna}ˈdʒ\~{\textturna}]}{(adv.)}{2}{}{Depressa.}{}{}%
\verb{djandjan}{}{[dʒ\~{\textturna}ˈdʒ\~{\textturna}]}{(adv.)}{3}{}{Rapidamente.}{}{}
\verb{djasu}{}{[dʒaˈsu]}{(interj.)}{1}{}{Bolas!}{}{}
\verb{djasu}{}{[dʒaˈsu]}{(interj.)}{1}{}{Porra!}{}{}
\verb{djasu}{}{[dʒaˈsu]}{(interj.)}{1}{}{Raios!}{}{}
\verb{djanga}{}{[dʒ\~{\textturna}ˈga]}{(interj.)}{2}{}{Diabo!}{}{}
\verb{djanga}{}{[dʒ\~{\textturna}ˈga]}{(interj.)}{2}{}{Merda!}{}{}
\verb{djanga}{}{[dʒ\~{\textturna}ˈga]}{(interj.)}{3}{}{Porra!}{}{}%
\verb{djanga}{}{[dʒ\~{\textturna}ˈga]}{(n.)}{2}{}{Canto.}{}{}
\verb{djanga}{}{[dʒ\~{\textturna}ˈga]}{(n.)}{2}{}{Esquina.}{}{}
\verb{djanga}{}{[dʒ\~{\textturna}ˈga]}{(n.)}{2}{}{Lugar recôndito.}{}{}
\verb{djê}{}{[ˈdʒe]}{(v.)}{1}{}{Apanhar.}{}{}
\verb{djê}{}{[ˈdʒe]}{(v.)}{2}{}{Buscar.}{}{}
\verb{djê}{}{[ˈdʒe]}{(v.)}{3}{}{Recolher.}{}{}
\verb{djêlu}{}{[ˈdʒelu]}{(n.)}{1}{}{Dinheiro.}{}{}
\verb{djesu}{}{[dʒɛˈsu]}{(interj.)}{1}{}{Diabo!}{}{}
\verb{djesu}{}{[dʒɛˈsu]}{(interj.)}{1}{}{Merda!}{}{}
\verb{djesu}{}{[dʒɛˈsu]}{(interj.)}{1}{}{Porra!}{Cf. \textbf{djasu}.}{}
\verb{dji}{}{[ˈdʒi]}{(prep.)}{1}{}{De.}{}{}
\verb{djibela}{}{[dʒiˈbɛla]}{(n.)}{1}{}{Algibeira.}{}{}%
\verb{djibela}{}{[dʒiˈbɛla]}{(n.)}{2}{}{Bolso.}{}{}
\verb{djibon}{}{[dʒiˈbõ]}{(n.)}{2}{}{Casaco.}{}{}
\verb{djibon}{}{[dʒiˈbõ]}{(n.)}{1}{}{Gibão.}{}{}%
\verb{djidali}{}{[dʒiˈdali]}{(n.)}{1}{}{Dedal.}{}{}
\verb{djikitxi}{}{[ˈdʒikitʃi]}{(adj.)}{1}{}{Genuíno.}{}{}%
\verb{djikitxi}{}{[ˈdʒikitʃi]}{(adj.)}{2}{}{Rude.}{}{}%
\verb{djimboa}{}{[dʒĩboˈa]}{(n.)}{1}{}{Jimboa.}{\textbf{\textit{Amaranthus
caudatus}.}}{}
%\verb{djimboa-matu}{}{[dʒĩboˈa ˈmatu]}{(n.)}{1}{}{Jimboa-do-mato.}{\textbf{\textit{Amaranthus viridis}.}}{}
\verb{djimola}{}{[dʒiˈmɔla]}{(n.)}{1}{}{Esmola.}{}{}
\verb{djina}{}{[ˈdʒina]}{(prep.)}{1}{}{De.}{\textbf{Djina liba antê basu}.
\textit{De cima a baixo}.}{}
\verb{djina}{}{[ˈdʒina]}{(prep.)}{1}{}{Desde.}{}{}
\verb{djina txintxintxin}{}{[ˈdʒina tʃĩtʃĩˈtʃĩ]}{(expr.)}{1}{}{Há muito
tempo.}{}{}
\verb{djina bixkôkô}{}{[ˈdʒina biʃˈkoko]}{(expr.)}{1}{}{Outrora.}{}{}
\verb{djinebla}{}{[dʒiˈnɛbla]}{(n.)}{1}{}{Genebra.}{}{}
\verb{djinga}{}{[dʒĩˈga]}{(v.)}{1}{}{Abanar.}{}{}
\verb{djinga}{}{[dʒĩˈga]}{(v.)}{2}{}{Agitar(-se).}{}{}
\verb{djinga}{}{[dʒĩˈga]}{(v.)}{3}{}{Balançar.}{}{}
\verb{djinga}{}{[dʒĩˈga]}{(v.)}{4}{}{Ficar perturbado.}{}{}
\verb{djinga}{}{[dʒĩˈga]}{(v.)}{5}{}{Gingar.}{}{}
\verb{djinga}{}{[dʒĩˈga]}{(v.)}{6}{}{Mover-se.}{}{}
\verb{djinga}{}{[dʒĩˈga]}{(v.)}{7}{}{Sacudir.}{}{}
\verb{djinga gidigidi}{}{[dʒĩˈga giˈdigiˈdi]}{(expr.)}{}{Abanar agitadamente
o corpo.}{}{}
\verb{djinola}{}{[dʒiˈnɔla]}{(adv.)}{1}{}{Desde essa altura.}{}{}
\verb{djinola}{}{[dʒiˈnɔla]}{(adv.)}{2}{}{Há muito tempo.}{}{}
\verb{djogo}{}{[ˈdʒɔgɔ]}{(n.)}{1}{}{\textit{Idjogo}.}{Cf. \textbf{idjogo}.}{}\verb{djômblo}{}{[dʒ\~oˈblo]}{(adj.)}{1}{}{Desajeitado.}{}{}
\verb{djômblo}{}{[dʒ\~oˈblo]}{(adj.)}{1}{}{Desengonçado.}{}{}
\verb{djômblo}{}{[dʒ\~oˈblo]}{(adj.)}{1}{}{Desleixado.}{}{}
\verb{djômblo}{}{[dʒ\~oˈblo]}{(n.)}{1}{}{Desajeitado.}{}{}
\verb{djômblo}{}{[dʒ\~oˈblo]}{(n.)}{1}{}{Desengonçado.}{}{}
\verb{djômblo}{}{[dʒ\~oˈblo]}{(n.)}{1}{}{Desleixado.}{}{}
\verb{dlegadu}{}{[dlɛˈgadu]}{(adj.)}{1}{}{Delgado.}{}{}
\verb{dlegadu}{}{[dlɛˈgadu]}{(adj.)}{2}{}{Magro.}{}{}
\verb{dlegeda}{}{[dlɛgɛˈda]}{(v.)}{1}{}{Degredar.}{}{}
\verb{dlegedu}{}{[dlɛˈgɛdu]}{(n.)}{1}{}{Degredo.}{}{}
\verb{dlentu}{}{[ˈdlẽtu]}{(adv.)}{1}{}{Dentro (de).}{Cf. \textbf{glentu}.}{}
\verb{dlêtê}{}{[dleˈte]}{(v.)}{1}{}{Derreter.}{}{}
\verb{dlêtu}{}{[ˈdletu]}{(adj.)}{1}{}{Direito.}{Cf. \textbf{glêtu}.}{}
\verb{dluba}{}{[dluˈba]}{(v.)}{1}{}{Derrubar.}{}{}
\verb{dobla}{}{[ˈdɔbla]}{(n.)}{1}{}{Dobra.}{Unidade monetária de São Tomé e
Príncipe.}{}
\verb{dobla}{}{[dɔˈbla]}{(v.)}{1}{}{Dobrar.}{}{}
\verb{dôdô}{}{[ˈdodo]}{(adj.)}{1}{}{Doido.}{}{}%
\verb{dôdô}{}{[ˈdodo]}{(adj.)}{3}{}{Lunático.}{}{}
\verb{dôdô}{}{[ˈdodo]}{(adj.)}{2}{}{Maluco.}{}{}%
\verb{dôdôdô}{}{[dodoˈdo]}{(adj.)}{1}{}{Fofo.}{}{}
\verb{dôdôdô}{}{[dodoˈdo]}{(adj.)}{2}{}{Querido.}{}{}
\verb{dôdô-dôdô}{}{[doˈdo doˈdo]}{(adv.)}{1}{}{Amalucadamente.}{}{}%
\verb{dôdô-dôdô}{}{[doˈdo doˈdo]}{(adv.)}{2}{}{Desnorteadamente.}{}{}
\verb{dôkê}{}{[ˈdoke]}{(conj.)}{1}{}{Do que.}{\textbf{Ê sa maxi bluku dôkê
ami}. \textit{Ele é mais falso do que eu}.}{}
\verb{dôlô}{}{[ˈdolo]}{(n.)}{1}{}{Dor.}{}{}
\verb{dôlô}{}{[doˈlo]}{(v.)}{1}{}{Acariciar.}{}{}
\verb{dôlô}{}{[doˈlo]}{(v.)}{2}{}{Aconchegar.}{}{}
\verb{dôlô}{}{[doˈlo]}{(v.)}{2}{}{Roçar suavemente.}{}{}
\verb{doma}{}{[dɔˈma]}{(v.)}{1}{}{Amansar.}{}{}
\verb{doma}{}{[dɔˈma]}{(v.)}{2}{}{Domar.}{}{}
\verb{dombo}{}{[d\~ɔˈbɔ]}{(n.)}{1}{}{Ramos verdes da palmeira.}{}{}
\verb{domine}{}{[dɔˈminɛ]}{(n.)}{1}{}{Sábio.}{}{}
\verb{dona}{}{[ˈdɔna]}{(n.)}{1}{}{Avó.}{}{}
\verb{dona-nglandji}{}{[ˈdɔna ˈŋgl\~{\textturna}dʒi]}{(n.)}{1}{}{Bisavó.}{}{}\verb{dondolo}{}{[d\~ɔdɔˈlɔ]}{(n.)}{1}{}{\textit{Dondolo}.}{Dança, ligada aos
espíritos, executada durante o \textbf{djambi}.}{}
\verb{dongodongo}{}{[d\~ɔˈgɔd\~ɔˈgɔ]}{(adj.)}{1}{}{Pegajoso.}{}{}
\verb{dongodongo}{}{[d\~ɔˈgɔd\~ɔˈgɔ]}{(adj.)}{2}{}{Ranhoso.}{}{}
\verb{dongodongo}{}{[d\~ɔˈgɔd\~ɔˈgɔ]}{(adj.)}{3}{}{Viscoso.}{}{}
\verb{dongodongo}{}{[d\~ɔˈgɔd\~ɔˈgɔ]}{(n.)}{1}{}{Baba.}{}{}
\verb{dongodongo}{}{[d\~ɔˈgɔd\~ɔˈgɔ]}{(n.)}{1}{}{Ranho.}{}{}
\verb{dongô-moli}{}{[ˈdõgo ˈmɔli]}{(adj.)}{1}{}{Apalermado.}{}{}
\verb{donoxa}{}{[dɔnɔˈʃa]}{(v.)}{1}{}{Atrofiar.}{}{}
\verb{donoxa}{}{[dɔnɔˈʃa]}{(v.)}{2}{}{Definhar.}{}{}%
\verb{donoxa}{}{[dɔnɔˈʃa]}{(v.)}{3}{}{Destruir.}{}{}%
\verb{donoxa}{}{[dɔnɔˈʃa]}{(v.)}{4}{}{Esmorecer.}{}{}
\verb{donoxadu}{}{[dɔnɔˈʃadu]}{(adj.)}{1}{}{Atrofiado.}{}{}
\verb{donoxadu}{}{[dɔnɔˈʃadu]}{(adj.)}{1}{}{Raquítico.}{}{}
\verb{donoxadu}{}{[dɔnɔˈʃadu]}{(adj.)}{2}{}{Subdesenvolvido.}{}{}
\verb{donu}{}{[ˈdɔnu]}{(n.)}{1}{}{Avô.}{}{}
\verb{donu}{}{[ˈdɔnu]}{(n.)}{2}{}{Dono.}{}{}
\verb{donu-nglandji}{}{[ˈdɔnu ˈŋgl\~{\textturna}dʒi]}{(n.)}{1}{}{Bisavô.}{}{}\verb{dôsu}{}{[ˈdosu]}{(num.)}{1}{}{Dois.}{}{}
\verb{dôsu-dexi}{}{[ˈdosu ˈdɛʃi]}{(num.)}{1}{}{Vinte.}{Cf.
\textbf{vintxi}.}{}
\verb{dôtôlô}{}{[doˈtolo]}{(n.)}{1}{}{Doutor.}{}{}
\verb{dôtôlô}{}{[doˈtolo]}{(n.)}{2}{}{Médico.}{}{}%
\verb{doxi}{}{[ˈdɔʃi]}{(adj.)}{1}{}{Agradável.}{}{}%
\verb{doxi}{}{[ˈdɔʃi]}{(adj.)}{2}{}{Belo.}{}{}
\verb{doxi}{}{[ˈdɔʃi]}{(adj.)}{3}{}{Bom.}{}{}%
\verb{doxi}{}{[ˈdɔʃi]}{(adj.)}{4}{}{Delicioso.}{}{}%
\verb{doxi}{}{[ˈdɔʃi]}{(adj.)}{5}{}{Doce.}{}{}%
\verb{doxi}{}{[ˈdɔʃi]}{(n.)}{1}{}{Bolo.}{}{}%
\verb{doxi}{}{[ˈdɔʃi]}{(n.)}{2}{}{Doce.}{}{}
\verb{doxi menemene}{}{[ˈdɔʃi mɛˈnɛmɛˈnɛ]}{(expr.)}{1}{}{Dulcíssimo.}{}{}
\verb{dôzê}{}{[ˈdoze]}{(num.)}{1}{}{Doze.}{}{}
\verb{dudji}{}{[ˈdudʒi]}{(n.)}{1}{}{Comida sem acompanhamento.}{}{}
\verb{dudji}{}{[ˈdudʒi]}{(adv.)}{1}{}{Debalde.}{}{}
\verb{dudu}{}{[ˈdudu]}{(n.)}{1}{}{Bilha.}{}{}%
\verb{dudu}{}{[ˈdudu]}{(n.)}{2}{}{Jarro.}{}{}%
\verb{dudu}{}{[ˈdudu]}{(n.)}{3}{}{Pote.}{}{}
\verb{dudu}{}{[ˈdudu]}{(n.)}{4}{}{Vaso.}{}{}
\verb{dududu}{}{[duduˈdu]}{(id.)}{1}{}{Cf. \textbf{xa dududu}.}{}
\verb{duji}{}{[ˈduʒi]}{(adj.)}{1}{}{Inútil.}{}{}
\verb{dujiduji}{}{[ˈduʒiˈduʒi]}{(adv.)}{1}{}{Injustificável.}{}{}%
\verb{dujiduji}{}{[ˈduʒiˈduʒi]}{(adv.)}{2}{}{Sem motivo.}{}{}
\verb{duki}{}{[ˈduki]}{(n.)}{1}{}{Duque.}{}{}
\verb{dukunu}{}{[dukuˈnu]}{(v.)}{1}{}{Arruinar financeiramente.}{}{}%
\verb{dukunu}{}{[dukuˈnu]}{(v.)}{2}{}{Desgraçar.}{}{}%
\verb{dukunu}{}{[dukuˈnu]}{(v.)}{3}{}{Espoliar.}{}{}
\verb{dumba}{}{[dũˈba]}{(v.)}{2}{}{Adubar.}{}{}
\verb{dumba}{}{[dũˈba]}{(v.)}{1}{}{Amontoar.}{}{}%
\verb{dumbu}{}{[dũˈbu]}{(n.)}{1}{}{\textit{Dumbu}.}{\textbf{\textit{Solanum
americanum}.}}{}
\verb{dumini}{}{[ˈdumini]}{(v.)}{1}{}{Dormir.}{}{}
\verb{dumu}{}{[ˈdumu]}{(n.)}{1}{}{Mão de pilão.}{}{}
\verb{dumu}{}{[duˈmu]}{(v.)}{1}{}{Acabar.}{}{}
\verb{dumu}{}{[duˈmu]}{(v.)}{2}{}{Encerrar.}{}{}
\verb{dumu}{}{[duˈmu]}{(v.)}{3}{}{Finalizar.}{}{}
\verb{dumu}{}{[duˈmu]}{(v.)}{4}{}{Parar.}{}{}%
\verb{dumu}{}{[duˈmu]}{(v.)}{5}{}{Terminar.}{}{}%
\verb{dumu}{}{[duˈmu]}{(v.)}{6}{}{Moer.}{}{}%
\verb{dumu}{}{[duˈmu]}{(v.)}{7}{}{Pilar.}{}{}%
\verb{dumudu}{}{[duˈmudu]}{(adj.)}{1}{}{Moído.}{}{}%
\verb{dumudu}{}{[duˈmudu]}{(adj.)}{2}{}{Pilado.}{}{}%
\verb{dumudu}{}{[duˈmudu]}{(adj.)}{3}{}{Pisado.}{}{}%
\verb{dumu-dumu}{}{[ˈdumu ˈdumu]}{(adj.)}{1}{}{Grosso.}{}{}
\verb{dumu-mwala}{}{[ˈdumu ˈmwala]}{(n.)}{1}{}{\textit{Dumu}
vermelho.}{\textbf{\textit{Ouratea vogelii}.}}{}
\verb{duzentu}{}{[duˈzẽtu]}{(num.)}{1}{}{Duzentos.}{}{}
\verb{dwala}{}{[ˈdwala]}{(n.)}{1}{}{\textit{Dwala}.}{Dança tradicional.}{}
\verb{dwê}{}{[ˈdwe]}{(v.)}{1}{}{Doer.}{}{}
\verb{dwentxi}{}{[ˈdwẽtʃi]}{(adj.)}{1}{}{Doente.}{}{}%
\verb{dwentxi}{}{[ˈdwẽtʃi]}{(n.)}{1}{}{Doença.}{}{}
\verb{dwentxi-bluku}{}{[ˈdwẽtʃi ˈbluku]}{(n.)}{1}{}{Doença incurável.}{}{}
\verb{dwentxi nfelumu}{}{[ˈdwẽtʃi ˈnfɛlumu]}{(expr.)}{1}{}{Muito doente.}{}{}\verb{dwentxi kwenkwenkwen}{}{[ˈdwẽtʃi kw\~ɛkw\~ɛˈkw\~ɛ]}{(expr.)}{1}{}{Muito doente.}{}{}
\verb{dwentxi kwenkwenkwen}{}{[ˈdwẽtʃi
kw\~ɛkw\~ɛˈkw\~ɛ]}{(expr.)}{1}{}{Doença prolongada.}{}{}
\verb{dyeta}{}{[ˈdjɛta]}{(n.)}{1}{}{Dieta.}{}{}
\end{letra}

\begin{letra}{e}


\verb{ê}{}{[ˈe]}{(part.)}{1}{}{Ó.}{\textbf{Inen mosu ê, a bili wê ô!}
{\textit{Ó rapazes, abram os olhos!}}}{}
\verb{ê}{}{[ˈe]}{(part.)}{2}{}{Partícula final.}{\textbf{Lega mu pa n be mu
ê!} \textit{Larga-me para eu me ir embora!}}{}
\verb{ê}{}{[ˈe]}{(pron.)}{1}{}{-a.}{Terceira pessoa do singular com a função
de complemento direto.}{}{}
\verb{ê}{}{[ˈe]}{(pron.)}{2}{}{-o.}{Terceira pessoa do singular com a função
de complemento direto.}{}{}
\verb{ê}{}{[ˈe]}{(pron.)}{3}{}{Ela.}{}{}
\verb{ê}{}{[ˈe]}{(pron.)}{4}{}{Ele.}{}{}
\verb{ê}{}{[ˈe]}{(pron.)}{5}{}{-lhe.}{Terceira pessoa do singular com a
função de complemento indireto.}{}{}
\verb{e}{}{[ˈɛ]}{(adv.)}{1}{}{Sim.}{}{}
\verb{e}{}{[ˈɛ]}{(pron.)}{1}{}{-a.}{Terceira pessoa do singular com a função
de complemento direto.}{}{}
\verb{e}{}{[ˈɛ]}{(pron.)}{2}{}{-lhe.}{Terceira pessoa do singular com a
função de complemento indireto.}{\textbf{N fad'e kwa se}. \textit{Eu
disse-lhe isso}.}{}{}
\verb{e}{}{[ˈɛ]}{(pron.)}{3}{}{-o.}{Terceira pessoa do singular com a função
de complemento direto.}{\textbf{Sun padê xka ba butxiz’e}. \textit{O senhor
padre vai baptizá-lo}.}{}{}%
\verb{ê byê pema}{}{[eˈbje ˈpɛma]}{(expr.)}{1}{}{Adeus!}{}{}
\verb{êê}{}{[eˈe]}{(adv.)}{1}{}{Sim.}{Resposta que se dá quando alguém chama
por nós.}{}{}
\verb{efan}{}{[ɛˈf\~{\textturna}]}{(adv.)}{2}{}{Certamente.}{}{}%
\verb{efan}{}{[ɛˈf\~{\textturna}]}{(adv.)}{4}{}{De fato.}{}{}{}
\verb{efan}{}{[ɛˈf\~{\textturna}]}{(adv.)}{3}{}{Efetivamente.}{}{}{}
\verb{efan}{}{[ɛˈf\~{\textturna}]}{(adv.)}{1}{}{Sim.}{}{}%
\verb{ejitu}{}{[ˈɛʒitu]}{(n.)}{1}{}{Habilidade.}{}{}
\verb{ejitu}{}{[ˈɛʒitu]}{(n.)}{1}{}{Jeito.}{}{}
\verb{eku}{}{[ˈɛku]}{(n.)}{1}{}{Eco.}{}{}%
\verb{êlê}{}{[ˈele]}{(pron.)}{1}{}{-a.}{Terceira pessoa do singular com a
função de complemento direto. {\textbf{Da mu êlê}. \textit{Dê-ma}.}}{}%
\verb{êlê}{}{[ˈele]}{(pron.)}{2}{}{Ela.}{\textbf{Êlê ten ka fe kume dê}.
\textit{Ela também faz a sua comida}.}{}{}{}
\verb{êlê}{}{[ˈele]}{(pron.)}{3}{}{Ele.}{}{}{}{}
\verb{êlê}{}{[ˈele]}{(pron.)}{4}{}{-o.}{Terceira pessoa do singular com a
função de complemento direto.}{\textbf{Da mu êlê}. \textit{Dê-mo}.}{}{}{}
%\verb{êlê manda}{}{[ˈelem\~{\textturna}ˈda]}{(conj.)}{1}{}{É por isso que.}{}{}
\verb{êlê manda}{}{[ˈelem\~{\textturna}ˈda]}{(conj.)}{1}{}{Por isso.}{}{}
\verb{eli}{}{[ˈɛli]}{(n.)}{1}{}{Brilho.}{}{}
\verb{eli}{}{[ˈɛli]}{(n.)}{2}{}{Luminosidade.}{}{}
\verb{eli}{}{[ˈɛli]}{(n.)}{3}{}{Reflexo da lua.}{}{}
\verb{êmisôra}{}{[emiˈsora]}{(n.)}{1}{}{Emissora de rádio.}{}{}
\verb{en}{}{[ˈẽ]}{(part.)}{1}{}{Partícula que enfatiza expressões
deíticas.}{\textbf{Ê xê dai myole-myole en!} \textit{Ele saiu daqui agora
mesmo!}}{}
\verb{enjolo}{}{[ẽʒɔˈlɔ]}{(n.)}{1}{}{Anjolo.}{\textbf{\textit{Neospiza
concolor}.}}{}
\verb{enkwantu}{}{[\~ɛˈk\~w\~{\textturna}tu]}{(conj.)}{1}{}{Enquanto.}{}{}
\verb{êya}{}{[ˈeja]}{(interj.)}{1}{}{Ei!}{Interjeição para chamar a atenção
de alguém.}{}{}
\end{letra}

\begin{letra}{f}

\verb{fa}{}{[ˈfa]}{(neg.)}{1}{}{Partícula de negaç\~{\textturna}o correlativa
de \textbf{na} ou no fim de oração.}{\textbf{N na sêbê fa}. \textit{Não
sei}.}{}
\verb{fa}{}{[ˈfa]}{(part.)}{2}{}{Partícula de asserção.}{Cf.
\textbf{fan}.}{}
\verb{fããã}{}{[ˈf\~{\textturna}\~{\textturna}\~{\textturna}]}{(id.)}{1}{}{Cf.
\textbf{mulatu fããã.}}{}
\verb{fablika}{}{[ˈfablika]}{(n.)}{1}{}{Fábrica.}{}{}
\verb{fada}{}{[ˈfada]}{(n.)}{1}{}{Farda.}{}{}
\verb{fada}{}{[faˈda]}{(v.)}{1}{}{Dizer.}{}{}
\verb{faka}{}{[ˈfaka]}{(n.)}{1}{}{Faca.}{}{}
\verb{faka-fôgon}{}{[ˈfaka foˈgõ]}{(n.)}{1}{}{Faca de cozinha.}{}{}
\verb{fakatxa}{}{[fakaˈtʃa]}{(v.)}{1}{}{Sujar.}{}{}
\verb{fakon}{}{[faˈk\~ɔ]}{(n.)}{1}{}{Facão.}{}{}
\verb{fala}{}{[ˈfala]}{(n.)}{1}{}{Fala.}{}{}%
\verb{fala}{}{[ˈfala]}{(n.)}{2}{}{Palavra.}{}{}
\verb{fala-tendê}{}{[ˈfala tẽˈde]}{(n.)}{1}{}{Boato.}{}{}%
\verb{fala-tendê}{}{[ˈfala tẽˈde]}{(n.)}{2}{}{Rumor.}{}{}%
\verb{fala-tendê}{}{[ˈfala tẽˈde]}{(n.)}{3}{}{Sussurro.}{}{}
\verb{fala vonte dêsu}{}{[ˈfala v\~ɔˈtɛ ˈdesu]}{(expr.)}{1}{}{Viveram felizes
para sempre.}{}{}
\verb{falu}{}{[ˈfalu]}{(n.)}{1}{}{Esconderijo.}{}{}
\verb{falufalu}{}{[faˈlufaˈlu]}{(adv.)}{1}{}{De forma dissimulada.}{}{}
\verb{faluza}{}{[faˈluza]}{(n.)}{1}{}{Ferrugem.}{}{}
\verb{fama}{}{[ˈfama]}{(n.)}{1}{}{Fama.}{}{}%
\verb{fama}{}{[ˈfama]}{(n.)}{2}{}{Nome.}{}{}%
\verb{fama}{}{[ˈfama]}{(n.)}{3}{}{Prestígio.}{}{}
\verb{fama}{}{[ˈfama]}{(n.)}{4}{}{Renome.}{}{}
\verb{famadu}{}{[faˈmadu]}{(adj.)}{1}{}{Afamado.}{}{}
\verb{famadu}{}{[faˈmadu]}{(adj.)}{2}{}{Famoso.}{}{}
\verb{familya}{}{[faˈmilja]}{(n.)}{1}{}{Família.}{}{}
\verb{fan}{}{[ˈf\~{\textturna}]}{(part.)}{2}{}{Partícula de
asserção.}{\textbf{Ê ka têndê fôlô fan}. \textit{De fato, ele compreende
santome}.}{}
\verb{fan}{}{[ˈf\~{\textturna}]}{(part.)}{1}{}{Partícula
imperativa.}{\textbf{Kaboka fan.} \textit{Cala-te!}}{}
\verb{fana}{}{[faˈna]}{(v.)}{1}{}{Abrir.}{}{}%
\verb{fana}{}{[faˈna]}{(v.)}{2}{}{Arregalar.}{}{}%
\verb{fana}{}{[faˈna]}{(v.)}{1}{}{Desatar.}{}{}
\verb{fana}{}{[faˈna]}{(v.)}{1}{}{Surrupiar.}{}{}
\verb{fanalixi}{}{[fanaˈliʃi]}{(n.)}{1}{}{Morcego de nariz
chato.}{\textbf{\textit{Hipposideros thomensis}.}}{}
\verb{fana wê}{}{[faˈna ˈwe]}{(expr.)}{1}{}{Acordar.}{}{}%
\verb{fanha}{}{[f\~{\textturna}ˈɲa]}{(n.)}{1}{}{Farinha.}{}{}
\verb{fanha-mandjoka}{}{[f\~{\textturna}ˈɲa
m\~{\textturna}ˈdjɔka]}{(n.)}{1}{}{Farinha de mandioca.}{}{}
\verb{fanha-min}{}{[f\~{\textturna}ˈɲa ˈm\~i]}{(n.)}{1}{}{Farinha de
milho.}{}{}
\verb{fanha-min}{}{[f\~{\textturna}ˈɲa ˈm\~i]}{(n.)}{1}{}{Fuba.}{}{}
\verb{fanha-min}{}{[f\~{\textturna}ˈɲa ˈm\~i]}{(n.)}{1}{}{Fubá.}{}{}
\verb{fanha-putuga}{}{[f\~{\textturna}ˈɲa putuˈga]}{(n.)}{1}{}{Farinha de
trigo.}{}{}
\verb{fanha-sevada}{}{[f\~{\textturna}ˈɲa sɛˈvada]}{(n.)}{1}{}{Farinha de
cevada.}{}{}
\verb{fanha-tligu}{}{[f\~{\textturna}ˈɲa ˈtligu]}{(n.)}{1}{}{Farinha de
trigo.}{}{}
\verb{fante}{}{[f\~{\textturna}ˈtɛ]}{(n.)}{1}{}{Desdém.}{}{}
\verb{fante}{}{[f\~{\textturna}ˈtɛ]}{(n.)}{2}{}{Mania.}{}{}
\verb{fasu}{}{[ˈfasu]}{(adj.)}{1}{}{Falso.}{}{}%
\verb{fasu}{}{[ˈfasu]}{(adj.)}{2}{}{Inseguro.}{}{}
\verb{fasu}{}{[ˈfasu]}{(n.)}{1}{}{Calúnia.}{}{}
\verb{fasu}{}{[ˈfasu]}{(n.)}{1}{}{Falsidade.}{}{}
\verb{fasu}{}{[ˈfasu]}{(n.)}{2}{}{Mentira.}{}{}
\verb{fasufasu}{}{[ˈfasuˈfasu]}{(adj.)}{1}{}{Débil.}{}{}
\verb{fasufasu}{}{[ˈfasuˈfasu]}{(adj.)}{2}{}{Frágil.}{}{}
\verb{fasufasu}{}{[ˈfasuˈfasu]}{(adj.)}{2}{}{Inconsistente.}{}{}
\verb{fata}{}{[faˈta]}{(v.)}{1}{}{Faltar.}{}{}
\verb{fata}{}{[faˈta]}{(v.)}{2}{}{Não estar presente.}{}{}
\verb{fata}{}{[faˈta]}{(v.)}{3}{}{Ser insuficiente.}{}{}
\verb{fata}{}{[ˈfata]}{(n.)}{2}{}{Ausência.}{}{}
\verb{fata}{}{[ˈfata]}{(n.)}{3}{}{Carência.}{}{}
\verb{fata}{}{[ˈfata]}{(n.)}{1}{}{Falta.}{}{}
\verb{fatu}{}{[ˈfatu]}{(n.)}{1}{}{Fato.}{}{}%
\verb{fatu}{}{[ˈfatu]}{(n.)}{2}{}{Terno.}{}{}
\verb{fatu-banhu}{}{[ˈfatu ˈb\~{\textturna}ɲu]}{(n.)}{1}{}{Fato de
banho.}{}{}%
\verb{fatula}{}{[faˈtula]}{(n.)}{2}{}{Fatura.}{}{}
\verb{fatula}{}{[faˈtula]}{(n.)}{1}{}{Fratura.}{}{}
\verb{favôlô}{}{[faˈvolo]}{(n.)}{1}{}{Favor.}{}{}
\verb{faxa}{}{[faˈʃa]}{(v.)}{1}{}{Enfastiar.}{}{}
\verb{faxadu}{}{[faˈʃadu]}{(adj.)}{1}{}{Enfastiado.}{}{}
\verb{faxadu}{}{[faˈʃadu]}{(adj.)}{2}{}{Farto.}{}{}
\verb{faxi}{}{[ˈfaʃi]}{(adj.)}{1}{}{Encostado.}{}{}
\verb{faxi}{}{[ˈfaʃi]}{(n.)}{1}{}{Abertura.}{}{}
\verb{faxi}{}{[ˈfaʃi]}{(n.)}{1}{}{Face.}{}{}
\verb{faxi}{}{[ˈfaʃi]}{(n.)}{1}{}{Fenda.}{}{}
\verb{faxi}{}{[ˈfaʃi]}{(n.)}{1}{}{Fresta.}{}{}
\verb{faxixta}{}{[faˈʃiʃta]}{(adj.)}{1}{}{Fascista.}{}{}
\verb{faxtozu}{}{[faʃˈtɔsu]}{(adj.)}{1}{}{Caprichoso.}{}{}
\verb{faxtozu}{}{[faʃˈtɔsu]}{(adj.)}{2}{}{Teimoso.}{}{}
\verb{fazenda}{}{[faˈzẽda]}{(n.)}{1}{}{Empresa.}{}{}%
\verb{fazenda}{}{[faˈzẽda]}{(n.)}{2}{}{Fazenda.}{}{}%
\verb{fazenda}{}{[faˈzẽda]}{(n.)}{3}{}{Finanças.}{}{}
\verb{fazenda}{}{[faˈzẽda]}{(n.)}{4}{}{Tecido.}{}{}%
\verb{fê}{}{[ˈfe]}{(adj.)}{1}{}{Feio.}{}{}
\verb{fe}{}{[ˈfɛ]}{(v.)}{2}{}{Construir.}{}{}
\verb{fe}{}{[ˈfɛ]}{(v.)}{1}{}{Fazer.}{}{}
\verb{fe}{}{[ˈfɛ]}{(v.)}{3}{}{Parecer.}{}{}
\verb{fe bengula}{}{[ˈfɛ b\~ɛˈgula]}{(expr.)}{1}{}{Ter relações sexuais.}{}{}
\verb{feble}{}{[ˈfɛblɛ]}{(n.)}{1}{}{Febre.}{}{}
\verb{fede}{}{[ˈfɛdɛ]}{(n.)}{1}{}{Mau cheiro.}{}{}
\verb{fede}{}{[fɛˈdɛ]}{(v.)}{2}{}{Cheirar mal.}{}{}
\verb{fede}{}{[fɛˈdɛ]}{(v.)}{1}{}{Feder.}{}{}%
\verb{fêdêgôsu}{}{[fedeˈgosu]}{(n.)}{1}{}{Fedegoso.}{\textbf{\textit{Cassia occidentalis}.}}{}
\verb{fêdêgôsu}{}{[fedeˈgosu]}{(n.)}{2}{}{Maioba.}{\textbf{\textit{Cassiaoccidentalis}.}}{}
\verb{fede kalumanu}{}{[fɛˈdɛ kaluˈmanu]}{(expr.)}{1}{}{Cheirar a carne humana em decomposição.}{}{}
\verb{fede kulumanu}{}{[fɛˈdɛ kuluˈmanu]}{(expr.)}{2}{}{Cheirar a carne humana em decomposição.}{Cf. \textbf{fede kalumanu}.}{}
\verb{fe demanda}{}{[ˈfɛ dɛˈm\~{\textturna}da]}{(expr.)}{1}{}{Discutir.}{}{}
\verb{fe demanda}{}{[ˈfɛ dɛˈm\~{\textturna}da]}{(expr.)}{2}{}{Disputar.}{}{}
\verb{fede tuntuntun}{}{[fɛˈdɛ ˈt\~ut\~ut\~u]}{(expr.)}{1}{}{Cheirar muito
mal.}{}{}
\verb{fe dja}{}{[ˈfɛ ˈdʒa]}{(expr.)}{1}{}{Amanhecer.}{}{}
\verb{fedô}{}{[fɛˈdo]}{(n.)}{1}{}{Artista.}{}{}
\verb{fedô}{}{[fɛˈdo]}{(n.)}{1}{}{Fazedor.}{}{}
\verb{fedô-ben}{}{[fɛˈdo ˈb\~ɛ]}{(n.)}{1}{}{Benfeitor.}{}{}
\verb{fedô-kume}{}{[fɛˈdo kuˈmɛ]}{(n.)}{1}{}{Cozinheiro.}{}{}
\verb{fedô-lata}{}{[fɛˈdo ˈlata]}{(n.)}{1}{}{Funileiro.}{}{}
\verb{fedô-lata}{}{[fɛˈdo ˈlata]}{(n.)}{2}{}{Latoeiro.}{}{}%
\verb{fedô-mali}{}{[fɛˈdo ˈmali]}{(n.)}{1}{}{Malfeitor.}{}{}
\verb{fedu}{}{[ˈfɛdu]}{(adj.)}{1}{}{Feito.}{}{}
\verb{fefe}{}{[fɛˈfɛ]}{(v.)}{1}{}{Chupar.}{}{}
\verb{fefe}{}{[fɛˈfɛ]}{(v.)}{2}{}{Comer e mastigar bem.}{}{}
\verb{fe fina}{}{[ˈfɛ ˈfina]}{(expr.)}{1}{}{Armar-se em pessoa fina.}{}{}
\verb{fe finta}{}{[ˈfɛ ˈf\~ita]}{(expr.)}{1}{}{Contribuir.}{}{}
\verb{fe finta}{}{[ˈfɛ ˈf\~ita]}{(expr.)}{2}{}{Cooperar.}{}{}
\verb{fe finta}{}{[ˈfɛ ˈf\~ita]}{(expr.)}{3}{}{Cotizar.}{}{}
\verb{fe fitxin}{}{[ˈfɛ fiˈtʃ\~i]}{(expr.)}{1}{}{Fazer intrigas.}{}{}
\verb{fe fya}{}{[ˈfɛ ˈfja]}{(expr.)}{1}{}{Namorar.}{}{}
\verb{fêgula}{}{[feˈgula]}{(n.)}{1}{}{Boneco utilizado no ritual do
\textbf{paga-dêvê}.}{}{}{}
%\verb{fêgula}{}{[feˈgula]}{(n.)}{1}{}{Estátua.}{}{}
\verb{fêgula}{}{[feˈgula]}{(n.)}{2}{}{Figura.}{}{}
\verb{fe keda}{}{[ˈfɛ ˈkɛda]}{(expr.)}{1}{}{Equilibrar(-se).}{}{}
\verb{fe kompanhe}{}{[ˈfɛ ˈkõpaɲɛ]}{(expr.)}{1}{}{Ter relações sexuais.}{}{}
\verb{fe konta}{}{[ˈfɛ ˈkõta]}{(expr.)}{1}{}{Calcular.}{}{}
\verb{fe kumbinason}{}{[ˈfɛ k\~ubinaˈsõ]}{(expr.)}{1}{}{Combinar.}{}{}
\verb{fela}{}{[ˈfɛla]}{(n.)}{1}{}{Feira.}{}{}%
\verb{fela}{}{[ˈfɛla]}{(n.)}{2}{}{Mercado.}{}{}
\verb{fela}{}{[fɛˈla]}{(v.)}{1}{}{Aquecer.}{}{}%
\verb{fela}{}{[fɛˈla]}{(v.)}{2}{}{Causar ardor.}{}{}
\verb{fela}{}{[fɛˈla]}{(v.)}{3}{}{Estar quente.}{}{}
\verb{fela}{}{[fɛˈla]}{(v.)}{3}{}{Ferrar.}{}{}
\verb{fela}{}{[fɛˈla]}{(v.)}{3}{}{Fortalecer.}{}{}
\verb{fela}{}{[fɛˈla]}{(v.)}{3}{}{Recrudescer.}{}{}
\verb{feladu}{}{[fɛˈladu]}{(adj.)}{1}{}{Aquecido.}{}{}
\verb{feladu}{}{[fɛˈladu]}{(adj.)}{1}{}{Ferrado.}{}{}
\verb{feladu}{}{[fɛˈladu]}{(adj.)}{2}{}{Lascivo.}{}{}
\verb{feladu}{}{[fɛˈladu]}{(n.)}{1}{}{Lascívia.}{}{}
\verb{feladu}{}{[fɛˈladu]}{(n.)}{1}{}{Mulherengo.}{}{}
\verb{fela madlê}{}{[fɛˈla maˈdle]}{(expr.)}{1}{}{Aquecer o útero.}{Prática tradicional pós-parto.}{}%
\verb{fela zuzuzu}{}{[fɛˈla zuzuˈzu]}{(expr.)}{1}{}{Quentíssimo.}{}{}
\verb{felon}{}{[fɛˈl\~ɔ]}{(n.)}{1}{}{Adulto.}{}{}
\verb{felon}{}{[fɛˈl\~ɔ]}{(n.)}{2}{}{Espora.}{}{}
\verb{felon}{}{[fɛˈl\~ɔ]}{(n.)}{3}{}{Ferrão.}{}{}
\verb{felu}{}{[ˈfɛlu]}{(n.)}{1}{}{Ferro.}{}{}%
\verb{felu}{}{[ˈfɛlu]}{(n.)}{2}{}{Pênis.}{}{}
\verb{felumu}{}{[ˈfɛlumu]}{(adj.)}{1}{}{Doente.}{}{}{}
\verb{felumu}{}{[ˈfɛlumu]}{(adj.)}{1}{}{Enfermo.}{Cf. \textbf{nfelumu}.}{}{}
\verb{felu tõõõ}{}{[ˈfɛlu ˈtõõõ]}{(expr.)}{1}{}{Pênis ereto.}{}{}
\verb{fe luvon}{}{[ˈfɛ luˈv\~ɔ]}{(expr.)}{1}{}{Armar-se em valentão.}{}{}
\verb{fe luvon}{}{[ˈfɛ luˈv\~ɔ]}{(expr.)}{1}{}{Fazer-se de forte.}{}{}
\verb{fe luxu}{}{[ˈfɛ ˈluʃu]}{(expr.)}{1}{}{Ostentar.}{}{}
\verb{fe luxu}{}{[ˈfɛ ˈluʃu]}{(expr.)}{1}{}{Requestar.}{}{}
\verb{fêlyadu}{}{[feˈljadu]}{(n.)}{1}{}{Feriado.}{}{}
\verb{fe mangason}{}{[ˈfɛ m\~{\textturna}gaˈs\~ɔ]}{(expr.)}{1}{}{Debochar.}{}{}
\verb{fe mangason}{}{[ˈfɛ m\~{\textturna}gaˈs\~ɔ]}{(expr.)}{2}{}{Fazer troça.}{}{}
\verb{fe mangason}{}{[ˈfɛ m\~{\textturna}gaˈs\~ɔ]}{(expr.)}{3}{}{Insultar.}{}{}
\verb{fe mangason}{}{[ˈfɛ m\~{\textturna}gaˈs\~ɔ]}{(expr.)}{4}{}{Menosprezar.}{}{}
\verb{fe matu}{}{[ˈfɛ ˈmatu]}{(expr.)}{1}{}{Defecar.}{}{}{}
\verb{fe migu}{}{[ˈfɛ ˈmigu]}{(expr.)}{1}{}{Fazer amizade.}{}{}
\verb{fenene}{}{[fɛnɛˈnɛ]}{(id.)}{1}{}{Cf. \textbf{blanku fenene.}}{}
\verb{fenene}{}{[fɛnɛˈnɛ]}{(id.)}{2}{}{Cf. \textbf{klalu fenene.}}{}
\verb{fênêtxiga}{}{[fenetʃiˈga]}{(v.)}{1}{}{Afligir.}{}{}
\verb{fênêtxiga}{}{[fenetʃiˈga]}{(v.)}{2}{}{Incomodar(-se).}{}{}
\verb{fênêtxiga}{}{[fenetʃiˈga]}{(v.)}{2}{}{Irritar(-se).}{}{}
\verb{fênêtxigadu}{}{[fenetʃiˈgadu]}{(adj.)}{1}{}{Aflito.}{}{}%
\verb{fênêtxigadu}{}{[fenetʃiˈgadu]}{(adj.)}{2}{}{Incomodado.}{}{}
\verb{fênêtxigadu}{}{[fenetʃiˈgadu]}{(adj.)}{2}{}{Irritado.}{}{}
\verb{fengêfengê}{}{[f\~eˈgef\~eˈge]}{(adj.)}{1}{}{Magro.}{}{}
\verb{fe odjo}{}{[ˈfɛ ɔˈdʒɔ]}{(expr.)}{1}{}{Odiar.}{}{}
\verb{fe olha klukutu}{}{[ˈfɛ ɔˈʎa klukuˈtu]}{(expr.)}{1}{}{Ignorar.}{}{}
\verb{fe poji}{}{[ˈfɛ ˈpɔʒi]}{(expr.)}{1}{}{Ostentar.}{}{}
\verb{fe poji}{}{[ˈfɛ ˈpɔʒi]}{(expr.)}{2}{}{Requestar.}{}{}
\verb{fe pôkô}{}{[ˈfɛ ˈpoko]}{(expr.)}{1}{}{Caçoar.}{}{}
\verb{fe pôkô}{}{[ˈfɛ ˈpoko]}{(expr.)}{1}{}{Troçar.}{}{}
\verb{ferya}{}{[ˈfɛrja]}{(n.)}{1}{}{Férias.}{}{}
\verb{fesa}{}{[ˈfɛsa]}{(n.)}{1}{}{Festa.}{}{}
\verb{fesu}{}{[ˈfɛsu]}{(n.)}{1}{}{Fecho.}{}{}
\verb{fêsu}{}{[ˈfesu]}{(n.)}{1}{}{Feixe.}{}{}
\verb{fêsu-basôla}{}{[ˈfesu baˈsola]}{(n.)}{1}{}{União.}{}{}
\verb{fêsu-basôla}{}{[ˈfesu baˈsola]}{(n.)}{1}{}{Vassoura tradicional.}{}{}
\verb{fetiu}{}{[feˈtiu]}{(n.)}{1}{}{Feitio.}{}{}
\verb{fêtwa}{}{[feˈtwa]}{(v.)}{1}{}{Arranjar problemas.}{}{}%
\verb{fêtwa}{}{[feˈtwa]}{(v.)}{2}{}{Provocar confusões.}{}{}
\verb{fêtwa}{}{[feˈtwa]}{(v.)}{3}{}{Tentar causar acidentes.}{}{}%
\verb{fêvêlêlu}{}{[feveˈlelu]}{(n.)}{1}{}{Fevereiro.}{}{}
\verb{fe vlegonha}{}{[ˈfɛ vlɛˈg\~ɔɲa]}{(expr.)}{1}{}{Envergonhar(-se).}{}{}
\verb{fe wê-lizu}{}{[ˈfɛ ˈwe ˈlizu]}{(expr.)}{1}{}{Encorajar.}{}{}
\verb{fe zawa}{}{[ˈfɛ ˈzawa]}{(expr.)}{1}{}{Urinar.}{}{}
\verb{fezon}{}{[fɛˈz\~ɔ]}{(n.)}{1}{}{Feijão.}{}{}
\verb{fezon-bongê}{}{[fɛˈz\~ɔ ˈbõge]}{(n.)}{1}{}{Feijão-cutelinho.}{\textit{\textbf{Phaseolus lunatus}}.}{}
\verb{fezon-floli}{}{[fɛˈz\~ɔ ˈflɔli]}{(n.)}{1}{}{Feijão-flor.}{\textit{\textbf{Centrosema pubescens}}.}{}
\verb{fezon-kongô}{}{[fɛˈz\~ɔ ˈkõgo]}{(n.)}{1}{}{Andú.}{\textit{\textbf{Cajanus cajan}}.}{}
\verb{fezon-kongô}{}{[fɛˈz\~ɔ ˈkõgo]}{(n.)}{2}{}{Ervilha do Congo.}{\textit{\textbf{Cajanus cajan}}.}{}
\verb{fezon-makundê}{}{[fɛˈz\~ɔ maˈk\~ude]}{(n.)}{2}{}{Feijão-macondê.}{\textit{\textbf{Vigna unguiculata}}.}{}
\verb{fezon-matu}{}{[fɛˈz\~ɔ ˈmatu]}{(n.)}{2}{}{Feijão-do-mato.}{\textit{\textbf{Clitoria falcata}}.}{}
\verb{fezon-sêsê}{}{[fɛˈz\~ɔ seˈse]}{(n.)}{1}{}{Feijão-frade.}{}{}
\verb{fezon-vêdê}{}{[fɛˈz\~ɔ veˈde]}{(n.)}{1}{}{Feijão-verde.}{}{}
\verb{fi}{}{[ˈfi]}{(n.)}{1}{}{Fio.}{}{}
\verb{fia}{}{[fiˈa]}{(v.)}{1}{}{Aliviar.}{}{}%
\verb{fia}{}{[fiˈa]}{(v.)}{2}{}{Arrefecer.}{}{}%
\verb{fia}{}{[fiˈa]}{(v.)}{4}{}{Consolar.}{}{}%
\verb{fia}{}{[fiˈa]}{(v.)}{3}{}{Esfriar.}{}{}%
\verb{fia}{}{[fiˈa]}{(v.)}{5}{}{Fiar.}{}{}%
\verb{fia}{}{[fiˈa]}{(v.)}{6}{}{Vender a crédito.}{}{}
\verb{fia kôkôkô}{}{[fiˈa kokoˈko]}{(expr.)}{1}{}{Arrefecer completamente.}{}{}
\verb{fibika}{}{[fibiˈka]}{(v.)}{1}{}{Chupar.}{}{}%
\verb{fibika}{}{[fibiˈka]}{(v.)}{2}{}{Sorver.}{}{}%
\verb{fifi}{}{[fiˈfi]}{(adv.)}{1}{}{Aos poucos.}{}{}
\verb{figu-plôkô}{}{[ˈfiguˈploko]}{(n.)}{1}{}{Figo-porco.}{\textbf{\textit{Ficus mucuso}.}}{}
\verb{figu-tôdô}{}{[ˈfigu ˈtodo]}{(n.)}{1}{}{Figo-tordo.}{\textbf{\textit{Ficus punila}.}}{}
\verb{fijiga}{}{[ˈfiʒiga]}{(n.)}{1}{}{Elástico.}{}{}%
\verb{fijiga}{}{[ˈfiʒiga]}{(n.)}{2}{}{Fisga.}{}{}
\verb{fijiga}{}{[ˈfiʒiga]}{(n.)}{3}{}{Flecha.}{}{}
\verb{fijiga}{}{[ˈfiʒiga]}{(n.)}{4}{}{Lança.}{}{}
\verb{fika}{}{[fiˈka]}{(v.)}{1}{}{Deixar.}{}{}
\verb{fika}{}{[fiˈka]}{(v.)}{2}{}{Ficar.}{}{}
\verb{fikadu}{}{[fiˈkadu]}{(adj.)}{1}{}{Deixado.}{}{}
\verb{fikadu}{}{[fiˈkadu]}{(adj.)}{2}{}{Ficado.}{}{}
\verb{fila}{}{[ˈfila]}{(n.)}{1}{}{Fila.}{}{}
\verb{fili}{}{[ˈfili]}{(adj.)}{2}{}{Novo.}{}{}%
\verb{fili}{}{[ˈfili]}{(adj.)}{1}{}{Tenro.}{}{}%
\verb{fili}{}{[ˈfili]}{(adj.)}{3}{}{Verde.}{}{}
\verb{fili}{}{[fiˈli]}{(v.)}{1}{}{Ferir(-se).}{}{}
\verb{fili bagasa}{}{[fiˈli bagaˈsa]}{(expr.)}{1}{}{Ferir(-se) gravemente.}{}{}\verb{filidu}{}{[fiˈlidu]}{(adj.)}{1}{}{Ferido.}{}{}
\verb{filidu nhanhanha}{}{[fiˈlidu ɲaɲaˈɲa]}{(expr.)}{1}{}{Muito ferido.}{}{}\verb{filiji}{}{[fiˈliʒi]}{(adj.)}{1}{}{Feliz.}{}{}
\verb{fili petepete}{}{[ˈfili pɛˈtɛpɛˈtɛ]}{(expr.)}{1}{}{Muito inexperiente.}{}{}
\verb{fili petepete}{}{[ˈfili pɛˈtɛpɛˈtɛ]}{(expr.)}{2}{}{Muito tenro.}{}{}
\verb{fili petepete}{}{[ˈfili pɛˈtɛpɛˈtɛ]}{(expr.)}{3}{}{Novíssimo.}{}{}
\verb{fin}{}{[ˈf\~i]}{(n.)}{1}{}{Fim.}{}{}%
\verb{fin}{}{[ˈf\~i]}{(n.)}{1}{}{Final.}{}{}
\verb{fina}{}{[fiˈna]}{(v.)}{1}{}{Afinar.}{}{}%
\verb{fina}{}{[ˈfina]}{(adj.)}{3}{}{Bom.}{}{}%
\verb{fina}{}{[ˈfina]}{(adj.)}{1}{}{Fino.}{}{}%
\verb{fina}{}{[ˈfina]}{(adj.)}{2}{}{Elegante.}{}{}%
\verb{fina}{}{[ˈfina]}{(adj.)}{4}{}{Importante.}{}{}
\verb{fina lekeleke}{}{[ˈfina lɛˈkɛlɛˈkɛ]}{(expr.)}{1}{}{Excelente.}{}{}
\verb{fina lekeleke}{}{[ˈfina lɛˈkɛlɛˈkɛ]}{(expr.)}{2}{}{Ótimo.}{}{}
\verb{finansa}{}{[fiˈn\~{\textturna}sa]}{(n.)}{1}{}{Finanças.}{}{}
\verb{finêtê}{}{[fiˈnete]}{(n.)}{1}{}{Alfinete.}{}{}
\verb{finfi}{}{[fi\~ˈfi]}{(adj.)}{1}{}{Delgado.}{}{}
\verb{finfi}{}{[fi\~ˈfi]}{(adj.)}{2}{}{Estreito.}{}{}
\verb{finfi}{}{[fi\~ˈfi]}{(adj.)}{3}{}{Fininho.}{}{}
\verb{finfi}{}{[fi\~ˈfi]}{(adj.)}{4}{}{Magro.}{}{}
\verb{fingi}{}{[fĩˈgi]}{(n.)}{1}{}{Ratinho.}{\textbf{\textit{Mus musculus}}.}{}
\verb{fingi lolo}{}{[fĩˈgi lᴐˈlᴐ]}{(expr.)}{1}{}{Pessoa dúbia.}{}
\verb{fingi lolo}{}{[fĩˈgi lᴐˈlᴐ]}{(expr.)}{2}{}{Pessoa fingida.}{}
\verb{finji}{}{[fĩˈʒi]}{(v.)}{1}{}{Fingir(-se).}{}{}
\verb{finjidu}{}{[fĩˈʒidu]}{(adj.)}{2}{}{Falso.}{}{}
\verb{finjidu}{}{[fĩˈʒidu]}{(adj.)}{1}{}{Fingido.}{}{}%
\verb{finka}{}{[fĩˈka]}{(v.)}{1}{}{Fincar.}{}{}
\verb{finka}{}{[fĩˈka]}{(v.)}{2}{}{Firmar.}{}{}
\verb{finkadu}{}{[fĩˈkadu]}{(adj.)}{1}{}{Fincado.}{}{}
\verb{finta}{}{[fĩˈta]}{(v.)}{1}{}{Desviar.}{}{}
\verb{finta}{}{[ˈfĩta]}{(n.)}{2}{}{Contribuição.}{}{}%
\verb{finta}{}{[ˈfĩta]}{(n.)}{3}{}{Cooperação.}{}{}
\verb{finta}{}{[ˈfĩta]}{(n.)}{1}{}{Quotização.}{}{}%
\verb{fintadu}{}{[fĩˈtadu]}{(adj.)}{1}{}{Desviado.}{}{}
\verb{fisa}{}{[fiˈsa]}{(v.)}{1}{}{Fechar.}{}{}%
\verb{fisa}{}{[fiˈsa]}{(v.)}{2}{}{Trancar.}{}{}
\verb{fisadu}{}{[fiˈsadu]}{(adj.)}{1}{}{Fechado.}{}{}
\verb{fisadu kôkôkô}{}{[fiˈsadu kokoˈko]}{(expr.)}{1}{}{Apertadíssimo.}{}{}
\verb{fisadu kôkôkô}{}{[fiˈsadu kokoˈko]}{(expr.)}{2}{}{Completamente
fechado.}{}{}
\verb{fisadula}{}{[fisaˈdula]}{(n.)}{1}{}{Fechadura.}{}{}
\verb{fisandja}{}{[fis\~{\textturna}ˈdʒa]}{(n.)}{1}{}{Adenia.}{\textbf{\textit{Adenia
cissampeloides}.}}{}
\verb{fisandja}{}{[fis\~{\textturna}ˈdʒa]}{(n.)}{1}{}{\textit{Fio-sardinha}.}{\textbf{\textit{Adenia
cissampeloides}.}}{}
\verb{fisandja}{}{[fis\~{\textturna}ˈdʒa]}{(n.)}{1}{}{\textit{Fisandja}.}{\textbf{\textit{Adenia
cissampeloides}.}}{}
\verb{fisa ubwê}{}{[fisuˈbwe]}{(expr.)}{1}{}{Compor(-se).}{}{}
\verb{fisa ubwê}{}{[fisuˈbwe]}{(expr.)}{2}{}{Curar.}{}{}
\verb{fisa ubwê}{}{[fisuˈbwe]}{(n.)}{2}{}{Massagem.}{}{}
\verb{fisu}{}{[ˈfisu]}{(n.)}{1}{}{Emprego.}{}{}%
\verb{fisu}{}{[ˈfisu]}{(n.)}{3}{}{Ritual fúnebre.}{}{}
\verb{fisu}{}{[ˈfisu]}{(n.)}{2}{}{Ofício.}{}{}%
\verb{fisu}{}{[ˈfisu]}{(n.)}{2}{}{Profissão.}{}{}%
\verb{fita}{}{[ˈfita]}{(n.)}{1}{}{Cena.}{}{}
\verb{fita}{}{[ˈfita]}{(n.)}{2}{}{Cenário.}{}{}
\verb{fita}{}{[ˈfita]}{(n.)}{3}{}{Faixa.}{}{}%
\verb{fita}{}{[ˈfita]}{(n.)}{4}{}{Fita.}{}{}%
\verb{fita}{}{[ˈfita]}{(n.)}{5}{}{Tira.}{}{}%
\verb{fi-tlaxi}{}{[ˈfiˈtlaʃi]}{(n.)}{2}{}{Coluna vertebral.}{}{}
\verb{fi-tlaxi}{}{[ˈfiˈtlaʃi]}{(n.)}{1}{}{Espinha dorsal.}{}{}%
\verb{fitôr}{}{[fiˈto{\textfishhookr}]}{(n.)}{1}{}{Feitor.}{}{}%
\verb{fitxin}{}{[fiˈtʃĩ]}{(n.)}{1}{}{Intriga.}{}{}
\verb{fitxin-flôgô}{}{[fiˈtʃĩ floˈgo]}{(n.)}{1}{}{Ilusão da vida.}{}{}
\verb{fitxisêla}{}{[fitʃiˈsela]}{(n.)}{1}{}{Feiticeira.}{}{}
\verb{fitxisêlu}{}{[fitʃiˈselu]}{(n.)}{2}{}{Feiticeiro.}{}{}
\verb{fitxisêlu aze}{}{[fitʃiˈselu aˈzɛ]}{(expr.)}{2}{}{Feiticeiro
temível.}{}{}
\verb{fitxisu}{}{[fiˈtʃisu]}{(n.)}{1}{}{Feitiço.}{}{}
\verb{fixali}{}{[fiˈʃali]}{(n.)}{1}{}{\textit{Fixali}.}{Denominação dos
responsáveis pelos participantes numa festa religiosa.}{}{}
\verb{fla}{}{[ˈfla]}{(v.)}{1}{}{Dizer.}{}{}
\verb{fla}{}{[ˈfla]}{(v.)}{2}{}{Falar.}{}{}%
\verb{fladin}{}{[flaˈd\~i]}{(adv.)}{1}{Ruidosamente.}{}{}{}
\verb{fladô}{}{[flaˈdo]}{(n.)}{1}{}{Falador.}{}{}
\verb{fladu}{}{[ˈfladu]}{(adj.)}{1}{}{Combinado.}{}{}%
\verb{fladu}{}{[ˈfladu]}{(adj.)}{2}{}{Conversado.}{}{}%
\verb{fladu}{}{[ˈfladu]}{(adj.)}{3}{}{Dito.}{}{}
\verb{fladu}{}{[ˈfladu]}{(adj.)}{4}{}{Falado.}{}{}
\verb{fladu}{}{[ˈfladu]}{(adj.)}{5}{}{Famoso.}{}{}
\verb{fla fala}{}{[ˈfla ˈfala]}{(expr.)}{1}{}{Criticar.}{}{}
\verb{fla fala}{}{[ˈfla ˈfala]}{(expr.)}{2}{}{Maldizer.}{}{}
\verb{flaji}{}{[ˈflaʒi]}{(n.)}{1}{}{Frase.}{}{}
\verb{flakêsê}{}{[flakeˈse]}{(v.)}{1}{}{Enfraquecer.}{}{}
\verb{flakêxidu}{}{[flakeˈʃidu]}{(adj.)}{1}{}{Enfraquecido.}{}{}
\verb{flakêxidu}{}{[flakeˈʃidu]}{(adj.)}{2}{}{Fraco.}{}{}
\verb{flakêza}{}{[flaˈkeza]}{(n.)}{1}{}{Fraqueza.}{}{}
\verb{flaki}{}{[ˈflaki]}{(n.)}{1}{}{Fraque.}{}{}
\verb{flakon}{}{[flaˈk\~ᴐ]}{(n.)}{1}{}{Falcão.}{\textbf{\textit{Milvus migrans}.}}{}
\verb{flakon}{}{[flaˈk\~ᴐ]}{(n.)}{2}{}{Milhafre negro.}{\textbf{\textit{Milvus migrans}.}}{}
\verb{flakon-benta}{}{[flaˈk\~ᴐ ˈbẽta]}{(n.)}{1}{}{Falcão-benta.}{Personagem de histórias tradicionais.}{}{}
\verb{flaku}{}{[ˈflaku]}{(adj.)}{1}{}{Covarde.}{}{}
\verb{flaku}{}{[ˈflaku]}{(adj.)}{2}{}{Débil.}{}{}
\verb{flaku}{}{[ˈflaku]}{(adj.)}{3}{}{Fraco.}{}{}
\verb{fla lason}{}{[ˈfla laˈs\~ᴐ]}{(expr.)}{1}{}{Orar.}{}{}
\verb{fla lason}{}{[ˈfla laˈs\~ᴐ]}{(expr.)}{2}{}{Rezar.}{}{}
\verb{fla mantxan}{}{[ˈfla m\~{\textturna}ˈtʃ\~{\textturna}]}{(expr.)}{1}{}{Cumprimentar.}{}{}
\verb{fla mantxan}{}{[ˈfla m\~{\textturna}ˈtʃ\~{\textturna}]}{(expr.)}{2}{}{Saudar.}{}{}%
\verb{flamason}{}{[flamaˈsõ]}{(n.)}{1}{}{Fama.}{}{}
\verb{flamason}{}{[flamaˈsõ]}{(n.)}{2}{}{Reputação.}{}{}
\verb{flamenta}{}{[flaˈmẽta]}{(n.)}{1}{}{Ferramenta.}{}{}
\verb{fla mintxila}{}{[ˈfla m\~{i}ˈtʃila]}{(expr.)}{1}{}{Mentir.}{}{}%
\verb{flanga}{}{[ˈfl\~{\textturna}ga]}{(n.)}{1}{}{Franga.}{Cf. \textbf{nganha-mosa}.}{}{}
\verb{flangin}{}{[fl\~{\textturna}ˈgĩ]}{(n.)}{1}{Pintainho.}{}{}{}
\verb{flangin}{}{[fl\~{\textturna}ˈgĩ]}{(n.)}{1}{Pintinho.}{Cf. \textbf{pinta}.}{}{}
\verb{flankotxi}{}{[fl\~{\textturna}ˈkᴐtʃi]}{(n.)}{1}{}{Pistola.}{}{}
\verb{flansêji}{}{[fl\~{\textturna}ˈseʒi]}{(adj.)}{1}{}{Francês.}{}{}
\verb{flansêji}{}{[fl\~{\textturna}ˈseʒi]}{(n.)}{1}{}{Francês.}{}{}
\verb{fla sasasa}{}{[ˈfla sasaˈsa]}{(expr.)}{1}{}{Falar com fluência.}{}{}
\verb{fla vonvon}{}{[ˈfla ˈvõˈvõ]}{(expr.)}{1}{Falar à toa.}{}{}
\verb{fla vonvon}{}{[ˈfla ˈvõˈvõ]}{(expr.)}{1}{Fofocar.}{}{}
\verb{fla vonvon}{}{[ˈfla ˈvõˈvõ]}{(expr.)}{1}{Mexericar.}{}{}
\verb{flaxkin}{}{[flaʃˈkĩ]}{(n.)}{1}{}{Frasco pequeno.}{}{}
\verb{fle}{}{[ˈflɛ]}{(n.)}{1}{}{Armadilha utilizada para caçar pequenos pássaros ou o \textbf{munken}.}{}{}{}
\verb{flê}{}{[ˈfle]}{(n.)}{1}{}{Freio.}{}{}
\verb{flêbê}{}{[fleˈbe]}{(v.)}{1}{}{Ferver.}{}{}
\verb{flêbê blublublu}{}{[fleˈbe blubluˈblu]}{(expr.)}{1}{}{Entrar em
ebulição.}{}{}
\verb{flebenta}{}{[flɛb\~ɛˈta]}{(v.)}{1}{}{Cozer.}{}{}
\verb{flebenta}{}{[flɛb\~ɛˈta]}{(v.)}{2}{}{Ferventar.}{}{}
\verb{flebentadu}{}{[flɛbẽˈtadu]}{(adj.)}{1}{}{Cozido.}{}{}
\verb{flebentadu}{}{[flɛbẽˈtadu]}{(adj.)}{2}{}{Ferventado.}{}{}
\verb{flêbidu}{}{[fleˈbidu]}{(adj.)}{1}{}{Fervido.}{}{}
\verb{flega}{}{[flɛˈga]}{(v.)}{1}{}{Esfregar.}{}{}%
\verb{flega}{}{[flɛˈga]}{(v.)}{2}{}{Fazer entranhar os temperos.}{}{}
\verb{flega}{}{[flɛˈga]}{(v.)}{3}{}{Massagear.}{}{}
\verb{flega}{}{[flɛˈga]}{(v.)}{4}{}{Massajar.}{}{}
\verb{flegadô}{}{[flɛgaˈdo]}{(n.)}{1}{}{Massagista.}{Terapeuta
tradicional.}{}
\verb{flegadu}{}{[flɛˈgadu]}{(adj.)}{1}{}{Esfregado.}{}{}
\verb{flegadu}{}{[flɛˈgadu]}{(adj.)}{2}{}{Massageado.}{}{}%
\verb{flegadu}{}{[flɛˈgadu]}{(adj.)}{3}{}{Massajado.}{}{}
\verb{flêgê}{}{[fleˈge]}{(n.)}{1}{}{Cliente.}{}{}
\verb{flêgê}{}{[fleˈge]}{(n.)}{2}{}{Freguês.}{}{}
\verb{flêgêja}{}{[flegeˈʒa]}{(n.)}{1}{}{Freguesia.}{}{}
\verb{flegon}{}{[flɛˈg\~ᴐ]}{(n.)}{1}{}{Esfregão.}{}{}
\verb{flêlu}{}{[ˈflelu]}{(n.)}{1}{}{Ferreiro.}{}{}
\verb{flema}{}{[flɛˈma]}{(v.)}{1}{}{Aborrecer.}{}{}
\verb{flema}{}{[flɛˈma]}{(v.)}{1}{}{Provocar.}{}{}
\verb{flêminga}{}{[fleˈmĩga]}{(n.)}{1}{}{Formiga.}{}{}
\verb{flenta}{}{[flẽˈta]}{(v.)}{1}{}{Cozer.}{}{}{}
\verb{flenta}{}{[flẽˈta]}{(v.)}{1}{}{Ferventar.}{Cf. \textbf{flebenta}.}{}{}
\verb{flentadu}{}{[fl\~eˈtadu]}{(n.)}{1}{}{Cozido.}{}{}
\verb{flentadu}{}{[fl\~eˈtadu]}{(n.)}{2}{}{Ferventado.}{}{}
\verb{flêsê}{}{[fleˈse]}{(v.)}{1}{}{Oferecer.}{}{}%
\verb{flêsê}{}{[fleˈse]}{(n.)}{2}{}{Oferecimento.}{Ritual no qual o bebê é
oferecido espiritualmente para receber proteção divina quando crescer.}{}
\verb{flesku}{}{[ˈflɛsku]}{(adj.)}{1}{}{Fresco.}{}{}
\verb{flesku tatata}{}{[ˈflɛsku tataˈta]}{(expr.)}{1}{}{Peixe
fresquíssimo.}{}{}\verb{flida}{}{[ˈflida]}{(n.)}{1}{}{Ferida.}{}{}
\verb{flijana}{}{[fliʒaˈna]}{(n.)}{1}{}{Panela.}{}{}
\verb{fliji}{}{[fliˈʒi]}{(id.)}{1}{}{Cf. \textbf{sola fliji}.}{}{}
\verb{fliji}{}{[fliˈʒi]}{(v.)}{1}{}{Afligir.}{}{}
\verb{fliji}{}{[fliˈʒi]}{(v.)}{2}{}{Fritar.}{}{}
\verb{flijidu}{}{[fliˈʒidu]}{(adj.)}{1}{}{Aflito.}{}{}
\verb{flijidu}{}{[fliˈʒidu]}{(adj.)}{2}{}{Frito.}{}{}
\verb{flima}{}{[fliˈma]}{(v.)}{1}{}{Firmar.}{}{}%
\verb{flima}{}{[fliˈma]}{(v.)}{2}{}{Fixar.}{}{}%
\verb{flima}{}{[fliˈma]}{(v.)}{3}{}{Virar adulto.}{}{}
\verb{flimadu}{}{[fliˈmadu]}{(adj.)}{1}{}{Crescido.}{}{}
\verb{flimi}{}{[ˈflimi]}{(adj.)}{1}{}{Firme.}{}{}%
\verb{flimi}{}{[ˈflimi]}{(adj.)}{2}{}{Fixo.}{}{}
\verb{flipotxi}{}{[fliˈpɔtʃi]}{(n.)}{1}{}{Filispote.}{Alimento preparado à base farinha de mandioca.}{}{}
\verb{floga}{}{[flɔˈga]}{(n.)}{1}{}{Brincadeira.}{}{}
\verb{floga}{}{[flɔˈga]}{(n.)}{2}{}{Diversão.}{}{}
\verb{floga}{}{[flɔˈga]}{(v.)}{1}{}{Brincar.}{}{}
\verb{floga}{}{[flɔˈga]}{(v.)}{2}{}{Desviar.}{}{}
\verb{floga}{}{[flɔˈga]}{(v.)}{3}{}{Divertir(-se).}{}{}
\verb{floga}{}{[flɔˈga]}{(v.)}{4}{}{Esquivar.}{}{}
\verb{floga}{}{[flɔˈga]}{(v.)}{5}{}{Folgar.}{}{}%
\verb{floga}{}{[flɔˈga]}{(v.)}{6}{}{Representar.}{}{}
\verb{flôgô}{}{[ˈflogo]}{(n.)}{1}{}{Fôlego.}{}{}%
\verb{flôgô}{}{[ˈflogo]}{(n.)}{2}{}{Respiração.}{}{}
\verb{flôgô libaliba}{}{[ˈflogo ˈlibaˈliba]}{(expr.)}{1}{}{Respiração
ofegante.}{}{}
\verb{floka}{}{[flɔˈka]}{(v.)}{1}{}{Enforcar.}{}{}
\verb{flokadu}{}{[flɔˈkadu]}{(adj.)}{1}{}{Enforcado.}{}{}
\verb{flokadu}{}{[flɔˈkadu]}{(n.)}{1}{}{Empurrão violento.}{}{}
\verb{floli}{}{[ˈflɔli]}{(n.)}{1}{}{Flor.}{}{}
\verb{floli-ku-dêsu-wanga-pê-mundu}{}{[ˈflɔli ˈku ˈdesu w\~{\textturna}ˈga
ˈpe
ˈm\~udu]}{(n.)}{1}{}{Flor-que-Deus-espalhou-pelo-mundo.}{\textit{\textbf{Gomphrena
globosa}}.}{}
\verb{floma}{}{[flɔˈma]}{(v.)}{1}{}{Formar.}{}{}
\verb{floma-pali}{}{[flɔˈma paˈli]}{(n.)}{1}{}{Sangue coagulado do
parto.}{}{}
\verb{flomenta}{}{[flɔmẽˈta]}{(v.)}{1}{}{Estimular.}{}{}%
\verb{flomenta}{}{[flɔmẽˈta]}{(v.)}{2}{}{Fomentar.}{}{}%
\verb{flomenta}{}{[flɔmẽˈta]}{(v.)}{3}{}{Friccionar com fomento ou
remédio.}{}{}%
\verb{flomenta}{}{[flɔmẽˈta]}{(v.)}{4}{}{Incitar.}{}{}
\verb{flomentu}{}{[flɔˈmẽtu]}{(n.)}{1}{}{Fomentação.}{}{}
\verb{flomentu}{}{[flɔˈmẽtu]}{(n.)}{2}{}{Remédio.}{}{}
\verb{flomentu}{}{[flɔˈmẽtu]}{(n.)}{3}{}{Unguento.}{}{}
\verb{flôminga}{}{[floˈmĩga]}{(n.)}{1}{}{Formiga.}{Cf.
\textbf{flêminga}.}{}{}
\verb{flonha}{}{[ˈflõɲa]}{(n.)}{1}{}{Fronha.}{}{}
\verb{flonta}{}{[ˈfl\~ɔta]}{(n.)}{1}{}{Assédio.}{}{}
\verb{flonta}{}{[ˈfl\~ɔta]}{(n.)}{2}{}{Chatice.}{}{}
\verb{flonta}{}{[ˈfl\~ɔta]}{(n.)}{3}{}{Incômodo.}{}{}
\verb{flonta}{}{[fl\~ɔˈta]}{(v.)}{1}{}{Assediar.}{}{}
\verb{flonta}{}{[fl\~ɔˈta]}{(v.)}{2}{}{Chatear.}{}{}
\verb{flonta}{}{[fl\~ɔˈta]}{(v.)}{3}{}{Incomodar.}{}{}
\verb{flontadu}{}{[fl\~ɔˈtadu]}{(adj.)}{1}{}{Egoísta.}{}{}
\verb{flontadu}{}{[fl\~ɔˈtadu]}{(adj.)}{2}{}{Glutão.}{}{}
\verb{flontadu}{}{[fl\~ɔˈtadu]}{(adj.)}{3}{}{Insaciável.}{}{}
\verb{flontadu}{}{[fl\~ɔˈtadu]}{(adj.)}{4}{}{Sôfrego.}{}{}
\verb{flontadu}{}{[fl\~ɔˈtadu]}{(adj.)}{5}{}{Voraz.}{}{}
\verb{flontadu}{}{[fl\~ɔˈtadu]}{(n.)}{1}{}{Egoísmo.}{}{}
\verb{flontadu}{}{[fl\~ɔˈtadu]}{(n.)}{2}{}{Gula.}{}{}
\verb{flontadu}{}{[fl\~ɔˈtadu]}{(n.)}{3}{}{Sofreguidão.}{}{}
\verb{flontamentu}{}{[fl\~ɔtaˈmẽtu]}{(n.)}{1}{}{Afrontamento.}{}{}
\verb{flontamentu}{}{[fl\~ɔtaˈmẽtu]}{(n.)}{2}{}{Incômodo.}{}{}
\verb{flôxô}{}{[ˈfloʃo]}{(adj.)}{1}{}{Frouxo.}{}{}
\verb{flôxô}{}{[ˈfloʃo]}{(n.)}{1}{}{Buraco.}{}{}
\verb{flukyan}{}{[fluˈkj\~{\textturna}]}{(n.)}{1}{}{Forquilha.}{}{}
\verb{fluta}{}{[ˈfluta]}{(n.)}{1}{}{Fruta-pão.}{Cf. \textbf{fluta-mpon}.}{}{}\verb{fluta-mpon}{}{[ˈfluta
ˈmpõ]}{(n.)}{1}{}{Árvore-do-pão.}{\textbf{\textit{Artocarpus altilis}.}}{}
\verb{fluta-mpon}{}{[ˈfluta ˈmpõ]}{(n.)}{2}{}{Fruta-pão.}{}{}%
\verb{fô}{}{[ˈfo]}{(neg.)}{1}{}{Partícula de negação no fim de oração ou
correlativa de \textbf{na}.}{\textbf{N na tê klupa fô.} \textit{Não tenho
culpa.}}{}
\verb{fô}{}{[ˈfo]}{(prep. v.)}{1}{}{De.}{}{}
\verb{fô}{}{[ˈfo]}{(prep. v.)}{1}{}{Sair de.}{}{}{}
\verb{fô}{}{[ˈfo]}{(prep. v.)}{1}{}{Vir de.}{Cf. \textbf{bi fô}.}{}{}
\verb{fô}{}{[ˈfo]}{(v.)}{1}{}{Afastar(-se).}{}{}
\verb{fô}{}{[ˈfo]}{(v.)}{2}{}{Sair de.}{}{}%
\verb{fôdê}{}{[foˈde]}{(v.)}{1}{}{Foder.}{}{}
\verb{fôfô}{}{[ˈfofo]}{(adj.)}{1}{}{Balofo.}{}{}%
\verb{fôfô}{}{[ˈfofo]}{(adj.)}{2}{}{Fofo.}{}{}%
\verb{fôfô}{}{[ˈfofo]}{(adj.)}{3}{}{Macio.}{}{}%
\verb{fôfô}{}{[ˈfofo]}{(adj.)}{4}{}{Roído pela traça.}{}{}
\verb{fôfô}{}{[ˈfofo]}{(adj.)}{5}{}{Traçado.}{}{}%
\verb{fôfô}{}{[foˈfo]}{(v.)}{1}{}{Peneirar soprando para separar o izaquente
ou o milho da casca.}{}{}{}
\verb{foga}{}{[fɔˈga]}{(v.)}{1}{}{Afogar(-se).}{}{}
\verb{foga}{}{[fɔˈga]}{(v.)}{2}{}{Cansar.}{}{}
\verb{fôgô}{}{[ˈfogo]}{(n.)}{1}{}{Fogo.}{}{}%
\verb{fôgô}{}{[ˈfogo]}{(n.)}{2}{}{Incêndio.}{}{}
\verb{fôgô}{}{[ˈfogo]}{(n.)}{3}{}{Lume.}{}{}%
\verb{fogon}{}{[fɔˈg\~ɔ]}{(n.)}{1}{}{Cozinha.}{}{}%
\verb{fogon}{}{[fɔˈg\~ɔ]}{(n.)}{2}{}{Fogão.}{}{}
\verb{fôgôzu}{}{[foˈgozu]}{(adj.)}{1}{}{Faceiro.}{}{}%
\verb{fôgôzu}{}{[foˈgozu]}{(adj.)}{2}{}{Muito acelerado.}{}{}%
\verb{fôgôzu}{}{[foˈgozu]}{(adj.)}{3}{}{Nervoso.}{}{}
\verb{foko}{}{[fɔˈkɔ]}{(v.)}{1}{}{Afastar.}{}{}
\verb{foko}{}{[fɔˈkɔ]}{(v.)}{2}{}{Esquivar.}{}{}
\verb{foko}{}{[fɔˈkɔ]}{(v.)}{3}{}{Folgar.}{}{}
\verb{fokofoko}{}{[fɔˈkɔfɔˈkɔ]}{(adj.)}{1}{}{Flácido.}{}{}%
\verb{fokofoko}{}{[fɔˈkɔfɔˈkɔ]}{(adj.)}{2}{}{Folgado.}{}{}%
\verb{fokofoko}{}{[fɔˈkɔfɔˈkɔ]}{(adj.)}{3}{}{Largo.}{}{}
\verb{fokoto}{}{[fɔˈkɔtɔ]}{(n.)}{1}{}{Confusão.}{}{}
\verb{fokoto}{}{[fɔˈkɔtɔ]}{(n.)}{2}{}{Mixórdia.}{}{}%
\verb{fokoto}{}{[fɔˈkɔtɔ]}{(n.)}{3}{}{Rebaldaria.}{}{}%
\verb{fokoto}{}{[fɔˈkɔtɔ]}{(n.)}{4}{}{Sarilho.}{}{}%
\verb{fola}{}{[fɔˈla]}{(v.)}{1}{}{Esfolar.}{}{}
\verb{fola}{}{[fɔˈla]}{(v.)}{1}{}{Forrar.}{}{}
\verb{fôlô}{}{[ˈfolo]}{(n.)}{1}{}{Estofo.}{}{}
\verb{fôlô}{}{[ˈfolo]}{(n.)}{2}{}{Forro.}{}{}%
\verb{fôlô}{}{[ˈfolo]}{(n.)}{3}{}{Forro.}{Indivíduo pertencente ao grupo
étnico \textbf{fôlô} de São Tomé e Príncipe.}{}%
\verb{fôlô}{}{[ˈfolo]}{(n.)}{4}{}{Língua dos \textbf{fôlô}.}{Cf.
\textbf{santome}.}{}{}
\verb{folufu}{}{[ˈfɔlufu]}{(n.)}{1}{}{Fósforo.}{}{}
\verb{fomi}{}{[ˈfɔmi]}{(n.)}{1}{}{Fome.}{}{}
\verb{Fonandopo}{}{[fɔˈn\~{\textturna}dɔ ˈpɔ]}{(top.)}{1}{}{Fernão do
Pó.}{}{}
\verb{fono}{}{[ˈfɔnɔ]}{(n.)}{1}{}{Forno.}{}{}
\verb{fono}{}{[fɔˈnɔ]}{(v.)}{2}{}{Arranhar.}{}{}
\verb{fono}{}{[fɔˈnɔ]}{(v.)}{2}{}{Estragar.}{}{}
\verb{fono}{}{[fɔˈnɔ]}{(v.)}{1}{}{Rasgar.}{}{}%
\verb{fonodu}{}{[fɔˈnɔdu]}{(adj.)}{3}{}{Arranhado.}{}{}
\verb{fonodu}{}{[fɔˈnɔdu]}{(adj.)}{3}{}{Estragado.}{}{}
\verb{fonodu}{}{[fɔˈnɔdu]}{(adj.)}{1}{}{Rasgado.}{}{}%
\verb{fonodu}{}{[fɔˈnɔdu]}{(adj.)}{2}{}{Roto.}{}{}%
\verb{fono-ngembu}{}{[ˈfɔnɔ ŋgẽˈbu]}{(n.)}{1}{}{Toca de morcego.}{}{}
\verb{fontxi}{}{[ˈf\~ɔtʃi]}{(n.)}{1}{}{Têmpora.}{}{}
\verb{fontxi}{}{[ˈf\~ɔtʃi]}{(n.)}{1}{}{Testa.}{}{}
\verb{fôsa}{}{[ˈfosa]}{(n.)}{1}{}{Força.}{}{}
\verb{fosa}{}{[ˈfɔsa]}{(n.)}{1}{}{Fossa.}{}{}
\verb{fotxi}{}{[ˈfɔtʃi]}{(adj.)}{1}{}{Forte.}{}{}
\verb{fotxi}{}{[ˈfɔtʃi]}{(n.)}{1}{}{Esquadra da Polícia.}{}{}
\verb{fotxi}{}{[ˈfɔtʃi]}{(n.)}{2}{}{Fortaleza.}{}{}
\verb{fotxi}{}{[ˈfɔtʃi]}{(n.)}{3}{}{Forte.}{}{}%
\verb{fotxi-losada}{}{[ˈfɔtʃi lɔˈsada]}{(n.)}{1}{}{Degredo.}{Referência ao Forte Roçadas, em Angola.}{}
\verb{fotxi-losada}{}{[ˈfɔtʃi lɔˈsada]}{(n.)}{1}{}{Prisão.}{}{}
\verb{foya}{}{[fɔˈja]}{(n.)}{1}{}{Quantidade grande.}{}{}
\verb{foya}{}{[fɔˈja]}{(v.)}{1}{}{Banalizar.}{}{}
\verb{foya}{}{[fɔˈja]}{(v.)}{2}{}{Vulgarizar.}{}{}
\verb{fu}{}{[ˈfu]}{(v.)}{1}{}{Branquear.}{}{}
\verb{fu}{}{[ˈfu]}{(v.)}{2}{}{Limpar.}{}{}
\verb{fuba}{}{[ˈfuba]}{(n.)}{1}{}{Farinha de milho.}{}{}
\verb{fuba}{}{[ˈfuba]}{(n.)}{2}{}{Fubá.}{}{}
\verb{fubuka}{}{[fubuˈka]}{(n.)}{1}{}{\textit{Fubuka}.}{Material do tronco de
uma bananeira em decomposição usado para se fazer uma canaleta para escoar a
água para o \textbf{kôsô}, no processo de produção de óleo de palma.}{}{}
\verb{fubuka}{}{[fubuˈka]}{(v.)}{1}{}{Amolgar.}{}{}%
\verb{fubuka}{}{[fubuˈka]}{(v.)}{2}{}{Amarrotar.}{}{}
\verb{fubuka}{}{[fubuˈka]}{(v.)}{3}{}{Amassar.}{}{}
\verb{fudu}{}{[ˈfudu]}{(adj.)}{1}{}{Claro.}{}{}
\verb{fudu}{}{[ˈfudu]}{(adj.)}{2}{}{Esbranquiçado.}{}{}
\verb{fudu}{}{[ˈfudu]}{(adj.)}{3}{}{Limpo.}{}{}
\verb{fudu txetxetxe}{}{[ˈfudu tʃɛtʃɛˈtʃɛ]}{(expr.)}{1}{}{Asseadíssimo.}{}{}
\verb{fugudu}{}{[fuˈgudu]}{(adj.)}{1}{}{Ofegante.}{}{}%
\verb{fuji}{}{[fuˈʒi]}{(v.)}{1}{}{Escapar.}{}{}
\verb{fuji}{}{[fuˈʒi]}{(v.)}{2}{}{Fugir.}{}{}%
\verb{fula}{}{[ˈfula]}{(adj.)}{1}{}{De pele acastanhada.}{}{}
\verb{fula}{}{[ˈfula]}{(adj.)}{2}{}{Mulato.}{}{}
\verb{fula}{}{[ˈfula]}{(adj.)}{2}{}{Pessoa negra de cor clara.}{}{}
\verb{fula}{}{[fuˈla]}{(v.)}{1}{}{Furar.}{}{}%
\verb{fula}{}{[fuˈla]}{(v.)}{2}{}{Perfurar.}{}{}
\verb{fuladô}{}{[fulaˈdo]}{(n.)}{1}{}{Furador.}{}{}
\verb{fuladu}{}{[fuˈladu]}{(adj.)}{1}{}{Furado.}{}{}
\verb{fulana}{}{[fuˈlana]}{(n.)}{1}{}{Fulana.}{}{}
\verb{fulana}{}{[fuˈlana]}{(n.)}{1}{}{Gaja.}{}{}
\verb{fulanu}{}{[fuˈlanu]}{(n.)}{1}{}{Fulano.}{}{}
\verb{fulanu}{}{[fuˈlanu]}{(n.)}{1}{}{Gajo.}{}{}
\verb{fula viji}{}{[fuˈla ˈviʒi]}{(expr.)}{1}{}{Deflorar.}{}{}%
\verb{fulu}{}{[ˈfulu]}{(n.)}{1}{}{Buraco.}{}{}
\verb{fulu}{}{[ˈfulu]}{(n.)}{2}{}{Furo.}{}{}
\verb{fulu}{}{[ˈfulu]}{(n.)}{3}{}{Vagina.}{}{}%
\verb{fulu}{}{[fuˈlu]}{(v.)}{1}{}{Açambarcar.}{}{}
\verb{fulu}{}{[fuˈlu]}{(v.)}{2}{}{Querer a todo custo.}{}{}
\verb{fulu-dentxi}{}{[ˈfulu ˈdẽtʃi]}{(n.)}{2}{}{Cárie.}{}{}
\verb{fulu-d'uku}{}{[ˈfulu ˈduku]}{(n.)}{1}{}{Ânus.}{}{}
\verb{fulufulu}{}{[ˈfuluˈfulu]}{(n.)}{1}{}{Atum-bonito.}{\textbf{\textit{Auxis
thazard}.}}{}
\verb{fulu-kadela}{}{[ˈfulu kaˈdɛla]}{(n.)}{1}{}{Ânus.}{}{}
\verb{fulu-poto}{}{[ˈfulu ˈpɔtɔ]}{(n.)}{1}{}{Buraco da fechadura.}{}{}
\verb{fulu-poto}{}{[ˈfulu ˈpɔtɔ]}{(n.)}{1}{}{Fechadura.}{}{}
\verb{fuma}{}{[fuˈma]}{(v.)}{1}{}{Defumar.}{}{}
\verb{fuma}{}{[fuˈma]}{(v.)}{2}{}{Encher.}{}{}
\verb{fuma}{}{[fuˈma]}{(v.)}{4}{}{Inchar.}{}{}
\verb{fumadu}{}{[fuˈmadu]}{(adj.)}{1}{}{Defumado.}{}{}
\verb{fumadu}{}{[fuˈmadu]}{(adj.)}{2}{}{Fumado.}{}{}
\verb{fumadu}{}{[fuˈmadu]}{(adj.)}{3}{}{Inchado.}{}{}
\verb{fumadu}{}{[fuˈmadu]}{(n.)}{1}{}{Inchaço.}{}{}
\verb{fumadu libita}{}{[fuˈmadu libiˈta]}{(expr.)}{1}{}{Cheiíssimo.}{}{}
\verb{fumadu libita}{}{[fuˈmadu libiˈta]}{(expr.)}{2}{}{Inchadíssimo.}{}{}
\verb{fumêlu}{}{[fuˈmelu]}{(n.)}{1}{}{Fumeiro.}{}{}
\verb{funda}{}{[fũˈda]}{(n.)}{1}{}{Embrulho.}{}{}
\verb{funda}{}{[fũˈda]}{(n.)}{2}{}{Feixe.}{}{}
\verb{funda}{}{[fũˈda]}{(n.)}{2}{}{\textit{Funda}.}{Borracha que marca o
local onde se encontra a âncora.}{}{}
\verb{funda}{}{[fũˈda]}{(n.)}{1}{}{Segredo.}{}{}
\verb{funda}{}{[fũˈda]}{(n.)}{4}{}{Trouxa.}{Recipiente de folha ou pano que
substitui a rodilha.}{}%
\verb{funda}{}{[fũˈda]}{(v.)}{1}{}{Afundar(-se).}{}{}
\verb{fundadô}{}{[fũdaˈdo]}{(n.)}{1}{}{Âncora.}{}{}{}
\verb{fundadô}{}{[fũdaˈdo]}{(n.)}{2}{}{Fundador.}{}{}{}
\verb{fundadu}{}{[fũˈdadu]}{(adj.)}{1}{}{Afundado.}{}{}{}
\verb{fundadu}{}{[fũˈdadu]}{(adj.)}{2}{}{Fundado.}{}{}{}
\verb{fundadu}{}{[fũˈdadu]}{(adj.)}{2}{}{Profundo.}{}{}{}
\verb{fundja}{}{[fũˈdʒa]}{(v.)}{1}{}{Fundear.}{}{}
\verb{fundjadô}{}{[fũdʒaˈdo]}{(n.)}{1}{}{Âncora.}{}{}
\verb{fundji}{}{[ˈfũdʒi]}{(n.)}{1}{}{Confusão.}{}{}
\verb{fundji}{}{[ˈfũdʒi]}{(n.)}{1}{}{Pirão de mandioca, milho ou
izaquente.}{}{}
\verb{fundji}{}{[fũˈdʒi]}{(v.)}{1}{}{Fundir.}{}{}
\verb{fundon}{}{[fũˈdõ]}{(n.)}{1}{}{Fundão.}{}{}
\verb{fundon}{}{[fũˈdõ]}{(n.)}{1}{}{Recinto de dança.}{}{}
\verb{fundu}{}{[ˈfũdu]}{(adj.)}{1}{}{Fundo.}{}{}
\verb{fundu}{}{[ˈfũdu]}{(adj.)}{2}{}{Profundo.}{}{}
\verb{fundu}{}{[ˈfũdu]}{(n.)}{1}{}{Base.}{}{}%
\verb{fundu}{}{[ˈfũdu]}{(n.)}{2}{}{Fundo.}{}{}
\verb{fundu}{}{[ˈfũdu]}{(n.)}{3}{}{Poço profundo.}{}{}%
\verb{fundu}{}{[ˈfũdu]}{(n.)}{4}{}{Talvegue.}{}{}%
\verb{fundula}{}{[fũˈdula]}{(n.)}{1}{}{Funduras.}{}{}
\verb{fundula}{}{[fũˈdula]}{(n.)}{2}{}{Profundezas.}{}{}
\verb{funhe}{}{[fũˈɲɛ]}{(n.)}{1}{}{Larva de um inseto.}{}{}
\verb{funhe}{}{[fũˈɲɛ]}{(n.)}{1}{}{\textit{Oso-moli}.}{Cf.
\textbf{oso-moli}.}{}{}
\verb{funilêlu}{}{[funiˈlelu]}{(n.)}{1}{}{Funileiro.}{}{}%
\verb{funilêlu}{}{[funiˈlelu]}{(n.)}{2}{}{Latoeiro.}{}{}
\verb{funinu}{}{[fuˈninu]}{(n.)}{1}{}{Funil.}{}{}
\verb{funkafunka}{}{[fũˈkafũˈka]}{(n.)}{2}{}{Interior.}{}{}%
\verb{funkafunka}{}{[fũˈkafũˈka]}{(n.)}{3}{}{Labirinto de caminhos em um
\textbf{luxan}.}{}{}
\verb{funkafunka}{}{[fũˈkafũˈka]}{(n.)}{1}{}{Lugar recôndito.}{}{}%
\verb{funson}{}{[fũˈsõ]}{(n.)}{1}{}{Cerimônia.}{}{}
\verb{funson}{}{[fũˈsõ]}{(n.)}{1}{}{Gala.}{}{}
\verb{funson}{}{[fũˈsõ]}{(n.)}{1}{}{Espetáculo.}{}{}
\verb{funson}{}{[fũˈsõ]}{(n.)}{2}{}{Festividade.}{}{}%
\verb{funson}{}{[fũˈsõ]}{(n.)}{3}{}{Função pública.}{}{}
\verb{funtxa}{}{[fũˈtʃa]}{(v.)}{1}{}{Focinhar.}{}{}
\verb{funtxin}{}{[f\~utˈʃĩ]}{(n.)}{1}{}{Focinho.}{}{}
\verb{funxa}{}{[fũˈʃa]}{(v.)}{1}{}{Focinhar.}{Cf. \textbf{funtxa}.}{}{}
\verb{funxyonalyu}{}{[fũʃjɔˈnalju]}{(n.)}{1}{}{Funcionário (público).}{}{}
\verb{fusa}{}{[ˈfusa]}{(n.)}{1}{}{Espuma.}{}{}
\verb{fusa}{}{[fuˈsa]}{(v.)}{1}{}{Espumar.}{}{}
\verb{fusula}{}{[fuˈsula]}{(n.)}{2}{}{Cambalhota.}{}{}
\verb{fusula}{}{[fuˈsula]}{(n.)}{2}{}{Confusão.}{}{}
\verb{fusula}{}{[fuˈsula]}{(n.)}{2}{}{Fígado de animais.}{}{}
\verb{futa}{}{[fuˈta]}{(v.)}{1}{}{Furtar.}{}{}%
\verb{futa}{}{[fuˈta]}{(v.)}{2}{}{Roubar.}{}{}
\verb{futadu}{}{[fuˈtadu]}{(adj.)}{1}{}{Furtado.}{}{}%
\verb{futadu}{}{[fuˈtadu]}{(adj.)}{2}{}{Roubado.}{}{}
\verb{futu}{}{[fuˈtu]}{(n.)}{1}{}{Cheiro a amoníaco, característico do
tubarão conservado cru e abafado durante algum tempo antes de ser
cozido.}{}{}
\verb{futu}{}{[fuˈtu]}{(n.)}{1}{}{Mau cheiro.}{}{}
\verb{futuna}{}{[fuˈtuna]}{(n.)}{1}{}{Fortuna.}{}{}%
\verb{futuna}{}{[fuˈtuna]}{(n.)}{2}{}{Riqueza.}{}{}
\verb{fuvêlu}{}{[fuˈvelu]}{(n.)}{1}{}{Irritação cutânea.}{}{}
\verb{fuvêlu}{}{[fuˈvelu]}{(n.)}{1}{}{Pé-de-atleta.}{}{}
\verb{fuzon}{}{[fuˈz\~ɔ]}{(n.)}{1}{}{Infusão.}{}{}
\verb{fya}{}{[ˈfja]}{(n.)}{1}{}{Folha.}{}{}
\verb{fya}{}{[ˈfja]}{(n.)}{2}{}{Planta.}{}{}
\verb{fyaba}{}{[ˈfjaba]}{(onom.)}{1}{}{Som de chicote.}{}{}%
\verb{fya-bambi}{}{[ˈfja ˈb\~{\textturna}bi]}{(n.)}{1}{}{Chile
branco.}{\textbf{\textit{Drymaria cordata}.}}{}
\verb{fya-bôba-blanku}{}{[ˈfja ˈboba
ˈbl\~{\textturna}ku]}{(n.)}{1}{}{Begônia.}{\textbf{\textit{Begonia
baccata}.}}{}
\verb{fya-bôba-d'ôbô}{}{[ˈfja ˈboba
doˈbo]}{(n.)}{1}{}{Folha-boba.}{\textbf{\textit{Piper umbellatum}.}}{}
\verb{fya-bôba-d'ôbô}{}{[ˈfja ˈboba
doˈbo]}{(n.)}{2}{}{Pimenta-dos-índios.}{\textbf{\textit{Piper
umbellatum}.}}{}
\verb{fya-bôba-nglandji}{}{[ˈfja ˈboba
ˈŋgl\~{\textturna}dʒi]}{(n.)}{1}{}{\textit{Fya-bôba-nglandji}.}{\textbf{\textit{Begonia
ampla}.}}{}
\verb{fya-bôba-pikina}{}{[ˈfja ˈboba piˈkina]}{(n.)}{1}{}{\textit{Fya-bôba-pikina}.}{\textbf{\textit{Piper capense}.}}{}
\verb{fya-bôba-vlêmê}{}{[ˈfja ˈboba vleˈme]}{(n.)}{1}{}{Begônia.}{\textbf{\textit{Begonia baccata}.}}{}
\verb{fya-bombon}{}{[ˈfjabõˈbõ]}{(n.)}{1}{}{Bôbôbôbô.}{\textbf{\textit{Casearia barteri}.}}{}
\verb{fya-budu}{}{[ˈfja ˈbudu]}{(n.)}{1}{}{Folha-pedra.}{\textit{\textbf{Elephantopus mollis}}.}{}
\verb{fyada}{}{[ˈfjada]}{(n.)}{1}{}{Afilhada.}{}{}%
\verb{fyadadji}{}{[fjaˈdadʒi]}{(n.)}{1}{}{Frialdade.}{}{}
\verb{fya-da-mina}{}{[ˈfja da ˈmina]}{(n.)}{1}{}{Folha-da-mina.}{\textit{\textbf{Bryophyllum pinnatum}.}}{}{}
\verb{fya-da-mina-galu}{}{[ˈfja daˈmina ˈgalu]}{(n.)}{2}{}{\textit{Fya-da-mina-galu}.}{\textit{\textbf{Kalanchoe crenata}.}}{}{}
\verb{fya-da-mina-ke}{}{[ˈfja da ˈmina ˈkɛ]}{(n.)}{1}{}{Folha da fortuna.}{\textbf{\textit{Kalanchoe pinnatum}.}}{}{}
\verb{fya-d'ami-so}{}{[ˈfja daˈmi ˈsɔ]}{(n.)}{1}{}{\textit{Fya-d'ami-so}.}{\textbf{\textit{Nervilia bicarinata}.}}{}
\verb{fya-dentxi}{}{[ˈfja ˈdẽtʃi]}{(n.)}{1}{}{\textit{Fya-dentxi}.}{\textbf{\textit{Acmella caulirhiza}}.}{}
\verb{fyadô}{}{[fjaˈdo]}{(n.)}{1}{}{Fiador.}{}{}
\verb{fya-d'olha}{}{[ˈfja dɔˈʎa]}{(n.)}{1}{}{Centela.}{\textbf{\textit{Centella asiatica}.}}{}{}
\verb{fya-d'olha}{}{[ˈfja dɔˈʎa]}{(n.)}{2}{}{Colágeno-de-Gotu.}{\textbf{\textit{Centella asiatica}.}}{}{}
\verb{fya-d'olha-ple}{}{[ˈfja dɔˈʎa ˈplɛ]}{(n.)}{1}{}{Erva-capitão.}{\textbf{\textit{Hydrocotyle bonariensis}.}}{}{}
\verb{fya-d'ôlô}{}{[ˈfja ˈdolo]}{(n.)}{1}{}{Ouros.}{Um dos naipes do baralho.}{}{}
\verb{fyadu}{}{[ˈfjadu]}{(adj.)}{1}{}{Abusado.}{}{}
\verb{fyadu}{}{[ˈfjadu]}{(adj.)}{2}{}{Atrevido.}{}{}
\verb{fyadu}{}{[ˈfjadu]}{(adj.)}{3}{}{Cretino.}{}{}
\verb{fyadu}{}{[ˈfjadu]}{(adj.)}{4}{}{Fiado.}{}{}
\verb{fyadu}{}{[ˈfjadu]}{(n.)}{1}{}{Afilhado.}{}{}
\verb{fya-fitxisu}{}{[ˈfja fiˈtʃisu]}{(n.)}{1}{}{Saia-roxa.}{Cf.
\textbf{fya-pletu}.}{}{}
\verb{fya-flakêza}{}{[ˈfja flaˈkeza]}{(n.)}{1}{}{Folha-fraqueza.}{\textit{\textbf{Laportea aestuans}}.}{}
\verb{fya-flêminga}{}{[ˈfja fleˈmĩga]}{(n.)}{1}{}{Pombinha.}{\textbf{\textit{Euphorbia serpens}}.}{}
\verb{fya-flêminga-blanku}{}{[ˈfja fleˈmĩga ˈbl\~{\textturna}ku]}{(n.)}{1}{}{Quebra-pedras.}{\textbf{\textit{Chamaesyce prostrata}}.}{}
\verb{fya-flêminga-vlêmê}{}{[ˈfja fleˈmĩga vleˈme]}{(n.)}{1}{}{Folha-formiga-vermelha.}{\textbf{\textit{Chamaesyce serpens}}.}{}
\verb{fya-fôgêtê}{}{[ˈfja foˈgete]}{(n.)}{1}{}{Pau-foguete.}{\textbf{\textit{Desmanthus virgatus}.}}{}{}
\verb{fya-fôgô}{}{[ˈfja foˈgo]}{(n.)}{1}{}{Barba-de-barata.}{\textbf{\textit{Acacia kamerunensis}.}}{}{}
\verb{fya-galu}{}{[ˈfja ˈgalu]}{(n.)}{1}{}{Heliotrópio-indiano.}{\textbf{\textit{Heliotropium indicum}.}}{}
\verb{fya-glavana}{}{[ˈfja glaˈv\~{\textturna}na]}{(n.)}{1}{}{\textit{Fya-glavana}.}{\textbf{\textit{Phaulopsis micrantha}.}}{}
\verb{fya-glêza}{}{[ˈfja ˈgleza]}{(n.)}{1}{}{Feto.}{\textbf{\textit{Dryopteris filixmas}.}}{}
\verb{fya-glêza}{}{[ˈfja ˈgleza]}{(n.)}{1}{}{Samambaia.}{\textbf{\textit{Dryopteris filixmas}.}}{}
\verb{fya-glêza-mwala}{}{[ˈfja ˈgleza
ˈmwala]}{(n.)}{1}{Folha-de-igreja.}{\textbf{\textit{Pneumatopteris
oppositifolia}.}}{}
\verb{fya-glêza-ome}{}{[ˈfja ˈgleza
ˈɔmɛ]}{(n.)}{1}{}{Folha-de-igreja.}{\textbf{\textit{Christella dentata}.}}{}
\verb{fya-ize}{}{[ˈfja
iˈzɛ]}{(n.)}{1}{}{Folha-camarão.}{\textbf{\textit{Nefrolepis bisserrata}.}}{}
\verb{fya-keza-mwala}{}{[ˈfja kɛˈza
ˈmwala]}{(n.)}{1}{Vassourinha-doce.}{\textbf{\textit{Scoparia dulcis}.}}{}
\verb{fya-keza-ome}{}{[ˈfja kɛˈza
ˈɔmɛ]}{(n.)}{1}{}{\textit{Fya-keza-ome}.}{\textbf{\textit{Spermacoce verticillata}.}}{}
\verb{fya-kopa}{}{[ˈfja ˈkɔpa]}{(n.)}{1}{}{Copas.}{Um dos naipes do baralho.}{}
\verb{fya-kopu}{}{[ˈfja ˈkɔpu]}{(n.)}{1}{}{\textit{Fya-kopu}.}{\textit{\textbf{Centella asiatica}}.}{}
\verb{fya-leve}{}{[ˈfja ˈlɛvɛ]}{(n.)}{2}{}{Musgo-do-mato.}{\textbf{\textit{Lycopodiella cernua}.}}{}
\verb{fya-leve-ome}{}{[ˈfja ˈlɛvɛ ˈɔmɛ]}{(n.)}{1}{}{Folha-leve-homem.}{\textbf{\textit{Dicranopteris linearis}}.}{}
\verb{fya-lixa}{}{[ˈfja ˈliʃa]}{(n.)}{1}{}{Pó-lixa.}{\textbf{\textit{Ficus exasperata}.}}{}
\verb{fya-male}{}{[ˈfja maˈlɛ]}{(n.)}{1}{}{Erva-de-São-João.}{\textbf{\textit{Ageratum conyzoides}.}}{}
\verb{fya-male}{}{[ˈfja maˈlɛ]}{(n.)}{3}{}{Mentrasto.}{\textbf{\textit{Ageratum conyzoides}.}}{}
\verb{fya-male-ome}{}{[ˈfja maˈlɛ ˈɔmɛ]}{(n.)}{1}{}{Folha-manuel-homem.}{\textbf{\textit{Synedrella nodiflora}.}}{}
\verb{fya-malivla}{}{[ˈfja ˈmalivla]}{(n.)}{1}{}{Malva.}{\textbf{\textit{Abutilon grandiflorum}}.}{}{}
\verb{fya-malixa}{}{[ˈfja maˈliʃa]}{(n.)}{1}{}{Folha-malícia.}{\textbf{\textit{Mimosa pudica}}.}{}{}
\verb{fya-mamblêblê}{}{[ˈfja m\~{\textturna}bleˈble]}{(n.)}{1}{}{\textit{Mamblêblê}.}{Cf. \textbf{mamblêblê}.}{}{}{}
\verb{fyamblê}{}{[ˈfj\~{\textturna}ble]}{(n.)}{1}{}{Fiambre.}{}{}{}
\verb{fya-miskitu}{}{[ˈfja misˈkitu]}{(n.)}{1}{}{Manjerico.}{\textbf{\textit{Ocimum americanum}.}}{}
\verb{fya-nginhon}{}{[ˈfja ŋgiˈɲõ]}{(n.)}{1}{}{Agrião.}{\textbf{\textit{Roripa nasturtium-aquaticum}.}}{}
\verb{fyansa}{}{[ˈfj\~{\textturna}sa]}{(n.)}{1}{}{Fiança.}{}{}
\verb{fyansa}{}{[fj\~{\textturna}ˈsa]}{(v.)}{1}{}{Afiançar.}{}{}
\verb{fya-paw}{}{[ˈfja ˈpaw]}{(n.)}{1}{}{Paus.}{Um dos naipes do baralho.}{}{}
\verb{fya-pletu}{}{[ˈfja ˈplɛtu]}{(n.)}{1}{}{Saia-roxa.}{\textbf{\textit{Datura metel}.}}{}
\verb{fya-pletu-blanku}{}{[ˈfja ˈplɛtu ˈbl\~{\textturna}ku]}{(n.)}{1}{}{Trombeteira.}{\textbf{\textit{Brugmansia x candida}.}}{}
\verb{fya-plôkô}{}{[ˈfja ˈploko]}{(n.)}{1}{}{Folha-porco.}{\textbf{\textit{Commelina congesta}.}}{}
\verb{fya-plôkô-son}{}{[ˈfja ˈploko ˈs\~ɔ]}{(n.)}{1}{}{Erva-tostão.}{\textbf{\textit{Boerhaavia diffusa}.}}{}
\verb{fya-pontu}{}{[ˈfja ˈp\~ɔtu]}{(n.)}{1}{}{Folha ponto.}{\textbf{\textit{Achyranthes aspera}.}}{}{}
\verb{fya-pyala}{}{[ˈfja ˈpjala]}{(n.)}{1}{}{\textit{Mpyala}.}{}{}{}
\verb{fya-santope}{}{[ˈfja s\~{\textturna}tɔˈpɛ]}{(n.)}{1}{}{Eufórbia.}{\textbf{\textit{Euphorbia hirta.}}}{}
\verb{fya-santope}{}{[ˈfja s\~{\textturna}tɔˈpɛ]}{(n.)}{2}{}{Folha-centopeia.}{\textbf{\textit{Chamaesyce
hirta.}}}{}
\verb{fya-supada}{}{[ˈfja suˈpada]}{(n.)}{1}{}{Espadas.}{}{}{}
\verb{fya-tatalugwa}{}{[ˈfja tataˈlugwa]}{(n.)}{1}{}{Espinafre do
Malabar.}{\textbf{\textit{Basella alba}.}}{}
\verb{fya-vela}{}{[ˈfja
ˈvɛla]}{(n.)}{1}{}{Folha-vela.}{\textbf{\textit{Tristemma litoralle.}}}{}
\verb{fya-vinte}{}{[ˈfja
v\~iˈtɛ]}{(n.)}{1}{}{Folha-vintém.}{\textbf{\textit{Desmodium
adscendens.}}}{}
\verb{fya-viola}{}{[ˈfja
viˈɔla]}{(n.)}{1}{}{Azedinha.}{\textbf{\textit{Oxalis corymbosa.}}}{}
\verb{fya-xalela}{}{[ˈfja
ʃaˈlɛla]}{(n.)}{1}{}{Chá-Gabão.}{\textbf{\textit{Cymbopogon ciitratus.}}}{}
\verb{fya-xalela}{}{[ˈfja
ʃaˈlɛla]}{(n.)}{1}{}{Chá-do-Príncipe.}{\textbf{\textit{Cymbopogon
ciitratus.}}}{}
\verb{fya-xalela}{}{[ˈfja
ʃaˈlɛla]}{(n.)}{1}{}{Citronela.}{\textbf{\textit{Cymbopogon ciitratus.}}}{}
\verb{fya-zaya}{}{[ˈfja
ˈzaja]}{(n.)}{1}{}{\emph{Fya-zaya}.}{\textbf{\textit{Cassia podocarpa.}}}{}{}
\verb{fyefyefye}{}{[fjɛfjɛˈfjɛ]}{(id.)}{1}{}{Cf. \textbf{bixi
fyefyefye}.}{}{}
\verb{fyefyefye}{}{[fjɛfjɛˈfjɛ]}{(id.)}{3}{}{Cf. \textbf{limpu
fyefyefye}.}{}{}
\verb{fyô}{}{[ˈfjo]}{(adj.)}{1}{}{Frio.}{}{}%
\verb{fyô}{}{[ˈfjo]}{(n.)}{1}{}{Flato.}{}{}
\verb{fyô}{}{[ˈfjo]}{(n.)}{2}{}{Peido.}{}{}
\verb{fyô-glosu}{}{[ˈfjo ˈglᴐsu]}{(n.)}{1}{}{Frieira.}{}{}
\verb{fyô-glosu}{}{[ˈfjo ˈglᴐsu]}{(n.)}{2}{}{Hipertensão.}{}{}
\verb{fyô-glosu}{}{[ˈfjo ˈglᴐsu]}{(n.)}{3}{}{Hipotensão.}{}{}
\verb{fyô-glosu}{}{[ˈfjo ˈglᴐsu]}{(n.)}{3}{}{Pneumonia.}{}{}
\verb{fyô-glosu}{}{[ˈfjo ˈglᴐsu]}{(n.)}{3}{}{Resfriamento agudo.}{}{}
\verb{fyô kôkôkô}{}{[ˈfjo kokoˈko]}{(expr.)}{1}{}{Frigidíssimo.}{}{}
\verb{fyô kôkôkô}{}{[ˈfjo kokoˈko]}{(expr.)}{1}{}{Gélido.}{}{}
\end{letra}

\begin{letra}{g}
\verb{ga}{}{[ˈga]}{(part.)}{1}{Partícula aspectual, alomorfe de \textbf{ka} que ocorre com a primeira pessoa do singular
\textbf{n}.}{}{\textbf{N ga kume}. \textit{Eu como.}}{}
\verb{ga}{}{[ˈga]}{(part.)}{1}{Partícula de modo, alomorfe de \textbf{ka}.}{\textbf{Xi n ga tava fla...}. \textit{Se eu tivesse falado...}}{}
\verb{gaba}{}{[gaˈba]}{(v.)}{1}{}{Gabar.}{}{}%
\verb{gaba}{}{[gaˈba]}{(v.)}{2}{}{Vangloriar(-se).}{}{}
\verb{gabinêtê}{}{[gabiˈnete]}{(n.)}{1}{}{Gabinete.}{}{}%
\verb{gabon}{}{[gaˈbõ]}{(adj.)}{1}{}{Estrangeiro africano.}{}{}{}
\verb{gabon}{}{[gaˈbõ]}{(adj.)}{1}{}{Gabonês.}{Designação genérica dos serviçais levados do continente africano para trabalhar nas roças do cacau e do café no tempo colonial.}{}{}
\verb{gabon}{}{[gaˈbõ]}{(n.)}{1}{}{Estrangeiro africano.}{}{}{}
\verb{Gabon}{}{[gaˈbõ]}{(top.)}{1}{}{Gabão.}{}{}
\verb{gadanha}{}{[gaˈd\~{\textturna}ɲa]}{(n.)}{1}{}{Gadanha.}{}{}
\verb{gadanhu}{}{[gaˈd\~{\textturna}ɲu]}{(n.)}{1}{}{Rastelo.}{}{}
\verb{gafu}{}{[ˈgafu]}{(n.)}{1}{}{Fungo.}{}{}
\verb{gafu}{}{[ˈgafu]}{(n.)}{2}{}{Gafa.}{}{}
\verb{gafu}{}{[ˈgafu]}{(n.)}{3}{}{Parasita.}{}{}
\verb{gagu}{}{[ˈgagu]}{(adj.)}{1}{}{Gago.}{}{}
\verb{gaja}{}{[gaˈʒa]}{(n.)}{1}{}{Vinho de palma fermentado.}{Cf. \textbf{uswa}.}{}
\verb{gaja}{}{[gaˈʒa]}{(v.)}{1}{}{De um dia para o outro.}{}{}
\verb{gajadu}{}{[gaˈʒadu]}{(adj.)}{1}{}{Guardado de um dia para o outro.}{}{}\verb{gajadu}{}{[gaˈʒadu]}{(adj.)}{2}{}{Vinho fermentado.}{}{}
\verb{gala}{}{[ˈgala]}{(n.)}{1}{}{Gala.}{}{}
\verb{gala}{}{[ˈgala]}{(n.)}{2}{}{Gala da palmeira.}{}{}
\verb{gala}{}{[ˈgala]}{(n.)}{3}{}{Guelra.}{}{}
\verb{gala}{}{[gaˈla]}{(v.)}{1}{}{Engravidar.}{}{}
\verb{galafa}{}{[gaˈlafa]}{(n.)}{1}{}{Garrafa.}{}{}
\verb{galafon}{}{[galaˈfõ]}{(n.)}{2}{}{Garrafão.}{}{}
\verb{galafon}{}{[galaˈfõ]}{(n.)}{1}{}{Vasilhame.}{}{}
\verb{galegu}{}{[gaˈlɛgu]}{(n.)}{2}{}{Estrangeiro.}{}{}
\verb{galegu}{}{[gaˈlɛgu]}{(n.)}{1}{}{Galego.}{}{}
\verb{galegu}{}{[gaˈlɛgu]}{(n.)}{1}{}{Branco pobre.}{}{}
\verb{galon}{}{[gaˈlõ]}{(adj.)}{1}{}{Galão.}{}{}
\verb{galu}{}{[ˈgalu]}{(n.)}{1}{}{Heliotrópio-indiano.}{Cf.
\textbf{fya-galu}.}{}{}
\verb{galu}{}{[ˈgalu]}{(n.)}{2}{}{Galo.}{}{}
\verb{galufu}{}{[ˈgalufu]}{(n.)}{1}{}{Garfo.}{}{}
\verb{gamala}{}{[gaˈmala]}{(n.)}{1}{}{Gamela.}{Cf. \textbf{ngama}.}{}{}%
\verb{gandu}{}{[g\~{\textturna}ˈdu]}{(n.)}{1}{}{Tubarão.}{Cf.
\textbf{ngandu}.}{}{}
\verb{ganga}{}{[g\~{\textturna}ˈga]}{(n.)}{1}{}{Mosca-da-fruta.}{\textit{\textbf{Drosophila
melanogaster.}}}{}
\verb{ganha}{}{[g\~{\textturna}ˈɲa]}{(n.)}{1}{}{Galinha.}{Cf.
\textbf{nganha}.}{}{}
\verb{ganha}{}{[g\~{\textturna}ˈɲa]}{(v.)}{1}{}{Arrecadar.}{}{}
\verb{ganha}{}{[g\~{\textturna}ˈɲa]}{(v.)}{2}{}{Ganhar.}{Cf.
\textbf{nganha}.}{}
\verb{gansu}{}{[ˈg\~{\textturna}su]}{(n.)}{1}{}{Gancho.}{Cf.
\textbf{ngansu}.}{}{}
\verb{gasa}{}{[ˈgasa]}{(n.)}{1}{}{Garça.}{}{}
\verb{gatela}{}{[gaˈtɛla]}{(n.)}{1}{}{Aquele que dá o mote em uma sessão
musical ou folclórica.}{}{}
\verb{gatela}{}{[gaˈtɛla]}{(n.)}{1}{}{Poeta.}{}{}
\verb{gatu}{}{[ˈgatu]}{(n.)}{2}{}{Gata.}{}{}
\verb{gatu}{}{[ˈgatu]}{(n.)}{1}{}{Gato.}{\textit{\textbf{Felis catus}}.}{}
\verb{gaxta}{}{[gaʃˈta]}{(v.)}{1}{}{Gastar.}{}{}
\verb{gayatu}{}{[gaˈjatu]}{(n.)}{1}{}{Gaiato.}{}{}
\verb{gayola}{}{[gaˈjɔla]}{(n.)}{1}{}{Gaiola.}{}{}
\verb{gazi}{}{[ˈgazi]}{(n.)}{1}{}{Gaze.}{}{}
\verb{gazolina}{}{[gazɔˈlina]}{(n.)}{1}{}{Gasolina.}{}{}
\verb{gêgê}{}{[ˈgege]}{(n.)}{1}{}{Cajá-mirim.}{}{}
\verb{gêgê}{}{[ˈgege]}{(n.)}{2}{}{Cajazeiro.}{\textbf{\textit{Spondias
mombim}.}}{}
\verb{gêgê}{}{[ˈgege]}{(n.)}{3}{}{Cajazeiro-mirim.}{\textbf{\textit{Spondias
luteas}.}}{}
\verb{gêgê}{}{[geˈge]}{(v.)}{1}{}{Impingir.}{}{}
\verb{gêgê-fasu}{}{[ˈgege
ˈfasu]}{(n.)}{1}{}{\textit{Gêgê-fasu}.}{\textbf{\textit{Polyscias
quintasii}.}}{}
\verb{gêgêgê}{}{[gegeˈge]}{(adv.)}{1}{}{Assim-assim.}{}{}%
\verb{gêgêgê}{}{[gegeˈge]}{(adv.)}{2}{}{Mais ou menos.}{}{}
\verb{gela}{}{[ˈgɛla]}{(n.)}{1}{}{Guerra.}{}{}
\verb{gembu}{}{[gẽˈbu]}{(n.)}{1}{}{Morcego.}{Cf. \textbf{ngembu}.}{}{}
\verb{gidigidi}{}{[giˈdigiˈdi]}{(id.)}{1}{}{Cf. \textbf{djinga
gidigidi.}}{}{}
\verb{gidigidi}{}{[giˈdigiˈdi]}{(id.)}{2}{}{Cf. \textbf{tlêmê gidigidi.}}{}
\verb{gigô}{}{[giˈgo]}{(n.)}{1}{}{Gligô.}{Cf. \textbf{gligô.}}{}
\verb{giya}{}{[ˈgija]}{(n.)}{1}{}{Fio de metal.}{}{}
\verb{giya}{}{[ˈgija]}{(n.)}{1}{}{Guia.}{}{}
\verb{giya}{}{[giˈja]}{(v.)}{1}{}{Conduzir.}{}{}
\verb{giya}{}{[giˈja]}{(v.)}{2}{}{Dirigir.}{}{}
\verb{giya}{}{[giˈja]}{(v.)}{3}{}{Guiar.}{}{}
\verb{giza}{}{[giˈza]}{(v.)}{1}{}{Guisar.}{}{}
\verb{gizadu}{}{[giˈzadu]}{(n.)}{2}{}{Guisado.}{}{}
\verb{glabadina}{}{[glabaˈdina]}{(n.)}{1}{}{Gabardina.}{}{}
\verb{glaganta}{}{[glaˈg\~{\textturna}ta]}{(n.)}{1}{}{Garganta.}{Cf.
\textbf{glagantxi}.}{}
\verb{glagantxa}{}{[glag\~{\textturna}ˈtʃa]}{(v.)}{1}{}{Gargarejar.}{}{}
\verb{glagantxi}{}{[glaˈg\~{\textturna}tʃi]}{(n.)}{1}{}{Garganta.}{}{}
\verb{glama}{}{[ˈglama]}{(n.)}{1}{}{Grama.}{}{}{}
\verb{glamatxika}{}{[glaˈmatʃika]}{(n.)}{1}{}{Gramática.}{}{}{}
\verb{glapon}{}{[glaˈpõ]}{(n.)}{1}{}{Carapau.}{\textit{\textbf{Selar crumenophthalmus}}.}{}{}
\verb{glasa}{}{[ˈglasa]}{(n.)}{1}{}{Graça de Deus.}{}{}
\verb{glava}{}{[glaˈva]}{(v.)}{1}{}{Agravar.}{}{}
\verb{glava}{}{[glaˈva]}{(v.)}{2}{}{Gravar.}{}{}
\verb{glava}{}{[glaˈva]}{(v.)}{3}{}{Ofender.}{}{}
\verb{glavana}{}{[glaˈvana]}{(n.)}{1}{}{Gravana.}{Estação seca e ventosa em São Tomé e Príncipe, mais ou menos de junho a setembro.}{}
\verb{glavata}{}{[glaˈvata]}{(n.)}{1}{}{Gravata.}{}{}
\verb{glavata}{}{[glavaˈta]}{(v.)}{1}{}{Agarrar pelo pescoço.}{}{}
\verb{glavêta}{}{[glaˈveta]}{(n.)}{1}{}{Gaveta.}{}{}
\verb{glavi}{}{[ˈglavi]}{(adj.)}{1}{}{Belo.}{}{}
\verb{glavi}{}{[ˈglavi]}{(adj.)}{1}{}{Bonito.}{}{}
\verb{glavi}{}{[ˈglavi]}{(adj.)}{2}{}{Formoso.}{}{}
\verb{glavi}{}{[ˈglavi]}{(adj.)}{2}{}{Lindo.}{}{}
\verb{glavi}{}{[ˈglavi]}{(n.)}{1}{}{Beleza.}{}{}
\verb{glavi}{}{[ˈglavi]}{(n.)}{1}{}{Formosura.}{}{}
\verb{glavidadji}{}{[glavidaˈdʒi]}{(n.)}{1}{}{Boniteza.}{}{}
\verb{glavidadji}{}{[glavidaˈdʒi]}{(n.)}{1}{}{Defeitos.}{}{}
\verb{glavidadji}{}{[glavidaˈdʒi]}{(n.)}{2}{}{Mania.}{}{}
\verb{glavi linda-floli}{}{[ˈglavi ˈl\~ida
ˈflɔli]}{(expr.)}{1}{}{Lindíssima.}{}{}
\verb{glavu}{}{[ˈglavu]}{(n.)}{1}{}{Agravo.}{}{}
\verb{glavu}{}{[ˈglavu]}{(n.)}{2}{}{Ofensa.}{}{}
\verb{glaxa}{}{[ˈglaʃa]}{(n.)}{1}{}{Graxa.}{}{}
\verb{gleba}{}{[ˈglɛba]}{(n.)}{1}{}{Gleba.}{}{}%
\verb{gleba}{}{[ˈglɛba]}{(n.)}{2}{}{Terreno.}{}{}
\verb{glêgu}{}{[ˈglegu]}{(adj.)}{1}{}{Grego.}{}{}
\verb{glêgu}{}{[ˈglegu]}{(n.)}{1}{}{Grego.}{}{}
\verb{glentu}{}{[ˈglẽtu]}{(adv.)}{1}{}{Dentro (de).}{Cf. \textbf{nglentu}.}{}\verb{glêtê}{}{[gleˈte]}{(v.)}{1}{}{Derreter.}{Cf. \textbf{dlêtê}.}{}{}
\verb{glêtu}{}{[ˈgletu]}{(n.)}{1}{}{Direito.}{}{}
\verb{glêtula}{}{[gleˈtula]}{(adv.)}{1}{}{Diretamente.}{}{}
\verb{glêtula}{}{[gleˈtula]}{(adv.)}{2}{}{Em frente.}{}{}
\verb{glêza}{}{[ˈgleza]}{(n.)}{1}{}{Igreja.}{}{}
\verb{gligô}{}{[gliˈgo]}{(n.)}{1}{}{Gligô.}{\textbf{\textit{Morinda
lucida}.}}{}
\verb{gligô-d'ôbô}{}{[gliˈgo
doˈbo]}{(n.)}{1}{}{Gligô-do-mato.}{\textbf{\textit{Sacosperma
paniculatum}.}}{}
\verb{glita}{}{[gliˈta]}{(n.)}{1}{}{Grito.}{}{}
\verb{glita}{}{[gliˈta]}{(v.)}{1}{}{Gritar.}{}{}
\verb{glopi}{}{[ˈglɔpi]}{(n.)}{1}{}{Gole.}{Cf. \textbf{nglopi}.}{}{}
\verb{glôpin}{}{[gloˈpĩ]}{(n.)}{1}{}{Garoupa.}{\textbf{\textit{Epinephelus
adscensionis}.}}{}
\verb{glosu}{}{[ˈglɔsu]}{(adj.)}{1}{}{Duro.}{}{}
\verb{glosu}{}{[ˈglɔsu]}{(adj.)}{2}{}{Forte.}{}{}
\verb{glosu}{}{[ˈglɔsu]}{(adj.)}{3}{}{Grosso.}{}{}
\verb{glupu}{}{[ˈglupu]}{(n.)}{1}{}{Grupo.}{}{}%
\verb{glusula}{}{[gluˈsula]}{(n.)}{1}{}{Grossura.}{}{}
\verb{gluton}{}{[gluˈt\~ɔ]}{(adj.)}{1}{}{Glutão.}{}{}
\verb{gluton}{}{[gluˈt\~ɔ]}{(adj.)}{2}{}{Glutonice.}{}{}
\verb{gô}{}{[ˈgo]}{(n.)}{2}{}{Queixa.}{}{}
\verb{gô}{}{[ˈgo]}{(n.)}{2}{}{Queixume.}{}{}
\verb{gô}{}{[ˈgo]}{(n.)}{2}{}{Reclamação.}{}{}
\verb{gô}{}{[ˈgo]}{(n.)}{3}{}{Tratamento para tirar alguém do estado de
transe.}{}{}{}
\verb{gô}{}{[ˈgo]}{(v.)}{1}{}{Choramingar.}{}{}
\verb{gô}{}{[ˈgo]}{(v.)}{2}{}{Queixar(-se).}{}{}%
\verb{gô}{}{[ˈgo]}{(v.)}{3}{}{Reclamar.}{}{}
\verb{gôdô}{}{[ˈgodo]}{(adj.)}{1}{}{Abastado.}{}{}
\verb{gôdô}{}{[ˈgodo]}{(adj.)}{3}{}{Gordo.}{}{}
\verb{gôdô}{}{[ˈgodo]}{(adj.)}{2}{}{Grande.}{}{}
\verb{gôdô}{}{[ˈgodo]}{(adj.)}{4}{}{Importante.}{}{}
\verb{gôdô}{}{[ˈgodo]}{(adj.)}{5}{}{Influente.}{}{}
\verb{gôdô}{}{[ˈgodo]}{(n.)}{1}{}{Gordura.}{}{}
\verb{gofi}{}{[ˈgɔfi]}{(n.)}{1}{}{Embaúba.}{\textbf{\textit{Cecropia
peltata}.}}{}
\verb{gofi-d'ôbô}{}{[ˈgɔfi
doˈbo]}{(n.)}{1}{}{Pau-sabrina.}{\textbf{\textit{Musanga cecropioides}.}}{}
\verb{gôgô}{}{[goˈgo]}{(n.)}{1}{}{Andiroba.}{\textbf{\textit{Carapa
procera}.}}{}
\verb{gôgô}{}{[goˈgo]}{(v.)}{1}{}{Amar.}{}{}
\verb{gôgô}{}{[goˈgo]}{(v.)}{2}{}{Ficar contente.}{}{}
\verb{gôgô}{}{[goˈgo]}{(v.)}{3}{}{Gostar.}{}{}
\verb{gôgô da}{}{[goˈgo ˈda]}{(expr.)}{1}{}{Felicitar.}{}{}
\verb{gôgô da}{}{[goˈgo ˈda]}{(expr.)}{2}{}{Parabenizar.}{}{}
\verb{gôgô-vlêmê}{}{[goˈgo ˈvleme]}{(n.)}{1}{}{Andiroba.}{Cf. \textbf{gôgô}.}{}{}
\verb{gola}{}{[ˈgɔla]}{(n.)}{1}{}{Gola.}{}{}
\verb{golo}{}{[gɔˈlɔ]}{(v.)}{1}{}{Procurar.}{}{}
\verb{golozu}{}{[gɔˈlɔzu]}{(adj.)}{1}{}{Pidão.}{}{}
\verb{golozu}{}{[gɔˈlɔzu]}{(n.)}{1}{}{Pidão.}{}{}
\verb{gongô}{}{[gõˈgo]}{(v.)}{1}{}{Amar.}{}{}{}
\verb{gongô}{}{[gõˈgo]}{(v.)}{1}{}{Gostar.}{Cf. \textbf{gôgô}.}{}{}
\verb{govena}{}{[gɔvɛˈna]}{(v.)}{1}{}{Chefiar.}{}{}%
\verb{govena}{}{[gɔvɛˈna]}{(v.)}{1}{}{Governar.}{}{}%
\verb{govena}{}{[gɔvɛˈna]}{(v.)}{2}{}{Mandar.}{}{}
\verb{govenadô}{}{[gɔvɛnaˈdo]}{(n.)}{1}{}{Governador.}{}{}
\verb{govenu}{}{[gɔˈvɛnu]}{(n.)}{1}{}{Governo.}{}{}
\verb{govenu}{}{[gɔˈvɛnu]}{(n.)}{2}{}{Regras.}{}{}
\verb{goxta}{}{[gɔʃˈta]}{(v.)}{1}{}{Adorar.}{}{}{}
\verb{goxta}{}{[gɔʃˈta]}{(v.)}{1}{}{Amar.}{}{}{}
\verb{goxta}{}{[gɔʃˈta]}{(v.)}{1}{}{Apreciar.}{}{}{}
\verb{goxta}{}{[gɔʃˈta]}{(v.)}{1}{}{Gostar.}{Cf. \textbf{ngoxta}.}{}{}
\verb{gôxtô}{}{[ˈgoʃto]}{(n.)}{1}{}{Alegria.}{}{}%
\verb{gôxtô}{}{[ˈgoʃto]}{(n.)}{2}{}{Contentamento.}{}{}%
\verb{gôxtô}{}{[ˈgoʃto]}{(n.)}{3}{}{Gosto.}{}{}
\verb{gôxtô}{}{[ˈgoʃto]}{(n.)}{5}{}{Prazer.}{}{}
\verb{gôxtô}{}{[ˈgoʃto]}{(n.)}{6}{}{Sabor.}{}{}
\verb{goza}{}{[gɔˈza]}{(v.)}{1}{}{Desfrutar.}{}{}
\verb{goza}{}{[gɔˈza]}{(v.)}{1}{}{Fazer pouco de.}{}{}
\verb{goza}{}{[gɔˈza]}{(v.)}{2}{}{Gozar com.}{}{}
\verb{goza}{}{[gɔˈza]}{(v.)}{3}{}{Gozar de.}{}{}%
\verb{goza}{}{[gɔˈza]}{(v.)}{4}{}{Ridicularizar.}{}{}
\verb{goza}{}{[gɔˈza]}{(v.)}{5}{}{Ter prazer.}{}{}
\verb{gugu}{}{[guˈgu]}{(n.)}{1}{}{Duende.}{(Figura mitológica.)}{}
\verb{gugu-d'awa}{}{[guˈgu ˈdawa]}{(n.)}{1}{}{Espírito de água doce}{Cf. \textbf{ôkôsô}.}{}
\verb{gulugulu}{}{[guˈluguˈlu]}{(n.)}{1}{}{Térmite.}{}{}
\verb{gumba}{}{[gũˈba]}{(n.)}{1}{}{Amendoim.}{Cf. \textbf{ngumba}.}{}{}
\verb{gumita}{}{[gumiˈta]}{(v.)}{1}{}{Vomitar.}{Cf. \textbf{ngumita}.}{}{}
\verb{gunda}{}{[gũˈda]}{(v.)}{1}{}{Acalmar.}{}{}%
\verb{gunda}{}{[gũˈda]}{(v.)}{2}{}{Acariciar.}{}{}%
\verb{gunda}{}{[gũˈda]}{(v.)}{3}{}{Atrair.}{}{}
\verb{gunda}{}{[gũˈda]}{(v.)}{4}{}{Chamar.}{}{}%
\verb{gunda}{}{[gũˈda]}{(v.)}{5}{}{Mandar.}{}{}%
\verb{gunda}{}{[gũˈda]}{(v.)}{6}{}{Seduzir.}{}{}
\verb{gungu}{}{[ˈgũgu]}{(n.)}{1}{}{\textit{Gungu}.}{(Ave.)}{}
\verb{guya}{}{[ˈguja]}{(n.)}{1}{}{Agulha.}{}{}%
\verb{guya}{}{[ˈguja]}{(n.)}{2}{}{Peixe-agulha.}{\textbf{\textit{Strongylura
crocodila}.}}{}
\verb{guya-bujina}{}{[ˈguja
buˈʒina]}{(n.)}{1}{}{Peixe-trombeta.}{\textbf{\textit{Fistularia
tabacaria}.}}{}
\verb{guya-kyô}{}{[ˈguja
ˈkjo]}{(n.)}{1}{}{Peixe-agulha-\textit{kyô}.}{\textbf{\textit{Tylosurus acus
rafale}.}}{}
\verb{guya-supada}{}{[ˈguja
suˈpada]}{(n.)}{1}{}{Peixe-espada.}{\textbf{\textit{Ablennes hians}.}}{}
\verb{gwada}{}{[ˈgwada]}{(n.)}{1}{}{Guarda.}{}{}
\verb{gwada}{}{[gwaˈda]}{(v.)}{1}{}{Aguardar.}{}{}%
\verb{gwada}{}{[gwaˈda]}{(v.)}{2}{}{Esperar.}{}{}%
\verb{gwada}{}{[gwaˈda]}{(v.)}{3}{}{Guardar.}{}{}
\verb{gwadadu}{}{[gwaˈdadu]}{(adj.)}{1}{}{Guardado.}{}{}
\verb{gwada-livlu}{}{[ˈgwadaˈlivlu]}{(n.)}{1}{}{Contabilista.}{}{}
\verb{gwada-livlu}{}{[ˈgwadaˈlivlu]}{(n.)}{1}{}{Guarda-livros.}{}{}
\verb{gwali}{}{[ˈgwali]}{(adv.)}{1}{}{Igual.}{}
\verb{gwalia}{}{[gwaˈlia]}{(n.)}{1}{}{Iguaria.}{Cf. \textbf{ngwalia}.}{}
\verb{gwarita}{}{[gwaˈrita]}{(n.)}{1}{}{Guarita.}{}{}
\verb{gwela}{}{[ˈgwɛla]}{(n.)}{1}{}{Goela.}{}{}
\verb{gwela}{}{[ˈgwɛla]}{(n.)}{1}{}{Guelra.}{}{}
\verb{gwenta}{}{[gwẽˈta]}{(v.)}{1}{}{Aguentar.}{}{}{}
\verb{gwenta}{}{[gwẽˈta]}{(v.)}{2}{}{Resistir.}{Cf. \textbf{ngwenta}.}{}{}
\verb{gweva}{}{[ˈgwɛva]}{(n.)}{1}{}{Goiaba.}{Cf. \textbf{ngweva}.}{}{}
\verb{gyêlu}{}{[ˈgjelu]}{(n.)}{1}{}{Agulheiro.}{}{}
\end{letra}

\begin{letra}{i}

\verb{i}{}{[ˈi]}{(conj.)}{1}{}{E.}{\textbf{I men mina bô}? \textit{E a mãe
dos teus filhos}?}{}
\verb{ia}{}{[iˈa]}{(n.)}{1}{}{Ilha.}{}{}
%\verb{Ia}{}{[iˈa]}{(top.)}{1}{}{Ilha do Príncipe.}{}{}
\verb{Ia Kabla}{}{[iˈa ˈkabla]}{(top.)}{1}{}{Ilhéu das Cabras.}{}{}
\verb{Ia Lôla}{}{[iˈa ˈlola]}{(top.)}{1}{}{Ilhéu das Rolas.}{}{}
\verb{identxidadji}{}{[iˈd\~etʃiˈdadʒi]}{(n.)}{1}{}{Identidade.}{}{}
\verb{idêya}{}{[iˈdeja]}{(n.)}{1}{}{Ideia.}{}{}
\verb{idjogo}{}{[iˈdʒɔgɔ]}{(n.)}{1}{}{\textit{Idjogo}.}{Prato típico, à base
de agrião e outras verduras, peixe e óleo de palma.}{}{}
\verb{idligu}{}{[ˈidligu]}{(n.)}{1}{}{Fumo.}{Cf. \textbf{igligu}.}{}{}
\verb{idu}{}{[ˈidu]}{(n.)}{1}{}{Piolho.}{}{}
\verb{idu-idu}{}{[ˈidu
ˈidu]}{(n.)}{1}{}{Silva-da-praia.}{\textbf{\textit{Caesalpinia bonduc}.}}{}{}
\verb{idu-idu}{}{[ˈidu
ˈidu]}{(n.)}{2}{}{Olho-de-gato.}{\textbf{\textit{Caesalpinia bonduc}.}}{}{}
\verb{Ie}{}{[ˈiɛ]}{(top.)}{1}{}{Ilha do Príncipe.}{}{}%
\verb{igleva}{}{[iˈglɛva]}{(n.)}{1}{}{Gémeo.}{Cf. \textbf{ingleva}.}{}{}
\verb{igligu}{}{[iˈgligu]}{(n.)}{1}{}{Fumaça.}{}{}
\verb{igligu}{}{[iˈgligu]}{(n.)}{2}{}{Fumo.}{}{}
\verb{igligu-kwami}{}{[iˈgligu ˈkwami]}{(n.)}{1}{}{Fumo negro.}{}{}
\verb{igligu-kwami}{}{[iˈgligu ˈkwami]}{(n.)}{1}{}{Tisna.}{}{}
\verb{ijiji}{}{[ˈiʒiʒi]}{(n.)}{1}{}{Arrepio.}{}{}
\verb{ijogo}{}{[iˈʒɔgɔ]}{(n.)}{1}{}{\textit{Idjogo}.}{Cf.
\textbf{idjogo}.}{}{}
\verb{ikili}{}{[ˈikili]}{(n.)}{1}{}{Rodilha.}{}{}
\verb{ikyabu}{}{[ˈikjabu]}{(n.)}{1}{}{Quiabo.}{\textbf{\textit{Abelmoschus
esculentus}}.}{}
\verb{ilangi-ilangi}{}{[il\~{\textturna}ˈgi
il\~{\textturna}ˈgi]}{(n.)}{1}{}{Ilangue-Ilangue.}{\textbf{\textit{Cananga
odorata}.}}{}
\verb{ilangi-ilangi}{}{[il\~{\textturna}ˈgi
il\~{\textturna}ˈgi]}{(n.)}{1}{}{Cananga.}{\textbf{\textit{Cananga
odorata}.}}{}
\verb{imeme}{}{[ˈimɛmɛ]}{(n.)}{1}{}{Andim muito tenro.}{}{}
\verb{impe}{}{[ĩˈpɛ]}{(n.)}{1}{}{Pau-Impé.}{\textbf{\textit{Olea
capensis}.}}{}
\verb{impete}{}{[ĩpɛˈtɛ]}{(n.)}{1}{}{Remela.}{}{}
\verb{inansê}{}{[in\~{\textturna}ˈse]}{(pron.)}{1}{}{Vocês.}{}{}
\verb{inansê}{}{[in\~{\textturna}ˈse]}{(pron.)}{2}{}{Vós.}{}{}
\verb{inansê}{}{[in\~{\textturna}ˈse]}{(pron.)}{3}{}{-vos.}{Segunda pessoa do plural com a função de complemento direto ou indireto. \textbf{Oze n ga mat'inansê tudaxi.} \emph{Hoje vou matar-vos todos.}}{}{}
\verb{inê}{}{[ˈine]}{(pron.)}{1}{}{Cf. \textbf{inen}.}{}{}{}
%\verb{inê}{}{[ˈine]}{(pron.)}{1}{}{Eles.}{Cf. \textbf{inen}.}{}{}
%\verb{inê}{}{[ˈine]}{(pron.)}{1}{}{-lhes.}{Cf. \textbf{inen}.}{}{}
%\verb{inê}{}{[ˈine]}{(pron.)}{1}{}{-os.}{Cf. \textbf{inen}.}{}{}
\verb{inen}{}{[ˈinẽ]}{(art.)}{1}{}{As.}{}{}
\verb{inen}{}{[ˈinẽ]}{(art.)}{1}{}{Os.}{\textbf{Inen mosu}. \emph{Os rapazes}.}{}
\verb{inen}{}{[ˈinẽ]}{(pron.)}{1}{}{-as.}{Terceira pessoa do plural com a
função de complemento direto.
}{}
\verb{inen}{}{[ˈinẽ]}{(pron.)}{1}{}{Elas.}{}{}
\verb{inen}{}{[ˈinẽ]}{(pron.)}{2}{}{Eles.}{}{}
\verb{inen}{}{[ˈinẽ]}{(pron.)}{1}{}{-lhes.}{Terceira pessoa do plural com a função de complemento indireto. \textbf{So a d'inen \~{u}a pingada.} \emph{Então deram-lhes uma espingarda}.}{}
\verb{inen}{}{[ˈinẽ]}{(pron.)}{1}{}{-os.}{Terceira pessoa do plural com a função de complemento direto. \textbf{N tênd'inen nala ka fla.} \emph{Ouvi-os lá a falar.}}{}{}
\verb{ingleva}{}{[ĩˈglɛva]}{(adj.)}{1}{}{Gémeo.}{}{}
\verb{ingleva}{}{[ĩˈglɛva]}{(n.)}{1}{}{Gémeo.}{}{}
\verb{inhe}{}{[ˈiɲɛ]}{(n.)}{1}{}{Lâmina de folha de palmeira.}{}{}
\verb{inhe}{}{[ˈiɲɛ]}{(n.)}{2}{}{Unha.}{}{}
\verb{inhe-bôbô}{}{[ˈiɲɛ boˈbo]}{(n.)}{1}{}{Pimenta-da-Guiné.}{\textbf{\textit{Xylopia aethiopica}.}}{}
\verb{inhe-pletu}{}{[ˈiɲɛ ˈplɛtu]}{(n.)}{2}{}{Unha-preta.}{\textbf{\textit{Polyalthia oliveri}.}}{}
\verb{injinhêlu}{}{[\~iʒ\~iˈɲelu]}{(n.)}{1}{}{Engenheiro.}{}{}
\verb{ino}{}{[ˈinɔ]}{(adv.)}{1}{}{Não.}{}{}
\verb{inon}{}{[ˈin\~ɔ]}{(adv.)}{1}{}{Não.}{Cf. \textbf{ino}.}{}{}
\verb{intima}{}{[\~itiˈma]}{(v.)}{1}{}{Intimar.}{}{}%
\verb{inxpetôla}{}{[\~iʃpɛˈtola]}{(n.)}{1}{}{Inspetora.}{}{}
\verb{inxpetôlu}{}{[\~iʃpɛˈtolu]}{(n.)}{1}{}{Inspetor.}{}{}
\verb{ipê}{}{[iˈpe]}{(n.)}{1}{}{Ipê.}{\textbf{\textit{Steganthus welwitschii}.}}{}
\verb{isaki}{}{[isaˈki]}{(dem.)}{1}{}{Aquela.}{}{}
\verb{isaki}{}{[isaˈki]}{(dem.)}{2}{}{Aquele.}{}{}
\verb{isaki}{}{[isaˈki]}{(dem.)}{3}{}{Esta.}{}{}
\verb{isaki}{}{[isaˈki]}{(dem.)}{3}{}{Estas.}{}{}
\verb{isaki}{}{[isaˈki]}{(dem.)}{3}{}{Este.}{}{}
\verb{isaki}{}{[isaˈki]}{(dem.)}{4}{}{Estes.}{}{}
\verb{isala}{}{[isaˈla]}{(dem.)}{1}{}{Aquela.}{}{}
\verb{isala}{}{[isaˈla]}{(dem.)}{2}{}{Aquelas.}{}{}
\verb{isala}{}{[isaˈla]}{(dem.)}{3}{}{Aquele.}{}{}
\verb{isala}{}{[isaˈla]}{(dem.)}{4}{}{Aqueles.}{}{}
\verb{isala}{}{[isaˈla]}{(dem.)}{1}{}{Essa.}{}{}
\verb{isala}{}{[isaˈla]}{(dem.)}{2}{}{Essas.}{}{}
\verb{isala}{}{[isaˈla]}{(dem.)}{3}{}{Esse.}{}{}
\verb{isala}{}{[isaˈla]}{(dem.)}{4}{}{Esses.}{}{}
\verb{ise}{}{[iˈsɛ]}{(dem.)}{1}{}{Esta.}{}{}
\verb{ise}{}{[iˈsɛ]}{(dem.)}{2}{}{Estas.}{}{}
\verb{ise}{}{[iˈsɛ]}{(dem.)}{3}{}{Este.}{\textbf{Ise ku tê beba ô?} \emph{Este aqui que tem barbas?}}{}{}
\verb{ise}{}{[iˈsɛ]}{(dem.)}{4}{}{Estes.}{}{}
\verb{ite}{}{[ˈitɛ]}{(n.)}{1}{}{Abdômen.}{}{}%
\verb{ite}{}{[ˈitɛ]}{(n.)}{1}{}{Baixo-ventre.}{}{}%
\verb{ite}{}{[ˈitɛ]}{(n.)}{2}{}{Colo.}{}{}
\verb{ite}{}{[ˈitɛ]}{(n.)}{3}{}{Virilha.}{}{}
\verb{ite-ite}{}{[ˈitɛ ˈitɛ]}{(n.)}{1}{}{Aconchego.}{}{}
\verb{ite-ite}{}{[ˈitɛ ˈitɛ]}{(n.)}{1}{}{Ao pé de.}{}{}
\verb{ite-ite}{}{[ˈitɛ ˈitɛ]}{(n.)}{1}{}{Apego.}{}{}
\verb{ixi}{}{[ˈiʃi]}{(dem.)}{1}{}{Aquela.}{}{}
\verb{ixi}{}{[ˈiʃi]}{(dem.)}{2}{}{Aquelas.}{}{}
\verb{ixi}{}{[ˈiʃi]}{(dem.)}{3}{}{Aquele.}{\textbf{Bô ka toma ixi ku bwaba bô.} \emph{Tu pegas aquele que te agradar.}}{}{}
\verb{ixi}{}{[ˈiʃi]}{(dem.)}{4}{}{Aqueles.}{}{}
\verb{ixi}{}{[ˈiʃi]}{(dem.)}{4}{}{A tal.}{}{}
\verb{ixi}{}{[ˈiʃi]}{(dem.)}{4}{}{As tais.}{}{}
\verb{ixi}{}{[ˈiʃi]}{(dem.)}{4}{}{O tal.}{}{}
\verb{ixi}{}{[ˈiʃi]}{(dem.)}{4}{}{Os tais.}{}{}
\verb{ixka}{}{[ˈiʃka]}{(n.)}{2}{}{Engodo.}{}{}
\verb{ixka}{}{[ˈiʃka]}{(n.)}{1}{}{Isca.}{}{}%
\verb{izê}{}{[ˈize]}{(n.)}{1}{}{Camarão.}{}{}
\verb{izê}{}{[ˈize]}{(n.)}{1}{}{Lagostim.}{}{}
\verb{izê-blanku}{}{[ˈize ˈbl\~{\textturna}ku]}{(n.)}{1}{}{Camarão
branco.}{}{}
\end{letra}

\begin{letra}{j}
\verb{ja}{}{[ˈʒa]}{(n.)}{1}{}{Dia.}{Cf. \textbf{dja}.}{}{}
\verb{jaba}{}{[ˈʒaba]}{(onom.)}{1}{}{Som do cacete.}{}{}
\verb{jabu}{}{[ˈʒabu]}{(n.)}{1}{}{Diabo.}{Cf. \textbf{djabu}.}{}{}
\verb{jadlin}{}{[ʒaˈdl\~i]}{(n.)}{1}{}{Jardim.}{}{}
\verb{jagli}{}{[ˈʒagli]}{(n.)}{1}{}{Sarna}{}{}{}
\verb{jagu}{}{[ˈʒagu]}{(n.)}{1}{}{Azar.}{}{}
\verb{ja-jingu}{}{[ʒaˈʒĩgu]}{(n.)}{1}{}{Domingo.}{Cf.
\textbf{dja-djingu}.}{}{}
\verb{jaka}{}{[ˈʒaka]}{(n.)}{1}{}{Jaca.}{}{}
\verb{jaka}{}{[ˈʒaka]}{(n.)}{1}{}{Jaqueira.}{\textbf{\textit{Artocarpus
heterophyla}}.}{}
\verb{jakare}{}{[ʒakaˈ{\textfishhookr}ɛ]}{(n.)}{1}{}{Crocodilo.}{}{}%
\verb{jalu}{}{[ˈʒalu]}{(n.)}{1}{}{Jarro.}{Cf. \textbf{djalu}.}{}{}
\verb{jamba}{}{[ˈʒ\~{\textturna}ba]}{(n.)}{1}{}{Liamba.}{Cf.
\textbf{lyamba}.}{}{}
\verb{jambi}{}{[ʒ\~{\textturna}ˈbi]}{(n.)}{1}{}{\textit{Djambi}.}{Cf.
\textbf{djambi}.}{}{}
\verb{jamblê}{}{[ʒ\~{\textturna}ˈble]}{(n.)}{1}{}{Verdete.}{}{}%
\verb{jampla}{}{[ˈʒ\~{\textturna}pla]}{(adj.)}{1}{}{Grande.}{}{}
\verb{janêlu}{}{[ʒaˈnelu]}{(n.)}{1}{}{Janeiro.}{}{}
\verb{janêlu}{}{[ʒaˈnelu]}{(n.)}{2}{}{Velhice.}{}{}
\verb{janja}{}{[ʒ\~{\textturna}ˈʒa]}{(adv.)}{1}{}{Apressado.}{}{}{}
\verb{janja}{}{[ʒ\~{\textturna}ˈʒa]}{(adv.)}{1}{}{Depressa.}{}{}{}
\verb{janja}{}{[ʒ\~{\textturna}ˈʒa]}{(adv.)}{1}{}{Rapidamente.}{Cf. \textbf{djandjan}.}{}{}
\verb{janjan}{}{[ʒ\~{\textturna}ˈʒ\~{\textturna}]}{(adv.)}{1}{}{Apressado.}{}{}{}
\verb{janjan}{}{[ʒ\~{\textturna}ˈʒ\~{\textturna}]}{(adv.)}{1}{}{Depressa.}{}{}{}
\verb{janjan}{}{[ʒ\~{\textturna}ˈʒ\~{\textturna}]}{(adv.)}{1}{}{Rapidamente.}{Cf. \textbf{djandjan}.}{}{}
\verb{janta}{}{[ʒ\~{\textturna}ˈta]}{(n.)}{1}{}{Janta.}{Cf.
\textbf{zanta}.}{}{}
\verb{janta}{}{[ʒ\~{\textturna}ˈta]}{(v.)}{1}{}{Jantar.}{Cf.
\textbf{zanta}.}{}{}
\verb{jasu!}{}{[ˈʒasu]}{(interj.)}{1}{}{Bolas!}{}{}{}
\verb{jasu!}{}{[ˈʒasu]}{(interj.)}{1}{}{Ora bolas!}{}{}{}
\verb{jasu!}{}{[ˈʒasu]}{(interj.)}{1}{}{Porra!}{}{}{}
\verb{jasu!}{}{[ˈʒasu]}{(interj.)}{1}{}{Raios!}{Cf. \textbf{djasu!}}{}{}
\verb{jatu}{}{[ˈʒatu]}{(n.)}{1}{}{Jato.}{}{}
\verb{jê}{}{[ˈʒe]}{(v.)}{1}{}{Apanhar.}{}{}{}
\verb{jê}{}{[ˈʒe]}{(v.)}{1}{}{Buscar.}{}{}{}
\verb{jê}{}{[ˈʒe]}{(v.)}{1}{}{Recolher.}{Cf. \textbf{djê}.}{}{}
\verb{jela}{}{[ʒɛˈla]}{(v.)}{1}{}{Gelar.}{}{}%
\verb{jelentxi}{}{[ʒɛˈl\~ɛtʃi]}{(n.)}{1}{}{Gerente.}{}{}
\verb{jêlu}{}{[ˈʒelu]}{(n.)}{2}{}{Dinheiro.}{Cf. \textbf{djêlu}.}{}{}
\verb{jêlu}{}{[ˈʒelu]}{(n.)}{1}{}{Gelo.}{}{}
\verb{jenson}{}{[ʒ\~ɛˈs\~ɔ]}{(n.)}{1}{}{Injeção.}{Cf. \textbf{njenson}.}{}{}
\verb{jenu}{}{[ˈʒɛnu]}{(n.)}{1}{}{Gênio.}{}{}%
\verb{jenu}{}{[ˈʒɛnu]}{(n.)}{2}{}{Temperamento.}{}{}
\verb{jera}{}{[ʒɛˈra]}{(v.)}{1}{}{Gerar.}{}{}%
\verb{jera}{}{[ʒɛˈra]}{(v.)}{2}{}{Reproduzir.}{}{}
\verb{jesu}{}{[ˈʒɛsu]}{(n.)}{1}{}{Cachimbo de barro.}{}{}%
\verb{jesu}{}{[ˈʒɛsu]}{(n.)}{2}{}{Gesso.}{}{}
\verb{jêtu}{}{[ˈʒetu]}{(n.)}{1}{}{Jeito.}{}{}
\verb{ji}{}{[ˈʒi]}{(prep.)}{1}{}{De.}{Cf. \textbf{dji}.}{}{}
\verb{jiba}{}{[ˈʒiba]}{(n.)}{1}{}{Elefantíase.}{}{}
\verb{jiba}{}{[ˈʒiba]}{(n.)}{1}{}{Erisipela.}{}{}
\verb{jibela}{}{[ʒiˈbɛla]}{(n.)}{1}{}{Algibeira.}{}{}{}
\verb{jibela}{}{[ʒiˈbɛla]}{(n.)}{1}{}{Bolso.}{Cf. \textbf{djibela}.}{}{}
\verb{jibon}{}{[ʒiˈbõ]}{(n.)}{1}{}{Casaco.}{}{}{}
\verb{jibon}{}{[ʒiˈbõ]}{(n.)}{1}{}{Gibão.}{Cf. \textbf{djibon}.}{}{}
\verb{jidali}{}{[ʒiˈdali]}{(n.)}{1}{}{Dedal.}{Cf. \textbf{djidali}.}{}{}
\verb{jiji}{}{[ˈʒiʒi]}{(adj.)}{1}{}{Cerrado.}{}{}
\verb{jiji}{}{[ˈʒiʒi]}{(adj.)}{1}{}{Denso.}{}{}
\verb{jiji}{}{[ˈʒiʒi]}{(adj.)}{1}{}{Impenetrável.}{}{}
\verb{jijimpli}{}{[ʒiˈʒĩpli]}{(n.)}{1}{}{Gengibre.}{\textbf{\textit{Zingiber
officinalis}.}}{}
\verb{jijimpli}{}{[ʒiˈʒĩpli]}{(n.)}{1}{}{Gengiva.}{}{}
\verb{jikitxi}{}{[ˈʒikitʃi]}{(adj.)}{1}{}{Castiço.}{}{}{}
\verb{jikitxi}{}{[ˈʒikitʃi]}{(adj.)}{1}{}{Genuíno.}{}{}{}
%\verb{jikitxi}{}{[ˈʒikitʃi]}{(adj.)}{1}{}{Grosso.}{Cf.\textbf{djikitxi}.}{}{}
\verb{jikitxi}{}{[ˈʒikitʃi]}{(adj.)}{1}{}{Rude.}{Cf. \textbf{djikitxi}.}{}{}
\verb{jilêra}{}{[ʒiˈlera]}{(n.)}{1}{}{Frigorífico.}{}{}
\verb{jilêra}{}{[ʒiˈlera]}{(n.)}{2}{}{Geladeira.}{}{}
\verb{jimbli}{}{[ˈʒĩbli]}{(n.)}{1}{}{Gengibre.}{Cf. \textbf{jijimpli}.}{}{}
\verb{jimboa}{}{[ʒĩbɔˈa]}{(n.)}{1}{}{\textit{Jimboa}.}{Cf.
\textbf{djimboa}.}{}{}
\verb{jimola}{}{[ʒiˈmɔla]}{(n.)}{1}{}{Esmola.}{Cf. \textbf{djimola}.}{}{}
\verb{jina}{}{[ˈʒina]}{(prep.)}{1}{}{De.}{}{}{}
\verb{jina}{}{[ˈʒina]}{(prep.)}{1}{}{Desde.}{Cf. \textbf{djina}.}{}{}
\verb{jinebla}{}{[ʒiˈnɛbla]}{(n.)}{1}{}{Genebra.}{Cf. \textbf{djinebla}.}{}{}\verb{jinga}{}{[ʒĩˈga]}{(v.)}{1}{}{Abanar.}{}{}{}
\verb{jinga}{}{[ʒĩˈga]}{(v.)}{1}{}{Agitar(-se).}{}{}{}
\verb{jinga}{}{[ʒĩˈga]}{(v.)}{1}{}{Balançar.}{}{}{}
\verb{jinga}{}{[ʒĩˈga]}{(v.)}{1}{}{Gingar.}{}{}{}
\verb{jinga}{}{[ʒĩˈga]}{(v.)}{1}{}{Mover-se.}{}{}{}
\verb{jinga}{}{[ʒĩˈga]}{(v.)}{1}{}{Sacudir.}{Cf. \textbf{djinga}.}{}{}
\verb{jingadô}{}{[ʒĩgaˈdo]}{(n.)}{1}{}{Cordoeiro.}{}{}
\verb{jingantxi}{}{[ʒĩˈg\~{\textturna}tʃi]}{(adj.)}{1}{}{Desmedido.}{}{}
\verb{jingantxi}{}{[ʒĩˈg\~{\textturna}tʃi]}{(adj.)}{2}{}{Gigante.}{}{}%
\verb{jingantxi}{}{[ʒĩˈg\~{\textturna}tʃi]}{(adj.)}{3}{}{Grande.}{}{}
\verb{jingantxi}{}{[ʒĩˈg\~{\textturna}tʃi]}{(n.)}{1}{}{Gigante.}{}{}%
\verb{jinklo}{}{[ʒĩˈklɔ]}{(n.)}{1}{}{Anato.}{\textbf{\textit{Bixa
orellana}.}}{}
\verb{jinklo}{}{[ʒĩˈklɔ]}{(n.)}{2}{}{\textit{Gincló}.}{\textbf{\textit{Bixa
orellana}.}}{}
\verb{jinklo}{}{[ʒĩˈklɔ]}{(n.)}{3}{}{Tintureira.}{\textbf{\textit{Bixa
orellana}.}}{}
\verb{jinklo}{}{[ʒĩˈklɔ]}{(n.)}{4}{}{Urucum.}{\textbf{\textit{Bixa
orellana}.}}{}
\verb{jinku}{}{[ˈʒĩku]}{(n.)}{1}{}{Chapa de zinco.}{}{}
\verb{jinku}{}{[ˈʒĩku]}{(n.)}{1}{}{Telha de zinco.}{}{}
\verb{jinku}{}{[ˈʒĩku]}{(n.)}{2}{}{Zinco.}{}{}%
\verb{jinola}{}{[ʒiˈnɔla]}{(adv.)}{1}{}{Desde essa altura.}{}{}{}
\verb{jinola}{}{[ʒiˈnɔla]}{(adv.)}{1}{}{Desde sempre.}{}{}{}
\verb{jinola}{}{[ʒiˈnɔla]}{(adv.)}{1}{}{Sempre.}{Cf. \textbf{djinola}.}{}{}
\verb{jintxin}{}{[ʒĩˈtʃĩ]}{(adj.)}{1}{}{Gentio.}{}{}
\verb{jintxin}{}{[ʒĩˈtʃĩ]}{(adj.)}{2}{}{Não-batizado.}{}{}
\verb{jintxin}{}{[ʒĩˈtʃĩ]}{(adj.)}{3}{}{Pagão.}{}{}
\verb{jipela}{}{[ʒiˈpɛla]}{(n.)}{1}{}{Elefantíase.}{}{}
\verb{jipela}{}{[ʒiˈpɛla]}{(n.)}{1}{}{Erisipela.}{}{}
\verb{jipi}{}{[ʒiˈpi]}{(n.)}{1}{}{Jipe.}{}{}
\verb{jita}{}{[ˈʒita]}{(n.)}{1}{}{Jita.}{\textbf{\textit{Boaedon lineatus
bedriagae}.}}{}
\verb{jizu}{}{[ʒiˈzu]}{(interj.)}{1}{}{Ai Jesus!}{}{}
\verb{Jizu}{}{[ʒiˈzu]}{(interj.)}{1}{}{Jesus!}{}{}
\verb{Jizu}{}{[ʒiˈzu]}{(n.)}{1}{}{Jesus}{}{}
\verb{jogo}{}{[ˈʒɔgɔ]}{(n.)}{1}{}{\textit{Idjogo}.}{Cf. \textbf{idjogo}.}{}{}\verb{jonali}{}{[ʒɔˈnali]}{(n.)}{1}{}{Jornal.}{}{}
\verb{julhu}{}{[ˈʒuʎu]}{(n.)}{1}{}{Julho.}{}{}
\verb{junhu}{}{[ˈʒũɲu]}{(n.)}{1}{}{Junho.}{}{}
\end{letra}

\begin{letra}{k}

\verb{ka}{}{[ˈka]}{(part.)}{1}{}{Partícula de modo.}{\textbf{Xi ê ka tava
fla...} \textit{Se ele tivesse falado...}}{}{}
\verb{ka}{}{[ˈka]}{(part.)}{1}{}{Partícula aspectual.}{\textbf{Non ka kume pixi.} \textit{Comemos peixe/Comeremos peixe.}}{}{}
\verb{kaba}{}{[kaˈba]}{(adv.)}{1}{}{Acertadamente.}{}{}%
\verb{kaba}{}{[kaˈba]}{(adv.)}{2}{}{Até o fim.}{}{}
\verb{kaba}{}{[kaˈba]}{(adv.)}{3}{}{De forma sensata.}{}{}
\verb{kaba}{}{[kaˈba]}{(adv.)}{4}{}{De uma vez por todas.}{}{}
\verb{kaba}{}{[kaˈba]}{(n.)}{1}{}{Fim.}{}{}%
\verb{kaba}{}{[kaˈba]}{(v.)}{1}{}{Acabar.}{}{}%
\verb{kaba}{}{[kaˈba]}{(v.)}{2}{}{Acabar de.}{}{}
\verb{kaba}{}{[kaˈba]}{(v.)}{3}{}{Terminar.}{}{}%
\verb{kaba}{}{[kaˈba]}{(v.)}{4}{}{Concluir.}{}{}%
\verb{kabadu}{}{[kaˈbadu]}{(adj.)}{1}{}{Acabado.}{}{}
\verb{kabadu}{}{[kaˈbadu]}{(adj.)}{1}{}{Sem dinheiro.}{}{}
\verb{kabaku}{}{[kaˈbaku]}{(n.)}{1}{}{Serragem.}{}{}
\verb{kabaku-lixi}{}{[kaˈbaku ˈliʃi]}{(n.)}{1}{}{Catarro seco.}{}{}
\verb{kaba lolo}{}{[kaˈba lɔˈlɔ]}{(expr.)}{1}{}{Acabar completamente.}{}{}
\verb{kaba lolo}{}{[kaˈba lɔˈlɔ]}{(expr.)}{1}{}{Emagrecer.}{}{}
\verb{kabalu}{}{[kaˈbalu]}{(n.)}{1}{}{Cavalo.}{}{}
\verb{kabalu-sun-dêsu}{}{[kaˈbalu ˈsũ ˈdesu]}{(n.)}{1}{}{Cavalinho-de-deus.}{}{}{}
\verb{kabalu-sun-dêsu}{}{[kaˈbalu ˈsũ ˈdesu]}{(n.)}{1}{}{Louva-a-deus.}{}{}{}\verb{kabamentu}{}{[kabaˈm\~etu]}{(n.)}{1}{}{Acabamento.}{}{}{}
\verb{kabamentu}{}{[kabaˈm\~etu]}{(n.)}{2}{}{Parte final.}{}{}{}
\verb{kaba plepleple}{}{[kaˈba plɛplɛˈplɛ]}{(expr.)}{1}{}{Esgotar(-se).}{}{}{}
\verb{kaba plepleple}{}{[kaˈba plɛplɛˈplɛ]}{(expr.)}{2}{}{Terminar.}{}{}{}
\verb{kabasu}{}{[kaˈbasu]}{(n.)}{1}{}{Cabaz.}{}{}
\verb{kabasu}{}{[kaˈbasu]}{(n.)}{2}{}{Embrulho.}{}{}
\verb{kabasu}{}{[kaˈbasu]}{(n.)}{3}{}{Hímen.}{}{}
\verb{kabasu}{}{[kaˈbasu]}{(n.)}{4}{}{Recipiente.}{}{}
\verb{kabêlu}{}{[kaˈbelu]}{(n.)}{1}{}{Cabelo.}{}{}%
\verb{kabêlu-limi}{}{[kaˈbelu ˈlimi]}{(n.)}{1}{}{Cabelo de bebê.}{}{}
\verb{kabêsa}{}{[kaˈbesa]}{(n.)}{1}{}{Cabeça.}{}{}
\verb{kabêsada}{}{[kabeˈsada]}{(n.)}{1}{}{Cabeçada.}{}{}
\verb{kabêsa-d'awa}{}{[kaˈbesa ˈdawa]}{(n.)}{1}{}{Nascente.}{}{}
\verb{kabêsa-dêfuntu}{}{[kaˈbesa deˈfũtu]}{(n.)}{1}{}{Caveira.}{}{}
\verb{kabêsa-kôlê}{}{[kaˈbesa koˈle]}{(n.)}{1}{}{Pensamento.}{}{}{}%
\verb{kabêsa-kôlê}{}{[kaˈbesa koˈle]}{(n.)}{1}{}{Preocupação.}{}{}{}%
\verb{kabêsa-nglandji}{}{[kaˈbesa
ˈŋgl\~{\textturna}dʒi]}{(n.)}{1}{}{Tartaruga- cabeçuda.}{\textbf{\textit{Caretta caraetta}.}}{}{}
\verb{kabêsa wôlôwôlô}{}{[kaˈbesa woˈlowoˈlo]}{(expr.)}{1}{}{Cabeça de vento.}{}{}{}%
\verb{kabêsa wôlôwôlô}{}{[kaˈbesa woˈlowoˈlo]}{(expr.)}{2}{}{Estroina.}{}{}{}\verb{kabidela}{}{[kabiˈdɛla]}{(n.)}{1}{}{Cabidela.}{}{}
\verb{kabisela}{}{[kabiˈsɛla]}{(n.)}{1}{}{Cabeceira.}{}{}
\verb{kabla}{}{[ˈkabla]}{(n.)}{1}{}{Cabra.}{}{}
\verb{kabla}{}{[ˈkabla]}{(n.)}{2}{}{Estragador.}{}{}
\verb{kabla-d'ope-longô}{}{[ˈkabla dɔˈpɛ ˈl\~ogo]}{(n.)}{1}{}{Carneiro.}{}{}
\verb{kaboka}{}{[kaˈbɔka]}{(v.)}{1}{}{Calar(-se).}{}{}
\verb{kaboka pipipi}{}{[kaˈbɔka pipiˈpi]}{(expr.)}{1}{}{Caluda!}{}{}
\verb{kabôkô}{}{[kaˈboko]}{(n.)}{1}{}{Alicerce.}{}{}
\verb{kabu}{}{[ˈkabu]}{(n.)}{1}{}{Cabo.}{}{}%
\verb{kabu}{}{[ˈkabu]}{(n.)}{1}{}{Cabo (patente militar).}{}{}%
\verb{kabu}{}{[ˈkabu]}{(n.)}{2}{}{Fio elétrico.}{}{}
\verb{kabuvêdê}{}{[ˈkabuˈvede]}{(adj.)}{1}{}{Cabo-verdiano.}{}{}
\verb{Kabuvêdê}{}{[ˈkabuˈvede]}{(n.)}{1}{}{Cabo-verdiano.}{}{}
\verb{Kabuvêdê}{}{[ˈkabuˈvede]}{(top.)}{1}{}{Cabo Verde.}{}{}
\verb{kada}{}{[ˈkada]}{(quant.)}{1}{}{Cada.}{}{}
\verb{kada}{}{[ˈkada]}{(n.)}{1}{}{Calda.}{}{}
\verb{kada}{}{[kaˈda]}{(v.)}{1}{}{Escaldar.}{}{}
\verb{kada vê ku}{}{[ˈkada ˈve ˈku]}{(conj.)}{1}{}{Sempre que.}{}{}
\verb{kadavelu}{}{[kaˈdavɛlu]}{(n.)}{1}{}{Cadáver.}{}{}
\verb{kadela}{}{[kaˈdɛla]}{(n.)}{1}{}{Ancas.}{}{}%
\verb{kadela}{}{[kaˈdɛla]}{(n.)}{2}{}{Bunda.}{}{}%
\verb{kadela}{}{[kaˈdɛla]}{(n.)}{3}{}{Cadeira.}{}{}%
\verb{kadela}{}{[kaˈdɛla]}{(n.)}{4}{}{Cadeiras.}{}{}%
\verb{kadela}{}{[kaˈdɛla]}{(n.)}{5}{}{Nádegas.}{}{}%
\verb{kadela}{}{[kaˈdɛla]}{(n.)}{6}{}{Rabo.}{}{}%
\verb{kadelada}{}{[kadɛˈlada]}{(n.)}{1}{}{Caldeirada.}{}{}
\verb{kadenu}{}{[kaˈdɛnu]}{(n.)}{1}{}{Caderno.}{}{}
\verb{kadja}{}{[kaˈdʒa]}{(n.)}{1}{}{Cadeia.}{}{}
\verb{kadja}{}{[kaˈdʒa]}{(n.)}{2}{}{Prisão.}{}{}%
\verb{kadu}{}{[ˈkadu]}{(n.)}{1}{}{Caldo.}{}{}
\verb{kafe}{}{[kaˈfɛ]}{(n.)}{1}{}{Café.}{}{}%
\verb{kafe}{}{[kaˈfɛ]}{(n.)}{2}{}{Cafezeiro.}{\textbf{\textit{Coffea
arabica}}.}{}
\verb{kafe-d'ôbô}{}{[kaˈfɛ
doˈbo]}{(n.)}{1}{}{Café-do-mato.}{\textbf{\textit{Bertiera racemosa}}.}{}
\verb{kafuka}{}{[kaˈfuka]}{(n.)}{1}{}{Candeeiro de petróleo feito de
lata.}{}{}
\verb{kafuka}{}{[kaˈfuka]}{(n.)}{2}{}{Mulato.}{}{}
\verb{kafungu}{}{[kaˈf\~ugu]}{(n.)}{1}{}{Alimento preparado à base de banana
madura pisada e farinha de milho.}{}{}
\verb{kafunhe}{}{[kafuˈɲɛ]}{(adj.)}{1}{}{Raquítico.}{}{}
\verb{kafunhe}{}{[kafuˈɲɛ]}{(n.)}{1}{}{Cabra-loira.}{\textbf{\textit{Lucanus
cervus}}.}{}
\verb{kain}{}{[kaˈĩ]}{(adj.)}{1}{}{Avarento.}{}{}%
\verb{kain}{}{[kaˈĩ]}{(adj.)}{2}{}{Mão-de-vaca.}{}{}
\verb{kain}{}{[kaˈĩ]}{(adj.)}{2}{}{Somítico.}{}{}
\verb{kain}{}{[kaˈĩ]}{(adj.)}{3}{}{Sovina.}{}{}
\verb{kajamanga}{}{[kaʒaˈm\~{\textturna}ga]}{(n.)}{1}{}{Cajá-manga.}{}{}
\verb{kajamanga}{}{[kaʒaˈm\~{\textturna}ga]}{(n.)}{1}{}{Cajá-mangueira.}{\textbf{\textit{Spondias
cytherea}}.}{}
\verb{kaji-vapô}{}{[ˈkaʒi vaˈpo]}{(n.)}{1}{}{Cais.}{}{}{}
\verb{kaju}{}{[kaˈʒu]}{(n.)}{1}{}{Caju.}{}{}
\verb{kaju}{}{[kaˈʒu]}{(n.)}{2}{}{Cajueiro.}{\textbf{\textit{Anacardium
occidentale}}.}{}
\verb{kaka}{}{[ˈkaka]}{(interj.)}{1}{}{Caramba!}{Cf. \textbf{akaka}.}{}{}
\verb{kaka}{}{[ˈkaka]}{(n.)}{1}{}{Excrementos.}{}{}
\verb{kakaw}{}{[kaˈkaw]}{(n.)}{1}{}{Cacau.}{Cf. \textbf{kakayu}.}{}{}
\verb{kakayu}{}{[kaˈkaju]}{(n.)}{1}{}{Cacau.}{}{}
\verb{kakayu}{}{[kaˈkaju]}{(n.)}{1}{}{Cacaueiro.}{\textbf{\textit{Theobroma
cacao}.}}{}
\verb{kakayu-ntêlu}{}{[kaˈkaju ˈntelu]}{(n.)}{1}{}{Cápsula do cacau.}{}{}
\verb{kaki}{}{[kaˈki]}{(interj.)}{1}{}{Caramba!}{Cf. \textbf{akaka}.}{}{}
\verb{kaki}{}{[kaˈki]}{(n.)}{1}{}{Jaguareçá.}{\textbf{\textit{Holocentrus
ascensionis}.}}{}
\verb{kaki}{}{[kaˈki]}{(n.)}{2}{}{Caqui.}{\textbf{\textit{Holocentrus
ascensionis}}.}{}
\verb{kaki}{}{[kaˈki]}{(n.)}{3}{}{Caqui.}{}{}
\verb{kaki}{}{[kaˈki]}{(n.)}{4}{}{Caquizeiro.}{\textbf{\textit{Diospyros
kaki}.}}{}
\verb{kaki}{}{[kaˈki]}{(n.)}{5}{}{Dióspiro.}{}{}
\verb{kaki}{}{[kaˈki]}{(n.)}{6}{}{Diospireiro.}{\textbf{\textit{Diospyros
kaki}.}}{}
\verb{ka konta ope}{}{[ˈka k\~ɔˈtɔpɛ]}{(expr.)}{1}{}{De mansinho.}{}{}
\verb{ka konta ope}{}{[ˈka k\~ɔˈtɔpɛ]}{(expr.)}{2}{}{Em bicos de pés.}{}{}
\verb{ka konta ope}{}{[ˈka k\~ɔˈtɔpɛ]}{(expr.)}{3}{}{Na ponta dos pés.}{}{}
\verb{ka konta ope}{}{[ˈka k\~ɔˈtɔpɛ]}{(expr.)}{4}{}{Sem ruído.}{}{}
\verb{ka konta ope}{}{[ˈka k\~ɔˈtɔpɛ]}{(expr.)}{5}{}{Silenciosamente.}{}{}
\verb{kaku}{}{[ˈkaku]}{(n.)}{1}{}{Caco.}{}{}
\verb{kaku}{}{[ˈkaku]}{(n.)}{1}{}{Fragmento.}{}{}
\verb{kakunda}{}{[kaˈkũda]}{(adj.)}{1}{}{Bossa.}{}{}
\verb{kakunda}{}{[kaˈkũda]}{(adj.)}{2}{}{Corcunda.}{}{}
\verb{kala}{}{[ˈkala]}{(n.)}{1}{}{Cara.}{}{}%
\verb{kala}{}{[ˈkala]}{(n.)}{2}{}{Rosto.}{}{}
\verb{kala}{}{[kaˈla]}{(v.)}{1}{}{Escalar o peixe.}{}{}
\verb{kalabana}{}{[ˈkalaˈbana]}{(n.)}{1}{}{Placebo.}{Cf. \textbf{awa
kalabana}.}{}
\verb{kalabusu}{}{[kalaˈbusu]}{(n.)}{1}{}{Calabouço.}{Cf.
\textbf{kalabuxu}.}{}{}%
\verb{kalabuxu}{}{[kalaˈbuʃu]}{(n.)}{2}{}{Calabouço.}{}{}
\verb{kalabuxu}{}{[kalaˈbuʃu]}{(n.)}{3}{}{Prisão.}{}{}%
\verb{kaladu}{}{[kaˈladu]}{(adj.)}{1}{}{Calado.}{}{}
\verb{kaladu}{}{[kaˈladu]}{(adj.)}{1}{}{Escalado o peixe.}{}{}
\verb{kalakala}{}{[ˈkalaˈkala]}{(v.)}{1}{}{Capinar.}{}{}
\verb{kalakala}{}{[ˈkalaˈkala]}{(v.)}{2}{}{Carpir.}{}{}
\verb{kalambola}{}{[kal\~{\textturna}ˈbɔla]}{(n.)}{1}{}{Carambola.}{}{}%
\verb{kalambola}{}{[kal\~{\textturna}ˈbɔla]}{(n.)}{2}{}{Caramboleira.}{\textbf{\textit{Averrhoa
carambola}.}}{}
\verb{kalapitu}{}{[kalaˈpitu]}{(n.)}{1}{}{Carrapito.}{}{}
\verb{kalema}{}{[kaˈlɛma]}{(n.)}{1}{}{Agitação.}{}{}
\verb{kalema}{}{[kaˈlɛma]}{(n.)}{2}{}{Carestia.}{}{}
\verb{kalema}{}{[kaˈlɛma]}{(n.)}{1}{}{Escarcéu.}{}{}
\verb{kalema}{}{[kaˈlɛma]}{(n.)}{3}{}{Maré alta.}{}{}
\verb{kalema}{}{[kaˈlɛma]}{(n.)}{4}{}{Maré cheia.}{}{}
\verb{kalema}{}{[kaˈlɛma]}{(n.)}{5}{}{Tornado.}{}{}
\verb{kaleta}{}{[kaˈlɛta]}{(adj.)}{1}{}{Careta.}{}{}
\verb{kaleta}{}{[kaˈlɛta]}{(adj.)}{2}{}{Desgraçado.}{}{}
\verb{kaleta}{}{[kaˈlɛta]}{(adj.)}{4}{}{Imprestável.}{}{}
\verb{kaleta}{}{[kaˈlɛta]}{(adj.)}{5}{}{Inútil.}{}{}
\verb{kaleta}{}{[kaˈlɛta]}{(n.)}{1}{}{Careta.}{}{}
\verb{kaleta}{}{[kaˈlɛta]}{(n.)}{2}{}{Desgraçado.}{}{}
\verb{kaleta}{}{[kaˈlɛta]}{(n.)}{3}{}{Enfezado.}{}{}
\verb{kaleta}{}{[kaˈlɛta]}{(n.)}{4}{}{Imprestável.}{}{}
\verb{kaleta}{}{[kaˈlɛta]}{(n.)}{5}{}{Inútil.}{}{}
\verb{kalêxidu}{}{[kaleˈʃidu]}{(adj.)}{1}{}{Sinusite.}{}{}
\verb{kali}{}{[ˈkali]}{(int.)}{1}{}{Qual.}{}{}
\verb{kali}{}{[ˈkali]}{(n.)}{1}{}{Cal.}{}{}%
\verb{kali}{}{[ˈkali]}{(n.)}{2}{}{Massa de cal.}{}{}
\verb{kalikali}{}{[ˈkaliˈkali]}{(n.)}{1}{}{Aposta.}{}{}%
\verb{kalima}{}{[kaliˈma]}{(n.)}{2}{}{Tabaco em pó.}{}{}
\verb{kaliptu}{}{[kaˈliptu]}{(n.)}{1}{}{Eucalipto.}{\textbf{\textit{Eucalyptus
globulus}}.}{}
\verb{kalôlô}{}{[kaˈlolo]}{(n.)}{1}{}{Calor.}{}{}%
\verb{kalôlô}{}{[kaˈlolo]}{(n.)}{2}{}{Transpiração.}{}{}
\verb{kalu}{}{[ˈkalu]}{(n.)}{1}{}{Calo.}{}{}%
\verb{kalu}{}{[ˈkalu]}{(n.)}{1}{}{Calulu.}{Prato típico preparado com óleo de
palma, folhas e outras plantas.}{}
\verb{kaluma}{}{[ˈkaluma]}{(n.)}{1}{}{Calma.}{}{}
\verb{kaluma}{}{[kaluˈma]}{(v.)}{1}{}{Acalmar(-se).}{}{}
\verb{kalumanu}{}{[kaluˈmanu]}{(n.)}{1}{}{Carne humana.}{}{}
\verb{kalu-mikoto}{}{[ˈkalu mikɔˈtɔ]}{(n.)}{1}{}{Calulu de pé de porco
(de)fumado.}{Cf. \textbf{kalu-mukoto}.}{}
\verb{kalu-mukoto}{}{[ˈkalu mukɔˈtɔ]}{(n.)}{1}{}{Calulu de pé de porco
(de)fumado.}{}{}
\verb{kalu-nganha}{}{[ˈkalu ˈŋg\~{\textturna}ɲa]}{(n.)}{1}{}{Calulu de
galinha.}{}{}
\verb{kalu-pixi}{}{[ˈkalu ˈpiʃi]}{(n.)}{1}{}{Calulu de peixe.}{}{}
\verb{kalu-pletu}{}{[ˈkalu ˈplɛtu]}{(n.)}{1}{}{Calulu para crianças e/ou
mulheres em dietas pós-parto.}{}{}
\verb{kama}{}{[ˈkama]}{(n.)}{1}{}{Cama.}{}{}%
\verb{kama}{}{[ˈkama]}{(n.)}{2}{}{Escama de peixe.}{}{}
\verb{kama}{}{[kaˈma]}{(v.)}{1}{}{Escamar.}{}{}
\verb{kamala}{}{[ˈkamala]}{(n.)}{1}{}{Câmara.}{}{}
\verb{kamalada}{}{[kamaˈlada]}{(n.)}{1}{}{Amante.}{}{}
\verb{kamalada}{}{[kamaˈlada]}{(n.)}{1}{}{Camarada.}{}{}
\verb{kamanda}{}{[kam\~{\textturna}ˈda]}{(int.)}{1}{}{Por que.}{}{}
\verb{kambla-sata}{}{[ˈk\~{\textturna}bla
saˈta]}{(interj.)}{1}{}{Caramba!}{}{}
\verb{kaminza}{}{[kaˈmĩza]}{(n.)}{1}{}{Camisa.}{}{}
\verb{kaminzola}{}{[kam\~iˈzɔla]}{(n.)}{1}{}{Camisola.}{}{}
\verb{kamiza}{}{[kaˈmiza]}{(n.)}{1}{}{Camisa.}{Cf. \textbf{kaminza}.}{}{}
\verb{kamizada}{}{[kamiˈzada]}{(n.)}{1}{}{Brincadeira.}{}{}
\verb{kamizola}{}{[kamiˈzɔla]}{(n.)}{1}{}{Camisola.}{Cf.
\textbf{kaminzola}.}{}{}
\verb{kampanha}{}{[k\~{\textturna}ˈp\~{\textturna}ɲa]}{(n.)}{1}{}{Campanha
eleitoral.}{}{}
\verb{kampon}{}{[k\~{\textturna}ˈpõ]}{(adv.)}{1}{}{Enorme.}{}{}
\verb{kampon}{}{[k\~{\textturna}ˈpõ]}{(n.)}{1}{}{Capô do carro.}{}{}
\verb{kampu}{}{[ˈk\~{\textturna}pu]}{(n.)}{1}{}{Campo.}{}{}
\verb{kampu-d'avyon}{}{[ˈk\~{\textturna}pu
daˈvjõ]}{(n.)}{1}{}{Aeroporto.}{}{}
\verb{kampyon}{}{[k\~{\textturna}ˈpj\~ɔ]}{(n.)}{1}{}{Campeão.}{}{}
\verb{kamunsela}{}{[kam\~uˈsɛla]}{(n.)}{1}{}{Tecelão-grande de São
Tomé.}{\textbf{\textit{Ploceus grandis}.} Cf. \textbf{kamusela}.}{}
\verb{kamusela}{}{[kamuˈsɛla]}{(n.)}{1}{}{Tecelão-grande de São
Tomé.}{\textbf{\textit{Ploceus grandis}.}}{}
\verb{kamuzenze}{}{[kamuz\~ɛˈzɛ]}{(adj.)}{1}{}{Definhado.}{}{}
\verb{kamya}{}{[kaˈmja]}{(int.)}{1}{}{Onde.}{}{}
\verb{kamya}{}{[kaˈmja]}{(n.)}{1}{}{Caminho.}{}{}%
\verb{kamya}{}{[kaˈmja]}{(n.)}{2}{}{Local.}{}{}
\verb{kamya-fe-zawa}{}{[kaˈmja ˈfɛ ˈzawa]}{(n.)}{1}{}{Sexo (feminino ou
masculino).}{}{}
\verb{kamyoneti}{}{[kamjɔˈnɛti]}{(n.)}{1}{}{Camião.}{}{}%
\verb{kamyoneti}{}{[kamjɔˈnɛti]}{(n.)}{2}{}{Caminhão.}{}{}
\verb{kana}{}{[ˈkana]}{(n.)}{1}{}{Fanfarronice.}{}{}
\verb{kana}{}{[kaˈna]}{(n.)}{1}{}{Cana-de-açúcar.}{\textbf{\textit{Saccharum
officinarum}}.}{}
\verb{kana-blanku}{}{[kaˈna
ˈbl\~{\textturna}ku]}{(n.)}{1}{}{Cana-de-açúcar.}{Cf. \textbf{kana}.}{}
\verb{kanada}{}{[kaˈnada]}{(interj.)}{1}{}{Êpa!}{}{}
\verb{kanalemi}{}{[kanaˈlɛmi]}{(adj.)}{1}{}{Franzino.}{}{}
\verb{kanalemi}{}{[kanaˈlɛmi]}{(n.)}{1}{}{Parasita.}{}{}
\verb{kanalemi-ngandu}{}{[kanaˈlɛmi
ŋg\~{\textturna}ˈdu]}{(n.)}{1}{}{Rêmora.}{\textbf{\textit{Remora remora}}.}{}
\verb{kana-makaku}{}{[kaˈna maˈkaku]}{(n.)}{1}{}{Cana silvestre.}{}{}
\verb{kanan}{}{[kaˈn\~{\textturna}]}{(n.)}{1}{}{Cana-de-açúcar.}{Cf.
\textbf{kana}}{}{}
\verb{kanapixtula}{}{[kanaˈpiʃtula]}{(n.)}{1}{}{Cássia-oficinal.}{\textit{\textbf{Cassia
fistula}}.}{}{}
\verb{kanapixtula}{}{[kanaˈpiʃtula]}{(n.)}{2}{}{Chuva-de-ouro.}{\textit{\textbf{Cassia
fistula}}.}{}{}
\verb{kana-pletu}{}{[kaˈna ˈplɛtu]}{(n.)}{1}{}{Cana-de-açúcar.}{Cf.
\textbf{kana}.}{}
\verb{kanaryu}{}{[kaˈna{\textfishhookr}ju]}{(n.)}{1}{}{Canário.}{}{}
\verb{kandja}{}{[k\~{\textturna}ˈdʒa]}{(n.)}{1}{}{Candeeiro.}{}{}%
\verb{kandja}{}{[k\~{\textturna}ˈdʒa]}{(n.)}{2}{}{Candeia.}{}{}%
\verb{kandja}{}{[k\~{\textturna}ˈdʒa]}{(n.)}{3}{}{Luz.}{}{}
\verb{kandja}{}{[ˈk\~{\textturna}dʒa]}{(n.)}{1}{}{Canja.}{}{}
\verb{kandleza}{}{[k\~{\textturna}ˈdlɛza]}{(v.)}{1}{}{Atrofiar.}{}{}
\verb{kandleza}{}{[k\~{\textturna}ˈdlɛza]}{(v.)}{1}{}{Inutilizar.}{}{}
\verb{kandlezadu}{}{[k\~{\textturna}dlɛˈzadu]}{(adj.)}{1}{}{Atrofiado.}{}{}
\verb{kandlezadu}{}{[k\~{\textturna}dlɛˈzadu]}{(v.)}{1}{}{Escanzelado.}{}{}
\verb{kaneka}{}{[kaˈnɛka]}{(n.)}{1}{}{Caneca.}{}{}
\verb{kaneku}{}{[kaˈnɛku]}{(n.)}{1}{}{Caneca.}{}{}
\verb{kanela}{}{[kaˈnɛla]}{(n.)}{1}{}{Canela.}{}{}%
\verb{kanela}{}{[kaˈnɛla]}{(n.)}{2}{}{Caneleira.}{\textbf{\textit{Cinnamomum verum}}.}{}
\verb{kanêlu}{}{[kaˈnelu]}{(n.)}{1}{}{Carneiro.}{}{}
\verb{kaneta}{}{[kaˈnɛta]}{(n.)}{1}{}{Caneta.}{}{}
\verb{kanfini}{}{[k\~{\textturna}fiˈni]}{(n.)}{1}{}{Cambalhota.}{}{}{}
%\verb{kanfini}{}{[k\~{\textturna}fiˈni]}{(n.)}{1}{}{Fazer o pino.}{}{}%
%\verb{kanfini}{}{[k\~{\textturna}fiˈni]}{(n.)}{2}{}{Ficar de ponta-cabeça.}{}{}
\verb{kanfini}{}{[k\~{\textturna}fiˈni]}{(n.)}{1}{}{Pino.}{}{}
\verb{kanfini}{}{[k\~{\textturna}fiˈni]}{(n.)}{1}{}{Ponta-cabeça.}{Cf. \textbf{bila kanfini}.}{}%
\verb{kanga}{}{[k\~{\textturna}ˈga]}{(n.)}{1}{}{\textit{Kanga}.}{\textbf{\textit{Pontinus
kuhlii}.}}{}
\verb{kanga}{}{[ˈk\~{\textturna}ga]}{(n.)}{1}{}{Canga.}{}{}
\verb{kanga}{}{[ˈk\~{\textturna}ga]}{(n.)}{1}{}{Cangalha.}{}{}
\verb{kanga}{}{[ˈk\~{\textturna}ga]}{(n.)}{2}{}{Jugo.}{}{}
\verb{kanga}{}{[k\~{\textturna}ˈga]}{(v.)}{1}{}{Amarrar.}{}{}
\verb{kanga}{}{[k\~{\textturna}ˈga]}{(v.)}{3}{}{Prender.}{}{}
\verb{kanha}{}{[kaˈɲa]}{(v.)}{1}{}{Acanhar.}{}{}
\verb{kanha}{}{[kaˈɲa]}{(v.)}{2}{}{Inibir.}{}{}
\verb{kanha}{}{[kaˈɲa]}{(v.)}{3}{}{Ter vergonha.}{}{}
\verb{kanhon}{}{[kaˈɲõ]}{(n.)}{1}{}{Canhão.}{}{}
\verb{kani}{}{[ˈkani]}{(n.)}{1}{}{Carne.}{}{}
\verb{kanidu}{}{[kaˈnidu]}{(n.)}{1}{}{Caniço.}{}{}
\verb{kanimbôtô}{}{[kanĩˈboto]}{(interj.)}{1}{}{Caramba!}{}{}
\verb{kanimbôtô}{}{[kanĩˈboto]}{(n.)}{1}{}{Atrapalhado.}{}{}
\verb{kanimbôtô}{}{[kanĩˈboto]}{(n.)}{2}{}{Tolo.}{}{}%
\verb{kanivêtê}{}{[kaniˈvete]}{(n.)}{1}{}{Canivete.}{}{}
\verb{kankankan}{}{[k\~{\textturna}k\~{\textturna}ˈk\~{\textturna}]}{(id.)}{1}{}{Cf.
\textbf{lizu kankankan.}}{}{}
\verb{kankankan}{}{[k\~{\textturna}k\~{\textturna}ˈk\~{\textturna}]}{(id.)}{2}{}{Cf.
\textbf{têdu kankankan.}}{}{}
\verb{kankankan}{}{[k\~{\textturna}k\~{\textturna}ˈk\~{\textturna}]}{(id.)}{3}{}{Cf.
\textbf{zedu kankankan.}}{}{}
\verb{kankankan}{}{[k\~{\textturna}k\~{\textturna}ˈk\~{\textturna}]}{(id.)}{3}{}{Cf.
\textbf{zulu kankankan.}}{}{}
\verb{kansa}{}{[k\~{\textturna}ˈsa]}{(v.)}{1}{}{Cansar(-se).}{}{}
\verb{kansa}{}{[k\~{\textturna}ˈsa]}{(v.)}{2}{}{Descansar.}{}{}%
\verb{kansadu}{}{[k\~{\textturna}ˈsadu]}{(adj.)}{1}{}{Cansado.}{}{}
\verb{kanson}{}{[k\~{\textturna}ˈs\~ɔ]}{(n.)}{1}{}{Canção.}{}{}
\verb{kanson}{}{[k\~{\textturna}ˈs\~ɔ]}{(n.)}{2}{}{Música.}{}{}
\verb{kansu}{}{[ˈk\~{\textturna}su]}{(n.)}{1}{}{Asma.}{}{}
\verb{kanta}{}{[k\~{\textturna}ˈta]}{(v.)}{1}{}{Cantar.}{}{}
\verb{kanta}{}{[k\~{\textturna}ˈta]}{(n.)}{1}{}{Cantiga.}{}{}
\verb{kantadô}{}{[k\~{\textturna}taˈdo]}{(n.)}{1}{}{Cantor.}{}{}
\verb{kantêlu}{}{[k\~{\textturna}ˈtelu]}{(n.)}{1}{}{Canteiro.}{}{}
\verb{kantlela}{}{[k\~{\textturna}ˈtlɛla]}{(n.)}{1}{}{Cantinho.}{}{}
\verb{kantlela}{}{[k\~{\textturna}ˈtlɛla]}{(n.)}{1}{}{Canto.}{}{}
\verb{kantlela}{}{[k\~{\textturna}ˈtlɛla]}{(n.)}{1}{}{Esquina da casa.}{}{}
\verb{kantlela}{}{[k\~{\textturna}ˈtlɛla]}{(n.)}{1}{}{Estante de parede.}{}{}\verb{kantôlô}{}{[k\~{\textturna}ˈtolo]}{(n.)}{1}{}{Aquele que entoa
ladainhas nos velórios.}{}{}
\verb{kantôlô}{}{[k\~{\textturna}ˈtolo]}{(n.)}{1}{}{Cantor.}{}{}
\verb{kantu}{}{[ˈk\~{\textturna}tu]}{(n.)}{1}{}{Canto.}{}{}
\verb{kantu}{}{[ˈk\~{\textturna}tu]}{(int.)}{1}{}{Quanto.}{}{}
\verb{kantxin}{}{[k\~{\textturna}ˈtʃĩ]}{(n.)}{2}{}{Cantinho.}{}{}
\verb{kantxin}{}{[k\~{\textturna}ˈtʃĩ]}{(n.)}{1}{}{Canto.}{}{}
\verb{kantxina}{}{[k\~{\textturna}ˈtʃĩna]}{(n.)}{1}{}{Cantina.}{}{}
\verb{kanu}{}{[ˈkanu]}{(n.)}{1}{}{Cano.}{}{}
\verb{kanvi}{}{[ˈk\~{\textturna}vi]}{(n.)}{1}{}{Palha do andim, depois de
extraído o óleo de palma.}{}
\verb{kanwa}{}{[ˈkanwa]}{(n.)}{1}{}{Canoa.}{}{}
\verb{kanwa}{}{[ˈkanwa]}{(n.)}{2}{}{Canoa onde se esmaga e se lava o andim,
para se extrair o óleo de palma.}{}{}
\verb{kanxika}{}{[k\~{\textturna}ˈʃika]}{(n.)}{1}{}{Canjica.}{}{}
\verb{kanza}{}{[k\~{\textturna}ˈza]}{(n.)}{1}{}{Reco-reco.}{}{}{}
\verb{kapa}{}{[ˈkapa]}{(n.)}{1}{}{Capa.}{}{}
\verb{kapa}{}{[ˈkapa]}{(n.)}{2}{}{Manto.}{}{}
\verb{kapa}{}{[kaˈpa]}{(v.)}{1}{}{Abafar.}{}{}
\verb{kapa}{}{[kaˈpa]}{(v.)}{1}{}{Castrar.}{}{}
\verb{kapa}{}{[kaˈpa]}{(v.)}{2}{}{Esterilizar.}{}{}
\verb{kapadu}{}{[kaˈpadu]}{(adj.)}{1}{}{Capado.}{}{}
\verb{kapadu}{}{[kaˈpadu]}{(adj.)}{2}{}{Castrado.}{}{}
\verb{kapanga}{}{[kaˈp\~{\textturna}ga]}{(n.)}{1}{}{Capanga.}{}{}
\verb{kapasêtê}{}{[kapaˈsete]}{(n.)}{1}{}{Capacete.}{}{}%
\verb{kapataji}{}{[kapaˈtaʒi]}{(n.)}{1}{}{Capataz.}{}{}
\verb{kapela}{}{[kaˈpɛla]}{(n.)}{1}{}{Capela.}{}{}%
\verb{kapela}{}{[kaˈpɛla]}{(n.)}{2}{}{Igreja.}{}{}
\verb{kapin}{}{[kaˈpĩ]}{(n.)}{1}{}{Capim.}{}{}
\verb{kapiton}{}{[kapiˈtõ]}{(n.)}{2}{}{Canário-do-mar.}{\textbf{\textit{Anthias
anthias}}.}{}%
\verb{kapiton}{}{[kapiˈtõ]}{(n.)}{1}{}{Capitão.}{}{}%
\verb{kapiton}{}{[kapiˈtõ]}{(n.)}{3}{}{Pau-capitão.}{Cf.
\textbf{po-kapiton}.}{}{}
\verb{kapiton-vapô}{}{[kapiˈtõ vaˈpo]}{(n.)}{1}{}{Almirante.}{}{}
\verb{kapotxi}{}{[kaˈpɔtʃi]}{(n.)}{1}{}{Capa de chuva.}{}{}
\verb{kapotxi}{}{[kaˈpɔtʃi]}{(n.)}{1}{}{Capote.}{}{}
\verb{kapwela}{}{[kaˈpwɛla]}{(n.)}{1}{}{Capoeira.}{}{}
\verb{kapwele}{}{[kapwɛˈlɛ]}{(n.)}{1}{}{Armadilha utilizada para capturar
morcegos.}{}{}
\verb{karu}{}{[ˈka{\textfishhookr}u]}{(adj.)}{1}{}{Caro.}{}{}
\verb{karu}{}{[ˈka{\textfishhookr}u]}{(adj.)}{2}{}{Dispendioso.}{}{}
\verb{karu}{}{[ˈka{\textfishhookr}u]}{(n.)}{1}{}{Carro.}{}{}
\verb{karu-di-plasa}{}{[ˈka{\textfishhookr}u di
ˈplasa]}{(n.)}{1}{}{Táxi.}{}{}
\verb{kasa}{}{[ˈkasa]}{(n.)}{1}{}{Caça.}{}{}
\verb{kasa}{}{[kaˈsa]}{(v.)}{1}{}{Caçar.}{}{}
\verb{kasadô}{}{[kasaˈdo]}{(n.)}{1}{}{Caçador.}{}{}
\verb{kasi}{}{[ˈkasi]}{(n.)}{1}{}{Cálice.}{}{}
\verb{kasku}{}{[ˈkasku]}{(n.)}{1}{}{Capacidade.}{}{}
\verb{kasku}{}{[ˈkasku]}{(n.)}{1}{}{Caspa.}{}{}
\verb{kasku}{}{[ˈkasku]}{(n.)}{1}{}{Cérebro.}{}{}
\verb{kasô}{}{[kaˈso]}{(n.)}{1}{}{Cachorro.}{}{}
\verb{kasô}{}{[kaˈso]}{(n.)}{2}{}{Cão.}{\textit{\textbf{Canis lupus
familiaris}}.}{}%
\verb{kasô-dêsu}{}{[kaˈso ˈdesu]}{(n.)}{1}{}{Criança.}{}{}
\verb{kasô-montxa}{}{[kaˈso ˈm\~otʃa]}{(n.)}{1}{}{Cão de caça.}{}{}
\verb{kasô-mwala}{}{[kaˈso ˈmwala]}{(n.)}{1}{}{Cadela.}{}{}
\verb{kason}{}{[kaˈsõ]}{(n.)}{1}{}{Caixão.}{}{}
\verb{kasu}{}{[ˈkasu]}{(adj.)}{1}{}{Avarento.}{}{}
\verb{kasu}{}{[ˈkasu]}{(adj.)}{2}{}{Mão-de-vaca.}{}{}
\verb{kasu}{}{[ˈkasu]}{(adj.)}{3}{}{Somítico.}{}{}
\verb{kasu}{}{[ˈkasu]}{(adj.)}{4}{}{Sovina.}{}{}
\verb{kaswada}{}{[kaˈswada]}{(n.)}{1}{}{Troça.}{}{}
\verb{kaswada}{}{[kaˈswada]}{(n.)}{1}{}{Zombaria.}{}{}
\verb{kata}{}{[ˈkata]}{(n.)}{1}{}{Carta.}{}{}%
\verb{kata}{}{[ˈkata]}{(n.)}{2}{}{Colher de madeira, geralmente de três
pontas, usada para triturar o \textbf{izakentxi}.}{}{}
\verb{kata}{}{[kaˈta]}{(v.)}{1}{}{Esmagar.}{}{}
\verb{kata}{}{[kaˈta]}{(v.)}{2}{}{Moer.}{}{}%
\verb{kata-d'ôbô}{}{[ˈkata
doˈbo]}{(n.)}{1}{}{\textit{Kata-d'obô}.}{\textbf{\textit{Tabernaemontana
stenosyphon}.}}{}
\verb{kata-d'ôbô}{}{[ˈkata doˈbo]}{(n.)}{2}{}{Pau-lírio.}{\textbf{\textit{Tabernaemontana stenosyphon}.}}{}
\verb{kata-kiyô}{}{[ˈkata kiˈjo]}{(n.)}{1}{}{\emph{Kata-kiyô}.}{\textbf{\textit{Voacanga lemosii}.}}{}
\verb{kata-kwene}{}{[ˈkata kwɛˈnɛ]}{(n.)}{1}{}{\textit{Kata-kwene}.}{\textbf{\textit{Rauwolfia caffra}.}}{}
\verb{katalu}{}{[kaˈtalu]}{(n.)}{1}{}{Catarro.}{}{}
\verb{kata-manginga}{}{[ˈkata m\~{\textturna}ˈg\~iga]}{(n.)}{1}{}{\textit{Kata-manginga}.}{\textbf{\textit{Rauwolfia vomitoria}.}}{}
\verb{kata-nglandji}{}{[ˈkata ˈŋgl\~{\textturna}dʒi]}{(n.)}{1}{}{\emph{Kata-nglandji}.}{\textbf{\textit{Rauwolfia caffra}.}}{}
\verb{kata-pikina}{}{[ˈkata piˈkina]}{(n.)}{1}{}{\emph{Kata-pikina}.}{\textbf{\textit{Rauwolfia vomitoria}.}}{}
\verb{katoliku}{}{[kaˈtᴐliku]}{(n.)}{1}{}{Católico.}{}{}
\verb{katon}{}{[kaˈtõ]}{(n.)}{1}{}{Cartão.}{}{}
\verb{katôzê}{}{[kaˈtoze]}{(num.)}{1}{}{Catorze.}{}{}
\verb{katuxu}{}{[kaˈtuʃu]}{(n.)}{1}{}{Cartucho.}{}{}
\verb{katxibu}{}{[kaˈtʃibu]}{(adj.)}{1}{}{Cativo.}{}{}%
\verb{katxibu}{}{[kaˈtʃibu]}{(adj.)}{2}{}{Escravo.}{}{}%
\verb{katxibu}{}{[kaˈtʃibu]}{(adj.)}{3}{}{Preso.}{}{}
\verb{katxibu}{}{[kaˈtʃibu]}{(n.)}{2}{}{Cativeiro.}{}{}
\verb{katxibu}{}{[kaˈtʃibu]}{(n.)}{1}{}{Cativo.}{}{}
\verb{katxibu}{}{[kaˈtʃibu]}{(n.)}{3}{}{Escravo.}{}{}
\verb{katxibu}{}{[kaˈtʃibu]}{(n.)}{4}{}{Subordinado.}{}{}
\verb{katxina-gêsa}{}{[kaˈtʃina
ˈgesa]}{(n.)}{1}{}{\textit{Katxina-gêsa}.}{\textit{\textbf{Clausena
anisata}}.}{}
\verb{katxinga}{}{[kaˈtʃ\~iga]}{(n.)}{1}{}{Catinga.}{}{}
\verb{katxisali}{}{[katʃiˈsali]}{(n.)}{1}{}{Castiçal.}{}{}
\verb{kavala}{}{[kaˈvala]}{(n.)}{1}{}{Cavala.}{\textbf{\textit{Decapterus
macarellus}}.}{}
\verb{kaxa}{}{[ˈkaʃa]}{(n.)}{1}{}{Caixa.}{}{}
\verb{kaxa}{}{[ˈkaʃa]}{(n.)}{2}{}{Corpo em transe.}{}{}
\verb{kaxalamba}{}{[kaʃaˈl\~{\textturna}ba]}{(n.)}{1}{}{Aguardente misturada
com água.}{}{}
\verb{kaxalamba}{}{[kaʃaˈl\~{\textturna}ba]}{(n.)}{1}{}{Cacharamba.}{}{}
\verb{kaxêlu}{}{[kaˈʃelu]}{(n.)}{1}{}{Balconista.}{}{}
\verb{kaxêlu}{}{[kaˈʃelu]}{(n.)}{1}{}{Caixeiro.}{}{}
\verb{kaxika}{}{[ˈkaʃika]}{(n.)}{1}{}{Casca.}{Cf. \textbf{kaxka}.}{}{}
\verb{kaxinha}{}{[kaˈʃ\~iɲa]}{(n.)}{1}{}{Calcinha.}{}{}
\verb{kaxka}{}{[ˈkaʃka]}{(n.)}{1}{}{Casca.}{}{}
\verb{kaxka-boka}{}{[ˈkaʃka ˈbɔka]}{(n.)}{1}{}{Lábios.}{}{}
\verb{kaxta}{}{[ˈkaʃta]}{(n.)}{1}{}{Casta.}{}{}%
\verb{kaxta}{}{[ˈkaʃta]}{(n.)}{2}{}{Espécie.}{}{}%
\verb{kaxta}{}{[ˈkaʃta]}{(n.)}{3}{}{Qualidade.}{}{}%
\verb{kaxta}{}{[ˈkaʃta]}{(n.)}{4}{}{Raça.}{}{}
\verb{kaxtanha}{}{[kaʃˈt\~{\textturna}ɲa]}{(n.)}{1}{}{Castanha.}{}{}
\verb{kaxtanha-kaju}{}{[kaʃˈt\~{\textturna}ɲa
kaˈʒu]}{(n.)}{1}{}{Castanha-de-caju.}{}{}
\verb{kaxtelu}{}{[kaʃˈtɛlu]}{(n.)}{1}{}{Castelo.}{}{}
\verb{kaxtiga}{}{[kaʃtiˈga]}{(v.)}{1}{}{Castigar.}{}{}
\verb{kaxtigu}{}{[kaʃˈtigu]}{(n.)}{1}{}{Castigo.}{}{}
\verb{kaxtiku}{}{[ˈkaʃtiku]}{(n.)}{1}{}{Pomada cáustica.}{}{}
\verb{kaxupa}{}{[kaˈʃupa]}{(n.)}{1}{}{Cachupa.}{}{}
\verb{kaya}{}{[kaˈja]}{(v.)}{1}{}{Caiar.}{}{}
\verb{kaya}{}{[kaˈja]}{(v.)}{3}{}{Encalhar.}{}{}
\verb{kayadu}{}{[kaˈjadu]}{(adj.)}{1}{}{Caiado.}{}{}
\verb{kayadu}{}{[kaˈjadu]}{(adj.)}{2}{}{Encalhado.}{}{}
\verb{kaza}{}{[kaˈza]}{(n.)}{1}{}{Casamento.}{}{}
\verb{kaza}{}{[kaˈza]}{(v.)}{1}{}{Casar(-se).}{}{}
\verb{kaza-blanku}{}{[kaˈza ˈbl\~{\textturna}ku]}{(n.)}{1}{}{Casamento à moda
europeia.}{}{}
\verb{kazadu}{}{[kaˈzadu]}{(adj.)}{1}{}{Casado.}{}{}
\verb{kazamentu}{}{[kazaˈmẽtu]}{(n.)}{1}{}{Casamento.}{}{}
\verb{kazena}{}{[kaˈzɛna]}{(n.)}{1}{}{Caserna.}{}{}
\verb{kazena}{}{[kaˈzɛna]}{(n.)}{2}{}{Quartel.}{}{}
\verb{kazu}{}{[ˈkazu]}{(n.)}{1}{}{Caso.}{}{}
\verb{kazumbi}{}{[kazũˈbi]}{(n.)}{1}{}{Alma.}{}{}
\verb{kazumbi}{}{[kazũˈbi]}{(n.)}{3}{}{Defunto.}{}{}
\verb{kazumbi}{}{[kazũˈbi]}{(n.)}{2}{}{Magia.}{}{}
\verb{ke}{}{[ˈkɛ]}{(n.)}{1}{}{Casa.}{}{}
\verb{ke}{}{[ˈkɛ]}{(n.)}{1}{}{Ninho.}{}{}
\verb{kê}{}{[ˈke]}{(int.)}{2}{}{Qual.}{}{}
\verb{kê}{}{[ˈke]}{(int.)}{1}{}{Que.}{}{}%
\verb{kê}{}{[ˈke]}{(interj.)}{1}{}{Nossa!}{Cf. \textbf{kyê}.}{}{}
\verb{ke-banhu}{}{[ˈkɛ ˈb\~{\textturna}ɲu]}{(n.)}{1}{}{Banheiro.}{}{}
\verb{ke-banhu}{}{[ˈkɛ ˈb\~{\textturna}ɲu]}{(n.)}{2}{}{Casa de banho.}{}{}
\verb{kebla}{}{[ˈkɛbla]}{(n.)}{1}{}{Gargalhada.}{}{}
\verb{kebla}{}{[kɛˈbla]}{(v.)}{1}{}{Fraturar.}{}{}
\verb{kebla}{}{[kɛˈbla]}{(v.)}{2}{}{Partir.}{}{}
\verb{kebla}{}{[kɛˈbla]}{(v.)}{3}{}{Quebrar.}{}{}
\verb{kebla dizê}{}{[ˈkɛbla diˈze]}{(expr.)}{1}{}{Ajoelhar.}{}{}
\verb{kebladu}{}{[kɛˈbladu]}{(adj.)}{2}{}{Partido.}{}{}
\verb{kebladu}{}{[kɛˈbladu]}{(adj.)}{1}{}{Quebrado.}{}{}%
\verb{kebla kaneku}{}{[kɛˈbla kaˈnɛku]}{(expr.)}{1}{}{Beber em excesso.}{}{}
\verb{keblankana}{}{[kɛbl\~{\textturna}ˈkana]}{(n.)}{1}{}{Bico-de-lacre.}{\textbf{\textit{Estrilda
astrild}}.}{}
\verb{keblankana}{}{[kɛbl\~{\textturna}ˈkana]}{(n.)}{1}{}{Erva-cão.}{\textbf{\textit{Eleusine
indica}}.}{}
\verb{keblankana-pletu}{}{[kɛbl\~{\textturna}ˈkana ˈplɛtu]}{(n.)}{1}{}{Freirinha.}{\textbf{\textit{Lonchura cucullata}}.}{}
\verb{kebla sono}{}{[kɛˈbla sᴐˈnᴐ]}{(expr.)}{1}{}{Dormitar.}{}{}
\verb{kebla winiwini}{}{[kɛˈbla wiˈniwiˈni]}{(expr.)}{1}{}{Fragmentar.}{}{}
\verb{kebla zegezege}{}{[kɛˈbla zɛˈgɛzɛˈgɛ]}{(expr.)}{1}{}{Partir por
completo, sem desassociar-se.}{}{}
\verb{keda}{}{[ˈkɛda]}{(n.)}{1}{}{Luta-livre.}{}{}
\verb{keda}{}{[ˈkɛda]}{(n.)}{2}{}{Queda.}{}{}%
\verb{kedadji}{}{[kɛˈdadʒi]}{(n.)}{1}{}{Claridade.}{}{}
\verb{kedadji-nwa}{}{[kɛˈdadʒi ˈnwa]}{(n.)}{1}{}{Luar.}{}{}
\verb{kedaji}{}{[kɛˈdaʒi]}{(n.)}{1}{}{Claridade.}{Cf. \textbf{kedadji}.}{}{}
\verb{kê dja}{}{[ˈke ˈdʒa]}{(int.)}{1}{}{Quando.}{}{}
\verb{kega}{}{[ˈkɛga]}{(n.)}{1}{}{Bagagem.}{}{}
\verb{kega}{}{[ˈkɛga]}{(n.)}{2}{}{Carga.}{}{}
\verb{kê kamya}{}{[ˈke ˈkamja]}{(int.)}{1}{}{Onde.}{}{}
\verb{kê kwa}{}{[ke ˈkwa]}{(int.)}{1}{}{O que.}{}{}
\verb{kela}{}{[ˈkɛla]}{(adv.)}{1}{}{E tal.}{}{}%
\verb{kêlê}{}{[keˈle]}{(v.)}{2}{}{Acreditar.}{}{}%
\verb{kêlê}{}{[keˈle]}{(v.)}{1}{}{Crer.}{}{}%
\verb{kema}{}{[ˈkɛma]}{(n.)}{1}{}{Doença venérea.}{}{}
\verb{kema}{}{[kɛˈma]}{(v.)}{1}{}{Queimar.}{}{}
\verb{kemadu}{}{[kɛˈmadu]}{(adj.)}{1}{}{Queimado.}{}{}
\verb{kemadu}{}{[kɛˈmadu]}{(n.)}{1}{}{Porção de comida que fica agarrada no
fundo da panela.}{}{}
\verb{kê mind'ola}{}{[ˈke m\~iˈdᴐla]}{(int.)}{1}{}{A que horas.}{}{}
\verb{kê mind'ola}{}{[ˈke m\~iˈdᴐla]}{(int.)}{1}{}{Quando.}{}{}
\verb{ke-mulu}{}{[ˈkɛ ˈmulu]}{(n.)}{1}{}{Casa de alvenaria.}{}{}
\verb{ken}{}{[ˈkẽ]}{(int.)}{1}{}{Quem.}{Cf. \textbf{kê ngê}.}{}{}
\verb{kendiji}{}{[kẽˈdiʒi]}{(adv.)}{2}{}{Isto é.}{}{}
\verb{kendiji}{}{[kẽˈdiʒi]}{(adv.)}{1}{}{Quer dizer.}{}{}%
\verb{ke-nganha}{}{[ˈkɛ ŋg\~{\textturna}ˈɲa]}{(n.)}{1}{}{Capoeira.}{}{}
\verb{kê ngê}{}{[k\~eˈge]}{(int.)}{1}{}{Quem.}{}{}
\verb{ke-nglandji}{}{[ˈkɛ ˈŋgl\~{\textturna}dʒi]}{(n.)}{1}{}{Casa grande.}{}{}
\verb{kenta}{}{[k\~ɛˈta]}{(v.)}{2}{}{Aquecer.}{}{}%
\verb{kenta}{}{[k\~ɛˈta]}{(v.)}{1}{}{Aquentar.}{}{}%
\verb{kenta}{}{[k\~ɛˈta]}{(v.)}{3}{}{Requentar.}{}{}
\verb{kentadu}{}{[k\~ɛˈtadu]}{(adj.)}{1}{}{Aquecido.}{}{}
\verb{kenta mina}{}{[k\~ɛˈta ˈmina]}{(expr.)}{1}{}{Aconchegar o bebê.}{}{}%
\verb{kenta zuzuzu}{}{[k\~ɛˈta zuzuˈzu]}{(expr.)}{1}{}{Sobreaquecer.}{}{}
\verb{kentula}{}{[k\~ɛˈtula]}{(n.)}{1}{}{Aquecimento.}{}{}
\verb{kentula}{}{[k\~ɛˈtula]}{(n.)}{2}{}{Calor.}{}{}
\verb{kentxi}{}{[ˈkẽtʃi]}{(adj.)}{1}{}{Quente.}{}{}%
\verb{kentxi}{}{[ˈkẽtʃi]}{(adj.)}{2}{}{Voluptuoso.}{}{}
\verb{kentxi zuzuzu}{}{[ˈkẽtʃi zuzuˈzu]}{(expr.)}{1}{}{Quentíssimo.}{}{}
\verb{kê ola}{}{[ˈke ˈᴐla]}{(int.)}{1}{}{Quando.}{}{}
\verb{kê ola}{}{[ˈke ˈᴐla]}{(int.)}{1}{}{Que horas.}{}{}
\verb{kêsê}{}{[keˈse]}{(v.)}{1}{}{Esquecer(-se).}{}{}
\verb{ke-sinema}{}{[ˈkɛ siˈnɛma]}{(n.)}{1}{}{Cinema.}{}{}
\verb{ke-sôlê}{}{[ˈkɛ soˈle]}{(n.)}{1}{}{Casa de solteiro.}{}{}
\verb{ke-taba}{}{[ˈkɛ ˈtaba]}{(n.)}{1}{}{Casa de madeira.}{}{}
\verb{ketekete}{}{[kɛˈtɛkɛˈtɛ]}{(id.)}{1}{}{Cf. \textbf{ve ketekete.}}{}{}
\verb{kêtêkêtê}{}{[keˈtekeˈte]}{(n.)}{1}{}{Briga.}{}{}
\verb{kêtêkêtê}{}{[keˈtekeˈte]}{(n.)}{1}{}{Conflito entre homem e mulher.}{}{}
\verb{ke-tôdô}{}{[ˈkɛ toˈdo]}{(n.)}{1}{}{Local onde permanecia a parturiente nos primeiros dias após o parto.}{}{}
\verb{ke-tôdô}{}{[ˈkɛ toˈdo]}{(n.)}{2}{}{Técnica de extração do vinho de palma.}{}{}
\verb{ke-vunvu}{}{[ˈkɛ vũˈvu]}{(n.)}{1}{}{Colmeia.}{}{}
\verb{kêxa}{}{[ˈkeʃa]}{(n.)}{1}{}{Queixa.}{}{}
\verb{kêxa}{}{[keˈʃa]}{(v.)}{1}{}{Queixar(-se).}{}{}%
\verb{kêxada}{}{[keˈʃada]}{(n.)}{1}{}{Bochecha.}{}{}
\verb{kêxada}{}{[keˈʃada]}{(n.)}{2}{}{Mandíbula.}{}{}
\verb{kêxada}{}{[keˈʃada]}{(n.)}{3}{}{Queixo.}{}{}%
\verb{kêxada-kobo}{}{[keˈʃada kᴐˈbᴐ]}{(n.)}{1}{}{Covinha.}{}{}
\verb{kêxidu}{}{[keˈʃidu]}{(adj.)}{1}{}{Esquecido.}{}{}
\verb{ke-xima}{}{[ˈkɛ ˈʃima]}{(n.)}{1}{}{Casa tradicional com primeiro piso.}{}{}
\verb{kê xitu}{}{[ke ˈʃitu]}{(int.)}{1}{}{Onde.}{}{}
\verb{kexton}{}{[kɛʃˈt\~ɔ]}{(n.)}{1}{}{Imbróglio.}{}{}
\verb{kexton}{}{[kɛʃˈt\~ɔ]}{(n.)}{2}{}{Problema.}{}{}
\verb{kexton}{}{[kɛʃˈt\~ɔ]}{(n.)}{3}{}{Questão.}{}{}
\verb{keza-mwala}{}{[kɛˈza ˈmwala]}{(n.)}{1}{Vassourinha-de-botão.}{\textbf{\textit{Scoparia dulcis}.}}{}
\verb{keza-ome}{}{[kɛˈza ˈɔmɛ]}{(n.)}{1}{}{Vassourinha-doce.}{\textbf{\textit{Borreria verticillata}.}}{}{}
\verb{kêzu}{}{[ˈkezu]}{(n.)}{1}{}{Queijo.}{}{}
\verb{kia}{}{[kiˈa]}{(part.)}{1}{Partícula prospectiva.}{}{\textbf{N kia kloga.} \textit{Eu ia escorregando.}}{}
\verb{kia}{}{[kiˈa]}{(v.)}{1}{}{Criar.}{}{}%
\verb{kia}{}{[kiˈa]}{(v.)}{2}{}{Educar.}{}{}
\verb{kiadô}{}{[kiaˈdo]}{(n.)}{1}{}{Criador.}{}{}
\verb{kiadô}{}{[kiaˈdo]}{(n.)}{1}{}{Protetor.}{}{}
\verb{kiadu}{}{[ˈkiadu]}{(adj.)}{1}{}{Criado.}{}{}
\verb{kiadu}{}{[ˈkiadu]}{(adj.)}{1}{}{Educado.}{}{}
\verb{kidadu}{}{[kiˈdadu]}{(n.)}{1}{}{Cuidado.}{Cf. \textbf{kwidadu}.}{}{}
\verb{kidalê}{}{[kidaˈle]}{(interj.)}{1}{}{Acudam!}{}{}
\verb{kidalê}{}{[kidaˈle]}{(interj.)}{1}{}{Ai meu deus!}{}{}
\verb{kidalê}{}{[kidaˈle]}{(interj.)}{1}{}{Aqui d'el rei!}{}{}
\verb{kidalê}{}{[kidaˈle]}{(interj.)}{1}{}{Boa!}{}{}
\verb{kidalê}{}{[kidaˈle]}{(interj.)}{1}{}{Credo!}{}{}
\verb{kidalê}{}{[kidaˈle]}{(interj.)}{1}{}{Caramba!}{}{}
\verb{kidalê}{}{[kidaˈle]}{(interj.)}{1}{}{Eia!}{}{}
\verb{kidalê}{}{[kidaˈle]}{(interj.)}{1}{}{Socorro!}{}{}
\verb{kifiki}{}{[kiˈfiki]}{(n.)}{1}{}{Esquife.}{}{}
\verb{kilala}{}{[kilaˈla]}{(n.)}{1}{}{Armazém.}{}{}
\verb{kilala}{}{[kilaˈla]}{(n.)}{2}{}{Cabana de palha.}{}{}
\verb{kilambu}{}{[kil\~{\textturna}ˈbu]}{(n.)}{1}{}{A maneira de as mulheres
dobrarem o vestido ou a saia de forma que o sexo não fique visível.}{}{}
\verb{kilambu}{}{[kil\~{\textturna}ˈbu]}{(n.)}{1}{}{Roupa íntima.}{}{}
\verb{kilambu}{}{[kil\~{\textturna}ˈbu]}{(n.)}{1}{}{Tanga de pano.}{}{}
\verb{kilêlê}{}{[kileˈle]}{(n.)}{1}{Dança.}{}{}{}
\verb{kili}{}{[kiˈli]}{(v.)}{1}{}{Encolher.}{}{}
\verb{kili}{}{[kiˈli]}{(v.)}{2}{}{Enrolar.}{}{}
\verb{kilidu}{}{[kiˈlidu]}{(adj.)}{1}{}{Encolhido.}{}{}
\verb{kilidu}{}{[kiˈlidu]}{(adj.)}{2}{}{Enrolado.}{}{}
\verb{kilometlu}{}{[kiˈlɔmɛtlu]}{(n.)}{1}{}{Quilômetro.}{}{}
\verb{kilu}{}{[ˈkilu]}{(n.)}{1}{}{Quilo.}{}{}
\verb{kimi-pletu}{}{[ˈkimi
ˈplɛtu]}{(n.)}{1}{}{\textit{Kimi-pletu}.}{\textbf{\textit{Newbouldia
laevis}}.}{}
\verb{kimoni}{}{[kiˈmɔni]}{(n.)}{1}{}{Blusa.}{}{}{}
\verb{kimoni}{}{[kiˈmɔni]}{(n.)}{1}{}{Quimono.}{Cf. \textbf{kimono}.}{}{}
\verb{kimono}{}{[kiˈmɔnɔ]}{(n.)}{1}{}{Blusa.}{}{}
\verb{kimono}{}{[kiˈmɔnɔ]}{(n.)}{2}{}{Quimono.}{}{}%
\verb{kina}{}{[ˈkina]}{(n.)}{1}{}{Quina.}{\textbf{\textit{Cinchona
calisaya}}.}{}
\verb{kina}{}{[kiˈna]}{(n.)}{1}{}{\textit{Kina}.}{Dança típica do grupo
étnico \textit{angolar}.}{}
\verb{kinda}{}{[kĩˈda]}{(n.)}{1}{}{Cestinho.}{}{}
\verb{kinga bô}{}{[ˈkĩga ˈbo]}{(expr.)}{1}{}{Estás marcado!}{}{}
\verb{kinga bô}{}{[ˈkĩga ˈbo]}{(expr.)}{1}{}{Tu que me aguardes!}{}{}
\verb{kinhentu}{}{[kĩˈɲẽtu]}{(num.)}{1}{}{Quinhentos.}{}{}
\verb{kinhon}{}{[kĩˈɲõ]}{(n.)}{1}{}{Pedaço.}{}{}
\verb{kinhon}{}{[kĩˈɲõ]}{(n.)}{1}{}{Pênis.}{}{}
\verb{kinhon}{}{[kĩˈɲõ]}{(n.)}{2}{}{Quinhão.}{}{}%
\verb{kinhon}{}{[kĩˈɲõ]}{(n.)}{3}{}{Testículos.}{}{}
\verb{kinji}{}{[ˈkĩʒi]}{(num.)}{1}{}{Quinze.}{}{}
\verb{kinta-fela}{}{[ˈkĩta ˈfɛla]}{(n.)}{1}{}{Quinta-feira.}{}{}
\verb{kinte}{}{[kĩˈtɛ]}{(n.)}{1}{}{Cerca.}{}{}%
\verb{kinte}{}{[kĩˈtɛ]}{(n.)}{2}{}{Quintal.}{}{}
\verb{kinte-môlê}{}{[kĩˈtɛ ˈmole]}{(n.)}{1}{}{Lugar do velório.}{}{}
\verb{kinte-nglandji}{}{[kĩˈtɛ ˈŋgl\~{\textturna}dʒi]}{(n.)}{1}{}{Quintal
onde as crianças são orientadas pelos mais velhos.}{}{}
\verb{kintu}{}{[ˈkĩtu]}{(num.)}{1}{}{Quinto.}{}{}
\verb{kintximon}{}{[ˈkĩtʃiˈmõ]}{(n.)}{1}{}{Cachimbo.}{}{}
\verb{kion}{}{[kiˈ\~ɔ]}{(n.)}{1}{}{Pênis.}{}{}
\verb{kiovo}{}{[kiɔˈvɔ]}{(v.)}{1}{}{Desanimar.}{}{}
\verb{kiovo}{}{[kiɔˈvɔ]}{(v.)}{2}{}{Entristecer.}{}{}
\verb{kiovo}{}{[kiɔˈvɔ]}{(v.)}{3}{}{Esmorecer.}{}{}
\verb{kiovo}{}{[kiɔˈvɔ]}{(v.)}{4}{}{Murchar.}{}{}
\verb{kiovo}{}{[kiɔˈvɔ]}{(v.)}{5}{}{Recuar.}{}{}
\verb{kiovodu}{}{[kiɔˈvɔdu]}{(adj.)}{1}{}{Abatido.}{}{}
\verb{kiovodu}{}{[kiɔˈvɔdu]}{(adj.)}{2}{}{Desanimado.}{}{}
\verb{kiovodu}{}{[kiɔˈvɔdu]}{(adj.)}{3}{}{Entristecido.}{}{}
\verb{kiovodu}{}{[kiɔˈvɔdu]}{(adj.)}{4}{}{Esmorecido.}{}{}
\verb{kiovodu}{}{[kiɔˈvɔdu]}{(adj.)}{5}{}{Introvertido.}{}{}
\verb{kisama}{}{[kisaˈma]}{(n.)}{1}{}{Toca.}{}{}
\verb{kisanda}{}{[kiˈs\~{\textturna}da]}{(n.)}{1}{}{Esteira feita com fios de
fibra de palmeira.}{}{}
\verb{kisenglê}{}{[kisẽˈgle]}{(n.)}{1}{}{Machadinho.}{Cf.
\textbf{klisengê}.}{}{}
\verb{kitaka}{}{[kitaˈka]}{(n.)}{1}{}{Armadilha em forma de gancho para
apanhar ratos e mussaranhos.}{}{}
\verb{kitali}{}{[kitaˈli]}{(n.)}{2}{}{Local isolado.}{}{}
\verb{kitali}{}{[kitaˈli]}{(n.)}{1}{}{Mocho.}{Cf. \textbf{kyê kitali}.}{}
\verb{kitali}{}{[kitaˈli]}{(n.)}{2}{}{Ribanceira.}{}{}
\verb{kitanda}{}{[kiˈt\~{\textturna}da]}{(n.)}{1}{}{Pequena loja.}{}{}
\verb{kitanda}{}{[kiˈt\~{\textturna}da]}{(n.)}{1}{}{Quitanda.}{}{}
\verb{kitembu}{}{[kitẽˈbu]}{(n.)}{1}{}{Mutirão.}{Trabalho de caráter
mutualista.}{}
\verb{kitoli}{}{[kitᴐˈli]}{(n.)}{1}{}{Mocho de São
Tomé.}{\textbf{\textit{Otus hartlaubi}}.}{}
\verb{kitxiba}{}{[kitʃiˈba]}{(n.)}{1}{}{Banana-prata.}{\textbf{\textit{Musa
balbisiana}}.}{}
\verb{kixiba}{}{[kiʃiˈba]}{(n.)}{1}{}{Banana-prata.}{Cf.
\textbf{kitxiba}.}{}{}
\verb{kixpa}{}{[kiʃˈpa]}{(n.)}{1}{}{Barraca.}{}{}
\verb{kixpa}{}{[kiʃˈpa]}{(n.)}{2}{}{Cozinha tradicional.}{}{}
\verb{kixtipadu}{}{[kiʃtiˈpadu]}{(adj.)}{1}{}{Resfriado.}{Cf.
\textbf{kuxtipadu}.}{}
%\verb{kiya}{}{[ˈkija]}{(v.)}{1}{}{Criar.}{}{}
\verb{kiya}{}{[ˈkija]}{(v.)}{1}{}{Pretender.}{}{}
\verb{kiya}{}{[ˈkija]}{(v.)}{1}{}{Querer.}{}{}
\verb{kiya}{}{[ˈkija]}{(v.)}{1}{}{Tencionar.}{}{}
%\verb{kiyadu}{}{[kiˈjadu]}{(v.)}{1}{}{Criado.}{}{}%
%\verb{kiyadu}{}{[kiˈjadu]}{(v.)}{2}{}{Educado.}{}{}
\verb{kizaka}{}{[kizaˈka]}{(n.)}{1}{}{\textit{Kizaka}.}{Prato à base de folha
de mandioca.}{}{}
\verb{kla}{}{[ˈkla]}{(id.)}{1}{}{Cf. \textbf{kota kla.}}{}{}
\verb{klafasa}{}{[klaˈfasa]}{(n.)}{1}{}{Cartilagem.}{}{}
\verb{klafatxi}{}{[klaˈfatʃi]}{(n.)}{1}{}{Carrapato.}{}{}%
\verb{klafatxi}{}{[klaˈfatʃi]}{(n.)}{2}{}{Diabinhos.}{Personagens de
histórias tradicionais.}{}
\verb{klaga}{}{[klaˈga]}{(v.)}{1}{}{Carregar.}{}{}
\verb{klagadô}{}{[klagaˈdo]}{(n.)}{1}{}{Carregador.}{}{}
\verb{klagadu}{}{[klaˈgadu]}{(adj.)}{1}{}{Carregado.}{}{}
\verb{klakanhon}{}{[klakaˈɲõ]}{(n.)}{1}{}{Calcanhar.}{}{}
\verb{klakata}{}{[klakaˈta]}{(id.)}{2}{}{Cf. \textbf{seku klakata.}}{}{}
\verb{klakla}{}{[klaˈkla]}{(n.)}{1}{}{Camarão muito pequeno.}{}{}
\verb{klala}{}{[ˈklala]}{(n.)}{1}{}{Clara.}{}{}
\verb{klala-d'ovu}{}{[ˈklala ˈdᴐvu]}{(n.)}{1}{}{Clara de ovo.}{}{}
\verb{klalu}{}{[ˈklalu]}{(adj.)}{1}{}{Claro.}{}{}
\verb{klalu fenene}{}{[ˈklalu fɛnɛˈnɛ]}{(expr.)}{1}{}{Claríssimo.}{}{}
\verb{klama}{}{[klaˈma]}{(v.)}{1}{}{Abrandar.}{}{}
\verb{klama}{}{[klaˈma]}{(v.)}{2}{}{Acalmar da chuva.}{}{}
\verb{klamuzu}{}{[klaˈmuzu]}{(n.)}{1}{}{Caramujo.}{}{}
\verb{klamuzu-d'omali}{}{[klaˈmuzu dᴐˈmali]}{(n.)}{1}{}{Caramujo do mar.}{}{}\verb{klan}{}{[ˈkl\~{\textturna}]}{(id.)}{1}{}{Cf. \textbf{bili klan}.}{}{}
\verb{klan}{}{[ˈkl\~{\textturna}]}{(id.)}{1}{}{Cf. \textbf{wê betu
klan}.}{}{}
\verb{klan}{}{[ˈkl\~{\textturna}]}{(id.)}{1}{}{Cf. \textbf{wê klan}.}{}{}
\verb{klana}{}{[klaˈna]}{(v.)}{2}{}{Afincar-se.}{}{}
\verb{klana}{}{[klaˈna]}{(v.)}{3}{}{Amainar.}{}{}
\verb{klana}{}{[klaˈna]}{(v.)}{4}{}{Inclinar(-se).}{}{}
\verb{klana}{}{[klaˈna]}{(v.)}{5}{}{Resistir.}{}{}
\verb{klani}{}{[klaˈni]}{(id.)}{1}{}{Cf. \textbf{kulu klani.}}{}{}
\verb{klanixi}{}{[klaˈniʃi]}{(n.)}{1}{}{Artrose reumática.}{}{}
\verb{klapinhe}{}{[klaˈpĩɲɛ]}{(n.)}{1}{}{Carapinha.}{}{}
\verb{klapintêlu}{}{[klapĩˈtelu]}{(n.)}{1}{}{Carpinteiro.}{}{}
\verb{klasa}{}{[klaˈsa]}{(v.)}{1}{}{Calçar.}{}{}
\verb{klasi}{}{[ˈklasi]}{(n.)}{1}{}{Classe escolar.}{}{}
\verb{klason}{}{[klaˈsõ]}{(n.)}{1}{}{Calção.}{}{}
\verb{klason}{}{[klaˈsõ]}{(n.)}{1}{}{Calças.}{}{}
\verb{klason-d'ope-kutu}{}{[klaˈsõ dᴐˈpɛ ˈkutu]}{(n.)}{1}{}{Calção.}{}{}
\verb{klason-d'ope-longô}{}{[klaˈsõ dᴐˈpɛ ˈlõgo]}{(n.)}{1}{}{Calças.}{}{}
\verb{klason-zwatxi}{}{[klaˈsõ ˈzwatʃi]}{(n.)}{1}{}{Calças de ganga.}{}{}
\verb{klason-zwatxi}{}{[klaˈsõ ˈzwatʃi]}{(n.)}{2}{}{Jeans.}{}{}
\verb{klavon}{}{[klaˈvõ]}{(n.)}{1}{}{Carvão.}{}{}
\verb{klaya}{}{[klaˈja]}{(v.)}{1}{}{Clarear.}{}{}
\verb{klebenta}{}{[klɛb\~ɛˈta]}{(v.)}{1}{}{Esconjurar.}{}{}
\verb{klebenta}{}{[klɛb\~ɛˈta]}{(v.)}{1}{}{Quebrantar.}{}{}
\verb{klêbentu}{}{[kleˈbẽtu]}{(n.)}{2}{}{Exantema.}{}{}
\verb{klêbentu}{}{[kleˈbẽtu]}{(n.)}{3}{}{Quisto.}{}{}
\verb{klêklê}{}{[kleˈkle]}{(n.)}{1}{}{Broto da palmeira.}{}{}%
\verb{klêklê}{}{[kleˈkle]}{(n.)}{2}{}{Costela da palmeira.}{}{}%
\verb{klêklê}{}{[kleˈkle]}{(n.)}{3}{}{Ramo seco da palmeira.}{}{}%
\verb{klêklê}{}{[kleˈkle]}{(n.)}{4}{}{Ramos da palmeira.}{}{}
\verb{klê ni dêsu padê klusu}{}{[ˈkle ˈni ˈdesu ˈpade ˈklusu]}{(interj.)}{1}{}{Abrenúncio!}{}
\verb{klê ni dêsu padê klusu}{}{[ˈkle ˈni ˈdesu ˈpade ˈklusu]}{(interj.)}{1}{}{Credo!}{}
\verb{klê ni dêsu padê klusu}{}{[ˈkle ˈni ˈdesu ˈpade ˈklusu]}{(interj.)}{1}{}{Crê em deus-pai e na cruz.}{}
\verb{klêsê}{}{[kleˈse]}{(v.)}{1}{}{Crescer.}{}{}
\verb{klêsenta}{}{[klesẽˈta]}{(v.)}{1}{}{Acrescentar.}{}{}
\verb{klêsentadu}{}{[klesẽˈtadu]}{(adj.)}{1}{}{Acrescentado.}{}{}
\verb{klesentu}{}{[klɛˈs\~ɛtu]}{(n.)}{1}{}{Acréscimo.}{}{}
\verb{klesentxi}{}{[klɛˈs\~ɛtʃi]}{(adj.)}{1}{}{Crescente.}{}{}
\verb{klesentxi}{}{[klɛˈs\~ɛtʃi]}{(n.)}{2}{}{Quarto crescente.}{}{}
\verb{klete}{}{[klɛˈtɛ]}{(id.)}{1}{}{Banana-prata.}{}{}
\verb{kliadu}{}{[kliˈadu]}{(n.)}{1}{}{Criado.}{}{}
\verb{kliadu}{}{[kliˈadu]}{(n.)}{2}{}{Empregado.}{}{}
\verb{kliason}{}{[kliaˈs\~ɔ]}{(n.)}{1}{}{Criação.}{Cf. \textbf{kyason}.}{}
\verb{kliba}{}{[ˈkliba]}{(n.)}{1}{}{Brincadeira movimentada e ruidosa.}{}{}
\verb{kliji}{}{[ˈkliʒi]}{(n.)}{1}{}{Crise.}{}{}
\verb{klikoto-d'ôbô}{}{[klikɔˈtɔ
doˈbo]}{(n.)}{1}{}{Borracha-do-mato.}{\textit{\textbf{Oxyanthus
speciosus}}.}{}
\verb{klimi}{}{[ˈklimi]}{(n.)}{1}{}{Crime.}{}{}
\verb{klina}{}{[kliˈna]}{(v.)}{1}{}{Fincar.}{}{}
\verb{klinkata}{}{[klĩkaˈta]}{(n.)}{1}{}{Criança raquítica.}{}{}
\verb{klinkêtê}{}{[klĩkeˈte]}{(n.)}{1}{}{Larva.}{}{}
\verb{klinkinha}{}{[klĩkĩˈɲa]}{(v.)}{1}{}{Amarrotar.}{}{}
\verb{klinkinhadu}{}{[klĩkĩˈɲadu]}{(adj.)}{1}{}{Amarrotado.}{}{}
\verb{klisakli}{}{[klisaˈkli]}{(n.)}{1}{}{\textit{Klisakli}.}{Utensílio de
pesca utilizado para a captura de camarões e outras espécies.}{}
\verb{klisengê}{}{[klisẽˈge]}{(n.)}{1}{}{Machadinho.}{}{}
\verb{klixton}{}{[kliʃˈtõ]}{(n.)}{1}{}{Cristão.}{}{}
\verb{klixton}{}{[kliʃˈtõ]}{(n.)}{1}{}{Ser humano.}{}{}
\verb{Klixtu}{}{[ˈkliʃtu]}{(n.)}{1}{}{Cristo.}{}{}
\verb{klôbôklôbô}{}{[kloˈbokloˈbo]}{(adj.)}{1}{}{Desarrumado.}{}{}
\verb{klôbôklôbô}{}{[kloˈbokloˈbo]}{(n.)}{1}{}{Desmazelado.}{}{}
\verb{kloga}{}{[klɔˈga]}{(v.)}{1}{}{Escorregar.}{}{}
\verb{klogadu}{}{[klɔˈgadu]}{(adj.)}{1}{}{Escorregado.}{}{}
\verb{kloklo}{}{[klɔˈklɔ]}{(v.)}{1}{}{Rapar.}{}{}%
\verb{kloklo}{}{[klɔˈklɔ]}{(v.)}{2}{}{Raspar.}{}{}
\verb{kloklodu}{}{[klɔˈklɔdu]}{(adj.)}{1}{}{Rapado.}{}{}%
\verb{kloklodu}{}{[klɔˈklɔdu]}{(adj.)}{2}{}{Raspado.}{}{}
\verb{kloko}{}{[klɔˈkɔ]}{(v.)}{1}{}{Rapar.}{}{}{}
\verb{kloko}{}{[klɔˈkɔ]}{(v.)}{1}{}{Raspar.}{Cf. \textbf{kloklo}.}{}{}
\verb{klokoto}{}{[klɔkɔˈtɔ]}{(n.)}{1}{}{Barata grande.}{}{}
\verb{klômisa}{}{[kloˈmisa]}{(n.)}{1}{}{Minhoca.}{}{}
\verb{klongôji}{}{[klõgoˈʒi]}{(n.)}{1}{}{Cupim.}{}{}
\verb{klongondo}{}{[kl\~ɔg\~ɔˈdɔ]}{(id.)}{1}{}{Cf. \textbf{bila
klongondo.}}{}{}
\verb{klonha}{}{[ˈkl\~ɔɲa]}{(n.)}{1}{}{Coronha.}{}{}%
\verb{klonha}{}{[ˈkl\~ɔɲa]}{(n.)}{1}{}{Coronhada.}{}{}
\verb{klonha}{}{[ˈkl\~ɔɲa]}{(n.)}{1}{}{Pessoa idosa.}{}{}%
\verb{klonha}{}{[kl\~ɔˈɲa]}{(v.)}{1}{}{Dar uma coronhada.}{}{}%
\verb{klonklo}{}{[kl\~ɔˈklɔ]}{(n.)}{1}{}{Pescoço.}{}{}%
\verb{klonklo}{}{[kl\~ɔˈklɔ]}{(n.)}{2}{}{Angina.}{}{}
\verb{klonkôji}{}{[klõkoˈʒi]}{(n.)}{1}{}{Cupim.}{}{}{}
\verb{klonvesa}{}{[kl\~ɔˈvɛsa]}{(n.)}{1}{}{Conversa.}{}{}
\verb{klonvesa}{}{[kl\~ɔvɛˈsa]}{(v.)}{1}{}{Conversar.}{}{}
\verb{klonvesadu}{}{[kl\~ɔvɛˈsadu]}{(adj.)}{1}{}{Conversado.}{}{}
\verb{klonvesadu}{}{[kl\~ɔvɛˈsadu]}{(adj.)}{2}{}{Falado.}{}{}
\verb{klonveson}{}{[kl\~ɔvɛˈs\~ɔ]}{(n.)}{1}{}{Conversa.}{}{}
\verb{klonzu}{}{[ˈklõzu]}{(n.)}{1}{}{Fezes.}{}{}
\verb{klôpaxê}{}{[klopaˈʃe]}{(n.)}{1}{}{Espingarda.}{}{}
\verb{klôpô}{}{[ˈklopo]}{(n.)}{1}{}{Corpo humano.}{}{}
\verb{klôsô}{}{[ˈkloso]}{(n.)}{1}{}{Amendoeira-da-Índia.}{\textbf{\textit{Terminalia
catappa}}.}{}
\verb{klôsô}{}{[ˈkloso]}{(n.)}{2}{}{Caroço.}{}{}
\verb{kloson}{}{[klɔˈs\~ɔ]}{(n.)}{1}{}{Coração.}{}{}%
\verb{kloson}{}{[klɔˈs\~ɔ]}{(n.)}{1}{}{Coragem.}{}{}%
\verb{kloson}{}{[klɔˈs\~ɔ]}{(n.)}{2}{}{Sensibilidade.}{}{}
\verb{kloson-felu}{}{[klɔˈs\~ɔ ˈfɛlu]}{(adj.)}{1}{}{Corajoso.}{}{}
\verb{kloson-felu}{}{[klɔˈs\~ɔ ˈfɛlu]}{(adj.)}{1}{}{Destemido.}{}{}
\verb{kloson-felu}{}{[klɔˈs\~ɔ ˈfɛlu]}{(adj.)}{1}{}{Impiedoso.}{}{}
\verb{kloson-felu}{}{[klɔˈs\~ɔ ˈfɛlu]}{(adj.)}{1}{}{Implacável.}{}{}
\verb{kloson-felu}{}{[klɔˈs\~ɔ ˈfɛlu]}{(adj.)}{1}{}{Intrépido.}{}{}
\verb{kloson-felu}{}{[klɔˈs\~ɔ ˈfɛlu]}{(adj.)}{1}{}{Mau.}{}{}
\verb{kloson-felu}{}{[klɔˈs\~ɔ ˈfɛlu]}{(adj.)}{1}{}{Sem compaixão.}{}{}
\verb{kloson-felu}{}{[klɔˈs\~ɔ ˈfɛlu]}{(adj.)}{1}{}{Valente.}{}{}
\verb{kloson-kebla}{}{[klɔˈs\~ɔ kɛˈbla]}{(n.)}{1}{}{Palpitação.}{}{}
\verb{kloson-kema}{}{[klɔˈs\~ɔ ˈkɛma]}{(n.)}{1}{}{Azia.}{}{}
\verb{kloson-lizu}{}{[klɔˈs\~ɔ
ˈlizu]}{(n.)}{1}{}{Coração-rijo.}{\textit{\textbf{Warneckea
membranifolia}}.}{}
\verb{kloson-lizu}{}{[klɔˈs\~ɔ ˈlizu]}{(n.)}{1}{}{Pessoa empedernida.}{}{}
\verb{kloson-lizu}{}{[klɔˈs\~ɔ ˈlizu]}{(n.)}{1}{}{Pessoa rígida.}{}{}
\verb{kloson-ngandu}{}{[klɔˈs\~ɔ ŋg\~{\textturna}ˈdu]}{(adj.)}{1}{}{Mau.}{}{}\verb{kloson-son}{}{[klɔˈs\~ɔ
ˈs\~ɔ]}{(n.)}{1}{}{Coração-do-chão.}{\textit{\textbf{Pleurotus
tuberregium}}.}{}
\verb{kloson-tatalugwa}{}{[klɔˈs\~ɔ ˈtataˈlugwa]}{(n.)}{1}{}{Destemido.}{}{}
\verb{kloson-tatalugwa}{}{[klɔˈs\~ɔ
ˈtataˈlugwa]}{(n.)}{1}{}{Inquebrantável.}{}{}
\verb{kloson-tefitefi}{}{[klɔˈs\~ɔ ˈtɛfiˈtɛfi]}{(adj.)}{1}{}{Ganancioso.}{}{}\verb{kloson-uxi}{}{[klɔˈs\~ɔ ˈuʃi]}{(n.)}{1}{}{Nojo.}{}{}
\verb{klôvina}{}{[kloˈvina]}{(n.)}{1}{}{Corvina.}{}{}
\verb{klôwa}{}{[ˈklowa]}{(n.)}{1}{}{Coroa.}{}{}%
\verb{klôwa}{}{[ˈklowa]}{(n.)}{2}{}{Coroa de palmeira.}{}{}
\verb{klôwa}{}{[ˈklowa]}{(n.)}{3}{}{Grinalda.}{}{}
\verb{klôwa-kaza}{}{[ˈklowa kaˈza]}{(n.)}{1}{}{Buquê de noiva.}{}{}
\verb{klôzê}{}{[kloˈze]}{(v.)}{1}{}{Coser.}{}{}
\verb{klôzê}{}{[kloˈze]}{(v.)}{2}{}{Costurar.}{}{}
\verb{kluklu}{}{[kluˈklu]}{(id.)}{1}{}{Cf. \textbf{mon kluklu.}}{}{}
\verb{kluklu}{}{[kluˈklu]}{(n.)}{1}{}{Toco.}{}{}
\verb{kluklute}{}{[klukluˈtɛ]}{(n.)}{1}{}{Lepra.}{}{}
\verb{klukutu}{}{[klukuˈtu]}{(id.)}{1}{}{Cf. \textbf{olha klukutu.}}{}{}
\verb{klumuklumu}{}{[kluˈmukluˈmu]}{(n.)}{1}{}{Cartilagem.}{}{}
\verb{klumunka}{}{[klumũˈka]}{(v.)}{1}{}{Forçar.}{}{}%
\verb{klumunka}{}{[klumũˈka]}{(v.)}{2}{}{Esfregar com esforço.}{}{}
\verb{klumunka}{}{[klumũˈka]}{(v.)}{3}{}{Fazer força.}{}{}
\verb{klumunka}{}{[klumũˈka]}{(v.)}{4}{}{Lutar.}{}{}%
\verb{klupa}{}{[ˈklupa]}{(n.)}{1}{}{Culpa.}{}{}
\verb{klupa}{}{[kluˈpa]}{(v.)}{1}{}{Culpar.}{}{}
\verb{klupadu}{}{[kluˈpadu]}{(adj.)}{1}{}{Culpado.}{}{}
\verb{klupu}{}{[ˈklupu]}{(n.)}{1}{}{Caneco de lata.}{}{}
\verb{klupu}{}{[ˈklupu]}{(n.)}{1}{}{Medida para óleo de palma (cerca de um
decilitro).}{}{}
\verb{klusu}{}{[ˈklusu]}{(n.)}{1}{}{Crucifixo.}{}{}
\verb{klusu}{}{[ˈklusu]}{(n.)}{2}{}{Cruz.}{}{}
\verb{klusu}{}{[ˈklusu]}{(n.)}{3}{}{Encargo pesado.}{}{}
\verb{klusu}{}{[ˈklusu]}{(n.)}{4}{}{Provação.}{}{}
\verb{kluva}{}{[ˈkluva]}{(n.)}{1}{}{Curva.}{}{}
\verb{kluza}{}{[kluˈza]}{(v.)}{1}{}{Cruzar.}{}{}
\verb{kluzadu}{}{[kluˈzadu]}{(adj.)}{1}{}{Cruzado.}{}{} 
\verb{kluzêlu}{}{[kluˈzelu]}{(n.)}{1}{}{Cruzamento.}{}{}
\verb{kluzêlu}{}{[kluˈzelu]}{(n.)}{1}{}{Cruzeiro.}{}{}
\verb{kô}{}{[ˈko]}{(v.)}{1}{}{Assemelhar(-se).}{}{}
\verb{kô}{}{[ˈko]}{(v.)}{2}{}{Parecer-se com.}{}{}
\verb{kobla}{}{[kɔˈbla]}{(v.)}{1}{}{Cobrar.}{}{}
\verb{kobli}{}{[ˈkɔbli]}{(n.)}{1}{}{Cobre.}{}{}
\verb{koblo}{}{[ˈkɔblɔ]}{(n.)}{1}{}{Cobra.}{}{}
\verb{koblo-bôbô}{}{[ˈkɔblɔ
boˈbo]}{(n.)}{1}{}{Cobra-amarela.}{\textbf{\textit{Schistometopum
thomensis}}.}{}{}
\verb{koblo-d'omali}{}{[ˈkɔblɔ dɔˈmali]}{(n.)}{1}{}{Moreia.}{}{}{}
\verb{koblo-pletu}{}{[ˈkɔblɔ
ˈplɛtu]}{(n.)}{1}{}{Cobra-preta.}{\textbf{\textit{Naja melanoleuca}}.}{}{}
\verb{kobo}{}{[ˈkɔbɔ]}{(n.)}{1}{}{Buraco.}{}{}
\verb{kobo}{}{[ˈkɔbɔ]}{(n.)}{2}{}{Cova.}{}{}%
\verb{kobo}{}{[kɔˈbɔ]}{(v.)}{1}{}{Cavar.}{}{}%
\verb{kobo}{}{[kɔˈbɔ]}{(v.)}{2}{}{Copular.}{}{}
\verb{kobo}{}{[kɔˈbɔ]}{(v.)}{3}{}{Esburacar.}{}{}%
\verb{kobo}{}{[kɔˈbɔ]}{(v.)}{3}{}{Escavar.}{}{}%
\verb{kobo-d'awa}{}{[ˈkɔbɔ ˈdawa]}{(n.)}{1}{}{Estrangeiro.}{}{}
\verb{kobo-d'awa}{}{[ˈkɔbɔ ˈdawa]}{(n.)}{1}{}{Terras além-mar.}{}{}
\verb{kobo-dentxi}{}{[ˈkɔbɔ ˈd\~etʃi]}{(n.)}{1}{}{Cárie.}{}{}
\verb{kobo-ke}{}{[ˈkɔbɔ ˈkɛ]}{(n.)}{1}{}{Interior da casa.}{}{}
\verb{kobo-mon}{}{[ˈkɔbɔ ˈm\~ɔ]}{(n.)}{1}{}{Mão em forma de concha para
receber algo.}{}{}
\verb{kobo-mon}{}{[ˈkɔbɔ ˈm\~ɔ]}{(n.)}{1}{}{Palma da mão.}{}{}
\verb{koda}{}{[kɔˈda]}{(v.)}{1}{}{Acordar.}{}{}
\verb{kodadu}{}{[kɔˈdadu]}{(adj.)}{1}{}{Acordado.}{}{}
\verb{kode}{}{[kɔˈdɛ]}{(n.)}{1}{}{Benjamim.}{}{}
\verb{kode}{}{[kɔˈdɛ]}{(n.)}{2}{}{Caçula.}{}{}
\verb{kodelu}{}{[kɔˈdɛlu]}{(adj.)}{1}{}{Manso.}{}{}
\verb{kodelu}{}{[kɔˈdɛlu]}{(n.)}{1}{}{Cordeiro.}{}{}
\verb{kodo}{}{[ˈkɔdɔ]}{(n.)}{1}{}{Corda.}{}{}
\verb{kodo-binku}{}{[ˈkɔdɔ ˈb\~iku]}{(n.)}{1}{}{Cordão umbilical.}{}{}
\verb{kodo-d'awa}{}{[ˈkɔdɔ
ˈdawa]}{(n.)}{1}{}{\textit{Kodo-d'awa}.}{\textbf{\textit{Psydrax
acutiflora}}.}{}
\verb{kodo-d'onso}{}{[ˈkɔdɔ d\~ɔˈsɔ]}{(n.)}{1}{}{Corda utilizada pelos
vinhateiros para subir as palmeiras.}{}
\verb{kodo-d'ope}{}{[ˈkɔdɔ dɔˈpɛ]}{(n.)}{1}{}{Tendão de Aquiles.}{}{}
\verb{kodo-ke}{}{[ˈkɔdɔ
ˈkɛ]}{(n.)}{2}{}{\textit{Kodo-ke}.}{\textbf{\textit{Paullinia pinnata}}.}{}
\verb{kodo-ke-d'ôbô}{}{[ˈkɔdɔ ˈkɛ
doˈbo]}{(n.)}{1}{}{Corda-de-casa-do-mato.}{\textbf{\textit{Jasminum
bakeri}}.}{}
\verb{kodo-kloson}{}{[ˈkɔdɔ klɔˈs\~ɔ]}{(n.)}{1}{}{Amor.}{}{}%
\verb{kodo-kloson}{}{[ˈkɔdɔ klɔˈs\~ɔ]}{(n.)}{2}{}{Bem.}{}{}%
\verb{kodo-kloson}{}{[ˈkɔdɔ klɔˈs\~ɔ]}{(n.)}{3}{}{Querido.}{}{}
\verb{kodon}{}{[kɔˈd\~ɔ]}{(n.)}{1}{}{Cordão.}{}{}
\verb{kôdôni}{}{[kodoˈni]}{(n.)}{1}{}{Codorniz.}{}{}
\verb{kofesa}{}{[kɔfɛˈsa]}{(v.)}{1}{}{Confessar.}{}{}
\verb{kofli}{}{[ˈkɔfli]}{(n.)}{1}{}{Cofre.}{}{}
\verb{kôfô}{}{[ˈkofo]}{(n.)}{1}{}{Colo.}{}{}
\verb{kôfô}{}{[ˈkofo]}{(n.)}{2}{}{Esconderijo.}{}{}%
\verb{kôfô}{}{[ˈkofo]}{(n.)}{3}{}{Regaço.}{}{}
\verb{kôkô}{}{[koˈko]}{(n.)}{1}{}{Cará.}{}{}
\verb{koko}{}{[ˈkɔkɔ]}{(n.)}{1}{}{Vagina.}{}{}
\verb{koko}{}{[kɔˈkɔ]}{(v.)}{1}{}{Defecar.}{}{}
\verb{kôkô}{}{[koˈko]}{(v.)}{1}{}{Engatinhar.}{}{}
\verb{kôkô}{}{[koˈko]}{(v.)}{2}{}{Gatinhar.}{}{}
\verb{kôkô-blanku}{}{[koˈko
ˈbl\~{\textturna}ku]}{(n.)}{1}{}{Cará-branco.}{\textbf{\textit{Xanthosoma
sagittifolium}.}}{}
\verb{kôkôi-mon}{}{[kokoˈi ˈmõ]}{(n.)}{1}{}{Cotovelo.}{}{}
\verb{kôkôkô}{}{[kokoˈko]}{(id.)}{2}{}{Cf. \textbf{fia kôkôkô.}}{}{}
\verb{kôkôkô}{}{[kokoˈko]}{(id.)}{1}{}{Cf. \textbf{fisadu kôkôkô.}}{}{}
\verb{kôkôkô}{}{[kokoˈko]}{(id.)}{3}{}{Cf. \textbf{fyô kôkôkô.}}{}{}
\verb{kokolo}{}{[kɔkɔˈlɔ]}{(n.)}{1}{}{Galo velho e grande.}{}{}
\verb{kokoloti}{}{[kɔkɔˈlɔti]}{(n.)}{1}{}{Porção de comida que fica agarrada
ao fundo da panela.}{}{}
\verb{kôkondja}{}{[koˈkõdʒa]}{(n.)}{1}{}{Coco.}{}{}%
\verb{kôkondja}{}{[koˈkõdʒa]}{(n.)}{2}{}{Coqueiro.}{\textit{\textbf{Cocos nucifera}}.}{}
\verb{kôkonja}{}{[koˈkõʒa]}{(n.)}{1}{}{Coco.}{Cf. \textbf{kôkondja}.}{}{}
\verb{kokonzuku}{}{[kɔk\~ɔˈzuku]}{(n.)}{1}{}{Rabo-de-junco.}{\textit{\textbf{Phaeton lepturus}}.}{}{}
\verb{kokovadu}{}{[kɔkɔˈvadu]}{(n.)}{1}{}{Corcovado.}{\textbf{\textit{Caranx hippos}}.}{}
\verb{kokovadu}{}{[kɔkɔˈvadu]}{(n.)}{1}{}{Xaréu.}{\textbf{\textit{Caranx hippos}}.}{}
\verb{kôkô-venenu}{}{[koˈko v\~ɛˈn\~ɛnu]}{(n.)}{1}{}{Cará-veneno.}{}{}{}
\verb{kôkô-vlêmê}{}{[koˈko vleˈme]}{(n.)}{1}{}{Cará-vermelho.}{\textbf{\textit{Xanthosoma sagittifolium}.}}{}
\verb{kola}{}{[ˈkɔla]}{(n.)}{1}{}{Coleira.}{\textbf{\textit{Cola acuminata.}}}{}
\verb{kola}{}{[ˈkɔla]}{(n.)}{2}{}{Noz-de-cola.}{\textbf{\textit{Cola acuminata.}}}{}
\verb{kola}{}{[kɔˈla]}{(v.)}{1}{}{Colar.}{}{}
\verb{kolaji}{}{[kɔˈlaʒi]}{(v.)}{1}{}{Coragem.}{}{}
\verb{kola-kongô}{}{[ˈkɔla kõˈgo]}{(n.)}{1}{}{Cola-do-congo.}{\textbf{\textit{Buchholzia coriacea.}}}{}
\verb{kola-makaku}{}{[ˈkɔla maˈkaku]}{(n.)}{1}{}{Cola-macaco.}{\textbf{\textit{Trichilia grandifolia}.}}{}
\verb{kôlê}{}{[koˈle]}{(n.)}{1}{}{Corrida.}{}{}
\verb{kôlê}{}{[koˈle]}{(v.)}{1}{}{Correr.}{}{}
\verb{kôlêdô}{}{[koleˈdo]}{(n.)}{1}{}{Corredor.}{}{}%
\verb{kolega}{}{[kɔˈlɛga]}{(n.)}{1}{}{Colega.}{}{}
\verb{kôlê kabêsa lembla}{}{[koˈle kaˈbesa l\~eˈbla]}{(expr.)}{1}{}{Refletir.}{}{}
\verb{kôlê ku}{}{[koˈle ˈku]}{(expr.)}{1}{}{Expulsar.}{}{}
\verb{kolema}{}{[ˈkɔlɛma]}{(n.)}{3}{}{Quaresma.}{}{}
\verb{kolema-dôdô}{}{[ˈkɔlɛma
doˈdo]}{(n.)}{1}{}{\textit{Kolema-dôdô}.}{\textbf{\textit{Millettia
barteri}}.}{}
\verb{kôlêmon}{}{[koleˈmõ]}{(n.)}{1}{}{Corrimão.}{}{}
\verb{kôlê ni tlaxi}{}{[koˈle ˈni ˈtlaʃi ]}{(expr.)}{1}{}{Perseguir.}{}{}
\verb{kolenta}{}{[kɔˈl\~ɛta]}{(num.)}{1}{}{Quarenta.}{}{}
\verb{kolentxi}{}{[kɔˈl\~ɛtʃi]}{(n.)}{1}{}{Corrente.}{}{}
\verb{kôlêpyan-ba-labu}{}{[koleˈpj\~{\textturna} ˈba
ˈlabu]}{(n.)}{1}{}{\textit{Kôlêpyan-ba-labu}.}{\textit{\textbf{Elops
senegalensis}}.}{}
\verb{kôlê sêlêlê}{}{[koˈle seleˈle]}{(expr.)}{1}{}{Fluir (curso d'água).}{}{}
%\verb{kôlêtê}{}{[koleˈte]}{(n.)}{1}{}{Colete.}{}{}%
%\verb{kôlêtê}{}{[koleˈte]}{(n.)}{2}{}{Sutiã.}{}{}
\verb{kôlikô}{}{[koliˈko]}{(n.)}{1}{}{Piparote.}{}{}
\verb{kolima}{}{[ˈkɔlima]}{(n.)}{1}{}{Cólima.}{\textbf{\textit{Lonchocarpus
sericeus}}.}{}
\verb{kolima}{}{[ˈkɔlima]}{(n.)}{2}{}{Cólima-fria.}{\textbf{\textit{Millettia
thonningii}.}}{}
\verb{kôlô}{}{[ˈkolo]}{(n.)}{1}{}{Amuleto em forma de almofada com enchimento
de ervas, usado para proteção do recém-nascido.}{}{}
\verb{kôlô}{}{[ˈkolo]}{(n.)}{2}{}{Cor.}{}{}%
\verb{kôlô}{}{[ˈkolo]}{(n.)}{3}{}{Espécie.}{}{}%
\verb{kôlô}{}{[ˈkolo]}{(n.)}{4}{}{Qualidade.}{}{}%
\verb{kôlô}{}{[ˈkolo]}{(n.)}{5}{}{Tipo.}{}{}
\verb{kôlô-bôbô}{}{[ˈkolo boˈbo]}{(n.)}{1}{}{\textit{Kota-wê}.}{Cf.
\textbf{kota-wê}.}{}{}
\verb{kôlomba}{}{[koˈlõba]}{(n.)}{2}{}{Estrangeiro.}{}{}
\verb{kôlomba}{}{[koˈlõba]}{(n.)}{1}{}{Europeu.}{}{}
\verb{kôlomba}{}{[koˈlõba]}{(n.)}{3}{}{Pessoa branca.}{}{}
\verb{kolombeta}{}{[kɔl\~ɔˈbɛta]}{(n.)}{1}{}{Colombeta.}{\textit{\textbf{Coryphaena
equiselis}}.}{}
\verb{kôlombêya}{}{[kolõˈbeja]}{(n.)}{1}{}{Algazarra.}{}{}
\verb{kôlombêya}{}{[kolõˈbeja]}{(n.)}{1}{}{Grupo de amigos.}{}{}
\verb{kôlônu}{}{[koˈlonu]}{(n.)}{1}{}{Colono.}{}{}
\verb{kôlô-xindja}{}{[ˈkolo ˈʃĩdʒa]}{(adj.)}{1}{}{Cinzento.}{}{}
\verb{komandantxi}{}{[kɔm\~{\textturna}ˈd\~{\textturna}tʃi]}{(n.)}{1}{}{Comandante.}{}{}
\verb{komandantxi-vapô}{}{[kɔm\~{\textturna}ˈd\~{\textturna}tʃi
vaˈpo]}{(n.)}{1}{}{Almirante.}{}{}
\verb{komesa}{}{[kɔmɛˈsa]}{(v.)}{1}{}{Começar.}{}{}%
\verb{komesa}{}{[kɔmɛˈsa]}{(v.)}{2}{}{Iniciar.}{}{}
\verb{komesu}{}{[kɔˈmɛsu]}{(n.)}{1}{}{Começo.}{}{}
\verb{komesu}{}{[kɔˈmɛsu]}{(n.)}{2}{}{Início.}{}{}
\verb{komesu}{}{[kɔˈmɛsu]}{(n.)}{3}{}{Princípio.}{}{}
\verb{komoda}{}{[kɔmɔˈda]}{(v.)}{1}{}{Acomodar.}{}{}
\verb{komoda}{}{[kɔmɔˈda]}{(v.)}{2}{}{Incomodar.}{}{}
\verb{kompa}{}{[kõˈpa]}{(n.)}{1}{}{Compadre.}{}{}
\verb{kompanhe}{}{[kõp\~{\textturna}ˈɲɛ]}{(n.)}{1}{}{Companheiro.}{}{}
\verb{kondê}{}{[kõˈde]}{(n.)}{1}{}{Esconderijo.}{}{}
\verb{kondê}{}{[kõˈde]}{(v.)}{1}{}{Esconder(-se).}{}{}
\verb{kondena}{}{[k\~ɔdɛˈna]}{(v.)}{1}{}{Condenar.}{}{}
\verb{kondidu}{}{[kõˈdidu]}{(adj.)}{1}{}{Escondido.}{}{}
\verb{kondji}{}{[ˈkõdʒi]}{(n.)}{1}{}{Conde.}{}{}
\verb{koneta}{}{[kɔˈnɛta]}{(n.)}{1}{}{Chifre.}{}{}
\verb{koneta}{}{[kɔˈnɛta]}{(n.)}{2}{}{Clarim.}{}{}
\verb{koneta}{}{[kɔˈnɛta]}{(n.)}{3}{}{Corneta.}{}{}
\verb{koneta}{}{[kɔˈnɛta]}{(n.)}{4}{}{Corno.}{}{}
\verb{konetêlu}{}{[kɔnɛˈtelu]}{(n.)}{1}{}{Corneteiro.}{}{}
\verb{kongô}{}{[ˈkõgo]}{(id.)}{1}{}{Cf. \textbf{pletu kongô.}}{}{}
\verb{konjuntu}{}{[kõˈʒũtu]}{(n.)}{1}{}{Banda.}{}{}%
\verb{konjuntu}{}{[kõˈʒũtu]}{(n.)}{2}{}{Conjunto musical.}{}{}
\verb{konki}{}{[kõˈki]}{(adj.)}{1}{}{Corcovado.}{}{}
\verb{konki}{}{[kõˈki]}{(adj.)}{2}{}{Corcunda.}{}{}
\verb{konko}{}{[k\~ɔˈkɔ]}{(n.)}{1}{}{\textit{Konko}.}{\textbf{\textit{Dactylopterus
volitans}}.}{}
\verb{konkonkon}{}{[kõkõˈkõ]}{(onom.)}{1}{}{Truz-truz.}{}{}
\verb{konkusu}{}{[k\~ɔˈkusu]}{(n.)}{1}{}{Competição.}{}{}%
\verb{konkusu}{}{[k\~ɔˈkusu]}{(n.)}{2}{}{Concurso.}{}{}
\verb{kono}{}{[ˈkɔnɔ]}{(n.)}{1}{}{Vagina.}{}{}
\verb{kono}{}{[kɔˈnɔ]}{(v.)}{1}{}{Arrancar.}{}{}%
\verb{kono}{}{[kɔˈnɔ]}{(v.)}{2}{}{Colher.}{}{}%
\verb{kono}{}{[kɔˈnɔ]}{(v.)}{3}{}{Juntar.}{}{}
\verb{konobya}{}{[kɔˈnɔbja]}{(n.)}{1}{}{Pica-peixe-pigmeu.}{\textbf{\textit{Ispidina
picta}}.}{}
\verb{konomia}{}{[kɔnɔˈmia]}{(n.)}{1}{}{Economia.}{}{}
\verb{konsê}{}{[kõˈse]}{(n.)}{1}{}{Conselho.}{}{}
\verb{konsê}{}{[kõˈse]}{(v.)}{1}{}{Conhecer.}{}{}%
\verb{konsê}{}{[kõˈse]}{(v.)}{2}{}{Perceber.}{}{}%
\verb{konsê}{}{[kõˈse]}{(v.)}{3}{}{Reconhecer.}{}{}
\verb{konseta}{}{[k\~ɔsɛˈta]}{(v.)}{1}{}{Consertar.}{}{}%
\verb{konseta}{}{[k\~ɔsɛˈta]}{(v.)}{2}{}{Reparar.}{}{}
\verb{konsetadu}{}{[k\~ɔsɛˈtadu]}{(adj.)}{1}{}{Consertado.}{}{}
\verb{konsetadu}{}{[k\~ɔsɛˈtadu]}{(adj.)}{2}{}{Reparado.}{}{}
\verb{konsola}{}{[k\~ɔsɔˈla]}{(v.)}{1}{}{Consolar.}{}{}
\verb{konta}{}{[kõˈta]}{(v.)}{1}{}{Contar.}{}{}%
\verb{konta}{}{[kõˈta]}{(v.)}{2}{}{Narrar.}{}{}
\verb{konta}{}{[ˈkõta]}{(n.)}{1}{}{Cálculo.}{}{}%
\verb{konta}{}{[ˈkõta]}{(n.)}{2}{}{Conta.}{}{}%
\verb{konta}{}{[ˈkõta]}{(n.)}{3}{}{Dívida.}{}{}
\verb{kontaji}{}{[kõˈtaʒi]}{(n.)}{1}{}{Conto.}{}{}%
\verb{kontaji}{}{[kõˈtaʒi]}{(n.)}{2}{}{Fábula.}{}{}%
\verb{kontaji}{}{[kõˈtaʒi]}{(n.)}{3}{}{Lenda.}{}{}
\verb{kontenta}{}{[kõtẽˈta]}{(v.)}{1}{}{Conformar(-se).}{}{}
\verb{kontenta}{}{[kõtẽˈta]}{(v.)}{2}{}{Contentar(-se).}{}{}
\verb{kontentadu}{}{[kõtẽˈtadu]}{(adj.)}{1}{}{Conformado.}{}{}
\verb{kontentxi}{}{[kõˈtẽtʃi]}{(adj.)}{1}{}{Contente.}{}{}
\verb{kontinensya}{}{[k\~ɔtiˈn\~ɛsja]}{(n.)}{1}{}{Continência.}{}{}
\verb{kontinwa}{}{[k\~ɔtiˈnwa]}{(v.)}{1}{}{Continuar.}{}{}
\verb{kontla}{}{[ˈkõtla]}{(n.)}{1}{}{Amuleto.}{}{}
\verb{kontla}{}{[ˈkõtla]}{(n.)}{2}{}{\textit{Kontla}.}{Remédio tradicional
contra feitiços.}{}
\verb{kontla}{}{[ˈkõtla]}{(prep.)}{1}{}{Contra.}{}{}
\verb{kontla}{}{[kõˈtla]}{(v.)}{1}{}{Encontrar.}{}{}
\verb{kontlata}{}{[kõtlaˈta]}{(v.)}{1}{}{Contratar.}{}{}
\verb{kontlatu}{}{[kõˈtlatu]}{(v.)}{1}{}{Contrato.}{}{}
\verb{kontlê}{}{[kõˈtle]}{(n.)}{1}{}{Ódio.}{}{}
\verb{kontlê}{}{[kõˈtle]}{(v.)}{1}{}{Odiar.}{}{}
\verb{kontu}{}{[ˈkõtu]}{(n.)}{1}{}{Conto.}{Unidade monetária correspondente a mil dobras.}{}
\verb{konvêsê}{}{[kõveˈse]}{(v.)}{1}{}{Conformar(-se).}{}{}
\verb{konveta}{}{[k\~ɔvɛˈta]}{(v.)}{1}{}{Benzer.}{}{}
\verb{konveta}{}{[k\~ɔvɛˈta]}{(v.)}{2}{}{Esconjurar.}{}{}
\verb{konveta}{}{[k\~ɔvɛˈta]}{(v.)}{3}{}{Ficar admirado.}{}{}
\verb{konvidu}{}{[k\~ɔˈvidu]}{(n.)}{1}{}{Convite.}{}{}
\verb{konvitxi}{}{[k\~ɔˈvitʃi]}{(n.)}{1}{}{Convite.}{}{}
\verb{konvlesa}{}{[k\~ɔvlɛˈsa]}{(v.)}{1}{}{Conversar.}{Cf.
\textbf{klonvesa}.}{}{}
\verb{konxa}{}{[ˈkõʃa]}{(n.)}{1}{}{Concha.}{}{}
\verb{konxidu}{}{[k\~ɔˈʃidu]}{(adj.)}{1}{}{Conhecido.}{}{}
\verb{konxtleva}{}{[k\~ɔʃtlɛˈva]}{(v.)}{1}{}{Conservar.}{}{}
\verb{konxtluson}{}{[k\~ɔʃtluˈs\~ɔ]}{(n.)}{1}{}{Construção.}{}{}
\verb{kopa}{}{[ˈkɔpa]}{(n.)}{1}{}{Copa.}{}{}
\verb{kopa}{}{[ˈkɔpa]}{(n.)}{1}{}{Copas.}{Cf. \textbf{fya-kopa}.}{}{}
\verb{kopla}{}{[kɔˈpla]}{(v.)}{1}{}{Comprar.}{}{}
\verb{kopladu}{}{[kɔˈpladu]}{(adj.)}{1}{}{Comprado.}{}{}
\verb{kopu}{}{[ˈkɔpu]}{(n.)}{1}{}{Copo.}{}{}
\verb{kopu-mon}{}{[ˈkɔpu ˈmõ]}{(n.)}{1}{}{Pulso.}{}{}
\verb{kosa}{}{[ˈkɔsa]}{(n.)}{1}{}{Ombro.}{}{}
\verb{kosa}{}{[kɔˈsa]}{(n.)}{1}{}{Sarna.}{}{}
\verb{kosa}{}{[kɔˈsa]}{(v.)}{1}{}{Coçar.}{}{}
\verb{kosakosa}{}{[ˈkɔsaˈkɔsa]}{(n.)}{1}{}{Comigo-ninguém-pode.}{\textbf{\textit{Dieffenbachia
seguine}}.}{}
\verb{kôsô}{}{[ˈkoso]}{(n.)}{1}{}{Canoa onde se lava o andim.}{}{}%
\verb{kôsô}{}{[ˈkoso]}{(n.)}{3}{}{Coxa.}{}{}
\verb{koson}{}{[kɔˈs\~ɔ]}{(n.)}{1}{}{Colchão.}{}{}
\verb{kota}{}{[kɔˈta]}{(v.)}{1}{}{Cortar.}{}{}
\verb{kota}{}{[kɔˈta]}{(v.)}{2}{}{Esconjurar.}{}{}
\verb{kota}{}{[kɔˈta]}{(v.)}{3}{}{Movimento dos quadris durante o ato
sexual.}{}{}%
\verb{kota-bambi}{}{[kɔˈta ˈb\~{\textturna}bi]}{(v.)}{1}{}{Esconjurar pessoa
acometida por \textbf{bambi}.}{}{}
\verb{kota-bega}{}{[ˈkɔta ˈbɛga]}{(n.)}{1}{}{Caçula.}{}{}
\verb{kota-bega}{}{[ˈkɔta ˈbɛga]}{(n.)}{1}{}{Benjamim.}{}{}{}
\verb{kota-bega}{}{[ˈkɔta ˈbɛga]}{(n.)}{1}{}{Caçula.}{Cf. \textbf{kode}.}{}{}\verb{kotadu}{}{[kɔˈtadu]}{(adj.)}{1}{}{Bonito.}{}{}
\verb{kotadu}{}{[kɔˈtadu]}{(adj.)}{2}{}{Cortado.}{}{}
\verb{kota kabêlu-limi}{}{[kɔˈta kaˈbɛlu ˈlimi]}{(expr.)}{1}{}{\textit{Kota
kabêlu-limi}.}{Ritual no qual os cabelos da criança recém-nascida são
cortados e, posteriormente, enterrados.}{}
\verb{kota kadela}{}{[kɔˈta kaˈdɛla]}{(v.)}{1}{}{Saracotear.}{}{}
\verb{kota kla}{}{[kɔˈta ˈkla]}{(expr.)}{1}{}{Cortar ao meio.}{}{}
\verb{kota-wê}{}{[ˈkɔta
ˈwe]}{(n.)}{1}{}{\textit{Kota-wê}.}{\textit{\textbf{Cephalopholis nigri}}.}{}
\verb{kota-wê}{}{[ˈkɔta ˈwe]}{(n.)}{1}{}{Olhar de poucos amigos.}{}{}
\verb{kota winiwini}{}{[kɔˈta wiˈniwiˈni]}{(expr.)}{1}{}{Cortar em
pedacinhos.}{}{}
\verb{kotokoto}{}{[kɔˈtɔkɔˈtɔ]}{(id.)}{1}{}{Cf. \textbf{dana kotokoto.}}{}{}
\verb{kotokoto}{}{[kɔˈtɔkɔˈtɔ]}{(id.)}{2}{}{Cf. \textbf{suzu kotokoto.}}{}{}
\verb{kotxi}{}{[ˈkɔtʃi]}{(n.)}{1}{}{Corte.}{}{}
\verb{kotxi}{}{[ˈkɔtʃi]}{(n.)}{1}{}{Corte real.}{}{}
\verb{kôtxi}{}{[ˈkotʃi]}{(n.)}{2}{}{Palácio.}{}{}
\verb{kovadu}{}{[kɔˈvadu]}{(adj.)}{1}{}{Covarde.}{}{}
\verb{kovadu}{}{[kɔˈvadu]}{(n.)}{1}{}{Covarde.}{}{}
\verb{kôvêlu}{}{[koˈvelu]}{(n.)}{1}{}{Coveiro.}{}{}
\verb{koventa}{}{[kɔˈv\~ɛta]}{(v.)}{1}{}{Converter.}{}{}
\verb{koventa}{}{[kɔˈv\~ɛta]}{(v.)}{2}{}{Esconjurar.}{}{}
\verb{kôvi}{}{[ˈkovi]}{(n.)}{1}{}{Couve.}{}{}
\verb{kôxô}{}{[ˈkoʃo]}{(adj.)}{1}{}{Coxo.}{}{}
\verb{kôxô}{}{[ˈkoʃo]}{(adj.)}{1}{}{Manco.}{}{}
\verb{koxta}{}{[kɔʃˈta]}{(v.)}{1}{}{Apoiar(-se).}{}{}
\verb{koxta}{}{[kɔʃˈta]}{(v.)}{2}{}{Encostar(-se).}{}{}
\verb{koxtadu}{}{[kɔʃˈtadu]}{(adj.)}{1}{}{Apoiado.}{}{}
\verb{koxtadu}{}{[kɔʃˈtadu]}{(adj.)}{2}{}{Associado.}{}{}
\verb{koxtadu}{}{[kɔʃˈtadu]}{(adj.)}{3}{}{Encostado.}{}{}
\verb{kôy}{}{[ˈkoj]}{(n.)}{1}{}{Bossa.}{}{}
\verb{kôy}{}{[ˈkoj]}{(n.)}{2}{}{Corcunda.}{}{}
\verb{kôy}{}{[ˈkoj]}{(n.)}{2}{}{Saliência.}{}{}
\verb{kôyê}{}{[koˈje]}{(v.)}{1}{}{Colher.}{}{}
\verb{kôyê}{}{[koˈje]}{(v.)}{2}{}{Encolher.}{}{}%
\verb{kôyê}{}{[koˈje]}{(v.)}{3}{}{Escolher.}{}{}
\verb{kôyidu}{}{[koˈjidu]}{(adj.)}{2}{}{Encolhido.}{}{}
\verb{kôyidu}{}{[koˈjidu]}{(adj.)}{1}{}{Escolhido.}{}{}%
\verb{kôykôy}{}{[kojˈkoj]}{(adj.)}{1}{}{Cheio de saliências.}{}{}
\verb{kôykôy}{}{[kojˈkoj]}{(adj.)}{1}{}{Desajeitado.}{}{}
\verb{kôykôy}{}{[kojˈkoj]}{(adj.)}{2}{}{Feio.}{}{}{}
\verb{kôzu}{}{[ˈkozu]}{(n.)}{1}{}{Cós.}{}{}%
\verb{ku}{}{[ˈku]}{(conj.)}{1}{}{E.}{Coordena nomes e pronomes. \textbf{Ami
ku ê na ka be fa}. \textit{Eu e ele não vamos}.}{}
\verb{ku}{}{[ˈku]}{(conj.)}{2}{}{Que.}{Introduz orações completivas, forma
reduzida de \textbf{kuma}.}{}
\verb{ku}{}{[ˈku]}{(conj.)}{2}{}{Que.}{Introduz orações relativas. \textbf{N
sa ome ku ka kume muntu}. \emph{Sou um homem que come muito}.}{}
\verb{ku}{}{[ˈku]}{(int.)}{1}{}{É que.}{Elemento que segue pronomes
interrogativos. \textbf{Andji ku neni sa n'ê}? \textit{Onde é que está o
anel}?}{}
\verb{ku}{}{[ˈku]}{(prep.)}{1}{}{Com.}{}{}
\verb{kua}{}{[kuˈa]}{(v.)}{1}{}{Coar.}{}{}
\verb{kua}{}{[kuˈa]}{(v.)}{2}{}{Escoar.}{}{}%
\verb{kuadu}{}{[kuˈadu]}{(adj.)}{1}{}{Coado.}{}{}
\verb{kuadu}{}{[kuˈadu]}{(adj.)}{2}{}{Escoado.}{}{}
\verb{kuba}{}{[kuˈba]}{(n.)}{1}{}{Ato sexual de animais.}{}{}
\verb{kuba}{}{[kuˈba]}{(v.)}{1}{}{Chocar ovos.}{}{}
\verb{kuba}{}{[kuˈba]}{(v.)}{1}{}{Fermentar.}{}{}
\verb{kuba}{}{[kuˈba]}{(v.)}{2}{}{Incubar.}{}{}
\verb{kubangu}{}{[kuˈb\~{\textturna}gu]}{(n.)}{1}{}{Incenso.}{\textbf{\textit{Croton
stellulifer}.}}{}
\verb{kubli}{}{[kuˈbli]}{(v.)}{1}{}{Cobrir.}{}{}%
\verb{kubli}{}{[kuˈbli]}{(v.)}{2}{}{Encobrir.}{}{}%
\verb{kubli}{}{[kuˈbli]}{(v.)}{3}{}{Incubar.}{}{}%
\verb{kubli ovu}{}{[kuˈbli ˈᴐvu]}{(expr.)}{1}{}{Chocar ovo(s).}{}{}
\verb{kubli-ovu}{}{[kuˈbli ˈᴐvu]}{(n.)}{1}{}{Ato sexual de animais.}{}{}
\verb{kubli-wê}{}{[kuˈbli ˈwe]}{(n.)}{1}{}{Bofetada.}{}{}
\verb{kubu}{}{[ˈkubu]}{(n.)}{1}{}{Investida.}{}{}
\verb{kudia}{}{[kudiˈa]}{(n.)}{1}{}{Alimento sagrado oferecido aos
antepassados no \textbf{djambi} e repartidos depois da meia-noite pela
audiência.}{}{}
\verb{kudjan}{}{[kuˈdʒ\~{\textturna}]}{(n.)}{1}{}{Cozinha.}{}{}
\verb{kudji}{}{[kuˈdʒi]}{(v.)}{1}{}{Acudir.}{}{}%
\verb{kudji}{}{[kuˈdʒi]}{(v.)}{2}{}{Cozinhar.}{}{}
\verb{kudji}{}{[kuˈdʒi]}{(v.)}{3}{}{Responder.}{}{}%
\verb{kudjidela}{}{[kudʒiˈdɛla]}{(n.)}{1}{}{Cozedura.}{}{}
\verb{kudjidela}{}{[kudʒiˈdɛla]}{(n.)}{1}{}{Cozinheiro.}{}{}
\verb{kudjidela}{}{[kudʒiˈdɛla]}{(n.)}{1}{}{Fervura.}{}{}
\verb{kudjidu}{}{[kuˈdʒidu]}{(adj.)}{1}{}{Cozido.}{}{}
\verb{kudjidu}{}{[kuˈdʒidu]}{(adj.)}{2}{}{Cozinhado.}{}{}
\verb{kudjimentu}{}{[kudʒiˈmẽtu]}{(n.)}{1}{}{Cozimento medicinal.}{}{}
\verb{kudjimentu}{}{[kudʒiˈmẽtu]}{(n.)}{2}{}{Medicamento.}{}{}
\verb{kudji ome}{}{[kuˈdʒi ˈᴐmɛ]}{(expr.)}{1}{}{Amantizar.}{}{}
\verb{kudji ome}{}{[kuˈdʒi ˈᴐmɛ]}{(expr.)}{1}{}{Amigar.}{}{}
\verb{kudji ome}{}{[kuˈdʒi ˈᴐmɛ]}{(expr.)}{2}{}{Iniciar uma relação marital.}{}{}
\verb{kujidu}{}{[kuˈʒidu]}{(adj.)}{1}{}{Cozido.}{}{}{}
\verb{kujidu}{}{[kuˈʒidu]}{(adj.)}{1}{}{Cozinhado.}{Cf.
\textbf{kudjidu}.}{}{}
\verb{kuku}{}{[kuˈku]}{(n.)}{1}{}{Larvas de inseto que vivem no interior da madeira seca.}{}{}
\verb{kuku}{}{[kuˈku]}{(n.)}{1}{}{Mocho de Ano Bom.}{\textbf{\textit{Otus senegalensis feae}.}}{}
\verb{kukuku}{}{[kuˈkuku]}{(n.)}{1}{}{Coruja.}{}{}
\verb{kukumba}{}{[kuˈkũba]}{(n.)}{1}{}{Saltão.}{\textbf{\textit{Periophthalmus barbarus.}}}{}
\verb{kukunu}{}{[kukuˈnu]}{(v.)}{1}{}{Abaixar(-se).}{}{}
\verb{kukunu}{}{[kukuˈnu]}{(v.)}{2}{}{Acocorar.}{}{}%
\verb{kukunudu}{}{[kukuˈnudu]}{(adj.)}{1}{}{Abaixado.}{}{}
\verb{kukunudu}{}{[kukuˈnudu]}{(adj.)}{2}{}{Acocorado.}{}{}
\verb{kukunudu}{}{[kukuˈnudu]}{(n.)}{1}{}{Modo de preparar polvo.}{}{}
\verb{kula}{}{[ˈkula]}{(n.)}{1}{}{Cura.}{}{}
\verb{kula}{}{[kuˈla]}{(v.)}{1}{}{Curar.}{}{}
\verb{kuladu}{}{[kuˈladu]}{(adj.)}{1}{}{Curado.}{}{}
\verb{kulandêlu}{}{[kul\~{\textturna}ˈdelu]}{(n.)}{1}{}{Curandeiro.}{}{}
\verb{kulêtê}{}{[kuˈlete]}{(n.)}{1}{}{Casaco.}{}{}
\verb{kulêtê}{}{[kuˈlete]}{(n.)}{2}{}{Colete.}{}{}
\verb{kulêtê}{}{[kuˈlete]}{(n.)}{2}{}{Sutiã.}{}{}
\verb{kulu}{}{[kuˈlu]}{(adj.)}{1}{}{Cru.}{}{}
\verb{kulu}{}{[ˈkulu]}{(adj.)}{1}{}{Escuro.}{}{}
\verb{kulu dĩĩĩ}{}{[ˈkulu ˈdĩĩĩ]}{(expr.)}{1}{}{Escuríssimo.}{}
\verb{kulu klani}{}{[ˈkulu klaˈni]}{(expr.)}{1}{}{Cruíssimo.}{}{}
\verb{kulukuku}{}{[ˈkuluˈkuku]}{(n.)}{1}{}{Rola cinzenta.}{\textit{\textbf{Streptopelia senegalensis}}.}{}
\verb{kulukulu}{}{[ˈkuluˈkulu]}{(n.)}{1}{}{Cheiro.}{Normalmente de carne crua, de peixe cru ou da parturiente.}{}{}
\verb{kulukulu}{}{[ˈkuluˈkulu]}{(n.)}{2}{}{Jogo infantil com grãos de milho.}{}{}
\verb{kuma}{}{[ˈkuma]}{(conj.)}{1}{}{Que.}{{Introduz orações completivas. \textbf{Ê sêbê kuma n ga fe tenda se di pêdlêlu}. \textit{Ele sabe que trabalho como pedreiro}}.}
\verb{kuma}{}{[ˈkuma]}{(int.)}{1}{}{Como.}{}{}%
\verb{kuma}{}{[ˈkuma]}{(int.)}{2}{}{De que modo.}{}{}%
\verb{kuma}{}{[ˈkuma]}{(int.)}{3}{}{Então.}{}{}
\verb{kuma}{}{[ˈkuma]}{(n.)}{1}{}{Espuma.}{}{}%
\verb{kuma}{}{[kuˈma]}{(n.)}{1}{}{Comadre.}{}{}%
\verb{kumba}{}{[ˈkũba]}{(n.)}{1}{}{Umbigada.}{}{}
\verb{kumbila}{}{[ˈkũbila]}{(n.)}{1}{}{Confusão.}{}{}%
\verb{kumbila}{}{[ˈkũbila]}{(n.)}{2}{}{Agitação.}{}{}
\verb{kumbila}{}{[ˈkũbila]}{(n.)}{1}{}{\textit{Kumbila}.}{Tipo de dança em que se bate o peito de uma pessoa contra a outra.}{}{}
\verb{kumbina}{}{[kũbiˈna]}{(v.)}{1}{}{Combinar.}{}{}
\verb{kumbinadu}{}{[kũbiˈnadu]}{(adj.)}{1}{}{Combinado.}{}{}
\verb{kumbinanson}{}{[k\~ubin\~{\textturna}ˈsõ]}{(n.)}{1}{}{Roupa interior feminina.}{}{}
\verb{kumbinason}{}{[kũbinaˈsõ]}{(n.)}{1}{}{Acordo.}{}{}
\verb{kumbinason}{}{[kũbinaˈsõ]}{(n.)}{1}{}{Combinação.}{}{}
\verb{kumbu}{}{[k\~uˈbu]}{(adj.)}{1}{}{Cortado.}{}{} 
\verb{kumbu}{}{[k\~uˈbu]}{(adj.)}{2}{}{Reduzido.}{}{} 
\verb{kumbudu}{}{[k\~uˈbudu]}{(adj.)}{1}{}{Cortado curto.}{}{} 
\verb{kumbudu}{}{[k\~uˈbudu]}{(adj.)}{2}{}{Reduzido.}{}{}
\verb{kume}{}{[kuˈmɛ]}{(n.)}{1}{}{Comida.}{}{}
\verb{kume}{}{[kuˈmɛ]}{(v.)}{1}{}{Comer.}{}{}
\verb{kume awa}{}{[kuˈmɛ ˈawa]}{(expr.)}{1}{}{Embriagar(-se).}{}{}
\verb{kume dentxi}{}{[kuˈmɛ ˈd\~etʃi]}{(expr.)}{1}{}{Rilhar os dentes.}{}{}
\verb{kume dentxi}{}{[kuˈmɛ ˈd\~etʃi]}{(expr.)}{1}{}{Zangar(-se).}{}{}
\verb{kumedô}{}{[kumɛˈdo]}{(n.)}{1}{}{Comedor.}{}{}
\verb{kumedu}{}{[kuˈmɛdu]}{(adj.)}{1}{}{Comido.}{}{}
\verb{kumedu awa}{}{[kuˈmɛdu ˈawa]}{(expr.)}{1}{}{Embriagado.}{}{}
\verb{kumedu dentxi}{}{[kuˈmɛdu ˈd\~etʃi]}{(expr.)}{1}{}{Zangado.}{}{}
\verb{kume-môlê}{}{[kuˈmɛ
ˈmole]}{(n.)}{1}{}{Come-morre.}{\textbf{\textit{Scorpaena laevis}.}}{}
\verb{kume-môlê}{}{[kuˈmɛ
ˈmole]}{(n.)}{2}{}{Peixe-escorpião.}{\textbf{\textit{Scorpaena laevis}.}}{}
\verb{kumison}{}{[kumiˈsõ]}{(n.)}{1}{}{Comichão.}{}{}
\verb{kumisu}{}{[kuˈmisu]}{(n.)}{1}{}{Comício.}{}{}
\verb{kumpli}{}{[kũˈpli]}{(v.)}{1}{}{Cumprir.}{}{}%
\verb{kumpli}{}{[kũˈpli]}{(v.)}{2}{}{Obedecer.}{}{}
\verb{kunda}{}{[kũˈda]}{(v.)}{1}{}{Julgar.}{}{}
\verb{kunda}{}{[kũˈda]}{(v.)}{2}{}{Pensar.}{}{}%
\verb{kunda}{}{[kũˈda]}{(v.)}{3}{}{Presumir.}{}{}%
\verb{kunda}{}{[kũˈda]}{(v.)}{4}{}{Supor.}{}{}
\verb{kundava}{}{[kũˈdava]}{(v.)}{1}{}{Julgava.}{}{}
\verb{kundava}{}{[kũˈdava]}{(v.)}{2}{}{Pensava.}{}{}
\verb{kundava}{}{[kũˈdava]}{(v.)}{3}{}{Presumia.}{}{}
\verb{kundava}{}{[kũˈdava]}{(v.)}{4}{}{Supunha.}{}{}%
\verb{kundu}{}{[kũˈdu]}{(n.)}{1}{}{Pentelho.}{}{}
\verb{kundu-di-mwala-ve}{}{[k\~uˈdu di ˈmwala
ˈvɛ]}{(n.)}{1}{\textit{Kundu-di-mwala-ve}.}{\textbf{\textit{Acanthus
montanus}}.}{}
\verb{kunduta}{}{[kũˈduta]}{(n.)}{1}{}{Comportamento.}{}{}
\verb{kunfuzon}{}{[k\~ufuˈz\~ɔ]}{(n.)}{1}{}{Confusão.}{}{}
\verb{kunfya}{}{[kũˈfja]}{(v.)}{1}{}{Confiar.}{}{}
\verb{kunfyansa}{}{[kũˈfj\~{\textturna}sa]}{(n.)}{1}{}{Confiança.}{}{}
\verb{kunga}{}{[kũˈga]}{(v.)}{2}{}{Esfregar com força.}{}{}%
\verb{kunga}{}{[kũˈga]}{(v.)}{3}{}{Pôr.}{}{}
\verb{kunga pê}{}{[kũˈga ˈpe]}{(expr.)}{1}{}{Assinalar.}{}{}
\verb{kunha}{}{[ˈkũɲa]}{(n.)}{1}{}{Auxílio de pessoa influente.}{}{}
\verb{kunha}{}{[ˈkũɲa]}{(n.)}{2}{}{Cacetada.}{}{}
\verb{kunha}{}{[ˈkũɲa]}{(n.)}{3}{}{Cunha.}{}{}
\verb{kunha}{}{[kuˈɲa]}{(v.)}{1}{}{Encravar.}{}{}
\verb{kunhada}{}{[kuˈɲada]}{(n.)}{1}{}{Cunhada.}{}{}
\verb{kunhadu}{}{[kuˈɲadu]}{(n.)}{1}{}{Cunhado.}{}{}
\verb{kunsuta}{}{[k\~uˈsuta]}{(n.)}{1}{}{Consulta.}{}{}
\verb{kunsuta}{}{[k\~usuˈta]}{(v.)}{1}{}{Consultar.}{}{}
\verb{kunsuta}{}{[k\~usuˈta]}{(v.)}{2}{}{Dar consultas.}{}{}
\verb{kunu}{}{[kuˈnu]}{(v.)}{1}{}{Arrumar.}{}{}
\verb{kunu}{}{[kuˈnu]}{(v.)}{2}{}{Juntar.}{}{}
\verb{kunu}{}{[kuˈnu]}{(v.)}{3}{}{Recolher.}{}{}
\verb{kunu}{}{[kuˈnu]}{(v.)}{4}{}{Reunir.}{}{}
\verb{kunudu}{}{[kuˈnudu]}{(adj.)}{1}{}{Aconchegado.}{}{}
\verb{kunudu}{}{[kuˈnudu]}{(adj.)}{2}{}{Arrumado.}{}{}
\verb{kunvidadu}{}{[kũviˈdadu]}{(n.)}{1}{}{Convidado.}{}{}
\verb{kunvitxi}{}{[kũˈvitʃi]}{(n.)}{1}{}{Convite.}{Cf. \textbf{konvidu}.}{}{}\verb{kunxensa}{}{[kũˈʃ\~ɛsa]}{(n.)}{1}{}{Consciência.}{}{}
\verb{kunxensa}{}{[kũˈʃ\~ɛsa]}{(n.)}{1}{}{Ética.}{}{}
\verb{kunxensa}{}{[kũˈʃ\~ɛsa]}{(n.)}{1}{}{Moral.}{}{}
\verb{kunxintxi}{}{[k\~uʃ\~iˈtʃi]}{(v.)}{1}{}{Consentir.}{}{}
\verb{kunzula}{}{[k\~uzuˈla]}{(v.)}{1}{}{Esconjurar.}{}{}
\verb{kupa}{}{[kuˈpa]}{(v.)}{1}{}{Ocupar.}{}{}
\verb{kupa}{}{[kuˈpa]}{(v.)}{1}{}{Preocupar(-se).}{}{}
\verb{kupadu}{}{[kuˈpadu]}{(adj.)}{1}{}{Ocupado.}{}{}
\verb{kupadu}{}{[kuˈpadu]}{(adj.)}{1}{}{Preocupado.}{}{}
\verb{kupi}{}{[kuˈpi]}{(n.)}{1}{}{Cuspe.}{}{}
\verb{kupi}{}{[kuˈpi]}{(n.)}{1}{}{Saliva.}{}{}
\verb{kupi}{}{[kuˈpi]}{(v.)}{1}{}{Cuspir.}{}{}
\verb{kupli}{}{[kuˈpli]}{(v.)}{1}{}{Cumprir.}{}{}
\verb{kusu}{}{[ˈkusu]}{(n.)}{1}{}{Curso.}{}{}
\verb{kusu}{}{[ˈkusu]}{(n.)}{2}{}{Diarréia.}{}{}
\verb{kusukusu}{}{[ˈkusuˈkusu]}{(n.)}{1}{}{Cuscuz.}{}{}
\verb{kusupi}{}{[kusuˈpi]}{(v.)}{1}{}{Esforçar-se muito.}{}{}{}
\verb{kusupi}{}{[kusuˈpi]}{(v.)}{1}{}{Sacrificar.}{}{}{}
\verb{kuta}{}{[kuˈta]}{(v.)}{1}{}{Escutar.}{}{}%
\verb{kuta}{}{[kuˈta]}{(v.)}{2}{}{Ouvir.}{}{}
\verb{kutu}{}{[ˈkutu]}{(adj.)}{1}{}{Baixo.}{}{}
\verb{kutu}{}{[ˈkutu]}{(adj.)}{2}{}{Curto.}{}{}
\verb{kutu}{}{[kuˈtu]}{(v.)}{1}{}{Engrossar.}{}{}
\verb{kutu}{}{[kuˈtu]}{(v.)}{2}{}{Turvar.}{}{}
\verb{kutudja}{}{[kutuˈdʒa]}{(n.)}{1}{}{Vergonha.}{}{}
\verb{kutuja}{}{[kutuˈʒa]}{(n.)}{1}{}{Vergonha.}{Cf. \textbf{kutudja}.}{}{}
\verb{kutukutu}{}{[ˈkutuˈkutu]}{(n.)}{1}{}{Sarampo.}{}{}
\verb{kutukutu}{}{[ˈkutuˈkutu]}{(n.)}{2}{}{Manchas vermelhas.}{}{}
\verb{kutukutu}{}{[ˈkutuˈkutu]}{(n.)}{3}{}{Varicela.}{}{}%
\verb{kutura}{}{[kuˈtura]}{(n.)}{1}{}{Cultura.}{}{}
\verb{kuvida}{}{[kuviˈda]}{(v.)}{1}{}{Convidar.}{}{}
\verb{kuvidadu}{}{[kuviˈdadu]}{(n.)}{1}{}{Convidado.}{Cf.
\textbf{kunvidadu}.}{}{}
\verb{kuvidu}{}{[kuˈvidu]}{(n.)}{1}{}{Boa recepção.}{}{}
\verb{kuvisozu}{}{[kuviˈsɔzu]}{(adj.)}{1}{}{Ganancioso.}{}{}
\verb{kuvisozu}{}{[kuviˈsɔzu]}{(adj.)}{2}{}{Invejoso.}{}{}
\verb{kuxpila}{}{[ˈkuʃpila]}{(n.)}{1}{}{\textit{Kuxpila}.}{\textbf{\textit{Tetrapleura
tetraptera}.}}{}
\verb{kuxta}{}{[ˈkuʃta]}{(n.)}{1}{}{Custa.}{}{}
\verb{kuxta}{}{[kuʃˈta]}{(v.)}{1}{}{Custar.}{}{}
\verb{kuxtipadu}{}{[kuʃtiˈpadu]}{(adj.)}{1}{}{Resfriado.}{}{}
\verb{kuxtipason}{}{[kuʃtipaˈsõ]}{(n.)}{1}{}{Constipação.}{}{}%
\verb{kuxtipason}{}{[kuʃtipaˈsõ]}{(n.)}{2}{}{Gripe.}{}{}%
\verb{kuxtipason}{}{[kuʃtipaˈsõ]}{(n.)}{3}{}{Resfriado.}{}{}%
\verb{kuxtu}{}{[ˈkuʃtu]}{(n.)}{1}{}{Custo.}{}{}
\verb{kuxtu}{}{[ˈkuʃtu]}{(n.)}{2}{}{Sacrifício.}{}{}
\verb{kuxtu}{}{[ˈkuʃtu]}{(n.)}{3}{}{Valor.}{}{}
\verb{kuxtumadu}{}{[kuʃtuˈmadu]}{(adj.)}{1}{}{Acostumado.}{}{}
\verb{kuxtumadu}{}{[kuʃtuˈmadu]}{(adj.)}{2}{}{Habituado.}{}{}
\verb{kuxtumi}{}{[kuʃˈtumi]}{(n.)}{1}{}{Costume.}{Cf. \textbf{kuxtumu}.}{}{}%\verb{kuxtumi}{}{[kuʃˈtumi]}{(n.)}{1}{}{Hábito.}{Cf. \textbf{kuxtumu}.}{}{}%
\verb{kuxtumu}{}{[kuʃˈtumu]}{(n.)}{1}{}{Costume.}{}{}
\verb{kuxtumu}{}{[kuʃˈtumu]}{(n.)}{1}{}{Hábito.}{}{}
\verb{kuyê}{}{[kuˈje]}{(n.)}{1}{}{Colher.}{}{}
\verb{kuza}{}{[kuˈza]}{(v.)}{1}{}{Acusar.}{}{}
\verb{kuzidu}{}{[kuˈzidu]}{(n.)}{1}{}{Cozido.}{}{}
\verb{kuzinhêra}{}{[kuziˈɲera]}{(n.)}{1}{}{Cozinheira.}{}{}
\verb{kuzinhêru}{}{[kuziˈɲeru]}{(n.)}{1}{}{Cozinheiro.}{}{}
\verb{kwa}{}{[ˈkwa]}{(int.)}{1}{}{O que.}{}{}
\verb{kwa}{}{[ˈkwa]}{(n.)}{1}{}{Algo.}{}{}%
\verb{kwa}{}{[ˈkwa]}{(n.)}{2}{}{Aquilo.}{}{}%
\verb{kwa}{}{[ˈkwa]}{(n.)}{3}{}{Assunto.}{}{}
\verb{kwa}{}{[ˈkwa]}{(n.)}{4}{}{Coisa.}{}{}%
\verb{kwa-bêbê}{}{[ˈkwa beˈbe]}{(n.)}{1}{}{Bebida.}{}{}%
\verb{kwada}{}{[ˈkwada]}{(n.)}{1}{}{Cunhada.}{}{}%
\verb{kwadlu}{}{[ˈkwadlu]}{(n.)}{1}{}{Quadro.}{}{}
\verb{kwadô}{}{[kwaˈdo]}{(n.)}{1}{}{Coador.}{}{}
\verb{kwadu}{}{[ˈkwadu]}{(n.)}{1}{}{Cunhado.}{}{}%
\verb{kwadu}{}{[ˈkwadu]}{(n.)}{2}{}{Genro.}{}{}
\verb{kwa-fe}{}{[ˈkwa ˈfɛ]}{(n.)}{1}{}{Ocupação.}{}{}
\verb{kwa-fedu}{}{[ˈkwa ˈfɛdu]}{(n.)}{1}{}{Conspiração.}{}{}%
\verb{kwa-fedu}{}{[ˈkwa ˈfɛdu]}{(n.)}{2}{}{Feitiço.}{}{}
\verb{kwa-floga}{}{[ˈkwa flᴐˈga]}{(n.)}{1}{}{Brincadeira.}{}{}
\verb{kwaji}{}{[ˈkwaʒi]}{(adj.)}{1}{}{Ansioso.}{}{}
\verb{kwaji}{}{[ˈkwaʒi]}{(adv.)}{1}{}{Quase.}{}{}
\verb{kwa-kala}{}{[ˈkwa ˈkala]}{(n.)}{1}{}{Bofetada.}{}{}%
\verb{kwaku}{}{[ˈkwaku]}{(n.)}{1}{}{Escova de dentes tradicional.}{}{}
\verb{kwaku}{}{[ˈkwaku]}{(n.)}{2}{}{Licor de ervas.}{}{}
\verb{kwaku-bangana}{}{[ˈkwaku
b\~{\textturna}ˈgana]}{(n.)}{1}{}{\textit{Kwaku-bangana}.}{\textit{\textbf{Ophiobotrys
zenkeri}.}}{}
\verb{kwaku-blanku}{}{[ˈkwaku
ˈbl\~{\textturna}ku]}{(n.)}{1}{}{\textit{Kwaku-blanku}.}{\textit{\textbf{Celtis
prantlii}.}}{}
\verb{kwaku-magita}{}{[ˈkwaku
maˈgita]}{(n.)}{1}{}{\textit{Kwaku-magita}.}{\textbf{\textit{Psychotria
subobliqua}.}}{}
\verb{kwa-kume}{}{[ˈkwa kuˈmɛ]}{(n.)}{1}{}{Comida.}{}{}
\verb{kwakwa}{}{[kwaˈkwa]}{(n.)}{1}{}{Carapaça de crustáceo.}{}{}
\verb{kwakwa}{}{[kwaˈkwa]}{(n.)}{2}{}{Modo de preparar o peixe-voador}{}{}
\verb{kwakwa-klôsô}{}{[kwaˈkwa
kloˈso]}{(n.)}{1}{}{\textit{Kwakwa-klôsô}.}{\textbf{\textit{Alternanthera
sessilis}.}}{}
\verb{kwakwakwa}{}{[kwakwaˈkwa]}{(id.)}{1}{}{Cf. \textbf{da kebla
kwakwakwa.}}{}{}
\verb{kwakwali}{}{[kwaˈkwali]}{(quant.)}{1}{}{Qualquer.}{\textbf{Kwakwali soya di sun alê ku san lenha.} \emph{Qualquer história do rei e da rainha.}}{}{}
\verb{kwa-leda}{}{[ˈkwa lɛˈda]}{(n.)}{1}{}{Herança.}{}{}
\verb{kwali}{}{[ˈkwali]}{(n.)}{1}{}{Cesto.}{}{}
\verb{kwamanda}{}{[kwam\~{\textturna}ˈda]}{(int.)}{1}{}{Por que.}{Cf.
\textbf{kamanda}.}{}{}
\verb{kwa-ple}{}{[ˈkwa ˈplɛ]}{(n.)}{1}{}{Frutos do mar.}{}{}
\verb{kwata}{}{[kwaˈta]}{(n.)}{1}{}{Alqueire.}{}{}
\verb{kwata}{}{[kwaˈta]}{(v.)}{1}{}{Incitar um cão.}{}{}
\verb{kwata-fela}{}{[ˈkwata ˈfɛla]}{(n.)}{1}{}{Quarta-feira.}{}{}
\verb{kwatlu}{}{[ˈkwatlu]}{(num.)}{1}{}{Quatro.}{}{}
\verb{kwatlu-dexi}{}{[ˈkwatlu ˈdɛʃi]}{(num.)}{1}{}{Quarenta.}{}{}
\verb{kwatlusentu}{}{[ˈkwatluˈs\~ɛtu]}{(num.)}{1}{}{Quatrocentos.}{}{}
\verb{kwatu}{}{[ˈkwatu]}{(n.)}{2}{}{Quarto.}{Cf. \textbf{nglentu}.}{}{}
\verb{kwê}{}{[ˈkwe]}{(n.)}{1}{}{Coelho.}{\textbf{\textit{Oryctolagus cuniculus}}.}{}%
\verb{kwê}{}{[ˈkwe]}{(n.)}{2}{}{Coelho.}{\textbf{\textit{Lagocephalus laevigatus}}.}{}
\verb{kwedanu}{}{[kwɛˈdanu]}{(n.)}{1}{}{Coedano.}{\textbf{\textit{Cestrum laevigatum}.}}{}
\verb{kweka}{}{[ˈkwɛka]}{(n.)}{1}{}{Cuecas.}{}{}
\verb{kwekwe}{}{[kwɛˈkwɛ]}{(n.)}{1}{}{Cuecas.}{}{}
\verb{kwekwe}{}{[kwɛˈkwɛ]}{(n.)}{1}{}{Silvo do morcego.}{}{}
%\verb{kweme}{}{[kwɛˈmɛ]}{(n.)}{1}{}{Sofrer por pancadas.}{}{}
\verb{kwenkwenkwen}{}{[kw\~ɛkw\~ɛˈkw\~ɛ]}{(id.)}{1}{}{Cf. \textbf{dwentxi
kwenkwenkwen.}}{}{}
\verb{kwêtadu}{}{[kweˈtadu]}{(adj.)}{1}{}{Coitado.}{}{}
\verb{kwidadu}{}{[kwiˈdadu]}{(n.)}{1}{}{Cuidado.}{}{}
\verb{kwini}{}{[ˈkwini]}{(n.)}{1}{}{Inhame-selvagem.}{Cf.
\textbf{nhami-kwini}.}{}
\verb{kyabu}{}{[ˈkjabu]}{(n.)}{1}{}{Quiabo.}{Cf. \textbf{ikyabu}.}{}{}
\verb{kyason}{}{[ˈkjasõ]}{(n.)}{1}{}{Criação.}{}{}
\verb{kyê}{}{[ˈkje]}{(interj.)}{1}{}{Credo!}{}{}
\verb{kyê}{}{[ˈkje]}{(interj.)}{1}{}{Nossa!}{}{}
\verb{kyê}{}{[ˈkje]}{(interj.)}{1}{}{Oh!}{}{}
\verb{kyê}{}{[ˈkje]}{(v.)}{1}{}{Cair.}{}{}
\verb{kyê bega}{}{[ˈkje ˈbɛga]}{(expr.)}{1}{}{Engravidar.}{}{}
\verb{kyê kitali}{}{[ˈkje kiˈtali]}{(expr.)}{1}{}{Ser apanhado.}{}{}
\verb{kyê kitali}{}{[ˈkje kiˈtali]}{(expr.)}{1}{}{Ser preso.}{}{}
\verb{kyê ku}{}{[ˈkje ˈku]}{(expr.)}{1}{}{Ajeitar.}{}{}
\verb{kyê ku}{}{[ˈkje ˈku]}{(expr.)}{2}{}{Ficar bem.}{}{}
\verb{kyê toma kadu}{}{[ˈkje tɔˈma ˈkadu]}{(expr.)}{1}{}{Agradar.}{}{}
\verb{kyêy}{}{[ˈkjej]}{(interj.)}{1}{}{Credo!}{}{}
\verb{kyêy}{}{[ˈkjej]}{(interj.)}{1}{}{Nossa!}{}{}
\verb{kyêy}{}{[ˈkjej]}{(interj.)}{1}{}{Oh!}{Cf. \textbf{kyê}.}{}
\verb{kyolakyola}{}{[ˈkjɔlaˈkjɔla]}{(adj.)}{1}{}{Ansioso.}{}{}
\verb{kyolakyola}{}{[ˈkjɔlaˈkjɔla]}{(adj.)}{1}{}{Impaciente.}{}{}
\verb{kyolakyola}{}{[ˈkjɔlaˈkjɔla]}{(adv.)}{1}{}{Quase.}{}{}
\verb{kyomba}{}{[ˈkjõba]}{(n.)}{1}{}{Flor da palmeira.}{}{}%
\verb{kyomba}{}{[ˈkjõba]}{(n.)}{2}{}{Vinho de palma imbebível.}{}{}
\verb{kyon}{}{[ˈkj\~ɔ]}{(n.)}{1}{}{Montinho.}{}{}
\verb{kyonkyonkyon}{}{[kjõkjõˈkjõ]}{(adj.)}{1}{}{Mórbido.}{}{}
\end{letra}

\begin{letra}{l}
\verb{la}{}{[ˈla]}{(v.)}{1}{}{Lavar.}{Cf. \textbf{laba}.}{}{}
\verb{laba}{}{[laˈba]}{(v.)}{1}{}{Lavar.}{}{}
\verb{labadu}{}{[laˈbadu]}{(adj.)}{1}{}{Lavado.}{}{}
\verb{labadu txe}{}{[laˈbadu ˈtʃɛ]}{(expr.)}{1}{}{Bem lavado.}{}{}
\verb{labandêla}{}{[lab\~{\textturna}ˈdela]}{(n.)}{1}{}{Lavadeira.}{}{}
\verb{labeka}{}{[laˈbɛka]}{(n.)}{1}{}{Rabeca.}{}{}
\verb{labu}{}{[ˈlabu]}{(n.)}{1}{}{Cauda.}{}{}
\verb{labu}{}{[ˈlabu]}{(n.)}{2}{}{Fim.}{}{}
\verb{labu}{}{[ˈlabu]}{(n.)}{3}{}{Rabo.}{}{}%
\verb{labu}{}{[ˈlabu]}{(n.)}{4}{}{Término.}{}{}
\verb{labu}{}{[ˈlabu]}{(n.)}{5}{}{Última parte.}{}{}
\verb{ladenha}{}{[laˈd\~ɛɲa]}{(n.)}{1}{}{Ladainha.}{}{}%
\verb{ladenha}{}{[laˈd\~ɛɲa]}{(n.)}{2}{}{Litania.}{}{}
\verb{ladla}{}{[laˈdla]}{(v.)}{1}{}{Ladrar.}{}{}
\verb{ladlon}{}{[laˈdlõ]}{(n.)}{1}{}{Ladrão.}{}{}
\verb{ladu}{}{[ˈladu]}{(n.)}{1}{}{Lado.}{}{}
\verb{laga}{}{[laˈga]}{(v.)}{1}{}{Descuidar.}{}{}
\verb{laga}{}{[laˈga]}{(v.)}{2}{}{Descurar.}{}{}
\verb{laga}{}{[laˈga]}{(v.)}{3}{}{Enganar(-se).}{}{}
\verb{lagasa}{}{[lagaˈsa]}{(v.)}{1}{}{Desacertar.}{}{}
\verb{lagasa}{}{[lagaˈsa]}{(v.)}{2}{}{Desairar.}{}{}
\verb{lagasa}{}{[lagaˈsa]}{(v.)}{3}{}{Desbragar.}{}{}
\verb{lagasa}{}{[lagaˈsa]}{(v.)}{4}{}{Penetrar.}{}{}
\verb{lagasa}{}{[lagaˈsa]}{(v.)}{5}{}{Saltar.}{}{}
\verb{lagasadu}{}{[lagaˈsadu]}{(adj.)}{1}{}{Desbragado.}{}{}%
\verb{lagatlisa}{}{[lagaˈtlisa]}{(n.)}{1}{}{Lagartixa do mato.}{}{}
\verb{lagaya}{}{[laˈgaja]}{(n.)}{2}{}{Lagaia.}{\textbf{\textit{Civetictis
civetta}}.}{}
\verb{laglima-dêsu}{}{[ˈlaglima
ˈdesu]}{(n.)}{1}{}{Lágrima-de-Cristo.}{\textbf{\textit{Breynia disticha}}.}{}
\verb{lagôxta}{}{[laˈgoʃta]}{(n.)}{1}{}{Lagosta.}{}{}
\verb{lajigatxi}{}{[laʒiˈgatʃi]}{(n.)}{1}{}{Resgate.}{}{}
\verb{lakla}{}{[laˈkla]}{(v.)}{1}{}{Lacrar.}{}{}
\verb{la kwa}{}{[ˈla ˈkwa]}{(expr.)}{1}{}{Lavar loiça ou roupa no rio.}{Cf.
\textbf{ba awa la kwa}.}{}
\verb{lala}{}{[ˈlala]}{(n.)}{1}{}{Ralador.}{}{}
\verb{lala}{}{[laˈla]}{(v.)}{1}{}{Chamuscar.}{}{}
\verb{lala}{}{[laˈla]}{(v.)}{2}{}{Ralar.}{}{}
\verb{laladu}{}{[laˈladu]}{(adj.)}{1}{}{Ralado.}{}{}
\verb{lalala}{}{[lalaˈla]}{(id.)}{1}{}{Cf. \textbf{bôbô lalala}.}{}{}
\verb{lalaxa}{}{[lalaˈʃa]}{(v.)}{1}{}{Abandalhar.}{}{}
\verb{lalaxadu}{}{[lalaˈʃadu]}{(adj.)}{1}{}{Abandalhado.}{}{}
\verb{lalu}{}{[ˈlalu]}{(n.)}{1}{}{Doença de pele.}{}{}
\verb{lalugu}{}{[ˈlalugu]}{(adj.)}{1}{}{Amplo.}{}{}
\verb{lalugu}{}{[ˈlalugu]}{(adj.)}{2}{}{Largo.}{}{}
\verb{lama}{}{[ˈlama]}{(n.)}{1}{}{Lama.}{}{}
\verb{lamalon}{}{[lamaˈlõ]}{(n.)}{1}{}{Charco.}{}{}
\verb{lamalon}{}{[lamaˈlõ]}{(n.)}{2}{}{Lamarão.}{}{}
\verb{lamboya}{}{[l\~{\textturna}ˈbɔja]}{(n.)}{1}{}{Vadiagem.}{}{}
\verb{lambu}{}{[l\~{\textturna}ˈbu]}{(n.)}{1}{}{Tanga.}{}{}
\verb{lamentason}{}{[lam\~ɛtaˈs\~ɔ]}{(n.)}{1}{}{Lamento.}{}{}
\verb{lamile}{}{[lamiˈlɛ]}{(n.)}{1}{}{Inconfidências.}{}{}
\verb{lamina}{}{[ˈlamina]}{(n.)}{1}{}{Lâmina.}{}{}
\verb{lampada}{}{[ˈl\~{\textturna}pada]}{(n.)}{1}{}{Lâmpada.}{}{}
\verb{lampyon}{}{[l\~{\textturna}ˈpjõ]}{(n.)}{1}{}{Lampião.}{}{}
\verb{lamu}{}{[ˈlamu]}{(n.)}{1}{}{Ramo.}{}{}
\verb{lan}{}{[ˈl\~{\textturna}]}{(n.)}{1}{}{Lã.}{}{}
\verb{landa}{}{[l\~{\textturna}ˈda]}{(v.)}{1}{}{Nadar.}{}{}
\verb{landadô}{}{[l\~{\textturna}daˈdo]}{(n.)}{1}{}{Nadador.}{}{}
\verb{langutangu}{}{[ˈl\~{\textturna}guˈt\~{\textturna}gu]}{(n.)}{1}{}{Orangotango.}{}{}
\verb{lanhu}{}{[ˈl\~{\textturna}ɲu]}{(n.)}{1}{}{Ranho.}{}{}
\verb{lanja}{}{[l\~{\textturna}ˈʒa]}{(v.)}{1}{}{Arranjar.}{}{}
\verb{lanka}{}{[l\~{\textturna}ˈka]}{(v.)}{1}{}{Arrancar.}{}{}
\verb{lanka bega}{}{[l\~{\textturna}ˈka ˈbɛga]}{(expr.)}{1}{}{Abortar.}{}{}
\verb{lanka da son}{}{[l\~{\textturna}ˈka ˈda ˈs\~ɔ]}{(expr.)}{1}{}{Cair ao
chão.}{}{}
\verb{lankadu}{}{[l\~{\textturna}ˈkadu]}{(adj.)}{1}{}{Arrancado.}{}{}
\verb{lansa}{}{[ˈl\~{\textturna}sa]}{(n.)}{1}{}{Lança.}{}{}
\verb{lansa}{}{[l\~{\textturna}ˈsa]}{(v.)}{1}{}{Lançar.}{}{}
\verb{lansa}{}{[l\~{\textturna}ˈsa]}{(v.)}{2}{}{Vomitar.}{}{}
\verb{lanseta}{}{[l\~{\textturna}ˈsɛta]}{(n.)}{1}{}{Lanceta.}{}{}
\verb{lansolo}{}{[l\~{\textturna}ˈsɔlɔ]}{(n.)}{1}{}{Lençol.}{}{}
\verb{lanta}{}{[l\~{\textturna}ˈta]}{(v.)}{1}{}{Erguer.}{}{}%
\verb{lanta}{}{[l\~{\textturna}ˈta]}{(v.)}{2}{}{Erigir.}{}{}
\verb{lanta}{}{[l\~{\textturna}ˈta]}{(v.)}{3}{}{Levantar(-se).}{}{}
\verb{lantadu}{}{[l\~{\textturna}ˈtadu]}{(adj.)}{1}{}{Erguido.}{}{}
\verb{lantadu}{}{[l\~{\textturna}ˈtadu]}{(adj.)}{2}{}{Levantado.}{}{}
\verb{lanta fasu}{}{[l\~{\textturna}ˈta ˈfasu]}{(expr.)}{1}{}{Caluniar.}{}{}
\verb{lanta fasu}{}{[l\~{\textturna}ˈta ˈfasu]}{(expr.)}{1}{}{Difamar.}{}{}
\verb{lantena}{}{[l\~{\textturna}ˈtɛna]}{(n.)}{1}{}{Lanterna.}{}{}
\verb{lanxa}{}{[ˈl\~{\textturna}ʃa]}{(n.)}{3}{}{Lancha.}{}{}
\verb{lanza}{}{[ˈl\~{\textturna}za]}{(n.)}{1}{}{Laranja.}{}{}
\verb{lanza-matu}{}{[ˈl\~{\textturna}za ˈmatu]}{(n.)}{2}{}{Laranja-do-mato.}{}{}
\verb{lanza-mukambu}{}{[ˈl\~{\textturna}za muk\~{\textturna}ˈbu]}{(n.)}{1}{}{\textit{Lanza-mukambu}.}{Cf.
\textbf{libô}.}{}{}
\verb{lapa}{}{[ˈlapa]}{(adj.)}{1}{}{Achatado.}{}{}
\verb{lapa}{}{[ˈlapa]}{(adj.)}{2}{}{Plano.}{}{}
\verb{lapa}{}{[ˈlapa]}{(adj.)}{3}{}{Raso.}{}{}
\verb{lapa}{}{[laˈpa]}{(v.)}{1}{}{Achatar.}{}{}
\verb{lape}{}{[laˈpɛ]}{(n.)}{1}{}{Rapé.}{}{}
\verb{lapi}{}{[ˈlapi]}{(n.)}{1}{}{Lápis.}{}{}
\verb{lapujin}{}{[lapuˈʒĩ]}{(n.)}{1}{}{Sabor desagradável.}{}{}
\verb{lasa}{}{[ˈlasa]}{(n.)}{1}{}{Raça.}{}{}
\verb{lason}{}{[laˈsõ]}{(n.)}{1}{}{Oração.}{}{}
\verb{lason}{}{[laˈsõ]}{(n.)}{2}{}{Prece.}{}{}
\verb{lasu}{}{[ˈlasu]}{(n.)}{3}{}{Laço.}{}{}
\verb{lasu}{}{[ˈlasu]}{(n.)}{3}{}{Nó.}{}{}
\verb{lasu-kôlê}{}{[ˈlasu koˈle]}{(n.)}{1}{}{Laço corredio.}{}{}
\verb{lata}{}{[ˈlata]}{(n.)}{1}{}{Lata.}{}{}
\verb{late}{}{[laˈtɛ]}{(n.)}{1}{}{Absorvente íntimo.}{}{}
\verb{late}{}{[laˈtɛ]}{(n.)}{2}{}{Farrapo.}{}{}
\verb{late}{}{[laˈtɛ]}{(n.)}{3}{}{Retalhos.}{}{}
\verb{late}{}{[laˈtɛ]}{(n.)}{4}{}{Trapo.}{}{}
\verb{latêya}{}{[laˈteja]}{(n.)}{1}{}{Adorno.}{}{}
\verb{latêya}{}{[laˈteja]}{(n.)}{1}{}{Enfeite.}{}{}
\verb{latêyadu}{}{[lateˈjadu]}{(adj.)}{2}{}{Adornado.}{}{}
\verb{latêyadu}{}{[lateˈjadu]}{(adj.)}{2}{}{Enfeitado.}{}{}
\verb{latu}{}{[ˈlatu]}{(n.)}{1}{}{Rato.}{}{}
\verb{latulatu}{}{[laˈtulaˈtu]}{(adj.)}{1}{}{Espevitado.}{}{}
\verb{latulatu}{}{[laˈtulaˈtu]}{(adj.)}{1}{}{Irrequieto.}{}{}
\verb{latulatu}{}{[laˈtulaˈtu]}{(adj.)}{1}{}{Traquinas.}{}{}
\verb{latulatu}{}{[laˈtulaˈtu]}{(adj.)}{1}{}{Vivaz.}{}{}
\verb{latwela}{}{[laˈtwɛla]}{(n.)}{1}{}{Ratoeira.}{}{}
\verb{lavla}{}{[laˈvla]}{(v.)}{1}{}{Aguçar.}{}{}
\verb{lavla}{}{[laˈvla]}{(v.)}{2}{}{Lavrar.}{}{}
\verb{laxpoxta}{}{[laʃˈpᴐʃta]}{(n.)}{1}{}{Resposta.}{Cf. \textbf{lexpoxta}.}{}{}
\verb{laxtiku}{}{[ˈlaʃtiku]}{(n.)}{1}{}{Elástico.}{}{}
\verb{laxtiku}{}{[ˈlaʃtiku]}{(n.)}{2}{}{Estilingue.}{}{}
\verb{laxtu}{}{[ˈlaʃtu]}{(n.)}{1}{}{Lastro.}{}{}
\verb{laza}{}{[ˈlaza]}{(n.)}{1}{}{Maldição.}{}{}%
\verb{laza}{}{[ˈlaza]}{(n.)}{2}{}{Reza.}{}{}
\verb{laza}{}{[laˈza]}{(v.)}{1}{}{Amaldiçoar.}{}{}
\verb{laza}{}{[laˈza]}{(v.)}{2}{}{Arrasar.}{}{}
\verb{laza}{}{[laˈza]}{(v.)}{3}{}{Ficar arrasado.}{}{}
\verb{laza}{}{[laˈza]}{(v.)}{4}{}{Orar.}{}{}
\verb{laza}{}{[laˈza]}{(v.)}{5}{}{Rezar.}{}{}
\verb{lazadu}{}{[laˈzadu]}{(adj.)}{1}{}{Amaldiçoado.}{}{}
\verb{lazadu}{}{[laˈzadu]}{(adj.)}{2}{}{Arrasado.}{}{}
\verb{lazon}{}{[laˈzõ]}{(n.)}{1}{}{Razão.}{}{}
\verb{lazu}{}{[ˈlazu]}{(adj.)}{1}{}{Raso.}{}{}
\verb{lê}{}{[ˈle]}{(v.)}{1}{}{Ler.}{}{}
\verb{lebenta}{}{[lɛb\~ɛˈta]}{(v.)}{1}{}{Arrebentar.}{}{}%
\verb{lebenta}{}{[lɛb\~ɛˈta]}{(v.)}{2}{}{Irromper.}{}{}
\verb{lebenta}{}{[lɛb\~ɛˈta]}{(v.)}{3}{}{Rebentar.}{}{}%
\verb{lêbilêbi}{}{[ˈlebileˈbi]}{(adj.)}{1}{}{Impertinente.}{}{}
\verb{lêbilêbi}{}{[ˈlebileˈbi]}{(adj.)}{2}{}{Traquinas.}{}{}
\verb{leda}{}{[lɛˈda]}{(n.)}{1}{}{Herança.}{}{}
\verb{leda}{}{[lɛˈda]}{(v.)}{1}{}{Herdar.}{}{}
\verb{lêdê}{}{[ˈlede]}{(n.)}{1}{}{Rede.}{}{}
\verb{lêdê}{}{[leˈde]}{(v.)}{1}{}{Acender.}{}{}
\verb{lêdê}{}{[leˈde]}{(v.)}{2}{}{Arder.}{}{}%
\verb{lêdêlu}{}{[leˈdelu]}{(n.)}{1}{}{Herdeiro.}{}{}
\verb{lêdê pitxipitxi}{}{[leˈde piˈtʃipiˈtʃi]}{(expr.)}{1}{}{Fogo que não arde bem.}{}{}
\verb{lêdê-pixka}{}{[ˈlede ˈpiʃka]}{(n.)}{1}{}{Rede de pesca.}{}{}
\verb{lêdê tatata}{}{[leˈde tataˈta]}{(expr. )}{1}{}{Brilhar do sol com intensidade.}{}{}
\verb{lêdê zazaza}{}{[leˈde zazaˈza]}{(expr.)}{1}{}{Ardor de uma ferida.}{}{}\verb{lêdê zazaza}{}{[leˈde zazaˈza]}{(expr.)}{2}{}{Ardor provocado por pimenta.}{}{}
\verb{lêdidu}{}{[leˈdidu]}{(adj.)}{1}{}{Aceso.}{}{}
\verb{lêdidu}{}{[leˈdidu]}{(adj.)}{2}{}{Ardido.}{}{}
\verb{lêdidu ng\~e\~e\~e}{}{[leˈdidu
ˈŋg\~ɛ\~ɛ\~ɛ]}{(expr.)}{1}{}{Acesíssimo.}{}{}
\verb{lêdja}{}{[ˈledʒa]}{(n.)}{1}{}{Corrida de cavalos.}{}{}
\verb{ledu}{}{[ˈlɛdu]}{(n.)}{2}{}{Atrito entre pessoas.}{}{}
\verb{ledu}{}{[ˈlɛdu]}{(n.)}{2}{}{Provocação.}{}{}
\verb{ledu}{}{[ˈlɛdu]}{(n.)}{1}{}{Soberbia.}{}{}%
\verb{lêfatxi}{}{[leˈfatʃi]}{(n.)}{1}{}{Alfaiate.}{Cf.
\textbf{lifyatxi}.}{}{}
\verb{lefegu}{}{[lɛˈfɛgu]}{(n.)}{1}{}{Raspão.}{Cf. \textbf{leflegu}.}{}
\verb{lefeta}{}{[lɛfɛˈta]}{(v.)}{1}{}{Afetar.}{}{}
\verb{lefeta}{}{[lɛfɛˈta]}{(v.)}{2}{}{Apoquentar.}{}{}
\verb{lefeta}{}{[lɛfɛˈta]}{(v.)}{3}{}{Incomodar.}{}{}
\verb{lefeta}{}{[lɛfɛˈta]}{(v.)}{4}{}{Prejudicar.}{}{}
\verb{lefeta}{}{[lɛfɛˈta]}{(v.)}{5}{}{Preocupar.}{}{}
\verb{leflegu}{}{[lɛˈflɛgu]}{(n.)}{1}{}{Raspão.}{}{}
\verb{lêflexka}{}{[leflɛʃˈka]}{(v.)}{1}{}{Refrescar.}{}{}%
\verb{lefoga}{}{[lɛfᴐˈga]}{(v.)}{1}{}{Refogar.}{}{}
\verb{lefogadu}{}{[lɛfᴐˈgadu]}{(adj.)}{1}{}{Refogado.}{}{}
\verb{lefogadu}{}{[lɛfᴐˈgadu]}{(n.)}{1}{}{Refogado.}{}{}
\verb{lega}{}{[ˈlɛga]}{(v.)}{1}{}{Abandonar.}{}{}%
\verb{lega}{}{[ˈlɛga]}{(v.)}{2}{}{Cantar.}{}{}
\verb{lega}{}{[ˈlɛga]}{(v.)}{3}{}{Largar.}{}{}%
\verb{lega}{}{[ˈlɛga]}{(v.)}{4}{}{Narrar.}{}{}%
\verb{lega}{}{[ˈlɛga]}{(v.)}{5}{}{Soltar.}{}{}%
\verb{legadu}{}{[lɛˈgadu]}{(adj.)}{1}{}{Largado.}{}{}
\verb{legadu}{}{[lɛˈgadu]}{(adj.)}{2}{}{Solto.}{}{}
\verb{legadu-bofi}{}{[lɛˈgadu ˈbᴐfi]}{(adv.)}{1}{}{Descontraidamente.}{}{}
\verb{legadu-bofi}{}{[lɛˈgadu ˈbᴐfi]}{(adv.)}{2}{}{Despreocupadamente.}{}{}
\verb{legadu-bofi}{}{[lɛˈgadu ˈbᴐfi]}{(adv.)}{3}{}{Relaxadamente.}{}{}
\verb{legela}{}{[lɛgɛˈla]}{(n.)}{1}{}{Abastança.}{}{}
\verb{legela}{}{[lɛgɛˈla]}{(n.)}{2}{}{Boa-vida.}{}{}
\verb{legela}{}{[lɛgɛˈla]}{(n.)}{3}{}{Conforto.}{}{}
\verb{legela}{}{[lɛgɛˈla]}{(n.)}{4}{}{Fartura.}{}{}
\verb{legela}{}{[lɛgɛˈla]}{(n.)}{5}{}{Gozo.}{}{}
\verb{legela}{}{[lɛgɛˈla]}{(n.)}{6}{}{Opulência.}{}{}
\verb{legela}{}{[lɛgɛˈla]}{(n.)}{7}{}{Regalo.}{}{}
\verb{legela}{}{[lɛgɛˈla]}{(n.)}{8}{}{Usufruto.}{}{}
\verb{legela}{}{[lɛgɛˈla]}{(v.)}{1}{}{Desfrutar.}{}{}
\verb{legela}{}{[lɛgɛˈla]}{(v.)}{2}{}{Gozar.}{}{}
\verb{legela}{}{[lɛgɛˈla]}{(v.)}{3}{}{Regalar(-se).}{}{}
\verb{legela}{}{[lɛgɛˈla]}{(v.)}{4}{}{Usufruir.}{}{}
\verb{legla}{}{[ˈlɛgla]}{(n.)}{1}{}{Regra.}{}{}%
\verb{legli}{}{[ˈlɛgli]}{(adj.)}{1}{}{Alegre.}{}{}%
\verb{lêglia}{}{[leˈglia]}{(n.)}{1}{}{Alegria.}{Cf. \textbf{alêglia}.}{}
\verb{lêgula}{}{[leguˈla]}{(v.)}{1}{}{Coordenar.}{}{}
\verb{lêgula}{}{[leguˈla]}{(v.)}{2}{}{Partilhar.}{}{}
\verb{lêgula}{}{[leguˈla]}{(v.)}{3}{}{Regular.}{}{}
\verb{lêguladô}{}{[legulaˈdo]}{(n.)}{1}{}{Regulador.}{}{}
\verb{lêguladu}{}{[leguˈladu]}{(adj.)}{1}{}{Medido.}{}{}
\verb{lêguladu}{}{[leguˈladu]}{(adj.)}{2}{}{Regulado.}{}{}
\verb{lêguladu}{}{[leguˈladu]}{(adj.)}{2}{}{Regular.}{}{}
\verb{lêgulamentu}{}{[legulaˈm\~etu]}{(n.)}{1}{}{Regulamento.}{}{}
\verb{lêja}{}{[ˈleʒa]}{(n.)}{1}{}{Corrida de cavalos.}{Cf.
\textbf{lêdja}.}{}{}
\verb{leji}{}{[ˈlɛʒi]}{(n.)}{1}{}{Raiz.}{}{}
\verb{lekadu}{}{[lɛˈkadu]}{(n.)}{1}{}{Recado.}{}{}
\verb{lekeleke}{}{[lɛˈkɛlɛˈkɛ]}{(id.)}{1}{}{Cf. \textbf{fina lekeleke.}}{}{}
\verb{lêklê}{}{[leˈkle]}{(v.)}{1}{}{Admoestar.}{}{}
\verb{lêklê}{}{[leˈkle]}{(v.)}{1}{}{Advertir.}{}{}
\verb{lêklê}{}{[leˈkle]}{(v.)}{2}{}{Repreender.}{}{}%
\verb{lêklêmentu}{}{[lekleˈm\~etu]}{(n.)}{1}{}{Requerimento.}{}{}
\verb{lêklidu}{}{[leˈklidu]}{(adj.)}{1}{}{Advertido.}{}{}
\verb{lêklidu}{}{[leˈklidu]}{(adj.)}{2}{}{Repreendido.}{}{}
\verb{lela}{}{[lɛˈla]}{(v.)}{1}{}{Lixar(-se).}{}{}
\verb{lelatu}{}{[lɛˈlatu]}{(n.)}{1}{}{Relato.}{}{}
\verb{lelatu}{}{[lɛˈlatu]}{(n.)}{2}{}{Relatório.}{}{}
\verb{lele}{}{[ˈlɛlɛ]}{(n.)}{1}{}{Vagina.}{}{}
\verb{lêlê}{}{[leˈle]}{(prep. v.)}{1}{}{Ao longo de.}{}{}
\verb{lêlê}{}{[leˈle]}{(v.)}{1}{}{Acompanhar.}{}{}%
\verb{lêlê}{}{[leˈle]}{(v.)}{2}{}{Seguir.}{}{}
\verb{lelele}{}{[lɛlɛˈlɛ]}{(id.)}{1}{}{Cf. \textbf{lekeleke}.}{}{}
\verb{lema}{}{[lɛˈma]}{(n.)}{1}{}{Armadilha.}{}{}
\verb{lema}{}{[lɛˈma]}{(v.)}{1}{}{Armar.}{}{}
\verb{lema}{}{[lɛˈma]}{(v.)}{2}{}{Começar.}{}{}
\verb{lema}{}{[lɛˈma]}{(v.)}{3}{}{Experimentar.}{}{}
\verb{lema}{}{[lɛˈma]}{(v.)}{4}{}{Fazer uma armadilha.}{}{}%
\verb{lema}{}{[lɛˈma]}{(v.)}{5}{}{Nublar.}{}{}
\verb{lema}{}{[lɛˈma]}{(v.)}{6}{}{Provar.}{}{}
\verb{lema}{}{[lɛˈma]}{(v.)}{7}{}{Remar.}{}{}
\verb{lemadô}{}{[lɛmaˈdo]}{(n.)}{1}{}{Armador.}{}{}
\verb{lemadô}{}{[lɛmaˈdo]}{(n.)}{2}{}{Remador.}{}{}
\verb{lemadu}{}{[lɛˈmadu]}{(adj.)}{1}{}{Armado.}{}{}
\verb{lemadu}{}{[lɛˈmadu]}{(adj.)}{2}{}{Nublado.}{}{}%
\verb{lema kidalê}{}{[lɛˈma kidaˈle]}{(expr.)}{1}{}{Começar a gritar.}{}{}
\verb{lema pê}{}{[lɛˈma ˈpe]}{(expr.)}{1}{}{Ajuntar-se.}{}{}
\verb{lêmatixmu}{}{[lemaˈtiʃmu]}{(n.)}{1}{}{Reumatismo.}{Cf. \textbf{lêmatixmu}}{}
\verb{lêmatizumu}{}{[lemaˈtizumu]}{(n.)}{1}{}{Reumatismo.}{}{}
\verb{lembalemba}{}{[ˈlẽbaˈlẽba]}{(n.)}{1}{}{\textit{Lembalemba}.}{\textbf{\textit{Ficus annobonensis}.}}{}
\verb{lembla}{}{[lẽˈbla]}{(v.)}{1}{}{Lembrar.}{}{}%
\verb{lembla}{}{[lẽˈbla]}{(v.)}{2}{}{Recordar.}{}{}
\verb{lembladu}{}{[lẽˈbladu]}{(adj.)}{1}{}{Lembrado.}{}{}%
\verb{lembladu}{}{[lẽˈbladu]}{(adj.)}{2}{}{Recordado.}{}{}%
\verb{lemblansa}{}{[lẽˈbl\~{\textturna}sa]}{(n.)}{1}{}{Lembrança.}{}{}%
\verb{lemblansa}{}{[lẽˈbl\~{\textturna}sa]}{(n.)}{2}{}{Recordação.}{}{}
\verb{lemeja}{}{[lɛmɛˈʒa]}{(v.)}{1}{}{Melhorar.}{}{}
\verb{lemeja}{}{[lɛmɛˈʒa]}{(v.)}{2}{}{Remediar.}{}{}
\verb{lemi}{}{[ˈlɛmi]}{(n.)}{3}{}{Leme.}{}{}%
\verb{lemi}{}{[ˈlɛmi]}{(n.)}{4}{}{Vagina.}{}{}%
\verb{lemu}{}{[ˈlɛmu]}{(n.)}{1}{}{Remo.}{Cf. \textbf{lemunha}.}{}
\verb{lemunha}{}{[lɛˈm\~uɲa]}{(n.)}{1}{}{Remo.}{}{}
\verb{lenda}{}{[ˈl\~ɛda]}{(n.)}{1}{}{Aluguer.}{}{}
\verb{lenda}{}{[ˈl\~ɛda]}{(n.)}{1}{}{Arrendamento.}{}{}
\verb{lenda}{}{[ˈl\~ɛda]}{(n.)}{1}{}{Renda.}{}{}
\verb{lenda}{}{[l\~ɛˈda]}{(v.)}{1}{}{Alugar.}{}{}
\verb{lenda}{}{[l\~ɛˈda]}{(v.)}{2}{}{Arrendar.}{}{}
\verb{lendadu}{}{[l\~ɛˈdadu]}{(v.)}{1}{}{Arrendado.}{}{}
\verb{lendê}{}{[l\~eˈde]}{(v.)}{1}{}{Dar um mau jeito.}{}{}
\verb{lendê}{}{[l\~eˈde]}{(v.)}{2}{}{Inutilizar.}{}{}
\verb{lendê}{}{[l\~eˈde]}{(v.)}{1}{}{Esforçar-se.}{}{}
\verb{lendê}{}{[l\~eˈde]}{(v.)}{3}{}{Render.}{}{}
\verb{lendêlu}{}{[l\~eˈdelu]}{(n.)}{1}{}{Inquilino.}{}{}
\verb{lendêlu}{}{[l\~eˈdelu]}{(n.)}{2}{}{Rendeiro.}{}{}
\verb{lendêlu}{}{[l\~eˈdelu]}{(n.)}{3}{}{Senhorio.}{}{}
\verb{lendidu}{}{[l\~eˈdidu]}{(adj.)}{1}{}{Rendido.}{}{}
\verb{lendondo}{}{[l\~ɛˈd\~ɔdɔ]}{(adj.)}{1}{}{Redondo.}{}{}
\verb{lengelenge}{}{[ˈl\~ɛgɛˈl\~ɛgɛ]}{(adj.)}{1}{}{Dependurado.}{}{}
\verb{lengelenge}{}{[ˈl\~ɛgɛˈl\~ɛgɛ]}{(adj.)}{2}{}{Pendurado.}{}{}%
\verb{lengelenge}{}{[ˈl\~ɛgɛˈl\~ɛgɛ]}{(adj.)}{3}{}{Suspenso.}{}{}%
\verb{lengelenge}{}{[ˈl\~ɛgɛˈl\~ɛgɛ]}{(adv.)}{2}{}{Assim-assim.}{}{}
\verb{lengelenge}{}{[ˈl\~ɛgɛˈl\~ɛgɛ]}{(adv.)}{1}{}{Mais ou menos.}{}{}%
\verb{lenha}{}{[ˈl\~ɛɲa]}{(n.)}{1}{}{Rainha.}{}{}
\verb{lenha}{}{[ˈl\~ɛɲa]}{(n.)}{1}{}{Rainha.}{\textit{\textbf{Uranoscopus
polli}}.}{}
\verb{lensu}{}{[ˈlẽsu]}{(n.)}{1}{}{Lenço.}{}{}
\verb{lentla}{}{[l\~ɛˈtla]}{(prep. n.)}{1}{}{Para dentro.}{\textbf{Mosu kôlê
lentla ke}. \textit{O rapaz correu para dentro de casa}.}{}
\verb{lentla}{}{[l\~ɛˈtla]}{(v.)}{1}{}{Entrar.}{}{}
\verb{lentxi}{}{[ˈlẽtʃi]}{(adv.)}{1}{}{Assim-assim.}{}{}{}
\verb{lentxi}{}{[ˈlẽtʃi]}{(adv.)}{2}{}{Mais ou menos.}{Cf. \textbf{mê-lentxi.}}{}{}
\verb{lenu}{}{[ˈlenu]}{(n.)}{1}{}{Reino.}{}{}
\verb{lenyon}{}{[leˈnjõ]}{(n.)}{1}{}{Reunião.}{}{}
\verb{lepala}{}{[lɛpaˈla]}{(v.)}{1}{}{Aperceber-se.}{}{}
\verb{lepala}{}{[lɛpaˈla]}{(v.)}{2}{}{Calcular.}{}{}
\verb{lepala}{}{[lɛpaˈla]}{(v.)}{3}{}{Lembrar.}{}{}
\verb{lepala}{}{[lɛpaˈla]}{(v.)}{4}{}{Reparar.}{}{}
\verb{lepalu}{}{[lɛˈpalu]}{(n.)}{1}{}{Atenção.}{}{}%
\verb{lepalu}{}{[lɛˈpalu]}{(n.)}{2}{}{Cuidado.}{}{}
\verb{lepalu}{}{[lɛˈpalu]}{(n.)}{3}{}{Juízo.}{}{}
\verb{lepalu}{}{[lɛˈpalu]}{(n.)}{4}{}{Reparo.}{}{}%
\verb{lêpendê}{}{[lepẽˈde]}{(v.)}{1}{}{Arrepender.}{}{}
\verb{lêpendidu}{}{[lepẽˈdidu]}{(adj.)}{1}{}{Arrependido.}{}{}
\verb{lepika}{}{[lɛpiˈka]}{(v.)}{1}{}{Repicar.}{}{}
\verb{lepla}{}{[ˈlɛpla]}{(n.)}{1}{}{Lepra.}{}{}
\verb{leplika}{}{[ˈlɛplika]}{(n.)}{1}{}{Réplica.}{}{}
\verb{lêpublika}{}{[leˈpublika]}{(n.)}{1}{}{República.}{}{}
\verb{lêsêbê}{}{[leseˈbe]}{(v.)}{1}{}{Receber.}{}{}
\verb{lêsenxadu}{}{[lesẽˈʃadu]}{(adj.)}{1}{}{Desleixado.}{}{}%
\verb{lêsenxadu}{}{[lesẽˈʃadu]}{(adj.)}{2}{}{Imprudente.}{}{}
\verb{lêsenxadu}{}{[lesẽˈʃadu]}{(adj.)}{4}{}{Jovem.}{Normalmente que se relaciona com mulher mais velha.}{}
\verb{lêsenxadu}{}{[lesẽˈʃadu]}{(adj.)}{4}{}{Ousado.}{}{}%
\verb{lêsêta}{}{[leˈseta]}{(n.)}{1}{}{Receita.}{}{}
\verb{letayadu}{}{[lɛtaˈjadu]}{(adj.)}{1}{}{Detalhado.}{}{}
\verb{lêtê}{}{[ˈlete]}{(n.)}{1}{}{Leite.}{}{}
\verb{lêtê}{}{[ˈlete]}{(n.)}{2}{}{Seiva.}{}{}
\verb{letla}{}{[ˈlɛtla]}{(n.)}{1}{}{Letra.}{}{}
\verb{lêtlatu}{}{[leˈtlatu]}{(n.)}{2}{}{Fotografia.}{}{}
\verb{lêtlatu}{}{[leˈtlatu]}{(n.)}{1}{}{Retrato.}{}{}%
\verb{letleti}{}{[lɛˈtlɛti]}{(n.)}{1}{}{Banheiro.}{}{}%
\verb{letleti}{}{[lɛˈtlɛti]}{(n.)}{2}{}{Casa de banho.}{}{}
\verb{letleti}{}{[lɛˈtlɛti]}{(n.)}{3}{}{Retrete.}{}{}%
\verb{lêtlina}{}{[leˈtlina]}{(n.)}{1}{Eritrina.}{\textit{\textbf{Eryhtrina
poeppigiana}}.}{}{}
\verb{lêton}{}{[leˈtõ]}{(n.)}{1}{}{Leitão.}{}{}
\verb{lêtu}{}{[ˈletu]}{(n.)}{1}{}{Gaiola.}{}{}
\verb{leva}{}{[ˈlɛva]}{(n.)}{2}{}{Fúria.}{}{}
\verb{leva}{}{[ˈlɛva]}{(n.)}{1}{}{Raiva.}{}{}
\verb{leva}{}{[lɛˈva]}{(v.)}{1}{}{Ultrapassar.}{}{}
\verb{levada}{}{[lɛˈvada]}{(n.)}{1}{}{Dreno para escoar água.}{}{}
\verb{levada}{}{[lɛˈvada]}{(n.)}{2}{}{Vala.}{}{}
\verb{leve}{}{[ˈlɛvɛ]}{(adj.)}{1}{}{Leve.}{}{}
\verb{leve}{}{[ˈlɛvɛ]}{(n.)}{1}{}{Urtiga.}{\textbf{\textit{Fleurya aestuam}.}}{}
\verb{lêvê}{}{[leˈve]}{(v.)}{1}{}{Vazar.}{}{}
\verb{lêvê}{}{[leˈve]}{(v.)}{1}{}{Verter.}{}{}
\verb{leveleve}{}{[ˈlɛvɛˈlɛvɛ]}{(adv.)}{2}{}{Assim-assim.}{}{}%
\verb{leveleve}{}{[ˈlɛvɛˈlɛvɛ]}{(adv.)}{3}{}{Devagar.}{}{}
\verb{leveleve}{}{[ˈlɛvɛˈlɛvɛ]}{(adv.)}{1}{}{Mais ou menos.}{}{}
\verb{levesadu}{}{[lɛvɛˈsadu]}{(adj.)}{1}{}{Arrevesado.}{}
\verb{levesadu}{}{[lɛvɛˈsadu]}{(adj.)}{1}{}{Torto.}{}
\verb{levoga}{}{[lɛvᴐˈga]}{(v.)}{1}{}{Fazer coro.}{}{}
\verb{levoga}{}{[lɛvᴐˈga]}{(v.)}{2}{}{Responder.}{}{}
\verb{levoga}{}{[lɛvᴐˈga]}{(v.)}{3}{}{Revogar.}{}{}
%\verb{lêxenxadu}{}{[leʃẽˈʃadu]}{(adj.)}{1}{}{\textbf{lêsenxadu}.}{}{}%
\verb{lêxenxadu}{}{[leʃẽˈʃadu]}{(adj.)}{1}{}{Desleixado.}{}{}%
\verb{lêxenxadu}{}{[leʃẽˈʃadu]}{(adj.)}{2}{}{Imprudente.}{}{}%
\verb{lêxenxadu}{}{[leʃẽˈʃadu]}{(adj.)}{4}{}{Jovem.}{Normalmente que se relaciona com mulher mais velha.}{}{}%
\verb{lêxenxadu}{}{[leʃẽˈʃadu]}{(adj.)}{4}{}{Ousado.}{Cf. \textbf{lêsenxadu}.}{}%
\verb{lêxibu}{}{[leˈʃibu]}{(n.)}{1}{}{Recibo.}{}{}
\verb{lexpoxta}{}{[lɛʃˈpᴐʃta]}{(n.)}{1}{}{Resposta.}{}{}
\verb{lexti}{}{[ˈlɛʃti]}{(n.)}{1}{}{Leste.}{}{}
\verb{lew}{}{[ˈlɛw]}{(adj.)}{1}{}{Favorável (referindo-se ao quadrante lunar ou ao signo).}{}{}
\verb{lêza}{}{[leˈza]}{(v.)}{1}{}{Aleijar.}{}{}
\verb{lêzadu}{}{[leˈzadu]}{(adj.)}{1}{}{Aleijado.}{}{}
\verb{lêzêdô}{}{[lezeˈdo]}{(n.)}{1}{}{Regedor.}{}{}
\verb{li}{}{[ˈli]}{(v.)}{1}{}{Rir.}{}{}
\verb{li}{}{[ˈli]}{(v.)}{2}{}{Sorrir.}{}{}
\verb{liba}{}{[ˈliba]}{(n.)}{1}{}{Zona distante.}{}{}
\verb{liba}{}{[ˈliba]}{(prep. n.)}{1}{}{Sobre.}{}{}%
\verb{liba}{}{[ˈliba]}{(prep. n.)}{2}{}{Em cima de.}{}{}%
\verb{liba}{}{[ˈliba]}{(prep. n.)}{3}{}{Por cima de.}{}{}
\verb{liba}{}{[liˈba]}{(v.)}{1}{}{Adiar.}{}{}%
\verb{liba-d'atali}{}{[ˈliba daˈtali]}{(n.)}{1}{}{Púlpito.}{}{}
\verb{liba-d'ôkê}{}{[ˈliba doˈke]}{(n.)}{1}{}{Cimo.}{}{}
\verb{liba-d'uku}{}{[ˈliba ˈduku]}{(n.)}{1}{}{Lixeira.}{}{}
\verb{liba-ke}{}{[ˈliba ˈkɛ]}{(n.)}{1}{}{Telhado.}{}{}
\verb{liba-ke}{}{[ˈliba ˈkɛ]}{(n.)}{2}{}{Teto.}{}{}%
\verb{libaliba}{}{[ˈlibaˈliba]}{(id.)}{1}{Cf. \textbf{flôgô libaliba}.}{}{}
\verb{liba-pentxi}{}{[ˈliba ˈp\~ɛtʃi]}{(n.)}{1}{}{Púbis.}{}{}
\verb{libason}{}{[libaˈs\~ɔ]}{(n.)}{1}{}{Libação.}{}{}
\verb{liba-wê}{}{[ˈliba ˈwe]}{(n.)}{2}{}{Pálpebra.}{}{}%
\verb{libedadji}{}{[libɛˈdadʒi]}{(n.)}{1}{}{Liberdade.}{}{}
\verb{libêlinha}{}{[libeˈl\~iɲa]}{(n.)}{1}{}{Libélula.}{}{}
\verb{libita}{}{[libiˈta]}{(id.)}{1}{}{Cf. \textbf{fumadu libita}.}{}{}
\verb{libita}{}{[libiˈta]}{(id.)}{1}{}{Cf. \textbf{xa libita}.}{}{}
\verb{libô}{}{[ˈlibo]}{(n.)}{1}{}{\textit{Libô}.}{\textbf{\textit{Vernonia
amygdalina}.}}{}
\verb{libô-d'awa}{}{[ˈlibo
ˈdawa]}{(n.)}{1}{}{\textit{Libô-d'awa}.}{\textbf{\textit{Struchium
sparganaphorum}.}}{}
\verb{libô-ke}{}{[ˈlibo
ˈkɛ]}{(n.)}{1}{}{\textit{Libô-ke}.}{\textbf{\textit{Vernonia amygdalina.}}}{}
\verb{libô-mukambu}{}{[ˈlibo
muˈk\~{\textturna}bu]}{(n.)}{1}{}{\textit{Libô}.}{Cf. \textbf{libô}.}{}{}
\verb{libô-tela}{}{[ˈlibo
ˈtɛla]}{(n.)}{1}{}{\textit{Libô-tela}.}{\textbf{\textit{Vernonia
amygdalina.}}}{}
\verb{lida}{}{[ˈlida]}{(n.)}{1}{}{Afazeres domésticos.}{}{}
\verb{lida}{}{[ˈlida]}{(n.)}{2}{}{Lida.}{}{}
\verb{lifa}{}{[ˈlifa]}{(n.)}{1}{}{Rifa.}{}{}
\verb{lifila}{}{[lifiˈla]}{(v.)}{1}{}{Refilar.}{}{}
\verb{lifilon}{}{[lifiˈlõ]}{(adj.)}{1}{}{Refilão.}{}{}
\verb{lifyatxi}{}{[liˈfjatʃi]}{(n.)}{1}{}{Alfaiate.}{}{}
\verb{liga}{}{[liˈga]}{(v.)}{1}{}{Ligar.}{}{}
\verb{ligadula}{}{[ligaˈdula]}{(n.)}{1}{}{Ligadura.}{}{}
\verb{ligi}{}{[liˈgi]}{(v.)}{1}{}{Erguer.}{}{}
\verb{ligi}{}{[liˈgi]}{(v.)}{1}{}{Levantar.}{}{}
\verb{ligi}{}{[liˈgi]}{(v.)}{1}{}{Suspender.}{}{}
\verb{ligida}{}{[ligiˈda]}{(n.)}{1}{}{Alguidar.}{}{}
\verb{lijibitadu}{}{[liʒibiˈtadu]}{(n.)}{1}{}{Febril.}{}{}
\verb{lijibitadu}{}{[liʒibiˈtadu]}{(n.)}{1}{}{Gripado.}{}{}
\verb{lijimentu}{}{[liʒiˈmẽtu]}{(n.)}{1}{}{Dieta.}{}{}
\verb{lijimentu}{}{[liʒiˈmẽtu]}{(n.)}{2}{}{Regime alimentar.}{}{}
\verb{lijimi}{}{[liˈʒimi]}{(n.)}{1}{}{Regime.}{}{}
\verb{lijinga}{}{[liʒ\~iˈga]}{(v.)}{1}{}{Refilar.}{}{}
%\verb{lijixtu}{}{[liˈʒiʃtu]}{(n.)}{1}{}{Registo.}{}{}
\verb{lijixtu}{}{[liˈʒiʃtu]}{(n.)}{1}{}{Registro.}{}{}
%\verb{lijixtu-sivil}{}{[liˈʒiʃtu siˈvil]}{(n.)}{1}{}{Registo civil.}{}{}
\verb{lijixtu-sivil}{}{[liˈʒiʃtu siˈvil]}{(n.)}{1}{}{Registro civil.}{}{}
\verb{lijon}{}{[liˈʒõ]}{(n.)}{1}{}{Religião.}{}{}
\verb{likatxi}{}{[liˈkatʃi]}{(n.)}{1}{}{Alicate.}{}{}
\verb{likêza}{}{[liˈkeza]}{(n.)}{1}{}{Riqueza.}{}{}
\verb{likidu}{}{[ˈlikidu]}{(n.)}{2}{}{Líquido.}{}{}
\verb{likli}{}{[liˈkli]}{(n.)}{1}{}{Alecrim.}{}{}
\verb{liku}{}{[ˈliku]}{(adj.)}{1}{}{Rico.}{}{}
\verb{liku sonosono}{}{[ˈliku sɔˈnɔsɔˈnɔ]}{(expr.)}{1}{}{Riquíssimo.}{}{}
\verb{lima}{}{[ˈlima]}{(n.)}{2}{}{Lima.}{}{}
\verb{lima}{}{[ˈlima]}{(n.)}{1}{}{Limo.}{Cf. \textbf{limi}.}{}{}
\verb{lima}{}{[liˈma]}{(v.)}{1}{}{Limar.}{}{}
\verb{limadu}{}{[liˈmadu]}{(adj.)}{1}{}{Limado.}{}{}
\verb{limba}{}{[l\~iˈba]}{(v.)}{1}{}{Escapar.}{}{}
\verb{limba}{}{[l\~iˈba]}{(v.)}{2}{}{Livrar.}{}{}
\verb{limi}{}{[ˈlimi]}{(n.)}{1}{}{Limo.}{}{}
\verb{limixu}{}{[liˈmiʃu]}{(adj.)}{1}{}{Introvertido.}{}{}
\verb{limon}{}{[liˈm\~ᴐ]}{(n.)}{2}{}{Lima.}{\textbf{\textit{Citrus
aurantifolia}}.}{}
\verb{limon}{}{[liˈm\~ᴐ]}{(n.)}{1}{}{Limão.}{}{}%
\verb{limon-blabu}{}{[liˈm\~ᴐ
ˈblabu]}{(n.)}{1}{}{Limão.}{\textbf{\textit{Citrus limon}}.}{}
\verb{limon-d'ôbô}{}{[liˈm\~ᴐ
doˈbo]}{(n.)}{1}{}{Limão.}{\textbf{\textit{Citrus limon}}.}{}
\verb{limon-ple}{}{[liˈm\~ᴐ
ˈplɛ]}{(n.)}{1}{}{Ameixoeira-da-baía.}{\textbf{\textit{Ximenia
americana}}.}{}
\verb{limon-ple}{}{[liˈm\~ᴐ
ˈplɛ]}{(n.)}{2}{}{Limão-da-praia.}{\textbf{\textit{Ximenia americana}}.}{}
\verb{limpa}{}{[lĩˈpa]}{(v.)}{1}{}{Limpar.}{}{}
\verb{limpêza}{}{[lĩˈpeza]}{(n.)}{1}{}{Limpeza.}{}{}
\verb{limpu}{}{[ˈlĩpu]}{(adj.)}{1}{}{Limpo.}{}{}
\verb{limpu fyefyefye}{}{[ˈlĩpu fjɛfjɛˈfjɛ]}{(expr.)}{1}{}{Limpíssimo.}{}{}
\verb{limpu pyenepyene}{}{[ˈlĩpu
pjɛˈnɛpjɛˈnɛ]}{(expr.)}{1}{}{Limpíssimo.}{}{}
\verb{limpu pyepyepye}{}{[ˈlĩpu pjɛpjɛˈpjɛ]}{(expr.)}{1}{}{Limpíssimo.}{}{}
%\verb{linda-floli}{}{[ˈl\~ida ˈflɔli]}{(adj.)}{1}{}{Lindíssima.}{Cf. \textbf{glavi linda-floli}.}{}
\verb{linga}{}{[lĩˈga]}{(v.)}{1}{}{Dependurar.}{}{}
\verb{linga}{}{[lĩˈga]}{(v.)}{2}{}{Içar.}{}{}
\verb{linga}{}{[lĩˈga]}{(v.)}{3}{}{Suspender.}{}{}
\verb{lingadu}{}{[lĩˈgadu]}{(adj.)}{1}{}{Dependurado.}{}{}%
\verb{lingadu}{}{[lĩˈgadu]}{(adj.)}{2}{}{Içado.}{}{}%
\verb{lingadu}{}{[lĩˈgadu]}{(adj.)}{3}{}{Suspenso.}{}{}
\verb{ling'ie}{}{[l\~iˈgiɛ]}{(n.)}{1}{}{Língua da Ilha do Príncipe.}{Cf. \textbf{lung'ie}.}{}{}
\verb{lingw'ie}{}{[l\~iˈgwiɛ]}{(n.)}{1}{}{Língua da Ilha do Príncipe.}{Cf. \textbf{lung'ie}.}{}{}
\verb{linhava}{}{[lĩɲaˈva]}{(v.)}{1}{}{Alinhavar.}{}{}
\verb{lin-kadela}{}{[ˈlĩ kaˈdɛla]}{(n.)}{1}{}{Cóccix.}{}{}
\verb{lin-tlaxi}{}{[ˈlĩ ˈtlaʃi]}{(n.)}{1}{}{Espinha dorsal.}{}{}%
\verb{lipa}{}{[ˈlipa]}{(n.)}{1}{}{Ripa.}{}{}
\verb{lisensa}{}{[liˈsẽsa]}{(n.)}{1}{}{Autorização.}{}{}
\verb{lisensa}{}{[liˈsẽsa]}{(n.)}{1}{}{Licença.}{}{}
\verb{lison}{}{[liˈsõ]}{(n.)}{1}{}{Lição.}{}{}
\verb{litlu}{}{[ˈlitlu]}{(n.)}{1}{}{Litro.}{}{}
\verb{livla}{}{[liˈvla]}{(v.)}{1}{}{Aliviar.}{}{}
\verb{livla}{}{[liˈvla]}{(v.)}{2}{}{Livrar.}{}{}
\verb{livli}{}{[ˈlivli]}{(adj.)}{1}{}{Livre.}{}{}
\verb{livlu}{}{[ˈlivlu]}{(n.)}{1}{}{Livro.}{}{}
\verb{livlu-nglandji}{}{[ˈlivlu ˈŋgl\~{\textturna}dʒi]}{(n.)}{1}{}{Dicionário.}{}{}
\verb{lixbitadu}{}{[liʃbiˈtadu]}{(n.)}{1}{}{Constipação.}{}{}%
\verb{lixbitadu}{}{[liʃbiˈtadu]}{(n.)}{2}{}{Febre.}{}{}
\verb{lixbitadu}{}{[liʃbiˈtadu]}{(n.)}{3}{}{Gripe.}{}{}
\verb{lixbitadu}{}{[liʃbiˈtadu]}{(n.)}{4}{}{Resfriado.}{}{}
\verb{lixgwadu}{}{[liʃˈgwadu]}{(n.)}{1}{}{Resguardo.}{}{}%
\verb{lixi}{}{[ˈliʃi]}{(n.)}{1}{}{Nariz.}{}{}
\verb{lixipêtu}{}{[liʃiˈpetu]}{(n.)}{1}{}{Respeito.}{}{}
\verb{lixi-tapa}{}{[ˈliʃi taˈpa]}{(n.)}{1}{}{Constipação nasal.}{}{}
\verb{lixpêta}{}{[liʃpeˈta]}{(v.)}{1}{}{Respeitar.}{}{}
\verb{lixpêtu}{}{[liʃˈpetu]}{(n.)}{1}{}{Respeito.}{Cf.
\textbf{lixipêtu}.}{}{}
\verb{lixtlison}{}{[liʃˈtlisõ]}{(n.)}{1}{}{Restrição.}{}{}
\verb{lizu}{}{[ˈlizu]}{(adj.)}{1}{}{Duro.}{}{}
\verb{lizu}{}{[ˈlizu]}{(adj.)}{2}{}{Rijo.}{}{}
\verb{lizu kankankan}{}{[ˈlizu
k\~{\textturna}k\~{\textturna}ˈk\~{\textturna}]}{(expr.)}{1}{}{Duríssimo.}{}%
\verb{lô}{}{[ˈlo]}{(v.)}{1}{}{Amontoar.}{}{}
\verb{lô}{}{[ˈlo]}{(v.)}{1}{}{Desenvolver(-se).}{}{}
\verb{lôbô}{}{[ˈlobo]}{(n.)}{1}{}{Formiga-vermelha.}{}{}
\verb{lôbô}{}{[ˈlobo]}{(n.)}{1}{}{Formigão.}{}{}
\verb{lôbô}{}{[ˈlobo]}{(n.)}{1}{}{Lobo.}{}{}
\verb{lôbonji}{}{[loˈbõʒi]}{(n.)}{1}{}{Musgo.}{}{}
\verb{loda}{}{[ˈlɔda]}{(n.)}{1}{}{Roda.}{}{}%
\verb{loda}{}{[ˈlɔda]}{(n.)}{2}{}{Volta.}{}{}
\verb{lodja}{}{[lɔˈdʒa]}{(v.)}{1}{}{Cercar.}{}{}{}
\verb{lodja}{}{[lɔˈdʒa]}{(v.)}{1}{}{Rodear.}{Cf. \textbf{loja}.}{}{}
\verb{lôdô}{}{[ˈlodo]}{(n.)}{1}{}{Aglomeração de casas.}{}{}
\verb{lôdô}{}{[ˈlodo]}{(n.)}{1}{}{Charco.}{}{}
\verb{lôdô}{}{[ˈlodo]}{(n.)}{2}{}{Poça.}{}{}
\verb{lôdô}{}{[ˈlodo]}{(n.)}{3}{}{Vila.}{}{}%
\verb{lôdô-d'awa}{}{[ˈlodo ˈdawa]}{(n.)}{1}{}{Poça de água.}{}{}%
\verb{lodoma}{}{[lɔˈdɔma]}{(n.)}{1}{}{Garrafa.}{}{}%
\verb{lodoma}{}{[lɔˈdɔma]}{(n.)}{2}{}{Redoma.}{}{}
\verb{loga}{}{[lɔˈga]}{(v.)}{1}{}{Rogar.}{}{}
\verb{loga plaga}{}{[lɔˈga ˈplaga]}{(expr.)}{1}{}{Praguejar.}{}{}
\verb{loga plaga}{}{[lɔˈga ˈplaga]}{(expr.)}{1}{}{Rogar praga.}{}{}
\verb{lôgô}{}{[ˈlogo]}{(adv.)}{1}{}{Afinal.}{}{}
\verb{lôgô}{}{[ˈlogo]}{(adv.)}{2}{}{Portanto.}{}{}
\verb{logo}{}{[ˈlɔgɔ]}{(adv.)}{1}{}{Depois.}{}{}
\verb{logo}{}{[ˈlɔgɔ]}{(adv.)}{2}{}{Logo.}{}{}
\verb{lôgôzô}{}{[loˈgozo]}{(n.)}{1}{Carrasco.}{Personagem do
\textbf{dansu-kongô} que passa de guardião a usurpador.}{}{}
\verb{lôja}{}{[ˈloʒa]}{(n.)}{1}{}{Rodilha.}{}{}
\verb{loja}{}{[lɔˈʒa]}{(prep. v.)}{1}{}{À volta de.}{\textbf{Nansê nda loja
ke}. \textit{Vocês andaram à volta da casa}.}{}
\verb{loja}{}{[lɔˈʒa]}{(prep. v.)}{2}{}{Ao redor de.}{}{}
\verb{loja}{}{[lɔˈʒa]}{(v.)}{1}{}{Cercar.}{}{}%
\verb{loja}{}{[lɔˈʒa]}{(v.)}{2}{}{Rodear.}{}{}%
\verb{lojadu}{}{[lɔˈʒadu]}{(adj.)}{1}{}{Cercado.}{}{}
\verb{loke}{}{[ˈlɔkɛ]}{(n.)}{1}{}{Olho-de-pombo.}{\textbf{\textit{Abrus
precatorius}.}}{}
\verb{loke}{}{[lɔˈkɛ]}{(v.)}{1}{}{Desaparecer.}{}
\verb{loke}{}{[lɔˈkɛ]}{(v.)}{2}{}{Partir.}{}
\verb{lôklê}{}{[loˈkle]}{(v.)}{1}{}{Arrecadar.}{}{}
\verb{lôklê}{}{[loˈkle]}{(v.)}{2}{}{Guardar.}{}{}%
\verb{lôklê}{}{[loˈkle]}{(v.)}{3}{}{Recolher.}{}{}
\verb{lôklidu}{}{[loˈklidu]}{(adj.)}{1}{}{Arrecadado.}{}{}
\verb{lôklidu}{}{[loˈklidu]}{(adj.)}{2}{}{Guardado.}{}{}%
\verb{lôklidu}{}{[loˈklidu]}{(adj.)}{3}{}{Recolhido.}{}{}
\verb{lola}{}{[ˈlɔla]}{(n.)}{1}{}{Pombo-de-nuca-bronzeada.}{\textbf{\textit{Columba malherbii}.}}{}
\verb{lola}{}{[ˈlɔla]}{(n.)}{2}{}{Rola.}{\textbf{\textit{Columba malherbii}.}}{}
\verb{lola}{}{[lɔˈla]}{(v.)}{1}{}{Desviar.}{}{}
\verb{lola}{}{[lɔˈla]}{(v.)}{2}{}{Rolar.}{}{}
\verb{lolo}{}{[lɔˈlɔ]}{(id.)}{1}{}{Cf. \textbf{kaba lolo}.}{}{}
\verb{lolo}{}{[lɔˈlɔ]}{(n.)}{1}{}{Pênis de criança.}{}{}
\verb{lolo}{}{[lɔˈlɔ]}{(v.)}{1}{}{Lamber.}{}{}
\verb{lolodu}{}{[lɔˈlɔdu]}{(adj.)}{1}{}{Lambido.}{}{}%
\verb{lolodu}{}{[lɔˈlɔdu]}{(adj.)}{2}{}{Sem-vergonha.}{}{}
\verb{lolojêlu}{}{[lɔlɔˈʒelu]}{(n.)}{1}{}{Relojoeiro.}{}{}
\verb{lôlôlô}{}{[loloˈlo]}{(id.)}{1}{}{Cf. \textbf{xa lôlôlô.}}{}{}
\verb{lôlongôma}{}{[lolõˈgoma]}{(n.)}{1}{}{Banana-pão.}{}{}
\verb{lôlongôma}{}{[lolõˈgoma]}{(n.)}{2}{}{Fingidor.}{}{}
\verb{lolozu}{}{[lɔˈlɔzu]}{(n.)}{1}{}{Relógio.}{}{}
\verb{lomba}{}{[lõˈba]}{(v.)}{1}{}{Arrombar.}{}{}
\verb{lomba}{}{[lõˈba]}{(v.)}{1}{}{Rebentar.}{}{}
\verb{lombin}{}{[lõˈb\~i]}{(n.)}{1}{}{Rim.}{}{}
\verb{lombliga}{}{[lõˈbliga]}{(n.)}{1}{}{Lombriga.}{}{}
\verb{lombo}{}{[ˈlõbo]}{(n.)}{1}{}{Bocado.}{}{}
\verb{lombo}{}{[ˈlõbo]}{(n.)}{1}{}{Lombo.}{}{}
\verb{lombo}{}{[ˈlõbo]}{(n.)}{1}{}{Porção.}{}{}
\verb{lomosa}{}{[lɔmɔˈsa]}{(v.)}{1}{}{Almoçar.}{}{}
\verb{lomosu}{}{[lɔˈmɔsu]}{(n.)}{1}{}{Almoço.}{}{}
\verb{lomplanu}{}{[l\~ɔˈplanu]}{(n.)}{1}{}{Aeroplano.}{}{}
\verb{lona}{}{[ˈlɔna]}{(n.)}{1}{}{Lona.}{}{}
\verb{londa}{}{[ˈl\~ɔda]}{(n.)}{1}{}{Ronda.}{}{}
\verb{londa}{}{[l\~ɔˈda]}{(v.)}{1}{}{Rondar.}{}{}
\verb{londji}{}{[ˈlõdʒi]}{(adj.)}{2}{}{Distante.}{}{}
\verb{londji}{}{[ˈlõdʒi]}{(adj.)}{1}{}{Longe.}{}{}%
\verb{londondo}{}{[l\~ɔˈd\~ɔdɔ]}{(adj.)}{1}{}{Redondo.}{}{}
\verb{longô}{}{[ˈlõgo]}{(adj.)}{1}{}{Alto.}{}{}%
\verb{longô}{}{[ˈlõgo]}{(adj.)}{2}{}{Comprido.}{}{}%
\verb{longô}{}{[ˈlõgo]}{(adj.)}{3}{}{Longo.}{}{}
\verb{longô}{}{[ˈlõgo]}{(adv.)}{1}{}{Demorado.}{}{}%
\verb{longô}{}{[ˈlõgo]}{(n.)}{1}{}{Anágua.}{}{}
\verb{longô}{}{[ˈlõgo]}{(n.)}{1}{}{Saiote interior.}{}{}
\verb{longô}{}{[ˈlõgo]}{(n.)}{2}{}{Sardinela.}{\textbf{\textit{Sardinella aurita}.}}{}
\verb{longô}{}{[lõˈgo]}{(v.)}{1}{}{Espreitar.}{}{}
\verb{lonji}{}{[ˈlõʒi]}{(adj.)}{1}{}{Distante.}{}{}{}%
\verb{lonji}{}{[ˈlõʒi]}{(adj.)}{1}{}{Longe.}{Cf. \textbf{londji}.}{}{}
\verb{lonka}{}{[lõˈka]}{(v.)}{2}{}{Ressonar.}{}{}
\verb{lonka}{}{[lõˈka]}{(v.)}{1}{}{Roncar.}{}{}%
\verb{lonkadô}{}{[lõkaˈdo]}{(n.)}{1}{}{Roncador.}{\textbf{\textit{Pomadasys
rogeri}}.}{}
\verb{lôpa}{}{[ˈlopa]}{(n.)}{1}{}{Roupa.}{}{}
\verb{losa}{}{[ˈlɔsa]}{(n.)}{1}{}{Fazenda.}{}{}
\verb{losa}{}{[ˈlɔsa]}{(n.)}{2}{}{Propriedade rural.}{}{}%
\verb{losa}{}{[ˈlɔsa]}{(n.)}{3}{}{Roça.}{}{}
\verb{losa}{}{[lɔˈsa]}{(v.)}{2}{}{Esfregar.}{}{}
\verb{losa}{}{[lɔˈsa]}{(v.)}{1}{}{Roçar.}{}{}
\verb{lôsô}{}{[ˈloso]}{(n.)}{1}{}{Arroz.}{}{}
\verb{lota}{}{[lɔˈta]}{(v.)}{1}{}{Arrotar.}{}{}
\verb{loti}{}{[ˈlɔti]}{(n.)}{1}{}{Lote.}{}{}%
\verb{loti}{}{[ˈlɔti]}{(n.)}{2}{}{Parcela de terreno.}{}{}
\verb{lôtlina}{}{[loˈtlina]}{(n.)}{1}{}{Eritrina.}{Cf. \textbf{lêtlina}.}{}{}\verb{lotu}{}{[ˈlɔtu]}{(n.)}{1}{}{Arroto.}{}{}
\verb{lotxiga}{}{[loˈtʃiga]}{(n.)}{1}{}{Urtiga.}{\textbf{\textit{Urera
mannii}.}}{}
\verb{lova}{}{[lɔˈva]}{(v.)}{1}{}{Louvar.}{}{}
\verb{lovadu}{}{[lɔˈvadu]}{(adj.)}{1}{}{Louvado.}{}{}
\verb{love}{}{[lɔˈvɛ]}{(n.)}{1}{}{Orvalho.}{}{}
\verb{lovlosa}{}{[lɔvlɔˈsa]}{(v.)}{1}{}{Alvoroçar.}{}{}
\verb{lovlosa}{}{[lɔvlɔˈsa]}{(v.)}{2}{}{Remexer.}{}{}
\verb{lovlosa}{}{[lɔvlɔˈsa]}{(v.)}{3}{}{Revirar.}{}{}
\verb{lovlosa}{}{[lɔvlɔˈsa]}{(v.)}{4}{}{Revoltar(-se).}{}{}
\verb{lovlosa}{}{[lɔvlɔˈsa]}{(v.)}{5}{}{Vasculhar.}{}{}
\verb{lôvlôsô}{}{[lovloˈso]}{(v.)}{1}{}{Mexer.}{}{}
\verb{lôvlôsô}{}{[lovloˈso]}{(v.)}{2}{}{Procurar.}{}{}
\verb{loya}{}{[ˈlɔja]}{(n.)}{1}{}{Rolha.}{}{}
\verb{loza}{}{[ˈlɔza]}{(n.)}{1}{}{Rosa.}{}{}
\verb{loza-bilanza}{}{[ˈlɔza biˈl\~{\textturna}za]}{(n.)}{1}{}{Dama-da-noite.}{\textbf{\textit{Mirabilis jalapa}}.}{}
\verb{loze}{}{[lɔˈzɛ]}{(n.)}{1}{}{Rosário.}{}{}
\verb{lôzôvê}{}{[lozoˈve]}{(v.)}{1}{}{Decidir.}{}{}
\verb{lôzôvê}{}{[lozoˈve]}{(v.)}{2}{}{Resolver.}{}{}
\verb{lôzôvidu}{}{[lozoˈvidu]}{(adj.)}{1}{}{Decidido.}{}{}
\verb{lôzôvidu}{}{[lozoˈvidu]}{(adj.)}{1}{}{Resolvido.}{}{}
\verb{lua}{}{[luˈa]}{(v.)}{1}{}{Estar menstruada.}{}{}%
\verb{luadu}{}{[luˈadu]}{(adj.)}{1}{}{Menstruada.}{}{}
\verb{luason}{}{[luaˈs\~o]}{(n.)}{1}{}{Menstruação.}{}{}%
\verb{luba}{}{[luˈba]}{(n.)}{1}{}{\textit{Luba}.}{\textbf{\textit{Parkia oliveri}.}}{}
\verb{lubêla}{}{[luˈbela]}{(n.)}{1}{}{Ribeira.}{}{}
\verb{luda}{}{[ˈluda]}{(n.)}{1}{}{Arruda.}{\textbf{\textit{Ruta chapelensis}}.}{}
\verb{lufu}{}{[ˈlufu]}{(n.)}{1}{}{Grunhido.}{}{}
\verb{lufugôzu}{}{[lufuˈgozu]}{(adv.)}{1}{}{Apressadamente.}{}{}%
\verb{lufugôzu}{}{[lufuˈgozu]}{(adv.)}{2}{}{Depressa.}{}{}
\verb{lufugôzu}{}{[lufuˈgozu]}{(adv.)}{3}{}{Precipitadamente.}{}{}
\verb{luga}{}{[luˈga]}{(v.)}{1}{}{Alugar.}{}{}
\verb{luga}{}{[luˈga]}{(v.)}{1}{}{Arrendar.}{}{}
\verb{luge}{}{[luˈgɛ]}{(n.)}{1}{}{Lugar.}{}{}
\verb{lugulugu}{}{[luˈguluˈgu]}{(adj.)}{1}{}{Gelatinoso.}{}{}
\verb{luji}{}{[luˈʒi]}{(v.)}{1}{}{Brilhar.}{}{}
\verb{luji}{}{[luˈʒi]}{(v.)}{2}{}{Luzir.}{}{}
\verb{lujidu}{}{[luˈʒidu]}{(adj.)}{1}{}{Brilhante.}{}{}
\verb{lujiga}{}{[ˈluʒiga]}{(n.)}{1}{}{Rusga.}{}{}
\verb{luji myêgêmyêgê}{}{[luˈʒi mjeˈgemjeˈge]}{(expr.)}{1}{}{Brilhar
intensamente.}{}{}
\verb{luklu}{}{[ˈluklu]}{(n.)}{1}{}{Lucro.}{}{}
\verb{lula}{}{[ˈlula]}{(n.)}{1}{}{Lula.}{}{}
\verb{lula}{}{[ˈlula]}{(n.)}{1}{}{Parte da flor de bananeira.}{}{}
\verb{lululu}{}{[luluˈlu]}{(id.)}{1}{}{Cf. \textbf{pletu lululu.}}{}{}
\verb{luma}{}{[luˈma]}{(adv.)}{1}{}{Bastante.}{}{}
\verb{luma}{}{[luˈma]}{(adv.)}{2}{}{Muito.}{}{}
\verb{luma}{}{[luˈma]}{(v.)}{1}{}{Arrumar.}{}{}
\verb{lumadu}{}{[luˈmadu]}{(adj.)}{1}{}{Arrumado.}{}{}
\verb{lumadu}{}{[luˈmadu]}{(adv.)}{1}{}{Muito.}{}{}
\verb{lumon}{}{[luˈmõ]}{(n.)}{1}{}{Irmã.}{}{}%
\verb{lumon}{}{[luˈmõ]}{(n.)}{2}{}{Irmão.}{}{}%
\verb{lumon}{}{[luˈmõ]}{(n.)}{2}{}{Parente.}{}{}%
\verb{lumon}{}{[luˈmõ]}{(n.)}{1}{}{Prima.}{}{}
\verb{lumon}{}{[luˈmõ]}{(n.)}{2}{}{Primo.}{}{}
\verb{lumonhon}{}{[lumõˈɲõ]}{(n.)}{2}{}{Reunião.}{Cf. \textbf{lenyon}.}{}{}
\verb{lumu}{}{[ˈlumu]}{(n.)}{1}{}{Rumo.}{}{}
\verb{lumya}{}{[luˈmja]}{(v.)}{1}{}{Citar.}{}{}
\verb{lumya}{}{[luˈmja]}{(v.)}{2}{}{Mencionar.}{}{}
\verb{lumya}{}{[luˈmja]}{(v.)}{3}{}{Nomear.}{}{}%
\verb{lumya}{}{[luˈmja]}{(v.)}{4}{}{Referir.}{}{}
\verb{lundu}{}{[lũˈdu]}{(n.)}{1}{Lundum.}{}{}{}
\verb{lunfa}{}{[l\~uˈfa]}{(v.)}{1}{}{Rufar.}{}{}%
\verb{lung'ie}{}{[lũˈgiɛ]}{(n.)}{1}{}{Lung'ie.}{}{}
\verb{lung'ie}{}{[lũˈgiɛ]}{(n.)}{2}{}{Língua da Ilha do Príncipe.}{}{}
\verb{lung'ie}{}{[lũˈgiɛ]}{(n.)}{3}{}{Principense.}{}{}
\verb{lungwa}{}{[ˈlũgwa]}{(n.)}{1}{}{Idioma.}{}{}
\verb{lungwa}{}{[ˈlũgwa]}{(n.)}{1}{}{Língua.}{}{}
\verb{lungwa-gatu}{}{[ˈlũgwa ˈgatu]}{(n.)}{1}{}{Terreno pequeno.}{}{}
\verb{lungwalaji}{}{[l\~ugwaˈlaʒi]}{(n.)}{1}{}{Fofoqueiro.}{}{}{}
\verb{lungwalaji}{}{[l\~ugwaˈlaʒi]}{(n.)}{1}{}{Linguado.}{Cf. \textbf{lingwalaji}.}{}{}
\verb{lungwalaji}{}{[l\~ugwaˈlaʒi]}{(n.)}{1}{}{Linguarudo.}{}{}{}
\verb{lungwa-vaka}{}{[ˈlũgwa ˈvaka]}{(n.)}{1}{}{Biscoito.}{}{}
\verb{lungwa-vaka}{}{[ˈlũgwa ˈvaka]}{(n.)}{2}{}{Rebento da palmeira.}{}{}
\verb{lungw'ie}{}{[lũˈgwiɛ]}{(n.)}{1}{}{Língua da Ilha do Príncipe.}{Cf. \textbf{lung'ie}.}{}{}
\verb{lupuye}{}{[lupuˈjɛ]}{(n.)}{1}{}{Pé rapado.}{}{}
\verb{lupuye}{}{[lupuˈjɛ]}{(n.)}{1}{}{Vira-lata.}{}{}
\verb{lusua}{}{[luˈsua]}{(n.)}{1}{}{Erva-moira.}{\textbf{\textit{Solanum
nigrum}}.}{}
\verb{lusua}{}{[luˈsua]}{(n.)}{2}{}{Prato típico à base de
\textbf{lusua}.}{}{}
\verb{luta}{}{[ˈluta]}{(n.)}{1}{}{Briga.}{}{}
\verb{luta}{}{[ˈluta]}{(n.)}{2}{}{Luta.}{}{}
\verb{luta}{}{[ˈluta]}{(n.)}{3}{}{Sacrifício.}{}{}
\verb{luta}{}{[luˈta]}{(v.)}{1}{}{Brigar.}{}{}
\verb{luta}{}{[luˈta]}{(v.)}{2}{}{Lutar.}{}{}%
\verb{lutadô}{}{[lutaˈdo]}{(n.)}{1}{}{Lutador.}{}{}
\verb{luta kunda}{}{[ˈluta ˈkũda]}{(expr.)}{1}{}{Engatinhar.}{}{}
\verb{lutu}{}{[ˈlutu]}{(n.)}{1}{}{Luto.}{}{}
\verb{luva}{}{[ˈluva]}{(n.)}{1}{}{Luva.}{}{}
\verb{luvesa}{}{[luˈvɛsa]}{(adv.)}{1}{}{Ao avesso.}{(Usado apenas para
roupas.)}{}%
\verb{luvesa}{}{[luˈvɛsa]}{(adv.)}{2}{}{Ao contrário.}{(Usado apenas para roupas.)}{}
\verb{luvon}{}{[luˈv\~ɔ]}{(n.)}{1}{}{Valentia.}{}{}
\verb{luvon-fasu}{}{[luˈv\~ɔ ˈfasu]}{(n.)}{1}{}{Fanfarronice.}{}{}
\verb{luvuson}{}{[luvuˈs\~ɔ]}{(n.)}{1}{}{Revolução.}{}{}
\verb{luxan}{}{[luˈʃ\~{\textturna}]}{(n.)}{1}{}{Aldeia.}{}{}
\verb{luxan}{}{[luˈʃ\~{\textturna}]}{(n.)}{1}{}{Zona não-urbanizada.}{}{}
\verb{luxu}{}{[ˈluʃu]}{(n.)}{1}{}{Luxo.}{}{}%
\verb{luxu}{}{[ˈluʃu]}{(n.)}{2}{}{Ostentação.}{}{}
\verb{lwa}{}{[ˈlwa]}{(n.)}{1}{}{Rua.}{}{}
\verb{lwalwa}{}{[lwaˈlwa]}{(v.)}{1}{}{Esquivar.}{}{}
\verb{lwalwa}{}{[lwaˈlwa]}{(v.)}{1}{}{Fugir.}{}{}
\verb{lwelwe}{}{[lwɛˈlwɛ]}{(v.)}{2}{}{Aproximar(-se).}{}{}
\verb{lwelwe}{}{[lwɛˈlwɛ]}{(v.)}{1}{}{Dissimular.}{}{}
\verb{lyali}{}{[ljaˈli]}{(n.)}{1}{}{Ouriço.}{\textbf{\textit{Paracentropis
cabrilla}}.}{}
\verb{lyamba}{}{[lj\~{\textturna}ˈba]}{(n.)}{1}{}{Liamba.}{}{}
\verb{lyamba}{}{[lj\~{\textturna}ˈba]}{(n.)}{2}{}{Maconha.}{}{}
\verb{lyamba}{}{[lj\~{\textturna}ˈba]}{(n.)}{3}{}{Marijuana.}{\textbf{\textit{Cannabis
sativa}}.}{}
\verb{lyon}{}{[ˈljõ]}{(n.)}{1}{}{Leão.}{}{}
\end{letra}

\begin{letra}{m}

\verb{m}{}{[ˈm]}{(poss.)}{1}{}{Cf. \textbf{mu}.}{}{}
%\verb{m}{}{[ˈm]}{(poss.)}{1}{}{Meus.}{Cf. \textbf{mu}.}{}{}
%\verb{m}{}{[ˈm]}{(poss.)}{1}{}{Minha.}{Cf. \textbf{mu}.}{}{}
%\verb{m}{}{[ˈm]}{(poss.)}{1}{}{Minhas.}{Cf. \textbf{mu}.}{}{}
\verb{m}{}{[ˈm]}{(pron.)}{1}{}{Cf. \textbf{mu}.}{}{}
\verb{ma}{}{[ˈma]}{(adj.)}{1}{}{Má.}{}{}
\verb{ma}{}{[ˈma]}{(adj.)}{2}{}{Mau.}{}{}
\verb{ma}{}{[ˈma]}{(adv.)}{1}{}{Muito.}{\textbf{Ngê se sa fê ma}. \textit{Essa pessoa é feiíssima}.}{}
\verb{mabôbô}{}{[maboˈbo]}{(n.)}{1}{}{Camarão amarelo.}{}{}
\verb{madadji}{}{[maˈdadʒi]}{(n.)}{1}{}{Maldade.}{}{}
\verb{madêla}{}{[maˈdela]}{(n.)}{1}{}{Madeira.}{}{}
\verb{madison}{}{[madiˈsõ]}{(n.)}{1}{}{Maldição.}{}{}
\verb{madlaxta}{}{[maˈdlaʃta]}{(n.)}{1}{}{Madrasta.}{}{}
\verb{madlê}{}{[ˈmadle]}{(n.)}{1}{}{Freira.}{}{}
\verb{madlê}{}{[ˈmadle]}{(n.)}{2}{}{Madre.}{}{}
\verb{madlê}{}{[ˈmadle]}{(n.)}{3}{}{Útero.}{}{}
\verb{madluga}{}{[madluˈga]}{(v.)}{1}{}{Madrugar.}{Cf. \textbf{mlazuga}.}{}{}\verb{madlugadô}{}{[madluˈgado]}{(n.)}{1}{}{Madrugador.}{Cf.
\textbf{mlazugadô}.}{}{}
\verb{madlugadu}{}{[madluˈgadu]}{(n.)}{1}{}{Madrugada.}{Cf.
\textbf{mlazugadu}.}{}{}
\verb{madô}{}{[maˈdo]}{(n.)}{1}{}{Combatente.}{}{}
\verb{madô}{}{[maˈdo]}{(n.)}{2}{}{Guerreiro.}{}{}
\verb{madô}{}{[maˈdo]}{(n.)}{3}{}{Indomável.}{}{}
\verb{madô}{}{[maˈdo]}{(n.)}{4}{}{Intrépido.}{}{}
\verb{madô}{}{[maˈdo]}{(n.)}{5}{}{Valente.}{}{}
\verb{ma-fala}{}{[ˈma ˈfala]}{(n.)}{1}{}{Insulto.}{}{}
\verb{mafe}{}{[maˈfɛ]}{(n.)}{1}{}{Má-fé.}{}{}
\verb{mafunji}{}{[maˈfũʒi]}{(n.)}{1}{}{Metamorfose da \textbf{klokoto}.}{}{}
\verb{magita}{}{[maˈgita]}{(n.)}{1}{}{Malagueta.}{\textbf{\textit{Capsicum
annuum}}.}{}
\verb{magita-pali}{}{[maˈgita paˈli]}{(n.)}{1}{}{\textit{Magita-pali}.}{Medicamento tradicional para
parturientes.}{}{}
\verb{magita-twatwa}{}{[maˈgita twaˈtwa]}{(n.)}{1}{}{Pessoa malcriada.}{}{}
\verb{magita-twatwa}{}{[maˈgita twaˈtwa]}{(n.)}{1}{}{Malagueta.}{\textbf{\textit{Capsicum frutescens}}.}{}
\verb{magitôyô}{}{[magiˈtojo]}{(n.)}{1}{}{Dongos-do-Congo.}{\textbf{\textit{Aframomum melegueta}}.}{}
\verb{maglita}{}{[maˈglita]}{(n.)}{1}{}{Malagueta.}{Cf. \textbf{magita}.}{}{}
\verb{magula}{}{[ˈmagula]}{(n.)}{1}{}{Amargura.}{}{}
\verb{magula}{}{[ˈmagula]}{(n.)}{2}{}{Mágoa.}{}{}
\verb{magwa}{}{[ˈmagwa]}{(n.)}{1}{}{Mágoa.}{}{}
\verb{magwa}{}{[ˈmagwa]}{(n.)}{1}{}{Ofensa.}{}{}
\verb{magwa}{}{[maˈgwa]}{(v.)}{1}{}{Ofender.}{}{}
\verb{magwa}{}{[maˈgwa]}{(v.)}{1}{}{Magoar.}{}{}
\verb{maji}{}{[ˈmaʒi]}{(conj.)}{1}{}{Mas.}{}{}
\verb{maka}{}{[ˈmaka]}{(n.)}{1}{}{Maca.}{}{}
\verb{makabali}{}{[makabaˈli]}{(n.)}{1}{}{Cambalhota.}{}{}
\verb{makabali}{}{[makabaˈli]}{(n.)}{2}{}{Erva-de-colégio.}{\textbf{\textit{Elephantopus
mollis}}.}{}%
\verb{makabali}{}{[makabaˈli]}{(n.)}{4}{}{\textit{Fya-dentxi}.}{}{}
\verb{makabungu}{}{[makab\~uˈgu]}{(n.)}{1}{}{Preparação de cascas cozidas
lentamente em uma panela de barro e usada pelo massagista para
tratar problemas de ossos, dores musculares etc.}{}{}
\verb{makaku}{}{[maˈkaku]}{(n.)}{1}{}{Doença que tolhe o desenvolvimento de
recém-nascidos, assim chamada porque as crianças nascem muito peludas.}{}{}
\verb{makaku}{}{[maˈkaku]}{(n.)}{1}{}{Macaco.}{}{}
\verb{makalon}{}{[makaˈlõ]}{(n.)}{1}{}{Macarrão.}{}{}
\verb{makamblala}{}{[mak\~{\textturna}blaˈla]}{(n.)}{1}{\textit{Macambrara}.}{\textbf{\textit{Craterispermum
montanum}.}}{}
%\verb{makaya}{}{[makaˈja]}{(n.)}{1}{}{Tabaco.}{\textbf{\textit{Nicotiana tabacum}}.}{}{}
\verb{makêkê}{}{[makeˈke]}{(n.)}{1}{}{Beringela
amarga.}{\textbf{\textit{Solanum macrocarpon}.}}{}
\verb{ma-kloson}{}{[ˈma klᴐˈs\~ᴐ]}{(adj.)}{1}{}{Malvado.}{}{}
\verb{ma-kloson}{}{[ˈma klᴐˈs\~ᴐ]}{(n.)}{1}{}{Malvadez.}{}{}
\verb{makoya}{}{[maˈkᴐja]}{(n.)}{1}{}{Sarna.}{}{}
\verb{makoya}{}{[maˈkᴐja]}{(n.)}{1}{}{Tabaco.}{\textbf{\textit{Nicotiana tabacum}}.}{}{}
\verb{makubungu}{}{[makub\~uˈgu]}{(n.)}{1}{}{Mistura de cascas para massagens.}{}{}
\verb{makuku}{}{[makuˈku]}{(n.)}{1}{}{Fogão tradicional formado por pedras, habitualmente três, dispostas triangularmente e que suportam a panela.}{}{}
%\verb{makulu}{}{[makuˈlu]}{(n.)}{1}{}{Hemorróidas.}{}{}
\verb{makulu}{}{[makuˈlu]}{(n.)}{1}{}{Infecção anal com sangramento.}{}{}
\verb{makundja}{}{[maˈkũdʒa]}{(n.)}{1}{}{\textit{Makundja}.}{\textbf{\textit{Mucuna pruriens}}.}{}
\verb{makunga}{}{[makũˈga]}{(n.)}{1}{}{Alimento cozido à base de farinha de milho, enrolado em folha de bananeira.}{}{}
\verb{makuta}{}{[makuˈta]}{(n.)}{1}{}{Antiga moeda de cobre de cinquenta centavos.}{}{}
\verb{makuta}{}{[makuˈta]}{(n.)}{1}{}{Seiva da palmeira.}{}{}
\verb{mala}{}{[ˈmala]}{(n.)}{1}{}{Caixa.}{}{}
\verb{mala}{}{[ˈmala]}{(n.)}{2}{}{Mala.}{}{}
\verb{mala}{}{[maˈla]}{(v.)}{1}{}{Amarrar.}{}{}%
\verb{mala}{}{[maˈla]}{(v.)}{2}{}{Atar.}{}{}%
\verb{mala}{}{[maˈla]}{(v.)}{3}{}{Capturar.}{}{}
\verb{mala}{}{[maˈla]}{(v.)}{4}{}{Prender.}{}{}%
\verb{maladu}{}{[maˈladu]}{(adj.)}{1}{}{Amarrado.}{}{}%
\verb{maladu}{}{[maˈladu]}{(adj.)}{2}{}{Preso.}{}{}%
\verb{maladu}{}{[maˈladu]}{(adj.)}{3}{}{Capturado.}{}{}
\verb{malakundja}{}{[malakũˈdʒa]}{(n.)}{1}{}{Maracujá.}{\textbf{\textit{Passiflora edulis}.}}{}
\verb{malakundja-blabu}{}{[malakũˈdʒa ˈblabu]}{(n.)}{1}{}{Maracujá-roxo.}{\textbf{\textit{Passiflora edulis}.}}{}
\verb{malakundja-koblo}{}{[malakũˈdʒa kɔˈblɔ]}{(n.)}{1}{}{Maracujá-de-cobra.}{\textbf{\textit{Passiflora foetida}.}}{}
\verb{malakundja-nglandji}{}{[malakũˈdʒa ˈŋgl\~{\textturna}dʒi]}{(n.)}{1}{}{Maracujá-grande.}{\textbf{\textit{Passiflora quadrangularis}.}}{}
\verb{malakunja}{}{[malakũˈʒa]}{(n.)}{1}{}{Maracujá.}{Cf.
\textbf{malakundja}.}{}{}
\verb{mala mon}{}{[maˈla ˈmõ]}{(expr.)}{1}{}{Casar.}{}{}
\verb{mala mon}{}{[maˈla ˈmõ]}{(expr.)}{2}{}{Contrair matrimônio.}{}{}
\verb{mala-mon}{}{[maˈla ˈmõ]}{(n.)}{1}{}{Casamento.}{}{}
\verb{mala-mon}{}{[maˈla ˈmõ]}{(n.)}{1}{}{Matrimônio.}{}{}
\verb{malapyon}{}{[malaˈpj\~ɔ]}{(n.)}{1}{Marapião.}{\textbf{\textit{Zanthoxylum gilletii}.}}{}
\verb{malapyon-mwala}{}{[malaˈpj\~ɔ ˈmwala]}{(n.)}{1}{Marapião-mulher.}{\textbf{\textit{Zanthoxylum rubescens}.}}{}
\verb{male}{}{[maˈlɛ]}{(n.)}{1}{}{Jovem.}{}{}
\verb{male}{}{[maˈlɛ]}{(n.)}{2}{}{Moço.}{}{}
\verb{male}{}{[maˈlɛ]}{(n.)}{3}{}{Rapaz.}{}{}
\verb{malelu}{}{[maˈlɛlu]}{(adj.)}{1}{}{Amarelo.}{}{}
\verb{mali}{}{[ˈmali]}{(adv.)}{1}{}{Mal.}{}{}
\verb{mali}{}{[ˈmali]}{(n.)}{1}{}{Erro.}{}{}
\verb{mali}{}{[ˈmali]}{(n.)}{1}{}{Mal.}{}{}
\verb{mali-bega}{}{[ˈmali ˈbɛga]}{(n.)}{1}{}{Dor de barriga.}{}{}
\verb{mali-bega}{}{[ˈmali ˈbɛga]}{(n.)}{2}{}{Prisão de ventre.}{}{}
\verb{malidu}{}{[maˈlidu]}{(n.)}{1}{}{Marido.}{}{}
\verb{malikyadu}{}{[maliˈkjadu]}{(adj.)}{1}{}{Malcriado.}{}{}
\verb{malikyadu}{}{[maliˈkjadu]}{(n.)}{1}{}{Malcriado.}{}{}
\verb{malimboki}{}{[malĩˈbɔki]}{(n.)}{1}{}{\textit{Malimboki}.}{\textbf{\textit{Oncoba spinosa}}.}{}
\verb{malivla}{}{[ˈmalivla]}{(n.)}{1}{}{Malva.}{\textbf{\textit{Abutilon grandiflorum}.}}{}
\verb{malixa}{}{[maˈliʃa]}{(n.)}{1}{}{Capricho.}{}{}
\verb{malixa}{}{[maˈliʃa]}{(n.)}{2}{}{Malícia.}{}{}
\verb{malixa}{}{[maˈliʃa]}{(n.)}{3}{}{Malvadez.}{}{}
\verb{malôkô}{}{[maˈloko]}{(adj.)}{1}{}{Ingênuo.}{}{}
\verb{malôkô}{}{[maˈloko]}{(adj.)}{2}{}{Inocente.}{}{}
\verb{malôkô}{}{[maˈloko]}{(adj.)}{3}{}{Maluco.}{}{}
\verb{malôkô}{}{[maˈloko]}{(adj.)}{4}{}{Pateta.}{}{}
\verb{malôkô}{}{[maˈloko]}{(adj.)}{5}{}{Tonto.}{}{}
\verb{malôtô}{}{[maˈloto]}{(adj.)}{1}{}{Maroto.}{}{}
\verb{malu}{}{[ˈmalu]}{(n.)}{1}{}{Cônjuge.}{}{}{}
\verb{malu}{}{[ˈmalu]}{(n.)}{1}{}{Esposa.}{}{}{}
\verb{malu}{}{[ˈmalu]}{(n.)}{1}{}{Esposo.}{}{}{}
\verb{malu}{}{[ˈmalu]}{(n.)}{1}{}{Marido.}{Cf. \textbf{malun}.}{}{}
\verb{malun}{}{[ˈmal\~u]}{(n.)}{1}{}{Cônjuge.}{}{}
\verb{malun}{}{[ˈmal\~u]}{(n.)}{2}{}{Esposa.}{}{}
\verb{malun}{}{[ˈmal\~u]}{(n.)}{3}{}{Esposo.}{}{}
\verb{malun}{}{[ˈmal\~u]}{(n.)}{4}{}{Marido.}{}{}
\verb{ma-lungwa}{}{[ˈma ˈl\~ugwa]}{(adj.)}{1}{}{Maledicente.}{}{}
\verb{ma-lungwa}{}{[ˈma ˈl\~ugwa]}{(n.)}{1}{}{Maledicência.}{}{}
\verb{maluvada}{}{[maluˈvada]}{(adj.)}{1}{}{Malvada.}{}{}%
\verb{maluvada}{}{[maluˈvada]}{(adj.)}{2}{}{Má.}{}{}
\verb{maluvadu}{}{[maluˈvadu]}{(adj.)}{1}{}{Malvado.}{}{}%
\verb{maluvadu}{}{[maluˈvadu]}{(adj.)}{2}{}{Mau.}{}{}
\verb{mama}{}{[maˈma]}{(n.)}{1}{}{Mãe.}{}{}%
\verb{mama}{}{[maˈma]}{(n.)}{2}{}{Senhora.}{}{}
\verb{mama}{}{[ˈmama]}{(n.)}{1}{}{Mama.}{}{}%
\verb{mama}{}{[ˈmama]}{(n.)}{2}{}{Seio.}{}{}
\verb{mama}{}{[maˈma]}{(v.)}{1}{}{Amamentar.}{}{}
\verb{mama}{}{[maˈma]}{(v.)}{2}{}{Mamar.}{}{}
\verb{mama-kadela}{}{[ˈmama kaˈdɛla]}{(n.)}{1}{}{Nádegas.}{}{}
\verb{mamalongô}{}{[mamaˈlõgo]}{(n.)}{1}{}{\textit{Mamalongô}.}{\textbf{\textit{Luffa aegyptiaca}.}}{}
\verb{mamblêblê}{}{[m\~{\textturna}bleˈble]}{(n.)}{1}{}{\textit{Mamblêblê}.}{\textbf{\textit{Brillantaisia patula}.}}{}
\verb{mamon}{}{[maˈmõ]}{(n.)}{1}{}{Mamão.}{}{}
\verb{mamon}{}{[maˈmõ]}{(n.)}{2}{}{Mamoeiro.}{\textbf{\textit{Carica papaya}.}}{}
\verb{mamon}{}{[maˈmõ]}{(n.)}{1}{}{Papaia.}{}{}
\verb{mamon}{}{[maˈmõ]}{(n.)}{2}{}{Mamoeiro-papaia.}{\textbf{\textit{Carica papaya}.}}{}
\verb{mamon-d'ôbô}{}{[maˈmõ doˈbo]}{(n.)}{1}{}{Mamão-do-obô.}{\textbf{\textit{Drypetes glabra}.}}{}
\verb{mamônô}{}{[mamoˈno]}{(n.)}{1}{}{Mamona.}{\textbf{\textit{Ricinus communis}.}}{}
\verb{mamônô}{}{[mamoˈno]}{(n.)}{2}{}{Rícino.}{\textbf{\textit{Ricinus communis}.}}{}
\verb{mampyan}{}{[m\~{\textturna}ˈp\~j\~{\textturna}]}{(n.)}{1}{}{Carapinha.}{}{}
\verb{mampyan}{}{[m\~{\textturna}ˈp\~j\~{\textturna}]}{(n.)}{2}{Marapião.}{\textbf{\textit{Zanthoxylum gillettii}}.}{}{}
\verb{mana}{}{[ˈmana]}{(n.)}{1}{}{Irmã.}{}{}
\verb{mana}{}{[ˈmana]}{(n.)}{2}{}{Mana.}{}{}
\verb{manaka}{}{[manaˈka]}{(n.)}{1}{}{\textit{Manaka}.}{\textbf{\textit{Brunfelsia uniflora}}.}{}
\verb{manda}{}{[m\~{\textturna}ˈda]}{(conj.)}{1}{}{Por isso.}{Cf. \textbf{êlê manda}.}{}
\verb{manda}{}{[m\~{\textturna}ˈda]}{(v.)}{1}{}{Enviar.}{}{}%
\verb{manda}{}{[m\~{\textturna}ˈda]}{(v.)}{2}{}{Mandar.}{}{}%
\verb{manda}{}{[m\~{\textturna}ˈda]}{(v.)}{3}{}{Ordenar.}{}{}
\verb{manda bi}{}{[m\~{\textturna}ˈda ˈbi]}{(v.)}{1}{}{Convocar.}{}{}
\verb{manda bi}{}{[m\~{\textturna}ˈda ˈbi]}{(v.)}{1}{}{Importar.}{}{}
\verb{mandjan}{}{[m\~{\textturna}ˈdʒ\~{\textturna}]}{(n.)}{1}{}{Madrinha.}{}{}
\verb{mandjinga}{}{[m\~{\textturna}ˈdʒĩga]}{(n.)}{1}{}{Ataque de fúria.}{}{}
\verb{mandjinga}{}{[m\~{\textturna}ˈdʒĩga]}{(n.)}{2}{}{Cólera.}{}{}
\verb{mandjinga}{}{[m\~{\textturna}ˈdʒĩga]}{(n.)}{3}{}{Fúria.}{}{}
\verb{mandjinga}{}{[m\~{\textturna}ˈdʒĩga]}{(n.)}{4}{}{Nervosismo.}{}{}
\verb{mandjingêlu}{}{[m\~{\textturna}dʒĩˈgelu]}{(adj.)}{1}{}{Desordeiro.}{}{}%
\verb{mandjingêlu}{}{[m\~{\textturna}dʒĩˈgelu]}{(adj.)}{2}{}{Destemido.}{}{}
\verb{mandjingêlu}{}{[m\~{\textturna}dʒĩˈgelu]}{(adj.)}{3}{}{Furioso.}{}{}
\verb{mandjingêlu}{}{[m\~{\textturna}dʒĩˈgelu]}{(adj.)}{4}{}{Insubmisso.}{}{}\verb{mandjingêlu}{}{[m\~{\textturna}dʒĩˈgelu]}{(adj.)}{5}{}{Nervoso.}{}{}
\verb{mandjingêlu}{}{[m\~{\textturna}dʒĩˈgelu]}{(adj.)}{6}{}{Rebelde.}{}{}
\verb{mandjingêlu}{}{[m\~{\textturna}dʒĩˈgelu]}{(adj.)}{7}{}{Reguila.}{}{}
\verb{mandjoka}{}{[m\~{\textturna}ˈdʒɔka]}{(n.)}{1}{}{Mandioca.}{\textbf{\textit{Manihot esculenta}}.}{}
\verb{mandjoka-zaya}{}{[m\~{\textturna}ˈdʒɔka ˈzaja]}{(n.)}{1}{}{Mandioca-brava.}{\textbf{\textit{Janipha manihot}}.}{}
\verb{mandjolo}{}{[m\~{\textturna}dʒɔˈlɔ]}{(n.)}{1}{}{\textit{Mandjolo}.}{\textbf{\textit{Solenostemon monostachyus}}.}{}
\verb{manduku}{}{[m\~{\textturna}ˈduku]}{(n.)}{1}{}{Acha.}{}{}
\verb{manduku}{}{[m\~{\textturna}ˈduku]}{(n.)}{2}{}{Bastão.}{}{}%
\verb{manduku}{}{[m\~{\textturna}ˈduku]}{(n.)}{3}{}{Bordão.}{}{}%
\verb{manduku}{}{[m\~{\textturna}ˈduku]}{(n.)}{4}{}{Cacete.}{}{}%
\verb{manga}{}{[ˈm\~{\textturna}ga]}{(n.)}{1}{}{Manga.}{}{}%
\verb{manga}{}{[ˈm\~{\textturna}ga]}{(n.)}{4}{}{Manga.}{(Parte da peça de vestuário.)}{}%
\verb{manga}{}{[ˈm\~{\textturna}ga]}{(n.)}{2}{}{Mangueira.}{\textbf{\textit{Mangifera indica}}.}{}%
\verb{manga}{}{[ˈm\~{\textturna}ga]}{(n.)}{3}{}{Pega.}{}{}%
\verb{manga}{}{[ˈm\~{\textturna}ga]}{(n.)}{5}{}{Ramo.}{}{}
\verb{manga}{}{[m\~{\textturna}ˈga]}{(v.)}{1}{}{Provocar.}{}{}
\verb{manga-makaku}{}{[ˈm\~{\textturna}ga maˈkaku]}{(n.)}{1}{}{Manga-maluca.}{\textbf{\textit{Irvingia gabonensis}}.}{}
\verb{mangason}{}{[m\~{\textturna}gaˈsõ]}{(n.)}{1}{}{Gozo.}{}{}
\verb{mangason}{}{[m\~{\textturna}gaˈsõ]}{(n.)}{2}{}{Ironia.}{}{}
\verb{mangason}{}{[m\~{\textturna}gaˈsõ]}{(n.)}{3}{}{Menosprezo.}{}{}
\verb{mangason}{}{[m\~{\textturna}gaˈsõ]}{(n.)}{4}{}{Troça.}{}{}
\verb{mangenge}{}{[m\~{\textturna}g\~eˈge]}{(n.)}{1}{}{Aranha \textit{mangenge}.}{}{}
\verb{mangi}{}{[ˈm\~{\textturna}gi]}{(n.)}{1}{}{Mangue.}{}{}
\verb{mangi-d'ôbô}{}{[ˈm\~{\textturna}gi doˈbo]}{(n.)}{1}{}{Nêspera-do-mato.}{\textbf{\textit{Corynanthe paniculata}}.}{}
\verb{mangineza}{}{[m\~{\textturna}giˈnɛza]}{(n.)}{1}{}{Magnésio.}{}{}
\verb{mangi-ple}{}{[ˈm\~{\textturna}gi ˈplɛ]}{(n.)}{1}{}{Mangue-da-praia.}{\textbf{\textit{Rizophora harrisonii}}.}{}
\verb{manglôlô}{}{[m\~{\textturna}gloˈlo]}{(n.)}{1}{}{Camarão de rio}{(espécie).}{}
\verb{mangugu}{}{[m\~{\textturna}ˈgugu]}{(n.)}{1}{}{\textit{Mangugu}.}{\textbf{\textit{Thaumatococcus danielii}.}}{}
\verb{manha}{}{[ˈm\~{\textturna}ɲa]}{(n.)}{1}{}{Manha.}{}{}%
\verb{manha}{}{[ˈm\~{\textturna}ɲa]}{(n.)}{2}{}{Mania.}{}{}
\verb{manha}{}{[ˈm\~{\textturna}ɲa]}{(n.)}{3}{}{Truques.}{}{}
\verb{manha}{}{[m\~{\textturna}ˈɲa]}{(v.)}{1}{}{Deixar de.}{}{}
\verb{manha}{}{[m\~{\textturna}ˈɲa]}{(v.)}{1}{}{Reduzir.}{}{}
\verb{manhêlu}{}{[m\~{\textturna}ˈɲelu]}{(n.)}{1}{}{Marinheiro.}{}{}
\verb{maniwini}{}{[maniˈwini]}{(n.)}{1}{}{Mármore.}{}{}
\verb{manjinga}{}{[m\~{\textturna}ˈʒĩga]}{(n.)}{1}{}{Ataque de fúria.}{}{}{}
\verb{manjinga}{}{[m\~{\textturna}ˈʒĩga]}{(n.)}{1}{}{Cólera.}{}{}{}
\verb{manjinga}{}{[m\~{\textturna}ˈʒĩga]}{(n.)}{1}{}{Fúria.}{}{}{}
\verb{manjinga}{}{[m\~{\textturna}ˈʒĩga]}{(n.)}{1}{}{Nervosismo.}{Cf. \textbf{mandjinga}.}{}{}
\verb{manjinkon}{}{[m\~{\textturna}ʒ\~iˈk\~o]}{(n.)}{1}{}{Manjericão.}{Cf. \textbf{mlanjinkon}.}{}{}
\verb{manka}{}{[m\~{\textturna}ˈka]}{(v.)}{1}{}{Mancar.}{}{}%
\verb{manka}{}{[m\~{\textturna}ˈka]}{(v.)}{1}{}{Manquejar.}{}{}%
\verb{mankêlê}{}{[m\~{\textturna}keˈle]}{(adj.)}{1}{}{Aleijado.}{}{}%
\verb{mankêlê}{}{[m\~{\textturna}keˈle]}{(adj.)}{1}{}{Manco.}{}{}%
\verb{mankina}{}{[ˈm\~{\textturna}kina]}{(n.)}{1}{}{Máquina.}{}{}
\verb{manklutu}{}{[m\~{\textturna}ˈklutu]}{(adj.)}{1}{Encruado.}{}{}{}
\verb{manklutu}{}{[m\~{\textturna}ˈklutu]}{(adj.)}{2}{Imaturo.}{}{}{}
\verb{mankwete}{}{[m\~{\textturna}kwɛˈtɛ]}{(adv.)}{1}{}{Abundantemente.}{}{}%\verb{mankwete}{}{[m\~{\textturna}kwɛˈtɛ]}{(adv.)}{2}{}{Demasiadamente.}{}{}%\verb{mankwete}{}{[m\~{\textturna}kwɛˈtɛ]}{(adv.)}{3}{}{Muito.}{}{}
\verb{mansa}{}{[m\~{\textturna}ˈsa]}{(v.)}{1}{}{Amansar.}{}{}
\verb{mansa}{}{[m\~{\textturna}ˈsa]}{(v.)}{2}{}{Amassar.}{}{}
\verb{mansa}{}{[m\~{\textturna}ˈsa]}{(v.)}{3}{}{Bater.}{}{}
\verb{mansa}{}{[m\~{\textturna}ˈsa]}{(v.)}{4}{}{Domar.}{}{}
\verb{mansa}{}{[m\~{\textturna}ˈsa]}{(v.)}{5}{}{Domesticar.}{}{}
\verb{mansa}{}{[m\~{\textturna}ˈsa]}{(v.)}{6}{}{Espancar.}{}{}
\verb{mansa}{}{[m\~{\textturna}ˈsa]}{(v.)}{7}{}{Estar entre.}{}{}
\verb{mansa}{}{[m\~{\textturna}ˈsa]}{(v.)}{8}{}{Estar no meio.}{}{}
\verb{mansadêlu}{}{[m\~{\textturna}saˈdelu]}{(n.)}{1}{}{Lenhador.}{}{}%
\verb{mansadu}{}{[m\~{\textturna}ˈsadu]}{(n.)}{1}{}{Machado.}{}{}%
\verb{manse}{}{[m\~{\textturna}ˈsɛ]}{(n.)}{1}{}{Estrangeiro.}{}{}
\verb{manse}{}{[m\~{\textturna}ˈsɛ]}{(n.)}{2}{}{Serviçal.}{}{}
\verb{manson}{}{[m\~{\textturna}ˈsõ]}{(n.)}{1}{}{Maçã.}{}{}
\verb{manson}{}{[m\~{\textturna}ˈsõ]}{(n.)}{1}{}{Mansão.}{}{}
\verb{mansu}{}{[ˈm\~{\textturna}su]}{(adj.)}{1}{}{Dissimulado.}{\textbf{Numigu mansu}. \textit{Inimigo dissimulado}.}{}
\verb{mansu}{}{[ˈm\~{\textturna}su]}{(adj.)}{1}{}{Manso.}{}{}
\verb{manta}{}{[ˈm\~{\textturna}ta]}{(n.)}{1}{}{Manta.}{}{}
\verb{mantê}{}{[m\~{\textturna}ˈte]}{(v.)}{1}{}{Manter.}{}{}
\verb{mantega}{}{[m\~{\textturna}ˈtɛga]}{(n.)}{1}{}{Manteiga.}{}{}
\verb{mantine}{}{[m\~{\textturna}tiˈnɛ]}{(n.)}{1}{}{Matinê.}{}{}
\verb{mantxan}{}{[m\~{\textturna}ˈtʃ\~{\textturna}]}{(n.)}{1}{}{Cumprimentos.}{}{}%
\verb{mantxan}{}{[m\~{\textturna}ˈtʃ\~{\textturna}]}{(n.)}{2}{}{Saudações.}{}{}
\verb{mantxi}{}{[m\~{\textturna}ˈtʃi]}{(n.)}{1}{}{Catana.}{}{}%
\verb{mantxi}{}{[m\~{\textturna}ˈtʃi]}{(n.)}{3}{}{Manche.}{}{}
\verb{manu}{}{[ˈmanu]}{(n.)}{1}{}{Irmão.}{}{}
\verb{manu}{}{[ˈmanu]}{(n.)}{2}{}{Mano.}{}{}
\verb{manve}{}{[m\~{\textturna}ˈvɛ]}{(n.)}{1}{}{Cãibra.}{}{}
\verb{manxi}{}{[m\~{\textturna}ˈʃi]}{(n.)}{1}{}{Catana.}{}{}{}
\verb{manxi}{}{[m\~{\textturna}ˈʃi]}{(n.)}{1}{}{Manche.}{Cf. \textbf{mantxi}.}{}{}
\verb{manzenze}{}{[m\~{\textturna}z\~ɛˈzɛ]}{(n.)}{1}{}{Vinho de palma muito doce.}{}{}
\verb{mapinta}{}{[mapĩˈta]}{(n.)}{1}{}{Tubarão-baleia.}{\textbf{\textit{Rhincodon typus}}.}{}
\verb{maplamina}{}{[maplaˈmina]}{(n.)}{1}{}{Camarão de rio}{(espécie).}{}
\verb{masa}{}{[ˈmasa]}{(n.)}{1}{}{Argamassa.}{}{}%
\verb{masa}{}{[ˈmasa]}{(n.)}{2}{}{Gesso.}{}{}
\verb{masa}{}{[ˈmasa]}{(n.)}{3}{}{Massa alimentar.}{}{}%
\verb{masa}{}{[maˈsa]}{(v.)}{1}{}{Amassar.}{}{}
\verb{masa}{}{[maˈsa]}{(v.)}{2}{}{Incomodar.}{}{}
\verb{masa}{}{[maˈsa]}{(v.)}{3}{}{Maçar.}{}{}
\verb{masada}{}{[maˈsada]}{(n.)}{1}{}{Cansaço.}{}{}
\verb{masada}{}{[maˈsada]}{(n.)}{2}{}{Incômodo.}{}{}
\verb{masada}{}{[maˈsada]}{(n.)}{3}{}{Maçada.}{}{}
\verb{masada}{}{[masaˈda]}{(v.)}{1}{}{Cansar.}{}{}
\verb{masada}{}{[masaˈda]}{(v.)}{1}{}{Incomodar.}{}{}
\verb{masadadu}{}{[masaˈdadu]}{(adj.)}{1}{}{Cansado.}{}{}
\verb{masadadu}{}{[masaˈdadu]}{(adj.)}{1}{}{Incomodado.}{}{}
\verb{masadadu}{}{[masaˈdadu]}{(adj.)}{2}{}{Maçado.}{}{}%
\verb{masadô}{}{[masaˈdo]}{(n.)}{1}{}{Argamassador.}{}{}
\verb{masakle}{}{[maˈsaklɛ]}{(n.)}{1}{}{Massacre.}{}{}
\verb{masêtê}{}{[maˈsete]}{(n.)}{1}{}{Cacete.}{}{}
\verb{masêtê}{}{[maˈsete]}{(n.)}{1}{}{Pau para matar peixes.}{}{}%
\verb{ma-sonhu}{}{[ˈma ˈsõɲu]}{(n.)}{1}{}{Pesadelo.}{}{}
\verb{masoniku}{}{[maˈsoniku]}{(n.)}{1}{}{Feiticeiro.}{}{}
\verb{masoniku}{}{[maˈsoniku]}{(n.)}{2}{}{Maçônico.}{}{}
\verb{masoniku}{}{[maˈsoniku]}{(n.)}{3}{}{Mágico.}{}{}
\verb{ma-sotxi}{}{[ˈma ˈsᴐtʃi]}{(n.)}{1}{}{Azar.}{}{}
\verb{masu}{}{[ˈmasu]}{(n.)}{1}{}{Maço.}{}{}
\verb{masu}{}{[ˈmasu]}{(n.)}{2}{}{Março.}{}{}
\verb{maswenswe}{}{[masw\~{ɛ}ˈswɛ]}{(n.)}{1}{}{Sorgo.}{\textbf{\textit{Sorghum sp.-Exccic}.}}{}
\verb{mata}{}{[maˈta]}{(v.)}{1}{}{Acabar.}{}{}
\verb{mata}{}{[maˈta]}{(v.)}{2}{}{Matar.}{}{}
\verb{mata}{}{[maˈta]}{(v.)}{3}{}{Terminar.}{}{}
\verb{mata-bisu}{}{[ˈmataˈbisu]}{(n.)}{1}{}{Café-da-manhã.}{}{}
\verb{mata-bisu}{}{[ˈmataˈbisu]}{(n.)}{2}{}{Gratificação.}{}{}
\verb{mata-bisu}{}{[ˈmataˈbisu]}{(n.)}{2}{}{Mata-bicho.}{}{}
\verb{mata-bisu}{}{[ˈmataˈbisu]}{(n.)}{3}{}{Pequeno-almoço.}{}{}
\verb{mata-bwê}{}{[ˈmata ˈbwê]}{(n.)}{1}{}{Mata-boi.}{\textit{\textbf{Abutilon striatum}}.}{}
\verb{matadô}{}{[mataˈdo]}{(n.)}{1}{}{Matador.}{}{}
\verb{matakumbi}{}{[mataˈk\~ubi]}{(n.)}{1}{}{\textit{Matakumbi}.}{Ritmo musical.}{}
\verb{matapasu}{}{[mataˈpasu]}{(n.)}{1}{}{Mata-passo.}{\textbf{\textit{Pentadesma butyracea}.}}{}
\verb{mata ubwê}{}{[matuˈbwê]}{(expr.)}{1}{}{Suicidar-se.}{}{}
\verb{matawula}{}{[mataˈwula]}{(n.)}{1}{}{Peixe salgado.}{}{}
\verb{matazen}{}{[mataˈzẽ]}{(n.)}{1}{}{\textit{Matazen}.}{\textbf{\textit{Merremia aegyptia}.}}{}
\verb{matelu}{}{[maˈtɛlu]}{(n.)}{1}{}{Martelo.}{}{}
\verb{matete}{}{[matɛˈtɛ]}{(n.)}{1}{}{Refugo do óleo de palma.}{}{}
\verb{matêya}{}{[maˈteja]}{(n.)}{1}{}{Pus.}{}{}
\verb{matimatika}{}{[matiˈmatika]}{(n.)}{1}{}{Matemática.}{}{}
\verb{matli}{}{[ˈmatli]}{(n.)}{1}{}{Fitolaca.}{\textbf{\textit{Phytolacca dodecandra}.}}{}
\verb{matli-mwala}{}{[ˈmatli ˈmwala]}{(n.)}{1}{}{Plumbago-branca.}{\textbf{\textit{Plumbago zeylanica}.}}{}
\verb{matlusu}{}{[maˈtlusu]}{(n.)}{1}{}{Erva-de-Santa-Maria.}{\textbf{\textit{Chenopodium ambrosioides}.}}{}
\verb{matlusu}{}{[maˈtlusu]}{(n.)}{2}{}{Matruço.}{\textbf{\textit{Chenopodium ambrosioides}.}}{}
\verb{matotadji}{}{[matɔˈtadʒi]}{(n.)}{1}{}{Sujeira.}{}{}
\verb{matotaji}{}{[matɔˈtaʒi]}{(n.)}{1}{}{Sujeira.}{Cf. \textbf{matotadji}.}{}
\verb{matu}{}{[ˈmatu]}{(n.)}{1}{}{Campo.}{}{}
\verb{matu}{}{[ˈmatu]}{(n.)}{2}{}{Mato.}{}{}
\verb{matu}{}{[ˈmatu]}{(n.)}{3}{}{Terreno baldio.}{}{}
\verb{matu-bana}{}{[ˈmatu ˈbana]}{(n.)}{1}{}{Folha-ponto.}{\textbf{\textit{Achyrantes aspera}.}}{}
\verb{matu-kana}{}{[ˈmatu kaˈna]}{(n.)}{1}{}{\textit{Matu-kana}.}{\textit{\textbf{Mikania chenopodiifolia}}.}{}
\verb{matxanzoxi}{}{[matʃ\~{\textturna}ˈzɔʃi]}{(n.)}{1}{}{\textit{Matxanzoxi}.}{\textbf{\textit{Syzygium guineense}.}}{}
\verb{matxi}{}{[ˈmatʃi]}{(n.)}{1}{}{Dificuldade.}{}{}
\verb{matxi}{}{[ˈmatʃi]}{(n.)}{2}{}{Sacrifício.}{}{}
\verb{matxikula}{}{[matʃikuˈla]}{(v.)}{1}{}{Matricular.}{}{}
\verb{matxoka}{}{[matʃɔˈka]}{(v.)}{1}{}{Amarrotar.}{}{}
\verb{matxoka}{}{[matʃɔˈka]}{(v.)}{1}{}{Machucar.}{}{}
\verb{matxokadu}{}{[matʃɔˈkadu]}{(adj.)}{1}{}{Amarrotado.}{}{}
\verb{matxokadu}{}{[matʃɔˈkadu]}{(adj.)}{1}{}{Machucado.}{}{}
\verb{maw}{}{[ˈmaw]}{(adj.)}{1}{}{Má.}{}{}
\verb{maw}{}{[ˈmaw]}{(adj.)}{2}{}{Mau.}{}{}
\verb{ma-wê}{}{[ˈma ˈwe]}{(n.)}{1}{}{Mau-olhado.}{}{}
\verb{maxi}{}{[ˈmaʃi]}{(adv.)}{1}{}{Mais.}{}{}{}
\verb{maxibin}{}{[maʃiˈbĩ]}{(n.)}{2}{}{Jovem.}{}{}
\verb{maxibin}{}{[maʃiˈbĩ]}{(n.)}{2}{}{Mancebo.}{}{}
\verb{maxibin}{}{[maʃiˈbĩ]}{(n.)}{1}{}{Rapaz.}{}{}
\verb{maxi-montxi}{}{[ˈmaʃi ˈmõtʃi]}{(quant.)}{2}{}{Maioria.}{}{}{}
\verb{maxi-montxi}{}{[ˈmaʃi ˈmõtʃi]}{(quant.)}{1}{}{Muitos.}{\textbf{Maxi montxi nen sangê ve.} \emph{Muitas das senhoras velhas.}}{}{}
\verb{maxipombô}{}{[ˈmaʃiˈpõbo]}{(n.)}{1}{}{\textit{Maxipombô}.}{\textbf{\textit{Hemiramphus balao}.}}{}
\verb{maxkelenxa}{}{[maʃkɛˈl\~ɛʃa]}{(n.)}{1}{}{Dívida moral.}{}{}
\verb{maxkelenxa}{}{[maʃkɛˈl\~ɛʃa]}{(n.)}{2}{}{Maldição.}{}{}
\verb{maxkelenxa}{}{[maʃkɛˈl\~ɛʃa]}{(n.)}{3}{}{Pecado.}{}{}
\verb{maxkovadu}{}{[maʃkɔˈvadu]}{(n.)}{1}{}{Mulato.}{}{}
\verb{maxpadu}{}{[maʃˈpadu]}{(n.)}{1}{}{Adulto.}{}{}
\verb{maxpadu}{}{[maʃˈpadu]}{(n.)}{2}{}{Sênior.}{}{}
\verb{maxtlu}{}{[ˈmaʃtlu]}{(n.)}{1}{}{Mastro.}{}{}
\verb{mayaga}{}{[majaˈga]}{(n.)}{1}{}{Molho à base de folhas e peixe.}{}{}{}
\verb{maya-wê-son}{}{[maˈja ˈwe ˈs\~ɔ]}{(n.)}{1}{}{Vagina.}{}{}
\verb{mayoba}{}{[maˈjɔba]}{(n.)}{1}{}{Fedegoso.}{\textbf{\textit{Cassia occidentalis}.}}{}
\verb{mayoba-beni}{}{[maˈjɔba ˈbɛni]}{(n.)}{1}{}{\textit{Mayoba-beni}.}{\textbf{\textit{Cassia sophera}.}}{}
\verb{mayu}{}{[ˈmaju]}{(n.)}{1}{}{Maio.}{}{}
\verb{mbaxada}{}{[mbaˈʃada]}{(n.)}{1}{}{Embaixada.}{}{}
%\verb{mbila}{}{[mbiˈla]}{(n.)}{2}{}{Pagã.}{}{}
\verb{mbila}{}{[mbiˈla]}{(n.)}{2}{}{Tumba.}{}{}
\verb{mbila}{}{[mbiˈla]}{(n.)}{3}{}{Túmulo pagão.}{}{}
%\verb{mbon}{}{[ˈmbõ]}{(adv.)}{1}{}{Em suma.}{}{}
\verb{mbon}{}{[ˈmbõ]}{(adv.)}{2}{}{Enfim.}{}{}
\verb{mbon}{}{[ˈmbõ]}{(adv.)}{3}{}{Ora bem.}{}{}
\verb{mbon}{}{[ˈmbõ]}{(adv.)}{4}{}{Vejamos.}{}{}
\verb{me}{}{[ˈmɛ]}{(adv.)}{1}{}{Mesmo.}{}{}%
\verb{me}{}{[ˈmɛ]}{(adv.)}{2}{}{Próprio.}{}{}
\verb{me-blugadu}{}{[ˈmɛ bluˈgadu]}{(n.)}{1}{}{Pênis.}{}{}
\verb{mê-dja}{}{[meˈdʒa]}{(n.)}{1}{}{Meio-dia.}{}{}
\verb{mê-d'ola}{}{[meˈdɔla]}{(adv.)}{1}{}{De repente.}{}{}%
\verb{mê-d'ola}{}{[meˈdɔla]}{(adv.)}{2}{}{Rapidamente.}{}{}
\verb{mega}{}{[ˈmɛga]}{(n.)}{1}{}{Melga.}{}{}
\verb{mêgêmêgê}{}{[meˈgemeˈge]}{(id.)}{1}{}{Cf. \textbf{myêgêmyêgê}.}{}{}
\verb{mêji}{}{[ˈmeʒi]}{(conj.)}{1}{}{Mas.}{Cf. \textbf{maji}.}{}{}
\verb{mêji}{}{[ˈmeʒi]}{(n.)}{1}{}{Mês.}{}{}%
\verb{meka}{}{[ˈmɛka]}{(n.)}{1}{}{Cicatriz.}{}{}%
\verb{meka}{}{[ˈmɛka]}{(n.)}{2}{}{Marca.}{}{}
\verb{mela}{}{[mɛˈla]}{(v.)}{1}{}{Amadurecer bastante.}{}{}
\verb{mele}{}{[ˈmɛlɛ]}{(n.)}{1}{}{Mel.}{}{}{}
\verb{mê-lentxi}{}{[ˈme ˈlẽtʃi]}{(adv.)}{1}{}{Assim-assim.}{}{}%
\verb{mê-lentxi}{}{[ˈme ˈlẽtʃi]}{(adv.)}{2}{}{Mais ou menos.}{}{}
\verb{mele-vunvu}{}{[mɛlɛv\~uˈvu]}{(n.)}{1}{}{Mel de abelha.}{}{}
\verb{melon}{}{[mɛˈl\~ɔ]}{(n.)}{1}{}{Melão.}{}{}
\verb{membla}{}{[ˈm\~ɛbla]}{(n.)}{1}{}{Roupa obrigatória para os membros de uma confraria.}{}{}{}%
\verb{membla}{}{[ˈm\~ɛbla]}{(n.)}{2}{}{Membro.}{Elemento feminino do folclore.}{}{}{}
\verb{memblu}{}{[ˈm\~ɛblu]}{(n.)}{1}{}{Membro.}{}{}{}
\verb{meme}{}{[mɛˈmɛ]}{(n.)}{1}{}{Mãe.}{}{}
\verb{meme}{}{[mɛˈmɛ]}{(n.)}{1}{}{Pó-lixa.}{\textbf{\textit{Ficus exasperata}}.}{}
\verb{memen}{}{[mɛˈm\~{ɛ}]}{(n.)}{1}{}{Enorme.}{}{}
\verb{men}{}{[ˈm\~{ɛ}]}{(n.)}{1}{}{Mãe.}{}{}
\verb{menda}{}{[ˈm\~{ɛ}da]}{(n.)}{1}{}{Emenda.}{}{}
\verb{menda}{}{[m\~{ɛ}ˈda]}{(v.)}{1}{}{Corrigir.}{}{}
\verb{menda}{}{[m\~{ɛ}ˈda]}{(v.)}{2}{}{Emendar.}{}{}
\verb{men-dawa}{}{[m\~{ɛ}ˈdawa]}{(n.)}{1}{}{Água-viva.}{}{}
\verb{men-dawa}{}{[m\~{ɛ}ˈdawa]}{(n.)}{2}{}{Alforreca.}{}{}
\verb{men-dawa}{}{[m\~{ɛ}ˈdawa]}{(n.)}{3}{}{Fonte.}{}{}
\verb{men-dawa}{}{[m\~{ɛ}ˈdawa]}{(n.)}{4}{}{Mãe-d'água.}{}{}
\verb{mendu}{}{[ˈm\~{ɛ}du]}{(n.)}{1}{}{Medo.}{}{}
\verb{mendu}{}{[ˈm\~{ɛ}du]}{(n.)}{1}{}{Receio.}{}{}
\verb{mendu}{}{[ˈm\~{ɛ}du]}{(v.)}{1}{}{Ter medo.}{}{}
\verb{menemene}{}{[mɛˈnɛmɛˈnɛ]}{(id.)}{1}{}{Cf. \textbf{doxi menemene.}}{}{}
\verb{men-kaki}{}{[ˈm\~ɛ
kaˈki]}{(n.)}{1}{}{Mãe-de-caqui.}{\textbf{\textit{Myripristis jacobus}.}}{}
\verb{menlôfi}{}{[m\~eˈlofi]}{(n.)}{1}{}{Redemoinho.}{}{}
\verb{mensê}{}{[m\~eˈse]}{(v.)}{1}{}{Amar.}{}{}{}
\verb{mensê}{}{[m\~eˈse]}{(v.)}{1}{}{Desejar.}{}{}{}
\verb{mensê}{}{[m\~eˈse]}{(v.)}{1}{}{Precisar de.}{}{}{}
\verb{mensê}{}{[m\~eˈse]}{(v.)}{1}{}{Querer.}{Cf. \textbf{mêsê}.}{}{}
\verb{menu}{}{[mɛˈnu]}{(n.)}{1}{}{Tipo de carnaval são-tomense.}{}{}
\verb{menu}{}{[ˈmɛnu]}{(quant.)}{1}{}{Menos.}{}{}
\verb{mesa}{}{[mɛˈsa]}{(v.)}{2}{}{Mostrar a língua.}{}{}
\verb{mesa}{}{[mɛˈsa]}{(v.)}{1}{}{Surgir.}{}{}
\verb{mese}{}{[ˈmɛsɛ]}{(n.)}{1}{}{Herbalista.}{}{}%
\verb{mese}{}{[ˈmɛsɛ]}{(n.)}{2}{}{Mestre.}{}{}
\verb{mese}{}{[ˈmɛsɛ]}{(n.)}{3}{}{Professor.}{}{}
\verb{mese}{}{[ˈmɛsɛ]}{(n.)}{4}{}{Sabedoria.}{}{}
\verb{mêsê}{}{[meˈse]}{(n.)}{1}{}{Amor.}{}{}
\verb{mêsê}{}{[meˈse]}{(n.)}{2}{}{Desejo.}{}{}
\verb{mêsê}{}{[meˈse]}{(n.)}{3}{}{Vontade.}{}{}
\verb{mêsê}{}{[meˈse]}{(v.)}{1}{}{Amar.}{}{}
\verb{mêsê}{}{[meˈse]}{(v.)}{2}{}{Desejar.}{}{}
\verb{mêsê}{}{[meˈse]}{(v.)}{3}{}{Precisar de.}{}{}
\verb{mêsê}{}{[meˈse]}{(v.)}{4}{}{Querer.}{}{}%
\verb{mêsê}{}{[meˈse]}{(v.)}{5}{}{Requerer.}{}{}
\verb{mesesela}{}{[mɛsɛˈsɛla]}{(n.)}{1}{}{Inseto.}{}{}
\verb{metadji}{}{[mɛˈtadʒi]}{(n.)}{1}{}{Metade.}{}{}
\verb{mêtê}{}{[meˈte]}{(v.)}{1}{}{Meter.}{}{}
\verb{mê-txibi}{}{[ˈme ˈtʃibi]}{(adv.)}{1}{}{Assim-assim.}{}{}%
\verb{mê-txibi}{}{[ˈme ˈtʃibi]}{(adv.)}{2}{}{Mais ou menos.}{}{}
\verb{mextlason}{}{[mɛʃtlaˈs\~ɔ]}{(n.)}{1}{}{Menstruação.}{}{}
\verb{mêya}{}{[ˈmeja]}{(n.)}{1}{}{Meia.}{}{}
\verb{mêya}{}{[ˈmeja]}{(n.)}{1}{}{Peúga.}{}{}
\verb{mêya-nôtxi}{}{[mejaˈnotʃi]}{(n.)}{1}{}{Meia-noite.}{}{}
\verb{meza}{}{[ˈmɛza]}{(n.)}{1}{}{Mesa.}{}{}
\verb{mezada}{}{[mɛˈzada]}{(n.)}{1}{}{Mensalidade.}{}{}
\verb{mezada}{}{[mɛˈzada]}{(n.)}{1}{}{Mesada.}{}{}
\verb{midji}{}{[miˈdʒi]}{(v.)}{1}{}{Medir.}{}{}
\verb{midjida}{}{[miˈdʒida]}{(n.)}{1}{}{Medida.}{}{}
\verb{miga}{}{[ˈmiga]}{(n.)}{1}{}{Amiga.}{}{}
\verb{miga}{}{[miˈga]}{(v.)}{1}{}{Amigar.}{}{}
\verb{miga}{}{[miˈga]}{(v.)}{2}{}{Viver maritalmente.}{}{}
\verb{mige}{}{[miˈgɛ]}{(n.)}{1}{}{Migalha.}{}{}
\verb{migu}{}{[ˈmigu]}{(n.)}{1}{}{Amigo.}{}{}{}
\verb{miji}{}{[miˈʒi]}{(v.)}{1}{}{Medir.}{Cf. \textbf{midji}.}{}{}
\verb{mikoko}{}{[mikɔˈkɔ]}{(n.)}{1}{}{Alfavaca-cravo.}{\textbf{\textit{Ocimum
gratissimum}.}}{}
\verb{mikoko-kampu}{}{[mikɔˈkɔ
ˈk\~{\textturna}pu]}{(n.)}{1}{}{Cambará.}{\textbf{\textit{Lantana
camara}.}}{}
\verb{mikolo}{}{[mikɔˈlɔ]}{(n.)}{1}{}{Saia utilizada nas festas religiosas e
populares.}{}{}
\verb{mikondo}{}{[mik\~{ɔ}ˈdɔ]}{(n.)}{3}{}{Baobá.}{}{}
\verb{mikondo}{}{[mik\~{ɔ}ˈdɔ]}{(n.)}{2}{}{Imbondeiro.}{}{}%
\verb{mikondo}{}{[mik\~{ɔ}ˈdɔ]}{(n.)}{1}{}{\textit{Mikondo}.}{\textbf{\textit{Adansonia
digitata}}.}{}%
\verb{mila}{}{[miˈla]}{(v.)}{1}{}{Mirrar.}{}{}
\verb{milagli}{}{[miˈlagli]}{(n.)}{1}{}{Milagre.}{}{}
\verb{milantxi}{}{[miˈl\~{\textturna}tʃi]}{(n.)}{1}{}{Meliante.}{}{}
\verb{mile}{}{[miˈlɛ]}{(n.)}{1}{}{Mil-réis.}{Antiga unidade monetária.}{}
\verb{milhon}{}{[miˈʎõ]}{(adv.)}{1}{}{Antes.}{}{}
\verb{milhon}{}{[miˈʎõ]}{(adv.)}{2}{}{Melhor.}{}{}
\verb{milhon}{}{[miˈʎõ]}{(adv.)}{1}{}{Ser necessário.}{}{}
\verb{milhon}{}{[miˈʎõ]}{(adv.)}{1}{}{Ser preciso.}{}{}
\verb{milhon}{}{[miˈʎõ]}{(num.)}{1}{}{Milhão.}{}{}
\verb{mili}{}{[ˈmili]}{(num.)}{1}{}{Mil.}{}{}
\verb{milondo}{}{[mil\~{ɔ}ˈdɔ]}{(n.)}{1}{}{\textit{Milondo}.}{\textbf{\textit{Acridocarpus
longifolius}}.}{}
\verb{mimoya}{}{[miˈmɔja]}{(n.)}{1}{}{Brinco.}{}{}
\verb{mimu}{}{[ˈmimu]}{(n.)}{1}{}{Mimo.}{}{}
\verb{min}{}{[ˈmĩ]}{(n.)}{2}{}{Milho.}{}{}%
\verb{min}{}{[ˈmĩ]}{(n.)}{2}{}{Milheiro.}{\textbf{\textit{Zea mays}}.}{}%
\verb{mina}{}{[ˈmina]}{(n.)}{1}{}{Criança.}{}{}%
\verb{mina}{}{[ˈmina]}{(n.)}{2}{}{Filha.}{}{}%
\verb{mina}{}{[ˈmina]}{(n.)}{3}{}{Filho.}{}{}%
\verb{mina}{}{[ˈmina]}{(n.)}{4}{}{Menina.}{}{}
\verb{mina}{}{[ˈmina]}{(n.)}{5}{}{Menino.}{}{}%
\verb{mina}{}{[ˈmina]}{(n.)}{6}{}{Pequeno.}{}{}
\verb{mina}{}{[ˈmina]}{(n.)}{7}{}{Um bocadinho de.}{}{}
\verb{mina}{}{[ˈmina]}{(n.)}{8}{}{Um pouco de.}{}{}
\verb{mina}{}{[miˈna]}{(v.)}{1}{}{Armar uma armadilha.}{}{}{}
\verb{mina bodobodo}{}{[ˈmina bɔˈdɔbɔˈdɔ]}{(expr.)}{1}{}{Sensualíssima.}{}{}
\verb{mina-dedu}{}{[ˈmina ˈdɛdu]}{(n.)}{1}{}{Dedo mindinho.}{}{}
\verb{mina-fili}{}{[ˈmina ˈfili]}{(n.)}{1}{}{Bebê.}{}{}
\verb{mina-fili}{}{[ˈmina ˈfili]}{(n.)}{2}{}{Criança pequena.}{}{}%
\verb{mina-fili petepete}{}{[ˈmina ˈfili pɛˈtɛpɛˈtɛ]}{(expr.)}{1}{}{Criança muito pequena.}{}{}
\verb{mina-kasô}{}{[ˈmina kaˈso]}{(n.)}{1}{}{Cachorrinho.}{}{}
\verb{mina-kono}{}{[ˈmina kɔˈnɔ]}{(n.)}{1}{}{Clítoris.}{}{}
\verb{mina-kya}{}{[ˈmina ˈkja]}{(n.)}{1}{}{Criado.}{}{}
\verb{mina-kya}{}{[ˈmina ˈkja]}{(n.)}{2}{}{Menino de recados.}{}{}
\verb{mina-lemi}{}{[ˈmina ˈlɛmi]}{(n.)}{1}{}{Clítoris.}{}{}
\verb{mina-mindjan}{}{[ˈmina m\~{i}ˈdʒ\~{\textturna}]}{(n.)}{1}{}{Prematuro.}{}{}
\verb{mina-mosa}{}{[ˈmina ˈmɔsa]}{(n.)}{1}{}{Moça.}{}{}
\verb{mina-mosu}{}{[ˈmina ˈmɔsu]}{(n.)}{1}{}{Rapaz.}{}{}
\verb{mina-mwala}{}{[ˈmina ˈmwala]}{(n.)}{1}{}{Filha.}{}{}
\verb{mina-mwala}{}{[ˈmina ˈmwala]}{(n.)}{2}{}{Menina.}{}{}%
\verb{mina-mwala}{}{[ˈmina ˈmwala]}{(n.)}{3}{}{Moça.}{}{}%
\verb{mina-ome}{}{[ˈmina ˈɔmɛ]}{(n.)}{1}{}{Filho.}{}{}
\verb{mina-ome}{}{[ˈmina ˈɔmɛ]}{(n.)}{2}{}{Menino.}{}{}%
\verb{mina-ome}{}{[ˈmina ˈɔmɛ]}{(n.)}{3}{}{Moço.}{}{}%
\verb{mina-pikina}{}{[ˈmina piˈkina]}{(n.)}{1}{}{Bebê.}{}{}%
\verb{mina-pikina}{}{[ˈmina piˈkina]}{(n.)}{2}{}{Criança.}{}{}
\verb{mina-pinta}{}{[ˈmina ˈpĩta]}{(n.)}{1}{}{Pintinho.}{}{}
\verb{mina-pixi}{}{[ˈmina ˈpiʃi]}{(n.)}{1}{}{Pênis de criança.}{}{}
\verb{mina-platu}{}{[ˈmina ˈplatu]}{(n.)}{1}{}{Pires.}{}{}
\verb{mina-santome}{}{[ˈmina s\~{\textturna}tɔˈmɛ]}{(n.)}{1}{}{Filho da
terra.}{}{}
\verb{mina-santome}{}{[ˈmina s\~{\textturna}tɔˈmɛ]}{(n.)}{1}{}{Forro.}{}{}
\verb{mina-santome}{}{[ˈmina
s\~{\textturna}tɔˈmɛ]}{(n.)}{1}{}{São-tomense.}{}{}
\verb{mina-sentenxa}{}{[ˈmina s\~ɛˈt\~ɛʃa]}{(n.)}{1}{}{Criança que não pagou
as dívidas espirituais da vida anterior.}{}{}
\verb{mina-tela}{}{[ˈmina ˈtɛla]}{(n.)}{1}{}{Filho da terra.}{}{}
\verb{mina-tela}{}{[ˈmina ˈtɛla]}{(n.)}{2}{}{São-tomense.}{}{}
\verb{mina-tlabe}{}{[ˈmina tlaˈbɛ]}{(n.)}{1}{}{Criança de penitência.}{}{}
\verb{minda}{}{[ˈmĩda]}{(n.)}{1}{}{Medida.}{}{}
\verb{minda}{}{[ˈmĩda]}{(n.)}{2}{}{Quantidade.}{}{}
\verb{mindjan}{}{[mĩˈdʒ\~{\textturna}]}{(n.)}{1}{}{Mezinha.}{}{}
\verb{mindjan}{}{[mĩˈdʒ\~{\textturna}]}{(n.)}{2}{}{Remédio caseiro.}{}{}
\verb{mindjan-matu}{}{[mĩˈdʒ\~{\textturna} ˈmatu]}{(n.)}{1}{}{Mezinha
tradicional.}{}{}%
\verb{mindjan-saka}{}{[mĩˈdʒ\~{\textturna}
saˈka]}{(n.)}{1}{}{Vomitório.}{}{}%
\verb{mingwa}{}{[mĩˈgwa]}{(v.)}{1}{}{Diminuir.}{}{}
\verb{mingwa}{}{[mĩˈgwa]}{(v.)}{1}{}{Minguar.}{}{}
\verb{mingwantxi}{}{[mĩˈgw\~{\textturna}tʃi]}{(n.)}{1}{}{Minguante.}{}{}
\verb{minhon}{}{[mĩˈɲõ]}{(adv.)}{1}{}{Antes.}{}{}{}
\verb{minhon}{}{[mĩˈɲõ]}{(adv.)}{1}{}{Melhor.}{}{}{}
\verb{minhon}{}{[mĩˈɲõ]}{(adv.)}{1}{}{Ser necessário.}{}{}{}
\verb{minhon}{}{[mĩˈɲõ]}{(adv.)}{1}{}{Ser preciso.}{Cf. \textbf{milhon}.}{}{}
\verb{minimu}{}{[miˈnimu]}{(n.)}{1}{}{Mínimo.}{}{}
\verb{minixtelyu}{}{[miniʃˈtɛlju]}{(n.)}{1}{}{Ministério.}{}{}
\verb{minixtlu}{}{[miˈniʃtlu]}{(n.)}{1}{}{Ministro.}{}{}
\verb{minjan}{}{[mĩˈʒ\~{\textturna}]}{(n.)}{1}{}{Mezinha.}{}{}{}
\verb{minjan}{}{[mĩˈʒ\~{\textturna}]}{(n.)}{1}{}{Remédio caseiro.}{Cf.
\textbf{mindjan}.}{}{}
\verb{mintxi}{}{[m\~{i}ˈtʃi]}{(n.)}{1}{}{Luxação.}{}{}
\verb{mintxi}{}{[m\~{i}ˈtʃi]}{(v.)}{1}{}{Desarticular.}{}{}
\verb{mintxi}{}{[m\~{i}ˈtʃi]}{(v.)}{2}{}{Deslocar.}{}{}
\verb{mintxi}{}{[m\~{i}ˈtʃi]}{(v.)}{3}{}{Luxar.}{}{}
\verb{mintxidu}{}{[mĩˈtʃidu]}{(adj.)}{1}{}{Desarticulado.}{}{}
\verb{mintxidu}{}{[mĩˈtʃidu]}{(adj.)}{2}{}{Deslocado.}{}{}
\verb{mintxidu}{}{[mĩˈtʃidu]}{(adj.)}{3}{}{Luxado.}{}{}
\verb{mintxidu}{}{[mĩˈtʃidu]}{(n.)}{1}{}{Desarticulação.}{}{}
\verb{mintxidu}{}{[mĩˈtʃidu]}{(n.)}{1}{}{Deslocação.}{}{}
\verb{mintxila}{}{[mĩˈtʃila]}{(n.)}{1}{}{Mentira.}{}{}
\verb{mintxila}{}{[mĩˈtʃila]}{(v.)}{1}{}{Mentir.}{}{}
\verb{minutu}{}{[miˈnutu]}{(n.)}{1}{}{Minuto.}{}{}
\verb{misa}{}{[ˈmisa]}{(n.)}{1}{}{Missa.}{}{}
\verb{misali}{}{[miˈsali]}{(n.)}{1}{}{Missal.}{}{}
\verb{misanga}{}{[miˈs\~{\textturna}ga]}{(n.)}{1}{}{Cortina usada para espantar moscas.}{}{}
\verb{misanga}{}{[miˈs\~{\textturna}ga]}{(n.)}{2}{}{Missanga.}{}{}
\verb{mixagla}{}{[miˈʃagla]}{(n.)}{1}{}{Dobradiça.}{}{}
\verb{mixidadji}{}{[miʃiˈdadʒi]}{(n.)}{1}{}{Carência.}{}{}%
\verb{mixidadji}{}{[miʃiˈdadʒi]}{(n.)}{2}{}{Necessidade.}{}{}
\verb{mixidadji}{}{[miʃiˈdadʒi]}{(n.)}{3}{}{Privação.}{}{}
\verb{mixidadji}{}{[miʃiˈdadʒi]}{(n.)}{4}{}{Pobreza.}{}{}
\verb{mixidaji}{}{[miʃiˈdaʒi]}{(n.)}{1}{}{Carência.}{}{}{}
\verb{mixidaji}{}{[miʃiˈdaʒi]}{(n.)}{1}{}{Necessidade.}{}{}{}
\verb{mixidaji}{}{[miʃiˈdaʒi]}{(n.)}{1}{}{Privação.}{}{}{}
\verb{mixidaji}{}{[miʃiˈdaʒi]}{(n.)}{1}{}{Pobreza.}{Cf. \textbf{mixidadji}.}{}{}
\verb{mixikodji}{}{[miʃiˈkɔdʒi]}{(n.)}{1}{}{Misericórdia.}{}{}
\verb{mixkitu}{}{[miʃˈkitu]}{(n.)}{1}{}{Mosquito.}{}{}
\verb{mixtula}{}{[miʃˈtula]}{(n.)}{1}{}{Mistura.}{}{}
\verb{miza}{}{[miˈza]}{(n.)}{1}{}{Esperma.}{}{}
\verb{mizelya}{}{[miˈzɛlja]}{(n.)}{1}{}{Miséria.}{}{}
\verb{mizonge}{}{[miz\~ɔˈgɛ]}{(n.)}{1}{}{Prato típico feito à base de
verduras e óleo de palma.}{Cf. \textbf{muzonge}.}{}{}
\verb{mlaga}{}{[mlaˈga]}{(n.)}{1}{}{Pâncreas.}{}{}
\verb{mlaga}{}{[mlaˈga]}{(v.)}{1}{}{Amargar.}{}{}
\verb{mlagadu}{}{[mlaˈgadu]}{(adj.)}{2}{}{Amargado.}{}{}
\verb{mlagadu}{}{[mlaˈgadu]}{(adj.)}{1}{}{Amargoso.}{}{}%
\verb{mlagôzu}{}{[mlaˈgozu]}{(n.)}{1}{}{Amargoso.}{\textbf{\textit{Mammea
africana}}.}{}
\verb{mlagu}{}{[ˈmlagu]}{(adj.)}{1}{}{Magro.}{}{}
\verb{mlagu benfebenfe}{}{[ˈmlagu b\~ɛˈfɛb\~ɛˈfɛ]}{(expr.)}{1}{}{Estreito.}{}\verb{mlagu benfebenfe}{}{[ˈmlagu
b\~ɛˈfɛb\~ɛˈfɛ]}{(expr.)}{2}{}{Magricelo.}{}
\verb{mlagu benfebenfe}{}{[ˈmlagu b\~ɛˈfɛb\~ɛˈfɛ]}{(expr.)}{3}{}{Magrinho.}{}\verb{mlagu benfebenfe}{}{[ˈmlagu
b\~ɛˈfɛb\~ɛˈfɛ]}{(expr.)}{4}{}{Raquítico.}{}
\verb{mlagu txeketxeke}{}{[ˈmlagu
tʃɛˈkɛtʃɛˈkɛ]}{(expr.)}{1}{}{Macérrimo.}{}{}
\verb{mlaka}{}{[mlaˈka]}{(v.)}{1}{}{Bordar.}{}{}
\verb{mlaka}{}{[mlaˈka]}{(v.)}{2}{}{Demarcar.}{}{}
\verb{mlaka}{}{[mlaˈka]}{(v.)}{3}{}{Marcar.}{}{}
\verb{mlakason}{}{[mlakaˈs\~ɔ]}{(n.)}{2}{}{Demarcação de terreno.}{}{}
\verb{mlakason}{}{[mlakaˈs\~ɔ]}{(n.)}{1}{}{Fronteira.}{}{}
\verb{mlakason}{}{[mlakaˈs\~ɔ]}{(n.)}{3}{}{Marcação.}{}{}
\verb{mlanjinkon}{}{[ml\~{\textturna}ʒ\~iˈk\~o]}{(n.)}{1}{}{Manjericão.}{\textbf{\textit{Ocimum
minimum}}.}{}
\verb{mlaxka}{}{[ˈmlaʃka]}{(n.)}{1}{}{Máscara.}{}{}
\verb{mlaxka}{}{[mlaʃˈka]}{(v.)}{1}{}{Mascarar.}{}{}
\verb{mlazuga}{}{[mlazuˈga]}{(v.)}{1}{}{Madrugar.}{}{}
\verb{mlazugadu}{}{[mlazuˈgadu]}{(n.)}{1}{}{Madrugada.}{}{}
\verb{mlazugadô}{}{[mlazugaˈdo]}{(n.)}{1}{}{Madrugador.}{}{}
\verb{mo}{}{[ˈmɔ]}{(conj.)}{2}{}{À semelhança de.}{}{}%
\verb{mo}{}{[ˈmɔ]}{(conj.)}{1}{}{Assim como.}{}{}%
\verb{mo}{}{[ˈmɔ]}{(conj.)}{3}{}{Como.}{}{}%
\verb{mo}{}{[ˈmɔ]}{(conj.)}{4}{}{Tal como.}{}{}
\verb{mo}{}{[ˈmɔ]}{(n.)}{1}{}{Maneira.}{}{}{}
\verb{mo}{}{[ˈmɔ]}{(n.)}{1}{}{Modo.}{Cf. \textbf{modu}.}{}{}
\verb{moda}{}{[ˈmɔda]}{(n.)}{1}{}{Costume.}{}{}
\verb{moda}{}{[ˈmɔda]}{(n.)}{2}{}{Forma.}{Cf. \textbf{modu}.}{}{}
\verb{moda}{}{[ˈmɔda]}{(n.)}{3}{}{Hábito.}{}{}
\verb{moda}{}{[ˈmɔda]}{(n.)}{4}{}{Moda.}{}{}
\verb{moda}{}{[ˈmɔda]}{(n.)}{5}{}{Modo.}{}{}
\verb{môdê}{}{[moˈde]}{(v.)}{1}{}{Morder.}{}{}
\verb{modu}{}{[ˈmɔdu]}{(n.)}{1}{}{Maneira.}{}{}%	
\verb{modu}{}{[ˈmɔdu]}{(n.)}{2}{}{Modo.}{}{}%
\verb{modu}{}{[ˈmɔdu]}{(n.)}{3}{}{Forma.}{}{}
\verb{modu ku modu}{}{[ˈmɔdu ku ˈmɔdu]}{(adv.)}{1}{}{De qualquer forma.}{}{}%\verb{modu ku modu}{}{[ˈmɔdu ku ˈmɔdu]}{(adv.)}{2}{}{Seja como for.}{}{}
\verb{modu-modu}{}{[ˈmɔduˈmɔdu]}{(adv.)}{1}{}{De qualquer maneira.}{}{}%
\verb{modu-modu}{}{[ˈmɔduˈmɔdu]}{(adv.)}{2}{}{Desajeitadamente.}{}{}
\verb{mo fala mo klonvesa}{}{[ˈmɔ ˈfala ˈmɔ klõvɛˈsa]}{(adv.)}{1}{}{Digamos.}{}{}
\verb{mo fala mo klonvesa}{}{[ˈmɔ ˈfala ˈmɔ klõvɛˈsa]}{(adv.)}{2}{}{Por exemplo.}{}{}
\verb{mogomogo}{}{[mɔˈgɔmɔˈgɔ]}{(id.)}{1}{}{Cf. \textbf{moli mogomogo.}}{}{}
\verb{mola}{}{[ˈmɔla]}{(n.)}{1}{}{Mola.}{}{}
\verb{mola}{}{[mɔˈla]}{(v.)}{1}{}{Afiar.}{}{}
\verb{mola}{}{[mɔˈla]}{(v.)}{2}{}{Amolar.}{}{}
\verb{mola}{}{[mɔˈla]}{(v.)}{3}{}{Desfazer.}{}{}
\verb{mola}{}{[mɔˈla]}{(v.)}{4}{}{Esmagar.}{}{}
\verb{mola}{}{[mɔˈla]}{(v.)}{5}{}{Moer.}{}{}
\verb{mola}{}{[mɔˈla]}{(v.)}{6}{}{Pisar.}{}{}
\verb{mola}{}{[mɔˈla]}{(v.)}{7}{}{Triturar.}{}{}
\verb{môladô}{}{[molaˈdo]}{(n.)}{1}{}{Cidadão.}{}{}{}
\verb{môladô}{}{[molaˈdo]}{(n.)}{2}{}{Forro pertencente à elite sócio-económica.}{}{}
\verb{môladô}{}{[molaˈdo]}{(n.)}{3}{}{Morador.}{}{}{}
\verb{moladu}{}{[mɔˈladu]}{(adj.)}{1}{}{Amolado.}{}{}%
\verb{moladu}{}{[mɔˈladu]}{(adj.)}{2}{}{Afiado.}{}{}
\verb{môlê}{}{[moˈle]}{(n.)}{1}{}{Morte.}{}{}
\verb{môlê}{}{[moˈle]}{(v.)}{1}{}{Morrer.}{}{}
\verb{molextya}{}{[mɔˈlɛʃtja]}{(n.)}{1}{}{Doença.}{}{}
\verb{molextya}{}{[mɔˈlɛʃtja]}{(n.)}{2}{}{Moléstia.}{}{}
\verb{molextyadu}{}{[mɔlɛʃˈtjadu]}{(adj.)}{1}{}{Doente.}{}{}
\verb{molextyadu}{}{[mɔlɛʃˈtjadu]}{(adj.)}{2}{}{Molestado.}{}{}
\verb{moli}{}{[ˈmɔli]}{(adj.)}{1}{}{Mole.}{}{}
\verb{moli mogomogo}{}{[ˈmɔli mɔˈgɔmɔˈgɔ]}{(expr.)}{1}{}{Molíssimo.}{}{}
\verb{molimoli}{}{[mɔliˈmɔli]}{(adv.)}{1}{}{Devagar.}{}{}
\verb{môlô}{}{[ˈmolo]}{(id.)}{1}{}{Cf. \textbf{zedon môlô.}}{}{}
\verb{môlô}{}{[ˈmolo]}{(n.)}{1}{}{Mouro.}{}{}
\verb{môlôkentxi}{}{[ˈmoloˈkẽtʃi]}{(n.)}{1}{}{Abcesso.}{}{}%
\verb{môlôkentxi}{}{[ˈmoloˈkẽtʃi]}{(n.)}{3}{}{Sífilis.}{}{}
\verb{môlôkentxi}{}{[ˈmoloˈkẽtʃi]}{(n.)}{2}{}{Úlcera.}{}{}
\verb{mo-modu}{}{[mɔˈmɔdu]}{(adv.)}{1}{}{De qualquer maneira.}{}{}{}
\verb{mo-modu}{}{[mɔˈmɔdu]}{(adv.)}{1}{}{Desajeitadamente.}{Cf. \textbf{modu-modu}.}{}{}
\verb{momoli}{}{[mɔˈmɔli]}{(adv.)}{1}{}{Devagar.}{Cf. \textbf{molimoli}.}{}{}
\verb{mon}{}{[ˈmõ]}{(n.)}{2}{}{Braço.}{}{}
\verb{mon}{}{[ˈmõ]}{(n.)}{1}{}{Mão.}{}{}%
\verb{mon}{}{[ˈmõ]}{(n.)}{3}{}{Vez.}{}{}
\verb{mon-betu}{}{[ˈmõ ˈbɛtu]}{(adj.)}{1}{}{Generoso.}{}{}%
\verb{mon-betu}{}{[ˈmõ ˈbɛtu]}{(adj.)}{2}{}{Pródigo.}{}{}
\verb{mon-fisadu}{}{[ˈmõ fiˈsadu]}{(adj.)}{1}{}{Avarento.}{}{}
\verb{mon-fisadu}{}{[ˈmõ fiˈsadu]}{(adj.)}{2}{}{Mão-de-vaca.}{}{}
\verb{mon-fisadu}{}{[ˈmõ fiˈsadu]}{(adj.)}{3}{}{Sovina.}{}{}
\verb{mon-fisadu}{}{[ˈmõ fiˈsadu]}{(adj.)}{4}{}{Somítico.}{}{}
\verb{monha}{}{[m\~{ɔ}ˈɲa]}{(v.)}{1}{}{Molhar.}{}{}
\verb{monhadu}{}{[m\~{ɔ}ˈɲadu]}{(adj.)}{1}{}{Molhado.}{}{}
\verb{monha potopoto}{}{[m\~{ɔ}ˈɲa pɔˈtɔpɔˈtɔ]}{(expr.)}{1}{}{Encharcar.}{}{}
\verb{mon-klaniji}{}{[ˈmõ klaˈniʒi]}{(n.)}{1}{}{Artrose reumática.}{}{}
\verb{mon-kluklu}{}{[ˈmõ kluˈklu]}{(n.)}{1}{}{Braço amputado.}{}{}
\verb{monko}{}{[m\~{ɔ}ˈkɔ]}{(n.)}{1}{}{Moncó.}{Designação pejorativa para os naturais da ilha do Príncipe e a sua língua.}{}
\verb{monko}{}{[m\~{ɔ}ˈkɔ]}{(n.)}{1}{}{Principense.}{}{}
\verb{mon-longô}{}{[ˈmõ ˈlõgo]}{(n.)}{2}{}{Ladrão.}{}{}
\verb{mono}{}{[ˈmɔnɔ]}{(adj.)}{1}{}{Morno.}{}{}
\verb{monsonson}{}{[mõsõˈsõ]}{(n.)}{1}{}{Erva-cacho.}{\textbf{\textit{Paspalum
paniculatum}}.}{}
\verb{monsonson}{}{[mõsõˈsõ]}{(n.)}{2}{}{Massagem terapêutica.}{}{}
\verb{monta}{}{[m\~{ɔ}ˈta]}{(v.)}{1}{}{Entrar em transe.}{}{}
\verb{monta}{}{[m\~{ɔ}ˈta]}{(v.)}{2}{}{Montar.}{}{}
\verb{montalha}{}{[m\~{ɔ}ˈtaʎa]}{(n.)}{1}{}{Mortalha.}{}{}%
\verb{montxa}{}{[m\~{ɔ}ˈtʃa]}{(v.)}{2}{}{Caçar.}{}{} 
\verb{montxa}{}{[m\~{ɔ}ˈtʃa]}{(v.)}{1}{}{Perseguir.}{}{}%
\verb{montxadô}{}{[m\~{ɔ}tʃaˈdo]}{(n.)}{1}{}{Caçador.}{}{}
\verb{montxi}{}{[ˈm\~{ɔ}tʃi]}{(adv.)}{1}{}{Aos montes.}{}{}
\verb{montxi}{}{[ˈm\~{ɔ}tʃi]}{(adv.)}{2}{}{Monte.}{}{}%
\verb{montxi}{}{[ˈm\~{ɔ}tʃi]}{(adv.)}{3}{}{Muito.}{}{}%
\verb{mosa}{}{[ˈmɔsa]}{(n.)}{1}{}{Jovem.}{}{}
\verb{mosa}{}{[ˈmɔsa]}{(n.)}{2}{}{Moça.}{}{}%
\verb{mosu}{}{[ˈmɔsu]}{(n.)}{1}{}{Jovem.}{}{}%
\verb{mosu}{}{[ˈmɔsu]}{(n.)}{1}{}{Moço.}{}{}%
\verb{mosu-kata}{}{[ˈmɔsu ˈkata]}{(n.)}{1}{}{Menino de recados.}{Personagem
do \textbf{Txiloli} que leva a carta de D. Carloto a D. Roldão.}{}{}
\verb{mosu-kya}{}{[ˈmɔsu ˈkja]}{(n.)}{1}{}{Criado.}{}{}
\verb{motali}{}{[mɔˈtali]}{(adj.)}{1}{}{Imundo.}{}{}
\verb{moto}{}{[ˈmɔtɔ]}{(n.)}{1}{}{Motocicleta.}{}{}
\verb{moto di plasa}{}{[ˈmɔtɔ di ˈplasa]}{(n.)}{1}{}{Moto-táxi.}{}{}
\verb{motxi}{}{[ˈmɔtʃi]}{(n.)}{2}{}{Morte.}{}{}%
\verb{motxi}{}{[ˈmɔtʃi]}{(n.)}{3}{}{Morto.}{}{}
\verb{môvê}{}{[moˈve]}{(n.)}{1}{}{Aborto.}{}{}
\verb{môvê}{}{[moˈve]}{(v.)}{1}{}{Abortar.}{}{}
\verb{môvê}{}{[moˈve]}{(v.)}{2}{}{Atrofiar.}{}{}
\verb{môvidu}{}{[moˈvidu]}{(adj.)}{1}{}{Atrofiado.}{}{}
\verb{môvidu}{}{[moˈvidu]}{(adj.)}{2}{}{Enfezado.}{}{}
\verb{moxka}{}{[ˈmɔʃka]}{(n.)}{1}{}{Mosca.}{}{}
\verb{môyô}{}{[ˈmojo]}{(n.)}{1}{}{Molho.}{}{}
\verb{mpali}{}{[ˈmpali]}{(n.)}{1}{}{Par.}{}{}
\verb{mpanampana}{}{[ˈmpan\~{\textturna}ˈpana]}{(n.)}{1}{}{Fatia.}{}{}
\verb{mpavu}{}{[ˈmpavu]}{(n.)}{1}{}{Cobertura de folhas de palmeira.}{}{}%
\verb{mpêlu}{}{[mpeˈlu]}{(n.)}{1}{}{Peru.}{}{}
\verb{mpenampena}{}{[mpɛˈn\~{\textturna}pɛˈna]}{(id.)}{1}{}{Cf. \textbf{blaga mpenampena}.}{}{}
\verb{mpenhu}{}{[ˈmp\~{ɛ}ɲu]}{(n.)}{1}{}{Capricho.}{}{}
\verb{mpenhu}{}{[ˈmp\~{ɛ}ɲu]}{(n.)}{2}{}{Determinação.}{}{}
\verb{mpenhu}{}{[ˈmp\~{ɛ}ɲu]}{(n.)}{3}{}{Empenho.}{}{}
\verb{mpênumpênu}{}{[mpeˈn\~upeˈnu]}{(n.)}{1}{}{Cílios.}{}{}
\verb{mpênumpênu}{}{[mpeˈn\~upeˈnu]}{(n.)}{2}{}{Pestanas.}{}{}%
\verb{mplega}{}{[ˈmplɛga]}{(n.)}{1}{}{Prega.}{}{}
\verb{mplega}{}{[ˈmplɛga]}{(n.)}{2}{}{Ruga.}{}{}
\verb{mpon}{}{[ˈmpõ]}{(n.)}{1}{}{Pão.}{}{}
\verb{mpon-dolo}{}{[ˈmpõ dɔˈlɔ]}{(n.)}{1}{}{Pão-de-ló.}{}{}
\verb{mpyala}{}{[ˈmpjala]}{(n.)}{1}{}{\textit{Mpyala}.}{\textit{\textbf{Olyra latifolia}.}}{}
\verb{mpyan}{}{[ˈmpj\~{\textturna}]}{(n.)}{2}{}{Cacho.}{}{}%
\verb{mpyan}{}{[ˈmpj\~{\textturna}]}{(n.)}{2}{}{Espinha de peixe.}{}{}%
\verb{mpyan}{}{[ˈmpj\~{\textturna}]}{(n.)}{1}{}{Espinho.}{}{}%
\verb{mpyan}{}{[ˈmpj\~{\textturna}]}{(n.)}{3}{}{Pinha.}{}{}%
\verb{mpyan-kabla}{}{[ˈmpj\~{\textturna} ˈkabla]}{(n.)}{1}{}{\textit{Mpyan-kabla}.}{\textbf{\textit{Alternanthera pungens}}.}{}
\verb{mpyan-kana}{}{[ˈmpj\~{\textturna} kaˈna]}{(n.)}{1}{}{Acne.}{}{}
\verb{mpyon}{}{[ˈmpjõ]}{(n.)}{1}{}{Destro.}{}{}
\verb{mpyon}{}{[ˈmpjõ]}{(n.)}{2}{}{Perigoso.}{}{}
\verb{mpyon}{}{[ˈmpjõ]}{(n.)}{3}{}{Pião.}{}{}
\verb{mpyon}{}{[ˈmpjõ]}{(n.)}{4}{}{Redemoinho.}{}{}
\verb{mu}{}{[ˈmu]}{(pron.)}{1}{}{-me.}{Primeira pessoa do singular com a função de complemento direto ou indireto. \textbf{Kê ngê ka ngana mu?} \emph{Quem é que me engana?}}{}{}
\verb{mu}{}{[ˈmu]}{(poss.)}{2}{}{Meu.}{}{}
\verb{mu}{}{[ˈmu]}{(poss.)}{3}{}{Meus.}{}{}
\verb{mu}{}{[ˈmu]}{(poss.)}{4}{}{Minha.}{}{}
\verb{mu}{}{[ˈmu]}{(poss.)}{5}{}{Minhas.}{}{}
\verb{muda}{}{[muˈda]}{(v.)}{1}{}{Mudar.}{}{}
\verb{mufa}{}{[muˈfa]}{(v.)}{1}{}{Mofar.}{}{}
\verb{mufada}{}{[muˈfada]}{(n.)}{1}{}{Almofada.}{Cf. \textbf{munfada}.}{}{}
\verb{mufinu}{}{[muˈfinu]}{(adj.)}{1}{}{Avaro.}{}{}{}
\verb{mufinu}{}{[muˈfinu]}{(adj.)}{1}{}{Sovina.}{Cf. \textbf{munfinu}.}{}{}
\verb{mufuku}{}{[mufuˈku]}{(id.)}{1}{}{Cf. \textbf{pema mufuku.}}{}{} 
\verb{mufuku}{}{[mufuˈku]}{(id.)}{2}{}{Cf. \textbf{plêjida mufuku.}}{}{}
\verb{mukamba}{}{[muk\~{\textturna}ˈba]}{(n.)}{1}{}{Espírito protetor dos vinhateiros.}{}{}%
\verb{mukamba-vlêmê}{}{[muk\~{\textturna}ˈba vleˈme]}{(n.)}{1}{}{\textit{Mukamba-vlêmê}.}{\textbf{\textit{Chlorophora excelsa}.}}{}{}%
\verb{mukambu}{}{[muk\~{\textturna}ˈbu]}{(n.)}{1}{}{\textit{Libô}.}{Cf. \textbf{libô}.}{}{}
\verb{mukluklu}{}{[mukluˈklu]}{(n.)}{1}{}{Elefantíase do escroto.}{}{}
\verb{mukumbli}{}{[mukũˈbli]}{(n.)}{1}{}{\textit{Mukumbli}.}{\textbf{\textit{Lannea welwitschii}.}}{}
\verb{mula}{}{[ˈmula]}{(n.)}{1}{}{Mula.}{}{}
\verb{mulangu}{}{[muˈl\~{\textturna}gu]}{(n.)}{1}{}{Framboesa-brava.}{\textbf{\textit{Rubus rosifolius}.}}{}
\verb{mulangu-d'ôbô}{}{[muˈl\~{\textturna}gu doˈbo]}{(n.)}{1}{}{Amora.}{\textbf{\textit{Rubus pinnatus}.}}{}
\verb{mulata}{}{[muˈlata]}{(adj.)}{1}{}{Mulata.}{}{}
\verb{mulata}{}{[muˈlata]}{(n.)}{1}{}{Mulata.}{}{}
\verb{mulatu}{}{[muˈlatu]}{(adj.)}{1}{}{Mulato.}{}{}
\verb{mulatu}{}{[muˈlatu]}{(n.)}{1}{}{Mulato.}{}{}
\verb{mulatu}{}{[muˈlatu]}{(n.)}{1}{}{Mulato.} {\textbf{\textit{Paranthias furcifer}.}}{}
\verb{mulatu fããã}{}{[muˈlatu ˈf\~{\textturna}\~{\textturna}\~{\textturna}]}{(expr.)}{1}{}{Mulato pálido.}{}{}
\verb{mulela}{}{[muˈlɛla]}{(n.)}{1}{}{Amoreira.}{\textbf{\textit{Milicia excelsa}.}}{}
\verb{mulela}{}{[muˈlɛla]}{(n.)}{1}{}{\textit{Mukamba-vlêmê}.}{\textbf{\textit{Chlorophora excelsa}.}}{}
\verb{mulu}{}{[ˈmulu]}{(n.)}{1}{}{Muro.}{}{}%
\verb{mulu}{}{[ˈmulu]}{(n.)}{2}{}{Parede.}{}{}%
\verb{mulu}{}{[ˈmulu]}{(n.)}{3}{}{Piso.}{}{}
\verb{mumu}{}{[ˈmumu]}{(adj.)}{1}{}{Mudo.}{}{}
\verb{mun}{}{[ˈmũ]}{(poss.)}{1}{}{Cf. \textbf{mu}.}{}{}
%\verb{mun}{}{[ˈmũ]}{(poss.)}{1}{}{Meus.}{Cf. \textbf{mu}.}{}{}
%\verb{mun}{}{[ˈmũ]}{(poss.)}{1}{}{Minha.}{Cf. \textbf{mu}.}{}{}
%\verb{mun}{}{[ˈmũ]}{(poss.)}{1}{}{Minhas.}{Cf. \textbf{mu}.}{}{}
\verb{mun}{}{[ˈmũ]}{(pron.)}{1}{}{Cf. \textbf{mu}.}{}{}
\verb{mundja}{}{[mũˈdʒa]}{(v.)}{1}{}{Estar de pé.}{}{}%
\verb{mundja}{}{[mũˈdʒa]}{(v.)}{2}{}{Ficar de pé.}{}{}
\verb{mundja}{}{[mũˈdʒa]}{(v.)}{3}{}{Parar.}{}{}
\verb{mundjadu}{}{[mũˈdʒadu]}{(adj.)}{1}{}{Estar parado.}{}{}
\verb{mundjadu t\~i\~i\~i}{}{[mũˈdʒadu ˈt\~i\~i\~i]}{(expr.)}{1}{}{Imobilizado.}{}{}
\verb{mundu}{}{[ˈmũdu]}{(n.)}{1}{}{Mundo.}{}{}
\verb{mundu}{}{[ˈmũdu]}{(n.)}{2}{}{Planeta.}{}{}%
\verb{mundu}{}{[ˈmũdu]}{(n.)}{3}{}{Terra.}{}{}%
\verb{munfada}{}{[mũˈfada]}{(n.)}{1}{}{Almofada.}{}{}
\verb{munfinu}{}{[mũˈfinu]}{(adj.)}{1}{}{Avaro.}{}{}%
\verb{munfinu}{}{[mũˈfinu]}{(adj.)}{2}{}{Sovina.}{}{}
\verb{munja}{}{[mũˈʒa]}{(v.)}{1}{}{Estar de pé.}{}{}{}
\verb{munja}{}{[mũˈʒa]}{(v.)}{1}{}{Ficar de pé.}{}{}{}
\verb{munja}{}{[mũˈʒa]}{(v.)}{1}{}{Parar.}{Cf. \textbf{mundja}.}{}{}
\verb{munken}{}{[mũˈk\~{ɛ}]}{(n.)}{1}{}{Pomba-preta.}{\textbf{\textit{Aplopelia larvata simplex}.}}{}
\verb{munsa}{}{[mũˈsa]}{(v.)}{1}{}{Mostrar.}{Cf. \textbf{musa}.}{}{}
\verb{muntu}{}{[ˈmũtu]}{(adv.)}{1}{}{Muito.}{}{}
\verb{musa}{}{[muˈsa]}{(v.)}{1}{}{Mostrar.}{}{}
\verb{musambê}{}{[muˈs\~{\textturna}be]}{(n.)}{1}{}{Peixe salgado.}{}{}
\verb{musampyan}{}{[mus\~{\textturna}ˈpj\~{\textturna}]}{(n.)}{1}{}{\textit{Musampyan}.}{\textbf{\textit{Hibiscus surattensis}}.}{}
\verb{musanda}{}{[mus\~{\textturna}ˈda]}{(n.)}{1}{}{\textit{Musanda}.}{\textbf{\textit{Ficus annobonensis}.}}{}
\verb{musanfi}{}{[muˈs\~{\textturna}fi]}{(n.)}{1}{}{\textit{Musanfi}.}{\textbf{\textit{Cleome rutidosperma}}.}{}
\verb{musinika}{}{[musiˈnika]}{(n.)}{1}{}{Ameixoeira-africana.}{\textbf{\textit{Prunus
africana}}.}{}
\verb{musumba}{}{[muˈsũba]}{(n.)}{1}{}{Timbalão tradicional.}{}{}
\verb{musungu}{}{[muˈsũgu]}{(n.)}{1}{}{Cântaro.}{}{}
\verb{muswa}{}{[ˈmuswa]}{(n.)}{1}{}{Azeda-da-Guiné.}{\textbf{\textit{Hibiscus acetosella}.}}{}
\verb{muswa}{}{[muˈswa]}{(n.)}{1}{}{Armadilha feita de fibra de folha da palmeira utilizada na captura do camarão.}{}{}
\verb{muta}{}{[ˈmuta]}{(n.)}{1}{}{\textit{Muta}.}{Mezinha, preparada com urina envelhecida, utilizada para tratamentos tradicionais, sobretudo para evitar o mau-olhado.}{}{}
\verb{mutambu}{}{[mut\~{\textturna}ˈbu]}{(n.)}{1}{}{Arapuca.}{}{}{}
\verb{mutendê}{}{[mutẽˈde]}{(n.)}{1}{}{Palmeira jovem.}{}{}
\verb{mutendê-d'ôbô}{}{[mutẽˈde doˈbo]}{(n.)}{1}{}{Palmeira-d’obô.}{\textit{\textbf{Mapania ferruginea}}.}{}
\verb{mutete}{}{[mutɛˈtɛ]}{(n.)}{1}{}{Cesto de \textbf{ndala}.}{}{}
\verb{mutopa}{}{[muˈtɔpa]}{(n.)}{1}{}{\textit{Mutopa}.}{\textbf{\textit{Maesa lanceolata}.}}{}
\verb{mutopa}{}{[muˈtɔpa]}{(n.)}{2}{}{Pau-cabra.}{\textbf{\textit{Maesa lanceolata}.}}{}
\verb{mutoru}{}{[muˈtɔru]}{(n.)}{1}{}{Motor.}{}{}
\verb{mutxi}{}{[muˈtʃi]}{(n.)}{1}{}{Motim.}{}{}
\verb{mutxivu}{}{[muˈtʃivu]}{(n.)}{1}{}{Motivo.}{}{}
\verb{muxila}{}{[muˈʃila]}{(n.)}{1}{}{Mochila.}{}{}
\verb{muxinji}{}{[muʃĩˈʒi]}{(n.)}{1}{}{Naco.}{}{}{}
\verb{muxkitu}{}{[muʃˈkitu]}{(n.)}{1}{}{Mosquito.}{Cf. \textbf{mixkitu}.}{}{}
\verb{muxtlada}{}{[muʃˈtlada]}{(n.)}{1}{}{Mostarda.}{\textbf{\textit{Brassica juncea}.}}{}
\verb{muzonge}{}{[muzõˈgɛ]}{(n.)}{1}{}{Caldo de peixe.}{}{}
\verb{muzula}{}{[muˈzula]}{(n.)}{1}{}{Aldrabice.}{}{}
\verb{muzula}{}{[muˈzula]}{(n.)}{2}{}{Fingimento.}{}{}
\verb{muzula}{}{[muˈzula]}{(n.)}{3}{}{Sorna.}{}{}
\verb{mwala}{}{[ˈmwala]}{(n.)}{1}{}{Fêmea.}{}{}
\verb{mwala}{}{[ˈmwala]}{(n.)}{2}{}{Mulher.}{}{}%
\verb{mwala-bega}{}{[ˈmwala ˈbɛga]}{(n.)}{1}{}{Grávida.}{}{}
\verb{mwala-palidu-fili}{}{[ˈmwala paˈlidu ˈfili]}{(n.)}{1}{}{Parturiente.}{}{}
\verb{mwala-sendê-mon-sendê-ope}{}{[ˈmwala s\~eˈde ˈm\~ɔ s\~eˈde ɔˈpɛ]}{(n.)}{1}{}{Mulher-estendeu-mão-estendeu-pé.}{\textbf{\textit{Paspalum conjugatum}}.}{}
\verb{mwandjin}{}{[m\~w\~{\textturna}ˈdʒĩ]}{(n.)}{1}{}{Sucupira.}{\textbf{\textit{Pentaclettra macrophylla}.}}{}
\verb{mwandjin-ome}{}{[m\~w\~{\textturna}ˈdʒĩ ˈɔmɛ]}{(n.)}{1}{}{\textit{Mwandjin-ome}.}{\textbf{\textit{Cnestis ferruginea.}}}{}
\verb{mwe}{}{[ˈmwɛ]}{(v.)}{1}{}{Amolecer.}{}{}
\verb{mwe}{}{[ˈmwɛ]}{(v.)}{2}{}{Domesticar.}{}{}
\verb{mweda}{}{[ˈmwɛda]}{(n.)}{1}{}{Moeda.}{}{}
\verb{mwindlu}{}{[m\~wĩˈdlu]}{(n.)}{1}{}{\textit{Mwindlu}.}{\textbf{\textit{Bridelia micrantha}.}}{}
\verb{mya}{}{[ˈmja]}{(v.)}{1}{}{Abrir as pernas.}{}{}
\verb{myamya}{}{[mjaˈmja]}{(n.)}{1}{}{Relâmpago.}{}{}
\verb{myamyamya}{}{[mjamjaˈmja]}{(id.)}{1}{}{Cf. \textbf{vlêmê myamyamya.}}{}{}
\verb{myawa}{}{[ˈmjawa]}{(n.)}{1}{}{Anexo.}{}{}
\verb{myêgêmyêgê}{}{[mjeˈgemjeˈge]}{(id.)}{1}{}{Cf. \textbf{luji myêgêmyêgê.}}{}{}
\verb{myole}{}{[mjɔˈlɛ]}{(adv.)}{1}{}{Agora.}{}{}%
\verb{myole}{}{[mjɔˈlɛ]}{(adv.)}{2}{}{Neste momento.}{}{}
\verb{myôlô}{}{[mjoˈlo]}{(n.)}{1}{}{Cabeça.}{}{}
\verb{myôlô}{}{[mjoˈlo]}{(n.)}{2}{}{Cérebro.}{}{}%
\verb{myôlô}{}{[mjoˈlo]}{(n.)}{4}{}{Miolo da palmeira.}{}{}
\verb{myôlô}{}{[mjoˈlo]}{(n.)}{5}{}{Miolo de pão.}{}{}
\verb{myôlô}{}{[mjoˈlo]}{(n.)}{3}{}{Miolos.}{}{}
\verb{myôtô-kadela}{}{[ˈmjoto kaˈdɛla]}{(n.)}{1}{}{Ânus.}{}{}
\end{letra}

\begin{letra}{n}

\verb{n}{}{[n]}{(prep.)}{1}{}{De.}{}{}{}
\verb{n}{}{[n]}{(prep.)}{1}{}{Em.}{Cf. \textbf{ni}.}{}{}
\verb{N}{}{[n]}{(pron.)}{1}{}{Eu.}{\textbf{N na xê fa.} \textit{Eu não
saí.}}{}{}
\verb{na}{}{[ˈna]}{(neg.)}{1}{}{Não.}{Partícula de negação pré-verbal
correlativa de \textbf{fa}. \textbf{N na tê mina fa}. \textit{Não tenho
filhos}.}{}{}
\verb{nadaxi}{}{[ˈnadaʃi]}{(indef.)}{1}{}{Nada.}{\textbf{N na fla nadaxi fa}.
\textit{Eu não disse nada}.}{}
\verb{nai}{}{[naˈi]}{(adv.)}{1}{}{Aqui.}{}{}{}
\verb{nai}{}{[naˈi]}{(adv.)}{1}{}{Cá.}{}{}{}
\verb{nala}{}{[naˈla]}{(adv.)}{1}{}{Acolá.}{}{}{}
\verb{nala}{}{[naˈla]}{(adv.)}{1}{}{Lá.}{Cf. \textbf{ala}.}{}{}
\verb{namplakata}{}{[n\~{\textturna}plaˈkata]}{(adv.)}{1}{}{De repente.}{}{}
%\verb{namplakata}{}{[n\~{\textturna}plaˈkata]}{(adv.)}{2}{}{Subitamente.}{}{}\verb{nanaji}{}{[naˈnaʒi]}{(n.)}{1}{}{Abacaxi.}{}{}
\verb{nanaji}{}{[naˈnaʒi]}{(n.)}{2}{}{Ananás.}{}{}
\verb{nanaji}{}{[naˈnaʒi]}{(n.)}{3}{}{Ananaseiro.}{\textit{\textbf{Ananas comosus}}.}{}
\verb{nandji}{}{[ˈn\~{\textturna}dʒi]}{(int.)}{1}{}{Onde.}{\textbf{Nandji ku nansê sa nê?} \textit{Onde é que vocês estão?}}{}
\verb{nankô}{}{[n\~{\textturna}ˈko]}{(adj.)}{1}{}{Encardido.}{}{}
\verb{nansê}{}{[n\~{\textturna}ˈse]}{(n.)}{1}{}{Destino.}{}{}
\verb{nansê}{}{[n\~{\textturna}ˈse]}{(n.)}{2}{}{Nascimento.}{}{}
\verb{nansê}{}{[n\~{\textturna}ˈse]}{(pron.)}{1}{}{Cf. \textbf{inansê}.}{}{}
%\verb{nansê}{}{[n\~{\textturna}ˈse]}{(pron.)}{1}{}{Vós.}{Cf. \textbf{inansê}.}{}{}
%\verb{nansê}{}{[n\~{\textturna}ˈse]}{(pron.)}{1}{}{-vos.}{Cf. \textbf{inansê}.}{}{}
\verb{nansê}{}{[n\~{\textturna}ˈse]}{(v.)}{1}{}{Nascer.}{}{}
\verb{nansolo}{}{[n\~{\textturna}ˈsɔlɔ]}{(n.)}{1}{}{Lençol.}{Cf. \textbf{lansolo}.}{}{}
\verb{nanson}{}{[n\~{\textturna}ˈsõ]}{(n.)}{1}{}{Casta.}{}{}
\verb{nanson}{}{[n\~{\textturna}ˈsõ]}{(n.)}{1}{}{Enorme.}{}{}
\verb{nanson}{}{[n\~{\textturna}ˈsõ]}{(n.)}{2}{}{Espécie.}{}{}
\verb{nanson}{}{[n\~{\textturna}ˈsõ]}{(n.)}{3}{}{Nação.}{}{}
\verb{nanta}{}{[ˈn\~{\textturna}ta]}{(neg.)}{1}{}{Jamais.}{}{}{}
\verb{nanta}{}{[ˈn\~{\textturna}ta]}{(neg.)}{1}{}{Nunca.}{}{}{}
\verb{nanta}{}{[ˈn\~{\textturna}ta]}{(neg.)}{1}{}{Nunca mais.}{Cf. \textbf{nantan}.}{}{}
\verb{nantan}{}{[n\~{\textturna}ˈt\~{\textturna}]}{(neg.)}{1}{}{Jamais.}{}{}
\verb{nantan}{}{[n\~{\textturna}ˈt\~{\textturna}]}{(neg.)}{2}{}{Nunca.}{}{}%
\verb{nantan}{}{[n\~{\textturna}ˈt\~{\textturna}]}{(neg.)}{3}{}{Nunca mais.}{\textbf{Ê nantan ka bila fe bô mali fa}. \textit{Ele nunca mais volta a fazer-te mal}.}{}
\verb{nanve}{}{[n\~{\textturna}ˈvɛ]}{(n.)}{1}{}{Navalha.}{}{}
\verb{nanxidu}{}{[n\~{\textturna}ˈʃidu]}{(adj.)}{1}{}{Nascido.}{}{}
\verb{nasika}{}{[naˈsika]}{(n.)}{1}{}{\textit{Nasika}.}{\textbf{\textit{Amaurocichla bocagii}.}}{}
\verb{nata}{}{[naˈta]}{(n.)}{1}{}{Natal.}{}{}
\verb{natula}{}{[naˈtula]}{(n.)}{1}{}{Época.}{}{}
\verb{nave}{}{[naˈvɛ]}{(n.)}{1}{}{Navalha.}{Cf. \textbf{nanve}.}{}
\verb{navega}{}{[navɛˈga]}{(v.)}{1}{}{Navegar.}{}{}
\verb{navegadô}{}{[navɛgaˈdo]}{(n.)}{1}{}{Navegador.}{}{}
\verb{navin}{}{[naˈvĩ]}{(n.)}{1}{}{Alma penada.}{}{}
\verb{navin}{}{[naˈvĩ]}{(n.)}{2}{}{Navio.}{}{}
\verb{nawa}{}{[ˈnawa]}{(n.)}{1}{}{Anágua.}{}{}
\verb{nawa}{}{[ˈnawa]}{(n.)}{2}{}{Saiote.}{}{}
\verb{naxi}{}{[ˈnaʃi]}{(neg.)}{1}{}{Ainda não.}{\textbf{N naxi tê mwala fa}. \textit{Ainda não tenho mulher}.}{}
\verb{nda}{}{[ˈnda]}{(v.)}{1}{}{Andar.}{}{}
\verb{nda}{}{[ˈnda]}{(v.)}{2}{}{Arrepiar.}{}{}
\verb{ndadu}{}{[ˈndadu]}{(adj.)}{1}{}{Andado.}{}{}
\verb{ndaga}{}{[ndaˈga]}{(v.)}{1}{}{Indagar.}{}{}%
\verb{ndaga}{}{[ndaˈga]}{(v.)}{2}{}{Investigar.}{}{}%
\verb{ndaga}{}{[ndaˈga]}{(v.)}{3}{}{Sondar.}{}{}
\verb{ndaga}{}{[ndaˈga]}{(v.)}{4}{}{Vasculhar.}{}{}%
\verb{ndakla}{}{[ndaˈkla]}{(n.)}{1}{}{Lacrau.}{}{}
\verb{ndala}{}{[ˈndala]}{(n.)}{1}{}{Ramos de coqueiro ou palmeira.}{}{}
\verb{ndependenxa}{}{[ndɛp\~ɛˈd\~ɛʃa]}{(n.)}{1}{}{Independência.}{Cf. \textbf{dependenxa}.}{}{}
\verb{ndêwa}{}{[ˈndewa]}{(n.)}{1}{}{Vagina.}{}{}
\verb{ndika}{}{[ndiˈka]}{(v.)}{1}{}{Apontar.}{}{}
\verb{ndika}{}{[ndiˈka]}{(v.)}{2}{}{Indicar.}{}{}
\verb{ndombo}{}{[nd\~ɔˈbɔ]}{(n.)}{1}{}{Folhas tenras da palmeira.}{}{}
\verb{ndonkli}{}{[ˈnd\~ɔkli]}{(id.)}{1}{}{Cf. \textbf{zedu ndonkli.}}{}{}
\verb{ndufa}{}{[nduˈfa]}{(n.)}{1}{}{Adufe.}{}{}
\verb{ndufa}{}{[nduˈfa]}{(n.)}{1}{}{Baqueta.}{}{}
\verb{ndufa}{}{[nduˈfa]}{(v.)}{1}{}{Bater.}{}{}
\verb{nduka}{}{[nduˈka]}{(v.)}{1}{}{Educar.}{}{}
\verb{ndukadu}{}{[nduˈkadu]}{(adj.)}{1}{}{Educado.}{}{}
\verb{ndukason}{}{[ndukaˈsõ]}{(n.)}{1}{}{Educação.}{}{}
\verb{ndumba}{}{[ˈnd\~uba]}{(adj.)}{1}{}{Grande.}{}{}
\verb{nê}{}{[ˈne]}{(conj.)}{1}{}{E.}{\textbf{Êlê so sa kapataji di tudu ngê ku sa vivu nê ku sa motxi}. \textit{Ele é que é o capataz de todos os que estão vivos e que estão mortos}.}{}
\verb{nê}{}{[ˈne]}{(conj.)}{2}{}{Inclusive.}{\textbf{Yô-yô ngê, nê minixtlu, ku a mata}. \textit{Muitas pessoas, inclusive ministros, foram mortas}.}{}%
\verb{nê}{}{[ˈne]}{(conj.)}{3}{}{Nem.}{\textbf{Nê zuji nê avogadu na pô blaga demanda se fa}. \textit{Nem o juiz, nem o advogado conseguiram solucionar a questão}.}{}%
\verb{nêblina}{}{[neˈblina]}{(n.)}{1}{}{Neblina.}{}{}%
\verb{nêblina}{}{[neˈblina]}{(n.)}{2}{}{Nuvem.}{}{}
\verb{nega}{}{[nɛˈga]}{(v.)}{1}{}{Negar.}{}{}%
\verb{nega}{}{[nɛˈga]}{(v.)}{2}{}{Proibir.}{}{}%
\verb{nega}{}{[nɛˈga]}{(v.)}{3}{}{Recusar.}{}{}
\verb{negla}{}{[ˈnɛgla]}{(n.)}{1}{}{Negra.}{}{}%
\verb{neglu}{}{[ˈnɛglu]}{(n.)}{1}{}{Negro.}{}{}%
\verb{nen}{}{[ˈnẽ]}{(pron.)}{1}{}{Cf. \textbf{inen}.}{}{}
%\verb{nen}{}{[ˈnẽ]}{(pron.)}{1}{}{Elas.}{Cf. \textbf{inen}.}{}{}
%\verb{nen}{}{[ˈnẽ]}{(pron.)}{1}{}{Eles.}{Cf. \textbf{inen}.}{}{}
%\verb{nen}{}{[ˈnẽ]}{(pron.)}{1}{}{-lhes.}{Cf. \textbf{inen}.}{}{}
%\verb{nen}{}{[ˈnẽ]}{(pron.)}{1}{}{-os.}{Cf. \textbf{inen}.}{}{}
\verb{nene}{}{[nɛˈnɛ]}{(n.)}{1}{}{Bebê.}{}{}
\verb{nene}{}{[nɛˈnɛ]}{(n.)}{2}{}{Boneca.}{}{}
\verb{nengla}{}{[nẽˈgla]}{(v.)}{1}{}{Enfeitar.}{}{}
\verb{neni}{}{[ˈn\~ɛni]}{(n.)}{1}{}{Anel.}{}{}
\verb{nen-ke-mu}{}{[ˈnẽ ˈkɛ ˈmu]}{(n.)}{1}{}{Antepassados familiares.}{}{}
\verb{nen-ke-mu}{}{[ˈnẽ ˈkɛ ˈmu]}{(n.)}{1}{}{Família.}{}{}
\verb{nen-ke-mu}{}{[ˈnẽ ˈkɛ ˈmu]}{(n.)}{1}{}{Gentes.}{}{}
\verb{nentxi}{}{[ˈnẽtʃi]}{(n.)}{1}{}{Gente.}{Cf. \textbf{zentxi}.}{}{}
\verb{netu-ome}{}{[ˈnɛtu ˈɔmɛ]}{(n.)}{1}{}{Neto.}{}{}
\verb{netu-mwala}{}{[ˈnɛtuˈmwala]}{(n.)}{1}{}{Neta.}{}{}
\verb{nê \~{u}a}{}{[ˈne ˈ\~{u}a]}{(quant.)}{1}{}{Nenhum.}{\textbf{N naxi bêbê nê \~{u}a tampa kaxalamba plaman se fa.} \emph{Ainda não bebi nenhuma tampa de cacharamba esta manhã.}}{}{}
%\verb{nê \~{u}a}{}{[ˈne ˈ\~{u}a]}{(quant.)}{2}{}{Nenhuma.}{}{}
\verb{nê \~{u}a ngê}{}{[ˈne ˈ\~{u}a ˈŋge]}{(indef.)}{1}{}{Ninguém.}{\textbf{N na mêsê nê \~ua ngê fa}. \textit{Não quero ninguém}.}{}
\verb{neva}{}{[ˈnɛva]}{(n.)}{1}{}{Noiva.}{}{}
\verb{neva}{}{[nɛˈva]}{(v.)}{1}{}{Alinhavar.}{}{}
\verb{nêxpla}{}{[ˈneʃpla]}{(n.)}{1}{}{Nêspera.}{}{}
\verb{nêxpla}{}{[ˈneʃpla]}{(n.)}{2}{}{Nespereira.}{\textbf{\textit{Eriobotrya japonica}}.}{}
\verb{nêxpla-d'ôbô}{}{[ˈneʃpla doˈbo]}{(n.)}{1}{}{Nêspera-do-bosque.}{\textbf{\textit{Uapaca guineensis}}.}{}
\verb{nfelumu}{}{[ˈnfɛlumu]}{(n.)}{1}{}{Doente.}{}{}
\verb{nfelumu}{}{[ˈnfɛlumu]}{(n.)}{2}{}{Enfermo.}{}{}
\verb{nfenu}{}{[ˈnfɛnu]}{(n.)}{1}{}{Inferno.}{}{}
\verb{nflimêla}{}{[nfliˈmela]}{(n.)}{1}{}{Enfermeira.}{}{}
\verb{nflimêlu}{}{[nfliˈmelu]}{(n.)}{1}{}{Enfermeiro.}{}{}
\verb{ngama}{}{[ŋgaˈma]}{(n.)}{2}{}{Gamela.}{}{}
\verb{ngamala}{}{[ŋgaˈmala]}{(n.)}{1}{}{Gamela.}{Cf. \textbf{ngama}.}{}{}
\verb{ngaman}{}{[ŋgaˈm\~{\textturna}]}{(n.)}{1}{}{Gamela.}{Cf. \textbf{ngama}.}{}{}
\verb{ngana}{}{[ˈŋgana]}{(n.)}{1}{}{Gana.}{}{}
\verb{ngana}{}{[ˈŋgana]}{(n.)}{2}{}{Ímpeto.}{}{}
\verb{ngana}{}{[ŋgaˈna]}{(v.)}{1}{}{Enganar(-se).}{}{}
\verb{nganadu}{}{[ŋgaˈnadu]}{(adj.)}{1}{}{Enganado.}{}{}
\verb{ngandu}{}{[ŋg\~{\textturna}ˈdu]}{(n.)}{1}{}{Tubarão.}{}{}
\verb{ngandu-d'alya}{}{[ŋg\~{\textturna}ˈdu daˈlja]}{(n.)}{1}{}{Tubarão-areia.}{}{}
\verb{ngandu-futu}{}{[ŋg\~{\textturna}ˈdu fuˈtu]}{(n.)}{1}{}{Iguaria preparada à base de carne de tubarão fermentada e cozida.}{}{}
\verb{ngandu-kwa-kota}{}{[ŋg\~{\textturna}ˈdu ˈkwa kɔˈta]}{(n.)}{1}{}{Tubarão-serra.}{}{}
\verb{ngandu-toto}{}{[ŋg\~{\textturna}ˈdu ˈtɔtɔ]}{(n.)}{1}{}{Tubarão-martelo.}{}{}
\verb{nganha}{}{[ŋg\~{\textturna}ˈɲa]}{(n.)}{1}{}{Galinha.}{}{}
\verb{nganha}{}{[ŋg\~{\textturna}ˈɲa]}{(v.)}{1}{}{Alcançar.}{}{}
\verb{nganha}{}{[ŋg\~{\textturna}ˈɲa]}{(v.)}{1}{}{Arrecadar.}{}{}
\verb{nganha}{}{[ŋg\~{\textturna}ˈɲa]}{(v.)}{2}{}{Chegar.}{}{}
\verb{nganha}{}{[ŋg\~{\textturna}ˈɲa]}{(v.)}{3}{}{Ganhar.}{}{}
\verb{nganha-balele}{}{[ŋgaˈɲa baˈlɛlɛ]}{(n.)}{1}{}{Galinha-garnisé.}{}{}
\verb{nganhadu}{}{[ŋg\~{\textturna}ˈɲadu]}{(adj.)}{1}{}{Chegado.}{}{}
\verb{nganha-mosa}{}{[ŋgaˈɲ\~{\textturna} ˈmɔsa]}{(n.)}{1}{}{Franga.}{}{}
\verb{nganhan}{}{[ŋgaˈɲ\~{\textturna}]}{(n.)}{1}{}{Galinha.}{Cf.
\textbf{nganha}.}{}{}
\verb{nganha-ngene}{}{[ŋg\~{\textturna}ˈɲ\~{\textturna} ŋgɛˈnɛ]}{(n.)}{2}{}{Galinha d'Angola.}{\textbf{\textit{Numida meleagris}}.}{}%
\verb{nganha-ngene}{}{[ŋg\~{\textturna}ˈɲ\~{\textturna} ŋgɛˈnɛ]}{(n.)}{1}{}{Galinha-da-Guiné.}{\textbf{\textit{Numida meleagris}}.}{}%
\verb{nganha-ngene}{}{[ŋg\~{\textturna}ˈɲ\~{\textturna} ŋgɛˈnɛ]}{(n.)}{3}{}{Perdiz.}{}{}
\verb{nganhu}{}{[ˈŋg\~{\textturna}ɲu]}{(n.)}{1}{}{Ganho.}{}{}
\verb{nganozu}{}{[ŋgaˈnɔzu]}{(adj.)}{1}{}{Enganoso.}{}{}
\verb{ngansa}{}{[ŋg\~{\textturna}ˈsa]}{(v.)}{1}{}{Agarrar.}{}{}
\verb{ngansa}{}{[ŋg\~{\textturna}ˈsa]}{(v.)}{2}{}{Enganchar.}{}{}
\verb{ngansu}{}{[ˈŋg\~{\textturna}su]}{(n.)}{1}{}{Gancho.}{}{}
\verb{ngê}{}{[ˈŋge]}{(n.)}{1}{}{Gente.}{}{}
\verb{ngê}{}{[ˈŋge]}{(n.)}{1}{}{Pessoa.}{}{}
\verb{ng\~e\~e\~e}{}{[ˈŋg\~ɛ\~ɛ\~ɛ]}{(id.)}{1}{}{Cf. \textbf{lêdidu ng\~e\~e\~e.}}{}{}
\verb{ngê-gôdô}{}{[ˈŋge ˈgodo]}{(adj.)}{1}{}{Influente.}{}{}
\verb{ngê-gôdô}{}{[ˈŋge ˈgodo]}{(adj.)}{1}{}{Poderoso.}{}{}
\verb{ngê-gôdô}{}{[ˈŋge ˈgodo]}{(n.)}{1}{}{Rico.}{}{}
\verb{ngêlê}{}{[ŋgeˈle]}{(n.)}{1}{}{\textit{Ngêlê}.}{\textbf{\textit{Warneckea memecyloides}.}}{}
\verb{ngê-lwa}{}{[ˈŋge ˈlwa]}{(n.)}{1}{}{Estranho.}{}{}
\verb{ngembu}{}{[ŋgẽˈbu]}{(n.)}{1}{}{Morcego.}{\textbf{\textit{Myonycteris brachycephala}.}}{}
\verb{ngen}{}{[ˈŋgẽ]}{(n.)}{1}{}{Galho.}{}{}
\verb{ngen}{}{[ˈŋgẽ]}{(n.)}{2}{}{Penca.}{}{}
\verb{ngene}{}{[ŋgɛˈnɛ]}{(n.)}{1}{}{\textit{Ngene}.}{Instrumento musical de cordas.}{}
\verb{ngenengene}{}{[ŋgɛˈnɛ ŋgɛˈnɛ]}{(id.)}{1}{}{Cf. \textbf{wê ngenengene.}}{}{}
\verb{ngenge}{}{[ŋg\~ɛˈgɛ]}{(n.)}{1}{}{Copo feito com o invólucro da flor de coqueiro, usado pelos vinhateiros.}{}{}
\verb{ngê-tamen}{}{[ˈŋge ˈtamẽ]}{(n.)}{1}{}{Adulto.}{}{}
\verb{ngê-tamen}{}{[ˈŋge ˈtamẽ]}{(n.)}{2}{}{Idoso.}{}{}
\verb{ngê-tamen}{}{[ˈŋge ˈtamẽ]}{(n.)}{2}{}{Pessoa influente.}{}{}
\verb{ngê-tamen}{}{[ˈŋge ˈtamẽ]}{(n.)}{2}{}{Pessoa prestigiada.}{}{}
\verb{nginda}{}{[ŋgĩˈda]}{(v.)}{1}{}{Atar.}{}{}
\verb{nginda}{}{[ŋgĩˈda]}{(v.)}{2}{}{Guindar.}{}{}
\verb{nginhon}{}{[ŋgĩˈɲõ]}{(n.)}{1}{}{Agrião.}{Cf. \textbf{fya-nginhon}.}{}{}\verb{nginhon}{}{[ŋgĩˈɲõ]}{(n.)}{2}{}{Corrente.}{}{}
\verb{nginhon}{}{[ŋgĩˈɲõ]}{(n.)}{3}{}{Grilhão.}{}{}
\verb{nginhon-matu}{}{[ŋgĩˈɲõ
ˈmatu]}{(n.)}{1}{}{Agrião-do-mato.}{\textbf{\textit{Peperomia pellucida}}.}{}
\verb{ngipa}{}{[ngiˈpa]}{(n.)}{1}{}{\textit{Ngipa}.}{Instrumento musical,
típico dos Angolares.}{}
\verb{nglandji}{}{[ˈŋgl\~{\textturna}dʒi]}{(adj.)}{1}{}{Grande.}{}{}
\verb{nglatu}{}{[ˈŋglatu]}{(adj.)}{1}{}{Ingrato.}{}{}
\verb{nglêji}{}{[ˈŋgleʒi]}{(adj.)}{1}{}{Inglês.}{}{}
\verb{nglêji}{}{[ˈŋgleʒi]}{(n.)}{1}{}{Inglês.}{}{}
\verb{nglêji-pletu}{}{[ˈŋgleʒi ˈplɛtu ]}{(n.)}{1}{}{Inglês-negro.}{Nome dado
a africanos de língua inglesa no período colonial.}{}
\verb{nglentu}{}{[ˈŋglẽtu]}{(n.)}{1}{}{Interior.}{}{}
\verb{nglentu}{}{[ˈŋglẽtu]}{(n.)}{2}{}{Quarto.}{}{}
\verb{nglentu}{}{[ˈŋglẽtu]}{(prep. n.)}{1}{}{Dentro (de).}{}{}
\verb{nglentu-ke}{}{[ˈŋglẽtu ˈkɛ]}{(n.)}{1}{}{Portas adentro.}{\textbf{Numigu
nglentu-ke}. \textit{Inimigo portas adentro}.}{}
\verb{nglexti}{}{[ˈŋglɛʃti]}{(adj.)}{1}{}{Áspero.}{}{}
\verb{nglexti}{}{[ˈŋglɛʃti]}{(adj.)}{2}{}{Picante.}{}{}
\verb{nglexti}{}{[ˈŋglɛʃti]}{(adj.)}{3}{}{Temperamental.}{}{}
\verb{nglexti}{}{[ˈŋglɛʃti]}{(adj.)}{4}{}{Terrível.}{}{}
\verb{ngleva}{}{[ˈŋglɛva]}{(adj.)}{1}{}{Gémeo.}{Cf. \textbf{ingleva}.}{}{}
\verb{nglimi}{}{[ŋgliˈmi]}{(n.)}{1}{}{Labaredas.}{}{}
\verb{nglimi}{}{[ŋgliˈmi]}{(v.)}{1}{}{Atiçar o fogo.}{}{}%
\verb{nglimi}{}{[ŋgliˈmi]}{(v.)}{2}{}{Encolerizar(-se).}{}{}%
\verb{nglimi}{}{[ŋgliˈmi]}{(v.)}{3}{}{Por fogo em.}{}{}%
\verb{nglimi}{}{[ŋgliˈmi]}{(v.)}{4}{}{Subir de tom.}{}{}
\verb{nglon}{}{[ˈŋgl\~ɔ]}{(n.)}{1}{}{Grão.}{}{}
\verb{nglon-floli}{}{[ˈŋgl\~ɔ ˈflɔli ]}{(n.)}{1}{}{Flor-de-coral.}{\textbf{\textit{Jatropha multifida}.}}{}
\verb{nglon-kongô-mwala}{}{[ˈŋgl\~ɔ ˈkõgo ˈmwala
]}{(n.)}{1}{}{Flor-de-coral.}{}{}
\verb{nglon-pluga}{}{[ˈŋgl\~ɔ
ˈpluga]}{(n.)}{1}{}{Purgueira.}{\textbf{\textit{Jatropha curcas}.}}{}
\verb{nglopi}{}{[ˈŋglɔpi]}{(n.)}{1}{}{Gole.}{}{}
\verb{nglopi}{}{[ˈŋglɔpi]}{(n.)}{1}{}{Trago.}{}{}
\verb{ngloya}{}{[ˈŋglɔja]}{(n.)}{1}{}{Glória.}{}{}
\verb{ngluda}{}{[ŋgluˈda]}{(v.)}{1}{}{Grudar.}{}{}%
\verb{ngola}{}{[ŋgɔˈla]}{(n.)}{1}{}{Angolar.}{Grupo étnico Angolar
de São Tomé e Príncipe.}{}
\verb{Ngola}{}{[ˈŋgɔla]}{(top.)}{1}{}{Angola.}{}{}
\verb{ngoma}{}{[ŋgɔˈma]}{(n.)}{1}{}{\textit{Ngoma}.}{Ritual no qual os
participantes entram em transe ao ritmo dos tambores.}{}{}
\verb{ngoma}{}{[ŋgɔˈma]}{(n.)}{1}{}{Rico.}{}{}
\verb{ngoma}{}{[ŋgɔˈma]}{(v.)}{1}{}{Engomar.}{}{}
\verb{ngomadu}{}{[ŋgɔˈmadu]}{(adj.)}{1}{}{Engomado.}{}{}
\verb{ngombe}{}{[ŋg\~ɔˈbɛ]}{(n.)}{1}{}{Cambalhota.}{}{}%
\verb{ngombe}{}{[ŋg\~ɔˈbɛ]}{(n.)}{2}{}{Movimento rápido.}{}{}
\verb{ngomitu}{}{[ˈŋgɔmitu]}{(n.)}{1}{}{Vômito.}{}{}
\verb{ngon}{}{[ˈŋgõ]}{(n.)}{1}{}{Primeiras bananas de uma penca.}{}{}
\verb{ngongô}{}{[ŋgõˈgo]}{(n.)}{1}{}{Caneca feita de fibra da coqueiro.}{}{}
\verb{ngoxta}{}{[ŋgɔʃˈta]}{(v.)}{1}{}{Adorar.}{}{}
\verb{ngoxta}{}{[ŋgɔʃˈta]}{(v.)}{2}{}{Amar.}{}{}%
\verb{ngoxta}{}{[ŋgɔʃˈta]}{(v.)}{3}{}{Apreciar.}{}{}%
\verb{ngoxta}{}{[ŋgɔʃˈta]}{(v.)}{4}{}{Gostar.}{}{}%
\verb{ngugu}{}{[ˈŋgugu]}{(n.)}{1}{}{Perdiz.}{}{}
\verb{ngugu}{}{[ˈŋgugu]}{(n.)}{2}{}{Promíscuo.}{}{}
\verb{nguli}{}{[ˈŋguli]}{(n.)}{1}{}{Espíritos maus.}{}{}
\verb{nguli}{}{[ˈŋguli]}{(n.)}{2}{}{Insaciável.}{}{}
\verb{nguli}{}{[ˈŋguli]}{(n.)}{3}{}{Ogro.}{}{}
\verb{nguli}{}{[ˈŋguli]}{(n.)}{4}{}{Sôfrego.}{}{}
\verb{nguli}{}{[ŋguˈli]}{(v.)}{1}{}{Engolir.}{}{}
\verb{nguli wê}{}{[ŋguˈli ˈwe]}{(v.)}{1}{}{Morrer.}{}{}
\verb{ngumba}{}{[ŋgũˈba]}{(n.)}{1}{}{Amendoim.}{\textbf{\textit{Arachis
hypogaea}.}}{}
\verb{ngumi}{}{[ˈŋgumi]}{(n.)}{1}{}{Teto.}{}{}
\verb{ngumita}{}{[ŋgumiˈta]}{(n.)}{1}{}{Vômito.}{}{}
\verb{ngumita}{}{[ŋgumiˈta]}{(v.)}{1}{}{Vomitar.}{}{}
\verb{ngunda}{}{[ˈŋgũda]}{(n.)}{1}{}{Engodo.}{}{}%
\verb{ngunda}{}{[ˈŋgũda]}{(n.)}{2}{}{Isca.}{}{}
\verb{ngunda}{}{[ŋgũˈda]}{(v.)}{1}{}{Acalmar.}{}{}
\verb{ngunda}{}{[ŋgũˈda]}{(v.)}{2}{}{Acarinhar.}{}{}%
\verb{ngunda}{}{[ŋgũˈda]}{(v.)}{3}{}{Aconselhar.}{}{}
\verb{ngunda}{}{[ŋgũˈda]}{(v.)}{4}{}{Conquistar.}{}{}
\verb{ngunda}{}{[ŋgũˈda]}{(v.)}{5}{}{Embalar.}{}{}
\verb{ngunda}{}{[ŋgũˈda]}{(v.)}{6}{}{Engodar.}{}{}
\verb{ngunda}{}{[ŋgũˈda]}{(v.)}{6}{}{Mimar.}{}{}
\verb{ngungunu}{}{[ŋgũguˈnu]}{(v.)}{1}{}{Murmurar.}{}{}
\verb{ngungunu}{}{[ŋgũguˈnu]}{(v.)}{1}{}{Resmungar.}{}{}
\verb{ngunhadu}{}{[ŋguˈɲadu]}{(adj.)}{1}{}{Encolhido.}{}{}
\verb{ngunu}{}{[ŋguˈnu]}{(n.)}{3}{}{Facho.}{}{}
\verb{ngunu}{}{[ŋguˈnu]}{(n.)}{1}{}{Lampião.}{}{}%
\verb{ngunu}{}{[ŋguˈnu]}{(n.)}{2}{}{Tocha.}{}{}
\verb{ngwalia}{}{[ŋgwaˈlia]}{(n.)}{1}{}{Iguaria.}{}{}
\verb{ngwangwangwan}{}{[ŋgw\~{\textturna}gw\~{\textturna}ˈgw\~{\textturna}]}{(id.)}{1}{}{Cf.
\textbf{solo mê-dja ngwangwangwan.}}{}{}
\verb{ngwenta}{}{[ŋgwẽˈta]}{(v.)}{1}{}{Aguentar.}{}{}%
\verb{ngwenta}{}{[ŋgwẽˈta]}{(v.)}{2}{}{Resistir.}{}{}
\verb{ngweva}{}{[ˈŋgwɛva]}{(n.)}{1}{}{Goiaba.}{}{}
\verb{ngweva}{}{[ˈŋgwɛva]}{(n.)}{2}{}{Goiabeira.}{\textit{\textbf{Psidium
guajava}}.}{}{}
\verb{ngyon}{}{[ˈŋgj\~ɔ]}{(onom.)}{1}{Som de engolir algo.}{}{}{}
\verb{nha}{}{[ˈɲa]}{(n.)}{1}{}{Lenha.}{}{}
\verb{nhami}{}{[ˈɲ\~{\textturna}mi]}{(n.)}{1}{}{Inhame.}{}{}
\verb{nhami-bini}{}{[ˈɲ\~{\textturna}mi
biˈni]}{(n.)}{1}{}{\textit{Inhame-bini}.}{}{}
\verb{nhami-blanku}{}{[ˈɲ\~{\textturna}mi
ˈbl\~{\textturna}ku]}{(n.)}{1}{}{Inhame-branco.}{\textit{\textbf{Discorea cayenensis}.}}{}
\verb{nhami-gundu}{}{[ˈɲ\~{\textturna}mi g\~uˈdu]}{(n.)}{1}{}{\textit{Inhame-gundu}.}{\textit{\textbf{Discorea
alata}.}}{}
\verb{nhami-gundu-sangi}{}{[ˈɲ\~{\textturna}mi g\~uˈduˈs\~{\textturna}gi]}{(n.)}{1}{}{\textit{Inhame-gundu-sangi}.}{}{}
\verb{nhami-klobo}{}{[ˈɲ\~{\textturna}mi ˈklɔbɔ]}{(n.)}{1}{}{\textit{Inhame-klobo}.}{}{}
\verb{nhami-kwini}{}{[ˈɲ\~{\textturna}mi ˈkwini]}{(n.)}{1}{}{Inhame-selvagem.}{\textit{\textbf{Dioscorea dumetorum}}.}{}
\verb{nhami-ngêlêwa}{}{[ˈɲami ŋgeleˈwa]}{(n.)}{1}{}{\textit{Inhame-ngêlêwa}.}{}{}
\verb{nhami-ofo}{}{[ˈɲami ɔˈfɔ]}{(n.)}{1}{}{Inhame-selvagem.}{\textbf{\textit{Dioscorea bulbifera}}.}{}{}
\verb{nhami-otoni-liba-kafe}{}{[ˈɲami ɔˈtɔni ˈliba kaˈfɛ]}{(n.)}{1}{}{\textit{Inhame-otoni-liba-kafe}.}{}{}
\verb{nhami-sangi}{}{[ˈɲ\~{\textturna}mi ˈs\~{\textturna}gi]}{(n.)}{1}{}{\textit{Inhame-sangi}.}{}{}
\verb{nhami-son-longô}{}{[ˈɲ\~{\textturna}mi ˈsõ ˈlõgo]}{(n.)}{1}{}{\textit{Inhame-son-longô}.}{}{}
\verb{nhami-zambluku}{}{[ˈɲ\~{\textturna}mi z\~{\textturna}ˈbluku]}{(n.)}{1}{}{Inhame-amarelo.}{\textit{\textbf{Dioscorea
cayenensis}.}}{}
\verb{nhamu}{}{[ɲaˈmu]}{(onom.)}{1}{}{Som que se faz ao comer.}{}{}
\verb{nhamunhamu}{}{[ɲaˈmuɲaˈmu]}{(onom.)}{1}{}{Som que se faz ao comer.}{Cf. \textbf{nhamu}.}{}{}
\verb{nhanga}{}{[ɲ\~{\textturna}ˈga]}{(v.)}{1}{}{Desarticular.}{}{}
\verb{nhange-d'ôbô}{}{[ɲãˈgɛ doˈbo]}{(n.)}{1}{}{\emph{Nhange-d'ôbô}.}{\textbf{\textit{Marattia fraxinia}}.}{}
\verb{nhanhadu}{}{[ɲaˈɲadu]}{(adj.)}{1}{}{Bêbado.}{}{}
\verb{nhanhanha}{}{[ɲaɲaˈɲa]}{(id.)}{1}{}{Cf. \textbf{filidu nhanhanha.}}{}{}
\verb{nhe}{}{[ˈɲɛ]}{(v.)}{1}{}{Calcar.}{}{}
\verb{nhe}{}{[ˈɲɛ]}{(v.)}{2}{}{Comprimir.}{}{}
\verb{nhe}{}{[ˈɲɛ]}{(v.)}{3}{}{Esmagar.}{}{}
\verb{nhe}{}{[ˈɲɛ]}{(v.)}{3}{}{Espremer.}{}{}
\verb{nhe}{}{[ˈɲɛ]}{(v.)}{4}{}{Pressionar.}{}{}%
\verb{nhendlu}{}{[ɲẽˈdlu]}{(n.)}{1}{}{Morcego.}{Cf. \textbf{ngembu}.}{}{}
\verb{nhe pê}{}{[ˈɲɛ ˈpe]}{(expr.)}{1}{}{Assinalar.}{}
\verb{nhe pê}{}{[ˈɲɛ ˈpe]}{(expr.)}{2}{}{Calcar.}{}
\verb{nho}{}{[ˈɲɔ]}{(v.)}{1}{}{Ausentar(-se).}{}{}
\verb{nho}{}{[ˈɲɔ]}{(v.)}{2}{}{Faltar.}{\textbf{Djêlu nho}. \textit{Falta dinheiro}.}{}
\verb{nho}{}{[ˈɲɔ]}{(v.)}{3}{}{Não estar.}{\textbf{Non nho ke}. \textit{Não estamos em casa}.}{}
\verb{nho}{}{[ˈɲɔ]}{(v.)}{4}{}{Não existir.}{}{}
\verb{nhongônhongô}{}{[ɲõˈgoɲõˈgo]}{(adj.)}{1}{}{Apalermado.}{}{}%
\verb{nhongônhongô}{}{[ɲõˈgoɲõˈgo]}{(adj.)}{2}{}{Idiota.}{}{}
\verb{nhongônhongô}{}{[ɲõˈgoɲõˈgo]}{(adj.)}{3}{}{Molengo.}{}{}%
\verb{nhongônhongô}{}{[ɲõˈgoɲõˈgo]}{(adj.)}{3}{}{Patético.}{}{}%
\verb{nhongônhongô}{}{[ɲõˈgoɲõˈgo]}{(n.)}{2}{}{Idiota.}{}{}
\verb{nhongônhongô}{}{[ɲõˈgoɲõˈgo]}{(n.)}{3}{}{Molengo.}{}{}
\verb{nhongônhongô}{}{[ɲõˈgoɲõˈgo]}{(n.)}{3}{}{Palerma.}{}{}
\verb{nhongônhongô}{}{[ɲõˈgoɲõˈgo]}{(n.)}{1}{}{Pateta.}{}{}%
\verb{nhonho}{}{[ɲ\~ɔˈɲɔ]}{(n.)}{1}{}{Caracol.}{}{}
\verb{ni}{}{[ˈni]}{(prep.)}{2}{}{De.}{}{}
\verb{ni}{}{[ˈni]}{(prep.)}{1}{}{Em.}{}{}%
\verb{nina}{}{[niˈna]}{(v.)}{1}{}{Fazer cócegas.}{}{}
\verb{ningê}{}{[nĩˈge]}{(n.)}{1}{}{Pessoa.}{Cf. \textbf{ngê}.}{}{}
\verb{njenson}{}{[nʒ\~ɛˈs\~ɔ]}{(n.)}{1}{}{Injeção.}{}{}
\verb{njeson}{}{[nʒɛˈs\~ɔ]}{(n.)}{1}{}{Injeção.}{Cf. \textbf{njenson}.}{}{}
\verb{njizu}{}{[ˈnʒizu]}{(n.)}{1}{}{Jejum.}{}{}
\verb{njozu}{}{[ˈnʒɔzu]}{(adj.)}{1}{}{Curioso.}{}{}
\verb{njozu}{}{[ˈnʒɔzu]}{(adj.)}{1}{}{Habilidoso.}{}{}
\verb{nkome}{}{[ŋkɔˈmɛ]}{(n.)}{1}{}{Soco.}{}{}{}
\verb{nkon}{}{[ˈŋk\~ɔ]}{(n.)}{1}{}{Alimento mal cozido.}{}{}%
\verb{nkon}{}{[ˈŋk\~ɔ]}{(v.)}{1}{}{Atenuar.}{}{}%
\verb{nkon}{}{[ˈŋk\~ɔ]}{(v.)}{2}{}{Atrofiar.}{}{}
\verb{nkon}{}{[ˈŋk\~ɔ]}{(v.)}{3}{}{Diminuir.}{}{}%
\verb{nkoni}{}{[ˈŋkɔni]}{(n.)}{2}{}{Chifre.}{}{}%
\verb{nkoni}{}{[ˈŋkɔni]}{(n.)}{1}{}{Saliência.}{}{}
\verb{nkon-kadela}{}{[ˈŋkõ  kaˈdɛla]}{(n.)}{1}{}{Ancas.}{}{}
\verb{nkyonkyon}{}{[ˈŋkj\~ɔˈkj\~ɔ]}{(id.)}{1}{}{Cf. \textbf{oso nkyonkyon.}}{}{}
\verb{noda}{}{[ˈnɔda]}{(n.)}{1}{}{Nódoa.}{}{}
\verb{nogoxo}{}{[nɔˈgɔʃɔ]}{(n.)}{1}{}{Negócio.}{}{}
\verb{noja}{}{[nɔˈʒa]}{(v.)}{1}{}{Enojar.}{}{}
\verb{nôjô}{}{[ˈnoʒo]}{(n.)}{1}{}{Nojo.}{}{}
\verb{nomi}{}{[ˈnɔmi]}{(n.)}{1}{}{Nome.}{}{}
\verb{nomi-familya}{}{[ˈnɔmi faˈmilja]}{(n.)}{1}{}{Apelido.}{}{}
\verb{nomi-familya}{}{[ˈnɔmi faˈmilja]}{(n.)}{1}{}{Nome de família.}{}{}
\verb{nomi-familya}{}{[ˈnɔmi faˈmilja]}{(n.)}{1}{}{Sobrenome.}{}{}
\verb{nomi-ke}{}{[ˈnɔmi ˈkɛ]}{(n.)}{1}{}{Alcunha.}{}{}
\verb{non}{}{[ˈnõ]}{(pron.)}{1}{}{-nos.}{Primeira pessoa do plural com a função de complemento direto ou indireto.}{}%
\verb{non}{}{[ˈnõ]}{(pron.)}{2}{}{Nós.}{}{}%
\verb{non}{}{[ˈnõ]}{(poss.)}{3}{}{Nossa.}{}{}
\verb{non}{}{[ˈnõ]}{(poss.)}{4}{}{Nossas.}{}{}
\verb{non}{}{[ˈnõ]}{(poss.)}{5}{}{Nosso.}{}{}
\verb{non}{}{[ˈnõ]}{(poss.)}{6}{}{Nossos.}{}{}
\verb{nona}{}{[ˈnɔna]}{(n.)}{1}{}{Anona.}{\textbf{\textit{Anona squamosa}.}}{}
\verb{nona}{}{[ˈnɔna]}{(n.)}{2}{}{Pinha.}{\textbf{\textit{Anona squamosa}.}}{}
\verb{nona-konxa}{}{[ˈnɔna ˈk\~ɔʃa]}{(n.)}{1}{}{Araticum-da-praia.}{\textbf{\textit{Annona glabra}.}}{}
\verb{nondji}{}{[ˈnõdʒi]}{(int.)}{1}{}{Onde.}{Cf. \textbf{nandji}.}{}
\verb{nono}{}{[nɔˈnɔ]}{(n.)}{1}{}{\textit{Nono}.}{\textit{\textbf{Canthium subcordatum}}.}{}
\verb{nono}{}{[nɔˈnɔ]}{(v.)}{2}{}{Coçar.}{}{}
\verb{nonu}{}{[ˈnɔnu]}{(num.)}{1}{}{Nono.}{}{}
\verb{nôô}{}{[noˈo]}{(adv.)}{1}{}{Não.}{}{}
\verb{Nosa-Xola}{}{[ˈnɔsa ˈʃɔla]}{(n.)}{1}{}{Nossa Senhora.}{}{}
\verb{nôsentxi}{}{[noˈsẽtʃi]}{(adj.)}{1}{}{Inocente.}{}{}
\verb{nôsentxi}{}{[noˈsẽtʃi]}{(n.)}{1}{}{Inocente.}{}{}
\verb{nota}{}{[ˈnɔta]}{(n.)}{1}{}{Nota.}{}{}
\verb{notxi}{}{[ˈnɔtʃi]}{(n.)}{1}{}{Norte.}{}{}
\verb{nôtxi}{}{[ˈnotʃi]}{(n.)}{1}{}{Noite.}{}{}
\verb{notxi}{}{[ˈnɔtʃi]}{(top.)}{2}{}{Portugal.}{}{}
\verb{nova}{}{[ˈnɔva]}{(n.)}{1}{}{Notícia.}{}{}
\verb{nova}{}{[ˈnɔva]}{(n.)}{2}{}{Novidade.}{}{}
\verb{nove}{}{[ˈnɔvɛ]}{(num.)}{1}{}{Nove.}{}{}
\verb{nove-dexi}{}{[nɔvɛˈdɛʃi]}{(num.)}{1}{}{Noventa.}{}{}
\verb{novemblu}{}{[nɔˈvẽblu]}{(n.)}{1}{}{Novembro.}{}{}
\verb{novena}{}{[nɔˈvɛna]}{(n.)}{1}{}{Novena.}{}{}
\verb{noventa}{}{[nɔˈvẽta]}{(num.)}{1}{}{Noventa.}{}{}
\verb{novesentu}{}{[nɔvɛˈs\~ɛtu]}{(num.)}{1}{}{Novecentos.}{}{}
\verb{nôvi}{}{[ˈnovi]}{(n.)}{1}{}{Nuvem.}{}{}
\verb{novu}{}{[ˈnɔvu]}{(adj.)}{1}{}{Jovem.}{}{}
\verb{novu}{}{[ˈnɔvu]}{(adj.)}{2}{}{Novo.}{}{}
\verb{novu}{}{[ˈnɔvu]}{(adj.)}{3}{}{Recente.}{}{}
\verb{novu xtlinki}{}{[ˈnɔvu ˈʃtlĩki]}{(expr.)}{1}{}{Novíssimo.}{}{}
\verb{noxtempu}{}{[nɔʃˈt\~ɛpu]}{(adv.)}{1}{}{Antigamente.}{}{}
\verb{nozadu}{}{[nɔˈzadu]}{(n.)}{1}{}{Cerimônia de luto.}{}{}	
\verb{nozadu}{}{[nɔˈzadu]}{(n.)}{2}{}{Velório.}{}{}	
\verb{nozo}{}{[nɔˈzɔ]}{(v.)}{1}{}{Fraturar.}{}{}
\verb{nsanga}{}{[ˈns\~{\textturna}ga]}{(n.)}{1}{}{Moreia.}{}{}
\verb{ntela}{}{[ntɛˈla]}{(v.)}{1}{}{Enterrar.}{}{}
\verb{nteladu}{}{[ntɛˈladu]}{(v.)}{1}{}{Enterrado.}{}{}
\verb{ntêlu}{}{[ˈntelu]}{(adj.)}{1}{}{Inteiro.}{}{}
\verb{ntelu}{}{[ˈntɛlu]}{(n.)}{1}{}{Enterro.}{}{}
\verb{ntenu}{}{[ˈntɛnu]}{(n.)}{1}{}{Marmita.}{}{}
\verb{ntlega}{}{[ntlɛˈga]}{(v.)}{1}{}{Entregar.}{Cf. \textbf{tlega}.}{}{}
\verb{numeru}{}{[ˈnumɛru]}{(n.)}{1}{}{Número.}{}{}
\verb{numiga}{}{[nuˈmiga]}{(n.)}{1}{}{Inimiga.}{}{}
\verb{numigu}{}{[nuˈmigu]}{(n.)}{1}{}{Inimigo.}{}{}
\verb{nunson}{}{[n\~uˈsõ]}{(n.)}{1}{}{Noção.}{}{}
\verb{nutixa}{}{[nuˈtiʃa]}{(n.)}{1}{}{Notícia.}{}{}
\verb{nutuxa}{}{[nuˈtuʃa]}{(n.)}{1}{}{Notícia.}{Cf. \textbf{nutixa}.}{}
\verb{nuvidadji}{}{[nuviˈdadʒi]}{(adj.)}{1}{}{Novidade.}{}{}%
\verb{nventa}{}{[nv\~eˈta]}{(v.)}{1}{}{Inventar.}{}{}
\verb{nwa}{}{[ˈnwa]}{(id.)}{1}{}{Cf. \textbf{betu nwa.}}{}{}
\verb{nwa}{}{[ˈnwa]}{(n.)}{1}{}{Lua.}{}{}
\verb{nyuku}{}{[ˈnjuku]}{(adv.)}{1}{}{Absolutamente nada.}{}{}
\verb{nyuku}{}{[ˈnjuku]}{(adv.)}{2}{}{Nada.}{}{}%
\verb{nzali}{}{[nzaˈli]}{(n.)}{1}{}{Larva.}{}{}
\verb{nzali}{}{[nzaˈli]}{(n.)}{1}{}{Verme.}{}{}
\verb{nzolo}{}{[ˈnzɔlɔ]}{(n.)}{1}{}{Anzol.}{}{}
\verb{nzuku}{}{[ˈnzuku]}{(n.)}{1}{}{Excrementos.}{}{}
\verb{nzuku}{}{[ˈnzuku]}{(n.)}{1}{}{Fezes.}{}{}
\end{letra}

\begin{letra}{o}

\verb{ô}{}{[ˈo]}{(conj.)}{1}{}{Ou.}{}{}
\verb{ô}{}{[ˈo]}{(part.)}{1}{}{Partícula de asserção.}{\textbf{N xka be mu ke ô.} \textit{Estou a ir para casa.}}{}{}
\verb{ô}{}{[ˈo]}{(pron.)}{1}{}{Cf. \textbf{bô}.}{}{}
%\verb{ô}{}{[ˈo]}{(pron.)}{1}{}{Tu.}{Cf. \textbf{bô}.}{}{}
%\verb{ô}{}{[ˈo]}{(pron.)}{1}{}{Você.}{Cf. \textbf{bô}.}{}{}
\verb{oali}{}{[ɔˈali]}{(n.)}{1}{}{Ar.}{}{}
\verb{oali}{}{[ɔˈali]}{(n.)}{1}{}{Atmosfera.}{}{}
\verb{oali}{}{[ɔˈali]}{(n.)}{1}{}{Firmamento.}{}{}
\verb{oali}{}{[ɔˈali]}{(n.)}{1}{}{Céu.}{}{}
\verb{oba}{}{[ɔˈba]}{(n.)}{1}{}{Cacau.}{}{}
\verb{oba}{}{[ɔˈba]}{(n.)}{2}{}{Dinheiro.}{}{}
\verb{oba}{}{[ɔˈba]}{(n.)}{3}{}{Obá.}{\textbf{\textit{Mammea africana}}.}{}
\verb{ôbata}{}{[obaˈta]}{(n.)}{1}{}{\textit{Ôbata}.}{\textbf{\textit{Ficus chlamydocarpa}}.}{}
\verb{obla}{}{[ˈɔbla]}{(n.)}{1}{}{Obra.}{}{}
\verb{obla}{}{[ɔˈbla]}{(n.)}{1}{}{Excremento.}{}{}
\verb{ôbliga}{}{[obliˈga]}{(v.)}{1}{}{Obrigar.}{}{}
\verb{ôbligadu}{}{[obliˈgadu]}{(adj.)}{1}{}{Obrigado.}{}{}
\verb{ôbligason}{}{[obligaˈsõ]}{(n.)}{1}{}{Dever.}{}{}
\verb{ôbligason}{}{[obligaˈsõ]}{(n.)}{1}{}{Obrigação.}{}{}
\verb{ôbô}{}{[oˈbo]}{(n.)}{1}{}{Mato.}{}{}%
\verb{ôbô}{}{[oˈbo]}{(n.)}{2}{}{Floresta.}{}{}
\verb{ôbô jiji}{}{[oˈbo ʒiˈʒi]}{(expr.)}{1}{}{Floresta densa.}{}{}%
\verb{odji}{}{[ˈɔdʒi]}{(n.)}{1}{}{Direito.}{}{}
\verb{odji}{}{[ˈɔdʒi]}{(n.)}{1}{}{Direitos.}{}{}
\verb{odji}{}{[ˈɔdʒi]}{(n.)}{1}{}{Ordem.}{}{}
\verb{odjo}{}{[ɔˈdʒɔ]}{(n.)}{1}{}{Inveja.}{}{}%
\verb{odjo}{}{[ɔˈdʒɔ]}{(n.)}{2}{}{Ódio.}{}{}%
\verb{odjo}{}{[ɔˈdʒɔ]}{(n.)}{3}{}{Rancor.}{}{}
\verb{odo}{}{[ɔˈdɔ]}{(n.)}{1}{}{Dó.}{}{}
\verb{odo}{}{[ɔˈdɔ]}{(n.)}{2}{}{Luto.}{}{}
\verb{ôdô}{}{[oˈdo]}{(n.)}{1}{}{Almofariz.}{}{}%
\verb{ôdô}{}{[oˈdo]}{(n.)}{2}{}{Potinho.}{}{}
\verb{ofisina}{}{[ɔfiˈsina]}{(n.)}{1}{}{Oficina.}{}{}
\verb{ofixali}{}{[ɔfiˈʃali]}{(n.)}{1}{}{Oficial.}{}{}
\verb{ofo}{}{[ɔˈfɔ]}{(n.)}{1}{}{Inhame-selvagem.}{Cf. \textbf{nhami-ofo}.}{}{}
\verb{oji}{}{[ˈɔʒi]}{(n.)}{1}{}{Ordem.}{Cf. \textbf{odji}.}{}{}
\verb{ojo}{}{[ˈɔʒɔ]}{(n.)}{1}{}{Ódio.}{Cf. \textbf{odjo}.}{}{}
\verb{oka}{}{[ɔˈka]}{(n.)}{1}{}{Mafumeira.}{\textbf{\textit{Ceiba pentandra}.}}{}%
\verb{oka}{}{[ɔˈka]}{(n.)}{2}{}{Sumaúma.}{\textbf{\textit{Ceiba pentandra}.}}{}
\verb{okala}{}{[ˈɔkala]}{(n.)}{1}{}{Cicatriz.}{}{}
\verb{okala}{}{[ˈɔkala]}{(n.)}{2}{}{Nódoa.}{}{}
\verb{ôkê}{}{[ˈoke]}{(n.)}{1}{}{Colina.}{}{}%
\verb{ôkê}{}{[ˈoke]}{(n.)}{2}{}{Ladeira.}{}{}%
\verb{ôkê}{}{[ˈoke]}{(n.)}{3}{}{Monte.}{}{}
\verb{ôkê}{}{[ˈoke]}{(n.)}{3}{}{Subida.}{}{}
\verb{oki}{}{[ˈɔki]}{(n.)}{1}{}{Gato-do-mato.}{}{}
\verb{okloklo}{}{[ɔklɔˈklɔ]}{(n.)}{1}{}{Cicatriz.}{}{}
\verb{oklu}{}{[ˈɔklu]}{(n.)}{1}{}{Óculos.}{}{}
\verb{ôkô}{}{[ˈoko]}{(n.)}{1}{}{Buraco.}{}{}
\verb{ôkô}{}{[ˈoko]}{(n.)}{2}{}{Orifício.}{}{}
\verb{ôkô}{}{[oˈko]}{(n.)}{1}{}{Cabaça.}{}{}
\verb{ôkô-budu}{}{[ˈoko ˈbudu]}{(n.)}{1}{}{Caverna.}{}{}
\verb{ôkô-d'olha}{}{[ˈoko ˈdɔʎa]}{(n.)}{1}{}{Ouvido.}{}{}
\verb{okoli}{}{[ˈɔkɔli]}{(n.)}{1}{}{Larva de besouro.}{}{}
\verb{ôkô-lixi}{}{[ˈoko ˈliʃi]}{(n.)}{1}{}{Narinas.}{}{}
\verb{ôkôsô}{}{[ˈokoso]}{(n.)}{2}{}{Albino.}{}{}
\verb{ôkôsô}{}{[ˈokoso]}{(n.)}{1}{}{Espírito de água doce.}{}{}%
\verb{ola}{}{[ˈɔla]}{(conj.)}{1}{}{Quando.}{}{}
\verb{ola}{}{[ˈɔla]}{(n.)}{1}{}{Hora.}{}{}
\verb{ôlêdê}{}{[oleˈde]}{(n.)}{1}{}{Atum.}{\textbf{\textit{Thunnus albacares}}.}{}%
\verb{olha}{}{[ɔˈʎa]}{(n.)}{1}{}{Orelha.}{}{}
\verb{olha-d'atô}{}{[ɔˈʎa daˈto]}{(n.)}{1}{}{\textit{Olha-d'atô}.}{\textbf{\textit{Geophila repens}}.}{}
\verb{olha-lizu}{}{[ɔˈʎa ˈlizu]}{(adj.)}{1}{}{Obstinado.}{}{}
\verb{olha-lizu}{}{[ɔˈʎa ˈlizu]}{(adj.)}{1}{}{Teimoso.}{}{}
\verb{olha-lizu}{}{[ɔˈʎa ˈlizu]}{(n.)}{1}{}{Obstinação.}{}{}
\verb{olha-lizu}{}{[ɔˈʎa ˈlizu]}{(n.)}{2}{}{Teimosia.}{}{}
\verb{olientxi}{}{[ɔliˈ\~ɛtʃi]}{(n.)}{1}{}{Oriente.}{}{}
\verb{ôlô}{}{[ˈolo]}{(n.)}{1}{}{Ouro.}{}{}
\verb{ôlô}{}{[ˈolo]}{(n.)}{1}{}{Ouros.}{Um dos naipes do baralho.}{}
\verb{olota}{}{[ˈɔlɔta]}{(n.)}{1}{}{Horta.}{}{}
\verb{olufu}{}{[ˈɔlufu]}{(n.)}{1}{}{Órfão.}{}{}
\verb{olyo}{}{[ˈɔljɔ]}{(n.)}{1}{}{Óleo.}{}{}
\verb{ôlyô}{}{[oˈljo]}{(n.)}{1}{}{Água parada coberta de limo.}{}{}
\verb{ôlyô}{}{[oˈljo]}{(n.)}{1}{}{Lago.}{}{}
\verb{ôlyô}{}{[oˈljo]}{(n.)}{1}{}{Lagoa.}{}{}
\verb{omali}{}{[ɔˈmali]}{(n.)}{1}{}{Mar.}{}{}
\verb{ome}{}{[ˈɔmɛ]}{(n.)}{1}{}{Homem.}{}{}
\verb{ome}{}{[ˈɔmɛ]}{(n.)}{2}{}{Macho.}{}{}
\verb{ômê}{}{[oˈme]}{(n.)}{1}{}{Meio.}{}{}
\verb{ômê}{}{[oˈme]}{(prep. n.)}{1}{}{No meio de.}{}{}
\verb{ome-d'\~ua-oso}{}{[ˈɔmɛ ˈd\~ua ˈɔsɔ]}{(n.)}{1}{}{Homem-de-um-osso.}{\textbf{\textit{Culcasia scandens}}.}{}{}
\verb{ômu}{}{[ˈomu]}{(n.)}{1}{}{Sabão em pó.}{}{}
\verb{ondji}{}{[ˈõdʒi]}{(int.)}{1}{}{Onde.}{Cf. \textbf{andji}.}{}{}
\verb{ondlega}{}{[ˈ\~ɔdlɛga]}{(n.)}{2}{}{Osga.}{\textbf{\textit{Hemidactylus greefi}.}}{}
\verb{ono}{}{[ɔˈnɔ]}{(n.)}{1}{}{Nó.}{}{}
\verb{onoxi}{}{[ɔˈnɔʃi]}{(n.)}{1}{}{Concha tradicional feita com casca de coco.}{}
\verb{onsa}{}{[ˈ\~ɔsa]}{(n.)}{1}{}{Onça.}{}{}
\verb{onso}{}{[\~ɔˈsɔ]}{(n.)}{1}{}{Enxó.}{}{}
\verb{onso}{}{[\~ɔˈsɔ]}{(n.)}{2}{}{Corda utilizada pelos vinhateiros para subir as palmeiras.}{Cf. \textbf{kodo-d'onso}.}{}{}
\verb{onte}{}{[\~ɔˈtɛ]}{(adv.)}{1}{}{Ontem.}{Cf. \textbf{onten}.}{}{}
\verb{onten}{}{[\~ɔˈt\~ɛ]}{(adv.)}{1}{}{Ontem.}{}{}
\verb{onze}{}{[ˈ\~ɔzɛ]}{(num.)}{1}{}{Onze.}{}{}
\verb{ope}{}{[ɔˈpɛ]}{(n.)}{1}{}{Pé.}{}{}%
\verb{ope}{}{[ɔˈpɛ]}{(n.)}{2}{}{Perna.}{}{}
\verb{ope}{}{[ɔˈpɛ]}{(n.)}{3}{}{Junto de.}{}{}
\verb{ope}{}{[ɔˈpɛ]}{(n.)}{4}{}{Perto a.}{}{}
\verb{ope-kabla}{}{[ɔˈpɛ ˈkabla]}{(n.)}{1}{}{Pé-de-cabra.}{}{}
\verb{ope-po}{}{[ɔˈpɛ ˈpɔ]}{(n.)}{1}{}{\textit{Perna-de-pau}.}{Personagem do \textbf{dansu-kongô} que executa acrobacias sobre pernas-de-pau.}{}{}
\verb{ope-po}{}{[ɔˈpɛ ˈpɔ]}{(n.)}{1}{}{Tronco.}{}{}{}
\verb{opleson}{}{[ɔplɛˈs\~ɔ]}{(n.)}{1}{}{Opressão.}{}{}
\verb{opo}{}{[ɔˈpɔ]}{(n.)}{1}{}{Pó.}{}{}
\verb{opo}{}{[ɔˈpɔ]}{(n.)}{2}{}{Poeira.}{}{}%
\verb{opo}{}{[ɔˈpɔ]}{(n.)}{3}{}{Resíduos em pó.}{}{}
\verb{opo-tabaku}{}{[ɔˈpɔ taˈbaku]}{(n.)}{1}{}{Rapé.}{}{}
\verb{ôsa}{}{[ˈosa]}{(n.)}{1}{}{Voo alto.}{}{}
\verb{ôsa}{}{[oˈsa]}{(v.)}{1}{}{Voar muito alto.}{}{}
\verb{osami}{}{[ˈɔsami]}{(n.)}{1}{}{\textit{Osami}.}{\textbf{\textit{Afromonum danielli}.}}{}
\verb{osana}{}{[ɔˈs\~{\textturna}na]}{(n.)}{1}{}{Hosana.}{}{}
\verb{ose}{}{[ɔˈsɛ]}{(n.)}{1}{}{Céu.}{}{}
\verb{oso}{}{[ˈɔsɔ]}{(n.)}{1}{}{Osso.}{}{}
\verb{ôsôbô}{}{[osoˈbo]}{(n.)}{1}{}{Cuco-esmeraldino.}{\textbf{\textit{Chrysococcyx cupreus insularum}.}}{}
\verb{oso-di-liba-d'amblu}{}{[ˈɔsɔ di ˈliba ˈd\~{\textturna}blu]}{(n.)}{1}{}{Clavícula.}{}{}
\verb{oso-d'ope}{}{[ˈɔsɔ dɔˈpɛ]}{(n.)}{1}{}{Tornozelo.}{}{}
\verb{oso-moli}{}{[ˈɔsɔ ˈmɔli]}{(n.)}{1}{}{\textit{Oso-moli}.}{\textbf{\textit{Uraspis secunda}.}}{}
\verb{oso nkyonkyon}{}{[ˈɔsɔ ˈŋkj\~ɔˈkj\~ɔ]}{(expr.)}{1}{}{Ossos amontoados.}{}{}
\verb{oso-tlaxi}{}{[ˈɔsɔ ˈtlaʃi]}{(n.)}{1}{}{Coluna vertebral.}{}{}
\verb{otaji}{}{[ˈɔtaʒi]}{(n.)}{1}{}{\textit{Otaji}.}{\textbf{\textit{Gongronema latifolium}.}}{}
\verb{otelu}{}{[ɔˈtɛlu]}{(n.)}{1}{}{Hotel.}{}{}
\verb{ôtlô}{}{[ˈotlo]}{(indef.)}{1}{}{Outro.}{}{}
\verb{ôtô}{}{[ˈoto]}{(num.)}{1}{}{Oito.}{Cf. \textbf{wôtô}.}{}{}
\verb{ototo}{}{[ɔˈtɔtɔ]}{(n.)}{1}{}{\textit{Ototo}.}{\textbf{\textit{Malvastrum coromandelianum.}}}{}%
\verb{ototo-nglandji}{}{[ɔˈtɔtɔ ˈŋgl\~{\textturna}dʒi]}{(n.)}{1}{}{\textit{Ototo-nglandji}.}{\textbf{\textit{Sida acuta}.}}{}
\verb{ototo-pikina}{}{[ɔˈtɔtɔ piˈkina]}{(n.)}{1}{}{\textit{Ototo-pikina}.}{\textbf{\textit{Urena lobata}.}}{}
\verb{ototo-ventu}{}{[ɔˈtɔtɔ ˈv\~ɛtu]}{(n.)}{1}{}{Malva-roxa.}{\textbf{\textit{Urena lobata}.}}{}
\verb{ôtublu}{}{[oˈtublu]}{(n.)}{1}{}{Outubro.}{}{}
\verb{ovu}{}{[ˈɔvu]}{(n.)}{1}{}{Ovo.}{}{}%
\verb{ovu}{}{[ˈɔvu]}{(n.)}{2}{}{Testículos.}{}{}
\verb{ovu-d'idu}{}{[ˈɔvu ˈdidu]}{(n.)}{1}{}{Lêndea.}{}{}
\verb{oxtya}{}{[ˈɔʃtja]}{(n.)}{1}{}{Hóstia.}{}{}
\verb{oze}{}{[ˈɔzɛ]}{(adv.)}{1}{}{Hoje.}{}{}
\end{letra}

\begin{letra}{p}

\verb{pa}{}{[ˈpa]}{(conj.)}{1}{}{A fim de.}{}{}%
\verb{pa}{}{[ˈpa]}{(conj.)}{2}{}{Para.}{\textbf{N ba ala pa n ga kopla vinpema.} \textit{Fui lá para comprar vinho de palma}.}{}%
\verb{pa}{}{[ˈpa]}{(conj.)}{2}{}{Para que.}{}{}%
\verb{pa}{}{[ˈpa]}{(conj.)}{4}{}{Que.}{Introduz orações completivas.}{\textbf{N mêsê pa san be.} \textit{Quero que ela vá.}}{}
\verb{padaria}{}{[padaˈria]}{(n.)}{1}{}{Padaria.}{}{}
\verb{pade}{}{[paˈdɛ]}{(n.)}{1}{}{Canário-castanho de São Tomé e Príncipe.}{\textit{\textbf{Serinus rufobrunneus thomensis.}}}{}{}
\verb{pade}{}{[paˈdɛ]}{(n.)}{1}{}{Pardal.}{}{}
\verb{padê}{}{[ˈpade]}{(n.)}{1}{}{Padre.}{}{}
\verb{pade-kampu}{}{[paˈdɛ ˈk\~{\textturna}pu]}{(n.)}{1}{}{Bispo-de-coroa-vermelha.}{\textit{\textbf{Euplectus hordeaceus.}}}{}{}
\verb{padê-nosu}{}{[ˈpade ˈnɔsu]}{(n.)}{1}{}{Pai-nosso.}{}{}
\verb{padisê}{}{[padiˈse]}{(v.)}{1}{}{Padecer.}{}{}
\verb{padjin}{}{[paˈdʒĩ]}{(n.)}{1}{}{Padrinho.}{}{}
\verb{padlaxtu}{}{[paˈdlaʃtu]}{(n.)}{1}{}{Padrasto.}{}{}
\verb{padlwêla}{}{[paˈdlwela]}{(n.)}{1}{}{Padroeira.}{}{}
\verb{padlwêlu}{}{[paˈdlwelu]}{(n.)}{1}{}{Padroeiro.}{}{}
\verb{paga}{}{[paˈga]}{(v.)}{1}{}{Apagar.}{}{}%
\verb{paga}{}{[paˈga]}{(v.)}{2}{}{Desligar.}{}{}
\verb{paga}{}{[paˈga]}{(v.)}{3}{}{Pagar.}{}{}
\verb{paga-dêvê}{}{[paˈga deˈve]}{(n.)}{1}{}{\textit{Paga-dêvê}.}{Ritual conduzido por um mestre que simula o pagamento de uma dívida, através de orações e oferendas depositadas em encruzilhadas e outros locais.}{}{}
\verb{pagadô}{}{[pagaˈdo]}{(n.)}{1}{}{Pagador.}{}{}
\verb{pagadô-dêvê}{}{[pagaˈdo deˈve]}{(n.)}{1}{}{Devedor do \textbf{paga-dêvê}.}{}{}
\verb{pagadô-dêvê}{}{[pagaˈdo deˈve]}{(n.)}{2}{}{Quimbandeiro.}{}{}
\verb{pagadu}{}{[paˈgadu]}{(adj.)}{1}{}{Apagado.}{}{}
\verb{pagadu}{}{[paˈgadu]}{(v.)}{2}{}{Desligado.}{}{}
\verb{pagadu}{}{[paˈgadu]}{(v.)}{3}{}{Pago.}{}{}%
\verb{paga-santu}{}{[paˈga ˈs\~{\textturna}tu]}{(n.)}{1}{}{\textit{Paga-santu}.}{Ritual para curar doenças nos olhos conduzido, em geral, perto de uma fonte de água ou praia.}{}{}
\verb{paga-wê}{}{[paˈga ˈwe]}{(n.)}{1}{}{Margoso.}{\textbf{\textit{Elaephorbia grandifolia}.}}{}
\verb{paga-wê-blanku}{}{[paˈga ˈwe ˈbl\~{\textturna}ku]}{(n.)}{1}{}{Apaga-olho-branco.}{\textbf{\textit{Elaeophorbia drupifera}.}}{}
\verb{Page}{}{[paˈgɛ]}{(top.)}{1}{}{Santo Antônio do Príncipe.}{}{}
\verb{pagu}{}{[ˈpagu]}{(n.)}{1}{}{Pago.}{}{}%
\verb{paji}{}{[ˈpaʒi]}{(n.)}{1}{}{Paz.}{}{}{}
\verb{pajin}{}{[paˈʒĩ]}{(n.)}{1}{}{Padrinho.}{Cf. \textbf{padjin}.}{}{}
\verb{pajina}{}{[ˈpaʒina]}{(n.)}{1}{}{Página.}{}{}{}
\verb{pakata}{}{[pakaˈta]}{(v.)}{1}{}{Esmagar-se.}{}{}%
\verb{pakata}{}{[pakaˈta]}{(v.)}{2}{}{Espalmar.}{}{}
\verb{pakatadu}{}{[pakaˈtadu]}{(adj.)}{1}{}{Esmagado.}{}{}%
\verb{pakatadu}{}{[pakaˈtadu]}{(adj.)}{2}{}{Espalmado.}{}{}
\verb{pala}{}{[ˈpala]}{(n.)}{1}{}{Pala.}{}{}
\verb{pala}{}{[ˈpala]}{(prep.)}{1}{}{Para.}{\textbf{Pala semple}. \emph{Para sempre}.}{}
\verb{pala}{}{[paˈla]}{(v.)}{1}{}{Agarrar.}{}{}
\verb{pala}{}{[paˈla]}{(v.)}{2}{}{Segurar.}{}{}
\verb{palada}{}{[paˈlada]}{(n.)}{1}{}{Palmada.}{}{}
\verb{palanki}{}{[paˈl\~{\textturna}ki]}{(n.)}{1}{}{Instrumento de
pesca.}{}{}
\verb{palanki}{}{[paˈl\~{\textturna}ki]}{(n.)}{2}{}{Palanque.}{}{}
\verb{palavla}{}{[paˈlavla]}{(n.)}{1}{}{Palavra.}{}{}
\verb{palaxu}{}{[paˈlaʃu]}{(n.)}{1}{}{Palácio.}{}{}
\verb{palayê}{}{[palaˈje]}{(n.)}{1}{}{Quitandeiro.}{}{}
\verb{palayê}{}{[palaˈje]}{(n.)}{1}{}{Revendedor(a).}{}{}
\verb{palayê}{}{[palaˈje]}{(n.)}{2}{}{Vendedor(a) ambulante.}{}{}%
\verb{palayê}{}{[palaˈje]}{(n.)}{3}{}{Vendedor(a) de mercado.}{}{}
\verb{pala-zawa}{}{[ˈpala ˈzawa]}{(n.)}{1}{}{Calcinha.}{}{}
\verb{palêdê}{}{[paˈlede]}{(n.)}{1}{}{Parede.}{}{}
\verb{palêsê}{}{[paˈlese]}{(adv.)}{1}{}{Aparentemente.}{}{}
\verb{pali}{}{[ˈpali]}{(n.)}{1}{}{Parto.}{}{}
\verb{pali}{}{[ˈpali]}{(n.)}{1}{}{Par.}{Cf. \textbf{mpali}.}{}
\verb{pali}{}{[paˈli]}{(v.)}{1}{}{Dar à luz.}{}{}
\verb{pali}{}{[paˈli]}{(v.)}{2}{}{Gerar.}{}{}
\verb{pali}{}{[paˈli]}{(v.)}{3}{}{Parir.}{}{}%
\verb{pali divida}{}{[paˈli ˈdivida]}{(expr.)}{1}{}{Contrair dívidas.}{}{}
\verb{palidu}{}{[paˈlidu]}{(adj.)}{1}{}{Parida.}{}{}
\verb{palitu}{}{[paˈlitu]}{(n.)}{1}{}{Palito.}{}{}
\verb{paludixmu}{}{[paluˈdiʃmu]}{(n.)}{1}{}{Malária.}{}{}
\verb{paludixmu}{}{[paluˈdiʃmu]}{(n.)}{2}{}{Paludismo.}{}{}
\verb{palugu}{}{[ˈpalugu]}{(n.)}{1}{}{Pargo.}{\textbf{\textit{Pagrus
caeruleostictus}}.}{}
\verb{palya}{}{[paˈlja]}{(v.)}{2}{}{Alisar.}{}{}
\verb{palya}{}{[paˈlja]}{(v.)}{1}{}{Aplainar.}{}{}
\verb{pampôlê}{}{[ˈp\~{\textturna}pole]}{(n.)}{1}{}{Peixe pampolê.}{}{}
\verb{panampanan}{}{[paˈn\~{\textturna}paˈn\~{\textturna}]}{(adj.)}{1}{}{Delicado.}{}{}
\verb{panampanan}{}{[paˈn\~{\textturna}paˈn\~{\textturna}]}{(adj.)}{2}{}{Leve.}{}{}
\verb{pane}{}{[paˈnɛ]}{(n.)}{1}{}{Crepúsculo.}{}{}
\verb{pane}{}{[paˈnɛ]}{(n.)}{1}{}{Penumbra.}{}{}
\verb{pane-di-wê}{}{[paˈnɛ ˈdi ˈwe]}{(n.)}{1}{}{Catarata.}{}{}
\verb{panela}{}{[paˈnɛla]}{(n.)}{1}{}{Panela.}{}{}
\verb{panha}{}{[p\~{\textturna}ˈɲa]}{(v.)}{1}{}{Apanhar.}{}{}
\verb{panha}{}{[p\~{\textturna}ˈɲa]}{(v.)}{2}{}{Ficar com.}{}{}%
\verb{panha}{}{[p\~{\textturna}ˈɲa]}{(v.)}{3}{}{Pegar.}{}{}%
\verb{panha pê}{}{[p\~{\textturna}ˈɲa ˈpe]}{(expr.)}{1}{}{Colocar em.}{}{}
\verb{panhonho}{}{[p\~{\textturna}ɲɔˈɲɔ]}{(adj.)}{1}{}{Idiota.}{}{}
\verb{panhonho}{}{[p\~{\textturna}ɲɔˈɲɔ]}{(adj.)}{2}{}{Imbecil.}{}{}
\verb{panhonho}{}{[p\~{\textturna}ɲɔˈɲɔ]}{(adj.)}{3}{}{Parvo.}{}{}
\verb{pankada}{}{[p\~{\textturna}ˈkada]}{(n.)}{1}{}{Pancada.}{}{}%
\verb{pankada}{}{[p\~{\textturna}ˈkada]}{(n.)}{2}{}{Porrada.}{}{}%
\verb{pankada}{}{[p\~{\textturna}ˈkada]}{(n.)}{3}{}{Sova.}{}{}%
\verb{pankada}{}{[p\~{\textturna}ˈkada]}{(n.)}{4}{}{Tareia.}{}{}
\verb{panta}{}{[p\~{\textturna}ˈta]}{(v.)}{1}{}{Espantar(-se).}{}{}
\verb{pantula}{}{[p\~{\textturna}tuˈla]}{(n.)}{1}{}{Indigestão.}{}{}
\verb{pantula}{}{[p\~{\textturna}tuˈla]}{(v.)}{1}{}{Empanturrar.}{}{}
\verb{pantuladu}{}{[p\~{\textturna}tuˈladu]}{(adj.)}{1}{}{Empanturrado.}{}{}
\verb{panu}{}{[ˈpanu]}{(n.)}{1}{}{Pano.}{}{}
\verb{panxadu}{}{[p\~{\textturna}ˈʃadu]}{(adj.)}{1}{}{Muito cheio.}{}{}%
\verb{papa}{}{[ˈpapa]}{(n.)}{1}{}{Manta.}{}{}%
\verb{papa}{}{[paˈpa]}{(n.)}{1}{}{Pai.}{}{}%
\verb{papa}{}{[paˈpa]}{(n.)}{2}{}{Papá.}{}{}%
\verb{papa}{}{[paˈpa]}{(n.)}{3}{}{Papai.}{}{}
\verb{papa}{}{[paˈpa]}{(v.)}{1}{}{Mastigar.}{}{}
\verb{papa}{}{[paˈpa]}{(v.)}{2}{}{Papar.}{}{}
\verb{papafigu}{}{[papaˈfigu]}{(n.)}{1}{}{Papa-figo de São Tomé.}{\textbf{\textit{Oriolus crassirostris}.}}{}
\verb{papage}{}{[papaˈgɛ]}{(n.)}{1}{}{Papagaio.}{\textbf{\textit{Psittacus eritachus princeps}}.}{}
\verb{papa-ventu}{}{[ˈpapa ˈvẽtu]}{(n.)}{2}{}{Ventilador.}{}{}
\verb{papa-ventu}{}{[ˈpapa ˈvẽtu]}{(n.)}{1}{}{Ventoinha.}{}{}%
\verb{papelu}{}{[paˈpɛlu]}{(n.)}{2}{}{Documento.}{}{}
\verb{papelu}{}{[paˈpɛlu]}{(n.)}{1}{}{Papel.}{}{}%
\verb{papla}{}{[paˈpla]}{(v.)}{2}{}{Apalpar.}{}{}{}
\verb{papla}{}{[paˈpla]}{(v.)}{2}{}{Experimentar.}{}{}{}
\verb{papla}{}{[paˈpla]}{(v.)}{1}{}{Preparar.}{}{}
\verb{papla}{}{[paˈpla]}{(v.)}{2}{}{Tentar.}{Cf. \textbf{plapa}.}{}{}
\verb{papu}{}{[ˈpapu]}{(n.)}{1}{}{Cordas vocais.}{}{}
\verb{papu}{}{[ˈpapu]}{(n.)}{2}{}{Garganta.}{}{}
\verb{papu}{}{[ˈpapu]}{(n.)}{3}{}{Papo.}{}{}
\verb{papuda}{}{[paˈpuda]}{(n.)}{1}{}{Papeira.}{}{}
\verb{papu-doxi}{}{[ˈpapu ˈdɔʃi]}{(n.)}{1}{}{Voz melodiosa.}{}{}
\verb{papuni}{}{[paˈpuni]}{(n.)}{1}{}{Cão novo.}{}{}
\verb{pasa}{}{[paˈsa]}{(adv.)}{1}{}{Demasiado.}{}{}
\verb{pasa}{}{[paˈsa]}{(adv.)}{2}{}{Muito.}{\textbf{Xefi mu sa bluku pasa}.
\textit{O meu chefe é muito mau}.}{}%
\verb{pasa}{}{[paˈsa]}{(conj.)}{1}{}{Do que.}{\textbf{Ê sa longô pasa mu.}
\textit{Ele é mais alto do que eu.}}{}
\verb{pasa}{}{[paˈsa]}{(v.)}{1}{}{Ocorrer.}{}{}%
\verb{pasa}{}{[paˈsa]}{(v.)}{2}{}{Passar(-se).}{}{}%
\verb{pasa}{}{[paˈsa]}{(v.)}{3}{}{Ultrapassar.}{}{}
\verb{pasadu}{}{[paˈsadu]}{(adj.)}{1}{}{Passado.}{}{}
\verb{pasadu}{}{[paˈsadu]}{(adj.)}{2}{}{Ultrapassado.}{}{}
\verb{pasadu}{}{[paˈsadu]}{(adj.)}{3}{}{Estragado.}{}{}
\verb{pasaji}{}{[paˈsaʒi]}{(n.)}{1}{}{Passagem.}{}{}
\verb{pasu}{}{[ˈpasu]}{(n.)}{1}{}{Paço.}{}{}
\verb{pasu}{}{[ˈpasu]}{(n.)}{2}{}{Passo.}{}{}
\verb{pasu}{}{[ˈpasu]}{(n.)}{3}{}{Presépio.}{}{}
\verb{pata}{}{[ˈpata]}{(n.)}{1}{}{Pata.}{}{}
\verb{patadu}{}{[paˈtadu]}{(adj.)}{1}{}{Apartado.}{}{}
\verb{patadu}{}{[paˈtadu]}{(adj.)}{2}{}{Separado.}{}{}
\verb{pata-galu}{}{[ˈpata ˈgalu]}{(n.)}{1}{}{Pato.}{}{}
\verb{pataka}{}{[paˈtaka]}{(n.)}{1}{}{Pataca.}{Antiga unidade monetária.}{}
\verb{pata-mwala}{}{[ˈpata ˈmwala]}{(n.)}{1}{}{Pata.}{}{}
\verb{patapata}{}{[ˈpataˈpata]}{(n.)}{1}{}{\textit{Patapata}.}{\textbf{\textit{Selene
dorsalis}}.}{}
\verb{patela}{}{[paˈtɛla]}{(n.)}{1}{}{Parteira.}{}{}
\verb{patlisu}{}{[paˈtlisu]}{(n.)}{1}{}{Co-cidadão.}{}{}
\verb{patlisu}{}{[paˈtlisu]}{(n.)}{2}{}{Patrício.}{}{}%
\verb{patlon}{}{[paˈtl\~ɔ]}{(n.)}{1}{}{Patrão.}{}{}
\verb{patu}{}{[ˈpatu]}{(n.)}{1}{}{Parto.}{}{}
\verb{patxa}{}{[paˈtʃa]}{(n.)}{1}{}{Partilha.}{}{}
\verb{patxi}{}{[ˈpatʃi]}{(n.)}{1}{}{Notícia.}{}{}%
\verb{patxi}{}{[ˈpatʃi]}{(n.)}{2}{}{Parte.}{}{}
\verb{patxi}{}{[paˈtʃi]}{(v.)}{1}{}{Distribuir.}{}{}
\verb{patxi}{}{[paˈtʃi]}{(v.)}{2}{}{Dividir.}{}{}
\verb{patxi}{}{[paˈtʃi]}{(v.)}{3}{}{Oferecer.}{}{}
\verb{patxi}{}{[paˈtʃi]}{(v.)}{4}{}{Repartir.}{}{}
\verb{pavon}{}{[paˈvõ]}{(n.)}{1}{}{Pavão.}{}{}
\verb{pavu}{}{[ˈpavu]}{(n.)}{1}{}{Cobertura de folhas de palmeira.}{Cf.
\textbf{mpavu}.}{}{}
\verb{paxa}{}{[paˈʃa]}{(v.)}{1}{}{Passear.}{}{}
\verb{paxensa}{}{[paˈʃẽsa]}{(n.)}{1}{}{Paciência.}{}{}
\verb{paxon}{}{[paˈʃ\~ɔ]}{(n.)}{1}{}{Paixão.}{}{}
\verb{paxte}{}{[paʃˈtɛ]}{(n.)}{1}{}{Pastel.}{}{}
\verb{paxtlu}{}{[ˈpaʃtlu]}{(n.)}{1}{}{\textit{Paxtlu}.}{\textit{\textbf{Onychognathus
fulgidus}}.}{}
\verb{paxtu}{}{[ˈpaʃtu]}{(n.)}{1}{}{Pasto.}{}{}
\verb{pawen}{}{[paˈ\~w\~ɛ]}{(n.)}{1}{}{Canibal.}{}{}
\verb{pawen}{}{[paˈ\~w\~ɛ]}{(n.)}{2}{}{Pessoa insaciável.}{}{}
\verb{pawen}{}{[paˈ\~w\~ɛ]}{(n.)}{3}{}{Pessoa voraz.}{}{}
\verb{paya}{}{[ˈpaja]}{(n.)}{1}{}{Palha.}{}{}
\verb{paya-min}{}{[ˈpaja ˈmĩ]}{(n.)}{1}{}{Palha de milho.}{}{}
\verb{paya-sela}{}{[ˈpaja ˈsɛla]}{(n.)}{1}{}{Besouro.}{\textbf{\textit{Cerembex cerdo}.}}{}
\verb{paya-sela}{}{[ˈpaja ˈsɛla]}{(n.)}{1}{}{Pau-esteira.}{\textbf{\textit{Pandanus thomensis}.}}{}
\verb{pazuma}{}{[pazuˈma]}{(v.)}{1}{}{Pasmar.}{}{}
\verb{pazumadu}{}{[pazuˈmadu]}{(adj.)}{1}{}{Inerte.}{}{}
\verb{pazumadu}{}{[pazuˈmadu]}{(adj.)}{2}{}{Pasmado.}{}{}
\verb{pe}{}{[ˈpɛ]}{(n.)}{2}{}{Grande.}{}{}
\verb{pe}{}{[ˈpɛ]}{(n.)}{1}{}{Pai.}{}{}%
\verb{pê}{}{[ˈpe]}{(prep. v.)}{1}{}{Em.}{\textbf{Sun tufu ope pê awa.}
\textit{Ele colocou os pés na água}.}{}%
\verb{pê}{}{[ˈpe]}{(prep. v.)}{2}{}{Em cima.}{}{}%
\verb{pê}{}{[ˈpe]}{(prep. v.)}{3}{}{Por cima.}{\textbf{Bô ka plêmê limon pê.}
\textit{Espremes o limão por cima}.}{}%
\verb{pê}{}{[ˈpe]}{(prep. v.)}{4}{}{Para.}{}{}
\verb{pê}{}{[ˈpe]}{(prep. v.)}{5}{}{Sobre.}{}{}
\verb{pê}{}{[ˈpe]}{(v.)}{1}{}{Colocar.}{}{}%
\verb{pê}{}{[ˈpe]}{(v.)}{2}{}{Pôr.}{}{}%
\verb{pedasu}{}{[pɛˈdasu]}{(n.)}{1}{}{Gleba.}{}{}
\verb{pedasu}{}{[pɛˈdasu]}{(n.)}{2}{}{Pedaço.}{}{}
\verb{pedasu}{}{[pɛˈdasu]}{(n.)}{3}{}{Parcela.}{}{}%
\verb{pêdlêlu}{}{[peˈdlelu]}{(n.)}{1}{}{Pedreiro.}{}{}
\verb{pedon}{}{[pɛˈd\~ɔ]}{(n.)}{1}{}{Perdão.}{}{}
\verb{pêdu}{}{[peˈdu]}{(adj.)}{1}{}{Colocado.}{}{}
\verb{pêdu}{}{[peˈdu]}{(adj.)}{2}{}{Posto.}{}{}
\verb{pega}{}{[ˈpɛga]}{(n.)}{2}{}{Briga.}{}{}
\verb{pega}{}{[ˈpɛga]}{(n.)}{1}{}{Conflito.}{}{}
\verb{pega}{}{[pɛˈga]}{(v.)}{1}{}{Acender.}{}{}
\verb{pega}{}{[pɛˈga]}{(v.)}{2}{}{Ligar.}{}{}
\verb{pega}{}{[pɛˈga]}{(v.)}{3}{}{Pegar.}{}{}%
\verb{pega}{}{[pɛˈga]}{(v.)}{4}{}{Segurar.}{}{}%
\verb{pega}{}{[pɛˈga]}{(v.)}{5}{}{Tomar.}{}{}%
\verb{pega kabu}{}{[pɛˈga kaˈbu]}{(expr.)}{1}{}{Agarrar com força.}{}{}
\verb{pega-latu}{}{[ˈpɛga
ˈlatu]}{(n.)}{1}{}{Pega-rato.}{\textbf{\textit{Pupalia lappacea}.}}{}
\verb{pega-pega}{}{[ˈpɛgaˈpɛga]}{(n.)}{1}{}{Picão
preto.}{\textbf{\textit{Desmodium ramosissimum}.}}{}
\verb{pekadô}{}{[pɛkaˈdo]}{(n.)}{1}{}{Pecador.}{}{}%
\verb{pekadô}{}{[pɛkaˈdo]}{(n.)}{1}{}{Pessoa.}{}{}%
\verb{pekadô}{}{[pɛkaˈdo]}{(n.)}{2}{}{Ser humano.}{}{}
\verb{pekadu}{}{[pɛˈkadu]}{(n.)}{1}{}{Pecado.}{}{}
\verb{peki}{}{[ˈpɛki]}{(n.)}{1}{}{Espremedor.}{}{}
\verb{pêlêja}{}{[peˈleʒa]}{(n.)}{1}{}{Peleja.}{}{}
\verb{pêlêja}{}{[peleˈʒa]}{(v.)}{1}{}{Pelejar.}{}{}
\verb{peli}{}{[ˈpɛli]}{(n.)}{1}{}{Pele.}{}{}
\verb{pêlu}{}{[peˈlu]}{(n.)}{1}{}{Peru.}{Cf. \textbf{mpêlu}.}{}{}
\verb{pema}{}{[ˈpɛma]}{(n.)}{1}{}{Palmeira.}{}{}%
\verb{pema-d'anji}{}{[ˈpɛma d\~{\textturna}ˈʒi]}{(n.)}{1}{}{Palmeira andim.}{\textbf{\textit{Elaeis guineense}.}}{}
\verb{pema-d'ôbô}{}{[ˈpɛma doˈbo]}{(n.)}{1}{}{Palmeira-d'ôbô.}{\textbf{\textit{Mapania ferruginea}.}}{}
\verb{pema mufuku}{}{[ˈpɛma mufuˈku]}{(expr.)}{1}{}{Palmeira ainda não tratada pelo vinhateiro.}{}{}
\verb{pema-vunvun}{}{[ˈpɛma v\~uˈv\~u]}{(n.)}{1}{}{Palmeira apta a se extrair o vinho de palma.}{}{}
\verb{pempe}{}{[p\~ɛˈpɛ]}{(adj.)}{1}{}{Estreito.}{}{}%
\verb{pempe}{}{[p\~ɛˈpɛ]}{(adj.)}{2}{}{Magro.}{}{}
\verb{pempe}{}{[p\~ɛˈpɛ]}{(n.)}{1}{}{Carriço utilizado para o \textbf{vin-pema} entrar em seu recipiente.}{}{}
\verb{pempen}{}{[p\~ɛˈpẽ]}{(n.)}{1}{}{\textit{Pempen}.}{\textbf{\textit{Dracaena
laxissima}.}}{}
\verb{pena}{}{[ˈpɛna]}{(n.)}{1}{}{Pena.}{}{}%
\verb{pena}{}{[ˈpɛna]}{(n.)}{2}{}{Tristeza.}{}{}%
\verb{pena}{}{[pɛˈna]}{(n.)}{3}{}{Sarna.}{}{}
\verb{pena}{}{[pɛˈna]}{(v.)}{1}{}{Depenar.}{}{}
\verb{pena}{}{[pɛˈna]}{(v.)}{2}{}{Sentir dó.}{}{}
\verb{pena}{}{[pɛˈna]}{(v.)}{3}{}{Sofrer.}{}{}
\verb{penadu}{}{[pɛˈnadu]}{(adj.)}{1}{}{Depenado.}{}{}%
\verb{penadu}{}{[pɛˈnadu]}{(adj.)}{2}{}{Sarnento.}{}{}
\verb{penadu}{}{[pɛˈnadu]}{(adj.)}{3}{}{Tinhoso.}{}{}
\verb{pena-d'ubwê}{}{[ˈpɛna ˈdubwe]}{(n.)}{1}{}{Pelo.}{}{}%
\verb{pena-limi}{}{[ˈpɛnaˈlimi]}{(n.)}{1}{}{Pelos púbicos da puberdade.}{}{}
\verb{pena-limi}{}{[ˈpɛnaˈlimi]}{(n.)}{1}{}{Penugem.}{}{}
\verb{penapena}{}{[ˈpɛnaˈpɛna]}{(n.)}{1}{}{Franjas do \textbf{kodo d'onso},
utilizadas como proteção às mãos.}{}{}
\verb{peneta}{}{[pɛˈnɛta]}{(n.)}{1}{}{Destino.}{}{}%
\verb{peneta}{}{[pɛˈnɛta]}{(n.)}{2}{}{Infortúnio.}{}{}
\verb{peneta}{}{[pɛˈnɛta]}{(n.)}{3}{}{Sorte.}{}{}%
\verb{peneta}{}{[pɛnɛˈta]}{(v.)}{1}{}{Sofrer.}{}{}
\verb{penhu}{}{[ˈp\~ɛɲu]}{(n.)}{1}{}{Capricho.}{}{}{}
\verb{penhu}{}{[ˈp\~ɛɲu]}{(n.)}{1}{}{Determinação.}{}{}{}
\verb{penhu}{}{[ˈp\~ɛɲu]}{(n.)}{1}{}{Empenho.}{Cf. \textbf{mpenhu}.}{}{}
\verb{pensa}{}{[p\~ɛˈsa]}{(v.)}{1}{}{Julgar.}{}{}
\verb{pensa}{}{[p\~ɛˈsa]}{(v.)}{2}{}{Pensar.}{}{}%
\verb{pensa}{}{[p\~ɛˈsa]}{(v.)}{3}{}{Preocupar(-se).}{}{}
\verb{pensa}{}{[p\~ɛˈsa]}{(v.)}{4}{}{Presumir.}{}{}
\verb{pensadu}{}{[p\~ɛˈsadu]}{(adj.)}{1}{}{Pensado.}{}{}
\verb{pensamentu}{}{[p\~ɛsaˈm\~ɛtu]}{(n.)}{1}{}{Pensamento.}{}{}
\verb{pensamentu}{}{[p\~ɛsaˈm\~ɛtu]}{(n.)}{2}{}{Preocupação.}{}{}
\verb{pentxa}{}{[pẽˈtʃa]}{(v.)}{1}{}{Adornar.}{}{}
\verb{pentxa}{}{[pẽˈtʃa]}{(v.)}{2}{}{Arrumar.}{}{}
\verb{pentxa}{}{[pẽˈtʃa]}{(v.)}{3}{}{Pentear.}{}{}
\verb{pentxa}{}{[pẽˈtʃa]}{(v.)}{4}{}{Polir.}{}{}
\verb{pentxi}{}{[ˈpẽtʃi]}{(n.)}{1}{}{Pente.}{}{}
\verb{pentxi}{}{[ˈpẽtʃi]}{(n.)}{1}{}{Púbis.}{}{}
%\verb{pentxin}{}{[pẽˈtʃĩ]}{(n.)}{1}{}{Pelos púbicos.}{}{}{}
%\verb{pentxin}{}{[pẽˈtʃĩ]}{(n.)}{1}{}{Pentelho.}{}{}{}
\verb{pênupênu}{}{[penupeˈnu]}{(n.)}{1}{}{Cílios.}{}{}{}
\verb{pênupênu}{}{[penupeˈnu]}{(n.)}{1}{}{Pálpebra.}{}{}{}
\verb{pênupênu}{}{[penupeˈnu]}{(n.)}{1}{}{Sobrancelhas.}{Cf. \textbf{mpênumpênu}.}{}{}
\verb{pepe}{}{[ˈpɛpɛ]}{(n.)}{1}{}{Pai.}{}{}
\verb{pepe}{}{[ˈpɛpɛ]}{(n.)}{2}{}{Papá.}{}{}
\verb{pê poxta}{}{[ˈpe ˈpɔʃta]}{(expr.)}{1}{}{Apostar.}{}{}
\verb{pesa}{}{[ˈpɛsa]}{(n.)}{1}{}{Canhão.}{}{}
\verb{pesa}{}{[ˈpɛsa]}{(n.)}{2}{}{Peça.}{}{}
\verb{pesku}{}{[ˈpɛsku]}{(n.)}{1}{}{Pêssego-de-São-Tomé.}{\textit{\textbf{Chytranthus mannii}.}}{}
\verb{peta}{}{[pɛˈta]}{(v.)}{1}{}{Apertar.}{}{}
\verb{peta}{}{[pɛˈta]}{(v.)}{2}{}{Espetar.}{}{}
\verb{petadu}{}{[pɛˈtadu]}{(v.)}{1}{}{Apertado.}{}{}
\verb{petadu}{}{[pɛˈtadu]}{(v.)}{1}{}{Espetado.}{}{}
\verb{petepete}{}{[pɛˈtɛpɛˈtɛ]}{(id.)}{1}{}{Cf. \textbf{fili petepete.}}{}{}
\verb{petepete}{}{[pɛˈtɛpɛˈtɛ]}{(id.)}{1}{}{Cf. \textbf{mina-fili petepete.}}{}{}
\verb{petloli}{}{[pɛˈtlɔli]}{(n.)}{1}{}{Petróleo.}{}{}
\verb{petu}{}{[ˈpɛtu]}{(adv.)}{1}{}{Perto.}{}{}
\verb{petu}{}{[ˈpɛtu]}{(n.)}{1}{}{Espeto.}{}{}
\verb{pêtu}{}{[ˈpetu]}{(n.)}{1}{}{Peito.}{}{}
\verb{pextana}{}{[pɛʃˈt\~{\textturna}na]}{(n.)}{1}{}{Sobrancelhas.}{}{}%
\verb{pexti}{}{[ˈpɛʃti]}{(n.)}{1}{}{Peste.}{Cf. \textbf{pextli}.}{}{}
\verb{pextli}{}{[ˈpɛʃtli]}{(adj.)}{1}{}{Grosseiro.}{}{}%
\verb{pextli}{}{[ˈpɛʃtli]}{(adj.)}{2}{}{Malcriado.}{}{}
\verb{pextli}{}{[ˈpɛʃtli]}{(n.)}{3}{}{Peste.}{}{}
\verb{peza}{}{[pɛˈza]}{(v.)}{1}{}{Pesar.}{}{}
\verb{pezu}{}{[ˈpɛzu]}{(n.)}{2}{}{Peso.}{}{}%
\verb{pidji}{}{[piˈdʒi]}{(v.)}{1}{}{Pedir.}{}{}
\verb{pidji kasa}{}{[piˈdʒi kaˈza]}{(expr.)}{1}{}{Cio.}{}{}
\verb{pidji plaga}{}{[piˈdʒi ˈplaga]}{(expr.)}{1}{}{Esconjurar.}{}{}
\verb{pidji plaga}{}{[piˈdʒi ˈplaga]}{(expr.)}{1}{}{Praguejar.}{}{}
\verb{pidji poda}{}{[piˈdʒi ˈpɔda]}{(expr.)}{1}{}{Desculpar(-se).}{}{}
\verb{pidu}{}{[ˈpidu]}{(n.)}{1}{}{Flato.}{}{}
\verb{pidu}{}{[ˈpidu]}{(n.)}{1}{}{Peido.}{}{}
\verb{pijami}{}{[piˈʒ\~{\textturna}mi]}{(n.)}{1}{}{Pijama.}{}{}
\verb{piji}{}{[piˈʒi]}{(v.)}{1}{}{Pedir.}{Cf. \textbf{pidji}.}{}{}
\verb{pika}{}{[piˈka]}{(v.)}{1}{}{Beliscar.}{}{}
\verb{pika}{}{[piˈka]}{(v.)}{2}{}{Desconfiar.}{}{}
\verb{pika}{}{[piˈka]}{(v.)}{3}{}{Picar.}{}{}
\verb{pikadu}{}{[piˈkadu]}{(adj.)}{1}{}{Desconfiado.}{}{}
\verb{pikadu}{}{[piˈkadu]}{(adj.)}{2}{}{Picado.}{}{}
\verb{pikaleta}{}{[pikaˈlɛta]}{(n.)}{1}{}{Picareta.}{}{}
\verb{pika letlatu}{}{[piˈka lɛˈtlatu]}{(expr.)}{1}{}{Fotografar.}{}{}
\verb{pikina}{}{[piˈkina]}{(adj.)}{1}{}{Pequeno.}{}{}
\verb{pikina}{}{[piˈkina]}{(adv.)}{1}{}{Bocadinho.}{}{}
\verb{pikina}{}{[piˈkina]}{(adv.)}{2}{}{Pouco.}{}{}
\verb{pikina}{}{[piˈkina]}{(n.)}{1}{}{Pedaço.}{}{}
\verb{pikiniki}{}{[pikiˈniki]}{(n.)}{1}{}{Piquenique.}{}{}
\verb{piku}{}{[ˈpiku]}{(n.)}{1}{}{Pico.}{}{}
\verb{pilha}{}{[ˈpiʎa]}{(n.)}{1}{}{Pilha.}{}{}
\verb{pilixilina}{}{[piliʃiˈlina]}{(n.)}{1}{}{Penicilina.}{}{}
\verb{pilolo}{}{[pilɔˈlɔ]}{(n.)}{1}{}{Pênis de criança.}{}{}
\verb{pilon}{}{[piˈlõ]}{(n.)}{3}{}{Pirão.}{Prato típico com farinha de mandioca e caldo.}{}
\verb{pilôtô}{}{[piˈloto]}{(n.)}{1}{}{Piloto.}{}{}
\verb{pilula}{}{[ˈpilula]}{(n.)}{1}{}{Pílula.}{}{}
\verb{pimbi}{}{[pĩˈbi]}{(n.)}{1}{}{Pênis.}{}{}
\verb{pimenton}{}{[pimẽˈtõ]}{(n.)}{1}{}{Pimentão.}{}{}
\verb{pimpinela}{}{[p\~ipiˈnɛla]}{(n.)}{1}{}{Chuchu.}{\textbf{\textit{Sechium edule}}.}{}
\verb{pimpinela}{}{[p\~ipiˈnɛla]}{(n.)}{1}{}{Pimpinela.}{\textbf{\textit{Sechium edule}}.}{}
\verb{pinela}{}{[piˈnɛla]}{(n.)}{1}{}{Peneira.}{}{}
\verb{pinga}{}{[ˈpĩga]}{(n.)}{1}{}{Vinho.}{}{}
\verb{pingada}{}{[pĩˈgada]}{(n.)}{1}{}{Espingarda.}{}{}
\verb{pingada-nglandji}{}{[pĩˈgada
ˈŋgl\~{\textturna}dʒi]}{(n.)}{1}{}{Canhão.}{}{}
\verb{pingininu}{}{[pĩgiˈninu]}{(n.)}{1}{}{Diabinho.}{}{}{}
\verb{piniku}{}{[piˈniku]}{(n.)}{1}{}{Penico.}{}{}
\verb{pininkanu}{}{[pin\~iˈk\~{\textturna}nu]}{(n.)}{1}{}{Cordão-de-frade.}{\textbf{\textit{Leonotis
nepetifolia}.}}{}
\verb{pinji}{}{[ˈpĩʒi]}{(n.)}{1}{}{Impigem.}{}{}
\verb{pinsa}{}{[pĩˈsa]}{(v.)}{1}{}{Empurrar.}{}{}
\verb{pinsu}{}{[ˈpĩsu]}{(n.)}{1}{}{Empurrão.}{}{}
\verb{pinsu}{}{[ˈpĩsu]}{(n.)}{2}{}{Encontrão.}{}{}
\verb{pinta}{}{[ˈpĩta]}{(n.)}{1}{}{Pintainho.}{}{}
\verb{pinta}{}{[ˈpĩta]}{(n.)}{2}{}{Pintinho.}{}{}
\verb{pinta}{}{[pĩˈta]}{(v.)}{1}{}{Pintar.}{}{}
\verb{pinta-wê}{}{[ˈpĩtaˈwe]}{(n.)}{1}{}{Pupila.}{}{}
\verb{pintenxa}{}{[pĩˈt\~eʃa]}{(n.)}{1}{}{Penitência.}{}{}
\verb{pinton}{}{[pĩˈtõ]}{(n.)}{1}{}{Clítoris.}{}{}
\verb{pipi}{}{[piˈpi]}{(n.)}{1}{}{Vagina.}{}{}
\verb{pipipi}{}{[pipiˈpi]}{(id.)}{1}{}{Cf. \textbf{kaboka pipipi.}}{}{}
\verb{pita}{}{[piˈta]}{(v.)}{1}{}{Apitar.}{}{}
\verb{pita}{}{[piˈta]}{(v.)}{1}{}{Alertar.}{}{}
\verb{pita}{}{[piˈta]}{(v.)}{1}{}{Avisar.}{}{}
\verb{pitanga}{}{[piˈt\~{\textturna}ga]}{(n.)}{1}{}{Pitanga.}{\textbf{\textit{Eugenia
uniflora}}.}{}
\verb{pitu}{}{[ˈpitu]}{(n.)}{1}{}{Apito.}{}{}
\verb{pitu}{}{[ˈpitu]}{(n.)}{2}{}{Flauta.}{}{}
\verb{pitu-doxi}{}{[ˈpitu ˈdɔʃi]}{(n.)}{1}{}{Música de \textbf{pitu}.}{}{}
\verb{pitu-pempe}{}{[ˈpitu p\~ɛˈpɛ]}{(n.)}{1}{}{Flauta de bambu.}{}{}
\verb{pitxipitxi}{}{[piˈtʃipiˈtʃi]}{(id.)}{1}{}{Cf. \textbf{lêdê
pitxipitxi.}}{}{}
\verb{pitxipitxi}{}{[piˈtʃipiˈtʃi]}{(id.)}{2}{}{Cf. \textbf{wê
pitxipitxi.}}{}{}
\verb{pixi}{}{[ˈpiʃi]}{(n.)}{1}{}{Peixe.}{}{}
\verb{pixi-fumu}{}{[ˈpiʃi
ˈfumu]}{(n.)}{1}{}{Peixe-fumo.}{\textit{\textbf{Acanthocybium solandri}}.}{}
\verb{pixi-gôdô}{}{[ˈpiʃi ˈgodo]}{(n.)}{1}{}{Rico.}{Cf.
\textbf{ngê-gôdô}.}{}{}
\verb{pixi-kabla}{}{[ˈpiʃi
ˈklaba]}{(n.)}{1}{}{Peixe-cabra.}{\textit{\textbf{Branchiostegus
semifasciatus}}.}{}
\verb{pixi-magita}{}{[ˈpiʃi
maˈgita]}{(n.)}{1}{}{\textit{Pixi-magita}.}{\textbf{\textit{Pagellus
bellottii bellottii}}.}{}
\verb{pixin}{}{[piˈʃ\~i]}{(n.)}{1}{}{Larvas de peixe, localmente denominado
`peixinho'.}{}{}
\verb{pixina}{}{[piˈʃina]}{(n.)}{1}{}{Piscina.}{}{}
\verb{pixinadu}{}{[piʃiˈnadu]}{(n.)}{1}{}{Pelo-sinal.}{}{}
\verb{pixinadu}{}{[piʃiˈnadu]}{(n.)}{2}{}{Sinal da cruz.}{}{}
\verb{pixi-ndala}{}{[ˈpiʃi
ˈndala]}{(n.)}{1}{}{Agulhão-bandeira.}{\textbf{\textit{Istiophorus
albicans}}.}{}
\verb{pixi-ndala}{}{[ˈpiʃi
ˈndala]}{(n.)}{2}{}{Marlim-azul.}{\textbf{\textit{Istiophorus albicans}}.}{}
\verb{pixi-novu}{}{[ˈpiʃi
ˈnɔvu]}{(n.)}{1}{}{Peixe-novo.}{\textbf{\textit{Apsilus fuscus}}.}{}
\verb{pixi-sela}{}{[ˈpiʃi
ˈsɛla]}{(n.)}{1}{}{Peixe-serra.}{\textbf{\textit{Scomberomorus tritor}}.}{}
\verb{pixka}{}{[ˈpiʃka]}{(n.)}{1}{}{Pesca.}{}{}
\verb{pixka}{}{[piʃˈka]}{(v.)}{1}{}{Cochilar.}{}{}
\verb{pixka}{}{[piʃˈka]}{(v.)}{2}{}{Dormitar.}{}{}
\verb{pixka}{}{[piʃˈka]}{(v.)}{3}{}{Pescar.}{}{}
\verb{pixkadô}{}{[piʃkaˈdo]}{(n.)}{1}{}{Pescador.}{}{}
\verb{pixkela}{}{[piʃˈkɛla]}{(n.)}{1}{}{Pesqueiro.}{}{}
\verb{pixtola}{}{[piʃˈtɔla]}{(n.)}{1}{}{Pistola.}{}{}
\verb{pixtola}{}{[piʃˈtɔla]}{(n.)}{2}{}{Revólver.}{}{}
\verb{pla Konsa}{}{[ˈpla ˈkõsa]}{(top.)}{1}{}{Praia das Conchas.}{}{}
\verb{pla}{}{[ˈpla]}{(prep.)}{1}{}{Para.}{\textbf{Ise sa pla ome tamen.}
\textit{Isso é para homens adultos.}}{}
\verb{plafuzu}{}{[plaˈfuzu]}{(n.)}{1}{}{Parafuso.}{}{}
\verb{plaga}{}{[ˈplaga]}{(n.)}{1}{}{Praga.}{}{}
\verb{plaka}{}{[plaˈka]}{(n.)}{1}{}{Peixe-voador (de)fumado seco.}{}{}
\verb{plakê}{}{[plaˈke]}{(prep.)}{2}{}{Perante.}{}{}
\verb{plakê}{}{[plaˈke]}{(prep.)}{1}{}{Por.}{\textbf{Plakê dêsu.} \textit{Por
Cristo}.}{}%
\verb{plakini}{}{[plakiˈni]}{(n.)}{1}{}{Diabinho.}{}{}
\verb{plama}{}{[plaˈma]}{(n.)}{1}{}{Manhã.}{}{}
\verb{plama bili}{}{[plaˈma biˈli]}{(expr.)}{1}{}{Amanhecer.}{}{}
\verb{plama bili wan}{}{[plaˈma biˈli ˈw\~{\textturna}]}{(expr.)}{1}{}{Amanhecer depressa.}{}{}
\verb{plama da kôdon}{}{[plaˈma da koˈdõ]}{(expr.)}{1}{}{Amanhecer.}{}{}
\verb{plama klaya}{}{[plaˈma klaˈja]}{(expr.)}{1}{}{Amanhecer.}{}{}
\verb{plamatoya}{}{[plamaˈtɔja]}{(n.)}{1}{}{Palmatória.}{}{}
\verb{plamitu}{}{[plaˈmitu]}{(n.)}{1}{}{Palmito.}{}{}
\verb{plana}{}{[ˈplana]}{(n.)}{1}{}{Plaina.}{}{}
\verb{plana}{}{[plaˈna]}{(v.)}{1}{}{Aplainar.}{}{}
\verb{planta}{}{[ˈpl\~{\textturna}ta]}{(n.)}{1}{}{Planta.}{}{}
\verb{planta}{}{[ˈpl\~{\textturna}ta]}{(n.)}{2}{}{Vegetal.}{}{}
\verb{plantu}{}{[ˈpl\~{\textturna}tu]}{(n.)}{1}{}{Alarde.}{}{}
\verb{plantu}{}{[ˈpl\~{\textturna}tu]}{(n.)}{1}{}{Espalhafato.}{}{}
\verb{plantu}{}{[ˈpl\~{\textturna}tu]}{(n.)}{1}{}{Espavento.}{}{}
\verb{plapa}{}{[plaˈpa]}{(v.)}{1}{}{Apalpar.}{}{}
\verb{plapa}{}{[plaˈpa]}{(v.)}{2}{}{Experimentar.}{}{}
\verb{plapa}{}{[plaˈpa]}{(v.)}{3}{}{Tentar.}{}{}
\verb{plaplapla}{}{[plaplaˈpla]}{(id.)}{1}{}{Cf. \textbf{blaga awa-wê
plaplapla}.}{}{}
\verb{plapa son}{}{[plaˈpa s\~ɔ]}{(expr.)}{1}{}{Apalpar o terreno.}{}{}
\verb{plapa son}{}{[plaˈpa s\~ɔ]}{(expr.)}{2}{}{Sondar.}{}{}
\verb{plasa}{}{[ˈplasa]}{(n.)}{1}{}{Praça.}{}{}%
\verb{plasê}{}{[plaˈse]}{(v.)}{1}{}{Aparecer.}{}{}%
\verb{plasê}{}{[plaˈse]}{(v.)}{2}{}{Surgir.}{}{}
\verb{plasela}{}{[plaˈsɛla]}{(n.)}{2}{}{Rival.}{Mulheres que compartilham ou
compartilharam o mesmo homem.}{}{}%
\verb{plasêlu}{}{[plaˈselu]}{(n.)}{1}{}{Rival.}{Homens que compartilham ou
compartilharam a mesma parceira.}{}{}
\verb{plasenta}{}{[plaˈsẽta]}{(n.)}{1}{}{Placenta.}{}{}
\verb{plata}{}{[ˈplata]}{(n.)}{1}{}{Prata.}{}{}
\verb{platoya}{}{[plaˈtɔja]}{(n.)}{1}{}{Palmatória.}{Cf.
\textbf{plamatoya}.}{}{}
\verb{platu}{}{[ˈplatu]}{(n.)}{1}{}{Prato.}{}{}
\verb{plaxtiku}{}{[ˈplaʃtiku]}{(n.)}{2}{}{Plástico.}{}{}
\verb{plaxtiku}{}{[ˈplaʃtiku]}{(n.)}{1}{}{Saco plástico.}{}{}
\verb{plazu}{}{[ˈplazu]}{(n.)}{1}{}{Prazo.}{}{}
\verb{ple}{}{[ˈplɛ]}{(n.)}{1}{}{Praia.}{}{}
\verb{plêdê}{}{[pleˈde]}{(v.)}{1}{}{Desaparecer.}{}{}
%\verb{plêdê}{}{[pleˈde]}{(v.)}{2}{}{Perder.}{}{}
\verb{plêdê}{}{[pleˈde]}{(v.)}{2}{}{Perder(-se).}{}{}
\verb{plêdê mêji}{}{[pleˈde ˈmeʒi]}{(expr.)}{1}{}{Estar grávida.}{}{}
\verb{plêdê xintxidu}{}{[pleˈde ʃĩˈtʃidu]}{(expr.)}{1}{}{Desmaiar.}{}{}
\verb{plêdidu}{}{[pleˈdidu]}{(adj.)}{1}{}{Perdido.}{}{}
\verb{plêdjida}{}{[pleˈdʒida]}{(n.)}{1}{}{Má cozinheira.}{}{}
\verb{plêdjida mufuku}{}{[pleˈdʒida mufuˈku]}{(expr.)}{1}{}{Péssima
cozinheira.}{}{}
\verb{plêfêtu}{}{[pleˈfetu]}{(n.)}{1}{}{Prefeito.}{}{}
\verb{plega}{}{[ˈplɛga]}{(n.)}{1}{}{Prega.}{}{}
\verb{plega}{}{[plɛˈga]}{(v.)}{1}{}{Pregar.}{}{}
\verb{plegadu}{}{[plɛˈgadu]}{(adj.)}{1}{}{Pregado.}{}{}
\verb{plegu}{}{[ˈplɛgu]}{(n.)}{1}{}{Prego.}{}{}
\verb{plêjida}{}{[pleˈʒida]}{(n.)}{1}{}{Má cozinheira.}{Cf.
\textbf{plêdjida}.}{}{}
\verb{plêmê}{}{[pleˈme]}{(n.)}{1}{}{Disenteria.}{}{}
\verb{plêmê}{}{[pleˈme]}{(v.)}{1}{}{Espremer.}{}{}%
\verb{plêmê}{}{[pleˈme]}{(v.)}{2}{}{Prensar.}{}{}%
\verb{plêmê}{}{[pleˈme]}{(v.)}{3}{}{Apertar.}{}{}
\verb{plêmêtê}{}{[plemeˈte]}{(v.)}{1}{}{Prometer.}{}{}
\verb{plêmidu}{}{[pleˈmidu]}{(adj.)}{1}{}{Espremido.}{}{}
\verb{plêndê}{}{[plẽˈde]}{(v.)}{1}{}{Perder.}{Cf. \textbf{plêdê}.}{}
\verb{plêndê}{}{[plẽˈde]}{(v.)}{2}{}{Prender.}{}
\verb{plepala}{}{[plɛpaˈla]}{(v.)}{1}{}{Preparar.}{}{}
\verb{plepala}{}{[plɛpaˈla]}{(v.)}{2}{}{Prevenir.}{}{}
\verb{plepleple}{}{[plɛplɛˈplɛ]}{(id.)}{1}{}{Cf. \textbf{kaba
plepleple}.}{}{}
\verb{plepoxta}{}{[plɛˈpɔʃta]}{(n.)}{1}{}{Proposta.}{}{}
\verb{plesa}{}{[plɛˈsa]}{(v.)}{1}{}{Emprestar.}{}{}
\verb{plesadu}{}{[plɛˈsadu]}{(adj.)}{1}{}{Emprestado.}{}{}
\verb{plesu}{}{[ˈplɛsu]}{(n.)}{1}{}{Preço.}{}{}
\verb{pletu}{}{[ˈplɛtu]}{(adj.)}{1}{}{Negro.}{}{}
\verb{pletu}{}{[ˈplɛtu]}{(adj.)}{1}{}{Preto.}{}{}
\verb{pletu}{}{[ˈplɛtu]}{(n.)}{1}{}{Folha-preta.}{\textbf{\textit{Datura metel}.}}{}
\verb{pletu}{}{[ˈplɛtu]}{(n.)}{1}{}{Negro.}{}{}
\verb{pletu}{}{[ˈplɛtu]}{(n.)}{1}{}{Preto.}{}{}
%\verb{pletu}{}{[ˈplɛtu]}{(n.)}{1}{}{Xarroco.}{\textbf{\textit{Lophius piscatorius.}}}{}
\verb{pletu kongô}{}{[ˈplɛtu ˈkõgo]}{(expr.)}{1}{}{Nigérrimo.}{}{}
\verb{pletu kongô}{}{[ˈplɛtu ˈkõgo]}{(expr.)}{2}{}{Pretíssimo.}{}{}
\verb{pletu lululu}{}{[ˈplɛtu luluˈlu]}{(expr.)}{1}{}{Pretíssimo.}{}{}
\verb{pletu lululu}{}{[ˈplɛtu luluˈlu]}{(expr.)}{2}{}{Nigérrimo.}{}{}
\verb{plêzenta}{}{[pleˈzẽta]}{(v.)}{1}{}{Apresentar.}{}{}
\verb{plêzentxi}{}{[pleˈzẽtʃi]}{(n.)}{2}{}{Prenda.}{}{}
\verb{plêzentxi}{}{[pleˈzẽtʃi]}{(n.)}{1}{}{Presente.}{}{}%
\verb{plezu}{}{[ˈplɛzu]}{(adj.)}{1}{}{Preso.}{}{}
\verb{plezu}{}{[ˈplɛzu]}{(n.)}{1}{}{Preso.}{}{}
\verb{pligisa}{}{[pliˈgisa]}{(n.)}{1}{}{Preguiça.}{}{}
\verb{pligisa}{}{[pligiˈsa]}{(v.)}{1}{}{Desmaiar.}{}{}
\verb{pligisa}{}{[pligiˈsa]}{(v.)}{1}{}{Morrer.}{}{}
\verb{pligisozu}{}{[pligiˈsɔzu]}{(adj.)}{1}{}{Preguiçoso.}{}{}
\verb{pligitu}{}{[pliˈgitu]}{(n.)}{1}{}{Periquito.}{}{}
\verb{pligôzu}{}{[pliˈgozu]}{(adj.)}{1}{}{Perigoso.}{}{}
\verb{pligu}{}{[ˈpligu]}{(n.)}{1}{}{Perigo.}{}{}
\verb{plijidentxi}{}{[pliʒid\~ɛˈtʃi]}{(n.)}{1}{}{Presidente.}{Cf.
\textbf{plizidentxi}.}{}{}
\verb{plima}{}{[ˈplima]}{(n.)}{1}{}{Prima.}{}{}
\verb{plimê}{}{[pliˈme]}{(num.)}{1}{}{Primeiro.}{Cf. \textbf{plumêlu}.}{}{}
\verb{plimenta}{}{[pliˈm\~eta]}{(n.)}{1}{}{Pimenta.}{\textbf{\textit{Piper
nigrum}}.}{}
\verb{plimu}{}{[ˈplimu]}{(n.)}{1}{}{Primo.}{}{}
\verb{plinxêza}{}{[plĩˈʃeza]}{(n.)}{1}{}{Princesa.}{}{}
\verb{plinxipi}{}{[ˈplĩʃipi]}{(n.)}{1}{}{Príncipe.}{}{}
\verb{Plinxipi}{}{[ˈplĩʃipi]}{(top.)}{1}{}{Ilha do Príncipe.}{}{}
\verb{plison}{}{[pliˈsõ]}{(n.)}{1}{}{Procissão.}{}{}
\verb{plivini}{}{[pliviˈni]}{(v.)}{1}{}{Preparar.}{}{}
\verb{plivini}{}{[pliviˈni]}{(v.)}{2}{}{Prevenir.}{}{}%
\verb{plivinidu}{}{[pliviˈnidu]}{(adj.)}{1}{}{Preparado.}{}{}
\verb{plivinidu}{}{[pliviˈnidu]}{(adj.)}{2}{}{Prevenido.}{}{}%
\verb{plixigi}{}{[pliʃiˈgi]}{(v.)}{1}{}{Perseguir.}{}{}
\verb{plixigidu}{}{[pliʃiˈgidu]}{(adj.)}{1}{}{Perseguido.}{}{}
\verb{plixiza}{}{[pliʃiˈza]}{(v.)}{1}{}{Necessitar.}{}{}
\verb{plixiza}{}{[pliʃiˈza]}{(v.)}{2}{}{Precisar.}{}{}
\verb{plixizadu}{}{[pliʃiˈzadu]}{(adj.)}{1}{}{Necessitado.}{}{}
\verb{plixizon}{}{[pliʃiˈz\~o]}{(n.)}{1}{}{Necessidade.}{}{}
\verb{plixizon}{}{[pliʃiˈz\~o]}{(n.)}{2}{}{Precisão.}{}{}
\verb{plixizu}{}{[pliˈʃizu]}{(adj.)}{1}{}{Preciso.}{}{}
\verb{plizidentxi}{}{[plizid\~ɛˈtʃi]}{(n.)}{1}{}{Presidente.}{}{}
\verb{plô}{}{[ˈplo]}{(prep.)}{1}{}{Por.}{\textbf{Vintxi kontu plô dja}.
\textit{Vinte contos por dia}.}{}
\verb{plodja}{}{[plɔˈdʒa]}{(n.)}{1}{}{Desdém.}{}{}
\verb{plodja}{}{[plɔˈdʒa]}{(n.)}{2}{}{Mania.}{}{}
\verb{plôdôzu}{}{[ploˈdozu]}{(adj.)}{1}{}{Poderoso.}{}{}
\verb{plofana}{}{[plɔfaˈna]}{(v.)}{1}{}{Criticar.}{}{}
\verb{plofesa}{}{[plɔfɛˈsa]}{(v.)}{1}{}{Professar.}{}{}%
\verb{plofesa}{}{[plɔfɛˈsa]}{(v.)}{2}{}{Prostrar.}{}{}
\verb{ploglama}{}{[plɔˈglama]}{(n.)}{1}{}{Programa.}{}{}
\verb{plôkê}{}{[ploˈke]}{(int.)}{1}{}{Por que.}{}{}%
\verb{plôkô}{}{[ˈploko]}{(n.)}{1}{}{Porco.}{}{}%
\verb{plôkô-balon}{}{[ˈploko baˈlõ]}{(n.)}{1}{}{Varrão.}{}{}
\verb{plôkô-matu}{}{[ˈploko
ˈmatu]}{(n.)}{1}{}{Porco-do-mato.}{\textbf{\textit{Sus scrofa}}.}{}
\verb{plôkôson}{}{[plokoˈsõ]}{(n.)}{1}{}{Erva-tostão.}{\textbf{\textit{Boerhaavia
diffusa}.}}{}
\verb{plôkô-supin}{}{[ˈploko suˈp\~i]}{(n.)}{1}{}{Porco-espinho.}{}{}
\verb{plokuladô}{}{[plɔkulaˈdo]}{(n.)}{1}{}{Procurador.}{}{}
\verb{plôli}{}{[ploˈli]}{(n.)}{1}{}{Disenteria.}{}{}
\verb{plomesa}{}{[plɔˈmɛsa]}{(n.)}{1}{}{Promessa.}{}{}
\verb{plômêtê}{}{[plomeˈte]}{(v.)}{1}{}{Prometer.}{Cf. \textbf{plêmêtê}.}{}{}\verb{plômondêsu}{}{[plomõˈdesu]}{(n.)}{1}{}{\textit{Plômondêsu}.}{Festa
típica de São Tomé em que são recitados provérbios ao som do batuque.}{}%
\verb{plopi}{}{[ˈplɔpi]}{(adj.)}{1}{}{Próprio.}{}{}
\verb{plôsêdê}{}{[ploseˈde]}{(n.)}{1}{}{Juízo.}{}{}
\verb{plôsêdê}{}{[ploseˈde]}{(v.)}{1}{}{Proceder.}{}{}
\verb{plosesu}{}{[plɔˈsɛsu]}{(n.)}{1}{}{Processo.}{}{}%
\verb{plova}{}{[plɔˈva]}{(v.)}{1}{}{Experimentar.}{}{}
\verb{plova}{}{[plɔˈva]}{(v.)}{2}{}{Provar.}{}{}%
\verb{ploveta}{}{[plɔvɛˈta]}{(v.)}{1}{}{Aproveitar.}{}{}
\verb{plovetu}{}{[plɔˈvɛtu]}{(n.)}{1}{}{Proveito.}{}{}
\verb{plovidensya}{}{[plɔviˈd\~ɛsja]}{(n.)}{1}{}{Providência.}{}{}
\verb{plovinxa}{}{[plɔˈv\~iʃa]}{(n.)}{1}{}{Província.}{}{}
\verb{plôvô}{}{[ˈplovo]}{(n.)}{1}{}{Polvo.}{}{}
\verb{plovoka}{}{[plɔvɔˈka]}{(v.)}{1}{}{Afligir.}{}{}
\verb{plovoka}{}{[plɔvɔˈka]}{(v.)}{2}{}{Brigar.}{}{}%
\verb{plovoka}{}{[plɔvɔˈka]}{(v.)}{3}{}{Provocar.}{}{}%
\verb{plôvya}{}{[ˈplovja]}{(conj.)}{2}{}{Devido a.}{}{}
\verb{plôvya}{}{[ˈplovja]}{(conj.)}{1}{}{Por causa de.}{}{}%
\verb{plôwa}{}{[ˈplowa]}{(n.)}{1}{}{Proa.}{}{}
\verb{ploximu}{}{[ˈplɔʃimu]}{(n.)}{1}{}{Próximo.}{}{}
\verb{plufêsôlu}{}{[plufeˈsolu]}{(n.)}{1}{}{Professor.}{}{}
\verb{plufya}{}{[pluˈfja]}{(n.)}{1}{}{Discussão.}{}{}%
\verb{plufya}{}{[pluˈfja]}{(n.)}{1}{}{Disputa.}{}{}%
\verb{plufya}{}{[pluˈfja]}{(n.)}{1}{}{Insistência.}{}{}%
\verb{plufya}{}{[pluˈfja]}{(n.)}{1}{}{Porfia.}{}{}%
\verb{plufya}{}{[pluˈfja]}{(v.)}{2}{}{Discutir.}{}{}%
\verb{plufya}{}{[pluˈfja]}{(v.)}{2}{}{Disputar.}{}{}%
\verb{plufya}{}{[pluˈfja]}{(v.)}{3}{}{Insistir.}{}{}
\verb{plufya}{}{[pluˈfja]}{(v.)}{4}{}{Porfiar.}{}{}%
\verb{pluga}{}{[ˈpluga]}{(n.)}{1}{}{Pulga.}{}{}
\verb{pluga}{}{[pluˈga]}{(n.)}{1}{}{Purgante.}{}{}
\verb{pluga}{}{[pluˈga]}{(v.)}{1}{}{Purgar.}{}{}%
\verb{pluga}{}{[pluˈga]}{(v.)}{2}{}{Ter diarreia.}{}{}
\verb{pluga-matu}{}{[ˈpluga
ˈmatu]}{(n.)}{1}{}{Purgante-do-mato.}{\textbf{\textit{Croton
draconopsis}}.}{}
\verb{plugatoli}{}{[plugaˈtɔli]}{(n.)}{1}{}{Purgatório.}{}{}
\verb{plujizu}{}{[pluˈʒizu]}{(n.)}{1}{}{Prejuízo.}{}{}
\verb{plukê}{}{[pluˈke]}{(conj.)}{1}{}{Porque.}{Cf. \textbf{plukêlu}.}{}{}
\verb{plukêlu}{}{[pluˈkelu]}{(conj.)}{1}{}{Porque.}{}{}
\verb{pluku}{}{[ˈpluku]}{(n.)}{3}{}{Púcaro.}{}{}
\verb{plukutu}{}{[ˈplukutu]}{(n.)}{1}{}{Altar.}{}{}
\verb{plukutu}{}{[ˈplukutu]}{(n.)}{2}{}{Estrado.}{}{}
\verb{plukutu}{}{[ˈplukutu]}{(n.)}{3}{}{Palanque.}{}{}
\verb{plukutu}{}{[ˈplukutu]}{(n.)}{4}{}{Palco.}{}{}
\verb{plukutu}{}{[ˈplukutu]}{(n.)}{5}{}{Pódio.}{}{}
\verb{plukutu}{}{[ˈplukutu]}{(n.)}{6}{}{Tribuna.}{}{}
\verb{plumê}{}{[pluˈme]}{(num.)}{1}{}{Primeiro.}{Cf. \textbf{plumêlu}.}{}{}
\verb{plumêlu}{}{[pluˈmelu]}{(num.)}{1}{}{Primeiro.}{}{}
\verb{plumu}{}{[ˈplumu]}{(n.)}{1}{}{Prumo.}{}{}%
\verb{plumu}{}{[ˈplumu]}{(n.)}{2}{}{Viga.}{}{}
\verb{plunda}{}{[plũˈda]}{(v.)}{1}{}{Pendurar.}{}{}
\verb{plundadu}{}{[plũˈdadu]}{(adj.)}{1}{}{Pendurado.}{}{}
\verb{plundadu}{}{[plũˈdadu]}{(n.)}{1}{}{Prateleira.}{}{}
\verb{pluvidu}{}{[pluˈvidu]}{(adj.)}{1}{}{Proibido.}{}{}
\verb{pluvimentu}{}{[pluviˈm\~ɛtu]}{(adj.)}{1}{}{Proibição.}{}{}
\verb{po}{}{[ˈpɔ]}{(n.)}{1}{}{Árvore.}{}{}%
\verb{po}{}{[ˈpɔ]}{(n.)}{4}{}{Paus.}{Um dos naipes do baralho. Cf. \textbf{fya-paw}.}{}{}
\verb{po}{}{[ˈpɔ]}{(n.)}{2}{}{Madeira.}{}{}%
\verb{po}{}{[ˈpɔ]}{(n.)}{3}{}{Pau.}{}{}
\verb{po}{}{[ˈpɔ]}{(n.)}{5}{}{Pênis.}{}{}
\verb{po}{}{[ˈpɔ]}{(n.)}{6}{}{Ramo.}{}{}%
\verb{pô}{}{[ˈpo]}{(v.)}{1}{}{Poder.}{}{}%
\verb{pô}{}{[ˈpo]}{(v.)}{2}{}{Ser capaz de.}{}{}
\verb{po-ama}{}{[ˈpɔ ˈama]}{(n.)}{1}{}{Pau-ama.}{\textbf{\textit{Premna
angolensis}.}}{}
\verb{po-awa}{}{[ˈpɔ ˈawa]}{(n.)}{1}{}{Pau-água.}{\textbf{\textit{Grumilea
venosa}.}}{}
\verb{po-bakatxi}{}{[ˈpɔ
baˈkatʃi]}{(n.)}{1}{}{Abacateiro.}{\textbf{\textit{Persea americana}}.}{}
\verb{po-blanku}{}{[ˈpɔ
ˈbl\~{\textturna}ku]}{(n.)}{1}{}{Pau-branco.}{\textbf{\textit{Tetrorchidium
didymostemom}.}}{}
\verb{pôblêza}{}{[poˈbleza]}{(n.)}{1}{}{Pobreza.}{}{}
\verb{pobli}{}{[ˈpɔbli]}{(adj.)}{1}{}{Pobre.}{}{}%
\verb{pobli}{}{[ˈpɔbli]}{(n.)}{2}{}{Pobreza.}{}{}
\verb{pobli vantenadu}{}{[ˈpɔbli
v\~{\textturna}tɛˈnadu]}{(expr.)}{1}{}{Paupérrimo.}{}{}
\verb{pobli zegezege}{}{[ˈpɔbli zɛˈgɛzɛˈgɛ]}{(expr.)}{1}{}{Paupérrimo.}{}{}
\verb{poda}{}{[ˈpɔda]}{(n.)}{1}{}{Poda.}{}{}%
\verb{poda}{}{[pɔˈda]}{(v.)}{2}{}{Castrar.}{}{}
\verb{poda}{}{[pɔˈda]}{(v.)}{1}{}{Perdoar.}{}{}
\verb{poda}{}{[pɔˈda]}{(v.)}{1}{}{Podar.}{}{}%
\verb{po-dadu}{}{[ˈpɔ ˈdadu]}{(n.)}{1}{}{Pau-dado.}{\textbf{\textit{Ouratea
nutans}}.}{}
\verb{podadu}{}{[pɔˈdadu]}{(adj.)}{1}{}{Castrado.}{}{}
\verb{podadu}{}{[pɔˈdadu]}{(adj.)}{2}{}{Perdoado.}{}{}
\verb{podadu}{}{[pɔˈdadu]}{(adj.)}{3}{}{Podado.}{}{}
\verb{pôdê}{}{[poˈde]}{(n.)}{1}{}{Poder.}{}{}
\verb{pôdêtê}{}{[poˈdete]}{(n.)}{1}{}{Leitão.}{}{}
\verb{po-di-kola}{}{[ˈpɔ di ˈkɔla]}{(n.)}{1}{}{Pau-de-cola.}{\textbf{\textit{Sterculia acuminata}.}}{}
\verb{pôdja}{}{[poˈdʒa]}{(v.)}{1}{}{Devia.}{}{}
\verb{pôdja}{}{[poˈdʒa]}{(v.)}{2}{}{Podia.}{}{}%
\verb{po-djaka}{}{[ˈpɔ ˈdʒaka]}{(n.)}{1}{}{Jaqueira.}{\textbf{\textit{Artocarpus heterophylla}}.}{}
\verb{podji}{}{[ˈpɔdʒi]}{(v.)}{1}{}{Aguentar.}{}{}%
\verb{podji}{}{[ˈpɔdʒi]}{(v.)}{2}{}{Conseguir.}{}{}
\verb{podle}{}{[ˈpɔdlɛ]}{(adj.)}{1}{}{Podre.}{}{}
\verb{po-d'olho}{}{[ˈpɔ ˈdɔʎɔ]}{(n.)}{1}{}{Pau-óleo.}{\textbf{\textit{Santiria trimera}}.}{}
\verb{po-d'olho-d'ôbô}{}{[ˈpɔ ˈdɔʎɔ doˈbo]}{(n.)}{1}{}{Óleo-barão.}{\textbf{\textit{Symphonia globulifera}}.}{}
\verb{po-dumu}{}{[ˈpɔ duˈmu]}{(n.)}{1}{}{Pau-dumo.}{\textbf{\textit{Ochna membranacea}.}}{}
\verb{po-dumu}{}{[ˈpɔ duˈmu]}{(n.)}{1}{}{Pau-dumo.}{\textbf{\textit{Psychotria molleri}.}}{}
\verb{po-fede}{}{[ˈpɔ fɛˈdɛ]}{(n.)}{1}{}{Pau-fede.}{\textbf{\textit{Celtis gomphophylla}.}}{}
\verb{po-felu}{}{[ˈpɔ ˈfɛlu]}{(n.)}{1}{}{Pau-ferro.}{\textbf{\textit{Margaritaria discoidea}.}}{}
\verb{po-flêminga}{}{[ˈpɔ fleˈmĩga]}{(n.)}{1}{}{Pau-formiga.}{\textbf{\textit{Pauridiantha floribunda}.}}{}
\verb{po-floku}{}{[ˈpɔ ˈflɔku]}{(n.)}{1}{}{Forca.}{}{}
\verb{po-floli}{}{[ˈpɔ ˈflɔli]}{(n.)}{1}{}{Pau-flor.}{\textit{\textbf{Breynia disticha}}.}{}
\verb{po-fluta}{}{[ˈpɔ ˈfluta]}{(n.)}{1}{}{Árvore-do-pão.}{\textit{\textbf{Artocarpus incisa}}.}{}
\verb{po-gamela}{}{[ˈpɔ ˈgamɛla]}{(n.)}{1}{}{Gameleira.}{\textbf{\textit{Ficus doliaria}.}}{}
\verb{po-impe}{}{[ˈpɔ \~iˈpɛ]}{(n.)}{1}{}{\textit{Po-impe}.}{\textbf{\textit{Olea capensis}.}}{}
\verb{pôja}{}{[poˈʒa]}{(v.)}{1}{}{Devia.}{}{}{}
\verb{pôja}{}{[poˈʒa]}{(v.)}{1}{}{Podia.}{Cf. \textbf{pôdja}.}{}{}
\verb{poji}{}{[ˈpɔʒi]}{(n.)}{1}{}{Luxo.}{}{}%
\verb{poji}{}{[ˈpɔʒi]}{(n.)}{2}{}{Ostentação.}{}{}%
\verb{poji}{}{[ˈpɔʒi]}{(n.)}{3}{}{Vaidade.}{}{}
\verb{po-kabla}{}{[ˈpɔ ˈkabla]}{(n.)}{1}{}{Pau-cabra.}{\textbf{\textit{Trema orientalis}}.}{}
\verb{po-kabole}{}{[ˈpɔ kabɔˈlɛ]}{(n.)}{1}{}{Cabolé.}{\textbf{\textit{Anisophyllea cabole}.}}{}
\verb{po-kadela}{}{[ˈpɔ kaˈdɛla]}{(n.)}{1}{}{Cintura.}{}{}
\verb{po-kadela}{}{[ˈpɔ kaˈdɛla]}{(n.)}{2}{}{Pau-leite.}{\textbf{\textit{Funtumia elastica}}.}{}
\verb{po-kadela}{}{[ˈpɔ kaˈdɛla]}{(n.)}{3}{}{Quadril.}{}{}
\verb{po-kafe}{}{[ˈpɔ kaˈfɛ]}{(n.)}{1}{}{Cafeeiro.}{\textbf{\textit{Coffea arabica}}.}{}
\verb{po-kafe}{}{[ˈpɔ kaˈfɛ]}{(n.)}{2}{}{Pé-de-café.}{}{}
\verb{po-kakaw}{}{[ˈpɔ kaˈkaw]}{(n.)}{1}{}{Cacaueiro.}{\textit{\textbf{Theobroma cacao}}.}{}
\verb{po-kali}{}{[ˈpɔ kaˈli]}{(n.)}{1}{}{\textit{Po-kali}.}{\textbf{\textit{Alchonea laxiflora}.}}{}
\verb{po-kanela}{}{[ˈpɔ kaˈnɛla]}{(n.)}{1}{}{Caneleira.}{\textbf{\textit{Cinnamomum verum}}.}{}%
\verb{po-kanela-d'ôbô}{}{[ˈpɔ kaˈnɛla doˈbo]}{(n.)}{1}{}{Caneleira-brava.}{\textbf{\textit{Cinnamomum burmanni}}.}{}{}
\verb{po-kapiton}{}{[ˈpɔ kapiˈt\~ɔ]}{(n.)}{1}{}{Pau-capitão.}{\textbf{\textit{Celtis mildbraedii}.}}{}
\verb{po-kason}{}{[ˈpɔ kaˈs\~ɔ]}{(n.)}{1}{}{Pau-caixão.}{\textbf{\textit{Pycnanthus angolensis}.}}{}
\verb{po-kaxtanha}{}{[ˈpɔ kaʃˈt\~{\textturna}ɲa]}{(n.)}{1}{}{Pau-castanha.}{\textbf{\textit{Artocarpus altilis}}.}{}
\verb{po-kazu}{}{[ˈpɔ kaˈzu]}{(n.)}{1}{}{Cajueiro.}{\textbf{\textit{Anacardium occidentale}}.}{}
\verb{po-kina}{}{[ˈpɔ ˈkina]}{(n.)}{1}{}{Quineira.}{\textbf{\textit{Chinchona carabayensis}}.}{}
\verb{po-kitxiba}{}{[ˈpɔ kiˈtʃiba]}{(n.)}{1}{}{Bananeira.}{\textit{\textbf{Musa paradisiaca}}.}{}
\verb{po-klusu}{}{[ˈpɔ kluˈsu]}{(n.)}{1}{}{Alho-d'ôbô.}{Cf. \textbf{ayu-d'ôbô}.}{}{}
\verb{po-klusu}{}{[ˈpɔ kluˈsu]}{(n.)}{2}{}{Pau-cruz.}{\textbf{\textit{Psychotria peduncularis}}.}{}
\verb{pôkô}{}{[ˈpoko]}{(adv.)}{1}{}{Pouco.}{\textbf{Sun ka nganha pôkô.} \emph{Ele ganha pouco.}}{}{}
\verb{pôkô}{}{[ˈpoko]}{(quant.)}{1}{}{Pouco.}{\textbf{Kuma, n tê pôkô sotxi.} \emph{Comadre, tenho pouca sorte.}}{}{}
\verb{po-kola}{}{[ˈpɔ ˈkɔla]}{(n.)}{1}{}{Coleira.}{\textbf{\textit{Cola acuminata}}.}{}
\verb{poladêsu}{}{[ˈpɔla ˈdesu]}{(n.)}{1}{}{Defunto velado, após o \textbf{fisu}.}{}{}
\verb{po-lanza-d'ôbô}{}{[ˈpɔ ˈl\~{\textturna}za doˈbo]}{(n.)}{1}{}{Laranjeira-do-mato.}{\textbf{\textit{Citrus aurantium}.}}{}
\verb{po-lanza-doxi}{}{[ˈpɔ ˈl\~{\textturna}za ˈdɔʃi]}{(n.)}{2}{}{Laranjeira.}{\textbf{\textit{Citrus sinensis}.}}{}
\verb{po-lanza-matu}{}{[ˈpɔ ˈl\~{\textturna}za ˈmatu]}{(n.)}{1}{}{Laranjeira-do-mato.}{}{}%
\verb{po-limon}{}{[ˈpᴐ liˈm\~ᴐ]}{(n.)}{1}{}{Limoeiro.}{\textbf{\textit{Citrus limonum}.}}{}
\verb{politika}{}{[pɔˈlitika]}{(n.)}{1}{}{Política.}{}{}
\verb{polivla}{}{[ˈpɔlivla]}{(n.)}{1}{}{Pólvora.}{}{}
\verb{polixa}{}{[pɔˈliʃa]}{(n.)}{2}{}{Agente de polícia.}{}{}
\verb{polixa}{}{[pɔˈliʃa]}{(n.)}{1}{}{Polícia.}{}{}%
\verb{po-lixa}{}{[ˈpᴐ ˈliʃa]}{(n.)}{1}{}{Folha-lixa.}{\textbf{\textit{Ficus exasperata}.}}{}
\verb{po-lukula}{}{[ˈpᴐ lukuˈla]}{(n.)}{1}{}{Pau-nicolau.}{\textbf{\textit{Pauridiantha floribunda}.}}{}
\verb{po-mampyan}{}{[ˈpɔ m\~{\textturna}ˈpj\~{\textturna}]}{(n.)}{1}{}{Marapião.}{\textbf{\textit{Zanthoxylum gillettii}}.}{}
\verb{po-manga}{}{[ˈpɔ ˈm\~{\textturna}ga]}{(n.)}{1}{}{Mangueira.}{\textit{\textbf{Mangifera indica}}.}{}
\verb{pombin}{}{[põˈbĩ]}{(n.)}{1}{}{Pombo de São Tomé.}{\textbf{\textit{Columba thomensis}}.}{}
\verb{pombô}{}{[põˈbo]}{(n.)}{1}{}{Pombo.}{}{}%
\verb{po-min}{}{[ˈpɔ ˈmĩ]}{(n.)}{2}{}{Milheiro.}{\textbf{\textit{Zea mays}}.}{}%
\verb{po-min}{}{[ˈpɔ ˈmĩ]}{(n.)}{1}{}{Sabugo.}{}{}%
\verb{po-moli}{}{[ˈpɔ ˈmɔli]}{(n.)}{1}{}{Pau-branco.}{\textbf{\textit{Tetrorchidium didymostemon}.}}{}
\verb{po-mpyan}{}{[ˈpɔ ˈmpj\~{\textturna}]}{(n.)}{1}{}{Pau-espinho.}{\textbf{\textit{Dalbergia ecastaphyllum}.}}{}
\verb{pondja}{}{[ˈpõdʒa]}{(n.)}{1}{}{Esponja.}{}{}
\verb{po-ngweva}{}{[ˈpɔ ˈŋgwɛva]}{(n.)}{1}{}{Goiabeira.}{\textit{\textbf{Psidium guajava}}.}{}
\verb{ponja}{}{[ˈpõʒa]}{(n.)}{1}{}{Esponja.}{Cf. \textbf{pondja}.}{}{}
\verb{ponta}{}{[ˈp\~ɔta]}{(n.)}{1}{}{Bico.}{}{}%
\verb{ponta}{}{[ˈp\~ɔta]}{(n.)}{2}{}{Extremidade.}{}{}
\verb{ponta}{}{[ˈp\~ɔta]}{(n.)}{3}{}{Ponta.}{}{}
\verb{ponta}{}{[p\~ɔˈta]}{(v.)}{1}{}{Apontar.}{}{}%
\verb{ponta}{}{[p\~ɔˈta]}{(v.)}{2}{}{Indicar.}{}{}
\verb{pontada}{}{[põˈtada]}{(n.)}{1}{}{Flanco do tórax.}{}{}%
\verb{pontada}{}{[põˈtada]}{(n.)}{2}{}{Pontada.}{}{}%
\verb{pontadu}{}{[põˈtadu]}{(adj.)}{1}{}{Quase maduro.}{}{}
\verb{ponta-mama}{}{[ˈp\~ɔta ˈmama]}{(n.)}{1}{}{Mamilo.}{}{}
\verb{ponta-mama}{}{[ˈp\~ɔta ˈmama]}{(n.)}{2}{}{Teta.}{}{}
\verb{pontope}{}{[p\~ɔtɔˈpɛ]}{(n.)}{1}{}{Pontapé.}{}{}
\verb{pontu}{}{[ˈp\~ɔtu]}{(n.)}{1}{}{Ponto.}{}{}
\verb{pontxi}{}{[ˈp\~otʃi]}{(n.)}{1}{}{Ponte.}{}{}
\verb{po-olyo}{}{[ˈpɔ ˈɔljɔ]}{(n.)}{1}{}{Bálsamo-de-São-Tomé.}{\textbf{\textit{Santiriops trimera}.}}{}
\verb{po-ova}{}{[ˈpɔ ɔˈva]}{(n.)}{1}{}{Mata-passo.}{\textbf{\textit{Pentadesma butyracea}.}}{}
\verb{pôpa}{}{[ˈpopa]}{(n.)}{1}{}{Popa.}{}{}
\verb{popa}{}{[pɔˈpa]}{(v.)}{1}{}{Poupar.}{}{}
\verb{po-patela}{}{[ˈpɔ paˈtɛla]}{(n.)}{1}{}{Pau-parteira.}{\textbf{\textit{Trema orientalis}}.}{}
\verb{po-pedasu}{}{[ˈpɔ ˈpɛdasu]}{(n.)}{1}{}{Tronco.}{}{}
\verb{po-pletu}{}{[ˈpɔ ˈplɛtu]}{(n.)}{1}{}{Pau-preto.}{\textbf{\textit{Polyalthia oliveri}.}}{}
\verb{po-plimenta}{}{[ˈpɔ pliˈmẽta]}{(n.)}{1}{}{Pimenteira-brava.}{\textbf{\textit{Piper guineense}}.}{}
\verb{po-pluga}{}{[ˈpɔ pluˈga]}{(n.)}{1}{}{Pau-purga.}{\textbf{\textit{Croton draconopsis}.}}{}
\verb{popoy}{}{[pɔˈpɔj]}{(adj.)}{1}{}{Barrigudo.}{}{}
\verb{po-sabon}{}{[ˈpɔ saˈbõ]}{(n.)}{1}{}{Pau-sabão.}{\textbf{\textit{Dracaena arborea}.}}{}
\verb{po-safu}{}{[ˈpɔ saˈfu]}{(n.)}{1}{}{Safuzeiro.}{\textit{\textbf{Canarium edule}}.}{}
\verb{po-sangi}{}{[ˈpɔ ˈs\~{\textturna}gi]}{(n.)}{1}{}{Pau-sangue.}{\textbf{\textit{Harungana madagascariensis}.}}{}
\verb{po-sapilin}{}{[ˈpɔ sapiˈl\~i]}{(n.)}{1}{}{Pau-chapelinho.}{\textbf{\textit{Cola digitata}.}}{}
\verb{pôsô}{}{[ˈposo]}{(n.)}{1}{}{Aglomerado.}{}{}
\verb{pôsô}{}{[ˈposo]}{(n.)}{1}{}{Poço.}{}{}
\verb{po-soda}{}{[ˈpɔ ˈsɔda]}{(n.)}{1}{}{\textit{Gêgê-fasu}.}{Cf. \textbf{gêgê-fasu}.}{}{}
\verb{Poson}{}{[pɔˈs\~ɔ]}{(top.)}{1}{}{Cidade de São Tomé.}{}{}
\verb{pota}{}{[pɔˈta]}{(v.)}{1}{}{Importar(-se).}{}{}
\verb{po-tabaki}{}{[ˈpɔ taˈbaki]}{(n.)}{1}{}{Pau-tabaque.}{\textbf{\textit{Cordia platythyrsa}.}}{}
\verb{po-tlêxi}{}{[ˈpɔ ˈtleʃi]}{(n.)}{1}{}{Pau-três.}{\textbf{\textit{Allophylus africanus}.}}{}
\verb{po-tlêxi-d'ôbô}{}{[ˈpɔ ˈtleʃi doˈbo]}{(n.)}{1}{}{Pau-três-do-mato.}{\textbf{\textit{Allophylus grandifolius}.}}{}
\verb{poto}{}{[ˈpɔtɔ]}{(n.)}{1}{}{Porta.}{}{}%
\verb{poto}{}{[pɔˈtɔ]}{(v.)}{1}{}{Apodrecer.}{}{}
\verb{poto}{}{[pɔˈtɔ]}{(v.)}{2}{}{Pisar.}{}{}%
\verb{potodu}{}{[pɔˈtɔdu]}{(adj.)}{2}{}{Apodrecido.}{}{}
\verb{potodu}{}{[pɔˈtɔdu]}{(adj.)}{1}{}{Pisado.}{}{}%
\verb{poto-nglandji}{}{[pɔˈtɔ ˈŋgl\~{\textturna}dʒi]}{(n.)}{1}{}{Portão.}{}{}
\verb{potopoto}{}{[pɔˈtɔpɔˈtɔ]}{(id.)}{1}{}{Cf. \textbf{monha potopoto.}}{}{}
\verb{potopoto}{}{[pɔˈtɔpɔˈtɔ]}{(id.)}{2}{}{Cf. \textbf{sola potopoto.}}{}{}
\verb{potxi}{}{[ˈpɔtʃi]}{(n.)}{1}{}{Pote.}{}{}
\verb{potxi}{}{[ˈpɔtʃi]}{(n.)}{2}{}{Vaso de barro.}{}{}
\verb{po-vilo}{}{[ˈpɔ viˈlɔ]}{(n.)}{1}{}{\textit{Po-vilo}.}{\textbf{\textit{Scytopetalum klaineanum}.}}{}
\verb{po-vlêmê}{}{[ˈpɔ vleˈme]}{(n.)}{1}{}{Pau-vermelho.}{\textbf{\textit{Stauditia pterocarpa}.}}{}
\verb{pôvô}{}{[ˈpovo]}{(n.)}{1}{}{Povo.}{}{}
\verb{poxi}{}{[ˈpɔʃi]}{(n.)}{1}{}{Capacidade.}{}{}
\verb{poxi}{}{[ˈpɔʃi]}{(n.)}{2}{}{Posses.}{}{}
\verb{poxi}{}{[ˈpɔʃi]}{(n.)}{3}{}{Possibilidades.}{}{}
\verb{poxta}{}{[ˈpɔʃta]}{(n.)}{1}{}{Aposta.}{}{}
\verb{poxta}{}{[ˈpɔʃta]}{(n.)}{2}{}{Posta.}{}{}
\verb{poxta}{}{[pɔʃˈta]}{(v.)}{1}{}{Apostar.}{}{}
\verb{pôxtêma}{}{[poʃˈtema]}{(n.)}{1}{}{Abcesso.}{}{}
\verb{pôxtêma}{}{[poʃˈtema]}{(n.)}{1}{}{Furúnculo.}{}{}
\verb{pôxtu}{}{[ˈpoʃtu]}{(adj.)}{1}{}{Posto.}{}{}
\verb{pôxtu}{}{[ˈpoʃtu]}{(n.)}{1}{}{Posto.}{}{}
\verb{poya}{}{[pɔˈja]}{(v.)}{1}{}{Apoiar.}{}{}
\verb{pôza}{}{[poˈza]}{(v.)}{1}{}{Pousar.}{}{}
\verb{pôzê}{}{[poˈze]}{(v.)}{1}{}{Aproximar(-se).}{}{}%
\verb{pôzê}{}{[poˈze]}{(v.)}{2}{}{Situar(-se).}{}{}
\verb{po-zêkentxi}{}{[ˈpɔ
zeˈkẽtʃi]}{(n.)}{1}{}{Izaquenteiro.}{\textbf{\textit{Treculia africana}.}}{}
\verb{po-zetona}{}{[ˈpɔ
zɛˈt\~ɔna]}{(n.)}{1}{}{Pau-azeitona.}{\textbf{\textit{Manilkara
multinervis}.}}{}
\verb{pu}{}{[ˈpu]}{(id.)}{1}{}{Cf. \textbf{xa pu.}}{}{}
\verb{pujita}{}{[puʒiˈta]}{(v.)}{1}{}{Amantizar.}{}{}
\verb{pujita}{}{[puʒiˈta]}{(v.)}{2}{}{Amigar.}{}{}{}
\verb{pujita}{}{[puʒiˈta]}{(v.)}{3}{}{Viver maritalmente.}{}{}
\verb{pujitadu}{}{[puʒiˈtadu]}{(adj.)}{1}{}{Amantizado.}{}{}
\verb{pujitadu}{}{[puʒiˈtadu]}{(adj.)}{1}{}{Amigado.}{}{}
\verb{puke}{}{[ˈpukɛ]}{(n.)}{1}{}{Púcaro.}{}{}
\verb{pulidu}{}{[puˈlidu]}{(adj.)}{1}{}{Polido.}{}{}
\verb{pulimentu}{}{[puliˈm\~ɛtu]}{(n.)}{1}{}{Polimento.}{}{}
\verb{pulimentu}{}{[puliˈm\~ɛtu]}{(n.)}{2}{}{Verniz.}{}{}
\verb{pulu}{}{[ˈpulu]}{(n.)}{1}{}{Pulo.}{}{}
\verb{pulumon}{}{[puluˈmõ]}{(n.)}{1}{}{Pulmão.}{}{}
\verb{pumbu}{}{[p\~uˈbu]}{(n.)}{1}{}{Vinho de palma de má qualidade.}{}{}
\verb{punda}{}{[ˈpũda]}{(conj.)}{1}{}{Porque.}{}{}
\verb{punda kamanda}{}{[ˈpũda
kam\~{\textturna}ˈda]}{(int.)}{1}{}{Porquê?}{}{}
\verb{punhu}{}{[ˈpuɲu]}{(n.)}{1}{}{Manga.}{}{}
\verb{punhu}{}{[ˈpuɲu]}{(n.)}{2}{}{Punho.}{}{}
\verb{punta}{}{[pũˈta]}{(v.)}{1}{}{Perguntar.}{}{}
\verb{punta}{}{[pũˈta]}{(v.)}{2}{}{Solicitar.}{}{}
\verb{purton}{}{[pu{\textfishhookr}ˈt\~ɔ]}{(n.)}{1}{}{Portão.}{}{}
\verb{pusela}{}{[puˈsɛla]}{(n.)}{1}{}{Pulseira.}{}{}
\verb{pusentu}{}{[puˈs\~ɛtu]}{(n.)}{1}{}{Aposento.}{}{}
\verb{pusentu}{}{[puˈs\~ɛtu]}{(n.)}{2}{}{Reforma.}{}{}
\verb{pusu}{}{[ˈpusu]}{(n.)}{1}{}{Hemorróidas.}{}{}
\verb{pusu}{}{[ˈpusu]}{(n.)}{1}{}{Pulso.}{}{}
\verb{pusuku}{}{[ˈpusuku]}{(n.)}{1}{}{Caixinha.}{}{}
\verb{pusu-mon}{}{[ˈpusu ˈmõ]}{(n.)}{1}{}{Pulso.}{}{}
\verb{Putuga}{}{[putuˈga]}{(top.)}{1}{}{Portugal.}{}{}
\verb{putugêji}{}{[putuˈgeʒi]}{(adj.)}{1}{}{Português.}{}{}
\verb{putugêji}{}{[putuˈgeʒi]}{(n.)}{1}{}{Português.}{}{}
\verb{puxa}{}{[ˈpuʃa]}{(n.)}{1}{}{\textit{Puxa}.}{Dança tradicional.}{}
\verb{pwela}{}{[ˈpwɛla]}{(n.)}{1}{}{Poeira.}{}{}
\verb{pwela}{}{[ˈpwɛla]}{(n.)}{2}{}{Vazio.}{}{}
\verb{pwêta}{}{[ˈpweta]}{(n.)}{1}{}{Puíta.}{}{}
\verb{pya}{}{[ˈpja]}{(v.)}{1}{}{Observar.}{}{}%
\verb{pya}{}{[ˈpja]}{(v.)}{2}{}{Olhar.}{}{}%
\verb{pya}{}{[ˈpja]}{(v.)}{3}{}{Ver.}{}{}
\verb{pya babaka}{}{[ˈpja babaˈka]}{(expr.)}{1}{}{Estar pasmo.}{}
\verb{pyadadji}{}{[pjaˈdadʒi]}{(n.)}{1}{}{Piedade.}{}{}
\verb{pyadô}{}{[pjaˈdo]}{(n.)}{1}{}{Espectador.}{}{}
\verb{pyadô-zawa}{}{[pjaˈdo ˈzawa]}{(n.)}{1}{}{\textit{Pyadô-zawa}.}{Terapeuta tradicional que examina a urina.}{}{}
\verb{pyala}{}{[ˈpjala]}{(n.)}{1}{}{\textit{Pyala}.}{Cf. \textbf{mpyala}.}{}{}
\verb{pyan}{}{[ˈpj\~{\textturna}]}{(n.)}{2}{}{Cacho.}{}{}{}%
\verb{pyan}{}{[ˈpj\~{\textturna}]}{(n.)}{2}{}{Espinha de peixe.}{}{}{}%
\verb{pyan}{}{[ˈpj\~{\textturna}]}{(n.)}{1}{}{Espinho.}{}{}{}%
\verb{pyan}{}{[ˈpj\~{\textturna}]}{(n.)}{3}{}{Pinha.}{Cf. \textbf{mpyan}.}{}{}%
\verb{pyan-tlaxi}{}{[ˈpj\~{\textturna} ˈtlaʃi]}{(n.)}{2}{}{Coluna vertebral.}{}{}
\verb{pyan-tlaxi}{}{[ˈpj\~{\textturna} ˈtlaʃi]}{(n.)}{1}{}{Espinha dorsal.}{}{}%
\verb{pya s\~u\~u\~u}{}{[ˈpja ˈs\~u\~u\~u]}{(expr.)}{1}{}{Olhar fixamente.}{}{}
\verb{pyenepyene}{}{[pjɛˈnɛpjɛˈnɛ]}{(id.)}{1}{}{Cf. \textbf{limpu pyenepyene.}}{}{}
\verb{pyepyepye}{}{[pjɛpjɛˈpjɛ]}{(id.)}{1}{}{Cf. \textbf{limpu pyepyepye.}}{}{}
\verb{pyola}{}{[pjɔˈla]}{(v.)}{2}{}{Agravar-se.}{}{}
\verb{pyola}{}{[pjɔˈla]}{(v.)}{1}{}{Piorar.}{}{}%
\verb{pyoladu}{}{[pjɔˈladu]}{(adj.)}{1}{}{Piorado.}{}{}
\verb{pyolo}{}{[ˈpjɔlɔ]}{(adv.)}{1}{}{Pior.}{}{}
\verb{pyolopyolo}{}{[pjɔˈlɔpjɔˈlɔ]}{(adv.)}{1}{}{Frequentemente.}{}{}
\verb{pyonelu}{}{[pjɔˈnɛlu]}{(n.)}{1}{}{Pioneiro.}{}{}
\verb{pyonpyonpyon}{}{[pjõpjõˈpjõ]}{(adj.)}{1}{}{Aguçado.}{}{}
\verb{pyonpyonpyon}{}{[pjõpjõˈpjõ]}{(adj.)}{2}{}{Pontiagudo.}{}{}
\verb{pyuku}{}{[pjuˈku]}{(n.)}{1}{}{Coscuvilheiro.}{}{}%
\verb{pyuku}{}{[pjuˈku]}{(n.)}{2}{}{Curioso.}{}{}
\end{letra}

\begin{letra}{r}
\verb{radiu}{}{[{\textfishhookr}aˈdiu]}{(n.)}{1}{}{Rádio}{}{}
\verb{radiu}{}{[{\textfishhookr}aˈdiu]}{(n.)}{1}{}{Emissora de rádio.}{Cf. \textbf{misôla}.}{}
\verb{raix-keblada}{}{[{\textfishhookr}aˈiʃ kɛˈblada]}{(n.)}{1}{}{Raiz quadrada.}{}{}
\verb{rew}{}{[ˈ{\textfishhookr}ɛw]}{(n.)}{1}{}{Réu.}{}{}
\verb{rivixta}{}{[{\textfishhookr}iˈviʃta]}{(n.)}{1}{}{Revista.}{}{}
\verb{rumba}{}{[ˈ{\textfishhookr}\~uba]}{(n.)}{1}{}{Rumba.}{}{}
\end{letra}

\begin{letra}{s}

\verb{sa}{}{[ˈsa]}{(v.)}{1}{}{Assar.}{}{}
\verb{sa}{}{[ˈsa]}{(v.)}{2}{}{Custar.}{}{}
\verb{sa}{}{[ˈsa]}{(v.)}{3}{}{Estar.}{}{}%
\verb{sa}{}{[ˈsa]}{(v.)}{4}{}{Ser.}{}{}%
\verb{sabadu}{}{[ˈsabadu]}{(n.)}{1}{}{Sábado.}{}{}
\verb{sabanku}{}{[saˈb\~{\textturna}ku]}{(n.)}{1}{}{Chanca.}{}{}
\verb{sabi}{}{[ˈsabi]}{(n.)}{1}{}{Chave.}{}{}
\verb{sabola}{}{[saˈbɔla]}{(n.)}{1}{}{Cebola.}{}{}
\verb{sabola-sensê}{}{[saˈbɔla s\~eˈse]}{(n.)}{1}{}{Flor-branca.}{\textit{\textbf{Crinum jagus}}.}{}
\verb{sabon}{}{[saˈb\~ɔ]}{(n.)}{1}{}{Sabão.}{}{}
\verb{sabunêtê}{}{[sabuˈnete]}{(n.)}{1}{}{Sabonete.}{}{}
\verb{sada}{}{[ˈsada]}{(n.)}{1}{}{Enxada.}{}{}
\verb{sada}{}{[ˈsada]}{(n.)}{2}{}{Pegada.}{}{}
\verb{sada}{}{[ˈsada]}{(n.)}{3}{}{Tartaruga-de-couro.}{\textbf{\textit{Dermochelys coriacea}.}}{}
\verb{sa di}{}{[ˈsa ˈdi]}{(expr.)}{1}{}{Estar para.}{}{}
\verb{sa di}{}{[ˈsa ˈdi]}{(expr.)}{2}{}{Estar prestes a.}{}{}
\verb{sadu}{}{[ˈsadu]}{(adj.)}{1}{}{Assado.}{}{}
\verb{safa}{}{[saˈfa]}{(v.)}{1}{}{Branquear.}{}{}
\verb{safa}{}{[saˈfa]}{(v.)}{1}{}{Clarear.}{}{}
\verb{saflon}{}{[saˈfl\~ɔ]}{(n.)}{1}{}{Açafrão.}{\textbf{\textit{Curcuma longa}}.}{}
\verb{safu}{}{[saˈfu]}{(n.)}{1}{}{Safu.}{}{}
\verb{safu-d'ôbô}{}{[saˈfu doˈbo]}{(n.)}{1}{}{Safu-do-mato.}{\textbf{\textit{Pseudospondias microcarpa}.}}{}
\verb{sagudji}{}{[saguˈdʒi]}{(v.)}{1}{}{Sacudir.}{}{}
\verb{saguji}{}{[saguˈʒi]}{(v.)}{1}{}{Sacudir.}{Cf. \textbf{sagudji}.}{}{}
\verb{sagwa}{}{[saˈgwa]}{(v.)}{1}{}{Bochechar.}{}{}%
\verb{sagwa}{}{[saˈgwa]}{(v.)}{2}{}{Enxaguar.}{}{}
\verb{saka}{}{[ˈsaka]}{(n.)}{1}{}{Bolsa.}{}{}
\verb{saka}{}{[ˈsaka]}{(n.)}{2}{}{Saca.}{}{}
\verb{saka}{}{[saˈka]}{(n.)}{1}{}{Vômito.}{}{}%
\verb{saka}{}{[ˈsaka]}{(part.)}{1}{}{Partícula que exprime o progressivo apectual .}{\textbf{Bega saka klêsê.} \textit{A barriga está a crescer.}{Cf. \textbf{xka}.}}{}
\verb{saka}{}{[saˈka]}{(v.)}{1}{}{Sacar.}{}{}%
\verb{saka}{}{[saˈka]}{(v.)}{2}{}{Tirar.}{}{}%
\verb{saka}{}{[saˈka]}{(v.)}{3}{}{Vomitar.}{}{}
\verb{saka-nzolo}{}{[saˈka ˈnzɔlɔ]}{(n.)}{1}{}{Anzolama.}{}{}
\verb{saka-nzolo}{}{[saˈka ˈnzɔlɔ]}{(n.)}{2}{}{Puxa-conversa.}{}{}
\verb{sakapuli}{}{[sakapuˈli]}{(v.)}{1}{}{Escapulir.}{}{}
\verb{sakapuli}{}{[sakapuˈli]}{(v.)}{2}{}{Fugir.}{}{}
\verb{sakapuli}{}{[sakapuˈli]}{(v.)}{3}{}{Resolver.}{}{}
\verb{sakasaka}{}{[saˈkasaˈka]}{(n.)}{1}{}{Bruto.}{}{}
\verb{sakasaka}{}{[saˈkasaˈka]}{(n.)}{2}{}{Desajeitado.}{}{}
\verb{sakasaka}{}{[saˈkasaˈka]}{(n.)}{3}{}{Pessoa grosseira.}{}{}
\verb{sakaya}{}{[saˈkaja]}{(n.)}{2}{}{Chocalho.}{}{}%
\verb{sakaya}{}{[saˈkaja]}{(n.)}{3}{}{Maraca.}{}{}
%\verb{sakaya}{}{[saˈkaja]}{(n.)}{1}{}{\textit{Sakaya}.}{Instrumento musical semelhante ao \textbf{xkalhu}, que consiste em um pequeno receptáculo feito de ramo da palmeira ou de coqueiro repleto de sementes.}{}{}%
\verb{saklamentu}{}{[saklaˈm\~ɛtu]}{(n.)}{1}{}{Sacramento.}{}{}
\verb{saku}{}{[ˈsaku]}{(n.)}{1}{}{Saco.}{}{}
\verb{sakusaku}{}{[saˈkusaˈku]}{(adj.)}{1}{}{Desajeitado.}{}{}
\verb{sakusaku}{}{[saˈkusaˈku]}{(adj.)}{2}{}{Estabanado.}{}{}
\verb{sala}{}{[saˈla]}{(dem.)}{1}{}{Aquela.}{}{}%
\verb{sala}{}{[saˈla]}{(dem.)}{2}{}{Aquelas.}{}{}%
\verb{sala}{}{[saˈla]}{(dem.)}{3}{}{Aquele.}{}{}%
\verb{sala}{}{[saˈla]}{(dem.)}{4}{}{Aqueles.}{}{}%
\verb{sala}{}{[ˈsala]}{(n.)}{1}{}{Dança.}{}{}
\verb{sala}{}{[ˈsala]}{(n.)}{2}{}{Sala.}{}{}%
\verb{sala}{}{[ˈsala]}{(n.)}{3}{}{Sala de estar.}{}{}
\verb{sala}{}{[saˈla]}{(v.)}{1}{}{Acrescentar.}{}{}
\verb{sala}{}{[saˈla]}{(v.)}{2}{}{Acrescer.}{}{}
\verb{sala}{}{[saˈla]}{(v.)}{3}{}{Completar.}{}{}
\verb{sala}{}{[saˈla]}{(v.)}{4}{}{Perfazer.}{}{}%
\verb{sala-d'ope}{}{[ˈsala dɔˈpɛ]}{(n.)}{1}{}{Planta do pé.}{}{}%
\verb{sala-ke}{}{[ˈsala ˈkɛ]}{(n.)}{1}{}{Sala de estar.}{}{}
\verb{salakonta}{}{[salaˈkõta]}{(n.)}{1}{}{Cana-de-jardim.}{\textbf{\textit{Canna bidentata}.}}{}
\verb{salamba}{}{[sal\~{\textturna}ˈba]}{(n.)}{1}{}{\textit{Salamba}.}{\textbf{\textit{Dialium guineense}.}}{}
\verb{sala-mon}{}{[ˈsala ˈmõ]}{(n.)}{1}{}{Destino.}{}{}
\verb{sala-mon}{}{[ˈsala ˈmõ]}{(n.)}{2}{}{Palma da mão.}{}{}
\verb{sala-mon}{}{[ˈsala ˈmõ]}{(n.)}{3}{}{Sina.}{}{}
\verb{salasa}{}{[saˈlasa]}{(n.)}{1}{}{Chalaça.}{}{}
\verb{salmu}{}{[ˈsalmu]}{(n.)}{1}{}{Salmo.}{}{}
\verb{salon}{}{[saˈl\~ɔ]}{(n.)}{1}{}{Salão.}{}{}
\verb{salu}{}{[ˈsalu]}{(n.)}{1}{}{Sal.}{}{}
\verb{salu}{}{[saˈlu]}{(v.)}{1}{}{Arrebentar.}{}{}
\verb{salu}{}{[saˈlu]}{(v.)}{2}{}{Extravazar.}{}{}
\verb{salu}{}{[saˈlu]}{(v.)}{3}{}{Vazar.}{}{}
\verb{sama}{}{[saˈma]}{(v.)}{1}{}{Chamar.}{}{}
\verb{samada}{}{[saˈmada]}{(n.)}{1}{}{Chamada.}{}{}%
\verb{samada}{}{[saˈmada]}{(n.)}{2}{}{Morte.}{}{}
\verb{samangungu}{}{[sam\~{\textturna}gũˈgu]}{(n.)}{1}{}{Tarântula de São Tomé.}{\textit{\textbf{Hysterocrates apostolicus}}.}{}
\verb{samba}{}{[ˈs\~{\textturna}ba]}{(n.)}{1}{}{Puxão.}{}{}
\verb{samba}{}{[ˈs\~{\textturna}ba]}{(n.)}{1}{}{Samba.}{}{}
\verb{sambawa}{}{[s\~{\textturna}ˈbawa]}{(n.)}{1}{}{Mulher favorita.}{}{}
\verb{sambawa}{}{[s\~{\textturna}ˈbawa]}{(n.)}{2}{}{Mulher grande.}{}{}
\verb{sambu}{}{[s\~{\textturna}ˈbu]}{(n.)}{1}{}{Embrulho.}{}{}
\verb{sambu}{}{[s\~{\textturna}ˈbu]}{(n.)}{2}{}{Grande quantidade.}{}{}
\verb{sambu}{}{[s\~{\textturna}ˈbu]}{(n.)}{3}{}{Reservatório.}{}{}
\verb{samu}{}{[ˈsamu]}{(n.)}{1}{}{Senhora.}{}{}
\verb{samungê}{}{[samũˈge]}{(n.)}{1}{}{Senhora.}{}{}
\verb{san}{}{[ˈs\~{\textturna}]}{(n.)}{2}{}{Senhora.}{}{}
\verb{san}{}{[ˈs\~{\textturna}]}{(poss.)}{1}{}{Seu.}{}{}
\verb{san}{}{[ˈs\~{\textturna}]}{(poss.)}{2}{}{Seus.}{}{}
\verb{san}{}{[ˈs\~{\textturna}]}{(poss.)}{3}{}{Sua.}{}{}
\verb{san}{}{[ˈs\~{\textturna}]}{(poss.)}{4}{}{Suas.}{}{}
\verb{san}{}{[ˈs\~{\textturna}]}{(pron.)}{1}{}{-a (formal).}{Terceira pessoa do singular com a função de complemento direto.}{}{}
\verb{san}{}{[ˈs\~{\textturna}]}{(pron.)}{2}{}{Ela (formal).}{\textbf{San nganha kinte.} \emph{Ela chegou ao quintal.}}{}{}
\verb{san}{}{[ˈs\~{\textturna}]}{(pron.)}{3}{}{-lhe (formal).}{Terceira pessoa do singular com a função de complemento indireto.}{\textbf{N fada san kuma n na sêbê fa.} \emph{Eu disse-lhe que não sabia.}}{}{}
\verb{sana}{}{[saˈna]}{(v.)}{1}{}{Empanturrar(-se).}{}{}{}
\verb{sanatoryu}{}{[sanaˈtɔrju]}{(n.)}{1}{}{Sanatório.}{}{}
\verb{sanda}{}{[ˈs\~{\textturna}da]}{(n.)}{1}{}{Enxada.}{}{}
\verb{sandaleti}{}{[s\~{\textturna}daˈlɛti]}{(n.)}{1}{}{Sandália.}{}{}
\verb{sandja}{}{[s\~{\textturna}ˈdʒa]}{(n.)}{1}{}{Sardinha.}{}{}
\verb{sandja-kasa}{}{[s\~{\textturna}ˈdʒa ˈkasa]}{(n.)}{1}{}{Sardinha pequena.}{}{}%
\verb{sandja-longô}{}{[s\~{\textturna}ˈdʒa ˈlõgo]}{(n.)}{1}{}{Sardinela.}{\textit{\textbf{Sardinella aurita}}.}{}%
\verb{sangê}{}{[s\~{\textturna}ˈge]}{(n.)}{1}{}{Senhora.}{}{}
\verb{sangi}{}{[ˈs\~{\textturna}gi]}{(n.)}{1}{}{Sangue.}{}{}
\verb{sanklixtia}{}{[s\~{\textturna}kliʃˈtia]}{(n.)}{1}{}{Sacristia.}{}{}
\verb{sanklixton}{}{[s\~{\textturna}kliʃˈt\~o]}{(n.)}{1}{}{Sacristão.}{}{}
\verb{san-lenha}{}{[ˈs\~{\textturna} ˈl\~ɛɲa]}{(n.)}{1}{}{Imperatriz.}{}{}
\verb{san-lenha}{}{[ˈs\~{\textturna} ˈl\~ɛɲa]}{(n.)}{2}{}{Rainha.}{Personagem de histórias tradicionais.}{}%
\verb{San-Ma}{}{[ˈs\~{\textturna} ˈma]}{(n.)}{1}{}{Nossa Senhora.}{}{}
\verb{San Maya}{}{[ˈs\~{\textturna} ˈmaja]}{(n.)}{1}{}{Santa Maria.}{}{}
\verb{santa}{}{[ˈs\~{\textturna}ta]}{(adj.)}{1}{}{Santa.}{}{}
\verb{santaji-basu-kafe}{}{[s\~{\textturna}ˈtaʒi ˈbasu kaˈfɛ]}{(n.)}{1}{}{\textit{Santaji-basu-kafe}.}{\textit{\textbf{Elytraria
marginata.}}}{}
\verb{santifikadu}{}{[s\~{\textturna}tifiˈkadu]}{(adj.)}{1}{}{Santificado.}{}{}
\verb{santome}{}{[s\~{\textturna}tɔˈmɛ]}{(n.)}{1}{}{Santome.}{A língua \textbf{santome} de São Tomé e Príncipe.}{}
\verb{santome}{}{[s\~{\textturna}tɔˈmɛ]}{(n.)}{2}{}{Forro.}{Grupo étnico de São Tomé e Príncipe.}{}%
\verb{Santome}{}{[s\~{\textturna}tɔˈmɛ]}{(top.)}{1}{}{Ilha de São Tomé.}{}{}
\verb{santope}{}{[s\~{\textturna}tɔˈpɛ]}{(n.)}{1}{}{Centopéia.}{\textbf{\textit{Otostigmus productus}.}}{}
\verb{santu}{}{[ˈs\~{\textturna}tu]}{(adj.)}{1}{}{Santo.}{}{}
\verb{santu}{}{[ˈs\~{\textturna}tu]}{(n.)}{1}{}{Espírito.}{}{}
\verb{santu}{}{[ˈs\~{\textturna}tu]}{(n.)}{2}{}{Santo.}{}{}
%\verb{San Zon}{}{[ˈs\~{\textturna} ˈzõ]}{(n.)}{1}{}{São João.}{}{}
\verb{saôdji}{}{[saˈodʒi]}{(n.)}{1}{}{Saúde.}{}{}
\verb{saôji}{}{[saˈoʒi]}{(n.)}{1}{}{Saúde.}{Cf. \textbf{saôdji}.}{}{}
\verb{sapatêlu}{}{[sapaˈtelu]}{(n.)}{1}{}{Sapateiro.}{}{}
\verb{sapatu}{}{[saˈpatu]}{(n.)}{1}{}{Sapato.}{}{}
\verb{sapatu-klonklo-longô}{}{[saˈpatu kl\~ɔˈklɔ ˈl\~ogo]}{(n.)}{1}{}{Bota.}{}{}
\verb{sapê}{}{[saˈpe]}{(n.)}{1}{}{Chapéu.}{}{}
\verb{sapê-kôkô}{}{[saˈpe ˈkoko]}{(n.)}{1}{}{Capacete.}{}{}
\verb{sapê-solo}{}{[saˈpe ˈsɔlɔ]}{(n.)}{1}{}{Guarda-sol.}{}{}
\verb{sapê-suba}{}{[saˈpe ˈsuba]}{(n.)}{1}{}{Guarda-chuva.}{}{}
\verb{saplamada}{}{[saplaˈmada]}{(n.)}{1}{}{Palmada.}{}{}
\verb{sasa}{}{[saˈsa]}{(v.)}{2}{}{Capinar.}{}{}%
\verb{sasa}{}{[saˈsa]}{(v.)}{1}{}{Esquartejar.}{}{}%
\verb{sasa}{}{[saˈsa]}{(v.)}{5}{}{Golpear.}{}{}%
\verb{sasa}{}{[saˈsa]}{(v.)}{3}{}{Podar.}{}{}%
\verb{sasa}{}{[saˈsa]}{(v.)}{4}{}{Retalhar.}{}{}%
\verb{sasa bengula}{}{[saˈsa b\~eguˈla]}{(expr.)}{1}{}{Ter relações sexuais.}{}{}
\verb{sasasa}{}{[sasaˈsa]}{(adj.)}{1}{}{Esperto.}{}{}
\verb{sasasa}{}{[sasaˈsa]}{(adj.)}{1}{}{Espevitado.}{}{}
\verb{sasasa}{}{[sasaˈsa]}{(adj.)}{1}{}{Vivo.}{}{}
\verb{sasasa}{}{[sasaˈsa]}{(id.)}{1}{}{Cf. \textbf{fla sasasa.}}{}{}
\verb{sata}{}{[saˈta]}{(v.)}{1}{}{Atravessar.}{}{}
\verb{sata}{}{[saˈta]}{(v.)}{2}{}{Pular.}{}{}
\verb{sata}{}{[saˈta]}{(v.)}{3}{}{Saltar.}{}{}
\verb{sata awa}{}{[saˈta ˈawa]}{(expr.)}{1}{}{Atravessar um curso de água ou o mar.}{}{}
\verb{sata awa}{}{[saˈta ˈawa]}{(expr.)}{1}{}{Viajar.}{}{}
\verb{sata-awa}{}{[saˈta ˈawa]}{(n.)}{1}{}{Calças curtas.}{}{}
\verb{satanaji}{}{[sataˈnaʒi]}{(n.)}{1}{}{Satanás.}{}{}
\verb{sata-sata}{}{[saˈta saˈta]}{(adj.)}{1}{}{Espevitado.}{}{}
\verb{sata-sata}{}{[saˈta saˈta]}{(adj.)}{1}{}{Saltitante.}{}{}
\verb{sata-sata}{}{[saˈta saˈta]}{(n.)}{1}{}{Pessoa saltitante.}{}{}
\verb{satida}{}{[ˈsatida]}{(n.)}{1}{}{Sátira.}{}{}
\verb{satu}{}{[ˈsatu]}{(n.)}{1}{}{Passo.}{}{}
\verb{satu}{}{[ˈsatu]}{(n.)}{2}{}{Salto.}{}{}
\verb{sawdadji}{}{[sawˈdadʒi]}{(n.)}{1}{}{Saudade.}{}{}
\verb{saya}{}{[ˈsaja]}{(n.)}{1}{}{Saia.}{}{}
\verb{saya}{}{[saˈja]}{(v.)}{1}{}{Arrastar.}{}{}
\verb{saya}{}{[saˈja]}{(v.)}{2}{}{Esticar.}{}{}
\verb{saya}{}{[saˈja]}{(v.)}{3}{}{Puxar.}{}{}%
\verb{se}{}{[ˈsɛ]}{(dem.)}{1}{}{Essa.}{}{}
\verb{se}{}{[ˈsɛ]}{(dem.)}{2}{}{Essas.}{}{}{}
\verb{se}{}{[ˈsɛ]}{(dem.)}{3}{}{Esse.}{\textbf{N ga toma fi se.} \emph{Eu vou pegar esse fio.}}{}{}
\verb{se}{}{[ˈsɛ]}{(dem.)}{4}{}{Esses.}{}{}{}
\verb{se}{}{[ˈsɛ]}{(dem.)}{5}{}{Esta.}{}{}
\verb{se}{}{[ˈsɛ]}{(dem.)}{6}{}{Estas.}{}{}
\verb{se}{}{[ˈsɛ]}{(dem.)}{7}{}{Este.}{}{}
\verb{se}{}{[ˈsɛ]}{(dem.)}{8}{}{Estes.}{}{}
\verb{sê}{}{[ˈse]}{(prep.)}{1}{}{Sem.}{}{}
\verb{sê}{}{[ˈse]}{(v.)}{1}{}{Saber.}{Cf. \textbf{sêbê}.}{}{}
\verb{sêbê}{}{[seˈbe]}{(n.)}{1}{}{Ciência.}{}{}
\verb{sêbê}{}{[seˈbe]}{(n.)}{2}{}{Sabedoria.}{}{}
\verb{sêbê}{}{[seˈbe]}{(n.)}{3}{}{Saber.}{}{}
\verb{sêbê}{}{[seˈbe]}{(v.)}{1}{}{Cuidar.}{}{}
\verb{sêbê}{}{[seˈbe]}{(v.)}{2}{}{Saber.}{}{}
\verb{sêbê}{}{[seˈbe]}{(v.)}{3}{}{Tratar.}{}{}
\verb{sêbêdô}{}{[sebeˈdo]}{(adj.)}{1}{}{Sábio.}{}{}
\verb{sêbêdô}{}{[sebeˈdo]}{(n.)}{1}{}{Sábio.}{}{}
\verb{sêbidu}{}{[seˈbidu]}{(adj.)}{1}{}{Sabido.}{}{}
\verb{sêbidu}{}{[seˈbidu]}{(n.)}{1}{}{Sabichão.}{}{}
\verb{sebu}{}{[ˈsɛbu]}{(n.)}{2}{}{Sebo.}{}{}
\verb{seda}{}{[ˈsɛda]}{(n.)}{1}{}{Cio.}{}{}
\verb{seda}{}{[ˈsɛda]}{(n.)}{3}{}{Seda.}{}{}
\verb{sêdji}{}{[ˈsedʒi]}{(n.)}{1}{}{Apetite sexual.}{}{}
\verb{sêdji}{}{[ˈsedʒi]}{(n.)}{2}{}{Sede.}{}{}%
\verb{sedu}{}{[ˈsɛdu]}{(adv.)}{1}{}{Cedo.}{}{}
\verb{sega}{}{[sɛˈga]}{(v.)}{1}{}{Cegar.}{}{}
\verb{sega}{}{[sɛˈga]}{(v.)}{1}{}{Distrair.}{}{}
\verb{segu}{}{[ˈsɛgu]}{(adj.)}{1}{}{Cego.}{}{}
\verb{sêgunda-fela}{}{[seˈgũda ˈfɛla]}{(n.)}{1}{}{Segunda-feira.}{}{}
\verb{sêgundu}{}{[seˈgũdu]}{(num.)}{1}{}{Segundo.}{}{}
\verb{sêja fêta}{}{[ˈseʒa ˈfeta]}{(expr.)}{1}{}{Seja feita.}{\textbf{Sêja fêta sa vonte Dêsu}. \emph{Seja feita a vontade de Deus}.}{}
\verb{sêja lôvadu}{}{[ˈseʒa loˈvadu]}{(expr.)}{1}{}{Seja louvado!}{}{}
\verb{sêji}{}{[ˈseʒi]}{(n.)}{1}{}{Apetite sexual.}{}{}{}
\verb{sêji}{}{[ˈseʒi]}{(n.)}{1}{}{Sede.}{Cf. \textbf{sêdji}.}{}{}
\verb{seka}{}{[ˈsɛka]}{(n.)}{1}{}{Seca.}{}{}
\verb{sekelu}{}{[sɛˈkɛlu]}{(adv.)}{1}{}{Ao menos.}{}{}
\verb{sekletalia}{}{[sɛklɛtaˈlia]}{(n.)}{1}{}{Secretaria.}{}{}
\verb{sêklêtalyu}{}{[sekleˈtalju]}{(n.)}{1}{}{Secretário.}{}{}
\verb{sê konta}{}{[ˈse ˈkõta]}{(expr.)}{1}{}{Imensurável.}{}{}
\verb{seku}{}{[ˈsɛku]}{(adj.)}{1}{}{Seco.}{}{}
\verb{seku klakata}{}{[ˈsɛku klakaˈta]}{(expr.)}{1}{}{Sequíssimo.}{}{}
\verb{sekulu}{}{[ˈsɛkulu]}{(n.)}{1}{}{Século.}{}{}
\verb{sela}{}{[sɛˈla]}{(adv.)}{1}{}{Apenas.}{\textbf{Sun na ka tlaba ku kwa di vlidu fa, sela di felu}. \textit{Ele não trabalha com coisas de vidro, apenas com as de ferro}.}{}
\verb{sela}{}{[sɛˈla]}{(adv.)}{2}{}{Ter de.}{\textbf{Sela pa n be ku bô}. \textit{Tenho de ir contigo}.}{}
\verb{sela}{}{[sɛˈla]}{(adv.)}{3}{}{Ter que.}{}{}
\verb{sela}{}{[ˈsɛla]}{(n.)}{1}{}{Cela.}{}{}
\verb{sela}{}{[ˈsɛla]}{(n.)}{1}{}{Serra.}{}{}
\verb{sêla}{}{[ˈsela]}{(n.)}{1}{}{Esteira.}{}{}
%\verb{sela}{}{[sɛˈla]}{(conj.)}{2}{}{A menos que.}{}{}
%\verb{sela}{}{[sɛˈla]}{(conj.)}{1}{}{A não ser que.}{}{}%
\verb{sela}{}{[sɛˈla]}{(v.)}{1}{}{Cheirar.}{}{}
\verb{sela}{}{[sɛˈla]}{(v.)}{2}{}{Serrar.}{}{}
\verb{selada}{}{[sɛˈlada]}{(n.)}{1}{}{Salada.}{}{}
\verb{seladô}{}{[sɛlaˈdo]}{(n.)}{2}{}{Serrador.}{}{}
\verb{sela tententen}{}{[sɛˈla t\~ɛt\~ɛˈt\~ɛ]}{(v.)}{1}{}{Cheirar bem.}{}{}
\verb{sêlê-alê}{}{[seˈle
aˈle]}{(n.)}{1}{}{\textit{Sêlê-alê}.}{\textbf{\textit{Leea tinctoria}.}}{}
\verb{sêlêlê}{}{[seleˈle]}{(id.)}{1}{}{Cf. \textbf{kôlê sêlêlê.}}{}{}
\verb{sêlêsêlê}{}{[seˈleseˈle]}{(n.)}{1}{}{\textit{Sêlê-alê.}}{Cf.\textbf{sêlê-alê}.}{}
\verb{sêlêvêja}{}{[seleˈveʒa]}{(n.)}{1}{}{Cerveja.}{}{}
\verb{selivi}{}{[ˈsɛlivi]}{(n.)}{1}{}{Servo.}{}{}
\verb{selon}{}{[sɛˈl\~ɔ]}{(n.)}{1}{}{Serrão.}{}{}
\verb{selotxi}{}{[sɛˈlɔtʃi]}{(n.)}{1}{}{Serrote.}{}{}
\verb{sêlu}{}{[ˈselu]}{(n.)}{1}{}{Cheiro.}{}{}
\verb{sêlu}{}{[ˈselu]}{(n.)}{2}{}{Selo.}{}{}
\verb{sêlu-sun-zon-maya}{}{[ˈselu ˈsũ ˈz\~ɔ maˈja]}{(n.)}{1}{}{Coentro de São
Tomé.}{\textbf{\textit{Eryngium foetidum}.}}{}
\verb{semba}{}{[ˈs\~ɛba]}{(n.)}{1}{}{Semba.}{}{}
\verb{sembleya}{}{[s\~ɛˈblɛja]}{(n.)}{1}{}{Assembleia.}{}{}
\verb{semple}{}{[ˈs\~ɛplɛ]}{(adv.)}{1}{}{Sempre.}{}{}
\verb{semplu}{}{[ˈs\~ɛplu]}{(n.)}{3}{}{Piada.}{}{}
\verb{semplu}{}{[ˈs\~ɛplu]}{(n.)}{4}{}{Provérbio.}{}{}
\verb{semplu}{}{[ˈs\~ɛplu]}{(n.)}{4}{}{Metáforas provocatórias.}{}{}
\verb{sen}{}{[ˈs\~ɛ]}{(num.)}{1}{}{Cem.}{}{}
\verb{sen}{}{[ˈs\~ɛ]}{(v.)}{1}{}{Existir.}{}{}%
\verb{sen}{}{[ˈs\~ɛ]}{(v.)}{2}{}{Haver.}{\textbf{Tempu na sen fa}.
\textit{Não há tempo}.}{}
\verb{sena}{}{[ˈsɛna]}{(n.)}{1}{}{Cena.}{}{}
\verb{sena}{}{[ˈsɛna]}{(n.)}{2}{}{Cerimônias.}{}{}
\verb{sena}{}{[ˈsɛna]}{(n.)}{3}{}{Manha.}{}{}
\verb{sena}{}{[ˈsɛna]}{(n.)}{4}{}{Vergonha.}{}{}
\verb{senadô}{}{[sɛnaˈdo]}{(n.)}{1}{}{Senador.}{}{}
\verb{sendê}{}{[sẽˈde]}{(v.)}{1}{}{Acender.}{}{}
\verb{sendê}{}{[sẽˈde]}{(v.)}{2}{}{Aumentar.}{}{}
\verb{sendê}{}{[sẽˈde]}{(v.)}{3}{}{Estender(-se).}{}{}
\verb{sendê}{}{[sẽˈde]}{(v.)}{4}{}{Esticar.}{}{}
\verb{sendê byololo}{}{[sẽˈde bjɔlɔˈlɔ]}{(expr.)}{1}{}{Estender-se muito.}{}
\verb{sendê pligisa}{}{[sẽˈde pliˈgisa]}{(expr.)}{1}{}{Espreguiçar(-se).}{}
\verb{senge}{}{[s\~ɛˈgɛ]}{(n.)}{1}{}{\textit{Senge}.}{Prato tradicional à
base de milho.}{}{}
\verb{sengundu}{}{[sẽˈgũdu]}{(num.)}{1}{}{Segundo.}{}{}{}
\verb{seni}{}{[ˈsɛni]}{(n.)}{1}{}{Cherne.}{}{}
\verb{senkwa}{}{[sẽˈkwa]}{(n.)}{1}{}{Percevejo.}{}{}
\verb{senson}{}{[s\~ɛˈs\~ɔ]}{(n.)}{1}{}{Cerimônia.}{}{}
\verb{senson}{}{[s\~ɛˈs\~ɔ]}{(n.)}{2}{}{Comício.}{}{}
\verb{senson}{}{[s\~ɛˈs\~ɔ]}{(n.)}{3}{}{Reunião.}{}{}
\verb{senson}{}{[s\~ɛˈs\~ɔ]}{(n.)}{4}{}{Sessão.}{}{}
\verb{sentenalyu}{}{[s\~ɛtɛˈnalju]}{(n.)}{1}{}{Centenário.}{}{}
\verb{sentenxa}{}{[s\~ɛˈt\~ɛʃa]}{(n.)}{1}{}{Carma.}{}{}
\verb{sentenxa}{}{[s\~ɛˈt\~ɛʃa]}{(n.)}{2}{}{Espírito.}{}{}%
\verb{sentenxa}{}{[s\~ɛˈt\~ɛʃa]}{(n.)}{3}{}{Sentença.}{}{}
\verb{sentlu}{}{[ˈs\~ɛtlu]}{(n.)}{1}{}{Centro.}{}{}
\verb{senzala}{}{[s\~ɛˈzala]}{(n.)}{1}{}{Senzala.}{}{}%
\verb{senzon}{}{[s\~ɛˈz\~ɔ]}{(n.)}{1}{}{Febre.}{}{}%
\verb{senzon}{}{[s\~ɛˈz\~ɔ]}{(n.)}{2}{}{Sintomas de doença.}{}{}
\verb{sê pa}{}{[ˈse ˈpa]}{(conj.)}{1}{}{Sem que.}{}
\verb{sê-plôsêdê}{}{[ˈse ploseˈde]}{(adj.)}{1}{}{Desalinhado.}{}
\verb{sê-plôsêdê}{}{[ˈse ploseˈde]}{(adj.)}{2}{}{Desatinado.}{}
\verb{sê-plôsêdê}{}{[ˈse ploseˈde]}{(adj.)}{3}{}{Descuidado.}{}
\verb{sê-plôsêdê}{}{[ˈse ploseˈde]}{(adj.)}{4}{}{Desleixado.}{}
\verb{sesa}{}{[ˈsɛsa]}{(adj.)}{1}{}{Insignificante.}{}
\verb{sesa}{}{[ˈsɛsa]}{(adj.)}{2}{}{Inútil.}{}
\verb{sêsa}{}{[ˈsesa]}{(n.)}{1}{}{Pombo-verde de São
Tomé.}{\textbf{\textit{Treron sanctithomae}}.}{}
\verb{sêsê-limê}{}{[seˈse
ˈlime]}{(n.)}{1}{}{\textit{Sêsê-limê}.}{\textbf{\textit{Psophocarpus
scandens}}.}{}
\verb{sesenta}{}{[seˈs\~eta]}{(num.)}{1}{}{Sessenta.}{}{}
\verb{sêsentu}{}{[seˈs\~etu]}{(num.)}{1}{}{Seiscentos.}{}{}
\verb{seta}{}{[sɛˈta]}{(v.)}{1}{}{Aceitar.}{}{}
\verb{seta}{}{[sɛˈta]}{(v.)}{1}{}{Saber.}{}{}
\verb{sete}{}{[ˈsɛtɛ]}{(num.)}{1}{}{Sete.}{}{}
\verb{sete-dexi}{}{[ˈsɛtɛˈdɛʃi]}{(num.)}{1}{}{Setenta.}{}{}
\verb{setemblu}{}{[sɛˈt\~ɛblu]}{(n.)}{1}{}{Setembro.}{}{}
\verb{setenta}{}{[sɛˈt\~ɛta]}{(num.)}{1}{}{Setenta.}{}{}
\verb{setesentu}{}{[sɛtɛˈs\~ɛtu]}{(num.)}{1}{}{Setecentos.}{}{}
\verb{setimu}{}{[ˈsɛtimu]}{(num.)}{1}{}{Sétimo.}{}{}
\verb{setu}{}{[ˈsɛtu]}{(adj.)}{1}{}{Certo.}{}{}
\verb{setu}{}{[ˈsɛtu]}{(adj.)}{2}{}{Correto.}{}{}
\verb{setu}{}{[ˈsɛtu]}{(n.)}{1}{}{Cetro.}{}{}
\verb{seva}{}{[sɛˈva]}{(v.)}{1}{}{Ficar estéril.}{}{}
\verb{sêvada}{}{[seˈvada]}{(n.)}{1}{}{Cevada.}{}{}
\verb{sêvadu}{}{[seˈvadu]}{(adj.)}{1}{}{Estéril.}{}{}
\verb{sêxi}{}{[ˈseʃi]}{(num.)}{1}{}{Seis.}{}{}
\verb{sêxi-dexi}{}{[ˈseʃi ˈdɛʃi]}{(num.)}{1}{}{Sessenta.}{}{}
\verb{sêxta-fela}{}{[ˈseʃta ˈfɛla]}{(n.)}{1}{}{Sexta-feira.}{}{}
\verb{sêxtêmunha}{}{[seʃteˈm\~uɲa]}{(n.)}{1}{}{Testemunha.}{}{}
\verb{sêxtu}{}{[ˈseʃtu]}{(num.)}{1}{}{Sexto.}{}{}
\verb{sidlela}{}{[siˈdlɛla]}{(n.)}{2}{}{Cedro-rosa.}{\textbf{\textit{Cedrela
odorata}}.}{}
\verb{sidlela}{}{[siˈdlɛla]}{(n.)}{1}{}{Cidreira.}{\textbf{\textit{Cedrela
odorata}}.}{}
\verb{siganu}{}{[siˈg\~{\textturna}nu]}{(n.)}{1}{}{Cigano.}{}{}
\verb{sigi}{}{[siˈgi]}{(v.)}{2}{}{Continuar.}{}{}
\verb{sigi}{}{[siˈgi]}{(v.)}{1}{}{Seguir.}{}{}%
\verb{simêla}{}{[siˈmela]}{(n.)}{1}{}{Cimeira.}{}{}
\verb{simenta}{}{[sim\~ɛˈta]}{(v.)}{1}{}{Cimentar.}{}{}
\verb{sinôra}{}{[siˈno{\textfishhookr}a]}{(n.)}{1}{}{Cenoura.}{}{}
\verb{sivika}{}{[ˈsivika]}{(adj.)}{1}{}{Cívica.}{}{}
\verb{sivil}{}{[siˈvil]}{(adj.)}{1}{}{Civil.}{}{}
\verb{sixti}{}{[siʃˈti]}{(v.)}{1}{}{Assistir.}{}{}
\verb{so}{}{[ˈsɔ]}{(adv.)}{1}{}{Apenas.}{\textbf{Sun pasa mu n’\~ua anu so}. \textit{Ele é apenas um ano mais velho}.}{}
\verb{so}{}{[ˈsɔ]}{(adv.)}{2}{}{Só.}{}{}%
\verb{so}{}{[ˈsɔ]}{(adv.)}{3}{}{Somente.}{}{}%
\verb{so}{}{[ˈsɔ]}{(conj.)}{1}{}{Aí.}{}{}
\verb{so}{}{[ˈsɔ]}{(conj.)}{2}{}{Então.}{\textbf{So a ka laba ôdô muntu ben}. \textit{Então lavaram a almofariz muito bem}.}{}%
\verb{so}{}{[ˈsɔ]}{(foc.)}{1}{}{É que.}{\textbf{Ami so kota sun mon.} \textit{Eu é que lhe cortei o braço.}}{}{}
\verb{sôbê}{}{[soˈbe]}{(v.)}{1}{}{Bater.}{}{}
\verb{sôbê}{}{[soˈbe]}{(v.)}{1}{}{Chover.}{}{}
\verb{sobelubu}{}{[sɔˈbɛlubu]}{(adj.)}{1}{}{Arrogante.}{}{}%
\verb{sobelubu}{}{[sɔˈbɛlubu]}{(adj.)}{2}{}{Soberbo.}{}{}%
\verb{sôbê po}{}{[soˈbe ˈpɔ]}{(expr.)}{1}{}{Bater com o pau.}{}{}
\verb{sôbê po}{}{[soˈbe ˈpɔ]}{(expr.)}{1}{}{Espancar.}{}{}
\verb{sobladu}{}{[sɔˈbladu]}{(n.)}{1}{}{Piso.}{}{}%
\verb{sobladu}{}{[sɔˈbladu]}{(n.)}{2}{}{Soalho.}{}{}
\verb{sobladu}{}{[sɔˈbladu]}{(n.)}{3}{}{Sobrado.}{}{}
\verb{soda}{}{[ˈsɔda]}{(n.)}{1}{}{Soda cáustica.}{}{}
\verb{soda}{}{[sɔˈda]}{(v.)}{1}{}{Atar.}{}{}
\verb{soda}{}{[sɔˈda]}{(v.)}{2}{}{Soldar.}{}{}
\verb{sode}{}{[sɔˈdɛ]}{(n.)}{1}{}{Polícia.}{}{}%
\verb{sode}{}{[sɔˈdɛ]}{(n.)}{2}{}{Soldado.}{}{}
\verb{sôdon-kampu}{}{[soˈdõ
ˈk\~{\textturna}pu]}{(n.)}{1}{}{\textit{Sôdon-kampu}.}{\textbf{\textit{Rynchosia
minima}}.}{}{}
\verb{sogla}{}{[ˈsɔgla]}{(n.)}{1}{}{Nora.}{}{}
\verb{sogla}{}{[ˈsɔgla]}{(n.)}{1}{}{Sogra.}{Cf. \textbf{xtloga}.}{}
\verb{soglu}{}{[ˈsɔglu]}{(n.)}{1}{}{Genro.}{}{}
\verb{soglu}{}{[ˈsɔglu]}{(n.)}{1}{}{Sogro.}{Cf. \textbf{xtlogu}.}{}
\verb{soka}{}{[sɔˈka]}{(v.)}{1}{}{Amarrar.}{}{}
\verb{soka}{}{[sɔˈka]}{(v.)}{1}{}{Apertar.}{}{}
\verb{soka}{}{[sɔˈka]}{(v.)}{1}{}{Atar.}{}{}
\verb{soka}{}{[sɔˈka]}{(v.)}{2}{}{Beber.}{}{}
\verb{sokadu}{}{[sɔˈkadu]}{(adj.)}{1}{}{Amarrado.}{}{}
\verb{sokadu}{}{[sɔˈkadu]}{(adj.)}{1}{}{Apertado.}{}{}
\verb{sokadu}{}{[sɔˈkadu]}{(adj.)}{1}{}{Atado.}{}{}
\verb{soki}{}{[ˈsɔki]}{(n.)}{1}{}{Choco.}{\textbf{\textit{Sepia
oficinalis}}.}{}{}
\verb{sôkô}{}{[ˈsoko]}{(n.)}{1}{}{Soco.}{}{}
\verb{sokope}{}{[sɔkɔˈpɛ]}{(n.)}{1}{}{Socopé.}{Dança tradicional.}{}{}
\verb{soku}{}{[ˈsɔku]}{(conj.)}{1}{}{Aí.}{}{}
\verb{soku}{}{[ˈsɔku]}{(conj.)}{2}{}{Depois.}{}{}
\verb{soku}{}{[ˈsɔku]}{(conj.)}{3}{}{Então.}{\textbf{Soku ê ka tason liba
budu se}. \textit{Então ele senta-se em cima dessa pedra}.}{}%
\verb{soku}{}{[ˈsɔku]}{(foc.)}{1}{}{É que.}{\textbf{Bô soku n fla ku ê}. \textit{Contigo é que falei}.}{}{}
\verb{sola}{}{[ˈsɔla]}{(n.)}{1}{}{Sola.}{}{}
\verb{sola}{}{[sɔˈla]}{(n.)}{1}{}{Choro.}{}{}
\verb{sola}{}{[sɔˈla]}{(v.)}{1}{}{Chorar.}{}{}
\verb{sola fliji}{}{[sɔˈla fliˈʒi]}{(expr.)}{1}{}{Chorar desalmadamente.}{}
\verb{sola potopoto}{}{[sɔˈla pɔˈtɔpɔˈtɔ]}{(expr.)}{1}{}{Chorar desalmadamente.}{}{}
\verb{sôlê}{}{[ˈsole]}{(adj.)}{1}{}{Solitário.}{}{}
\verb{sôlê}{}{[ˈsole]}{(adj.)}{1}{}{Solteiro.}{}{}
\verb{sôlê}{}{[ˈsole]}{(adj.)}{1}{}{Sozinho.}{}{}
\verb{sôlê}{}{[ˈsole]}{(n.)}{1}{}{Solitário.}{}{}
\verb{sôlê}{}{[ˈsole]}{(n.)}{1}{}{Solteiro.}{}{}
\verb{sôlêta}{}{[soˈleta]}{(n.)}{1}{}{Suporte.}{}{}%
\verb{sôlêta}{}{[soˈleta]}{(n.)}{2}{}{Viga.}{}{}
\verb{soletla}{}{[sɔlɛˈtla]}{(v.)}{1}{}{Soletrar.}{}{}
\verb{soletladu}{}{[sɔlɛˈtladu]}{(adj.)}{1}{}{Soletrado.}{}{}
\verb{sôlisô}{}{[soˈliso]}{(n.)}{1}{}{Chouriço.}{}{}
\verb{sôlisô}{}{[soˈliso]}{(n.)}{2}{}{Linguiça.}{}{}
\verb{solo}{}{[ˈsɔlɔ]}{(n.)}{1}{}{Sol.}{}{}
\verb{sôlôbatu}{}{[soloˈbatu]}{(n.)}{1}{}{Celibato.}{}{}
\verb{solo mê-dja ngwangwangwan}{}{[ˈsɔlɔ meˈdʒa ŋgw\~{\textturna}gw\~{\textturna}ˈgw\~{\textturna}]}{(expr.)}{1}{}{Sol abrasador.}{}{}
%\verb{solo xka benzê awa}{}{[ˈsɔlɔ ˈʃka b\~eˈze ˈawa]}{(expr.)}{1}{}{Pôr-do-sol.}{}{}
\verb{soma}{}{[ˈsɔma]}{(n.)}{1}{}{Soma.}{}{}
\verb{somana}{}{[sɔˈmana]}{(n.)}{1}{}{Semana.}{}{}
\verb{sombla}{}{[ˈs\~ɔbla]}{(n.)}{1}{}{Sombra.}{}{}
\verb{sombla}{}{[s\~ɔˈbla]}{(v.)}{1}{}{Assombrar.}{}{}
\verb{sombla}{}{[s\~ɔˈbla]}{(v.)}{2}{}{Assustar(-se).}{}{}
\verb{sombla}{}{[s\~ɔˈbla]}{(v.)}{3}{}{Espantar.}{}{}
\verb{sombladu}{}{[s\~ɔˈbladu]}{(adj.)}{1}{}{Assombrado.}{}{}
\verb{sombladu}{}{[s\~ɔˈbladu]}{(adj.)}{2}{}{Assustado.}{}{}
\verb{sombladu}{}{[s\~ɔˈbladu]}{(adj.)}{3}{}{Espantado.}{}{}
\verb{son}{}{[ˈs\~ɔ]}{(n.)}{1}{}{Chão.}{}{}
\verb{son}{}{[ˈs\~ɔ]}{(n.)}{2}{}{Solo.}{}{}%
\verb{son}{}{[ˈs\~ɔ]}{(n.)}{3}{}{Terra.}{}{}%
\verb{sonha}{}{[s\~ɔˈɲa]}{(v.)}{1}{}{Sonhar.}{}{}
\verb{sonhu}{}{[ˈs\~ɔɲu]}{(n.)}{1}{}{Sonho.}{}{}
\verb{sono}{}{[ˈsɔnɔ]}{(n.)}{1}{}{Sono.}{}{}
\verb{sono}{}{[sɔˈnɔ]}{(v.)}{1}{}{Ruído produzido com os lábios para exprimir
desagrado, desprezo ou enfado perante uma situação ou pessoa.}{}{}
\verb{sonosono}{}{[sɔˈnɔsɔˈnɔ]}{(id.)}{1}{}{Cf. \textbf{liku sonosono.}}{}{}
\verb{sôpa}{}{[ˈsopa]}{(n.)}{1}{}{Sopa.}{}{}%
\verb{sôpa}{}{[ˈsopa]}{(n.)}{2}{}{Sopa.}{\textbf{\textit{Kyphosus
incisor}.}}{}
\verb{sopla}{}{[sɔˈpla]}{(v.)}{1}{}{Assoprar.}{}{}
\verb{sopla}{}{[sɔˈpla]}{(v.)}{2}{}{Atirar.}{}{}
\verb{sopla}{}{[sɔˈpla]}{(v.)}{3}{}{Dizer.}{}{}
\verb{sopla}{}{[sɔˈpla]}{(v.)}{4}{}{Falar.}{}{}%
\verb{sopla}{}{[sɔˈpla]}{(v.)}{5}{}{Soprar.}{}{}
\verb{sopla olha}{}{[sɔˈpla ɔˈʎa]}{(expr.)}{1}{}{Namorar.}{}{}%
\verb{sopla pê}{}{[sɔˈpla ˈpe]}{(expr.)}{1}{}{Disparar.}{}{}
\verb{sosega}{}{[sɔsɛˈga]}{(v.)}{1}{}{Acalmar.}{}{}
\verb{sosega}{}{[sɔsɛˈga]}{(v.)}{2}{}{Sossegar.}{}{}%
\verb{sosegadu}{}{[sɔsɛˈgadu]}{(adj.)}{1}{}{Quieto.}{}{}
\verb{sosegadu}{}{[sɔsɛˈgadu]}{(adj.)}{2}{}{Sossegado.}{}{}
\verb{sosegu}{}{[sɔˈsɛgu]}{(n.)}{1}{}{Calma.}{}{}
\verb{sosegu}{}{[sɔˈsɛgu]}{(n.)}{2}{}{Sossego.}{}{}
\verb{soso}{}{[sɔˈsɔ]}{(conj.)}{1}{}{Aí.}{}{}{}
\verb{soso}{}{[sɔˈsɔ]}{(conj.)}{1}{}{Então.}{Cf. \textbf{so}.}{}{}
\verb{sôsô}{}{[soˈso]}{(adj.)}{1}{}{Aventureiro.}{}{}
\verb{sôsô}{}{[soˈso]}{(adj.)}{2}{}{Vadio.}{}{}
\verb{sôsô}{}{[soˈso]}{(adj.)}{3}{}{Vagabundo.}{}{}
\verb{sôsô}{}{[soˈso]}{(v.)}{1}{}{Aventureirar.}{}{}
\verb{sôsô}{}{[soˈso]}{(v.)}{1}{}{Vadiar.}{}{}
\verb{sôsô}{}{[soˈso]}{(v.)}{1}{}{Vagabundear.}{}{}
\verb{sôsô}{}{[soˈso]}{(v.)}{1}{}{Vaguear.}{}{}
\verb{sososo}{}{[sɔsɔˈsɔ]}{(conj.)}{1}{}{Aí.}{}{}{}
\verb{sososo}{}{[sɔsɔˈsɔ]}{(conj.)}{1}{}{Então.}{Cf. \textbf{so}.}{}{}
\verb{sososo}{}{[sɔsɔˈsɔ]}{(conj.)}{1}{}{Por isso.}{}{}
\verb{sota}{}{[sɔˈta]}{(v.)}{1}{}{Soltar.}{}{}
\verb{sotxi}{}{[ˈsɔtʃi]}{(n.)}{1}{}{Destino.}{}{}
\verb{sotxi}{}{[ˈsɔtʃi]}{(n.)}{2}{}{Sorte.}{}{}{}
\verb{sôtxi}{}{[ˈsotʃi]}{(n.)}{1}{}{Açoite.}{}{}%
\verb{sôtxi}{}{[ˈsotʃi]}{(n.)}{2}{}{Sova.}{}{}
\verb{sôtxi}{}{[ˈsotʃi]}{(n.)}{3}{}{Surra.}{}{}
\verb{sôtxi-flima}{}{[ˈsotʃi fliˈma]}{(n.)}{1}{}{\textit{Sôtxi-flima.}}{Ritual de transição da puberdade para a vida adulta.}{}{}{}
\verb{sôvêtê}{}{[soveˈte]}{(v.)}{1}{}{Esgotar.}{}{}
\verb{soveza}{}{[sɔvɛˈza]}{(v.)}{1}{}{Restar.}{}{}
\verb{soveza}{}{[sɔvɛˈza]}{(v.)}{2}{}{Sobejar.}{}{}
\verb{soveza}{}{[sɔvɛˈza]}{(v.)}{3}{}{Sobrar.}{}{}
\verb{sovezu}{}{[sɔvɛˈzu]}{(n.)}{1}{}{Sobejos.}{}{}
\verb{sovezu}{}{[sɔvɛˈzu]}{(n.)}{1}{}{Sobra.}{}{}
\verb{sovezu}{}{[sɔvɛˈzu]}{(n.)}{2}{}{Resto.}{}{}
\verb{sôwô}{}{[ˈsowo]}{(n.)}{1}{}{\textit{Sôwô}.}{Prato típico que pode, alternativamente, ser preparado com fruta-pão, mandioca, inhame, matabala, batata-doce, banana-pão, com peixe e óleo de palma e algumas ervas aromáticas, acompanhado de farinha de mandioca ou banana assada.}{}{}
\verb{soxi}{}{[ˈsɔʃi]}{(n.)}{1}{}{Sócio.}{Cf. \textbf{xoxi}.}{}%
\verb{soya}{}{[ˈsɔja]}{(n.)}{1}{}{Conto.}{}{}%
\verb{soya}{}{[ˈsɔja]}{(n.)}{2}{}{História.}{}{}%
\verb{soya}{}{[ˈsɔja]}{(n.)}{3}{}{Narrativa tradicional.}{}{}
\verb{soya}{}{[sɔˈja]}{(v.)}{1}{}{Assoalhar.}{}{}%
\verb{soya}{}{[sɔˈja]}{(v.)}{1}{}{Deitar.}{}{}
\verb{soyu}{}{[ˈsɔju]}{(n.)}{1}{}{Assoalho.}{}{}
\verb{soyu}{}{[ˈsɔju]}{(n.)}{2}{}{Soalho.}{}{}
\verb{suba}{}{[ˈsuba]}{(n.)}{1}{}{Chuva.}{}{}%
\verb{suba}{}{[ˈsuba]}{(n.)}{2}{}{Estação chuvosa.}{}{}
\verb{subli}{}{[suˈbli]}{(prep. v.)}{1}{}{Para cima.}{\textbf{Inen nda subli}. \textit{Andaram para cima}.}{}
\verb{subli}{}{[suˈbli]}{(n.)}{1}{}{Subida.}{}{}
\verb{subli}{}{[suˈbli]}{(v.)}{1}{}{Montar.}{}{}
\verb{subli}{}{[suˈbli]}{(v.)}{2}{}{Subir.}{}{}%
\verb{subli}{}{[suˈbli]}{(v.)}{3}{}{Superar.}{}{}%
\verb{sublidô}{}{[subliˈdo]}{(v.)}{2}{}{Escalador.}{}{}
\verb{sublidô}{}{[subliˈdo]}{(v.)}{1}{}{Trepador.}{}{}%
\verb{sublidu}{}{[suˈblidu]}{(adj.)}{1}{}{Caro.}{}{}
\verb{sublidu}{}{[suˈblidu]}{(adj.)}{2}{}{Subido.}{}{}
\verb{sublinha}{}{[suˈbl\~iɲa]}{(n.)}{1}{}{Sobrinha.}{}{}
\verb{sublinhu}{}{[suˈbl\~iɲu]}{(n.)}{1}{}{Sobrinho.}{}{}
\verb{subli-taku}{}{[suˈbli ˈtaku]}{(n.)}{1}{}{Posição sexual.}{}{}
\verb{subyan}{}{[suˈbj\~{\textturna}]}{(n.)}{1}{}{Afilhado do mesmo padrinho ou da mesma madrinha.}{}{}
\verb{sudu}{}{[ˈsudu]}{(adj.)}{1}{}{Surdo.}{}{}
\verb{sudu}{}{[ˈsudu]}{(n.)}{1}{}{Bofetada.}{}{}
\verb{sufatu}{}{[suˈfatu]}{(n.)}{1}{}{Sulfato.}{}{}
\verb{sufli}{}{[suˈfli]}{(v.)}{1}{}{Sofrer.}{}{}
\verb{suflidu}{}{[suˈflidu]}{(adj.)}{1}{}{Sofrido.}{}{}
\verb{sufli kloson}{}{[suˈfli klɔˈs\~ɔ]}{(expr.)}{1}{}{Encorajar(-se).}{}{}
\verb{sufli kloson}{}{[suˈfli klɔˈs\~ɔ]}{(expr.)}{2}{}{Ter coragem.}{}{}
%\verb{suflimentu}{}{[sufliˈm\~ɛtu]}{(n.)}{1}{}{Sofrimento.}{}{}
\verb{suga}{}{[suˈga]}{(v.)}{1}{}{Enxugar.}{}{}
\verb{suga}{}{[suˈga]}{(v.)}{2}{}{Secar.}{}{}%
\verb{suga}{}{[suˈga]}{(v.)}{3}{}{Sugar.}{}{}
\verb{sugadu}{}{[suˈgadu]}{(adj.)}{1}{}{Enxuto.}{}{}
\verb{sugadu}{}{[suˈgadu]}{(adj.)}{2}{}{Seco.}{}{}
\verb{suinsuin}{}{[suˈ\~isuˈ\~i]}{(n.)}{1}{}{\textit{Suinsuin}.}{\textit{\textbf{Uraeginthus
angolensis}}.}{}
\verb{sukata}{}{[suˈkata]}{(n.)}{1}{}{Coisa estragada.}{}{}
\verb{sukata}{}{[suˈkata]}{(n.)}{2}{}{Sucata.}{}{}%
\verb{sukli}{}{[ˈsukli]}{(n.)}{1}{}{Açúcar.}{}{}
\verb{sukumbi}{}{[suk\~uˈbi]}{(v.)}{1}{}{Sumir.}{}{}
\verb{sulon}{}{[suˈl\~ɔ]}{(adj.)}{1}{}{Lascivo.}{}{}
\verb{sulon}{}{[suˈl\~ɔ]}{(adj.)}{1}{}{Com forte apetite sexual.}{}{}
\verb{sulon}{}{[suˈl\~ɔ]}{(n.)}{1}{}{Pessoa lasciva.}{}{}
\verb{sulon}{}{[suˈl\~ɔ]}{(n.)}{1}{}{Pessoa com forte apetite sexual.}{}{}
\verb{sulu}{}{[ˈsulu]}{(n.)}{1}{}{Sul.}{}{}
\verb{sulu}{}{[ˈsulu]}{(top.)}{1}{}{Angola.}{}{}
\verb{sumba}{}{[sũˈba]}{(v.)}{1}{}{Chumbar.}{}{}
\verb{sumba}{}{[sũˈba]}{(v.)}{1}{}{Encarar.}{}{}
\verb{sumbada}{}{[sũˈbada]}{(n.)}{1}{}{Frígida.}{}{}{}
\verb{sumbadu}{}{[sũˈbadu]}{(n.)}{1}{}{Chumbado.}{}{}{}
\verb{sumbadu}{}{[sũˈbadu]}{(n.)}{2}{}{Inerte.}{}{}{}
\verb{sumbu}{}{[ˈsũbu]}{(n.)}{1}{}{Chumbo.}{}{}
\verb{sumbu}{}{[ˈsũbu]}{(n.)}{2}{}{Mulher sexualmente passiva.}{}{}
\verb{sumi}{}{[ˈsumi]}{(adj.)}{1}{}{Ciumento.}{}{}
\verb{sumi}{}{[ˈsumi]}{(n.)}{1}{}{Ciúmes.}{}{}
\verb{sumi}{}{[suˈmi]}{(v.)}{1}{}{Consumir.}{}{}
\verb{sumi}{}{[suˈmi]}{(v.)}{2}{}{Desaparecer.}{}{}
\verb{sumi}{}{[suˈmi]}{(v.)}{2}{}{Disputar.}{}{}
\verb{sumi}{}{[suˈmi]}{(v.)}{2}{}{Enciumar.}{}{}
\verb{sumi}{}{[suˈmi]}{(v.)}{3}{}{Sumir.}{}{}
\verb{sumi}{}{[suˈmi]}{(v.)}{3}{}{Ter ciúmes de.}{}{}
\verb{sumu}{}{[ˈsumu]}{(n.)}{1}{}{Deus.}{}{}%
\verb{sumu}{}{[ˈsumu]}{(n.)}{2}{}{Excelentíssimo.}{}{}
\verb{sumu}{}{[ˈsumu]}{(n.)}{3}{}{Senhor.}{}{}
\verb{sumu}{}{[ˈsumu]}{(n.)}{4}{}{Suco de fruta.}{}{}
\verb{sumu}{}{[ˈsumu]}{(n.)}{5}{}{Sumo de fruta.}{}{}
\verb{sun}{}{[ˈsũ]}{(n.)}{1}{}{Senhor.}{}{}
\verb{sun}{}{[ˈsũ]}{(poss.)}{1}{}{Seu.}{}{}
\verb{sun}{}{[ˈsũ]}{(poss.)}{2}{}{Seus.}{}{}
\verb{sun}{}{[ˈsũ]}{(poss.)}{3}{}{Sua.}{}{}
\verb{sun}{}{[ˈsũ]}{(poss.)}{4}{}{Suas.}{}{}
\verb{sun}{}{[ˈs\~u]}{(pron.)}{2}{}{Ele (formal).}{\textbf{Kê kwa ku sun se fe ni vida sun?} \emph{O que é que ele fez na sua vida?}}{}{}
\verb{sun}{}{[ˈs\~u]}{(pron.)}{3}{}{-lhe (formal).}{Terceira pessoa do singular com a função de complemento indireto.}{}{}{}
\verb{sun}{}{[ˈs\~u]}{(pron.)}{1}{}{-o (formal).}{Terceira pessoa do singular com a função de complemento direto.}{}
\verb{sun-alê}{}{[ˈsũ aˈle]}{(n.)}{1}{}{Imperador.}{Personagem de histórias tradicionais.}{}{}%
\verb{sun-alê}{}{[ˈsũ aˈle]}{(n.)}{2}{}{Rei.}{}{}%
\verb{sungê}{}{[sũˈge]}{(n.)}{1}{}{Senhor.}{}{}
\verb{supa}{}{[suˈpa]}{(v.)}{1}{}{Chupar.}{}{}
\verb{supa}{}{[suˈpa]}{(v.)}{2}{}{Soluçar.}{}{}
\verb{supa}{}{[suˈpa]}{(v.)}{3}{}{Ter soluços.}{}{}
\verb{supada}{}{[suˈpada]}{(n.)}{1}{}{Espada.}{}{}
\verb{supada}{}{[suˈpada]}{(n.)}{2}{}{Espadas.}{Um dos naipes do baralho. Cf. \textbf{fya-supada}.}{}{}
\verb{supadô}{}{[supaˈdo]}{(n.)}{1}{}{Aproveitador.}{}{}
\verb{supapu}{}{[supaˈpu]}{(n.)}{1}{}{Sopapo.}{}{}
\verb{supê}{}{[suˈpe]}{(n.)}{1}{}{Espelho.}{}{}
\verb{supetu}{}{[suˈpɛtu]}{(adj.)}{1}{}{Astuto.}{}{}
\verb{supetu}{}{[suˈpɛtu]}{(adj.)}{2}{}{Esperto.}{}{}
\verb{supiga}{}{[suˈpiga]}{(n.)}{1}{}{Espiga.}{}{}
\verb{suplitu}{}{[suˈplitu]}{(n.)}{1}{}{Espírito.}{}{}
\verb{suplitu}{}{[suˈplitu]}{(n.)}{2}{}{Força.}{}{}
\verb{supriôr}{}{[sup{\textfishhookr}iˈo{\textfishhookr}]}{(n.)}{1}{}{Superior.}{}{}
\verb{supya}{}{[suˈpja]}{(n.)}{1}{}{Assobio.}{}{}
\verb{supya}{}{[suˈpja]}{(v.)}{1}{}{Assobiar.}{}{}
\verb{s\~u\~u\~u}{}{[ˈs\~u\~u\~u]}{(id.)}{1}{}{Cf. \textbf{pya s\~u\~u\~u}.}{}
\verb{suxtu}{}{[ˈsuʃtu]}{(n.)}{1}{}{Susto.}{}{}
\verb{suza}{}{[suˈza]}{(v.)}{1}{}{Sujar.}{}{}
\verb{suzu}{}{[ˈsuzu]}{(adj.)}{1}{}{Indecente.}{}{}
\verb{suzu}{}{[ˈsuzu]}{(adj.)}{2}{}{Sujo.}{}{}
\verb{suzu}{}{[ˈsuzu]}{(n.)}{1}{}{Excremento.}{}{}
\verb{suzu-d'olha}{}{[ˈsuzu dɔˈʎa]}{(n.)}{1}{}{Cera de ouvido.}{}{}
\verb{suzu-d'olha}{}{[ˈsuzu dɔˈʎa]}{(n.)}{2}{}{Cerume.}{}{}
\verb{suzu kotokoto}{}{[ˈsuzu kɔˈtɔkɔˈtɔ]}{(expr.)}{1}{}{Sujíssimo.}{}{}
\verb{swa}{}{[ˈswa]}{(v.)}{1}{}{Suar.}{}{}
\verb{swa}{}{[ˈswa]}{(v.)}{2}{}{Transpirar.}{}{}
\verb{swaswa}{}{[swaˈswa]}{(n.)}{1}{}{Cobra-verde.}{\textbf{\textit{Philothamnus thomensis}}.}{}{}%
\verb{swaswa}{}{[swaˈswa]}{(n.)}{2}{}{\textit{Swaswa}.}{\textbf{\textit{Rinorea molleri}.}}{}
\verb{syensya}{}{[ˈsj\~esja]}{(n.)}{1}{}{Ciência.}{}{}
\end{letra}

\begin{letra}{t}

\verb{ta}{}{[ˈta]}{(part.)}{1}{}{Cf. \textbf{tava}.}{}{}
\verb{ta}{}{[ˈta]}{(v.)}{1}{}{Estar.}{}{}%
\verb{ta}{}{[ˈta]}{(v.)}{2}{}{Ficar.}{}{}
\verb{tá}{}{[ˈta]}{(v.)}{1}{}{Ter relações sexuais.}{}{}
\verb{taba}{}{[ˈtaba]}{(n.)}{1}{}{Tábua.}{}{}
\verb{tabaki}{}{[taˈbaki]}{(n.)}{1}{}{Atabaque.}{}{}%
\verb{tabaku}{}{[taˈbaku]}{(n.)}{1}{}{Tabaco.}{}{}
\verb{tabulêlu}{}{[tabuˈlelu]}{(n.)}{1}{}{Tabuleiro.}{}{}
\verb{tabwada}{}{[taˈbwada]}{(n.)}{1}{}{Tabuada.}{}{}
\verb{tada}{}{[ˈtada]}{(n.)}{1}{}{Metade.}{}{}%
\verb{tada}{}{[ˈtada]}{(n.)}{2}{}{Pedaço.}{}{}
\verb{tada}{}{[taˈda]}{(v.)}{1}{}{Atrasar.}{}{}
\verb{tada}{}{[taˈda]}{(v.)}{2}{}{Demorar.}{}{}
\verb{tada}{}{[taˈda]}{(v.)}{3}{}{Tardar.}{}{}
\verb{tadadu}{}{[taˈdadu]}{(adj.)}{1}{}{Atrasado.}{}{}
\verb{tadadu}{}{[taˈdadu]}{(adj.)}{2}{}{Demorado.}{}{}
\verb{tadji}{}{[ˈtadʒi]}{(n.)}{1}{}{Tarde.}{}{}
\verb{tafon}{}{[taˈfõ]}{(n.)}{1}{}{Mosca grande.}{}{}
\verb{taji}{}{[ˈtaʒi]}{(adv.)}{1}{}{Tarde.}{Cf. \textbf{tadji}.}{}{}
\verb{taka}{}{[taˈka]}{(v.)}{1}{}{Atacar.}{}{}
\verb{taka boto}{}{[taˈka bɔˈtɔ]}{(expr.)}{1}{}{Abotoar.}{}{}
\verb{takadu}{}{[taˈkadu]}{(adj.)}{1}{}{Atacado.}{}{}
\verb{taki}{}{[ˈtaki]}{(n.)}{1}{}{Epilepsia.}{Cf. \textbf{xtaki}.}{}
\verb{taku-po}{}{[ˈtaku ˈpɔ]}{(n.)}{1}{}{Cepo.}{}{}
\verb{talabaxi}{}{[talaˈbaʃi]}{(n.)}{1}{}{Saca grande.}{}{}
\verb{talafa}{}{[taˈlafa]}{(n.)}{1}{}{Rede de pesca.}{}{}
\verb{talafa}{}{[taˈlafa]}{(n.)}{2}{}{Tarrafa.}{}{}
\verb{talanta}{}{[tal\~{\textturna}ˈta]}{(v.)}{1}{}{Amedrontar.}{}{}
\verb{talanta}{}{[tal\~{\textturna}ˈta]}{(v.)}{2}{}{Assustar.}{}{}
\verb{talanta}{}{[tal\~{\textturna}ˈta]}{(v.)}{3}{}{Atarantar.}{}{}
\verb{talantadu}{}{[tal\~{\textturna}ˈtadu]}{(adj.)}{1}{}{Amedrontado.}{}{}
\verb{talantadu}{}{[tal\~{\textturna}ˈtadu]}{(adj.)}{2}{}{Assustado.}{}{}
\verb{talantadu}{}{[tal\~{\textturna}ˈtadu]}{(adj.)}{3}{}{Atarantado.}{}{}
\verb{talhadera}{}{[taʎaˈdɛra]}{(n.)}{1}{}{Talhadeira.}{}{}{}
\verb{tali}{}{[ˈtali]}{(dem.)}{1}{}{Tal.}{\textbf{Maji n bê mwala bô ni tali xitu.}\emph{ Mas encontrei a tua mulher no tal lugar.}}{}{}
\verb{talimba}{}{[taˈl\~iba]}{(n.)}{1}{}{Tarimba.}{}{}{}
\verb{talitali}{}{[ˈtaliˈtali]}{(adv.)}{1}{}{Completamente.}{}{}
\verb{talon}{}{[taˈl\~ɔ]}{(n.)}{1}{}{Recibo.}{}{}
\verb{talon}{}{[taˈl\~ɔ]}{(n.)}{2}{}{Talão.}{}{}
\verb{taluvê}{}{[taluˈve]}{(adv.)}{2}{}{Talvez.}{}{}
\verb{tamanhan}{}{[tam\~{\textturna}ˈɲ\~{\textturna}]}{(n.)}{1}{}{Formiga
grande preta.}{}{}
\verb{tamanhan}{}{[tam\~{\textturna}ˈɲ\~{\textturna}]}{(n.)}{1}{}{Tamarindo.}{}{}
\verb{tamanhan}{}{[tam\~{\textturna}ˈɲ\~{\textturna}]}{(n.)}{2}{}{Tamarindeiro.}{\textbf{\textit{Tamarindus
indica}}.}{}%
\verb{tamanhan-ome}{}{[tam\~{\textturna}ˈɲ\~{\textturna}
ˈɔmɛ]}{(n.)}{1}{}{Tamarindo-homem.}{\textbf{\textit{Ormocarpum
sennoides}}.}{}%
\verb{tambôlô}{}{[t\~{\textturna}ˈbolo]}{(n.)}{1}{}{Tambor.}{}{}
\verb{tambwê}{}{[t\~{\textturna}ˈbwe]}{(n.)}{1}{}{Armadilha.}{}{}
\verb{tambwê}{}{[t\~{\textturna}ˈbwe]}{(n.)}{1}{}{Jugo.}{}{}
\verb{tambwê}{}{[t\~{\textturna}ˈbwe]}{(n.)}{1}{}{Ratoeira.}{}{}
\verb{tamen}{}{[taˈmẽ]}{(adj.)}{1}{}{Adulto.}{}
\verb{tamen}{}{[taˈmẽ]}{(adj.)}{1}{}{Crescido.}{}
\verb{tamen}{}{[taˈmẽ]}{(adj.)}{2}{}{Grande.}{}{}
\verb{tamen}{}{[taˈmẽ]}{(adj.)}{3}{}{Velho.}{}{}
\verb{tampa}{}{[ˈt\~{\textturna}pa]}{(n.)}{1}{}{Tampa.}{}{}
\verb{tampu}{}{[ˈt\~{\textturna}pu]}{(n.)}{1}{}{Tampa.}{}{}
\verb{tampu}{}{[ˈt\~{\textturna}pu]}{(n.)}{2}{}{Tampão.}{}{}
\verb{tamyan}{}{[taˈmj\~{\textturna}]}{(n.)}{1}{}{Marmita.}{}{}
\verb{tamyan}{}{[taˈmj\~{\textturna}]}{(n.)}{2}{}{Tigela.}{}{}
\verb{tan}{}{[ˈt\~{\textturna}]}{(adv.)}{1}{}{Apenas.}{}{}%
\verb{tan}{}{[ˈt\~{\textturna}]}{(adv.)}{2}{}{Só.}{}{}%
\verb{tan}{}{[ˈt\~{\textturna}]}{(adv.)}{3}{}{Sozinho.}{}{}
\verb{tanaji}{}{[taˈnaʒi]}{(adj.)}{1}{}{Tenaz.}{}{}
\verb{tanaza}{}{[tanaˈza]}{(v.)}{1}{}{Danificar.}{}{}
\verb{tanaza}{}{[tanaˈza]}{(v.)}{2}{}{Escangalhar.}{}{}
\verb{tanda}{}{[t\~{\textturna}ˈda]}{(v.)}{1}{}{Entregar.}{}{}
\verb{tangana}{}{[t\~{\textturna}gaˈna]}{(adv.)}{1}{}{Apenas.}{\textbf{Sun
tangana tan ku vida sun}. \textit{Apenas ele com a sua vida}.}{}
\verb{tanji}{}{[t\~{\textturna}ˈʒi]}{(v.)}{1}{}{Chamar de longe.}{}{}
\verb{tanji}{}{[t\~{\textturna}ˈʒi]}{(v.)}{2}{}{Clamar.}{}{}
\verb{tanji}{}{[t\~{\textturna}ˈʒi]}{(v.)}{2}{}{Tanger.}{}{}
\verb{tanjilina}{}{[t\~{\textturna}ʒiˈlina]}{(n.)}{1}{}{Tangerina.}{}{}
\verb{tanjilina}{}{[t\~{\textturna}ʒiˈlina]}{(n.)}{2}{}{Tangerineira.}{\textbf{\textit{Citrus
reticulata}.}}{}
\verb{tanki}{}{[ˈt\~{\textturna}ki]}{(n.)}{1}{}{Chafariz.}{}{}
\verb{tanki}{}{[ˈt\~{\textturna}ki]}{(n.)}{2}{}{Tanque.}{}{}
\verb{tansu}{}{[ˈt\~{\textturna}su]}{(n.)}{1}{}{Idiota.}{}{}
\verb{tansu}{}{[ˈt\~{\textturna}su]}{(n.)}{1}{}{Tanso.}{}{}
\verb{tantan}{}{[t\~{\textturna}ˈt\~{\textturna}]}{(n.)}{1}{}{Gafanhoto.}{}{}\verb{tantu}{}{[ˈt\~{\textturna}tu]}{(quant.)}{1}{}{Tanto.}{}{}
\verb{tapa}{}{[taˈpa]}{(v.)}{1}{}{Tapar.}{}{}
\verb{tapa}{}{[taˈpa]}{(v.)}{2}{}{Vedar.}{}{}
\verb{tapadu}{}{[taˈpadu]}{(adj.)}{1}{}{Tapado.}{}{}
\verb{tapadu}{}{[taˈpadu]}{(adj.)}{1}{}{Vedado.}{}{}
\verb{tapa-wê}{}{[ˈtapa ˈwe]}{(n.)}{1}{}{Bofetada.}{}{}
\verb{tapêtê}{}{[taˈpete]}{(n.)}{1}{}{Tapete.}{}{}
\verb{tapona}{}{[taˈpɔna]}{(n.)}{1}{}{Palmada na cara com as costas da
mão.}{}{}
\verb{tasa}{}{[ˈtasa]}{(n.)}{1}{}{Taça.}{}{}
\verb{tason}{}{[taˈs\~ɔ]}{(v.)}{1}{}{Sentar(-se).}{}{}
\verb{tasondu}{}{[taˈs\~ɔdu]}{(adj.)}{1}{}{Sentado.}{}{}
\verb{tason zekete}{}{[taˈs\~ɔ zɛkɛˈtɛ]}{(expr.)}{1}{}{Sentar-se
imobilizado.}{}{}
\verb{tasu}{}{[ˈtasu]}{(n.)}{1}{}{Tacho.}{}{}
\verb{tata}{}{[taˈta]}{(n.)}{1}{}{Fezes.}{}{}
\verb{tata}{}{[taˈta]}{(v.)}{1}{}{Defecar.}{}{}
\verb{tatali}{}{[ˈtatali]}{(adv.)}{1}{}{Completamente.}{Cf.\textbf{talitali}.}{}{}
\verb{tataluga}{}{[tataˈluga]}{(n.)}{1}{}{Tartaruga.}{Personagem nas narrativas tradicionais. Cf. \emph{tatalugwa}.}{}{}
\verb{tatalugwa}{}{[tataˈlugwa]}{(n.)}{2}{}{Tartaruga.}{}{}%
\verb{tatalugwa}{}{[tataˈlugwa]}{(n.)}{1}{}{Tartaruga.}{Personagem nas narrativas tradicionais.}{}{}
\verb{tatata}{}{[tataˈta]}{(id.)}{1}{}{Cf. \textbf{flexku tatata.}}{}{}
\verb{tatata}{}{[tataˈta]}{(id.)}{2}{}{Cf. \textbf{lêdê tatata.}}{}{}
\verb{tatata}{}{[tataˈta]}{(id.)}{3}{}{Cf. \textbf{tlêmê tatata.}}{}{}
\verb{tatata}{}{[tataˈta]}{(id.)}{4}{}{Cf. \textbf{vivu tatata.}}{}{}
\verb{tava}{}{[ˈtava]}{(part.)}{3}{}{Partícula temporal.}{\textbf{Bô naxi tava nansê.} \textit{Ainda não tinhas
nascido.} \textbf{N tava ka vivê nala}. \emph{Eu estava a viver lá}.}{}
\verb{tava}{}{[ˈtava]}{(v.)}{1}{}{Ser (passado do verbo copulativo \textbf{sa}).}{\textbf{Ê tava ome tamen za}. \textit{Ele já era um homem crescido}.}{}
\verb{tava}{}{[ˈtava]}{(v.)}{2}{}{Estar (passado do verbo copulativo \textbf{sa}).}{\textbf{Men mu tava Plinxipi}. A
\textit{minha mãe estava na Ilha do Príncipe}.}{}
\verb{taxka}{}{[ˈtaʃka]}{(n.)}{1}{}{Pequena loja.}{}{}
\verb{taxka}{}{[ˈtaʃka]}{(n.)}{1}{}{Tasca.}{}{}
\verb{tayu}{}{[ˈtaju]}{(n.)}{1}{}{Açougue.}{}{}
\verb{tayu}{}{[ˈtaju]}{(n.)}{2}{}{Talho.}{}{}
\verb{tayu}{}{[ˈtaju]}{(n.)}{3}{}{Vinco.}{}{}
\verb{tê}{}{[ˈte]}{(v.)}{1}{}{Agarrar.}{}{}%
\verb{tê}{}{[ˈte]}{(v.)}{2}{}{Existir.}{}{}%
\verb{tê}{}{[ˈte]}{(v.)}{3}{}{Haver.}{}{}
\verb{tê}{}{[ˈte]}{(v.)}{4}{}{Possuir.}{}{}%
\verb{tê}{}{[ˈte]}{(v.)}{5}{}{Segurar.}{}{}
\verb{tê}{}{[ˈte]}{(v.)}{6}{}{Ter.}{}{}%
\verb{tê}{}{[ˈte]}{(v.)}{8}{}{Ter que.}{}{}
\verb{tebo}{}{[ˈtɛbɔ]}{(n.)}{1}{}{Impotente.}{}{}
\verb{tê di}{}{[ˈte ˈdi]}{(expr.)}{7}{}{Ter de.}{\textbf{Bô tê di fla}. \textit{Tens de falar}.}{}%
\verb{têdu}{}{[ˈtedu]}{(adj.)}{1}{}{Agarrado.}{}{}
\verb{têdu}{}{[ˈtedu]}{(adj.)}{2}{}{Amparado.}{}{}
\verb{têdu}{}{[ˈtedu]}{(adj.)}{3}{}{Protegido.}{}{}
\verb{têdu}{}{[ˈtedu]}{(adj.)}{4}{}{Segurado.}{}{}
\verb{têdu kankankan}{}{[ˈtedu
k\~{\textturna}k\~{\textturna}ˈk\~{\textturna}]}{(expr.)}{1}{}{Agarradíssimo.}{}{}
\verb{tefitefi}{}{[ˈtɛfiˈtɛfi]}{(id.)}{1}{}{Cf. \textbf{kloson
tefitefi.}}{}{}
\verb{tê fitxin}{}{[ˈte fiˈtʃ\~i]}{(expr.)}{1}{}{Fazer intrigas.}{}{}%
\verb{tê konta ku}{}{[ˈte ˈkõta ˈku]}{(expr.)}{1}{}{Ter responsabilidades
com.}{}{}
\verb{tela}{}{[ˈtɛla]}{(n.)}{2}{}{Nação.}{}{}
\verb{tela}{}{[ˈtɛla]}{(n.)}{3}{}{País.}{}{}
\verb{tela}{}{[ˈtɛla]}{(n.)}{5}{}{Planeta Terra.}{}{}
\verb{tela}{}{[ˈtɛla]}{(n.)}{4}{}{Terra.}{}{}
\verb{telefona}{}{[tɛlɛfɔˈna]}{(v.)}{1}{}{Telefonar.}{}{}
\verb{telefoni}{}{[tɛlɛˈfɔni]}{(n.)}{1}{}{Telefone.}{}{}
\verb{televizon}{}{[tɛlɛviˈz\~ɔ]}{(n.)}{1}{}{Televisão.}{}{}
\verb{tema}{}{[ˈtɛma]}{(n.)}{1}{}{Teimosia.}{}{}
\verb{tê matxi}{}{[ˈte ˈmatʃi]}{(expr.)}{1}{}{Ser difícil.}{}{}%
\verb{tembe}{}{[t\~ɛˈbɛ]}{(adv.)}{1}{}{Também.}{Cf.\textbf{tembeten}.}{}{}
\verb{tembeten}{}{[t\~ɛbɛˈt\~ɛ]}{(adv.)}{1}{}{Também.}{}{}
\verb{templa}{}{[ˈt\~ɛpla]}{(n.)}{1}{}{Condimento.}{}{}
\verb{templa}{}{[ˈt\~ɛpla]}{(n.)}{2}{}{Pimenta.}{}{}%
\verb{templa}{}{[ˈt\~ɛpla]}{(n.)}{3}{}{Tempero.}{}{}%
\verb{templa}{}{[t\~ɛˈpla]}{(v.)}{1}{}{Preparar a palmeira para a extração do
vinho de palma.}{}{}
\verb{templa}{}{[t\~ɛˈpla]}{(v.)}{2}{}{Temperar.}{}{}%
\verb{templadu}{}{[t\~ɛˈpladu]}{(adj.)}{1}{}{Preparada a palmeira para a
extração do vinho de palma.}{}{}
\verb{templadu}{}{[t\~ɛˈpladu]}{(adj.)}{2}{}{Temperado.}{}{}%
\verb{templêlu}{}{[t\~ɛˈplelu]}{(n.)}{1}{}{Tempero.}{}{}
\verb{tempu}{}{[ˈtẽpu]}{(n.)}{1}{}{Clima.}{}{}
\verb{tempu}{}{[ˈtẽpu]}{(n.)}{2}{}{Tempo.}{}{}
\verb{ten}{}{[ˈt\~ɛ]}{(adv.)}{1}{}{Também.}{}{}
\verb{tenda}{}{[ˈt\~ɛda]}{(n.)}{1}{}{Barraca.}{}{}
\verb{tenda}{}{[ˈt\~ɛda]}{(n.)}{2}{}{Emprego.}{}{}
\verb{tenda}{}{[ˈt\~ɛda]}{(n.)}{3}{}{Local onde se juntam cápsulas de
cacau.}{}{}
\verb{tenda}{}{[ˈt\~ɛda]}{(n.)}{4}{}{Loja rural.}{}{}
\verb{tenda}{}{[ˈt\~ɛda]}{(n.)}{5}{}{Tenda.}{}{}
\verb{tenda}{}{[ˈt\~ɛda]}{(n.)}{1}{}{Trabalho.}{}{}
\verb{tendê}{}{[tẽˈde]}{(v.)}{1}{}{Compreender.}{}{}%
\verb{tendê}{}{[tẽˈde]}{(v.)}{2}{}{Entender.}{}{}%
\verb{tendê}{}{[tẽˈde]}{(v.)}{3}{}{Ouvir.}{}{}%
\verb{tendê}{}{[tẽˈde]}{(v.)}{4}{}{Sentir.}{}{}
\verb{tendêdô}{}{[t\~edeˈdo]}{(n.)}{1}{}{Entendedor.}{}{}
\verb{tengu}{}{[tẽˈgu]}{(v.)}{1}{}{Coxear.}{}{}
\verb{tenha}{}{[ˈt\~ɛɲa]}{(n.)}{1}{}{Tainha.}{\textbf{\textit{Mugil
liza}.}}{}
\verb{tenson}{}{[t\~ɛˈs\~ɔ]}{(n.)}{1}{}{Atenção.}{}{}
\verb{tenson}{}{[t\~ɛˈs\~ɔ]}{(n.)}{1}{}{Intenção.}{}{}
\verb{tenson}{}{[t\~ɛˈs\~ɔ]}{(n.)}{1}{}{Pretensão.}{}{}
\verb{tenson}{}{[t\~ɛˈs\~ɔ]}{(n.)}{1}{}{Tensão.}{}{}
\verb{tenta}{}{[tẽˈta]}{(v.)}{1}{}{Tentar.}{}{}
\verb{tentason}{}{[t\~ɛtaˈs\~ɔ]}{(n.)}{1}{}{Tentação.}{}{}
\verb{tententen}{}{[t\~ɛt\~ɛˈt\~ɛ]}{(adj.)}{1}{}{Afoito.}{}{} 
\verb{tententen}{}{[t\~ɛt\~ɛˈt\~ɛ]}{(adj.)}{1}{}{Espevitado.}{}{} 
\verb{tententen}{}{[t\~ɛt\~ɛˈt\~ɛ]}{(adj.)}{2}{}{Impertinente.}{}{} 
\verb{tententen}{}{[t\~ɛt\~ɛˈt\~ɛ]}{(adj.)}{3}{}{Intrometido.}{}{} 
\verb{tententen}{}{[t\~ɛt\~ɛˈt\~ɛ]}{(adj.)}{4}{}{Metediço.}{}{} 
\verb{tententen}{}{[t\~ɛt\~ɛˈt\~ɛ]}{(id.)}{2}{}{Cf. \textbf{sêla
tententen.}}{}{}
\verb{tententen}{}{[t\~ɛt\~ɛˈt\~ɛ]}{(n.)}{1}{}{Afoito.}{}{} 
\verb{tententen}{}{[t\~ɛt\~ɛˈt\~ɛ]}{(n.)}{2}{}{Impertinente.}{}{} 
\verb{tententen}{}{[t\~ɛt\~ɛˈt\~ɛ]}{(n.)}{3}{}{Intrometido.}{}{} 
\verb{tententen}{}{[t\~ɛt\~ɛˈt\~ɛ]}{(n.)}{4}{}{Intromissão.}{}{} 
\verb{tententen}{}{[t\~ɛt\~ɛˈt\~ɛ]}{(n.)}{5}{}{Metediço.}{}{} 
\verb{tê pena}{}{[ˈte ˈpɛna]}{(expr.)}{1}{}{Apiedar-se de.}{}{}%
\verb{tê pena}{}{[ˈte ˈpɛna]}{(expr.)}{1}{}{Lamentar.}{}{}%
\verb{tesa}{}{[ˈtɛsa]}{(n.)}{1}{}{Fachada.}{}{}
\verb{tesa}{}{[ˈtɛsa]}{(n.)}{3}{}{Testa.}{}{}
\verb{tesa-fela}{}{[ˈtɛsa ˈfɛla]}{(n.)}{1}{}{Terça-feira.}{}{}
\verb{tese}{}{[tɛˈsɛ]}{(n.)}{1}{}{\textit{Tese}.}{\textbf{\textit{Rinorea
thomensis}.}}{}
\verb{têsê}{}{[teˈse]}{(v.)}{1}{}{Entrançar.}{}{}
\verb{têsê}{}{[teˈse]}{(v.)}{1}{}{Tecer.}{}{}
\verb{têsêdô}{}{[teseˈdo]}{(n.)}{1}{}{Tecedor.}{}{}
\verb{têsidu}{}{[teˈsidu]}{(adj.)}{1}{}{Tecido.}{}{}
\verb{têsidu}{}{[teˈsidu]}{(n.)}{1}{}{Tecido.}{}{}
\verb{tesu}{}{[ˈtɛsu]}{(n.)}{1}{}{Terço.}{}{}
\verb{tete}{}{[tɛˈtɛ]}{(v.)}{1}{}{Andar de forma lenta e arrastada.}{}{}
\verb{tete}{}{[tɛˈtɛ]}{(v.)}{2}{}{Engatinhar.}{}{}
\verb{tete}{}{[tɛˈtɛ]}{(v.)}{3}{}{Gatinhar.}{}{}
\verb{tê tema}{}{[ˈte tɛˈma]}{(expr.)}{1}{}{Teimar.}{}{}
\verb{tetu}{}{[ˈtɛtu]}{(n.)}{1}{}{Teto.}{}{}%
\verb{textu}{}{[ˈtɛʃtu]}{(n.)}{1}{}{Texto.}{}{}%
\verb{têya}{}{[ˈteja]}{(n.)}{1}{}{Telha.}{}{}
\verb{teyateya}{}{[ˈtɛjaˈtɛja]}{(n.)}{1}{}{Mucumba.}{\textbf{\textit{Rothmannia urcelliformis}.}}{}
\verb{teyateya}{}{[ˈtɛjaˈtɛja]}{(n.)}{2}{}{\textit{Teya-teya}.}{\textbf{\textit{Rothmannia urcelliformis}.}}{}
\verb{teza}{}{[tɛˈza]}{(v.)}{1}{}{Entesar.}{}{}%
\verb{teza}{}{[tɛˈza]}{(v.)}{2}{}{Esticar.}{}{}
\verb{tezadu tõõõ}{}{[tɛˈzadu ˈtõõõ]}{(adj.)}{1}{}{Esticadíssimo.}{}{}
\verb{t\~i\~i\~i}{}{[ˈt\~i\~i\~i]}{(id.)}{1}{}{Cf. \textbf{mundjadu t\~i\~i\~i.}}{}{}
%\verb{tin}{}{[ˈtĩ]}{(id.)}{1}{}{Cf. \textbf{t\~i\~i\~i}.}{}{}
\verb{tijigadu}{}{[tiʒiˈgadu]}{(n.)}{1}{}{Tísico.}{}{}
\verb{tijigu}{}{[tiˈʒigu]}{(n.)}{1}{}{Tuberculose.}{}{}
\verb{tindji}{}{[tĩˈdʒi]}{(n.)}{1}{}{Pênis de criança.}{Cf.
\textbf{txintxi}.}{}{}
\verb{tinha}{}{[ˈt\~iɲa]}{(v.)}{1}{}{Tinha.}{}{}
\verb{tintu}{}{[ˈtĩtu]}{(adj.)}{1}{}{Tinto.}{}{}
\verb{tipu}{}{[ˈtipu]}{(n.)}{1}{}{Fulano.}{}{}
\verb{tipu}{}{[ˈtipu]}{(n.)}{2}{}{Tipo.}{}{}
\verb{tisa}{}{[tiˈsa]}{(v.)}{1}{}{Atiçar.}{}{}
\verb{titiya}{}{[tiˈtija]}{(n.)}{1}{}{Tia-avó.}{}{}
\verb{titiyu}{}{[tiˈtiju]}{(n.)}{1}{}{Tio-avô.}{}{}
\verb{tixi}{}{[tiˈʃi]}{(v.)}{1}{}{Espirrar.}{}{}
\verb{tixidu}{}{[tiˈʃidu]}{(n.)}{1}{}{Tecido.}{Cf. \textbf{têsidu}.}{}{}
\verb{tiya}{}{[ˈtija]}{(n.)}{1}{}{Tia.}{}{}
\verb{tiya-nglandji}{}{[ˈtija ˈŋgl\~{\textturna}dʒi]}{(n.)}{1}{}{Tia-avó.}{}{}
\verb{tiyu}{}{[ˈtiju]}{(n.)}{1}{}{Tio.}{}{}
\verb{tiyu-nglandji}{}{[ˈtiju ˈŋgl\~{\textturna}dʒi]}{(n.)}{1}{}{Tio-avô.}{}{}
\verb{tlaba}{}{[tlaˈba]}{(v.)}{1}{}{Funcionar.}{}{}
\verb{tlaba}{}{[tlaˈba]}{(v.)}{2}{}{Trabalhar.}{}{}%
\verb{tlabadô}{}{[tlabaˈdo]}{(n.)}{1}{}{Trabalhador.}{}{}
\verb{tlabadu}{}{[tlaˈbadu]}{(adj.)}{1}{}{Trabalhado.}{}{}
\verb{tlabe}{}{[tlaˈbɛ]}{(n.)}{1}{}{Infelicidade.}{}{}%
\verb{tlabe}{}{[tlaˈbɛ]}{(n.)}{2}{}{Infortúnio.}{}{}%
\verb{tlabe}{}{[tlaˈbɛ]}{(n.)}{3}{}{Padecimento.}{}{}
\verb{tlabe}{}{[tlaˈbɛ]}{(n.)}{4}{}{Trabalho braçal.}{}{}
\verb{tlanka}{}{[tl\~{\textturna}ˈka]}{(v.)}{1}{}{Trancar.}{}{}
\verb{tlansadu}{}{[tl\~{\textturna}ˈsadu]}{(n.)}{1}{}{Espada.}{}{}
\verb{tlapasa}{}{[tlaˈpasa]}{(n.)}{1}{}{Falsidade.}{}{}
\verb{tlapasa}{}{[tlaˈpasa]}{(n.)}{2}{}{Mentira.}{}{}%
\verb{tlapasa}{}{[tlaˈpasa]}{(n.)}{3}{}{Trapaça.}{}{}
\verb{tlapaya}{}{[tlapaˈja]}{(v.)}{1}{}{Atrapalhar.}{}{}
\verb{tlapayadu}{}{[tlapaˈjadu]}{(adj.)}{1}{}{Atrapalhado.}{}{}
\verb{tlata}{}{[tlaˈta]}{(v.)}{1}{}{Cuidar.}{}{}
\verb{tlata}{}{[tlaˈta]}{(v.)}{2}{}{Tratar.}{}{}
\verb{tlatadu}{}{[tlaˈtadu]}{(adj.)}{1}{}{Tratado.}{}{}
\verb{tlatamentu}{}{[tlataˈm\~ɛtu]}{(n.)}{1}{}{Tratamento.}{}{}
\verb{tlatason}{}{[tlataˈs\~ɔ]}{(n.)}{1}{}{Cuidados.}{}{}
\verb{tlatason}{}{[tlataˈs\~ɔ]}{(n.)}{1}{}{Tratamento.}{}{}
\verb{tlavesa}{}{[tlaˈvɛsa]}{(n.)}{1}{}{Travessa.}{}{}
\verb{tlavon}{}{[tlaˈv\~ɔ]}{(n.)}{1}{}{Freio.}{}{}
\verb{tlavon}{}{[tlaˈv\~ɔ]}{(n.)}{2}{}{Travão.}{}{}
\verb{tlaxi}{}{[ˈtlaʃi]}{(n.)}{1}{}{Costas.}{}{}
\verb{tlaxi}{}{[ˈtlaʃi]}{(n.)}{2}{}{Parte de trás.}{}{}
\verb{tlaxi}{}{[ˈtlaʃi]}{(prep. n.)}{1}{}{Atrás.}{}{}
\verb{tlaxi-glêza}{}{[ˈtlaʃi ˈgleza]}{(n.)}{1}{}{Confraternização realizada atrás da igreja, após as cerimónias religiosas, que consiste na repartição de comida e bebidas.}{}{}
\verb{tlaxi-kabêsa}{}{[ˈtlaʃi kaˈbesa]}{(n.)}{1}{}{Nuca.}{}{}
\verb{tlaxilin}{}{[tlaxiˈlĩ]}{(n.)}{1}{}{Cordão de ouro.}{}{}
\verb{tlaza}{}{[tlaˈza]}{(v.)}{1}{}{Atrasar.}{}{}
\verb{tlazadu}{}{[tlaˈzadu]}{(adj.)}{1}{}{Atrasado.}{}{}
\verb{tlebesa}{}{[tlɛbɛˈsa]}{(n.)}{1}{}{Peixe-voador.}{\textbf{\textit{Cheilopogon melanurus}}.}{}
\verb{tlebesa}{}{[tlɛbɛˈsa]}{(v.)}{1}{}{Atravessar.}{}{}%
\verb{tlêbêsubê}{}{[tlebeˈsube]}{(n.)}{1}{}{Afoito.}{}{}
\verb{tlêbêsubê}{}{[tlebeˈsube]}{(n.)}{2}{}{Impertinente.}{}{}
\verb{tlêbêsubê}{}{[tlebeˈsube]}{(n.)}{3}{}{Intrometido.}{}{}%
\verb{tlêbêsubê}{}{[tlebeˈsube]}{(n.)}{4}{}{Irrequieto.}{}{}%
\verb{tlêbêsubê}{}{[tlebeˈsube]}{(n.)}{5}{}{Metediço.}{}{}
\verb{tlêbêsubê}{}{[tlebeˈsube]}{(n.)}{6}{}{Travesso.}{}{}
\verb{tlega}{}{[tlɛˈga]}{(v.)}{1}{}{Entregar.}{}{}
\verb{tlêmê}{}{[tleˈme]}{(v.)}{1}{}{Tremer.}{}{}
\verb{tlêmê gidigidi}{}{[tleˈme giˈdigiˈdi]}{(expr.)}{1}{}{Tremer intensamente.}{}{}
\verb{tlêmê tatata}{}{[tleˈme tataˈta]}{(expr.)}{1}{}{Tiritar.}{}{}
\verb{tlen}{}{[ˈtlẽ]}{(n.)}{1}{}{Comboio.}{}{}
\verb{tlen}{}{[ˈtlẽ]}{(n.)}{2}{}{Trem.}{}{}
\verb{tlêsêlu}{}{[tleˈselu]}{(num.)}{1}{}{Terceiro.}{Cf. \textbf{tlusêlu}.}{}{}
\verb{tlêxi}{}{[ˈtleʃi]}{(num.)}{1}{}{Três.}{}{}
\verb{tlêxi-dexi}{}{[ˈtleʃi ˈdɛʃi]}{(num.)}{2}{}{Trinta.}{Cf. \textbf{tlinta}.}{}{}
\verb{tlezaman-pasa}{}{[tlɛzaˈm\~{\textturna} paˈsa]}{(adv.)}{1}{}{De hoje a três dias.}{}{}
\verb{tlezantonte}{}{[ˈtlɛz\~{\textturna}ˈt\~ɔtɛ]}{(adv.)}{1}{}{trás-anteonten.}{}{}
\verb{tlezanu-pasadu}{}{[ˈtlɛzanu paˈsadu]}{(adv.)}{1}{}{Há dois anos.}{}{}
\verb{tlêzê}{}{[ˈtleze]}{(num.)}{1}{}{Treze.}{}{}
\verb{tlezentu}{}{[tlɛˈz\~ɛtu]}{(num.)}{1}{}{Trezentos.}{}{}
\verb{tliatu}{}{[tliˈatu]}{(n.)}{1}{}{Teatro.}{}{}
\verb{tligi}{}{[ˈtligi]}{(n.)}{1}{}{Tigre.}{}{}
\verb{tligi-mwala}{}{[ˈtligi ˈmwala]}{(n.)}{1}{}{Tigresa.}{}{}
\verb{tligu}{}{[ˈtligu]}{(n.)}{1}{}{Trigo.}{}{}
\verb{tlinku}{}{[ˈtlĩku]}{(n.)}{1}{}{Trinco.}{}{}
\verb{tlinta}{}{[ˈtlĩta]}{(num.)}{1}{}{Trinta.}{}{}
\verb{tlipa}{}{[ˈtlipa]}{(n.)}{1}{}{Intestinos.}{}{}
\verb{tlipa}{}{[ˈtlipa]}{(n.)}{2}{}{Tripas.}{}{}
\verb{tlisa}{}{[ˈtlisa]}{(n.)}{1}{}{Crista.}{}{}
\verb{tlisa}{}{[ˈtlisa]}{(n.)}{2}{}{Hepatite.}{}{}
\verb{tlisa}{}{[ˈtlisa]}{(n.)}{3}{}{Icterícia.}{}{}
\verb{tlividu}{}{[tliˈvidu]}{(adj.)}{1}{}{Atrevido.}{}{}
\verb{tlixtêza}{}{[tliʃˈteza]}{(n.)}{1}{}{Desgosto.}{}{}%
\verb{tlixtêza}{}{[tliʃˈteza]}{(n.)}{2}{}{Tristeza.}{}{}
\verb{tloka}{}{[tlɔˈka]}{(v.)}{1}{}{Trocar.}{}{}
\verb{tlokadu}{}{[tlɔˈkadu]}{(n.)}{1}{}{Trocado.}{}{}
\verb{tloku}{}{[ˈtlɔku]}{(n.)}{1}{}{Troco.}{}{}
\verb{tlomentu}{}{[tlɔˈm\~ɛtu]}{(n.)}{1}{}{Barulho.}{}{}%
\verb{tlomentu}{}{[tlɔˈm\~ɛtu]}{(n.)}{2}{}{Desordem.}{}{}
\verb{tlosa}{}{[ˈtlɔsa]}{(n.)}{1}{}{Trouxa de roupa.}{}{}
\verb{tlôsê}{}{[tloˈse]}{(v.)}{1}{}{Torcer.}{}{}
\verb{tlotlo}{}{[tlɔˈtlɔ]}{(v.)}{1}{}{Economizar.}{}{}
\verb{tlotlo}{}{[tlɔˈtlɔ]}{(v.)}{2}{}{Poupar.}{}{}
\verb{tlovada}{}{[tlɔˈvada]}{(n.)}{1}{}{Chuva forte.}{}{}
\verb{tlovada}{}{[tlɔˈvada]}{(n.)}{2}{}{Tempestade.}{}{}
\verb{tlovada}{}{[tlɔˈvada]}{(n.)}{3}{}{Trovão.}{}{}
\verb{tlovu}{}{[ˈtlɔvu]}{(n.)}{1}{}{Dente de serra.}{}{}
\verb{tlovu}{}{[ˈtlɔvu]}{(n.)}{2}{}{Lâmina de serra.}{}{}
\verb{tlôxidu}{}{[tloˈʃidu]}{(adj.)}{1}{}{Torcido.}{}{}
\verb{tlôxidu}{}{[tloˈʃidu]}{(n.)}{1}{}{Mecha.}{}{}
\verb{tlôxidu}{}{[tloˈʃidu]}{(n.)}{2}{}{Rastilho.}{}{}%
\verb{tlubon-tunha}{}{[tluˈbõ ˈt\~uɲa]}{(n.)}{1}{}{Tubarão (espécie).}{}{}
\verb{tlubuladu}{}{[tlubuˈladu]}{(adj.)}{1}{}{Agitado.}{}{}
\verb{tlubuladu}{}{[tlubuˈladu]}{(adj.)}{2}{}{Atribulado.}{}{}
\verb{tlubuladu}{}{[tlubuˈladu]}{(adj.)}{3}{}{Desconcertado.}{}{}
\verb{tlubuladu}{}{[tlubuˈladu]}{(adj.)}{4}{}{Impaciente.}{}{}
\verb{tlubuladu}{}{[tlubuˈladu]}{(adj.)}{5}{}{Perturbado.}{}{}
\verb{tlubuladu}{}{[tlubuˈladu]}{(adj.)}{6}{}{Rabugento.}{}{}
\verb{tlubulason}{}{[tlubulaˈs\~ɔ]}{(n.)}{1}{}{Agitação.}{}{}
\verb{tlubulason}{}{[tlubulaˈs\~ɔ]}{(n.)}{2}{}{Atribulação.}{}{}
\verb{tlubulason}{}{[tlubulaˈs\~ɔ]}{(n.)}{3}{}{Desconcentração.}{}{}
\verb{tlubulason}{}{[tlubulaˈs\~ɔ]}{(n.)}{4}{}{Perturbação.}{}{}
\verb{tlubutu-boka}{}{[tluˈbutu ˈbɔka]}{(n.)}{1}{}{Afta.}{}{}
\verb{tluki-sun-dêsu}{}{[ˈtluki ˈsũ ˈdesu]}{(n.)}{1}{}{Truqui.}{\textbf{\textit{Prinia molleri}.}}{}
\verb{tlundu}{}{[ˈtlũdu]}{(n.)}{1}{}{Carnaval.}{}{}%
\verb{tlundu}{}{[ˈtlũdu]}{(n.)}{2}{}{Entrudo.}{}{}
\verb{tlunfu}{}{[ˈtlũfu]}{(n.)}{1}{}{Triunfo.}{}{}
\verb{tlusêlu}{}{[tluˈselu]}{(adj.)}{1}{}{Terceiro.}{}{}
\verb{tluxi}{}{[ˈtluʃi]}{(n.)}{1}{}{Cueca.}{}{}
\verb{to}{}{[ˈtɔ]}{(v.)}{1}{}{Bicar.}{}{}
\verb{to}{}{[ˈtɔ]}{(v.)}{2}{}{Gotejar.}{}{}%
\verb{to}{}{[ˈtɔ]}{(v.)}{3}{}{Pingar.}{}{}
\verb{tobo}{}{[ˈtɔbɔ]}{(n.)}{1}{}{Izaquente.}{}{}
\verb{tôdô}{}{[ˈtodo]}{(n.)}{1}{}{Período pós-parto em que a mulher
tradicionalmente não sai de casa durante oito dias.}{}{}
\verb{tôdô}{}{[ˈtodo]}{(n.)}{2}{}{Repouso pós-parto.}{}{}
\verb{tôdô}{}{[ˈtodo]}{(n.)}{3}{}{Resguardo.}{}{}
\verb{tôdô}{}{[ˈtodo]}{(n.)}{4}{}{Toldo.}{}{}
\verb{tôdô}{}{[ˈtodo]}{(n.)}{5}{}{Tordo de São Tomé.}{\textbf{\textit{Turdus
olivaceofuscus}.}}{}
\verb{tôdô-santu}{}{[ˈtodo
ˈs\~{\textturna}tu]}{(n.)}{1}{}{Todos-os-Santos.}{}{}%
\verb{toka}{}{[tɔˈka]}{(v.)}{1}{}{Beber muito.}{}{}%
\verb{toka}{}{[tɔˈka]}{(v.)}{2}{}{Caber a.}{}{}%
\verb{toka}{}{[tɔˈka]}{(v.)}{3}{}{Deveria.}{}{}%
\verb{toka}{}{[tɔˈka]}{(v.)}{4}{}{Tocar.}{}{}
\verb{toka}{}{[tɔˈka]}{(v.)}{5}{}{Tocar um instrumento musical.}{}%
\verb{tokadô}{}{[tɔkaˈdo]}{(n.)}{1}{}{Músico.}{}{}
\verb{tokadô}{}{[tɔkaˈdo]}{(n.)}{2}{}{Tocador.}{}{}
\verb{tokadu}{}{[tɔˈkadu]}{(adj.)}{1}{}{Tocado.}{}{}
\verb{toki}{}{[ˈtɔki]}{(n.)}{1}{}{Música.}{}{}
\verb{toki}{}{[ˈtɔki]}{(n.)}{2}{}{Ritmo musical.}{}{}
\verb{toki}{}{[ˈtɔki]}{(n.)}{3}{}{Toque.}{}{}
\verb{tôkô}{}{[ˈtoko]}{(n.)}{1}{}{Pedacinho.}{}{}
\verb{tôkô}{}{[ˈtoko]}{(n.)}{1}{}{Toco.}{}{}
\verb{tôkô}{}{[ˈtoko]}{(n.)}{1}{}{Tronco.}{}{}
\verb{tola}{}{[ˈtɔla]}{(n.)}{1}{}{Tora.}{}{}
\verb{tola}{}{[tɔˈla]}{(v.)}{1}{}{Torar.}{}{}
\verb{tola}{}{[tɔˈla]}{(v.)}{2}{}{Torrar.}{}{}
\verb{toladu}{}{[tɔˈladu]}{(adj.)}{1}{}{Torrado.}{}{}
\verb{tôli}{}{[ˈtoli]}{(n.)}{1}{}{Torre.}{}{}
%\verb{toma}{}{[tɔˈma]}{(v.)}{1}{}{Ficar possuído.}{}{}
\verb{toma}{}{[tɔˈma]}{(v.)}{2}{}{Pegar.}{}{}%
\verb{toma}{}{[tɔˈma]}{(v.)}{3}{}{Receber.}{}{}%
\verb{toma}{}{[tɔˈma]}{(v.)}{4}{}{Retirar.}{}{}%
\verb{toma}{}{[tɔˈma]}{(v.)}{5}{}{Tirar.}{}{}%
\verb{toma}{}{[tɔˈma]}{(v.)}{6}{}{Tomar.}{}{}%
\verb{tomadu}{}{[tɔˈmadu]}{(adj.)}{1}{}{Bêbado.}{}{}
\verb{tomadu}{}{[tɔˈmadu]}{(adj.)}{2}{}{Enfeitiçado.}{}{}
\verb{tomadu}{}{[tɔˈmadu]}{(adj.)}{3}{}{Tomado.}{}{}
\verb{toma fôsa}{}{[tɔˈma ˈfosa]}{(expr.)}{1}{}{Recuperar-se.}{}{}
\verb{toma kinjila}{}{[tɔˈma kĩˈʒila]}{(expr.)}{1}{}{Desrespeitar.}{}{}
\verb{toma kôlô dixi}{}{[tɔˈma ˈkolo ˈdiʃi]}{(expr.)}{1}{}{Ressurgir.}{}{}
\verb{toma mina}{}{[tɔˈma ˈmina]}{(expr.)}{1}{}{Amantizar.}{}{}
\verb{toma mina}{}{[tɔˈma ˈmina]}{(expr.)}{1}{}{Amigar.}{}{}
\verb{toma mina}{}{[tɔˈma ˈmina]}{(expr.)}{1}{}{Casar.}{}{}
\verb{toma mwala}{}{[tɔˈma ˈmwala]}{(expr.)}{1}{}{Amantizar.}{}{}
\verb{toma mwala}{}{[tɔˈma ˈmwala]}{(expr.)}{1}{}{Amigar.}{}{}
\verb{toma mwala}{}{[tɔˈma ˈmwala]}{(expr.)}{1}{}{Casar.}{}{}
\verb{toma pêtu bala}{}{[tɔˈma ˈpetu ˈbala]}{(expr.)}{1}{}{Assumir.}{}{}
\verb{toma pêtu bala}{}{[tɔˈma ˈpetu ˈbala]}{(expr.)}{1}{}{Chegar-se à frente com destemor.}{}{}
\verb{toma santu}{}{[tɔˈma ˈs\~{\textturna}tu]}{(expr.)}{1}{}{Entrar em transe.}{}{}
\verb{tome-gaga}{}{[tɔˈmɛ gaˈga]}{(n.)}{1}{}{Papa-moscas de São Tomé.}{\textbf{\textit{Terpsiphone atrochalybeia}.}}{}
\verb{ton}{}{[ˈtõ]}{(n.)}{1}{}{Tom.}{}{}
\verb{ton}{}{[ˈtõ]}{(v.)}{1}{}{Pegar.}{}{}{}
\verb{ton}{}{[ˈtõ]}{(v.)}{1}{}{Receber.}{Cf. \textbf{toma}.}{}{}
\verb{tônêla}{}{[toˈnela]}{(n.)}{1}{}{Torneira.}{}{}
\verb{tonfonsu}{}{[tõˈfõsu]}{(n.)}{1}{}{\textit{Tonfonsu}.}{\textbf{\textit{Adenostemma perrottetii}.}}{}
\verb{tonga}{}{[ˈtõga]}{(n.)}{1}{}{Tonga.}{Descendente dos trabalhadores contratados.}{}{}
\verb{tonifonsu}{}{[toniˈfõsu]}{(n.)}{1}{}{\textit{Tonifonsu}.}{\textbf{\textit{Adenostemma perrottetii}.}}{}
\verb{toniku}{}{[ˈtɔniku]}{(n.)}{1}{}{Tônico.}{}{}
\verb{toniku}{}{[ˈtɔniku]}{(n.)}{2}{}{Xarope.}{}{}
\verb{tonitoni}{}{[ˈtõniˈtõni]}{(n.)}{1}{}{Feridas na pele.}{}{}
\verb{tonitoni}{}{[ˈtõniˈtõni]}{(n.)}{2}{}{Reação cutânea.}{}{}
\verb{tono}{}{[tɔˈnɔ]}{(v.)}{1}{}{Beliscar.}{}{}
\verb{tono}{}{[tɔˈnɔ]}{(v.)}{2}{}{Esmigalhar.}{}{}
\verb{tono}{}{[tɔˈnɔ]}{(v.)}{3}{}{Espicaçar.}{}{}
\verb{tono}{}{[tɔˈnɔ]}{(v.)}{4}{}{Picar.}{}{}
\verb{tono}{}{[tɔˈnɔ]}{(v.)}{5}{}{Tirar um pedaço.}{}{}
\verb{tônôtônô}{}{[toˈnotoˈno]}{(n.)}{1}{}{Feridas.}{}{}%
\verb{tõõõ}{}{[ˈtõõõ]}{(id.)}{1}{}{Cf. \textbf{felu tõõõ}.}{}{}
\verb{tõõõ}{}{[ˈtõõõ]}{(id.)}{2}{}{Cf. \textbf{tezadu tõõõ.}}{}{}
\verb{topi}{}{[ˈtɔpi]}{(n.)}{1}{}{Topo.}{}{}
\verb{topi}{}{[ˈtɔpi]}{(n.)}{2}{}{Tropeção.}{}{}
\verb{topi}{}{[ˈtɔpi]}{(n.)}{3}{}{Tropeço.}{}{}
\verb{toson}{}{[tɔˈs\~ɔ]}{(n.)}{1}{}{Dinheiro.}{}{}
\verb{toson}{}{[tɔˈs\~ɔ]}{(n.)}{2}{}{Tostão.}{}{}
\verb{tota}{}{[tɔˈta]}{(v.)}{1}{}{Entortar(-se).}{}{}%
\verb{tota}{}{[tɔˈta]}{(v.)}{2}{}{Virar.}{}{}
\verb{toto}{}{[ˈtɔtɔ]}{(adj.)}{1}{}{Torto.}{}{}
\verb{toto}{}{[tɔˈtɔ]}{(n.)}{1}{}{Vagina.}{}{}
\verb{tôtô}{}{[toˈto]}{(adj.)}{1}{}{Anão.}{}{}
\verb{tôtô}{}{[toˈto]}{(adj.)}{1}{}{Baixo.}{}{}
\verb{tôtô}{}{[toˈto]}{(adj.)}{2}{}{Pequeno.}{}{}
\verb{tôwôtôwô}{}{[toˈwotoˈwo]}{(adj.)}{1}{}{Estouvado.}{}{}
\verb{tôwôtôwô}{}{[toˈwotoˈwo]}{(adj.)}{2}{}{Precipitado.}{}{}
\verb{toxa}{}{[ˈtɔʃa]}{(n.)}{1}{}{Candeeiro.}{}{}
\verb{toxa}{}{[ˈtɔʃa]}{(n.)}{2}{}{Lamparina.}{}{}
\verb{toxa}{}{[ˈtɔʃa]}{(n.)}{3}{}{Tocha.}{}{}%
\verb{toxi}{}{[ˈtɔʃi]}{(n.)}{1}{}{Tosse.}{}{}
\verb{toxi}{}{[tɔˈʃi]}{(v.)}{1}{}{Tossir.}{}{}
\verb{toya}{}{[ˈtɔja]}{(n.)}{1}{}{Toalha.}{}{}
\verb{tudaxi}{}{[tudaˈʃi]}{(adv.)}{1}{}{Igualmente.}{}{}%
\verb{tudaxi}{}{[tudaˈʃi]}{(adv.)}{2}{}{Também.}{\textbf{Êlê tudaxi ka kume}.
\textit{Ele também comerá}.}{}%
\verb{tudaxi}{}{[tudaˈʃi]}{(indef.)}{1}{}{Tudo.}{\textbf{Bô ka ba paga tudaxi}. \textit{Vais pagar tudo}.}{}%
%\verb{tudaxi}{}{[tudaˈʃi]}{(quant.)}{1}{}{Todas.}{}{}
\verb{tudaxi}{}{[tudaˈʃi]}{(quant.)}{2}{}{Todo.}{\textbf{Sode tudaxi be dinen}. \textit{Os soldados todos foram-se embora}.}{}
\verb{tudu}{}{[ˈtudu]}{(indef.)}{1}{}{Tudo.}{}{}
%\verb{tudu}{}{[ˈtudu]}{(quant.)}{1}{}{Todas.}{}{}
\verb{tudu}{}{[ˈtudu]}{(quant.)}{2}{}{Todo.}{}{}
\verb{tudu modu}{}{[ˈtudu ˈmɔdu]}{(adv.)}{1}{}{Ao menos.}{}{}
\verb{tufu}{}{[ˈtufu]}{(n.)}{1}{}{Estocada.}{}{}
\verb{tufu}{}{[tuˈfu]}{(v.)}{1}{}{Copular rapidamente.}{}{}
\verb{tufu}{}{[tuˈfu]}{(v.)}{2}{}{Enfiar.}{}{}
\verb{tufu}{}{[tuˈfu]}{(v.)}{3}{}{Introduzir.}{}{}
\verb{tufu}{}{[tuˈfu]}{(v.)}{4}{}{Meter.}{}{}%
\verb{tufu}{}{[tuˈfu]}{(v.)}{4}{}{Penetrar.}{}{}%
\verb{tugu}{}{[tuˈgu]}{(v.)}{1}{}{Agitar.}{}{}
\verb{tugu}{}{[tuˈgu]}{(v.)}{2}{}{Bulir.}{}{}
\verb{tugu}{}{[tuˈgu]}{(v.)}{3}{}{Complicar.}{}{}%
\verb{tugu}{}{[tuˈgu]}{(v.)}{4}{}{Intensificar.}{}{}%
\verb{tugu}{}{[tuˈgu]}{(v.)}{5}{}{Recrudescer.}{}{}%
\verb{tugudu}{}{[tuˈgudu]}{(adj.)}{1}{}{Agitado.}{}{}%
\verb{tugudu}{}{[tuˈgudu]}{(adj.)}{2}{}{Bravo.}{}{}
\verb{tuji}{}{[tuˈʒi]}{(v.)}{1}{}{Mexer.}{}{}
\verb{tuji}{}{[tuˈʒi]}{(v.)}{2}{}{Tugir.}{}{}
\verb{tula}{}{[ˈtula]}{(adv.)}{1}{}{Bastante.}{}{}
\verb{tula}{}{[tuˈla]}{(v.)}{1}{}{Encher.}{}{}
\verb{tula}{}{[tuˈla]}{(v.)}{1}{}{Entupir.}{}{}
\verb{tumatu}{}{[tuˈmatu]}{(n.)}{1}{}{Tomate.}{\textbf{\textit{Lycopersicum
esculentum}}.}{}
\verb{tumbu}{}{[tũˈbu]}{(n.)}{1}{}{Cheiro desagradável.}{}{}
\verb{tumbu}{}{[tũˈbu]}{(n.)}{1}{}{Poeira.}{}{}
\verb{tumba}{}{[ˈtũba]}{(n.)}{2}{}{Tumba.}{}{}
\verb{tumba}{}{[ˈtũba]}{(n.)}{3}{}{Túmulo.}{}{}
\verb{tunga}{}{[tũˈga]}{(v.)}{1}{}{Atiçar.}{}{}
\verb{tunga}{}{[tũˈga]}{(v.)}{2}{}{Convidar.}{}{}%
\verb{tunga}{}{[tũˈga]}{(v.)}{3}{}{Estimular.}{}{}
\verb{tunha}{}{[t\~uˈɲa]}{(n.)}{1}{}{Golfinho.}{}{}
\verb{tuntuntun}{}{[tũtũˈtũ]}{(adj.)}{1}{}{Desconcentrado.}{}{}%
\verb{tuntuntun}{}{[tũtũˈtũ]}{(adj.)}{2}{}{Distraído.}{}{}
\verb{tunxadu}{}{[t\~uˈʃadu]}{(adj.)}{1}{}{Farto.}{}{}
\verb{tutana}{}{[tuˈtana]}{(n.)}{1}{}{Cérebro.}{}{}
\verb{tutu}{}{[tuˈtu]}{(n.)}{1}{}{Fezes.}{}{}%
\verb{tutu}{}{[tuˈtu]}{(v.)}{1}{}{Aglomerar.}{}{}%
\verb{tutu}{}{[tuˈtu]}{(v.)}{2}{}{Ajuntar.}{}{}
\verb{tutu}{}{[tuˈtu]}{(v.)}{3}{}{Juntar.}{}{}
\verb{tutubi}{}{[tutuˈbi]}{(n.)}{1}{}{Criançada.}{}{}
\verb{tutubi}{}{[tutuˈbi]}{(n.)}{2}{}{Crianças.}{}{}
\verb{t\~u\~u\~u}{}{[ˈt\~u\~u\~u]}{(id.)}{1}{}{Cf. \textbf{fede t\~u\~u\~u}.}{}{}
\verb{twa}{}{[ˈtwa]}{(adv.)}{1}{}{Diretamente.}{}{}
\verb{twa}{}{[ˈtwa]}{(v.)}{1}{}{Entoar.}{}{}
\verb{twatwa}{}{[twaˈtwa]}{(n.)}{1}{}{Malagueta pequena.}{\textbf{\textit{Capsicum frutescens}.}}{}
\verb{txada}{}{[ˈtʃada]}{(n.)}{1}{}{Acampamento montado nas praias pelos
pescadores para a pesca de peixe-voador na gravana.}{Cf.
\textbf{xada}.}{}{}
\verb{txakatxaka}{}{[tʃaˈkatʃaˈka]}{(adj.)}{1}{}{Atabalhoado.}{}{}
\verb{txakatxaka}{}{[tʃaˈkatʃaˈka]}{(adj.)}{2}{}{Desgarrado.}{}{}
\verb{txakatxaka}{}{[tʃaˈkatʃaˈka]}{(adj.)}{3}{}{Desinteressante.}{}{}
\verb{txakatxaka}{}{[tʃaˈkatʃaˈka]}{(adj.)}{4}{}{Impróprio.}{}{}
\verb{txakatxaka}{}{[tʃaˈkatʃaˈka]}{(adj.)}{5}{}{Inconveniente.}{}{}
\verb{txakatxaka}{}{[tʃaˈkatʃaˈka]}{(adj.)}{6}{}{Lamacento.}{}{}%
\verb{txanga}{}{[tʃ\~{\textturna}ˈga]}{(n.)}{1}{}{Instrumento de pesca para
apanhar \textbf{pixin}.}{}{}
\verb{txe}{}{[ˈtʃɛ]}{(id.)}{1}{}{Cf. \textbf{labadu txe.}}{}{}
\verb{txeketxeke}{}{[tʃɛˈkɛtʃɛˈkɛ]}{(adj.)}{1}{}{Franzino.}{}{}
\verb{txeketxeke}{}{[tʃɛˈkɛtʃɛˈkɛ]}{(id.)}{1}{}{Cf. \textbf{mlagu
txeketxeke}.}{}{}
\verb{txentxa}{}{[tʃẽˈtʃa]}{(v.)}{1}{}{Poupar.}{}{}
\verb{txentxa}{}{[tʃẽˈtʃa]}{(v.)}{2}{}{Economizar.}{}{}
\verb{txetxetxe}{}{[tʃɛtʃɛˈtʃɛ]}{(id.)}{1}{}{Cf. \textbf{fudu
txetxetxe.}}{}{}
\verb{txi}{}{[ˈtʃi]}{(n.)}{1}{}{Tio.}{}{}
\verb{txibadu}{}{[tʃiˈbadu]}{(adj.)}{1}{}{Baço.}{}{}
\verb{txibadu}{}{[tʃiˈbadu]}{(adj.)}{1}{}{Encardido.}{}{}
\verb{txibadu}{}{[tʃiˈbadu]}{(adj.)}{2}{}{Entupido.}{}{}
\verb{txibadu}{}{[tʃiˈbadu]}{(adj.)}{3}{}{Húmido.}{}{}
\verb{txibadu}{}{[tʃiˈbadu]}{(adj.)}{1}{}{Sem brilho.}{}{}
\verb{txibi}{}{[ˈtʃibi]}{(adv.)}{1}{}{Assim-assim.}{}{}{}
\verb{txibi}{}{[ˈtʃibi]}{(adv.)}{1}{}{Mais ou menos.}{Cf. \textbf{txibitxibi}.}{}{}
\verb{txibitxibi}{}{[ˈtʃibiˈtʃibi]}{(adv.)}{1}{}{Assim-assim.}{}{}%
\verb{txibitxibi}{}{[ˈtʃibiˈtʃibi]}{(adv.)}{2}{}{Mais ou menos.}{}{}
\verb{txibitxibi}{}{[tʃiˈbitʃiˈbi]}{(n.)}{1}{}{Osso parietal.}{}{}
\verb{txibitxobo}{}{[tʃibiˈtʃɔbɔ]}{(adv.)}{1}{}{Assim-assim.}{}{}
\verb{txifina}{}{[ˈtʃifina]}{(n.)}{1}{}{Acicate.}{}{}
\verb{txifina}{}{[ˈtʃifina]}{(n.)}{1}{}{Instigação.}{}{}
\verb{txifungu}{}{[tʃiˈf\~ugu]}{(n.)}{1}{}{Alimento preparado à base de banana madura pisada e farinha de milho.}{Cf. \textbf{kafungu}.}{}
\verb{txika}{}{[tʃiˈka]}{(n.)}{1}{}{Palmadas amigáveis.}{}{}
\verb{txila}{}{[tʃiˈla]}{(v.)}{1}{}{Desbotar.}{}{}
\verb{txila}{}{[tʃiˈla]}{(v.)}{2}{}{Tirar.}{}{}
\verb{txila awa}{}{[tʃiˈla ˈawa]}{(expr.)}{1}{}{Atingir o orgasmo.}{}{}{}
\verb{txila awa}{}{[tʃiˈla ˈawa]}{(expr.)}{2}{}{Gozar.}{}{}{}
\verb{txiladô-ventoza}{}{[tʃilaˈdo v\~ɛˈtɔza]}{(n.)}{1}{}{\textit{Txiladô-ventoza}.}{Terapeuta tradicional que pratica a sangria ou ventosa.}{}
\verb{txiladu}{}{[tʃiˈladu]}{(adj.)}{1}{}{Tirado.}{}{}
\verb{txila lema}{}{[tʃiˈla ˈlɛma]}{(expr.)}{1}{}{Provar.}{}{}{}
\verb{txila semplu}{}{[tʃiˈla ˈs\~ɛplu]}{(expr.)}{1}{}{Satirizar.}{}{}{}
\verb{txila ventoza}{}{[tʃiˈla v\~ɛˈtɔza]}{(expr.)}{1}{}{Sangrar.}{Fazer uma sangria terapêutica.}{}{}
\verb{txila vesu}{}{[tʃiˈla ˈvɛsu]}{(expr.)}{1}{}{Caluniar.}{}{}
\verb{txila vesu}{}{[tʃiˈla ˈvɛsu]}{(expr.)}{1}{}{Depreciar alguém numa canção.}{}{}
\verb{txili}{}{[ˈtʃili]}{(n.)}{1}{}{Bocadinho.}{Cf. \textbf{txilitxoko}.}{}{}
\verb{txilitxili}{}{[ˈtʃiliˈtʃili]}{(n.)}{1}{}{Olho-branco-pequeno de São Tomé.}{\textit{\textbf{Zosterops feae}}.}{}
\verb{txilitxoko}{}{[ˈtʃiliˈtʃɔkɔ]}{(n.)}{1}{}{Bocadinho.}{}{}
\verb{txiloli}{}{[ˈtʃilɔli]}{(n.)}{1}{}{Tchiloli.}{Representação do conflito entre a corte medieval do Imperador Carlos Magno e a do Marquês de Mântua provocado pelo assassínio de Valdevinos.}{}{}
\verb{txilu}{}{[ˈtʃilu]}{(n.)}{1}{}{Tiro.}{}{}
\verb{tximbadu}{}{[tʃ\~iˈbadu]}{(adj.)}{1}{}{Imóvel.}{}{}
\verb{tximbadu}{}{[tʃ\~iˈbadu]}{(adj.)}{2}{}{Inábil.}{}{}
\verb{tximbadu}{}{[tʃ\~iˈbadu]}{(adj.)}{3}{}{Parado.}{}{}
\verb{tximbôtô}{}{[tʃĩˈboto]}{(n.)}{1}{}{Desajeitado.}{}{}
\verb{tximbôtô}{}{[tʃĩˈboto]}{(n.)}{2}{}{Idiota.}{}{}
\verb{tximbôtô}{}{[tʃĩˈboto]}{(n.)}{3}{}{Inábil.}{}{}
\verb{tximbôtô}{}{[tʃĩˈboto]}{(n.)}{4}{}{Inativo.}{}{}
\verb{tximbôtô}{}{[tʃĩˈboto]}{(n.)}{5}{}{Palerma.}{}{}
\verb{tximbôtô}{}{[tʃĩˈboto]}{(n.)}{6}{}{Pateta.}{}{}
\verb{tximbôtô}{}{[tʃĩˈboto]}{(n.)}{7}{}{Tolo.}{}{}
\verb{txinantxinan}{}{[tʃiˈn\~{\textturna}tʃiˈn\~{\textturna}]}{(n.)}{1}{}{Seiva do safuzeiro.}{}{}
\verb{txinantxinan}{}{[tʃiˈn\~{\textturna}tʃiˈn\~{\textturna}]}{(n.)}{2}{}{Seiva venenosa.}{}{}
\verb{txini}{}{[ˈtʃini]}{(n.)}{1}{}{Cachorrinho.}{}{}
\verb{txinta}{}{[ˈtʃĩta]}{(n.)}{1}{}{Tinta.}{}{}
\verb{txintxi}{}{[tʃĩˈtʃi]}{(n.)}{1}{}{Pênis de criança.}{}{}{}
\verb{txintxin}{}{[tʃĩˈtʃĩ]}{(id.)}{1}{}{Cf. \textbf{bega txintxin}.}{}{}
\verb{txintxin}{}{[tʃĩˈtʃĩ]}{(id.)}{2}{}{Cf. \textbf{txintxintxin}.}{}{}
\verb{txintxin}{}{[tʃĩˈtʃĩ]}{(n.)}{1}{}{\textit{Txintxin}.}{\textbf{\textit{Stegastes imbricatus}}.}{}
\verb{txintxintxin}{}{[tʃĩtʃĩˈtʃĩ]}{(id.)}{1}{}{Cf. \textbf{djina txintxintxin.}}{}{}
\verb{txintxintxolo}{}{[tʃĩtʃĩˈtʃɔlɔ]}{(n.)}{1}{}{Tecelão de São Tomé.}{\textbf{\textit{Thomasophantes sanctithomae}.}}{}
\verb{txisa}{}{[tʃiˈsa]}{(v.)}{1}{}{Atiçar.}{}{}
\verb{txita}{}{[ˈtʃita]}{(n.)}{1}{}{Corrimento.}{}{}
\verb{txitu}{}{[ˈtʃitu]}{(n.)}{1}{}{Título.}{}{}
\verb{txitxi}{}{[tʃiˈtʃi]}{(id.)}{1}{}{Cf. \textbf{txintxintxin}.}{}{}
\verb{txizola}{}{[tʃiˈzɔla]}{(n.)}{1}{}{Tesoura.}{}{}
\verb{txizolo}{}{[tʃiˈzɔlɔ]}{(n.)}{2}{}{Tesouro.}{}{}
\verb{txofodu}{}{[tʃɔˈfɔdu]}{(adj.)}{1}{}{Furado.}{}{}%
\verb{txofodu}{}{[tʃɔˈfɔdu]}{(adj.)}{1}{}{Perfurado.}{}{}%
\verb{txoko}{}{[ˈtʃɔkɔ]}{(adj.)}{1}{}{Pequenino.}{}{}%
\verb{txoko}{}{[ˈtʃɔkɔ]}{(adj.)}{2}{}{Pequeno.}{}{}%
\verb{txoko}{}{[ˈtʃɔkɔ]}{(n.)}{1}{}{Bocadinho.}{}{}
\verb{txonzu}{}{ˈ[tʃõzu]}{(adj.)}{1}{}{Mofino.}{}{}
\verb{txonzu}{}{ˈ[tʃõzu]}{(n.)}{1}{}{Garça-de-cabeça-negra.}{\textbf{\textit{Butorides striatus}.}}{}
\verb{txonzu}{}{ˈ[tʃõzu]}{(n.)}{1}{}{Mofino.}{}{}
\end{letra}

\begin{letra}{u}
\verb{ũa}{}{[ˈũa]}{(adv.)}{1}{}{Tão.}{\textbf{Kasô sa \~ua bluku}. \textit{O cão é tão mau}.}{}%
\verb{ũa}{}{[ˈũa]}{(art.)}{1}{}{Um.}{}{}%
\verb{ũa}{}{[ˈũa]}{(art.)}{2}{}{Uma.}{}{}
\verb{ũa}{}{[ˈũa]}{(num.)}{2}{}{Um.}{}{}
\verb{ũa}{}{[ˈũa]}{(num.)}{2}{}{Uma.}{}{}
\verb{ũa data}{}{[ˈũa ˈdata]}{(adv.)}{1}{}{Muito.}{\textbf{Mwala ka blôsê ku ami \~ua data}. \textit{A mulher chateia-se muito comigo}.}{}
\verb{ũa data}{}{[ˈũa ˈdata]}{(quant.)}{1}{}{Muito.}{\textbf{N sa ku feble \~ua data}. \textit{Estou com muita febre}.}{}
%\verb{ũa data}{}{[ˈũa ˈdata]}{(quant.)}{2}{}{Muitas.}{\textbf{Ê ka bê lanza bôbôdu ni matu \~ua data}. \textit{Ele encontra muitas laranjas maduras no mato}.}{}
%\verb{ũa data}{}{[ˈũa ˈdata]}{(quant.)}{3}{}{Muito.}{}{}{}
%\verb{ũa data}{}{[ˈũa ˈdata]}{(quant.)}{4}{}{Muitos.}{}{}
\verb{ũa-dôsu}{}{[ˈũa ˈdosu]}{(quant.)}{2}{}{Alguns.}{\textbf{N tê \~ua-dôsu kopu za}. \textit{Já tenho alguns copos}.}{}%
%\verb{ũa-dôsu}{}{[ˈũa ˈdosu]}{(quant.)}{4}{}{Algumas.}{}{}
\verb{ũa-kwatu}{}{[ˈũa ˈkwatu]}{(n.)}{1}{}{Metade.}{}{}{}
\verb{ũa-ũa}{}{[ˈũa ˈũa]}{(indef.)}{1}{}{Certo.}{\textbf{N tê ũa-ũa pema mu nala liba.} \emph{Tenho uma certa palmeira minha lá em cima.}}{}{}
%\verb{ũa-ũa}{}{[ˈũa ˈũa]}{(quant.)}{2}{}{Alguma.}{}{}
%\verb{ũa-ũa}{}{[ˈũa ˈũa]}{(quant.)}{3}{}{Algumas.}{\textbf{N tê \~ua-\~ua pema mu nala liba}. \textit{Tenho algumas palmeiras minhas lá em cima}.}{}
%\verb{ũa-ũa}{}{[ˈũa ˈũa]}{(quant.)}{4}{}{Alguns.}{}{}
\verb{ubaga}{}{[uˈbaga]}{(n.)}{1}{}{Panela.}{}{}{}
\verb{ubaga-plelele}{}{[uˈbaga plɛlɛˈlɛ]}{(n.)}{1}{}{Caçarola.}{}{}
\verb{ubaga-plelele}{}{[uˈbaga plɛlɛˈlɛ]}{(n.)}{1}{}{Tacho.}{}{}
\verb{ubaga-tela}{}{[uˈbaga ˈtɛla]}{(n.)}{1}{}{Panela de barro.}{}{}
\verb{ubwa}{}{[ˈubwa]}{(n.)}{1}{}{Cerca.}{}{}%
\verb{ubwa}{}{[ˈubwa]}{(n.)}{2}{}{Cercado.}{}{}%
\verb{ubwa}{}{[ˈubwa]}{(n.)}{3}{}{Pocilga.}{}{}
\verb{ubwa}{}{[ˈubwa]}{(n.)}{4}{}{Tapume.}{}{}%
\verb{ubwami}{}{[uˈbwami]}{(n.)}{1}{}{Bochecha.}{}{}
\verb{ubwê}{}{[ˈubwe]}{(n.)}{1}{}{Caule.}{}{}
\verb{ubwê}{}{[ˈubwe]}{(n.)}{2}{}{Corpo humano.}{}{}
\verb{ubwê}{}{[ˈubwe]}{(n.)}{3}{}{Pênis.}{}{}
\verb{ubwê-betu}{}{[ˈubwe ˈbɛtu]}{(adj.)}{1}{}{Desandado.}{}{}%
\verb{ubwê-betu}{}{[ˈubwe ˈbɛtu]}{(adj.)}{2}{}{Desligado.}{}{}%
\verb{ubwê-d'ome}{}{[ˈubwe ˈdɔmɛ]}{(n.)}{1}{}{Pênis.}{}{}%
\verb{ubwê-mwala}{}{[ˈubwe ˈmwala]}{(n.)}{1}{}{Vagina.}{}{}
\verb{uexti}{}{[uˈɛʃti]}{(n.)}{1}{}{Oeste.}{}{}
\verb{uku}{}{[ˈuku]}{(n.)}{1}{}{Lixo.}{}{}%
\verb{uku}{}{[ˈuku]}{(n.)}{2}{}{Sujidade.}{}{}
\verb{uku}{}{[uˈku]}{(n.)}{1}{}{Ânus.}{}{}%
\verb{uku}{}{[uˈku]}{(n.)}{1}{}{Cu.}{}{}%
\verb{ukwe}{}{[ˈukwɛ]}{(n.)}{1}{}{Bago.}{}{}
\verb{ukwe}{}{[ˈukwɛ]}{(n.)}{2}{}{Dragão.}{Personagem do
\textbf{dansu-kongô}.}{}
\verb{ukwe}{}{[ˈukwɛ]}{(n.)}{3}{}{Gigante.}{Personagem de narrativas tradicionais.}{}{}%
\verb{ukwe}{}{[ˈukwɛ]}{(n.)}{5}{}{Testículos.}{}{}
\verb{ukwe-d'ovu}{}{[ˈukwɛ ˈdɔvu]}{(n.)}{1}{}{Testículos.}{}{}
\verb{ukwêtê}{}{[ukweˈte]}{(n.)}{1}{}{\textit{Ukwêtê}.}{\textbf{\textit{Palisota pedicillata}}.}{}{}
\verb{ukwêtê-d'awa}{}{[ukweˈte ˈdawa]}{(n.)}{1}{}{\textit{Ukwêtê-d'awa}.}{\textit{\textbf{Costus afer}}.}{}
\verb{ukwêtê-d'ôbô}{}{[ukweˈte doˈbo]}{(n.)}{1}{}{\textit{Ukwêtê-d'ôbô}.}{\textit{\textbf{Pollia condensata}}.}{}
\verb{ukwêtê-makaku}{}{[ukweˈte maˈkaku]}{(n.)}{1}{}{\textit{Ukwêtê-makaku}.}{\textit{\textbf{Palisota pedicellata}}.}{}{}
\verb{ukwêtê-nglandji}{}{[ukweˈte ˈŋgl\~{\textturna}dʒi]}{(n.)}{1}{}{Bordão-de-macaco.}{\textbf{\textit{Costus giganteus}.}}{}
\verb{ukwe-tlaxi}{}{[uˈkwɛ ˈtlaʃi]}{(n.)}{1}{}{\textit{Ukwe-tlaxi}.}{\textbf{\textit{Phyllanthus amarus}.}}{}{}
\verb{ukwe-wê}{}{[ˈukwɛ ˈwe]}{(n.)}{1}{}{Íris.}{}{}
\verb{uluba}{}{[ˈuluba]}{(n.)}{3}{}{Terçol.}{}{}%
\verb{ululu}{}{[ˈululu]}{(n.)}{1}{}{Placenta.}{}{}
\verb{ulwa}{}{[ˈulwa]}{(n.)}{1}{}{Bananeira-de-leque.}{\textbf{\textit{Ravenala madagascariensis}.}}{}
\verb{umeme}{}{[ˈumɛmɛ]}{(n.)}{1}{}{Andim muito tenro.}{Cf. \textbf{imeme}.}{}
\verb{umida}{}{[ˈumida]}{(n.)}{1}{}{Nevoeiro.}{}{}
\verb{umpete}{}{[ˈ\~upɛtɛ]}{(n.)}{1}{}{Remela.}{Cf. \textbf{impete}.}{}
\verb{unidu}{}{[uˈnidu]}{(adj.)}{1}{}{Unido.}{}{}
\verb{untwe}{}{[ˈũtwɛ]}{(n.)}{1}{}{\textit{Untwe}.}{\textbf{\textit{Chrysophyllum albidum}.}}{}
\verb{untwe-d'ôbô}{}{[ˈũtwɛ doˈbo]}{(n.)}{1}{}{\textit{Untwe-d'ôbô}.}{\textbf{\textit{Crysophyllum africanum}.}}{}
\verb{unu}{}{[uˈnu]}{(adj.)}{1}{}{Despido.}{}{}
\verb{unu}{}{[uˈnu]}{(adj.)}{2}{}{Nu.}{}{}%
\verb{upa}{}{[ˈupa]}{(id.)}{1}{}{Cf. \textbf{dentxi upa}.}{}{}
\verb{upa}{}{[ˈupa]}{(n.)}{1}{}{Paina.}{}{}
\verb{uswa}{}{[ˈuswa]}{(n.)}{1}{}{Vinho de palma fermentado.}{}{}%
\verb{uswa}{}{[ˈuswa]}{(n.)}{2}{}{\textit{Uswa}.}{Dança tradicional.}{}{}
\verb{utu}{}{[uˈtu]}{(n.)}{1}{}{Bolor.}{}{}%
\verb{utu}{}{[uˈtu]}{(n.)}{2}{}{Cogumelo.}{}{}%
\verb{utu}{}{[uˈtu]}{(n.)}{3}{}{Mofo.}{}{}
\verb{utu}{}{[uˈtu]}{(n.)}{4}{}{Musgo.}{}{}%
\verb{uva}{}{[ˈuva]}{(n.)}{1}{}{Uva.}{}{}
\verb{uxi}{}{[uˈʃi]}{(v.)}{1}{}{Agitar.}{}{}
\verb{uxi}{}{[uˈʃi]}{(v.)}{2}{}{Mexer.}{}{}%
\verb{uxi}{}{[uˈʃi]}{(v.)}{3}{}{Misturar.}{}{}%
\verb{uxidu}{}{[uˈʃidu]}{(adj.)}{1}{}{Agitado.}{}{}
\verb{uxidu}{}{[uˈʃidu]}{(adj.)}{2}{}{Mexido.}{}{}
\verb{uxidu}{}{[uˈʃidu]}{(adj.)}{3}{}{Misturado.}{}{}
\verb{uwu}{}{[uˈwu]}{(n.)}{1}{}{Linha.}{}{}%
\verb{uwu}{}{[uˈwu]}{(n.)}{2}{}{Fio.}{}{}
\verb{uwu-dêsu}{}{[uˈwu ˈdesu]}{(n.)}{1}{}{Algodão.}{\textbf{\textit{Gossypium barbadense}.}}{}
\verb{uza}{}{[ˈuza]}{(n.)}{2}{}{Arraia.}{}{}
\verb{uza}{}{[ˈuza]}{(n.)}{1}{}{Raia.}{}{}%
\verb{uzadu}{}{[uˈzadu]}{(adj.)}{1}{}{Usado.}{}{}
\verb{uzuzu}{}{[ˈuzuzu]}{(n.)}{1}{}{Fio desfiado.}{}{}
\verb{uzuzu}{}{[ˈuzuzu]}{(n.)}{3}{}{Fitinhas nas pontas de um xaile.}{}{}
\verb{uzuzu}{}{[ˈuzuzu]}{(n.)}{3}{}{Franja.}{}{}
\end{letra}

\begin{letra}{v}

\verb{va}{}{[ˈva]}{(v.)}{1}{}{Rachar.}{}{}
\verb{vadjin}{}{[vaˈdʒĩ]}{(adj.)}{2}{}{Vadio.}{}{}
\verb{vadjin}{}{[vaˈdʒĩ]}{(adj.)}{3}{}{Vão.}{}{}
\verb{vadjin}{}{[vaˈdʒĩ]}{(n.)}{1}{}{Inútil.}{}{}%
\verb{vadjin}{}{[vaˈdʒĩ]}{(n.)}{2}{}{Vadio.}{}{}%
\verb{vadô}{}{[vaˈdo]}{(n.)}{1}{}{Peixe-voador.}{\textbf{\textit{Cheilopogon melanurus}}.}{}
\verb{vadô}{}{[vaˈdo]}{(n.)}{2}{}{Rachador de lenha.}{}{}
\verb{vadô-guya}{}{[vaˈdo ˈguja]}{(n.)}{1}{}{Peixe-voador-agulha.}{}{}
\verb{vadô-panha}{}{[vaˈdo p\~aˈɲa]}{(n.)}{1}{}{Peixe-voador.}{\textbf{\textit{Cheilopogon melanurus}}.}{}
\verb{vadu}{}{[ˈvadu]}{(adj.)}{1}{}{Aberto.}{}{}
\verb{vadu}{}{[ˈvadu]}{(adj.)}{2}{}{Rachado.}{}{}%
\verb{vagudu}{}{[ˈvagudu]}{(n.)}{1}{}{Desmaio.}{}{}
\verb{vagudu}{}{[ˈvagudu]}{(n.)}{2}{}{Fraqueza.}{}{}%
\verb{vaji}{}{[ˈvaʒi]}{(n.)}{1}{}{Grota.}{}{}
\verb{vaji}{}{[ˈvaʒi]}{(n.)}{1}{}{Limite da roça.}{}{}
\verb{vajin}{}{[vaˈʒĩ]}{(adj.)}{1}{}{Inútil.}{}{}{}
\verb{vajin}{}{[vaˈʒĩ]}{(adj.)}{1}{}{Vadio.}{}{}{}
\verb{vajin}{}{[vaˈʒĩ]}{(adj.)}{1}{}{Vão.}{Cf. \textbf{vadjin}.}{}{}
\verb{vala}{}{[ˈvala]}{(n.)}{1}{}{Açoite.}{}{}
\verb{vala}{}{[ˈvala]}{(n.)}{2}{}{Vala.}{}{}%
\verb{vala}{}{[ˈvala]}{(n.)}{3}{}{Vara.}{}{}%
\verb{vala}{}{[vaˈla]}{(v.)}{1}{}{Passar por.}{}{}%
\verb{vala}{}{[vaˈla]}{(v.)}{2}{}{Ultrapassar.}{}{}
\verb{valadu}{}{[vaˈladu]}{(n.)}{1}{}{Valado.}{}{}
\verb{vala-kazê}{}{[ˈvala
kaˈze]}{(n.)}{1}{}{Carrapichão.}{\textbf{\textit{Desmodium incanum}.}}{}
\verb{valanda}{}{[vaˈl\~{\textturna}da]}{(n.)}{1}{}{Varanda.}{}{}
\verb{vala-ple}{}{[ˈvala ˈplɛ]}{(n.)}{1}{}{Vara-da-praia.} {\textbf{\textit{Turraea vogelii}.}}{}
\verb{valapo}{}{[valaˈpɔ]}{(n.)}{1}{}{Varapau.}{}{}
\verb{valê}{}{[vaˈle]}{(v.)}{1}{}{Acudir.}{}{}
\verb{valê}{}{[vaˈle]}{(v.)}{2}{}{Ajudar.}{}{}
\verb{valê}{}{[vaˈle]}{(v.)}{3}{}{Socorrer.}{}{}
\verb{valê}{}{[vaˈle]}{(v.)}{4}{}{Valer.}{}{}
\verb{valeta}{}{[vaˈlɛta]}{(n.)}{1}{}{Valeta.}{}{}
\verb{valha}{}{[vaˈʎa]}{(v.)}{1}{}{Variar.}{}{}
\verb{valu}{}{[ˈvalu]}{(n.)}{1}{}{Vale.}{}{}
\verb{valudu}{}{[vaˈludu]}{(n.)}{1}{}{Coco seco.}{}{}
\verb{vamplega}{}{[v\~{\textturna}ˈplɛga]}{(n.)}{1}{}{Técnica de construção
de paredes à base de ramos de palmeira.}{}{}
\verb{vangana}{}{[v\~{\textturna}gaˈna]}{(v.)}{1}{}{Cambalear.}{}{}
\verb{vanjeli}{}{[v\~{\textturna}ˈʒɛli]}{(n.)}{1}{}{Evangelho.}{}{}
\verb{vantaji}{}{[v\~{\textturna}ˈtaʒi]}{(n.)}{1}{}{Vantagem.}{}{}
\verb{vantenadu}{}{[v\~{\textturna}tɛˈnadu]}{(id.)}{1}{}{Cf. \textbf{pobli
vantenadu.}}{}{}
\verb{vantxi}{}{[ˈv\~{\textturna}tʃi]}{(adv.)}{1}{}{Adiante.}{}{}
\verb{vantxi}{}{[ˈv\~{\textturna}tʃi]}{(adv.)}{2}{}{Avante.}{}{}%
\verb{vantxi}{}{[ˈv\~{\textturna}tʃi]}{(adv.)}{3}{}{Em frente.}{}{}%
\verb{vapô}{}{[vaˈpo]}{(n.)}{1}{}{Barco.}{}{}%
\verb{vapô}{}{[vaˈpo]}{(n.)}{2}{}{Barco grande.}{}{}%
\verb{vapô}{}{[vaˈpo]}{(n.)}{3}{}{Navio.}{}{}
\verb{va wê}{}{[ˈva ˈwe]}{(expr.)}{1}{}{Arregalar.}{}{}
\verb{ve}{}{[ˈvɛ]}{(adj.)}{2}{}{Velho.}{}{}
\verb{ve}{}{[ˈvɛ]}{(n.)}{1}{}{Idoso.}{}{}
\verb{ve}{}{[ˈvɛ]}{(n.)}{2}{}{Velho.}{}{}
\verb{vê}{}{[ˈve]}{(n.)}{1}{}{Vez.}{}{}
\verb{vede}{}{[vɛˈdɛ]}{(adj.)}{1}{}{Verdade.}{}{}
\verb{vede}{}{[vɛˈdɛ]}{(adj.)}{2}{}{Verdadeiro.}{}{}
\verb{vede}{}{[vɛˈdɛ]}{(n.)}{1}{}{Verdade.}{}{}
\verb{vêdê}{}{[ˈvede]}{(adj.)}{1}{}{Imaturo.}{}{}
\verb{vêdê}{}{[ˈvede]}{(adj.)}{2}{}{Verde.}{}{}
%\verb{vedevede}{}{[vɛdɛvɛˈdɛ]}{(adv.)}{1}{}{De fato.}{}{}
\verb{vede-vede}{}{[vɛˈdɛ vɛˈdɛ]}{(adv.)}{1}{}{De verdade.}{}{}
\verb{vede-vede}{}{[vɛˈdɛ vɛˈdɛ]}{(adv.)}{1}{}{Mesmo.}{}{}
\verb{vede-vede}{}{[vɛˈdɛ vɛˈdɛ]}{(adv.)}{1}{}{Realmente.}{}{}
\verb{vede-vede}{}{[vɛˈdɛ vɛˈdɛ]}{(adv.)}{1}{}{Verdadeiramente.}{}{}
\verb{ve ketekete}{}{[ˈvɛ kɛˈtɛkɛˈtɛ]}{(expr.)}{1}{}{Velhíssimo.}{}{}
\verb{ve-kwa}{}{[ˈvɛ ˈkwa]}{(n.)}{1}{}{Resto.}{}{}
\verb{ve-kwa}{}{[ˈvɛ ˈkwa]}{(n.)}{2}{}{Trapo.}{}{}
\verb{vela}{}{[ˈvɛla]}{(n.)}{1}{}{Vela.}{}{}
\verb{ve-late}{}{[ˈvɛ laˈtɛ]}{(n.)}{1}{}{Trapo.}{}{}
\verb{veludu}{}{[vɛˈludu]}{(n.)}{1}{}{Veludo.}{}{}
\verb{veludu}{}{[vɛˈludu]}{(n.)}{1}{}{\textit{Veludu}.}{\textbf{\textit{Trichilia
grandifolia}}.}{}
\verb{vendê}{}{[ˈvẽde]}{(n.)}{1}{}{Loja.}{}{}
\verb{venena}{}{[vɛnɛˈna]}{(v.)}{1}{}{Envenenar.}{}{}
\verb{venenadu}{}{[vɛnɛˈnadu]}{(v.)}{1}{}{Envenenado.}{}{}
\verb{venenu}{}{[vɛˈnɛnu]}{(n.)}{1}{}{Veneno.}{}{}
\verb{venka}{}{[v\~ɛˈka]}{(n.)}{1}{}{Avenca.}{\textbf{\textit{Adiantum
raddianum}.}}{}
\verb{vensê}{}{[vẽˈse]}{(v.)}{1}{}{Ganhar.}{}{}
\verb{vensê}{}{[vẽˈse]}{(v.)}{2}{}{Vencer.}{}{}
\verb{vensêmentu}{}{[vẽseˈm\~etu]}{(n.)}{1}{}{Salário.}{}{}
\verb{vensêmentu}{}{[vẽseˈm\~etu]}{(n.)}{2}{}{Vencimento.}{}{}
\verb{venta}{}{[v\~ɛˈta]}{(n.)}{1}{}{Cigarro.}{}{}
\verb{venta}{}{[v\~ɛˈta]}{(v.)}{1}{}{Fumar.}{}{}
\verb{venta}{}{[v\~ɛˈta]}{(v.)}{2}{}{Soprar de.}{}{}
\verb{venta}{}{[v\~ɛˈta]}{(v.)}{3}{}{Ventar.}{}{}
\verb{ventoza}{}{[v\~ɛˈtɔza]}{(n.)}{1}{}{Ventosa.}{Cf. \textbf{txila
ventoza}.}{}{}
\verb{ventozu}{}{[v\~ɛˈtɔzu]}{(adj.)}{1}{}{Ventoso.}{}{}
\verb{ventu}{}{[ˈv\~ɛtu]}{(n.)}{1}{}{Vento.}{}{}
\verb{venxa}{}{[ˈv\~ɛʃa]}{(n.)}{1}{}{Amante.}{}{}{}
\verb{venxa}{}{[ˈv\~ɛʃa]}{(n.)}{1}{}{Amásia.}{}{}{}
\verb{venxa}{}{[ˈv\~ɛʃa]}{(n.)}{1}{}{Concubina.}{Cf. \textbf{vivenxa}.}{}{}
\verb{vesu}{}{[ˈvɛsu]}{(n.)}{1}{}{Expressão idiomática.}{}{}
\verb{vesu}{}{[ˈvɛsu]}{(n.)}{2}{}{Máxima.}{}{}
\verb{vesu}{}{[ˈvɛsu]}{(n.)}{3}{}{Provérbio.}{}{}%
\verb{vesu}{}{[ˈvɛsu]}{(n.)}{3}{}{Sátira.}{}{}%
\verb{vesu}{}{[ˈvɛsu]}{(n.)}{4}{}{Verso.}{}{}%
\verb{ve-tlapu}{}{[ˈvɛ ˈtlapu]}{(n.)}{1}{}{Trapo.}{}{}
\verb{veyaku}{}{[vɛˈjaku]}{(n.)}{1}{}{Velhaco.}{}{}
\verb{vida}{}{[ˈvida]}{(n.)}{1}{}{Vida.}{}{}
\verb{vidja}{}{[viˈdʒa]}{(v.)}{1}{}{Vigiar.}{}{}
\verb{vidjadu}{}{[viˈdʒadu]}{(adj.)}{1}{}{Vigiado.}{}{}
\verb{vidjan}{}{[viˈdʒ\~{\textturna}]}{(n.)}{1}{}{Vizinho.}{}{}
\verb{vidola}{}{[viˈdɔla]}{(n.)}{1}{}{Libelinha.}{}{}
\verb{vidola}{}{[viˈdɔla]}{(n.)}{1}{}{Libélula.}{}{}
\verb{viga}{}{[ˈviga]}{(n.)}{1}{}{Barrote.}{}{}
\verb{viga}{}{[ˈviga]}{(n.)}{2}{}{Prumo.}{}{}
\verb{viga}{}{[ˈviga]}{(n.)}{3}{}{Viga.}{}{}
\verb{vigalu}{}{[viˈgalu]}{(n.)}{1}{}{Vigário.}{}{}
\verb{vija}{}{[viˈʒa]}{(v.)}{1}{}{Vigiar.}{Cf. \textbf{vidja}.}{}{}
\verb{vijan}{}{[viˈʒ\~{\textturna}]}{(n.)}{1}{}{Vizinho.}{Cf.
\textbf{vidjan}.}{}{}
\verb{viji}{}{[ˈviʒi]}{(n.)}{1}{}{Hímen.}{}{}
\verb{viji}{}{[ˈviʒi]}{(n.)}{2}{}{Virgem.}{}{}%
\verb{viji}{}{[ˈviʒi]}{(n.)}{3}{}{Virgindade.}{}{}
\verb{vijilanxya}{}{[viʒiˈl\~{\textturna}ʃja]}{(n.)}{1}{}{Vigilância.}{}{}
\verb{vila}{}{[ˈvila]}{(n.)}{3}{}{Vila.}{}{}%
\verb{vilotxi}{}{[viˈlɔtʃi]}{(n.)}{1}{}{Broto.}{}{}
\verb{vilotxi}{}{[viˈlɔtʃi]}{(n.)}{2}{}{Rebento.}{}{}
\verb{vin}{}{[ˈvĩ]}{(n.)}{1}{}{Vinho.}{}{}
\verb{vinagli}{}{[viˈnagli]}{(n.)}{1}{}{Vinagre.}{}{}
\verb{vinda}{}{[ˈvĩda]}{(n.)}{1}{}{Vinda.}{}{}%
\verb{vinga}{}{[vĩˈga]}{(v.)}{1}{}{Vingar.}{}{}
\verb{vingansa}{}{[vĩˈg\~{\textturna}sa]}{(n.)}{1}{}{Vingança.}{}{}
\verb{vin-pema}{}{[ˈvĩ ˈpɛma]}{(n.)}{1}{}{Vinho de palma.}{}{}
\verb{vinte}{}{[vĩˈtɛ]}{(n.)}{1}{}{\textit{Fya-vinte}.}{\textbf{\textit{Desmodium adscendens}.}}{}
\verb{vinte}{}{[vĩˈtɛ]}{(n.)}{2}{}{Pega-pega.}{\textbf{\textit{Desmodium adscendens}.}}{}
\verb{vinte}{}{[vĩˈtɛ]}{(n.)}{3}{}{Vintém.}{}{}%
\verb{vintxi}{}{[ˈvĩtʃi]}{(num.)}{1}{}{Vinte.}{}{}
\verb{visu}{}{[ˈvisu]}{(n.)}{1}{}{Vício.}{}{}
\verb{vita}{}{[viˈta]}{(v.)}{1}{}{Evitar.}{}{}
\verb{vita}{}{[viˈta]}{(v.)}{2}{}{Repreender.}{}{}
\verb{vitamina}{}{[vitaˈmina]}{(n.)}{1}{}{Vitamina.}{}{}
\verb{viva}{}{[ˈviva]}{(n.)}{1}{}{Viva.}{}{}
\verb{viva}{}{[ˈviva]}{(v.)}{1}{}{Dar vivas a.}{}{}
\verb{vivê}{}{[viˈve]}{(n.)}{1}{}{Vida.}{}{}
\verb{vivê}{}{[viˈve]}{(v.)}{1}{}{Viver.}{}{}
\verb{vivêmentu}{}{[viveˈm\~etu]}{(n.)}{1}{}{Convivência.}{}{}
\verb{vivêmentu}{}{[viveˈm\~etu]}{(n.)}{1}{}{Vivência.}{}{}
\verb{vivenxa}{}{[viˈv\~ɛʃa]}{(n.)}{1}{}{Amante.}{}{}%
\verb{vivenxa}{}{[viˈv\~ɛʃa]}{(n.)}{2}{}{Amásia.}{}{}%
\verb{vivenxa}{}{[viˈv\~ɛʃa]}{(n.)}{3}{}{Concubina.}{}{}
\verb{vivu}{}{[ˈvivu]}{(adj.)}{1}{}{Vivo.}{}{}
\verb{vivu tatata}{}{[ˈvivu tataˈta]}{(expr.)}{1}{}{Vivíssimo.}{}{}
\verb{vixkondji}{}{[viʃˈk\~odʒi]}{(n.)}{1}{}{Visconde.}{}{}
\verb{vixtidu}{}{[viʃˈtidu]}{(n.)}{1}{}{Vestido.}{}{}
\verb{vixu}{}{[ˈviʃu]}{(n.)}{1}{}{Vício.}{}{}
\verb{viya}{}{[ˈvija]}{(n.)}{1}{}{Hérnia.}{}{}
\verb{viya}{}{[ˈvija]}{(n.)}{2}{}{Vagina.}{}{}
\verb{viyuva}{}{[viˈjuva]}{(n.)}{1}{}{Viúva.}{}{}
\verb{viyuvu}{}{[viˈjuvu]}{(n.)}{1}{}{Viúvo.}{}{}
\verb{viza}{}{[viˈza]}{(v.)}{1}{}{Avisar.}{}{}
\verb{vizadu}{}{[viˈzadu]}{(adj.)}{1}{}{Avisado.}{}{}
\verb{vizita}{}{[viˈzita]}{(n.)}{1}{}{Visita.}{}{}
\verb{vizita}{}{[viziˈta]}{(v.)}{1}{}{Visitar.}{}{}
\verb{vlega}{}{[vlɛˈga]}{(v.)}{1}{}{Abaixar.}{}{}
\verb{vlega}{}{[vlɛˈga]}{(v.)}{1}{}{Curvar.}{}{}
\verb{vlega}{}{[vlɛˈga]}{(v.)}{2}{}{Envergar.}{}{}
\verb{vlega}{}{[vlɛˈga]}{(v.)}{3}{}{Inclinar(-se).}{}{}%
\verb{vlega}{}{[vlɛˈga]}{(v.)}{4}{}{Vergar.}{}{}
\verb{vlegonha}{}{[vlɛˈg\~ɔɲa]}{(n.)}{1}{}{Vergonha.}{}{}
\verb{vlegonha}{}{[vlɛg\~ɔˈɲa]}{(v.)}{1}{}{Envergonhar(-se).}{}{}
\verb{vlêmê}{}{[vleˈme]}{(adj.)}{1}{}{Vermelho.}{}{}%
\verb{vlêmê}{}{[vleˈme]}{(n.)}{1}{}{Vermelho.}{\textbf{\textit{Apsilus fuscus}}.}{}
\verb{vlêmê bababa}{}{[vleˈme babaˈba]}{(expr.)}{1}{}{Vermelhíssimo.}{}{}
\verb{vlêmê myamyamya}{}{[vleˈme mjamjaˈmja]}{(expr.)}{1}{}{Vermelho garrido.}{}{}
\verb{vlentêji}{}{[vlẽˈteʒi]}{(n.)}{1}{}{Entranhas.}{}{}
\verb{vlêtê}{}{[vleˈte]}{(v.)}{1}{}{Verter.}{}{}%
\verb{vlidu}{}{[ˈvlidu]}{(n.)}{1}{}{Vidro.}{}{}
\verb{vôdô}{}{[ˈvodo]}{(n.)}{1}{}{Balbúrdia.}{}{}
\verb{vôdô}{}{[ˈvodo]}{(n.)}{1}{}{Bando.}{}{}
\verb{vôlô}{}{[voˈlo]}{(v.)}{1}{}{Repreender.}{}{}
\verb{vôlô}{}{[voˈlo]}{(v.)}{2}{}{Zangar(-se).}{}{}%
\verb{vonte}{}{[v\~ɔˈtɛ]}{(n.)}{1}{}{Interesse.}{}{}
\verb{vonte}{}{[v\~ɔˈtɛ]}{(n.)}{2}{}{Vontade.}{}{}%
\verb{vonvon}{}{[võˈvõ]}{(adv.)}{1}{}{À toa.}{}{}%
\verb{vonvon}{}{[võˈvõ]}{(id.)}{1}{}{Cf. \textbf{fla vonvon.}}{}{}
\verb{vota}{}{[ˈvɔta]}{(n.)}{1}{}{Vez.}{}{}
\verb{vota}{}{[ˈvɔta]}{(n.)}{2}{}{Volta.}{}{}
\verb{vota}{}{[vɔˈta]}{(v.)}{1}{}{Votar.}{}{}
\verb{votu}{}{[ˈvɔtu]}{(n.)}{1}{}{Voto.}{}{}
\verb{vovo}{}{[vɔˈvɔ]}{(v.)}{1}{Desintegrar.}{}{}
\verb{vovo}{}{[vɔˈvɔ]}{(v.)}{1}{Infectar.}{}{}
\verb{vozu}{}{[ˈvɔzu]}{(n.)}{1}{}{Voz.}{}{}
\verb{vu}{}{[ˈvu]}{(id.)}{1}{}{Cf. \textbf{xê vu.}}{}{}
\verb{vugu}{}{[vuˈgu]}{(v.)}{1}{}{Agitar.}{}{}%
\verb{vugu}{}{[vuˈgu]}{(v.)}{2}{}{Lutar pela vida.}{}{}
\verb{vugu}{}{[vuˈgu]}{(v.)}{3}{}{Mover.}{}{}%
\verb{vuguvugu}{}{[ˈvuguˈvugu]}{(n.)}{1}{}{Toco ou pedaço de madeira que é arremessado para tirar frutos das árvores ou utilizado como arma.}{}{}
\verb{vumba}{}{[vũˈba]}{(v.)}{1}{}{Desferir.}{}{}%
\verb{vumba}{}{[vũˈba]}{(v.)}{2}{}{Enfiar.}{}{}%
\verb{vumbada}{}{[vũˈbada]}{(n.)}{1}{}{Purgante.}{}{}
\verb{vumbada}{}{[vũˈbada]}{(n.)}{1}{}{\textit{Vumbada}.}{Mistura de folhas utilizada para o tratamento de cólicas em crianças.}{}{}
\verb{vunga}{}{[ˈvũga]}{(n.)}{1}{}{Balanço.}{}{}
\verb{vunga}{}{[ˈvũga]}{(n.)}{2}{}{Baloiço.}{}{}
\verb{vunga}{}{[vũˈga]}{(v.)}{1}{}{Ter relações sexuais.}{}{}
\verb{vungu}{}{[ˈvũgu]}{(n.)}{1}{}{Canção.}{}{}
\verb{vungu}{}{[ˈvũgu]}{(n.)}{2}{}{Cantiga.}{}{}
\verb{vungu}{}{[ˈvũgu]}{(n.)}{3}{}{Música.}{}{}
\verb{vunun}{}{[vuˈnũ]}{(n.)}{1}{}{\textit{Vunun}.}{\textbf{\textit{Ehretia scrobiculata}.}}{}
\verb{vunvu}{}{[vũˈvu]}{(n.)}{1}{}{Abelha.}{Cf. \textbf{vunvun}.}{}{}
\verb{vunvun}{}{[vũˈvũ]}{(n.)}{1}{}{Abelha.}{}{}
\verb{vunvun-wangadu}{}{[vũˈvũ w\~{\textturna}ˈgadu]}{(n.)}{1}{}{Têmpera do mosto da palmeira.}{}
\verb{vutu}{}{[ˈvutu]}{(n.)}{1}{}{Corpulência.}{}{}
\verb{vutu}{}{[ˈvutu]}{(n.)}{2}{}{Sombra.}{}{}
\verb{vutu}{}{[ˈvutu]}{(n.)}{3}{}{Vulto.}{}{}
\verb{vuva}{}{[vuˈva]}{(v.)}{1}{}{Uivar.}{}{}
\verb{vuza}{}{[vuˈza]}{(v.)}{1}{}{Celebrar.}{}{}
\verb{vuza}{}{[vuˈza]}{(v.)}{2}{}{Comemorar.}{}{}
\verb{vuza}{}{[vuˈza]}{(v.)}{3}{}{Inaugurar.}{}{}
\verb{vwa}{}{[ˈvwa]}{(v.)}{1}{}{Apressar(-se).}{}{}
\verb{vwa}{}{[ˈvwa]}{(v.)}{1}{}{Correr.}{}{}
\verb{vwa}{}{[ˈvwa]}{(v.)}{1}{}{Saltar de.}{}{}
\verb{vwa}{}{[ˈvwa]}{(v.)}{2}{}{Voar.}{}{}
\verb{vwa-sata}{}{[ˈvwa saˈta]}{(n.)}{1}{}{Atalho.}{}{}
\verb{vwa-sata}{}{[ˈvwa saˈta]}{(n.)}{1}{}{Corta-mato.}{}{}
\verb{vya}{}{[ˈvja]}{(n.)}{1}{}{Veia.}{}{}
\verb{vyaji}{}{[ˈvjaʒi]}{(n.)}{1}{}{Viagem.}{}{}
\verb{vyantêlu}{}{[vj\~{\textturna}ˈtelu]}{(n.)}{1}{}{Vinhateiro.}{}{}%
\verb{vyola}{}{[ˈvjɔla]}{(n.)}{1}{}{Guitarra.}{}{}
\verb{vyola}{}{[ˈvjɔla]}{(n.)}{2}{}{Violão.}{}{}%
\verb{vyolinu}{}{[vjɔˈlinu]}{(n.)}{1}{}{Violino.}{}{}
\verb{vyuva}{}{[ˈvjuva]}{(n.)}{1}{}{Viúva.}{Cf. \textbf{viyuva}.}{}{}
\verb{vyuvu}{}{[ˈvjuvu]}{(n.)}{1}{}{Viúvo.}{Cf. \textbf{viyuvu}.}{}{}
\end{letra}

\begin{letra}{w}
\verb{wagawaga}{}{[ˈwagaˈwaga]}{(n.)}{1}{}{Erva-boi.}{\textbf{\textit{Setaria
megaphylla}.}}{}
\verb{walawala}{}{[ˈwalaˈwala]}{(adv.)}{1}{}{Desarrumadamente.}{}{}%
\verb{walawala}{}{[ˈwalaˈwala]}{(adv.)}{2}{}{Desorganizadamente.}{}{}
\verb{walawala}{}{[ˈwalaˈwala]}{(adv.)}{3}{}{Dispersamente.}{}{}%
\verb{walawala}{}{[ˈwalaˈwala]}{(adv.)}{4}{}{Espalhadamente.}{}{}
\verb{wamba}{}{[w\~{\textturna}ˈba]}{(n.)}{1}{}{Água turva resultante do
processo de fabricação do óleo de palma.}{}{}
\verb{wan}{}{[ˈw\~{\textturna}]}{(id.)}{1}{}{Cf. \textbf{betu wan}}{}{}
\verb{wan}{}{[ˈw\~{\textturna}]}{(id.)}{2}{}{Cf. \textbf{plama bili wan}}{}{}\verb{wanga}{}{[w\~{\textturna}ˈga]}{(v.)}{1}{}{Derramar.}{}{}%
\verb{wanga}{}{[w\~{\textturna}ˈga]}{(v.)}{2}{}{Entornar.}{}{}
\verb{wanga}{}{[w\~{\textturna}ˈga]}{(v.)}{3}{}{Espalhar.}{}{}%
\verb{wangadu}{}{[w\~{\textturna}ˈgadu]}{(adj.)}{1}{}{Espalhado.}{}{}%
\verb{wangadu}{}{[w\~{\textturna}ˈgadu]}{(adj.)}{2}{}{Derramado.}{}{}%
\verb{wangila}{}{[w\~{\textturna}giˈla]}{(adv.)}{1}{}{Em grande
quantidade.}{}{}
\verb{wangila}{}{[w\~{\textturna}giˈla]}{(adv.)}{2}{}{Muito.}{}{}%
\verb{wê}{}{[ˈwe]}{(n.)}{1}{}{Broto.}{}{}%
\verb{wê}{}{[ˈwe]}{(n.)}{2}{}{Folhas.}{}{}%
\verb{wê}{}{[ˈwe]}{(n.)}{4}{}{Frente.}{}{}%
\verb{wê}{}{[ˈwe]}{(n.)}{3}{}{Olho.}{}{}%
\verb{wê}{}{[ˈwe]}{(n.)}{4}{}{Primeiro.}{}{}
\verb{wê}{}{[ˈwe]}{(n.)}{5}{}{Rebento.}{}{}
\verb{wê}{}{[ˈwe]}{(prep. n.)}{1}{}{Diante (de).}{}{}%
\verb{wê}{}{[ˈwe]}{(prep. n.)}{2}{}{Em frente a.}{}{}%
\verb{wê}{}{[ˈwe]}{(prep. n.)}{3}{}{Em frente de.}{}{}%
\verb{wê-betu}{}{[ˈwe ˈbɛtu]}{(adj.)}{1}{}{Acordado.}{}{}
\verb{wê-betu}{}{[ˈwe ˈbɛtu]}{(adj.)}{1}{}{Atento.}{}{}
\verb{wê-betu}{}{[ˈwe ˈbɛtu]}{(adj.)}{1}{}{Esperto.}{}{}
\verb{wê betu klan}{}{[ˈwe ˈbɛtu ˈkl\~{\textturna}]}{(expr.)}{1}{}{De olhos
bem abertos.}{}{}
\verb{wê-bila}{}{[ˈwe biˈla]}{(n.)}{1}{}{Vertigem.}{}{}
\verb{wê-bluku}{}{[ˈwe ˈbluku]}{(n.)}{1}{}{Mau-olhado.}{}{}
\verb{wê-d'ope}{}{[ˈwe dɔˈpɛ]}{(n.)}{1}{}{Canela.}{}{}
\verb{wê-glosu}{}{[ˈwe ˈglɔsu]}{(adj.)}{1}{}{Ganancioso.}{}{}
\verb{wê-glosu}{}{[ˈwe ˈglɔsu]}{(adj.)}{2}{}{Invejoso.}{}{}
\verb{wê-glosu}{}{[ˈwe ˈglɔsu]}{(adj.)}{3}{}{Orgulhoso.}{}{}
\verb{wê-glosu}{}{[ˈwe ˈglɔsu]}{(n.)}{1}{}{Ganância.}{}{}
\verb{wê-glosu}{}{[ˈwe ˈglɔsu]}{(n.)}{2}{}{Inveja.}{}{}
\verb{wê-glosu}{}{[ˈwe ˈglɔsu]}{(n.)}{3}{}{Olho-grosso.}{\textbf{\textit{Speirops lugubris}}.}{}
\verb{wê-glosu}{}{[ˈwe ˈglɔsu]}{(n.)}{4}{}{Orgulho.}{}{}
\verb{wê-gôdô}{}{[ˈwe ˈgodo]}{(adj.)}{1}{}{Guloso.}{}{}
\verb{wê-gôdô}{}{[ˈwe ˈgodo]}{(n.)}{1}{}{Gula.}{}{}
\verb{wê klan}{}{[ˈwe ˈkl\~{\textturna}]}{(expr.)}{1}{}{De olhos bem abertos.}{}{}
\verb{wê-kobo}{}{[ˈwe ˈkɔbɔ]}{(n.)}{1}{}{Olheiras.}{}{}%
\verb{wê-kobo}{}{[ˈwe ˈkɔbɔ]}{(n.)}{2}{}{Olhos fundos.}{}{}
\verb{wê-kobo}{}{[ˈwe kɔˈbɔ]}{(n.)}{1}{}{Covinha.}{}{}%
\verb{wê-kobo}{}{[ˈwe kɔˈbɔ]}{(n.)}{2}{}{Jogo de crianças.}{}{}
\verb{wê-kota}{}{[ˈwe kɔˈta]}{(n.)}{1}{}{Má cara.}{}{}%
\verb{wê-kubli}{}{[ˈwe kuˈbli]}{(n.)}{1}{}{Ao contrário.}{}{}
\verb{wê-kubli}{}{[ˈwe kuˈbli]}{(n.)}{2}{}{De borco.}{}{}
\verb{wê-kubli}{}{[ˈwe kuˈbli]}{(n.)}{3}{}{De cabeça para baixo.}{}{}
\verb{wê-kume}{}{[ˈwe kuˈmɛ]}{(n.)}{1}{}{\textit{Wê-kume}.}{Primeira concha que se retira da panela oferecida aos \textbf{nen-ke-mu}.}{}
\verb{wele}{}{[ˈwɛlɛ]}{(adv.)}{1}{}{Agora.}{}{}
\verb{wê-leve}{}{[ˈwe ˈlɛvɛ]}{(n.)}{1}{}{Vidente.}{}{}
\verb{welewele}{}{[ˈwɛlɛˈwɛlɛ]}{(adv.)}{1}{}{Agora.}{}{}
\verb{welewele}{}{[ˈwɛlɛˈwɛlɛ]}{(adv.)}{2}{}{Há pouco.}{}{}
\verb{welewele}{}{[ˈwɛlɛˈwɛlɛ]}{(adv.)}{3}{}{Recentemente.}{}{}%
\verb{wê-lizu}{}{[ˈwe ˈlizu]}{(adj.)}{1}{}{Corajoso.}{}{}
\verb{wê-lizu}{}{[ˈwe ˈlizu]}{(n.)}{1}{}{Coragem.}{}{}
\verb{wê-longô}{}{[ˈwe ˈl\~ogo]}{(adj.)}{1}{}{Invejoso.}{}{}
\verb{wê-longô}{}{[ˈwe ˈl\~ogo]}{(n.)}{1}{}{Inveja.}{}{}
\verb{wê-longô}{}{[ˈwe ˈl\~ogo]}{(n.)}{1}{}{Saudades.}{}{}
%\verb{wê-longô}{}{[ˈwe ˈl\~ogo]}{(n.)}{1}{}{Olhar saudoso.}{}{}
\verb{wê-lujidu}{}{[ˈwe luˈʒidu]}{(adj.)}{1}{}{Culto.}{}{}
\verb{wê-lujidu}{}{[ˈwe luˈʒidu]}{(adj.)}{2}{}{Instruído.}{}{}
\verb{wê-lujidu}{}{[ˈwe luˈʒidu]}{(n.)}{1}{}{Culto.}{}{}
\verb{wê-lujidu}{}{[ˈwe luˈʒidu]}{(n.)}{2}{}{Instruído.}{}{}
\verb{wembe}{}{[ˈw\~ɛbɛ]}{(adj.)}{1}{}{Grande.}{}{}
\verb{wembe}{}{[ˈw\~ɛbɛ]}{(adj.)}{2}{}{Insistente.}{}{}
\verb{wembe}{}{[ˈw\~ɛbɛ]}{(adj.)}{3}{}{Perseverante.}{}{}
\verb{wembe}{}{[ˈw\~ɛbɛ]}{(n.)}{1}{}{Coragem.}{}{}
\verb{wembe}{}{[ˈw\~ɛbɛ]}{(n.)}{2}{}{Insistência.}{}{}
\verb{wembe}{}{[ˈw\~ɛbɛ]}{(n.)}{3}{}{Perseverância.}{}{}
\verb{wembe}{}{[ˈw\~ɛbɛ]}{(n.)}{4}{}{Tambor.}{}{}
\verb{wê ngenengene}{}{[ˈwe ŋgɛˈnɛ ŋgɛˈnɛ]}{(expr.)}{1}{}{Olhos reluzentes.}{}{}
\verb{wê-pinta}{}{[ˈwe ˈp\~ita]}{(n.)}{1}{}{Cego.}{}{}
\verb{wê pitxipitxi}{}{[ˈwe piˈtʃipiˈtʃi]}{(expr.)}{1}{}{Olhos semi-cerrados.}{}{}
\verb{wê-poto}{}{[ˈwe ˈpɔtɔ]}{(n.)}{1}{}{Vista de frente.}{}{}
\verb{wê-son}{}{[ˈwe ˈs\~ɔ]}{(adj.)}{1}{}{Castanho.}{}{}
\verb{wê-sonson}{}{[ˈwe s\~ɔˈs\~ɔ]}{(adj.)}{1}{}{Dissimulado.}{}{}
\verb{wê-sonson}{}{[ˈwe s\~ɔˈs\~ɔ]}{(adj.)}{2}{}{Sorrateiro.}{}{}
\verb{wêtavu}{}{[weˈtavu]}{(adj.)}{1}{}{Oitavo.}{}{}
\verb{wêtenta}{}{[weˈt\~eta]}{(num.)}{1}{}{Oitenta.}{}{}
\verb{wê-txofodu}{}{[ˈwe tʃɔˈfɔdu]}{(n.)}{1}{}{Caolho.}{}{}
\verb{wê-txofodu}{}{[ˈwe tʃɔˈfɔdu]}{(n.)}{2}{}{Zarolho.}{}{}
\verb{wê-vilo}{}{[ˈwe viˈlɔ]}{(adj.)}{1}{}{Estrábico.}{}{}
\verb{wê-vilo}{}{[ˈwe viˈlɔ]}{(adj.)}{1}{}{Vesgo.}{}{}
\verb{wê-vilo}{}{[ˈwe viˈlɔ]}{(adj.)}{1}{}{Pessoa estrábica.}{}{}
\verb{wê-xa}{}{[ˈwe ˈʃa]}{(adj.)}{1}{}{Ganancioso.}{}{}
\verb{wê-xa}{}{[ˈwe ˈʃa]}{(adj.)}{2}{}{Invejoso.}{}{}
\verb{wê-xa}{}{[ˈwe ˈʃa]}{(adj.)}{2}{}{Orgulhoso.}{}{}
\verb{wê-xa}{}{[ˈwe ˈʃa]}{(n.)}{1}{}{Ganância.}{}{}
\verb{wê-xa}{}{[ˈwe ˈʃa]}{(n.)}{2}{}{Inveja.}{}{}
\verb{wê-xa}{}{[ˈwe ˈʃa]}{(n.)}{3}{}{Orgulho.}{}{}
\verb{wêyê}{}{[weˈje]}{(v.)}{1}{}{Remediar.}{}{}
\verb{winiwini}{}{[ˈwiniˈwini]}{(id.)}{1}{}{Cf. \textbf{kebla nwininwini}.}{}{}
\verb{winiwini}{}{[ˈwiniˈwini]}{(id.)}{1}{}{Cf. \textbf{kota nwininwini}.}{}{}
\verb{winiwini}{}{[wiˈniwiˈni]}{(n.)}{1}{}{Estilhaço.}{}{}
\verb{winiwini}{}{[wiˈniwiˈni]}{(n.)}{2}{}{Fragmento.}{}{}
\verb{wixiwaxa}{}{[ˈwiʃiˈwaʃa]}{(n.)}{1}{}{Desordem.}{}{}
\verb{wo}{}{[ˈwɔ]}{(adv.)}{1}{}{Agora.}{}{}
\verb{wô}{}{[ˈwo]}{(v.)}{1}{}{Estar maduro.}{}{}%
\verb{wô}{}{[ˈwo]}{(v.)}{2}{}{Evoluir.}{}{}%
\verb{wô}{}{[ˈwo]}{(v.)}{3}{}{Progredir.}{}{}
\verb{wôdu}{}{[ˈwodu]}{(adj.)}{1}{}{Crescido.}{}{}%
\verb{wôdu}{}{[ˈwodu]}{(adj.)}{2}{}{Idoso.}{}{}%
\verb{wôdu}{}{[ˈwodu]}{(adj.)}{3}{}{Maduro.}{}{}
\verb{wôkô}{}{[ˈwoko]}{(n.)}{1}{}{Buraco.}{}{}{}
\verb{wôkô}{}{[ˈwoko]}{(n.)}{1}{}{Orifício.}{Cf. \textbf{ôkô}.}{}{}
\verb{wôlôwôlô}{}{[woˈlowoˈlo]}{(adj.)}{1}{}{Desconcentrado.}{}{}
\verb{wôlôwôlô}{}{[woˈlowoˈlo]}{(id.)}{1}{}{Cf. \textbf{kabêsa wôlôwôlô.}}{}{}
\verb{wôtô}{}{[ˈwoto]}{(num.)}{1}{}{Oito.}{}{}
\verb{wôtô-dexi}{}{[ˈwotoˈdɛʃi]}{(num.)}{1}{}{Oitenta.}{}{}
\verb{wôtôsentu}{}{[wotoˈs\~etu]}{(num.)}{1}{}{Oitocentos.}{}{}
\end{letra}

\begin{letra}{x}

\verb{xa}{}{[ˈʃa]}{(adj.)}{1}{}{Cheio.}{}{}
\verb{xa}{}{[ˈʃa]}{(n.)}{1}{}{Chá.}{}{}
\verb{xa}{}{[ˈʃa]}{(v.)}{1}{}{Encher.}{}{}
\verb{xa}{}{[ˈʃa]}{(v.)}{2}{}{Inflar.}{}{}%
\verb{xa}{}{[ˈʃa]}{(v.)}{3}{}{Insuflar.}{}{}
\verb{xada}{}{[ˈxada]}{(n.)}{1}{}{Acampamento montado nas praias pelos pescadores para a pesca de peixe-voador na gravana.}{}{}
\verb{xadu}{}{[ˈʃadu]}{(adj.)}{1}{}{Cheio.}{}{}
\verb{xa dududu}{}{[ˈʃa duduˈdu]}{(expr.)}{1}{}{Repletíssimo.}{}{}
\verb{xalanga}{}{[ʃaˈl\~{\textturna}ga]}{(n.)}{1}{}{Charanga.}{}{}
\verb{xale}{}{[ʃaˈlɛ]}{(n.)}{1}{}{Casa pequena.}{}{}
\verb{xale}{}{[ʃaˈlɛ]}{(n.)}{2}{}{Chalé.}{}{}%
\verb{xalela}{}{[ʃaˈlɛla]}{(n.)}{1}{}{Belgata.}{\textbf{\textit{Cymbopogon citratus}.}}{}
\verb{xalela}{}{[ʃaˈlɛla]}{(n.)}{2}{}{Chá do Príncipe.}{\textbf{\textit{Cymbopogon citratus}.}}{}
\verb{xalela}{}{[ʃaˈlɛla]}{(n.)}{3}{}{Erva-limão.}{\textbf{\textit{Cymbopogon citratus}.}}{}
\verb{xa libita}{}{[ˈʃa  libiˈta]}{(expr.)}{1}{}{Cheiíssimo.}{}{}
\verb{xa lôlôlô}{}{[ˈʃa loloˈlo]}{(expr.)}{1}{}{Cheiíssimo.}{}{}
\verb{xa lôlôlô}{}{[ˈʃa loloˈlo]}{(expr.)}{1}{}{Inchadíssimo.}{}{}
\verb{xa-plaga}{}{[ˈʃa ˈplaga]}{(adj.)}{1}{}{Azarado.}{}{}
\verb{xa pu}{}{[ˈʃa ˈpu]}{(expr.)}{1}{}{Muito cheio.}{}{}
\verb{xapuxapu}{}{[ˈʃapuˈʃapu]}{(n.)}{1}{}{Graviola.}{}{}%
\verb{xapuxapu}{}{[ˈʃapuˈʃapu]}{(n.)}{2}{}{Gravioleira.}{\textbf{\textit{Anona Muricata}}.}{}
\verb{xatu}{}{[ˈʃatu]}{(n.)}{1}{}{Chato.}{\textbf{\textit{Pthirus pubis}}.}{}\verb{xavina}{}{[ˈʃavina]}{(n.)}{1}{}{Chávena.}{}{}
\verb{xdluga}{}{[ʃdluˈga]}{(v.)}{1}{}{Julgar.}{}{}
\verb{xê}{}{[ˈʃe]}{(interj.)}{1}{}{\textit{Xê}.}{Interjeição de espanto ou dúvida.}{}{}
\verb{xê}{}{[ˈʃe]}{(v.)}{1}{}{Sair.}{}{}
\verb{xêdu}{}{[ˈʃedu]}{(adj.)}{1}{}{Saído.}{}{}
\verb{xefi}{}{[ˈʃɛfi]}{(n.)}{1}{}{Chefe.}{}{}
\verb{xeki}{}{[ˈʃɛki]}{(n.)}{1}{}{Cheque.}{}{}
\verb{xê kubu}{}{[ˈʃe ˈkubu]}{(expr.)}{1}{}{Arremeter.}{}{}
\verb{xê kubu}{}{[ˈʃe ˈkubu]}{(expr.)}{2}{}{Fazer investidas.}{Cf. \textbf{da kubu}.}{}
\verb{xele}{}{[ˈʃɛlɛ]}{(n.)}{1}{}{Dinheiro.}{}{}
\verb{xeli}{}{[ˈʃɛli]}{(n.)}{1}{}{Xaile.}{}{}
\verb{xeli}{}{[ˈʃɛli]}{(n.)}{2}{}{Xale.}{}{}
\verb{xê liba}{}{[ˈʃe ˈliba]}{(expr.)}{1}{}{Deparar com.}{}
\verb{xê lwa}{}{[ˈʃe ˈlwa]}{(expr.)}{1}{}{Sair.}{}
\verb{xembe}{}{[ʃ\~ɛˈbɛ]}{(n.)}{1}{}{\textit{Kolombeta}.}{Cf. \textbf{kolombeta}.}{}{}
\verb{xempli}{}{[ˈʃ\~ɛpli]}{(n.)}{1}{}{Caldeirão.}{}{}
\verb{xê pla}{}{[ˈʃe ˈpla]}{(expr.)}{1}{}{Sair intempestivamente.}{}
\verb{xê sangi}{}{[ˈʃe ˈs\~{\textturna}gi]}{(expr.)}{1}{}{Sangrar.}{}{}
\verb{xê vu}{}{[ˈʃe ˈvu]}{(expr.)}{1}{}{Aparecer intempestivamente.}{}{}
\verb{xê vu}{}{[ˈʃe ˈvu]}{(expr.)}{1}{}{Sair repentinamente.}{}{}
\verb{xi}{}{[ˈʃi]}{(conj.)}{1}{}{Caso.}{}{}
\verb{xi}{}{[ˈʃi]}{(conj.)}{2}{}{Se.}{}{}{}
\verb{xi}{}{[ˈʃi]}{(dem.)}{3}{}{Essa.}{\textbf{Kume xi ku ê kume.} \emph{Essa comida que ele comeu.}}{}
\verb{xi}{}{[ˈʃi]}{(dem.)}{4}{}{Essas.}{}{}{}
\verb{xi}{}{[ˈʃi]}{(dem.)}{1}{}{Esse.}{}{}{}
\verb{xi}{}{[ˈʃi]}{(dem.)}{2}{}{Esses.}{}{}{}
\verb{xi}{}{[ˈʃi]}{(n.)}{1}{}{Disenteria.}{}{}{}
\verb{xibata}{}{[ʃibaˈta]}{(v.)}{1}{}{Chicotear.}{}{}
\verb{xidadon}{}{[ʃidaˈd\~ɔ]}{(n.)}{1}{}{Cidadão.}{}{}{}
\verb{xidadon}{}{[ʃidaˈd\~ɔ]}{(n.)}{1}{}{Morador.}{Cf. \textbf{môladô}.}{}{}
\verb{xiga}{}{[ʃiˈga]}{(v.)}{1}{}{Bastar.}{}{}
\verb{xiga}{}{[ʃiˈga]}{(v.)}{2}{}{Chegar.}{}{}%
\verb{xigalu}{}{[ʃiˈgalu]}{(n.)}{1}{}{Cigarro.}{}{}
\verb{xigunu}{}{[ˈʃigunu]}{(n.)}{1}{}{Signo.}{}{}
\verb{xika}{}{[ʃiˈka]}{(v.)}{1}{}{Instigar.}{}{}
\verb{xikêma}{}{[ʃiˈkema]}{(n.)}{1}{}{Esquema.}{}{}
\verb{xikila}{}{[ʃikiˈla]}{(n.)}{1}{}{Cooperativa em que, a cada mês, algum membro dá uma parte dos seus vencimentos para ser guardado num fundo que poderá ser utilizado futuramente.}{}{}{}
\verb{xikixiki}{}{[ˈʃikiˈʃiki]}{(n.)}{1}{}{Pega-pega.}{Brincadeira de criança.}{}
\verb{xikla}{}{[ˈʃikla]}{(n.)}{1}{}{Chávena.}{}{}
\verb{xikla}{}{[ˈʃikla]}{(n.)}{1}{}{Xícara.}{}{}
\verb{xikotxi}{}{[ʃiˈkɔtʃi]}{(n.)}{1}{}{Chicote.}{}{}
\verb{xili}{}{[ˈʃili]}{(n.)}{1}{}{Chile branco.}{\textbf{\textit{Drymaria cordata}.}}{}
\verb{xilinga}{}{[ʃiˈl\~iga]}{(n.)}{1}{}{Seringa.}{}{}
\verb{xilola}{}{[ʃiˈlɔla]}{(n.)}{1}{}{Ceroulas.}{}{}
\verb{xima}{}{[ˈʃima]}{(n.)}{1}{}{Andar superior de uma casa tradicional.}{}{}
\verb{xima}{}{[ˈʃima]}{(n.)}{1}{}{Parte de cima.}{}{}
\verb{ximba}{}{[ʃ\~iˈba]}{(n.)}{1}{}{Polícias de segurança pública no tempo colonial.}{}{}
\verb{ximbôtô}{}{[ʃ\~iˈboto]}{(n.)}{1}{}{Desajeitado.}{}{}{}
\verb{ximbôtô}{}{[ʃ\~iˈboto]}{(n.)}{1}{}{Idiota.}{}{}{}
\verb{ximbôtô}{}{[ʃ\~iˈboto]}{(n.)}{1}{}{Palerma.}{}{}{}
\verb{ximbôtô}{}{[ʃ\~iˈboto]}{(n.)}{1}{}{Pateta.}{}{}{}
\verb{ximbôtô}{}{[ʃ\~iˈboto]}{(n.)}{1}{}{Tolo.}{Cf. \textbf{tximbôtô}.}{}{}
\verb{ximentxi}{}{[ʃiˈm\~ɛtʃi]}{(adj.)}{1}{}{Muito.}{}{}
\verb{ximentxi}{}{[ʃiˈm\~ɛtʃi]}{(adj.)}{2}{}{Semelhante.}{}
\verb{ximentxi}{}{[ʃiˈm\~ɛtʃi]}{(adj.)}{3}{}{Tal.}{}{}
\verb{ximidô}{}{[ʃimiˈdo]}{(n.)}{1}{}{Consumidor.}{}{}
\verb{ximidô}{}{[ʃimiˈdo]}{(n.)}{2}{}{Ralo.}{}{}%
\verb{ximidô}{}{[ʃimiˈdo]}{(n.)}{3}{}{Sumidouro.}{}{}%
\verb{ximine}{}{[ʃimiˈnɛ]}{(n.)}{1}{}{Chaminé.}{}{}
\verb{ximine}{}{[ʃimiˈnɛ]}{(n.)}{2}{}{Latrina ao ar livre.}{}{}
\verb{ximinteli}{}{[ʃimĩˈtɛli]}{(n.)}{1}{}{Cemitério.}{}{}
\verb{ximon-koya}{}{[ʃiˈm\~ɔ kɔˈja]}{(n.)}{3}{}{Simão-Correia.}{\textit{\textbf{Lagenaria breviflora}.}}{}
\verb{ximpli}{}{[ˈʃĩpli]}{(adj.)}{1}{}{Insosso.}{}{}
\verb{ximpli}{}{[ˈʃĩpli]}{(adj.)}{2}{}{Simples.}{}{}
\verb{ximya}{}{[ʃiˈmja]}{(v.)}{1}{}{Plantar.}{}{}
\verb{ximya}{}{[ʃiˈmja]}{(v.)}{2}{}{Semear.}{}{}
\verb{ximyadô}{}{[ʃimjaˈdo]}{(n.)}{1}{}{Agricultor.}{}{}
\verb{ximyadu}{}{[ʃiˈmjadu]}{(adj.)}{1}{}{Semeado.}{}{}
\verb{xina}{}{[ʃiˈna]}{(v.)}{1}{}{Aperfilhar.}{}{}
\verb{xina}{}{[ʃiˈna]}{(v.)}{2}{}{Aprender.}{}{}
\verb{xina}{}{[ʃiˈna]}{(v.)}{3}{}{Assinar.}{}{}
\verb{xina}{}{[ʃiˈna]}{(v.)}{4}{}{Ensinar.}{}{}
\verb{xinadu}{}{[ʃiˈnadu]}{(adj.)}{1}{}{Aprendido.}{}{}
\verb{xinadu}{}{[ʃiˈnadu]}{(adj.)}{2}{}{Assinado.}{}{}
\verb{xinadu}{}{[ʃiˈnadu]}{(adj.)}{3}{}{Ensinado.}{}{}%
\verb{xinali}{}{[ʃiˈnali]}{(n.)}{1}{}{Sinal.}{}{}
\verb{xindja}{}{[ˈʃĩdʒa]}{(n.)}{1}{}{Cinzas.}{}{}%
\verb{xinelu}{}{[ʃiˈnɛlu]}{(n.)}{1}{}{Chinela.}{}{}
\verb{xinelu}{}{[ʃiˈnɛlu]}{(n.)}{2}{}{Chinelo.}{}{}
\verb{xinga}{}{[ʃ\~iˈga]}{(n.)}{1}{}{Músculo.}{}{}
\verb{xinga}{}{[ʃ\~iˈga]}{(n.)}{1}{}{Tendão.}{}{}
\verb{xingila}{}{[ʃĩgiˈla]}{(v.)}{1}{}{Entrar em transe.}{}{}%}
\verb{xinhô}{}{[ʃ\~iˈɲo]}{(n.)}{1}{}{Festa religiosa de freguesia.}{}{}
\verb{xinhô}{}{[ʃ\~iˈɲo]}{(n.)}{1}{}{Ostensório.}{}{}
\verb{xinhô}{}{[ʃ\~iˈɲo]}{(n.)}{1}{}{Santíssimo sacramento.}{}{}
\verb{xinjeli}{}{[ʃĩˈʒɛli]}{(n.)}{2}{}{Parcelas.}{}{}%
\verb{xinjeli}{}{[ʃĩˈʒɛli]}{(n.)}{1}{}{Pedaço.}{}{}
\verb{xinku}{}{[ˈʃĩku]}{(num.)}{1}{}{Cinco.}{}{}
\verb{xinku-dexi}{}{[ˈʃĩku ˈdɛʃi]}{(num.)}{1}{}{Cinquenta.}{}{}
\verb{xinkwenta}{}{[ʃĩˈkw\~ɛta]}{(num.)}{1}{}{Cinquenta.}{}{}
\verb{xinta}{}{[ˈʃĩta]}{(n.)}{1}{}{Cinta.}{}{}
\verb{xinta-kadela}{}{[ˈʃĩta kaˈdɛla]}{(n.)}{1}{}{Ancas.}{}{}
\verb{xintu}{}{[ˈʃĩtu]}{(n.)}{1}{}{Cinto.}{}{}
\verb{xintxi}{}{[ˈʃĩtʃi]}{(n.)}{1}{}{Acinte.}{}{}
\verb{xintxi}{}{[ˈʃĩtʃi]}{(n.)}{2}{}{Capricho.}{}{}
\verb{xintxi}{}{[ˈʃĩtʃi]}{(n.)}{3}{}{Casmurrice.}{}{}
\verb{xintxi}{}{[ˈʃĩtʃi]}{(n.)}{4}{}{Teimosia.}{}{}
\verb{xintxi}{}{[ʃĩˈtʃi]}{(v.)}{1}{}{Aperceber-se de.}{}{}
\verb{xintxi}{}{[ʃĩˈtʃi]}{(v.)}{2}{}{Sentir.}{}{}%
\verb{xintxidu}{}{[ʃĩˈtʃidu]}{(n.)}{1}{}{Intenção.}{}{}
\verb{xintxidu}{}{[ʃĩˈtʃidu]}{(n.)}{2}{}{Percepção.}{}{}
\verb{xintxidu}{}{[ʃĩˈtʃidu]}{(n.)}{3}{}{Sentido.}{}{}
\verb{xintxidu}{}{[ʃĩˈtʃidu]}{(n.)}{4}{}{Significado.}{}{}
\verb{xintximentu}{}{[ʃĩtʃiˈm\~etu]}{(n.)}{1}{}{Sentimento.}{}{}
\verb{xinu}{}{[ˈʃinu]}{(n.)}{1}{}{Ferrinhos.}{}{}
\verb{xinu}{}{[ˈʃinu]}{(n.)}{1}{}{Sineta.}{}{}
\verb{xinu}{}{[ˈʃinu]}{(n.)}{1}{}{Sino.}{}{}
\verb{xipitali}{}{[ʃipiˈtali]}{(n.)}{1}{}{Hospital.}{}{}
\verb{xitla}{}{[ˈʃitla]}{(n.)}{1}{}{Tecido de chita.}{}{}
\verb{xitu}{}{[ˈʃitu]}{(n.)}{1}{}{Local.}{}{}
\verb{xitu}{}{[ˈʃitu]}{(n.)}{2}{}{Lugar.}{}{}%
\verb{xitu}{}{[ˈʃitu]}{(n.)}{3}{}{Sítio.}{}{}%
\verb{xka}{}{[ˈʃka]}{(part.)}{1}{}{Partícula preverbal que exprime o progressivo aspectual.}{\textbf{N xka be mu ke}. \textit{Estou a ir para casa}.}{Cf. \textbf{saka}.}{}{}
%\verb{xka}{}{[ˈʃka]}{(part.)}{2}{}{Partícula que exprime o aspecto progressivo.}{Cf. \textbf{saka}.}{}{}{}
\verb{xkada}{}{[ˈʃkada]}{(n.)}{1}{}{Escada.}{}{}
\verb{xkalhu}{}{[ˈʃkaʎu]}{(n.)}{1}{}{\textit{Xkalhu}.}{Instrumento musical, feito de casca de coco e sementes. Usa-se um pedaço de pau como pega.}{}{}
\verb{xkapa}{}{[ʃkaˈpa]}{(v.)}{1}{}{Escapar.}{}{}
\verb{xkedu}{}{[ˈʃkɛdu]}{(adj.)}{1}{}{Esquerdo.}{}{}
\verb{xkentamentu}{}{[ʃk\~etaˈm\~etu]}{(n.)}{1}{}{Gonorréia.}{}{}
\verb{xkindi}{}{[ʃkĩˈdi]}{(n.)}{1}{}{Mania.}{}{}
\verb{xklêvê}{}{[ʃkleˈve]}{(v.)}{1}{}{Escrever.}{}{}
\verb{xklêvê}{}{[ʃkleˈve]}{(v.)}{2}{}{Descrever.}{}{}
\verb{xklêvê}{}{[ʃkleˈve]}{(v.)}{3}{}{Desenhar.}{}{}
\verb{xklêvêdô}{}{[ʃkleveˈdo]}{(n.)}{1}{}{Escritor.}{}{}
\verb{xklêvidu}{}{[ʃkleˈvidu]}{(adj.)}{1}{}{Escrito.}{}{}
\verb{xklitoli}{}{[ʃkliˈtɔli]}{(n.)}{1}{}{Escritório.}{}{}
\verb{xklitu}{}{[ˈʃklitu]}{(adj.)}{1}{}{Escrito.}{}{}
\verb{xklivon}{}{[ʃkliˈvõ]}{(n.)}{1}{}{Escrivão.}{}{}
\verb{xkola}{}{[ˈʃkɔla]}{(n.)}{1}{}{Escola.}{}{}
\verb{xkôva}{}{[ˈʃkova]}{(n.)}{1}{}{Escova.}{}{}
\verb{xkudu}{}{[ˈʃkudu]}{(n.)}{1}{}{Escudo.}{}{}
\verb{xlava}{}{[ʃlaˈva]}{(v.)}{1}{}{Salvar.}{Cf. \textbf{xtlava}.}{}{}
\verb{xlavadô}{}{[ʃlavaˈdo]}{(n.)}{1}{}{Salvador.}{Cf. \textbf{xtlavadô}.}{}{}
\verb{xlavason}{}{[ʃlavaˈsõ]}{(n.)}{1}{}{Salvação.}{Cf. \textbf{xtlavason}.}{}{}
\verb{xlivisu}{}{[ʃliˈvisu]}{(n.)}{1}{}{Serviço.}{}{}{}
\verb{xlivisu}{}{[ʃliˈvisu]}{(n.)}{1}{}{Trabalho.}{Cf. \textbf{xtluvisu}.}{}{}%
\verb{xofelu}{}{[ʃɔˈfɛlu]}{(n.)}{1}{}{Motorista.}{}{}
\verb{xoki}{}{[ˈʃɔki]}{(n.)}{1}{}{Acidente.}{}{}
\verb{xola}{}{[ˈʃɔla]}{(n.)}{1}{}{Desfile de canoas em festas religiosas de pescadores.}{}{}
\verb{xola}{}{[ˈʃɔla]}{(n.)}{1}{}{Festa de Nossa Senhora.}{}{}
\verb{xola}{}{[ˈʃɔla]}{(n.)}{2}{}{Senhora.}{}{}
\verb{xota}{}{[ˈʃɔta]}{(n.)}{1}{}{Sótão.}{}{}
\verb{xoxi}{}{[ˈʃɔʃi]}{(n.)}{1}{}{Parceiro.}{}{}
\verb{xoxi}{}{[ˈʃɔʃi]}{(n.)}{2}{}{Sócio.}{}{}
\verb{xoxo}{}{[ʃɔˈʃɔ]}{(n.)}{1}{}{Pica-peixe-de-peito-azul.}{\textbf{\textit{Halcyon malimbica dryas}.}}{}
\verb{xpanze}{}{[ʃp\~{\textturna}ˈzɛ]}{(n.)}{1}{}{Chimpanzé.}{}{}
\verb{xplika}{}{[ʃpliˈka]}{(v.)}{1}{}{Explicar.}{}{}
\verb{xplikadu}{}{[ʃpliˈkadu]}{(adj.)}{1}{}{Explicado.}{}{}
\verb{xplikason}{}{[ʃplikaˈs\~ɔ]}{(n.)}{1}{}{Explicação.}{}{}
\verb{xtadu}{}{[ˈʃtadu]}{(n.)}{1}{}{Estado.}{}{}%
\verb{xtadu}{}{[ˈʃtadu]}{(n.)}{2}{}{Governo.}{}{}
\verb{xtaka}{}{[ˈʃtaka]}{(n.)}{1}{}{Estaca.}{}{}
\verb{xtaka}{}{[ʃtaˈka]}{(v.)}{1}{}{Permanecer.}{}{}
\verb{xtaki}{}{[ˈʃtaki]}{(n.)}{1}{}{Epilepsia.}{}{}
\verb{xtanhu}{}{[ˈʃt\~{\textturna}ɲu]}{(n.)}{1}{}{Estanho.}{}{}
\verb{xtanka}{}{[ʃt\~{\textturna}ˈka]}{(v.)}{1}{}{Estancar.}{}{}
\verb{xtatwa}{}{[ˈʃtatwa]}{(n.)}{1}{}{Estátua.}{}{}{}
\verb{xtê}{}{[ˈʃte]}{(n.)}{1}{}{Amparo.}{}{}
\verb{xtê}{}{[ˈʃte]}{(n.)}{2}{}{Batente.}{}{}%
\verb{xtê}{}{[ˈʃte]}{(n.)}{3}{}{Esteio.}{}{}%
\verb{xtê}{}{[ˈʃte]}{(n.)}{4}{}{Prumo.}{}{}%
\verb{xtê}{}{[ˈʃte]}{(n.)}{5}{}{Segurança.}{}{}
\verb{xtenson}{}{[ʃt\~ɛˈs\~ɔ]}{(n.)}{1}{}{Extensão.}{}{}
\verb{xtenta}{}{[ˈʃt\~ɛta]}{(num.)}{1}{}{Setenta.}{Cf. \textbf{setenta}.}{}
\verb{xtêpô}{}{[ʃteˈpo]}{(n.)}{1}{}{Achaque.}{}{}
\verb{xtêpô}{}{[ʃteˈpo]}{(n.)}{2}{}{Fraqueza.}{}{}
\verb{xtêpô}{}{[ʃteˈpo]}{(n.)}{2}{}{Doença.}{}{}
\verb{xtêpô}{}{[ʃteˈpo]}{(n.)}{3}{}{Incômodo.}{}{}%
\verb{xtêpô}{}{[ʃteˈpo]}{(n.)}{4}{}{Maleita.}{}{}%
\verb{xtetika}{}{[ˈʃtɛtika]}{(n.)}{1}{}{Estética.}{}{}
\verb{xtika}{}{[ʃtiˈka]}{(v.)}{1}{}{Estender.}{}{}
\verb{xtika}{}{[ʃtiˈka]}{(v.)}{2}{}{Esticar.}{}{}%
\verb{xtika}{}{[ʃtiˈka]}{(v.)}{3}{}{Morrer.}{}{}
\verb{xtima}{}{[ʃtiˈma]}{(v.)}{1}{}{Estimar.}{}{}
\verb{xtivadô}{}{[ʃtivaˈdo]}{(n.)}{1}{}{Estivador.}{Cf. \textbf{xtivadô-bodo}.}{}{}
\verb{xtivadô-bodo}{}{[ʃtivaˈdo ˈbɔdɔ]}{(n.)}{1}{}{Estivador.}{}{}
\verb{xtlada}{}{[ˈʃtlada]}{(n.)}{1}{}{Caminho.}{}{}
\verb{xtlada}{}{[ˈʃtlada]}{(n.)}{2}{}{Estrada.}{}{}
\verb{xtlada-klusu}{}{[ˈʃtlada ˈklusu]}{(n.)}{1}{}{Cruzamento.}{}{}
\verb{xtlada-klusu}{}{[ˈʃtlada ˈklusu]}{(n.)}{2}{}{Encruzilhada.}{}{}
\verb{xtlada-lasu}{}{[ˈʃtlada ˈlasu]}{(n.)}{2}{}{Pequenos atalhos.}{}{}
\verb{xtlafa}{}{[ʃtlaˈfa]}{(v.)}{1}{}{Cortar.}{}{}
\verb{xtlafa}{}{[ʃtlaˈfa]}{(v.)}{1}{}{Lavrar.}{}{}
\verb{xtlafasa}{}{[ʃtlafaˈsa]}{(v.)}{1}{}{Sangrar.}{}{}
\verb{xtlafason}{}{[ʃtlafaˈsõ]}{(n.)}{1}{}{Escarificação.}{}{}
\verb{xtlafon}{}{[ʃtlaˈfõ]}{(n.)}{1}{}{Açafrão.}{Cf. \textbf{saflon}.}{}{}
\verb{xtlaga}{}{[ʃtlaˈga]}{(v.)}{1}{}{Salgar.}{}{}
\verb{xtlagadu}{}{[ʃtlaˈgadu]}{(adj.)}{1}{}{Salgado.}{}{}
\verb{xtlagu}{}{[ˈʃtlagu]}{(n.)}{1}{}{Estrago.}{}{}
\verb{xtlaka}{}{[ʃtlaˈka]}{(n.)}{1}{}{Formiga preta e grande.}{}{}
\verb{xtlakamentu}{}{[ʃtlakaˈm\~ɛtu]}{(n.)}{1}{}{Sacramento.}{Cf. \textbf{saklamentu}.}{}
\verb{xtlala}{}{[ʃtlaˈla]}{(v.)}{1}{}{Estalar.}{}{}
\verb{xtlala}{}{[ʃtlaˈla]}{(v.)}{1}{}{Romper.}{}{}
\verb{xtlalaxtlala}{}{[ˈʃtlalaˈʃtlala]}{(n.)}{1}{}{\textit{Xtlalaxtlala}.}{\textbf{\textit{Ophiobotrys zenkeri}.}}{}
\verb{xtlamunka}{}{[ʃtlam\~uˈka]}{(v.)}{1}{}{Cozer mal.}{}{}
\verb{xtlamunkadu}{}{[ʃtlam\~uˈkadu]}{(adj.)}{1}{}{Mal cozido.}{}{}
\verb{xtlanhu}{}{[ˈʃtl\~{\textturna}ɲu]}{(adj.)}{1}{}{Desconhecido.}{}{}
\verb{xtlanhu}{}{[ˈʃtl\~{\textturna}ɲu]}{(adj.)}{2}{}{Estranho.}{}{}
\verb{xtlanjêlu}{}{[ʃtl\~{\textturna}ˈʒelu]}{(n.)}{1}{}{Estrangeiro.}{}{}
\verb{xtlava}{}{[ʃtlaˈva]}{(v.)}{1}{}{Salvar.}{}{}
\verb{xtlavadô}{}{[ʃtlavaˈdo]}{(adj.)}{1}{}{Salvador.}{}{}
\verb{xtlavadu}{}{[ʃtlaˈvadu]}{(adj.)}{1}{}{Salvo.}{}{}
\verb{xtlavason}{}{[ʃtlavaˈs\~ɔ]}{(n.)}{1}{}{Salvação.}{}{}
\verb{xtlegedu}{}{[ʃtlɛˈgɛdu]}{(n.)}{1}{}{Segredo.}{}{}
\verb{xtleka}{}{[ʃtlɛˈka]}{(v.)}{1}{}{Cercar.}{}{}
\verb{xtlekadu}{}{[ʃtlɛˈkadu]}{(adj.)}{1}{}{Cercado.}{}{}
\verb{xtlekadu}{}{[ʃtlɛˈkadu]}{(n.)}{1}{}{Cercado.}{}{}
\verb{xtlekadu}{}{[ʃtlɛˈkadu]}{(n.)}{2}{}{Curral.}{}{}
\verb{xtlekon}{}{[ʃtlɛˈk\~ɔ]}{(n.)}{1}{}{Cerca.}{}{}
\verb{xtlekon}{}{[ʃtlɛˈk\~ɔ]}{(n.)}{2}{}{Cercado.}{}{}
\verb{xtlela}{}{[ˈʃtlɛla]}{(n.)}{1}{}{Estrela.}{}{}
\verb{xtlela-xtlela}{}{[ˈʃtlɛlaˈʃtlɛla]}{(n.)}{1}{}{\textit{Sêlê-alê}.}{}{}{}
\verb{xtlele}{}{[ʃtlɛˈlɛ]}{(n.)}{1}{}{Formiga-branca.}{}{}
\verb{xtlele}{}{[ʃtlɛˈlɛ]}{(n.)}{2}{}{Térmite.}{}{}
\verb{xtlêlê}{}{[ʃtleˈle]}{(n.)}{1}{}{Formiga-branca.}{}{}{}
\verb{xtlêlê}{}{[ʃtleˈle]}{(n.)}{1}{}{Térmite.}{Cf. \textbf{xtlele}.}{}{}
\verb{xtlêlê}{}{[ʃtleˈle]}{(n.)}{1}{}{Beija-flor-pequeno de São Tomé.}{\textbf{\textit{Anabathmis newtonii}.}}{}
\verb{xtlêlê-mangotxi}{}{[ʃtleˈle
m\~{\textturna}ˈgɔtʃi]}{(n.)}{1}{}{Beija-flor-preto de São
Tomé.}{\textbf{\textit{Nectarinia thomensis}.}}{}
\verb{xtlena}{}{[ʃtlɛˈna]}{(v.)}{1}{}{Chuviscar.}{}{}
\verb{xtlena}{}{[ʃtlɛˈna]}{(v.)}{2}{}{Garoar.}{}{}
\verb{xtlena}{}{[ʃtlɛˈna]}{(v.)}{3}{}{Serenar.}{}{}
\verb{xtlenu}{}{[ˈʃtlɛnu]}{(n.)}{1}{}{Chuvisco.}{}{}
\verb{xtlenu}{}{[ˈʃtlɛnu]}{(n.)}{2}{}{Garoa.}{}{}
\verb{xtlenu}{}{[ˈʃtlɛnu]}{(n.)}{3}{}{Sereno.}{}{}
\verb{xtleson}{}{[ʃtlɛˈs\~ɔ]}{(n.)}{1}{}{Garoa.}{Sinaliza o início do período das chuvas.}{}
\verb{xtleva}{}{[ˈʃtlɛva]}{(n.)}{1}{}{Escuridão.}{}{}%
\verb{xtleva}{}{[ˈʃtlɛva]}{(n.)}{2}{}{Trevas.}{}{}%
\verb{xtleva}{}{[ˈʃtlɛva]}{(n.)}{3}{}{Trevas.}{Representação teatral que ocorre na Quarta-Feira de Cinzas.}{}{}
\verb{xtlijon}{}{[ʃtliˈʒ\~ɔ]}{(n.)}{1}{}{Terapeuta tradicional.}{}{}{}
\verb{xtlijon-matu}{}{[ʃtliˈʒ\~ɔ ˈmatu]}{(n.)}{1}{}{Terapeuta tradicional especializado em coletar remédios no mato.}{}{}
\verb{xtlimonha}{}{[ʃtliˈm\~ɔɲa]}{(n.)}{1}{}{Cerimônia.}{}{}
\verb{xtlinki}{}{[ˈʃtlĩki]}{(id.)}{1}{}{Cf. \textbf{novu xtlinki.}}{}{}
\verb{xtlivi}{}{[ʃtliˈvi]}{(v.)}{2}{}{Servir.}{}{}
\verb{xtlivi}{}{[ʃtliˈvi]}{(v.)}{1}{}{Trabalhar.}{}{}
\verb{xtlividô}{}{[ʃtliviˈdo]}{(n.)}{1}{}{Empregado.}{}{}
\verb{xtlividô}{}{[ʃtliviˈdo]}{(n.)}{2}{}{Servidor.}{}{}%
\verb{xtlivisu}{}{[ʃtliˈvisu]}{(n.)}{1}{}{Serviço.}{}{}{}
\verb{xtlivisu}{}{[ʃtliˈvisu]}{(n.)}{1}{}{Trabalho.}{Cf. \textbf{xtluvisu}.}{}{}
\verb{xtloa}{}{[ʃtlɔˈa]}{(v.)}{1}{}{Atordoar.}{}{}%
\verb{xtloa}{}{[ʃtlɔˈa]}{(v.)}{1}{}{Ficar atordoado.}{}{}%
\verb{xtloadu}{}{[ʃtlɔˈadu]}{(adj.)}{1}{}{Atordoado.}{}{}
\verb{xtlofi}{}{[ˈʃtlɔfi]}{(n.)}{1}{}{Melão-de-São-Caetano.}{\textbf{\textit{Momordica charantia}.}}{}
\verb{xtlofi-d'ôbô}{}{[ˈʃtlɔfi doˈbo]}{(n.)}{1}{}{\textit{Xtlofi-d'ôbô}.}{\textbf{\textit{Diplocyclos palmatus}.}}{}
\verb{xtloga}{}{[ˈʃtlɔga]}{(n.)}{1}{}{Nora.}{}{}
\verb{xtloga}{}{[ˈʃtlɔga]}{(n.)}{1}{}{Sogra.}{}{}
\verb{xtlogu}{}{[ˈʃtlɔgu]}{(n.)}{1}{}{Genro.}{}{}
\verb{xtlogu}{}{[ˈʃtlɔgu]}{(n.)}{1}{}{Sogro.}{}{}
\verb{xtlôkô}{}{[ˈʃtloko]}{(n.)}{1}{}{Xaroco.}{\textbf{\textit{Eleotris vittata}.}}{}
\verb{xtlôlô}{}{[ˈʃtlolo]}{(n.)}{1}{}{Esforço.}{}{}%
\verb{xtlôlô}{}{[ˈʃtlolo]}{(n.)}{2}{}{Sacrifício.}{}{}
\verb{xtlôlô}{}{[ˈʃtlolo]}{(n.)}{3}{}{Suor.}{}{}%
\verb{xtlova}{}{[ʃtlɔˈva]}{(v.)}{1}{}{Estorvar.}{}{}%
\verb{xtlôvê}{}{[ʃtloˈve]}{(v.)}{1}{}{Converter.}{}{}%
\verb{xtlôvê}{}{[ʃtloˈve]}{(v.)}{1}{}{Esconjurar um malefício com a ajuda de um padre.}{}{}
\verb{xtlôvê}{}{[ʃtloˈve]}{(v.)}{2}{}{Solver.}{}{}%
\verb{xtlumu}{}{[ˈʃtlumu]}{(n.)}{1}{}{Estrume.}{}{}
\verb{xtluvisu}{}{[ʃtluˈvisu]}{(n.)}{1}{}{Serviço.}{}{}%
\verb{xtluvisu}{}{[ʃtluˈvisu]}{(n.)}{2}{}{Trabalho.}{}{}
\verb{xtofa}{}{[ʃtɔˈfa]}{(v.)}{1}{}{Estufar.}{}{}
\verb{xtofadu}{}{[ʃtɔˈfadu]}{(adj.)}{1}{}{Estufado.}{}{}
\verb{xtuda}{}{[ʃtuˈda]}{(v.)}{1}{}{Estudar.}{}{}
\verb{xtudantxi}{}{[ʃtuˈd\~{\textturna}tʃi]}{(n.)}{1}{}{Estudante.}{}{}
\verb{xtudantxi}{}{[ʃtuˈd\~{\textturna}tʃi]}{(n.)}{2}{}{Sacristão.}{}{}%
\verb{xtudu}{}{[ˈʃtudu]}{(n.)}{1}{}{Estudo.}{}{}
\verb{xuxu}{}{[ˈʃuʃu]}{(n.)}{1}{}{Faca.}{}{}%
\verb{xuxu}{}{[ˈʃuʃu]}{(n.)}{2}{}{Punhal.}{}{}
\verb{xuxu}{}{[ˈʃuʃu]}{(n.)}{1}{}{Garça-de-cabeça-negra.}{\textbf{\textit{Butorides
striatus}.}}{}

\end{letra}

\begin{letra}{y}

\verb{ya}{}{[ˈja]}{(adv.)}{1}{}{Aqui.}{}{}
\verb{ya}{}{[ˈja]}{(adv.)}{2}{}{Eis.}{\textbf{Ya mina mu en}. \textit{Eis a minha filha}.}{}%
\verb{yaga}{}{[ˈjaga]}{(n.)}{1}{}{Andim seco.}{}{}{}
\verb{yaga}{}{[jaˈga]}{(v.)}{1}{}{Abandonar.}{}{}%
\verb{yaga}{}{[jaˈga]}{(v.)}{2}{}{Banalizar.}{}{}{}
\verb{yaga}{}{[jaˈga]}{(v.)}{3}{}{Desprezar.}{}{}
\verb{yaga}{}{[jaˈga]}{(v.)}{4}{}{Extrair.}{}{}{}
\verb{yala}{}{[jaˈla]}{(adv.)}{1}{}{Eis ali.}{}{}
\verb{yalala}{}{[jalaˈla]}{(adv.)}{1}{}{Ali.}{}{}%
\verb{yalala}{}{[jalaˈla]}{(adv.)}{1}{}{Lá.}{\textbf{Yalala ka kume}.
\textit{Lá está a comer}.}{}%
\verb{yale}{}{[jaˈlɛ]}{(adv.)}{1}{}{Eis.}{\textbf{Yale nansê bi xê mu ke}.
\textit{Eis que vocês me apareceram}.}{}%
\verb{yale}{}{[jaˈlɛ]}{(adv.)}{1}{}{Eis aqui.}{}{}%
\verb{yanga}{}{[j\~{\textturna}ˈga]}{(v.)}{1}{}{Abrir.}{}{}%
\verb{yanga}{}{[j\~{\textturna}ˈga]}{(v.)}{2}{}{Desarticular.}{}{}%
\verb{yanga}{}{[j\~{\textturna}ˈga]}{(v.)}{3}{}{Escancarar.}{}{}
\verb{yaya}{}{[jaˈja]}{(n.)}{1}{}{Sementes subdesenvolvidas de izaquente.}{}{}
\verb{yayaya}{}{[jajaˈja]}{(v.)}{1}{}{Pinicar.}{}{}
\verb{yê}{}{[ˈje]}{(v.)}{1}{}{Agradar.}{}{}%
\verb{yê}{}{[ˈje]}{(v.)}{2}{}{Amontoar.}{}{}
\verb{yê}{}{[ˈje]}{(v.)}{3}{}{Apanhar.}{}{}%
\verb{yê}{}{[ˈje]}{(v.)}{4}{}{Apetecer.}{}{}%
\verb{yê}{}{[ˈje]}{(v.)}{5}{}{Arrumar.}{}{}%
\verb{yê}{}{[ˈje]}{(v.)}{5}{}{Tomar.}{\textbf{yê pikina}. \textit{Tomar um bocadinho}.}{}%
\verb{yêdu}{}{[ˈjedu]}{(adj.)}{1}{}{Amontoado.}{}{}
\verb{yeta}{}{[jɛˈta]}{(v.)}{1}{}{Apegar(-se).}{}{}
\verb{yeta}{}{[jɛˈta]}{(v.)}{1}{}{Esconder(-se).}{}{}
\verb{yeta}{}{[jɛˈta]}{(v.)}{2}{}{Grudar.}{}{}
\verb{yetayeta}{}{[ˈjɛtaˈjɛta]}{(n.)}{1}{}{Cantinho.}{}{}
\verb{yetayeta}{}{[ˈjɛtaˈjɛta]}{(n.)}{2}{}{Esconderijo.}{}{}
\verb{yetayeta}{}{[ˈjɛtaˈjɛta]}{(n.)}{2}{}{Labirinto.}{}{}
\verb{yô}{}{[ˈjo]}{(adv.)}{1}{}{Sim.}{}{}
\verb{yô}{}{[ˈjo]}{(n.)}{1}{}{Ilhéu.}{}{}
\verb{yô}{}{[ˈjo]}{(n.)}{2}{}{Ribeira.}{}{}
\verb{yô}{}{[ˈjo]}{(n.)}{3}{}{Rio.}{}{}{}
\verb{yô}{}{[ˈjo]}{(quant.)}{1}{}{Muito.}{\textbf{Sun tê yô likêza.} \emph{Ele tem muita riqueza.}}{}{}
\verb{yobo}{}{[jɔˈbɔ]}{(n.)}{1}{}{Noz-moscada-da-Jamaica.}{\textbf{\textit{Monodora myristica}.}}{}
\verb{yogo}{}{[jɔˈgɔ]}{(v.)}{1}{}{Afrouxar.}{}{}
\verb{yogo}{}{[jɔˈgɔ]}{(v.)}{2}{}{Convalescer.}{}{}
\verb{yogo}{}{[jɔˈgɔ]}{(v.)}{3}{}{Curar.}{}{}
\verb{yogo}{}{[jɔˈgɔ]}{(v.)}{4}{}{Decair.}{}{}
\verb{yogo}{}{[jɔˈgɔ]}{(v.)}{5}{}{Esmorecer.}{}{}
\verb{yogo}{}{[jɔˈgɔ]}{(v.)}{7}{}{Melhorar.}{}{}
\verb{yogo}{}{[jɔˈgɔ]}{(v.)}{8}{}{Pender.}{}{}
\verb{yôlô}{}{[joˈlo]}{(v.)}{1}{}{Afastar(-se).}{}{}
\verb{yôlô}{}{[joˈlo]}{(v.)}{2}{}{Desamarrar.}{}{}%
\verb{yôlô}{}{[joˈlo]}{(v.)}{3}{}{Despir(-se).}{}{}
\verb{yono}{}{[jɔˈnɔ]}{(v.)}{1}{}{Beliscar.}{}{}
\verb{yono}{}{[jɔˈnɔ]}{(v.)}{1}{}{Retirar um naco.}{}{}
\verb{yôxi}{}{[ˈjoʃi]}{(adv.)}{1}{}{Sim.}{}{}
\verb{yôyô}{}{[joˈjo]}{(n.)}{1}{}{Bala de chumbo.}{}{}%
\verb{yoyo}{}{[jɔˈjɔ]}{(n.)}{1}{}{Pênis de criança.}{}{}
\end{letra}

\begin{letra}{z}
\verb{za}{}{[ˈza]}{(adv.)}{1}{}{Agora.}{}{}
\verb{za}{}{[ˈza]}{(adv.)}{2}{}{Já.}{}{}
\verb{zage}{}{[zaˈgɛ]}{(n.)}{1}{}{Lança.}{}{}
\verb{zage}{}{[zaˈgɛ]}{(n.)}{2}{}{Zagaia.}{}{}%
\verb{zagli}{}{[ˈzagli]}{(n.)}{1}{}{Doença cutânea grave.}{}{}{}%
\verb{zagli}{}{[ˈzagli]}{(n.)}{1}{}{Lepra.}{}{}{}%
\verb{zagli}{}{[ˈzagli]}{(n.)}{1}{}{Sarna.}{}{}{}%
\verb{zaguli}{}{[zaguˈli]}{(v.)}{1}{}{Desfazer(-se).}{}{}
\verb{zaguli}{}{[zaguˈli]}{(v.)}{2}{}{Desfiar.}{}{}
\verb{zaguli}{}{[zaguˈli]}{(v.)}{3}{}{Destruir.}{}{}%
\verb{zaguli}{}{[zaguˈli]}{(v.)}{4}{}{Rasgar.}{}{}
\verb{zala}{}{[zaˈla]}{(v.)}{1}{}{Ciscar.}{}{}
\verb{zala}{}{[zaˈla]}{(v.)}{2}{}{Desfazer.}{}{}%
\verb{zala}{}{[zaˈla]}{(v.)}{3}{}{Dispersar.}{}{}%
\verb{zala}{}{[zaˈla]}{(v.)}{4}{}{Esgravatar.}{}{}%
\verb{zala}{}{[zaˈla]}{(v.)}{4}{}{Espalhar.}{}{}%
\verb{zaladu}{}{[zaˈladu]}{(adj.)}{2}{}{Ciscado.}{}{}
\verb{zaladu}{}{[zaˈladu]}{(adj.)}{2}{}{Desfeito.}{}{}
\verb{zaladu}{}{[zaˈladu]}{(adj.)}{3}{}{Disperso.}{}{}
\verb{zaladu}{}{[zaˈladu]}{(adj.)}{2}{}{Esgravatado.}{}{}
\verb{zaladu}{}{[zaˈladu]}{(adj.)}{2}{}{Espalhado.}{}{}
\verb{zali}{}{[ˈzali]}{(n.)}{1}{}{Verme.}{Cf. \textbf{nzali}.}{}{}
\verb{zalima}{}{[ˈzalima]}{(n.)}{1}{}{Alma.}{}{}
\verb{zalima}{}{[ˈzalima]}{(n.)}{2}{}{Fantasma.}{}{}
\verb{zalima-bluku}{}{[ˈzalima ˈbluku]}{(n.)}{1}{}{Alma dos que tiveram uma
morte violenta.}{}{}
\verb{zalima-bluku}{}{[ˈzalima ˈbluku]}{(n.)}{2}{}{Alma penada.}{}{}
\verb{zalu}{}{[ˈzalu]}{(n.)}{1}{}{Jarro.}{}{}
\verb{zamba}{}{[z\~{\textturna}ˈba]}{(adj.)}{2}{}{Muito grande.}{}{}
\verb{zamba}{}{[z\~{\textturna}ˈba]}{(n.)}{1}{}{Elefante.}{}{}%
\verb{zambomba}{}{[z\~{\textturna}ˈbõba]}{(n.)}{1}{}{Zabumba.}{}{}
\verb{zami}{}{[ˈzami]}{(n.)}{1}{}{Exame.}{}{}
\verb{zamumu}{}{[zamuˈmu]}{(n.)}{1}{}{\textit{Zamumu}.}{\textbf{\textit{Gambeya
africana}.}}{}
\verb{zanela}{}{[zaˈnɛla]}{(n.)}{1}{}{Janela.}{}{}
\verb{zanove}{}{[zaˈnɔvɛ]}{(num.)}{1}{}{Dezanove.}{Cf. \textbf{dezanovi}.}{}
\verb{zanta}{}{[z\~{\textturna}ˈta]}{(n.)}{1}{}{Jantar.}{}{}
\verb{zanta}{}{[z\~{\textturna}ˈta]}{(v.)}{1}{}{Jantar.}{}{}
\verb{zanve}{}{[z\~{\textturna}ˈvɛ]}{(n.)}{1}{}{\textit{Zanve}.}{\textbf{\textit{Tylosurus
acus rafale}.}}{}
\verb{zanvyadu}{}{[z\~{\textturna}ˈvjadu]}{adj.}{1}{}{Enviesado.}{}{}
\verb{zanvyadu}{}{[z\~{\textturna}ˈvjadu]}{adj.}{1}{}{Torto.}{}{}
\verb{zao}{}{[zaˈɔ]}{(adv.)}{1}{}{A seguir.}{}{}{}
\verb{zao}{}{[zaˈɔ]}{(adv.)}{1}{}{Depois.}{}{}{}
\verb{zao}{}{[zaˈɔ]}{(adv.)}{1}{}{De seguida.}{}{}{}
\verb{zao}{}{[zaˈɔ]}{(adv.)}{1}{}{Então.}{Cf. \textbf{zawo}.}{}{}
\verb{zasete}{}{[zaˈsɛtɛ]}{(num.)}{1}{}{Dezassete.}{Cf. \textbf{dezasete}.}{}
\verb{zasêxi}{}{[zaˈseʃi]}{(num.)}{1}{}{Dezasseis.}{Cf. \textbf{dezasêxi}.}{}
\verb{zatona}{}{[zaˈtɔna]}{(n.)}{1}{}{Azeitona.}{}{}
\verb{zaxpela}{}{[zaʃpɛˈla]}{(v.)}{1}{}{Desesperar.}{}{}
\verb{zaxpeladu}{}{[zaʃpɛˈladu]}{(adj.)}{1}{}{Desesperado.}{}{}
\verb{zawa}{}{[ˈzawa]}{(n.)}{1}{}{Urina.}{}{}
\verb{zawo}{}{[zaˈwɔ]}{(adv.)}{1}{}{A seguir.}{}{}
\verb{zawo}{}{[zaˈwɔ]}{(adv.)}{1}{}{Depois.}{\textbf{Nganha ka plapla aza zawo pa ê vwa}. \textit{A galinha experimenta as asas para depois voar}.}{}
\verb{zawo}{}{[zaˈwɔ]}{(adv.)}{1}{}{De seguida.}{}{}
\verb{zawo}{}{[zaˈwɔ]}{(adv.)}{2}{}{Então.}{\textbf{Zawo bô ka punta non}. \textit{Então perguntas-nos}.}{}
\verb{zawo}{}{[zaˈwɔ]}{(adv.)}{1}{}{Seguidamente.}{}{}
\verb{zaya}{}{[ˈzaja]}{(n.)}{1}{}{\textit{Zaya}.}{\textbf{\textit{Cassia podocarpa}.}}{}
\verb{zazaza}{}{[zazaˈza]}{(id.)}{1}{Cf. \textbf{lêdê zazaza}.}{}{}
\verb{zeda}{}{[zɛˈda]}{(v.)}{1}{}{Azedar.}{}{}
\verb{zedadu}{}{[zɛˈdadu]}{(adj.)}{1}{}{Azedado.}{}{}
\verb{zedadu}{}{[zɛˈdadu]}{(adj.)}{2}{}{Cansado.}{}{}
\verb{zedon}{}{[zɛˈd\~ɔ]}{(adj.)}{1}{}{Azedo.}{}{}
\verb{zedon môlô}{}{[zɛˈd\~ɔ ˈmolo]}{(expr.)}{1}{}{Acérrimo.}{}{}
\verb{zedu}{}{[ˈzɛdu]}{(adj.)}{1}{}{Azedo.}{}{}
\verb{zedu kankankan}{}{[ˈzɛdu k\~{\textturna}k\~{\textturna}ˈk\~{\textturna}]}{(expr.)}{1}{}{Acérrimo.}{}{}
\verb{zedu ndonkli}{}{[ˈzɛdu ˈnd\~ɔkli]}{(expr.)}{1}{}{Acérrimo.}{}{}
\verb{zegezege}{}{[zɛˈgɛzɛˈgɛ]}{(id.)}{1}{}{Cf. \textbf{kebla zegezege.}}{}{}\verb{zegezege}{}{[zɛˈgɛzɛˈgɛ]}{(id.)}{1}{}{Cf. \textbf{pobli zegezege.}}{}{}\verb{zêkentxi}{}{[zeˈkẽtʃi]}{(n.)}{1}{}{Izaquente.}{\textbf{\textit{Treculia africana}}.}{}
\verb{zekete}{}{[zɛkɛˈtɛ]}{(id.)}{1}{}{Cf. \textbf{tason zekete.}}{}{}
\verb{zelason}{}{[zɛˈlas\~ɔ]}{(n.)}{1}{}{Geração.}{}{}
\verb{zele}{}{[zɛˈlɛ]}{(v.)}{1}{}{Apodrecer.}{}{}%
\verb{zele}{}{[zɛˈlɛ]}{(v.)}{2}{}{Desfazer(-se).}{}{}
\verb{zema}{}{[ˈzɛma]}{(n.)}{1}{}{Gema.}{}{}
\verb{zema}{}{[zɛˈma]}{(v.)}{1}{}{Algemar.}{}{}
\verb{zema}{}{[zɛˈma]}{(v.)}{2}{}{Cair inerte.}{}{}%
\verb{zemadu}{}{[zɛˈmadu]}{(adj.)}{1}{}{Algemado.}{}{}
\verb{zemadu}{}{[zɛˈmadu]}{(adj.)}{2}{}{Inerte.}{}{}%
\verb{zeme}{}{[zɛˈmɛ]}{(n.)}{1}{}{Gemido.}{}{}
\verb{zeme}{}{[zɛˈmɛ]}{(v.)}{1}{}{Gemer.}{}{}
\verb{zenha}{}{[z\~ɛˈɲa]}{(v.)}{1}{}{Engendrar.}{}{}
\verb{zentria}{}{[z\~eˈtria]}{(n.)}{1}{}{Disenteria.}{}{}
\verb{zentxi}{}{[ˈzẽtʃi]}{(n.)}{1}{}{Gente.}{}{}
\verb{zenze}{}{[z\~ɛˈzɛ]}{(n.)}{1}{}{\textit{Zenze}.}{\textbf{\textit{Pachylobus
edulis}.}}{}
\verb{zeta}{}{[zɛˈta]}{(v.)}{1}{}{Abandonar.}{}{}
\verb{zeta}{}{[zɛˈta]}{(v.)}{2}{}{Detestar.}{}{}
\verb{zeta}{}{[zɛˈta]}{(v.)}{3}{}{Rejeitar.}{}{}%
\verb{zetadu}{}{[zɛˈtadu]}{(adj.)}{1}{}{Abandonado.}{}{}
\verb{zetadu}{}{[zɛˈtadu]}{(adj.)}{2}{}{Enjeitado.}{}{}
\verb{zetadu}{}{[zɛˈtadu]}{(adj.)}{3}{}{Rejeitado.}{}{}%
\verb{zêtê}{}{[ˈzete]}{(n.)}{1}{}{Azeite.}{}{}
\verb{zêtê}{}{[ˈzete]}{(n.)}{1}{}{Óleo de palma.}{}{}
\verb{zêtê-doxi}{}{[ˈzete ˈdɔʃi]}{(n.)}{1}{}{Azeite de oliva.}{}{}
\verb{zêtê-pema}{}{[ˈzete ˈpɛma]}{(n.)}{2}{}{Óleo de palma.}{}{}
\verb{zêtu}{}{[ˈzetu]}{(n.)}{1}{}{Jeito.}{Cf. \textbf{jêtu}.}{}
\verb{zimblon}{}{[zĩˈblõ]}{(n.)}{1}{}{Maçã-da-Índia.}{\textbf{\textit{Ziziphus
abyssinica}.}}{}
\verb{zo}{}{[ˈzɔ]}{(adv.)}{1}{}{A seguir.}{}{}{}
\verb{zo}{}{[ˈzɔ]}{(adv.)}{1}{}{Depois.}{}{}{}
\verb{zo}{}{[ˈzɔ]}{(adv.)}{1}{}{De seguida.}{}{}{}
\verb{zo}{}{[ˈzɔ]}{(adv.)}{1}{}{Então.}{Cf. \textbf{zawo}.}{}{}
\verb{zobla}{}{[ˈzɔbla]}{(n.)}{1}{}{Obras públicas.}{}{}
\verb{zobozobo}{}{[zɔˈbɔzɔˈbɔ]}{(adj.)}{1}{}{Flácido.}{}{}
\verb{zôgô}{}{[ˈzogo]}{(n.)}{1}{}{Jogo.}{}{}
\verb{zolo}{}{[ˈzɔlɔ]}{(n.)}{1}{}{Anzol.}{Cf. \textbf{nzolo}.}{}{}
\verb{zolo}{}{[zɔˈlɔ]}{(v.)}{1}{}{Misturar.}{}{}
\verb{zolodu}{}{[zɔˈlɔdu]}{(adj.)}{1}{}{Misto.}{}{}
\verb{zolodu}{}{[zɔˈlɔdu]}{(adj.)}{2}{}{Misturado.}{}{}
\verb{zona}{}{[ˈzɔna]}{(n.)}{1}{}{Zona.}{}{}
\verb{zôplô}{}{[ˈzoplo]}{(n.)}{1}{}{Simulação.}{}{}
\verb{zozo}{}{[zɔˈzɔ]}{(v.)}{2}{}{Desfazer(-se).}{}{}
\verb{zozo}{}{[zɔˈzɔ]}{(v.)}{3}{}{Desintegrar.}{}{}
\verb{zozo}{}{[zɔˈzɔ]}{(v.)}{4}{}{Espalhar.}{}{}
\verb{zozodu}{}{[zɔˈzɔdu]}{(adj.)}{2}{}{Desfeito.}{}{}
\verb{zozodu}{}{[zɔˈzɔdu]}{(adj.)}{3}{}{Desintegrado.}{}{}
\verb{zozodu}{}{[zɔˈzɔdu]}{(adj.)}{4}{}{Espalhado.}{}{}
\verb{zu}{}{[ˈzu]}{(v.)}{1}{}{Latejar.}{}{}
\verb{zuda}{}{[ˈzuda]}{(n.)}{1}{}{Ajuda.}{}{}
\verb{Zuda}{}{[ˈzuda]}{(n.)}{2}{}{Judas.}{}{}
\verb{zuda}{}{[ˈzuda]}{(n.)}{3}{}{Pessoa falsa.}{}{}
\verb{zuda}{}{[ˈzuda]}{(n.)}{4}{}{Pessoa traiçoeira.}{}{}
\verb{zuda}{}{[zuˈda]}{(v.)}{1}{}{Ajudar.}{}{}%
\verb{zuda pê}{}{[zuˈda ˈpe]}{(expr.)}{1}{}{Descarregar.}{}{}
\verb{zudê}{}{[zuˈde]}{(n.)}{2}{}{Atum-judeu.}{\textbf{\textit{Katsuwonus pelamis}}.}{}
\verb{zudê}{}{[zuˈde]}{(n.)}{1}{}{Judeu.}{}{}%
%\verb{zudon}{}{[zuˈd\~ɔ]}{(n.)}{1}{}{Jordão.}{}{}
\verb{zuga}{}{[zuˈga]}{(v.)}{1}{}{Atirar.}{}{}
\verb{zuga}{}{[zuˈga]}{(v.)}{2}{}{Jogar.}{}{}%
\verb{zuga buta}{}{[zuˈga buˈta]}{(expr.)}{1}{}{Deitar fora.}{}{}
\verb{zuga kupi}{}{[zuˈga kuˈpi]}{(expr.)}{1}{}{Cuspir.}{}{}
\verb{zuga vitô}{}{[zuˈga ˈvito]}{(expr.)}{1}{}{Mentir.}{}{}
%\verb{zuguzugu}{}{[ˈzuguˈzugu]}{(adj.)}{1}{}{Coisa muito flexível, muito macia e abundante.}{}{}
\verb{zuguzugu-dansu}{}{[ˈzuguˈzugu ˈd\~{\textturna}su]}{(n.)}{1}{}{\textit{Zuguzugu-dansu}.}{Personagem ajudante do feiticeiro no \textbf{dansu-kongô}.}{}{}
\verb{zuji}{}{[ˈzuʒi]}{(n.)}{1}{}{Juiz.}{}{}
\verb{zuji}{}{[ˈzuʒi]}{(n.)}{2}{}{Juízo.}{}{}
\verb{zuku}{}{[ˈzuku]}{(n.)}{1}{}{Excrementos.}{}{}{}
\verb{zuku}{}{[ˈzuku]}{(n.)}{1}{}{Fezes.}{Cf. \textbf{nzuku}.}{}{}
\verb{zula}{}{[zuˈla]}{(v.)}{1}{}{Jurar.}{}{}
\verb{zuladu}{}{[zuˈladu]}{(adj.)}{1}{}{Jurado.}{}{}
\verb{zulu}{}{[ˈzulu]}{(n.)}{1}{}{Azul.}{}{}
\verb{zulu kankankan}{}{[ˈzulu k\~{\textturna}k\~{\textturna}ˈk\~{\textturna}]}{(expr.)}{1}{}{Azulíssimo.}{}{}
\verb{zumba}{}{[ˈz\~uba]}{(n.)}{1}{}{Aflição.}{}{}
\verb{zumba}{}{[ˈz\~uba]}{(n.)}{2}{}{Desgraça.}{}{}
\verb{zumba}{}{[ˈz\~uba]}{(n.)}{3}{}{Problema.}{}{}
\verb{zumbi}{}{[zũˈbi]}{(n.)}{1}{}{Espírito.}{}{}
\verb{zumbi}{}{[zũˈbi]}{(n.)}{2}{}{Fantasma.}{}{}
\verb{zumbu}{}{[zũˈbu]}{(n.)}{1}{}{Fralda.}{}{}
\verb{zumbu}{}{[zũˈbu]}{(n.)}{1}{}{Tanga.}{}{}
\verb{zungu}{}{[ˈzũgu]}{(n.)}{1}{}{Quebra-machado.}{\textbf{\textit{Cynometra mannii}.}}{}
\verb{zunta}{}{[ˈzũta]}{(n.)}{1}{}{Junta.}{}{}
\verb{zunta}{}{[ˈzũta]}{(n.)}{1}{}{Joelho.}{}{}
\verb{zunta}{}{[zũˈta]}{(v.)}{1}{}{Ajuntar.}{}{}%
\verb{zunta}{}{[zũˈta]}{(v.)}{2}{}{Juntar.}{}{}%
\verb{zunta}{}{[zũˈta]}{(v.)}{3}{}{Misturar.}{}{}
\verb{zunta}{}{[zũˈta]}{(v.)}{4}{}{Reunir.}{}{}
\verb{zuntadu}{}{[zũˈtadu]}{(adj.)}{1}{}{Juntado.}{}{}
\verb{zuntadu}{}{[zũˈtadu]}{(adj.)}{2}{}{Junto.}{}{}%
\verb{zuntadu}{}{[zũˈtadu]}{(adj.)}{3}{}{Misturado.}{}{}
\verb{zunta fôgô}{}{[zũˈta ˈfogo]}{(expr.)}{1}{}{Preparar o fogo.}{}{}
\verb{zuntamentu}{}{[zũtaˈm\~etu]}{(n.)}{1}{}{Ajuntamento.}{}{}
\verb{zuntamentu}{}{[zũtaˈm\~etu]}{(n.)}{1}{}{Rebaldaria.}{}{}
\verb{zuntu}{}{[ˈzũtu]}{(adj.)}{1}{}{Junto.}{}{}
\verb{zunzwa}{}{[zũˈzwa]}{(v.)}{1}{}{Jejuar.}{}{}
\verb{zusa}{}{[ˈzusa]}{(n.)}{1}{}{Tiririca-comum.}{\textbf{\textit{Cyperus rotundus}}.}{}
\verb{zuxtisa}{}{[zuʃˈtisa]}{(n.)}{1}{}{Justiça.}{}{}
\verb{zuxtu}{}{[ˈzuʃtu]}{(adj.)}{1}{}{Apertado.}{}{}
\verb{zuxtu}{}{[ˈzuʃtu]}{(adj.)}{1}{}{Exato.}{}{}
\verb{zuxtu}{}{[ˈzuʃtu]}{(adj.)}{1}{}{Justo.}{}{}
\verb{zuzuzu}{}{[zuzuˈzu]}{(id.)}{1}{}{Cf. \textbf{fela zuzuzu.}}{}{}
\verb{zuzuzu}{}{[zuzuˈzu]}{(id.)}{2}{}{Cf. \textbf{kenta zuzuzu.}}{}{}
\verb{zuzuzu}{}{[zuzuˈzu]}{(id.)}{3}{}{Cf. \textbf{kentxi zuzuzu.}}{}{}
\end{letra}
