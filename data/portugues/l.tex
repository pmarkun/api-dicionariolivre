\verb{l}{}{}{}{}{s.m.}{Décima segunda letra do alfabeto português.}{l}{0}
\verb{l}{}{}{}{}{}{Abrev. de \textit{litro}.}{l}{0}
\verb{L}{}{}{}{}{}{Algarismo romano equivalente a 50.}{L}{0}
\verb{L}{}{}{}{}{}{Com ponto, abrev. de \textit{leste}.}{L}{0}
\verb{lá}{}{Mús.}{}{}{s.m.}{A sexta nota musical na escala de \textit{dó}.}{lá}{0}
\verb{lá}{}{}{}{}{adv.}{Naquele lugar, ali.}{lá}{0}
\verb{lá}{}{}{}{}{}{Então; em tempo afastado (passado ou futuro); nesse tempo.}{lá}{0}
\verb{La}{}{Quím.}{}{}{}{Símb. do \textit{lantânio}. }{La}{0}
\verb{lã}{}{}{}{}{s.f.}{Pelo animal, especialmente de carneiros e ovelhas.}{lã}{0}
\verb{lã}{}{}{}{}{}{O tecido que se faz desse pelo.}{lã}{0}
\verb{labareda}{ê}{}{}{}{s.f.}{Chama comprida de fogo; língua de fogo.}{la.ba.re.da}{0}
\verb{lábaro}{}{}{}{}{s.m.}{Pedaço de pano que serve de símbolo; estandarte, bandeira, pendão.}{lá.ba.ro}{0}
\verb{labéu}{}{}{}{}{s.m.}{Mancha na reputação; desonra, mácula.}{la.béu}{0}
\verb{lábia}{}{}{}{}{s.f.}{Maneira de falar com que se engana alguém; manha, astúcia, esperteza.}{lá.bia}{0}
\verb{labiada}{}{Bot.}{}{}{s.f.}{Espécime das labiadas, plantas cujas flores têm a corola em forma de lábios, como o alecrim, a erva"-cidreira, a sálvia, o orégano. }{la.bi.a.da}{0}
\verb{labiado}{}{}{}{}{adj.}{Que tem aspecto ou forma de lábios.}{la.bi.a.do}{0}
\verb{labial}{}{}{"-ais}{}{adj.2g.}{Relativo aos lábios.}{la.bi.al}{0}
\verb{labial}{}{Gram.}{"-ais}{}{}{Diz"-se do som ou fonema que se articula com os lábios.}{la.bi.al}{0}
\verb{lábil}{}{}{"-eis}{}{adj.2g.}{Que escorrega, desliza facilmente.}{lá.bil}{0}
\verb{lábil}{}{}{"-eis}{}{}{Variável, adaptável, instável.}{lá.bil}{0}
\verb{lábio}{}{Anat.}{}{}{s.m.}{Cada um dos segmentos carnudos que formam o contorno da boca; beiço.}{lá.bio}{0}
\verb{lábio}{}{}{}{}{}{Borda de uma ferida.}{lá.bio}{0}
\verb{labioso}{ô}{}{"-osos ⟨ó⟩}{"-osa ⟨ó⟩}{adj.}{Que tem lábios grandes e grossos.}{la.bi.o.so}{0}
\verb{labioso}{ô}{}{"-osos ⟨ó⟩}{"-osa ⟨ó⟩}{}{Que tem muita lábia; esperto, astucioso.}{la.bi.o.so}{0}
\verb{labiríntico}{}{}{}{}{adj.}{Que tem aspecto ou forma de labirinto; emaranhado, complexo.}{la.bi.rín.ti.co}{0}
\verb{labirintite}{}{Med.}{}{}{s.f.}{Inflamação do labirinto.}{la.bi.rin.ti.te}{0}
\verb{labirinto}{}{}{}{}{s.m.}{Lugar com tantos caminhos cruzados e divisões que dificultam a saída.}{la.bi.rin.to}{0}
\verb{labirinto}{}{}{}{}{}{Bordado nordestino feito sobre um pano que se desfia para aparecer um tipo de rede.}{la.bi.rin.to}{0}
\verb{labirinto}{}{Anat.}{}{}{}{Parte óssea do ouvido interno, que contém os canais semicirculares, o vestíbulo e o caracol.}{la.bi.rin.to}{0}
\verb{labor}{ô}{}{}{}{s.m.}{Tarefa árdua e demorada; faina, trabalho.}{la.bor}{0}
\verb{laboração}{}{}{"-ões}{}{s.f.}{Ato ou efeito de laborar; faina, trabalho.}{la.bo.ra.ção}{0}
\verb{laborar}{}{}{}{}{v.t.}{Fazer alguma obra; trabalhar, realizar.}{la.bo.rar}{\verboinum{1}}
\verb{laboratório}{}{}{}{}{s.m.}{Local onde se fazem pesquisas científicas, análises químicas, preparo de medicamentos etc.}{la.bo.ra.tó.rio}{0}
\verb{laboratório}{}{}{}{}{}{Local onde se fazem trabalhos de qualquer ramo científico ou artístico.}{la.bo.ra.tó.rio}{0}
\verb{laboratorista}{}{}{}{}{s.2g.}{Médico que se dedica a fazer análises clínicas.}{la.bo.ra.to.ris.ta}{0}
\verb{laboriosidade}{}{}{}{}{s.f.}{Qualidade de laborioso; zelo, esforço, diligência.}{la.bo.ri.o.si.da.de}{0}
\verb{laborioso}{ô}{}{"-osos ⟨ó⟩}{"-osa ⟨ó⟩}{adj.}{Que gosta de trabalhar; ativo, diligente.}{la.bo.ri.o.so}{0}
\verb{laborioso}{ô}{}{"-osos ⟨ó⟩}{"-osa ⟨ó⟩}{}{Que é feito com muito esforço; cansativo, árduo.}{la.bo.ri.o.so}{0}
\verb{laborterapia}{}{Med.}{}{}{s.f.}{Tratamento de enfermidades nervosas pelo trabalho; terapia ocupacional.}{la.bor.te.ra.pi.a}{0}
\verb{labrego}{ê}{}{}{}{s.m.}{Indivíduo rústico, grosseiro; aldeão, campônio.}{la.bre.go}{0}
\verb{labuta}{}{}{}{}{s.f.}{Trabalho árduo; lida, faina, labor.}{la.bu.ta}{0}
\verb{labutar}{}{}{}{}{v.i.}{Trabalhar arduamente; lidar, laborar.}{la.bu.tar}{\verboinum{1}}
\verb{laca}{}{}{}{}{s.f.}{Resina vermelha ou fécula extraída das sementes de várias plantas; goma"-laca.}{la.ca}{0}
\verb{laçada}{}{}{}{}{s.f.}{Laço ou nó que se desata facilmente.}{la.ça.da}{0}
\verb{laçador}{ô}{}{}{}{adj.}{Que maneja o laço com destreza, com muito jeito.}{la.ça.dor}{0}
\verb{lacaio}{}{}{}{}{s.m.}{Criado que acompanhava o amo em passeios e viagens.}{la.cai.o}{0}
\verb{lacaio}{}{Fig.}{}{}{}{Indivíduo servil, subserviente, bajulador.}{la.cai.o}{0}
\verb{laçar}{}{}{}{}{v.t.}{Prender animal ou pessoa com laço.}{la.çar}{\verboinum{3}}
\verb{laçarada}{}{}{}{}{s.f.}{Conjunto de laços para enfeite.}{la.ça.ra.da}{0}
\verb{laçaria}{}{}{}{}{s.f.}{Enfeites em forma de laço.}{la.ça.ri.a}{0}
\verb{laçarote}{ó}{}{}{}{s.m.}{Laço de grandes pontas, usado como enfeite.}{la.ça.ro.te}{0}
\verb{laceração}{}{}{"-ões}{}{s.f.}{Ato ou efeito de lacerar, rasgar; dilaceração.}{la.ce.ra.ção}{0}
\verb{lacerante}{}{}{}{}{adj.2g.}{Que lacera; dilacerante, aflitivo, tormentoso.}{la.ce.ran.te}{0}
\verb{lacerar}{}{}{}{}{v.t.}{Rasgar fortemente; dilacerar.}{la.ce.rar}{\verboinum{1}}
\verb{lacertílio}{}{Zool.}{}{}{s.m.}{Divisão dos répteis à qual pertencem lagartos, lagartixas e camaleões; sáurio.}{la.cer.tí.lio}{0}
\verb{lacete}{ê}{}{}{}{s.m.}{Laço pequeno.}{la.ce.te}{0}
\verb{lacete}{ê}{}{}{}{}{Parte da fechadura por onde entra o fecho.}{la.ce.te}{0}
\verb{laço}{}{}{}{}{s.m.}{Nó, com uma alça ou mais, que se desata facilmente puxando"-se uma das pontas.}{la.ço}{0}
\verb{laço}{}{}{}{}{}{Corda comprida com uma alça na ponta e que se atira ao animal para prendê"-lo.}{la.ço}{0}
\verb{laço}{}{}{}{}{}{Aliança, compromisso, ligação.}{la.ço}{0}
\verb{lacônico}{}{}{}{}{adj.}{De poucas palavras; conciso, breve, resumido.}{la.cô.ni.co}{0}
\verb{laconismo}{}{}{}{}{s.m.}{Maneira breve, concisa, de falar e escrever.}{la.co.nis.mo}{0}
\verb{lacraia}{}{Zool.}{}{}{s.f.}{Animal invertebrado de corpo comprido e achatado, com muitas pernas de cada lado.}{la.crai.a}{0}
\verb{lacrar}{}{}{}{}{v.t.}{Fechar hermeticamente com lacre ou outro material.}{la.crar}{\verboinum{1}}
\verb{lacrau}{}{Zool.}{}{}{s.m.}{Animal invertebrado com cauda terminada em aguilhão provido de veneno; escorpião.}{la.crau}{0}
\verb{lacre}{}{}{}{}{s.m.}{Mistura de substância resinosa com matéria corante usada para selar cartas, fechar garrafas etc.}{la.cre}{0}
\verb{lacre}{}{Por ext.}{}{}{}{Aquilo que se usa para fechar hermeticamente alguma coisa, garantindo sua inviolabilidade.}{la.cre}{0}
\verb{lacrimação}{}{}{"-ões}{}{s.f.}{Ato ou efeito de lacrimar; derramamento de lágrimas.}{la.cri.ma.ção}{0}
\verb{lacrimal}{}{}{"-ais}{}{adj.2g.}{Relativo a lágrima.}{la.cri.mal}{0}
\verb{lacrimal}{}{Anat.}{"-ais}{}{}{Diz"-se do órgão produtor e condutor de lágrimas.}{la.cri.mal}{0}
\verb{lacrimar}{}{}{}{}{v.i.}{Lacrimejar.}{la.cri.mar}{\verboinum{1}}
\verb{lacrimejante}{}{}{}{}{adj.2g.}{Que lacrimeja, verte lágrimas.}{la.cri.me.jan.te}{0}
\verb{lacrimejar}{}{}{}{}{v.i.}{Verter lágrimas; chorar, lacrimar.}{la.cri.me.jar}{\verboinum{1}}
\verb{lacrimogêneo}{}{}{}{}{adj.}{Que provoca lágrimas, faz chorar.}{la.cri.mo.gê.neo}{0}
\verb{lacrimoso}{ô}{}{"-osos ⟨ó⟩}{"-osa ⟨ó⟩}{adj.}{Em lágrimas; choroso, lastimoso.}{la.cri.mo.so}{0}
\verb{lactação}{}{}{"-ões}{}{s.f.}{Ato ou efeito de lactar; amamentação.}{lac.ta.ção}{0}
\verb{lactante}{}{}{}{}{adj.2g.}{Diz"-se da criança que ainda mama ou que está na fase da primeira dentição.}{lac.tan.te}{0}
\verb{lactar}{}{}{}{}{v.t.}{Dar leite do peito; amamentar.}{lac.tar}{0}
\verb{lactar}{}{}{}{}{v.i.}{Sugar o leite do peito; mamar.}{lac.tar}{\verboinum{1}}
\verb{lactário}{}{}{}{}{adj.}{Que segrega leite.}{lac.tá.rio}{0}
\verb{lactário}{}{}{}{}{s.m.}{Instituição que dá assistência a lactantes.}{lac.tá.rio}{0}
\verb{lactente}{}{}{}{}{}{Var. de \textit{lactante}.}{lac.ten.te}{0}
\verb{lácteo}{}{}{}{}{adj.}{Relativo ao leite.}{lác.te.o}{0}
\verb{lácteo}{}{}{}{}{}{Que tem a cor do leite; branco, leitoso.}{lác.te.o}{0}
\verb{lactescente}{}{}{}{}{adj.2g.}{Que produz leite; lactário.}{lac.tes.cen.te}{0}
\verb{lactescente}{}{}{}{}{}{Da cor do leite; leitoso, lácteo.}{lac.tes.cen.te}{0}
\verb{lacticínio}{}{}{}{}{}{Var. de \textit{laticínio}.}{lac.ti.cí.nio}{0}
\verb{láctico}{}{Quím.}{}{}{adj.}{Diz"-se do ácido encontrado no leite, usado como acidulante e intermediário químico.}{lác.ti.co}{0}
\verb{lactífago}{}{}{}{}{adj.}{Que se alimenta, se nutre de leite.}{lac.tí.fa.go}{0}
\verb{lactífero}{}{}{}{}{adj.}{Que produz leite, suco leitoso ou látex.}{lac.tí.fe.ro}{0}
\verb{lactômetro}{}{}{}{}{s.m.}{Instrumento com o qual se avalia a pureza e a densidade do leite.}{lac.tô.me.tro}{0}
\verb{lactose}{ó}{Quím.}{}{}{s.f.}{Açúcar sólido e cristalino que se encontra no leite.}{lac.to.se}{0}
\verb{lacuna}{}{}{}{}{s.f.}{Espaço vazio ou em branco; falha, hiato.}{la.cu.na}{0}
\verb{lacunar}{}{}{}{}{adj.2g.}{Que apresenta lacunas.}{la.cu.nar}{0}
\verb{lacunoso}{ô}{}{"-osos ⟨ó⟩}{"-osa ⟨ó⟩}{adj.}{Que tem falhas; lacunar.}{la.cu.no.so}{0}
\verb{lacustre}{}{}{}{}{adj.2g.}{Relativo a lago.}{la.cus.tre}{0}
\verb{lacustre}{}{}{}{}{}{Que vive nos lagos ou à sua margem.}{la.cus.tre}{0}
\verb{ladainha}{}{}{}{}{s.f.}{Oração com uma série de invocações curtas e respostas repetidas.}{la.da.i.nha}{0}
\verb{ladainha}{}{Fig.}{}{}{}{Enumeração ou lista enfadonha; lengalenga.}{la.da.i.nha}{0}
\verb{ladear}{}{}{}{}{v.t.}{Estar situado ao lado.}{la.de.ar}{0}
\verb{ladear}{}{}{}{}{}{Passar ao lado; contornar.}{la.de.ar}{\verboinum{4}}
\verb{ladeira}{ê}{}{}{}{s.f.}{Descida íngreme; rampa, declive.}{la.dei.ra}{0}
\verb{ladeirento}{}{}{}{}{adj.}{Disposto em ladeira; inclinado, declivoso.}{la.dei.ren.to}{0}
\verb{ladineza}{ê}{}{}{}{s.f.}{Ladinice.}{la.di.ne.za}{0}
\verb{ladinice}{}{}{}{}{s.f.}{Qualidade ou caráter de ladino; astúcia, manha.}{la.di.ni.ce}{0}
\verb{ladino}{}{}{}{}{adj.}{Que é esperto, astuto, manhoso.}{la.di.no}{0}
\verb{ladino}{}{}{}{}{s.m.}{Língua neolatina falada no leste da Suíça; rético.}{la.di.no}{0}
\verb{lado}{}{}{}{}{s.m.}{A parte direita ou esquerda de qualquer coisa, pessoa ou animal; flanco.}{la.do}{0}
\verb{lado}{}{Geom.}{}{}{}{Cada uma das faces de um sólido.}{la.do}{0}
\verb{lado}{}{}{}{}{}{Aspecto, feição.}{la.do}{0}
\verb{ladra}{}{}{}{}{s.f.}{Mulher que rouba ou furta.}{la.dra}{0}
\verb{ladrado}{}{}{}{}{s.m.}{Som produzido pelo cão; latido, ladrido.}{la.dra.do}{0}
\verb{ladrador}{ô}{}{}{}{adj.}{Diz"-se do animal que ladra ou late.}{la.dra.dor}{0}
\verb{ladrão}{}{}{"-ões}{}{}{Homem que furta ou rouba; larápio, gatuno.}{la.drão}{0}
\verb{ladrão}{}{}{"-ões}{}{s.m.}{Cano nas caixas"-d'água, banheiras, canalizações diversas, por onde se escoa o líquido excedente.}{la.drão}{0}
\verb{ladrar}{}{}{}{}{v.i.}{Produzir ladrados; latir.}{la.drar}{\verboinum{1}}
\verb{ladravaz}{}{}{}{}{s.m.}{Grande ladrão.}{la.dra.vaz}{0}
\verb{ladrido}{}{}{}{}{s.m.}{Ladrado, latido.}{la.dri.do}{0}
\verb{ladrilhar}{}{}{}{}{v.t.}{Revestir piso ou parede de ladrilhos.}{la.dri.lhar}{\verboinum{1}}
\verb{ladrilheiro}{ê}{}{}{}{s.m.}{Fabricante ou vendedor de ladrilhos.}{la.dri.lhei.ro}{0}
\verb{ladrilho}{}{}{}{}{s.m.}{Peça de cerâmica retangular ou quadrada, usada para revestir paredes ou pisos; azulejo.}{la.dri.lho}{0}
\verb{ladro}{}{}{}{}{s.m.}{Ladrado, latido.}{la.dro}{0}
\verb{ladroaço}{}{}{}{}{s.m.}{Grande ladrão; ladravaz.}{la.dro.a.ço}{0}
\verb{ladroagem}{}{}{"-ens}{}{s.f.}{Ato ou efeito de roubar; extorsão, usurpação.}{la.dro.a.gem}{0}
\verb{ladroagem}{}{}{"-ens}{}{}{A classe dos ladrões.}{la.dro.a.gem}{0}
\verb{ladroeira}{ê}{}{}{}{s.f.}{Ladroagem.}{la.dro.ei.ra}{0}
\verb{ladroíce}{}{}{}{}{s.f.}{Ladroagem.}{la.dro.í.ce}{0}
\verb{lagamar}{}{}{}{}{s.m.}{Cova no fundo do mar ou de um rio; pego.}{la.ga.mar}{0}
\verb{lagamar}{}{}{}{}{}{Baía, golfo.}{la.ga.mar}{0}
\verb{lagar}{}{}{}{}{s.m.}{Tanque onde se espremem as uvas para extrair"-lhes o suco.}{la.gar}{0}
\verb{lagarta}{}{Zool.}{}{}{s.f.}{Larva das borboletas e mariposas que apresenta corpo mole e alongado.}{la.gar.ta}{0}
\verb{lagarta}{}{}{}{}{}{Corrente larga que cobre a roda de tratores e tanques de guerra facilitando seu deslocamento.}{la.gar.ta}{0}
\verb{lagarta"-de"-fogo}{ô}{Zool.}{lagartas"-de"-fogo ⟨ô⟩}{}{s.f.}{Lagarta que produz queimadura ao ser tocada; taturana.}{la.gar.ta"-de"-fo.go}{0}
\verb{lagartear}{}{}{}{}{v.i.}{Ficar sem fazer nada, estirado ao sol, aquecendo"-se como um lagarto.}{la.gar.te.ar}{\verboinum{4}}
\verb{lagartixa}{ch}{Zool.}{}{}{s.f.}{Pequeno réptil que corre por paredes, muros e árvores, caçando insetos. }{la.gar.ti.xa}{0}
\verb{lagarto}{}{Zool.}{}{}{s.m.}{Réptil de pele escamosa, cauda comprida, pernas curtas e língua bipartida.}{la.gar.to}{0}
\verb{lago}{}{}{}{}{s.m.}{Grande extensão de água, natural ou criada pelo homem, cercada de terras.}{la.go}{0}
\verb{lago}{}{}{}{}{}{Grande porção de líquido derramado ou empoçado.}{la.go}{0}
\verb{lagoa}{}{}{}{}{s.f.}{Lago pouco extenso.}{la.go.a}{0}
\verb{lagoa}{}{}{}{}{}{Porção de água estagnada; charco.}{la.go.a}{0}
\verb{lagoeiro}{ê}{}{}{}{s.m.}{Depósito temporário de águas de chuva  em depressões de terreno.}{la.go.ei.ro}{0}
\verb{lagosta}{ô}{Zool.}{}{}{s.f.}{Crustáceo de dez pernas, antenas compridas e carapaça espessa, que vive no fundo do mar e tem carne muito apreciada.}{la.gos.ta}{0}
\verb{lagostim}{}{Zool.}{}{}{s.m.}{Crustáceo parecido com uma pequena lagosta, sem antenas longas.}{la.gos.tim}{0}
\verb{lágrima}{}{}{}{}{s.f.}{Gota do líquido produzido pelas glândulas lacrimais.}{lá.gri.ma}{0}
\verb{lagrimejar}{}{}{}{}{}{Var. de \textit{lacrimejar}.}{la.gri.me.jar}{0}
\verb{lagrimoso}{ô}{}{}{}{}{Var. de \textit{lacrimoso}.}{la.gri.mo.so}{0}
\verb{laguna}{}{}{}{}{s.f.}{Canal ou lago de barragem, pouco profundo, formado entre bancos de areia ou ilhotas por acúmulo de água do mar.}{la.gu.na}{0}
\verb{laia}{}{}{}{}{s.f.}{Qualidade desprezível; espécie, categoria.}{lai.a}{0}
\verb{laical}{}{}{"-ais}{}{adj.2g.}{Relativo a leigo.}{lai.cal}{0}
\verb{laicidade}{}{}{}{}{s.f.}{Qualidade do que é laico ou leigo; secularidade.}{lai.ci.da.de}{0}
\verb{laicismo}{}{}{}{}{s.m.}{Estado ou caráter de leigo.}{lai.cis.mo}{0}
\verb{laicismo}{}{}{}{}{}{Sistema administrativo em que não há influência da Igreja.}{lai.cis.mo}{0}
\verb{laicizar}{}{}{}{}{v.t.}{Tornar leigo, isentando de influências religiosas; secularizar.}{lai.ci.zar}{\verboinum{1}}
\verb{laico}{}{}{}{}{adj.}{De caráter não religioso; secular, laical.}{lai.co}{0}
\verb{laivo}{}{}{}{}{s.m.}{Marca deixada por alguma substância; mancha, nódoa, resquício.}{lai.vo}{0}
%\verb{}{}{}{}{}{}{}{}{0}
\verb{laje}{}{}{}{}{s.f.}{Pedra lisa, de superfície plana e pouca espessura; lajem.}{la.je}{0}
\verb{laje}{}{}{}{}{}{Cobertura ou piso feito de cimento e areia e reforçado com barras de ferro.}{la.je}{0}
\verb{lajeado}{}{}{}{}{s.m.}{Pavimento coberto de lajes; lajedo.}{la.je.a.do}{0}
\verb{lajear}{}{}{}{}{v.t.}{Revestir ou cobrir com lajes.}{la.je.ar}{\verboinum{4}}
\verb{lajedo}{ê}{}{}{}{s.m.}{Lajeado.}{la.je.do}{0}
\verb{lajem}{}{}{"-ens}{}{s.f.}{Laje.}{la.jem}{0}
\verb{lajota}{ó}{}{}{}{s.f.}{Pequena laje usada para revestir pisos ou caminhos.}{la.jo.ta}{0}
\verb{lajoteiro}{ê}{}{}{}{s.m.}{Fabricante ou assentador de lajes ou lajotas.}{la.jo.tei.ro}{0}
\verb{lama}{}{}{}{}{s.f.}{Mistura pegajosa de terra com água; lodo.}{la.ma}{0}
\verb{lama}{}{}{}{}{s.m.}{Sacerdote budista. }{la.ma}{0}
\verb{lamaçal}{}{}{"-ais}{}{s.m.}{Grande extensão de lama; atoleiro, lamaceira, lamaceiro.}{la.ma.çal}{0}
\verb{lamaceira}{ê}{}{}{}{s.f.}{Lamaçal.}{la.ma.cei.ra}{0}
\verb{lamaceiro}{ê}{}{}{}{s.m.}{Lamaçal.}{la.ma.cei.ro}{0}
\verb{lamacento}{}{}{}{}{adj.}{Cheio de lama; enlameado.}{la.ma.cen.to}{0}
\verb{lambada}{}{}{}{}{s.f.}{Golpe forte com chicote ou vara; cacetada, chicotada.}{lam.ba.da}{0}
\verb{lambada}{}{}{}{}{}{Música e dança brasileiras de movimentos rápidos e ritmo agitado, em que as pernas dos dançarinos se entrecuzam.}{lam.ba.da}{0}
\verb{lambança}{}{}{}{}{s.f.}{Desordem, confusão, tumulto.}{lam.ban.ça}{0}
\verb{lambão}{}{}{"-ões}{}{s.m.}{Que gosta de comer muito; guloso, comilão.}{lam.bão}{0}
\verb{lambão}{}{}{"-ões}{}{}{Tolo, palerma.}{lam.bão}{0}
\verb{lambari}{}{Zool.}{}{}{s.m.}{Peixe pequeno muito comum nos rios brasileiros.}{lam.ba.ri}{0}
\verb{lambateria}{}{}{}{}{s.f.}{Casa noturna onde se dança lambada.}{lam.ba.te.ri.a}{0}
%\verb{}{}{}{}{}{}{}{}{0}
\verb{lambda}{}{}{}{}{s.m.}{Décima primeira letra do alfabeto grego.}{lamb.da}{0}
\verb{lambdacismo}{}{Gram.}{}{}{s.m.}{Pronúncia viciosa  que consiste em articular \textit{l} em lugar de \textit{r}.}{lamb.da.cis.mo}{0}
\verb{lambedor}{ô}{}{}{}{adj.}{Que lambe, bajula; bajulador.}{lam.be.dor}{0}
\verb{lambedura}{}{}{}{}{s.f.}{Ato ou efeito de lamber; lambidela.}{lam.be.du.ra}{0}
\verb{lambe"-lambe}{}{}{lambe"-lambes}{}{s.m.}{Indivíduo que fotografa transeuntes e lhes oferece a compra de suas fotografias; fotógrafo ambulante que faz trabalhos em lugares públicos.}{lam.be"-lam.be}{0}
\verb{lamber}{ê}{}{}{}{v.t.}{Passar a língua sobre.}{lam.ber}{0}
\verb{lamber}{ê}{}{}{}{}{Adular, bajular.}{lam.ber}{0}
\verb{lamber}{ê}{}{}{}{v.pron.}{Sentir alegria; ficar contente.}{lam.ber}{\verboinum{12}}
\verb{lambeta}{ê}{}{}{}{adj.2g.}{Que fala mal de outrem; mexeriqueiro, bisbilhoteiro.}{lam.be.ta}{0}
\verb{lambição}{}{}{"-ões}{}{s.f.}{Ato de viver falando bem de outrem; bajulação, adulação.}{lam.bi.ção}{0}
\verb{lambida}{}{}{}{}{s.f.}{Ato ou efeito de lamber; lambidura, lambidela. (\textit{O menino deu uma lambida no sorvete do pai.})}{lam.bi.da}{0}
\verb{lambidela}{é}{}{}{}{s.f.}{Ato de lamber rapidamente; lambida.}{lam.bi.de.la}{0}
\verb{lambido}{}{}{}{}{adj.}{Que se lambeu.}{lam.bi.do}{0}
\verb{lambido}{}{}{}{}{}{Diz"-se do cabelo muito alisado.}{lam.bi.do}{0}
\verb{lambiscar}{}{}{}{}{v.t.}{Comer pouco; beliscar. (\textit{A menina lambiscava a torta.})}{lam.bis.car}{\verboinum{2}}
\verb{lambisco}{}{}{}{}{s.m.}{Pequena porção de comida.}{lam.bis.co}{0}
\verb{lambisco}{}{}{}{}{}{Pouca coisa.}{lam.bis.co}{0}
\verb{lambisgoia}{ó}{}{}{}{s.f.}{Pessoa, especialmente mulher, intrometida, antipática e pretensiosa.}{lam.bis.goi.a}{0}
\verb{lambrequins}{}{}{}{}{s.m.pl.}{Ornatos de recortes de madeira ou de lâmina metálica para beiras de telhados, cortinas, cantoneiras etc.}{lam.bre.quins}{0}
\verb{lambreta}{ê}{}{}{}{s.f.}{Veículo parecido com a motocicleta, mas com assento e rodas menores.}{lam.bre.ta}{0}
\verb{lambretista}{}{}{}{}{s.2g.}{Pessoa que guia lambreta.}{lam.bre.tis.ta}{0}
\verb{lambri}{}{}{}{}{s.m.}{Revestimento de madeira, azulejos, mármore etc., aplicado até certa altura das paredes internas da peça de um edifício.}{lam.bri}{0}
\verb{lambril}{}{}{}{}{}{Var. de \textit{lambri}.}{lam.bril}{0}
\verb{lambris}{}{}{}{}{}{Var. de \textit{lambri}.}{lam.bris}{0}
\verb{lambrisar}{}{}{}{}{v.t.}{Revestir uma parede de lambri.}{lam.bri.sar}{\verboinum{1}}
\verb{lambuja}{}{}{}{}{s.f.}{Vantagem dada ao outro jogador.}{lam.bu.ja}{0}
\verb{lambuja}{}{}{}{}{}{Coisa que se ganha a mais que o combinado.}{lam.bu.ja}{0}
\verb{lambujem}{}{}{"-ens}{}{s.f.}{Lambuja.}{lam.bu.jem}{0}
\verb{lambuzada}{}{}{}{}{s.f.}{Ato ou efeito de lambuzar.}{lam.bu.za.da}{0}
\verb{lambuzada}{}{}{}{}{}{Aquilo que suja, que lambuza.}{lam.bu.za.da}{0}
\verb{lambuzar}{}{}{}{}{v.t.}{Sujar pessoa ou coisa com líquido, comida, cola, graxa etc.}{lam.bu.zar}{\verboinum{1}}
%\verb{}{}{}{}{}{}{}{}{0}
\verb{lamê}{}{}{}{}{s.m.}{Tecido brilhoso, enfeitado com lâminas prateadas ou douradas, ou feito de um fio de metal.}{la.mê}{0}
\verb{lameira}{ê}{}{}{}{s.f.}{Lameiro.}{la.mei.ra}{0}
\verb{lameiro}{ê}{}{}{}{s.m.}{Lamaçal.}{la.mei.ro}{0}
\verb{lameiro}{ê}{}{}{}{}{Terra cultivada na vazante do rio.}{la.mei.ro}{0}
\verb{lamela}{é}{}{}{}{s.f.}{Lâmina muito fina.}{la.me.la}{0}
\verb{lameliforme}{ó}{}{}{}{adj.2g.}{Que tem forma de lâmina ou lamela.}{la.me.li.for.me}{0}
\verb{lamentação}{}{}{"-ões}{}{s.f.}{Ato ou efeito de lamentar.}{la.men.ta.ção}{0}
\verb{lamentação}{}{}{"-ões}{}{}{Queixa, lamúria.}{la.men.ta.ção}{0}
\verb{lamentação}{}{}{"-ões}{}{}{Canto fúnebre.}{la.men.ta.ção}{0}
\verb{lamentar}{}{}{}{}{v.t.}{Mostrar"-se triste e descontente com o que acontece a pessoa ou coisa; deplorar, lastimar. (\textit{Lamentamos a sua derrota no campeonato.})}{la.men.tar}{\verboinum{1}}
\verb{lamentável}{}{}{"-eis}{}{adj.2g.}{Que é digno de ser lamentado; digno de dó, de compaixão.}{la.men.tá.vel}{0}
\verb{lamentável}{}{}{"-eis}{}{}{Digno de ser censurado; deplorável.}{la.men.tá.vel}{0}
\verb{lamento}{}{}{}{}{s.m.}{Palavras ditas com voz de choro; lamentação, queixa.}{la.men.to}{0}
\verb{lamentoso}{ô}{}{"-osos ⟨ó⟩}{"-osa ⟨ó⟩}{adj.}{Em que há lamentação; lastimoso.}{la.men.to.so}{0}
\verb{lâmina}{}{}{}{}{s.f.}{Placa delgada e chata.}{lâ.mi.na}{0}
\verb{lâmina}{}{}{}{}{}{Placa de vidro usada para observação em microscópio.}{lâ.mi.na}{0}
\verb{lâmina}{}{}{}{}{}{Parte achatada de um objeto cortante.}{lâ.mi.na}{0}
\verb{laminação}{}{}{"-ões}{}{s.f.}{Ato ou efeito de laminar ou reduzir o metal a lâmina.}{la.mi.na.ção}{0}
\verb{laminação}{}{}{"-ões}{}{}{Estabelecimento onde se laminam metais.}{la.mi.na.ção}{0}
\verb{laminado}{}{}{}{}{adj.}{Que se laminou.}{la.mi.na.do}{0}
\verb{laminado}{}{}{}{}{}{Composto de lâminas.}{la.mi.na.do}{0}
\verb{laminado}{}{}{}{}{s.m.}{Chapa de metal reduzida a lâmina.}{la.mi.na.do}{0}
\verb{laminador}{ô}{}{}{}{adj.}{Diz"-se de quem lamina.}{la.mi.na.dor}{0}
\verb{laminador}{ô}{}{}{}{s.m.}{Instrumento ou máquina de laminar.}{la.mi.na.dor}{0}
\verb{laminar}{}{}{}{}{v.t.}{Transformar alguma coisa em lâmina.}{la.mi.nar}{\verboinum{1}}
\verb{laminoso}{ô}{}{"-osos ⟨ó⟩}{"-osas ⟨ó⟩}{adj.}{Que tem lâminas.}{la.mi.no.so}{0}
\verb{lamiré}{}{Mús.}{}{}{s.m.}{Peça metálica que emite som em uma frequência fixa e constante, usado para afinar instrumentos musicais; diapasão.}{la.mi.ré}{0}
\verb{lamiré}{}{Fig.}{}{}{}{Sinal para começar alguma coisa.}{la.mi.ré}{0}
\verb{lamiré}{}{Pop.}{}{}{}{Repreensão.}{la.mi.ré}{0}
\verb{lamoso}{ô}{}{"-osos ⟨ó⟩}{"-osas ⟨ó⟩}{adj.}{Cheio ou coberto de lama.}{la.mo.so}{0}
\verb{lamoso}{ô}{}{"-osos ⟨ó⟩}{"-osas ⟨ó⟩}{}{Semelhante a lama.}{la.mo.so}{0}
\verb{lâmpada}{}{}{}{}{s.f.}{Globo de vidro com um fio de metal ou gás que se acende para iluminar.}{lâm.pa.da}{0}
\verb{lampadário}{}{}{}{}{s.m.}{Peça suspensa do teto, em que há uma ou mais lâmpadas; candelabro, lustre.}{lam.pa.dá.rio}{0}
\verb{lamparina}{}{}{}{}{s.f.}{Pequena lâmpada.}{lam.pa.ri.na}{0}
\verb{lamparina}{}{}{}{}{}{Pavio fixo a uma boia que, mergulhada no azeite ou querosene contidos em um pequeno recipiente, serve para iluminar ambientes.}{lam.pa.ri.na}{0}
\verb{lampeiro}{ê}{}{}{}{adj.}{Que é agitado e alegre; faceiro. (\textit{A menina veio toda lampeira com a boneca que ganhou.})}{lam.pei.ro}{0}
\verb{lampejante}{}{}{}{}{adj.2g.}{Que lampeja.}{lam.pe.jan.te}{0}
\verb{lampejar}{}{}{}{}{v.i.}{Provocar um clarão repentino; brilhar, faiscar.}{lam.pe.jar}{\verboinum{1}}
\verb{lampejo}{ê}{}{}{}{s.m.}{Clarão que aparece de repente; brilho, cintilação.}{lam.pe.jo}{0}
\verb{lampejo}{ê}{Fig.}{}{}{}{Manifestação rápida ou brilhante de uma ideia.}{lam.pe.jo}{0}
\verb{lampião}{}{}{"-ões}{}{s.m.}{Lanterna grande, portátil ou fixa, que funciona a eletricidade ou a gás, querosene ou outro combustível. }{lam.pi.ão}{0}
\verb{lampreia}{ê/ ou /é}{Zool.}{}{}{s.f.}{Peixe da Europa, de forma cilíndrica e alongada, muito saboroso e apreciado.}{lam.prei.a}{0}
\verb{lamúria}{}{}{}{}{s.f.}{Demonstração de descontentamento com lamentações, choros ou gemidos; lamento, queixa.}{la.mú.ria}{0}
\verb{lamuriante}{}{}{}{}{adj.2g.}{Que faz lamúria, que lamuria.}{la.mu.ri.an.te}{0}
\verb{lamuriante}{}{}{}{}{}{Que tem caráter de lamúria, que envolve lamúria.}{la.mu.ri.an.te}{0}
\verb{lamuriar}{}{}{}{}{v.t.}{Falar alguma coisa com lamúrias; lamentar, lastimar, queixar"-se.}{la.mu.ri.ar}{\verboinum{6}}
\verb{lança}{}{}{}{}{s.f.}{Arma formada por uma vara comprida, de ferro ou madeira, terminada em ponta afiada.}{lan.ça}{0}
\verb{lança"-chamas}{}{}{lança"-chamas}{}{s.m.}{Arma que atira um líquido que pega fogo.}{lan.ça"-cha.mas}{0}
\verb{lançada}{}{}{}{}{s.f.}{Golpe de lança.}{lan.ça.da}{0}
\verb{lançadeira}{ê}{}{}{}{s.f.}{Peça de tear que faz passar o fio da tecelagem.}{lan.ça.dei.ra}{0}
\verb{lançadeira}{ê}{}{}{}{}{Peça semelhante, nas máquinas de costura. }{lan.ça.dei.ra}{0}
\verb{lançador}{ô}{}{}{}{adj.}{Que atira com lança.}{lan.ça.dor}{0}
\verb{lançador}{ô}{}{}{}{}{Que é o primeiro a colocar uma novidade no mercado.}{lan.ça.dor}{0}
\verb{lançamento}{}{}{}{}{s.m.}{Projeção, movimento para a frente.}{lan.ça.men.to}{0}
\verb{lançamento}{}{}{}{}{}{Arremesso, jogada.}{lan.ça.men.to}{0}
\verb{lançamento}{}{}{}{}{}{Introdução de algo novo no mercado.}{lan.ça.men.to}{0}
\verb{lançamento}{}{}{}{}{}{O produto lançado.}{lan.ça.men.to}{0}
\verb{lança"-perfume}{}{}{lança"-perfumes}{}{s.m.}{Recipiente cilíndrico, de vidro ou de metal, que contém éter perfumado mantido sob pressão e lançado em jato, e que se usa especialmente durante o carnaval.}{lan.ça"-per.fu.me}{0}
\verb{lançar}{}{}{}{}{v.t.}{Fazer pessoa, animal ou coisa sair com violência em determinada direção; arremessar, atirar, jogar.}{lan.çar}{0}
\verb{lançar}{}{}{}{}{}{Apresentar pessoa ou coisa pela primeira vez ao público.}{lan.çar}{0}
\verb{lançar}{}{}{}{}{}{Escrever alguma nota em um lugar; anotar, registrar.}{lan.çar}{\verboinum{3}}
\verb{lança"-torpedos}{ê}{}{lança"-torpedos ⟨ê⟩}{}{s.m.}{Aparelho a bordo de navios e submarinos, para lançar torpedos.}{lan.ça"-tor.pe.dos}{0}
\verb{lance}{}{}{}{}{s.m.}{Cada um dos movimentos de um jogo; jogada.}{lan.ce}{0}
\verb{lance}{}{}{}{}{}{Cada uma das partes de um acontecimento; momento.}{lan.ce}{0}
\verb{lance}{}{}{}{}{}{Cada uma das ofertas de preço num leilão.}{lan.ce}{0}
\verb{lancear}{}{}{}{}{v.t.}{Golpear com lança.}{lan.ce.ar}{\verboinum{4}}
\verb{lanceiro}{ê}{}{}{}{s.m.}{Soldado armado de lança.}{lan.cei.ro}{0}
\verb{lanceolado}{}{}{}{}{adj.}{Que tem feitio semelhante ao da lança.}{lan.ce.o.la.do}{0}
\verb{lanceta}{ê}{Med.}{}{}{s.f.}{Instrumento de dois gumes, usado em pequenas cirurgias. }{lan.ce.ta}{0}
\verb{lancetar}{}{}{}{}{v.t.}{Cortar ou abrir com lanceta.}{lan.ce.tar}{\verboinum{1}}
\verb{lancha}{}{}{}{}{s.f.}{Embarcação pequena, movida a motor, própria para navegação costeira, para transporte ou para outros serviços dentro dos portos.}{lan.cha}{0}
\verb{lanchão}{}{}{"-ões}{}{s.m.}{Lancha aberta de grande porte.}{lan.chão}{0}
\verb{lanchar}{}{}{}{}{v.i.}{Comer lanche. (\textit{Os alunos lancham na cantina da escola.})}{lan.char}{\verboinum{1}}
\verb{lanche}{}{}{}{}{s.m.}{Refeição pequena e leve; merenda.}{lan.che}{0}
\verb{lancheira}{ê}{}{}{}{s.f.}{Maleta onde se leva lanche.}{lan.chei.ra}{0}
\verb{lanchonete}{é}{}{}{}{s.f.}{Pequeno restaurante que serve refeições ligeiras, geralmente no balcão.}{lan.cho.ne.te}{0}
\verb{lancinante}{}{}{}{}{adj.2g.}{Que faz sofrer muito.}{lan.ci.nan.te}{0}
\verb{lancinante}{}{}{}{}{}{Em que se nota muita dor.}{lan.ci.nan.te}{0}
\verb{lancinar}{}{}{}{}{v.t.}{Fazer um corte em pessoa ou animal; golpear, pungir.}{lan.ci.nar}{0}
\verb{lancinar}{}{}{}{}{}{Afligir, atormentar, torturar.}{lan.ci.nar}{\verboinum{1}}
\verb{lanço}{}{}{}{}{s.m.}{Ato ou efeito de lançar; arremesso, jogada, lançamento.}{lan.ço}{0}
\verb{lanço}{}{}{}{}{}{Oferta de preço em leilão ou venda; lance.}{lan.ço}{0}
\verb{lanço}{}{}{}{}{}{Conjunto de degraus que termina num pequeno piso.}{lan.ço}{0}
\verb{landa}{}{}{}{}{s.f.}{Descampado onde só crescem ervas silvestres.}{lan.da}{0}
\verb{landau}{}{}{landaus}{}{s.m.}{Carruagem de quatro rodas, com capota que se levanta e abaixa.}{lan.dau}{0}
\verb{lande}{}{}{}{}{s.m.}{Bolota de carvalho, do sobreiro.}{lan.de}{0}
\verb{langanho}{}{Bras.}{}{}{s.m.}{Carne de baixa qualidade, com nervos ou peles.}{lan.ga.nho}{0}
\verb{langor}{ô}{}{}{}{s.m.}{Languidez.}{lan.gor}{0}
\verb{langoroso}{ô}{}{"-osos ⟨ó⟩}{"-osa ⟨ó⟩}{adj.}{Cheio de langor; frouxo, lânguido.}{lan.go.ro.so}{0}
\verb{languescer}{ê}{}{}{}{v.i.}{Tornar"-se lânguido; enfraquecer.}{lan.gues.cer}{\verboinum{15}}
\verb{languidez}{ê}{}{}{}{s.f.}{Estado de lânguido; moleza, frouxidão, langor.}{lan.gui.dez}{0}
\verb{lânguido}{}{}{}{}{adj.}{Abatido, debilitado, mole, sem força.}{lân.gui.do}{0}
\verb{lanhar}{}{}{}{}{v.t.}{Golpear, cortar, ferir.}{la.nhar}{0}
\verb{lanhar}{}{}{}{}{}{Afligir, magoar, maltratar.}{la.nhar}{0}
\verb{lanhar}{}{}{}{}{}{Alterar, deturpar, desvirtuar.}{la.nhar}{\verboinum{1}}
\verb{lanho}{}{}{}{}{s.m.}{Golpe com instrumento cortante.}{la.nho}{0}
\verb{lanho}{}{}{}{}{}{O ferimento produzido por esse golpe.}{la.nho}{0}
\verb{lanho}{}{Bras.}{}{}{}{Carne bovina cortada em tiras.}{la.nho}{0}
\verb{lanífero}{}{}{}{}{adj.}{Lanígero.}{la.ní.fe.ro}{0}
\verb{lanifício}{}{}{}{}{s.m.}{Técnica de fabricação de lã.}{la.ni.fí.cio}{0}
\verb{lanifício}{}{}{}{}{}{Fábrica de artigos de lã.}{la.ni.fí.cio}{0}
\verb{lanifício}{}{}{}{}{}{Artigo de lã.}{la.ni.fí.cio}{0}
\verb{lanígero}{}{}{}{}{adj.}{Que produz lã; lanífero.}{la.ní.ge.ro}{0}
\verb{lanígero}{}{}{}{}{}{Provido de lã ou lanugem.}{la.ní.ge.ro}{0}
\verb{lanolina}{}{Quím.}{}{}{s.f.}{Substância de origem animal usada na fabricação de pomadas, cremes e cosméticos.}{la.no.li.na}{0}
\verb{lanoso}{ô}{}{"-osos ⟨ó⟩}{"-osa ⟨ó⟩}{adj.}{Relativo a lã.}{la.no.so}{0}
\verb{lanoso}{ô}{}{"-osos ⟨ó⟩}{"-osa ⟨ó⟩}{}{Que tem lã.}{la.no.so}{0}
\verb{lantanídeo}{}{Quím.}{}{}{s.m.}{Qualquer elemento dos lantanídeos, série de quinze elementos químicos de caráter metálico, dos quais o primeiro é o lantânio, com propriedades muito semelhantes, encontrados na areia monazítica; também se denominam terras"-raras.}{lan.ta.ní.deo}{0}
\verb{lantânio}{}{Quím.}{}{}{s.m.}{Elemento metálico prateado, mole, bastante radioativo, o primeiro da família dos lantanídeos (terras"-raras), a que dá o nome; é  natural e radioativo; usado em certas ligas para pedra de isqueiro e em lâmpadas especiais. \elemento{57}{138.9055}{La}.}{lan.tâ.nio}{0}
\verb{lantejoula}{ô}{}{}{}{s.f.}{Pequena lâmina de material cintilante e com furo no meio, usada para enfeitar roupas e bordados.}{lan.te.jou.la}{0}
\verb{lanterna}{é}{}{}{}{s.f.}{Lampião portátil.}{lan.ter.na}{0}
\verb{lanterna}{é}{}{}{}{}{Aparelho de iluminação com lâmpada elétrica alimentada por pilhas.}{lan.ter.na}{0}
\verb{lanterna}{é}{}{}{}{}{Luz de posição dos veículos.}{lan.ter.na}{0}
%\verb{}{}{}{}{}{}{}{}{0}
\verb{lanterneiro}{ê}{}{}{}{s.m.}{Indivíduo que carrega as lanternas numa procissão.}{lan.ter.nei.ro}{0}
\verb{lanterninha}{}{Esport.}{}{}{s.2g.}{Indivíduo ou clube em último lugar em uma competição.}{lan.ter.ni.nha}{0}
\verb{lanterninha}{}{}{}{}{}{Indivíduo que ajudava os espectadores a encontrar lugares desocupados nos cinemas.}{lan.ter.ni.nha}{0}
\verb{lanudo}{}{}{}{}{adj.}{Lanoso.}{la.nu.do}{0}
\verb{lanugem}{}{}{"-ens}{}{s.f.}{Pelo fino e macio; penugem.}{la.nu.gem}{0}
\verb{laosiano}{}{}{}{}{adj.}{Relativo à República do Laos (sudoeste da Ásia).}{la.o.si.a.no}{0}
\verb{laosiano}{}{}{}{}{s.m.}{Indivíduo natural ou habitante desse país.}{la.o.si.a.no}{0}
\verb{laosiano}{}{}{}{}{}{Língua falada nesse país.}{la.o.si.a.no}{0}
\verb{lapa}{}{}{}{}{s.f.}{Pedra grande e saliente que forma um abrigo.}{la.pa}{0}
\verb{lapa}{}{}{}{}{}{Gruta, caverna.}{la.pa}{0}
\verb{láparo}{}{}{}{}{s.m.}{Filhote de coelho.}{lá.pa.ro}{0}
\verb{laparoscopia}{}{Med.}{}{}{s.f.}{Exame interno da cavidade abdominal feito com auxílio de um laparoscópio.}{la.pa.ros.co.pi.a}{0}
\verb{laparotomia}{}{Med.}{}{}{s.f.}{Abertura cirúrgica da cavidade abdominal.}{la.pa.ro.to.mi.a}{0}
\verb{lapela}{é}{}{}{}{s.f.}{Parte superior e dianteira de casaco ou paletó, dobrada e voltada para fora.}{la.pe.la}{0}
\verb{lapidação}{}{}{"-ões}{}{s.f.}{Ato de traçar as facetas de pedra preciosa.}{la.pi.da.ção}{0}
\verb{lapidação}{}{Fig.}{"-ões}{}{}{Educação, aperfeiçoamento dos modos.}{la.pi.da.ção}{0}
\verb{lapidação}{}{}{"-ões}{}{}{Suplício em que a vítima é morta por apedrejamento.}{la.pi.da.ção}{0}
\verb{lapidar}{}{}{}{}{v.t.}{Traçar as facetas de pedra preciosa.}{la.pi.dar}{0}
\verb{lapidar}{}{}{}{}{adj.2g.}{Relativo a lápide.}{la.pi.dar}{0}
\verb{lapidar}{}{Fig.}{}{}{}{Educar, aperfeiçoar os modos.}{la.pi.dar}{\verboinum{1}}
\verb{lapidar}{}{}{}{}{}{Gravado em pedra.}{la.pi.dar}{0}
\verb{lapidar}{}{}{}{}{}{Modelar, primoroso, perfeito.}{la.pi.dar}{0}
\verb{lapidaria}{}{}{}{}{s.f.}{Técnica de lapidar pedras.}{la.pi.da.ri.a}{0}
\verb{lapidaria}{}{}{}{}{}{Oficina do lapidário.}{la.pi.da.ri.a}{0}
\verb{lapidária}{}{}{}{}{s.f.}{Ramo da paleografia que estuda, data, decifra e interpreta as inscrições e escritas antigas.}{la.pi.dá.ria}{0}
\verb{lapidário}{}{}{}{}{s.m.}{Indivíduo que lapida pedras preciosas.}{la.pi.dá.rio}{0}
\verb{lápide}{}{}{}{}{s.f.}{Pedra que cobre uma sepultura.}{lá.pi.de}{0}
\verb{lápide}{}{}{}{}{}{Pedra com inscrição comemorativa.}{lá.pi.de}{0}
\verb{lapinha}{}{Bras.}{}{}{s.f.}{Presépio ou nicho preparado para festa de Natal ou de Reis.}{la.pi.nha}{0}
\verb{lápis}{}{}{}{}{s.m.}{Vareta de grafite ou substância colorida embutida em madeira cilíndrica que se aponta em uma extremidade e se usa para escrever ou desenhar.}{lá.pis}{0}
\verb{lapisar}{}{}{}{}{v.t.}{Escrever ou desenhar usando lápis.}{la.pi.sar}{\verboinum{1}}
\verb{lapiseira}{ê}{}{}{}{s.f.}{Instrumento de escrita semelhante a uma caneta, mas com bastão de grafite em vez de tinta.}{la.pi.sei.ra}{0}
\verb{lápis"-lazúli}{}{}{lápis"-lazúlis}{}{s.m.}{Mineral constituído de sódio, cálcio e enxofre, muito utilizado na fabricação de objetos ornamentais, colares e pulseiras.}{lá.pis"-la.zú.li}{0}
\verb{lapso}{}{}{}{}{s.m.}{Intervalo de tempo.}{lap.so}{0}
\verb{lapso}{}{}{}{}{}{Erro cometido por descuido ou falha da memória.}{lap.so}{0}
\verb{laptop}{}{Informát.}{}{}{s.m.}{Computador portátil em forma de uma pequena pasta de mão e alimentado por bateria recarregável.}{\textit{laptop}}{0}
\verb{laquê}{}{Bras.}{}{}{s.m.}{Produto que se pulveriza sobre os cabelos para fixar o penteado.}{la.quê}{0}
\verb{laqueação}{}{}{"-ões}{}{s.f.}{Ato ou efeito de laquear.}{la.que.a.ção}{0}
\verb{laqueador}{ô}{}{}{}{adj.}{Que laqueia, enverniza objetos e móveis.}{la.que.a.dor}{0}
\verb{laqueadura}{}{Med.}{}{}{s.f.}{Ato ou efeito de laquear; ligação de vasos do corpo.}{la.que.a.du.ra}{0}
\verb{laquear}{}{Med.}{}{}{v.t.}{Ligar vasos do corpo como veias, artérias, trompas.}{la.que.ar}{\verboinum{4}}
\verb{laquear}{}{Bras.}{}{}{v.t.}{Pintar com laca.}{la.que.ar}{0}
\verb{laquear}{}{Por ext.}{}{}{}{Pintar com verniz ou tinta esmalte.}{la.que.ar}{\verboinum{4}}
\verb{lar}{}{}{}{}{s.m.}{Casa onde habita um grupo de pessoas geralmente unidas por laços de parentesco.}{lar}{0}
\verb{lar}{}{Fig.}{}{}{}{A terra natal; pátria.}{lar}{0}
\verb{lar}{}{}{}{}{}{Lugar na cozinha onde se acende o fogo.}{lar}{0}
\verb{laranja}{}{}{}{}{s.f.}{Fruto da laranjeira.}{la.ran.ja}{0}
\verb{laranja}{}{}{}{}{s.m.}{A cor da laranja; alaranjado.}{la.ran.ja}{0}
\verb{laranja}{}{Pop.}{}{}{}{Indivíduo cujo nome é utilizado em práticas ilícitas para proteger o nome de gente mais poderosa e influente.}{la.ran.ja}{0}
\verb{laranja"-cravo}{}{}{laranjas"-cravo \textit{ou} laranjas"-cravos}{}{s.f.}{Tangerina, mexerica, bergamota.}{la.ran.ja"-cra.vo}{0}
\verb{laranjada}{}{}{}{}{s.f.}{Suco de laranja com água e açúcar.}{la.ran.ja.da}{0}
\verb{laranjal}{}{}{"-ais}{}{s.m.}{Coletivo de laranja.}{la.ran.jal}{0}
\verb{laranjeira}{ê}{Bot.}{}{}{s.f.}{Árvore de copa arredondada, caule com espinhos finos e longos, flores brancas e frutos esféricos comestíveis.}{la.ran.jei.ra}{0}
\verb{laranjeiro}{ê}{Bras.}{}{}{s.m.}{Cultivador ou vendedor de laranjas.}{la.ran.jei.ro}{0}
\verb{larápio}{}{}{}{}{s.m.}{Ladrão, gatuno.}{la.rá.pio}{0}
\verb{lardear}{}{}{}{}{v.t.}{Colocar pedaços de toucinho em peça de carne.}{lar.de.ar}{\verboinum{4}}
\verb{lardo}{}{}{}{}{s.m.}{Toucinho, geralmente em tiras ou pedaços.}{lar.do}{0}
\verb{lareira}{ê}{}{}{}{s.f.}{Local próprio para acender fogo a lenha e aquecer o ambiente.}{la.rei.ra}{0}
\verb{lares}{}{Relig.}{}{}{s.m.pl.}{Deuses domésticos cultuados pelos romanos e etruscos.}{la.res}{0}
\verb{larga}{}{}{}{}{s.f.}{Liberdade, soltura.}{lar.ga}{0}
\verb{larga}{}{}{}{}{loc. adv.}{(\textit{à larga}) Com abundância.}{lar.ga}{0}
\verb{largada}{}{}{}{}{s.f.}{Momento em que se inicia uma corrida; arrancada.}{lar.ga.da}{0}
\verb{largado}{}{}{}{}{adj.}{Abandonado, desprezado.}{lar.ga.do}{0}
\verb{largar}{}{}{}{}{v.t.}{Deixar de segurar; soltar.}{lar.gar}{0}
\verb{largar}{}{}{}{}{}{Abandonar, deixar.}{lar.gar}{\verboinum{5}}
\verb{largo}{}{}{}{}{adj.}{Que tem grande dimensão.}{lar.go}{0}
\verb{largo}{}{}{}{}{}{Que tem grande dimensão na direção perpendicular ao comprimento e horizontal.}{lar.go}{0}
\verb{largo}{}{}{}{}{s.m.}{Local público geralmente de dimensões maiores do que as das vias próximas.}{lar.go}{0}
\verb{largueza}{ê}{}{}{}{s.f.}{Qualidade de largo; largura.}{lar.gue.za}{0}
\verb{largura}{}{}{}{}{s.f.}{Dimensão perpendicular ao comprimento.}{lar.gu.ra}{0}
\verb{laringe}{}{Anat.}{}{}{s.f.}{Cavidade situada na parte superior da traqueia e na qual ficam as cordas vocais.}{la.rin.ge}{0}
\verb{laríngeo}{}{}{}{}{adj.}{Relativo a laringe.}{la.rín.geo}{0}
\verb{laringite}{}{Med.}{}{}{s.f.}{Inflamação da laringe.}{la.rin.gi.te}{0}
\verb{laringologia}{}{Med.}{}{}{s.f.}{Ramo da medicina que estuda a laringe e seus distúrbios.}{la.rin.go.lo.gi.a}{0}
\verb{laringologista}{}{Med.}{}{}{s.2g.}{Especialista em laringologia.}{la.rin.go.lo.gis.ta}{0}
\verb{larva}{}{}{}{}{s.f.}{Forma dos insetos no estágio inicial da vida, logo após sair do ovo.}{lar.va}{0}
\verb{larvado}{}{Med.}{}{}{adj.}{Diz"-se de doença com sintomas incomuns ou sem eles.}{lar.va.do}{0}
\verb{larvado}{}{Pop.}{}{}{s.m.}{Indivíduo louco com intervalos de lucidez.}{lar.va.do}{0}
\verb{larval}{}{}{"-ais}{}{adj.2g.}{Relativo a larva.}{lar.val}{0}
\verb{larvar}{}{}{}{}{adj.2g.}{Relativo a larva; larval.}{lar.var}{0}
\verb{larvário}{}{}{}{}{adj.}{Relativo a larva; larval.}{lar.vá.rio}{0}
\verb{larvário}{}{}{}{}{s.m.}{Ninho de larvas.}{lar.vá.rio}{0}
\verb{lasanha}{}{}{}{}{s.f.}{Massa alimentícia em forma de tiras finas e largas, como folhas.}{la.sa.nha}{0}
\verb{lasanha}{}{Cul.}{}{}{}{Prato preparado com essa massa, geralmente com queijo, presunto, carne e molho.}{la.sa.nha}{0}
\verb{lasca}{}{}{}{}{s.f.}{Fragmento de madeira, pedra ou metal.}{las.ca}{0}
\verb{lascar}{}{}{}{}{v.t.}{Tirar lasca; quebrar, rachar.}{las.car}{0}
\verb{lascar}{}{Pop.}{}{}{}{Aplicar, assentar. (\textit{Lascou um pontapé no juiz.})}{las.car}{\verboinum{2}}
\verb{lascívia}{}{}{}{}{s.f.}{Qualidade de lascivo.}{las.cí.via}{0}
\verb{lascivo}{}{}{}{}{adj.}{Inclinado ao sexo; libidinoso, sensual.}{las.ci.vo}{0}
\verb{laser}{lêiser}{Fís.}{}{}{s.m.}{Fonte luminosa que produz um feixe de luz geralmente de grande intensidade e coerência.}{\textit{laser}}{0}
\verb{lassear}{}{}{}{}{v.t.}{Fazer ficar frouxo; afrouxar, desapertar.}{las.se.ar}{\verboinum{4}}
\verb{lassidão}{}{}{"-ões}{}{s.f.}{Qualidade de lasso.}{las.si.dão}{0}
\verb{lassidão}{}{}{"-ões}{}{}{Fadiga, esgotamento.}{las.si.dão}{0}
\verb{lassitude}{}{}{}{}{s.f.}{Qualidade de lasso; lassidão.}{las.si.tu.de}{0}
\verb{lasso}{}{}{}{}{adj.}{Frouxo, relaxado.}{las.so}{0}
\verb{lasso}{}{}{}{}{}{Fatigado, esgotado, cansado.}{las.so}{0}
\verb{lasso}{}{Fig.}{}{}{}{Devasso, dissoluto.}{las.so}{0}
\verb{lástima}{}{}{}{}{}{Lamentação, queixa.}{lás.ti.ma}{0}
\verb{lástima}{}{}{}{}{s.f.}{Pena, dó.}{lás.ti.ma}{0}
\verb{lástima}{}{Pejor.}{}{}{}{Pessoa ou objeto inútil.}{lás.ti.ma}{0}
\verb{lastimar}{}{}{}{}{v.t.}{Lamentar, deplorar, queixar"-se.}{las.ti.mar}{0}
\verb{lastimar}{}{}{}{}{}{Ter pena; condoer"-se.}{las.ti.mar}{\verboinum{1}}
\verb{lastimável}{}{}{"-eis}{}{adj.2g.}{Lamentável, deplorável.}{las.ti.má.vel}{0}
\verb{lastimoso}{ô}{}{"-osos ⟨ó⟩}{"-osa ⟨ó⟩}{adj.}{Lastimável, lamentável.}{las.ti.mo.so}{0}
\verb{lastimoso}{ô}{}{"-osos ⟨ó⟩}{"-osa ⟨ó⟩}{}{Que se lastima; choroso.}{las.ti.mo.so}{0}
\verb{lastrar}{}{}{}{}{}{Carregar com lastro.}{las.trar}{0}
\verb{lastrar}{}{}{}{}{v.t.}{Pôr lastro.}{las.trar}{0}
\verb{lastrar}{}{}{}{}{}{Aumentar o peso para tornar mais firme.}{las.trar}{\verboinum{1}}
\verb{lastreamento}{}{}{}{}{s.m.}{Ato ou efeito de lastrear.}{las.tre.a.men.to}{0}
\verb{lastrear}{}{}{}{}{v.t.}{Lastrar.}{las.tre.ar}{\verboinum{4}}
\verb{lastro}{}{}{}{}{s.m.}{Conjunto de pesos que ficam no porão do navio para dar"-lhe estabilidade.}{las.tro}{0}
\verb{lastro}{}{Econ.}{}{}{}{Depósito em ouro que serve de garantia ao papel"-moeda.}{las.tro}{0}
\verb{lastro}{}{Fig.}{}{}{}{Fundamento, embasamento.}{las.tro}{0}
\verb{lata}{}{}{}{}{s.f.}{Chapa de metal laminado coberto com camada de estanho; folha"-de"-flandres.}{la.ta}{0}
\verb{lata}{}{}{}{}{}{Recipiente feito com essa chapa.}{la.ta}{0}
\verb{latada}{}{}{}{}{s.f.}{Grade horizontal feita de paus ou varas que serve de suporte para plantas trepadeiras.}{la.ta.da}{0}
\verb{latagão}{}{}{"-ões}{}{s.m.}{Homem alto e forte.}{la.ta.gão}{0}
\verb{latão}{}{}{"-ões}{}{s.m.}{Liga metálica feita de cobre e zinco.}{la.tão}{0}
\verb{lataria}{}{}{}{}{s.f.}{Coletivo de lata.}{la.ta.ri.a}{0}
\verb{lataria}{}{}{}{}{}{Designação comum aos alimentos enlatados.}{la.ta.ri.a}{0}
\verb{lataria}{}{Pop.}{}{}{}{Carroceria do automóvel.}{la.ta.ri.a}{0}
\verb{látego}{}{}{}{}{s.m.}{Corda ou correia para açoitar; açoite, chicote.}{lá.te.go}{0}
\verb{látego}{}{Fig.}{}{}{}{Punição aplicada a pessoa ou animal, com ou sem esse instrumento; castigo.}{lá.te.go}{0}
\verb{latejante}{}{}{}{}{adj.2g.}{Que lateja; pulsante, palpitante.}{la.te.jan.te}{0}
\verb{latejar}{}{}{}{}{v.i.}{Dilatar"-se em intervalos regulares e curtos; pulsar, palpitar.}{la.te.jar}{\verboinum{1}}
\verb{latejo}{ê}{}{}{}{s.m.}{Ato de latejar; pulsação, palpitação.}{la.te.jo}{0}
\verb{latente}{}{}{}{}{adj.2g.}{Não manifesto, não explícito; oculto.}{la.ten.te}{0}
\verb{latente}{}{}{}{}{}{Que está presente mas inativo, suscetível, entretanto, de ativar"-se; potencial.}{la.ten.te}{0}
\verb{láteo}{}{}{}{}{}{Var. de \textit{lácteo}.}{lá.teo}{0}
\verb{lateral}{}{}{"-ais}{}{adj.2g.}{Relativo a lado.}{la.te.ral}{0}
\verb{lateral}{}{}{"-ais}{}{}{Que está do lado.}{la.te.ral}{0}
\verb{lateral}{}{}{"-ais}{}{}{Que está à margem; marginal, secundário.}{la.te.ral}{0}
\verb{látex}{cs}{Bot.}{}{}{s.m.}{Substância líquida, geralmente de cor leitosa ou incolor, extraída do caule ou folhas de alguns vegetais, usada para fabricar resinas e borrachas.}{lá.tex}{0}
\verb{laticínio}{}{}{}{}{s.m.}{Produto comestível cujo principal componente é o leite.}{la.ti.cí.nio}{0}
\verb{latido}{}{}{}{}{s.m.}{A voz do cão.}{la.ti.do}{0}
\verb{latifundiário}{}{}{}{}{adj.}{Relativo a latifúndio.}{la.ti.fun.di.á.rio}{0}
\verb{latifundiário}{}{}{}{}{s.m.}{Proprietário de latifúndio.}{la.ti.fun.di.á.rio}{0}
\verb{latifúndio}{}{}{}{}{s.m.}{Território rural de grandes dimensões, não produtivo ou utilizado para cultura que exige baixo investimento e utiliza mão de obra barata.}{la.ti.fún.dio}{0}
\verb{latim}{}{Gram.}{}{}{s.m.}{Língua falada pelo antigo povo romano, da qual surgiram as chamadas línguas românicas, como português, galego, castelhano, catalão, francês, italiano, romeno.}{la.tim}{0}
\verb{latinismo}{}{Gram.}{}{}{s.m.}{Palavra, locução ou construção própria do latim.}{la.ti.nis.mo}{0}
\verb{latinista}{}{}{}{}{s.2g.}{Indivíduo especializado em língua, literatura ou cultura latina.}{la.ti.nis.ta}{0}
\verb{latinizar}{}{}{}{}{v.t.}{Tornar latino, dar forma latina a palavras de outras línguas.}{la.ti.ni.zar}{0}
\verb{latinizar}{}{}{}{}{}{Submeter à influência da língua, cultura ou religião latina.}{la.ti.ni.zar}{\verboinum{1}}
\verb{latino}{}{}{}{}{adj.}{Relativo ao latim.}{la.ti.no}{0}
\verb{latino}{}{}{}{}{}{Dito ou escrito em latim.}{la.ti.no}{0}
\verb{latino}{}{}{}{}{}{Relativo aos povos de origem latina.}{la.ti.no}{0}
\verb{latino}{}{}{}{}{}{Relativo à igreja de Roma.}{la.ti.no}{0}
\verb{latino}{}{}{}{}{s.m.}{Indivíduo natural ou habitante da região do Lácio.}{la.ti.no}{0}
\verb{latino}{}{}{}{}{}{Indivíduo natural ou habitante de qualquer país latino.}{la.ti.no}{0}
\verb{latino"-americano}{}{}{latino"-americanos}{}{adj.}{Relativo aos países americanos de línguas neolatinas.}{la.ti.no"-a.me.ri.ca.no}{0}
\verb{latino"-americano}{}{}{latino"-americanos}{}{s.m.}{Indivíduo natural ou habitante de um desses países.}{la.ti.no"-a.me.ri.ca.no}{0}
\verb{latir}{}{}{}{}{v.i.}{Dar latidos; ladrar.}{la.tir}{0}
\verb{latir}{}{Fig.}{}{}{}{Gritar.}{la.tir}{\verboinum{34}\verboirregular{\emph{def.} late, latem}} 
\verb{latitude}{}{Geogr.}{}{}{s.f.}{Distância medida em graus na direção norte"-sul entre um ponto qualquer da Terra e o equador.}{la.ti.tu.de}{0}
\verb{latitude}{}{}{}{}{}{Amplitude, largura, largueza.}{la.ti.tu.de}{0}
\verb{latitude}{}{}{}{}{}{Região, clima.}{la.ti.tu.de}{0}
\verb{lato}{}{}{}{}{adj.}{Largo, extenso, amplo, dilatado.}{la.to}{0}
\verb{latoaria}{}{}{}{}{s.f.}{Ofício ou oficina de latoeiro.}{la.to.a.ri.a}{0}
\verb{latoeiro}{ê}{}{}{}{s.m.}{Fabricante ou comerciante de lata ou latão.}{la.to.ei.ro}{0}
\verb{latria}{}{Relig.}{}{}{s.f.}{Culto de adoração a Deus.}{la.tri.a}{0}
\verb{latria}{}{Por ext.}{}{}{}{Amor excessivo, adoração, idolatria.}{la.tri.a}{0}
\verb{latrina}{}{}{}{}{s.f.}{Escavação ou vaso sanitário destinado a dejeções.}{la.tri.na}{0}
\verb{latrina}{}{}{}{}{}{Recinto, cômodo ou local público com essa escavação ou vaso.}{la.tri.na}{0}
\verb{latrocínio}{}{Jur.}{}{}{s.m.}{Roubo associado a homicídio ou a graves lesões corporais.}{la.tro.cí.nio}{0}
\verb{lauda}{}{}{}{}{s.f.}{Página de livro.}{lau.da}{0}
\verb{lauda}{}{}{}{}{s.f.}{Canto ou poema religioso na literatura medieval italiana.}{lau.da}{0}
\verb{lauda}{}{}{}{}{}{Cada lado escrito ou impresso de uma folha de papel.}{lau.da}{0}
\verb{lauda}{}{}{}{}{}{Folha padronizada para redação de matéria jornalística.}{lau.da}{0}
\verb{lauda}{}{}{}{}{}{Unidade correspondente a determinada quantidade de texto.}{lau.da}{0}
\verb{láudano}{}{}{}{}{s.m.}{Medicamento sedativo à base de ópio.}{láu.da.no}{0}
\verb{laudatório}{}{}{}{}{adj.}{Relativo a louvor.}{lau.da.tó.rio}{0}
\verb{laudêmio}{}{}{}{}{s.m.}{Quantia paga pelo foreiro ao senhorio direto quando há alienação do respectivo prédio.}{lau.dê.mio}{0}
\verb{laudo}{}{}{}{}{s.m.}{Parecer técnico, emitido por perito ou especialista, baseado em observações e investigações.}{lau.do}{0}
\verb{laudo}{}{}{}{}{}{Parecer do louvado ou árbitro; louvação.}{lau.do}{0}
\verb{láurea}{}{}{}{}{s.f.}{Coroa de louros; laurel.}{láu.re.a}{0}
\verb{laureado}{}{}{}{}{adj.}{Que recebeu láurea ou laurel.}{lau.re.a.do}{0}
\verb{laureado}{}{}{}{}{}{Louvado, aplaudido.}{lau.re.a.do}{0}
\verb{laurear}{}{}{}{}{v.t.}{Coroar de louros.}{lau.re.ar}{0}
\verb{laurear}{}{}{}{}{}{Premiar por mérito, geralmente artístico.}{lau.re.ar}{0}
\verb{laurear}{}{}{}{}{}{Festejar, aplaudir, louvar.}{lau.re.ar}{0}
\verb{laurear}{}{}{}{}{}{Adornar, enfeitar.}{lau.re.ar}{\verboinum{4}}
\verb{laurel}{é}{}{"-éis}{}{s.m.}{Coroa de louros.}{lau.rel}{0}
\verb{laurel}{é}{Fig.}{"-éis}{}{}{Prêmio, recompensa.}{lau.rel}{0}
\verb{laurêncio}{}{Quím.}{}{}{s.m.}{Elemento químico radioativo, do grupo dos actinídeos, obtido artificialmente. \elemento{103}{(262)}{Lr}.}{lau.rên.cio}{0}
\verb{láureo}{}{}{}{}{adj.}{Relativo a louro (a planta).}{láu.re.o}{0}
\verb{lauto}{}{}{}{}{adj.}{Abundante, suntuoso, farto.}{lau.to}{0}
\verb{lava}{}{}{}{}{s.f.}{Magma que se encontra na cratera do vulcão.}{la.va}{0}
\verb{lava}{}{Fig.}{}{}{}{Enxurrada, torrente.}{la.va}{0}
\verb{lavabo}{}{}{}{}{s.m.}{Pequeno cômodo na entrada da casa, com sanitário e pia para o visitante.}{la.va.bo}{0}
\verb{lavada}{}{Esport.}{}{}{s.f.}{Derrota por muita diferença de pontos ou de nota.}{la.va.da}{0}
\verb{lavada}{}{Pop.}{}{}{}{Repreensão, bronca.}{la.va.da}{0}
\verb{lavada}{}{}{}{}{}{Rede de pesca.}{la.va.da}{0}
\verb{lavadeira}{ê}{}{}{}{s.f.}{Mulher que lava roupa.}{la.va.dei.ra}{0}
\verb{lavadeira}{ê}{}{}{}{}{Máquina para lavagem de roupa.}{la.va.dei.ra}{0}
\verb{lavadeira}{ê}{Zool.}{}{}{}{Libélula.}{la.va.dei.ra}{0}
\verb{lavadela}{é}{}{}{}{s.f.}{Lavagem ligeira e superficial.}{la.va.de.la}{0}
\verb{lavado}{}{}{}{}{adj.}{Que se lavou; limpo, asseado.}{la.va.do}{0}
\verb{lavado}{}{}{}{}{}{Límpido, claro.}{la.va.do}{0}
\verb{lavadora}{ô}{Bras.}{}{}{s.f.}{Máquina de lavar roupas.}{la.va.do.ra}{0}
\verb{lavadouro}{ô}{}{}{}{s.m.}{Local, pedra ou tanque em que se ensaboa e lava roupa.}{la.va.dou.ro}{0}
\verb{lavadura}{}{}{}{}{s.f.}{Ato ou efeito de lavar; lavagem.}{la.va.du.ra}{0}
\verb{lavadura}{}{}{}{}{}{Água com restos de alimentos na qual se lavou a louça, e que é dada como alimento aos porcos.}{la.va.du.ra}{0}
\verb{lavagem}{}{}{"-ens}{}{s.f.}{Ato ou efeito de lavar.}{la.va.gem}{0}
\verb{lavagem}{}{}{"-ens}{}{}{Água com resto de alimentos dada como comida aos porcos.}{la.va.gem}{0}
\verb{lavagem}{}{Med.}{"-ens}{}{}{Irrigação de órgãos.}{la.va.gem}{0}
\verb{lavanda}{}{Bot.}{}{}{s.f.}{Arbusto aromático, do qual se extrai um óleo utilizado na fabricação de perfumes e na medicina; alfazema.}{la.van.da}{0}
\verb{lavanda}{}{}{}{}{}{Cosmético feito com essência dessa planta.}{la.van.da}{0}
\verb{lavandeira}{ê}{}{}{}{s.f.}{Lavadeira.}{la.van.dei.ra}{0}
\verb{lavandeira}{ê}{Zool.}{}{}{}{Certo pássaro branco com a coroa da cabeça avermelhada.}{la.van.dei.ra}{0}
\verb{lavanderia}{}{}{}{}{s.f.}{Estabelecimento em que se lavam roupas.}{la.van.de.ri.a}{0}
\verb{lavanderia}{}{}{}{}{}{Cômodo equipado para a lavagem de roupas.}{la.van.de.ri.a}{0}
\verb{lava"-pés}{}{Relig.}{}{}{s.m.}{Celebração do dia em que Jesus Cristo teria lavado os pés de seus discípulos, na quinta"-feira santa.}{la.va"-pés}{0}
\verb{lavar}{}{}{}{}{v.t.}{Limpar com água, geralmente em abundância.}{la.var}{0}
\verb{lavar}{}{}{}{}{}{Banhar, região ou cidade. (\textit{O riacho lavava as pequenas vilas.})}{la.var}{0}
\verb{lavar}{}{}{}{}{}{Purificar, expurgar.}{la.var}{\verboinum{1}}
\verb{lavatório}{}{}{}{}{s.m.}{Móvel ou peça sanitária própria para lavar as mãos e o rosto.}{la.va.tó.rio}{0}
\verb{lavatório}{}{Fig.}{}{}{}{Purificação, expurgação.}{la.va.tó.rio}{0}
\verb{lavatório}{}{}{}{}{}{Ato de lavar; lavagem.}{la.va.tó.rio}{0}
\verb{lavável}{}{}{"-eis}{}{adj.2g.}{Que pode ser lavado sem danificar"-se.}{la.vá.vel}{0}
\verb{lavoira}{ô}{}{}{}{}{Var. de \textit{lavoura}.}{la.voi.ra}{0}
\verb{lavor}{ô}{}{}{}{s.m.}{Trabalho, especialmente manual, artesanal.}{la.vor}{0}
\verb{lavoura}{ô}{}{}{}{s.f.}{Trabalho de cuidar da terra para plantar e colher; cultivo. }{la.vou.ra}{0}
\verb{lavoura}{ô}{}{}{}{}{Terreno em que se planta para alimentação; plantação.}{la.vou.ra}{0}
\verb{lavra}{}{}{}{}{s.f.}{Ato ou efeito de lavrar.}{la.vra}{0}
\verb{lavra}{}{}{}{}{}{Terreno onde se tira algum mineral; jazida, mina.}{la.vra}{0}
\verb{lavradio}{}{}{}{}{s.m.}{Que é próprio para ser lavrado.}{la.vra.di.o}{0}
\verb{lavrador}{ô}{}{}{}{s.m.}{Trabalhador que se ocupa com a lavoura; agricultor, camponês.}{la.vra.dor}{0}
\verb{lavragem}{}{}{"-ens}{}{s.f.}{Ato ou efeito de lavrar a terra ou madeiras.}{la.vra.gem}{0}
\verb{lavrar}{}{}{}{}{v.t.}{Virar e revirar a terra para plantar; arar.}{la.vrar}{0}
\verb{lavrar}{}{}{}{}{}{Fazer alguma coisa ficar lisa, pronta para ser usada.}{la.vrar}{0}
\verb{lavrar}{}{}{}{}{}{Escrever documento em livro próprio.}{la.vrar}{0}
\verb{lavrar}{}{}{}{}{v.i.}{Estender"-se pouco a pouco para outros lugares; alastrar"-se.}{la.vrar}{\verboinum{1}}
\verb{lavratura}{}{}{}{}{s.f.}{Ato de lavrar um documento, uma escritura, com a intervenção de serventuário e, em certos casos, do juiz.}{la.vra.tu.ra}{0}
\verb{laxante}{ch}{}{}{}{adj.2g.}{Que laxa, afrouxa, dilata.}{la.xan.te}{0}
\verb{laxante}{ch}{}{}{}{s.m.}{Remédio que induz evacuação.}{la.xan.te}{0}
\verb{laxar}{ch}{}{}{}{v.t.}{Tornar frouxo; alargar, relaxar.}{la.xar}{\verboinum{1}}
\verb{laxativo}{ch}{}{}{}{adj.}{Laxante.}{la.xa.ti.vo}{0}
\verb{laxo}{ch}{}{}{}{adj.}{Frouxo; solto; lasso.}{la.xo}{0}
\verb{layout}{}{}{}{}{s.m.}{Esboço de anúncio publicitário que é submetido à aprovação do cliente; leiaute.}{\textit{layout}}{0}
\verb{lazarento}{}{}{}{}{adj.}{Diz"-se de indivíduo que tem uma doença que deixa o corpo cheio de feridas; lázaro, leproso.}{la.za.ren.to}{0}
\verb{lazareto}{ê}{}{}{}{s.m.}{Casa para recolher pessoas suspeitas de doença contagiosa.}{la.za.re.to}{0}
\verb{lazarina}{}{}{}{}{s.f.}{Espingarda de cano fino e longo, para caçar passarinhos.}{la.za.ri.na}{0}
\verb{lázaro}{}{}{}{}{adj.}{Diz"-se de indivíduo que tem lepra, uma doença infecciosa; lazarento, leproso.}{lá.za.ro}{0}
\verb{lazeira}{ê}{}{}{}{s.f.}{Miséria, desgraça.}{la.zei.ra}{0}
\verb{lazeira}{ê}{Fig.}{}{}{}{Fome.}{la.zei.ra}{0}
\verb{lazeirento}{}{Pop.}{}{}{adj.}{Que tem lazeira; esfomeado.}{la.zei.ren.to}{0}
\verb{lazer}{ê}{}{}{}{s.m.}{Tempo livre de compromissos; folga, descanso.}{la.zer}{0}
\verb{lazurita}{}{}{}{}{s.f.}{Silicato de sódio, alumínio e enxofre; lápis"-lazúli.}{la.zu.ri.ta}{0}
\verb{lead}{}{}{}{}{s.m.}{No teatro, papel ou personagem principal.}{\textit{lead}}{0}
\verb{lead}{}{}{}{}{}{Resumo que apresenta os principais tópicos da matéria desenvolvida no texto jornalístico; lide.}{\textit{lead}}{0}
\verb{leal}{}{}{"-ais}{}{adj.2g.}{Que é sincero, honesto.}{le.al}{0}
\verb{leal}{}{}{"-ais}{}{}{Que é fiel aos seus compromissos.}{le.al}{0}
\verb{lealdade}{}{}{}{}{s.f.}{Qualidade de leal.}{le.al.da.de}{0}
\verb{leão}{}{Astron.}{}{}{s.m.}{Quinta constelação zodiacal.}{le.ão}{0}
\verb{leão}{}{Astrol.}{}{}{}{O signo do zodíaco referente a essa constelação.}{le.ão}{0}
\verb{leão}{}{Zool.}{"-ões}{}{s.m.}{Grande mamífero felino selvagem, carnívoro e muito feroz, que habita as estepes e as savanas da África.}{le.ão}{0}
\verb{leão"-de"-chácara}{}{}{leões"-de"-chácara}{}{s.m.}{Indivíduo encarregado da vigilância e manutenção de casas de diversões.}{le.ão"-de"-chá.ca.ra}{0}
\verb{leão"-marinho}{}{Zool.}{leões"-marinhos}{}{s.m.}{Mamífero carnívoro, semelhante à foca, com pequenas orelhas externas, cauda curta e nadadeiras posteriores que auxiliam a locomoção terrestre.}{le.ão"-ma.ri.nho}{0}
\verb{leasing}{}{}{}{}{s.m.}{Contrato que associa aluguel e venda a prestação, por meio de uma técnica especial de financiamento.}{\textit{leasing}}{0}
\verb{lebracho}{}{}{}{}{s.m.}{O macho da lebre, ainda novo.}{le.bra.cho}{0}
\verb{lebrão}{}{}{"-ões}{}{s.m.}{Macho da lebre.}{le.brão}{0}
\verb{lebre}{é}{Zool.}{}{}{s.f.}{Mamífero, da família do coelho, de grande velocidade devido às pernas traseiras muito fortes.}{le.bre}{0}
\verb{lebreiro}{ê}{}{}{}{adj.}{Diz"-se do cão que caça lebres.}{le.brei.ro}{0}
\verb{lebrel}{é}{}{"-éis}{}{s.m.}{Lebréu.}{le.brel}{0}
\verb{lebréu}{}{}{}{}{s.m.}{Cão amestrado na caça das lebres.}{le.bréu}{0}
\verb{lechia}{}{}{}{}{s.f.}{Lichia.}{le.chi.a}{0}
\verb{lecionar}{}{}{}{}{v.t.}{Dar aula de alguma coisa a alguém; ensinar. (\textit{O professor leciona Matemática em várias escolas.})}{le.ci.o.nar}{\verboinum{1}}
\verb{lecitina}{}{}{}{}{s.f.}{Substância gordurosa encontrada nas células do cérebro e da medula dos animais.}{le.ci.ti.na}{0}
\verb{ledo}{ê}{}{}{}{adj.}{Que revela ou sente alegria, júbilo; contente, risonho.}{le.do}{0}
\verb{ledor}{ô}{}{}{}{s.f.}{Diz"-se daquele que lê; leitor.}{le.dor}{0}
\verb{legação}{}{}{"-ões}{}{s.f.}{Ato de legar.}{le.ga.ção}{0}
\verb{legação}{}{}{"-ões}{}{}{Representação de um Estado em outro em que não tem embaixada.}{le.ga.ção}{0}
\verb{legação}{}{}{"-ões}{}{}{Prédio dessa representação.}{le.ga.ção}{0}
\verb{legado}{}{}{}{}{s.m.}{Herança.}{le.ga.do}{0}
\verb{legado}{}{}{}{}{}{Enviado do governo.}{le.ga.do}{0}
\verb{legado}{}{}{}{}{}{Representante do Papa.}{le.ga.do}{0}
\verb{legal}{}{}{"-ais}{}{adj.2g.}{Que é feito conforme a lei.}{le.gal}{0}
\verb{legal}{}{Pop.}{"-ais}{}{}{Apreciado por suas qualidades; bacana.}{le.gal}{0}
\verb{legalidade}{}{}{}{}{s.f.}{Característica do que se faz dentro da lei.}{le.ga.li.da.de}{0}
\verb{legalismo}{}{}{}{}{s.m.}{Respeito à lei vigente.}{le.ga.lis.mo}{0}
\verb{legalismo}{}{}{}{}{}{Apego exagerado e não crítico às normas legais.}{le.ga.lis.mo}{0}
\verb{legalista}{}{}{}{}{adj.2g.}{Relativo à lei, às normas legais.}{le.ga.lis.ta}{0}
\verb{legalista}{}{}{}{}{}{Diz"-se daquele que observa rigorosamente as leis.}{le.ga.lis.ta}{0}
\verb{legalista}{}{}{}{}{}{Diz"-se daquele que, especialmente em revolução, apoia o governo legal e combate por ele.}{le.ga.lis.ta}{0}
\verb{legalização}{}{}{"-ões}{}{s.f.}{Regularização conforme a lei.}{le.ga.li.za.ção}{0}
\verb{legalizar}{}{}{}{}{v.t.}{Fazer algo ficar de acordo com a lei.}{le.ga.li.zar}{\verboinum{1}}
\verb{legar}{}{}{}{}{v.t.}{Deixar como legado; transmitir.}{le.gar}{\verboinum{5}}
\verb{legatário}{}{}{}{}{s.m.}{Pessoa a quem se deixou um legado.}{le.ga.tá.rio}{0}
\verb{legenda}{}{}{}{}{s.f.}{Frase explicativa que acompanha uma ilustração, uma gravura.}{le.gen.da}{0}
\verb{legenda}{}{}{}{}{}{Letreiro de filme.}{le.gen.da}{0}
\verb{legenda}{}{}{}{}{}{Letras iniciais que identificam o nome de um partido político.}{le.gen.da}{0}
\verb{legendado}{}{}{}{}{adj.}{Diz"-se de filme que possui legenda.}{le.gen.da.do}{0}
\verb{legendar}{}{}{}{}{v.t.}{Pôr legenda; criar legenda.}{le.gen.dar}{\verboinum{1}}
\verb{legendário}{}{}{}{}{adj.}{Que se transformou em uma lenda; fabuloso, lendário.}{le.gen.dá.rio}{0}
\verb{legging}{}{}{}{}{s.m.}{Calça de malha, justa e comprida.}{\textit{legging}}{0}
\verb{legião}{}{}{"-ões}{}{s.f.}{Tropa de soldados do antigo exército romano.}{le.gi.ão}{0}
\verb{legião}{}{}{"-ões}{}{}{Grande quantidade de pessoas.}{le.gi.ão}{0}
\verb{legionário}{}{}{}{}{adj.}{Relativo à legião.}{le.gi.o.ná.rio}{0}
\verb{legionário}{}{}{}{}{s.m.}{Soldado legionário.}{le.gi.o.ná.rio}{0}
\verb{legislação}{}{}{"-ões}{}{s.f.}{Conjunto de leis sobre determinado assunto.}{le.gis.la.ção}{0}
\verb{legislador}{ô}{}{}{}{adj.}{Diz"-se de autor de leis.}{le.gis.la.dor}{0}
\verb{legislar}{}{}{}{}{v.i.}{Estabelecer ou fazer leis.}{le.gis.lar}{\verboinum{1}}
\verb{legislativo}{}{}{}{}{adj.}{Que legisla, que faz leis.}{le.gis.la.ti.vo}{0}
\verb{legislativo}{}{}{}{}{s.m.}{Poder da república encarregado de elaborar as leis.}{le.gis.la.ti.vo}{0}
\verb{legislatura}{}{}{}{}{s.f.}{Conjunto de deputados e senadores numa sessão.}{le.gis.la.tu.ra}{0}
\verb{legislatura}{}{}{}{}{}{Tempo em que deputados e senadores continuam no cargo.}{le.gis.la.tu.ra}{0}
\verb{legisperito}{}{Jur.}{}{}{s.m.}{Indivíduo perito em leis; jurisconsulto.}{le.gis.pe.ri.to}{0}
\verb{legista}{}{Jur.}{}{}{s.2g.}{Indivíduo que estuda e conhece as leis; jurisconsulto.}{le.gis.ta}{0}
\verb{legista}{}{Med.}{}{}{}{Médico especialista em medicina legal.}{le.gis.ta}{0}
\verb{legítima}{}{}{}{}{s.f.}{Parte da herança reservada por lei aos herdeiros necessários.}{le.gí.ti.ma}{0}
\verb{legitimação}{}{}{"-ões}{}{s.f.}{Ato ou efeito de legitimar; legalização.}{le.gi.ti.ma.ção}{0}
\verb{legitimar}{}{}{}{}{v.t.}{Reconhecer como legítimo para todos os efeitos da lei; legalizar.}{le.gi.ti.mar}{\verboinum{1}}
\verb{legitimidade}{}{}{}{}{s.f.}{Qualidade ou caráter do que é legítimo; legalidade.}{le.gi.ti.mi.da.de}{0}
\verb{legítimo}{}{}{}{}{adj.}{Que tem caráter ou força de lei; legal.}{le.gí.ti.mo}{0}
\verb{legítimo}{}{}{}{}{}{Fundado na razão, no direito ou na justiça.}{le.gí.ti.mo}{0}
\verb{legítimo}{}{}{}{}{}{Genuíno, autêntico.}{le.gí.ti.mo}{0}
\verb{legível}{}{}{"-eis}{}{adj.2g.}{Que se pode ler; escrito em caligrafia nítida e clara.}{le.gí.vel}{0}
\verb{legorne}{ó}{Zool.}{}{}{s.2g.}{Raça de galinha poedeira originária de Livorno, na Itália.}{le.gor.ne}{0}
\verb{légua}{}{}{}{}{s.f.}{Medida itinerária equivalente a 6000 metros.}{lé.gu.a}{0}
\verb{leguelhé}{}{}{}{}{s.m.}{Indivíduo sem importância; joão"-ninguém.}{le.gue.lhé}{0}
\verb{legume}{}{}{}{}{s.m.}{Fruto seco, típico das leguminosas, que se abre por duas fendas, como o feijão, a ervilha, o amendoim etc.; vagem. }{le.gu.me}{0}
\verb{legume}{}{Bras.}{}{}{}{Qualquer hortaliça; verdura.}{le.gu.me}{0}
\verb{leguminosa}{ó}{Bot.}{}{}{s.f.}{Espécime das leguminosas, família de plantas dicotiledôneas de grande importância na alimentação, cuja principal característica é ser o fruto uma vagem, e cujos representantes mais comuns são o feijão, o amendoim e a ervilha.}{le.gu.mi.no.sa}{0}
\verb{leguminoso}{ô}{}{"-osos ⟨ó⟩}{"-osa ⟨ó⟩}{adj.}{Que frutifica em vagem ou legume.}{le.gu.mi.no.so}{0}
\verb{lei}{ê}{}{}{}{s.m.}{Regra, prescrição escrita que emana do Poder Legislativo e impõe a todos os indivíduos a obrigação de submeter"-se a ela.}{lei}{0}
\verb{lei}{ê}{}{}{}{}{Conjunto das condições necessárias que determinam os fenômenos. (\textit{Uma das leis mais conhecidas da Física é a da gravidade.})}{lei}{0}
\verb{lei}{ê}{}{}{}{}{Obrigação imposta pela consciência e pela sociedade.}{lei}{0}
\verb{lei}{ê}{}{}{}{}{Norma, regra.}{lei}{0}
\verb{leiaute}{}{}{}{}{s.m.}{Esboço de anúncio publicitário que é submetido à aprovação do cliente; \textit{layout}.}{lei.au.te}{0}
\verb{leiautista}{}{}{}{}{s.2g.}{Indivíduo que faz leiautes.}{lei.au.tis.ta}{0}
\verb{leicenço}{}{Med.}{}{}{s.m.}{Nódulo dolorido formado na pele por inflamação; furúnculo.}{lei.cen.ço}{0}
\verb{leigo}{ê}{}{}{}{adj.}{Que não pertence a ordens religiosas; laico.}{lei.go}{0}
\verb{leigo}{ê}{}{}{}{}{Estranho ou alheio a um assunto.}{lei.go}{0}
\verb{leilão}{}{}{"-ões}{}{s.m.}{Venda pública a quem oferecer maior lance; hasta. (\textit{Haverá um leilão de bens confiscados pelo governo.})}{lei.lão}{0}
\verb{leiloar}{}{}{}{}{v.t.}{Pôr ou apregoar em leilão.}{lei.lo.ar}{\verboinum{7}}
\verb{leiloeiro}{ê}{}{}{}{s.m.}{Organizador ou pregoeiro de leilões.}{lei.lo.ei.ro}{0}
\verb{leira}{ê}{}{}{}{s.f.}{Sulco na terra para se deitar a semente durante o cultivo.}{lei.ra}{0}
\verb{leirar}{}{}{}{}{v.t.}{Abrir sulcos na terra; arar, sulcar.}{lei.rar}{\verboinum{1}}
\verb{leishmaniose}{ó}{Med.}{}{}{s.f.}{Doença infecciosa causada por determinados protozoários, e cuja forma cutânea provoca ulcerações no nariz, no septo nasal e no palato.}{leish.ma.ni.o.se}{0}
\verb{leitão}{}{Zool.}{"-ões}{leitoa}{s.m.}{Porco pequeno, quando ainda está sendo amamentado; bácoro.}{lei.tão}{0}
\verb{leitaria}{}{}{}{}{}{Var. de \textit{leiteria}.}{lei.ta.ri.a}{0}
\verb{leite}{}{}{}{}{s.m.}{Líquido branco, opaco, segregado pelas glândulas mamárias da fêmea dos mamíferos.}{lei.te}{0}
\verb{leite}{}{}{}{}{}{Suco branco de alguns vegetais.}{lei.te}{0}
\verb{leite}{}{}{}{}{}{Qualquer líquido leitoso.}{lei.te}{0}
\verb{leiteira}{ê}{}{}{}{s.f.}{Recipiente próprio para leite que se leva à mesa. }{lei.tei.ra}{0}
\verb{leiteira}{ê}{}{}{}{}{Mulher que vende leite.}{lei.tei.ra}{0}
\verb{leiteiro}{ê}{}{}{}{adj.}{Que produz leite. (\textit{Minas Gerais possui o maior rebanho de gado leiteiro do país.})}{lei.tei.ro}{0}
\verb{leiteiro}{ê}{}{}{}{s.m.}{Vendedor ou entregador de leite.}{lei.tei.ro}{0}
\verb{leitelho}{ê}{}{}{}{s.m.}{Líquido desnatado que sobra na batedura da nata no fabrico da manteiga, muito usado sob a forma de leite em pó.}{lei.te.lho}{0}
\verb{leiteria}{}{}{}{}{s.f.}{Estabelecimento comercial especializado na venda de leite e de laticínios.}{lei.te.ri.a}{0}
\verb{leito}{ê}{}{}{}{s.m.}{Armação de madeira ou ferro que sustenta o colchão; cama.}{lei.to}{0}
\verb{leito}{ê}{}{}{}{}{Canal escavado por um rio por onde corre a água. (\textit{Alguns minérios são extraídos do leito do rio.})}{lei.to}{0}
\verb{leitoa}{}{}{}{}{s.f.}{A fêmea do leitão.}{lei.to.a}{0}
\verb{leitor}{ô}{}{}{}{adj.}{Que lê.}{lei.tor}{0}
\verb{leitor}{ô}{}{}{}{s.m.}{Pessoa que lê.}{lei.tor}{0}
\verb{leitor}{ô}{}{}{}{}{Pessoa que tem a ocupação de ler em voz alta para outros ouvirem.}{lei.tor}{0}
\verb{leitoso}{ô}{}{"-osos ⟨ó⟩}{"-osa ⟨ó⟩}{adj.}{Semelhante ao leite na cor ou no aspecto; lácteo.}{lei.to.so}{0}
\verb{leitura}{}{}{}{}{s.f.}{Ato ou efeito de ler.}{lei.tu.ra}{0}
\verb{leitura}{}{}{}{}{}{Hábito ou gosto de ler. (\textit{A leitura é uma atividade que estimula a imaginação.})}{lei.tu.ra}{0}
\verb{leitura}{}{}{}{}{}{Aquilo que se lê; texto, livro.}{lei.tu.ra}{0}
\verb{leiva}{ê}{}{}{}{s.f.}{Elevação de terra entre sulcos; arado.}{lei.va}{0}
\verb{leiva}{ê}{}{}{}{}{Terra adequada para cultivo; gleba.}{lei.va}{0}
\verb{lelé}{}{Pop.}{}{}{adj.2g.}{Que age de forma insensata; doido, biruta.}{le.lé}{0}
\verb{lema}{}{}{}{}{s.m.}{Sentença que se toma por norma de comportamento. (\textit{O lema dos escoteiros é "sempre alerta".})}{le.ma}{0}
\verb{lema}{}{}{}{}{}{Divisa, emblema, preceito.}{le.ma}{0}
\verb{lembrança}{}{}{}{}{s.f.}{Ato ou efeito de lembrar.}{lem.bran.ça}{0}
\verb{lembrança}{}{}{}{}{}{Recordação de fatos passados que se conserva na memória. (\textit{Ainda guardo a lembrança da conquista da copa de 1970.})}{lem.bran.ça}{0}
\verb{lembrança}{}{}{}{}{}{Ideia, inspiração, pensamento.}{lem.bran.ça}{0}
\verb{lembrança}{}{}{}{}{}{Presente, brinde, mimo.}{lem.bran.ça}{0}
\verb{lembrar}{}{}{}{}{v.t.}{Trazer à memória; recordar.}{lem.brar}{0}
\verb{lembrar}{}{}{}{}{}{Dar a ideia; sugerir, propor.}{lem.brar}{0}
\verb{lembrar}{}{}{}{}{}{Informar alguém acerca de algo; advertir, prevenir.}{lem.brar}{0}
\verb{lembrar}{}{}{}{}{}{Mandar lembranças; recomendar.}{lem.brar}{\verboinum{1}}
\verb{lembrete}{ê}{}{}{}{s.m.}{Nota para auxiliar a memória; apontamento.}{lem.bre.te}{0}
\verb{lembrete}{ê}{}{}{}{}{Censura leve; admoestação, repreensão. }{lem.bre.te}{0}
\verb{leme}{}{}{}{}{s.m.}{Aparelho localizado na popa de barco ou na cauda do avião e que serve para lhes dar direção.}{le.me}{0}
\verb{leme}{}{}{}{}{}{Direção, governo.}{le.me}{0}
\verb{lêmure}{}{Zool.}{}{}{s.m.}{Nome comum a numerosos mamíferos primatas, da família dos lemurídeos, de cauda longa e espessa, que gostam de viver em árvores e têm hábitos noturnos.}{lê.mu.re}{0}
\verb{lemurídeo}{}{Zool.}{}{}{s.m.}{Família de primatas herbívoros, que vivem em árvores e habitam Madagascar.}{le.mu.rí.deo}{0}
\verb{lenço}{}{}{}{}{s.m.}{Pedaço de pano, em forma de quadrado, usado para assoar o nariz e abrigar ou ornar a cabeça ou o pescoço. (\textit{A moça deixou cair o lenço na poça de água.})}{len.ço}{0}
\verb{lençol}{ó}{}{lençóis}{}{s.m.}{Peça de tecido usada para forrar a cama ou servir de coberta.}{len.çol}{0}
\verb{lençol}{ó}{}{lençóis}{}{}{Depósito natural de água, petróleo etc. no subsolo.}{len.çol}{0}
\verb{lenda}{}{}{}{}{s.f.}{Narrativa de tradição oral, de caráter maravilhoso, envolvendo fatos históricos, que não se pode provar se são autênticos.}{len.da}{0}
\verb{lenda}{}{}{}{}{}{História fantástica, imaginosa.}{len.da}{0}
\verb{lendário}{}{}{}{}{adj.}{Relativo a lenda.}{len.dá.rio}{0}
\verb{lendário}{}{}{}{}{}{Popularizado pela tradição.}{len.dá.rio}{0}
\verb{lêndea}{}{Zool.}{}{}{s.f.}{Ovo do piolho que se adere à base dos cabelos ou pelos.}{lên.dea}{0}
\verb{lene}{ê}{}{}{}{adj.2g.}{Marcado pela brandura; suave, macio, agradável.}{le.ne}{0}
\verb{lengalenga}{}{}{}{}{s.f.}{Conversa comprida e sem interesse; lero"-lero. (\textit{A lengalenga do candidato cansou a plateia.})}{len.ga.len.ga}{0}
\verb{lenha}{}{}{}{}{s.f.}{Porção de fragmentos de troncos de árvores, galhos, achas, usados como combustível.}{le.nha}{0}
\verb{lenhador}{ô}{}{}{}{s.m.}{Indivíduo que racha ou corta lenha.}{le.nha.dor}{0}
\verb{lenhar}{}{}{}{}{v.i.}{Cortar ou rachar lenha.}{le.nhar}{\verboinum{1}}
\verb{lenheiro}{ê}{}{}{}{s.m.}{Indivíduo que corta ou negocia lenha.}{le.nhei.ro}{0}
\verb{lenheiro}{ê}{}{}{}{}{Local onde se empilha e guarda lenha.}{le.nhei.ro}{0}
\verb{lenho}{}{}{}{}{s.m.}{Tronco de árvore cortado e limpo de galhos e folhas; madeiro.}{le.nho}{0}
\verb{lenhoso}{ô}{}{"-osos ⟨ó⟩}{"-osa ⟨ó⟩}{adj.}{Que tem a natureza, o aspecto ou a consistência do lenho.}{le.nho.so}{0}
\verb{lenimento}{}{}{}{}{s.m.}{Aquilo que suaviza, abranda, acalma; lenitivo, sedativo.}{le.ni.men.to}{0}
\verb{lenir}{}{}{}{}{v.t.}{Tornar mais fácil de suportar; mitigar, abrandar, suavizar.}{le.nir}{\verboinum{35}}
\verb{lenitivo}{}{}{}{}{adj.}{Que serve para abrandar, mitigar, suavizar.}{le.ni.ti.vo}{0}
\verb{lenitivo}{}{Fig.}{}{}{s.m.}{Alívio, consolação. (\textit{Seu sorriso era um lenitivo para o sofrimento do irmão.})}{le.ni.ti.vo}{0}
\verb{lenocínio}{}{}{}{}{s.m.}{Crime que consiste em favorecer a prostituição ou a corrupção.}{le.no.cí.nio}{0}
\verb{lente}{}{}{}{}{s.f.}{Corpo de vidro usado em instrumentos ópticos, óculos etc. para modificar a direção dos raios luminosos, aumentando ou diminuindo aparentemente o tamanho dos objetos vistos através dele.}{len.te}{0}
\verb{lente}{}{}{}{}{s.2g.}{Professor de universidade ou de escola secundária.}{len.te}{0}
\verb{lentejoula}{ô}{}{}{}{}{Var. de \textit{lantejoula}.}{len.te.jou.la}{0}
\verb{lentícula}{}{}{}{}{s.f.}{Pequena lente.}{len.tí.cu.la}{0}
\verb{lentidão}{}{}{"-ões}{}{s.f.}{Qualidade de lento; vagarosidade, demora.}{len.ti.dão}{0}
\verb{lentilha}{}{Bot.}{}{}{s.f.}{Planta trepadeira da família das leguminosas, com semente comestível.}{len.ti.lha}{0}
\verb{lentilha}{}{}{}{}{}{A semente dessa planta.}{len.ti.lha}{0}
\verb{lento}{}{}{}{}{adj.}{A que falta rapidez; moroso, vagaroso.}{len.to}{0}
\verb{lento}{}{}{}{}{}{De efeito progressivo; prolongado, demorado.}{len.to}{0}
\verb{leoa}{ô}{Zool.}{}{}{s.f.}{A fêmea do leão.}{le.o.a}{0}
\verb{leonês}{}{}{}{}{adj.}{Relativo ao antigo reino ou à atual província de Leão, na Espanha.}{le.o.nês}{0}
\verb{leonês}{}{}{}{}{s.m.}{Indivíduo natural ou habitante dessa província.}{le.o.nês}{0}
\verb{leonino}{}{}{}{}{adj.}{Relativo a ou próprio de leão.}{le.o.ni.no}{0}
\verb{leonino}{}{Astrol.}{}{}{s.m.}{Indivíduo que nasceu sob o signo de leão.}{le.o.ni.no}{0}
\verb{leonino}{}{Astrol.}{}{}{adj.}{Relativo ou pertencente a esse signo.}{le.o.ni.no}{0}
\verb{leopardo}{}{Zool.}{}{}{s.m.}{Mamífero felino de grande porte, de hábitos noturnos, de pelagem amarelada com manchas negras; pantera.}{le.o.par.do}{0}
\verb{lépido}{}{}{}{}{adj.}{Que apresenta agilidade; ligeiro, expedito.}{lé.pi.do}{0}
\verb{lépido}{}{}{}{}{}{Que denota jovialidade; alegre, risonho.}{lé.pi.do}{0}
\verb{lepidóptero}{}{Zool.}{}{}{s.m.}{Ordem de insetos, que reúne as borboletas e mariposas, cuja larva é a lagarta e a ninfa, a crisálida.}{le.pi.dóp.te.ro}{0}
\verb{leporídeo}{}{Zool.}{}{}{s.m.}{Família de mamíferos à qual pertencem a lebre e o coelho.}{le.po.rí.deo}{0}
\verb{leporino}{}{}{}{}{adj.}{Relativo à lebre.}{le.po.ri.no}{0}
\verb{lepra}{é}{Med.}{}{}{s.f.}{Doença infecciosa que ataca a pele e os nervos, de caráter crônico e contagioso.}{le.pra}{0}
\verb{leprosário}{}{}{}{}{s.m.}{Hospital destinado ao tratamento de pessoas com lepra.}{le.pro.sá.rio}{0}
\verb{leproso}{ô}{}{"-osos ⟨ó⟩}{"-osa ⟨ó⟩}{adj.}{Que sofre de lepra.}{le.pro.so}{0}
\verb{leptospirose}{ó}{Med.}{}{}{s.f.}{Doença transmitida pela urina do rato.}{lep.tos.pi.ro.se}{0}
\verb{leque}{é}{}{}{}{s.m.}{Conjunto de varetas cobertas por pano ou papel e presas numa ponta para abrir ou fechar, que serve para abanar.}{le.que}{0}
\verb{ler}{ê}{}{}{}{v.t.}{Percorrer com a vista algo escrito, interpretando"-o por uma relação estabelecida entre as sequências dos sinais gráficos escritos e os sinais linguísticos próprios de uma língua natural.       }{ler}{0}
\verb{ler}{ê}{}{}{}{}{Adivinhar, predizer.}{ler}{0}
\verb{ler}{ê}{}{}{}{}{Identificar, notar, perceber.}{ler}{\verboinum{12}}
\verb{lerdeza}{ê}{}{}{}{s.f.}{Qualidade ou condição de lerdo, característica do que apresenta movimentos retardados ou lentos.}{ler.de.za}{0}
\verb{lerdo}{é}{}{}{}{adj.}{Que se move com dificuldade; lento, vagaroso.}{ler.do}{0}
\verb{lerdo}{é}{}{}{}{}{Que se mostra tolo; estúpido.}{ler.do}{0}
\verb{léria}{}{}{}{}{s.f.}{Fala astuciosa; falácia, lábia.}{lé.ria}{0}
\verb{léria}{}{}{}{}{}{Conversa mole.}{lé.ria}{0}
\verb{léria}{}{}{}{}{s.2g.}{Indivíduo que fala muito, mas não faz nada que preste.}{lé.ria}{0}
\verb{lero"-lero}{é\ldots{}é}{}{lero"-leros ⟨é\ldots{}é⟩}{}{s.m.}{Conversa comprida e sem interesse; lengalenga, conversa fiada. (\textit{Em vez de trabalhar, a garota ficava de lero"-lero com a amiga.})}{le.ro"-le.ro}{0}
\verb{lesado}{}{}{}{}{adj.}{Que sofreu lesão.}{le.sa.do}{0}
\verb{lesado}{}{}{}{}{}{Ferido ou prejudicado física ou moralmente.}{le.sa.do}{0}
\verb{lesão}{}{}{"-ões}{}{s.f.}{Ferimento ou traumatismo.}{le.são}{0}
\verb{lesão}{}{}{"-ões}{}{}{Dano, prejuízo.}{le.são}{0}
\verb{lesa"-pátria}{é}{}{lesas"-pátrias ⟨é⟩}{}{s.f.}{Ofensa grave à pátria.}{le.sa"-pá.tria}{0}
\verb{lesar}{}{}{}{}{v.t.}{Causar lesão física; ferir, contundir.}{le.sar}{0}
\verb{lesar}{}{}{}{}{}{Fazer sofrer alguma perda; prejudicar.}{le.sar}{0}
\verb{lesar}{}{}{}{}{}{Fraudar, roubar.}{le.sar}{\verboinum{1}}
\verb{lesbianismo}{}{}{}{}{s.m.}{Atração ou prática sexual entre mulheres.}{les.bi.a.nis.mo}{0}
\verb{lésbica}{}{}{}{}{s.f.}{Mulher que se sente atraída física, emocional e espiritualmente por outra mulher.}{lés.bi.ca}{0}
\verb{lésbico}{}{}{}{}{adj.}{Relativo ao lesbianismo.}{lés.bi.co}{0}
\verb{lesivo}{}{}{}{}{adj.}{Que provoca lesão física ou moral.}{le.si.vo}{0}
\verb{lesivo}{}{}{}{}{}{Que prejudica, que danifica.}{le.si.vo}{0}
\verb{lesma}{ê}{Zool.}{}{}{s.f.}{Nome comum a diversos moluscos terrestres, alguns providos de concha muito reduzida e oculta sob o manto, outros não; vivem em lugares úmidos e alimentam"-se de vegetais.}{les.ma}{0}
\verb{lesma}{ê}{Pop.}{}{}{s.2g.}{Indivíduo muito lento, vagaroso.}{les.ma}{0}
\verb{lés"-nordeste}{é}{}{lés"-nordestes ⟨é⟩}{}{s.m.}{Vento ou rumo de entre o leste e o nordeste; és"-nordeste.}{lés"-nor.des.te}{0}
\verb{leso}{é}{}{}{}{adj.}{Ferido ou prejudicado física ou moralmente.}{le.so}{0}
\verb{leso}{é}{}{}{}{}{Aquele que é idiota, amalucado, lesado.}{le.so}{0}
\verb{lés"-sueste}{é}{}{lés"-suestes ⟨é⟩}{}{s.m.}{Vento ou direção de entre o leste e o sueste; és"-sueste.}{lés"-su.es.te}{0}
\verb{leste}{é}{}{}{}{s.m.}{Região que fica à direita de quem olha para o norte, e onde, na esfera celeste, nascem os astros.}{les.te}{0}
\verb{leste}{é}{}{}{}{adj.2g.}{Diz"-se de região ou de conjunto de regiões que se situa a leste.}{les.te}{0}
\verb{lesto}{é}{}{}{}{adj.}{Que mostra agilidade, ligeireza, velocidade; ágil, rápido.}{les.to}{0}
\verb{lesto}{é}{}{}{}{}{Esperto.}{les.to}{0}
\verb{lesto}{é}{}{}{}{adv.}{De maneira lesta; agilmente, rapidamente.}{les.to}{0}
\verb{letal}{}{}{"-ais}{}{adj.2g.}{Que produz morte; mortal, fatal.}{le.tal}{0}
\verb{letal}{}{}{"-ais}{}{}{Relativo à morte; lúgubre.}{le.tal}{0}
\verb{letalidade}{}{}{}{}{s.f.}{Qualidade do que causa a morte.}{le.ta.li.da.de}{0}
\verb{letalidade}{}{}{}{}{}{Número de óbitos; mortalidade.}{le.ta.li.da.de}{0}
\verb{letão}{}{}{"-ões}{}{adj.}{Relativo à República da Letônia (Europa).}{le.tão}{0}
\verb{letão}{}{}{"-ões}{}{s.m.}{Língua falada nessa república.}{le.tão}{0}
\verb{letão}{}{}{"-ões}{}{}{Indivíduo natural ou habitante dessa república.}{le.tão}{0}
\verb{letargia}{}{}{}{}{s.f.}{Estado de profunda e prolongada inconsciência, semelhante ao sono profundo.}{le.tar.gi.a}{0}
\verb{letargia}{}{}{}{}{}{Incapacidade de reagir e de expressar emoções; apatia.}{le.tar.gi.a}{0}
\verb{letárgico}{}{}{}{}{adj.}{Relativo ou próprio de letargia.}{le.tár.gi.co}{0}
\verb{letárgico}{}{}{}{}{}{Que sofre de letargia.}{le.tár.gi.co}{0}
\verb{letárgico}{}{Fig.}{}{}{}{Que se mostra desanimado, esmorecido.}{le.tár.gi.co}{0}
\verb{letargo}{}{}{}{}{s.m.}{Letargia.}{le.tar.go}{0}
\verb{letivo}{}{}{}{}{adj.}{Relativo às atividades escolares.}{le.ti.vo}{0}
\verb{letra}{ê}{}{}{}{s.f.}{Cada um dos sinais gráficos que são usados para escrever a palavra.}{le.tra}{0}
\verb{letra}{ê}{}{}{}{}{Conjunto de palavras, geralmente em versos, que compõem o texto de uma canção.}{le.tra}{0}
\verb{letrado}{}{}{}{}{adj.}{Que possui cultura, erudição; instruído.}{le.tra.do}{0}
\verb{letrado}{}{}{}{}{s.m.}{Indivíduo de grande conhecimento jurídico; advogado.}{le.tra.do}{0}
\verb{letreiro}{ê}{}{}{}{s.m.}{Anúncio escrito com letras grandes, que serve aos mais diversos tipos de informação.}{le.trei.ro}{0}
\verb{letreiro}{ê}{}{}{}{}{Frase que aparece em cena de filme e serve para esclarecer o espectador sobre as situações vividas pelos personagens; legenda.}{le.trei.ro}{0}
\verb{letrista}{}{}{}{}{adj.2g.}{Diz"-se de artista gráfico especialista em desenho de letras. }{le.tris.ta}{0}
\verb{letrista}{}{}{}{}{}{Diz"-se de autor de letras musicais.}{le.tris.ta}{0}
\verb{léu}{}{Pop.}{}{}{s.m.}{Inação, ociosidade.}{léu}{0}
\verb{léu}{}{Pop.}{}{}{}{Ocasião, oportunidade.}{léu}{0}
\verb{leucemia}{}{Med.}{}{}{s.f.}{Doença que se caracteriza pela proliferação descontrolada, isto é, cancerosa, de células precursoras dos glóbulos brancos normais na medula óssea e no sangue.}{leu.ce.mi.a}{0}
\verb{leucêmico}{}{}{}{}{adj.}{Relativo à leucemia.}{leu.cê.mi.co}{0}
\verb{leucêmico}{}{}{}{}{}{Que tem leucemia.}{leu.cê.mi.co}{0}
\verb{leucócito}{}{}{}{}{s.m.}{Glóbulo branco do sangue, que participa nos processos de defesa imunitária do organismo.}{leu.có.ci.to}{0}
\verb{leucocitose}{ó}{Med.}{}{}{s.f.}{Aumento da taxa de leucócitos no sangue.}{leu.co.ci.to.se}{0}
\verb{leva}{é}{}{}{}{s.f.}{Alistamento de tropa; recrutamento.}{le.va}{0}
\verb{leva}{é}{}{}{}{}{Grupo, conjunto.}{le.va}{0}
\verb{levada}{}{}{}{}{s.f.}{Ato ou efeito de levar.}{le.va.da}{0}
\verb{levada}{}{}{}{}{}{Corrente de água que se desvia de um rio para regar ou para mover algum engenho.}{le.va.da}{0}
\verb{levadiço}{}{}{}{}{adj.}{Que se levanta ou baixa facilmente.}{le.va.di.ço}{0}
\verb{levadiço}{}{}{}{}{}{Que se pode mover; movediço, móvel.}{le.va.di.ço}{0}
\verb{levadio}{}{}{}{}{adj.}{Diz"-se do telhado formado de telhas soltas.}{le.va.di.o}{0}
\verb{levado}{}{}{}{}{adj.}{Que foi transportado de um local a outro.}{le.va.do}{0}
\verb{levado}{}{}{}{}{}{Diz"-se de indivíduo travesso.}{le.va.do}{0}
\verb{leva"-e"-traz}{}{}{}{}{s.2g.2n}{Indivíduo que leva uma conversa e traz outra; fofoqueiro, mexeriqueiro.}{le.va"-e"-traz}{0}
\verb{levantado}{}{}{}{}{adj.}{Posto em pé.}{le.van.ta.do}{0}
\verb{levantado}{}{}{}{}{}{Alto, elevado.}{le.van.ta.do}{0}
\verb{levantado}{}{}{}{}{}{Que se encontra em estado de insubordinação; revolto, rebelde.}{le.van.ta.do}{0}
\verb{levantador}{ô}{}{}{}{adj.}{Que levanta, que eleva.}{le.van.ta.dor}{0}
\verb{levantador}{ô}{}{}{}{}{Que incita alguém à revolta.}{le.van.ta.dor}{0}
\verb{levantador}{ô}{Esport.}{}{}{s.m.}{Jogador que tem a função de levantar a bola para que outro a golpeie.}{le.van.ta.dor}{0}
\verb{levantamento}{}{}{}{}{s.m.}{Ato ou efeito de levantar; colocação na vertical ou em pé.}{le.van.ta.men.to}{0}
\verb{levantamento}{}{}{}{}{}{Motim, revolta.}{le.van.ta.men.to}{0}
\verb{levantamento}{}{}{}{}{}{Listagem, arrolamento.}{le.van.ta.men.to}{0}
\verb{levantar}{}{}{}{}{v.t.}{Deixar algo mais alto; erguer, elevar.}{le.van.tar}{0}
\verb{levantar}{}{}{}{}{}{Fazer uma obra; construir. (\textit{O pedreiro já levantou duas paredes hoje.})}{le.van.tar}{0}
\verb{levantar}{}{}{}{}{}{Revoltar, rebelar. (\textit{Os subalternos se levantaram contra os chefes.})}{le.van.tar}{0}
\verb{levantar}{}{}{}{}{}{Conseguir algo. (\textit{Precisamos levantar fundos para a festa da formatura.})}{le.van.tar}{0}
\verb{levantar}{}{}{}{}{}{Pôr"-se de pé. (\textit{O menino levanta cedo todos os dias.})}{le.van.tar}{\verboinum{1}}
\verb{levante}{}{}{}{}{s.m.}{Lugar no horizonte onde o Sol se levanta; nascente.}{le.van.te}{0}
\verb{levante}{}{}{}{}{}{Revolta, motim.}{le.van.te}{0}
\verb{levantino}{}{}{}{}{adj.}{Relativo aos países do Levante (Turquia, Síria, Egito) e Ásia Menor.}{le.van.ti.no}{0}
\verb{levantino}{}{}{}{}{s.m.}{Indivíduo natural ou habitante do Levante.}{le.van.ti.no}{0}
\verb{levar}{}{}{}{}{v.pron.}{Deixar"-se influenciar.}{le.var}{\verboinum{1}}
\verb{levar}{}{}{}{}{}{Carregar consigo.}{le.var}{0}
\verb{levar}{}{}{}{}{}{Passar, consumir o tempo, a vida etc.}{le.var}{0}
\verb{levar}{}{}{}{}{}{Sofrer golpe, queda, castigo etc.}{le.var}{0}
\verb{levar}{}{}{}{}{v.t.}{Transportar pessoas ou coisas a determinado lugar; carregar, conduzir.}{le.var}{0}
\verb{leve}{é}{}{}{}{adj.2g.}{Que tem pouco peso.}{le.ve}{0}
\verb{leve}{é}{}{}{}{}{Pouco acentuado; delicado, suave.}{le.ve}{0}
\verb{leve}{é}{}{}{}{}{Que tem fácil digestão.}{le.ve}{0}
\verb{levedação}{}{}{"-ões}{}{s.f.}{Ato ou efeito de levedar; fermentação.}{le.ve.da.ção}{0}
\verb{levedação}{}{}{"-ões}{}{}{Melhoramento, desenvolvimento.}{le.ve.da.ção}{0}
\verb{levedar}{}{}{}{}{v.t.}{Tornar lêvedo; fazer fermentar; tornar volumoso; aumentar, crescer.}{le.ve.dar}{0}
\verb{levedar}{}{Fig.}{}{}{v.i.}{Aperfeiçoar, desenvolver.}{le.ve.dar}{\verboinum{1}}
\verb{levedo}{ê}{}{}{}{s.m.}{Fungo que provoca fermentação, utilizado na alimentação.}{le.ve.do}{0}
\verb{lêvedo}{}{}{}{}{}{Var. de \textit{levedo}.}{lê.ve.do}{0}
\verb{levedura}{}{}{}{}{s.f.}{Levedo.}{le.ve.du.ra}{0}
\verb{leveza}{ê}{}{}{}{s.f.}{Qualidade do que tem pouco peso.}{le.ve.za}{0}
\verb{leveza}{ê}{Fig.}{}{}{}{Caráter do que é singelo, delicado.}{le.ve.za}{0}
\verb{leveza}{ê}{Pejor.}{}{}{}{Leviandade, irreflexão.}{le.ve.za}{0}
\verb{leviandade}{}{}{}{}{s.f.}{Qualidade ou caráter do que é leviano; falta de seriedade; irresponsabilidade.}{le.vi.an.da.de}{0}
\verb{leviandade}{}{}{}{}{}{Comportamento ou dito leviano; insensatez, irreflexão.}{le.vi.an.da.de}{0}
\verb{leviano}{}{}{}{}{adj.}{Que julga ou procede irresponsavelmente; que age sem seriedade.}{le.vi.a.no}{0}
\verb{leviatã}{}{}{}{}{s.m.}{Monstro do caos, na mitologia fenícia, identificado na Bíblia, como um animal aquático ou réptil.}{le.vi.a.tã}{0}
\verb{leviatã}{}{}{}{}{}{Qualquer ser ou coisa de aparência monstruosa.}{le.vi.a.tã}{0}
\verb{levita}{}{}{}{}{s.m.}{Membro da tribo hebraica sacerdotal de Levi.}{le.vi.ta}{0}
\verb{levita}{}{Por ext.}{}{}{}{Sacerdote, padre.}{le.vi.ta}{0}
\verb{levitação}{}{}{"-ões}{}{s.f.}{Ato ou efeito de levitar; erguimento sem sustentação.}{le.vi.ta.ção}{0}
\verb{levitar}{}{}{}{}{v.t.}{Erguer"-se por cima do solo, sem sustentação visível.}{le.vi.tar}{\verboinum{1}}
\verb{lexical}{cs}{}{"-ais}{}{adj.2g.}{Relativo a vocabulário; léxico.}{le.xi.cal}{0}
\verb{léxico}{cs}{}{}{}{adj.}{Lexical.}{lé.xi.co}{0}
\verb{léxico}{cs}{Gram.}{}{}{s.m.}{Conjunto de vocábulos de um idioma.}{lé.xi.co}{0}
\verb{léxico}{cs}{}{}{}{}{Dicionário de vocábulos usados por um autor ou por uma escola literária.}{lé.xi.co}{0}
\verb{lexicografia}{cs}{}{}{}{s.f.}{Técnica de elaboração de dicionários.}{le.xi.co.gra.fi.a}{0}
\verb{lexicografia}{cs}{}{}{}{}{O estudo dessa técnica.}{le.xi.co.gra.fi.a}{0}
\verb{lexicográfico}{cs}{}{}{}{adj.}{Relativo à lexicografia ou a lexicógrafo.}{le.xi.co.grá.fi.co}{0}
\verb{lexicógrafo}{cs}{}{}{}{s.m.}{Autor de dicionário ou de trabalho a respeito de palavras de uma língua.}{le.xi.có.gra.fo}{0}
\verb{lexicologia}{cs}{}{}{}{s.f.}{Estudo que se ocupa do valor etimológico e das várias acepções das palavras.}{le.xi.co.lo.gi.a}{0}
\verb{lexicologia}{cs}{}{}{}{}{Estudo dos elementos de formação das palavras.}{le.xi.co.lo.gi.a}{0}
\verb{lezíria}{}{}{}{}{s.f.}{Terra plana e alagadiça, nas margens de um rio.}{le.zí.ria}{0}
\verb{lhama}{}{Zool.}{}{}{s.f.}{Mamífero ruminante, encontrado ao Sul do Peru e ao Noroeste da Argentina, de pelagem longa e lanosa.}{lha.ma}{0}
\verb{lhaneza}{ê}{}{}{}{s.f.}{Qualidade do que é lhano, afável; singeleza, candura.}{lha.ne.za}{0}
\verb{lhano}{}{}{}{}{adj.}{Movido pela franqueza; sincero, verdadeiro.}{lha.no}{0}
\verb{lhano}{}{}{}{}{}{Em que há simplicidade; singelo.}{lha.no}{0}
\verb{lhano}{}{}{}{}{}{Afável, amável.}{lha.no}{0}
\verb{lhano}{}{Geogr.}{}{}{s.m.}{Planície extensa de vegetação herbácea, no norte da América do Sul (mais usado no plural).}{lha.no}{0}
\verb{lhe}{}{}{}{}{pron.}{Pronome pessoal que se refere a ele ou a ela. (\textit{A aluna apresentou o trabalho e o professor lhe fez um elogio.})}{lhe}{0}
\verb{lhe}{}{}{}{}{}{Pronome pessoal que se refere a você. (\textit{Filha, trouxe"-lhe o livro que me pediu.})}{lhe}{0}
\verb{lhe}{}{}{}{}{}{Dele(a). (\textit{O amigo tomou"-lhe o brinquedo.})}{lhe}{0}
\verb{lhe}{}{}{}{}{}{De você. (\textit{Menino, quem lhe tomou o brinquedo?})}{lhe}{0}
\verb{lho}{}{}{}{}{}{Contração dos pronomes \textit{lhe} e \textit{o}. (\textit{A menina apaixonou"-se pelo cachorro e o pai comprou"-lho.})}{lho}{0}
\verb{Li}{}{Quím.}{}{}{}{Símb. do \textit{lítio}.}{Li}{0}
\verb{liame}{}{}{}{}{s.m.}{Aquilo que liga ou prende uma coisa a outra.}{li.a.me}{0}
\verb{liana}{}{}{}{}{s.f.}{Trepadeira lenhosa, geralmente de grande tamanho, semelhante a cipó.}{li.a.na}{0}
\verb{libação}{}{}{"-ões}{}{s.f.}{Entre os pagãos, ato que consistia na aspersão de um líquido de origem orgânica em intenção de uma divindade.}{li.ba.ção}{0}
\verb{libação}{}{}{"-ões}{}{}{O líquido espargido.}{li.ba.ção}{0}
\verb{libação}{}{}{"-ões}{}{}{Ato de tomar bebidas, especialmente alcoólicas, por prazer ou para se fazerem brindes.}{li.ba.ção}{0}
\verb{libanês}{}{}{}{}{adj.}{Relativo a Líbano.}{li.ba.nês}{0}
\verb{libanês}{}{}{}{}{s.m.}{Indivíduo natural ou habitante desse país.}{li.ba.nês}{0}
\verb{libar}{}{}{}{}{v.i.}{Fazer libações em honra de uma divindade.}{li.bar}{0}
\verb{libar}{}{}{}{}{v.t.}{Beber, tragar.}{li.bar}{0}
\verb{libar}{}{}{}{}{}{Experimentar. provar.}{li.bar}{\verboinum{1}}
\verb{libelo}{é}{}{}{}{s.m.}{Apresentação, oral ou escrita, de uma acusação contra uma pessoa ou empresa.}{li.be.lo}{0}
\verb{libélula}{}{Zool.}{}{}{s.f.}{Inseto de abdômen longo e estreito, quatro asas alongadas, transparentes e providas de rica nervação.}{li.bé.lu.la}{0}
\verb{líber}{}{Bot.}{}{}{s.m.}{Tecido vascular vegetal cuja função principal é transporte de água, sais minerais e compostos orgânicos produzidos pela fotossíntese, além das funções de sustentação e reserva.}{lí.ber}{0}
\verb{liberação}{}{}{"-ões}{}{s.f.}{Ato ou efeito de liberar; libertação.}{li.be.ra.ção}{0}
\verb{liberação}{}{}{"-ões}{}{}{Quitação de dívida ou obrigação.}{li.be.ra.ção}{0}
\verb{liberação}{}{}{"-ões}{}{}{Libertação de condenado após cumprimento de pena.}{li.be.ra.ção}{0}
\verb{liberal}{}{}{"-ais}{}{adj.2g.}{Que gosta de dar, que não se importa de gastar; generoso.}{li.be.ral}{0}
\verb{liberal}{}{}{"-ais}{}{}{Que é partidário do liberalismo, ou que nele se funda.}{li.be.ral}{0}
\verb{liberal}{}{}{"-ais}{}{}{Diz"-se de profissional que age com independência.}{li.be.ral}{0}
\verb{liberalidade}{}{}{}{}{s.f.}{Qualidade ou condição de liberal; generosidade.}{li.be.ra.li.da.de}{0}
\verb{liberalidade}{}{}{}{}{}{Transigência, flexibilidade.}{li.be.ra.li.da.de}{0}
\verb{liberalismo}{}{}{}{}{s.m.}{Doutrina baseada na defesa intransigente da liberdade individual nos campos econômico, político, religioso e intelectual, e contra a intervenção do Estado na economia. }{li.be.ra.lis.mo}{0}
\verb{liberalismo}{}{}{}{}{}{Comportamento pródigo; liberalidade.}{li.be.ra.lis.mo}{0}
\verb{liberalista}{}{}{}{}{adj.2g.}{Relativo ao liberalismo.}{li.be.ra.lis.ta}{0}
\verb{liberalista}{}{}{}{}{}{Que é partidário do liberalismo.}{li.be.ra.lis.ta}{0}
\verb{liberalizar}{}{}{}{}{v.t.}{Dar com liberalidade, distribuir com profusão; prodigalizar.}{li.be.ra.li.zar}{0}
\verb{liberalizar}{}{}{}{}{}{Tornar liberal ou mais liberal.}{li.be.ra.li.zar}{0}
\verb{liberalizar}{}{}{}{}{v.pron.}{Tornar"-se adepto do liberalismo.}{li.be.ra.li.zar}{\verboinum{1}}
\verb{liberar}{}{}{}{}{v.t.}{Permitir a saída de.}{li.be.rar}{0}
\verb{liberar}{}{}{}{}{}{Libertar de atitudes tradicionais.}{li.be.rar}{0}
\verb{liberar}{}{}{}{}{}{Dar folga a; dispensar.}{li.be.rar}{0}
\verb{liberar}{}{}{}{}{}{Retirar a proibição de alguma coisa; permitir.}{li.be.rar}{\verboinum{1}}
\verb{liberdade}{}{}{}{}{s.f.}{Poder de agir, segundo a própria determinação, dentro dos limites impostos por normas definidas.}{li.ber.da.de}{0}
\verb{liberdade}{}{}{}{}{}{Independência, autonomia.}{li.ber.da.de}{0}
\verb{liberdade}{}{}{}{}{}{Atitude de quem se sente à vontade para fazer alguma coisa.}{li.ber.da.de}{0}
\verb{liberiano}{}{}{}{}{adj.}{Relativo à Libéria.}{li.be.ri.a.no}{0}
\verb{liberiano}{}{}{}{}{s.m.}{Indivíduo natural ou habitante desse país.}{li.be.ri.a.no}{0}
\verb{líbero}{}{Esport.}{}{}{s.m.}{Jogador que, por não ter posição específica no campo, pode corrigir eventuais falhas de seus companheiros na zaga e no ataque.}{lí.be.ro}{0}
\verb{libertação}{}{}{"-ões}{}{s.f.}{Ato ou efeito de libertar; liberação.}{li.ber.ta.ção}{0}
\verb{libertador}{ô}{}{}{}{adj.}{Que liberta, que concede a liberdade.}{li.ber.ta.dor}{0}
\verb{libertar}{}{}{}{}{v.t.}{Tornar livre; pôr em liberdade.}{li.ber.tar}{0}
\verb{libertar}{}{}{}{}{}{Dar vazão a; expandir; soltar.}{li.ber.tar}{\verboinum{1}}
\verb{libertário}{}{}{}{}{adj.}{Que é partidário da liberdade absoluta.}{li.ber.tá.rio}{0}
\verb{libertário}{}{}{}{}{}{Anarquista.}{li.ber.tá.rio}{0}
\verb{libertinagem}{}{}{"-ens}{}{s.f.}{Conduta de quem se entrega sem moderação a prazeres sexuais.}{li.ber.ti.na.gem}{0}
\verb{libertino}{}{}{}{}{adj.}{Que se entrega imoderadamente aos prazeres do sexo.}{li.ber.ti.no}{0}
\verb{liberto}{é}{}{}{}{adj.}{Que foi posto em liberdade.}{li.ber.to}{0}
\verb{libidinagem}{}{}{"-ens}{}{s.f.}{Procura irrefreada de satisfações sexuais.}{li.bi.di.na.gem}{0}
\verb{libidinoso}{ô}{}{"-osos ⟨ó⟩}{"-osa ⟨ó⟩}{adj.}{Relativo ao prazer ou ao apetite sexual; voluptuoso, sensual.}{li.bi.di.no.so}{0}
\verb{libidinoso}{ô}{}{"-osos ⟨ó⟩}{"-osa ⟨ó⟩}{}{Que tem desejos sexuais intensos e constantes; devasso.}{li.bi.di.no.so}{0}
\verb{libido}{}{}{}{}{s.f.}{Instinto ou desejo sexual.}{li.bi.do}{0}
\verb{líbio}{}{}{}{}{adj.}{Relativo a Líbia.}{lí.bio}{0}
\verb{líbio}{}{}{}{}{s.m.}{Indivíduo natural ou habitante desse país.}{lí.bio}{0}
\verb{libra}{}{Astron.}{}{}{s.f.}{Sétima constelação zodiacal.}{li.bra}{0}
\verb{libra}{}{Astrol.}{}{}{}{O signo do zodíaco referente a essa constelação.}{li.bra}{0}
\verb{libra}{}{}{}{}{s.f.}{Unidade de massa utilizada no sistema inglês de pesos e medidas equivalente a 453 gramas.}{li.bra}{0}
\verb{libra}{}{}{}{}{}{Moeda inglesa.}{li.bra}{0}
\verb{libra}{}{}{}{}{}{A sétima constelação do Zodíaco.}{li.bra}{0}
\verb{librar}{}{}{}{}{v.t.}{Pôr em equilíbrio.}{li.brar}{0}
\verb{librar}{}{}{}{}{}{Sustentar no ar; suspender.}{li.brar}{\verboinum{1}}
\verb{libré}{}{}{}{}{s.m.}{Uniforme usado pelos criados de casas nobres e senhoriais.}{li.bré}{0}
\verb{libreto}{ê}{}{}{}{s.m.}{Pequeno livro com textos de óperas, oratórios ou cantatas.}{li.bre.to}{0}
\verb{libriano}{}{Astrol.}{}{}{s.m.}{Indivíduo que nasceu sob o signo de libra.   }{li.bri.a.no}{0}
\verb{libriano}{}{Astrol.}{}{}{adj.}{Relativo ou pertencente a esse signo.   }{li.bri.a.no}{0}
\verb{liça}{}{}{}{}{s.f.}{Arena destinada a torneios, justas e combates.}{li.ça}{0}
\verb{liça}{}{Por ext.}{}{}{}{Luta, briga, combate.}{li.ça}{0}
\verb{lição}{}{}{"-ões}{}{s.f.}{Ensinamento sobre alguma coisa.}{li.ção}{0}
\verb{lição}{}{}{"-ões}{}{}{Tarefa escolar.}{li.ção}{0}
\verb{lição}{}{}{"-ões}{}{}{Comportamento que deve ser imitado; exemplo.}{li.ção}{0}
\verb{lição}{}{}{"-ões}{}{}{Castigo.}{li.ção}{0}
\verb{licença}{}{}{}{}{s.f.}{Autorização, consentimento, permissão.}{li.cen.ça}{0}
\verb{licença}{}{}{}{}{}{Autorização para faltar ao emprego.}{li.cen.ça}{0}
\verb{licença"-prêmio}{}{}{licenças"-prêmios \textit{ou}  licenças"-prêmio}{}{s.f.}{Licença remunerada, de três meses, a que o funcionário público tem direito a cada cinco anos de trabalho sem faltas.}{li.cen.ça"-prê.mio}{0}
\verb{licenciado}{}{}{}{}{adj.}{Diz"-se de indivíduo que obteve a licenciatura.}{li.cen.ci.a.do}{0}
\verb{licenciado}{}{}{}{}{}{Que possui licença ou que está autorizado por licença.}{li.cen.ci.a.do}{0}
\verb{licenciamento}{}{}{}{}{s.m.}{Ato ou efeito de licenciar; licenciatura.}{li.cen.ci.a.men.to}{0}
\verb{licenciamento}{}{}{}{}{}{Concessão de licença.}{li.cen.ci.a.men.to}{0}
\verb{licenciamento}{}{}{}{}{}{Demissão.}{li.cen.ci.a.men.to}{0}
\verb{licenciar}{}{}{}{}{v.pron.}{Fazer licenciatura.}{li.cen.ci.ar}{0}
\verb{licenciar}{}{}{}{}{v.t.}{Conceder licença.}{li.cen.ci.ar}{\verboinum{6}}
\verb{licenciatura}{}{}{}{}{s.f.}{Grau universitário que dá o direito de exercer o magistério do ensino médio. }{li.cen.ci.a.tu.ra}{0}
\verb{licenciosidade}{}{}{}{}{s.f.}{Qualidade ou condição de licencioso; libertinagem.}{li.cen.ci.o.si.da.de}{0}
\verb{licencioso}{ô}{}{"-osos ⟨ó⟩}{"-osa ⟨ó⟩}{adj.}{Que usa de excessiva licença; indisciplinado, desregrado.}{li.cen.ci.o.so}{0}
\verb{licencioso}{ô}{}{"-osos ⟨ó⟩}{"-osa ⟨ó⟩}{}{Sensual, libidinoso.}{li.cen.ci.o.so}{0}
\verb{licencioso}{ô}{}{"-osos ⟨ó⟩}{"-osa ⟨ó⟩}{}{Indecente, depravado.}{li.cen.ci.o.so}{0}
\verb{liceu}{}{}{}{}{s.m.}{Estabelecimento no qual é ministrado o ensino do segundo grau ou o ensino profissionalizante.}{li.ceu}{0}
\verb{lichia}{}{}{}{}{}{O fruto dessa árvore.}{li.chi.a}{0}
\verb{lichia}{}{Bot.}{}{}{s.f.}{Árvore de flores brancas ou esverdeadas, originária do extremo Oriente, cultivada pelo fruto comestível e pela madeira. }{li.chi.a}{0}
\verb{licitação}{}{}{"-ões}{}{s.f.}{Ato ou efeito de licitar; dar lanços num leilão ou hasta pública.}{li.ci.ta.ção}{0}
\verb{licitador}{ô}{}{}{}{adj.}{Licitante.}{li.ci.ta.dor}{0}
\verb{licitante}{}{}{}{}{adj.2g.}{Que licita.}{li.ci.tan.te}{0}
\verb{licitante}{}{}{}{}{}{Que faz o lance ou oferta de compra pelo preço que indica; lançador.}{li.ci.tan.te}{0}
\verb{licitar}{}{}{}{}{v.t.}{Pôr em leilão ou em concorrência pública.}{li.ci.tar}{0}
\verb{licitar}{}{}{}{}{}{Oferecer ou cobrir lanço sobre.}{li.ci.tar}{\verboinum{1}}
\verb{lícito}{}{}{}{}{adj.}{Conforme a lei; legal.}{lí.ci.to}{0}
\verb{licitude}{}{}{}{}{s.f.}{Qualidade do que é lícito.}{li.ci.tu.de}{0}
\verb{licitude}{}{Jur.}{}{}{}{Conformidade ao direito, à lei; juricidade, legalidade.}{li.ci.tu.de}{0}
\verb{liço}{}{}{}{}{s.m.}{Cada um dos fios do tear, que sobem e descem para serem atravessados pelos fios da tecelagem.}{li.ço}{0}
\verb{licor}{ô}{}{}{}{s.m.}{Bebida alcoólica aromatizada e doce.}{li.cor}{0}
\verb{licoreira}{ê}{}{}{}{s.f.}{Garrafa para guardar licor.}{li.co.rei.ra}{0}
\verb{licoreiro}{ê}{}{}{}{s.m.}{Licoreira.}{li.co.rei.ro}{0}
\verb{licoroso}{ô}{}{"-osos ⟨ó⟩}{"-osa ⟨ó⟩}{adj.}{Que tem o aroma, o teor alcoólico, e é açucarado como o licor.}{li.co.ro.so}{0}
\verb{lida}{}{}{}{}{s.f.}{Uso da inteligência e da força no desenvolvimento de um trabalho; labor.}{li.da}{0}
\verb{lida}{}{}{}{}{}{Leitura rápida; superficial.}{li.da}{0}
\verb{lidador}{ô}{}{}{}{adj.}{Que lida; trabalhador, labutador.}{li.da.dor}{0}
\verb{lidador}{ô}{}{}{}{}{Lutador, combatente.}{li.da.dor}{0}
\verb{lidar}{}{}{}{}{v.t.}{Trabalhar com afã; batalhar.}{li.dar}{0}
\verb{lidar}{}{}{}{}{}{Mexer ou operar com.}{li.dar}{0}
\verb{lidar}{}{}{}{}{}{Conviver; tratar com.}{li.dar}{\verboinum{1}}
\verb{lide}{}{}{}{}{s.m.}{Resumo que apresenta os principais tópicos da matéria desenvolvida no texto jornalístico; \textit{lead}.}{li.de}{0}
\verb{líder}{}{}{}{}{s.m.}{Indivíduo que lidera um grupo; comandante, dirigente.}{lí.der}{0}
\verb{líder}{}{}{}{}{}{O primeiro colocado num campeonato.}{lí.der}{0}
\verb{liderança}{}{}{}{}{s.f.}{Função de líder.}{li.de.ran.ça}{0}
\verb{liderança}{}{}{}{}{}{Capacidade de liderar; espírito de chefia.}{li.de.ran.ça}{0}
\verb{liderar}{}{}{}{}{v.t.}{Proceder como ou ter a função de líder.}{li.de.rar}{0}
\verb{liderar}{}{}{}{}{}{Ocupar a posição de líder.}{li.de.rar}{\verboinum{1}}
\verb{lídimo}{}{}{}{}{adj.}{Reconhecido como legítimo; autêntico, verdadeiro.}{lí.di.mo}{0}
\verb{lídimo}{}{}{}{}{}{Vernáculo, puro, genuíno.}{lí.di.mo}{0}
\verb{lido}{}{}{}{}{adj.}{Que se leu ou se lê.}{li.do}{0}
\verb{lido}{}{}{}{}{}{Que tem conhecimentos adquiridos pela leitura; versado, erudito.}{li.do}{0}
\verb{liga}{}{}{}{}{s.f.}{Ato de ligar; junção, união, aliança.}{li.ga}{0}
\verb{liga}{}{}{}{}{}{União de pessoas que concentram esforços para determinado fim; associação, agremiação; sociedade.}{li.ga}{0}
\verb{liga}{}{Quím.}{}{}{}{Sólido cristalino formado por dois ou mais elementos, em geral de caráter metálico, e que pode ser uma solução sólida, uma combinação química ou uma mistura íntima dos cristais dos diferentes elementos.}{li.ga}{0}
\verb{liga}{}{}{}{}{}{Tira de elástico usada para prender a meia à perna.}{li.ga}{0}
\verb{ligação}{}{}{"-ões}{}{s.f.}{União entre duas coisas ou pessoas.}{li.ga.ção}{0}
\verb{ligação}{}{}{"-ões}{}{}{Aquilo que serve para ligar, unir.}{li.ga.ção}{0}
\verb{ligação}{}{}{"-ões}{}{}{Telefonema.}{li.ga.ção}{0}
\verb{ligada}{}{Pop.}{}{}{s.f.}{Ato ou efeito de ligar, de telefonar; telefonema.}{li.ga.da}{0}
\verb{ligado}{}{}{}{}{adj.}{Que se ligou; posto em contato; unido, junto.}{li.ga.do}{0}
\verb{ligado}{}{Fig.}{}{}{}{Com fortes vínculos que prendem a.}{li.ga.do}{0}
\verb{ligado}{}{Pop.}{}{}{}{Intensamente atento; concentrado.}{li.ga.do}{0}
\verb{ligado}{}{Pop.}{}{}{}{Que se acha sob efeito de droga.}{li.ga.do}{0}
\verb{ligadura}{}{}{}{}{s.f.}{Faixa, atadura, ligamento.}{li.ga.du.ra}{0}
\verb{ligadura}{}{}{}{}{}{Cirurgia que fecha canais, trompa, veia etc.}{li.ga.du.ra}{0}
\verb{ligamento}{}{}{}{}{s.m.}{Tudo o que serve para unir ou ligar; vínculo, conexão.}{li.ga.men.to}{0}
\verb{ligamento}{}{Anat.}{}{}{}{Feixe fbroso que liga ossos e órgãos.}{li.ga.men.to}{0}
\verb{ligamentoso}{ô}{}{"-osos ⟨ó⟩}{"-osa ⟨ó⟩}{adj.}{Pertencente ou da natureza de um ligamento.}{li.ga.men.to.so}{0}
\verb{ligar}{}{}{}{}{v.t.}{Unir, atar. (\textit{As ferrovias ligam diversas regiões.})}{li.gar}{0}
\verb{ligar}{}{}{}{}{}{Associar, relacionar. (\textit{Depois de um certo tempo, ela passou a ligar os fatos.})}{li.gar}{0}
\verb{ligar}{}{}{}{}{}{Acionar, fazer funcionar. (\textit{É preciso ligar o botão da máquina para que ela funcione.})}{li.gar}{0}
\verb{ligar}{}{}{}{}{}{Telefonar. (\textit{Mãe e filham se ligam todos os dias.})}{li.gar}{0}
\verb{ligar}{}{Pop.}{}{}{}{Dar importância. (\textit{As crianças nem ligam para a chuva e vão para a rua brincar.})}{li.gar}{0}
\verb{ligar}{}{}{}{}{}{Unir"-se por vínculos afetivos. (\textit{As viagens de estudo ligam os professores e os alunos.})}{li.gar}{\verboinum{5}}
\verb{ligeireza}{ê}{}{}{}{s.f.}{Qualidade de ligeiro.}{li.gei.re.za}{0}
\verb{ligeireza}{ê}{}{}{}{}{Rapidez, velocidade.}{li.gei.re.za}{0}
\verb{ligeireza}{ê}{}{}{}{}{Presteza de movimentos; agilidade.}{li.gei.re.za}{0}
\verb{ligeiro}{ê}{}{}{}{adj.}{Que é rápido, veloz.}{li.gei.ro}{0}
\verb{ligeiro}{ê}{}{}{}{}{Presto de movimentos; ágil.}{li.gei.ro}{0}
\verb{ligeiro}{ê}{}{}{}{}{Que é pouco acentuado; leve.}{li.gei.ro}{0}
\verb{lígneo}{}{}{}{}{adj.}{Que tem aspecto ou consistência de madeira.}{líg.neo}{0}
\verb{lignina}{}{}{}{}{s.f.}{Substância responsável pela rigidez da madeira.}{lig.ni.na}{0}
\verb{lignita}{}{}{}{}{s.f.}{Tipo de carvão fóssil. }{lig.ni.ta}{0}
\verb{lignito}{}{}{}{}{}{Var. de \textit{lignita}.}{lig.ni.to}{0}
\verb{lilá}{}{}{}{}{}{Var. de \textit{lilás}.}{li.lá}{0}
\verb{lilás}{}{Bot.}{}{}{s.m.}{Arbusto nativo da Europa, de flores arroxeadas ou brancas.}{li.lás}{0}
\verb{lilás}{}{}{}{}{}{A flor desse arbusto.}{li.lás}{0}
\verb{lilás}{}{}{}{}{}{A cor arroxeada dessa flor.}{li.lás}{0}
\verb{liliputiano}{}{}{}{}{adj.}{Relativo a Lilipute, ilha imaginária do livro \textit{Viagens de Gulliver}, do escritor inglês Jonathan Swift, e cujos habitantes eram extremamente pequenos.}{li.li.pu.ti.a.no}{0}
\verb{liliputiano}{}{Por ext.}{}{}{}{Extremamente pequeno.}{li.li.pu.ti.a.no}{0}
\verb{liliputiano}{}{}{}{}{s.m.}{Indivíduo natural ou habitante dessa ilha.}{li.li.pu.ti.a.no}{0}
\verb{lima}{}{}{}{}{s.f.}{Fruta de cor amarela, polpa esverdeada e sabor meio amargo.}{li.ma}{0}
\verb{lima}{}{}{}{}{s.f.}{Ferramenta metálica com lâmina estriada usada para desbastar ou serrar metal e outros materiais rígidos.}{li.ma}{0}
\verb{lima}{}{Fig.}{}{}{}{Coisa que desgasta, consome, corrói.}{li.ma}{0}
\verb{lima}{}{}{}{}{}{Ato ou efeito de limar.}{li.ma}{0}
\verb{limadura}{}{}{}{}{s.f.}{Ato ou efeito de limar.}{li.ma.du.ra}{0}
\verb{limadura}{}{Fig.}{}{}{}{Polimento, aperfeiçoamento, apuro.}{li.ma.du.ra}{0}
\verb{limadura}{}{}{}{}{}{Fragmentos de metal resultantes do ato de limar; limalha. [usa"-se geralmente no plural nesta acepção]}{li.ma.du.ra}{0}
\verb{limagem}{}{}{"-ens}{}{s.f.}{Limadura.}{li.ma.gem}{0}
\verb{limalha}{}{}{}{}{s.f.}{Fragmentos de metal resultantes do ato de limar.}{li.ma.lha}{0}
\verb{limão}{}{}{"-ões}{}{s.m.}{Fruta de cor amarela ou verde, polpa verde e sabor ácido.}{li.mão}{0}
\verb{limão"-galego}{ê}{}{limões"-galegos ⟨ê⟩}{}{s.m.}{Variedade de limão com casca e polpa de cor alaranjada e sabor ácido.}{li.mão"-ga.le.go}{0}
\verb{limar}{}{}{}{}{v.t.}{Desbastar ou serrar com a lima.}{li.mar}{0}
\verb{limar}{}{Fig.}{}{}{}{Corroer, gastar, desgastar.}{li.mar}{0}
\verb{limar}{}{Fig.}{}{}{}{Polir, aprimorar, aperfeiçoar.}{li.mar}{\verboinum{1}}
\verb{limar}{}{}{}{}{v.t.}{Temperar com limão e azeite.}{li.mar}{\verboinum{1}}
\verb{limbo}{}{}{}{}{s.m.}{Borda, orla.}{lim.bo}{0}
\verb{limbo}{}{Relig.}{}{}{}{Local para onde vão, segundo a religião cristã, as almas das pessoas que não foram batizadas.}{lim.bo}{0}
\verb{limbo}{}{}{}{}{}{Situação em que não se é lembrado; esquecimento, olvido.}{lim.bo}{0}
\verb{limbo}{}{}{}{}{}{Incerteza, indecisão, hesitação, indeterminação.}{lim.bo}{0}
\verb{limeira}{ê}{Bot.}{}{}{s.f.}{Árvore que dá a lima.}{li.mei.ra}{0}
\verb{limenho}{}{}{}{}{adj.}{Relativo a Lima, capital do Peru.}{li.me.nho}{0}
\verb{limenho}{}{}{}{}{s.m.}{Indivíduo natural ou habitante dessa cidade.}{li.me.nho}{0}
\verb{limiar}{}{}{}{}{s.m.}{Soleira da porta.}{li.mi.ar}{0}
\verb{limiar}{}{Fig.}{}{}{}{Entrada, início, começo.}{li.mi.ar}{0}
\verb{limiar}{}{}{}{}{}{Limite inferior de.}{li.mi.ar}{0}
\verb{liminar}{}{}{}{}{}{Relativo a ou localizado no limite.}{li.mi.nar}{0}
\verb{liminar}{}{Jur.}{}{}{}{Diz"-se de medida judicial que antecede o tratamento do objeto principal da ação.}{li.mi.nar}{0}
\verb{liminar}{}{}{}{}{s.f.}{Que constitui ou situado no início.}{li.mi.nar}{0}
\verb{limitação}{}{}{"-ões}{}{s.f.}{Ato ou efeito de limitar.}{li.mi.ta.ção}{0}
\verb{limitação}{}{}{"-ões}{}{}{Qualidade de limitado.}{li.mi.ta.ção}{0}
\verb{limitação}{}{}{"-ões}{}{}{Restrição, insuficiência.}{li.mi.ta.ção}{0}
\verb{limitado}{}{}{}{}{}{Finito, efêmero, transitório.}{li.mi.ta.do}{0}
\verb{limitado}{}{}{}{}{adj.}{Que tem limites; de extensão ou quantidade reduzida; restrito.}{li.mi.ta.do}{0}
\verb{limitado}{}{}{}{}{}{Que tem capacidade intelectual abaixo do necessário.}{li.mi.ta.do}{0}
\verb{limitar}{}{}{}{}{v.t.}{Determinar ou impor limites.}{li.mi.tar}{0}
\verb{limitar}{}{}{}{}{}{Fazer fronteira com.}{li.mi.tar}{0}
\verb{limitar}{}{}{}{}{v.pron.}{Dar"-se por satisfeito com.}{li.mi.tar}{0}
\verb{limitar}{}{}{}{}{}{Restringir"-se a; circunscrever"-se a.}{li.mi.tar}{\verboinum{1}}
\verb{limitativo}{}{}{}{}{adj.}{Que representa ou impõe limite; limitante.}{li.mi.ta.ti.vo}{0}
\verb{limite}{}{}{}{}{s.m.}{Fronteira, divisa.}{li.mi.te}{0}
\verb{limite}{}{Fig.}{}{}{}{Fim, termo.}{li.mi.te}{0}
\verb{limite}{}{Fig.}{}{}{}{Imperfeição, insuficiência, defeito.}{li.mi.te}{0}
\verb{limítrofe}{}{}{}{}{adj.2g.}{Que se situa próximo ao limite.}{li.mí.tro.fe}{0}
\verb{limítrofe}{}{}{}{}{}{Que faz fronteira com.}{li.mí.tro.fe}{0}
\verb{limnologia}{}{Biol.}{}{}{s.f.}{Ramo da biologia que estuda lagos, lagoas e pântanos.}{lim.no.lo.gi.a}{0}
\verb{limo}{}{}{}{}{s.m.}{Mistura viscosa de barro e matéria orgânica; lama, lodo.}{li.mo}{0}
\verb{limoal}{}{}{"-ais}{}{s.m.}{Aglomerado de limoeiros.}{li.mo.al}{0}
\verb{limoeiro}{ê}{Bot.}{}{}{s.m.}{Árvore que dá o limão.}{li.mo.ei.ro}{0}
\verb{limonada}{}{}{}{}{s.f.}{Refresco de sumo de limão, água e açúcar.}{li.mo.na.da}{0}
\verb{limoso}{ô}{}{"-osos ⟨ó⟩}{"-osa ⟨ó⟩}{adj.}{Que tem limo.}{li.mo.so}{0}
\verb{limpa}{}{}{}{}{s.f.}{Ato ou efeito de limpar.}{lim.pa}{0}
\verb{limpa}{}{Pop.}{}{}{}{Roubo em que se tira tudo da vítima.}{lim.pa}{0}
\verb{limpadela}{é}{}{}{}{s.f.}{Limpeza superficial e rápida.}{lim.pa.de.la}{0}
\verb{limpador}{}{}{}{}{adj.2g.}{Que limpa.}{lim.pa.dor}{0}
\verb{limpador}{}{}{}{}{s.m.}{Instrumento ou produto que limpa.}{lim.pa.dor}{0}
\verb{limpar}{}{}{}{}{v.t.}{Retirar sujeiras, impurezas, manchas.}{lim.par}{0}
\verb{limpar}{}{}{}{}{}{Tirar tudo ou tudo que está em excesso.}{lim.par}{0}
\verb{limpar}{}{}{}{}{v.i.}{Tornar"-se límpido e sem nuvens (o tempo, a atmosfera).}{lim.par}{\verboinum{1}}
\verb{limpa"-trilhos}{}{}{}{}{s.m.}{Grade ou chapa de ferro fixada à frente das locomotivas para desviar da linha obstáculos que possam oferecer riscos à circulação dos trens.}{lim.pa"-tri.lhos}{0}
\verb{limpeza}{ê}{}{}{}{s.f.}{Ato ou efeito de limpar.}{lim.pe.za}{0}
\verb{limpeza}{ê}{}{}{}{}{Qualidade de limpo.}{lim.pe.za}{0}
\verb{limpidez}{ê}{}{}{}{s.f.}{Qualidade de límpido.}{lim.pi.dez}{0}
\verb{límpido}{}{}{}{}{adj.}{Claro, transparente, puro, desanuviado, brilhante, polido, distinto.}{lím.pi.do}{0}
\verb{limpo}{}{}{}{}{adj.}{Sem sujeira, impureza, mancha.}{lim.po}{0}
\verb{limpo}{}{}{}{}{}{Diz"-se de atmosfera sem nuvens ou nebulosidade.}{lim.po}{0}
\verb{limpo}{}{Fig.}{}{}{}{Isento de culpa ou de fatores incriminatórios.}{lim.po}{0}
\verb{limusine}{}{}{}{}{s.f.}{Automóvel alongado, luxuoso, totalmente fechado e com cabine do motorista separada da dos passageiros por um vidro.}{li.mu.si.ne}{0}
\verb{lince}{}{Zool.}{}{}{s.m.}{Mamífero felino, com cauda curta e boa visão.}{lin.ce}{0}
\verb{lince}{}{Fig.}{}{}{}{Indivíduo que enxerga muito bem.}{lin.ce}{0}
\verb{linchamento}{}{}{}{}{s.m.}{Ato ou efeito de linchar.}{lin.cha.men.to}{0}
\verb{linchar}{}{}{}{}{v.t.}{Espancar ou executar uma pessoa por decisão impulsiva e coletiva, geralmente de uma multidão.}{lin.char}{\verboinum{1}}
\verb{linda}{}{}{}{}{s.f.}{Linde.}{lin.da}{0}
\verb{lindar}{}{}{}{}{v.t.}{Demarcar, delimitar.}{lin.dar}{\verboinum{1}}
\verb{linde}{}{}{}{}{s.m.}{Marco, limite, baliza.}{lin.de}{0}
\verb{lindeiro}{ê}{}{}{}{adj.}{Que limita.}{lin.dei.ro}{0}
\verb{lindeza}{ê}{}{}{}{s.f.}{Qualidade de lindo.}{lin.de.za}{0}
\verb{lindeza}{ê}{}{}{}{}{Pessoa ou coisa linda.}{lin.de.za}{0}
\verb{lindo}{}{}{}{}{adj.}{Muito bonito, muito belo, maravilhoso.}{lin.do}{0}
\verb{lindo}{}{}{}{}{}{Muito bom; agradável, aprazível.}{lin.do}{0}
\verb{lineamento}{}{}{}{}{s.m.}{Traços, linhas.}{li.ne.a.men.to}{0}
\verb{lineamento}{}{}{}{}{}{Primeiros traços de uma obra; esboço.}{li.ne.a.men.to}{0}
\verb{linear}{}{}{}{}{adj.2g.}{Relativo a linha.}{li.ne.ar}{0}
\verb{linear}{}{}{}{}{}{Disposto em linha.}{li.ne.ar}{0}
\verb{linfa}{}{Anat.}{}{}{s.f.}{Líquido transparente rico em glóbulos brancos e que circula pelos vasos linfáticos.}{lin.fa}{0}
\verb{linfa}{}{Poét.}{}{}{}{Água cristalina.}{lin.fa}{0}
\verb{linfático}{}{}{}{}{adj.}{Relativo a linfa.}{lin.fá.ti.co}{0}
\verb{linfático}{}{Anat.}{}{}{}{Diz"-se dos vasos por onde circula linfa.}{lin.fá.ti.co}{0}
\verb{linfócito}{}{Anat.}{}{}{s.m.}{Um entre os vários tipos de glóbulos brancos.}{lin.fó.ci.to}{0}
\verb{linfoide}{}{}{}{}{adj.2g.}{Semelhante a linfa.}{lin.foi.de}{0}
\verb{linfoma}{}{Med.}{}{}{s.m.}{Tumor dos tecidos linfáticos.}{lin.fo.ma}{0}
\verb{lingerie}{}{}{}{}{s.f.}{Roupa íntima feminina.}{\textit{lingerie}}{0}
\verb{lingote}{ó}{}{}{}{s.m.}{Barra de metal fundido.}{lin.go.te}{0}
\verb{língua}{}{Anat.}{}{}{s.f.}{Órgão muscular situado na cavidade bucal que serve aos processos de mastigação, deglutição e articulação dos sons da fala.}{lín.gua}{0}
\verb{língua}{}{}{}{}{}{Sistema de signos verbais que serve como meio de comunicação, expressão e organização do pensamento; linguagem, idioma.}{lín.gua}{0}
\verb{língua"-de"-sogra}{ó}{}{línguas"-de"-sogra ⟨ó⟩}{}{s.f.}{Brinquedo que consiste num apito com um tubo de papel enrolado na ponta, o qual se desenrola como uma língua quando soprado.}{lín.gua"-de"-so.gra}{0}
%\verb{}{}{}{}{}{}{}{}{0}
\verb{linguado}{}{Zool.}{}{}{s.m.}{Peixe marítimo ou fluvial de corpo achatado.}{lin.gua.do}{0}
\verb{linguado}{}{}{}{}{}{Tira de papel para escrever.}{lin.gua.do}{0}
\verb{linguado}{}{}{}{}{}{Lâmina comprida.}{lin.gua.do}{0}
\verb{linguafone}{}{}{}{}{s.m.}{Metodo de ensino de língua estrangeira baseado em textos escritos e gravações fonográficas.}{lin.gua.fo.ne}{0}
\verb{linguagem}{}{}{"-ens}{}{s.f.}{Faculdade humana de comunicação verbal.}{lin.gua.gem}{0}
\verb{linguagem}{}{}{"-ens}{}{}{Qualquer sistema de signos que serve para expressar ideias e sentimentos.}{lin.gua.gem}{0}
\verb{linguagem}{}{}{"-ens}{}{}{Idioma, língua, dialeto.}{lin.gua.gem}{0}
\verb{linguagem}{}{}{"-ens}{}{}{Maneira peculiar de se expressar.}{lin.gua.gem}{0}
\verb{linguajar}{}{}{}{}{s.m.}{Maneira peculiar de se expressar em palavras.}{lin.gua.jar}{0}
\verb{lingual}{}{}{"-ais}{}{adj.2g.}{Relativo a língua.}{lin.gual}{0}
\verb{linguarudo}{}{}{}{}{adj.}{Diz"-se de indivíduo que fala mais do que o conveniente; indiscreto, tagarela.}{lin.gua.ru.do}{0}
\verb{lingueta}{ê}{}{}{}{s.f.}{Qualquer objeto pequeno semelhante a uma língua.}{lin.gue.ta}{0}
\verb{lingueta}{ê}{}{}{}{}{Peça movediça, plana e delgada, que faz parte de alguns instrumentos de sopro e de certas máquinas.}{lin.gue.ta}{0}
\verb{lingueta}{ê}{}{}{}{}{Parte da fechadura que a chave move para abrir ou fechar.}{lin.gue.ta}{0}
\verb{linguiça}{}{}{}{}{s.f.}{Tripa fina, geralmente de porco ou carneiro, recheada de carne e gordura temperadas.}{lin.gui.ça}{0}
\verb{linguiforme}{ó}{}{}{}{adj.2g.}{Que tem forma de língua.  }{lin.gui.for.me}{0}
\verb{linguista}{}{}{}{}{s.2g.}{Pessoa versada em linguística.}{lin.guis.ta}{0}
\verb{linguística}{}{}{}{}{s.f.}{Estudo da linguagem humana que procura  estabelecer uma teoria que explique as características gerais dessa linguagem e a gramática de cada língua particular.}{lin.guís.ti.ca}{0}
\verb{linguístico}{}{}{}{}{adj.}{Relativo à linguística.}{lin.guís.ti.co}{0}
\verb{linguístico}{}{}{}{}{}{Próprio da língua.}{lin.guís.ti.co}{0}
\verb{linguodental}{}{}{"-ais}{}{adj.2g.}{Que se refere à língua e aos dentes. }{lin.guo.den.tal}{0}
\verb{linguodental}{}{Gram.}{"-ais}{}{}{Diz"-se do fonema que se pronuncia encostando a ponta da língua na face interna dos dentes superiores, como /d/, /t/, /n/.}{lin.guo.den.tal}{0}
\verb{linha}{}{}{}{}{s.f.}{Fio de linho, seda, algodão etc. que se usa nos trabalhos de costura.}{li.nha}{0}
\verb{linha}{}{}{}{}{}{Fio metálico usado nos serviços de telefonia e telegrafia.}{li.nha}{0}
\verb{linha}{}{}{}{}{}{Traço, real ou imaginário, que separa duas coisas; limite.}{li.nha}{0}
\verb{linha}{}{}{}{}{}{Sequência de caracteres escritos de um lado a outro em uma coluna ou página.}{li.nha}{0}
\verb{linha}{}{}{}{}{}{Serviço regular de transportes entre dois pontos.}{li.nha}{0}
\verb{linha}{}{}{}{}{}{Forma correta de comportamento segundo preceitos sociais; norma, diretriz.}{li.nha}{0}
\verb{linhaça}{}{}{}{}{s.f.}{A semente do linho.}{li.nha.ça}{0}
\verb{linhada}{}{Bras.}{}{}{s.f.}{Lance de anzol.}{li.nha.da}{0}
%\verb{}{}{}{}{}{}{}{}{0}
%\verb{}{}{}{}{}{}{}{}{0}
\verb{linhagem}{}{}{"-ens}{}{s.f.}{Linha de parentesco; genealogia.}{li.nha.gem}{0}
\verb{linhita}{}{}{}{}{}{Var. de \textit{lignita}.}{li.nhi.ta}{0}
\verb{linhito}{}{}{}{}{}{Var. de \textit{lignita}.}{li.nhi.to}{0}
\verb{linho}{}{Bot.}{}{}{s.m.}{Nome comum às plantas cujo caule fornece a fibra de mesmo nome, muito usada na indústria de tecidos, e cujas sementes fornecem o óleo de linhaça.}{li.nho}{0}
\verb{linho}{}{}{}{}{}{O tecido fabricado com as fibras dessa planta.}{li.nho}{0}
\verb{linhol}{ó}{}{"-óis}{}{s.m.}{Fio grosso usado para costurar sapato e lona.}{li.nhol}{0}
\verb{linimento}{}{}{}{}{s.m.}{Medicamento oleoso usado em fricções sobre a pele.}{li.ni.men.to}{0}
\verb{link}{}{Informát.}{}{}{s.m.}{Qualquer elemento de uma página da Internet (palavra, frase ou elemento gráfico) que, uma vez clicado, remete a outra página.}{\textit{link}}{0}
\verb{linóleo}{}{}{}{}{s.m.}{Tecido impermeável de juta, untado com óleo de linhaça e cortiça em pó.}{li.nó.leo}{0}
\verb{linotipia}{}{}{}{}{s.f.}{Arte de compor em linotipo, ou a oficina onde esse trabalho é realizado.}{li.no.ti.pi.a}{0}
\verb{linotipia}{}{}{}{}{}{}{li.no.ti.pi.a}{0}
\verb{linotipista}{}{}{}{}{s.2g.}{Pessoa que opera a máquina de linotipo.}{li.no.ti.pis.ta}{0}
\verb{linotipo}{}{}{}{}{s.f.}{Máquina de compor e fundir caracteres tipográficos.}{li.no.ti.po}{0}
\verb{liofilização}{}{Quím.}{"-ões}{}{s.f.}{Ato ou efeito de liofilizar.}{li.o.fi.li.za.ção}{0}
\verb{liofilizar}{}{Quím.}{}{}{v.t.}{Desidratar uma substância em baixa temperatura e sob pressão reduzida para mantê"-la conservada.  }{li.o.fi.li.zar}{\verboinum{1}}
\verb{lipídio}{}{Bioquím.}{}{}{s.m.}{Qualquer substância orgânica constituída de ácidos graxos unidos a outros compostos, cuja função é armazenar energia; comumente chamado de gordura.}{li.pí.dio}{0}
\verb{lipoaspiração}{}{Med.}{"-ões}{}{s.f.}{Tipo de cirurgia, com finalidade estética, que consiste em aspirar o tecido gorduroso superficial de uma pessoa.}{li.po.as.pi.ra.ção}{0}
\verb{lipoide}{}{}{}{}{adj.2g.}{Que tem aparência ou consistência semelhante a da gordura.}{li.poi.de}{0}
\verb{lipoma}{}{Med.}{}{}{s.m.}{Tumor benigno formado pelo acúmulo anormal de tecido adiposo.}{li.po.ma}{0}
\verb{liquefação}{}{}{}{}{}{Var. de \textit{liquefação}.}{li.que.fa.ção}{0}
\verb{liquefação}{}{}{"-ões}{}{s.f.}{Ato ou efeito de liquefazer, de tornar líquido. }{li.que.fa.ção}{0}
\verb{liquefazer}{ê}{}{}{}{v.t.}{Tornar líquido; liquescer.}{li.que.fa.zer}{\verboinum{42}}
\verb{liquefazer}{ê}{}{}{}{}{Var. de \textit{liquefazer}.}{li.que.fa.zer}{0}
\verb{liquefeito}{ê}{}{}{}{adj.}{Que se tornou líquido.}{li.que.fei.to}{0}
\verb{liquefeito}{ê}{}{}{}{}{Var. de \textit{liquefeito}.}{li.que.fei.to}{0}
\verb{liquefeito}{ê}{}{}{}{}{Derretido.}{li.que.fei.to}{0}
\verb{líquen}{}{Bot.}{"-ens}{}{s.m.}{Forma vegetal resultante da simbiose de uma alga e um cogumelo, que cobre geralmente rochas e troncos de árvores.}{lí.quen}{0}
\verb{liquescer}{ê}{}{}{}{v.i.}{Tornar"-se líquido; liquefazer"-se.}{li.ques.cer}{\verboinum{15}}
\verb{liquidação}{}{}{"-ões}{}{s.f.}{Ato ou efeito de liquidar.}{li.qui.da.ção}{0}
\verb{liquidação}{}{Bras.}{"-ões}{}{}{Venda de mercadorias a preços reduzidos, geralmente para renovação dos estoques.}{li.qui.da.ção}{0}
\verb{liquidação}{}{}{}{}{}{Var. de \textit{liquidação}.}{li.qui.da.ção}{0}
\verb{liquidado}{}{}{}{}{adj.}{Arrasado, inutilizado.}{li.qui.da.do}{0}
\verb{liquidado}{}{}{}{}{}{Acabado, terminado.}{li.qui.da.do}{0}
\verb{liquidado}{}{}{}{}{}{Var. de \textit{liquidado}.}{li.qui.da.do}{0}
\verb{liquidante}{}{}{}{}{adj.2g.}{Que liquida; conclusivo.}{li.qui.dan.te}{0}
\verb{liquidante}{}{Jur.}{}{}{s.2g.}{Pessoa física ou jurídica encarregada da liquidação de uma sociedade comercial, quando esta se dissolve.}{li.qui.dan.te}{0}
\verb{liquidante}{}{}{}{}{}{Var. de \textit{liquidante}.}{li.qui.dan.te}{0}
\verb{liquidar}{}{}{}{}{v.t.}{Resolver, dar cabo.}{li.qui.dar}{0}
\verb{liquidar}{}{}{}{}{}{Var. de \textit{liquidar}.}{li.qui.dar}{0}
\verb{liquidar}{}{}{}{}{}{Ajustar, pagar todas as contas.}{li.qui.dar}{0}
\verb{liquidar}{}{}{}{}{}{Tornar líquido.}{li.qui.dar}{0}
\verb{liquidar}{}{}{}{}{v.i.}{Vender a preços bem baixos; queimar.}{li.qui.dar}{\verboinum{1}}
\verb{liquidez}{ê}{}{}{}{s.f.}{Qualidade ou estado de líquido.}{li.qui.dez}{0}
\verb{liquidez}{ê}{}{}{}{}{Var. de \textit{liquidez}.}{li.qui.dez}{0}
\verb{liquidez}{ê}{Jur.}{}{}{}{Qualidade ou estado de uma sentença em que o objeto e o valor da condenação estão claramente determinados. }{li.qui.dez}{0}
\verb{liquidificador}{ô}{}{}{}{adj.}{Que liquidifica.}{li.qui.di.fi.ca.dor}{0}
\verb{liquidificador}{ô}{}{}{}{s.m.}{Aparelho eletrodoméstico que serve para liquidificar frutas e outros alimentos.}{li.qui.di.fi.ca.dor}{0}
\verb{liquidificador}{ô}{}{}{}{}{Var. de \textit{liquidificador}.}{li.qui.di.fi.ca.dor}{0}
\verb{liquidificar}{}{}{}{}{v.t.}{Tornar líquido; liquefazer.}{li.qui.di.fi.car}{\verboinum{2}}
\verb{liquidificar}{}{}{}{}{}{Var. de \textit{liquidificar}.}{li.qui.di.fi.car}{0}
\verb{líquido}{}{}{}{}{adj.}{Que corre ou flui.}{lí.qui.do}{0}
\verb{líquido}{}{}{}{}{}{Var. de \textit{líquido}.}{lí.qui.do}{0}
\verb{líquido}{}{}{}{}{}{Diz"-se do valor ou quantia que sobra depois de descontados os encargos e as despesas.}{lí.qui.do}{0}
\verb{líquido}{}{}{}{}{s.m.}{Qualquer substância líquida para beber; bebida.}{lí.qui.do}{0}
\verb{lira}{}{Mús.}{}{}{s.f.}{Instrumento de cordas.}{li.ra}{0}
\verb{lira}{}{}{}{}{s.f.}{Padrão monetário usado na Itália antes da adoção do euro.}{li.ra}{0}
\verb{lirial}{}{}{}{}{adj.2g.}{Que tem a cor do lírio.}{li.ri.al}{0}
\verb{lírica}{}{}{}{}{s.f.}{Conjunto de poesias líricas.}{lí.ri.ca}{0}
\verb{lírica}{}{}{}{}{}{Poesia cujo gênero é lírico.}{lí.ri.ca}{0}
\verb{lírico}{}{}{}{}{adj.}{Diz"-se do gênero de poesia emotivo, sentimental e no qual o poeta exprime seus pensamentos íntimos.}{lí.ri.co}{0}
\verb{lírico}{}{Fig.}{}{}{}{Sonhador, sentimental, romântico.}{lí.ri.co}{0}
\verb{lírico}{}{}{}{}{}{Relativo a ópera.}{lí.ri.co}{0}
\verb{lírico}{}{}{}{}{s.m.}{Poeta que cultiva o gênero lírico.}{lí.ri.co}{0}
\verb{lírio}{}{Bot.}{}{}{s.m.}{Planta ornamental cultivada por suas flores alvas e perfumadas.}{lí.rio}{0}
\verb{lírio}{}{}{}{}{}{A flor dessa planta; açucena.}{lí.rio}{0}
\verb{lirismo}{}{}{}{}{s.m.}{Caráter da poesia lírica.}{li.ris.mo}{0}
\verb{lirismo}{}{}{}{}{}{Subjetivismo, sentimentalismo.}{li.ris.mo}{0}
\verb{lis}{}{}{}{}{s.m.}{Lírio.}{lis}{0}
\verb{lisboeta}{ê}{}{}{}{adj.2g.}{Relativo a Lisboa, capital de Portugal.}{lis.bo.e.ta}{0}
\verb{lisboeta}{ê}{}{}{}{s.2g.}{Indivíduo natural ou habitante dessa cidade. }{lis.bo.e.ta}{0}
\verb{lisbonense}{}{}{}{}{adj.2g. e s.2g.}{Lisboeta.}{lis.bo.nen.se}{0}
\verb{lisérgico}{}{Quím.}{}{}{adj.}{Diz"-se de um ácido com propriedades alucinógenas.}{li.sér.gi.co}{0}
\verb{liso}{}{}{}{}{adj.}{Que é plano e não apresenta asperezas.}{li.so}{0}
\verb{liso}{}{}{}{}{}{Que é de uma única cor, sem estampas nem desenhos.}{li.so}{0}
\verb{liso}{}{Pop.}{}{}{}{Sem dinheiro; duro.}{li.so}{0}
\verb{lisonja}{}{}{}{}{s.f.}{Ato ou efeito de lisonjear.}{li.son.ja}{0}
\verb{lisonja}{}{}{}{}{}{Elogio exagerado; adulação, bajulação.}{li.son.ja}{0}
\verb{lisonjaria}{}{}{}{}{s.f.}{Ato de lisonjear.}{li.son.ja.ri.a}{0}
\verb{lisonjaria}{}{}{}{}{}{Louvor exagerado; adulação. }{li.son.ja.ri.a}{0}
\verb{lisonjeador}{ô}{}{}{}{adj.}{Que lisonjeia; adulador, lisonjeiro.}{li.son.je.a.dor}{0}
\verb{lisonjear}{}{}{}{}{v.t.}{Elogiar com exagero; bajular, adular.}{li.son.je.ar}{0}
\verb{lisonjear}{}{}{}{}{}{Agradar, satisfazer.}{li.son.je.ar}{\verboinum{4}}
\verb{lisonjeiro}{ê}{}{}{}{adj.}{Lisonjeador.}{li.son.jei.ro}{0}
\verb{lista}{}{}{}{}{s.f.}{Sequência de nomes escritos numa folha; rol, relação.}{lis.ta}{0}
\verb{lista}{}{}{}{}{}{Tira comprida e estreita.}{lis.ta}{0}
\verb{lista}{}{}{}{}{}{Risca, listra, traço.}{lis.ta}{0}
\verb{listado}{}{}{}{}{adj.}{Listrado.}{lis.ta.do}{0}
\verb{listagem}{}{}{"-ens}{}{s.f.}{Lista.}{lis.ta.gem}{0}
\verb{listar}{}{}{}{}{v.t.}{Pôr em uma lista; arrolar, relacionar, inscrever, registrar.}{lis.tar}{0}
\verb{listra}{}{}{}{}{s.f.}{Num tecido, risca ou lista de cor diferente. }{lis.tra}{0}
\verb{listrado}{}{}{}{}{adj.}{Que tem listras; listado.}{lis.tra.do}{0}
\verb{listrar}{}{}{}{}{v.t.}{Entrelaçar ou adornar com listras.}{lis.trar}{0}
\verb{lisura}{}{}{}{}{s.f.}{Qualidade do que é liso.}{li.su.ra}{0}
\verb{lisura}{}{Fig.}{}{}{}{Integridade de caráter; retidão, honestidade, honradez.}{li.su.ra}{0}
\verb{litania}{}{}{}{}{s.f.}{Ladainha.}{li.ta.ni.a}{0}
\verb{liteira}{ê}{}{}{}{s.f.}{Antiga cadeirinha portátil sustentada por dois varais e carregada por dois homens ou dois animais.}{li.tei.ra}{0}
\verb{literal}{}{}{"-ais}{}{adj.2g.}{Expresso em letras.}{li.te.ral}{0}
\verb{literal}{}{}{"-ais}{}{}{Conforme o sentido original e próprio das palavras; exato, rigoroso.}{li.te.ral}{0}
\verb{literário}{}{}{}{}{adj.}{Relativo a literatura.}{li.te.rá.rio}{0}
\verb{literato}{}{}{}{}{s.m.}{Indivíduo que escreve obras literárias; escritor.}{li.te.ra.to}{0}
\verb{literato}{}{}{}{}{}{Indivíduo que tem bom conhecimento de literatura; letrado.}{li.te.ra.to}{0}
\verb{literatura}{}{}{}{}{s.f.}{Arte de compor obras em prosa ou verso.}{li.te.ra.tu.ra}{0}
\verb{literatura}{}{}{}{}{}{Conjunto das obras sobre determinado assunto; bibliografia.}{li.te.ra.tu.ra}{0}
\verb{literatura}{}{}{}{}{}{Conjunto das obras ou escritos literários de determinada época ou país.}{li.te.ra.tu.ra}{0}
\verb{lítico}{}{}{}{}{adj.}{Relativo a pedra.}{lí.ti.co}{0}
\verb{litigante}{}{}{}{}{adj.2g.}{Que litiga.}{li.ti.gan.te}{0}
\verb{litigante}{}{}{}{}{}{Que se refere a litígio.}{li.ti.gan.te}{0}
\verb{litigante}{}{}{}{}{s.2g.}{Pessoa que sustenta litígio; contendor, parte.}{li.ti.gan.te}{0}
\verb{litigar}{}{}{}{}{v.t.}{Entrar em litígio; disputar, pleitear.}{li.ti.gar}{\verboinum{5}}
\verb{litígio}{}{}{}{}{s.m.}{Conflito de interesses; questão, disputa.}{litígio}{0}
\verb{litigioso}{ô}{}{}{}{adj.}{Que envolve litígio.}{li.ti.gi.o.so}{0}
\verb{litigioso}{ô}{}{}{}{}{Que depende de sentença judicial.}{li.ti.gi.o.so}{0}
\verb{lítio}{}{Quím.}{}{}{s.m.}{Elemento químico da família dos alcalinos, usado em ligas metálicas, em baterias etc. \elemento{3}{6.941}{Li.}}{lí.tio}{0}
\verb{litografar}{}{}{}{}{v.t.}{Reproduzir por meio da litografia.}{li.to.gra.far}{0}
\verb{litografia}{}{}{}{}{s.f.}{Processo de gravação de texto ou desenho em pedra ou metal.}{li.to.gra.fi.a}{0}
\verb{litografia}{}{}{}{}{}{Gravura feita por esse processo.}{li.to.gra.fi.a}{0}
\verb{litográfico}{}{}{}{}{adj.}{Relativo a litografia.}{li.to.grá.fi.co}{0}
\verb{litógrafo}{}{}{}{}{s.m.}{Pessoa que grava, desenha ou imprime por meio da litografia.}{li.tó.gra.fo}{0}
\verb{litogravura}{}{}{}{}{s.f.}{Litografia.}{li.to.gra.vu.ra}{0}
\verb{litologia}{}{}{}{}{s.f.}{Petrologia.}{li.to.lo.gi.a}{0}
\verb{litologia}{}{}{}{}{}{}{li.to.lo.gi.a}{0}
\verb{litoral}{}{}{}{}{adj.2g.}{Relativo a beira"-mar; litorâneo.}{li.to.ral}{0}
\verb{litoral}{}{}{}{}{s.m.}{Faixa de terra banhada pelo mar ou situada à beira"-mar; costa.}{li.to.ral}{0}
\verb{litorâneo}{}{}{}{}{adj.}{Litoral.}{li.to.râ.neo}{0}
\verb{litorina}{}{Bras.}{}{}{s.f.}{Automotriz.}{li.to.ri.na}{0}
\verb{litosfera}{é}{Geogr.}{}{}{s.f.}{A parte externa consolidada da Terra; crosta da Terra, crosta terrestre, orosfera. }{li.tos.fe.ra}{0}
\verb{litro}{}{}{}{}{s.m.}{Unidade de medida de capacidade do sistema métrico, em geral usada para líquidos, e que equivale a um decímetro cúbico.[Símb.: l] }{li.tro}{0}
\verb{litro}{}{}{}{}{}{Garrafa que contém um litro.}{li.tro}{0}
\verb{lituano}{}{}{}{}{adj.}{Relativo à República da Lituânia.}{li.tu.a.no}{0}
\verb{lituano}{}{}{}{}{s.m.}{Indivíduo natural ou habitante desse país.}{li.tu.a.no}{0}
\verb{lituano}{}{}{}{}{}{A língua báltica falada nesse país.}{li.tu.a.no}{0}
\verb{liturgia}{}{}{}{}{s.f.}{O conjunto das cerimônias religiosas instituídas por uma Igreja.}{li.tur.gi.a}{0}
\verb{liturgia}{}{}{}{}{}{O conjunto das formas de culto.}{li.tur.gi.a}{0}
\verb{litúrgico}{}{}{}{}{adj.}{Que se refere a liturgia. }{li.túr.gi.co}{0}
\verb{livermório}{}{Quím.}{}{}{s.m.}{Elemento químico sintetizado em 2012, chamado até então de ``unun-héxio'', de carácter radioativo, superpesado, provavelmente metálico e sólido. \elemento{116}{(293)}{Lv}.}{li.ver.mó.rio}{0}
\verb{lividez}{ê}{}{}{}{s.f.}{Qualidade ou estado de lívido.}{li.vi.dez}{0}
\verb{lívido}{}{}{}{}{adj.}{Que perdeu a cor natural; muito pálido.}{lí.vi.do}{0}
\verb{livramento}{}{}{}{}{s.m.}{Ato ou efeito de livrar; libertação, soltura.}{li.vra.men.to}{0}
\verb{livrar}{}{}{}{}{v.t.}{Tornar livre; libertar.}{li.vrar}{0}
\verb{livrar}{}{}{}{}{}{Desvencilhar de situação difícil ou perigosa; pôr a salvo; defender.}{li.vrar}{\verboinum{1}}
\verb{livraria}{}{}{}{}{s.f.}{Estabelecimento onde se vendem livros.}{li.vra.ri.a}{0}
\verb{livre}{}{}{}{}{adj.2g.}{Que tem poder de escolha; independente.}{li.vre}{0}
\verb{livre}{}{}{}{}{}{Que não está sujeito a controle, restrições ou limitações.}{li.vre}{0}
\verb{livre}{}{}{}{}{}{Desocupado, vago, disponível.}{li.vre}{0}
\verb{livre}{}{}{}{}{}{Desimpedido, desobstruído, liberado.}{li.vre}{0}
\verb{livre}{}{}{}{}{}{Que não requer pagamento; gratuito, franco.}{li.vre}{0}
\verb{livre"-arbítrio}{}{}{livres"-arbítrios}{}{s.m.}{Capacidade que as pessoas têm de escolher o que fazer, baseadas apenas em suas vontades.}{li.vre"-ar.bí.trio}{0}
\verb{livre"-câmbio}{}{}{livres"-câmbios}{}{s.m.}{Prática econômica que se caracteriza pela liberdade de comércio entre duas nações, sem barreiras alfandegárias.}{li.vre"-câm.bio}{0}
%\verb{}{}{}{}{}{}{}{}{0}
\verb{livreco}{é}{}{}{}{s.m.}{Livro pequeno.}{li.vre.co}{0}
\verb{livreco}{é}{Pejor.}{}{}{}{Livro sem valor, insignificante ou sem mérito.}{li.vre.co}{0}
\verb{livre"-docência}{}{}{livres"-docências}{}{s.f.}{Título universitário que se obtém por concurso ou mérito. }{li.vre"-do.cên.cia}{0}
\verb{livre"-docência}{}{}{livres"-docências}{}{}{Atividade exercida por livre"-docente.}{li.vre"-do.cên.cia}{0}
\verb{livre"-docente}{}{}{livres"-docentes}{}{s.2g.}{Professor universitário a quem, por concurso ou mérito, foi concedida a livre"-docência.}{li.vre"-do.cen.te}{0}
\verb{livreiro}{ê}{}{}{}{adj.}{Relativo a produção de livros; livresco.}{li.vrei.ro}{0}
\verb{livreiro}{ê}{}{}{}{s.m.}{Pessoa que vende livros ou proprietário de livraria.}{li.vrei.ro}{0}
\verb{livre"-pensador}{ô}{}{livres"-pensadores ⟨ô⟩}{}{s.m.}{Indivíduo que, a respeito da religião, pensa livremente, só aceitando as doutrinas que se harmonizam com a sua razão.}{li.vre"-pen.sa.dor}{0}
\verb{livresco}{ê}{}{}{}{adj.}{Relativo a livro.}{li.vres.co}{0}
\verb{livresco}{ê}{}{}{}{}{Diz"-se da experiência adquirida exclusivamente por meio de livros.}{li.vres.co}{0}
\verb{livro}{}{}{}{}{s.m.}{Conjunto de folhas impressas e encadernadas.}{li.vro}{0}
\verb{livro}{}{}{}{}{}{Obra escrita em prosa ou verso sobre qualquer assunto.}{li.vro}{0}
\verb{livro}{}{}{}{}{}{Cada uma das partes em que se divide uma obra extensa.}{li.vro}{0}
\verb{livro}{}{}{}{}{}{Caderno em que se registram atividades, assinaturas, endereços etc.}{li.vro}{0}
\verb{lixa}{ch}{}{}{}{s.f.}{Papel coberto com uma camada de produto abrasivo, que se usa para polir metais, madeiras, unhas etc.}{li.xa}{0}
\verb{lixão}{ch}{}{ões}{}{s.m.}{Grande depósito de lixo de uma cidade; aterro sanitário.}{li.xão}{0}
\verb{lixar}{ch}{}{}{}{v.t.}{Raspar ou polir com lixa.}{li.xar}{0}
\verb{lixar}{ch}{Pop.}{}{}{v.pron.}{Não dar importância; não se incomodar.}{li.xar}{\verboinum{1}}
\verb{lixeira}{ch}{}{}{}{s.f.}{Recipiente onde se põe lixo.}{li.xei.ra}{0}
\verb{lixeira}{ch}{}{}{}{}{Lugar onde o lixo é depositado; monturo.}{li.xei.ra}{0}
\verb{lixeira}{ch}{Por ext.}{}{}{}{Lugar sujo, imundo.}{li.xei.ra}{0}
\verb{lixeiro}{ch}{Bras.}{}{}{s.m.}{Pessoa encarregada de recolher o lixo.}{li.xei.ro}{0}
\verb{lixívia}{ch}{}{}{}{s.f.}{Água fervida com cinzas vegetais e soda, usada para branquear roupa; barrela.}{li.xí.via}{0}
\verb{lixiviação}{ch}{}{"-ões}{}{s.f.}{Ato ou efeito de lixiviar. }{li.xi.vi.a.ção}{0}
\verb{lixiviar}{ch}{}{}{}{v.t.}{Lavar com lixívia.}{li.xi.vi.ar}{0}
\verb{lixo}{ch}{}{}{}{s.m.}{Conjunto de coisas ou restos que não se aproveitam mais e se jogam fora.}{li.xo}{0}
\verb{lixo}{ch}{}{}{}{}{Recipiente em que se jogam ou se amontoam esses restos.}{li.xo}{0}
\verb{lo}{}{}{}{}{pron.}{Forma oblíqua de terceira pessoa do singular, equivalente a \textit{o}, usada: após formas verbais terminadas em \textit{r}, \textit{s} e \textit{z}; após os pronomes \textit{nos} e \textit{vos} e após o advérbio \textit{eis}.}{lo}{0}
\verb{ló}{}{}{}{}{s.m.}{Tipo de tecido fino e transparente.}{ló}{0}
\verb{loa}{}{}{}{}{s.f.}{Conjunto de versos ou cânticos em louvor às divindades.}{lo.a}{0}
\verb{loa}{}{}{}{}{}{Discurso feito para elogiar ou louvar alguém.}{lo.a}{0}
\verb{loba}{ô}{}{}{}{s.f.}{A fêmea do lobo.}{lo.ba}{0}
\verb{loba}{ô}{}{}{}{s.f.}{Batina de membro da Igreja.}{lo.ba}{0}
\verb{loba}{ô}{Desus.}{}{}{}{Prostituta.}{lo.ba}{0}
\verb{loba}{ô}{}{}{}{}{Roupa antiga que se arrastava pelo chão.}{lo.ba}{0}
\verb{lobacho}{}{}{}{}{s.m.}{Pequeno lobo; lobinho.}{lo.ba.cho}{0}
\verb{lobby}{}{}{}{}{s.m.}{Grande salão à entrada de hotéis, teatros, salões.}{\textit{lobby}}{0}
\verb{lobby}{}{}{}{}{}{Atuação de um grupo organizado de parlamentares com a finalidade de influenciar seus colegas a votarem de acordo com interesses próprios ou de grupos ligados a esses parlamentares.}{\textit{lobby}}{0}
\verb{lobinho}{}{}{}{}{s.m.}{Cisto sob a pele; lombinho, calombo.}{lo.bi.nho}{0}
\verb{lobinho}{}{Bras.}{}{}{}{Escoteiro com menos de dez anos de idade.}{lo.bi.nho}{0}
\verb{lobismo}{}{}{}{}{s.m.}{Prática de fazer \textit{lobby}.}{lo.bis.mo}{0}
\verb{lobisomem}{}{}{}{}{s.m.}{Criatura mitológica meio homem meio lobo que anda pela noite.}{lo.bi.so.mem}{0}
\verb{lobista}{}{}{}{}{s.2g.}{Indivíduo que faz \textit{lobby}.}{lo.bis.ta}{0}
\verb{lobo}{ó}{Anat.}{}{}{s.m.}{Parte de um órgão delimitada por sulcos ou marcas bem definidas.}{lo.bo}{0}
\verb{lobo}{ó}{Anat.}{}{}{}{A parte inferior e mole da orelha; lóbulo.}{lo.bo}{0}
\verb{lobo}{ô}{Fig.}{}{}{}{Indivíduo que evita o convívio social. }{lo.bo}{0}
\verb{lobo}{ô}{Zool.}{}{}{s.m.}{Mamífero carnívoro, de pelagem longa e cinzenta, que vive em grupos.}{lo.bo}{0}
\verb{lobo"-do"-mar}{ô}{Zool.}{lobos"-do"-mar ⟨ô⟩}{}{s.m.}{Peixe marinho com espinhos na nadadeira dorsal.}{lo.bo"-do"-mar}{0}
\verb{lobo"-do"-mar}{ô}{}{lobos"-do"-mar ⟨ô⟩}{}{}{Marinheiro com bastante experiência na vida marítima.}{lo.bo"-do"-mar}{0}
\verb{lobo"-marinho}{ô}{Zool.}{lobos"-marinhos ⟨ô⟩}{}{s.m.}{Leão"-marinho.}{lo.bo"-ma.ri.nho}{0}
\verb{lôbrego}{}{}{}{}{adj.}{Mal iluminado; sombrio, escuro.}{lô.bre.go}{0}
\verb{lôbrego}{}{}{}{}{}{Assustador, lúgubre, soturno.}{lô.bre.go}{0}
\verb{lobrigar}{}{}{}{}{v.t.}{Ver com dificuldade, especialmente na escuridão; entrever, avistar.}{lo.bri.gar}{0}
\verb{lobrigar}{}{Fig.}{}{}{}{Dar"-se conta de; perceber, entender.}{lo.bri.gar}{\verboinum{5}}
\verb{lobular}{}{}{}{}{adj.2g.}{Relativo a lóbulo.}{lo.bu.lar}{0}
\verb{lóbulo}{}{Anat.}{}{}{s.m.}{Pequeno lobo; lóbulo.}{ló.bu.lo}{0}
\verb{lobuloso}{}{}{}{}{adj.}{Dividido em lóbulos; lobular.}{lo.bu.lo.so}{0}
\verb{loca}{}{}{}{}{s.f.}{Tipo de gruta submersa que serve de abrigo ou toca para peixes.}{lo.ca}{0}
\verb{locação}{}{}{"-ões}{}{s.f.}{Ato ou efeito de locar.}{lo.ca.ção}{0}
\verb{locação}{}{Jur.}{"-ões}{}{}{Contrato em que uma das partes cede à outra, mediante pagamento, o uso de um bem móvel ou imóvel.}{lo.ca.ção}{0}
\verb{locação}{}{}{"-ões}{}{}{Local aberto ou fechado, fora de estúdio apropriado, em que são filmadas cenas de cinema ou televisão.}{lo.ca.ção}{0}
\verb{locador}{ô}{Jur.}{}{}{s.m.}{Em um contrato de locação, a parte que cede à outra o uso do bem.}{lo.ca.dor}{0}
\verb{locadora}{ô}{}{}{}{s.f.}{Estabelecimento que faz locação de bens (automóveis, filmes).}{lo.ca.do.ra}{0}
\verb{local}{}{}{}{}{s.m.}{Área, região.}{lo.cal}{0}
\verb{local}{}{}{}{}{}{Área específica, com limites definidos.}{lo.cal}{0}
\verb{local}{}{}{}{}{adj.2g.}{Relativo ou pertencente à mesma região em que se está ou em que se vive.}{lo.cal}{0}
\verb{localidade}{}{}{}{}{s.f.}{Área específica; lugar.}{lo.ca.li.da.de}{0}
\verb{localidade}{}{}{}{}{}{Qualidade do que é local.}{localidade}{0}
\verb{localização}{}{}{"-ões}{}{s.f.}{Ato ou efeito de localizar.}{lo.ca.li.za.ção}{0}
\verb{localização}{}{}{"-ões}{}{}{Lugar, região, área.}{lo.ca.li.za.ção}{0}
\verb{localizar}{}{}{}{}{v.t.}{Determinar o local onde se encontra alguém ou algo.}{lo.ca.li.zar}{0}
\verb{localizar}{}{}{}{}{}{Determinar o local onde algo deve ficar; colocar, assentar.}{lo.ca.li.zar}{0}
\verb{localizar}{}{}{}{}{v.pron.}{Estar situado em determinada região.}{lo.ca.li.zar}{0}
\verb{localizar}{}{}{}{}{}{Tomar consciência do local em que se está, observando referências, coordenadas.}{lo.ca.li.zar}{\verboinum{1}}
\verb{loção}{}{}{"-ões}{}{s.f.}{Produto próprio para uso tópico.}{lo.ção}{0}
\verb{loção}{}{}{"-ões}{}{}{Ato ou efeito de lavar o corpo com tecido úmido.}{lo.ção}{0}
\verb{locar}{}{}{}{}{v.t.}{Ceder a outrem, mediante contrato e pagamento, o uso de um bem móvel ou imóvel; arrendar, alugar.}{lo.car}{0}
\verb{locar}{}{}{}{}{}{Determinar o local de; localizar.}{lo.car}{0}
\verb{locatário}{}{}{}{}{s.m.}{Em um contrato de locação, a parte que recebe o direito de uso de um bem.}{lo.ca.tá.rio}{0}
\verb{lockout}{}{}{}{}{s.m.}{Fechamento de um estabelecimento (fábrica, comércio, usina etc.) por decisão do proprietário ou da direção para pressionar os trabalhadores a aceitarem condições muito desfavoráveis de trabalho.}{\textit{lockout}}{0}
\verb{locomoção}{}{}{"-ões}{}{s.f.}{Ato ou efeito de locomover"-se.}{lo.co.mo.ção}{0}
\verb{locomotiva}{}{}{}{}{s.f.}{Vagão que, em uma composição de trem, puxa ou empurra os outros, sendo dotado de máquinas ou motores que produzem movimento.}{lo.co.mo.ti.va}{0}
\verb{locomotividade}{}{}{}{}{s.f.}{Habilidade de locomover"-se, própria aos animais.}{lo.co.mo.ti.vi.da.de}{0}
\verb{locomotor}{ô}{}{}{}{adj.}{Relativo a locomoção.}{lo.co.mo.tor}{0}
\verb{locomotor}{ô}{}{}{}{}{Que produz locomoção, movimento.}{lo.co.mo.tor}{0}
\verb{locomotriz}{}{}{}{}{adj.}{Forma feminina de \textit{locomotor}.}{lo.co.mo.triz}{0}
\verb{locomóvel}{}{}{"-eis}{}{adj.2g.}{Que pode locomover"-se.}{lo.co.mó.vel}{0}
\verb{locomóvel}{}{}{"-eis}{}{s.f.}{Máquina sobre rodas movida a vapor, usada em amplos locais de trabalho.}{lo.co.mó.vel}{0}
\verb{locomover}{ê}{}{}{}{v.pron.}{Mudar de posição; deslocar"-se.}{lo.co.mo.ver}{0}
\verb{locução}{}{}{"-ões}{}{s.f.}{Maneira de produzir os sons da fala; dicção, articulação.}{lo.cu.ção}{0}
\verb{locução}{}{}{"-ões}{}{}{Maneira própria de se expressar; estilo, linguagem.}{lo.cu.ção}{0}
\verb{locução}{}{Gram.}{"-ões}{}{}{Conjunto de palavras que equivalem, gramaticalmente, a um só vocábulo, por formarem, no conjunto, um significado único.}{lo.cu.ção}{0}
\verb{locução}{}{}{"-ões}{}{}{Ato de falar diante de microfones.}{lo.cu.ção}{0}
\verb{locução}{}{}{"-ões}{}{}{Em um roteiro de rádio ou televisão, as partes a serem faladas pelo locutor.}{lo.cu.ção}{0}
\verb{locupletar}{}{}{}{}{v.t.}{Tornar rico; enriquecer.}{lo.cu.ple.tar}{0}
\verb{locupletar}{}{}{}{}{}{Tornar cheio; encher, abarrotar.}{locupletar}{0}
\verb{locutor}{ô}{}{}{}{s.m.}{Indivíduo que faz a apresentação de programas em rádio ou televisão.}{lo.cu.tor}{0}
\verb{locutor}{ô}{Gram.}{}{}{}{Indivíduo que produz os enunciados; falante.}{lo.cu.tor}{0}
\verb{locutor}{ô}{}{}{}{adj.}{Relativo a locução.}{lo.cu.tor}{0}
\verb{locutório}{}{}{}{}{s.m.}{Local separado por grades, existente em prisões e conventos, no qual os internos ficam para conversar com pessoas de fora.}{locutório}{0}
\verb{lodaçal}{}{}{}{}{s.m.}{Lugar em que há muito lodo; atoleiro.}{lo.da.çal}{0}
\verb{lodaçal}{}{Fig.}{}{}{}{Lugar degradante, decadente.}{lodaçal}{0}
\verb{lodo}{ô}{}{}{}{s.m.}{Material formado por terra e matéria em decomposição, encontrado no fundo das águas.}{lo.do}{0}
\verb{lodoso}{}{}{}{}{adj.}{Em que há lodo.}{lo.do.so}{0}
\verb{logaritmo}{}{Mat.}{}{}{s.m.}{Expoente ao qual se deve elevar a base para obter um número determinado.}{lo.ga.rit.mo}{0}
\verb{lógica}{}{}{}{}{s.f.}{Parte da filosofia que estuda as operações intelectuais como dedução, indução, hipótese etc.}{ló.gi.ca}{0}
\verb{lógica}{}{}{}{}{}{Encadeamento coerente de ideias; harmonia.}{ló.gi.ca}{0}
\verb{lógico}{}{}{}{}{adj.}{Relativo a lógica.}{ló.gi.co}{0}
\verb{lógico}{}{}{}{}{}{Racional, justo, coerente, correto, consistente.}{ló.gi.co}{0}
\verb{lógico}{}{}{}{}{adv.}{Evidentemente, obviamente, naturalmente, claro, certamente.}{ló.gi.co}{0}
\verb{lógico}{}{}{}{}{s.m.}{Indivíduo especialista em lógica.}{ló.gi.co}{0}
\verb{logística}{}{}{}{}{s.f.}{Planejamento e organização dos aspectos relacionados a transporte e abastecimento de tropas em uma ação militar.}{lo.gís.ti.ca}{0}
\verb{logística}{}{Por ext.}{}{}{}{Planejamento e administração dos detalhes necessários à execução adequada de qualquer operação.}{lo.gís.ti.ca}{0}
\verb{logo}{}{}{}{}{adv.}{Em um tempo posterior, imediato ou relativamente imediato.}{lo.go}{0}
\verb{logo}{}{}{}{}{conj.}{Portanto.}{lo.go}{0}
\verb{logogrifo}{}{}{}{}{s.m.}{Certo jogo de adivinhar palavras.}{lo.go.gri.fo}{0}
\verb{logogrifo}{}{}{}{}{}{Coisa obscura, linguagem enigmática.}{lo.go.gri.fo}{0}
\verb{logomarca}{}{}{}{}{s.f.}{Representação gráfica padronizada do nome de uma marca comercial acompanhada de seu símbolo visual.}{lo.go.mar.ca}{0}
\verb{logorreia}{é}{}{}{}{s.f.}{Fala demasiadamente abundante, com frases sem sentido ou em excesso.}{lo.gor.rei.a}{0}
\verb{logotipo}{}{}{}{}{s.m.}{Representação gráfica que identifica uma empresa ou marca, com o nome escrito em letra padronizada e estilizada.}{lo.go.ti.po}{0}
\verb{logradouro}{ô}{}{}{}{s.m.}{Qualquer lugar público, como praça, jardim, calçada etc.}{lo.gra.dou.ro}{0}
\verb{logradouro}{ô}{}{}{}{}{Aquilo que se pode lograr.}{lo.gra.dou.ro}{0}
\verb{lograr}{}{}{}{}{v.t.}{Conseguir, obter, alcançar.}{lo.grar}{0}
\verb{lograr}{}{}{}{}{}{Desfrutar, aproveitar, usufruir.}{lo.grar}{0}
\verb{lograr}{}{}{}{}{}{Enganar, iludir, ludibriar.}{lo.grar}{0}
\verb{lograr}{}{}{}{}{v.i.}{Surtir efeito.}{lo.grar}{0}
\verb{logro}{}{}{}{}{s.m.}{Ato ou efeito de lograr.}{lo.gro}{0}
\verb{logro}{}{}{}{}{}{Embuste, fraude, burla.}{logro}{0}
\verb{loiça}{}{}{}{}{}{Var. de \textit{louça}.}{loi.ça}{0}
\verb{loireiro}{ê}{}{}{}{}{Var. de \textit{loureiro}.}{loi.rei.ro}{0}
\verb{loiro}{ô}{}{}{}{adj.}{Louro.}{loi.ro}{0}
\verb{loisa}{}{}{}{}{}{Var. de \textit{lousa}.}{loi.sa}{0}
\verb{loja}{}{}{}{}{s.f.}{Estabelecimento em que se vendem mercadorias.}{lo.ja}{0}
\verb{loja}{}{}{}{}{}{Pavimento térreo de uma edificação, utilizado como oficina, escritório, armazém etc.}{lo.ja}{0}
\verb{lojista}{}{}{}{}{adj.2g.}{Relativo a atividade de comércio feita em lojas.}{lo.jis.ta}{0}
\verb{lojista}{}{}{}{}{s.2g.}{Indivíduo proprietário de loja comercial.}{lo.jis.ta}{0}
\verb{lomba}{}{}{}{}{s.f.}{Dorso de montanha ou colina.}{lom.ba}{0}
\verb{lomba}{}{}{}{}{}{Monte de areia ou terra formado pelo vento.}{lom.ba}{0}
\verb{lombada}{}{}{}{}{s.f.}{Dorso de rês bovina.}{lom.ba.da}{0}
\verb{lombada}{}{}{}{}{}{Longo dorso de montanha ou colina; lombada.}{lom.ba.da}{0}
\verb{lombada}{}{}{}{}{}{Pequena elevação feita no leito de rua ou estrada para obrigar os veículos a reduzirem a velocidade.}{lom.ba.da}{0}
\verb{lombada}{}{}{}{}{}{Lado da espessura do livro em que fica a costura, revestido pela encadernação e no qual se escrevem o nome da obra e do autor.}{lom.ba.da}{0}
\verb{lombar}{}{}{}{}{adj.2g.}{Relativo a lombo.}{lom.bar}{0}
\verb{lombar}{}{Anat.}{}{}{}{Diz"-se da região posterior do abdômen, na metade inferior baixa das costas.}{lom.bar}{0}
\verb{lombardo}{}{}{}{}{adj.}{Relativo à Lombardia, região da Itália.}{lom.bar.do}{0}
\verb{lombardo}{}{}{}{}{s.m.}{Indivíduo natural ou habitante dessa região.}{lom.bar.do}{0}
\verb{lombardo}{}{}{}{}{}{Língua falada nessa região.}{lom.bar.do}{0}
\verb{lombeira}{ê}{Pop.}{}{}{s.f.}{Preguiça física; indolência, sonolência, moleza.}{lom.bei.ra}{0}
\verb{lombilho}{}{Bras.}{}{}{s.m.}{Parte principal dos arreios que pode substituir a sela comum.}{lom.bi.lho}{0}
\verb{lombilho}{}{}{}{}{}{Músculo lombar da rês, apreciado na culinária.}{lom.bi.lho}{0}
\verb{lombinho}{}{}{}{}{s.m.}{Porção de carne macia próxima da espinha dorsal de animais de corte, muito apreciada em culinária.}{lom.bi.nho}{0}
\verb{lombo}{}{Pop.}{}{}{s.m.}{Região das costas abaixo das costelas.}{lom.bo}{0}
\verb{lombo}{}{}{}{}{}{Dorso, costas.}{lom.bo}{0}
\verb{lombo}{}{}{}{}{}{Carne localizada na região da coluna vertebral em animais de corte, muito apreciada em culinária.}{lom.bo}{0}
\verb{lombrical}{}{}{}{}{adj.2g.}{Relativo a lombriga.}{lom.bri.cal}{0}
\verb{lombricida}{}{}{}{}{adj.2g.}{Que mata lombrigas; lumbricida.}{lom.bri.ci.da}{0}
\verb{lombriga}{}{Biol.}{}{}{s.f.}{Designação comum de várias espécies de vermes parasitas do intestino.}{lom.bri.ga}{0}
\verb{lombrigueiro}{ê}{}{}{}{s.m.}{Remédio para eliminar lombrigas.}{lom.bri.guei.ro}{0}
\verb{lombudo}{}{}{}{}{adj.}{Que tem grandes lombos.}{lom.bu.do}{0}
\verb{lona}{}{}{}{}{s.f.}{Tecido resistente, de algodão, linho ou cânhamo, usado na fabricação de sacos, velas, toldos.}{lo.na}{0}
\verb{londrino}{}{}{}{}{adj.}{Relativo a Londres, capital da Inglaterra.}{lon.dri.no}{0}
\verb{londrino}{}{}{}{}{s.m.}{Indivíduo natural ou habitante dessa cidade.}{lon.dri.no}{0}
\verb{longa}{}{}{}{}{s.m.}{Redução de \textit{longa"-metragem}}{lon.ga}{0}
\verb{longa}{}{}{}{}{}{Certa arma de cano comprido e estreito.}{longa}{0}
\verb{longa}{}{Mús.}{}{}{s.f.}{Nota que tem o valor de duas breves.}{longa}{0}
\verb{longa"-metragem}{}{}{longas"-metragens}{}{s.m.}{Filme cinematográfico com mais de 70 minutos de duração.}{lon.ga"-me.tra.gem}{0}
\verb{longânime}{}{}{}{}{adj.}{Generoso, altruísta, bondoso.}{lon.gâ.ni.me}{0}
\verb{longânime}{}{}{}{}{}{Paciente, resignado.}{lon.gâ.ni.me}{0}
\verb{longarina}{}{}{}{}{s.f.}{Viga longitudinal que constitui geralmente o principal elemento estrutural de pontes, veículos, embarcações, edificações.}{lon.ga.ri.na}{0}
\verb{longe}{}{}{}{}{adv.}{A grande distância de um ponto; distante. (\textit{Algumas crianças moram muito longe da escola em que estudam.})}{lon.ge}{0}
\verb{longe}{}{}{}{}{adj.2g.}{Distante, longínquo. (\textit{O rapaz partiu para longes terras.})}{lon.ge}{0}
\verb{longevidade}{}{}{}{}{s.f.}{Duração da vida, especialmente aquela acima da média.}{lon.ge.vi.da.de}{0}
\verb{longevo}{é}{}{}{}{adj.}{Que tem ou teve vida longa.}{lon.ge.vo}{0}
\verb{longevo}{é}{}{}{}{}{Duradouro.}{lon.ge.vo}{0}
\verb{longilíneo}{}{}{}{}{adj.}{De forma longa e delgada.}{lon.gi.lí.neo}{0}
\verb{longínquo}{}{}{}{}{adj.}{Afastado no tempo ou no espaço; remoto.}{lon.gín.quo}{0}
\verb{longínquo}{}{Fig.}{}{}{}{Alheio, distante, ausente.}{lon.gín.quo}{0}
\verb{longitude}{}{Geogr.}{}{}{s.f.}{Ângulo entre o meridiano de \textit{Greenwich} e o ponto em questão.}{lon.gi.tu.de}{0}
\verb{longitude}{}{}{}{}{}{Distância, loujura.}{longitude}{0}
\verb{longitudinal}{}{}{"-ais}{}{adj.2g.}{Relativo a longitude.}{lon.gi.tu.di.nal}{0}
\verb{longitudinal}{}{}{"-ais}{}{}{Posicionado no sentido do comprimento.}{lon.gi.tu.di.nal}{0}
\verb{longo}{}{}{}{}{adj.}{Extenso, comprido.}{lon.go}{0}
\verb{longo}{}{}{}{}{}{Que tem grande duração; prolongado, demorado.}{lon.go}{0}
\verb{long"-play}{}{}{}{}{s.m.}{Disco de 10 ou 12 polegadas de diâmetro, com gravação fonográfica feita em microssulcos no material plástico, geralmente vinil; elepê.}{\textit{long"-play}}{0}
\verb{lonjura}{}{Pop.}{}{}{s.f.}{Grande distância.}{lon.ju.ra}{0}
\verb{lontra}{}{Zool.}{}{}{s.f.}{Designação comum a certas espécies de mamíferos carnívoros aquáticos ou semiaquáticos.}{lon.tra}{0}
\verb{loquacidade}{}{}{}{}{s.f.}{Qualidade de loquaz; eloquência.}{lo.qua.ci.da.de}{0}
\verb{loquaz}{}{}{}{}{adj.}{Que fala bastante; falador.}{lo.quaz}{0}
\verb{loquaz}{}{}{}{}{}{Que fala com facilidade e habilidade; eloquente, facundo.}{lo.quaz}{0}
\verb{loquela}{é}{}{}{}{s.f.}{Faculdade de falar; fala.}{lo.que.la}{0}
\verb{lorde}{}{}{}{}{s.m.}{Título de nobreza na Inglaterra.}{lor.de}{0}
\verb{lorde}{}{Pop.}{}{}{adj.2g.}{Elegante, luxento, vistoso.}{lor.de}{0}
\verb{lordose}{}{Med.}{}{}{s.f.}{Desvio da coluna vertebral com convexidade anterior.}{lor.do.se}{0}
\verb{loro}{}{}{}{}{s.m.}{Correia dupla afivelada à sela para sustentar o estribo.}{lo.ro}{0}
\verb{loro}{}{}{}{}{}{Parte da cabeça das aves entre os olhos e a base do bico.}{lo.ro}{0}
\verb{lorota}{}{}{}{}{s.f.}{Conversa fiada; mentira, falsidade.}{lo.ro.ta}{0}
\verb{loroteiro}{ê}{Pop.}{}{}{adj.}{Que conta lorotas; mentiroso.}{lo.ro.tei.ro}{0}
\verb{lorpa}{}{}{}{}{adj.}{Que demonstra pouca inteligência; idiota, imbecil.}{lor.pa}{0}
\verb{losango}{}{Geom.}{}{}{s.m.}{Quadrilátero com todos os lados iguais, dois ângulos agudos e dois obtusos.}{lo.san.go}{0}
\verb{losna}{ó}{Bot.}{}{}{s.f.}{Erva aromática com propriedades medicinais; absinto.}{los.na}{0}
\verb{lotação}{}{}{"-ões}{}{s.f.}{Ato ou efeito de lotar.}{lo.ta.ção}{0}
\verb{lotação}{}{}{"-ões}{}{}{Número de pessoas que comporta um recinto ou veículo.}{lo.ta.ção}{0}
\verb{lotação}{}{}{"-ões}{}{}{Cálculo para controlar lotes ou rendimentos.}{lo.ta.ção}{0}
\verb{lotação}{}{Bras.}{"-ões}{}{s.m.}{Tipo de transporte coletivo feito em veículo com poucos lugares.}{lo.ta.ção}{0}
\verb{lotado}{}{}{}{}{adj.}{Que se lotou.}{lo.ta.do}{0}
\verb{lotado}{}{}{}{}{}{Cuja capacidade está completa; cheio.}{lo.ta.do}{0}
\verb{lotado}{}{}{}{}{}{Designado para determinado setor, departamento.}{lo.ta.do}{0}
\verb{lotado}{}{}{}{}{}{Orçado, calculado.}{lo.ta.do}{0}
\verb{lotar}{}{}{}{}{v.t.}{Esgotar completamente a capacidade (de recinto ou veículo).}{lo.tar}{0}
\verb{lotar}{}{}{}{}{}{Designar ou alocar (funcionário) em determinado setor, departamento.}{lo.tar}{0}
\verb{lotar}{}{}{}{}{}{Repartir em lotes.}{lo.tar}{0}
\verb{lote}{}{}{}{}{s.m.}{Porção de objetos ou animais semelhantes.}{lo.te}{0}
\verb{lote}{}{}{}{}{}{Porção de terra desmembrada de uma extensão maior.}{lo.te}{0}
\verb{lote}{}{}{}{}{}{Cada parte sorteada; quinhão.}{lo.te}{0}
\verb{loteamento}{}{}{}{}{s.m.}{Ato, processo ou efeito de lotear, dividir porções de terra.}{lo.te.a.men.to}{0}
\verb{loteamento}{}{Por ext.}{}{}{}{A extensão de terra loteada.}{lo.te.a.men.to}{0}
\verb{lotear}{}{}{}{}{v.t.}{Dividir uma extensão de terra em partes menores para venda.}{lo.te.ar}{\verboinum{4}}
\verb{loteca}{}{Pop.}{}{}{s.f.}{Redução de \textit{loteria esportiva}, jogo em que os participantes tentam adivinhar o resultado de partidas de futebol previamente estabelecidas.}{lo.te.ca}{0}
\verb{loteria}{}{}{}{}{s.f.}{Sistema de distribuir bens ou prêmios entre indivíduos mediante sorteio.}{lo.te.ri.a}{0}
\verb{loteria}{}{}{}{}{}{Jogo de azar oficial em que o participante compra bilhetes numerados, com sorteios em catas previamente estabelecidas.}{lo.te.ri.a}{0}
\verb{loteria}{}{Fig.}{}{}{}{Acontecimento de resultado imprevisível, determinado pelo acaso.}{lo.te.ri.a}{0}
\verb{lotérico}{}{}{}{}{adj.}{Relativo a loteria.}{lo.té.ri.co}{0}
\verb{loto}{ó}{Bot.}{}{}{s.m.}{Lótus.}{lo.to}{0}
\verb{loto}{ô}{}{}{}{s.m.}{Jogo de azar que utiliza peças numeradas sorteadas de um saco e cartões numerados que vão sendo marcados pelos jogadores conforme o sorteio das peças.}{lo.to}{0}
\verb{loto}{ó}{Bras.}{}{}{s.f.}{Modalidade de loteria em que os jogadores escolhem números e são premiados se os números escolhidos forem sorteados em data previamente estabelecida.}{lo.to}{0}
\verb{lótus}{}{Bot.}{}{}{s.m.}{Certa planta aquática com flores brancas, rosadas, azuis ou violáceas.}{ló.tus}{0}
\verb{louça}{}{}{}{}{s.f.}{Produto de cerâmica cozida e esmaltada, geralmente para uso doméstico.}{lou.ça}{0}
\verb{louça}{}{}{}{}{}{O conjunto dos recipientes de porcelana ou material semelhante utilizado no serviço de mesa.}{lou.ça}{0}
\verb{louçania}{}{}{}{}{s.f.}{Qualidade de loução.}{lou.ça.ni.a}{0}
\verb{loução}{}{}{"-ãos}{louçã}{adj.}{Provido de adorno; enfeitado.}{lou.ção}{0}
\verb{loução}{}{Pejor.}{"-ãos}{louçã}{}{Excessivamente elegante; garboso.}{lou.ção}{0}
\verb{louco}{ô}{}{}{}{adj.}{Que perdeu a razão; alienado, doido.}{lou.co}{0}
\verb{louco}{ô}{}{}{}{}{Fora de si; transtornado.}{lou.co}{0}
\verb{louco}{ô}{}{}{}{}{Descontrolado.}{lou.co}{0}
\verb{loucura}{}{}{}{}{s.f.}{Estado ou condição de louco; insanidade mental.}{lou.cu.ra}{0}
\verb{loucura}{}{}{}{}{}{Ato próprio de louco.}{lou.cu.ra}{0}
\verb{loureiro}{ê}{Bot.}{}{}{s.m.}{Árvore, nativa de regiões áridas do Mediterrâneo, cuja folha é usada como tempero.}{lou.rei.ro}{0}
\verb{lourejar}{}{}{}{}{v.t.}{Tornar louro; apresentar a cor loura.}{lou.re.jar}{0}
\verb{louro}{ô}{}{}{}{s.m.}{A folha do loureiro usada como tempero.}{lou.ro}{0}
\verb{louro}{ô}{Pop.}{}{}{}{Papagaio.}{lou.ro}{0}
\verb{louro}{ô}{}{}{}{adj.}{Que tem os cabelos claros.}{lou.ro}{0}
\verb{louros}{}{Fig.}{}{}{s.m.pl.}{Glórias, triunfos.}{lou.ros}{0}
\verb{lousa}{ô}{}{}{}{s.f.}{Quadro de ardósia de tamanhos variados, com moldura de madeira, que se usa nas escolas, para nele se escrever a giz; quadro"-negro.}{lou.sa}{0}
\verb{louva"-a"-deus}{}{Zool.}{}{}{s.m.}{Inseto verde que tem o hábito de manter as patas dianteiras levantadas, lembrando uma pessoa ajoelhada em oração.}{lou.va"-a"-deus}{0}
\verb{louvação}{}{}{"-ões}{}{s.f.}{Ato ou efeito de louvar; louvor, exaltação.}{lou.va.ção}{0}
\verb{louvado}{}{}{}{}{adj.}{Que recebeu louvor.}{lou.va.do}{0}
\verb{louvado}{}{}{}{}{s.m.}{Indivíduo nomeado ou escolhido para avaliar, decidir alguma demanda ou sobre ela apresentar laudo; avaliador.}{louvado}{0}
\verb{louvaminha}{}{}{}{}{s.f.}{Louvor afetado; adulação, bajulação.}{lou.va.mi.nha}{0}
\verb{louvaminhar}{}{}{}{}{v.t.}{Tecer louvaminhas; lisonjear.}{lou.va.mi.nhar}{\verboinum{1}}
\verb{louvaminheiro}{ê}{}{}{}{adj.}{Que encerra ou envolve louvaminha.}{lou.va.mi.nhei.ro}{0}
\verb{louvaminheiro}{ê}{}{}{}{}{Que é dado a louvaminhar; adulador, bajulador.}{lou.va.mi.nhei.ro}{0}
\verb{louvar}{}{}{}{}{v.t.}{Enaltecer com palavras; dirigir louvores a; elogiar, gabar.}{lou.var}{0}
\verb{louvar}{}{}{}{}{}{Bendizer, glorificar.}{lou.var}{\verboinum{1}}
\verb{louvável}{}{}{"-eis}{}{adj.2g.}{que se deve louvar; digno de louvor.}{lou.vá.vel}{0}
\verb{louvor}{ô}{}{}{}{s.m.}{Celebração ou manifestação honrosa; homenagem, honraria.}{lou.vor}{0}
\verb{louvor}{ô}{}{}{}{}{Enaltecimento dos méritos de alguém; elogio.}{lou.vor}{0}
\verb{Lr}{}{Quím.}{}{}{}{Símb. do \textit{laurêncio}.}{Lr}{0}
\verb{LSD}{}{}{}{}{s.m.}{Psicotrópico artificial, conhecido como \textit{ácido}.}{LSD}{0}
\verb{LSD}{}{Quím.}{}{}{}{Abrev. de \textit{Lysergic Saure Diethylamide}.}{LSD}{0}
\verb{Lu}{}{Quím.}{}{}{}{Símb. do \textit{lutécio}.}{Lu}{0}
\verb{lua}{}{}{}{}{s.f.}{Satélite natural da Terra, com quatro fases.}{lu.a}{0}
\verb{lua}{}{}{}{}{}{Espaço de um mês.}{lu.a}{0}
\verb{lua}{}{}{}{}{}{Satélite de um planeta qualquer.}{lu.a}{0}
\verb{lua"-de"-mel}{é}{}{luas"-de"-mel ⟨é⟩}{}{s.f.}{Período de lazer e intimidade desfrutado pelos casados, logo depois do casamento.}{lu.a"-de"-mel}{0}
\verb{lua"-de"-mel}{é}{Por ext.}{luas"-de"-mel ⟨é⟩}{}{}{Viagem que o casal costuma fazer depois do casamento ou em outro período da vida conjugal.}{lu.a"-de"-mel}{0}
\verb{lua"-de"-mel}{é}{Fig.}{luas"-de"-mel ⟨é⟩}{}{}{Qualquer momento ou situação, em relações ou negócios, vivido com extrema harmonia e bom humor.}{lu.a"-de"-mel}{0}
\verb{luar}{}{}{}{}{}{O clarão que Lua espalha sobre a Terra.}{lu.ar}{0}
\verb{luar}{}{}{}{}{s.m.}{Luminosidade refletida sobre a Lua ao ser iluminada pelo Sol.}{lu.ar}{0}
\verb{lubricidade}{}{}{}{}{s.f.}{Qualidade de lúbrico; escorregadio, úmido ou liso aponta de fazer escorregar.}{lu.bri.ci.da.de}{0}
\verb{lubricidade}{}{Fig.}{}{}{}{Lascívia, sensualidade.}{lu.bri.ci.da.de}{0}
\verb{lúbrico}{}{}{}{}{adj.}{Escorregadio, resvaladiço.}{lú.bri.co}{0}
\verb{lúbrico}{}{}{}{}{}{Úmido ou liso a ponto de fazer escorregar.}{lú.bri.co}{0}
\verb{lúbrico}{}{Fig.}{}{}{}{Lascivo, sensual.}{lú.bri.co}{0}
\verb{lubrificação}{}{}{"-ões}{}{s.f.}{Ato ou efeito de lubrificar.}{lu.bri.fi.ca.ção}{0}
\verb{lubrificante}{}{}{}{}{adj.2g.}{Que lubrifica, que torna escorregadio.}{lu.bri.fi.can.te}{0}
\verb{lubrificante}{}{}{}{}{}{Diz"-se de substância empregada para reduzir o atrito de eixos e peças que se movem uma sobre a outra.}{lu.bri.fi.can.te}{0}
\verb{lubrificar}{}{}{}{}{v.t.}{Tornar lúbrico ou escorregadio.}{lu.bri.fi.car}{0}
\verb{lubrificar}{}{}{}{}{}{Untar com substância oleosa a fim de atenuar o atrito.}{lu.bri.fi.car}{\verboinum{2}}
\verb{lucarna}{}{Por ext.}{}{}{}{Fresta.}{lu.car.na}{0}
\verb{lucarna}{}{}{}{}{s.f.}{Abertura no teto de uma casa para dar luz.}{lu.car.na}{0}
\verb{lucerna}{é}{}{}{}{s.f.}{Pequena luz ou fonte de luz, posta em local bem alto; candeia.}{lu.cer.na}{0}
\verb{lucerna}{é}{}{}{}{}{Armação com várias lâmpadas; candelabro.}{lu.cer.na}{0}
\verb{lucidez}{ê}{}{}{}{s.f.}{Qualidade ou estado de lúcido.}{lu.ci.dez}{0}
\verb{lúcido}{}{}{}{}{adj.}{Que conhece, compreende, apreende; consciente, racional.}{lú.ci.do}{0}
\verb{lúcifer}{}{}{}{}{s.m.}{Diabo.}{lú.ci.fer}{0}
\verb{lucilar}{}{}{}{}{v.i.}{Brilhar com pouca intensidade; luzir frouxamente.}{lu.ci.lar}{\verboinum{1}}
\verb{lucrar}{}{}{}{}{v.t.}{Obter alguma coisa de um empreendimento qualquer; ganhar.}{lu.crar}{\verboinum{1}}
\verb{lucrativo}{}{}{}{}{adj.}{Que proporciona lucro ou vantagem; vantajoso, rentável.}{lu.cra.ti.vo}{0}
\verb{lucrativo}{}{}{}{}{}{De que se tira proveito; interessante, fútil.}{lu.cra.ti.vo}{0}
\verb{lucro}{}{}{}{}{s.m.}{Qualquer vantagem, benefício, que se pode tirar de alguma coisa.}{lu.cro}{0}
\verb{lucro}{}{}{}{}{}{Ganho comercial.}{lu.cro}{0}
\verb{lucubração}{}{}{"-ões}{}{s.f.}{Meditação longa e profunda; elucubração.}{lu.cu.bra.ção}{0}
\verb{lucubrar}{}{}{}{}{v.t.}{Refletir demoradamente sobre algo; elucubrar.}{lu.cu.brar}{\verboinum{1}}
\verb{ludibriante}{}{}{}{}{adj.2g.}{Que ludibria; enganador, zombador.}{lu.di.bri.an.te}{0}
\verb{ludibriar}{}{}{}{}{v.t.}{Fazer acreditar em algo que não é verdadeiro; enganar.}{lu.di.bri.ar}{0}
\verb{ludibriar}{}{}{}{}{}{Fazer troça de; escarnecer, zombar.}{lu.di.bri.ar}{\verboinum{6}}
\verb{ludíbrio}{}{}{}{}{s.m.}{Ato ou efeito de ludibriar; enganar.}{lu.dí.brio}{0}
\verb{ludíbrio}{}{}{}{}{}{Escárnio, zombaria.}{lu.dí.brio}{0}
\verb{lúdico}{}{}{}{}{adj.}{Que tem o caráter de jogos, brinquedos ou divertimento.}{lú.di.co}{0}
\verb{lues}{}{Med.}{}{}{s.f.pl.}{Doença infecciosa, transmissível por contato sexual ou à descendência, causada por bactéria e caracterizada por lesões na pele e mucosas; sífilis.}{lu.es}{0}
\verb{lufada}{}{}{}{}{s.f.}{Incidência súbita e curta de vento, luz etc.}{lu.fa.da}{0}
\verb{lufa"-lufa}{}{}{lufa"-lufas}{}{s.f.}{Grande pressa; agitação.}{lu.fa"-lu.fa}{0}
\verb{lugar}{}{}{}{}{s.m.}{Ponto em que alguma coisa está.}{lu.gar}{0}
\verb{lugar}{}{}{}{}{}{Espaço que pode ser ocupado.}{lu.gar}{0}
\verb{lugar}{}{}{}{}{}{Posição que se ocupa num conjunto.}{lu.gar}{0}
\verb{lugar}{}{}{}{}{}{Região onde mora uma população; localidade.}{lu.gar}{0}
\verb{lugar"-comum}{}{}{lugares"-comuns}{}{s.m.}{Fórmula, argumento ou ideia já muito conhecida e muito usada; clichê, chavão. }{lu.gar"-co.mum}{0}
\verb{lugarejo}{ê}{}{}{}{s.m.}{Povoado pequeno.}{lu.ga.re.jo}{0}
\verb{lugar"-tenente}{}{}{lugar"-tenentes \textit{ou}lugares"-tenentes}{}{s.2g.}{Pessoa que desempenha provisoriamente as funções de outra.}{lu.gar"-te.nen.te}{0}
\verb{lúgubre}{}{}{}{}{adj.2g.}{Que se refere a morte e a luto; fúnebre.}{lú.gu.bre}{0}
\verb{lúgubre}{}{}{}{}{}{Que é triste; sombrio.}{lú.gu.bre}{0}
\verb{lula}{}{Zool.}{}{}{s.f.}{Molusco de corpo alongado e mole, provido de dez tentáculos com ventosas, capaz de tomar a cor do lugar em que está.}{lu.la}{0}
\verb{lula}{}{Cul.}{}{}{}{Comida que é preparada com esse animal.}{lu.la}{0}
\verb{lumbago}{}{}{}{}{s.m.}{Dor lombar aguda.}{lum.ba.go}{0}
\verb{lumbricida}{}{}{}{}{adj.2g.}{Lombricida.}{lum.bri.ci.da}{0}
\verb{lume}{}{}{}{}{s.m.}{Calor e luz produzidos pela combustão; fogo.}{lu.me}{0}
\verb{lume}{}{}{}{}{}{Jato de luz; brilho, claridade.}{lu.me}{0}
\verb{luminar}{}{}{}{}{adj.2g.}{Que dá lume, que espalha luz.}{lu.mi.nar}{0}
\verb{luminar}{}{Fig.}{}{}{}{Sábio, doutor.}{lu.mi.nar}{0}
\verb{luminária}{}{}{}{}{s.f.}{Aparelho de iluminação que queima combustível e permite regular a distribuição da luz.}{lu.mi.ná.ria}{0}
\verb{luminária}{}{}{}{}{}{Conjunto de lâmpadas ligadas a um suporte central; lustre.}{lu.mi.ná.ria}{0}
\verb{luminária}{}{}{}{}{}{Coisa que ilumina.}{lu.mi.ná.ria}{0}
\verb{luminescência}{}{}{}{}{s.f.}{Emissão de luz por uma substância, provocada por qualquer processo que não seja o aquecimento}{luminescência}{0}
\verb{luminescente}{}{}{}{}{adj.2g.}{Que tem a propriedade de emitir luz em temperatura ordinária.}{luminescente}{0}
\verb{luminosidade}{}{}{}{}{s.f.}{Qualidade de luminoso.}{luminosidade}{0}
\verb{luminoso}{ô}{}{"-osos ⟨ó⟩}{"-osa ⟨ó⟩}{adj.}{Que espalha claridade.}{lu.mi.no.so}{0}
\verb{luminoso}{ô}{}{"-osos ⟨ó⟩}{"-osa ⟨ó⟩}{}{Em que se nota grande inteligência.}{lu.mi.no.so}{0}
\verb{luminoso}{ô}{}{"-osos ⟨ó⟩}{"-osa ⟨ó⟩}{s.m.}{Anúncio iluminado.}{lu.mi.no.so}{0}
\verb{lunação}{}{Astron.}{"-ões}{}{s.f.}{O tempo decorrido entre uma lua nova e a lua nova seguinte (aproximadamente 29 dias e meio).}{lu.na.ção}{0}
\verb{lunar}{}{}{}{}{adj.2g.}{Relativo à Lua.}{lu.nar}{0}
\verb{lunático}{}{}{}{}{adj.}{Que tem ideias incoerentes; maluco, maníaco.}{lu.ná.ti.co}{0}
\verb{lundu}{}{Mús.}{}{}{s.m.}{Canto e dança rural de origem africana.}{lun.du}{0}
\verb{lundum}{}{}{}{}{s.m.}{Lundu.}{lun.dum}{0}
\verb{luneta}{ê}{}{}{}{s.f.}{Instrumento óptico usado para se ver distintamente objetos a distância.}{lu.ne.ta}{0}
\verb{luniforme}{}{}{}{}{adj.2g.}{Que tem a forma da Lua.}{lu.ni.for.me}{0}
\verb{lupa}{}{}{}{}{s.f.}{Lente biconvexa que aumenta a imagem dos objetos.}{lupa}{0}
\verb{lupanar}{}{}{}{}{s.m.}{Prostíbulo.}{lu.pa.nar}{0}
\verb{lupino}{}{}{}{}{adj.}{Relativo ou pertencente a lobo.}{lu.pi.no}{0}
\verb{lúpulo}{}{}{}{}{s.m.}{Planta cujas inflorescências são usadas na fabricação de cerveja, para conferir sabor amargo à bebida.}{lú.pu.lo}{0}
\verb{lúpus}{}{Med.}{}{}{s.m.}{Doença autoimune do tecido conjuntivo, com manifestações cutâneas, cardiovasculares, renais, nervosas e articulares; lúpus eritematoso.}{lú.pus}{0}
\verb{lura}{}{}{}{}{s.f.}{Esconderijo de animais; toca, covil.}{lu.ra}{0}
\verb{lúrido}{}{}{}{}{adj.}{Descorado, pálido, lívido. }{lú.ri.do}{0}
\verb{lusco"-fusco}{}{}{lusco"-fuscos}{}{s.m.}{Hora de pouca luz, ao amanhecer ou entardecer.}{lus.co"-fus.co}{0}
\verb{lusíada}{}{}{}{}{adj.2g. e s.2g.}{Lusitano.}{lu.sí.a.da}{0}
\verb{lusitanismo}{}{}{}{}{s.m.}{Costume próprio dos lusitanos, dos portugueses.}{lu.si.ta.nis.mo}{0}
\verb{lusitanismo}{}{}{}{}{}{Palavra, locução ou construção própria do português de Portugal.}{lu.si.ta.nis.mo}{0}
\verb{lusitano}{}{}{}{}{adj.}{Relativo à Lusitânia.}{lu.si.ta.no}{0}
\verb{lusitano}{}{}{}{}{}{Relativo a Portugal.}{lu.si.ta.no}{0}
\verb{lusitano}{}{}{}{}{s.m.}{Indivíduo natural ou habitante da Lusitânia ou de Portugal. }{lu.si.ta.no}{0}
\verb{luso}{}{}{}{}{adj. e s.m.  }{Lusitano.}{lu.so}{0}
\verb{luso"-brasileiro}{ê}{}{luso"-brasileiros}{luso"-brasileira}{adj.}{Relativo a Portugal e ao Brasil, ou de origem portuguesa e brasileira.}{lu.so"-bra.si.lei.ro}{0}
%\verb{}{}{}{}{}{}{}{}{0}
\verb{lustração}{}{}{"-ões}{}{s.f.}{Ato ou efeito de lustrar, de dar brilho; polimento, enceramento.}{lus.tra.ção}{0}
\verb{lustral}{}{}{}{}{adj.2g.}{Que serve para lustrar ou para purificar.}{lus.tral}{0}
\verb{lustra"-móveis}{}{}{}{}{s.m.}{Produto químico usado para limpar móveis e lhes dar lustre. }{lus.tra"-mó.veis}{0}
\verb{lustrar}{}{}{}{}{v.t.}{Dar lustro, brilho; fazer luzir.}{lus.trar}{\verboinum{1}}
\verb{lustre}{}{}{}{}{s.m.}{Brilho que um objeto tem quando polido; lustro.}{lus.tre}{0}
\verb{lustre}{}{}{}{}{}{Luminária suspensa no teto.}{lus.tre}{0}
\verb{lustro}{}{}{}{}{s.m.}{Período de cinco anos; quinquênio.}{lus.tro}{0}
\verb{lustroso}{ô}{}{"-osos ⟨ó⟩}{"-osa ⟨ó⟩}{adj.}{Que tem lustre ou lustro; polido, luzidio.}{lus.tro.so}{0}
\verb{lustroso}{ô}{Por ext.}{"-osos ⟨ó⟩}{"-osa ⟨ó⟩}{}{Que é brilhante, luzidio.}{lus.tro.so}{0}
\verb{lua}{}{}{}{}{s.f.}{Ato ou efeito de lutar; conflito, combate.}{lu.a}{0}
\verb{lua}{}{}{}{}{}{Empenho, esforço.}{lu.a}{0}
\verb{lutador}{ô}{}{}{}{adj.}{Que luta.}{lu.ta.dor}{0}
\verb{lutador}{ô}{}{}{}{s.m.}{Esportista que pratica a luta.}{lu.ta.dor}{0}
\verb{lutador}{ô}{Fig.}{}{}{}{Pessoa determinada a alcançar um fim, que se empenha com afinco.}{lu.ta.dor}{0}
\verb{lutar}{}{}{}{}{v.t.}{Travar combate; brigar, pelejar.}{lu.tar}{0}
\verb{lutar}{}{}{}{}{}{Trabalhar muito para conseguir o que se deseja; empenhar"-se, esforçar"-se.}{lu.tar}{\verboinum{1}}
\verb{lutécio}{}{Quím.}{}{}{s.m.}{Elemento químico metálico, prateado, da família dos lantanídeos (terras"-raras); usado em radioterapia, tecnologia nuclear etc. \elemento{71}{174.967}{Lu}.}{lu.té.cio}{0}
\verb{luteranismo}{}{Relig.}{}{}{s.m.}{A doutrina religiosa propagada por Martinho Lutero (1483--1546), teólogo e reformador alemão.  }{lu.te.ra.nis.mo}{0}
\verb{luterano}{}{}{}{}{adj.}{Relativo a Lutero ou ao luteranismo.}{lu.te.ra.no}{0}
\verb{luterano}{}{}{}{}{s.m.}{Pessoa adepta do luteranismo.}{lu.te.ra.no}{0}
\verb{luto}{}{}{}{}{s.m.}{Sentimento de dor e pesar pelo falecimento de alguém.}{lu.to}{0}
\verb{luto}{}{}{}{}{}{Sinais que demonstram esse sentimento.}{lu.to}{0}
\verb{lutulento}{}{}{}{}{adj.}{Que está cheio de lodo; lodoso, lamacento.}{lu.tu.len.to}{0}
\verb{lutuoso}{ô}{}{}{}{adj.}{Coberto de luto.}{lu.tu.o.so}{0}
\verb{lutuoso}{ô}{Fig.}{}{}{}{Triste, fúnebre, lúgubre.}{lu.tu.o.so}{0}
\verb{luva}{}{}{}{}{s.f.}{Peça do vestuário, feita de tecido, lã ou couro, com que se cobre a mão.}{lu.va}{0}
\verb{luva}{}{}{}{}{}{Tubo com roscas internas que serve de conexão entre dois canos.}{lu.va}{0}
\verb{luxação}{ch}{}{"-ões}{}{s.f.}{Deslocamento de dois ou mais ossos do corpo.}{lu.xa.ção}{0}
\verb{luxar}{ch}{Med.}{}{}{v.t.}{Fazer um osso do corpo sair do lugar em que se junta com outro; desarticular; deslocar.}{lu.xar}{0}
\verb{luxar}{ch}{}{}{}{v.i.}{Ostentar luxo.}{lu.xar}{\verboinum{1}}
\verb{luxar}{ch}{}{}{}{v.i.}{Ficar deslocado.}{lu.xar}{\verboinum{1}}
\verb{luxemburguês}{ch}{}{}{}{adj.}{Relativo ao grão"-ducado de Luxemburgo (Europa).}{lu.xem.bur.guês}{0}
\verb{luxemburguês}{ch}{}{}{}{s.m.}{Indivíduo natural ou habitante de Luxemburgo.}{lu.xem.bur.guês}{0}
\verb{luxento}{ch}{Bras.}{}{}{adj.}{Que é manhoso e exigente; melindroso, manhoso.}{lu.xen.to}{0}
\verb{luxo}{ch}{}{}{}{s.m.}{Forma de viver caracterizada pela ostentação de riqueza.}{lu.xo}{0}
\verb{luxo}{ch}{}{}{}{}{Recusa fingida; melindre.}{lu.xo}{0}
\verb{luxuosidade}{ch}{}{}{}{s.f.}{Qualidade de luxuoso.}{lu.xu.o.si.da.de}{0}
\verb{luxuoso}{ch\ldots{}ô}{}{"-osos ⟨ó⟩}{"-osa ⟨ó⟩}{adj.}{Em que há luxo; que ostenta luxo; faustuoso, requintado, pomposo.}{lu.xu.o.so}{0}
\verb{luxúria}{ch}{}{}{}{s.f.}{Exuberância dos vegetais; viço.}{lu.xú.ria}{0}
\verb{luxúria}{ch}{}{}{}{}{Lascívia, sensualidade, libertinagem.}{lu.xú.ria}{0}
\verb{luxuriante}{ch}{}{}{}{adj.2g.}{Luxurioso.}{lu.xu.ri.an.te}{0}
\verb{luxuriar}{ch}{}{}{}{v.i.}{Desenvolver"-se com força e viço; vicejar.}{lu.xu.ri.ar}{0}
\verb{luxuriar}{ch}{Fig.}{}{}{v.t.}{Estimular, incentivar a luxúria.}{lu.xu.ri.ar}{\verboinum{1}}
\verb{luxurioso}{ch\ldots{}ô}{}{"-osos ⟨ó⟩}{"-osa ⟨ó⟩}{adj.}{Que é dado a ou demonstra luxúria; lascivo, libidinoso.}{lu.xu.ri.o.so}{0}
\verb{luxurioso}{ch\ldots{}ô}{}{"-osos ⟨ó⟩}{"-osa ⟨ó⟩}{}{Que é exuberante, viçoso; luxuriante.}{lu.xu.ri.o.so}{0}
\verb{luz}{}{}{}{}{s.f.}{Claridade emitida ou refletida por certos corpos; luminosidade.}{luz}{0}
\verb{luz}{}{}{}{}{}{Qualquer objeto que produz claridade.}{luz}{0}
\verb{luz}{}{Fig.}{}{}{}{Tudo o que esclarece o espírito; saber, inteligência.}{luz}{0}
\verb{luzeiro}{ê}{}{}{}{s.m.}{Qualquer objeto que emite luz; candeeiro, lâmpada.}{lu.zei.ro}{0}
\verb{luzeiro}{ê}{Fig.}{}{}{}{Homem ilustre; luminar.}{lu.zei.ro}{0}
\verb{luzente}{}{}{}{}{adj.2g.}{Que luz ou brilha; fulgurante, lúcido, luminoso, refulgente, luzidio.}{lu.zen.te}{0}
\verb{luzerna}{é}{}{}{}{s.f.}{Luz muito intensa; grande clarão.}{lu.zer.na}{0}
\verb{luzerna}{é}{Bot.}{}{}{s.f.}{Planta leguminosa usada como forragem; alfafa.}{lu.zer.na}{0}
\verb{luzidio}{}{}{}{}{adj.}{Que brilha, que reluz; brilhante, polido.}{lu.zi.di.o}{0}
\verb{luzido}{}{}{}{}{}{Que chama a atenção pelo esplendor; pomposo, vistoso, aparatoso.}{lu.zi.do}{0}
\verb{luzido}{}{}{}{}{adj.}{Que é brilhante; luzidio.}{lu.zi.do}{0}
\verb{luziluzir}{}{Bras.}{}{}{v.i.}{Brilhar tremulamente; tremeluzir.}{lu.zi.lu.zir}{\verboinum{21}}
\verb{luzimento}{}{}{}{}{s.m.}{Ato ou efeito de luzir.}{lu.zi.men.to}{0}
\verb{luzimento}{}{Fig.}{}{}{}{Esplendor, brilho.}{lu.zi.men.to}{0}
\verb{luzir}{}{}{}{}{v.i.}{Emitir luz; brilhar, resplandecer.}{lu.zir}{\verboinum{21}}
\verb{lycra}{}{}{}{}{s.f.}{Tecido sintético de grande elasticidade, muito usado em maiôs, cintas, biquínis, sungas etc.}{\textit{lycra}}{0}
