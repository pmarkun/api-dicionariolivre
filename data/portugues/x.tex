\verb{x}{}{}{}{}{s.m.}{Vigésima quarta letra do alfabeto português.}{x}{0}
\verb{X}{}{}{}{}{}{Algarismo romano equivalente a 10.}{X}{0}
\verb{x}{}{}{}{}{}{Aquilo que falta conhecer; resposta ou solução ignorada; incógnita.}{x}{0}
\verb{x}{}{}{}{}{}{Quantidade ou quantia indeterminada, ou não informada diretamente.}{x}{0}
\verb{xá}{ch}{}{}{}{s.m.}{Título do soberano do Irã (antiga Pérsia).}{xá}{0}
\verb{xácara}{ch}{Liter.}{}{}{s.f.}{Narrativa em verso, de origem árabe, popular na península Ibérica.}{xá.ca.ra}{0}
\verb{xadrez}{ch\ldots{}ê}{}{}{}{adj.2g.}{Diz"-se do que é quadriculado como o tabuleiro de xadrez; axadrezado.}{xa.drez}{0}
\verb{xadrez}{ch\ldots{}ê}{}{}{}{s.m.}{Jogo disputado entre duas pessoas, que movem sobre um tabuleiro com casas quadradas pretas e brancas trinta e duas peças, de várias formas e valores, com o objetivo de atacar o rei do adversário e defender o seu.}{xa.drez}{0}
\verb{xadrez}{ch\ldots{}ê}{Pop.}{}{}{}{Cadeia, cela, prisão.}{xa.drez}{0}
\verb{xadrezista}{ch}{}{}{}{s.2g.}{Jogador ou jogadora de xadrez; enxadrista.}{xa.dre.zis.ta}{0}
\verb{xaile}{ch}{}{}{}{}{Var. de \textit{xale}.}{xai.le}{0}
\verb{xairel}{ch\ldots{}é}{}{"-éis}{}{s.m.}{Cobertura de cavalgadura, feita de tecido ou de couro, sobre a qual se põe a sela; xaréu.  }{xai.rel}{0}
\verb{xale}{ch}{}{}{}{s.m.}{Peça do vestuário feminino, espécie de manta que cobre e agasalha o pescoço, os ombros e, às vezes, a cabeça.}{xa.le}{0}
\verb{xamã}{ch}{}{}{}{s.m.}{Nos povos da Ásia central e setentrional, indivíduo que atua por meio de estados extáticos e que supostamente é capaz de curar, predizer o futuro e entrar em contato com os espíritos. }{xa.mã}{0}
\verb{xamã}{ch}{Por ext.}{}{}{}{Em diversos povos, indivíduo escolhido para exercer funções religiosas ou que tenha capacidade para isso.}{xa.mã}{0}
\verb{xamanismo}{ch}{}{}{}{s.m.}{Conjunto de crenças e práticas associadas aos xamãs.}{xa.ma.nis.mo}{0}
\verb{xamanismo}{ch}{}{}{}{}{Sistema religioso característico dos povos da Ásia setentrional, em que o xamã, supostamente capaz de interceder junto aos espíritos e às forças da natureza, é o líder.}{xa.ma.nis.mo}{0}
\verb{xampu}{ch}{}{}{}{s.m.}{Solução saponácea usada para lavar os cabelos e o couro cabeludo.}{xam.pu}{0}
\verb{xangô}{ch}{Relig.}{}{}{s.m.}{Poderoso orixá ioruba, filho de Iemanjá e Oxalá, cuja manifestação são os raios e os trovões.}{xan.gô}{0}
\verb{xanteína}{ch}{Quím.}{}{}{s.f.}{Corante que se extrai da dália amarela.}{xan.te.í.na}{0}
\verb{xantofila}{ch}{Bioquím.}{}{}{s.f.}{Pigmento amarelo encontrado em muitos organismos.}{xan.to.fi.la}{0}
\verb{xantungue}{ch}{}{}{}{s.m.}{Tecido de seda áspero, com fios irregulares e torcidos.}{xan.tun.gue}{0}
\verb{xará}{ch}{}{}{}{s.2g.}{Pessoa que tem o mesmo nome de batismo que outra. }{xa.rá}{0}
\verb{xarda}{ch}{}{}{}{s.f.}{Tipo de dança húngara, provavelmente de origem cigana; czarda.}{xar.da}{0}
\verb{xarelete}{ch\ldots{}ê}{Zool.}{}{}{s.m.}{Xaréu.}{xa.re.le.te}{0}
\verb{xaréu}{ch}{Zool.}{}{}{s.m.}{Nome comum a várias espécies de peixes migradores, que habitam o oceano Atlântico; xarelete.}{xa.réu}{0}
\verb{xaréu}{ch}{}{}{}{}{Xairel.}{xa.réu}{0}
\verb{xaropada}{ch}{}{}{}{s.f.}{Porção de xarope tomada de uma só vez.}{xa.ro.pa.da}{0}
\verb{xaropada}{ch}{Pop.}{}{}{}{Qualquer coisa muito chata, maçante, enfadonha, especialmente conversa ou discurso.}{xa.ro.pa.da}{0}
\verb{xarope}{ch\ldots{}ó}{}{}{}{s.m.}{Solução medicamentosa, concentrada e adocicada; tisana.}{xa.ro.pe}{0}
\verb{xarope}{ch\ldots{}ó}{Pop.}{}{}{}{Calda.}{xa.ro.pe}{0}
\verb{xarope}{ch\ldots{}ó}{Por ext.}{}{}{}{Bebida muito adocicada. }{xa.ro.pe}{0}
\verb{xarope}{ch\ldots{}ó}{}{}{}{}{Remédio caseiro.}{xa.ro.pe}{0}
\verb{xarope}{ch\ldots{}ó}{Pop.}{}{}{adj.2g.}{Diz"-se de coisa ou indivíduo maçante, enfadonho, chato.}{xa.ro.pe}{0}
\verb{xaroposo}{ch\ldots{}ô}{}{"-osos ⟨ó⟩}{"-osa ⟨ó⟩}{adj.}{Que tem consistência semelhante à do xarope; viscoso.}{xa.ro.po.so}{0}
\verb{xaveco}{ch\ldots{}é}{}{}{}{s.m.}{Espécie de embarcação muçulmana largamente empregada por piratas do mar Mediterrâneo nos séculos \textsc{xviii} e \textsc{xix}.}{xa.ve.co}{0}
\verb{xaveco}{ch\ldots{}é}{Bras.}{}{}{}{Qualquer embarcação pequena, mal construída ou mal conservada.}{xa.ve.co}{0}
\verb{xaveco}{ch\ldots{}é}{}{}{}{}{Coisa ou pessoa insignificante, sem importância.}{xa.ve.co}{0}
\verb{xaveco}{ch\ldots{}é}{Pop.}{}{}{}{Patifaria.}{xa.ve.co}{0}
\verb{xaveco}{ch\ldots{}é}{Pop.}{}{}{}{Cantada.}{xa.ve.co}{0}
\verb{xaxado}{ch\ldots{}ch}{Bras.}{}{}{s.m.}{Dança pernambucana, originariamente masculina, divulgada por cangaceiros em muitas partes do Nordeste.}{xa.xa.do}{0}
\verb{xaxim}{ch\ldots{}ch}{Bot.}{"-ins}{}{s.m.}{Nome de certa samambaia arborescente, nativa da Mata Atlântica.}{xa.xim}{0}
\verb{xaxim}{ch\ldots{}ch}{}{"-ins}{}{}{Massa do caule dessa samambaia, leve e fibrosa, de largo e variado emprego em jardinagem.}{xa.xim}{0}
\verb{Xe}{}{Quím.}{}{}{}{Símb. do \textit{xenônio}. }{Xe}{0}
\verb{xeique}{ch}{}{}{}{s.m.}{Soberano ou chefe de povo árabe; xeque.}{xei.que}{0}
\verb{xelim}{ch}{}{"-ins}{}{s.m.}{Antiga moeda inglesa, que valia a vigésima parte da libra.}{xe.lim}{0}
\verb{xelim}{ch}{}{"-ins}{}{}{Unidade monetária e moeda de Uganda, Tanzânia, Somália, Quênia e Áustria.}{xe.lim}{0}
\verb{xenartro}{ch}{Zool.}{}{}{s.m.}{Espécime dos xenartros, ordem de mamíferos desdentados ou com dentes reduzidos, com dedos providos de garras e olfato desenvolvido, representada pelas preguiças, tamanduás e tatus.}{xe.nar.tro}{0}
\verb{xenofilia}{ch}{}{}{}{s.f.}{Simpatia por estrangeiro ou pela cultura estrangeira.}{xe.no.fi.li.a}{0}
\verb{xenófilo}{ch}{}{}{}{adj.}{Que tem xenofilia.}{xe.nó.fi.lo}{0}
\verb{xenófilo}{ch}{}{}{}{s.m.}{Indivíduo que tem xenofilia.}{xe.nó.fi.lo}{0}
\verb{xenofobia}{ch}{}{}{}{s.f.}{Aversão ao estrangeiro ou à cultura estrangeira.}{xe.no.fo.bi.a}{0}
\verb{xenófobo}{ch}{}{}{}{adj.}{Que tem xenofobia.}{xe.nó.fo.bo}{0}
\verb{xenófobo}{ch}{}{}{}{s.m.}{Indivíduo que tem xenofobia.}{xe.nó.fo.bo}{0}
\verb{xenônio}{ch}{Quím.}{}{}{s.m.}{Elemento químico da família dos gases nobres, incolor e inodoro, utilizado em válvulas eletrônicas, em lâmpadas especiais de grande intensidade, em \textit{flashes} para fotografia, entre outros. \elemento{54}{131.29}{Xe}.}{xe.nô.nio}{0}
\verb{xepa}{chê}{Pop.}{}{}{s.f.}{Comida servida em quartel.}{xe.pa}{0}
\verb{xepa}{chê}{Pop.}{}{}{}{Comida de má qualidade; sobra; grude.}{xe.pa}{0}
\verb{xepa}{chê}{Pop.}{}{}{}{Papel usado que se recolhe para reciclagem.}{xe.pa}{0}
\verb{xepa}{chê}{Pop.}{}{}{}{Numa feira livre, as últimas mercadorias vendidas, de qualidade inferior e menor preço.}{xe.pa}{0}
\verb{xepa}{chê}{Pop.}{}{}{}{Sobras de alimentos perecíveis que as pessoas recolhem para consumo próprio ao término de uma feira livre ou em um mercado.}{xe.pa}{0}
\verb{xepeiro}{ch\ldots{}ê}{Bras.}{}{}{s.m.}{Soldado que come no quartel.}{xe.pei.ro}{0}
\verb{xepeiro}{ch\ldots{}ê}{Pop.}{}{}{}{Indivíduo que se sustenta com esmolas e dorme em qualquer parte.}{xe.pei.ro}{0}
\verb{xepeiro}{ch\ldots{}ê}{Pop.}{}{}{}{Indivíduo que pede objetos emprestados com frequência ou se aproveita do que não é seu. }{xe.pei.ro}{0}
\verb{xepeiro}{ch\ldots{}ê}{Pop.}{}{}{}{Indivíduo que recolhe papel para revender.}{xe.pei.ro}{0}
\verb{xepeiro}{ch\ldots{}ê}{Pop.}{}{}{}{Indivíduo que compra ou recolhe a xepa de feira livre ou mercado.}{xe.pei.ro}{0}
\verb{xeque}{ché}{}{}{}{s.m.}{Soberano ou chefe de povo árabe; xeique.}{xe.que}{0}
\verb{xeque}{ché}{}{}{}{}{Entre os árabes, homem respeitável por sua idade e seus conhecimentos.}{xe.que}{0}
\verb{xeque}{ché}{}{}{}{s.m.}{No jogo de xadrez, lance em que o rei é ameaçado por peça adversária.}{xe.que}{0}
\verb{xeque"-mate}{ché}{}{xeques"-mate(s)}{}{s.m.}{No jogo de xadrez, lance em que o rei é atacado, sem possibilidade de fuga ou defesa, e que determina o fim da partida e a derrota do jogador que recebeu esse lance.}{xe.que"-ma.te}{0}
\verb{xerém}{ch}{}{"-ens}{}{s.m.}{Milho pilado grosso; canjiquinha.}{xe.rém}{0}
\verb{xerém}{ch}{Bras.}{"-ens}{}{}{Dança nordestina de roda, com acompanhamento de sanfona.}{xe.rém}{0}
\verb{xereta}{ch\ldots{}ê}{}{}{}{s.2g.}{Indivíduo que se intromete em assuntos particulares ou na vida alheia; bisbilhoteiro, leva"-e"-traz.}{xe.re.ta}{0}
\verb{xeretar}{ch}{Pop.}{}{}{v.t.}{Bisbilhotar, intrometer"-se.}{xe.re.tar}{\verboinum{1}}
\verb{xeretear}{ch}{}{}{}{v.t.}{Xeretar.}{xe.re.te.ar}{\verboinum{4}}
\verb{xerez}{ch\ldots{}ê}{}{}{}{s.m.}{Casta de uva tinta.}{xe.rez}{0}
\verb{xerez}{ch\ldots{}ê}{}{}{}{}{Vinho licoroso da província espanhola Andaluzia.}{xe.rez}{0}
\verb{xerga}{chê}{Lus.}{}{}{s.f.}{Tecido grosseiro.}{xer.ga}{0}
\verb{xerga}{chê}{Bras.}{}{}{}{Almofada que se estende por baixo da sela dos animais de carga.}{xer.ga}{0}
\verb{xerife}{ch}{}{}{}{s.m.}{Principal indivíduo encarregado pelo cumprimento e pela execução da lei em cidades norte"-americanas.}{xe.ri.fe}{0}
\verb{xerocar}{ch}{}{}{}{v.t.}{Fazer cópia em máquina xerox; xerocopiar, xerografar, xeroxar.}{xe.ro.car}{\verboinum{2}}
\verb{xerocópia}{ch}{}{}{}{s.f.}{Cópia obtida pelo processo de xerografia.}{xe.ro.có.pia}{0}
\verb{xerocopiar}{ch}{}{}{}{v.t.}{Fazer cópia em máquina xerox; xerocar, xerografar, xeroxar.}{xe.ro.co.pi.ar}{\verboinum{1}}
\verb{xerófilo}{ch}{Ecol.}{}{}{adj.}{Diz"-se de organismo que vive em ambiente árido.}{xe.ró.fi.lo}{0}
\verb{xerófilo}{ch}{Bot.}{}{}{}{Diz"-se da estrutura das folhas de algumas plantas. }{xe.ró.fi.lo}{0}
\verb{xerófito}{ch}{Bot.}{}{}{adj.}{Diz"-se dos vegetais que vivem em ambientes áridos e têm adaptações para minimizar a perda de água.}{xe.ró.fi.to}{0}
\verb{xerófito}{ch}{}{}{}{s.m.}{O vegetal xerófito.}{xe.ró.fi.to}{0}
\verb{xerografar}{ch}{}{}{}{v.t.}{Fazer cópia em máquina xerox; xerocar, xerocopiar, xeroxar.}{xe.ro.gra.far}{\verboinum{1}}
\verb{xerografia}{ch}{}{}{}{s.f.}{Processo de reprodução de texto ou imagem em máquina xerox, ou cópia obtida por esse processo.}{xe.ro.gra.fi.a}{0}
\verb{xerografia}{ch}{Geogr.}{}{}{}{Ramo da geografia que estuda os ambientes áridos da Terra.}{xe.ro.gra.fi.a}{0}
\verb{xérox}{ch\ldots{}cs}{}{}{}{}{Var. de \textit{xerox}.}{xé.rox}{0}
\verb{xerox}{ch\ldots{}ócs}{}{}{}{adj.2g.}{Diz"-se de máquina usada na reprodução de texto ou imagem, que realiza um processo de reprografia a seco, por meio de fotocondutividade.}{xe.rox}{0}
\verb{xerox}{ch\ldots{}ócs}{}{}{}{s.2g.}{Máquina empregada no processo de cópia de texto ou imagem.}{xe.rox}{0}
\verb{xerox}{ch\ldots{}ócs}{}{}{}{}{Cópia obtida com essa máquina.}{xe.rox}{0}
\verb{xeroxar}{ch\ldots{}cs}{}{}{}{v.t.}{Fazer cópia em máquina xerox; xerocar, xerocopiar, xerografar.}{xe.ro.xar}{\verboinum{1}}
\verb{xexéu}{ch\ldots{}ch}{Zool.}{}{}{s.m.}{Nome comum a várias aves passeriformes, de porte médio, coloração negra e amarela ou vermelha, de larga distribuição no Brasil; japi, japim.}{xe.xéu}{0}
\verb{xexéu}{ch\ldots{}ch}{Pop.}{}{}{}{Mau cheiro de homem ou animal; bodum, catinga.}{xe.xéu}{0}
\verb{xi}{chi}{Bras.}{}{}{interj.}{Exprime admiração, espanto, inquietação, surpresa ou desagrado.}{xi}{0}
\verb{xi}{csi}{}{}{}{s.m.}{Nome da décima quarta letra do alfabeto grego.}{xi}{0}
\verb{xícara}{ch}{}{}{}{s.f.}{Recipiente pequeno, geralmente de louça ou de metal, com asa, no qual se servem bebidas, geralmente quentes, como café, chá e chocolate.  }{xí.ca.ra}{0}
\verb{xícara}{ch}{}{}{}{}{A quantidade de qualquer substância que uma xícara comporta.}{xí.ca.ra}{0}
\verb{xifoide}{ch}{}{}{}{adj.2g.}{Que tem forma semelhante à de uma espada; xifóideo, ensiforme.}{xi.foi.de}{0}
\verb{xifoideo}{ch}{}{}{}{}{Var. de \textit{xifoide}.}{xi.foi.de.o}{0}
\verb{xifopagia}{ch}{Med.}{}{}{s.f.}{Má formação genética que consiste em duplicação de parte superior do corpo, tórax ou região xifoide.}{xi.fo.pa.gi.a}{0}
\verb{xifópago}{ch}{}{}{}{adj.2g.}{Diz"-se do ser humano geneticamente alterado, que apresenta duplicação do corpo na região do tórax e da cabeça.}{xi.fó.pa.go}{0}
\verb{xifópagos}{ch}{}{}{}{adj.2g.}{Diz"-se de gêmeos que apresentam xifopagia.}{xi.fó.pa.gos}{0}
\verb{xiita}{ch}{Relig.}{}{}{adj.}{Relativo ou pertencente aos xiitas. }{xi.i.ta}{0}
\verb{xiita}{ch}{Por ext.}{}{}{}{Diz"-se do indivíduo que tem ideologia e atitudes extremistas, radicais, dogmáticas.}{xi.i.ta}{0}
\verb{xiita}{ch}{Relig.}{}{}{s.m.}{Membro dos xiitas, grupo muçulmano partidário de Ali, primo e genro de Maomé, em oposição aos sunitas.}{xi.i.ta}{0}
\verb{xiita}{ch}{Por ext.}{}{}{}{Indivíduo que pertence a uma minoria extremista radical.   }{xi.i.ta}{0}
\verb{xilema}{ch}{Bot.}{}{}{s.m.}{Tipo de tecido vascular vegetal responsável pelo transporte de seiva bruta das raízes para o restante da planta, envolvido também na sustentação.}{xi.le.ma}{0}
\verb{xileno}{ch}{Quím.}{}{}{s.m.}{Líquido incolor usado como solvente,  obtido na destilação do carvão ou de certos petróleos.}{xi.le.no}{0}
\verb{xilindró}{ch}{Pop.}{}{}{s.m.}{Cadeia, prisão.}{xi.lin.dró}{0}
\verb{xilófago}{ch}{Biol.}{}{}{adj.2g.}{Diz"-se de inseto que se nutre da madeira.  }{xi.ló.fa.go}{0}
\verb{xilofone}{ch}{Mús.}{}{}{s.m.}{Instrumento musical constituído de lâminas de madeira, que são percutidas por baquetas.}{xi.lo.fo.ne}{0}
\verb{xilografia}{ch}{Art.}{}{}{s.f.}{Técnica de estampa que utiliza gravura em relevo sobre madeira como matriz.}{xi.lo.gra.fi.a}{0}
\verb{xilografia}{ch}{Por ext.}{}{}{}{Gravura em madeira.}{xi.lo.gra.fi.a}{0}
\verb{xilógrafo}{ch}{}{}{}{adj.}{Diz"-se de quem grava sobre madeira.}{xi.ló.gra.fo}{0}
\verb{xilógrafo}{ch}{}{}{}{s.m.}{Trabalho de xilografia; xilogravura.}{xi.ló.gra.fo}{0}
\verb{xilogravar}{ch}{}{}{}{v.t.}{Gravar em madeira.}{xi.lo.gra.var}{\verboinum{1}}
\verb{xilogravura}{ch}{Art.}{}{}{s.f.}{Técnica ou processo de gravura em relevo sobre madeira.}{xi.lo.gra.vu.ra}{0}
\verb{xilogravura}{ch}{}{}{}{}{Estampa ou ilustração obtida com essa técnica.}{xi.lo.gra.vu.ra}{0}
\verb{xiloma}{ch}{Biol.}{}{}{s.m.}{Estrutura esporífera de alguns fungos.}{xi.lo.ma}{0}
\verb{ximango}{ch}{Zool.}{}{}{s.m.}{Ave de rapina com até 40 cm de comprimento, de coloração creme com manchas brancas nas asas, e que ocorre da Terra do Fogo ao Paraguai e também no Sul, Sudeste e Centro"-Oeste do Brasil.}{xi.man.go}{0}
\verb{ximbé}{ch}{Bras.}{}{}{adj.2g.}{Diz"-se de animais com focinho curto e chato.}{xim.bé}{0}
\verb{ximbica}{ch}{Bras.}{}{}{s.f.}{Certo jogo de cartas.}{xim.bi.ca}{0}
\verb{ximbica}{ch}{Pop.}{}{}{}{Carro muito velho; calhambeque.}{xim.bi.ca}{0}
\verb{ximburé}{ch}{Zool.}{}{}{s.m.}{Nome comum a diversos peixes de água doce.}{xim.bu.ré}{0}
\verb{xingação}{ch}{Bras.}{"-ões}{}{s.f.}{Ato ou efeito de xingar; xingamento.}{xin.ga.ção}{0}
\verb{xingadela}{ch\ldots{}é}{Bras.}{}{}{s.f.}{Insulto, xingamento dissimulado.}{xin.ga.de.la}{0}
\verb{xingamento}{ch}{Bras.}{}{}{s.m.}{Ato ou efeito de xingar; xingação.}{xin.ga.men.to}{0}
\verb{xingar}{ch}{}{}{}{v.t.}{Dizer insultos ou palavras afrontosas a; injuriar, insultar.}{xin.gar}{\verboinum{5}}
\verb{xingatório}{ch}{}{}{}{adj.}{Que envolve xingação; insultuoso, injurioso.}{xin.ga.tó.rio}{0}
\verb{xingatório}{ch}{}{}{}{s.m.}{Grande número de xingações; xingamento.}{xin.ga.tó.rio}{0}
\verb{xintoísmo}{ch}{Relig.}{}{}{s.m.}{Antiga religião politeísta do Japão, ainda professada nesse país, sem fundador nem caráter missionário, caracterizada pela veneração de divindades representadas por fenômenos da natureza.}{xin.to.ís.mo}{0}
\verb{xintoísta}{ch}{}{}{}{s.2g.}{Indivíduo adepto do xintoísmo.}{xin.to.ís.ta}{0}
\verb{xintoísta}{ch}{}{}{}{adj.2g.}{Relativo ou pertencente ao xintoísmo.}{xin.to.ís.ta}{0}
\verb{xinxim}{ch\ldots{}ch}{Cul.}{}{}{s.m.}{Prato típico baiano, de origem africana, que consiste num guisado de galinha ou outra carne, com alho e cebola ralados, camarão seco, amendoim e castanha de caju.}{xin.xim}{0}
\verb{xipaia}{ch}{}{}{}{adj.}{Relativo aos Xipaia.}{xi.pai.a}{0}
\verb{xipaia}{ch}{}{}{}{s.2g.}{Indivíduo pertencente ao povo xipaia, família linguística juruna.}{xi.pai.a}{0}
\verb{xique"-xique}{ch\ldots{}ch}{Bras.}{xique"-xiques ⟨ch\ldots{}ch⟩}{}{s.m.}{Chocalho, ganzá.}{xi.que"-xi.que}{0}
\verb{xiquexique}{ch\ldots{}ch}{Bot.}{}{}{s.m.}{Planta nativa do Brasil, característica das caatingas.}{xi.que.xi.que}{0}
\verb{xique"-xique}{ch\ldots{}ch}{Bot.}{}{}{}{Nome comum a diversas plantas leguminosas.}{xi.que"-xi.que}{0}
\verb{xis}{ch}{}{}{}{s.m.}{Nome da letra \textit{x}.}{xis}{0}
\verb{xisto}{ch}{Geol.}{}{}{s.m.}{Nome comum às rochas que se podem dividir em lâminas e cujos minerais são visíveis a olho nu.}{xis.to}{0}
\verb{xistosidade}{ch}{Geol.}{}{}{s.f.}{Qualidade do que se apresenta em camadas, como o xisto.}{xis.to.si.da.de}{0}
\verb{xistoso}{ch\ldots{}ô}{}{"-osos ⟨ó⟩}{"-osa ⟨ó⟩}{adj.}{Diz"-se do que tem a natureza do xisto.}{xis.to.so}{0}
\verb{xixi}{ch\ldots{}ch}{}{}{}{s.m.}{Designação infantil ou familiar da urina.}{xi.xi}{0}
\verb{xô}{ch}{}{}{}{interj.}{Expressão usada para enxotar galinhas ou outras aves.}{xô}{0}
\verb{xodó}{ch}{Bras.}{}{}{s.m.}{Qualquer envolvimento amoroso; namoro, namorico.}{xo.dó}{0}
\verb{xodó}{ch}{Pop.}{}{}{}{Namorado, amante.}{xo.dó}{0}
\verb{xodó}{ch}{Pop.}{}{}{}{Sentimento ou objeto de afeição, estima, apreço.}{xo.dó}{0}
\verb{xodó}{ch}{Pop.}{}{}{}{Mexerico, intriga.}{xo.dó}{0}
\verb{xokleng}{ch}{}{}{}{adj.}{Relativo aos Xokleng.}{xo.kleng}{0}
\verb{xokleng}{ch}{}{}{}{s.2g.}{Indivíduo pertencente ao povo xokleng, família linguística jê.}{xo.kleng}{0}
\verb{xokó}{ch}{}{}{}{adj.}{Relativo aos Xokó.}{xo.kó}{0}
\verb{xokó}{ch}{}{}{}{s.2g.}{Indivíduo pertencente ao povo xokó.}{xo.kó}{0}
\verb{xote}{chó}{}{}{}{s.m.}{Dança de salão de origem europeia, com passos semelhantes aos da polca, difundida no Nordeste do Brasil, onde é dançada nos bailes populares com acompanhamento de sanfona.}{xo.te}{0}
\verb{xote}{chó}{Mús.}{}{}{}{Música em compasso binário e andamento lento que acompanha essa dança. }{xo.te}{0}
\verb{xucrice}{ch}{Bras.}{}{}{s.f.}{Qualidade de xucro; ignorância, rudeza, grosseria.}{xu.cri.ce}{0}
\verb{xucro}{ch}{Bras.}{}{}{adj.}{Diz"-se de animal de sela não domesticado ou domado.}{xu.cro}{0}
\verb{xucro}{ch}{Por ext.}{}{}{}{Diz"-se daquele que é desprovido de educação; ignorante, abrutalhado, grosseiro, bronco.}{xu.cro}{0}
\verb{xucro}{ch}{Por ext.}{}{}{}{Diz"-se de café amargo, sem açúcar.}{xu.cro}{0}
\verb{xukuru}{ch}{}{}{}{adj.}{Relativo aos Xukuru.}{xu.ku.ru}{0}
\verb{xukuru}{ch}{}{}{}{s.2g.}{Indivíduo pertencente ao povo xukuru.}{xu.ku.ru}{0}
\verb{xukuru kariri}{ch}{}{}{}{adj.}{Relativo aos Xukuru Kariri.}{xu.ku.ru ka.ri.ri}{0}
\verb{xukuru kariri}{ch}{}{}{}{s.2g.}{Indivíduo pertencente ao povo xukuru kariri.}{xu.ku.ru ka.ri.ri}{0}
