\verb{b}{}{}{}{}{s.m.}{Segunda letra do alfabeto português.}{b}{0}
\verb{b}{}{Informát.}{}{}{}{Símb. de \textit{bit}.}{b}{0}
\verb{B}{}{Mat.}{}{}{}{No sistema hexadecimal representa o décimo segundo algarismo, equivalente ao número decimal 11.}{B}{0}
\verb{B}{}{Mús.}{}{}{}{A nota ou acorde referente ao \textit{si}, ou à sétima nota da escala de \textit{dó}.}{B}{0}
\verb{B}{}{Quím.}{}{}{}{Símb. do \textit{boro}.}{B}{0}
\verb{B}{}{Informát.}{}{}{}{Símb. de \textit{byte}.}{B}{0}
\verb{bá}{}{Pop.}{}{}{s.f.}{Babá; ama"-seca.}{bá}{0}
\verb{Ba}{}{Quím.}{}{}{}{Símb. do \textit{bário}.}{Ba}{0}
\verb{BA}{}{}{}{}{}{Sigla do estado da Bahia.}{BA}{0}
\verb{baamense}{}{}{}{}{adj.2g. e s.2g.}{Baamiano.}{ba.a.men.se}{0}
\verb{baamês}{}{}{}{}{adj. e s.m.  }{Baamiano.}{ba.a.mês}{0}
\verb{baamiano}{}{}{}{}{adj.}{Relativo às Baamas, ou Bahamas (arquipélago das Antilhas).}{ba.a.mi.a.no}{0}
\verb{baamiano}{}{}{}{}{s.m.}{Indivíduo natural ou habitante desse país; baamense, baamês.}{ba.a.mi.a.no}{0}
\verb{baba}{}{}{}{}{s.f.}{Saliva que escorre da boca involuntariamente.}{ba.ba}{0}
\verb{baba}{}{}{}{}{}{Gosma secretada por certos animais.}{ba.ba}{0}
\verb{babá}{}{Bras.}{}{}{s.f.}{Profissional especialmente contratada para cuidar de crianças em residências particulares.}{ba.bá}{0}
\verb{babaca}{}{}{}{}{adj.2g.}{Que é ingênuo, que não tem inteligência; simplório, tolo.}{ba.ba.ca}{0}
\verb{babaca}{}{}{}{}{s.2g.}{Indivíduo tolo, simplório.}{ba.ba.ca}{0}
\verb{babaçu}{}{Bot.}{}{}{s.m.}{Palmeira cujos frutos contêm sementes oleaginosas e comestíveis, das quais se extrai um óleo empregado na alimentação, e de cujas folhas se fabricam esteiras, cestos, chapéus.}{ba.ba.çu}{0}
\verb{babaçu}{}{Por ext.}{}{}{}{O fruto do babaçu.}{ba.ba.çu}{0}
\verb{babaçual}{}{}{"-ais}{}{s.m.}{Coletivo de babaçu.}{ba.ba.çu.al}{0}
\verb{baba"-de"-moça}{ô}{Cul.}{babas"-de"-moça ⟨ô⟩}{}{s.f.}{Doce feito com calda de açúcar, leite de coco e gemas de ovos.  }{ba.ba"-de"-mo.ça}{0}
\verb{babado}{}{}{}{}{adj.}{Molhado de baba, saliva.}{ba.ba.do}{0}
\verb{babado}{}{Pop.}{}{}{}{Apaixonado, enamorado.}{ba.ba.do}{0}
\verb{babado}{}{}{}{}{s.m.}{Guarnição de pregas para enfeitar vestes, toalhas etc.}{ba.ba.do}{0}
\verb{babador}{ô}{}{}{}{s.m.}{Peça de pano com que se cobre o peito de crianças e enfermos para evitar que se sujem de baba ou de comida; babadouro.}{ba.ba.dor}{0}
\verb{babadouro}{ô}{}{}{}{s.m.}{Babador.}{ba.ba.dou.ro}{0}
\verb{babalorixá}{ch}{Relig.}{}{}{s.m.}{Nos candomblés, xangôs e em alguns centros de umbanda, nome dado ao chefe espiritual e administrador da casa; pai"-de"-santo.}{ba.ba.lo.ri.xá}{0}
\verb{babão}{}{}{"-ões}{"-ona}{adj.}{Que baba muito.}{ba.bão}{0}
\verb{babão}{}{}{"-ões}{"-ona}{}{Dengoso, melindroso.}{ba.bão}{0}
\verb{babaquara}{}{}{}{}{s.m.}{Habitante do campo, pouco instruído e simples; matuto, caipira.}{ba.ba.qua.ra}{0}
\verb{babaquara}{}{}{}{}{adj.}{Bobo, pateta, babaca.}{ba.ba.qua.ra}{0}
\verb{babaquice}{}{}{}{}{s.f.}{Ato ou dito sem conteúdo; bobagem, asneira, tolice.}{ba.ba.qui.ce}{0}
\verb{babar}{}{}{}{}{v.t.}{Molhar ou umedecer com baba.}{ba.bar}{0}
\verb{babar}{}{}{}{}{}{Falar com hesitação; gaguejar, balbuciar.}{ba.bar}{0}
\verb{babar}{}{}{}{}{v.pron.}{Estar apaixonado, encantado.}{ba.bar}{\verboinum{1}}
\verb{babau}{}{}{}{}{interj.}{Expressão que indica perda irreversível de algo. (\textit{Depois que o leite derramou, babau, não há como recuperar.})}{ba.bau}{0}
\verb{babel}{é}{}{"-éis}{}{s.f.}{Grande confusão ou algazarra; mistura de línguas ou de vozes.}{ba.bel}{0}
\verb{babel}{é}{}{"-éis}{}{}{Conjunto de elementos muito diversos; complexidade.}{ba.bel}{0}
\verb{babélico}{}{}{}{}{adj.}{Relativo a babel; confuso, caótico.}{ba.bé.li.co}{0}
\verb{babilônia}{}{}{}{}{s.f.}{Cidade grande, mal planejada e desordenada.}{ba.bi.lô.nia}{0}
\verb{babilônico}{}{}{}{}{adj.}{Relativo à cidade ou ao império da Babilônia.}{ba.bi.lô.ni.co}{0}
\verb{babilônico}{}{}{}{}{s.m.}{Indivíduo natural ou habitante dessa cidade.}{ba.bi.lô.ni.co}{0}
\verb{babilônico}{}{Fig.}{}{}{adj.}{Imponente, majestoso, luxuoso.}{ba.bi.lô.ni.co}{0}
\verb{babosa}{ó}{Bot.}{}{}{s.f.}{Planta arbustiva suculenta, da qual se extrai uma resina usada para fins medicinais e de cujas fibras confeccionam"-se cordas e esteiras.}{ba.bo.sa}{0}
\verb{baboseira}{ê}{}{}{}{s.f.}{Dito insignificante, irrelevante; asneira, tolice, bobagem.}{ba.bo.sei.ra}{0}
\verb{babuge}{}{}{}{}{}{Var. de \textit{babugem}.}{ba.bu.ge}{0}
\verb{babugem}{}{}{"-ens}{}{s.f.}{Espuma formada pela água agitada. (\textit{Aos poucos as folhas secas iam se misturando à babugem do mar e desapareciam de nossos olhos.})}{ba.bu.gem}{0}
\verb{babugem}{}{}{"-ens}{}{}{Restos, especialmente de comida. (\textit{Depois da feira, ficava somente aquela babugem que alguns ainda se esforçavam para recolher.})}{ba.bu.gem}{0}
\verb{babuíno}{}{Zool.}{}{}{s.m.}{Espécie de macaco africano, de focinho longo, cauda curta, temperamento selvagem, que vive em bandos numerosos.}{ba.bu.í.no}{0}
\verb{babujar}{}{}{}{}{v.t.}{Desejar alguma comida, salivar. (\textit{O pobre animal babujava o capim solitariamente.})}{ba.bu.jar}{0}
\verb{babujar}{}{Fig.}{}{}{}{Cobiçar, desejar com avidez. (\textit{O demônio babujava a alma daqueles que o procuravam.})}{ba.bu.jar}{0}
\verb{babujar}{}{Fig.}{}{}{}{Tratar com desrespeito; ofender. (\textit{Se alguém o babujasse novamente, ele sairia.})}{ba.bu.jar}{0}
\verb{babujar}{}{}{}{}{}{Falar com a boca cheia, balbuciar. (\textit{Ele babujou algumas palavras no almoço.})}{ba.bu.jar}{\verboinum{1}}
\verb{baby"-sitter}{}{}{}{}{s.2g.}{Pessoa contratada temporariamente para cuidar de crianças na ausência dos pais.}{\textit{baby"-sitter}}{0}
\verb{bacaba}{}{Bot.}{}{}{s.f.}{Palmeira de até 20 metros, cujos fruto e semente oleaginosa são comestíveis, e de cujo lenho e de cujas folhas confeccionam"-se cestos, esteiras etc.}{ba.ca.ba}{0}
\verb{bacaba}{}{}{}{}{}{O fruto dessa palmeira.}{ba.ca.ba}{0}
\verb{bacalhau}{}{Zool.}{}{}{s.m.}{Peixe nativo dos mares frios, cuja carne, seca e salgada ou fresca, é muito apreciada, e cujo óleo é rico em vitaminas \textsc{a} e \textsc{d}.}{ba.ca.lhau}{0}
\verb{bacalhau}{}{Fig.}{}{}{}{Pessoa muito magra.}{ba.ca.lhau}{0}
\verb{bacalhoada}{}{Cul.}{}{}{s.f.}{Prato preparado com bacalhau cozido, batatas, cebolas, regado com azeite.}{ba.ca.lho.a.da}{0}
\verb{bacalhoeiro}{ê}{}{}{}{s.m.}{Negociante de bacalhau a varejo.}{ba.ca.lho.ei.ro}{0}
\verb{bacalhoeiro}{ê}{}{}{}{}{Barco utilizado na pesca e no transporte de bacalhau.}{ba.ca.lho.ei.ro}{0}
\verb{bacamarte}{}{}{}{}{s.m.}{Antiga arma de fogo de cano largo, reforçada na coronha.}{ba.ca.mar.te}{0}
\verb{bacana}{}{}{}{}{adj.}{Palavra que exprime vários atributos positivos; bom, bonito, elegante, simpático, bem"-educado etc. (\textit{Era um rapaz bacana que impressionou muito bem a todos.})}{ba.ca.na}{0}
\verb{bacanal}{}{}{"-ais}{}{s.f.}{Festa marcada pela devassidão e pela volúpia; orgia.}{ba.ca.nal}{0}
\verb{bacanal}{}{Mit.}{"-ais}{}{}{Na Roma Antiga, festa em honra de Baco, deus do vinho.}{ba.ca.nal}{0}
\verb{bacante}{}{}{}{}{s.f.}{Mulher licenciosa, devassa, libertina. (\textit{Era conveniente prevenir os amigos daquela dama, transfigurada em bacante.})}{ba.can.te}{0}
\verb{bacante}{}{Mit.}{}{}{}{Na antiga Roma, sacerdotisa do culto de Baco, o deus do vinho.}{ba.can.te}{0}
\verb{bacará}{}{}{}{}{s.m.}{Jogo carteado entre um banqueiro e vários jogadores, que consiste em totalizar um número de pontos próximo de nove.}{ba.ca.rá}{0}
\verb{bacará}{}{}{}{}{s.m.}{Variedade de cristal produzida em Baccarat, cidade da França. }{ba.ca.rá}{0}
\verb{bacelo}{ê}{}{}{}{s.m.}{Vara tirada de uma videira velha utilizada como muda para o plantio.}{ba.ce.lo}{0}
\verb{bacelo}{ê}{}{}{}{}{Videira nova e pequena.}{ba.ce.lo}{0}
\verb{bacharel}{é}{}{"-éis}{}{s.m.}{Indivíduo que obteve o primeiro grau de formatura dos estudos universitários.}{ba.cha.rel}{0}
\verb{bacharelado}{}{}{}{}{s.m.}{Grau de bacharel; bacharelato.}{ba.cha.re.la.do}{0}
\verb{bacharelado}{}{}{}{}{}{Curso que concede esse grau.}{ba.cha.re.la.do}{0}
\verb{bacharelando}{}{}{}{}{s.m.}{Indivíduo que vai tomar o grau de bacharel; graduando, formando.}{ba.cha.re.lan.do}{0}
\verb{bacharelar"-se}{}{}{}{}{v.pron.}{Obter grau de bacharel; formar"-se, graduar"-se.}{ba.cha.re.lar"-se}{\verboinum{1}}
\verb{bacharelato}{}{}{}{}{s.m.}{Bacharelado.}{ba.cha.re.la.to}{0}
\verb{bacia}{}{}{}{}{s.f.}{Recipiente arredondado, com fundo chato e bordas relativamente altas.}{ba.ci.a}{0}
\verb{bacia}{}{Por ext.}{}{}{}{O vaso sanitário, privada.}{ba.ci.a}{0}
\verb{bacia}{}{Por ext.}{}{}{}{Cavidade de paredes ósseas localizada na parte inferior do tronco,  formada anterior e lateralmente, pelos ossos ilíacos, e, posteriormente, pelo sacro e pelo cóccix.}{ba.ci.a}{0}
\verb{bacia}{}{Anat.}{}{}{}{Toda a região formada pelos ossos que servem de base à coluna e de ponto de apoio aos membros inferiores; pelve.}{ba.ci.a}{0}
\verb{bacia}{}{Geogr.}{}{}{}{Conjunto de todos os rios e seus afluentes que correm para um mesmo rio principal, formando um sistema bem definido de drenagem.}{ba.ci.a}{0}
\verb{bacia}{}{}{}{}{}{Depressão acentuada de um terreno, que provoca acúmulo de água.}{ba.ci.a}{0}
\verb{baciada}{}{}{}{}{s.f.}{O conteúdo que enche uma bacia.}{ba.ci.a.da}{0}
\verb{bacilar}{}{}{}{}{adj.2g.}{Relativo a bacilo.}{ba.ci.lar}{0}
\verb{bacilar}{}{}{}{}{}{Que tem forma cilíndrica e longa como um bastonete.}{ba.ci.lar}{0}
\verb{bacilo}{}{}{}{}{s.m.}{Bactéria que apresenta forma de bastonete.}{ba.ci.lo}{0}
\verb{bacilo}{}{Biol.}{}{}{}{Nome dado às bactérias cujas espécies são saprófitas ou patogênicas para os seres humanos e outros mamíferos.}{ba.ci.lo}{0}
\verb{bacio}{}{}{}{}{s.m.}{Recipiente apropriado para se recolherem excrementos humanos; urinol, penico.}{ba.ci.o}{0}
\verb{backup}{}{Informát.}{}{}{s.m.}{Cópia de um arquivo, mantida como reserva para o caso de dano ou perda do original; cópia de segurança, becape. }{\textit{backup}}{0}
\verb{baço}{}{}{}{}{adj.}{Que não tem brilho; opaco, fosco.}{ba.ço}{0}
\verb{baço}{}{Anat.}{}{}{s.m.}{Víscera linfoide situada na porção esquerda do abdômen, e que tem como função principal a destruição dos glóbulos vermelhos inúteis.}{ba.ço}{0}
\verb{bacon}{}{}{}{}{s.m.}{Toicinho defumado.}{\textit{bacon}}{0}
\verb{bacorejar}{}{}{}{}{v.t.}{Ter pressentimento; adivinhar, prever. (\textit{Espero estar enganado, mas a mim bacoreja"-me que sua doença não tem cura.})}{ba.co.re.jar}{0}
\verb{bacorejar}{}{}{}{}{}{Fazer propostas; sugerir, insinuar. (\textit{Ela bacorejou que o casamento terminaria.})}{ba.co.re.jar}{0}
\verb{bacorejar}{}{}{}{}{}{Grunhir. (\textit{O leitãozinho bacorejava muito.})}{ba.co.re.jar}{\verboinum{1}}
\verb{bacorinho}{}{}{}{}{s.m.}{Filhote de porco; leitãozinho.}{ba.co.ri.nho}{0}
\verb{bacorinho}{}{Pop.}{}{}{}{Filho pequeno; criança, bebê.}{ba.co.ri.nho}{0}
\verb{bácoro}{}{}{}{}{s.m.}{Porco novo; leitão.}{bá.co.ro}{0}
\verb{bactéria}{}{Biol.}{}{}{s.f.}{Micro"-organismo unicelular, de vida livre ou parasita, cujo material genético encontra"-se espalhado no citoplasma, que se multiplica por cissiparidade.}{bac.té.ria}{0}
\verb{bactericida}{}{}{}{}{adj.2g.}{Diz"-se da substância que elimina as bactérias.}{bac.te.ri.ci.da}{0}
\verb{bacteriologia}{}{Biol.}{}{}{s.f.}{Parte da microbiologia que estuda as bactérias e suas propriedades.}{bac.te.ri.o.lo.gi.a}{0}
\verb{bacteriologista}{}{}{}{}{s.2g.}{Especialista no estudo das bactérias.}{bac.te.ri.o.lo.gis.ta}{0}
\verb{báculo}{}{}{}{}{s.m.}{Bastão alto, com a ponta recurvada, utilizado por bispos.}{bá.cu.lo}{0}
\verb{báculo}{}{Por ext.}{}{}{}{Cajado, bordão.}{bá.cu.lo}{0}
\verb{báculo}{}{Fig.}{}{}{}{Apoio moral ou financeiro; arrimo, amparo.}{bá.cu.lo}{0}
\verb{bacurau}{}{Zool.}{}{}{s.m.}{Ave noturna de plumagem macia e voo silencioso que se alimenta de uma grande variedade de insetos; curiango.}{ba.cu.rau}{0}
\verb{bacurau}{}{Fig.}{}{}{}{Indivíduo que tem o hábito de sair só à noite.}{ba.cu.rau}{0}
\verb{bacuri}{}{}{}{}{s.m.}{Fruto de baga amarela com polpa branca e macia, da qual se fazem doces e refrescos. }{ba.cu.ri}{0}
\verb{bacuri}{}{Pop.}{}{}{s.m.}{Menino pequeno, bebê.}{ba.cu.ri}{0}
\verb{bacurizeiro}{ê}{Bot.}{}{}{s.m.}{Grande árvore resinosa, que fornece madeira nobre amarelada e frutos comestíveis.}{ba.cu.ri.zei.ro}{0}
\verb{badalação}{}{Bras.}{"-ões}{}{s.f.}{Vida social; divertimento.}{ba.da.la.ção}{0}
\verb{badalação}{}{Bras.}{"-ões}{}{}{Frequência em eventos ou ambientes sociais de maneira assídua ou ostentatória.}{ba.da.la.ção}{0}
\verb{badalada}{}{}{}{}{s.f.}{Cada uma das batidas do badalo no sino.}{ba.da.la.da}{0}
\verb{badalada}{}{}{}{}{}{O som produzido por essa batida.}{ba.da.la.da}{0}
\verb{badalado}{}{}{}{}{adj.}{Que é objeto de muito comentário; famoso.}{ba.da.la.do}{0}
\verb{badalar}{}{}{}{}{v.i.}{Dar badaladas.}{ba.da.lar}{0}
\verb{badalar}{}{Fig.}{}{}{}{Proclamar enfaticamente.}{ba.da.lar}{0}
\verb{badalar}{}{}{}{}{}{Fofocar.}{ba.da.lar}{0}
\verb{badalar}{}{}{}{}{}{Frequentar eventos ou ambientes sociais de maneira ostentatória; exibir"-se.}{ba.da.lar}{0}
\verb{badalar}{}{}{}{}{v.t.}{Indicar as horas com badaladas.}{ba.da.lar}{0}
\verb{badalar}{}{}{}{}{}{Promover, divulgar, propalar.}{ba.da.lar}{\verboinum{1}}
\verb{badalo}{}{}{}{}{s.m.}{Peça metálica longa com extremidade em forma de bola que pende dentro de um sino para produzir as badaladas.}{ba.da.lo}{0}
\verb{badameco}{é}{Desus.}{}{}{s.m.}{Pasta usada por estudantes para transportar livros e materiais escolares.}{ba.da.me.co}{0}
\verb{badameco}{é}{Por ext.}{}{}{}{Rapaz.}{ba.da.me.co}{0}
\verb{badameco}{é}{Desus.}{}{}{}{Indivíduo ridículo por se vestir com apuro excessivo.}{ba.da.me.co}{0}
\verb{badameco}{é}{Lus.}{}{}{}{Indivíduo sem importância; joão"-ninguém.}{ba.da.me.co}{0}
\verb{badejo}{é/ ou /ê}{Bras.}{}{}{s.m.}{Peixe que vive em águas costeiras de regiões tropicais.}{ba.de.jo}{0}
\verb{badejo}{é/ ou /ê}{Pop.}{}{}{adj.}{Grande, extenso. (\textit{O problema é que aquilo ainda acabaria num escândalo badejo.})}{ba.de.jo}{0}
\verb{badejo}{é/ ou /ê}{Pop.}{}{}{}{Bonito, vistoso.}{ba.de.jo}{0}
\verb{baderna}{é}{Bras.}{}{}{s.f.}{Bagunça, confusão, desordem.}{ba.der.na}{0}
\verb{baderna}{é}{}{}{}{}{Grupo de rapazes.}{ba.der.na}{0}
\verb{baderna}{é}{}{}{}{}{Grupo de pessoas de má índole; corja, súcia.}{ba.der.na}{0}
\verb{baderna}{é}{}{}{}{}{Noitada, brincadeira, folia.}{ba.der.na}{0}
\verb{baderneiro}{ê}{Bras.}{}{}{adj.}{Que faz baderna, desordem, confusão.}{ba.der.nei.ro}{0}
\verb{baderneiro}{ê}{Bras.}{}{}{}{Que vive na noitada, na folia.}{ba.der.nei.ro}{0}
\verb{badulaque}{}{}{}{}{s.m.}{Coisa miúda, de pouco valor ou utilidade.}{ba.du.la.que}{0}
\verb{badulaque}{}{}{}{}{}{Enfeite de pouco valor, penduricalho.}{ba.du.la.que}{0}
\verb{badulaque}{}{Cul.}{}{}{}{Prato ensopado feito com vísceras do boi.}{ba.du.la.que}{0}
\verb{baeta}{ê}{}{}{}{s.f.}{Tecido felpudo de lã.}{ba.e.ta}{0}
\verb{bafafá}{}{Bras.}{}{}{s.m.}{Tumulto, confusão, rolo.}{ba.fa.fá}{0}
\verb{bafagem}{}{}{"-ens}{}{s.f.}{Vento de fraca intensidade.}{ba.fa.gem}{0}
\verb{bafagem}{}{}{"-ens}{}{}{Expiração pela boca; bafo.}{ba.fa.gem}{0}
\verb{bafagem}{}{Fig.}{"-ens}{}{}{Imaginação criativa; inspiração.}{ba.fa.gem}{0}
\verb{bafejar}{}{}{}{}{v.t.}{Aquecer com o bafo. (\textit{A criança bafejava o vidro, para fazer desenhos.})}{ba.fe.jar}{0}
\verb{bafejar}{}{Fig.}{}{}{}{Acariciar, estimular, encorajar, favorecer. (\textit{A fama bafeja tanto pessoas iluminadas e qualificadas quanto outras.})}{ba.fe.jar}{0}
\verb{bafejar}{}{Fig.}{}{}{}{Sugerir, inspirar. (\textit{Aquela cabeça heroica, bafejada pelo sopro da liberdade, agora bafejava a luta dos mais novos. })}{ba.fe.jar}{0}
\verb{bafejar}{}{}{}{}{v.i.}{Exalar bafo. (\textit{Mal ele sentiu seu sorriso bafejar"-lhe a face, apaixonou"-se.})}{ba.fe.jar}{\verboinum{1}}
\verb{bafejo}{ê}{}{}{}{s.m.}{Ato ou efeito de bafejar.}{ba.fe.jo}{0}
\verb{bafejo}{ê}{}{}{}{}{Ar expirado pela boca; bafo.}{ba.fe.jo}{0}
\verb{bafejo}{ê}{}{}{}{}{Vento muito leve; aragem.}{ba.fe.jo}{0}
\verb{bafejo}{ê}{Fig.}{}{}{}{Proteção; boa fortuna.}{ba.fe.jo}{0}
\verb{bafio}{}{}{}{}{s.m.}{Cheiro característico do que está úmido ou sem renovação de ar; bolor.}{ba.fi.o}{0}
\verb{bafo}{}{}{}{}{s.m.}{Ar exalado dos pulmões, com cheiro característico.}{ba.fo}{0}
\verb{bafo}{}{}{}{}{}{Hálito fétido; mau hálito.}{ba.fo}{0}
\verb{bafo}{}{Por ext.}{}{}{}{Ar quente, mormaço.}{ba.fo}{0}
\verb{bafo}{}{Fig.}{}{}{}{Sopro criador; inspiração.}{ba.fo}{0}
\verb{bafo}{}{Fig.}{}{}{}{Proteção, amparo, bafejo.}{ba.fo}{0}
\verb{bafo}{}{}{}{}{}{Jogo infantil que consiste em bater a mão, em forma de concha, sobre o monte de figurinhas do adversário, visando a virá"-las e tomá"-las para si.}{ba.fo}{0}
\verb{bafômetro}{}{Bras.}{}{}{s.m.}{Aparelho que permite determinar a concentração de álcool no organismo de uma pessoa, analisando o ar exalado dos pulmões.}{ba.fô.me.tro}{0}
\verb{baforada}{}{}{}{}{s.f.}{Exalação de ar pela boca ou narinas.}{ba.fo.ra.da}{0}
\verb{baforada}{}{}{}{}{}{Golfada de fumaça de cigarro, charuto ou cachimbo.}{ba.fo.ra.da}{0}
\verb{baforada}{}{Fig.}{}{}{}{Bravata, vantagem.}{ba.fo.ra.da}{0}
\verb{baforar}{}{}{}{}{v.t.}{Expelir bafo; soprar.}{ba.fo.rar}{0}
\verb{baforar}{}{Fig.}{}{}{v.i.}{Contar vantagem; dizer bravata; jactar"-se.}{ba.fo.rar}{\verboinum{1}}
\verb{baga}{}{}{}{}{s.f.}{Fruto carnoso, que não se abre por si mesmo ao alcançar a maturidade, com carpelos e sementes, e geralmente comestível, como o tomate, a banana, o mamão, a laranja etc.}{ba.ga}{0}
\verb{baga}{}{}{}{}{}{Gota de orvalho ou de suor; gotícula, camarinha.}{ba.ga}{0}
\verb{baga}{}{}{}{}{}{A semente descascada da mamona.}{ba.ga}{0}
\verb{baga}{}{Bras.}{}{}{}{Cachaça.}{ba.ga}{0}
\verb{bagaceira}{ê}{}{}{}{s.f.}{Lugar onde se junta o bagaço da uva.}{ba.ga.cei.ra}{0}
\verb{bagaceira}{ê}{Por ext.}{}{}{}{Resto, resíduo, sobra.}{ba.ga.cei.ra}{0}
\verb{bagaceira}{ê}{Por ext.}{}{}{}{Conjunto de coisas sem valor ou utilidade; cacarecos.}{ba.ga.cei.ra}{0}
\verb{bagaceira}{ê}{Bras.}{}{}{}{Aguardente feita do bagaço da uva.}{ba.ga.cei.ra}{0}
\verb{bagaceira}{ê}{}{}{}{}{Aguardente de cana; cachaça.}{ba.ga.cei.ra}{0}
\verb{bagaço}{}{}{}{}{s.m.}{Resíduo de frutos, ervas etc. depois de moído ou espremido para extrair o suco.}{ba.ga.ço}{0}
\verb{bagaço}{}{Por ext.}{}{}{}{Coisas velhas, surradas ou sem utilidade; resto.}{ba.ga.ço}{0}
\verb{bagageiro}{ê}{}{}{}{adj.}{Que transporta bagagem; carregador.}{ba.ga.gei.ro}{0}
\verb{bagageiro}{ê}{}{}{}{}{Diz"-se de veículo ou vagão em que se transporta bagagem.}{ba.ga.gei.ro}{0}
\verb{bagageiro}{ê}{}{}{}{s.m.}{Compartimento do carro ou estrutura metálica de qualquer veículo onde se transporta bagagem.}{ba.ga.gei.ro}{0}
\verb{bagagem}{}{}{"-ens}{}{s.f.}{Conjunto de objetos que um viajante leva em uma viagem, como malas, bolsas, mochilas, pacotes, caixas.}{ba.ga.gem}{0}
\verb{bagagem}{}{Fig.}{"-ens}{}{}{Conjunto das obras de um escritor, cientista, artista.}{ba.ga.gem}{0}
\verb{bagana}{}{}{}{}{s.f.}{Ponta de cigarro, charuto ou baseado que sobra depois de fumado; guimba.}{ba.ga.na}{0}
\verb{bagana}{}{Por ext.}{}{}{}{Cigarro, especialmente os feitos à mão.}{ba.ga.na}{0}
\verb{bagana}{}{}{}{}{}{Comida de má qualidade.}{ba.ga.na}{0}
\verb{bagana}{}{}{}{}{}{Coisa sem valor; bagatela, ninharia.}{ba.ga.na}{0}
\verb{bagatela}{é}{}{}{}{s.f.}{Objeto de pouco valor ou utilidade; bugiganga, ninharia, cacareco.}{ba.ga.te.la}{0}
\verb{bagatela}{é}{}{}{}{}{Quantia insignificante; preço baixo.}{ba.ga.te.la}{0}
\verb{bago}{}{}{}{}{s.m.}{Cada fruto do cacho de uva.}{ba.go}{0}
\verb{bago}{}{Por ext.}{}{}{}{Qualquer fruto carnoso semelhante à uva.}{ba.go}{0}
\verb{bago}{}{Por ext.}{}{}{}{Qualquer grão miúdo.}{ba.go}{0}
\verb{bago}{}{Pop.}{}{}{}{Testículo [Obs.: usa"-se geralmente no plural].    }{ba.go}{0}
\verb{bagre}{}{Zool.}{}{}{s.m.}{Peixe com corpo mole, pele sem escamas e grandes barbilhões na maxila inferior.}{ba.gre}{0}
\verb{bagre}{}{Pop.}{}{}{}{Pessoa muito feia.}{ba.gre}{0}
\verb{baguete}{é}{}{}{}{s.f.}{Tipo de enfeite colocado nas meias, acima do tornozelo.}{ba.gue.te}{0}
\verb{baguete}{é}{}{}{}{}{Tipo de pão francês cilíndrico, fino e longo.}{ba.gue.te}{0}
\verb{bagulho}{}{}{}{}{s.m.}{Semente contida no bago da uva ou de outros frutos.}{ba.gu.lho}{0}
\verb{bagulho}{}{Por ext.}{}{}{}{Objeto sem valor ou utilidade.}{ba.gu.lho}{0}
\verb{bagulho}{}{Pop.}{}{}{}{Maconha.}{ba.gu.lho}{0}
\verb{bagulho}{}{Pop.}{}{}{}{Objeto roubado.}{ba.gu.lho}{0}
\verb{bagunça}{}{}{}{}{s.f.}{Falta de ordem; desorganização.}{ba.gun.ça}{0}
\verb{bagunça}{}{}{}{}{}{Baderna, bagunça, confusão, farra.}{ba.gun.ça}{0}
\verb{bagunça}{}{Bras.}{}{}{}{Máquina para remover aterro.}{ba.gun.ça}{0}
\verb{bagunçar}{}{}{}{}{v.i.}{Promover bagunça ou desordem.}{ba.gun.çar}{\verboinum{3}}
\verb{bagunceiro}{ê}{}{}{}{adj.}{Que faz ou costuma fazer bagunça.}{ba.gun.cei.ro}{0}
\verb{bah}{}{}{}{}{interj.}{Expressão que denota admiração, surpresa, geralmente usada no Sul do Brasil. (\textit{Bah, não me digas que vai até lá sozinha.})}{bah}{0}
\verb{baia}{}{}{}{}{s.f.}{Cada um dos compartimentos onde se recolhem os animais nos estábulos.}{bai.a}{0}
\verb{baia}{}{Por ext.}{}{}{}{Qualquer compartimento individual de trabalho, geralmente separado por divisórias.}{bai.a}{0}
\verb{baía}{}{Geogr.}{}{}{s.f.}{Pequeno golfo, reentrância marítima.}{ba.í.a}{0}
\verb{baía}{}{Bras.}{}{}{}{Lago que se comunica com um rio através de canal.}{ba.í.a}{0}
\verb{baía}{}{}{}{}{}{Canal para escoamento de pântanos.}{ba.í.a}{0}
\verb{baiacu}{}{Zool.}{}{}{s.m.}{Peixe com corpo coberto por escamas, espinhos ou placas ósseas, que inflam a barriga quando se sentem ameaçados, cuja carne é considerada venenosa.}{bai.a.cu}{0}
\verb{baiacu}{}{Pop.}{}{}{}{Indivíduo muito gordo e geralmente de baixa estatura.}{bai.a.cu}{0}
\verb{baiano}{}{}{}{}{adj.}{Relativo à Bahia.}{bai.a.no}{0}
\verb{baiano}{}{}{}{}{s.m.}{Indivíduo natural ou habitante desse estado.}{bai.a.no}{0}
\verb{baião}{}{Bras.}{"-ões}{}{s.m.}{Dança e canto popular, executado com viola, acordeão, zabumba e outros instrumentos.}{bai.ão}{0}
\verb{baião}{}{Mús.}{"-ões}{}{}{Ritmo de origem nordestina.}{bai.ão}{0}
\verb{baila}{}{Desus.}{}{}{s.f.}{Baile, bailado.}{bai.la}{0}
\verb{baila}{}{}{}{}{}{Usado na expressão \textit{vir à baila}: introduzir na conversa, mencionar.}{bai.la}{0}
\verb{bailado}{}{}{}{}{adj.}{Em que há dança.}{bai.la.do}{0}
\verb{bailado}{}{}{}{}{s.m.}{Dança que faz parte de espetáculos, filmes, peças, rituais.}{bai.la.do}{0}
\verb{bailado}{}{}{}{}{}{Balé, dança.}{bai.la.do}{0}
\verb{bailar}{}{}{}{}{v.t.}{Movimentar o corpo ao som de música; dançar.}{bai.lar}{0}
\verb{bailar}{}{Por ext.}{}{}{v.i.}{Balançar, mover"-se, oscilar, tremer.}{bai.lar}{\verboinum{1}}
\verb{bailarino}{}{}{}{}{s.m.}{Indivíduo que se dedica profissionalmente à dança.}{bai.la.ri.no}{0}
\verb{bailarino}{}{Por ext.}{}{}{}{Indivíduo que dança muito bem.}{bai.la.ri.no}{0}
\verb{baile}{}{}{}{}{s.m.}{Festa, geralmente de caráter formal, em que há dança.}{bai.le}{0}
\verb{baile}{}{}{}{}{}{Dança alegre e festiva.}{bai.le}{0}
\verb{bailéu}{}{}{}{}{s.m.}{Designação geral de obras e peças em balanço, como sacadas, abas de telhado ou prateleiras suspensas em paredes.}{bai.léu}{0}
\verb{bailéu}{}{}{}{}{}{Andaime, palanque.}{bai.léu}{0}
\verb{bailéu}{}{}{}{}{}{Sacada, beirada do telhado; prateleira.}{bai.léu}{0}
\verb{bainha}{}{}{}{}{s.f.}{Dobra e costura feita na barra de um tecido para que não desfie.}{ba.i.nha}{0}
\verb{bainha}{}{}{}{}{}{Estojo onde se guarda a lâmina de arma branca.}{ba.i.nha}{0}
\verb{bainha}{}{Bot.}{}{}{}{Parte da folha que envolve o caule.}{ba.i.nha}{0}
\verb{baio}{}{}{}{}{adj.}{De cor castanha ou amarelada.}{bai.o}{0}
\verb{baio}{}{}{}{}{s.m.}{Cavalo de cor baia.}{bai.o}{0}
\verb{baioneta}{ê}{}{}{}{s.f.}{Arma branca pontiaguda que se adapta à ponta do cano do fuzil, utilizada em combates corpo a corpo.}{bai.o.ne.ta}{0}
\verb{baionetada}{}{}{}{}{s.f.}{Golpe de baioneta.}{bai.o.ne.ta.da}{0}
\verb{bairrismo}{}{}{}{}{s.m.}{Qualidade ou ação de bairrista.}{bair.ris.mo}{0}
\verb{bairrista}{}{}{}{}{adj.2g.}{Que habita ou frequenta um bairro.}{bair.ris.ta}{0}
\verb{bairrista}{}{}{}{}{}{Que defende os interesses de seu bairro ou região.}{bair.ris.ta}{0}
\verb{bairrista}{}{Bras.}{}{}{}{Diz"-se de quem é muito afeiçoado a sua cidade, região ou país de origem, hostilizando ou menosprezando aquilo que é ou vem de fora.}{bair.ris.ta}{0}
\verb{bairro}{}{}{}{}{s.m.}{Cada uma das subdivisões de uma cidade ou vila, para facilitar a localização e a administração.}{bair.ro}{0}
\verb{bairro}{}{}{}{}{}{Região habitada que se localiza longe do centro de uma cidade ou nas suas cercanias; arraial, distrito.}{bair.ro}{0}
\verb{baita}{}{Bras.}{}{}{adj.}{Muito grande; enorme.}{bai.ta}{0}
\verb{baita}{}{}{}{}{}{Muito competente; exímio, excelente.}{bai.ta}{0}
\verb{baitaca}{}{}{}{}{}{Var. de \textit{maritaca}.}{bai.ta.ca}{0}
\verb{baiuca}{}{}{}{}{s.f.}{Habitação humilde.}{bai.u.ca}{0}
\verb{baiuca}{}{}{}{}{}{Local onde se vendem bebidas alcoólicas; taberna, boteco.}{bai.u.ca}{0}
\verb{baiuca}{}{}{}{}{}{Casa de jogo ou de prostituição.}{bai.u.ca}{0}
\verb{baixa}{ch}{}{}{}{s.f.}{Depressão do terreno; lugar baixo. (\textit{Logo que chegamos na baixa daquela região, sentimos o calor que lhe era característico.})}{bai.xa}{0}
\verb{baixa}{ch}{}{}{}{}{Diminuição de valor ou altura; declínio. (\textit{A baixa dos preços não vai durar muito tempo.})}{bai.xa}{0}
\verb{baixa}{ch}{}{}{}{}{Desligamento de um membro da corporação militar, por esgotamento do tempo de serviço ou por solicitação. (\textit{Meu irmão vai dar baixa do exército na próxima semana.})}{bai.xa}{0}
\verb{baixa}{ch}{}{}{}{}{Perda de efetivo militar por morte, ferimento ou prisão pelo inimigo. (\textit{Foram muitas baixas no exército durante a primeira batalha.})}{bai.xa}{0}
\verb{baixada}{ch}{}{}{}{s.f.}{Planície em meio ou próxima a montanhas ou regiões relativamente mais altas.}{bai.xa.da}{0}
\verb{baixada}{ch}{}{}{}{}{Depressão de terreno junto a colinas ou coxilhas.}{bai.xa.da}{0}
\verb{baixa"-mar}{ch}{}{baixa"-mares ⟨ch⟩}{}{s.f.}{Nível mínimo da maré; maré baixa.}{bai.xa"-mar}{0}
\verb{baixar}{ch}{}{}{}{v.t.}{Dirigir para uma posição inferior, abaixar.}{bai.xar}{0}
\verb{baixar}{ch}{}{}{}{}{Tornar oficial uma decisão, por meio de documentos públicos.}{bai.xar}{0}
\verb{baixar}{ch}{Fig.}{}{}{v.i.}{Diminuir em altura, em cotação, em valor.}{bai.xar}{\verboinum{1}}
\verb{baixaria}{ch}{Bras.}{}{}{s.f.}{Pessoa, coisa ou gesto grosseiro, violento ou inconveniente.}{bai.xa.ri.a}{0}
\verb{baixela}{ché}{}{}{}{s.f.}{Conjunto de utensílios para o serviço de mesa, geralmente de material nobre.}{bai.xe.la}{0}
\verb{baixela}{ché}{}{}{}{}{Os objetos litúrgicos de valor usados em uma igreja.}{bai.xe.la}{0}
\verb{baixeza}{chê}{}{}{}{s.f.}{Qualidade do que é baixo; pequenez.}{bai.xe.za}{0}
\verb{baixio}{ch}{Geogr.}{}{}{s.m.}{Banco de areia ou rochedo.}{bai.xi.o}{0}
\verb{baixista}{ch}{Mús.}{}{}{s.2g.}{Instrumentista que toca contrabaixo.}{bai.xis.ta}{0}
\verb{baixista}{ch}{}{}{}{adj.2g.}{Diz"-se do profissional que, no mercado financeiro, visa desvalorizar títulos e mercadorias, ou que opera apenas no mercado em baixa.}{bai.xis.ta}{0}
\verb{baixo}{ch}{}{}{}{adj.}{De pequena altura.}{bai.xo}{0}
\verb{baixo}{ch}{}{}{}{}{Que se volta para o chão ou está mais próximo dele.}{bai.xo}{0}
\verb{baixo}{ch}{Por ext.}{}{}{}{Que tem pouca qualidade ou intensidade. (\textit{Ouro baixo. Voz baixa.})}{bai.xo}{0}
\verb{baixo}{ch}{}{}{}{}{Que fica ao sul de outra região.}{bai.xo}{0}
\verb{baixo}{ch}{}{}{}{}{De som grave.}{bai.xo}{0}
\verb{baixo}{ch}{}{}{}{s.m.}{Forma reduzida de \textit{contrabaixo}, instrumento de cordas, cuja afinação é feita em tom mais grave que todos os demais instrumentos.}{bai.xo}{0}
\verb{baixo"-astral}{ch}{}{baixos"-astrais ⟨ch⟩}{}{s.m.}{Mau humor, depressão, pessimismo, infelicidade, tédio.}{bai.xo"-as.tral}{0}
\verb{baixo"-astral}{ch}{}{baixos"-astrais ⟨ch⟩}{}{adj.2g.}{Diz"-se de pessoa, evento ou ambiente em que há baixo"-astral; mal"-humorado, depressivo, maçante, desagradável.}{bai.xo"-as.tral}{0}
\verb{baixo"-relevo}{ch\ldots{}ê}{Art.}{baixos"-relevos ⟨ch\ldots{}ê⟩}{}{s.m.}{Técnica de escultura executada sobre uma superfície de forma que as figuras se projetem ou se destaquem do plano de fundo.}{bai.xo"-re.le.vo}{0}
\verb{baixote}{chó}{Pop.}{}{}{adj.}{Diz"-se de indivíduo de baixa estatura.}{bai.xo.te}{0}
\verb{baixo"-ventre}{ch}{Anat.}{baixos"-ventres ⟨ch⟩}{}{s.m.}{Região hipogástrica, na parte inferior do abdômen.}{bai.xo"-ven.tre}{0}
\verb{bajoujar}{}{Desus.}{}{}{v.t.}{Cobrir de louvores, de palavras de admiração; lisonjear, adular. (\textit{Ele sempre baboujava as moças de sua sala.})}{ba.jou.jar}{0}
\verb{bajoujar}{}{}{}{}{}{Cobrir de carinho; amimar, acariciar. (\textit{A mãe não se cansava de bajoujar as crianças.})}{ba.jou.jar}{\verboinum{1}}
\verb{bajulação}{}{}{"-ões}{}{s.f.}{Ato ou efeito de bajular; adulação.}{ba.ju.la.ção}{0}
\verb{bajulador}{ô}{}{}{}{adj.}{Que bajula, que adula servilmente.}{ba.ju.la.dor}{0}
\verb{bajulador}{ô}{}{}{}{s.m.}{Indivíduo que bajula; adulador.}{ba.ju.la.dor}{0}
\verb{bajular}{}{}{}{}{v.t.}{Lisonjear para obter vantagem; adular.}{ba.ju.lar}{\verboinum{1}}
\verb{bala}{}{}{}{}{s.f.}{Projétil metálico arredondado ou ogival, próprio para ser disparado por uma arma de fogo.}{ba.la}{0}
\verb{bala}{}{}{}{}{}{Pequeno doce, de consistência firme, feito com calda de açúcar aromatizada e acrescida de corantes, ou de ingredientes com sabores diversos. }{ba.la}{0}
\verb{bala}{}{}{}{}{}{Fardo de papel equivalente a dez resmas, ou cinco mil folhas.}{ba.la}{0}
\verb{balaço}{}{}{}{}{s.m.}{Grande bala.}{ba.la.ço}{0}
\verb{balaço}{}{}{}{}{}{Tiro de bala.}{ba.la.ço}{0}
\verb{balada}{}{}{}{}{s.f.}{Forma musical simples, de andamento lento, composta de letra rimada e versificada.}{ba.la.da}{0}
\verb{balado}{}{}{}{}{s.m.}{Balido.}{ba.la.do}{0}
\verb{balaio}{}{}{}{}{s.m.}{Cesto grande feito de palha, taquara, bambu, cipó etc., usado para transporte ou para guardar objetos.}{ba.lai.o}{0}
\verb{balaio}{}{}{}{}{}{Partidário da Balaiada, rebelião popular que ocorreu no Maranhão.}{ba.lai.o}{0}
\verb{balalaica}{}{Mús.}{}{}{s.f.}{Instrumento musical de três cordas e forma triangular, muito usado pelos russos na execução de sua música popular.}{ba.la.lai.ca}{0}
\verb{balança}{}{}{}{}{s.f.}{Instrumento com que se determina a massa ou peso dos corpos.}{ba.lan.ça}{0}
\verb{balança}{}{Fig.}{}{}{}{Equilíbrio, prudência, ponderação.}{ba.lan.ça}{0}
\verb{balança}{}{Astrol.}{}{}{}{Constelação do Zodíaco; libra.}{ba.lan.ça}{0}
\verb{balançar}{}{}{}{}{v.t.}{Fazer com que algo se movimente de um lado a outro, repetidamente, como um pêndulo; oscilar; mover, embalar. (\textit{As ondas balançam o navio.})}{ba.lan.çar}{0}
\verb{balançar}{}{}{}{}{}{Equilibrar, compensar, contrabalançar.}{ba.lan.çar}{0}
\verb{balançar}{}{}{}{}{}{Examinar, comparando; pesar.}{ba.lan.çar}{0}
\verb{balançar}{}{}{}{}{v.i.}{Mover o corpo de um lado para o outro.}{ba.lan.çar}{\verboinum{3}}
\verb{balancê}{}{Mús.}{}{}{s.m.}{Passo de quadrilha, que consiste em movimentos balançados do corpo sem deslocamento dos pés.}{ba.lan.cê}{0}
\verb{balancê}{}{}{}{}{}{Antiga prensa de moedeiro.}{ba.lan.cê}{0}
\verb{balancê}{}{}{}{}{}{Máquina usada para cortar papelão, couro, metal etc.}{ba.lan.cê}{0}
\verb{balanceado}{}{}{}{}{adj.}{Que balança; que ginga; oscilante.}{ba.lan.ce.a.do}{0}
\verb{balanceado}{}{}{}{}{}{Diz"-se de alimentação, ração etc., cujos componentes são equilibrados nas quantidades e qualidades adequadas. }{ba.lan.ce.a.do}{0}
\verb{balanceamento}{}{}{}{}{s.m.}{Ato ou efeito de balancear.}{ba.lan.ce.a.men.to}{0}
\verb{balanceamento}{}{}{}{}{}{Uniformizar o peso da roda em toda a sua circunferência, para que não vibre em velocidades mais altas.}{ba.lan.ce.a.men.to}{0}
\verb{balancear}{}{}{}{}{v.t.}{Fazer os ajustes necessários para que todas as partes tenham as mesmas características.}{ba.lan.ce.ar}{\verboinum{4}}
\verb{balancete}{ê}{}{}{}{s.m.}{Levantamento contábil parcial de uma firma, relativo a um período inferior a um ano.}{ba.lan.ce.te}{0}
\verb{balancete}{ê}{Fig.}{}{}{}{Cálculo, avaliação.}{ba.lan.ce.te}{0}
\verb{balancim}{}{}{"-ins}{}{s.m.}{Peça a que se ligam tirantes que permitem a movimentação pendular do conjunto.}{ba.lan.cim}{0}
\verb{balanço}{}{}{}{}{s.m.}{Ato ou efeito de balançar.}{ba.lan.ço}{0}
\verb{balanço}{}{}{}{}{}{Peça para diversão que consiste num assento, suspenso por cordas ou correntes que produzem um movimento pendular.}{ba.lan.ço}{0}
\verb{balanço}{}{}{}{}{}{Verificação das contas de empresa, em que se consideram gastos e receitas.}{ba.lan.ço}{0}
\verb{balanço}{}{Fig.}{}{}{}{Ponderação a respeito de um fato, considerando seus aspectos positivos e negativos.}{ba.lan.ço}{0}
\verb{balangandã}{}{Bras.}{}{}{s.m.}{Ornamento ou amuleto de metal, em forma de figas, medalhas, chaves etc., usado pelas baianas em dias de festas.}{ba.lan.gan.dã}{0}
\verb{balangandã}{}{Por ext.}{}{}{}{Penduricalho de qualquer formato.}{ba.lan.gan.dã}{0}
\verb{balão}{}{}{"-ões}{}{s.m.}{Peça de tamanho variado e forma arredondada que se sustenta no ar e que consiste num envoltório de gás ou de ar aquecido.}{ba.lão}{0}
\verb{balão}{}{}{"-ões}{}{}{Veículo para navegação aérea que se caracteriza pela sustentação por meio de ar aquecido ou gases mais leves do que o ar.}{ba.lão}{0}
\verb{balão}{}{}{"-ões}{}{}{Recurso utilizado nas histórias em quadrinhos para acrescentar as falas das personagens.}{ba.lão}{0}
\verb{balão"-de"-ensaio}{}{}{balões"-de"-ensaio}{}{s.m.}{Pequeno balão que se solta para verificar a direção dos ventos.}{ba.lão"-de"-en.sai.o}{0}
\verb{balão"-de"-ensaio}{}{Fig.}{balões"-de"-ensaio}{}{}{Boato ou notícia que se faz circular para verificar as tendências da opinião pública.}{ba.lão"-de"-en.sai.o}{0}
\verb{balão"-sonda}{}{}{balões"-sondas}{}{s.m.}{Balão que conduz aparelhos meteorológicos para observações nas altas camadas da atmosfera.}{ba.lão"-son.da}{0}
\verb{balar}{}{}{}{}{v.i.}{Soltar balidos (a ovelha ou o cordeiro); balir.}{ba.lar}{\verboinum{1}}
\verb{balar}{}{}{}{}{s.m.}{A ação de balar.}{ba.lar}{0}
\verb{balata}{}{Bot.}{}{}{s.f.}{Árvore de terra firme, que fornece madeira útil quase roxa, usada na construção civil e naval, e cujo látex é usado na fabricação de isolantes e correias.}{ba.la.ta}{0}
\verb{balata}{}{}{}{}{}{Esse látex.}{ba.la.ta}{0}
\verb{balaustrada}{}{}{}{}{s.f.}{Série ou fileira de balaústres que forma um parapeito.}{ba.la.us.tra.da}{0}
\verb{balaustrada}{}{Por ext.}{}{}{}{Qualquer parapeito, corrimão ou grade de apoio ou proteção.}{ba.la.us.tra.da}{0}
\verb{balaústre}{}{}{}{}{s.m.}{Pequena coluna de madeira, pedra ou metal, que sustenta, com outras iguais, regularmente distribuídas, uma travessa, corrimão ou peitoril.}{ba.la.ús.tre}{0}
\verb{balaústre}{}{}{}{}{}{Haste vertical de metal ou de madeira, usada para auxiliar o passageiro no embarque e desembarque de veículos coletivos.}{ba.la.ús.tre}{0}
\verb{balaústre}{}{}{}{}{}{Cada uma das peças torneadas que formam o espaldar de cadeiras ou a cabeceira de cama.}{ba.la.ús.tre}{0}
\verb{balázio}{}{}{}{}{s.m.}{Balaço.}{ba.lá.zio}{0}
\verb{balbuciar}{}{}{}{}{v.t.}{Pronunciar imperfeitamente e com hesitação.}{bal.bu.ci.ar}{0}
\verb{balbuciar}{}{}{}{}{}{Pronunciar sons sem sentido.}{bal.bu.ci.ar}{0}
\verb{balbuciar}{}{Fig.}{}{}{}{Falar sobre algum assunto confusamente, sem o domínio suficiente.}{bal.bu.ci.ar}{\verboinum{6}}
\verb{balbucio}{}{}{}{}{s.m.}{Ato de balbuciar.}{bal.bu.ci.o}{0}
\verb{balbucio}{}{Fig.}{}{}{}{Experiência inicial; tentativa, ensaio.}{bal.bu.ci.o}{0}
\verb{balbúrdia}{}{}{}{}{s.f.}{Desordem barulhenta; vozearia, algazarra.}{bal.búr.dia}{0}
\verb{balbúrdia}{}{}{}{}{}{Situação confusa; complicação.}{bal.búr.dia}{0}
\verb{balcão}{}{}{"-ões}{}{s.m.}{Plataforma saliente da fachada de casa ou edifício, geralmente em balanço e sustentada por colunas, consolos, e guarnecida de um parapeito, à qual se tem acesso do interior, por uma porta; sacada.}{bal.cão}{0}
\verb{balcão}{}{}{"-ões}{}{}{Móvel, da altura de uma mesa ou um pouco mais alto, empregado em lojas, repartições ou outros estabelecimentos, para atendimento do público ou da clientela, e que eventualmente serve para expor mercadorias.}{bal.cão}{0}
\verb{balcão}{}{}{"-ões}{}{}{Numa sala de espetáculos, localidade da plateia situada entre os camarotes e as galerias.}{bal.cão}{0}
\verb{balconista}{}{}{}{}{s.2g.}{Indivíduo que atende fregueses nos estabelecimentos comerciais, com ou sem balcão; vendedor, caixeiro.}{bal.co.nis.ta}{0}
\verb{balda}{}{}{}{}{s.f.}{Defeito ou hábito arraigado; mania, veneta. (\textit{Ele sempre teve aquela balda de não sentar de costas para a porta.})}{bal.da}{0}
\verb{balda}{}{}{}{}{}{Falta, carência. (\textit{O país não sofre a balda de homens honestos.})}{bal.da}{0}
\verb{baldado}{}{}{}{}{adj.}{Que não teve êxito; fracassado, malogrado, inútil. (\textit{Até agora, só vimos tentativas baldadas de melhorar a questão social no Brasil.})}{bal.da.do}{0}
\verb{baldaquim}{}{}{"-ins}{}{s.m.}{Espécie de dossel sustentado por colunas, que serve de cúpula ou coroa de um altar, trono, sólio ou leito.}{bal.da.quim}{0}
\verb{baldaquim}{}{}{"-ins}{}{}{Cobertura leve por cima da porta externa, para protegê"-la da chuva.}{bal.da.quim}{0}
\verb{baldar}{}{}{}{}{v.t.}{Tornar inútil; malograr, frustrar.}{bal.dar}{0}
\verb{baldar}{}{}{}{}{}{Empregar com mau resultado.}{bal.dar}{0}
\verb{baldar}{}{}{}{}{v.pron.}{No carteado, livrar"-se de cartas inúteis.}{bal.dar}{\verboinum{1}}
\verb{balde}{}{}{}{}{s.m.}{Recipiente feito de metal, plástico ou madeira, usado para tirar ou carregar líquidos, areia etc., que possui alça na parte superior.}{bal.de}{0}
\verb{baldeação}{}{}{"-ões}{}{s.f.}{Ato ou efeito de baldear, de transferir pessoa ou coisa de uma embarcação para outra.}{bal.de.a.ção}{0}
\verb{baldeação}{}{}{"-ões}{}{}{Ato ou efeito de molhar ou lavar com balde.}{bal.de.a.ção}{0}
\verb{baldeação}{}{}{"-ões}{}{}{Faixa de terreno à volta das salinas, de onde se tira terra para reparo nestas.}{bal.de.a.ção}{0}
\verb{baldear}{}{}{}{}{}{Passar pessoas ou objetos de um veículo para outro, ou de um lugar para outro.}{bal.de.ar}{\verboinum{4}}
\verb{baldear}{}{}{}{}{v.t.}{Tirar líquido com baldes.}{bal.de.ar}{0}
\verb{baldio}{}{}{}{}{adj.}{Que não vale a pena; inútil.}{bal.di.o}{0}
\verb{baldio}{}{}{}{}{}{Sem cultivo; agreste.}{bal.di.o}{0}
\verb{baldio}{}{}{}{}{s.m.}{Terreno por cultivar.}{bal.di.o}{0}
\verb{balé}{}{}{}{}{s.m.}{Dança artística na qual poses e passos são combinados para criar uma forma de expressão.}{ba.lé}{0}
\verb{balé}{}{}{}{}{}{Conjunto de bailarinos que interpretam um balé.}{ba.lé}{0}
\verb{balear}{}{}{}{}{adj.2g.}{Que é próprio para dar impulso ou para se arremessar.}{ba.le.ar}{0}
\verb{balear}{}{}{}{}{v.t.}{Atingir com bala, ferindo ou matando.}{ba.le.ar}{\verboinum{4}}
\verb{baleeira}{ê}{}{}{}{s.f.}{Embarcação que permite manobras rápidas, concebida para a pesca.}{ba.le.ei.ra}{0}
\verb{baleeira}{ê}{Por ext.}{}{}{}{Embarcação usada para casos de desembarque de emergência. }{ba.le.ei.ra}{0}
\verb{baleeiro}{ê}{}{}{}{s.m.}{Pescador de baleias.}{ba.le.ei.ro}{0}
\verb{baleeiro}{ê}{}{}{}{}{Navio próprio para pesca de baleias.}{ba.le.ei.ro}{0}
\verb{baleia}{ê}{Zool.}{}{}{s.f.}{Mamífero marinho de grande porte, pele lisa, desprovido de pelo, que possui espessa camada de gordura sob a pele, membros posteriores atrofiados e anteriores adaptados para natação.}{ba.lei.a}{0}
\verb{baleia}{ê}{Pop.}{}{}{}{Indivíduo muito gordo.}{ba.lei.a}{0}
\verb{baleiro}{ê}{}{}{}{s.m.}{Vendedor ambulante de doces ou balas.}{ba.lei.ro}{0}
\verb{baleiro}{ê}{}{}{}{}{Recipiente em que se colocam balas, confeitos, chocolates.}{ba.lei.ro}{0}
\verb{balela}{é}{}{}{}{s.f.}{Notícia ou dito sem fundamento; boato.}{ba.le.la}{0}
\verb{baleote}{ó}{Zool.}{}{}{s.m.}{Filhote de baleia que ainda não desmamou.}{ba.le.o.te}{0}
\verb{baleote}{ó}{Zool.}{}{}{}{Baleia encontrada em todos os oceanos, de porte pequeno, alcança o comprimento máximo de dez metros, possui cabeça quase triangular, corpo cinza, mais escuro no dorso, ventre claro, e uma faixa branca em cada nadadeira peitoral.}{ba.le.o.te}{0}
\verb{balido}{}{}{}{}{s.m.}{Som emitido por ovelha ou por cordeiro; balado.}{ba.li.do}{0}
\verb{balido}{}{Fig.}{}{}{}{Reclamação dos paroquianos contra o pároco.}{ba.li.do}{0}
\verb{balir}{}{}{}{}{v.i.}{Soltar balidos (a ovelha ou o cordeiro); balar.}{ba.lir}{\verboinum{18}}
\verb{balística}{}{Fís.}{}{}{s.f.}{Ramo da física que estuda o movimento dos projéteis, particularmente os disparados por armas de fogo.}{ba.lís.ti.ca}{0}
\verb{baliza}{}{}{}{}{s.f.}{Estaca que marca um limite.}{ba.li.za}{0}
\verb{baliza}{}{}{}{}{}{Boia que serve de sinal na navegação.}{ba.li.za}{0}
\verb{baliza}{}{}{}{}{s.2g.}{Pessoa que faz acrobacias com um bastão, à frente de certos desfiles.}{ba.li.za}{0}
\verb{balizagem}{}{}{"-ens}{}{s.f.}{Balizamento.}{ba.li.za.gem}{0}
\verb{balizamento}{}{}{}{}{s.m.}{Ato ou efeito de balizar, de demarcar com balizas; balizagem.}{ba.li.za.men.to}{0}
\verb{balizar}{}{}{}{}{v.t.}{Marcar com balizas; demarcar, delimitar.}{ba.li.zar}{0}
\verb{balizar}{}{}{}{}{}{Separar com baliza; limitar, restringir.}{ba.li.zar}{\verboinum{1}}
\verb{balneário}{}{}{}{}{adj.}{Relativo a banho.}{bal.ne.á.rio}{0}
\verb{balneário}{}{}{}{}{s.m.}{Recinto público destinado a banhos; termas.}{bal.ne.á.rio}{0}
\verb{balneário}{}{}{}{}{}{Estância balneária hidromineral ou de águas medicinais.}{bal.ne.á.rio}{0}
\verb{balneoterapia}{}{Med.}{}{}{s.f.}{Tratamento por meio de banhos.}{bal.ne.o.te.ra.pi.a}{0}
\verb{balofo}{ô}{}{}{}{adj.}{Que tem volume desmedido em relação ao peso.}{ba.lo.fo}{0}
\verb{balofo}{ô}{}{}{}{}{Que tem consistência leve; fofo.}{ba.lo.fo}{0}
\verb{balofo}{ô}{Fig.}{}{}{}{Que só tem aparência e nenhum conteúdo; vão.}{ba.lo.fo}{0}
\verb{balofo}{ô}{Pop.}{}{}{}{Muito gordo, adiposo ou inchado.}{ba.lo.fo}{0}
\verb{balofo}{ô}{Pop.}{}{}{s.m.}{Indivíduo balofo; gordo.}{ba.lo.fo}{0}
\verb{balonismo}{}{}{}{}{s.m.}{Hábito de soltar balões ou esporte de neles voar.}{ba.lo.nis.mo}{0}
\verb{balouçar}{}{}{}{}{v.t.}{Fazer mover; balançar.}{ba.lou.çar}{\verboinum{3}}
\verb{balouço}{}{}{}{}{s.m.}{Balanço.}{ba.lou.ço}{0}
\verb{balsa}{}{}{}{}{s.f.}{Embarcação grande e plana, para travessia de carga e passageiros em rio ou braço de mar.}{bal.sa}{0}
\verb{balsâmico}{}{}{}{}{adj.}{Que tem as propriedades do bálsamo.}{bal.sâ.mi.co}{0}
\verb{balsâmico}{}{}{}{}{}{Que exala perfume; aromático, odorífico.}{bal.sâ.mi.co}{0}
\verb{bálsamo}{}{Bot.}{}{}{s.m.}{Líquido aromático e espesso que flui de muitas plantas, quer espontaneamente, quer por ferimento intencional.}{bál.sa.mo}{0}
\verb{bálsamo}{}{}{}{}{}{Aroma agradável e penetrante; perfume.}{bál.sa.mo}{0}
\verb{bálsamo}{}{Fig.}{}{}{}{Consolo, alívio.}{bál.sa.mo}{0}
\verb{baluarte}{}{}{}{}{s.m.}{Obra de fortificação; fortaleza.}{ba.lu.ar.te}{0}
\verb{baluarte}{}{}{}{}{}{Construção segura e alta, sustentada por muralhas.}{ba.lu.ar.te}{0}
\verb{baluarte}{}{Fig.}{}{}{}{Suporte, apoio, sustentáculo.}{ba.lu.ar.te}{0}
\verb{balzaquiana}{}{}{}{}{adj.}{Diz"-se de mulher de trinta anos (ou mais ou menos essa idade), como a personagem central do romance \textit{A mulher de trinta anos}, de Honoré de Balzac. }{bal.za.qui.a.na}{0}
\verb{balzaquiano}{}{}{}{}{adj.}{Relativo ao escritor francês Honoré de Balzac (1799--1850).}{bal.za.qui.a.no}{0}
\verb{balzaquiano}{}{Pop.}{}{}{}{Diz"-se de homem com trinta anos, ou mais ou menos essa idade.}{bal.za.qui.a.no}{0}
\verb{balzaquiano}{}{}{}{}{s.m.}{Grande admirador ou profundo conhecedor da obra de Balzac.}{bal.za.qui.a.no}{0}
\verb{bamba}{}{Pop.}{}{}{adj.}{Que é muito valente.}{bam.ba}{0}
\verb{bamba}{}{Fig.}{}{}{}{Diz"-se de autoridade em determinado assunto.}{bam.ba}{0}
\verb{bamba}{}{Pop.}{}{}{s.2g.}{Indivíduo valentão.}{bam.ba}{0}
\verb{bamba}{}{Fig.}{}{}{}{Indivíduo que domina algum assunto.}{bam.ba}{0}
\verb{bambear}{}{}{}{}{v.t.}{Tornar bambo; afrouxar.}{bam.be.ar}{0}
\verb{bambear}{}{}{}{}{v.i.}{Ficar bambo; hesitar, vacilar.}{bam.be.ar}{\verboinum{4}}
\verb{bambinela}{é}{}{}{}{s.f.}{Cortina, dividida em duas partes erguidas e presas dos lados, usada para sombrear e enfeitar janelas e portas.}{bam.bi.ne.la}{0}
\verb{bambo}{}{}{}{}{adj.}{Que não está firme; frouxo, relaxado.}{bam.bo}{0}
\verb{bambo}{}{Pop.}{}{}{}{Diz"-se daquele que hesita; indeciso, vacilante.}{bam.bo}{0}
\verb{bambochata}{}{}{}{}{s.f.}{Pintura que representa festas populares e cenas rústicas ou burlescas.}{bam.bo.cha.ta}{0}
\verb{bambochata}{}{}{}{}{}{Festa marcada por excessos; orgia, patuscada. (\textit{Era preciso que ele evitasse as bambochatas, tanto nos casarios abandonados, quanto nos palacetes da zona  sul. })}{bam.bo.cha.ta}{0}
\verb{bambolê}{}{}{}{}{s.m.}{Aro de plástico ou de metal, de cerca de um metro de diâmetro, usado como brinquedo por crianças e adolescentes, que o fazem girar em torno do corpo, especialmente na cintura.}{bam.bo.lê}{0}
\verb{bambolear}{}{}{}{}{v.t.}{Mover, mexendo os quadris; balancear, gingar, menear.}{bam.bo.le.ar}{0}
\verb{bambolear}{}{}{}{}{}{Fazer uma série de pequenas oscilações; tremular.}{bam.bo.le.ar}{\verboinum{4}}
\verb{bamboleio}{ê}{}{}{}{s.m.}{Ato ou efeito de bambolear; balanceio.}{bam.bo.lei.o}{0}
\verb{bambolim}{}{}{"-ins}{}{s.m.}{Faixa larga de tecido que se sobrepõe a cortinado de portas ou de janelas.}{bam.bo.lim}{0}
\verb{bambu}{}{Bot.}{}{}{s.m.}{Gramínea caracterizada pela altura excepcional do colmo, que alcança muitos metros, utilizada como ornamental ou para o aproveitamento do lenho e dos brotos comestíveis.}{bam.bu}{0}
\verb{bambual}{}{}{"-ais}{}{s.m.}{Coletivo de bambu.}{bam.bu.al}{0}
\verb{bambúrrio}{}{}{}{}{s.m.}{Fortuna inesperada.}{bam.búr.ri.o}{0}
\verb{bambúrrio}{}{}{}{}{}{Sorte no jogo.}{bam.búr.ri.o}{0}
\verb{bambuzal}{}{}{"-ais}{}{s.m.}{Bambual.}{bam.bu.zal}{0}
\verb{banal}{}{}{"-ais}{}{adj.2g.}{Que não tem originalidade; vulgar, trivial, corriqueiro.}{ba.nal}{0}
\verb{banalidade}{}{}{}{}{s.f.}{Caráter do que é banal; insignificância, trivialidade.}{ba.na.li.da.de}{0}
\verb{banalizar}{}{}{}{}{v.t.}{Tornar banal, comum, trivial; vulgarizar.}{ba.na.li.zar}{\verboinum{1}}
\verb{banana}{}{}{}{}{s.f.}{Fruto comestível da bananeira, de polpa carnosa e sem sementes.}{ba.na.na}{0}
\verb{banana}{}{Fig.}{}{}{s.2g.}{Pessoa sem energia e sem coragem para tomar decisões; frouxo.}{ba.na.na}{0}
\verb{bananada}{}{Cul.}{}{}{s.f.}{Doce feito da polpa da banana.}{ba.na.na.da}{0}
\verb{bananal}{}{}{"-ais}{}{s.m.}{Coletivo de banana.}{ba.na.nal}{0}
\verb{banana"-split}{}{}{bananas"-split}{}{s.f.}{Banana fatiada, servida com sorvete, nozes ou castanhas de caju picadas e calda.}{\textit{banana"-split}}{0}
\verb{bananeira}{ê}{Bot.}{}{}{s.f.}{Planta que produz a banana, originária do sudeste asiático, de folhas largas, caule cilíndrico, formado pelas bainhas que se enrolam umas nas outras.}{ba.na.nei.ra}{0}
\verb{bananeiro}{ê}{}{}{}{s.m.}{Indivíduo que cultiva ou comercializa bananas.}{ba.na.nei.ro}{0}
\verb{bananeiro}{ê}{}{}{}{}{Vendedor ambulante de bananas.}{ba.na.nei.ro}{0}
\verb{bananicultor}{ô}{}{}{}{s.m.}{Agricultor que cultiva a banana, que se dedica à bananicultura.}{ba.na.ni.cul.tor}{0}
\verb{bananicultura}{}{}{}{}{s.f.}{Plantação de bananeiras para comercialização ou industrialização.}{ba.na.ni.cul.tu.ra}{0}
\verb{banca}{}{}{}{}{s.f.}{Mesa de trabalho. (\textit{O marceneiro colocou tudo sobre a banca e começou a trabalhar.})}{ban.ca}{0}
\verb{banca}{}{}{}{}{}{Escritório de advocacia. (\textit{Ele tem uma banca de advocacia no centro da cidade.})}{ban.ca}{0}
\verb{banca}{}{}{}{}{}{Reserva de dinheiro para o pagamento de apostas em certos jogos de azar. (\textit{Em alguns jogos, é sempre a banca quem leva a melhor.})}{ban.ca}{0}
\verb{banca}{}{}{}{}{}{Conjunto de examinadores. (\textit{A banca do concurso foi muito rigorosa com os candidatos.})}{ban.ca}{0}
\verb{banca}{}{}{}{}{}{Local onde se vendem jornais e revistas. (\textit{Eu sempre compro meus jornais na banca da praça.})}{ban.ca}{0}
\verb{bancada}{}{}{}{}{s.f.}{Conjuntos de bancos dispostos em ordem. (\textit{Há uma bancada para idosos nas salas de espera para atendimento de saúde.})}{ban.ca.da}{0}
\verb{bancada}{}{}{}{}{}{Conjunto de vereadores do município ou de deputados ou senadores de um Estado ou de um partido. (\textit{A bancada de oposição do senado acabou cedendo às pressões do governo.})}{ban.ca.da}{0}
\verb{bancar}{}{}{}{}{v.t.}{Sustentar financeiramente algo ou alguém.}{ban.car}{0}
\verb{bancar}{}{}{}{}{}{Servir de banqueiro em jogo de azar.}{ban.car}{0}
\verb{bancar}{}{Pop.}{}{}{}{Fingir, simular.}{ban.car}{\verboinum{2}}
\verb{bancário}{}{}{}{}{adj.}{Relativo a bancos.}{ban.cá.rio}{0}
\verb{bancário}{}{}{}{}{s.m.}{Funcionário de banco, ou de casa bancária.}{ban.cá.rio}{0}
\verb{bancarrota}{ô}{}{}{}{s.f.}{Falência de negociante ou do Estado.}{ban.car.ro.ta}{0}
\verb{bancarrota}{ô}{}{}{}{}{Falência fraudulenta.}{ban.car.ro.ta}{0}
\verb{bancarrota}{ô}{Fig.}{}{}{}{Ruína, decadência.}{ban.car.ro.ta}{0}
\verb{banco}{}{}{}{}{s.m.}{Assento estreito e longo, próprio para várias pessoas. (\textit{Eles todos se sentaram num banco da praça.})}{ban.co}{0}
\verb{banco}{}{Fig.}{}{}{}{Elevação do fundo do mar ou de rios, que quase chega à superfície.}{ban.co}{0}
\verb{banco}{}{}{}{}{s.m.}{Empresa que guarda e empresta dinheiro. (\textit{Seu dinheiro está sempre guardado em bancos nacionais.})}{ban.co}{0}
\verb{banco}{}{}{}{}{}{Local em que se guardam coisas para se usar mais tarde. (\textit{O estoque do banco de sangue está diminuindo.})}{ban.co}{0}
\verb{banda}{}{}{}{}{s.f.}{Cada uma das partes opostas de alguma coisa; lado. (\textit{Eles pescaram na banda direta da lagoa.})}{ban.da}{0}
\verb{banda}{}{}{}{}{s.f.}{Conjunto de músicos com seus instrumentos. (\textit{A banda tocava na praça.})}{ban.da}{0}
\verb{banda}{}{}{}{}{}{Faixa larga e comprida. (\textit{Eles usam banda em torno da cintura, em vez de cintos.})}{ban.da}{0}
\verb{bandagem}{}{}{"-ens}{}{s.f.}{Faixa de gaze ou de outro tecido que se aplica sobre uma parte do corpo como curativo ou para sustentar um órgão.}{ban.da.gem}{0}
\verb{bandalha}{}{}{}{}{s.f.}{Ato irregular, fora da lei.}{ban.da.lha}{0}
\verb{bandalha}{}{}{}{}{}{Grupo de bandalhos; corja.}{ban.da.lha}{0}
\verb{bandalheira}{ê}{}{}{}{s.f.}{Ato ou modos de bandalho; ausência de dignidade; bandalha.}{ban.da.lhei.ra}{0}
\verb{bandalheira}{ê}{}{}{}{}{Negócio ilícito; roubalheira, imoralidade.}{ban.da.lhei.ra}{0}
\verb{bandalho}{}{}{}{}{s.m.}{Indivíduo maltrapilho; esfarrapado.}{ban.da.lho}{0}
\verb{bandalho}{}{}{}{}{}{Indivíduo sem dignidade nem brio; desprezível, patife.}{ban.da.lho}{0}
\verb{bandarilha}{}{}{}{}{s.f.}{Pequena haste de madeira, enfeitada com bandeirinhas, fitas e papéis de cores vistosas, que tem na extremidade uma ponta de ferro que o toureiro crava no cachaço do touro.}{ban.da.ri.lha}{0}
\verb{bandear}{}{}{}{}{v.t.}{Inclinar para o lado, para banda.}{ban.de.ar}{0}
\verb{bandear}{}{}{}{}{}{Formar bando; juntar em bando.}{ban.de.ar}{0}
\verb{bandear}{}{}{}{}{}{Hesitar entre duas bandas, partidos, facções etc.}{ban.de.ar}{0}
\verb{bandear}{}{}{}{}{v.i.}{Passar para o lado contrário, mudando de opinião ou de partido.}{ban.de.ar}{\verboinum{4}}
\verb{bandeira}{ê}{}{}{}{s.f.}{Pedaço de pano, de uma ou mais cores, que serve de símbolo para comunidades e países; estandarte, lábaro, pavilhão, pendão. (\textit{As cores das bandeiras dos times estão sempre presentes nas camisetas dos seus torcedores.})}{ban.dei.ra}{0}
\verb{bandeira}{ê}{}{}{}{}{Grupos que exploravam o interior do Brasil nos tempos coloniais para  procurar metais preciosos e aprisionar índios. (\textit{As bandeiras paulistas duravam muitos anos e percorriam distâncias enormes.})}{ban.dei.ra}{0}
\verb{bandeirada}{}{}{}{}{s.f.}{Quantia fixa previamente marcada pelo taxímetro dos automóveis de praça, e que constitui o preço mínimo que o passageiro deverá pagar.}{ban.dei.ra.da}{0}
\verb{bandeirante}{}{}{}{}{s.m.}{Indivíduo que, no Brasil colonial, tomou parte em uma bandeira, expedição.}{ban.dei.ran.te}{0}
\verb{bandeirante}{}{}{}{}{s.f.}{Menina ou mulher que pertence à Federação de Bandeirantes do Brasil, ou que se dedica ao bandeirantismo.}{ban.dei.ran.te}{0}
\verb{bandeirantismo}{}{}{}{}{s.m.}{Sistema baseado no método idealizado por Baden Powell, e que visa a desenvolver, entre meninas e moças, o espírito comunitário, a liberdade responsável, o esforço de progresso e as atitudes moldadas em valores éticos. }{ban.dei.ran.tis.mo}{0}
\verb{bandeirantismo}{}{}{}{}{}{Conjunto de fatos que dizem respeito às épocas das bandeiras.}{ban.dei.ran.tis.mo}{0}
\verb{bandeirinha}{}{Esport.}{}{}{s.m.}{Auxiliar do juiz, encarregado de acenar com uma pequena bandeira ao observar uma infração; juiz de linha.}{ban.dei.ri.nha}{0}
\verb{bandeirismo}{}{}{}{}{s.m.}{Conjunto de acontecimentos relacionados às bandeiras.}{ban.dei.ris.mo}{0}
\verb{bandeirismo}{}{}{}{}{}{Maneira de agir das bandeiras e dos bandeirantes.}{ban.dei.ris.mo}{0}
\verb{bandeirola}{ó}{}{}{}{s.f.}{Pequena bandeira, geralmente para sinalização.}{ban.dei.ro.la}{0}
\verb{bandeirola}{ó}{}{}{}{}{Bandeira de seda com franjas.}{ban.dei.ro.la}{0}
\verb{bandeja}{ê}{}{}{}{s.f.}{Tabuleiro de feitio variado para serviço de mesa.}{ban.de.ja}{0}
\verb{bandeja}{ê}{Esport.}{}{}{}{Jogada no basquete em que o atleta encesta a bola, conduzindo"-a por baixo, com uma das mãos.}{ban.de.ja}{0}
\verb{bandejão}{}{Pop.}{"-ões}{}{s.m.}{Restaurante que serve a refeição a preço popular em bandeja com divisões, geralmente em fábricas, universidades etc.}{ban.de.jão}{0}
\verb{bandejão}{}{}{"-ões}{}{}{Bandeja grande.}{ban.de.jão}{0}
\verb{bandido}{}{}{}{}{s.m.}{Indivíduo que pratica atividades criminosas; assaltante, malfeitor.}{ban.di.do}{0}
\verb{bandido}{}{Por ext.}{}{}{}{Indivíduo sem caráter, de maus sentimentos.}{ban.di.do}{0}
\verb{bandido}{}{}{}{}{adj.}{Que tem qualquer uma das características atribuíveis aos bandidos.}{ban.di.do}{0}
\verb{banditismo}{}{}{}{}{s.m.}{Ação, modo de vida de bandido.}{ban.di.tis.mo}{0}
\verb{banditismo}{}{}{}{}{}{Incidência de crimes num determinado lugar ou época; criminalidade.}{ban.di.tis.mo}{0}
\verb{bando}{}{}{}{}{s.m.}{Ajuntamento de pessoas ou animais.}{ban.do}{0}
\verb{bando}{}{}{}{}{}{Os integrantes de um partido ou facção.}{ban.do}{0}
\verb{bando}{}{}{}{}{}{Quadrilha de malfeitores.}{ban.do}{0}
\verb{bandó}{}{}{}{}{s.m.}{Cada uma das partes em que, num certo tipo de penteado, o cabelo é repartido ao meio e esticado para os lados da cabeça, cobrindo as orelhas, sendo preso atrás geralmente por meio de um coque.}{ban.dó}{0}
\verb{bandô}{}{}{}{}{s.m.}{Peça decorativa, rígida, feita de madeira ou substituto, pintada ou coberta de tecido, que arremata a parte superior de portas e janelas, em geral com o fim de ocultar o trilho das cortinas.}{ban.dô}{0}
\verb{bandoleira}{ê}{}{}{}{s.f.}{Correia, geralmente de lona ou de couro, passada a tiracolo, à qual se prende arma de fogo.}{ban.do.lei.ra}{0}
\verb{bandoleiro}{ê}{}{}{}{s.m.}{Indivíduo que pratica assaltos; bandido, cangaceiro.}{ban.do.lei.ro}{0}
\verb{bandoleiro}{ê}{}{}{}{}{Indivíduo que mente, trapaceia.}{ban.do.lei.ro}{0}
\verb{bandoleiro}{ê}{}{}{}{adj.}{Que é inconstante nos amores, nas amizades.}{ban.do.lei.ro}{0}
\verb{bandoleiro}{ê}{}{}{}{}{Que não tem ocupação; vadio, ocioso.}{ban.do.lei.ro}{0}
\verb{bandolim}{}{Mús.}{"-ins}{}{s.m.}{Instrumento de cordas surgido na Itália no séc. \textsc{xvi}, tocado com uma palheta.}{ban.do.lim}{0}
\verb{bandolinista}{}{}{}{}{s.2g.}{Indivíduo que toca bandolim.}{ban.do.li.nis.ta}{0}
\verb{bandulho}{}{Pop.}{}{}{s.m.}{Ventre acentuado; barriga, pança.}{ban.du.lho}{0}
\verb{bandurra}{}{Mús.}{}{}{s.f.}{Espécie de guitarra de braço curto, cordas de tripa e bordões.}{ban.dur.ra}{0}
\verb{bangalô}{}{}{}{}{s.m.}{Na Índia, casa baixa de um andar só, geralmente com grande varanda coberta.}{ban.ga.lô}{0}
\verb{bangalô}{}{}{}{}{}{Casa residencial cuja arquitetura lembra a do bangalô indiano.}{ban.ga.lô}{0}
\verb{banguê}{}{}{}{}{s.m.}{Padiola que era utilizada para transportar cadáveres de escravos.}{ban.guê}{0}
\verb{banguê}{}{}{}{}{}{Padiola de levar material de construção.}{ban.guê}{0}
\verb{banguê}{}{}{}{}{}{Padiola em que se conduz o bagaço verde da moenda para a bagaceira.}{ban.guê}{0}
\verb{bangue"-bangue}{}{}{bangue"-bangues}{}{s.m.}{Filme que retrata cenas da conquista do Oeste norte"-americano, em geral com muitos tiroteios e lutas.}{ban.gue"-ban.gue}{0}
\verb{banguela}{é}{}{}{}{adj.}{Diz"-se de pessoa cuja arcada dentária é falha na frente; desdentado.}{ban.gue.la}{0}
\verb{banguela}{é}{}{}{}{s.2g.}{Indivíduo que perdeu um ou vários dentes da frente.}{ban.gue.la}{0}
\verb{banha}{}{}{}{}{s.f.}{Gordura animal, especialmente a do porco.}{ba.nha}{0}
\verb{banha}{}{Por ext.}{}{}{}{Excesso de massa corporal dos seres humanos; gordura, obesidade. }{ba.nha}{0}
\verb{banhado}{}{Bras.}{}{}{s.m.}{Pântano coberto de vegetação; charco, brejo.}{ba.nha.do}{0}
\verb{banhar}{}{}{}{}{v.t.}{Dar banho em alguém. (\textit{A mãe banhou o filho.})}{ba.nhar}{0}
\verb{banhar}{}{}{}{}{}{Dar banho em um corpo ou em parte dele; lavar. (\textit{A mãe banhou somente os pés de seu filho.})}{ba.nhar}{0}
\verb{banhar}{}{}{}{}{}{Colocar alguma coisa dentro de um líquido. (\textit{Meu pai pediu para banhar suas medalhas.})}{ba.nhar}{0}
\verb{banhar}{}{}{}{}{}{Passar ou correr por algum lugar. (\textit{O rio Tietê banha muitas cidades do oeste paulista.})}{ba.nhar}{\verboinum{1}}
\verb{banheira}{ê}{}{}{}{s.f.}{Recipiente sanitário em forma de uma grande cuba para banho de imersão.}{ba.nhei.ra}{0}
\verb{banheira}{ê}{Pop.}{}{}{}{Automóvel de grande tamanho.}{ba.nhei.ra}{0}
\verb{banheiro}{ê}{Bras.}{}{}{s.m.}{Aposento da casa ou local público com vaso sanitário, lavatório, chuveiro; sanitário, toalete.}{ba.nhei.ro}{0}
\verb{banhista}{}{}{}{}{s.2g.}{Indivíduo que vai a praia, rio ou piscina para banhar"-se.}{ba.nhis.ta}{0}
\verb{banho}{}{}{}{}{s.m.}{Limpeza do corpo inteiro com água. (\textit{Vou dar um banho nos meus filhos.})}{ba.nho}{0}
\verb{banho}{}{}{}{}{}{Divertimento na água de mar, rio ou piscina. (\textit{Fomos à praia, tomar banhos de mar.})}{ba.nho}{0}
\verb{banho}{}{}{}{}{}{Ação de deixar os raios de luz bater no corpo ou em parte do corpo. (\textit{Algumas crianças, às vezes, têm de tomar banhos de luz logo após o nascimento.})}{ba.nho}{0}
\verb{banho}{}{Fig.}{}{}{}{Derrota por uma grande diferença de pontos. (\textit{Meu time deu um banho de gols no time adversário.})}{ba.nho}{0}
\verb{banho"-maria}{}{}{banhos"-marias \textit{ou} banhos"-maria}{}{s.m.}{Método de aquecer ou cozinhar indireta e lentamente, em que a vasilha com o alimento ou substância é colocada dentro de outro recipiente com água fervente.}{ba.nho"-ma.ri.a}{0}
\verb{banido}{}{}{}{}{adj.}{Que se baniu; exilado, desterrado, expatriado.}{ba.ni.do}{0}
\verb{banimento}{}{}{}{}{s.m.}{Ato ou efeito de banir.}{ba.ni.men.to}{0}
\verb{banir}{}{}{}{}{v.t.}{Expulsar de pátria, sociedade ou grupo; expatriar, desterrar, afastar.}{ba.nir}{0}
\verb{banir}{}{}{}{}{}{Eliminar, proscrever, abolir.}{ba.nir}{\verboinum{34}\verboirregular{\emph{def.} banimos, banis}}
\verb{banjo}{}{Mús.}{}{}{s.m.}{Instrumento de cordas, de origem estadunidense, com caixa de ressonância semelhante a um tambor e braço comprido e estreito.}{ban.jo}{0}
\verb{banjoísta}{}{Mús.}{}{}{s.2g.}{Indivíduo que toca banjo.}{ban.jo.ís.ta}{0}
\verb{banqueiro}{ê}{}{}{}{s.m.}{Dono ou alto executivo de uma instituição bancária.}{ban.quei.ro}{0}
\verb{banqueiro}{ê}{}{}{}{}{Indivíduo que realiza operações bancárias.}{ban.quei.ro}{0}
\verb{banqueiro}{ê}{Fig.}{}{}{}{Indivíduo muito rico.}{ban.quei.ro}{0}
\verb{banqueiro}{ê}{}{}{}{}{Indivíduo que, em um jogo, é responsável pela banca, distribuindo cartas e controlando apostas.}{ban.quei.ro}{0}
\verb{banqueiro}{ê}{}{}{}{}{Indivíduo que tem uma banca em jogo do bicho.}{ban.quei.ro}{0}
\verb{banqueta}{ê}{}{}{}{s.f.}{Pequeno banco sem encosto.}{ban.que.ta}{0}
\verb{banqueta}{ê}{}{}{}{}{Banco sem encosto, para uma pessoa.  }{ban.que.ta}{0}
\verb{banquete}{ê}{}{}{}{s.m.}{Refeição solene ou festiva, geralmente com muitos convidados.}{ban.que.te}{0}
\verb{banquete}{ê}{Por ext.}{}{}{}{Refeição suntuosa e aparatosa, geralmente de culinária refinada.}{ban.que.te}{0}
\verb{banquetear}{}{}{}{}{v.t.}{Oferecer um banquete.}{ban.que.te.ar}{0}
\verb{banquetear}{}{}{}{}{v.pron.}{Participar de banquete.}{ban.que.te.ar}{0}
\verb{banquetear}{}{Por ext.}{}{}{}{Comer muito bem ou gastando muito.}{ban.que.te.ar}{\verboinum{4}}
\verb{banto}{}{}{}{}{adj.}{Relativo aos bantos, grupo que engloba várias etnias africanas distribuídas por todo esse continente; muitos dos indivíduos trazidos para o Brasil como escravos pertenciam a esse grupo étnico.}{ban.to}{0}
\verb{banto}{}{Gram.}{}{}{}{Grupo linguístico que engloba várias línguas faladas na África.}{ban.to}{0}
\verb{banto}{}{}{}{}{}{Diz"-se das línguas pertencentes a esse grupo linguístico.}{ban.to}{0}
\verb{banto}{}{}{}{}{s.m.}{Indivíduo dos bantos.}{ban.to}{0}
\verb{banzar}{}{}{}{}{v.t.}{Surpreender, espantar, pasmar.}{ban.zar}{0}
\verb{banzar}{}{}{}{}{v.i.}{Ficar pensativo; meditar, matutar.}{ban.zar}{\verboinum{1}}
\verb{banzé}{}{Bras.}{}{}{s.m.}{Festa popular com música de viola e danças.}{ban.zé}{0}
\verb{banzé}{}{Por ext.}{}{}{}{Algazarra, bagunça, gritaria.}{ban.zé}{0}
\verb{banzeiro}{ê}{}{}{}{adj.}{Diz"-se de mar que se agita vagarosamente.}{ban.zei.ro}{0}
\verb{banzeiro}{ê}{Bras.}{}{}{}{Que sente banzo; triste, nostálgico, pensativo.}{ban.zei.ro}{0}
\verb{banzo}{}{}{}{}{s.m.}{Estado depressivo em que ficavam muitos escravos africanos, causado por saudade da terra natal, e que por vezes levava à loucura e à morte.}{ban.zo}{0}
\verb{banzo}{}{}{}{}{adj.}{Triste, melancólico, banzeiro.}{ban.zo}{0}
\verb{baobá}{}{Bot.}{}{}{s.m.}{Árvore de tronco espesso e madeira clara, rica em reservas de água, de cuja casca, com propriedades medicinais, se extrai uma fibra têxtil.}{ba.o.bá}{0}
\verb{baque}{}{}{}{}{s.m.}{Ruído de um corpo ao cair ou colidir.}{ba.que}{0}
\verb{baque}{}{}{}{}{}{Queda, tombo.}{ba.que}{0}
\verb{baque}{}{Fig.}{}{}{}{Revés súbito; dissabor, contratempo.}{ba.que}{0}
\verb{baque}{}{Fig.}{}{}{}{Desconfiança, suposição.}{ba.que}{0}
\verb{baquear}{}{}{}{}{v.i.}{Cair no chão; desabar, desmoronar.}{ba.que.ar}{0}
\verb{baquear}{}{Fig.}{}{}{}{Arruinar, falir.}{ba.que.ar}{0}
\verb{baquear}{}{}{}{}{v.t.}{Afetar, abater, enfraquecer, debilitar.}{ba.que.ar}{\verboinum{4}}
\verb{baquelita}{}{Quím.}{}{}{s.f.}{Resina sintética feita de fenol e formol, empregada como isolante térmico em alguns utensílios, ferramentas, panelas; baquelite.}{ba.que.li.ta}{0}
\verb{baquelite}{}{Quím.}{}{}{s.f.}{Baquelita.}{ba.que.li.te}{0}
\verb{baqueta}{ê}{Mús.}{}{}{s.f.}{Pequena vara de madeira usada para tocar instrumentos de percussão.}{ba.que.ta}{0}
\verb{baqueta}{ê}{}{}{}{}{Vareta do guarda"-sol; vaqueta.}{ba.que.ta}{0}
\verb{bar}{}{}{}{}{s.m.}{Estabelecimento com balcão e mesas onde se servem bebidas e pequenas refeições.}{bar}{0}
\verb{bar}{}{Fís.}{}{}{s.m.}{Unidade de medida de pressão equivalente a 10\textsuperscript{5} pascals.}{bar}{0}
\verb{bar}{}{}{}{}{}{Ambiente de uma residência, hotel, teatro etc., no qual se servem bebidas e pequenas refeições.}{bar}{0}
\verb{bar}{}{}{}{}{}{Móvel caseiro para guardar bebidas.}{bar}{0}
\verb{bar}{}{}{}{}{s.m.}{Unidade indiana de medida de peso que equivale a algo entre 141 e 330 kg, de acordo com a região.}{bar}{0}
\verb{baraço}{}{}{}{}{s.m.}{Corda ou laço usado em enforcamentos.}{ba.ra.ço}{0}
\verb{barafunda}{}{}{}{}{s.f.}{Tumulto, confusão, desordem.}{ba.ra.fun.da}{0}
\verb{barafunda}{}{}{}{}{}{Mistura desordenada de diversas coisas; bagunça, mixórdia.}{ba.ra.fun.da}{0}
\verb{barafustar}{}{}{}{}{v.t.}{Entrar violenta ou precipitadamente.}{ba.ra.fus.tar}{0}
\verb{barafustar}{}{}{}{}{v.i.}{Espernear, debater"-se.}{ba.ra.fus.tar}{\verboinum{1}}
\verb{baralhada}{}{}{}{}{s.f.}{Desordem, bagunça, barafunda.}{ba.ra.lha.da}{0}
\verb{baralhar}{}{}{}{}{v.t.}{Misturar as cartas do baralho; embaralhar.}{ba.ra.lhar}{0}
\verb{baralhar}{}{Por ext.}{}{}{}{Desarrumar, bagunçar.}{ba.ra.lhar}{0}
\verb{baralhar}{}{Fig.}{}{}{v.pron.}{Confundir"-se, equivocar"-se.}{ba.ra.lhar}{\verboinum{1}}
\verb{baralho}{}{}{}{}{s.m.}{Conjunto de cartas com diferentes figuras para jogar.}{ba.ra.lho}{0}
\verb{barão}{}{}{"-ões}{}{s.m.}{Título nobiliárquico imediatamente inferior ao de visconde, sendo o menos graduado nessa hierarquia.}{ba.rão}{0}
\verb{barão}{}{Por ext.}{"-ões}{}{}{Homem poderoso e notável, geralmente por sua riqueza.}{ba.rão}{0}
\verb{barão}{}{Bot.}{"-ões}{}{}{Variedade de algodoeiro.}{ba.rão}{0}
\verb{barão}{}{Desus.}{"-ões}{}{}{Nota de mil cruzeiros.}{ba.rão}{0}
\verb{barata}{}{Zool.}{}{}{s.f.}{Inseto onívoro, de corpo alongado e achatado, dotado de um par de antenas longas, que põe ovos e tem hábitos noturnos.}{ba.ra.ta}{0}
\verb{barateamento}{}{}{}{}{s.m.}{Ato ou efeito de baratear.}{ba.ra.te.a.men.to}{0}
\verb{baratear}{}{}{}{}{v.t.}{Baixar o preço; abater, desvalorizar.}{ba.ra.te.ar}{0}
\verb{baratear}{}{}{}{}{}{Dar pouco valor; menosprezar.}{ba.ra.te.ar}{0}
\verb{baratear}{}{}{}{}{v.i.}{Perder valor; desvalorizar.}{ba.ra.te.ar}{\verboinum{4}}
\verb{barateiro}{ê}{}{}{}{adj.}{Que vende a preços baixos.}{ba.ra.tei.ro}{0}
\verb{barateiro}{ê}{}{}{}{s.m.}{Nas casas de jogo, o indivíduo que cobra barato.}{ba.ra.tei.ro}{0}
\verb{barateza}{ê}{}{}{}{s.f.}{Qualidade de barato.}{ba.ra.te.za}{0}
\verb{barateza}{ê}{}{}{}{}{Baixeza de preços.}{ba.ra.te.za}{0}
\verb{baratinar}{}{Pop.}{}{}{v.i.}{Ficar ou sentir"-se alegre, agitado ou eufórico sob o efeito de álcool ou drogas.}{ba.ra.ti.nar}{0}
\verb{baratinar}{}{Por ext.}{}{}{}{Perturbar mentalmente; transtornar.}{ba.ra.ti.nar}{\verboinum{1}}
\verb{baratinha}{}{Zool.}{}{}{s.f.}{Espécie de crustáceo que, ao ser tocado, enrola o próprio corpo; tatuzinho.}{ba.ra.ti.nha}{0}
\verb{baratinha}{}{Pop.}{}{}{}{Antigo carro de corrida.  }{ba.ra.ti.nha}{0}
\verb{barato}{}{}{}{}{adj.}{Que tem preço baixo. (\textit{Nessa loja os artigos são muito baratos.})}{ba.ra.to}{0}
\verb{barato}{}{}{}{}{}{Por preço baixo. (\textit{Eles vendem barato para algumas pessoas.})}{ba.ra.to}{0}
\verb{barato}{}{Pop.}{}{}{s.m.}{Coisa muito boa. (\textit{O filme foi um barato.})}{ba.ra.to}{0}
\verb{baraúna}{}{Bot.}{}{}{s.f.}{Árvore de floresta pluvial com flores amarelas e madeira escura e muito dura.}{ba.ra.ú.na}{0}
\verb{barba}{}{}{}{}{s.f.}{Pelos que crescem no rosto do homem.}{bar.ba}{0}
\verb{barba}{}{}{}{}{}{Pelos do focinho ou bico de certos animais.}{bar.ba}{0}
\verb{barba"-azul}{}{}{barbas"-azuis}{}{s.m.}{Homem que ficou viúvo várias vezes.}{bar.ba"-a.zul}{0}
\verb{barba"-azul}{}{}{barbas"-azuis}{}{}{Homem que tem ou conquista várias mulheres; mulherengo, conquistador.}{bar.ba"-a.zul}{0}
\verb{barbaças}{}{}{}{}{s.m.}{Homem de barbas grandes.}{bar.ba.ças}{0}
\verb{barbada}{}{}{}{}{s.f.}{O beiço inferior do cavalo.}{bar.ba.da}{0}
\verb{barbado}{}{}{}{}{adj.}{Que usa barba ou está com ela crescida por não tê"-la feito.}{bar.ba.do}{0}
\verb{barbado}{}{Pop.}{}{}{s.m.}{Homem adulto; marmanjo.}{bar.ba.do}{0}
\verb{barbante}{}{}{}{}{s.m.}{Cordel fino usado para amarrar.}{bar.ban.te}{0}
\verb{barbaria}{}{}{}{}{s.f.}{Ato ou comportamento próprio de bárbaros; barbaridade, crueldade, selvageria.}{bar.ba.ri.a}{0}
\verb{barbaria}{}{}{}{}{}{Grande conjunto de bárbaros.}{bar.ba.ri.a}{0}
\verb{barbaridade}{}{}{}{}{s.f.}{Ato ou comportamento próprio de bárbaros; crueldade, selvageria, barbaria.}{bar.ba.ri.da.de}{0}
\verb{barbaridade}{}{}{}{}{}{Expressão grosseira.}{bar.ba.ri.da.de}{0}
\verb{barbaridade}{}{}{}{}{interj.}{Expressão que denota admiração, espanto ou estupefação, ideia de quantidade grande.}{bar.ba.ri.da.de}{0}
\verb{barbaridade}{}{}{}{}{}{Proposição absurda.}{bar.ba.ri.da.de}{0}
\verb{barbárie}{}{}{}{}{s.f.}{Qualidade, estado ou condição de bárbaro.}{bar.bá.rie}{0}
\verb{barbarismo}{}{}{}{}{s.m.}{Barbárie.}{bar.ba.ris.mo}{0}
\verb{barbarismo}{}{Gram.}{}{}{}{Nas gramáticas normativas, o uso de formas ou construções que são consideradas desvios condenáveis.}{bar.ba.ris.mo}{0}
\verb{barbarizar}{}{}{}{}{v.t.}{Tornar bárbaro; embrutecer.}{bar.ba.ri.zar}{0}
\verb{barbarizar}{}{Gram.}{}{}{v.i.}{Cometer barbarismos.}{bar.ba.ri.zar}{0}
\verb{barbarizar}{}{Pop.}{}{}{}{Mostrar ou obter excelente desempenho.}{bar.ba.ri.zar}{\verboinum{1}}
\verb{bárbaro}{}{}{}{}{adj.}{Diz"-se dos povos que conquistaram os territórios romanos no início da Idade Média. (\textit{Os povos bárbaros eram, principalmente, de origem germânica.})}{bár.ba.ro}{0}
\verb{bárbaro}{}{}{}{}{}{Aquilo que surpreende pelo seu ineditismo ou pelas características fora do comum. (\textit{Aquela foi uma festa bárbara.})}{bár.ba.ro}{0}
\verb{barbatana}{}{Zool.}{}{}{s.f.}{Membrana móvel que os peixes possuem e que servem para locomoverem"-se na água.}{bar.ba.ta.na}{0}
\verb{barbatana}{}{}{}{}{}{Vareta flexível que serve de armação em colarinhos e certas partes do vestuário.}{bar.ba.ta.na}{0}
\verb{barbatimão}{}{Bot.}{"-ões}{}{s.m.}{Árvore leguminosa de cujo fruto e casca se extrai o tanino, de madeira resistente à umidade e útil também para extração de tinta vermelha e por suas propriedades medicinais.}{bar.ba.ti.mão}{0}
\verb{barbeador}{ô}{}{}{}{s.m.}{Aparelho para barbear.}{bar.be.a.dor}{0}
\verb{barbear}{}{}{}{}{v.t.}{Cortar os pelos da barba; fazer a barba. (\textit{Naquela barbearia, são as moças que barbeiam os clientes.})}{bar.be.ar}{\verboinum{4}}
\verb{barbearia}{}{}{}{}{s.f.}{Salão ou ofício de barbeiro.}{bar.be.a.ri.a}{0}
\verb{barbeirada}{}{}{}{}{s.f.}{Barbeiragem.}{bar.bei.ra.da}{0}
\verb{barbeiragem}{}{Bras.}{"-ens}{}{s.f.}{Imperícia ou incompetência de qualquer profissional ou agente, especialmente a dos motoristas; barbeirada.}{bar.bei.ra.gem}{0}
\verb{barbeiro}{ê}{}{}{}{s.m.}{Indivíduo cuja profissão é raspar ou aparar barbas e cortar cabelos.}{bar.bei.ro}{0}
\verb{barbeiro}{ê}{Bras.}{}{}{}{Inseto hematófago, transmissor da doença de Chagas.}{bar.bei.ro}{0}
\verb{barbeiro}{ê}{Pop.}{}{}{}{Motorista descuidado, imperito ou imprudente.}{bar.bei.ro}{0}
\verb{barbela}{é}{}{}{}{s.f.}{Pele que pende do pescoço do boi e de outros ruminantes.}{bar.be.la}{0}
\verb{barbela}{é}{Por ext.}{}{}{}{Dobra adiposa sob o queixo; papada.}{bar.be.la}{0}
\verb{barbicacho}{}{}{}{}{s.m.}{Cabresto ou focinheira de corda para cavalgaduras.}{bar.bi.ca.cho}{0}
\verb{barbicacho}{}{Bras.}{}{}{}{Corda ou tira de couro que prende o chapéu à cabeça, atando"-a ao queixo.}{bar.bi.ca.cho}{0}
\verb{barbicha}{}{}{}{}{s.f.}{Barba rala, curta e pontuda.}{bar.bi.cha}{0}
\verb{barbicha}{}{}{}{}{}{Barba comprida do bode.}{bar.bi.cha}{0}
\verb{barbilhão}{}{Zool.}{"-ões}{}{s.m.}{Filamento que sobressai dos cantos da boca de certos peixes.}{bar.bi.lhão}{0}
\verb{barbilhão}{}{Zool.}{"-ões}{}{}{Apêndice carnoso pendente embaixo do bico de certas aves.}{bar.bi.lhão}{0}
\verb{barbitúrico}{}{Farm.}{}{}{s.m.}{Nome comum aos medicamentos de ação sedativa ou hipnótica.}{bar.bi.tú.ri.co}{0}
\verb{barbudo}{}{}{}{}{adj.}{Que tem a barba grande ou longa.}{bar.bu.do}{0}
\verb{barca}{}{}{}{}{s.f.}{Embarcação larga, de fundo chato, própria para transporte de passageiros e cargas a pequenas distâncias. (\textit{As barcas subiam o rio até o primeiro porto.})}{bar.ca}{0}
\verb{barcaça}{}{}{}{}{s.f.}{Barca muito grande. (\textit{Os animais foram levados em barcaças através do canal.})}{bar.ca.ça}{0}
\verb{barcarola}{ó}{}{}{}{s.f.}{Canção dos gondoleiros venezianos.}{bar.ca.ro.la}{0}
\verb{barcarola}{ó}{Mús.}{}{}{}{Peça vocal ou instrumental, em andamento moderado, cujo ritmo é inspirado no balanço de uma barca sobre as águas.}{bar.ca.ro.la}{0}
\verb{barcarola}{ó}{Liter.}{}{}{}{Cantiga medieval que se desenvolve em cenários marítimos.}{bar.ca.ro.la}{0}
\verb{barco}{}{}{}{}{s.m.}{Embarcação pequena, sem coberta. (\textit{Ele comprou um barco para pescar no rio.})}{bar.co}{0}
\verb{barco}{}{}{}{}{}{Qualquer embarcação. (\textit{Navios cargueiros  são barcos que transportam muitas coisas para muitos países.})}{bar.co}{0}
\verb{bardo}{}{}{}{}{s.m.}{Indivíduo que compunha e recitava poemas épicos, acompanhado de lira ou harpa.}{bar.do}{0}
\verb{bardo}{}{Por ext.}{}{}{}{Poeta, trovador, vate.}{bar.do}{0}
\verb{barganha}{}{Pop.}{}{}{s.f.}{Ato ou efeito de barganhar; troca.}{bar.ga.nha}{0}
\verb{barganha}{}{}{}{}{}{Transação ou negociação fraudulenta; trapaça.}{bar.ga.nha}{0}
\verb{barganhar}{}{}{}{}{v.t.}{Negociar uma coisa por outra, em operação que não envolva dinheiro; trocar. (\textit{É mais fácil barganhar coisas usadas do que tentar vendê"-las.})}{bar.ga.nhar}{\verboinum{1}}
\verb{bário}{}{Quím.}{}{}{s.m.}{Elemento químico metálico, tóxico, com a aparência prateada, maleável, do grupo dos alcalino"-terrosos; usado, sob a forma pura, para manter o vácuo em válvulas eletrônicas e, sob a forma de compostos, tem diversas aplicações. \elemento{56}{137.327}{Ba}.}{bá.rio}{0}
\verb{barisfera}{é}{Geol.}{}{}{s.f.}{Núcleo central da Terra, composto principalmente de níquel e ferro; centrosfera.}{ba.ris.fe.ra}{0}
\verb{barítono}{}{Mús.}{}{}{adj.}{Diz"-se da voz ou registro intermediário entre o baixo e o tenor.}{ba.rí.to.no}{0}
\verb{barítono}{}{}{}{}{s.m.}{Indivíduo que tem essa voz ou toca um instrumento com esse registro.}{ba.rí.to.no}{0}
\verb{barlavento}{}{}{}{}{s.m.}{Direção de onde sopra o vento. }{bar.la.ven.to}{0}
\verb{barman}{}{}{}{\textit{barmaid}}{s.m.}{Homem encarregado de preparar ou servir as bebidas em um bar.}{\textit{barman}}{0}
\verb{barnabé}{}{Pop.}{}{}{s.m.}{Funcionário público, especialmente o de categoria modesta.}{bar.na.bé}{0}
\verb{barnabita}{}{Relig.}{}{}{adj.}{Relativo ou pertencente aos religiosos da Ordem dos Clérigos Regulares de São Paulo.}{bar.na.bi.ta}{0}
\verb{barômetro}{}{}{}{}{s.m.}{Instrumento que mede a pressão atmosférica.}{ba.rô.me.tro}{0}
\verb{baronato}{}{}{}{}{s.m.}{Título ou dignidade de barão.}{ba.ro.na.to}{0}
\verb{baronesa}{ê}{}{}{}{s.f.}{Mulher com título ou dignidade de barão ou casada com aquele que detém essa dignidade.}{ba.ro.ne.sa}{0}
\verb{baronete}{ê}{}{}{}{s.m.}{Na Inglaterra, título de nobreza hereditário, intermediário entre cavaleiro e barão.}{ba.ro.ne.te}{0}
\verb{barqueiro}{ê}{}{}{}{s.m.}{Indivíduo que dirige um barco.}{bar.quei.ro}{0}
\verb{barqueta}{ê}{}{}{}{s.f.}{Barca pequena; barquinha.}{bar.que.ta}{0}
\verb{barquinha}{}{}{}{}{}{Cesto de um balão (aeróstato), onde viajam os tripulantes.}{bar.qui.nha}{0}
\verb{barquinha}{}{}{}{}{s.f.}{Barqueta.}{bar.qui.nha}{0}
\verb{barquinha}{}{Fig.}{}{}{}{Pequeno caixão usado para enterrar crianças, geralmente recém"-nascidas.}{bar.qui.nha}{0}
\verb{barra}{}{}{}{}{}{Extremidade inferior de roupa. (\textit{Ele tinha a barra da calça sempre suja.})}{bar.ra}{0}
\verb{barra}{}{}{}{}{s.f.}{Bloco retangular ou quadrado de um produto. (\textit{Comprei algumas barras de sabão.})}{bar.ra}{0}
\verb{barra}{}{}{}{}{}{Entrada de porto. (\textit{O navio entrou na barra para atracar.})}{bar.ra}{0}
\verb{barra}{}{}{}{}{}{Cano sustentado na horizontal por uma trave em cada ponta, próprio para ginástica. (\textit{Ele fazia exercícios na barra com perfeição.})}{bar.ra}{0}
\verb{barra}{}{}{}{}{}{Pequeno traço vertical ou inclinado usado na escrita. (\textit{Usa"-se a barra para muitos fins na computação.})}{bar.ra}{0}
\verb{barraca}{}{}{}{}{s.f.}{Tenda para acampar.}{bar.ra.ca}{0}
\verb{barraca}{}{}{}{}{}{Casa humilde, pobre.}{bar.ra.ca}{0}
\verb{barraca}{}{}{}{}{}{Tenda ou abrigo, de montagem prática, usado em feiras.}{bar.ra.ca}{0}
\verb{barracão}{}{}{"-ões}{}{s.m.}{Barraco grande.}{bar.ra.cão}{0}
\verb{barracão}{}{}{"-ões}{}{}{Abrigo ou construção provisório usado como depósito.}{bar.ra.cão}{0}
\verb{barracão}{}{}{"-ões}{}{}{Estabelecimento comercial em lugares pouco habitados.}{bar.ra.cão}{0}
\verb{barraco}{}{}{}{}{s.m.}{Construção improvisada, precária, erguida  com materiais de origem diversa, como palha, zinco ou telha; barracão. (\textit{Eles construíram um pequeno barraco nos fundos das casas para guardar as ferramentas.})}{bar.ra.co}{0}
\verb{barracuda}{}{Zool.}{}{}{s.f.}{Peixe de até 3 m, de corpo alongado, dorso cinza com listras escuras, de ampla distribuição no Atlântico ocidental, cuja carne é considerada tóxica.}{bar.ra.cu.da}{0}
\verb{barrado}{}{}{}{}{adj.}{Coberto ou revestido de barro.}{bar.ra.do}{0}
\verb{barrado}{}{}{}{}{}{Que tem barras ou listras.}{bar.ra.do}{0}
\verb{barragem}{}{}{"-ens}{}{s.f.}{Construção que fecha um vale onde passa um rio e impede a água de ir embora; represa. (\textit{Eles moram muito perto da barragem.})}{bar.ra.gem}{0}
\verb{barra"-limpa}{}{Pop.}{barra"-limpas}{}{adj.}{Diz"-se do que não cria empecilhos, problemas; leal, confiável, simpático, boa"-praça.}{bar.ra"-lim.pa}{0}
\verb{barranca}{}{}{}{}{s.f.}{Barranco.}{bar.ran.ca}{0}
\verb{barranco}{}{}{}{}{s.m.}{Escavação de terreno, provocada pela erosão ou pela ação do homem.}{bar.ran.co}{0}
\verb{barranco}{}{}{}{}{}{Ribanceira de um rio ou margem de estrada alta ou íngreme.}{bar.ran.co}{0}
\verb{barranqueira}{ê}{Bras.}{}{}{s.f.}{Sucessão de barrancos.}{bar.ran.quei.ra}{0}
\verb{barranqueira}{ê}{}{}{}{}{Margem elevada de rio; ribanceira.}{bar.ran.quei.ra}{0}
\verb{barranqueira}{ê}{}{}{}{}{Despenhadeiro, abismo, precipício.}{bar.ran.quei.ra}{0}
\verb{barranqueiro}{ê}{Bras.}{}{}{s.m.}{Indivíduo que habita próximo da margem de um rio; ribeirinho.}{bar.ran.quei.ro}{0}
\verb{barra"-pesada}{}{Pop.}{barras"-pesadas}{}{adj.2g.}{Diz"-se do que infunde temor, desconfiança, medo etc., por suas atitudes violentas ou escandalosas.}{bar.ra"-pe.sa.da}{0}
\verb{barra"-pesada}{}{}{barras"-pesadas}{}{}{Que é complicado, de difícil solução; barra.}{bar.ra"-pe.sa.da}{0}
\verb{barraqueiro}{ê}{}{}{}{s.m.}{Indivíduo que tem ou trabalha em barraca, especialmente feirante.}{bar.ra.quei.ro}{0}
\verb{barrar}{}{}{}{}{v.t.}{Não deixar alguma coisa acontecer; impedir. (\textit{Ele barrou o processo.})}{bar.rar}{0}
\verb{barrar}{}{}{}{}{}{Não deixar alguém entrar. (\textit{Ele barrou  as pessoas que não eram convidadas.})}{bar.rar}{\verboinum{1}}
\verb{barrear}{}{}{}{}{v.t.}{Cobrir de barro. (\textit{Ele barreou todas as paredes externas.})}{bar.re.ar}{\verboinum{4}}
\verb{barregã}{}{Desus.}{}{}{s.f.}{Mulher que vive em união estável com um homem, sem estar religiosa ou juridicamente casada com este; concubina.}{bar.re.gã}{0}
\verb{barregão}{}{Desus.}{"-ões}{barregã}{s.m.}{Homem amancebado que vive em união estável com uma mulher, sem estar religiosa ou juridicamente casado com esta; concubino.}{bar.re.gão}{0}
\verb{barreira}{ê}{}{}{}{s.f.}{O que se coloca em determinado lugar para impedir a passagem. (\textit{A polícia fez uma barreira logo no início da avenida.})}{bar.rei.ra}{0}
\verb{barreira}{ê}{}{}{}{}{Lugar em que se controlam veículos para controle fiscal. (\textit{Logo adiante há uma barreira da Secretaria da Fazenda.})}{bar.rei.ra}{0}
\verb{barreira}{ê}{}{}{}{s.f.}{Barreiro.}{bar.rei.ra}{0}
\verb{barreiro}{ê}{}{}{}{s.m.}{Lugar de onde se tira barro para fazer tijolos e telhas.   }{bar.rei.ro}{0}
\verb{barrela}{é}{}{}{}{s.f.}{Água fervida com cinzas vegetais e soda, usada para branquear roupa; lixívia.  }{bar.re.la}{0}
\verb{barrento}{}{}{}{}{adj.}{Que tem barro ou é da cor deste.}{bar.ren.to}{0}
\verb{barretada}{}{}{}{}{s.f.}{Ato ou efeito de descobrir a cabeça, tirando chapéu ou barrete, para saudar ou fazer cortesia.}{bar.re.ta.da}{0}
\verb{barretada}{}{}{}{}{}{Reverência exagerada; rapapé, salamaleque.}{bar.re.ta.da}{0}
\verb{barrete}{ê}{}{}{}{s.m.}{Cobertura flexível, de pano ou de malha, que se ajusta à cabeça.}{bar.re.te}{0}
\verb{barrete}{ê}{Zool.}{}{}{}{Segunda cavidade do estômago dos ruminantes.}{bar.re.te}{0}
\verb{barrica}{}{}{}{}{s.f.}{Vasilha de tanoaria em forma de pipa, para armazenar mercadorias.}{bar.ri.ca}{0}
\verb{barrica}{}{Fig.}{}{}{}{Indivíduo baixo e obeso.}{bar.ri.ca}{0}
\verb{barricada}{}{}{}{}{s.f.}{Tipo de trincheira improvisada com barricas, carros, sacos de areia, pedras etc., com a qual se protege a entrada de uma rua ou passagem.  }{bar.ri.ca.da}{0}
\verb{barricada}{}{}{}{}{}{Trincheira, barreira.}{bar.ri.ca.da}{0}
\verb{barrido}{}{}{}{}{}{Var. de \textit{barrito}.}{bar.ri.do}{0}
\verb{barriga}{}{}{}{}{s.f.}{Parte do corpo humano e de outros animais onde estão localizados o estômago, os intestinos e outros órgãos; ventre; pança.  }{bar.ri.ga}{0}
\verb{barriga}{}{}{}{}{}{Parte arredondada e saltada de alguma coisa; bojo. }{bar.ri.ga}{0}
\verb{barrigada}{}{}{}{}{s.f.}{Golpe ou pancada na barriga ou com a barriga.}{bar.ri.ga.da}{0}
\verb{barrigada}{}{}{}{}{}{Vísceras de reses abatidas.}{bar.ri.ga.da}{0}
\verb{barriga"-d'água}{}{Pop.}{barrigas"-d'água}{}{s.f.}{Esquistossomose.   }{bar.ri.ga"-d'á.gua}{0}
\verb{barriga"-verde}{ê}{}{barrigas"-verdes ⟨ê⟩}{}{adj.2g.}{Relativo a Santa Catarina; catarinense.}{bar.ri.ga"-ver.de}{0}
\verb{barriga"-verde}{ê}{}{barrigas"-verdes ⟨ê⟩}{}{s.2g.}{Indivíduo natural ou habitante desse estado.}{bar.ri.ga"-ver.de}{0}
\verb{barriguda}{}{}{}{}{s.f.}{Diz"-se da mulher grávida, prenhe.}{bar.ri.gu.da}{0}
\verb{barrigudo}{}{}{}{}{s.m.}{Que tem a barriga volumosa ou protuberante; ventrudo, pançudo.}{bar.ri.gu.do}{0}
\verb{barrigueira}{ê}{}{}{}{s.f.}{Peça de arreio que prende a sela à montaria; cilha.}{bar.ri.guei.ra}{0}
\verb{barril}{}{}{"-is}{}{s.m.}{Recipiente de madeira, abaulado, formado de aduelas, e que serve para transportar ou conservar líquidos.}{bar.ril}{0}
\verb{barril}{}{}{"-is}{}{}{Medida de capacidade de litros, usada na indústria petrolífera, correspondente a 158,98 litros.}{bar.ril}{0}
\verb{barrilete}{ê}{}{}{}{s.m.}{Pequeno barril.}{bar.ri.le.te}{0}
\verb{barrilete}{ê}{}{}{}{}{Peça de ferro usada pelos marceneiros e entalhadores para firmar no banco a madeira que será trabalhada.}{bar.ri.le.te}{0}
\verb{barrir}{}{}{}{}{v.i.}{Emitir barritos.}{bar.rir}{\verboinum{18}}
\verb{barrista}{}{}{}{}{s.2g.}{Ginasta que pratica exercícios em barras fixas.}{bar.ris.ta}{0}
\verb{barrista}{}{}{}{}{s.2g.}{Escultor que modela em barro.}{bar.ris.ta}{0}
\verb{barrito}{}{}{}{}{s.m.}{Grito do elefante e de outros animais grandes.}{bar.ri.to}{0}
\verb{barro}{}{}{}{}{s.m.}{Argila vermelha, branca ou amarela que é utilizada na fabricação de tijolos e telhas e em esculturas.}{bar.ro}{0}
\verb{barro}{}{}{}{}{}{Mistura de argila e água usada para assentar tijolos em construções provisórias.}{bar.ro}{0}
\verb{barro}{}{}{}{}{}{Objeto sem valor, falsificado.}{bar.ro}{0}
\verb{barroca}{ó}{}{}{}{s.f.}{Terreno cheio de barro ou piçarra.}{bar.ro.ca}{0}
\verb{barroca}{ó}{}{}{}{}{Escavação formada por enxurrada; cova.}{bar.ro.ca}{0}
\verb{barroca}{ó}{}{}{}{}{Despenhadeiro, abismo, precipício.}{bar.ro.ca}{0}
\verb{barroco}{ô}{}{}{}{s.m.}{Estilo literário e artístico ligado ao movimento da Contrarreforma, que predominou na Europa durante os séculos \textsc{xvii} e \textsc{xviii} e caracterizou"-se pelas formas rebuscadas e pomposas e pelos elementos contrastantes.}{bar.ro.co}{0}
\verb{barroco}{ô}{}{}{}{adj.}{Relativo a esse estilo.}{bar.ro.co}{0}
\verb{barroco}{ô}{Fig.}{}{}{}{Exuberante, rebuscado, afetado, exagerado.}{bar.ro.co}{0}
\verb{barroso}{ô}{}{"-osos ⟨ó⟩}{"-osa ⟨ó⟩}{adj.}{Da natureza do barro.}{bar.ro.so}{0}
\verb{barroso}{ô}{}{"-osos ⟨ó⟩}{"-osa ⟨ó⟩}{}{Cheio de barro; barrento.}{bar.ro.so}{0}
\verb{barrote}{ó}{}{}{}{s.m.}{Peça grossa de madeira com a qual se fixam assoalhos, forros etc.}{bar.ro.te}{0}
\verb{barulhada}{}{}{}{}{s.f.}{Grande barulho; zoeira, barulheira.}{ba.ru.lha.da}{0}
\verb{barulheira}{ê}{}{}{}{s.f.}{Grande barulho; gritaria, algazarra, barulhada.}{ba.ru.lhei.ra}{0}
\verb{barulhento}{}{}{}{}{adj.}{Que faz muito barulho; ruidoso, turbulento.}{ba.ru.lhen.to}{0}
\verb{barulho}{}{}{}{}{s.m.}{Rumor forte, estrondo, ruído.}{ba.ru.lho}{0}
\verb{barulho}{}{}{}{}{}{Desordem, alvoroço, tumulto.}{ba.ru.lho}{0}
\verb{basal}{}{}{"-ais}{}{adj.2g.}{Relativo a base, apoio, princípio.}{ba.sal}{0}
\verb{basal}{}{Med.}{"-ais}{}{}{Que indica o nível mínimo de atividade de um organismo em completo repouso.}{ba.sal}{0}
\verb{basalto}{}{Geol.}{}{}{s.m.}{Rocha vulcânica, geralmente preta (ocorre também em cinza ou castanho), usada na pavimentação de ruas e estradas, e também na escultura de obras de arte e estátuas.}{ba.sal.to}{0}
\verb{basbaque}{}{}{}{}{adj.}{Que se espanta com tudo; tolo, palerma, simplório.}{bas.ba.que}{0}
\verb{basbaquice}{}{}{}{}{s.f.}{Ato, dito ou comportamento de basbaque; tolice, palermice.}{bas.ba.qui.ce}{0}
\verb{basco}{}{}{}{}{adj.}{Relativo ao País Basco, região dos Pireneus, no norte da Espanha e no sul da França.}{bas.co}{0}
\verb{basco}{}{}{}{}{s.m.}{Indivíduo nativo ou habitante dessa região.}{bas.co}{0}
\verb{báscula}{}{}{}{}{s.f.}{Balança decimal com base horizontal para objetos pesados ou pessoas.}{bás.cu.la}{0}
\verb{basculante}{}{}{}{}{adj.2g.}{Que funciona com movimento de básculo.}{bas.cu.lan.te}{0}
\verb{basculante}{}{}{}{}{}{Diz"-se do caminhão equipado com esse dispositivo, usado para transportar areia,  terras, entulho.}{bas.cu.lan.te}{0}
\verb{basculante}{}{}{}{}{}{Diz"-se da janela com folhas horizontais que se inclinam ao abrir.}{bas.cu.lan.te}{0}
\verb{básculo}{}{}{}{}{s.m.}{Ponte elevadiça com dispositivo de contrapeso.}{bás.cu.lo}{0}
\verb{básculo}{}{}{}{}{}{Peça móvel de metal, apoiada num pino, que serve para abrir e fechar, alternadamente, dois ferrolhos de uma porta ou janela.}{bás.cu.lo}{0}
\verb{base}{}{}{}{}{s.f.}{Aquilo que serve de apoio para outra coisa; fundamento, suporte. }{ba.se}{0}
\verb{base}{}{}{}{}{}{Conjunto de conhecimentos que se deve ter para fazer alguma coisa. }{ba.se}{0}
\verb{base}{}{}{}{}{}{Lugar onde fica uma tropa de soldados para controlar as atividades militares da região.   }{ba.se}{0}
\verb{baseado}{}{}{}{}{adj.}{Que se assenta numa base; fundamentado, firme.}{ba.se.a.do}{0}
\verb{baseado}{}{Pop.}{}{}{s.m.}{Cigarro de maconha; jererê.}{ba.se.a.do}{0}
\verb{baseado}{}{}{}{}{}{Seguro do próprio valor ou de alguma habilidade.}{ba.se.a.do}{0}
\verb{baseado}{}{}{}{}{}{Sagaz, astucioso, perspicaz.}{ba.se.a.do}{0}
\verb{basear}{}{}{}{}{v.t.}{Servir de base; fundamentar.}{ba.se.ar}{0}
\verb{basear}{}{}{}{}{}{Estabelecer a base; apoiar, firmar.}{ba.se.ar}{\verboinum{4}}
\verb{básico}{}{}{}{}{adj.}{Que serve de base; fundamental.}{bá.si.co}{0}
\verb{básico}{}{}{}{}{}{Que é mais importante; essencial, primordial.}{bá.si.co}{0}
\verb{básico}{}{Quím.}{}{}{}{Que tem caráter de base; alcalino.}{bá.si.co}{0}
\verb{basilar}{}{}{}{}{adj.2g.}{Que serve de base; fundamental, básico.}{ba.si.lar}{0}
\verb{basilar}{}{}{}{}{}{Que se origina ou está situado na base.}{ba.si.lar}{0}
\verb{basílica}{}{Relig.}{}{}{s.f.}{Igreja que goza de certos privilégios sobre as outras, com exceção das catedrais, não estando submetida à jurisdição eclesiástica local.}{ba.sí.li.ca}{0}
\verb{basílica}{}{Anat.}{}{}{adj.}{Diz"-se da veia superficial da face interna do braço.}{ba.sí.li.ca}{0}
\verb{basilicão}{}{}{"-ões}{}{s.m.}{Unguento preparado com cera, azeite, resina e pez.}{ba.si.li.cão}{0}
\verb{basilisco}{}{Zool.}{}{}{s.m.}{Nome dado a certos tipos de lagartos, habitantes de terras desde o México até a Colômbia, que possuem coloração verde, longas patas traseiras e cauda muito comprida.}{ba.si.lis.co}{0}
\verb{basilisco}{}{Mit.}{}{}{}{Serpente fantástica, cujo olhar e cujo bafo tinham poder de matar.}{ba.si.lis.co}{0}
\verb{basilisco}{}{}{}{}{}{Antigo canhão de bronze que atirava pesadas balas de ferro.}{ba.si.lis.co}{0}
\verb{basquete}{é}{Esport.}{}{}{s.m.}{Esporte disputado por duas equipes com cinco jogadores, cujo objetivo é marcar o maior número possível de pontos, fazendo a bola entrar numa cesta; basquetebol.}{bas.que.te}{0}
\verb{basquetebol}{ó}{Esport.}{}{}{s.m.}{Basquete.}{bas.que.te.bol}{0}
\verb{bassê}{}{Zool.}{}{}{s.m.}{Raça de cão de corpo alongado, pernas curtas, orelhas compridas e pelo curto de coloração marrom ou preta.}{bas.sê}{0}
\verb{basta}{}{}{}{}{interj.}{Expressão que indica a alguém que deve parar de fazer algo ou calar"-se; chega.}{bas.ta}{0}
\verb{basta}{}{}{}{}{s.m.}{Interrupção, ponto final, limite.}{bas.ta}{0}
\verb{bastante}{}{}{}{}{adj.2g.}{Que basta, que satisfaz; suficiente.}{bas.tan.te}{0}
\verb{bastante}{}{}{}{}{pron.}{Muito, numeroso, abundante.}{bas.tan.te}{0}
\verb{bastante}{}{}{}{}{adv.}{Em quantidade suficiente; muito.}{bas.tan.te}{0}
\verb{bastante}{}{}{}{}{}{De maneira acima da média.}{bas.tan.te}{0}
\verb{bastão}{}{}{"-ões}{}{s.m.}{Pedaço fino de madeira, cilíndrico e alongado, que se leva na mão para auxiliar na marcha.}{bas.tão}{0}
\verb{bastão}{}{}{"-ões}{}{}{Pequena vara, cajado, bengala.}{bas.tão}{0}
\verb{bastão}{}{}{"-ões}{}{}{Insígnia ou distintivo de certas dignidades.}{bas.tão}{0}
\verb{bastar}{}{}{}{}{v.t.}{Ser suficiente; chegar; não precisar de ajuda.}{bas.tar}{\verboinum{1}}
\verb{bastardo}{}{}{}{}{adj.}{Diz"-se do filho que nasceu de uma união extraconjugal; ilegítimo.}{bas.tar.do}{0}
\verb{bastardo}{}{Biol.}{}{}{}{Que se tornou diferente da espécie ou raça a que pertence.}{bas.tar.do}{0}
\verb{bastião}{}{}{"-ões}{}{s.m.}{Parte saliente de uma fortificação que permite vigiar a parte externa da muralha.}{bas.ti.ão}{0}
\verb{bastião}{}{Fig.}{"-ões}{}{}{Indivíduo ou instituição que luta em prol de uma causa; baluarte.}{bas.ti.ão}{0}
\verb{bastidor}{ô}{}{}{}{s.m.}{Caixilho de madeira em que se prende um tecido para bordar.}{bas.ti.dor}{0}
\verb{bastidor}{ô}{}{}{}{}{Cada uma das armações móveis que servem de cenário para decorar as laterais do palco.}{bas.ti.dor}{0}
\verb{basto}{}{}{}{}{adj.}{Que possui espessura e densidade; compacto, cerrado.}{bas.to}{0}
\verb{basto}{}{}{}{}{}{Que apresenta abundância; numeroso, copioso.}{bas.to}{0}
\verb{bastonada}{}{}{}{}{s.f.}{Golpe, pancada aplicada com bastão.}{bas.to.na.da}{0}
\verb{bastonete}{ê}{}{}{}{s.m.}{Pequeno bastão, varinha.}{bas.to.ne.te}{0}
\verb{bastonete}{ê}{Biol.}{}{}{}{Bactéria alongada, em forma de bastão.}{bas.to.ne.te}{0}
\verb{bata}{}{}{}{}{s.f.}{Vestido inteiriço, solto, abotoado na frente de cima a baixo.}{ba.ta}{0}
\verb{bata}{}{}{}{}{}{Blusa solta e folgada que se usa por cima da saia ou da calça.}{ba.ta}{0}
\verb{bata}{}{}{}{}{}{Espécie de roupão, de tecido branco e leve, usado por dentistas, médicos etc., no exercício de suas funções.}{ba.ta}{0}
\verb{batalha}{}{}{}{}{s.f.}{Em uma guerra, combate entre forças oponentes.}{ba.ta.lha}{0}
\verb{batalha}{}{Por ext.}{}{}{}{Qualquer combate; luta, peleja, duelo.}{ba.ta.lha}{0}
\verb{batalha}{}{Fig.}{}{}{}{Discussão violenta; polêmica, controvérsia.}{ba.ta.lha}{0}
\verb{batalhador}{ô}{}{}{}{adj.}{Que batalha; lutador.}{ba.ta.lha.dor}{0}
\verb{batalhador}{ô}{}{}{}{}{Defensor convicto e audaz de uma ideia ou princípio.}{ba.ta.lha.dor}{0}
\verb{batalhador}{ô}{}{}{}{}{Que trabalha muito.}{ba.ta.lha.dor}{0}
\verb{batalhão}{}{}{"-ões}{}{s.m.}{Parte de um regimento, subdividida em companhias.}{ba.ta.lhão}{0}
\verb{batalhão}{}{Pop.}{"-ões}{}{}{Grande quantidade de pessoas; multidão.}{ba.ta.lhão}{0}
\verb{batalhar}{}{}{}{}{v.i.}{Entrar em combate; lutar, pelejar.}{ba.ta.lhar}{0}
\verb{batalhar}{}{}{}{}{v.t.}{Discutir incansavelmente; argumentar, polemizar.}{ba.ta.lhar}{0}
\verb{batalhar}{}{}{}{}{}{Tentar conseguir; esforçar"-se.}{ba.ta.lhar}{\verboinum{1}}
\verb{batata}{}{Bot.}{}{}{s.f.}{Planta rasteira, originária das Américas, largamente cultivada por seus tubérculos comestíveis, de forma e tamanho variados, ricos em fécula.}{ba.ta.ta}{0}
\verb{batata}{}{}{}{}{}{O tubérculo dessa planta.}{ba.ta.ta}{0}
\verb{batata"-baroa}{ô}{Bot.}{batatas"-baroas ⟨ô⟩}{}{s.f.}{Planta de tubérculos amarelos, utilizados na alimentação humana e como forragem.}{ba.ta.ta"-ba.ro.a}{0}
\verb{batata"-baroa}{ô}{}{batatas"-baroas ⟨ô⟩}{}{}{O tubérculo dessa planta; mandioquinha.}{ba.ta.ta"-ba.ro.a}{0}
\verb{batatada}{}{}{}{}{s.f.}{Grande quantidade de batatas.}{ba.ta.ta.da}{0}
\verb{batatada}{}{}{}{}{}{Doce feito de batata"-doce.}{ba.ta.ta.da}{0}
\verb{batatada}{}{Pop.}{}{}{}{Grande besteira; tolice, asneira.}{ba.ta.ta.da}{0}
\verb{batata"-doce}{ô}{Bot.}{batatas"-doces ⟨ô⟩}{}{s.f.}{Planta rasteira, cujas raízes são tubérculos comestíveis com sabor adocicado.}{ba.ta.ta"-do.ce}{0}
\verb{batata"-doce}{ô}{}{batatas"-doces ⟨ô⟩}{}{}{O tubérculo dessa planta.}{ba.ta.ta"-do.ce}{0}
\verb{batata"-inglesa}{ê}{}{batatas"-inglesas ⟨ê⟩}{}{s.f.}{Variedade de batata, largamente consumida na alimentação humana.}{ba.ta.ta"-in.gle.sa}{0}
\verb{batatal}{}{}{"-ais}{}{s.m.}{Terreno em que se cultivam batatas.}{ba.ta.tal}{0}
\verb{batatinha}{}{}{}{}{s.f.}{Batata pequena preparada principalmente como conserva.}{ba.ta.ti.nha}{0}
\verb{batavo}{}{}{}{}{adj.}{Relativo à Batávia, antigo nome da Holanda; holandês.}{ba.ta.vo}{0}
\verb{batavo}{}{}{}{}{s.m.}{Indivíduo natural ou habitante da Holanda; holandês.}{ba.ta.vo}{0}
\verb{bateada}{}{}{}{}{s.f.}{Quantidade de minério contida em uma bateia.}{ba.te.a.da}{0}
\verb{batear}{}{}{}{}{v.t.}{Lavar minério na bateia.}{ba.te.ar}{\verboinum{4}}
\verb{bate"-boca}{ô}{}{bate"-bocas ⟨ô⟩}{}{s.m.}{Discussão agressiva; altercação.}{ba.te"-bo.ca}{0}
\verb{bate"-boca}{ô}{}{bate"-bocas ⟨ô⟩}{}{}{Vozerio, clamor de briga.}{ba.te"-bo.ca}{0}
\verb{bate"-bola}{ó}{Esport.}{bate"-bolas ⟨ó⟩}{}{s.m.}{Futebol informal, jogado como diversão ou exercício; pelada.}{ba.te"-bo.la}{0}
\verb{bate"-bola}{ó}{}{bate"-bolas ⟨ó⟩}{}{}{Troca de passes feita pelos jogadores para aquecimento antes do início da partida.}{ba.te"-bo.la}{0}
\verb{batedeira}{ê}{}{}{}{s.f.}{Utensílio de cozinha, manual ou elétrico, usado para bater massas, ovos,  misturas etc.}{ba.te.dei.ra}{0}
\verb{batedeira}{ê}{}{}{}{}{Aparelho com que se bate o leite ou agita a nata, para obter a manteiga.}{ba.te.dei.ra}{0}
\verb{batedeira}{ê}{Fig.}{}{}{}{Palpitações do coração.}{ba.te.dei.ra}{0}
\verb{batedor}{ô}{}{}{}{adj.}{Que bate.}{ba.te.dor}{0}
\verb{batedor}{ô}{}{}{}{}{Que vai à frente da caravana ou da companhia, abrindo caminho.}{ba.te.dor}{0}
\verb{batedor}{ô}{}{}{}{}{Diz"-se do policial encarregado de abrir caminho para a passagem de autoridades ou visitantes ilustres.}{ba.te.dor}{0}
\verb{batedor}{ô}{}{}{}{}{Que cunha moedas.}{ba.te.dor}{0}
\verb{batedor}{ô}{}{}{}{s.m.}{Instrumento usado para debulhar milho.}{ba.te.dor}{0}
\verb{batedouro}{ô}{}{}{}{s.m.}{Pedra em que as lavadeiras batem as roupas.}{ba.te.dou.ro}{0}
\verb{bateeiro}{ê}{}{}{}{s.m.}{Garimpeiro que trabalha com a bateia nas lavras de ouro e diamantes.}{ba.te.ei.ro}{0}
\verb{bate"-estaca}{}{}{bate"-estacas}{}{s.m.}{Aparelho utilizado para cravar estacas no chão, como fundação de uma construção.}{ba.te"-es.ta.ca}{0}
\verb{bate"-estacas}{}{}{}{}{s.m.}{Bate"-estaca.}{ba.te"-es.ta.cas}{0}
\verb{bátega}{}{}{}{}{s.f.}{Pancada de chuva; aguaceiro forte; toró.}{bá.te.ga}{0}
\verb{bateia}{é/ ou /ê}{}{}{}{s.f.}{Vasilha de madeira ou de metal em que se lavam areias ou cascalho no garimpo.}{ba.tei.a}{0}
\verb{batel}{é}{}{"-éis}{}{s.m.}{Barco pequeno; canoa.}{ba.tel}{0}
\verb{batelada}{}{}{}{}{s.f.}{Carregamento de um batel.}{ba.te.la.da}{0}
\verb{batelada}{}{}{}{}{}{Grande quantidade.}{ba.te.la.da}{0}
\verb{batelão}{}{}{"-ões}{}{s.m.}{Embarcação grande usada para transportar cargas pesadas.}{ba.te.lão}{0}
\verb{bateleiro}{ê}{}{}{}{s.m.}{Indivíduo que dirige um batel ou é dono de um.}{ba.te.lei.ro}{0}
\verb{batente}{}{}{}{}{s.m.}{Encaixe de madeira para porta ou janela.}{ba.ten.te}{0}
\verb{batente}{}{}{}{}{}{A folha que se fecha primeiro, em portas ou janelas de duas folhas.}{ba.ten.te}{0}
\verb{batente}{}{Pop.}{}{}{}{Trabalho diário, de onde se tira o sustento.}{ba.ten.te}{0}
\verb{bate"-papo}{}{}{bate"-papos}{}{s.m.}{Conversa informal e amigável; prosa.}{ba.te"-pa.po}{0}
\verb{bate"-pé}{}{}{bate"-pés}{}{s.m.}{Dança popular que se caracteriza por um sapateado rápido e cadenciado; sapateado.}{ba.te"-pé}{0}
\verb{bater}{ê}{}{}{}{v.t.}{Dar pancada em alguém ou em alguma coisa. (\textit{Ele bateu no cachorro.})}{ba.ter}{0}
\verb{bater}{ê}{}{}{}{}{Fazer alguma coisa mover"-se de encontro a outra. (\textit{Os pássaros saíram batendo as asas.})}{ba.ter}{0}
\verb{bater}{ê}{}{}{}{}{Andar por um lugar para observar ou procurar; explorar. (\textit{Ele bateu por todas as ruas, mas não o encontrou.})}{ba.ter}{0}
\verb{bater}{ê}{}{}{}{}{Fazer alguma coisa soar. (\textit{Eu vou bater o sino.})}{ba.ter}{0}
\verb{bater}{ê}{}{}{}{}{Fazer uma máquina ou um instrumento funcionar com um movimento de pressão. (\textit{Bater um texto à máquina. Bater uma foto.})}{ba.ter}{0}
\verb{bater}{ê}{}{}{}{}{Vencer. (\textit{Meu time bateu todos os adversários com facilidade.})}{ba.ter}{0}
\verb{bater}{ê}{}{}{}{}{Mover"-se de encontro a alguma coisa. (\textit{Menino, pare de bater essas portas!})}{ba.ter}{0}
\verb{bater}{ê}{}{}{}{}{Enfrentar um adversário. (\textit{Não devemos ficar nos batendo nessa discussão tola.})}{ba.ter}{0}
\verb{bater}{ê}{}{}{}{v.i.}{Ter movimentos de contração e dilatação; palpitar, pulsar. (\textit{O coração parou de bater.})}{ba.ter}{\verboinum{12}}
\verb{bateria}{}{Mús.}{}{}{s.f.}{Conjunto de instrumentos de percussão numa banda ou orquestra.}{ba.te.ri.a}{0}
\verb{bateria}{}{}{}{}{}{Conjunto de testes, exames ou provas.}{ba.te.ri.a}{0}
\verb{bateria}{}{Esport.}{}{}{}{Cada uma das etapas de um torneio esportivo.}{ba.te.ri.a}{0}
\verb{bateria}{}{}{}{}{}{Conjunto dos utensílios de cozinha.}{ba.te.ri.a}{0}
\verb{bateria}{}{}{}{}{}{Associação de duas ou mais pilhas ligadas em série.}{ba.te.ri.a}{0}
\verb{bateria}{}{}{}{}{}{Conjunto de bocas"-de"-fogo, de mesmas características ou finalidades, a bordo de um navio.}{ba.te.ri.a}{0}
\verb{baterista}{}{Mús.}{}{}{s.2g.}{Instrumentista que toca bateria.}{ba.te.ris.ta}{0}
\verb{baticum}{}{}{"-uns}{}{s.m.}{Som de palmas e sapateados, como nos batuques.}{ba.ti.cum}{0}
\verb{baticum}{}{Pop.}{"-uns}{}{}{Pulsação forte do coração e das artérias; taquicardia.}{ba.ti.cum}{0}
\verb{batida}{}{}{}{}{s.f.}{Encontro violento de uma coisa com outra; choque, colisão. (\textit{Na batida contra o poste, o carro ficou danificado.})}{ba.ti.da}{0}
\verb{batida}{}{Pop.}{}{}{}{Exploração de algum lugar para procurar alguma coisa.  (\textit{A polícia fez uma batida em todas as casas.})}{ba.ti.da}{0}
\verb{batida}{}{}{}{}{}{Bebida alcoólica a que se acrescentam suco de fruta e açúcar.  (\textit{Na festa só serviram refrigerante e batida de maracujá.})}{ba.ti.da}{0}
\verb{batido}{}{}{}{}{adj.}{Que levou pancada; sovado, socado.}{ba.ti.do}{0}
\verb{batido}{}{}{}{}{}{Que sofreu derrota; vencido.}{ba.ti.do}{0}
\verb{batido}{}{}{}{}{}{Desgastado pelo uso; muito usado.}{ba.ti.do}{0}
\verb{batido}{}{}{}{}{}{Sem originalidade; trivial, banalizado.}{ba.ti.do}{0}
\verb{batimento}{}{}{}{}{s.m.}{Ato ou efeito de bater; batido, batida.}{ba.ti.men.to}{0}
\verb{batimento}{}{Med.}{}{}{}{Pulsação, palpitação.}{ba.ti.men.to}{0}
\verb{batina}{}{Relig.}{}{}{s.f.}{Veste usada por sacerdotes ou seminaristas.}{ba.ti.na}{0}
\verb{batismal}{}{}{"-ais}{}{adj.2g.}{Relativo a batismo.}{ba.tis.mal}{0}
\verb{batismo}{}{Relig.}{}{}{s.m.}{Rito de purificação ou iniciação presente em diversas religiões, que consiste na imersão ou ablução do iniciado em água.}{ba.tis.mo}{0}
\verb{batismo}{}{}{}{}{}{Na Igreja Católica, o primeiro dos sete sacramentos.}{ba.tis.mo}{0}
\verb{batismo}{}{}{}{}{}{Ato de dar nome a uma pessoa ou coisa.}{ba.tis.mo}{0}
\verb{batismo}{}{Fig.}{}{}{}{Adulteração de um líquido, geralmente vinho ou leite, pela adição de água.}{ba.tis.mo}{0}
\verb{batista}{}{}{}{}{adj.2g.}{Diz"-se do membro de igreja protestante, na qual o batismo só é ministrado aos adultos.}{ba.tis.ta}{0}
\verb{batista}{}{}{}{}{}{Relativo a essa igreja.}{ba.tis.ta}{0}
\verb{batistério}{}{Relig.}{}{}{s.m.}{Recinto da igreja, próximo à porta principal, onde se localiza a pia batismal.}{ba.tis.té.rio}{0}
\verb{batistério}{}{Pop.}{}{}{}{Certidão de batismo.}{ba.tis.té.rio}{0}
\verb{batizado}{}{}{}{}{adj.}{Que acaba de receber o batismo.}{ba.ti.za.do}{0}
\verb{batizado}{}{Pop.}{}{}{}{Diz"-se do vinho ou do leite adulterado com  água.}{ba.ti.za.do}{0}
\verb{batizado}{}{Relig.}{}{}{s.m.}{Cerimônia religiosa com que se celebra o batismo.}{ba.ti.za.do}{0}
\verb{batizando}{}{Relig.}{}{}{adj.}{Que está prestes a receber o batismo.}{ba.ti.zan.do}{0}
\verb{batizar}{}{}{}{}{v.t.}{Administrar o batismo.}{ba.ti.zar}{0}
\verb{batizar}{}{}{}{}{}{Dar um nome de batismo.}{ba.ti.zar}{0}
\verb{batizar}{}{Fig.}{}{}{}{Adulterar uma bebida, adicionando"-lhe água.}{ba.ti.zar}{\verboinum{1}}
\verb{batom}{}{}{"-ons}{}{s.m.}{Cosmético em forma de bastão, de diversas cores, para pintar os lábios.}{ba.tom}{0}
\verb{batoque}{ó}{}{}{}{s.m.}{Orifício redondo no bojo de tonéis, barris etc.}{ba.to.que}{0}
\verb{batoque}{ó}{}{}{}{}{Rolha grossa que tapa esse orifício.}{ba.to.que}{0}
\verb{batoque}{ó}{Fig.}{}{}{}{Indivíduo de baixa estatura; baixote.}{ba.to.que}{0}
\verb{batoque}{ó}{}{}{}{}{Botoque.}{ba.to.que}{0}
\verb{batota}{ó}{Pop.}{}{}{s.f.}{Trapaça, fraude no jogo.}{ba.to.ta}{0}
\verb{batota}{ó}{}{}{}{}{Jogo de azar.}{ba.to.ta}{0}
\verb{batoteiro}{ê}{}{}{}{adj.}{Que está envolvido em fraudes no jogo; trapaceiro.}{ba.to.tei.ro}{0}
\verb{batoteiro}{ê}{}{}{}{}{Que frequenta assiduamente casas de jogos de azar.}{ba.to.tei.ro}{0}
\verb{batráquio}{}{Zool.}{}{}{s.m.}{Ordem dos anfíbios que compreende os sapos, as rãs e as pererecas; anuro.}{ba.trá.quio}{0}
\verb{batucada}{}{}{}{}{s.f.}{Ato ou efeito de batucar; batuque.}{ba.tu.ca.da}{0}
\verb{batucada}{}{}{}{}{}{Ritmo de batuque.}{ba.tu.ca.da}{0}
\verb{batucada}{}{}{}{}{}{Reunião popular com dança ou canto acompanhado de batuque.}{ba.tu.ca.da}{0}
\verb{batucador}{ô}{}{}{}{adj.}{Que batuca.}{ba.tu.ca.dor}{0}
\verb{batucador}{ô}{}{}{}{}{Que toca mal o piano.}{ba.tu.ca.dor}{0}
\verb{batucar}{}{}{}{}{v.i.}{Tocar instrumento de percussão.}{ba.tu.car}{0}
\verb{batucar}{}{}{}{}{}{Marcar o ritmo, batendo em algo; tamborilar.}{ba.tu.car}{0}
\verb{batucar}{}{}{}{}{}{Dançar e cantar o batuque.}{ba.tu.car}{0}
\verb{batucar}{}{}{}{}{}{Tocar mal o piano.}{ba.tu.car}{\verboinum{2}}
\verb{batuque}{}{}{}{}{s.m.}{Ato ou efeito de batucar.}{ba.tu.que}{0}
\verb{batuque}{}{}{}{}{}{Dança afro"-brasileira acompanhada de instrumentos de percussão.}{ba.tu.que}{0}
\verb{batuque}{}{}{}{}{}{Ato de bater repetidamente, martelar.}{ba.tu.que}{0}
\verb{batuqueiro}{ê}{}{}{}{adj.}{Que toca, canta ou dança batuques.}{ba.tu.quei.ro}{0}
\verb{batuqueiro}{ê}{}{}{}{}{Que frequenta batuques ou batucadas.}{ba.tu.quei.ro}{0}
\verb{batuta}{}{Mús.}{}{}{s.f.}{Bastão leve e fino utilizado pelos regentes para dirigir orquestras.}{ba.tu.ta}{0}
\verb{batuta}{}{Pop.}{}{}{adj.}{Entendido em algum assunto; hábil, capaz.}{ba.tu.ta}{0}
\verb{baú}{}{}{}{}{s.m.}{Caixa retangular de madeira, com tampa convexa, geralmente revestida de couro, utilizada para guardar e transportar pertences; arca.}{ba.ú}{0}
\verb{baunilha}{}{Bot.}{}{}{s.f.}{Planta da família das orquidáceas, de cujas vagens se extrai um aromatizante muito utilizado em confeitaria.}{bau.ni.lha}{0}
\verb{baunilha}{}{}{}{}{}{A essência aromatizante preparada com a vagem dessa planta.}{bau.ni.lha}{0}
\verb{bauru}{}{Cul.}{}{}{s.m.}{Sanduíche preparado com pão francês, rosbife ou presunto, queijo, alface, tomate e ovo frito.}{bau.ru}{0}
\verb{bauxita}{ch}{Geol.}{}{}{s.f.}{Rocha mineral, com aparência de argila, mas sem plasticidade, formada de alumina hidratada misturada com uma proporção variável de óxido de ferro, de onde se extrai o alumínio.}{bau.xi.ta}{0}
\verb{bávaro}{}{}{}{}{adj.}{Relativo à região da Baviera, na Alemanha.}{bá.va.ro}{0}
\verb{bávaro}{}{}{}{}{s.m.}{Indivíduo natural ou habitante dessa região.}{bá.va.ro}{0}
\verb{bazar}{}{}{}{}{s.m.}{Estabelecimento onde se vende todo tipo de objetos, principalmente quinquilharias, louças, brinquedos.}{ba.zar}{0}
\verb{bazar}{}{}{}{}{}{No Oriente e no Norte da África, mercado público onde se vendem ou se trocam mercadorias variadas, além de gêneros alimentícios.}{ba.zar}{0}
\verb{bazófia}{}{}{}{}{s.f.}{Vaidade exagerada; presunção, fanfarrice.}{ba.zó.fia}{0}
\verb{bazófia}{}{Cul.}{}{}{}{Ensopado feito com sobras de comida.}{ba.zó.fia}{0}
\verb{bazofiar}{}{}{}{}{v.t.}{Dizer bazófia; vangloriar, alardear, fanfarrear.}{ba.zo.fi.ar}{\verboinum{1}}
\verb{bazuca}{}{}{}{}{s.f.}{Arma portátil de guerra, que consiste em um tubo de disparo aberto em ambas as extremidades que lança foguetes contra tanques.}{ba.zu.ca}{0}
\verb{BCG}{}{Med.}{}{}{s.f.}{Sigla de bacilo de Calmette e Guérin; vacina utilizada para a imunização contra a tuberculose.}{b.c.g.}{0}
\verb{Be}{}{Quím.}{}{}{}{Símb. do \textit{berílio}.}{Be}{0}
\verb{bê}{}{}{}{}{s.m.}{Nome da letra \textit{b}.}{bê}{0}
\verb{bê"-a"-bá}{}{}{}{}{s.m.}{Conjunto das letras do alfabeto; abecedário.}{bê"-a"-bá}{0}
\verb{bê"-a"-bá}{}{}{}{}{}{Exercício de soletração.}{bê"-a"-bá}{0}
\verb{bê"-a"-bá}{}{}{}{}{}{Primeiras noções de algum assunto, ciência ou arte.}{bê"-a"-bá}{0}
\verb{beata}{}{Relig.}{}{}{s.f.}{Mulher que foi beatificada pela Igreja Católica.}{be.a.ta}{0}
\verb{beata}{}{Pop.}{}{}{}{Ponta de cigarro; guimba.}{be.a.ta}{0}
\verb{beata}{}{}{}{}{}{Mulher que se dedica totalmente às práticas religiosas.}{be.a.ta}{0}
\verb{beata}{}{Pop.}{}{}{}{Mulher que finge grande devoção e demonstra um comportamento exageradamente puritano.}{be.a.ta}{0}
\verb{beatice}{}{}{}{}{s.f.}{Devoção exagerada às práticas religiosas, de forma afetada ou fingida, acompanhada de comportamento excessivamente puritano.}{be.a.ti.ce}{0}
\verb{beatificação}{}{}{"-ões}{}{s.f.}{Ato ou efeito de beatificar.}{be.a.ti.fi.ca.ção}{0}
\verb{beatificação}{}{Relig.}{"-ões}{}{}{Cerimônia católica na qual o papa concede a pessoa falecida posição no rol dos bem"-aventurados, reconhecendo"-lhe suas virtudes e recomendando"-a ao culto dos fiéis.}{be.a.ti.fi.ca.ção}{0}
\verb{beatificado}{}{Relig.}{}{}{adj.}{Que recebeu a beatificação; bem"-aventurado.}{be.a.ti.fi.ca.do}{0}
\verb{beatificar}{}{}{}{}{v.t.}{Declarar beato, colocando"-o no rol dos bem"-aventurados.}{be.a.ti.fi.car}{0}
\verb{beatificar}{}{}{}{}{}{Tornar feliz.}{be.a.ti.fi.car}{\verboinum{2}}
\verb{beatífico}{}{}{}{}{adj.}{Que torna feliz, bem"-aventurado.}{be.a.tí.fi.co}{0}
\verb{beatífico}{}{}{}{}{}{Relativo a êxtase, glória.}{be.a.tí.fi.co}{0}
\verb{beatitude}{}{Relig.}{}{}{s.f.}{Segundo a Igreja Católica, felicidade eterna desfrutada no céu pelos santos.}{be.a.ti.tu.de}{0}
\verb{beatitude}{}{Por ext.}{}{}{}{Estado de serenidade, de felicidade; placidez.}{be.a.ti.tu.de}{0}
\verb{beatitude}{}{}{}{}{}{Forma de tratamento dispensada ao papa.}{be.a.ti.tu.de}{0}
\verb{beato}{}{}{}{}{adj.}{Que goza de bem"-aventurança; feliz.}{be.a.to}{0}
\verb{beato}{}{}{}{}{}{Que recebeu beatificação.}{be.a.to}{0}
\verb{beato}{}{}{}{}{}{Que é muito devoto das práticas religiosas.}{be.a.to}{0}
\verb{bêbado}{}{}{}{}{adj.}{Diz"-se de indivíduo intoxicado por bebida alcóolica; ébrio, bêbado, embriagado, alcoolizado.}{bê.ba.do}{0}
\verb{bêbado}{}{}{}{}{s.m.}{Indivíduo viciado em bebidas alcoólicas; beberrão, borracho, pinguço.  }{bê.ba.do}{0}
\verb{bêbado}{}{Fig.}{}{}{}{Que se encontra em estado de exaltação devido a uma perturbação emocional.}{bê.ba.do}{0}
\verb{bêbado}{}{Fig.}{}{}{}{Atordoado, tonto, zonzo, atarantado.}{bê.ba.do}{0}
\verb{bebé}{}{}{}{}{}{Var. de \textit{bebê}.}{be.bé}{0}
\verb{bebê}{}{}{}{}{s.2g.}{Recém"-nascido, criança de peito; nenê.}{be.bê}{0}
\verb{bebedeira}{ê}{}{}{}{s.f.}{Ato ou efeito de embriagar"-se, alcoolizar"-se; embriaguez.}{be.be.dei.ra}{0}
\verb{bêbedo}{}{}{}{}{}{Var. de \textit{bêbado}.}{bê.be.do}{0}
\verb{bebedor}{ô}{}{}{}{adj.}{Que bebe.}{be.be.dor}{0}
\verb{bebedor}{ô}{}{}{}{s.m.}{Indivíduo dado a bebedeiras; ébrio.}{be.be.dor}{0}
\verb{bebedouro}{ô}{}{}{}{s.m.}{Tanque ou recipiente onde se coloca água para os animais beberem.}{be.be.dou.ro}{0}
\verb{bebedouro}{ô}{Bras.}{}{}{}{Aparelho com água canalizada, elétrico ou não, com torneira voltada para cima, no qual se pode beber água diretamente, sem auxílio de copos,  aproximando a boca do jato; bebedor.}{be.be.dou.ro}{0}
\verb{beber}{ê}{}{}{}{}{Engolir um líquido; tomar. }{be.ber}{\verboinum{12}}
\verb{beber}{ê}{}{}{}{v.t.}{Ter o costume de tomar bebida alcoólica.  }{be.ber}{0}
\verb{beberagem}{}{Med.}{"-ens}{}{s.f.}{Cozimento ou infusão medicinal.}{be.be.ra.gem}{0}
\verb{beberagem}{}{}{"-ens}{}{}{Bebida, composta de vários ingredientes, de sabor desagradável, intragável.}{be.be.ra.gem}{0}
\verb{beberagem}{}{Bras.}{"-ens}{}{}{Preparado medicinal, caseiro, geralmente elaborado por curandeiro; garrafada.}{be.be.ra.gem}{0}
\verb{bebericar}{}{}{}{}{v.i.}{Beber pouco e repetidamente.}{be.be.ri.car}{\verboinum{2}}
\verb{bebericar}{}{}{}{}{v.t.}{Beber aos poucos, a goles pequenos.}{be.be.ri.car}{0}
\verb{beberrão}{}{}{"-ões}{"-ona}{adj.}{Diz"-se daquele que bebe muito ou com frequência; ébrio, bêbado, pinguço.}{be.ber.rão}{0}
\verb{bebes}{é}{}{}{}{s.m.pl.}{Tudo o que se bebe em festas, reuniões, almoços etc.; bebidas.}{be.bes}{0}
\verb{bebida}{}{}{}{}{s.f.}{Ato de beber. }{be.bi.da}{0}
\verb{bebida}{}{}{}{}{}{Líquido que se pode beber.  }{be.bi.da}{0}
\verb{bebida}{}{}{}{}{}{Líquido alcoólico que se pode beber.  }{be.bi.da}{0}
\verb{bebido}{}{}{}{}{adj.}{Que se bebeu; ingerido, tomado.}{be.bi.do}{0}
\verb{beca}{é}{}{}{}{s.f.}{Veste preta, que vai até os calcanhares, usada por magistrados, advogados, funcionários judiciais, catedráticos e formandos de grau superior; toga.}{be.ca}{0}
\verb{beca}{é}{Fig.}{}{}{}{Ofício da magistratura.}{be.ca}{0}
\verb{beca}{é}{Pop.}{}{}{}{Roupa, vestimenta, traje.}{be.ca}{0}
\verb{beca}{é}{}{}{}{s.m.}{Magistrado ou bacharel em direito.}{be.ca}{0}
\verb{beça}{é}{}{}{}{}{Palavra usada apenas na locução adverbial \textit{à beça}: em grande quantidade, em profusão, muito.}{be.ça}{0}
\verb{becape}{}{Informát.}{}{}{s.m.}{Aportuguesamento da palavra inglesa \textit{backup}, que designa a cópia de um arquivo, mantida como reserva para o caso de dano ou perda do original; cópia de segurança.}{be.ca.pe}{0}
\verb{beco}{ê}{}{}{}{s.m.}{Rua curta e estreita, geralmente sem saída; ruela.}{be.co}{0}
\verb{bedel}{é}{Bras.}{"-éis}{}{s.m.}{Funcionário encarregado de manter a disciplina em estabelecimentos de ensino.}{be.del}{0}
\verb{bedelho}{ê}{}{}{}{s.m.}{Ferrolho de porta.}{be.de.lho}{0}
\verb{bedelho}{ê}{}{}{}{}{Criançola, menino, rapazinho. }{be.de.lho}{0}
\verb{beduíno}{}{}{}{}{s.m.}{Árabe nômade do deserto.}{be.du.í.no}{0}
\verb{beduíno}{}{}{}{}{adj.}{Relativo a ou pertencente a beduíno.}{be.du.í.no}{0}
\verb{bege}{é}{}{}{}{adj.2g.}{De cor amarela pálida.}{be.ge}{0}
\verb{bege}{é}{}{}{}{s.m.}{Essa cor.}{be.ge}{0}
\verb{begônia}{}{Bot.}{}{}{s.f.}{Nome comum a várias plantas nativas dos trópicos, cultivadas pelas folhas e flores de grande efeito decorativo e, alguns espécimes, por suas propriedades antitérmicas.}{be.gô.nia}{0}
\verb{beiço}{}{}{}{}{s.m.}{Cada uma das partes carnudas que formam o exterior da boca; lábio.}{bei.ço}{0}
\verb{beiçola}{ó}{}{}{}{s.f.}{Beiço grosso e proeminente; beiçorra.}{bei.ço.la}{0}
\verb{beiçudo}{}{Pop.}{}{}{adj.}{Que tem o beiço grande, grosso ou proeminente.}{bei.çu.do}{0}
\verb{beija"-flor}{ô}{Zool.}{beija"-flores ⟨ô⟩}{}{s.m.}{Nome comum às aves pequeninas, de voo veloz, bico longo e fino, dotado de língua comprida, usada para sugar o néctar das flores; colibri.}{bei.ja"-flor}{0}
\verb{beija"-mão}{}{}{beija"-mãos}{}{s.m.}{Ato, cerimônia ou costume de beijar a mão.}{bei.ja"-mão}{0}
\verb{beija"-pé}{}{}{beija"-pés}{}{s.m.}{Ato, cerimônia ou costume de beijar o pé.}{bei.ja"-pé}{0}
\verb{beija"-pé}{}{Relig.}{beija"-pés}{}{}{Cerimônia na qual os devotos beijam os pés do papa, em Roma.}{bei.ja"-pé}{0}
\verb{beijar}{}{}{}{}{v.t.}{Tocar pessoa ou coisa com os lábios, fazendo leve sucção.  }{bei.jar}{\verboinum{1}}
\verb{beijo}{ê}{}{}{}{s.m.}{Ato de tocar com os lábios em alguém ou algo, fazendo leve sucção; ósculo.}{bei.jo}{0}
\verb{beijoca}{ó}{Pop.}{}{}{s.f.}{Beijo ruidoso, estalado; bicota.}{bei.jo.ca}{0}
\verb{beijocar}{}{}{}{}{v.t.}{Dar ou trocar beijos estalados; beijar.}{bei.jo.car}{\verboinum{2}}
\verb{beijoqueiro}{ê}{}{}{}{adj.}{Que gosta de beijar ou de beijocar.}{bei.jo.quei.ro}{0}
\verb{beiju}{}{}{}{}{}{Var. de \textit{biju}.}{bei.ju}{0}
\verb{beira}{ê}{}{}{}{s.f.}{Parte que margeia; borda, limite, margem, orla.}{bei.ra}{0}
\verb{beira}{ê}{}{}{}{}{Beiral, aba de telhado.}{bei.ra}{0}
\verb{beira}{ê}{}{}{}{}{Proximidade, vizinha.}{bei.ra}{0}
\verb{beirada}{}{}{}{}{s.f.}{Parte, lado que está na beira; margem, borda. }{bei.ra.da}{0}
\verb{beirada}{}{Bras.}{}{}{}{Pedaço, bocado.}{bei.ra.da}{0}
\verb{beiral}{}{}{"-ais}{}{s.m.}{Parte do telhado que fica para fora da parede.}{bei.ral}{0}
\verb{beira"-mar}{}{}{beira"-mares}{}{s.f.}{Costa marítima, litoral, praia.}{bei.ra"-mar}{0}
\verb{beirão}{}{}{"-ões}{"-oa \textit{⟨ô⟩ ou} -ã}{adj.}{Pertencente ou relativo às Beiras (Portugal).}{bei.rão}{0}
\verb{beirão}{}{}{"-ões}{"-oa \textit{⟨ô⟩ ou} -ã}{s.m.}{Indivíduo natural ou habitante dessa região.}{bei.rão}{0}
\verb{beirar}{}{}{}{}{v.t.}{Estar na beira de algum lugar. (\textit{O caminho beirava o precipício.})}{bei.rar}{\verboinum{1}}
\verb{beirute}{}{Cul.}{}{}{s.m.}{Sanduíche feito com pão árabe, com recheio variado.}{bei.ru.te}{0}
\verb{beisebol}{ó}{Esport.}{}{}{s.m.}{Jogo de bola, derivado do críquete, disputado por dois times de nove jogadores cada, num campo com quatro bases.}{bei.se.bol}{0}
%\verb{}{}{}{}{}{}{}{}{0}
\verb{beladona}{}{Bot.}{}{}{s.f.}{Planta da família das solanáceas, originária da Europa e da Ásia, de flores violáceas e bagas roxas ou pretas com muitas sementes achatadas; contém o alcaloide atropina, usado medicinalmente.}{be.la.do.na}{0}
\verb{belas"-artes}{é}{}{}{}{s.f.pl.}{Manifestações artísticas (desenho, pintura, poesia, dança, música, arquitetura etc.) cujo objetivo exclusivo é a representação do belo. }{be.las"-ar.tes}{0}
\verb{belas"-letras}{é\ldots{}ê}{Liter.}{}{}{s.f.pl.}{A gramática, a retórica, a poesia, a prosa, a arte dramática e quaisquer outras manifestações letradas, estudadas ou ensinadas com um propósito exclusivamente estético, em oposição aos textos técnicos e científicos.}{be.las"-le.tras}{0}
\verb{belchior}{ó}{Desus.}{}{}{s.m.}{Negociante de objetos velhos e usados; ferro"-velho. }{bel.chi.or}{0}
\verb{belchior}{ó}{}{}{}{}{Dono de sebo; alfarrabista.}{bel.chi.or}{0}
\verb{beldade}{}{}{}{}{s.f.}{Qualidade do que é belo; beleza, formosura.}{bel.da.de}{0}
\verb{beldade}{}{}{}{}{}{Mulher muito bonita, que chama a atenção.}{bel.da.de}{0}
\verb{beleguim}{}{Desus.}{"-ins}{}{s.m.}{Agente de polícia; tira.}{be.le.guim}{0}
\verb{beleléu}{}{Pop.}{}{}{s.m.}{Palavra usada apenas na locução \textit{ir para o beleléu}: morrer, sumir, desaparecer ou não ter êxito num empreendimento qualquer.}{be.le.léu}{0}
\verb{belenense}{}{}{}{}{adj.2g.}{Relativo a Belém, capital do Pará.}{be.le.nen.se}{0}
\verb{belenense}{}{}{}{}{s.2g.}{Indivíduo natural ou habitante dessa cidade.}{be.le.nen.se}{0}
\verb{beleza}{ê}{}{}{}{s.f.}{Qualidade do que é belo. (\textit{É a beleza da paisagem que nos encanta.})}{be.le.za}{0}
\verb{beleza}{ê}{}{}{}{}{Mulher bela.}{be.le.za}{0}
\verb{beleza}{ê}{}{}{}{}{Coisa bela, agradável, boa.}{be.le.za}{0}
\verb{belezoca}{ó}{Pop.}{}{}{s.2g.}{Pessoa ou coisa bonita, charmosa; beleza. }{be.le.zo.ca}{0}
\verb{belga}{é}{}{}{}{adj.2g.}{Relativo à Bélgica.}{bel.ga}{0}
\verb{belga}{é}{}{}{}{s.2g.}{Indivíduo natural ou habitante desse país.}{bel.ga}{0}
\verb{beliche}{}{}{}{}{s.f.}{Conjunto de duas ou três camas superpostas, com lastros apoiados numa armação única.}{be.li.che}{0}
\verb{beliche}{}{}{}{}{}{Cama estreita e de fixação especial, própria para uso a bordo.}{be.li.che}{0}
\verb{bélico}{}{}{}{}{adj.}{Relativo a guerra.}{bé.li.co}{0}
\verb{belicosidade}{}{}{}{}{s.f.}{Característica do que é belicoso.}{be.li.co.si.da.de}{0}
\verb{belicoso}{ô}{}{"-osos ⟨ó⟩}{"-osa ⟨ó⟩}{adj.}{Que gosta de lutar; guerreiro.}{be.li.co.so}{0}
\verb{belida}{}{Med.}{}{}{s.f.}{Mancha permanente da córnea devida a traumatismos ou ulcerações.}{be.li.da}{0}
\verb{beligerância}{}{}{}{}{s.f.}{Estado ou qualidade de beligerante.}{be.li.ge.rân.cia}{0}
\verb{beligerante}{}{}{}{}{adj.2g.}{Que faz guerra, ou está em guerra.}{be.li.ge.ran.te}{0}
\verb{beliscão}{}{}{"-ões}{}{s.m.}{Ato ou efeito de beliscar, de apertar a pele com a ponta dos dedos polegar e indicador.}{be.lis.cão}{0}
\verb{beliscar}{}{}{}{}{v.t.}{Apertar a pele com as pontas dos dedos polegar e indicador.}{be.lis.car}{0}
\verb{beliscar}{}{}{}{}{}{Estimular, excitar.}{be.lis.car}{0}
\verb{beliscar}{}{}{}{}{}{Comer pouco; lambiscar.}{be.lis.car}{\verboinum{2}}
\verb{belizenho}{}{}{}{}{adj.}{Relativo a Belize.}{be.li.ze.nho}{0}
\verb{belizenho}{}{}{}{}{s.m.}{Indivíduo natural ou habitante desse país; belizense.}{be.li.ze.nho}{0}
\verb{belizense}{}{}{}{}{adj.2g. e s.2g.}{Belizenho.}{be.li.zen.se}{0}
\verb{belo}{é}{}{}{}{adj.}{Que encanta por suas formas ou características; formoso.}{be.lo}{0}
\verb{belo}{é}{}{}{}{}{Que provoca espanto por seu tamanho.}{be.lo}{0}
\verb{belo}{é}{}{}{}{}{Que se pode identificar, mas não se quer; certo, determinado. (\textit{Um belo dia, ela voltou.})}{be.lo}{0}
\verb{belo}{é}{}{}{}{s.m.}{A beleza.}{be.lo}{0}
\verb{belo"-horizontino}{é}{}{belo"-horizontinos ⟨é⟩}{}{adj.}{Relativo a Belo Horizonte, capital de Minas Gerais.}{be.lo"-ho.ri.zon.ti.no}{0}
\verb{belo"-horizontino}{é}{}{belo"-horizontinos ⟨é⟩}{}{s.m.}{Indivíduo natural ou habitante dessa cidade.}{be.lo"-ho.ri.zon.ti.no}{0}
\verb{belonave}{}{}{}{}{s.f.}{Navio apropriado para realizar operações de guerra.}{be.lo.na.ve}{0}
\verb{bel"-prazer}{é\ldots{}ê}{}{bel"-prazeres ⟨é\ldots{}ê⟩}{}{s.m.}{Vontade própria, escolha, arbítrio, capricho.}{bel"-pra.zer}{0}
\verb{beltrano}{}{}{}{}{s.m.}{Termo usado para se referir a pessoa indeterminada.}{bel.tra.no}{0}
\verb{belveder}{é}{}{}{}{s.m.}{Pequeno mirante de onde se aprecia um vasto panorama.}{bel.ve.der}{0}
\verb{belvedere}{ê}{}{}{}{}{Var. de \textit{belveder}.}{bel.ve.de.re}{0}
\verb{belzebu}{}{}{}{}{s.m.}{O principal dos espíritos infernais, segundo o Novo Testamento; diabo, demônio.}{bel.ze.bu}{0}
\verb{bem}{}{}{bens}{}{s.m.}{Conjunto das ações que fazem a pessoa merecer a aprovação e o respeito da sociedade. (\textit{Ele só fez o bem para todos nós.})}{bem}{0}
\verb{bem}{}{}{bens}{}{}{Vida segura e feliz; bem"-estar, felicidade. \textit{ }}{bem}{0}
\verb{bem}{}{}{bens}{}{}{Pessoa a quem se ama.}{bem}{0}
\verb{bem}{}{}{bens}{}{}{Objeto de compra e venda. (\textit{Ele colocou todos os seus bens à venda.})}{bem}{0}
\verb{bem}{}{}{bens}{}{adv.}{De maneira perfeita. (\textit{Ele faz bem os seus trabalhos.})}{bem}{0}
\verb{bem}{}{}{bens}{}{}{Em grau extremo; muito. (\textit{Ele é um homem bem forte.})}{bem}{0}
\verb{bem"-acabado}{}{}{bem"-acabados}{}{adj.}{Que é feito com acabamento bom ou perfeito; bem executado.}{bem"-a.ca.ba.do}{0}
\verb{bem"-afortunado}{}{}{bem"-afortunados}{}{adj.}{Que goza de prosperidade; ditoso, feliz.}{bem"-a.for.tu.na.do}{0}
\verb{bem"-amado}{}{}{bem"-amados}{}{adj.}{Que é objeto de grande estima ou valor.}{bem"-a.ma.do}{0}
\verb{bem"-amado}{}{}{bem"-amados}{}{s.m.}{Indivíduo a quem se quer muito; querido, predileto.}{bem"-a.ma.do}{0}
\verb{bem"-apessoado}{}{}{bem"-apessoados}{}{adj.}{Diz"-se daquele que tem boa aparência.}{bem"-a.pes.so.a.do}{0}
\verb{bem"-aventurado}{}{}{bem"-aventurados}{}{adj.}{Que goza de boa ventura; feliz.}{bem"-a.ven.tu.ra.do}{0}
\verb{bem"-aventurado}{}{}{bem"-aventurados}{}{}{Diz"-se daquele que, em vida ou após a morte, desfruta das bem"-aventuranças divinas.}{bem"-a.ven.tu.ra.do}{0}
\verb{bem"-aventurado}{}{}{bem"-aventurados}{}{s.m.}{Indivíduo muito feliz.}{bem"-a.ven.tu.ra.do}{0}
\verb{bem"-aventurado}{}{}{bem"-aventurados}{}{}{Indivíduo merecedor das graças divinas.}{bem"-a.ven.tu.ra.do}{0}
\verb{bem"-aventurança}{}{}{bem"-aventuranças}{}{s.f.}{Grande felicidade; a glória.}{bem"-a.ven.tu.ran.ça}{0}
\verb{bem"-aventurança}{}{Relig.}{bem"-aventuranças}{}{}{A felicidade eterna que os santos e justos gozam no Céu, junto a Deus.}{bem"-a.ven.tu.ran.ça}{0}
\verb{bem"-bom}{}{Bras.}{}{}{s.m.}{Vida aprazível, tranquila; comodidade.}{bem"-bom}{0}
\verb{bem"-comportado}{}{}{bem"-comportados}{}{adj.}{Diz"-se daquele que se porta ou se comporta bem.}{bem"-com.por.ta.do}{0}
\verb{bem"-criado}{}{}{bem"-criados}{}{adj.}{Que recebeu uma boa educação.}{bem"-cri.a.do}{0}
\verb{bem"-criado}{}{}{bem"-criados}{}{}{Que é gordo, nutrido.}{bem"-cri.a.do}{0}
\verb{bem"-disposto}{ô}{}{bem"-dispostos ⟨ó⟩}{bem"-disposta ⟨ó⟩}{adj.}{Diz"-se daquele que está em bom estado de ânimo e de saúde.}{bem"-dis.pos.to}{0}
\verb{bem"-dotado}{}{}{bem"-dotados}{}{adj.}{Que é cheio de dotes, prendas, aptidões.}{bem"-do.ta.do}{0}
\verb{bem"-educado}{}{}{bem"-educados}{}{adj.}{Diz"-se daquele que recebeu boa educação social; polido, cortês.}{bem"-e.du.ca.do}{0}
\verb{bem"-estar}{}{}{bem"-estares}{}{s.m.}{Estado de perfeita condição física ou moral.}{bem"-es.tar}{0}
\verb{bem"-estar}{}{}{bem"-estares}{}{}{Sensação de segurança; conforto, tranquilidade.}{bem"-es.tar}{0}
\verb{bem"-falante}{}{}{bem"-falantes}{}{adj.2g.}{Que fala bem, que é eloquente.}{bem"-fa.lan.te}{0}
\verb{bem"-falante}{}{}{bem"-falantes}{}{s.2g.}{Indivíduo que fala bem, com correção, com boa fluência.}{bem"-fa.lan.te}{0}
\verb{benfeito}{ê}{}{benfeitos}{}{adj.}{Que é feito com esmero; caprichado, bem"-acabado.}{ben.fei.to}{0}
\verb{benfeito}{ê}{}{benfeitos}{}{}{Que apresenta formas harmoniosas.}{ben.fei.to}{0}
\verb{benfeito}{ê}{}{benfeitos}{}{}{Que é elegante, gracioso.}{ben.fei.to}{0}
\verb{bem"-humorado}{}{}{bem"-humorados}{}{adj.}{Que tem ou está de bom humor, que goza de boa disposição de espírito.}{bem"-hu.mo.ra.do}{0}
\verb{bem"-intencionado}{}{}{bem"-intencionados}{}{adj.}{Diz"-se daquele que tem boas intenções ou propósitos verdadeiros.}{bem"-in.ten.ci.o.na.do}{0}
\verb{bem"-intencionado}{}{}{bem"-intencionados}{}{s.m.}{Indivíduo provido de boas intenções.}{bem"-in.ten.ci.o.na.do}{0}
\verb{bem"-me"-quer}{é}{Bot.}{bem"-me"-queres ⟨é⟩}{}{s.m.}{Erva de flores amarelas e frutos providos de pelos, também chamada de malmequer.}{bem"-me"-quer}{0}
\verb{bem"-nascido}{}{}{bem"-nascidos}{}{adj.}{Que nasceu de família boa ou nobre.}{bem"-nas.ci.do}{0}
\verb{bemol}{ó}{Mús.}{"-óis}{}{s.m.}{Sinal que indica o abaixamento de um semitom na altura de uma nota musical.}{be.mol}{0}
\verb{bem"-posto}{ô}{}{bem"-postos ⟨ó⟩}{bem"-posta ⟨ó⟩}{adj.}{Que é harmonioso, elegante nos movimentos, no deslocar"-se.}{bem"-pos.to}{0}
\verb{bem"-posto}{ô}{}{bem"-postos ⟨ó⟩}{bem"-posta ⟨ó⟩}{}{Que está vestido com elegância e boas roupas.}{bem"-pos.to}{0}
\verb{benquerer}{ê}{}{\textit{do s.m.:} benquereres ⟨ê⟩}{}{v.t.}{Querer bem a alguém ou algo; dedicar grande estima.}{ben.que.rer}{\verboinum{49}}
\verb{benquerer}{ê}{}{\textit{do s.m.:} benquereres ⟨ê⟩}{}{s.m.}{O querer bem; benquerença.}{ben.que.rer}{0}
\verb{benquerer}{ê}{}{\textit{do s.m.:} benquereres ⟨ê⟩}{}{}{Indivíduo a quem se ama; o bem"-amado.}{ben.que.rer}{0}
\verb{bem"-sucedido}{}{}{bem"-sucedidos}{}{adj.}{Que teve sucesso, bom êxito.}{bem"-su.ce.di.do}{0}
\verb{bem"-te"-vi}{}{Zool.}{bem"-te"-vis}{}{s.m.}{Ave largamente distribuída no Brasil, de bico longo e forte, dorso bege, abdômen amarelado, e cabeça preta e branca com uma mancha amarela no vértice.}{bem"-te"-vi}{0}
\verb{bem"-vindo}{}{}{bem"-vindos}{}{adj.}{Que é bem recebido, bem acolhido à chegada.}{bem"-vin.do}{0}
\verb{bem"-visto}{}{}{bem"-vistos}{}{adj.}{Que é tido em bom conceito; considerado.}{bem"-vis.to}{0}
\verb{bem"-visto}{}{}{bem"-vistos}{}{}{Que é querido, estimado por todos; benquisto.}{bem"-vis.to}{0}
\verb{bênção}{}{}{"-ãos}{}{s.f.}{Ato de benzer, de abençoar, de consagrar.}{bên.ção}{0}
\verb{bênção}{}{}{"-ãos}{}{}{Graça concedida por Deus.}{bên.ção}{0}
\verb{bênção}{}{}{"-ãos}{}{}{Palavras e sentimentos de gratidão.}{bên.ção}{0}
\verb{bendito}{}{}{}{}{adj.}{Diz"-se daquele ou daquilo a quem se abençoou; louvado.}{ben.di.to}{0}
\verb{bendito}{}{Relig.}{}{}{s.m.}{Oração que principia por aquela palavra.}{ben.di.to}{0}
\verb{bendizer}{ê}{}{}{}{v.t.}{Dizer bem; elogiar, enaltecer.}{ben.di.zer}{0}
\verb{bendizer}{ê}{}{}{}{}{Abençoar, louvar, glorificar.}{ben.di.zer}{\verboinum{41}}
\verb{beneditino}{}{Relig.}{}{}{s.m.}{Religioso pertencente à Ordem de São Bento.}{be.ne.di.ti.no}{0}
\verb{beneditino}{}{}{}{}{adj.}{Relativo a essa ordem ou a esse religioso.}{be.ne.di.ti.no}{0}
\verb{beneditino}{}{Por ext.}{}{}{}{Que é paciente, resignado.}{be.ne.di.ti.no}{0}
\verb{beneficência}{}{}{}{}{s.f.}{Ato de fazer o bem, de beneficiar o próximo}{be.ne.fi.cên.cia}{0}
\verb{beneficência}{}{}{}{}{}{Caridade, filantropia.}{be.ne.fi.cên.cia}{0}
\verb{beneficente}{}{}{}{}{adj.2g.}{Que traz benefício, que faz caridade.}{be.ne.fi.cen.te}{0}
\verb{beneficiado}{}{}{}{}{adj.}{Que recebeu benefício; favorecido.}{be.ne.fi.ci.a.do}{0}
\verb{beneficiado}{}{}{}{}{s.m.}{Beneficiário.}{be.ne.fi.ci.a.do}{0}
\verb{beneficiamento}{}{}{}{}{s.m.}{Ato ou efeito de beneficiar, de favorecer.}{be.ne.fi.ci.a.men.to}{0}
\verb{beneficiamento}{}{}{}{}{}{Benfeitoria ou reparo em propriedade.}{be.ne.fi.ci.a.men.to}{0}
\verb{beneficiamento}{}{}{}{}{}{Tratamento ao qual os produtos agrícolas são submetidos, para que se tornem próprios para consumo.}{be.ne.fi.ci.a.men.to}{0}
\verb{beneficiar}{}{}{}{}{v.t.}{Fazer benefício; favorecer.}{be.ne.fi.ci.ar}{0}
\verb{beneficiar}{}{}{}{}{}{Melhorar, reparar, consertar.}{be.ne.fi.ci.ar}{0}
\verb{beneficiar}{}{}{}{}{}{Submeter produtos agrícolas a processos destinados a dar"-lhes condições de serem consumidos.}{be.ne.fi.ci.ar}{\verboinum{1}}
\verb{beneficiário}{}{}{}{}{adj.}{Relativo a benefício.}{be.ne.fi.ci.á.rio}{0}
\verb{beneficiário}{}{}{}{}{}{Diz"-se daquele que recebe ou usufrui benefício ou vantagem.}{be.ne.fi.ci.á.rio}{0}
\verb{beneficiário}{}{}{}{}{s.m.}{Indivíduo que é favorecido por vantagem ou direito.}{be.ne.fi.ci.á.rio}{0}
\verb{benefício}{}{}{}{}{s.m.}{Ajuda que se recebe sem pagar por ela; favor, proveito, vantagem.}{be.ne.fí.cio}{0}
\verb{benefício}{}{}{}{}{}{Ajuda recebida por força de lei; auxílio.}{be.ne.fí.cio}{0}
\verb{benéfico}{}{}{}{}{adj.}{Que faz bem; salutar, favorável.}{be.né.fi.co}{0}
\verb{benéfico}{}{}{}{}{}{Que é bondoso; caridoso.}{be.né.fi.co}{0}
\verb{benemerência}{}{}{}{}{s.f.}{Qualidade de benemérito.}{be.ne.me.rên.cia}{0}
\verb{benemerente}{}{}{}{}{adj.2g.}{Que é merecedor de louvores e recompensas, por serviços relevantes prestados, ou por suas qualidades e virtudes.}{be.ne.me.ren.te}{0}
\verb{benemérito}{}{}{}{}{adj.}{Que é digno de honras, recompensas e aplausos, por serviços importantes ou por procedimento notável. }{be.ne.mé.ri.to}{0}
\verb{benemérito}{}{}{}{}{}{Que é muito distinto; ilustre.}{be.ne.mé.ri.to}{0}
\verb{beneplácito}{}{}{}{}{s.m.}{Expressão de consentimento; licença, aprovação, aprazimento. (\textit{Eles receberam o beneplácito do papa.})}{be.ne.plá.ci.to}{0}
\verb{benesse}{é}{}{}{}{s.2g.}{Rendimento paroquial.}{be.nes.se}{0}
\verb{benesse}{é}{Por ext.}{}{}{}{Aquilo que se doa; presente, dádiva.}{be.nes.se}{0}
\verb{benesse}{é}{Por ext.}{}{}{}{Vantagem ou lucro que não deriva de esforço ou trabalho.}{be.nes.se}{0}
\verb{benevolência}{}{}{}{}{s.f.}{Boa vontade para com alguém.}{be.ne.vo.lên.cia}{0}
\verb{benevolência}{}{}{}{}{}{Tolerância com inferiores.}{be.ne.vo.lên.cia}{0}
\verb{benevolência}{}{}{}{}{}{Manifestação de afeto; estima.}{be.ne.vo.lên.cia}{0}
\verb{benevolente}{}{}{}{}{adj.2g.}{Que tende a fazer o bem; bondoso, benévolo.}{be.ne.vo.len.te}{0}
\verb{benevolente}{}{}{}{}{}{Que é complacente, indulgente, tolerante, benigno.}{be.ne.vo.len.te}{0}
\verb{benévolo}{}{}{}{}{adj.}{Benevolente.}{be.né.vo.lo}{0}
\verb{benfazejo}{ê}{}{}{}{adj.}{Que faz o bem; caridoso.}{ben.fa.ze.jo}{0}
\verb{benfazejo}{ê}{}{}{}{}{Que tem ação favorável, benéfica ou útil.}{ben.fa.ze.jo}{0}
\verb{benfeitor}{ô}{}{}{}{s.m.}{Indivíduo que faz o bem.}{ben.fei.tor}{0}
\verb{benfeitor}{ô}{}{}{}{}{Indivíduo que faz benfeitoria.}{ben.fei.tor}{0}
\verb{benfeitor}{ô}{}{}{}{adj.}{Que tende a fazer o bem; bondoso, benévolo.}{ben.fei.tor}{0}
\verb{benfeitoria}{}{}{}{}{s.f.}{Obra realizada em propriedade para valorizá"-la.}{ben.fei.to.ri.a}{0}
\verb{bengala}{}{}{}{}{adj.2g. e s.2g.}{Bengali.}{ben.ga.la}{0}
\verb{bengala}{}{}{}{}{s.f.}{Bastão de madeira ou outro material, geralmente com a extremidade superior em forma de meio círculo, sobre o qual se apoia a mão ao andar.}{ben.ga.la}{0}
\verb{bengala}{}{Bras.}{}{}{}{Tipo de pão comprido; bisnaga.}{ben.ga.la}{0}
\verb{bengalada}{}{}{}{}{s.f.}{Bordoada, golpe dado com a bengala.}{ben.ga.la.da}{0}
\verb{bengalês}{}{}{}{}{adj.}{Relativo a Bangladesh; bengali.}{ben.ga.lês}{0}
\verb{bengalês}{}{}{}{}{s.m.}{Indivíduo natural ou habitante desse país.}{ben.ga.lês}{0}
\verb{bengali}{}{}{}{}{adj.2g. e s.2g.}{Bengalês.}{ben.ga.li}{0}
\verb{benignidade}{}{}{}{}{s.f.}{Qualidade de benigno.}{be.nig.ni.da.de}{0}
\verb{benigno}{}{}{}{}{}{Que é suave, brando, agradável.}{be.nig.no}{0}
\verb{benigno}{}{}{}{}{}{Que não apresenta gravidade, que não é perigoso nem maligno.}{be.nig.no}{0}
\verb{benigno}{}{}{}{}{adj.}{Que é bondoso, complacente, generoso.}{be.nig.no}{0}
\verb{beninense}{}{}{}{}{adj.2g.}{Relativo a Benin (África Ocidental).}{be.ni.nen.se}{0}
\verb{beninense}{}{}{}{}{s.2g.}{Indivíduo natural ou habitante desse país.}{be.ni.nen.se}{0}
\verb{benjamim}{}{}{"-ins}{}{s.m.}{O filho predileto, em geral o mais moço.}{ben.ja.mim}{0}
\verb{benjamim}{}{}{"-ins}{}{}{O membro mais jovem de uma agremiação.}{ben.ja.mim}{0}
\verb{benjamim}{}{}{"-ins}{}{}{Plugue ou extensão dupla ou tripla para tomadas elétricas.}{ben.ja.mim}{0}
\verb{benjoeiro}{ê}{Bot.}{}{}{s.m.}{Planta de flores alvas, cuja madeira é usada em obras internas e construção civil, e que produz uma resina aromática empregada em farmácia.}{ben.jo.ei.ro}{0}
\verb{benjoim}{}{}{}{}{s.m.}{Bálsamo aromático, amarelo, extraído do benjoeiro, utilizado na fabricação de perfumes e em medicina.}{ben.jo.im}{0}
\verb{benquerença}{}{}{}{}{s.f.}{O querer bem; estima, benevolência.}{ben.que.ren.ça}{0}
\verb{benquisto}{}{}{}{}{adj.}{Que é querido, estimado por todos.}{ben.quis.to}{0}
\verb{benquisto}{}{}{}{}{}{Que é bem considerado; bem"-visto.}{ben.quis.to}{0}
\verb{bens}{}{}{}{}{s.m.pl.}{O que é propriedade de alguém; posses.}{bens}{0}
\verb{bentinho}{}{}{}{}{s.m.}{Objeto que as pessoas devotas trazem ao pescoço, composto por dois saquinhos quadrados de pano, geralmente contendo orações escritas; escapulário.}{ben.ti.nho}{0}
\verb{bento}{}{}{}{}{adj.}{Que recebeu bênção eclesiástica; benzido.}{ben.to}{0}
\verb{benzedeira}{ê}{}{}{}{s.f.}{Mulher que pratica a benzedura para curar doenças e anular feitiços.}{ben.ze.dei.ra}{0}
\verb{benzedeiro}{ê}{}{}{}{s.m.}{Homem que pretende curar doenças e anular feitiços por meio de benzeduras.}{ben.ze.dei.ro}{0}
\verb{benzedura}{}{}{}{}{s.f.}{Ato de benzer, com ou sem o sinal"-da- cruz, acompanhado de orações e gestos característicos.}{ben.ze.du.ra}{0}
\verb{benzeno}{}{Quím.}{}{}{s.m.}{Líquido incolor, com cheiro característico, cuja molécula tem uma estrutura cíclica típica de hidrocarboneto aromático, usado como solvente e matéria"-prima para obtenção de outros compostos, como polímeros, detergentes, corantes etc. }{ben.ze.no}{0}
\verb{benzer}{ê}{}{}{}{v.t.}{Dizer palavras e fazer gestos para pedir a proteção de Deus para pessoa, animal ou coisa; abençoar.}{ben.zer}{0}
\verb{benzer}{ê}{}{}{}{}{Tratar um doente por meio de benzedura.}{ben.zer}{0}
\verb{benzer}{ê}{}{}{}{v.pron.}{Fazer sobre si mesmo o sinal"-da"-cruz.}{ben.zer}{\verboinum{12}}
\verb{benzina}{}{}{}{}{s.f.}{Benzeno impuro, vendido comercialmente como solvente industrial.}{ben.zi.na}{0}
\verb{benzina}{}{Quím.}{}{}{}{Éter muito volátil, resultante da destilação do petróleo.}{ben.zi.na}{0}
\verb{beócio}{}{}{}{}{adj.}{Relativo à Beócia, província da Grécia antiga.}{be.ó.cio}{0}
\verb{beócio}{}{Fig.}{}{}{}{Que é curto de inteligência; simplório.}{be.ó.cio}{0}
\verb{beócio}{}{}{}{}{s.m.}{Indivíduo natural ou habitante da Beócia.}{be.ó.cio}{0}
\verb{beócio}{}{}{}{}{}{O dialeto dessa província.}{be.ó.cio}{0}
\verb{beócio}{}{Fig.}{}{}{}{Indivíduo ignorante; simplório.}{be.ó.cio}{0}
\verb{bequadro}{}{Mús.}{}{}{s.m.}{Sinal gráfico que anula o efeito dos sustenidos e bemóis, e repõe no seu tom natural a nota elevada ou abaixada.}{be.qua.dro}{0}
\verb{beque}{é}{Esport.}{}{}{s.m.}{Jogador de defesa que ocupa a zaga; zagueiro. }{be.que}{0}
\verb{berçário}{}{}{}{}{s.m.}{Seção, nas maternidades, onde ficam os berços das crianças recém"-nascidas.}{ber.çá.rio}{0}
\verb{berço}{ê}{}{}{}{s.m.}{Cama pequena para criança de colo.}{ber.ço}{0}
\verb{berço}{ê}{}{}{}{}{Lugar onde alguma pessoa ou coisa teve origem, donde procede. (\textit{A Grécia é o berço da civilização ocidental.})}{ber.ço}{0}
\verb{bergamota}{ó}{}{}{}{s.f.}{Variedade de pera aromática e que tem muito caldo.}{ber.ga.mo.ta}{0}
\verb{bergamota}{ó}{}{}{}{}{Tangerina.}{ber.ga.mo.ta}{0}
\verb{bergantim}{}{}{"-ins}{}{s.m.}{Antiga embarcação a vela e remo, esguia e veloz, com dois mastros.}{ber.gan.tim}{0}
\verb{beribéri}{}{Med.}{}{}{s.m.}{Doença decorrente da deficiência de vitamina B1, que apresenta inflamação de vários nervos, cardiopatia e edema.}{be.ri.bé.ri}{0}
\verb{berílio}{}{Quím.}{}{}{s.m.}{Elemento químico metálico, cinzento"-azulado, radioativo, do grupo dos alcalino"-terrosos, utilizado em diversas ligas leves, como componente de aeronaves, em reatores nucleares etc. \elemento{4}{9.012182}{Be}.}{be.rí.lio}{\verboinum{1}}
\verb{berilo}{}{}{}{}{s.m.}{Pedra semipreciosa e o mais abundante dos minerais de berílio.}{be.ri.lo}{0}
\verb{berimbau}{}{}{}{}{s.m.}{Instrumento musical de percussão, composto de um aro com fio de arame e de uma cabaça na parte de baixo.}{be.rim.bau}{0}
\verb{berinjela}{é}{Bot.}{}{}{s.f.}{Planta ornamental, originária da Índia, dotada de flores roxas, e cujo fruto, que são bagas carnudas com forma oval e cilíndrica, tem largo emprego na alimentação humana.}{be.rin.je.la}{0}
\verb{berinjela}{é}{}{}{}{}{O fruto dessa planta.}{be.rin.je.la}{0}
\verb{berlinda}{}{}{}{}{s.f.}{Pequena carruagem de quatro rodas e vidraças laterais, com quatro ou seis lugares, suspensa por molas.}{ber.lin.da}{0}
\verb{berlinda}{}{}{}{}{}{Pequeno oratório envidraçado para imagens de santos.}{ber.lin.da}{0}
\verb{berlinense}{}{}{}{}{adj.2g.}{Relativo a Berlim (Alemanha).}{ber.li.nen.se}{0}
\verb{berlinense}{}{}{}{}{s.2g.}{Indivíduo natural ou habitante dessa cidade.}{ber.li.nen.se}{0}
\verb{berloque}{ó}{}{}{}{s.m.}{Pequeno enfeite delicado, de matéria e formas variadas, de pouco valor material, que se traz pendente da corrente do relógio, da pulseira etc.; penduricalho, pingente.}{ber.lo.que}{0}
\verb{bermuda}{}{}{}{}{s.f.}{Tipo de calça curta, que se estende quase até os joelhos.}{ber.mu.da}{0}
\verb{bernarda}{}{Hist.}{}{}{s.f.}{Movimento revolucionário ocorrido em Braga (Portugal) em 1862.}{ber.nar.da}{0}
\verb{bernarda}{}{}{}{}{}{Revolta popular; motim, desordem.}{ber.nar.da}{0}
\verb{bernarda}{}{}{}{}{}{Variedade de pera.}{ber.nar.da}{0}
\verb{berne}{é}{}{}{}{s.m.}{Larva cujo desenvolvimento final se processa debaixo da pele de mamíferos, incluindo o homem, após ser depositada por mosca ou mosquito que lhe serve de hospedeiro numa fase intermediária.}{ber.ne}{0}
\verb{berne}{é}{}{}{}{}{Tumor subcutâneo produzido em animais e no homem por ação dessa larva.}{ber.ne}{0}
\verb{berquélio}{}{Quím.}{}{}{s.m.}{Elemento químico radioativo, do grupo dos actinídeos, obtido artificialmente. \elemento{97}{(247)}{Bk}.}{ber.qué.lio}{0}
\verb{berrador}{ô}{}{}{}{adj.}{Que berra muito.}{ber.ra.dor}{0}
\verb{berrante}{}{}{}{}{adj.2g.}{Que berra.}{ber.ran.te}{0}
\verb{berrante}{}{}{}{}{}{Diz"-se de cor muito viva ou que chama muito a atenção.}{ber.ran.te}{0}
\verb{berrante}{}{}{}{}{s.m.}{Corneta de chifre com que os boiadeiros tangem o gado.}{ber.ran.te}{0}
\verb{berrar}{}{}{}{}{v.i.}{Soltar berros.}{ber.rar}{0}
\verb{berrar}{}{}{}{}{}{Falar muito alto; gritar.}{ber.rar}{0}
\verb{berrar}{}{}{}{}{}{Chorar alto e forte.}{ber.rar}{0}
\verb{berrar}{}{}{}{}{v.t.}{Pedir com muita instância.}{ber.rar}{\verboinum{1}}
\verb{berreiro}{ê}{}{}{}{s.m.}{Berros contínuos e altos.}{ber.rei.ro}{0}
\verb{berreiro}{ê}{}{}{}{}{Choro muito ruidoso.}{ber.rei.ro}{0}
\verb{berro}{é}{}{}{}{s.m.}{Grito de certos animais; rugido.}{ber.ro}{0}
\verb{berro}{é}{}{}{}{}{Grito rude e alto de uma pessoa.}{ber.ro}{0}
\verb{berruga}{}{}{}{}{}{Var. de \textit{verruga}.}{ber.ru.ga}{0}
\verb{bertalha}{}{Bot.}{}{}{s.f.}{Planta trepadeira, suculenta, mole e rica em água, muito cultivada como hortaliça, de flores esverdeadas, e cujos frutos são bagas negras.}{ber.ta.lha}{0}
\verb{besouro}{ô}{Zool.}{}{}{s.m.}{Inseto cujas asas anteriores são  córneas, possui corpo duro e aparelho bucal mastigador.}{be.sou.ro}{0}
\verb{besta}{é}{}{}{}{s.f.}{Arma antiga formada de arco, cabo e corda, com que se disparavam pelouros ou setas.}{bes.ta}{0}
\verb{besta}{ê}{Zool.}{}{}{s.f.}{Animal quadrúpede de grande porte, geralmente destinado ao transporte de carga.}{bes.ta}{0}
\verb{besta}{ê}{}{}{}{s.2g.}{Indivíduo tolo ou estúpido; ignorante.}{bes.ta}{0}
\verb{besta}{ê}{}{}{}{}{Indivíduo pretensioso, pedante.}{bes.ta}{0}
\verb{besta}{ê}{}{}{}{adj.}{Que é ingênuo, tolo.}{bes.ta}{0}
\verb{besta}{ê}{}{}{}{}{Que é pedante, presunçoso.}{bes.ta}{0}
\verb{besta"-fera}{ê\ldots{}é}{}{bestas"-feras ⟨ê\ldots{}é⟩}{}{s.2g.}{Animal feroz.}{bes.ta"-fe.ra}{0}
\verb{besta"-fera}{ê\ldots{}é}{Fig.}{bestas"-feras ⟨ê\ldots{}é⟩}{}{}{Indivíduo mau, desumano.}{bes.ta"-fe.ra}{0}
\verb{bestalhão}{}{}{"-ões}{"-ona}{adj.}{Que é ignorante, tolo, rústico.}{bes.ta.lhão}{0}
\verb{bestalhão}{}{}{"-ões}{"-ona}{s.m.}{Indivíduo inculto, parvo, paspalhão.}{bes.ta.lhão}{0}
\verb{bestar}{}{}{}{}{v.i.}{Dizer besteiras, tolices, asneiras.}{bes.tar}{0}
\verb{bestar}{}{}{}{}{}{Andar sem destino.}{bes.tar}{0}
\verb{bestar}{}{}{}{}{}{Estar ocioso.}{bes.tar}{\verboinum{1}}
\verb{besteira}{ê}{}{}{}{s.f.}{Bobagem, asneira, tolice.}{bes.tei.ra}{0}
\verb{besteira}{ê}{}{}{}{}{Quantia insignificante; ninharia.}{bes.tei.ra}{0}
\verb{bestial}{}{}{"-ais}{}{adj.2g.}{Relativo a besta.}{bes.ti.al}{0}
\verb{bestial}{}{}{"-ais}{}{}{Grosseiro, brutal, repugnante.}{bes.ti.al}{0}
\verb{bestialidade}{}{}{}{}{s.f.}{Comportamento que torna o homem semelhante ao animal irracional; brutalidade, estupidez.}{bes.ti.a.li.da.de}{0}
\verb{bestialidade}{}{}{}{}{}{Prática de atos libidinosos com animais.}{bes.ti.a.li.da.de}{0}
\verb{bestializar}{}{}{}{}{v.t.}{Tornar semelhante ao animal irracional; bestificar, animalizar.}{bes.ti.a.li.zar}{\verboinum{1}}
\verb{bestialógico}{}{Pop.}{}{}{adj.}{Relativo a tolices e asneiras; asneirento.}{bes.ti.a.ló.gi.co}{0}
\verb{bestialógico}{}{Pop.}{}{}{s.m.}{Discurso ou escrito sem nexo, cheio de absurdos, disparatado.}{bes.ti.a.ló.gi.co}{0}
\verb{bestice}{}{Pop.}{}{}{s.f.}{Ato ou dito impensado; besteira, asneira, tolice.}{bes.ti.ce}{0}
\verb{bestificado}{}{}{}{}{adj.}{Espantado, pasmado, embasbacado, boquiaberto.}{bes.ti.fi.ca.do}{0}
\verb{bestificar}{}{}{}{}{v.t.}{Causar admiração; pasmar, embasbacar.}{bes.ti.fi.car}{0}
\verb{bestificar}{}{}{}{}{}{Tornar estúpido, tolo.}{bes.ti.fi.car}{0}
\verb{bestificar}{}{}{}{}{}{Tornar semelhante a animal irracional; bestializar, brutalizar.}{bes.ti.fi.car}{\verboinum{2}}
\verb{best"-seller}{}{}{}{}{s.m.}{Livro ou autor que é sucesso de vendas.}{\textit{best"-seller}}{0}
\verb{bestunto}{}{Pop.}{}{}{s.m.}{Inteligência curta; cabeça de pouco juízo.}{bes.tun.to}{0}
\verb{besuntar}{}{}{}{}{v.t.}{Untar muito; sujar de gordura; lambuzar.}{be.sun.tar}{\verboinum{1}}
\verb{beta}{é}{}{}{}{s.m.}{Segunda letra do alfabeto grego.}{be.ta}{0}
\verb{beterraba}{}{Bot.}{}{}{s.f.}{Planta herbácea, cujas raízes são tubérculos comestíveis e dos quais se extrai açúcar semelhante ao da cana.}{be.ter.ra.ba}{0}
\verb{beterraba}{}{}{}{}{}{O tubérculo comestível dessa planta.}{be.ter.ra.ba}{0}
\verb{betoneira}{ê}{}{}{}{s.f.}{Máquina que consiste em um grande recipiente giratório empregado no preparo de concreto.}{be.to.nei.ra}{0}
\verb{betume}{}{Quím.}{}{}{s.m.}{Mistura líquida, viscosa e escura, composta de hidrocarbonetos pesados, usada especialmente como impermeabilizante, na pavimentação de estradas, na fabricação de borrachas etc.; asfalto.}{be.tu.me}{0}
\verb{betuminoso}{ô}{}{"-osos ⟨ó⟩}{"-osa ⟨ó⟩}{adj.}{Que contém betume.}{be.tu.mi.no.so}{0}
\verb{bexiga}{ch}{}{}{}{s.f.}{Balão de borracha, inflável e colorido, utilizado como enfeite ou brinquedo.}{be.xi.ga}{0}
\verb{bexiga}{ch}{Anat.}{}{}{}{Órgão musculo"-membranoso que recebe urina pelos ureteres; bexiga urinária.}{be.xi.ga}{0}
\verb{bexiga}{ch}{Med.}{}{}{}{Nome popular da varíola ou marca no rosto deixada por ela.}{be.xi.ga}{0}
\verb{bexiguento}{ch}{Pop.}{}{}{adj.}{Diz"-se daquele que está com varíola ou tem marcas no rosto deixadas por ela.}{be.xi.guen.to}{0}
\verb{bezerro}{ê}{}{}{}{s.m.}{Filhote de vaca, ainda em fase de amamentação; novilho, vitelo.}{be.zer.ro}{0}
\verb{bezerro}{ê}{}{}{}{}{O couro curtido desse animal, usado na confecção de calçados.}{be.zer.ro}{0}
\verb{Bi}{}{Quím.}{}{}{}{Símb. do \textit{bismuto}.}{Bi}{0}
\verb{bibelô}{}{}{}{}{s.m.}{Pequeno objeto usado como enfeite sobre móveis.}{bi.be.lô}{0}
\verb{bibelô}{}{}{}{}{}{Objeto sem utilidade nem valor.}{bi.be.lô}{0}
\verb{bibelô}{}{Fig.}{}{}{}{Pessoa bonita ou delicada.}{bi.be.lô}{0}
\verb{bíblia}{}{Relig.}{}{}{s.f.}{Conjunto dos livros sagrados do cristianismo, que contém o Antigo e o Novo Testamentos. (Usa"-se maiúscula nesta acepção.)}{bí.blia}{0}
\verb{bíblia}{}{Fig.}{}{}{}{Obra em que se reconhecem autoridade e importância no assunto versado.}{bí.blia}{0}
\verb{bíblico}{}{}{}{}{adj.}{Relativo à Bíblia ou que evoca características de época ou dos lugares em que ocorreram os fatos nela narrados.}{bí.bli.co}{0}
\verb{bibliofilia}{}{}{}{}{s.f.}{Arte de colecionar livros, especialmente os raros e preciosos.}{bi.bli.o.fi.li.a}{0}
\verb{bibliófilo}{}{}{}{}{s.m.}{Colecionador de livros raros e preciosos.}{bi.bli.ó.fi.lo}{0}
\verb{bibliografia}{}{}{}{}{s.f.}{Relação de obras recomendadas ou citadas por um autor na criação de determinado texto.}{bi.bli.o.gra.fi.a}{0}
\verb{bibliografia}{}{}{}{}{}{Registro dos livros escritos sobre determinado assunto ou autor.}{bi.bli.o.gra.fi.a}{0}
\verb{bibliografia}{}{}{}{}{}{Estudo sobre a história, descrição e classificação dos livros.}{bi.bli.o.gra.fi.a}{0}
\verb{bibliográfico}{}{}{}{}{adj.}{Relativo a bibliografia ou a livros.}{bi.bli.o.grá.fi.co}{0}
\verb{bibliógrafo}{}{}{}{}{s.m.}{Indivíduo versado no conhecimento de livros ou de edições.}{bi.bli.ó.gra.fo}{0}
\verb{bibliógrafo}{}{}{}{}{}{Compilador de obras bibliográficas.}{bi.bli.ó.gra.fo}{0}
\verb{bibliomania}{}{}{}{}{s.f.}{Mania de comprar e colecionar livros.}{bi.bli.o.ma.ni.a}{0}
\verb{biblioteca}{é}{}{}{}{s.f.}{Coleção de livros classificados em determinada ordem.}{bi.bli.o.te.ca}{0}
\verb{biblioteca}{é}{}{}{}{}{Edifício ou recinto onde se instala essa coleção para consulta.}{bi.bli.o.te.ca}{0}
\verb{bibliotecário}{}{}{}{}{s.m.}{Indivíduo que administra uma biblioteca.}{bi.bli.o.te.cá.rio}{0}
\verb{bibliotecário}{}{}{}{}{}{Membro do corpo de profissionais que trabalham em uma biblioteca.}{bi.bli.o.te.cá.rio}{0}
\verb{biblioteconomia}{}{}{}{}{s.f.}{Ciência e técnica que trata da armazenagem, do acesso e da circulação dos livros de uma biblioteca.}{bi.bli.o.te.co.no.mi.a}{0}
\verb{biblioteconomista}{}{}{}{}{s.2g.}{Especialista em biblioteconomia.}{bi.bli.o.te.co.no.mis.ta}{0}
\verb{biboca}{ó}{}{}{}{s.f.}{Habitação humilde e pequena, coberta de palha ou de sapê.}{bi.bo.ca}{0}
\verb{biboca}{ó}{}{}{}{}{Lugar ou vale distante e de difícil acesso.}{bi.bo.ca}{0}
\verb{biboca}{ó}{}{}{}{}{Escavação ou fenda de terreno; cova, buraco.}{bi.bo.ca}{0}
\verb{bica}{}{}{}{}{s.f.}{Calha ou cano por onde corre e cai a água.}{bi.ca}{0}
\verb{bica}{}{}{}{}{}{Fonte ou veio de água corrente.}{bi.ca}{0}
\verb{bicada}{}{}{}{}{s.f.}{Golpe ou picada desferida com o bico.}{bi.ca.da}{0}
\verb{bicada}{}{}{}{}{}{Aquilo que a ave carrega no bico de uma só vez.}{bi.ca.da}{0}
\verb{bicada}{}{Pop.}{}{}{}{Pequeno gole, geralmente de bebida alcoólica; trago.}{bi.ca.da}{0}
\verb{bicama}{}{}{}{}{s.f.}{Cama de altura normal que, na sua parte inferior, traz uma outra, embutida, a qual desliza para fora quando necessário.}{bi.ca.ma}{0}
\verb{bicampeão}{}{}{"-ões}{"-ã}{adj.}{Diz"-se do indivíduo, equipe ou clube campeão duas vezes.}{bi.cam.pe.ão}{0}
\verb{bicão}{}{}{"-ões}{}{s.m.}{Indivíduo que se intromete nos assuntos alheios; bisbilhoteiro, aproveitador.}{bi.cão}{0}
\verb{bicar}{}{}{}{}{v.t.}{Picar com o bico; dar bicadas.}{bi.car}{0}
\verb{bicar}{}{}{}{}{}{Beber em pequenos goles; bebericar.}{bi.car}{\verboinum{2}}
\verb{bicarbonato}{}{Quím.}{}{}{s.m.}{Nome comum dos sais que contenham dois ânions de ácido carbônico por um de uma substância básica.}{bi.car.bo.na.to}{0}
\verb{bicentenário}{}{}{}{}{adj.}{Que tem duzentos anos.}{bi.cen.te.ná.rio}{0}
\verb{bicentenário}{}{}{}{}{s.m.}{O segundo centenário de algum acontecimento.}{bi.cen.te.ná.rio}{0}
\verb{bíceps}{}{Anat.}{}{}{s.m.}{Nome de dois músculos do corpo, um da perna e outro do antebraço, ambos com duas porções na parte superior.}{bí.ceps}{0}
\verb{bicha}{}{}{}{}{s.f.}{Animal comprido e sem pernas; larva, verme, lombriga.}{bi.cha}{0}
\verb{bicha}{}{Bras.}{}{}{}{Homem efeminado.}{bi.cha}{0}
\verb{bicha}{}{Lus.}{}{}{}{Fila.}{bi.cha}{0}
\verb{bichado}{}{}{}{}{adj.}{Diz"-se do alimento atacado por  insetos ou vermes; estragado.}{bi.cha.do}{0}
\verb{bichado}{}{}{}{}{}{Diz"-se da pessoa atacada por lombrigas e que apresenta apetite insaciável.}{bi.cha.do}{0}
\verb{bichano}{}{}{}{}{s.m.}{Gato novo ou manso, geralmente doméstico.}{bi.cha.no}{0}
\verb{bichar}{}{}{}{}{v.i.}{Encher"-se ou infestar"-se de insetos, larvas ou lagartas.}{bi.char}{\verboinum{1}}
\verb{bicharada}{}{}{}{}{s.f.}{Grupo de animais, de bichos; bicharia.}{bi.cha.ra.da}{0}
\verb{bicharia}{}{}{}{}{s.f.}{Grande número de bichos, de animais; bicharada.}{bi.cha.ri.a}{0}
\verb{bicheira}{ê}{}{}{}{s.f.}{Feridas nos animais, infestadas de bichos ou vermes, em geral de larvas de moscas"-varejeiras.}{bi.chei.ra}{0}
\verb{bicheiro}{ê}{}{}{}{s.m.}{No jogo do bicho, indivíduo que registra e recebe as apostas.}{bi.chei.ro}{0}
\verb{bicheiro}{ê}{}{}{}{}{Indivíduo que banca as apostas desse jogo.}{bi.chei.ro}{0}
\verb{bicheiro}{ê}{}{}{}{}{Recipiente onde se guardam bichas ou sanguessugas.}{bi.chei.ro}{0}
\verb{bicho}{}{}{}{}{s.m.}{Qualquer animal terrestre.}{bi.cho}{0}
\verb{bicho}{}{}{}{}{}{Diz"-se de alguns insetos ou vermes, como o cupim, a traça ou larvas de certas moscas quando atacam madeira, livros, frutas etc. (\textit{Essa goiaba está cheia de bichos.})}{bi.cho}{0}
\verb{bicho}{}{Pop.}{}{}{}{Calouro.}{bi.cho}{0}
\verb{bicho}{}{Pop.}{}{}{}{Jogo do bicho.}{bi.cho}{0}
\verb{bicho}{}{}{}{}{}{Gratificação oferecida aos jogadores e ao técnico por um resultado favorável.}{bi.cho}{0}
\verb{bicho"-cabeludo}{}{Zool.}{bichos"-cabeludos}{}{s.m.}{Lagarta peluda; taturana.}{bi.cho"-ca.be.lu.do}{0}
\verb{bicho"-carpinteiro}{ê}{Zool.}{bichos"-carpinteiros \textit{ou} bichos"-carpinteiro}{}{s.m.}{Nome popular de determinados besouros que roem a madeira onde vivem durante a fase larvar; escaravelho.}{bi.cho"-car.pin.tei.ro}{0}
\verb{bicho"-da"-seda}{ê}{Zool.}{bichos"-da"-seda ⟨ê⟩}{}{s.m.}{Espécime de lagarta cuja larva produz um fio de seda na fabricação do casulo.}{bi.cho"-da"-se.da}{0}
\verb{bicho"-de"-pé}{}{}{bichos"-de"-pé}{}{s.m.}{Inseto cuja fêmea fecundada penetra na pele de homens e animais, principalmente na região dos pés, onde deposita os ovos.}{bi.cho"-de"-pé}{0}
\verb{bicho"-de"-sete"-cabeças}{é\ldots{}ê}{Pop.}{bichos"-de"-sete"-cabeças ⟨é\ldots{}ê⟩}{}{s.m.}{Coisa de difícil resolução, muito complicada.}{bi.cho"-de"-se.te"-ca.be.ças}{0}
\verb{bicho"-de"-sete"-cabeças}{é\ldots{}ê}{}{bichos"-de"-sete"-cabeças ⟨é\ldots{}ê⟩}{}{}{Coisa que amedronta.}{bi.cho"-de"-se.te"-ca.be.ças}{0}
\verb{bicho"-do"-mato}{}{Pop.}{bichos"-do"-mato}{}{s.m.}{Indivíduo arredio, retraído, que foge ao convívio social.}{bi.cho"-do"-ma.to}{0}
\verb{bicho"-do"-pé}{}{}{}{}{}{Var. de \textit{bicho"-de"-pé}.}{bi.cho"-do"-pé}{0}
\verb{bicho"-papão}{}{}{bichos"-papões}{}{s.m.}{Criação monstruosa do imaginário popular com que se assustam as crianças; ogro.}{bi.cho"-pa.pão}{0}
\verb{bicho"-pau}{}{Zool.}{bichos"-paus \textit{ou } bichos"-pau}{}{s.m.}{Nome comum dado a insetos cujo corpo lembra gravetos e folhas.}{bi.cho"-pau}{0}
\verb{bicicleta}{é}{}{}{}{s.f.}{Veículo leve, de duas rodas iguais, sendo a de trás acionada por uma corrente ligada ao pedal do veículo.}{bi.ci.cle.ta}{0}
\verb{bicicleta}{é}{Esport.}{}{}{}{No futebol, lance em que o jogador salta e, no ar, chuta a bola por cima da própria cabeça.}{bi.ci.cle.ta}{0}
\verb{bico}{}{}{}{}{s.m.}{Extremidade da boca das aves e de outros animais, própria para a alimentação.}{bi.co}{0}
\verb{bico}{}{}{}{}{}{Ponta ou extremidade aguda de vários objetos.}{bi.co}{0}
\verb{bico}{}{Fig.}{}{}{}{Boca.}{bi.co}{0}
\verb{bico}{}{Fig.}{}{}{}{Emprego temporário, passageiro; biscate.}{bi.co}{0}
\verb{bico"-de"-lacre}{}{Zool.}{bicos"-de"-lacre}{}{s.m.}{Ave de coloração parda e bico de cor vermelha intensa, bastante comum na região sudeste do Brasil.}{bi.co"-de"-la.cre}{0}
\verb{bico"-de"-papagaio}{}{Med.}{bicos"-de"-papagaio}{}{s.m.}{Saliência óssea na coluna vertebral, responsável por dores e fenômenos reflexos.}{bi.co"-de"-pa.pa.gai.o}{0}
\verb{bico"-de"-papagaio}{}{Bot.}{bicos"-de"-papagaio}{}{}{Planta ornamental de flores vermelhas, brancas e amarelas, que fornece tintura com propriedades medicinais.}{bi.co"-de"-pa.pa.gai.o}{0}
\verb{bico"-de"-papagaio}{}{}{bicos"-de"-papagaio}{}{}{Nariz curvo, aquilino.}{bi.co"-de"-pa.pa.gai.o}{0}
\verb{bicolor}{ô}{}{}{}{adj.2g.}{Que tem duas cores.}{bi.co.lor}{0}
\verb{bicôncavo}{}{}{}{}{adj.}{Que apresenta dois lados côncavos.}{bi.côn.ca.vo}{0}
\verb{biconvexo}{écs}{}{}{}{adj.}{Que apresenta dois lados convexos.}{bi.con.ve.xo}{0}
\verb{bicota}{ó}{Bras.}{}{}{s.f.}{Beijo estalado; beijinho; beijoca.}{bi.co.ta}{0}
\verb{bicudo}{}{}{}{}{adj.}{Que possui bico; bical.}{bi.cu.do}{0}
\verb{bicudo}{}{}{}{}{}{Com ponta(s); pontiagudo, aguçado.}{bi.cu.do}{0}
\verb{bicudo}{}{Pop.}{}{}{}{Complicado, árduo, difícil.}{bi.cu.do}{0}
\verb{bicudo}{}{Pop.}{}{}{}{Zangado, acabrunhado, amuado.}{bi.cu.do}{0}
\verb{bidé}{}{}{}{}{}{Var. de \textit{bidê}.}{bi.dé}{0}
\verb{bidê}{}{}{}{}{s.m.}{Aparelho sanitário em forma de bacia oblonga, usado para lavar a parte inferior do tronco.}{bi.dê}{0}
\verb{bidimensional}{}{}{"-ais}{}{adj.2g.}{Que apresenta duas dimensões.}{bi.di.men.si.o.nal}{0}
\verb{biela}{é}{}{}{}{s.f.}{Peça de máquina cuja função é transformar o movimento retilíneo alternado em movimento circular contínuo.}{bi.e.la}{0}
\verb{bielo"-russo}{é}{}{bielo"-russos ⟨é⟩}{}{adj.}{Relativo à Bielo"-Rússia.}{bi.e.lo"-rus.so}{0}
\verb{bielo"-russo}{é}{}{bielo"-russos ⟨é⟩}{}{s.m.}{Indivíduo natural ou habitante desse país.}{bi.e.lo"-rus.so}{0}
\verb{bielo"-russo}{é}{}{bielo"-russos ⟨é⟩}{}{}{A língua falada pelos bielo"-russos.}{bi.e.lo"-rus.so}{0}
\verb{bienal}{}{}{"-ais}{}{adj.2g.}{Que dura dois anos.}{bi.e.nal}{0}
\verb{bienal}{}{}{"-ais}{}{}{Que acontece ou é realizado a cada dois anos.}{bi.e.nal}{0}
\verb{bienal}{}{}{"-ais}{}{s.f.}{Exposição realizada a cada dois anos.}{bi.e.nal}{0}
\verb{biênio}{}{}{}{}{s.m.}{Período de dois anos seguidos.}{bi.ê.nio}{0}
\verb{bife}{}{}{}{}{s.m.}{Qualquer fatia ou pedaço de carne, frita, grelhada ou cozida, geralmente temperada, e servida como alimento.}{bi.fe}{0}
\verb{bífido}{}{}{}{}{adj.}{Que é dividido ou fendido em duas porções; bipartido.}{bí.fi.do}{0}
\verb{bifocal}{}{}{"-ais}{}{adj.2g.}{Que tem dois focos.}{bi.fo.cal}{0}
\verb{bifocal}{}{}{"-ais}{}{}{Diz"-se de lente dotada de duas distâncias focais, uma que corrige a visão a distância, e a outra, que corrige a visão próxima.}{bi.fo.cal}{0}
\verb{biforme}{ó}{}{}{}{adj.2g.}{Que apresenta duas formas.}{bi.for.me}{0}
\verb{bifurcação}{}{}{ões}{}{s.f.}{Ato ou efeito de bifurcar; bifurcamento.}{bi.fur.ca.ção}{0}
\verb{bifurcar}{}{}{}{}{v.t.}{Dividir, separar em dois ramos ou partes.}{bi.fur.car}{\verboinum{2}}
\verb{biga}{}{}{}{}{s.f.}{Entre os romanos da Antiguidade, carro de duas ou quatro rodas puxado por dois cavalos.}{bi.ga}{0}
\verb{bigamia}{}{}{}{}{s.f.}{Estado ou condição de bígamo.}{bi.ga.mi.a}{0}
\verb{bígamo}{}{}{}{}{adj.}{Diz"-se daquele que tem dois cônjuges simultaneamente.}{bí.ga.mo}{0}
\verb{bigode}{ó}{}{}{}{s.m.}{Conjunto de pelos que cresce sobre o lábio superior.}{bi.go.de}{0}
\verb{bigodear}{}{}{}{}{v.t.}{Pregar uma peça; enganar, iludir.}{bi.go.de.ar}{0}
\verb{bigodear}{}{}{}{}{}{Fazer pouco"-caso; escarnecer, caçoar.}{bi.go.de.ar}{\verboinum{4}}
\verb{bigodeira}{ê}{}{}{}{s.f.}{Bigode grande, espesso.}{bi.go.dei.ra}{0}
\verb{bigorna}{ó}{}{}{}{s.f.}{Ferramenta sobre a qual se malham e se amoldam metais.}{bi.gor.na}{0}
\verb{bigorna}{ó}{Anat.}{}{}{}{Pequeno osso que faz parte do ouvido médio.}{bi.gor.na}{0}
\verb{bigorrilha}{}{Pop.}{}{}{s.m.}{Indivíduo desprezível, baixo, vil; joão"-ninguém.}{bi.gor.ri.lha}{0}
\verb{biguá}{}{Zool.}{}{}{s.m.}{Nome comum às aves pelicaniformes, aquáticas, de distribuição mundial; corvo"-marinho, mergulhão.}{bi.guá}{0}
\verb{biju}{}{Cul.}{}{}{s.m.}{Bolo de massa de tapioca ou de mandioca.  }{bi.ju}{0}
\verb{bijuteria}{}{}{}{}{s.f.}{Ramo da ourivesaria que se utiliza de ligas de metal que imitam o ouro ou a prata, ou se vale de pedras semipreciosas, para produzir imitações de joias preciosas ou peças de fantasia e enfeites.}{bi.ju.te.ri.a}{0}
\verb{bijuteria}{}{}{}{}{}{As peças ou objetos assim fabricados.}{bi.ju.te.ri.a}{0}
\verb{bilabiado}{}{}{}{}{adj.}{Que apresenta dois lábios.}{bi.la.bi.a.do}{0}
\verb{bilabial}{}{Gram.}{"-ais}{}{adj.2g.}{Diz"-se da consoante em que a obstrução à passagem do ar resulta do movimento de um lábio contra o outro.}{bi.la.bi.al}{0}
\verb{bilateral}{}{}{"-ais}{}{adj.2g.}{Que apresenta dois lados, ou lados opostos.}{bi.la.te.ral}{0}
\verb{bilateral}{}{Jur.}{"-ais}{}{}{Diz"-se do contrato em que as obrigações são recíprocas para ambas as partes.}{bi.la.te.ral}{0}
\verb{bilboquê}{}{}{}{}{s.m.}{Brinquedo que consiste numa bola de madeira com um furo, amarrada por um cordel a um bastonete, no qual aquela se encaixa quando impulsionada.  }{bil.bo.quê}{0}
\verb{bile}{}{}{}{}{s.f.}{Bílis.}{bi.le}{0}
\verb{bilha}{}{}{}{}{s.f.}{Vaso bojudo, de gargalo estreito, geralmente de barro, usado para guardar líquidos; moringa.}{bi.lha}{0}
\verb{bilhão}{}{}{"-ões}{}{num.}{Mil milhões.}{bi.lhão}{0}
\verb{bilhar}{}{}{}{}{s.m.}{Jogo praticado com três bolas que se impulsionam com um taco sobre uma mesa forrada de pano verde.}{bi.lhar}{0}
\verb{bilhar}{}{}{}{}{}{Casa ou mesa em que se pratica esse jogo.}{bi.lhar}{0}
\verb{bilhete}{ê}{}{}{}{s.m.}{Pequena mensagem escrita, geralmente numa tira de papel.}{bi.lhe.te}{0}
\verb{bilhete}{ê}{}{}{}{}{Carta breve e informal.}{bi.lhe.te}{0}
\verb{bilhete}{ê}{}{}{}{}{Senha ou cartão de ingresso em espetáculos, transportes públicos, jogos etc.}{bi.lhe.te}{0}
\verb{bilhete}{ê}{}{}{}{}{Cédula de loteria, rifa etc.}{bi.lhe.te}{0}
\verb{bilheteiro}{ê}{}{}{}{s.m.}{Indivíduo que vende ingressos de espetáculos públicos, bilhetes de loteria, passagens de transportes etc.}{bi.lhe.tei.ro}{0}
\verb{bilheteria}{}{}{}{}{s.f.}{Local ou guichê onde se vendem ingressos para espetáculos públicos, passagens de transportes etc.}{bi.lhe.te.ri.a}{0}
\verb{bilião}{}{}{}{}{}{Var. de \textit{bilhão}.}{bi.li.ão}{0}
\verb{biliar}{}{}{}{}{adj.2g.}{Relativo à bílis.}{bi.li.ar}{0}
\verb{bilíngue}{}{}{}{}{adj.2g.}{Que fala duas línguas.}{bi.lín.gue}{0}
\verb{bilíngue}{}{}{}{}{}{Diz"-se do texto apresentado em duas línguas. (\textit{Tenho uma edição bilíngue da \textnormal{Odisseia}.})}{bi.lín.gue}{0}
\verb{bilíngue}{}{}{}{}{s.2g.}{Indivíduo que fala duas línguas.}{bi.lín.gue}{0}
\verb{bilinguismo}{}{}{}{}{s.m.}{Qualidade ou condição de bilíngue.}{bi.lin.guis.mo}{0}
\verb{bilinguismo}{}{}{}{}{}{Uso regular de duas línguas num país, ambas com \textit{status} de língua oficial.}{bi.lin.guis.mo}{0}
\verb{bilionário}{}{}{}{}{adj.}{Diz"-se do que é extremamente rico.}{bi.li.o.ná.rio}{0}
\verb{bilionário}{}{}{}{}{s.m.}{Indivíduo que possui grande fortuna; multimilionário, biliardário.}{bi.li.o.ná.rio}{0}
\verb{bilionésimo}{}{}{}{}{num.}{Ordinal e fracionário correspondente a 1 bilhão.  }{bi.li.o.né.si.mo}{0}
\verb{bilioso}{ô}{Fig.}{"-osos ⟨ó⟩}{"-osa ⟨ó⟩}{}{Mal"-humorado, irritadiço.}{bi.li.o.so}{0}
\verb{bilioso}{ô}{Med.}{"-osos ⟨ó⟩}{"-osa ⟨ó⟩}{adj.}{Que está cheio de bílis.}{bi.li.o.so}{0}
\verb{bílis}{}{Bioquím.}{}{}{s.f.}{Líquido viscoso secretado pelo fígado, que se acumula na vesícula biliar, de onde é levado para o duodeno no momento da digestão, auxiliando na absorção das gorduras.}{bí.lis}{0}
\verb{bílis}{}{Fig.}{}{}{}{Mau humor, irritação.}{bí.lis}{0}
\verb{bilontra}{}{}{}{}{adj.2g.}{Diz"-se do indivíduo ardiloso; espertalhão, finório, velhaco, patife.}{bi.lon.tra}{0}
\verb{bilontra}{}{}{}{}{}{Que é dado a conquistas amorosas.}{bi.lon.tra}{0}
\verb{bilro}{}{}{}{}{s.m.}{Peça de madeira ou metal usada para fazer rendas.}{bil.ro}{0}
\verb{biltre}{}{}{}{biltra}{adj.}{Que age de forma vil; canalha, infame, abjeto, desprezível.}{bil.tre}{0}
\verb{bímano}{}{}{}{}{adj.}{Que tem duas mãos.}{bí.ma.no}{0}
\verb{bimbalhada}{}{}{}{}{s.f.}{Ato ou efeito de tocarem muitos sinos ao mesmo tempo.}{bim.ba.lha.da}{0}
\verb{bimbalhar}{}{}{}{}{v.t.}{Fazer repicar ou soar o(s) sino(s).}{bim.ba.lhar}{\verboinum{1}}
\verb{bimensal}{}{}{"-ais}{}{adj.2g.}{Que se faz ou aparece duas vezes por mês; quinzenal.}{bi.men.sal}{0}
\verb{bimestral}{}{}{"-ais}{}{adj.2g.}{Que ocorre a cada dois meses.}{bi.mes.tral}{0}
\verb{bimestre}{é}{}{}{}{s.m.}{Período de dois meses.}{bi.mes.tre}{0}
\verb{bimotor}{ô}{}{}{}{adj.2g.}{Diz"-se de veículo que tem dois motores.}{bi.mo.tor}{0}
\verb{binacional}{}{}{"-ais}{}{adj.2g.}{Relativo ou pertencente a duas nações ou nacionalidades.}{bi.na.ci.o.nal}{0}
\verb{binário}{}{}{}{}{adj.}{Que tem dois elementos.}{bi.ná.rio}{0}
\verb{binário}{}{Mús.}{}{}{}{Que tem dois tempos iguais.}{bi.ná.rio}{0}
\verb{binário}{}{Mat.}{}{}{}{Sistema de numeração de base dois.}{bi.ná.rio}{0}
\verb{bingo}{}{}{}{}{s.m.}{Jogo de azar que consiste de um cartão com cinco fileiras com cinco números cada, ganhando o jogador que completar uma fileira, ou toda a cartela, conforme os números sorteados.}{bin.go}{0}
\verb{bingo}{}{}{}{}{interj.}{Grito ou exclamação para indicar que se completou uma fileira ou todo o cartão no jogo de bingo.}{bin.go}{0}
\verb{binóculo}{}{}{}{}{s.m.}{Instrumento óptico composto de duas lunetas focalizáveis simultaneamente para a visão com ambos os olhos, que permitem uma observação acurada de objetos distantes, com uma boa percepção da profundidade.}{bi.nó.cu.lo}{0}
\verb{binômio}{}{Mat.}{}{}{s.m.}{Expressão algébrica que consiste em dois termos ligados por um sinal de mais ou de menos.}{bi.nô.mio}{0}
\verb{binômio}{}{}{}{}{}{Nome científico de uma espécie (planta ou animal) composto de dois termos em latim: um substantivo, que designa o gênero, e um adjetivo, que designa a espécie. \textit{Hydrangea macrophylla} (\textit{hortência}).}{bi.nô.mio}{0}
\verb{biobibliografia}{}{}{}{}{s.f.}{Biografia de uma pessoa, acompanhada da relação de suas obras.}{bi.o.bi.bli.o.gra.fi.a}{0}
\verb{biociclo}{}{}{}{}{s.m.}{Sequência das etapas por que passam os seres vivos.}{bi.o.ci.clo}{0}
\verb{biociência}{}{Biol.}{}{}{s.f.}{Estudo científico dos seres vivos, nos seus diversos aspectos interdisciplinares.}{bi.o.ci.ên.cia}{0}
%\verb{}{}{}{}{}{}{}{}{0}
\verb{biodegradável}{}{Quím.}{"-eis}{}{adj.2g.}{Diz"-se de substância que pode ser decomposta por agente biológico.}{bi.o.de.gra.dá.vel}{0}
\verb{biodiversidade}{}{}{}{}{s.f.}{O conjunto de todas as espécies de seres vivos em determinada região e época.}{bi.o.di.ver.si.da.de}{0}
\verb{biogênese}{}{Biol.}{}{}{s.f.}{Princípio segundo o qual um ser vivo só pode ser originado a partir de outro ser vivo.}{bi.o.gê.ne.se}{0}
\verb{biogeografia}{}{Biol.}{}{}{s.f.}{Estudo da distribuição dos seres vivos nas várias regiões do planeta e sua relação com fatores climáticos e geológicos.}{bi.o.ge.o.gra.fi.a}{0}
\verb{biografar}{}{}{}{}{v.t.}{Fazer uma narrativa contando a história da vida real de alguém.}{bi.o.gra.far}{\verboinum{1}}
\verb{biografia}{}{}{}{}{s.f.}{História de vida de uma pessoa.}{bi.o.gra.fi.a}{0}
\verb{biografia}{}{Por ext.}{}{}{}{O livro que contém essa história.}{bi.o.gra.fi.a}{0}
\verb{biográfico}{}{}{}{}{adj.}{Relativo a biografia.}{bi.o.grá.fi.co}{0}
\verb{biográfico}{}{}{}{}{}{Que contém uma ou mais biografias. (\textit{Enciclopédia biográfica.})}{bi.o.grá.fi.co}{0}
\verb{biógrafo}{}{}{}{}{s.m.}{Autor de biografias.}{bi.ó.gra.fo}{0}
\verb{biologia}{}{}{}{}{s.f.}{Ciência que estuda os seres vivos e as leis que governam a vida.}{bi.o.lo.gi.a}{0}
\verb{biológico}{}{}{}{}{adj.}{Relativo à biologia.}{bi.o.ló.gi.co}{0}
\verb{biológico}{}{}{}{}{}{Relativo ou próprio de seres vivos.}{bi.o.ló.gi.co}{0}
\verb{biológico}{}{}{}{}{}{Não adotivo. (\textit{Pai biológico.})}{bi.o.ló.gi.co}{0}
\verb{biológico}{}{}{}{}{}{Que atua por meio de fenômenos biológicos. (\textit{Agente biológico, arma biológica.})}{bi.o.ló.gi.co}{0}
\verb{biologismo}{}{}{}{}{s.m.}{Doutrina segundo a qual os fenômenos psíquicos e sociais seriam determinados por fatores biológicos.}{bi.o.lo.gis.mo}{0}
\verb{biologista}{}{}{}{}{adj.2g.}{Relativo a biologismo.}{bi.o.lo.gis.ta}{0}
\verb{biologista}{}{}{}{}{}{Biólogo.}{bi.o.lo.gis.ta}{0}
\verb{biologista}{}{}{}{}{s.2g.}{Indivíduo adepto do biologismo.}{bi.o.lo.gis.ta}{0}
\verb{biólogo}{}{}{}{}{s.m.}{Indivíduo especializado em Biologia.}{bi.ó.lo.go}{0}
\verb{bioluminescência}{}{}{}{}{s.f.}{Fenômeno da produção de luz por seres vivos, como resultado de reações químicas, como nos vaga"-lumes e certas algas marinhas.}{bi.o.lu.mi.nes.cên.cia}{0}
\verb{bioma}{}{Biol.}{}{}{s.m.}{Comunidade estável de seres vivos em uma determinada área.}{bi.o.ma}{0}
\verb{biomassa}{}{}{}{}{s.f.}{Massa originada por matéria viva e que é utilizada como fonte de energia.}{bi.o.mas.sa}{0}
\verb{biombo}{}{}{}{}{s.m.}{Anteparo ou divisória móvel que tem a finalidade de dividir um ambiente ou criar uma área resguardada.}{bi.om.bo}{0}
\verb{biometria}{}{}{}{}{s.f.}{Estudo das dimensões das estruturas e órgãos dos seres vivos, com finalidades específicas.}{bi.o.me.tri.a}{0}
\verb{biométrico}{}{}{}{}{adj.}{Relativo à biometria.}{bi.o.mé.tri.co}{0}
\verb{biônica}{}{}{}{}{s.f.}{Estudo das funções e mecanismos existentes nos seres vivos com a finalidade de aplicá"-lo à construção de máquinas e aparelhos.}{bi.ô.ni.ca}{0}
\verb{biônico}{}{}{}{}{adj.}{Relativo à biônica.}{bi.ô.ni.co}{0}
\verb{biônico}{}{}{}{}{}{Criado por meio dos conhecimentos da biônica, em que há cooperação de mecanismos biológicos e eletrônicos.}{bi.ô.ni.co}{0}
\verb{biônico}{}{Fig.}{}{}{}{Diz"-se dos políticos que assumem cargos eletivos por nomeação.}{bi.ô.ni.co}{0}
\verb{biopsia}{}{}{}{}{}{Var. de \textit{biópsia}.}{bi.op.si.a}{0}
\verb{biópsia}{}{Med.}{}{}{s.f.}{Exame de um tecido retirado de um ser vivo para verificar alterações e fazer diagnósticos.}{bi.óp.sia}{0}
\verb{bioquímica}{}{Bioquím.}{}{}{s.f.}{Estudo das reações químicas que ocorrem nas substâncias e moléculas originadas por seres vivos e de seus processos metabólicos.}{bi.o.quí.mi.ca}{0}
\verb{bioquímico}{}{}{}{}{adj.}{Relativo à bioquímica.}{bi.o.quí.mi.co}{0}
\verb{bioquímico}{}{}{}{}{s.m.}{Indivíduo que se dedica à bioquímica.}{bi.o.quí.mi.co}{0}
\verb{biorritmo}{}{Biol.}{}{}{s.m.}{O ritmo característico de qualquer processo biológico.}{bi.or.rit.mo}{0}
\verb{biosfera}{é}{}{}{}{s.f.}{O conjunto de todos os ecossistemas do planeta.}{bi.os.fe.ra}{0}
\verb{biosfera}{é}{}{}{}{}{O conjunto de todas as áreas do planeta onde pode existir vida.}{bi.os.fe.ra}{0}
\verb{biota}{ó}{Biol.}{}{}{s.f.}{Conjunto de todos os seres vivos de uma determinada região.}{bi.o.ta}{0}
\verb{biotecnologia}{}{Biol.}{}{}{s.f.}{Estudo e criação de organismos geneticamente modificados.}{bi.o.tec.no.lo.gi.a}{0}
\verb{biotecnologia}{}{}{}{}{}{Aplicação de conhecimentos da biologia para fins produtivos.}{bi.o.tec.no.lo.gi.a}{0}
\verb{biotério}{}{Bras.}{}{}{s.m.}{Viveiro de cobaias.}{bi.o.té.rio}{0}
\verb{biotipo}{}{Biol.}{}{}{s.m.}{Grupo de indivíduos geneticamente iguais.}{bi.o.ti.po}{0}
\verb{biotipo}{}{Med.}{}{}{}{Tipo físico constitucional.}{bi.o.ti.po}{0}
\verb{biótipo}{}{}{}{}{}{Var. de \textit{biotipo}.}{bi.ó.ti.po}{0}
\verb{biotipologia}{}{}{}{}{s.f.}{Estudo dos tipos antropológicos que classifica os seres humanos em tipos biopsicológicos.}{bi.o.ti.po.lo.gi.a}{0}
\verb{bióxido}{cs}{Quím.}{}{}{s.m.}{Óxido em cuja molécula há dois átomos de oxigênio; dióxido.}{bi.ó.xi.do}{0}
\verb{bipartidarismo}{}{}{}{}{s.m.}{Situação política em que há apenas dois partidos ou dois partidos relevantes.}{bi.par.ti.da.ris.mo}{0}
\verb{bipartido}{}{}{}{}{adj.}{Dividido em duas partes; dividido ao meio.}{bi.par.ti.do}{0}
\verb{bipartir}{}{}{}{}{v.t.}{Dividir em dois.}{bi.par.tir}{\verboinum{18}}
\verb{bipe}{}{}{}{}{s.m.}{Som breve e agudo, geralmente produzido por aparelho eletrônico.}{bi.pe}{0}
\verb{bipe}{}{}{}{}{}{Aparelho eletrônico portátil de comunicação que recebe sinais ou mensagens escritas.}{bi.pe}{0}
\verb{bípede}{}{}{}{}{adj.2g.}{Que tem dois pés; que usa dois membros para deslocar"-se.}{bí.pe.de}{0}
\verb{bípede}{}{}{}{}{s.m.}{Animal que se desloca usando dois membros.}{bí.pe.de}{0}
\verb{biplano}{}{}{}{}{s.m.}{Aeroplano cujas asas de sustentação são formadas por dois planos paralelos, posicionados um sobre o outro.  }{bi.pla.no}{0}
\verb{bipolar}{}{}{}{}{adj.2g.}{Que tem dois polos.}{bi.po.lar}{0}
\verb{bipolaridade}{}{}{}{}{s.f.}{Existência de dois polos opostos em um corpo, sistema físico ou sistema político.}{bi.po.la.ri.da.de}{0}
\verb{biqueira}{ê}{}{}{}{s.f.}{Extremidade em forma de ponta ou terminação que se ajusta à extremidade de algo para dar acabamento.}{bi.quei.ra}{0}
\verb{biqueiro}{ê}{}{}{}{adj.}{Que come pouco, por costume, falta de apetite ou por ser alguém difícil de agradar.}{bi.quei.ro}{0}
\verb{biquíni}{}{}{}{}{s.m.}{Roupa de banho feminina composta de duas peças independentes que cobrem o busto e a região pélvica.}{bi.quí.ni}{0}
\verb{biriba}{}{Bras.}{}{}{s.f.}{Jogo de cartas com regras semelhantes às da canastra.}{bi.ri.ba}{0}
\verb{birita}{}{Pop.}{}{}{s.f.}{Cachaça.}{bi.ri.ta}{0}
\verb{birita}{}{Por ext.}{}{}{}{Qualquer bebida alcoólica, especialmente as destiladas.}{bi.ri.ta}{0}
\verb{birmanês}{}{}{}{}{adj.}{Relativo à Birmânia, país asiático atualmente denominado União de Myanmar (ou Myanma, ou Mianmá).}{bir.ma.nês}{0}
\verb{birmanês}{}{}{}{}{s.m.}{Indivíduo natural ou habitante desse país.}{bir.ma.nês}{0}
\verb{birmanês}{}{}{}{}{}{Língua falada nesse país.}{bir.ma.nês}{0}
\verb{birô}{}{}{}{}{s.m.}{Escrivaninha.}{bi.rô}{0}
\verb{birô}{}{Por ext.}{}{}{}{Escritório, gabinete.}{bi.rô}{0}
\verb{birô}{}{Por ext.}{}{}{}{Repartição pública; departamento, agência.}{bi.rô}{0}
\verb{birô}{}{Bras.}{}{}{}{Estabelecimento que presta serviços na área de computação gráfica e editoração eletrônica.}{bi.rô}{0}
\verb{birosca}{ó}{Bras.}{}{}{s.f.}{Pequena venda, geralmente em bairros pobres, onde se encontram bebidas e artigos de mercearia.}{bi.ros.ca}{0}
\verb{birosca}{ó}{Bras.}{}{}{}{Jogo de bolinhas de gude.}{bi.ros.ca}{0}
\verb{birote}{ó}{Bras.}{}{}{s.m.}{Penteado feminino que consiste em enrodilhar os cabelos no alto da cabeça.}{bi.ro.te}{0}
\verb{birra}{}{}{}{}{s.f.}{Teimosia, zanga, mau humor.}{bir.ra}{0}
\verb{birra}{}{}{}{}{}{Implicância, antipatia, aversão.}{bir.ra}{0}
\verb{birra}{}{}{}{}{}{Desavença, desentendimento.}{bir.ra}{0}
\verb{birrento}{}{}{}{}{adj.}{Que tem ou faz birra; teimoso, implicante, zangado.}{bir.ren.to}{0}
\verb{biruta}{}{}{}{}{s.f.}{Aparelho meteorológico feito com um tubo de tecido preso a um aro preso na vertical, e que serve para indicar a direção do vento.}{bi.ru.ta}{0}
\verb{biruta}{}{Pop.}{}{}{adj.2g.}{Diz"-se de indivíduo sem norte, sem constância de opinião ou de vontade.}{bi.ru.ta}{0}
\verb{biruta}{}{Pop.}{}{}{}{Inquieto.}{bi.ru.ta}{0}
\verb{bis}{}{}{}{}{interj.}{Expressão com que se pede a um artista que repita a execução de obra ou trecho dela.}{bis}{0}
\verb{bis}{}{}{}{}{s.m.}{A repetição executada pelo artista.}{bis}{0}
\verb{bisão}{}{Zool.}{"-ões}{}{s.m.}{Mamífero ruminante de grande porte com pelagem longa na região do pescoço, cabeça grande e chifres curtos.}{bi.são}{0}
\verb{bisar}{}{}{}{}{v.t.}{Pedir a repetição de trecho de um espetáculo, gritando \textit{bis}.}{bi.sar}{0}
\verb{bisar}{}{}{}{}{}{Executar novamente um trecho de espetáculo.}{bi.sar}{0}
\verb{bisar}{}{Por ext.}{}{}{}{Repetir qualquer coisa, como uma refeição, uma sobremesa, uma façanha.}{bi.sar}{\verboinum{1}}
\verb{bisavó}{}{}{}{}{s.f.}{Mãe do avô ou da avó.}{bi.sa.vó}{0}
\verb{bisavô}{}{}{}{}{s.m.}{Pai do avô ou da avó.}{bi.sa.vô}{0}
\verb{bisbilhotar}{}{}{}{}{v.t.}{Fazer intrigas, falar da vida alheia.}{bis.bi.lho.tar}{0}
\verb{bisbilhotar}{}{}{}{}{}{Intrometer"-se, esquadrinhar.}{bis.bi.lho.tar}{\verboinum{1}}
\verb{bisbilhoteiro}{ê}{}{}{}{adj.}{Que fala da vida alheia; mexeriqueiro.}{bis.bi.lho.tei.ro}{0}
\verb{bisbilhoteiro}{ê}{}{}{}{}{Que se intromete com curiosidade; enxerido, curioso.}{bis.bi.lho.tei.ro}{0}
\verb{bisbilhotice}{}{}{}{}{s.f.}{Ato de bisbilhotar.}{bis.bi.lho.ti.ce}{0}
\verb{bisbilhotice}{}{}{}{}{}{Qualidade de bisbilhoteiro.}{bis.bi.lho.ti.ce}{0}
\verb{bisca}{}{}{}{}{s.f.}{Nome comum a diversos jogos de cartas.}{bis.ca}{0}
\verb{bisca}{}{Pop.}{}{}{}{Pessoa de mau caráter; falsa, dissimulada.}{bis.ca}{0}
\verb{biscainho}{}{}{}{}{adj.}{Relativo à província ou à região de Biscaia, no País Basco, Espanha.}{bis.ca.i.nho}{0}
\verb{biscainho}{}{}{}{}{s.m.}{Indivíduo natural ou habitante dessa região.}{bis.ca.i.nho}{0}
\verb{biscainho}{}{Gram.}{}{}{}{Dialeto da língua basca falado em Biscaia.}{bis.ca.i.nho}{0}
\verb{biscate}{}{}{}{}{s.m.}{Serviço simples; ocupação eventual ou secundária; bico.}{bis.ca.te}{0}
\verb{biscate}{}{Bras.}{}{}{}{Prostituta.}{bis.ca.te}{0}
\verb{biscatear}{}{}{}{}{v.i.}{Fazer ou viver de biscates.}{bis.ca.te.ar}{\verboinum{4}}
\verb{biscateiro}{ê}{}{}{}{adj.}{Que faz ou vive de biscates.}{bis.ca.tei.ro}{0}
\verb{biscoiteira}{ê}{}{}{}{s.f.}{Recipiente para guardar biscoitos e similares.}{bis.coi.tei.ra}{0}
\verb{biscoiteira}{ê}{}{}{}{}{Mulher que faz ou vende biscoitos.}{bis.coi.tei.ra}{0}
\verb{biscoiteiro}{ê}{}{}{}{s.m.}{Indivíduo que fabrica ou vende biscoitos.}{bis.coi.tei.ro}{0}
\verb{biscoito}{ô}{}{}{}{s.m.}{Alimento de massa à base de farinha ou fécula com diversas especiarias e assado no forno.}{bis.coi.to}{0}
\verb{bisel}{é}{}{"-éis}{}{s.m.}{Corte oblíquo em borda de vidro ou quina; chanfro.}{bi.sel}{0}
\verb{bismuto}{}{Quím.}{}{}{s.m.}{Elemento químico metálico, quebradiço, usado em ligas e compostos para uso químico e farmacêutico. \elemento{83}{208.98038}{Bi}.}{bis.mu.to}{0}
\verb{bisnaga}{}{}{}{}{s.f.}{Tubo geralmente longo em que se embalam produtos líquidos ou pastosos.}{bis.na.ga}{0}
\verb{bisnaga}{}{}{}{}{}{Tipo de pão macio e comprido.}{bis.na.ga}{0}
\verb{bisneto}{é}{}{}{}{s.m.}{Filho de neto ou de neta.}{bis.ne.to}{0}
\verb{bisonho}{}{}{}{}{s.m.}{Soldado inexperiente; recruta.}{bi.so.nho}{0}
\verb{bisonho}{}{Por ext.}{}{}{adj.}{Novato, principiante.}{bi.so.nho}{0}
\verb{bisonho}{}{Pop.}{}{}{}{Malvestido.}{bi.so.nho}{0}
\verb{bispado}{}{}{}{}{s.m.}{Cargo ou dignidade de bispo.}{bis.pa.do}{0}
\verb{bispado}{}{}{}{}{adj.}{Que se bispou; vislumbrado, visto de longe.}{bis.pa.do}{0}
\verb{bispado}{}{}{}{}{}{Território eclesiástico sob a autoridade de um bispo.}{bis.pa.do}{0}
\verb{bispado}{}{}{}{}{}{Residência de um bispo.}{bis.pa.do}{0}
\verb{bispado}{}{}{}{}{}{Que pegou no fundo da panela e ficou queimado ou com gosto de queimado.}{bis.pa.do}{0}
\verb{bispar}{}{}{}{}{v.t.}{Exercer a função de bispo.}{bis.par}{0}
\verb{bispar}{}{}{}{}{}{Ver sem muita clareza; vislumbrar, avistar, entrever.}{bis.par}{\verboinum{1}}
\verb{bispo}{}{}{}{}{s.m.}{Padre que dirige uma diocese, com poder de transmitir o sacramento da ordem eclesiástica.}{bis.po}{0}
\verb{bispo}{}{}{}{}{}{Peça do jogo de xadrez que se movimenta somente na direção diagonal.}{bis.po}{0}
\verb{bissemanal}{}{}{"-ais}{}{adj.2g.}{Que se realiza duas vezes por semana.}{bis.se.ma.nal}{0}
\verb{bissemanal}{}{}{"-ais}{}{}{Que se realiza a cada duas semanas.}{bis.se.ma.nal}{0}
\verb{bissetriz}{}{Geom.}{}{}{s.f.}{Reta que divide um ângulo ao meio.}{bis.se.triz}{0}
\verb{bissexto}{ês}{}{}{}{adj.}{Diz"-se do ano que tem 366 dias.}{bis.sex.to}{0}
\verb{bissexto}{ês}{}{}{}{s.m.}{O dia que é acrescentado à contagem do ano solar a cada quatro anos.}{bis.sex.to}{0}
\verb{bissexto}{ês}{Fig.}{}{}{adj.}{Que exerce alguma atividade com pouca frequência.}{bis.sex.to}{0}
\verb{bissexual}{cs}{}{"-ais}{}{adj.2g.}{Relativo ao comportamento sexual com indivíduos de ambos os sexos.}{bis.se.xu.al}{0}
\verb{bissexual}{cs}{Biol.}{"-ais}{}{}{Diz"-se do ser que tem órgãos reprodutores dos dois sexos; hermafrodito.}{bis.se.xu.al}{0}
\verb{bissexual}{cs}{}{"-ais}{}{s.m.}{Indivíduo que se sente atraído física, emocional e espiritualmente tanto por pessoas do mesmo sexo como por pessoas de sexo diferente.   }{bis.se.xu.al}{0}
\verb{bissexualidade}{cs}{}{}{}{s.f.}{Qualidade ou condição de bissexual.}{bis.se.xu.a.li.da.de}{0}
\verb{bissexualismo}{cs}{}{}{}{s.m.}{Bissexualidade.}{bis.se.xu.a.lis.mo}{0}
\verb{bissexualismo}{cs}{Biol.}{}{}{}{Presença dos dois sexos num mesmo indivíduo; hermafroditismo.}{bis.se.xu.a.lis.mo}{0}
\verb{bisteca}{é}{Bras.}{}{}{s.f.}{Fatia de carne bovina, geralmente contrafilé, com osso.}{bis.te.ca}{0}
\verb{bistrô}{}{}{}{}{s.m.}{Restaurante pequeno e aconchegante.}{bis.trô}{0}
\verb{bisturi}{}{}{}{}{s.m.}{Instrumento de lâmina curta usado para cortar a pele e outros tecidos.}{bis.tu.ri}{0}
\verb{bit}{}{Informát.}{}{}{s.m.}{Menor unidade de informação processada por um microprocessador digital. Símb.: b.}{\textit{bit}}{0}
\verb{bit}{}{}{}{}{}{Algarismo do sistema numérico binário e que pode assumir os valores 0 e 1.}{\textit{bit}}{0}
\verb{bitácula}{}{}{}{}{s.f.}{Coluna de madeira que acomoda a bússola em um navio.}{bi.tá.cu.la}{0}
\verb{bitácula}{}{Fig.}{}{}{}{O rosto, especialmente quando se usam óculos.}{bi.tá.cu.la}{0}
\verb{bitola}{ó}{}{}{}{s.f.}{Medida"-padrão, utilizada na fabricação de peças e na construção.}{bi.to.la}{0}
\verb{bitola}{ó}{Fig.}{}{}{}{Padrão de conduta; norma, regra.}{bi.to.la}{0}
\verb{bitola}{ó}{}{}{}{}{A largura de uma ferrovia, que é a distância entre os trilhos.}{bi.to.la}{0}
\verb{bitola}{ó}{}{}{}{}{A largura de um filme fotográfico ou cinematográfico.}{bi.to.la}{0}
\verb{bitolado}{}{}{}{}{adj.}{Que se bitolou.}{bi.to.la.do}{0}
\verb{bitolado}{}{Fig.}{}{}{}{Que tem comportamentos e opiniões considerados estranhos, por serem demasiado rígidos, limitados ou ultrapassados; quadrado, careta.}{bi.to.la.do}{0}
\verb{bitolar}{}{}{}{}{v.t.}{Determinar ou aplicar uma bitola.}{bi.to.lar}{0}
\verb{bitolar}{}{}{}{}{}{Avaliar, estimar.}{bi.to.lar}{0}
\verb{bitolar}{}{}{}{}{v.pron.}{Tornar"-se bitolado.}{bi.to.lar}{\verboinum{1}}
\verb{bitransitivo}{}{Gram.}{}{}{adj.}{Diz"-se de verbo que admite dois complementos, um objeto direto e um objeto indireto.}{bi.tran.si.ti.vo}{0}
\verb{bituca}{}{Pop.}{}{}{s.f.}{A extremidade do cigarro que sobra depois de fumado.}{bi.tu.ca}{0}
\verb{bituca}{}{Pop.}{}{}{s.m.}{Indivíduo que não paga suas dívidas.}{bi.tu.ca}{0}
\verb{bivalente}{}{Gram.}{}{}{adj.2g.}{Diz"-se de elemento que funciona como predicado e requer dois argumentos.}{bi.va.len.te}{0}
\verb{bivalente}{}{Quím.}{}{}{}{Que possui valência dois.}{bi.va.len.te}{0}
\verb{bivalente}{}{Fig.}{}{}{}{Que possui duas funções.}{bi.va.len.te}{0}
\verb{bivalve}{}{Biol.}{}{}{adj.2g.}{Que apresenta duas valvas.}{bi.val.ve}{0}
\verb{bivalve}{}{Biol.}{}{}{}{Relativo aos bivalves, moluscos de vida aquática conhecidos como ostras e mexilhões.}{bi.val.ve}{0}
\verb{bivalve}{}{Biol.}{}{}{s.m.}{Espécime dos bivalves.}{bi.val.ve}{0}
\verb{bivaque}{}{}{}{}{s.m.}{Estacionamento provisório de tropas militares a céu aberto ou sob algum tipo de abrigo natural.}{bi.va.que}{0}
\verb{bizantino}{}{}{}{}{adj.}{Relativo a Bizâncio, cidade fundada às margens do rio Bósforo pelos gregos no séc. \textsc{vii} a.C. e que se tornou capital do Império Romano do Oriente. Atualmente chama"-se Istambul.}{bi.zan.ti.no}{0}
\verb{bizantino}{}{Hist.}{}{}{}{Relativo ao Império Romano do Oriente.}{bi.zan.ti.no}{0}
\verb{bizantino}{}{}{}{}{s.m.}{Indivíduo natural ou habitante da cidade de Bizâncio.}{bi.zan.ti.no}{0}
\verb{bizantino}{}{}{}{}{adj.}{Diz"-se de certa tonalidade de cor"-de"-rosa.}{bi.zan.ti.no}{0}
\verb{bizantino}{}{Fig.}{}{}{}{Inútil, frívolo, pretensioso.}{bi.zan.ti.no}{0}
\verb{bizarria}{}{}{}{}{s.f.}{Qualidade ou ação de bizarro; ostentação, pompa; esquisitice.}{bi.zar.ri.a}{0}
\verb{bizarro}{}{}{}{}{adj.}{Elegante, garboso, bem"-apessoado.}{bi.zar.ro}{0}
\verb{bizarro}{}{}{}{}{}{Nobre, generoso, valente.}{bi.zar.ro}{0}
\verb{bizarro}{}{}{}{}{}{Extravagante, esquisito.}{bi.zar.ro}{0}
\verb{Bk}{}{Quím.}{}{}{}{Símb. do \textit{berquélio}.}{Bk}{0}
\verb{blablablá}{}{Bras.}{}{}{s.m.}{Conversa sem conteúdo.}{bla.bla.blá}{0}
\verb{blandícia}{}{}{}{}{s.f.}{Gesto de carinho; carícia, afago.}{blan.dí.cia}{0}
\verb{blandícia}{}{Fig.}{}{}{}{Meiguice, brandura.}{blan.dí.cia}{0}
\verb{blandície}{}{}{}{}{}{Var. de \textit{blandícia}.}{blan.dí.cie}{0}
\verb{blasfemar}{}{}{}{}{v.i.}{Dizer blasfêmia.}{blas.fe.mar}{0}
\verb{blasfemar}{}{}{}{}{}{Dizer mal, rogar praga, amaldiçoar.}{blas.fe.mar}{\verboinum{1}}
\verb{blasfêmia}{}{}{}{}{s.f.}{Palavra ou expressão que insulta uma divindade ou a religião.}{blas.fê.mia}{0}
\verb{blasfêmia}{}{Por ext.}{}{}{}{Afirmação ilógica; contrassenso.}{blas.fê.mia}{0}
\verb{blasfemo}{}{}{}{}{adj.}{Que blasfema.}{blas.fe.mo}{0}
\verb{blasonar}{}{}{}{}{v.t.}{Ostentar, alardear.}{bla.so.nar}{0}
\verb{blasonar}{}{}{}{}{}{Vangloriar"-se; jactar"-se de valentias supostas.}{bla.so.nar}{\verboinum{1}}
\verb{blaterar}{}{}{}{}{v.i.}{Soltar a voz (o camelo).}{bla.te.rar}{0}
\verb{blaterar}{}{}{}{}{v.t.}{Falar mal; xingar, vociferar.}{bla.te.rar}{\verboinum{1}}
\verb{blecaute}{}{}{}{}{s.m.}{Medida de defesa antiaérea que consiste em deixar tudo às escuras.}{ble.cau.te}{0}
\verb{blecaute}{}{}{}{}{}{Interrupção no fornecimento de energia elétrica; apagão.   }{ble.cau.te}{0}
\verb{blefar}{}{}{}{}{v.i.}{Em um jogo, fazer aposta alta sem ter cartas que garantam a aposta, para que o oponente desista da jogada.}{ble.far}{0}
\verb{blefar}{}{Por ext.}{}{}{}{Fingir, enganar, iludir, ludibriar.}{ble.far}{\verboinum{1}}
\verb{blefe}{ê/ ou /é}{}{}{}{s.m.}{Ato ou efeito de blefar.}{ble.fe}{0}
\verb{blenorragia}{}{Med.}{}{}{s.f.}{Doença sexualmente transmissível caracterizada  por inflamação das membranas da uretra e da vagina e acompanhada de corrimentos; blenorreia.}{ble.nor.ra.gi.a}{0}
\verb{blenorreia}{é}{Med.}{}{}{s.f.}{Blenorragia.}{ble.nor.rei.a}{0}
\verb{blindado}{}{}{}{}{adj.}{Que é coberto ou protegido com revestimento de aço ou chapa metálica resistente a projéteis.}{blin.da.do}{0}
\verb{blindado}{}{Fig.}{}{}{}{Que é protegido, defendido.}{blin.da.do}{0}
\verb{blindado}{}{}{}{}{s.m.}{Carro blindado.}{blin.da.do}{0}
\verb{blindagem}{}{}{"-ens}{}{s.f.}{Ato ou efeito de blindar.}{blin.da.gem}{0}
\verb{blindagem}{}{}{"-ens}{}{}{Revestimento de chapa metálica, usado para proteger contra projéteis.}{blin.da.gem}{0}
\verb{blindar}{}{}{}{}{v.t.}{Revestir de chapas de aço.}{blin.dar}{0}
\verb{blindar}{}{Por ext.}{}{}{}{Proteger, resguardar, cobrir.}{blin.dar}{\verboinum{1}}
\verb{blitz}{}{}{}{}{s.f.}{Batida policial, de surpresa e com grande aparato bélico.}{\textit{blitz}}{0}
\verb{bloco}{ó}{}{}{}{s.m.}{Massa volumosa e sólida de uma substância.}{blo.co}{0}
\verb{bloco}{ó}{}{}{}{}{Reunião de folhas de papel presas por um dos lados.}{blo.co}{0}
\verb{bloco}{ó}{}{}{}{}{Cada um dos edifícios de um conjunto de prédios.}{blo.co}{0}
\verb{bloco}{ó}{}{}{}{}{Grupo carnavalesco.}{blo.co}{0}
\verb{bloquear}{}{}{}{}{v.t.}{Pôr bloqueio; cercar, sitiar.}{blo.que.ar}{0}
\verb{bloquear}{}{}{}{}{}{Impedir movimento ou circulação.}{blo.que.ar}{0}
\verb{bloquear}{}{}{}{}{}{Obstruir, dificultar, impedir.}{blo.que.ar}{\verboinum{4}}
\verb{bloqueio}{ê}{}{}{}{s.m.}{Ato ou efeito de bloquear.}{blo.quei.o}{0}
\verb{bloqueio}{ê}{}{}{}{}{Obstrução ou impedimento de entrada, saída ou passagem de alguém ou algo, por meio de algum obstáculo físico.}{blo.quei.o}{0}
\verb{bloqueio}{ê}{}{}{}{}{Interrupção de movimento, deslocamento, circulação, desenvolvimento, funcionamento de algo, especialmente por ação ou influência de um fator externo.}{blo.quei.o}{0}
\verb{bluejeans}{}{}{}{}{s.m.}{Calça comprida, feita de tecido grosso e resistente, de cor azul; \textit{jeans}.}{\textit{bluejeans}}{0}
\verb{blusa}{}{}{}{}{s.f.}{Peça do vestuário feminino ou masculino, com ou sem mangas, com ou sem gola, e que termina na cintura ou pouco abaixo.}{blu.sa}{0}
\verb{blusão}{}{}{"-ões}{}{s.m.}{Vestimenta esportiva, mais folgada e mais comprida que a blusa, usada geralmente por fora da calça ou da saia.}{blu.são}{0}
\verb{boa}{ô}{}{}{}{adj.}{Feminino de bom.}{bo.a}{0}
\verb{boa}{ô}{Pop.}{}{}{}{Diz"-se de mulher de físico sedutor, atraente.}{bo.a}{0}
\verb{boa"-fé}{ô}{}{boas"-fés ⟨ô⟩}{}{s.f.}{Retidão ou pureza de intenções; sinceridade. (\textit{Ele agia sempre com boa"-fé.})}{bo.a"-fé}{0}
\verb{boa"-fé}{ô}{}{boas"-fés ⟨ô⟩}{}{}{Certeza de agir com o amparo da lei, ou sem ofensa a ela.}{bo.a"-fé}{0}
\verb{boa"-fé}{ô}{}{boas"-fés ⟨ô⟩}{}{}{Ausência de intenção dolosa.}{bo.a"-fé}{0}
\verb{boa"-noite}{ô}{}{boas"-noites ⟨ô⟩}{}{s.m.}{Cumprimento com que se saúda alguém à noite.}{bo.a"-noi.te}{0}
\verb{boa"-pinta}{ô}{Pop.}{boas"-pintas ⟨ô⟩}{}{adj.2g.}{Diz"-se de pessoa elegante, que causa boa impressão.}{bo.a"-pin.ta}{0}
\verb{boa"-pinta}{ô}{Pop.}{boas"-pintas ⟨ô⟩}{}{s.2g.}{Indivíduo que agrada fisicamente, que tem boa aparência e boas maneiras.}{bo.a"-pin.ta}{0}
\verb{boa"-praça}{ô}{Pop.}{boas"-praças ⟨ô⟩}{}{adj.2g.}{Diz"-se de pessoa boa, agradável, simpática.}{bo.a"-pra.ça}{0}
\verb{boa"-praça}{ô}{Pop.}{boas"-praças ⟨ô⟩}{}{s.2g.}{Indivíduo que é bom companheiro ou camarada, simpático e confiável.}{bo.a"-pra.ça}{0}
\verb{boas"-festas}{ô\ldots{}é}{}{}{}{s.f.pl.}{Cumprimentos e felicitações por ocasião do Natal ou do Ano"-Novo.}{bo.as"-fes.tas}{0}
\verb{boas"-festas}{ô\ldots{}é}{}{}{}{interj.}{Expressão com que se saúdam as pessoas na época do Natal ou do Ano"-Novo.}{bo.as"-fes.tas}{0}
\verb{boas"-vindas}{ô}{}{}{}{s.f.pl.}{Saudação cordial pela chegada de alguém.}{bo.as"-vin.das}{0}
\verb{boa"-tarde}{ô}{}{boas"-tardes ⟨ô⟩}{}{s.m.}{Cumprimento que se usa para saudar alguém à tarde.}{bo.a"-tar.de}{0}
\verb{boataria}{}{}{}{}{s.f.}{Grande quantidade ou difusão de boatos.}{bo.a.ta.ri.a}{0}
\verb{boate}{}{}{}{}{s.f.}{Casa noturna que consta de bar, restaurante, pista de dança e palco para apresentação de atrações artísticas.}{bo.a.te}{0}
\verb{boateiro}{ê}{}{}{}{adj.}{Diz"-se daquele que veicula boatos.}{bo.a.tei.ro}{0}
\verb{boateiro}{ê}{}{}{}{s.m.}{Indivíduo que lança ou espalha boatos.}{bo.a.tei.ro}{0}
\verb{boato}{}{}{}{}{s.m.}{Notícia de fonte desconhecida, muitas vezes infundada, que se divulga entre o público.}{bo.a.to}{0}
\verb{boa"-vida}{ô}{}{boas"-vidas ⟨ô⟩}{}{s.2g.}{Pessoa que não gosta de trabalhar e procura viver de modo mais agradável com o mínimo de esforço.}{bo.a"-vi.da}{0}
\verb{boa"-vistense}{ô}{}{boa"-vistenses ⟨ô⟩}{}{adj.2g.}{Relativo a Boa Vista, capital de Roraima.}{bo.a"-vis.ten.se}{0}
\verb{boa"-vistense}{ô}{}{boa"-vistenses ⟨ô⟩}{}{s.2g.}{Indivíduo natural ou habitante dessa cidade.}{bo.a"-vis.ten.se}{0}
\verb{bobagem}{}{}{"-ens}{}{s.f.}{Asneira, tolice.}{bo.ba.gem}{0}
\verb{bobagem}{}{}{"-ens}{}{}{Fato ou palavra inconveniente.}{bo.ba.gem}{0}
\verb{bobagem}{}{}{"-ens}{}{}{Coisa supérflua ou sem importância.}{bo.ba.gem}{0}
\verb{bobalhão}{}{}{"-ões}{"-ona}{s.m.}{Indivíduo muito bobo; simplório, palerma.}{bo.ba.lhão}{0}
\verb{bobeada}{}{Pop.}{}{}{s.f.}{Falta de atenção; descuido, vacilada.}{bo.be.a.da}{0}
\verb{bobear}{}{}{}{}{v.i.}{Fazer ou dizer bobices; portar"-se como bobo.}{bo.be.ar}{0}
\verb{bobear}{}{}{}{}{}{Desperdiçar oportunidades.}{bo.be.ar}{0}
\verb{bobear}{}{}{}{}{}{Cometer descuidos.}{bo.be.ar}{0}
\verb{bobear}{}{}{}{}{v.t.}{Fazer crer no que é falso; enganar.}{bo.be.ar}{\verboinum{4}}
\verb{bobeira}{ê}{Pop.}{}{}{s.f.}{Atitude de tolo.}{bo.bei.ra}{0}
\verb{bobeira}{ê}{}{}{}{}{Desatenção, credulidade.}{bo.bei.ra}{0}
\verb{bobeira}{ê}{}{}{}{}{Coisa insignificante.}{bo.bei.ra}{0}
\verb{bobice}{}{}{}{}{s.f.}{Bobeira.}{bo.bi.ce}{0}
\verb{bobina}{}{}{}{}{s.f.}{Rolo de madeira ou de metal, sobre o qual se enrola fio de seda, linho, lã ou de qualquer outro material flexível.}{bo.bi.na}{0}
\verb{bobinagem}{}{}{"-ens}{}{s.f.}{Ato ou efeito de bobinar, de enrolar em bobina.}{bo.bi.na.gem}{0}
\verb{bobinar}{}{}{}{}{v.t.}{Enrolar folha de papel, fita, fio, filme etc., em bobina.}{bo.bi.nar}{\verboinum{1}}
\verb{bobó}{}{Cul.}{}{}{s.m.}{Comida africana feita com feijão e azeite"-de"-dendê.}{bo.bó}{0}
\verb{bobó}{}{Cul.}{}{}{}{Creme de inhame, aipim ou fruta"-pão, cozido com azeite"-de"-dendê e temperos, que se come puro, com frutos do mar, ou ainda como acompanhamento de pratos de peixe ou carne.}{bo.bó}{0}
\verb{bobo}{ô}{}{}{}{adj.}{Diz"-se daquele que tem pouca inteligência; bocó, tolo.}{bo.bo}{0}
\verb{bobo}{ô}{}{}{}{}{Que não se justifica por ter pouca importância; idiota, tolo.}{bo.bo}{0}
\verb{bobo}{ô}{}{}{}{s.m.}{Indivíduo que, na Idade Média, divertia os reis e grandes senhores.}{bo.bo}{0}
\verb{boboca}{ó}{}{}{}{adj.2g.}{Diz"-se daquele que é muito bobo.}{bo.bo.ca}{0}
\verb{boboca}{ó}{}{}{}{s.2g.}{Indivíduo muito tolo, ingênuo.}{bo.bo.ca}{0}
\verb{boca}{ô}{}{}{}{s.f.}{Cavidade na parte inferior da face ou da cabeça, entrada do tubo digestivo, pela qual os homens e outros animais ingerem os alimentos.}{bo.ca}{0}
\verb{boca}{ô}{}{}{}{}{A parte exterior dessa cavidade, constituída pelos lábios.}{bo.ca}{0}
\verb{boca}{ô}{}{}{}{}{Entrada. (\textit{O carro parou na boca do túnel.})}{bo.ca}{0}
\verb{boca}{ô}{}{}{}{}{Abertura por onde alguma coisa sai.}{bo.ca}{0}
\verb{boca"-aberta}{ô\ldots{}é}{Pop.}{bocas"-abertas ⟨ô\ldots{}é⟩}{}{s.2g.}{Indivíduo que se surpreende com tudo; palerma.}{bo.ca"-a.ber.ta}{0}
\verb{boca"-aberta}{ô\ldots{}é}{Pop.}{bocas"-abertas ⟨ô\ldots{}é⟩}{}{}{Indivíduo indolente e sem cuidados.}{bo.ca"-a.ber.ta}{0}
\verb{boca"-de"-fogo}{ô\ldots{}ô}{}{bocas"-de"-fogo ⟨ô\ldots{}ô⟩}{}{s.m.}{Qualquer peça de artilharia.}{bo.ca"-de"-fo.go}{0}
\verb{boca"-de"-forno}{ô\ldots{}ô}{}{bocas"-de"-forno ⟨ô\ldots{}ô⟩}{}{s.f.}{Brincadeira infantil em que são postas à prova a destreza e a coragem dos participantes.}{bo.ca"-de"-for.no}{0}
\verb{boca"-de"-fumo}{ô}{Pop.}{bocas"-de"-fumo ⟨ô⟩}{}{s.f.}{Ponto de venda de maconha.}{bo.ca"-de"-fu.mo}{0}
\verb{boca"-de"-leão}{ô}{Bot.}{bocas"-de"-leâo ⟨ô⟩}{}{s.f.}{Trepadeira ornamental, cujas flores são de várias cores e têm, na abertura, duas partes sobrepostas que dão a impressão de lábios.}{bo.ca"-de"-le.ão}{0}
\verb{boca"-de"-lobo}{ô\ldots{}ô}{Bras.}{bocas"-de"-lobo ⟨ô\ldots{}ô⟩}{}{s.f.}{Bueiro.}{bo.ca"-de"-lo.bo}{0}
\verb{bocado}{}{}{}{}{s.m.}{Porção de alimento que se leva à boca.}{bo.ca.do}{0}
\verb{bocado}{}{}{}{}{}{Pequena quantidade de qualquer coisa.}{bo.ca.do}{0}
\verb{boca"-do"-lixo}{ô\ldots{}ch}{Bras.}{bocas"-do"-lixo ⟨ô\ldots{}ch⟩}{}{s.f.}{Zona de uma cidade, onde se aglomeram marginais, prostitutas, viciados e traficantes de drogas.}{bo.ca"-do"-li.xo}{0}
\verb{bocaina}{}{}{}{}{s.f.}{Depressão que serve de passagem numa serra.}{bo.cai.na}{0}
\verb{bocaina}{}{}{}{}{}{Vale profundo entre duas elevações do terreno.}{bo.cai.na}{0}
\verb{bocaina}{}{}{}{}{}{Foz de um rio; entrada de um canal.}{bo.cai.na}{0}
\verb{bocaina}{}{}{}{}{}{Braço de água ou furo que liga um lago a um igarapé.}{bo.cai.na}{0}
\verb{bocaiúva}{}{Bot.}{}{}{s.f.}{Palmeira de até sete metros, nativa do Brasil e do Paraguai, com frutos de polpa comestível, amarela e doce, com propriedades expectorantes.}{bo.cai.ú.va}{0}
\verb{bocaiúva}{}{Por ext.}{}{}{}{O fruto dessa palmeira.}{bo.cai.ú.va}{0}
\verb{bocal}{}{}{"-ais}{}{s.m.}{Abertura de vaso, candeeiro, frasco, castiçal etc.}{bo.cal}{0}
\verb{bocal}{}{}{"-ais}{}{}{A parte côncava do castiçal, onde se fixa a vela.}{bo.cal}{0}
\verb{bocal}{}{Mús.}{"-ais}{}{}{Peça móvel que serve de embocadura para certos instrumentos de sopro de metal.}{bo.cal}{0}
\verb{bocal}{}{}{"-ais}{}{}{Peça rosqueada onde se encaixam as lâmpadas.}{bo.cal}{0}
\verb{boçal}{}{}{"-ais}{}{adj.2g.}{Que é ignorante, rude.}{bo.çal}{0}
\verb{boçal}{}{}{"-ais}{}{s.2g.}{Indivíduo que é desprovido de inteligência; grosseiro, estúpido.}{bo.çal}{0}
\verb{boçalidade}{}{}{}{}{s.f.}{Característica ou procedimento de pessoa boçal; ignorância, estupidez.}{bo.ça.li.da.de}{0}
\verb{boca"-livre}{ô}{}{bocas"-livres ⟨ô⟩}{}{s.2g.}{Evento com entrada livre, comida e bebida de graça.}{bo.ca"-li.vre}{0}
\verb{boca"-livre}{ô}{}{bocas"-livres ⟨ô⟩}{}{}{Lugar ou cargo em que se ganha dinheiro irregularmente; mamata.}{bo.ca"-li.vre}{0}
\verb{bocarra}{}{}{}{}{s.f.}{Boca muito grande ou muito aberta.}{bo.car.ra}{0}
\verb{bocejar}{}{}{}{}{v.i.}{Inspirar pela boca quantidade de ar maior que o normal, abrindo"-a bastante em processo geralmente involuntário causado por sono, cansaço, fome ou tédio.}{bo.ce.jar}{0}
\verb{bocejar}{}{Por ext.}{}{}{v.t.}{Dizer por entre bocejos ou em tom de enfado.}{bo.ce.jar}{\verboinum{1}}
\verb{bocejo}{ê}{}{}{}{s.m.}{Ato ou efeito de bocejar; inspiração involuntária longa com a boca aberta.}{bo.ce.jo}{0}
\verb{boceta}{ê}{Pop.}{}{}{s.f.}{Vulva.}{bo.ce.ta}{0}
\verb{boceta}{ê}{Desus.}{}{}{}{Caixa redonda, oval ou oblonga, feita de materiais diversos e usada para guardar pequenos objetos.}{bo.ce.ta}{0}
\verb{boceta}{ê}{Desus.}{}{}{}{Caixa em que se guardava rapé.}{bo.ce.ta}{0}
\verb{bocha}{ó}{}{}{}{s.f.}{Jogo em que cada parceiro, com três bolas de madeira, as atira a certa distância, tentando aproximá"-las tanto quanto possível de outra pequena, chamada bolim.}{bo.cha}{0}
\verb{bocha}{ó}{Por ext.}{}{}{}{Cada uma das bolas empregadas nesse jogo.}{bo.cha}{0}
\verb{bochecha}{ê}{}{}{}{s.f.}{Cada uma das partes laterais e salientes da face.}{bo.che.cha}{0}
\verb{bochechar}{}{}{}{}{v.t.}{Agitar líquido na boca, inflando e desinflando as bochechas.}{bo.che.char}{\verboinum{1}}
\verb{bochecho}{ê}{}{}{}{s.m.}{Ato ou efeito de bochechar.}{bo.che.cho}{0}
\verb{bochecho}{ê}{}{}{}{}{Porção de líquido, muitas vezes com ação medicamentosa, que se bochecha.}{bo.che.cho}{0}
\verb{bochechudo}{}{}{}{}{adj.}{Que tem as bochechas grandes.}{bo.che.chu.do}{0}
\verb{bochincho}{}{}{}{}{s.m.}{Baile popular.}{bo.chin.cho}{0}
\verb{bochincho}{}{}{}{}{}{Briga entre muitas pessoas; desordem.}{bo.chin.cho}{0}
\verb{bócio}{}{Med.}{}{}{s.m.}{Aumento crônico do volume da glândula tireoide, devido à carência de iodo; papeira, papo.}{bó.cio}{0}
\verb{bocó}{}{Bras.}{}{}{adj.}{Diz"-se de pessoa tola, apalermada; bobo, ignorante, infantil.}{bo.có}{0}
\verb{bocó}{}{}{}{}{s.2g.}{Essa pessoa.}{bo.có}{0}
\verb{boda}{ô}{}{}{}{s.f.}{Bodas.}{bo.da}{0}
\verb{bodas}{ô}{}{}{}{s.f.pl.}{Festa ou banquete para celebração do casamento.}{bo.das}{0}
\verb{bode}{ó}{Zool.}{}{}{s.m.}{Mamífero ruminante, de chifres grandes e barba no queixo.}{bo.de}{0}
\verb{bode}{ó}{Fig.}{}{}{}{Situação difícil; complicação, encrenca. (\textit{Deu um bode danado a sua saída do grupo de pesquisa de Geografia.})}{bo.de}{0}
\verb{bodega}{é}{}{}{}{s.f.}{Estabelecimento onde se vendem bebidas e petiscos; taberna.}{bo.de.ga}{0}
\verb{bodega}{é}{}{}{}{}{Pequeno armazém de secos e molhados.}{bo.de.ga}{0}
\verb{bodega}{é}{}{}{}{}{Comida malfeita; gororoba, grude.}{bo.de.ga}{0}
\verb{bodegueiro}{ê}{}{}{}{s.m.}{Proprietário, empregado ou freguês de bodega; taberneiro.}{bo.de.guei.ro}{0}
\verb{bodejar}{}{}{}{}{v.i.}{Emitir a voz (o bode); berrar.}{bo.de.jar}{0}
\verb{bodejar}{}{}{}{}{}{Falar com hesitação, gaguejar.}{bo.de.jar}{\verboinum{1}}
\verb{bodoque}{ó}{}{}{}{s.m.}{Brinquedo infantil que consiste numa forquilha de madeira que prende dois elásticos unidos por um pedaço de couro em que se colocam pequenas pedras para atirar; atiradeira, estilingue.  }{bo.do.que}{0}
\verb{bodum}{}{}{"-uns}{}{s.m.}{Qualquer mau cheiro forte e nauseabundo.}{bo.dum}{0}
\verb{boemia}{}{}{}{}{}{Var. de \textit{boêmia}.}{bo.e.mi.a}{0}
\verb{boêmia}{}{}{}{}{s.f.}{Vida despreocupada, alegre, com bastante diversão noturna.}{bo.ê.mia}{0}
\verb{boêmia}{}{}{}{}{}{Vadiagem, diversões, farras; vida noturna agitada.}{bo.ê.mia}{0}
\verb{boêmio}{}{}{}{}{adj.}{Relativo à Boêmia, região da República Checa.}{bo.ê.mio}{0}
\verb{boêmio}{}{}{}{}{s.m.}{Indivíduo natural ou habitante dessa região.}{bo.ê.mio}{0}
\verb{boêmio}{}{}{}{}{}{A língua falada na Boêmia.}{bo.ê.mio}{0}
\verb{boêmio}{}{}{}{}{adj.}{Que vive na vadiagem.}{bo.ê.mio}{0}
\verb{boêmio}{}{}{}{}{s.m.}{Indivíduo que leva a vida sem preocupação e com diversões e farras noturnas.}{bo.ê.mio}{0}
\verb{bôer}{}{}{}{}{adj.2g.}{Diz"-se de sul"-africano descendente de holandês.}{bô.er}{0}
\verb{bôer}{}{}{}{}{s.2g.}{Esse indivíduo.}{bô.er}{0}
\verb{bofe}{ó}{Pop.}{}{}{s.m.}{Pulmão.}{bo.fe}{0}
\verb{bofe}{ó}{Pop.}{}{}{}{Indivíduo sem atrativos físicos.}{bo.fe}{0}
\verb{bofes}{ó}{}{}{}{s.m.pl.}{Vísceras de animais; fressura, pacuera.}{bo.fes}{0}
\verb{bofes}{ó}{Pop.}{}{}{}{Temperamento, gênio, índole.}{bo.fes}{0}
\verb{bofetada}{}{}{}{}{s.f.}{Tapa dado no rosto com a mão aberta.}{bo.fe.ta.da}{0}
\verb{bofetada}{}{Fig.}{}{}{}{Afronta, insulto, injúria.}{bo.fe.ta.da}{0}
\verb{bofetão}{}{}{"-ões}{}{s.m.}{Bofetada dada com bastante força.}{bo.fe.tão}{0}
\verb{bogari}{}{Bot.}{}{}{s.m.}{Arbusto nativo da Índia, de flores brancas, muito perfumadas, cultivado como ornamental e pela essência que se extrai das flores. }{bo.ga.ri}{0}
\verb{bóhrio}{}{Quím.}{}{}{s.m.}{Elemento químico radioativo artificial, metálico, sólido, de aparência prateada, cuja utilização de restringe à pesquisa científica. \elemento{107}{(262)}{Bh}.}{bóh.rio}{0}
\verb{bói}{}{}{}{}{s.m.}{Indivíduo que trabalha em empresas, hotéis, repartições públicas, fazendo pequenos serviços; contínuo; \textit{boy}.}{bói}{0}
\verb{boi}{}{Zool.}{}{}{s.m.}{Mamífero ruminante, grande e forte, que tem dois chifres ocos, criado para se aproveitar a carne e para utilização no trabalho de carga.}{boi}{0}
\verb{boia}{ó}{}{}{}{s.f.}{Peça que não afunda na água, usada para diversas finalidades. (\textit{Minha irmãzinha ganhou uma boia nova para ir à piscina.})}{boi.a}{0}
\verb{boia}{ó}{Pop.}{}{}{}{Comida; rango.}{boi.a}{0}
\verb{boiada}{}{}{}{}{s.f.}{Manada de bois.}{boi.a.da}{0}
\verb{boiadeiro}{ê}{}{}{}{s.m.}{Indivíduo que guarda ou conduz a boiada; vaqueiro.}{boi.a.dei.ro}{0}
\verb{boiadeiro}{ê}{}{}{}{}{Proprietário, administrador ou revendedor de bovinos.       }{boi.a.dei.ro}{0}
\verb{boia"-fria}{}{}{boias"-frias}{}{s.m.}{Trabalhador rural que come no local de trabalho a comida, fria, que traz de casa.}{boi.a"-fri.a}{0}
\verb{boião}{}{}{"-ões}{}{s.m.}{Vaso bojudo, de boca larga, para guardar líquidos, armazenar conservas etc.           }{boi.ão}{0}
\verb{boião}{}{Pop.}{"-ões}{}{}{Refeição substanciosa, nutritiva.}{boi.ão}{0}
\verb{boiar}{}{}{}{}{v.i.}{Ficar na superfície da água; flutuar.}{boi.ar}{0}
\verb{boiar}{}{Pop.}{}{}{}{Não entender um assunto de que se fala; ficar na mesma.}{boi.ar}{0}
\verb{boiar}{}{Pop.}{}{}{}{Comer a refeição; almoçar ou jantar.}{boi.ar}{\verboinum{1}}
\verb{boi"-bumbá}{}{}{bois"-bumbás \textit{e} bois"-bumbá}{}{s.m.}{Bumba"-meu"-boi, no Norte e Nordeste.}{boi"-bum.bá}{0}
\verb{boicininga}{}{Zool.}{}{}{s.f.}{Cascavel.}{boi.ci.nin.ga}{0}
\verb{boicotagem}{}{}{"-ens}{}{s.f.}{Boicote.}{boi.co.ta.gem}{0}
\verb{boicotar}{}{}{}{}{v.t.}{Prejudicar uma pessoa, um grupo ou um país, criando dificuldades comerciais. (\textit{A imprensa internacional denuncia qualquer tentativa de um governo de querer boicotar ou dificultar o seu trabalho.  })}{boi.co.tar}{\verboinum{1}}
\verb{boicote}{ó}{}{}{}{s.m.}{Ato ou efeito de boicotar, punindo ou criando empecilhos para negócios; boicotagem.  }{boi.co.te}{0}
\verb{boina}{}{}{}{}{s.f.}{Tipo de boné ou gorro, chato ou redondo, sem costura e sem pala.}{boi.na}{0}
\verb{boitatá}{}{Mit.}{}{}{s.m.}{Entidade mitológica ameríndia, gênio protetor dos campos contra aqueles que o incendeiam. Tem a forma de uma cobra de fogo.  }{boi.ta.tá}{0}
\verb{boitatá}{}{}{}{}{}{Fogo"-fátuo.     }{boi.ta.tá}{0}
\verb{boiúna}{}{Mit.}{}{}{s.f.}{Figura mitológica ameríndia, que toma a forma de uma serpente escura, virando as embarcações e levando os náufragos para o fundo do rio.}{boi.ú.na}{0}
\verb{boiúna}{}{Zool.}{}{}{}{Sucuri.}{boi.ú.na}{0}
\verb{bojo}{ô}{}{}{}{s.m.}{Forma arredondada de alguma coisa.}{bo.jo}{0}
\verb{bojo}{ô}{}{}{}{}{Parte de dentro de alguma coisa arredondada.}{bo.jo}{0}
\verb{bojudo}{}{}{}{}{adj.}{Que tem bojo grande; arredondado.}{bo.ju.do}{0}
\verb{bola}{ó}{}{}{}{s.f.}{Qualquer coisa esférica.}{bo.la}{0}
\verb{bola}{ó}{}{}{}{}{Objeto todo redondo, usado em muitos esportes.}{bo.la}{0}
\verb{bola"-ao"-cesto}{ó\ldots{}ê}{Esport.}{bolas"-ao"-cesto ⟨ó\ldots{}ê⟩}{}{s.f.}{Basquetebol.}{bo.la"-ao"-ces.to}{0}
\verb{bolacha}{}{Pop.}{}{}{}{Tapa no rosto com a mão aberta; bofetada.}{bo.la.cha}{0}
\verb{bolacha}{}{}{}{}{s.f.}{Bolinho seco, em geral achatado, feito de farinha de trigo, ovos e açúcar ou sal, assado no forno.}{bo.la.cha}{0}
\verb{bolaço}{}{}{}{}{s.m.}{Pancada ou golpe com bola; bolada.}{bo.la.ço}{0}
\verb{bolaço}{}{Esport.}{}{}{}{No futebol, jogada ou passe executado com maestria ou precisão; bolão.}{bo.la.ço}{0}
\verb{bolada}{}{}{}{}{s.f.}{Pancada ou golpe dado com bola; bolaço.}{bo.la.da}{0}
\verb{bolada}{}{}{}{}{}{Soma vultosa de dinheiro.}{bo.la.da}{0}
\verb{bolandeira}{ê}{}{}{}{s.f.}{Nos engenhos de açúcar, grande roda dentada que gira sobre a moenda.}{bo.lan.dei.ra}{0}
\verb{bolandeira}{ê}{}{}{}{}{Máquina de descaroçar algodão. }{bo.lan.dei.ra}{0}
\verb{bolão}{}{}{"-ões}{}{s.m.}{Bola grande.}{bo.lão}{0}
\verb{bolão}{}{}{"-ões}{}{}{Grande soma de dinheiro, especialmente de loteria.}{bo.lão}{0}
\verb{bolar}{}{}{}{}{v.t.}{Tocar com a bola; acertar com a bola.}{bo.lar}{0}
\verb{bolar}{}{Pop.}{}{}{}{Inventar, arquitetar um plano, ideia, trabalho etc.}{bo.lar}{\verboinum{1}}
\verb{bolas}{ó}{}{}{}{interj.}{Designação de enfado ou reprovação.}{bo.las}{0}
\verb{bolbo}{ô}{Bot.}{}{}{s.m.}{Bulbo.}{bol.bo}{0}
\verb{bolchevique}{}{Hist.}{}{}{adj.2g.}{Relativo ou pertencente à ala majoritária do partido operário russo que tomou o poder em 1917.}{bol.che.vi.que}{0}
\verb{bolchevique}{}{}{}{}{s.2g.}{Membro ou simpatizante dessa ala ou de sua ideologia.}{bol.che.vi.que}{0}
\verb{bolchevismo}{}{Hist.}{}{}{s.m.}{Doutrina da ala majoritária do Partido Operário Social"-Democrata Russo, adepta do marxismo revolucionário de Lênin.}{bol.che.vis.mo}{0}
\verb{bolchevista}{}{}{}{}{adj.2g. e s.2g.}{Bolchevique.}{bol.che.vis.ta}{0}
\verb{boldo}{ô}{Bot.}{}{}{s.m.}{Árvore nativa do Chile, cultivada pelos ramos e folhas medicinais, dos quais se faz chá, usado contra os males digestivos; boldo"-do"-chile.}{bol.do}{0}
\verb{boldrié}{}{}{}{}{s.m.}{Correia a tiracolo, à qual se prende a espada ou outra arma; talabarte, cinturão.}{bol.dri.é}{0}
\verb{boleadeiras}{}{Desus.}{}{}{s.f.pl.}{Artefato constituído por três bolas, de ferro, pedra ou marfim, e ligadas entre si por cordão de couro, manejado pelos campeiros para laçar animais, ou como arma de guerra. }{bo.le.a.dei.ras}{0}
\verb{bolear}{}{}{}{}{v.t.}{Dar a forma de bola; arredondar.}{bo.le.ar}{0}
\verb{bolear}{}{}{}{}{v.t.}{Conduzir; dirigir.}{bo.le.ar}{\verboinum{4}}
\verb{bolear}{}{}{}{}{}{Arremessar as boleadeiras e prender um animal com elas. }{bo.le.ar}{0}
\verb{bolear}{}{Pop.}{}{}{}{Remexer, sacudir, rebolar.}{bo.le.ar}{\verboinum{4}}
\verb{boleeiro}{ê}{}{}{}{s.m.}{Indivíduo que conduz um veículo na boleia; cocheiro.}{bo.le.ei.ro}{0}
\verb{boleia}{é}{}{}{}{s.f.}{Cabina do motorista de caminhão.}{bo.lei.a}{0}
\verb{boleia}{é}{}{}{}{}{Assento do cocheiro na carruagem.}{bo.lei.a}{0}
\verb{bolero}{é}{Mús.}{}{}{s.m.}{Dança espanhola, ou a música que a acompanha, em compasso ternário.}{bo.le.ro}{0}
\verb{bolero}{é}{}{}{}{}{Tipo de casaco curto ou jaqueta, usada por cima de vestido ou  blusa.}{bo.le.ro}{0}
\verb{boletim}{}{}{"-ins}{}{s.m.}{Caderneta escolar em que, periodicamente, são registradas as notas dos estudantes.}{bo.le.tim}{0}
\verb{boletim}{}{}{"-ins}{}{}{Noticiário transmitido em rádio ou televisão, a intervalos regulares.}{bo.le.tim}{0}
\verb{boletim}{}{}{"-ins}{}{}{Publicação periódica de informações de entidades oficiais ou privadas.}{bo.le.tim}{0}
\verb{boleto}{ê}{}{}{}{s.m.}{Impresso, expedido por empresa ou instituição financeira, com instruções para pagamento de dívida em determinada data.}{bo.le.to}{0}
\verb{boleto}{ê}{Bot.}{}{}{}{Nome comum dado a diversos cogumelos, relativamente grandes e de cores variadas, que podem ser comestíveis ou altamente venenosos.}{bo.le.to}{0}
\verb{boleto}{ê}{Veter.}{}{}{}{Articulação arredondada da perna do cavalo, acima do pé.}{bo.le.to}{0}
\verb{bolha}{ô}{}{}{}{s.f.}{Pequeno globo de gás, ar ou vapor que se forma nos líquidos após serem agitados ou por motivo de ebulição ou fermentação.}{bo.lha}{0}
\verb{bolha}{ô}{Med.}{}{}{}{Pequena elevação, cheia de água ou pus, na superfície da pele, causada por inflamação, queimadura ou outros sintomas.}{bo.lha}{0}
\verb{bolha}{ô}{Pop.}{}{}{adj.}{Diz"-se de pessoa sem iniciativa, enfadonha.}{bo.lha}{0}
\verb{boliche}{}{Esport.}{}{}{s.m.}{Jogo que consiste em arremessar uma bola de madeira ou outro material com o objetivo de derrubar dez pinos dispostos na extremidade de uma pista.}{bo.li.che}{0}
\verb{boliche}{}{}{}{}{}{Local onde é praticado esse jogo.}{bo.li.che}{0}
\verb{bólide}{}{Astron.}{}{}{s.m.}{Meteorito de grandes dimensões que, ao entrar na atmosfera terrestre, produz ruído e intensa luminosidade; bólido.}{bó.li.de}{0}
\verb{bólido}{}{}{}{}{}{Var. de \textit{bólide}.}{bó.li.do}{0}
\verb{bolina}{}{}{}{}{s.f.}{Cada um dos cabos de sustentação das velas, destinados a dar"-lhes a curvatura necessária para colher o vento de maneira ideal.}{bo.li.na}{0}
\verb{bolinar}{}{}{}{}{v.t.}{Conduzir um navio ou barco a vela usando a bolina.}{bo.li.nar}{0}
\verb{bolinar}{}{}{}{}{}{Enganar, lograr, burlar.}{bo.li.nar}{0}
\verb{bolinar}{}{Pop.}{}{}{}{Apalpar ou encostar"-se a uma pessoa com fins libidinosos, principalmente em locais públicos.}{bo.li.nar}{\verboinum{1}}
\verb{bolinho}{}{Cul.}{}{}{s.m.}{Pequena porção de massa de forma arredondada, preparada com farinha ou outro ingrediente culinário, e geralmente frita.}{bo.li.nho}{0}
\verb{boliviano}{}{}{}{}{adj.}{Relativo à Bolívia.}{bo.li.vi.a.no}{0}
\verb{boliviano}{}{}{}{}{s.m.}{Indivíduo natural ou habitante desse país.}{bo.li.vi.a.no}{0}
\verb{bolo}{ô}{Cul.}{}{}{s.m.}{Massa feita à base de farinha, quase sempre com açúcar, leite, ovos, fermento e gordura, geralmente assada em forma.}{bo.lo}{0}
\verb{bolo}{ô}{Pop.}{}{}{}{Ajuntamento confuso de pessoas; multidão.}{bo.lo}{0}
\verb{bolo}{ô}{}{}{}{}{Soma de dinheiro em mesa de jogo, formada por rateio ou por apostas de parceiros; bolada.}{bo.lo}{0}
\verb{bolo}{ô}{}{}{}{}{Golpe que era aplicado com régua ou palmatória na palma da mão de criança.}{bo.lo}{0}
\verb{bolonhês}{}{}{}{}{adj.}{Relativo a Bolonha, região da Itália.}{bo.lo.nhês}{0}
\verb{bolonhês}{}{}{}{}{s.m.}{Indivíduo natural ou habitante dessa região.}{bo.lo.nhês}{0}
\verb{bolor}{ô}{}{}{}{s.m.}{Nome vulgar de fungos que causam a decomposição de alimentos, frutas e outros vegetais; mofo.}{bo.lor}{0}
\verb{bolor}{ô}{Fig.}{}{}{}{Condição ou estado do que é ultrapassado, antiquado; decadência.}{bo.lor}{0}
\verb{bolorento}{}{}{}{}{adj.}{Coberto ou cheio de bolor.}{bo.lo.ren.to}{0}
\verb{bolorento}{}{}{}{}{}{Decadente, antiquado, ultrapassado.}{bo.lo.ren.to}{0}
\verb{bolota}{ó}{}{}{}{s.f.}{Fruto do carvalho ou da azinheira.}{bo.lo.ta}{0}
\verb{bolota}{ó}{}{}{}{}{Elevação na pele; bolha, calombo.}{bo.lo.ta}{0}
\verb{bolsa}{ô}{}{}{}{s.f.}{Sacola de couro ou outro material, usada principalmente pelas mulheres para carregar dinheiro e pequenas miudezas.}{bol.sa}{0}
\verb{bolsa}{ô}{}{}{}{}{Quantia concedida a estudantes ou pesquisadores para que prossigam seus trabalhos específicos.}{bol.sa}{0}
\verb{bolsa}{ô}{}{}{}{}{Estabelecimento em que se negociam títulos, ações e mercadorias.}{bol.sa}{0}
\verb{bolsa}{ô}{}{}{}{}{Distensão da pele sob os olhos, causada pelo cansaço ou pela velhice.}{bol.sa}{0}
\verb{bolsão}{}{}{"-ões}{}{s.m.}{Bolsa grande.}{bol.são}{0}
\verb{bolsão}{}{}{"-ões}{}{}{Aquilo que se encontra isolado do todo a que pertence ou que o circunda.}{bol.são}{0}
\verb{bolsista}{}{}{}{}{s.2g.}{Indivíduo que recebe bolsa de estudos ou de viagem.}{bol.sis.ta}{0}
\verb{bolsista}{}{}{}{}{}{Indivíduo que aplica na bolsa de valores ou se dedica profissionalmente às operações da bolsa.}{bol.sis.ta}{0}
\verb{bolso}{ô}{}{}{}{s.m.}{Espécie de pequeno saco de tecido preso à parte interna ou externa da veste, para guardar objetos pessoais.}{bol.so}{0}
\verb{bom}{}{}{bons}{boa ⟨ô⟩}{adj.}{Que possui as qualidades necessárias para o que se precisa; útil.}{bom}{0}
\verb{bom}{}{}{bons}{boa ⟨ô⟩}{}{Que possui qualidades ou características que o tornam querido; caridoso, bondoso.}{bom}{0}
\verb{bom}{}{}{bons}{boa ⟨ô⟩}{}{Moralmente correto; honesto, honrado.}{bom}{0}
\verb{bomba}{}{}{}{}{s.f.}{Projétil cheio de material explosivo.}{bom.ba}{0}
\verb{bomba}{}{}{}{}{s.f.}{Aparelho usado para elevar o nível da água.}{bom.ba}{0}
\verb{bomba}{}{Fig.}{}{}{}{Coisa ruim, de má qualidade.}{bom.ba}{0}
\verb{bomba}{}{Cul.}{}{}{}{Doce com recheio de creme ou chocolate e cobertura de glacê.}{bom.ba}{0}
\verb{bomba}{}{}{}{}{}{Instrumento para encher câmaras"-de"-ar.}{bom.ba}{0}
\verb{bomba}{}{}{}{}{}{Canudo que se coloca na cuia para tomar o mate; bombilha.}{bom.ba}{0}
\verb{bomba}{}{Pop.}{}{}{}{Reprovação na escola.}{bom.ba}{0}
\verb{bombachas}{}{}{}{}{s.f.pl.}{Calças largas e folgadas, presas no tornozelo, usadas pelos gaúchos.}{bom.ba.chas}{0}
\verb{bombada}{}{}{}{}{s.f.}{Cada manobra completa da bomba ao movimentar fluidos.}{bom.ba.da}{0}
\verb{bombada}{}{}{}{}{}{Quantidade de fluido que a bomba movimenta em cada manobra.}{bom.ba.da}{0}
\verb{bombada}{}{Fig.}{}{}{}{Prejuízo, perda, logro.}{bom.ba.da}{0}
\verb{bombado}{}{Pop.}{}{}{adj.}{Que toma substância anabolizante para desenvolver exageradamente a massa muscular.}{bom.ba.do}{0}
\verb{bombar}{}{Pop.}{}{}{v.t.}{Ser reprovado em provas escolares; tomar bomba.}{bom.bar}{\verboinum{1}}
\verb{bombarda}{}{}{}{}{s.f.}{Antiga peça de artilharia do fim da Idade Média, que lançava projéteis de ferro ou pedra.}{bom.bar.da}{0}
\verb{bombarda}{}{Mús.}{}{}{}{Antigo instrumento de palheta dupla, precursor do oboé.}{bom.bar.da}{0}
\verb{bombardão}{}{Mús.}{"-ões}{}{s.m.}{Instrumento de sopro de metal; tuba.}{bom.bar.dão}{0}
\verb{bombardeado}{}{}{}{}{adj.}{Que foi alvo de bombardeio; arrasado com bombas; acabado, arruinado.}{bom.bar.de.a.do}{0}
\verb{bombardeamento}{}{}{}{}{s.m.}{Bombardeio.}{bom.bar.de.a.men.to}{0}
\verb{bombardear}{}{}{}{}{v.t.}{Lançar bombas ou projéteis de artilharia; canhonear.}{bom.bar.de.ar}{0}
\verb{bombardear}{}{Fig.}{}{}{}{Fazer críticas; combater com perguntas; arguir.}{bom.bar.de.ar}{0}
\verb{bombardear}{}{}{}{}{}{Causar prejuízo; transtornar, boicotar.}{bom.bar.de.ar}{0}
\verb{bombardear}{}{Fís.}{}{}{}{Submeter um alvo à irradiação de um feixe de partículas aceleradas, com o fim de obter reações nucleares.}{bom.bar.de.ar}{\verboinum{4}}
\verb{bombardeio}{ê}{}{}{}{s.m.}{Ato ou efeito de bombardear; lançamento de bombas ou projéteis; bombardeamento.}{bom.bar.dei.o}{0}
\verb{bombardeio}{ê}{Fís.}{}{}{}{Ato ou efeito de atingir um alvo atômico ou nuclear com um feixe de partículas aceleradas.}{bom.bar.dei.o}{0}
\verb{bombardeiro}{ê}{}{}{}{s.m.}{Avião empregado para bombardear.}{bom.bar.dei.ro}{0}
\verb{bombardeiro}{ê}{}{}{}{}{Membro da tripulação desse avião.}{bom.bar.dei.ro}{0}
\verb{bombardeiro}{ê}{}{}{}{}{Antiga designação dada ao soldado que manejava a bombarda.}{bom.bar.dei.ro}{0}
\verb{bombardino}{}{Mús.}{}{}{s.m.}{Instrumento metálico de sopro semelhante a uma trompa, usado nas bandas militares.}{bom.bar.di.no}{0}
\verb{bombástico}{}{}{}{}{adj.}{Estrondoso como o barulho de uma bomba; ruidoso.}{bom.bás.ti.co}{0}
\verb{bombástico}{}{Fig.}{}{}{}{De linguagem empolada, pretensiosa.}{bom.bás.ti.co}{0}
\verb{bombeamento}{}{}{}{}{s.m.}{Ato ou efeito de bombear; extração de líquido por meio de máquina ou aparelho próprio.}{bom.be.a.men.to}{0}
\verb{bombear}{}{}{}{}{v.t.}{Retirar um fluido por acionamento de bomba.}{bom.be.ar}{\verboinum{4}}
\verb{bombeiro}{ê}{}{}{}{s.m.}{Soldado que pertence a um corpo de combate de incêndios ou de salvamento.}{bom.bei.ro}{0}
\verb{bombeiro}{ê}{}{}{}{}{Indivíduo que trabalha com bombas hidráulicas; encanador.}{bom.bei.ro}{0}
\verb{bombilha}{}{}{}{}{s.f.}{Canudo de metal ou madeira com que se toma o chimarrão.}{bom.bi.lha}{0}
\verb{bombo}{}{Mús.}{}{}{s.m.}{Grande tambor, tocado em posição vertical com uma única baqueta, usado em orquestras sinfônicas ou bandas militares; bumbo, zabumba.}{bom.bo}{0}
\verb{bombo}{}{}{}{}{}{Músico que toca esse instrumento.}{bom.bo}{0}
\verb{bom"-bocado}{}{Cul.}{bons"-bocados}{}{s.m.}{Doce que leva gemas, maisena, coco ou amêndoas e que é assado no forno.}{bom"-bo.ca.do}{0}
\verb{bombom}{}{Cul.}{"-ons}{}{s.m.}{Confeito de chocolate, geralmente recheado de licor, amêndoas ou pedaços de frutas.}{bom.bom}{0}
\verb{bombordo}{ó}{}{}{}{s.m.}{Lado esquerdo do navio, para quem o observa da popa para a proa.}{bom.bor.do}{0}
\verb{bom"-dia}{}{}{bons"-dias}{}{s.m.}{Saudação que se dirige a alguém na parte da manhã.}{bom"-di.a}{0}
\verb{bom"-senso}{}{}{}{}{s.m.}{O sentimento verdadeiro do que é justo, permitido, conveniente; sensatez. (\textit{Se usarmos de bom"-senso nas negociações, tudo se resolverá em breve.})}{bom"-sen.so}{0}
\verb{bom"-tom}{}{}{}{}{s.m.}{Conjunto de regras que caracterizam a elegância de maneiras e a boa educação; bons modos.}{bom"-tom}{0}
\verb{bonachão}{}{}{"-ões}{}{s.m.}{Indivíduo que tem bondade natural.}{bo.na.chão}{0}
\verb{bonachão}{}{}{"-ões}{}{}{Indivíduo simples, ingênuo, paciente.}{bo.na.chão}{0}
\verb{bonacheirão}{}{}{"-ões}{}{s.m.}{Bonachão.}{bo.na.chei.rão}{0}
\verb{bonança}{}{}{}{}{s.f.}{Condição de tranquilidade do mar, favorável à navegação.}{bo.nan.ça}{0}
\verb{bonança}{}{}{}{}{}{Calmaria, ventura, serenidade.}{bo.nan.ça}{0}
\verb{bonançoso}{ô}{}{"-osos ⟨ó⟩}{"-osa ⟨ó⟩}{adj.}{Que favorece; calmo, tranquilo, estável, ameno.}{bo.nan.ço.so}{0}
\verb{bondade}{}{}{}{}{s.f.}{Qualidade de bom; benevolência, magnanimidade.}{bon.da.de}{0}
\verb{bondade}{}{}{}{}{}{Boa ação; indulgência, clemência.}{bon.da.de}{0}
\verb{bondade}{}{}{}{}{}{Atitude amável; cortesia, brandura.}{bon.da.de}{0}
\verb{bonde}{}{}{}{}{s.m.}{Veículo elétrico de transporte coletivo, urbano ou suburbano, que roda sobre trilhos.}{bon.de}{0}
\verb{bonde}{}{Pop.}{}{}{}{Mau negócio; prejuízo, logro.}{bon.de}{0}
\verb{bonde}{}{Pop.}{}{}{}{Pessoa feia, sem atrativos; bofe, bagulho.}{bon.de}{0}
\verb{bondoso}{ô}{}{"-osos ⟨ó⟩}{"-osa ⟨ó⟩}{adj.}{Que tem bondade; benévolo, humanitário, magnânimo.}{bon.do.so}{0}
\verb{boné}{}{}{}{}{s.m.}{Peça de vestuário para cobrir a cabeça, com uma pala sobre os olhos.}{bo.né}{0}
\verb{boneca}{é}{}{}{}{s.f.}{Figura de louça, papelão, plástico ou pano com a figura de um bebê, de uma criança ou de uma mulher e que serve de brinquedo para crianças.}{bo.ne.ca}{0}
\verb{boneca}{é}{Fig.}{}{}{}{Mulher muito bonita.}{bo.ne.ca}{0}
\verb{boneca}{é}{}{}{}{}{Nome dado à espiga de milho muito verde, ainda em formação.}{bo.ne.ca}{0}
\verb{boneco}{é}{}{}{}{s.m.}{Figura de trapo, plástico, madeira ou outros materiais que imita um indivíduo do sexo masculino; usado como brinquedo infantil.}{bo.ne.co}{0}
\verb{boneco}{é}{}{}{}{}{Indivíduo facilmente manipulado por outrem.}{bo.ne.co}{0}
\verb{boneco}{é}{}{}{}{}{Modelo para confecção de livro, revista ou  qualquer projeto gráfico.}{bo.ne.co}{0}
\verb{bonificação}{}{}{"-ões}{}{s.f.}{Ato ou efeito de bonificar; gratificação, prêmio.}{bo.ni.fi.ca.ção}{0}
\verb{bonificação}{}{}{"-ões}{}{}{Concessão que o vendedor faz ao comprador, fazendo um abatimento no preço da mercadoria ou entregando uma quantidade maior do que a adquirida.}{bo.ni.fi.ca.ção}{0}
\verb{bonificar}{}{}{}{}{v.t.}{Conceder prêmio ou vantagem; gratificar.}{bo.ni.fi.car}{0}
\verb{bonificar}{}{}{}{}{}{Tornar mais produtivo; melhorar, beneficiar.}{bo.ni.fi.car}{\verboinum{2}}
\verb{bonifrate}{}{}{}{}{s.m.}{Pessoa que se deixa manipular pelos outros; fantoche, boneco.}{bo.ni.fra.te}{0}
\verb{bonina}{}{Bot.}{}{}{s.f.}{Planta de flores vistosas, vermelhas ou brancas, que se abrem no final da tarde; maravilha.}{bo.ni.na}{0}
\verb{boniteza}{ê}{}{}{}{s.f.}{Qualidade do que é bonito; formosura, beleza.}{bo.ni.te.za}{0}
\verb{bonito}{}{}{}{}{adj.}{Que é agradável aos sentidos; formoso, belo.}{bo.ni.to}{0}
\verb{bonito}{}{}{}{}{}{Que revela nobreza; generoso, bondoso.}{bo.ni.to}{0}
\verb{bonito}{}{Zool.}{}{}{s.m.}{Nome comum dado a peixes oceânicos, de coloração azul ou esverdeada, cujo tamanho varia entre a cavala e o atum.}{bo.ni.to}{0}
\verb{bonito}{}{}{}{}{}{Bom, vantajoso.}{bo.ni.to}{0}
\verb{bonomia}{}{}{}{}{s.f.}{Qualidade de quem é simples, bom, crédulo.}{bo.no.mi.a}{0}
\verb{bônus}{}{}{}{}{s.m.}{Prêmio ou vantagem concedido aos portadores de certos títulos, cupons, bilhetes de transportes etc.}{bô.nus}{0}
\verb{bônus}{}{}{}{}{}{Título da dívida pública.}{bô.nus}{0}
\verb{bonzo}{}{}{}{}{s.m.}{Sacerdote ou monge budista.}{bon.zo}{0}
\verb{bonzo}{}{Fig.}{}{}{}{Indivíduo sonso, dissimulado, fingido.}{bon.zo}{0}
\verb{boom}{}{}{}{}{s.m.}{Período de expansão da economia.}{\textit{boom}}{0}
\verb{boom}{}{}{}{}{}{Elevação súbita nos preços.}{\textit{boom}}{0}
\verb{boot}{}{Informát.}{}{}{s.m.}{Ato de acionar um computador; inicialização.}{\textit{boot}}{0}
\verb{boqueira}{ê}{Pop.}{}{}{s.f.}{Pequena inflamação nos cantos da boca; afta.}{bo.quei.ra}{0}
\verb{boqueirão}{}{}{"-ões}{}{s.m.}{Abertura grande em costa marítima, rio ou canal.}{bo.quei.rão}{0}
\verb{boqueirão}{}{}{"-ões}{}{}{Passagem estreita entre vertentes.}{bo.quei.rão}{0}
\verb{boqueirão}{}{}{"-ões}{}{}{Grande boca, bocarra.}{bo.quei.rão}{0}
\verb{boquejar}{}{}{}{}{v.t.}{Falar entre os dentes; falar baixo; murmurar, sussurrar.}{bo.que.jar}{0}
\verb{boquejar}{}{}{}{}{}{Falar mal; censurar.}{bo.que.jar}{0}
\verb{boquejar}{}{}{}{}{v.i.}{Dar bocejos.}{bo.que.jar}{\verboinum{1}}
\verb{boquiaberto}{é}{}{}{}{adj.}{Com a boca aberta.}{bo.qui.a.ber.to}{0}
\verb{boquiaberto}{é}{}{}{}{}{Muito admirado, embasbacado, estupefato.}{bo.qui.a.ber.to}{0}
\verb{boquiabrir}{}{}{}{}{v.t.}{Causar grande espanto ou admiração.}{bo.qui.a.brir}{\verboinum{18}}
\verb{boquilha}{}{}{}{}{s.f.}{Parte do cachimbo que se insere na boca.}{bo.qui.lha}{0}
\verb{boquilha}{}{}{}{}{}{Encaixe dos caixilhos de portas e janelas.}{bo.qui.lha}{0}
\verb{boquilha}{}{}{}{}{}{Peça do instrumento de sopro que serve de embocadura.}{bo.qui.lha}{0}
\verb{boquinha}{}{}{}{}{s.f.}{Boca pequena.}{bo.qui.nha}{0}
\verb{boquinha}{}{}{}{}{}{Beijo delicado, terno.}{bo.qui.nha}{0}
\verb{boquinha}{}{}{}{}{}{Refeição ligeira.}{bo.qui.nha}{0}
\verb{bórax}{cs}{Quím.}{}{}{s.m.}{Borato de sódio em forma cristalina e incolor usado como branqueador e antisséptico.}{bó.rax}{0}
\verb{borboleta}{ê}{Zool.}{}{}{s.f.}{Inseto de asas coloridas e antenas compridas.}{bor.bo.le.ta}{0}
\verb{borboleta}{ê}{}{}{}{}{Trava de ferro das janelas de caixilhos, para mantê"-las abertas.}{bor.bo.le.ta}{0}
\verb{borboleta}{ê}{Fig.}{}{}{}{Aparelho que conta o número de passageiros ou de espectadores em cinemas, estádios etc.; catraca, roleta, torniquete.}{bor.bo.le.ta}{0}
\verb{borboletear}{}{}{}{}{v.i.}{Ir de um lugar a outro; andar a esmo; vaguear. (\textit{O rapaz ia aos bailes e ficava borboleteando, sem saber com quem ficar.})}{bor.bo.le.te.ar}{0}
\verb{borboletear}{}{}{}{}{}{Entrar em devaneios; divagar, fantasiar.}{bor.bo.le.te.ar}{0}
\verb{borboletear}{}{}{}{}{}{Dar voos curtos e sem direção certa; voejar, adejar.}{bor.bo.le.te.ar}{\verboinum{4}}
\verb{borborigmo}{}{Med.}{}{}{s.m.}{Ruído surdo produzido pela movimentação de líquidos ou gases contidos no intestino; borborismo.}{bor.bo.rig.mo}{0}
\verb{borborismo}{}{Med.}{}{}{s.m.}{Borborigmo.}{bor.bo.ris.mo}{0}
\verb{borbotão}{}{}{"-ões}{}{s.m.}{Jato impetuoso e volumoso; jorro, borbulhão.}{bor.bo.tão}{0}
\verb{borbotão}{}{}{"-ões}{}{}{Lufada, golfada, rajada.}{bor.bo.tão}{0}
\verb{borbotar}{}{}{}{}{v.t.}{Expelir em jatos fortes.}{bor.bo.tar}{0}
\verb{borbotar}{}{}{}{}{}{Falar ou dizer de forma impetuosa. (\textit{Quando se zangava, borbotava palavrões sem parar. })}{bor.bo.tar}{0}
\verb{borbotar}{}{}{}{}{}{Formar botões; borbulhar.}{bor.bo.tar}{\verboinum{1}}
\verb{borbulha}{}{}{}{}{s.f.}{Pequena bolha de vapor ou de gás.}{bor.bu.lha}{0}
\verb{borbulha}{}{Med.}{}{}{}{Pequena vesícula que se forma na pele, de conteúdo aquoso ou purulento.}{bor.bu.lha}{0}
\verb{borbulhante}{}{}{}{}{adj.2g.}{Que contém borbulhas; efervescente.}{bor.bu.lhan.te}{0}
\verb{borbulhar}{}{}{}{}{v.i.}{Sair em borbulhas; gotejar.}{bor.bu.lhar}{0}
\verb{borbulhar}{}{}{}{}{}{Produzir bolhas pela fervura.}{bor.bu.lhar}{0}
\verb{borbulhar}{}{}{}{}{v.t.}{Introduzir gás em um líquido, formando borbulhas.}{bor.bu.lhar}{\verboinum{1}}
\verb{borco}{ô}{}{}{}{s.m.}{Usado na expressão \textit{de borco}:  de boca para baixo. (\textit{Deixei os copos de borco para que secassem logo.})}{bor.co}{0}
\verb{borda}{ó}{}{}{}{s.f.}{Limite de uma superfície; beira, bordo.}{bor.da}{0}
\verb{borda}{ó}{}{}{}{}{Parte que remata um objeto; contorno, fímbria.}{bor.da}{0}
\verb{borda}{ó}{}{}{}{}{Área que contorna um lago ou rio; orla, margem.}{bor.da}{0}
\verb{bordadeira}{ê}{}{}{}{s.f.}{Mulher que faz bordados.}{bor.da.dei.ra}{0}
\verb{bordadeira}{ê}{}{}{}{}{Acessório que se adapta à máquina de costura, para fazer bordados.}{bor.da.dei.ra}{0}
\verb{bordado}{}{}{}{}{s.m.}{Trabalho decorativo, em relevo, feito sobre tecido com agulha e linha.}{bor.da.do}{0}
\verb{bordado}{}{}{}{}{adj.}{Enfeitado com bordado.}{bor.da.do}{0}
\verb{bordão}{}{}{"-ões}{}{s.m.}{Pedaço roliço de madeira resistente que se leva à mão para servir de apoio; bastão, cajado.}{bor.dão}{0}
\verb{bordão}{}{Fig.}{"-ões}{}{}{Proteção, arrimo, amparo.}{bor.dão}{0}
\verb{bordão}{}{}{"-ões}{}{}{Palavra ou frase repetida muitas vezes, na conversa, na música ou na escrita. (\textit{Ele usava sempre o mesmo bordão nas suas conversas: a culpa é do governo!})}{bor.dão}{0}
\verb{bordar}{}{}{}{}{v.t.}{Enfeitar tecido com elementos decorativos.}{bor.dar}{0}
\verb{bordar}{}{}{}{}{}{Guarnecer, adornar, ornamentar.}{bor.dar}{0}
\verb{bordar}{}{Fig.}{}{}{}{Aumentar uma narrativa com detalhes inventados, fantasiosos.}{bor.dar}{\verboinum{1}}
\verb{bordejar}{}{}{}{}{v.i.}{Navegar sem rumo certo, fazendo ziguezague.}{bor.de.jar}{0}
\verb{bordejar}{}{}{}{}{}{Andar de maneira vacilante; cambalear.}{bor.de.jar}{\verboinum{1}}
\verb{bordejo}{ê}{}{}{}{s.m.}{Ato ou efeito de navegar a esmo, seguindo a direção do vento.}{bor.de.jo}{0}
\verb{bordejo}{ê}{}{}{}{}{Ato ou efeito de andar sem direção certa.}{bor.de.jo}{0}
\verb{bordel}{é}{}{"-éis}{}{s.m.}{Casa de prostituição; prostíbulo.}{bor.del}{0}
\verb{borderô}{}{}{}{}{s.m.}{Nota na qual são discriminados os movimentos de débito e crédito feitos em uma conta bancária.}{bor.de.rô}{0}
\verb{bordo}{ó}{}{}{}{s.m.}{Cada um dos lados de uma embarcação.}{bor.do}{0}
\verb{bordo}{ó}{}{}{}{}{Beira, borda, extremidade.}{bor.do}{0}
\verb{bordo}{ó}{}{}{}{}{Interior do navio ou de qualquer outro tipo de transporte, como avião, trem etc.}{bor.do}{0}
\verb{bordô}{}{}{}{}{adj.2g.}{Que tem cor parecida à do vinho tinto da região de Bordéus, na França.}{bor.dô}{0}
\verb{bordô}{}{}{}{}{s.m.}{Essa cor.}{bor.dô}{0}
\verb{bordo}{ô}{Bot.}{}{}{s.m.}{Nome dado a árvores e arbustos de madeira branca, leve e compacta, própria para obras internas, e de seiva com alto teor de sacarose.}{bor.do}{0}
\verb{bordoada}{}{}{}{}{s.f.}{Golpe ou pancada desferida com bordão; paulada, cacetada.}{bor.do.a.da}{0}
\verb{bordoeira}{ê}{}{}{}{s.f.}{Série de bordoadas; surra.}{bor.do.ei.ra}{0}
\verb{borduna}{}{}{}{}{s.f.}{Arma indígena de ataque, defesa ou caça, feita de madeira dura; porrete, cacete.}{bor.du.na}{0}
\verb{boré}{}{}{}{}{s.m.}{Trombeta de bambu indígena.}{bo.ré}{0}
\verb{boré}{}{}{}{}{}{Mastro de jangada.}{bo.ré}{0}
\verb{boreal}{}{}{"-ais}{}{adj.2g.}{Relativo ou pertencente ao polo Norte; setentrional.}{bo.re.al}{0}
\verb{boreste}{é}{}{}{}{s.m.}{Lado direito da embarcação para quem olha da popa para a proa.}{bo.res.te}{0}
\verb{boricado}{}{Quím.}{}{}{adj.}{Diz"-se da solução que contém ácido bórico, utilizada como antisséptico.}{bo.ri.ca.do}{0}
\verb{bórico}{}{Quím.}{}{}{adj.}{Diz"-se do ácido em forma de cristais brancos que, em solução, forma a água boricada.}{bó.ri.co}{0}
\verb{borla}{ó}{}{}{}{s.f.}{Ornamento de passamanaria formado de um botão do qual pendem inúmeros fios.}{bor.la}{0}
\verb{borla}{ó}{}{}{}{}{Objeto redondo composto de fios ou pelos.}{bor.la}{0}
\verb{bornal}{}{}{"-ais}{}{s.m.}{Saco de pano ou couro, usado a tiracolo, para guardar provisões ou ferramentas.}{bor.nal}{0}
\verb{bornal}{}{}{"-ais}{}{}{Espécie de saco em que se prendem o focinho das cavalgaduras, para nele comerem. }{bor.nal}{0}
\verb{boro}{ó}{Quím.}{}{}{s.m.}{Elemento químico sólido, pouco reativo, usado na fabricação de tintas, esmaltes cerâmicos, vidros especiais, aço, reatores nucleares, semicondutores etc. \elemento{5}{10.811}{B}.}{bo.ro}{0}
\verb{borocoxô}{ch}{Pop.}{}{}{adj.}{Diz"-se de quem está sem coragem, desanimado, envelhecido. (\textit{Ela ficou borocoxô depois que viu o namorado com sua melhor amiga.})}{bo.ro.co.xô}{0}
\verb{bororó}{}{}{}{}{adj.}{Que é pouco instruído; matuto.}{bo.ro.ró}{0}
\verb{borra}{ô}{}{}{}{s.f.}{Sedimento espesso que fica em suspensão em um líquido ou depositado no fundo do recipiente; resíduo.}{bor.ra}{0}
\verb{borra}{ô}{Fig.}{}{}{}{Escória social, ralé.}{bor.ra}{0}
\verb{borra"-botas}{ó\ldots{}ó}{}{}{}{s.m.}{Engraxate inexperiente, trapalhão.}{bor.ra"-bo.tas}{0}
\verb{borra"-botas}{ó\ldots{}ó}{Fig.}{}{}{}{Indivíduo sem importância, reles; joão"-ninguém.}{bor.ra"-bo.tas}{0}
\verb{borracha}{}{}{}{}{s.f.}{Substância elástica natural, obtida do látex de certas plantas tropicais como a seringueira, ou sintética, obtida por processos químicos. }{bor.ra.cha}{0}
\verb{borracha}{}{}{}{}{}{Artigo de escritório, feito com esse material, usado para apagar traços de lápis ou de tinta.}{bor.ra.cha}{0}
\verb{borracha}{}{Bras.}{}{}{}{Cassetete.}{bor.ra.cha}{0}
\verb{borracharia}{}{}{}{}{s.f.}{Oficina mecânica especializada no reparo ou na venda de pneus, câmaras de ar e afins.}{bor.ra.cha.ri.a}{0}
\verb{borracheira}{ê}{}{}{}{s.f.}{Comportamento ou palavas de borracho, bêbado; embriaguez, bebedeira.}{bor.ra.chei.ra}{0}
\verb{borracheira}{ê}{}{}{}{}{Grosseria, brutalidade, desconsideração.}{bor.ra.chei.ra}{0}
\verb{borracheiro}{ê}{}{}{}{s.m.}{Indivíduo que conserta ou vende pneus e câmaras"-de"-ar.}{bor.ra.chei.ro}{0}
\verb{borracheiro}{ê}{}{}{}{}{Indivíduo que trabalha na extração de látex; seringueiro.}{bor.ra.chei.ro}{0}
\verb{borracho}{}{}{}{}{adj.}{Que está embriagado ou é dado ao vício de beber.}{bor.ra.cho}{0}
\verb{borracho}{}{}{}{}{s.m.}{Indivíduo bêbado, ébrio.}{bor.ra.cho}{0}
\verb{borracho}{}{Zool.}{}{}{}{Pombo implume, ou que ainda não voa.}{bor.ra.cho}{0}
\verb{borrachudo}{}{Zool.}{}{}{s.m.}{Inseto de pequeno porte e coloração negra, que ocorre na faixa litorânea, principalmente próximo a cachoeiras e rios, cuja fêmea se alimenta de sangue e tem hábitos diurnos.}{bor.ra.chu.do}{0}
\verb{borradela}{é}{}{}{}{s.f.}{Camada de tinta aplicada grosseiramente; mancha, borrão.}{bor.ra.de.la}{0}
\verb{borrador}{ô}{}{}{}{s.m.}{Livro em que os comerciantes anotam diariamente as operações.}{bor.ra.dor}{0}
\verb{borrador}{ô}{}{}{}{}{Caderno de rascunho.}{bor.ra.dor}{0}
\verb{borrador}{ô}{}{}{}{}{Mau pintor.}{bor.ra.dor}{0}
\verb{borrador}{ô}{}{}{}{}{Mau escritor.}{bor.ra.dor}{0}
\verb{borralha}{}{}{}{}{s.f.}{Cinzas do fogão. (\textit{A borralha estava espalhada pelo chão.})}{bor.ra.lha}{0}
\verb{borralheira}{ê}{}{}{}{s.f.}{Lugar onde se acumula a borralha ou as cinzas do forno ou da lareira.}{bor.ra.lhei.ra}{0}
\verb{borralheiro}{ê}{}{}{}{s.m.}{Borralheira.}{bor.ra.lhei.ro}{0}
\verb{borralheiro}{ê}{}{}{}{adj.}{Que gosta de ficar junto ao borralho, na cozinha.}{bor.ra.lhei.ro}{0}
\verb{borralheiro}{ê}{}{}{}{}{Que sai pouco de casa.}{bor.ra.lhei.ro}{0}
\verb{borralho}{}{}{}{}{s.m.}{Brasido coberto de cinzas, quase apagado.}{bor.ra.lho}{0}
\verb{borralho}{}{}{}{}{}{Cinzas quentes.}{bor.ra.lho}{0}
\verb{borralho}{}{Fig.}{}{}{}{Lar, lareira.}{bor.ra.lho}{0}
\verb{borrão}{}{}{"-ões}{}{s.m.}{Mancha de tinta; borradela.}{bor.rão}{0}
\verb{borrão}{}{}{"-ões}{}{}{Esboço, rascunho.}{bor.rão}{0}
\verb{borrão}{}{Fig.}{"-ões}{}{}{Ação indigna; desonra.}{bor.rão}{0}
\verb{borrar}{}{}{}{}{v.t.}{Sujar com borrão; manchar.}{bor.rar}{0}
\verb{borrar}{}{}{}{}{}{Riscar, rabiscar.}{bor.rar}{0}
\verb{borrar}{}{}{}{}{}{Pintar grosseiramente.}{bor.rar}{0}
\verb{borrar}{}{Pop.}{}{}{v.i.}{Sujar com fezes; defecar.}{bor.rar}{\verboinum{2}}
\verb{borrasca}{}{}{}{}{s.f.}{Vento forte e súbito acompanhado de chuva.}{bor.ras.ca}{0}
\verb{borrasca}{}{}{}{}{}{Tempestade no mar.}{bor.ras.ca}{0}
\verb{borrasca}{}{Fig.}{}{}{}{Contrariedade repentina; inquietação, desgosto.}{bor.ras.ca}{0}
\verb{borrascoso}{ô}{}{"-osos ⟨ó⟩}{"-osa ⟨ó⟩}{adj.}{Que traz, que promete borrasca, ou em que há borrasca.}{bor.ras.co.so}{0}
\verb{borrego}{ê}{}{}{}{s.m.}{Cordeiro com menos de um ano.}{bor.re.go}{0}
\verb{borrego}{ê}{Fig.}{}{}{}{Indivíduo sossegado, manso.}{bor.re.go}{0}
\verb{borrifador}{ô}{}{}{}{adj.}{Que borrifa, que molha com borrifos.}{bor.ri.fa.dor}{0}
\verb{borrifador}{ô}{}{}{}{s.m.}{Indivíduo ou coisa que borrifa, que umedece ou molha com borrifos.}{bor.ri.fa.dor}{0}
\verb{borrifador}{ô}{}{}{}{}{Recipiente, em geral cilíndrico, munido de um bico, que serve sobretudo para regar plantas.}{bor.ri.fa.dor}{0}
\verb{borrifar}{}{}{}{}{v.t.}{Umedecer ou molhar, aspergindo ou dispersando gotículas; lançar borrifos.}{bor.ri.far}{0}
\verb{borrifar}{}{}{}{}{}{Orvalhar, esborrifar, aspergir.}{bor.ri.far}{\verboinum{1}}
\verb{borrifo}{}{}{}{}{s.m.}{Ato ou efeito de borrifar; difusão ou aspersão de gotas de água ou de outro líquido.}{bor.ri.fo}{0}
\verb{borrifo}{}{}{}{}{}{Pequenas gotas de chuva ou de orvalho.}{bor.ri.fo}{0}
\verb{borrifo}{}{}{}{}{}{Conjunto de pequenos fios de água que passam pelo crivo do regador.}{bor.ri.fo}{0}
\verb{borzeguim}{}{}{"-ins}{}{s.m.}{Botina cujo cano é fechado com cordões.}{bor.ze.guim}{0}
\verb{bósnio}{}{}{}{}{adj.}{Relativo à Bósnia"-Herzegóvina.}{bós.nio}{0}
\verb{bósnio}{}{}{}{}{s.m.}{Indivíduo natural ou habitante desse país.}{bós.nio}{0}
\verb{bosque}{ó}{}{}{}{s.m.}{Quantidade mais ou menos considerável de árvores dispostas aproximadamente entre si; mata, floresta.}{bos.que}{0}
\verb{bosquejar}{}{}{}{}{v.t.}{Fazer bosquejo; delinear, esboçar, resumir. (\textit{Há em sua casa muitos fragmentos de obras que foram bosquejadas sobre assuntos de higiene caseira. })}{bos.que.jar}{\verboinum{1}}
\verb{bosquejo}{ê}{}{}{}{s.m.}{Primeiros traços, ainda imprecisos, que antecedem o plano geral de uma obra; esboço, rascunho.}{bos.que.jo}{0}
\verb{bossa}{ó}{}{}{}{s.f.}{Inchaço produzido por uma pancada; galo.}{bos.sa}{0}
\verb{bossa}{ó}{}{}{}{}{Elevação arredondada nas costas de alguns animais; corcova, corcunda.}{bos.sa}{0}
\verb{bossa}{ó}{}{}{}{}{Jeito natural que se tem para alguma coisa; aptidão, inclinação, vocação. (\textit{A garota tinha a bossa da música e da dança.})}{bos.sa}{0}
\verb{bosta}{ó}{}{}{}{s.f.}{Excremento de gado bovino ou de outros animais.}{bos.ta}{0}
\verb{bosta}{ó}{Pop.}{}{}{}{Coisa mal feita, de má qualidade.}{bos.ta}{0}
\verb{bosta}{ó}{Pop.}{}{}{interj.}{Expressão que denota desagrado, contrariedade.}{bos.ta}{0}
\verb{bota}{ó}{}{}{}{s.f.}{Calçado de couro ou borracha que envolve o pé, a perna, e às vezes a coxa, usado para proteger contra o frio, montar a cavalo etc.}{bo.ta}{0}
\verb{bota"-fora}{ó\ldots{}ó}{}{bota"-foras ⟨ó\ldots{}ó⟩}{}{s.m.}{Despedida de quem sai de viagem, com festa ou acompanhando"-o até o momento da partida. (\textit{Iremos ao bota"-fora da minha prima, que vai para a Austrália amanhã.})}{bo.ta"-fo.ra}{0}
\verb{botânica}{}{Biol.}{}{}{s.f.}{Campo da biologia que estuda a morfologia e a fisiologia do reino vegetal.}{bo.tâ.ni.ca}{0}
\verb{botânico}{}{Bot.}{}{}{adj.}{Relativo à botânica. (\textit{Os vocabulários botânicos são feitos por especialistas.})}{bo.tâ.ni.co}{0}
\verb{botânico}{}{Bot.}{}{}{s.m.}{Indivíduo especializado em botânica.}{bo.tâ.ni.co}{0}
\verb{botão}{}{}{"-ões}{}{s.m.}{A flor antes de abrir.}{bo.tão}{0}
\verb{botão}{}{}{"-ões}{}{}{Folha ou ramo que está nascendo numa planta; broto.}{bo.tão}{0}
\verb{botão}{}{}{"-ões}{}{}{Pequena peça que se prega sobre o pano, própria para unir as duas bandas de uma roupa, fechando"-a.}{bo.tão}{0}
\verb{botão}{}{}{"-ões}{}{}{Pequena peça de máquinas ou instrumentos que se aperta ou gira.}{bo.tão}{0}
\verb{botar}{}{}{}{}{v.t.}{Colocar pessoa ou coisa em algum lugar; pôr.}{bo.tar}{0}
\verb{botar}{}{}{}{}{}{Colocar alguma coisa em alguma parte de seu próprio corpo. (\textit{Ele botou a camisa verde.})}{bo.tar}{0}
\verb{botar}{}{}{}{}{}{Fazer pessoa ou coisa ficar em determinado estado. (\textit{Precisamos botar a casa em ordem.})}{bo.tar}{0}
\verb{botar}{}{}{}{}{}{Fazer alguma coisa sair de dentro do próprio corpo. (\textit{A galinha bota ovos.})}{bo.tar}{\verboinum{1}}
\verb{bote}{ó}{}{}{}{s.m.}{Embarcação miúda, sem cobertura, geralmente movida a remo. (\textit{Os pescadores desceram do bote para acampar. })}{bo.te}{0}
\verb{bote}{ó}{}{}{}{s.m.}{Golpe de arma branca.}{bo.te}{0}
\verb{bote}{ó}{}{}{}{}{Ataque do animal sobre a presa. (\textit{O bote de uma cobra pode ser fatal.})}{bo.te}{0}
\verb{bote}{ó}{Fig.}{}{}{}{Ataque, investida. (\textit{Os caçadores fizeram uma tocaia para dar um bote no animal.})}{bo.te}{0}
\verb{boteco}{é}{}{}{}{s.m.}{Botequim.}{bo.te.co}{0}
\verb{botelha}{ê}{}{}{}{s.f.}{Garrafa, frasco.}{bo.te.lha}{0}
\verb{botelha}{ê}{Por ext.}{}{}{}{O conteúdo dessa garrafa.}{bo.te.lha}{0}
\verb{botequim}{}{}{"-ins}{}{s.m.}{Estabelecimento comercial onde se servem bebidas em geral e pequenos lanches; bar.}{bo.te.quim}{0}
\verb{botequineiro}{ê}{}{}{}{s.m.}{Dono ou administrador de botequim.}{bo.te.qui.nei.ro}{0}
\verb{botica}{}{}{}{}{s.f.}{Estabelecimento onde se preparam e vendem medicamentos; farmácia.}{bo.ti.ca}{0}
\verb{boticão}{}{}{"-ões}{}{s.m.}{Tenaz usado para extrações em cirurgia óssea; fórceps.}{bo.ti.cão}{0}
\verb{boticão}{}{}{"-ões}{}{}{Espécie de alicate ou tenaz para arrancar dentes.}{bo.ti.cão}{0}
\verb{boticário}{}{}{}{}{s.m.}{Dono de botica.}{bo.ti.cá.rio}{0}
\verb{boticário}{}{}{}{}{}{Indivíduo que prepara e vende medicamentos na botica.}{bo.ti.cá.rio}{0}
\verb{botija}{}{}{}{}{s.f.}{Vaso de cerâmica, cilíndrico, de boca estreita, gargalo curto e uma pequena asa.}{bo.ti.ja}{0}
\verb{botija}{}{Fig.}{}{}{}{Indivíduo gordo.}{bo.ti.ja}{0}
\verb{botijão}{}{}{"-ões}{}{s.m.}{Recipiente metálico ou de outro material, usado para armazenar produtos voláteis.}{bo.ti.jão}{0}
\verb{botijão}{}{}{"-ões}{}{}{Recipiente metálico, usado para entrega de gás a domicílio.}{bo.ti.jão}{0}
\verb{botina}{}{}{}{}{s.f.}{Bota de cano baixo, usada por homens.}{bo.ti.na}{0}
\verb{boto}{ô}{Zool.}{}{}{s.m.}{Mamífero provido de dentes, marinho e de água doce.}{bo.to}{0}
\verb{botocudo}{}{}{}{}{s.m.}{Denominação dada pelos portugueses a indígena pertencente a grupos linguisticamente distintos, por usarem botoques.}{bo.to.cu.do}{0}
\verb{botocudo}{}{}{}{}{adj.}{Relativo a botocudo.}{bo.to.cu.do}{0}
\verb{botoeira}{ê}{}{}{}{s.f.}{Casa para botão nas roupas.}{bo.to.ei.ra}{0}
\verb{botoque}{ó}{}{}{}{s.m.}{Rodela grande, de uso entre os botocudos e outros indígenas brasileiros, para ser introduzida em furos artificiais feitos nos lóbulos da orelha, narinas e beiço inferior.}{bo.to.que}{0}
\verb{botulismo}{}{Med.}{}{}{s.m.}{Intoxicação alimentar de natureza infecciosa, causada pela toxina de um bacilo que se desenvolve em alimentos inadequadamente enlatados ou conservados.}{bo.tu.lis.mo}{0}
\verb{bouba}{ô}{Med.}{}{}{s.f.}{Doença tropical contagiosa, causada por um germe semelhante ao da sífilis, caracterizada por lesões cutâneas na face e nas extremidades.}{bou.ba}{0}
\verb{bovídeo}{}{Zool.}{}{}{s.m.}{Espécime dos bovídeos, mamíferos ruminantes, que possuem os dedos protegidos por cascos e são providos de chifres, estritamente herbívoros e de grande importância econômica, como os bois, os búfalos e os antílopes.}{bo.ví.deo}{0}
\verb{bovídeo}{}{}{}{}{adj.}{Relativo aos bovídeos.}{bo.ví.deo}{0}
\verb{bovino}{}{}{}{}{adj.}{Relativo a boi.}{bo.vi.no}{0}
\verb{bovinocultor}{ô}{}{}{}{s.m.}{Indivíduo que se dedica à bovinocultura.}{bo.vi.no.cul.tor}{0}
\verb{bovinocultura}{}{}{}{}{s.f.}{Criação de gado bovino.}{bo.vi.no.cul.tu.ra}{0}
\verb{boxe}{ócs}{}{}{}{s.m.}{Divisão que separa o local do chuveiro do resto do banheiro.}{bo.xe}{0}
\verb{boxe}{ócs}{}{}{}{}{Cada uma das divisões de um prédio, reservadas para a mesma finalidade.}{bo.xe}{0}
\verb{boxe}{ócs}{Esport.}{}{}{}{Luta esportiva em que duas pessoas trocam socos com luvas especiais.}{bo.xe}{0}
\verb{boxeador}{cs\ldots{}ô}{}{}{}{s.m.}{Indivíduo que luta boxe.}{bo.xe.a.dor}{0}
\verb{boy}{}{}{}{}{s.m.}{Indivíduo de qualquer idade, empregado num escritório para fazer trabalhos de entregas, visitas a bancos; contínuo. }{\textit{boy}}{0}
\verb{boy}{}{Pop.}{}{}{}{Indivíduo rico ou que ostenta riqueza, geralmente ocioso, jovem e solteiro, e de vida social intensa.  }{\textit{boy}}{0}
\verb{bozó}{}{}{}{}{s.m.}{Jogo de dados em que se atiram os cubos dentro de um cilindro de folha"-de"-flandres ou de um copo de couro, só se descobrindo o lance depois de feitas as apostas.}{bo.zó}{0}
\verb{Br}{}{Quím.}{}{}{}{Símb. do \textit{bromo}. }{Br}{0}
\verb{brabeza}{ê}{}{}{}{}{Var. de \textit{braveza}.}{bra.be.za}{0}
\verb{brabo}{}{}{}{}{}{Var. de \textit{bravo}.}{bra.bo}{0}
\verb{braça}{}{}{}{}{s.f.}{Antiga unidade de medida equivalente a dez palmos, ou seja, a 2,2 m.}{bra.ça}{0}
\verb{braçada}{}{}{}{}{s.f.}{Aquilo que se pode abranger com os braços.}{bra.ça.da}{0}
\verb{braçada}{}{}{}{}{}{Movimento dos braços, na natação.}{bra.ça.da}{0}
\verb{braçadeira}{ê}{}{}{}{s.f.}{Faixa distintiva que se usa no braço, sobre a manga.}{bra.ça.dei.ra}{0}
\verb{braçadeira}{ê}{}{}{}{}{Correia, faixa ou peça para reforçar, prender etc.}{bra.ça.dei.ra}{0}
\verb{braçadeira}{ê}{}{}{}{}{Correia ou argola fixada atrás do escudo e por onde se enfiava o braço.}{bra.ça.dei.ra}{0}
\verb{braçal}{}{}{"-ais}{}{adj.2g.}{Relativo a braço.}{bra.çal}{0}
\verb{braçal}{}{}{"-ais}{}{}{Diz"-se de trabalho, geralmente pesado, que se utiliza da força muscular, especialmente a dos braços.}{bra.çal}{0}
\verb{bracejar}{}{}{}{}{v.i.}{Agitar os braços; gesticular.}{bra.ce.jar}{\verboinum{1}}
\verb{bracelete}{ê}{}{}{}{s.m.}{Adorno em forma de aro e feito de materiais diversos, que se usa no pulso, braço ou antebraço; pulseira.}{bra.ce.le.te}{0}
\verb{bracelete}{ê}{Por ext.}{}{}{}{Algema, cadeia.}{bra.ce.le.te}{0}
\verb{braço}{}{Anat.}{}{}{s.m.}{Cada um dos membros superiores do corpo humano.}{bra.ço}{0}
\verb{braço}{}{Anat.}{}{}{}{A porção desses membros que fica entre o ombro e o cotovelo.}{bra.ço}{0}
\verb{braço}{}{Por ext.}{}{}{}{Parte alongada de algum objeto, especialmente com função de suporte.}{bra.ço}{0}
\verb{braço}{}{}{}{}{}{Ramificação de rio ou mar que se prolonga para fora de seu curso principal.}{bra.ço}{0}
\verb{braço"-de"-ferro}{é}{}{braços"-de"-ferro ⟨é⟩}{}{s.m.}{Jogo para medir força, em que cada um dos dois competidores, de mãos dadas, tenta encostar o antebraço do outro na superfície sobre a qual apoiam o cotovelo; queda"-de"-braço.}{bra.ço"-de"-fer.ro}{0}
\verb{braço"-de"-ferro}{é}{Fig.}{braços"-de"-ferro ⟨é⟩}{}{}{Luta, embate.}{bra.ço"-de"-fer.ro}{0}
\verb{bráctea}{}{Bot.}{}{}{s.f.}{Folha modificada, localizada abaixo de uma flor ou inflorescência.}{brác.te.a}{0}
\verb{bradar}{}{}{}{}{v.t.}{Transmitir em voz alta; divulgar, apregoar.}{bra.dar}{0}
\verb{bradar}{}{}{}{}{}{Reclamar em alta voz ou com veemência; exigir. (\textit{Os trabalhadores bradavam pedindo aumento de salário.})}{bra.dar}{0}
\verb{bradar}{}{Fig.}{}{}{}{Mostrar"-se discrepante; destoar.}{bra.dar}{\verboinum{1}}
\verb{brado}{}{}{}{}{s.m.}{Súplica, protesto, queixa.}{bra.do}{0}
\verb{braga}{}{}{}{}{s.f.}{Tipo de calção, geralmente curto e largo, usado até a Idade Média e, modernamente, por alguns muçulmanos; bragas.}{bra.ga}{0}
\verb{bragantino}{}{}{}{}{adj.}{Relativo às cidades de Bragança (Portugal), Bragança (\textsc{pa}) ou Bragança Paulista (\textsc{sp}).}{bra.gan.ti.no}{0}
\verb{bragantino}{}{}{}{}{s.m.}{Indivíduo natural ou habitante de uma dessas cidades.}{bra.gan.ti.no}{0}
\verb{bragantino}{}{}{}{}{adj.}{Relativo à dinastia portuguesa dos Braganças.}{bra.gan.ti.no}{0}
\verb{bragantino}{}{}{}{}{s.m.}{Membro dessa dinastia.}{bra.gan.ti.no}{0}
\verb{bragas}{}{}{}{}{s.f.pl.}{Braga.}{bra.gas}{0}
\verb{braguilha}{}{}{}{}{s.f.}{Abertura dianteira de calças, calções, cuecas.}{bra.gui.lha}{0}
\verb{braile}{}{}{}{}{s.m.}{Sistema de escrita que utiliza combinações de pontos em relevo para leitura por meio do tato, utilizado por portadores de deficiência visual.}{brai.le}{0}
\verb{brâmane}{}{}{}{}{adj.2g.}{Relativo aos brâmanes, membros hereditários da mais alta casta dos hindus, antigamente a dos sacerdotes e, modernamente, a dos homens livres.}{brâ.ma.ne}{0}
\verb{brâmane}{}{}{}{}{s.2g.}{Membro dessa casta.}{brâ.ma.ne}{0}
\verb{bramanismo}{}{Relig.}{}{}{s.m.}{A religião indiana.}{bra.ma.nis.mo}{0}
\verb{bramar}{}{}{}{}{v.i.}{Dar bramidos; bramir, rugir, berrar.}{bra.mar}{0}
\verb{bramar}{}{}{}{}{}{Estar no cio.}{bra.mar}{0}
\verb{bramar}{}{Por ext.}{}{}{}{Enfurecer"-se, zangar"-se.}{bra.mar}{\verboinum{1}}
\verb{bramido}{}{}{}{}{s.m.}{Rugido de fera.}{bra.mi.do}{0}
\verb{bramido}{}{}{}{}{}{Ato ou efeito de bramir; berro.}{bra.mi.do}{0}
\verb{bramido}{}{Por ext.}{}{}{}{Reclamação, acusação ou ofensa em alta voz.}{bra.mi.do}{0}
\verb{bramir}{}{}{}{}{v.i.}{Soltar bramidos.}{bra.mir}{0}
\verb{bramir}{}{Por ext.}{}{}{}{Gritar, vociferar.}{bra.mir}{0}
\verb{bramir}{}{Por ext.}{}{}{}{Exaltar"-se, irritar"-se, exigir aos gritos.}{bra.mir}{\verboinum{34}\verboirregular{\emph{def.} brame, bramem}}
\verb{brancacento}{}{}{}{}{adj.}{Quase branco.}{bran.ca.cen.to}{0}
\verb{brancarana}{}{Bras.}{}{}{s.f.}{Mulata clara.}{bran.ca.ra.na}{0}
\verb{brancarão}{}{}{"-ões}{}{adj.}{Diz"-se de indivíduo mulato claro.}{bran.ca.rão}{0}
\verb{branco}{}{}{}{}{adj.}{Da cor do leite; alvo.}{bran.co}{0}
\verb{branco}{}{}{}{}{}{Que tem cor mais clara que outro da mesma espécie. (\textit{Vinho branco.})}{bran.co}{0}
\verb{branco}{}{}{}{}{}{Que tem a pele mais clara. (\textit{A raça branca.})}{bran.co}{0}
\verb{branco}{}{}{}{}{s.m.}{Cor de leite. (\textit{O branco da parede.})}{bran.co}{0}
\verb{branco}{}{}{}{}{}{Espaço entre linhas escritas ou impressas; lacuna.}{bran.co}{0}
\verb{branco}{}{Fig.}{}{}{}{Momento de esquecimento.}{bran.co}{0}
\verb{brancura}{}{}{}{}{s.f.}{Qualidade de branco; alvura, branquidão.}{bran.cu.ra}{0}
\verb{brandir}{}{}{}{}{v.t.}{Erguer a arma antes de disparar.}{bran.dir}{0}
\verb{brandir}{}{}{}{}{}{Mover a mão ou algo com a mão; acenar, agitar. (\textit{As crianças brandiam bandeiras para receber as autoridades.})}{bran.dir}{0}
\verb{brandir}{}{}{}{}{v.i.}{Oscilar, balançar, agitar"-se.}{bran.dir}{\verboinum{34}\verboirregular{\emph{def.} brandimos, brandis}}
\verb{brando}{}{}{}{}{adj.}{De pouca intensidade; suave, fraco.}{bran.do}{0}
\verb{brando}{}{}{}{}{}{Macio, mole, tenro.}{bran.do}{0}
\verb{brando}{}{}{}{}{}{Dócil, flexível, afável.}{bran.do}{0}
\verb{brandura}{}{}{}{}{s.f.}{Qualidade de brando; baixa intensidade, lentidão, ternura. (\textit{A professora falava sempre com brandura.})}{bran.du.ra}{0}
\verb{branqueamento}{}{}{}{}{s.m.}{Ato, efeito ou processo de branquear.}{bran.que.a.men.to}{0}
\verb{branqueamento}{}{Fig.}{}{}{}{Purgação, purificação.}{bran.que.a.men.to}{0}
\verb{branquear}{}{}{}{}{v.t.}{Tornar branco; alvejar.}{bran.que.ar}{0}
\verb{branquear}{}{}{}{}{}{Pintar de branco; caiar.}{bran.que.ar}{0}
\verb{branquear}{}{}{}{}{}{Limpar, assear.}{bran.que.ar}{\verboinum{4}}
\verb{branquejar}{}{}{}{}{v.i.}{Tornar branco aos poucos; branquear.}{bran.que.jar}{0}
\verb{branquejar}{}{}{}{}{}{Mostrar"-se branco; destacar"-se, realçar.}{bran.que.jar}{\verboinum{1}}
\verb{brânquia}{}{Zool.}{}{}{s.f.}{Órgão respiratório da maioria dos animais aquáticos.}{brân.quia}{0}
\verb{branquinha}{}{Pop.}{}{}{s.f.}{Aguardente de cana; cachaça.}{bran.qui.nha}{0}
\verb{braquilogia}{}{Gram.}{}{}{s.f.}{Redução da forma de uma palavra ou expressão sem alteração do conteúdo.}{bra.qui.lo.gi.a}{0}
\verb{brasa}{}{}{}{}{s.f.}{Carvão incandescente sem chama.}{bra.sa}{0}
\verb{brasa}{}{}{}{}{}{O estado de incandescência.}{bra.sa}{0}
\verb{brasa}{}{Fig.}{}{}{}{Desejo ardente; ardor, paixão.}{bra.sa}{0}
\verb{brasa}{}{Fig.}{}{}{}{Cólera, ira.}{bra.sa}{0}
\verb{brasão}{}{}{"-ões}{}{s.m.}{Conjunto de figuras que compõem o distintivo de pessoa, família, cidade ou Estado.}{bra.são}{0}
\verb{brasão}{}{}{"-ões}{}{}{A peça ou insígnia composta com tais elementos.}{bra.são}{0}
\verb{brasão}{}{Fig.}{"-ões}{}{}{Honra, glória.}{bra.são}{0}
\verb{braseira}{ê}{}{}{}{s.f.}{Grande quantidade de brasas; braseiro.}{bra.sei.ra}{0}
\verb{braseira}{ê}{}{}{}{}{Recipiente em que ficam as brasas para aquecer o ambiente.}{bra.sei.ra}{0}
\verb{braseira}{ê}{Pop.}{}{}{}{Aguardente de cana; cachaça.}{bra.sei.ra}{0}
\verb{braseiro}{ê}{}{}{}{s.m.}{Grande quantidade de brasas.}{bra.sei.ro}{0}
\verb{braseiro}{ê}{Por ext.}{}{}{}{Fogaréu, incêndio.}{bra.sei.ro}{0}
\verb{braseiro}{ê}{Por ext.}{}{}{}{Calor intenso.}{bra.sei.ro}{0}
\verb{braseiro}{ê}{}{}{}{}{Recipiente em que ficam as brasas para aquecer o ambiente; braseira.}{bra.sei.ro}{0}
\verb{brasileirismo}{}{Gram.}{}{}{s.m.}{Fato linguístico próprio do português do Brasil.}{bra.si.lei.ris.mo}{0}
\verb{brasileirismo}{}{}{}{}{}{Qualidade peculiar de quem é brasileiro.}{bra.si.lei.ris.mo}{0}
\verb{brasileirismo}{}{}{}{}{}{Sentimento de afinidade em relação ao Brasil.}{bra.si.lei.ris.mo}{0}
\verb{brasileiro}{ê}{}{}{}{adj.}{Relativo ao Brasil.}{bra.si.lei.ro}{0}
\verb{brasileiro}{ê}{}{}{}{s.m.}{Indivíduo natural ou habitante desse país.}{bra.si.lei.ro}{0}
\verb{brasiliana}{}{}{}{}{s.f.}{Coleção de livros, publicações, filmes etc. sobre o Brasil.}{bra.si.li.a.na}{0}
\verb{brasilianista}{}{}{}{}{adj.2g.}{Diz"-se de especialista, geralmente estrangeiro, dedicado aos assuntos brasileiros.}{bra.si.li.a.nis.ta}{0}
\verb{brasílico}{}{}{}{}{adj.}{Diz"-se da gente e das coisas nativas do Brasil.}{bra.sí.li.co}{0}
\verb{brasílico}{}{}{}{}{}{Relativo ao Brasil.}{bra.sí.li.co}{0}
\verb{brasilidade}{}{}{}{}{s.f.}{Qualidade peculiar de quem é brasileiro.}{bra.si.li.da.de}{0}
\verb{brasilidade}{}{}{}{}{}{Sentimento de afinidade em relação ao Brasil.}{bra.si.li.da.de}{0}
\verb{brasiliense}{}{}{}{}{adj.2g.}{Relativo a Brasília, capital do Brasil.}{bra.si.li.en.se}{0}
\verb{brasiliense}{}{}{}{}{s.2g.}{Indivíduo natural ou habitante dessa cidade.}{bra.si.li.en.se}{0}
\verb{bravata}{}{}{}{}{s.f.}{Ato ou dito que envolve arrogância.}{bra.va.ta}{0}
\verb{bravata}{}{}{}{}{}{Ato ou dito caracterizado por presunção, jactância, fanfarronada.}{bra.va.ta}{0}
\verb{bravatear}{}{}{}{}{v.i.}{Fazer ou dizer bravatas.}{bra.va.te.ar}{\verboinum{4}}
\verb{braveza}{ê}{}{}{}{s.f.}{Qualidade de bravo; valentia, intrepidez, coragem.}{bra.ve.za}{0}
\verb{braveza}{ê}{}{}{}{}{Ferocidade, selvageria, violência.}{bra.ve.za}{0}
\verb{bravio}{}{}{}{}{adj.}{Selvagem, bruto, feroz.}{bra.vi.o}{0}
\verb{bravio}{}{}{}{}{}{Rústico, rude, agreste, árduo.}{bra.vi.o}{0}
\verb{bravo}{}{}{}{}{adj.}{Corajoso, valente, intrépido.}{bra.vo}{0}
\verb{bravo}{}{}{}{}{}{Furioso, exaltado, severo.}{bra.vo}{0}
\verb{bravo}{}{}{}{}{}{Rústico, agreste, silvestre, bravio.}{bra.vo}{0}
\verb{bravo}{}{}{}{}{s.m.}{Indivíduo valente.}{bra.vo}{0}
\verb{bravo}{}{}{}{}{interj.}{Expressão que denota aprovação, entusiasmo, dirigida a um artista após seu bom desempenho.}{bra.vo}{0}
\verb{bravura}{}{}{}{}{s.f.}{Qualidade de bravo; coragem.}{bra.vu.ra}{0}
\verb{bravura}{}{}{}{}{}{Ato de valentia.}{bra.vu.ra}{0}
\verb{bravura}{}{}{}{}{}{Ímpeto, ferocidade, violência, selvageria.}{bra.vu.ra}{0}
\verb{break}{}{}{}{}{s.m.}{No jazz e em outros gêneros musicais, interrupção que faz o conjunto para deixar solar um dos músicos.}{\textit{break}}{0}
\verb{break}{}{}{}{}{}{Pausa que se faz num espetáculo; intervalo.}{\textit{break}}{0}
\verb{break}{}{}{}{}{}{Dança de rua, própria dos grandes centros urbanos, que se notabilizou pelos passos com evoluções complexas e difíceis de se realizar.}{\textit{break}}{0}
\verb{brear}{}{}{}{}{v.t.}{Cobrir de breu ou material semelhante.}{bre.ar}{0}
\verb{brear}{}{Fig.}{}{}{}{Tornar escuro; sujar.}{bre.ar}{\verboinum{4}}
\verb{breca}{é}{}{}{}{s.f.}{Câimbra.}{bre.ca}{0}
\verb{breca}{é}{}{}{}{}{Usado na expressão \textit{levado da breca}: que faz travessuras; muito levado.}{bre.ca}{0}
\verb{breca}{é}{}{}{}{}{Usado na expressão \textit{levar a breca}: ficar completamente quebrado.}{bre.ca}{0}
\verb{brecar}{}{}{}{}{v.t.}{Deter ou diminuir o movimento; frear.}{bre.car}{0}
\verb{brecar}{}{Fig.}{}{}{}{Conter, moderar, controlar. (\textit{A mãe começou a brecar as saídas noturnas dos filhos.})}{bre.car}{\verboinum{2}}
\verb{brecha}{é}{}{}{}{s.f.}{Fenda, rachadura, abertura.}{bre.cha}{0}
\verb{brecha}{é}{}{}{}{}{Espaço vazio; lacuna.}{bre.cha}{0}
\verb{brechó}{}{Bras.}{}{}{s.m.}{Estabelecimento de belchior, negociante de roupas, livros e objetos usados.}{bre.chó}{0}
\verb{brega}{é}{}{}{}{adj.2g.}{Diz"-se de coisa ou pessoa deselegante, que revela mau gosto; cafona.}{bre.ga}{0}
\verb{brejeirice}{}{}{}{}{s.f.}{Ato ou dito próprio de brejeiro.}{bre.jei.ri.ce}{0}
\verb{brejeirice}{}{Bras.}{}{}{}{Qualidade de quem é alegre, brincalhão.}{bre.jei.ri.ce}{0}
\verb{brejeiro}{ê}{}{}{}{adj.}{Relativo a brejo.}{bre.jei.ro}{0}
\verb{brejeiro}{ê}{}{}{}{}{Vadio, vagabundo.}{bre.jei.ro}{0}
\verb{brejeiro}{ê}{}{}{}{}{Brincalhão, travesso.}{bre.jei.ro}{0}
\verb{brejeiro}{ê}{}{}{}{}{Malicioso, lúbrico.}{bre.jei.ro}{0}
\verb{brejo}{é}{}{}{}{s.m.}{Terreno pantanoso, alagadiço e lodoso; banhado, charco, pântano, paul.}{bre.jo}{0}
\verb{brenha}{}{}{}{}{s.f.}{Mata espessa, cerrada, de difícil trânsito; matagal, selva.}{bre.nha}{0}
\verb{brenha}{}{Fig.}{}{}{}{Complicação, confusão.}{bre.nha}{0}
\verb{brenha}{}{Fig.}{}{}{}{Segredo, mistério.}{bre.nha}{0}
\verb{breque}{é}{}{}{}{s.m.}{Freio.}{bre.que}{0}
\verb{breque}{é}{Bras.}{}{}{s.m.}{No samba, interrupção súbita na execução dos músicos para que o cantor faça alguma intervenção, geralmente falada e de caráter humorístico.}{bre.que}{0}
\verb{bretão}{}{}{"-ões}{"-ã}{adj.}{Pertencente ou relativo à Grã"-Bretanha, ou à Inglaterra.}{bre.tão}{0}
\verb{bretão}{}{}{"-ões}{"-ã}{}{Relativo ou pertencente à Bretanha (França).}{bre.tão}{0}
\verb{bretão}{}{}{"-ões}{"-ã}{s.m.}{Natural ou habitante da Grã"-Bretanha, ou da Inglaterra.}{bre.tão}{0}
\verb{bretão}{}{}{"-ões}{"-ã}{}{Natural ou habitante da Bretanha.}{bre.tão}{0}
\verb{breu}{ê}{}{}{}{s.m.}{Substância semelhante ao pez, obtida pela destilação do alcatrão de hulha ou a partir de secreções resinosas de certas plantas.}{breu}{0}
\verb{breu}{ê}{Fig.}{}{}{}{Grande escuridão; trevas.}{breu}{0}
\verb{breve}{é}{}{}{}{adj.2g.}{Que tem curta duração; rápido.}{bre.ve}{0}
\verb{brevê}{}{}{}{}{s.m.}{Diploma que certifica que um indivíduo está apto a pilotar aviões.}{bre.vê}{0}
\verb{breve}{é}{}{}{}{adv.}{Dentro de um curto espaço de tempo; brevemente, em breve.}{bre.ve}{0}
\verb{breve}{é}{Mús.}{}{}{s.f.}{Figura da notação musical, que tem a duração de duas semibreves.}{bre.ve}{0}
\verb{breviário}{}{Relig.}{}{}{s.m.}{Livro das rezas diárias dos sacerdotes católicos.}{bre.vi.á.rio}{0}
\verb{breviário}{}{}{}{}{}{Sinopse, resumo.}{bre.vi.á.rio}{0}
\verb{breviário}{}{Por ext.}{}{}{}{Livro de cabeceira; livro predileto, \textit{vade"-mécum}.}{bre.vi.á.rio}{0}
\verb{brevidade}{}{}{}{}{s.f.}{Qualidade de breve; concisão, laconismo.}{bre.vi.da.de}{0}
\verb{brevidade}{}{Cul.}{}{}{}{Bolinho de polvilho, assado no forno.}{bre.vi.da.de}{0}
\verb{bricabraque}{}{Bras.}{}{}{s.m.}{Estabelecimento que vende ou compra objetos usados e antiguidades, ou o conjunto desses objetos.}{bri.ca.bra.que}{0}
\verb{brida}{}{}{}{}{s.f.}{Rédea.}{bri.da}{0}
\verb{brida}{}{}{}{}{}{Usado na locução \textit{a toda a brida}: a toda velocidade, em disparada.}{bri.da}{0}
\verb{bridão}{}{}{"-ões}{}{s.m.}{Brida grande. }{bri.dão}{0}
\verb{bridão}{}{}{"-ões}{}{}{Freio simples usado nas corridas de cavalo.}{bri.dão}{0}
\verb{bridge}{}{}{}{}{s.m.}{Jogo de vazas, entre quatro jogadores divididos em duplas, que utiliza as 52 cartas do baralho.}{\textit{bridge}}{0}
\verb{briga}{}{}{}{}{s.f.}{Ato ou efeito de brigar; contenda, luta, rixa.}{bri.ga}{0}
\verb{brigada}{}{}{}{}{s.f.}{Corpo militar, geralmente com dois regimentos.}{bri.ga.da}{0}
\verb{brigada}{}{}{}{}{}{Conjunto de duas ou três baterias de campanha.}{bri.ga.da}{0}
\verb{brigadeiro}{ê}{}{}{}{s.m.}{Denominação comum a \textit{brigadeiro"-do"-ar}, \textit{major"-brigadeiro} e \textit{tenente"-brigadeiro}, as três mais altas patentes da Aeronáutica brasileira.}{bri.ga.dei.ro}{0}
\verb{brigadeiro}{ê}{Cul.}{}{}{s.m.}{Doce feito com leite condensado e chocolate cozidos, na forma de bolinhas cobertas de chocolate granulado. }{bri.ga.dei.ro}{0}
\verb{brigadeiro"-do"-ar}{ê}{}{brigadeiros"-do"-ar}{}{s.m.}{Posto da hierarquia da Aeronáutica  intermediário entre o de major"-brigadeiro e o de coronel"-aviador. }{bri.ga.dei.ro"-do"-ar}{0}
\verb{brigadeiro"-do"-ar}{ê}{}{brigadeiros"-do"-ar}{}{}{Militar que ocupa esse posto.}{bri.ga.dei.ro"-do"-ar}{0}
\verb{brigalhada}{}{}{}{}{s.f.}{Brigas ou altercações longas ou generalizadas.}{bri.ga.lha.da}{0}
\verb{brigão}{}{}{"-ões}{"-ona}{adj.}{Diz"-se daquele que é dado a brigas.}{bri.gão}{0}
\verb{brigar}{}{}{}{}{v.t.}{Atacar outra pessoa e defender"-se dela sem obedecer a nenhuma regra; trocar socos. (\textit{Os torcedores brigaram depois do jogo.})}{bri.gar}{0}
\verb{brigar}{}{}{}{}{}{Usar palavras violentas contra alguém. (\textit{O vizinho brigou com minha prima por causa do cachorro dela.})}{bri.gar}{\verboinum{5}}
\verb{brigue}{}{Desus.}{}{}{s.m.}{Antigo navio a vela, de dois mastros e velas redondas.}{bri.gue}{0}
\verb{briguento}{}{}{}{}{s.m.}{Brigão.}{bri.guen.to}{0}
\verb{brilhante}{}{}{}{}{adj.2g.}{Que brilha; cintilante, reluzente, resplandecente.}{bri.lhan.te}{0}
\verb{brilhante}{}{Fig.}{}{}{}{Talentoso, genial; muito habilidoso.}{bri.lhan.te}{0}
\verb{brilhante}{}{}{}{}{s.m.}{Diamante lapidado.}{bri.lhan.te}{0}
\verb{brilhantina}{}{}{}{}{s.f.}{Cosmético usado para lustrar ou fixar o cabelo.}{bri.lhan.ti.na}{0}
\verb{brilhantismo}{}{}{}{}{s.m.}{Qualidade do que é brilhante; brilho, resplandecência, cintilação. }{bri.lhan.tis.mo}{0}
\verb{brilhantismo}{}{Fig.}{}{}{}{Suntuosidade, magnificência.}{bri.lhan.tis.mo}{0}
\verb{brilhantismo}{}{Fig.}{}{}{}{Maestria, talento.}{bri.lhan.tis.mo}{0}
\verb{brilhar}{}{}{}{}{v.i.}{Lançar brilho; luzir, resplandecer.}{bri.lhar}{0}
\verb{brilhar}{}{}{}{}{}{Fazer"-se notar em sua atividade; destacar"-se, notabilizar"-se, sobressair.}{bri.lhar}{\verboinum{1}}
\verb{brilho}{}{}{}{}{s.m.}{Luz muito viva, que um corpo emite ou reflete; cintilação, resplandecência.}{bri.lho}{0}
\verb{brilho}{}{Fig.}{}{}{}{Luxo, suntuosidade, opulência.}{bri.lho}{0}
\verb{brim}{}{}{}{}{s.m.}{Tecido resistente de algodão, linho etc.}{brim}{0}
\verb{brincadeira}{ê}{}{}{}{s.f.}{Ato de brincar.}{brin.ca.dei.ra}{0}
\verb{brincadeira}{ê}{}{}{}{}{Ato que se pratica sem o desejo de ofender, apenas usando outra pessoa para fazer graça; gozação.}{brin.ca.dei.ra}{0}
\verb{brincalhão}{}{}{"-ões}{"-ona}{adj.}{Diz"-se daquele que é dado a fazer brincadeiras, a divertir a si mesmo e aos outros; folgazão. }{brin.ca.lhão}{0}
\verb{brincar}{}{}{}{}{v.i.}{Fazer alguma coisa por prazer, sem outra finalidade; divertir"-se.}{brin.car}{0}
\verb{brincar}{}{}{}{}{}{Fazer uma brincadeira com alguém; mexer, fazer uma gozação.}{brin.car}{\verboinum{2}}
\verb{brinco}{}{}{}{}{s.m.}{Joia ou bijuteria que se usa presa à  orelha ou pendente dela.}{brin.co}{0}
\verb{brinco}{}{Fig.}{}{}{}{Pessoa ou coisa fina, delicada.}{brin.co}{0}
\verb{brinco}{}{Fig.}{}{}{}{Qualquer coisa elaborada ou organizada com esmero, primor.}{brin.co}{0}
\verb{brinco"-de"-princesa}{ê}{Bot.}{brincos"-de"-princesa ⟨ê⟩}{}{s.m.}{Nome comum aos arbustos que se apoiam em suportes para subir, de flores vermelhas ou violáceas, muito apreciados como ornamentais; fúcsia.}{brin.co"-de"-prin.ce.sa}{0}
\verb{brindar}{}{}{}{}{v.t.}{Dirigir ou levantar um brinde à saúde de alguém, em comemoração a algo ou pelo bom êxito de alguém.}{brin.dar}{\verboinum{1}}
\verb{brinde}{}{}{}{}{s.m.}{Ato ou efeito de brindar.}{brin.de}{0}
\verb{brinde}{}{Por ext.}{}{}{}{Discurso ou palavras de saudação ou comemoração proferidas nesse ato. }{brin.de}{0}
\verb{brinde}{}{}{}{}{}{Presente, oferta, dádiva.}{brin.de}{0}
\verb{brinde}{}{}{}{}{}{Objeto ou oferta condicionada à compra de certa mercadoria.}{brin.de}{0}
\verb{brinde}{}{}{}{}{}{Objeto promocional oferecido a determinado público com objetivos de \textit{marketing}.}{brin.de}{0}
\verb{brinquedo}{ê}{}{}{}{s.m.}{Coisa com que se brinca.}{brin.que.do}{0}
\verb{brio}{}{}{}{}{s.m.}{Sentimento da prória dignidade, honra; amor"-próprio.}{bri.o}{0}
\verb{brio}{}{}{}{}{}{Coragem, bravura.}{bri.o}{0}
\verb{brio}{}{}{}{}{}{Altivez, garbo, galhardia.}{bri.o}{0}
\verb{brioche}{ó}{Cul.}{}{}{s.m.}{Pãozinho de massa leve, doce ou não, feito com farinha, manteiga e ovos.}{bri.o.che}{0}
\verb{brioso}{ô}{}{"-osos ⟨ó⟩}{"-osa ⟨ó⟩}{adj.}{Que tem brio.}{bri.o.so}{0}
\verb{brioso}{ô}{}{"-osos ⟨ó⟩}{"-osa ⟨ó⟩}{}{Orgulhoso, garboso.}{bri.o.so}{0}
\verb{brioso}{ô}{}{"-osos ⟨ó⟩}{"-osa ⟨ó⟩}{}{Corajoso, valente.}{bri.o.so}{0}
\verb{brisa}{}{}{}{}{s.f.}{Vento suave e fresco; aragem, viração, fresca.}{bri.sa}{0}
\verb{brisa}{}{Pop.}{}{}{}{Falta de dinheiro; pindaíba, quebradeira.}{bri.sa}{0}
\verb{brita}{}{}{}{}{s.f.}{Pedra quebrada manual ou mecanicamente em pequenos fragmentos de vários tamanhos, usada na construção de estradas, na composição de concretos etc.}{bri.ta}{0}
\verb{britadeira}{ê}{}{}{}{s.f.}{Máquina de britar.}{bri.ta.dei.ra}{0}
\verb{britador}{ô}{}{}{}{s.m.}{Britadeira.}{bri.ta.dor}{0}
\verb{britador}{ô}{}{}{}{adj.}{Que brita.}{bri.ta.dor}{0}
\verb{britânico}{}{}{}{}{adj.}{Relativo à Grã"-Bretanha (Inglaterra, Escócia e País de Gales), ou ao Reino Unido (Grã"-Bretanha e Irlanda do Norte).}{bri.tâ.ni.co}{0}
\verb{britânico}{}{}{}{}{}{Que apresenta certas características atribuídas aos ingleses.}{bri.tâ.ni.co}{0}
\verb{britânico}{}{}{}{}{s.m.}{Indivíduo natural ou habitante desse país.}{bri.tâ.ni.co}{0}
\verb{britar}{}{}{}{}{v.t.}{Quebrar pedra, reduzindo"-a a fragmentos de determinadas dimensões.}{bri.tar}{0}
\verb{britar}{}{}{}{}{}{Picar, fragmentar, partir, triturar.}{bri.tar}{\verboinum{1}}
\verb{broa}{ô}{Cul.}{}{}{s.f.}{Pão ou bolinho redondo, doce, geralmente de fubá de milho.}{bro.a}{0}
\verb{broa}{ô}{Pop.}{}{}{s.m.}{Indivíduo gorducho.}{bro.a}{0}
\verb{broca}{ó}{}{}{}{s.f.}{Instrumento composto de uma haste pontuda, acionada por manivela, usado para perfurar, abrir buracos circulares.}{bro.ca}{0}
\verb{broca}{ó}{Bras.}{}{}{}{Nome comum a todos os insetos, adultos ou em sua forma larvar, que se nutrem  perfurando ou corroendo madeira, papel etc.}{bro.ca}{0}
\verb{brocado}{}{}{}{}{s.m.}{Tecido de seda com relevos bordados a ouro ou prata.}{bro.ca.do}{0}
\verb{brocado}{}{}{}{}{adj.}{Com o feitio ou à semelhança de brocado.}{bro.ca.do}{0}
\verb{broca"-do"-café}{ó}{Zool.}{brocas"-do"-café ⟨ó⟩}{}{s.f.}{Besouro que habita os trópicos, originário da África, cuja larva se alimenta da semente verde do cafeeiro.}{bro.ca"-do"-ca.fé}{0}
\verb{brocar}{}{}{}{}{v.t.}{Perfurar com broca; furar, esburacar.}{bro.car}{0}
\verb{brocar}{}{}{}{}{}{Corroer, carcomer, estragar.}{bro.car}{0}
\verb{brocar}{}{}{}{}{v.i.}{Contar mentiras; lorotar.}{bro.car}{\verboinum{2}}
\verb{brocardo}{}{}{}{}{s.m.}{Máxima, axioma jurídico.}{bro.car.do}{0}
\verb{brocardo}{}{Por ext.}{}{}{}{Qualquer máxima, sentença, provérbio.}{bro.car.do}{0}
\verb{brocha}{ó}{}{}{}{s.f.}{Prego curto, de cabeça chata e larga.}{bro.cha}{0}
\verb{brochado}{}{}{}{}{adj.}{Livro não encadernado.}{bro.cha.do}{0}
\verb{brochar}{}{}{}{}{v.t.}{Cravar brochas; pregar, fixar.}{bro.char}{\verboinum{1}}
\verb{brochar}{}{}{}{}{v.t.}{Costurar as folhas dos livros depois de dobradas e ordenadas, colando"-lhes em seguida a capa.}{bro.char}{\verboinum{1}}
\verb{broche}{ó}{}{}{}{s.m.}{Enfeite de metal ou pedraria, com um alfinete de fecho, para adornar peça de vestuário.}{bro.che}{0}
\verb{brochura}{}{}{}{}{s.f.}{Ato, efeito ou arte de brochar livros; brochagem.}{bro.chu.ra}{0}
\verb{brochura}{}{}{}{}{}{Livro, folheto ou opúsculo brochado.}{bro.chu.ra}{0}
\verb{brócolis}{}{Bot.}{}{}{s.m.}{Nome comum aos cultivares da couve, da família das crucíferas, muito apreciadas como verdura.}{bró.co.lis}{0}
\verb{brócolos}{}{Bot.}{}{}{s.m.}{Brócolis.}{bró.co.los}{0}
\verb{bródio}{}{}{}{}{s.m.}{Refeição ou banquete farto e alegre; comezaina, patuscada.}{bró.dio}{0}
\verb{brometo}{ê}{Quím.}{}{}{s.m.}{Qualquer sal que contenha o ânion simples do bromo.}{bro.me.to}{0}
\verb{bromo}{}{Quím.}{}{}{s.m.}{Elemento químico do grupo dos halogênios, avermelhado, tóxico, de cheiro desagradável, utilizado na fabricação de corantes e produtos farmacêuticos. \elemento{35}{79.904}{Br}.}{bro.mo}{0}
\verb{bronca}{}{}{}{}{s.f.}{Ato ou efeito de bronquear; reprimenda, repreensão, descompostura.}{bron.ca}{0}
\verb{bronco}{}{}{}{}{adj.}{Áspero, agreste.}{bron.co}{0}
\verb{bronco}{}{Fig.}{}{}{}{Rude, ignorante, grosseiro, estúpido. }{bron.co}{0}
\verb{broncopneumonia}{}{Med.}{}{}{s.f.}{Inflamação pulmonar aguda.}{bron.co.pneu.mo.ni.a}{0}
\verb{bronquear}{}{}{}{}{v.t.}{Dar bronca; repreender.}{bron.que.ar}{\verboinum{4}}
\verb{brônquico}{}{Anat.}{}{}{adj.}{Relativo ao brônquio.}{brôn.qui.co}{0}
\verb{brônquio}{}{Anat.}{}{}{s.m.}{Cada um dos dois canais que resultam da bifurcação da traqueia, e também as subdivisões, formando a árvore brônquica.  }{brôn.quio}{0}
\verb{bronquíolo}{}{Anat.}{}{}{s.m.}{Cada uma das sub"-ramificações da árvore brônquica, que penetram nos alvéolos pulmonares.}{bron.quí.o.lo}{0}
\verb{bronquite}{}{Med.}{}{}{s.f.}{Inflamação da mucosa da traqueia e dos brônquios.}{bron.qui.te}{0}
\verb{brontossauro}{}{Paleo.}{}{}{s.m.}{Nome comum aos dinossauros quadrúpedes, de pescoço comprido e cauda forte, herbívoros, que podiam alcançar 30 t e 22 m de comprimento e que viveram no Jurássico.}{bron.tos.sau.ro}{0}
\verb{bronze}{}{Quím.}{}{}{s.m.}{Liga metálica de cobre e estanho.}{bron.ze}{0}
\verb{bronze}{}{Fig.}{}{}{}{Sino.}{bron.ze}{0}
\verb{bronze}{}{}{}{}{}{Escultura em bronze.}{bron.ze}{0}
\verb{bronze}{}{Pop.}{}{}{}{Bronzeado.}{bron.ze}{0}
\verb{bronzeado}{}{}{}{}{adj.}{Da cor do bronze; brônzeo.}{bron.ze.a.do}{0}
\verb{bronzeado}{}{}{}{}{}{Revestido ou guarnecido de bronze.}{bron.ze.a.do}{0}
\verb{bronzeado}{}{Bras.}{}{}{}{Amorenado, moreno, queimado.}{bron.ze.a.do}{0}
\verb{bronzeado}{}{}{}{}{s.m.}{Cor da pele exposta longa ou repetidamente ao sol.}{bron.ze.a.do}{0}
\verb{bronzeador}{ô}{}{}{}{adj.}{Que bronzeia.}{bron.ze.a.dor}{0}
\verb{bronzeador}{ô}{}{}{}{s.m.}{Indivíduo que faz bronzeamento de objetos.}{bron.ze.a.dor}{0}
\verb{bronzeador}{ô}{}{}{}{}{Substância própria para bronzear a pele.}{bron.ze.a.dor}{0}
\verb{bronzeamento}{}{}{}{}{s.m.}{Ato ou efeito de bronzear. }{bron.ze.a.men.to}{0}
\verb{bronzeamento}{}{}{}{}{}{Escurecimento da pele exposta aos raios solares.}{bron.ze.a.men.to}{0}
\verb{bronzeamento}{}{}{}{}{}{Bronzagem.}{bron.ze.a.men.to}{0}
\verb{bronzear}{}{}{}{}{v.t.}{Dar a cor escura do bronze a pessoa ou coisa.}{bron.ze.ar}{0}
\verb{bronzear}{}{}{}{}{v.pron.}{Ficar ao sol para ficar com a pele mais escura. (\textit{Os rapazes se bronzeiam na praia.})}{bron.ze.ar}{\verboinum{4}}
\verb{brônzeo}{}{}{}{}{adj.}{Da cor ou do feitio do bronze.}{brôn.ze.o}{0}
\verb{broquel}{é}{}{"-éis}{}{s.m.}{Escudo antigo, redondo e pequeno, com uma broca no centro.}{bro.quel}{0}
\verb{brotar}{}{}{}{}{v.i.}{Sair da semente; germinar, nascer.}{bro.tar}{0}
\verb{brotar}{}{}{}{}{}{Sair da terra; jorrar. (\textit{A água brota do solo.})}{bro.tar}{\verboinum{1}}
\verb{brotinho}{}{}{}{}{s.m.}{Broto pequeno.}{bro.ti.nho}{0}
\verb{brotinho}{}{Pop.}{}{}{}{Jovem no início da adolescência.}{bro.ti.nho}{0}
\verb{broto}{ô}{Bot.}{}{}{s.m.}{Órgão vegetal na fase inicial de seu desenvolvimento, ou rebento que dá origem à nova planta; gema, vergôntea, brotamento, abrolho, brotação, gomo, renovo, grelo, muda.  }{bro.to}{0}
\verb{broto}{ô}{Pop.}{}{}{}{Namorada ou namorado.}{bro.to}{0}
\verb{broto}{ô}{Pop.}{}{}{}{Brotinho.}{bro.to}{0}
\verb{brotoeja}{ê}{}{}{}{s.f.}{Erupção cutânea em que aparecem vesículas vermelhas seguidas de coceira; pereba.}{bro.to.e.ja}{0}
\verb{broxa}{óch}{}{}{}{s.f.}{Espécie de pincel grande, feito com cerdas grossas de qualidade inferior, empregado em caiação e noutros tipos de pintura pouco apurada.   }{bro.xa}{0}
\verb{broxa}{óch}{Pop.}{}{}{s.m.}{Indivíduo sem potência sexual.  }{bro.xa}{0}
\verb{broxante}{ch}{Pop.}{}{}{adj.2g.}{Cansativo, importuno.}{bro.xan.te}{0}
\verb{broxar}{ch}{}{}{}{v.t.}{Pintar, caiar com broxa.}{bro.xar}{0}
\verb{broxar}{ch}{Pop.}{}{}{v.i.}{Perder, ocasional ou definitivamente, a potência sexual.}{bro.xar}{\verboinum{1}}
\verb{bruaca}{}{}{}{}{s.f.}{Sacola de couro cru para transportar objetos ou mantimentos sobre bestas.}{bru.a.ca}{0}
\verb{bruaca}{}{Pop.}{}{}{}{Mulher feia, má ou faladeira, especialmente as idosas.}{bru.a.ca}{0}
\verb{brucelose}{ó}{Med.}{}{}{s.f.}{Infecção generalizada causada por bactérias, comum aos bovinos, caprinos e suínos, transmissível ao homem por estes, e cujos sintomas incluem febre, anemia, nevralgias, dores articulares e suores.  }{bru.ce.lo.se}{0}
\verb{bruços}{}{}{}{}{s.m.pl.}{Elemento usado apenas na expressão \textit{de bruços}: deitado com a barriga voltada para baixo.  }{bru.ços}{0}
\verb{bruma}{}{}{}{}{s.f.}{Nevoeiro, cerração, neblina (especialmente no mar).}{bru.ma}{0}
\verb{bruma}{}{}{}{}{}{Turvação na visibilidade atmosférica, causada por poeira, poluição, fumaça etc.}{bru.ma}{0}
\verb{bruma}{}{Fig.}{}{}{}{Falta de nitidez; incerteza, vagueza.}{bru.ma}{0}
\verb{brumoso}{ô}{}{"-osos ⟨ó⟩}{"-osa ⟨ó⟩}{adj.}{Coberto de bruma; enevoado, nublado.}{bru.mo.so}{0}
\verb{brumoso}{ô}{Fig.}{"-osos ⟨ó⟩}{"-osa ⟨ó⟩}{}{Impreciso, vago, confuso.}{bru.mo.so}{0}
\verb{brunido}{}{}{}{}{adj.}{Lustroso, polido, luzidio.}{bru.ni.do}{0}
\verb{brunidor}{ô}{}{}{}{adj.}{Que brune.}{bru.ni.dor}{0}
\verb{brunidura}{}{}{}{}{s.f.}{Ato ou efeito de brunir.}{bru.ni.du.ra}{0}
\verb{brunir}{}{}{}{}{v.t.}{Tornar, deixar brilhante; lustrar, polir.}{bru.nir}{\verboinum{34}\verboirregular{\emph{def.} brunimos, brunis}}
\verb{brusco}{}{}{}{}{adj.}{Súbito, inesperado, imprevisto, repentino.}{brus.co}{0}
\verb{brusco}{}{}{}{}{}{Áspero, ríspido nos modos ou gestos; indelicado, bruto, rude.}{brus.co}{0}
\verb{brusquidão}{}{}{"-ões}{}{s.f.}{Qualidade ou condição do que ou de quem é brusco; brusquidez.}{brus.qui.dão}{0}
\verb{brutal}{}{}{"-ais}{}{adj.2g.}{Próprio de bruto; bestial, animal.}{bru.tal}{0}
\verb{brutal}{}{}{"-ais}{}{}{Cruel, agressivo, violento.}{bru.tal}{0}
\verb{brutal}{}{}{"-ais}{}{}{Selvagem, grosseiro, rude.}{bru.tal}{0}
\verb{brutalidade}{}{}{}{}{s.f.}{Qualidade ou condição do que é bruto ou brutal; crueldade, desumanidade.}{bru.ta.li.da.de}{0}
\verb{brutalizar}{}{}{}{}{v.t.}{Tratar com brutalidade; maltratar, seviciar.}{bru.ta.li.zar}{0}
\verb{brutalizar}{}{}{}{}{}{Tornar bruto; embrutecer.}{bru.ta.li.zar}{\verboinum{1}}
\verb{brutamontes}{}{}{}{}{}{Indivíduo de modos brutos e rudes.  }{bru.ta.mon.tes}{0}
\verb{brutamontes}{}{}{}{}{s.2g.}{Homem muito alto, forte e corpulento.   }{bru.ta.mon.tes}{0}
\verb{bruto}{}{}{}{}{adj.}{Grosseiro, rude, malcriado. (\textit{Era um homem bruto com a esposa e com os filhos.})}{bru.to}{0}
\verb{bruto}{}{}{}{}{}{Que está tal como encontrado na natureza, não manipulado. (\textit{Diamante bruto. Pedra bruta.})}{bru.to}{0}
\verb{bruto}{}{}{}{}{}{Selvagem, bárbaro, inculto, agreste.}{bru.to}{0}
\verb{bruto}{}{}{}{}{s.m.}{Animal irracional.}{bru.to}{0}
\verb{bruxa}{ch}{}{}{}{s.f.}{Mulher que, de acordo com as histórias e lendas, faz bruxaria; feiticeira, maga.}{bru.xa}{0}
\verb{bruxa}{ch}{Por ext.}{}{}{}{Mulher feia ou rabugenta.}{bru.xa}{0}
\verb{bruxa}{ch}{Desus.}{}{}{}{Boneca de pano.}{bru.xa}{0}
\verb{bruxaria}{ch}{}{}{}{s.f.}{Ação, prática ou ritual próprio de bruxo ou bruxa; feitiçaria, feitiço, bruxedo.}{bru.xa.ri.a}{0}
\verb{bruxaria}{ch}{}{}{}{}{Acontecimento ou fenômeno atribuído a artes diabólicas ou a espíritos maléficos.}{bru.xa.ri.a}{0}
\verb{bruxedo}{chê}{}{}{}{s.m.}{Bruxaria.}{bru.xe.do}{0}
\verb{bruxismo}{cs}{Med.}{}{}{s.m.}{Ranger de dentes que ocorre geralmente durante o sono.}{bru.xis.mo}{0}
\verb{bruxismo}{ch}{}{}{}{s.m.}{Crença em bruxas e em bruxarias.}{bru.xis.mo}{0}
\verb{bruxo}{ch}{}{}{}{s.m.}{Indivíduo dado a práticas, ações ou rituais de bruxaria, a quem se atribui poderes sobrenaturais como prever o futuro, enfeitiçar etc.; mago, feiticeiro.}{bru.xo}{0}
\verb{bruxulear}{ch}{}{}{}{v.i.}{Oscilar fracamente (chama ou luz); tremeluzir.}{bru.xu.le.ar}{\verboinum{4}}
\verb{bubão}{}{Med.}{"-ões}{}{s.m.}{Inflamação de um gânglio linfático; bubo, íngua.}{bu.bão}{0}
\verb{bubônica}{}{Med.}{}{}{adj.}{Diz"-se da peste em que se manifestam bubões.}{bu.bô.ni.ca}{0}
\verb{bubônico}{}{}{}{}{adj.}{Relativo a bubão.}{bu.bô.ni.co}{0}
\verb{bucal}{}{}{"-ais}{}{adj.2g.}{Relativo ou pertencente a boca; oral.}{bu.cal}{0}
\verb{bucha}{}{}{}{}{s.f.}{Pedaço de madeira ou de outro material usado para encher uma abertura.}{bu.cha}{0}
\verb{bucha}{}{}{}{}{}{Rolo de pano molhado com alguma coisa que pega fogo, próprio para encher o balão de ar quente e fazê"-lo subir.}{bu.cha}{0}
\verb{bucha}{}{Bot.}{}{}{}{Planta trepadeira que tem um fruto comprido que é usado como esponja quando seco.}{bu.cha}{0}
\verb{buchada}{}{}{}{}{s.f.}{Bucho e as demais entranhas dos animais.}{bu.cha.da}{0}
\verb{buchada}{}{Cul.}{}{}{}{Iguaria feita com as vísceras e intestinos do carneiro ou do bode refogados.}{bu.cha.da}{0}
\verb{bucho}{}{}{}{}{s.m.}{Estômago dos peixes e dos mamíferos.}{bu.cho}{0}
\verb{bucho}{}{Pop.}{}{}{}{Barriga, pança.}{bu.cho}{0}
\verb{bucho}{}{Pop.}{}{}{}{Mulher muito feia ou velha.}{bu.cho}{0}
\verb{buço}{}{}{}{}{s.m.}{Penugem no lábio superior do homem e de algumas mulheres; lanugem. }{bu.ço}{0}
\verb{bucólica}{}{}{}{}{s.f.}{Poesia pastoril; écloga.}{bu.có.li.ca}{0}
\verb{bucólico}{}{}{}{}{adj.}{Relativo ou pertencente à vida e aos costumes do campo; campestre, pastoril.}{bu.có.li.co}{0}
\verb{bucólico}{}{Por ext.}{}{}{}{Relativo à natureza ou à vida natural.}{bu.có.li.co}{0}
\verb{bucolismo}{}{}{}{}{s.m.}{Caráter ou condição de bucólico.}{bu.co.lis.mo}{0}
\verb{bucolismo}{}{Liter.}{}{}{}{Gênero de poesia com temas pastoris ou campestres.}{bu.co.lis.mo}{0}
\verb{budismo}{}{Relig.}{}{}{s.m.}{Doutrina ética, filosófica e religiosa fundada por Sidarta Gautama, o Buda, difundida em muitas partes da Ásia, cujo ensinamento fundamental é o da busca da iluminação, por meio do conhecimento, para escapar da roda dos nascimentos e alcançar o nirvana.}{bu.dis.mo}{0}
\verb{budista}{}{}{}{}{adj.2g.}{Relativo ou pertencente ao budismo.}{bu.dis.ta}{0}
\verb{budista}{}{}{}{}{s.2g.}{Indivíduo adepto do budismo.}{bu.dis.ta}{0}
\verb{bueiro}{ê}{}{}{}{s.m.}{Tubulação subterrânea, coberta por uma grade, colocada à beira das calçadas para escoamento das águas das chuvas.}{bu.ei.ro}{0}
\verb{búfalo}{}{Zool.}{}{búfala}{s.m.}{Nome comum a duas espécies de mamíferos ruminantes, da família dos bovídeos, de cauda curta e de chifres achatados, direcionados para baixo.}{bú.fa.lo}{0}
\verb{bufão}{}{}{"-ões}{bufona}{s.m.}{Ator que faz a plateia rir com caretas, mímicas, ditos engraçados; bufo, truão, palhaço.}{bu.fão}{0}
\verb{bufão}{}{}{"-ões}{bufona}{adj.}{Que faz rir por sua própria figura; cômico, engraçado.}{bu.fão}{0}
\verb{bufar}{}{}{}{}{v.i.}{Soltar fortemente o ar dos pulmões. (\textit{Bufar de raiva. Bufar de cansaço.})}{bu.far}{\verboinum{1}}
\verb{bufarinheiro}{ê}{}{}{}{s.m.}{Vendedor ambulante de quinquilharias; mascate.}{bu.fa.ri.nhei.ro}{0}
\verb{bufê}{}{}{}{}{s.m.}{Móvel de sala de jantar constituído por tampo e armário na parte inferior, utilizado para apoiar ou guardar aparelhagem de jantar; aparador.}{bu.fê}{0}
\verb{bufê}{}{}{}{}{}{Mesa em que se servem as iguarias e as bebidas numa recepção ou reunião festiva.}{bu.fê}{0}
\verb{bufê}{}{Por ext.}{}{}{}{Conjunto das bebidas e iguarias servidas nessas reuniões.}{bu.fê}{0}
\verb{bufê}{}{Por ext.}{}{}{}{Esse serviço, oferecido nessas ocasiões.}{bu.fê}{0}
\verb{bufo}{}{}{}{}{}{Som que se produz ao bufar; bafo forte.}{bu.fo}{0}
\verb{bufo}{}{}{}{}{s.m.}{Ato ou efeito de bufar.}{bu.fo}{0}
\verb{bufo}{}{Zool.}{}{}{s.m.}{Ave noturna; corujão.}{bu.fo}{0}
\verb{bufo}{}{}{}{}{s.m.}{Ator encarregado de fazer rir o público com mímica e esgares; bufão.}{bu.fo}{0}
\verb{bufonaria}{}{}{}{}{s.f.}{Ato, dito de bufão.}{bu.fo.na.ri.a}{0}
\verb{bug}{}{Informát.}{}{}{s.m.}{Defeito, falha ou inconsistência no código de um programa que acarreta falhas na sua execução.}{\textit{bug}}{0}
\verb{bugalho}{}{}{}{}{s.m.}{Galha arredondada que se forma no tronco dos carvalhos, produzida por larva de inseto.}{bu.ga.lho}{0}
\verb{bugalho}{}{}{}{}{}{Qualquer coisa com forma arredondada.}{bu.ga.lho}{0}
\verb{bugalho}{}{Por ext.}{}{}{}{Globo ocular.}{bu.ga.lho}{0}
\verb{buganvília}{}{Bot.}{}{}{s.f.}{Planta trepadeira usada como ornamental ou como cerca viva.}{bu.gan.ví.lia}{0}
\verb{bugiar}{}{}{}{}{v.i.}{Fazer bugiarias, macaquices.}{bu.gi.ar}{\verboinum{1}}
\verb{bugiganga}{}{}{}{}{s.f.}{Objeto de pouco valor ou serventia; ninharia, quinquilharia.}{bu.gi.gan.ga}{0}
\verb{bugio}{}{Zool.}{}{}{s.m.}{Espécie de macaco; guariba.}{bu.gi.o}{0}
\verb{bugio}{}{}{}{}{}{Bate"-estacas de batente elevado; macaco.}{bu.gi.o}{0}
\verb{bugio}{}{Fig.}{}{}{}{Indivíduo que imita os outros.}{bu.gi.o}{0}
\verb{bugre}{}{}{}{}{s.m.}{Designação dada a índio selvagem ou aguerrido.}{bu.gre}{0}
\verb{bugre}{}{Fig.}{}{}{}{Indivíduo rude, grosseiro.}{bu.gre}{0}
\verb{bugre}{}{Fig.}{}{}{}{Indivíduo desconfiado, arredio.}{bu.gre}{0}
\verb{bujão}{}{}{"-ões}{}{s.m.}{Peça usada para vedar um orifício, fenda ou rombo.}{bu.jão}{0}
\verb{bujão}{}{}{"-ões}{}{}{Rolha de atarraxar usada em veículos automóveis.}{bu.jão}{0}
\verb{bujão}{}{}{"-ões}{}{}{Recipiente metálico usado para entrega de gás em domicílio.}{bu.jão}{0}
\verb{bujarrona}{}{}{}{}{s.f.}{Vela triangular que vai à frente dos barcos.}{bu.jar.ro.na}{0}
\verb{bula}{}{}{}{}{s.f.}{Impresso que acompanha um medicamento e contém informações acerca de sua composição e posologia.}{bu.la}{0}
\verb{bula}{}{}{}{}{}{Selo que se prendia aos antigos documentos emitidos por papas e reis.}{bu.la}{0}
\verb{bulbo}{}{Anat.}{}{}{s.m.}{Formação ou tecido de formato arredondado.}{bul.bo}{0}
\verb{bulbo}{}{Por ext.}{}{}{}{Qualquer objeto ou parte dele que tenha formato arredondado.}{bul.bo}{0}
\verb{bulbo}{}{Anat.}{}{}{}{Parte inferior do encéfalo, relacionada com a sensibilidade e motricidade de várias partes do organismo e onde se alojam alguns centros de controle vital, como o centro respiratório.}{bul.bo}{0}
\verb{bulboso}{ô}{}{osos ⟨ó⟩}{osa ⟨ó⟩}{adj.}{Relativo a bulbo.}{bul.bo.so}{0}
\verb{bulboso}{ô}{}{osos ⟨ó⟩}{osa ⟨ó⟩}{}{Que tem ou apresenta forma de bulbo.}{bul.bo.so}{0}
\verb{buldogue}{ó}{Zool.}{}{}{s.m.}{Raça de cães com focinho curto e achatado, aspecto feroz e pelos curtos.}{bul.do.gue}{0}
\verb{buldogue}{ó}{}{}{}{}{Cão dessa raça.}{bul.do.gue}{0}
\verb{bule}{}{}{}{}{s.m.}{Recipiente com bico, asa, tampa e geralmente bojudo, para servir café ou chá.}{bu.le}{0}
\verb{bulevar}{}{}{}{}{s.m.}{Rua ou avenida larga e geralmente arborizada.}{bu.le.var}{0}
\verb{búlgaro}{}{}{}{}{adj.}{Relativo à Bulgária.}{búl.ga.ro}{0}
\verb{búlgaro}{}{}{}{}{s.m.}{Indivíduo natural ou habitante desse país.}{búl.ga.ro}{0}
\verb{bulha}{}{}{}{}{s.f.}{Confusão de sons; barulho, gritaria.}{bu.lha}{0}
\verb{bulha}{}{}{}{}{}{Tumulto, desordem, confusão.}{bu.lha}{0}
\verb{bulhento}{}{}{}{}{adj.}{Que faz muito ruído; barulhento.}{bu.lhen.to}{0}
\verb{bulhento}{}{}{}{}{}{Briguento, desordeiro.}{bu.lhen.to}{0}
\verb{bulhufas}{}{Pop.}{}{}{pron.}{Coisa nenhuma; nada.}{bu.lhu.fas}{0}
\verb{bulício}{}{}{}{}{s.m.}{Ruído incessante e pouco distinto de coisas ou vozes; sussurro.}{bu.lí.cio}{0}
\verb{bulício}{}{}{}{}{}{Ausência de tranquilidade; agitação, inquietação.}{bu.lí.cio}{0}
\verb{buliçoso}{ô}{}{"-osos ⟨ó⟩}{"-osa ⟨ó⟩}{adj.}{Que bole ou se move sem parar; agitado.}{bu.li.ço.so}{0}
\verb{buliçoso}{ô}{}{"-osos ⟨ó⟩}{"-osa ⟨ó⟩}{}{Esperto, desenvolto.}{bu.li.ço.so}{0}
\verb{buliçoso}{ô}{}{"-osos ⟨ó⟩}{"-osa ⟨ó⟩}{}{Irrequieto, travesso.}{bu.li.ço.so}{0}
\verb{bulimia}{}{Med.}{}{}{s.f.}{Distúrbio do apetite, de origem neurótica ou orgânica, que se manifesta por uma necessidade compulsiva e incontrolável de ingerir alimentos, seguida de purgação.}{bu.li.mi.a}{0}
\verb{bulir}{}{}{}{}{v.t.}{Agitar ou mover de leve.}{bu.lir}{0}
\verb{bulir}{}{}{}{}{}{Pôr as mãos; tocar levemente.}{bu.lir}{\verboinum{33}}
\verb{bumba}{}{}{}{}{interj.}{Expressão que indica o som ou ruído de uma batida.}{bum.ba}{0}
\verb{bumba"-meu"-boi}{}{Bras.}{}{}{s.m.}{Bailado popular cômico"-dramático, com personagens humanas, animais e fantásticas, sobre a morte e a ressurreição de um boi; boi"-bumbá.}{bum.ba"-meu"-boi}{0}
\verb{bumba"-meu"-boi}{}{}{}{}{}{A festa que acompanha esse bailado. }{bum.ba"-meu"-boi}{0}
\verb{bumbo}{}{Mús.}{}{}{s.m.}{Tambor de grandes dimensões usado em bandas militares; bombo.}{bum.bo}{0}
\verb{bumbum}{}{}{"-uns}{}{s.m.}{Ruído ou estrondo repetido.}{bum.bum}{0}
\verb{bumbum}{}{Pop.}{"-uns}{}{}{Nádegas.}{bum.bum}{0}
\verb{bumerangue}{}{}{}{}{s.m.}{Arma de arremesso feita em madeira escavada que é utilizada por grupos indígenas australianos e que tem a propriedade de retornar a um ponto próximo de onde foi lançada.}{bu.me.ran.gue}{0}
\verb{bunda}{}{}{}{}{s.f.}{As nádegas.}{bun.da}{0}
\verb{bunda}{}{}{}{}{}{O ânus.}{bun.da}{0}
\verb{bunda}{}{}{}{}{adj.2g.}{Sem importância; ordinário, reles.}{bun.da}{0}
\verb{bundão}{}{Pop.}{"-ões}{}{adj.}{Covarde, tímido, medroso, desanimado, maçante.}{bun.dão}{0}
\verb{bundão}{}{Bras.}{"-ões}{}{s.m.}{Jagunço, criminoso.}{bun.dão}{0}
\verb{bundão}{}{Bras.}{"-ões}{}{}{Grupo de criminosos, garimpeiros e jagunços ligados a políticos [usa"-se no plural nesta acepção].}{bun.dão}{0}
\verb{buquê}{}{}{}{}{s.m.}{Ramo de flores.}{bu.quê}{0}
\verb{buquê}{}{}{}{}{}{Aroma adquirido pelo vinho ao envelhecer; perfume.}{bu.quê}{0}
\verb{buquê}{}{}{}{}{}{Arranjo de fogos de artifício que produzem um belo efeito.}{bu.quê}{0}
\verb{buraco}{}{}{}{}{s.m.}{Cavidade ou espaço vazio em uma superfície ou em um corpo.}{bu.ra.co}{0}
\verb{buraco}{}{Fig.}{}{}{}{Sentimento de perda; falta, vazio.}{bu.ra.co}{0}
\verb{buraco}{}{Pejor.}{}{}{}{Local afastado, de difícil acesso e de condições precárias.}{bu.ra.co}{0}
\verb{buraco}{}{}{}{}{}{Jogo de cartas semelhante à canastra.}{bu.ra.co}{0}
\verb{buraqueira}{ê}{Bras.}{}{}{s.f.}{Terreno repleto de buracos.}{bu.ra.quei.ra}{0}
\verb{buraqueira}{ê}{Bras.}{}{}{}{Lugar afastado, ermo, distante de cidades; buraco.}{bu.ra.quei.ra}{0}
\verb{burburejar}{}{}{}{}{v.i.}{Emitir barulho como de água em queda.}{bur.bu.re.jar}{\verboinum{1}}
\verb{burburinhar}{}{}{}{}{v.i.}{Fazer burburinho; rumorejar.}{bur.bu.ri.nhar}{\verboinum{1}}
\verb{burburinho}{}{}{}{}{s.m.}{Som contínuo de muitas vozes ao mesmo tempo.}{bur.bu.ri.nho}{0}
\verb{bureau}{}{}{}{}{s.m.}{Escrivaninha.}{\textit{bureau}}{0}
\verb{bureau}{}{Por ext.}{}{}{}{Escritório, gabinete.}{\textit{bureau}}{0}
\verb{bureau}{}{Por ext.}{}{}{}{Repartição pública; departamento, agência.}{\textit{bureau}}{0}
\verb{bureau}{}{Bras.}{}{}{}{Estabelecimento que presta serviços na área de computação gráfica e editoração eletrônica.}{\textit{bureau}}{0}
\verb{burel}{é}{}{"-éis}{}{s.m.}{Tecido grosseiro de lã.}{bu.rel}{0}
\verb{burel}{é}{}{"-éis}{}{}{O hábito de frade ou freira feito com esse tecido.}{bu.rel}{0}
\verb{burel}{é}{Fig.}{"-éis}{}{}{Luto, tristeza.}{bu.rel}{0}
\verb{bureta}{ê}{}{}{}{s.f.}{Recipiente de vidro, longo e fino, com torneira e graduação de volume, usado em laboratórios para lidar com líquidos ou gases.}{bu.re.ta}{0}
\verb{burgo}{}{}{}{}{s.m.}{Povoado, vila, aldeia.}{bur.go}{0}
\verb{burgo}{}{}{}{}{}{Castelo, casa de nobre ou mosteiro fortificado por muralhas.}{bur.go}{0}
\verb{burgo}{}{}{}{}{}{Arrabalde de cidade.}{bur.go}{0}
\verb{burgomestre}{é}{}{}{}{s.m.}{Principal magistrado em cidades de alguns países europeus, equivalente a um prefeito.}{bur.go.mes.tre}{0}
\verb{burguês}{}{}{}{}{s.m.}{Natural ou habitante de um burgo.}{bur.guês}{0}
\verb{burguês}{}{}{}{}{adj.}{Na Idade Média, relativo a burgo.}{bur.guês}{0}
\verb{burguês}{}{}{}{}{}{Relativo a burguesia; que desfruta de boa situação socioeconômica.}{bur.guês}{0}
\verb{burguês}{}{Pejor.}{}{}{}{Diz"-se de indivíduo preconceituoso, reacionário, individualista e excessivamente preocupado com êxito material.}{bur.guês}{0}
\verb{burguesia}{}{Hist.}{}{}{s.f.}{Grupo social formado no final da Idade Média, composto de pessoas ligadas a atividades comerciais e que moravam nos burgos, tendo passado a exercer dominação política e econômica sobre os outros grupos sociais.}{bur.gue.si.a}{0}
\verb{burguesia}{}{}{}{}{}{Nos dias atuais, grupo formado por pessoas de média e alta renda, com comportamento específico em relação ao consumo e às práticas sociais e mais ou menos ligadas às esferas políticas da sociedade.}{bur.gue.si.a}{0}
\verb{burguesia}{}{Pejor.}{}{}{}{Grupo de pessoas que apresentam comportamento ambicioso, arrogante e exibicionista.}{bur.gue.si.a}{0}
\verb{buril}{}{}{"-is}{}{s.m.}{Ferramenta cortante utilizada na gravação em metal ou madeira.}{bu.ril}{0}
\verb{burilar}{}{}{}{}{v.t.}{Gravar com buril.}{bu.ri.lar}{0}
\verb{burilar}{}{Fig.}{}{}{}{Aprimorar, aperfeiçoar.}{bu.ri.lar}{\verboinum{1}}
\verb{buriti}{}{Bot.}{}{}{s.m.}{Palmeira cujo fruto amarelo serve à fabricação de óleo e vinho de buriti.}{bu.ri.ti}{0}
\verb{buriti}{}{}{}{}{}{O fruto dessa palmeira.}{bu.ri.ti}{0}
\verb{buritizeiro}{ê}{Bot.}{}{}{s.m.}{A árvore buriti.}{bu.ri.ti.zei.ro}{0}
\verb{burla}{}{}{}{}{s.f.}{Ato ou efeito de burlar.}{bur.la}{0}
\verb{burla}{}{}{}{}{}{Embuste, fraude.}{bur.la}{0}
\verb{burla}{}{}{}{}{}{Brincadeira de mau gosto; peça.}{bur.la}{0}
\verb{burlão}{}{}{"-ões}{}{adj.}{Burlador.}{bur.lão}{0}
\verb{burlão}{}{Lus.}{"-ões}{}{s.m.}{Batedor de carteiras; ladrão.}{bur.lão}{0}
\verb{burlar}{}{}{}{}{v.t.}{Fraudar, lesar.}{bur.lar}{0}
\verb{burlar}{}{}{}{}{}{Enganar, ludibriar.}{bur.lar}{\verboinum{1}}
\verb{burlesco}{ê}{}{}{}{adj.}{Que provoca riso; cômico.}{bur.les.co}{0}
\verb{burlesco}{ê}{}{}{}{}{Satírico, zombeteiro.}{bur.les.co}{0}
\verb{burlesco}{ê}{}{}{}{}{Cômico ao ponto de ser ridículo; caricato.}{bur.les.co}{0}
\verb{burocracia}{}{}{}{}{s.f.}{Sistema de administração, geralmente em instituições públicas, em que os funcionários são sujeitos a rotina, hierarquia e regras rígidas.}{bu.ro.cra.ci.a}{0}
\verb{burocracia}{}{}{}{}{}{O conjunto dos funcionários que trabalham nesse tipo de organização, especialmente as públicas.}{bu.ro.cra.ci.a}{0}
\verb{burocracia}{}{Pejor.}{}{}{}{Morosidade ou excesso de exigências e formalidades na resolução de assuntos administrativos.}{bu.ro.cra.ci.a}{0}
\verb{burocrata}{}{}{}{}{s.2g.}{Indivíduo que faz parte da burocracia.}{bu.ro.cra.ta}{0}
\verb{burocrata}{}{}{}{}{}{Funcionário cujo comportamento no trabalho é excessivamente rígido e rotineiro.}{bu.ro.cra.ta}{0}
\verb{burocrata}{}{}{}{}{}{Funcionário que abusa do poder de seu cargo, por julgá"-lo de prestígio, e toma atitudes que prejudicam os outros ou o funcionamento do sistema.}{bu.ro.cra.ta}{0}
\verb{burocrático}{}{}{}{}{adj.}{Relativo a burocracia.}{bu.ro.crá.ti.co}{0}
\verb{burocratizar}{}{}{}{}{v.t.}{Dar caráter burocrático.}{bu.ro.cra.ti.zar}{\verboinum{1}}
\verb{burra}{}{}{}{}{s.f.}{Caixa para guardar dinheiro e valores.}{bur.ra}{0}
\verb{burrada}{}{}{}{}{s.f.}{Tropa de burros.}{bur.ra.da}{0}
\verb{burrada}{}{Pop.}{}{}{}{Ato estúpido; besteira, tolice.}{bur.ra.da}{0}
\verb{burrama}{}{Bras.}{}{}{s.f.}{Tropa de burros.}{bur.ra.ma}{0}
\verb{burrice}{}{}{}{}{s.f.}{Falta de inteligência; qualidade de burro.}{bur.ri.ce}{0}
\verb{burrice}{}{}{}{}{}{Ação pouco inteligente; ação própria de burro.}{bur.ri.ce}{0}
\verb{burrico}{}{Zool.}{}{}{s.m.}{Pequeno burro.}{bur.ri.co}{0}
\verb{burrico}{}{Bras.}{}{}{}{Qualquer burro.}{bur.ri.co}{0}
\verb{burrificar}{}{}{}{}{v.t.}{Tornar burro; embrutecer.}{bur.ri.fi.car}{\verboinum{2}}
\verb{burrinho}{}{}{}{}{s.m.}{Burro de pequeno porte; burrico.}{bur.ri.nho}{0}
\verb{burrinho}{}{}{}{}{}{Pequena bomba movida por energia elétrica usada para aspirar líquidos.}{bur.ri.nho}{0}
\verb{burrinho}{}{Bras.}{}{}{}{Compressor de óleo no sistema de freio dos automóveis.}{bur.ri.nho}{0}
\verb{burro}{}{Zool.}{}{}{s.m.}{Jumento, asno.}{bur.ro}{0}
\verb{burro}{}{Zool.}{}{}{}{Animal estéril nascido do cruzamento entre cavalo e jumenta ou entre égua e jumento.}{bur.ro}{0}
\verb{burro}{}{Fig.}{}{}{}{Estúpido, ignorante, bronco, teimoso.}{bur.ro}{0}
\verb{burundinês}{}{}{}{}{adj.}{Relativo ao Burundi.}{bu.run.di.nês}{0}
\verb{burundinês}{}{}{}{}{s.m.}{Indivíduo natural ou habitante desse país.}{bu.run.di.nês}{0}
\verb{busca}{}{}{}{}{s.f.}{Ato ou efeito de buscar.}{bus.ca}{0}
\verb{busca}{}{}{}{}{}{Procura, exame, revista.}{bus.ca}{0}
\verb{busca}{}{}{}{}{}{Investigação, pesquisa.}{bus.ca}{0}
\verb{busca}{}{}{}{}{}{Esforço íntimo para alcançar um objetivo.}{bus.ca}{0}
\verb{busca"-pé}{}{}{busca"-pés}{}{s.m.}{Fogo de artifício que salta e corre pelo chão.}{bus.ca"-pé}{0}
\verb{buscar}{}{}{}{}{v.t.}{Procurar, revistar, examinar.}{bus.car}{0}
\verb{buscar}{}{}{}{}{}{Investigar, pesquisar.}{bus.car}{0}
\verb{buscar}{}{}{}{}{}{Tratar de obter; empenhar"-se.}{bus.car}{0}
\verb{buscar}{}{}{}{}{}{Cobiçar, almejar, ambicionar.}{bus.car}{\verboinum{2}}
\verb{busílis}{}{}{}{}{s.m.}{O ponto principal do problema; o xis da questão. (\textit{É aí que está o busílis.})}{bu.sí.lis}{0}
\verb{bússola}{}{}{}{}{s.f.}{Agulha magnética móvel que sempre aponta na direção norte"-sul e serve para orientação.}{bús.so.la}{0}
\verb{bússola}{}{Fig.}{}{}{}{Aquilo que serve de referência, de guia, de norte.}{bús.so.la}{0}
\verb{bustiê}{}{}{}{}{s.m.}{Peça de vestuário feminino justa e sem alças que cobre apenas o busto.}{bus.ti.ê}{0}
\verb{busto}{}{}{}{}{s.m.}{A parte superior do tronco humano; torso.}{bus.to}{0}
\verb{busto}{}{}{}{}{}{Tipo de escultura em que se representa a cabeça, o pescoço e parte do peito de uma figura humana.}{bus.to}{0}
\verb{busto}{}{}{}{}{}{Os seios da mulher.}{bus.to}{0}
\verb{busto}{}{}{}{}{}{A medida externa do torso feminino na altura dos seios para uso em projetos de costura.}{bus.to}{0}
\verb{butanês}{}{}{}{}{adj.}{Relativo ao Butão, Estado asiático localizado na região sul do Himalaia.}{bu.ta.nês}{0}
\verb{butanês}{}{}{}{}{s.m.}{Indivíduo natural ou habitante do Butão.}{bu.ta.nês}{0}
\verb{butanês}{}{}{}{}{}{O dialeto da língua tibetana falado no Butão.}{bu.ta.nês}{0}
\verb{butano}{}{Quím.}{}{}{s.m.}{Gás incolor com pouco odor utilizado como combustível e como propelente de aerossóis.}{bu.ta.no}{0}
\verb{butiá}{}{Bot.}{}{}{s.m.}{Palmeira alta, de folhas compridas, cujo lenho fornece fécula comestível e cujos frutos possuem uma polpa comestível e amêndoa oleaginosa.}{bu.ti.á}{0}
\verb{butique}{}{}{}{}{s.f.}{Pequena loja que vende roupas, bijuterias e outros artigos femininos.}{bu.ti.que}{0}
\verb{butuca}{}{Zool.}{}{}{s.f.}{Espécie de mosca grande, de picada dolorosa, que ataca principalmente o homem  e o gado; mutuca.}{bu.tu.ca}{0}
\verb{buxo}{ch}{Bot.}{}{}{s.m.}{Espécie de arbusto de folhas pequenas e flores bancas, que fornece madeira amarela, muito utilizada em esculturas e na fabricação de instrumentos musicais de sopro.}{bu.xo}{0}
\verb{buzina}{}{}{}{}{s.f.}{Instrumento elétrico sonoro, acoplado em veículos para dar sinais de advertência.}{bu.zi.na}{0}
\verb{buzina}{}{}{}{}{}{Espécie de trombeta ou corneta feita de corno, marfim, metal etc. }{bu.zi.na}{0}
\verb{buzina}{}{}{}{}{}{Antigo instrumento de sopro feito de chifre ou metal retorcido.}{bu.zi.na}{0}
\verb{buzinada}{}{}{}{}{s.f.}{Barulho feito pela buzina.}{bu.zi.na.da}{0}
\verb{buzinar}{}{}{}{}{v.i.}{Tocar ou fazer soar a buzina.}{bu.zi.nar}{0}
\verb{buzinar}{}{}{}{}{}{Soprar fortemente, imitando o som da buzina.}{bu.zi.nar}{0}
\verb{buzinar}{}{}{}{}{}{Importunar alguém, repetindo insistentemente a mesma coisa.}{bu.zi.nar}{\verboinum{1}}
\verb{búzio}{}{Zool.}{}{}{s.m.}{Nome comum dado a moluscos marinhos, de concha piramidal, com uma saliência arredondada de um lado.}{bú.zio}{0}
\verb{búzio}{}{}{}{}{}{A concha desses moluscos, usada nos cultos afro"-brasileiros na técnica divinatória.}{bú.zio}{0}
\verb{buzo}{}{}{}{}{}{Var. de \textit{búzio}.}{bu.zo}{0}
\verb{byte}{}{Informát.}{}{}{s.m.}{Unidade básica de informação, constituída de 8 bits adjacentes, usada para representar um caractere.}{\textit{byte}}{0}
