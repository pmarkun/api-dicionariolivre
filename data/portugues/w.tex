\verb{w}{}{}{}{}{s.m.}{Vigésima terceira letra do alfabeto português.}{w}{0}
\verb{W}{}{Quím.}{}{}{}{Símb. do \textit{tungstênio}.}{W}{0}
\verb{W}{}{Fís.}{}{}{}{Símb. de \textit{watt}.}{W}{0}
\verb{W}{}{}{}{}{}{Abrev. do inglês \textit{west}, oeste.}{W}{0}
\verb{waffle}{}{Cul.}{}{}{s.m.}{Espécie de panqueca assada de massa grossa, consumida pura ou com geleia, mel, manteiga.}{\textit{waffle}}{0}
\verb{wagneriano}{v}{Mús.}{}{}{adj.}{Pertencente ou relativo ao compositor alemão Richard Wagner (1813--1883) ou a sua obra.}{wag.ne.ri.a.no}{0}
\verb{wagneriano}{v}{}{}{}{s.m.}{Indivíduo que admira ou estuda as teorias e o estilo musical de Richard Wagner.}{wag.ne.ri.a.no}{0}
\verb{wagnerismo}{v}{Mús.}{}{}{s.m.}{Conjunto das concepções estéticas e musicais do compositor alemão Richard Wagner.}{wag.ne.ris.mo}{0}
\verb{waiãpi}{}{}{}{}{adj.}{Relativo aos Waiãpi.}{wai.ã.pi}{0}
\verb{waiãpi}{}{}{}{}{s.2g.}{Indivíduo pertencente ao povo waiãpi, família linguística tupi"-guarani.}{wai.ã.pi}{0}
\verb{waimiri atroari}{}{}{}{}{adj.}{Relativo aos Waimiri Atroari.}{wai.mi.ri a.tro.a.ri}{0}
\verb{waimiri atroari}{}{}{}{}{s.2g.}{Indivíduo pertencente ao povo waimiri atroari, família linguística karib.}{wai.mi.ri a.tro.a.ri}{0}
\verb{wai wai}{}{}{}{}{adj.}{Relativo aos Wai Wai.}{wai wai}{0}
\verb{wai wai}{}{}{}{}{s.2g.}{Indivíduo pertencente ao povo wai wai, família linguística karib.}{wai wai}{0}
\verb{walkie"-talkie}{}{}{walkie"-talkies}{}{s.m.}{Emissor e receptor portátil para comunicação radiofônica a curta distância.}{\textit{walkie"-talkie}}{0}
\verb{walkman}{}{}{}{}{s.m.}{Aparelho de rádio ou toca"-fitas, pequeno e portátil, com fones de ouvido.}{\textit{walkman}}{0}
\verb{wanano}{}{}{}{}{adj.}{Relativo aos Wanano.}{wa.na.no}{0}
\verb{wanano}{}{}{}{}{s.2g.}{Indivíduo pertencente ao povo wanano, família linguística tukano.}{wa.na.no}{0}
\verb{wapixana}{ch}{}{}{}{adj.}{Relativo aos Wapixana.}{wa.pi.xa.na}{0}
\verb{wapixana}{ch}{}{}{}{s.2g.}{Indivíduo pertencente ao povo wapixana, família linguística aruak.}{wa.pi.xa.na}{0}
\verb{warekena}{ê}{}{}{}{adj.}{Relativo aos Warekena.}{wa.re.ke.na}{0}
\verb{warekena}{ê}{}{}{}{s.2g.}{Indivíduo pertencente ao povo warekena, família linguística aruak.}{wa.re.ke.na}{0}
\verb{warrant}{}{Jur.}{}{}{s.m.}{Título de crédito negociável, emitido por estabelecimento responsável pela guarda de mercadorias.}{\textit{warrant}}{0}
\verb{wassu}{}{}{}{}{adj.}{Relativo aos Wassu.}{was.su}{0}
\verb{wassu}{}{}{}{}{s.2g.}{Indivíduo pertencente ao povo wassu.}{was.su}{0}
\verb{water"-closet}{}{}{}{}{s.m.}{Banheiro. Abrev. \textsc{w.c}.}{\textit{water"-closet}}{0}
\verb{watt}{v}{Fís.}{}{}{s.m.}{Unidade de medida de potência. Símb.: \textsc{w}.}{watt}{0}
\verb{watt"-hora}{v\ldots{}ó}{Fís.}{watts"-horas \textit{ou} watts"-hora ⟨v\ldots{}ó⟩}{}{s.m.}{Unidade de medida de energia, equivalente a 3.600 J. Símb.: Wh.}{watt"-ho.ra}{0}
\verb{wattímetro}{v}{Fís.}{}{}{s.m.}{Aparelho que mede a potência elétrica.}{wat.tí.me.tro}{0}
\verb{waurá}{}{}{}{}{adj.}{Relativo aos Waurá.}{wau.rá}{0}
\verb{waurá}{}{}{}{}{s.2g.}{Indivíduo pertencente ao povo waurá, família linguística aruak.}{wau.rá}{0}
\verb{wayana}{}{}{}{}{adj.}{Relativo aos Wayana.}{wa.ya.na}{0}
\verb{wayana}{}{}{}{}{s.2g.}{Indivíduo pertencente ao povo wayana, família linguística karib.}{wa.ya.na}{0}
\verb{WC}{}{}{}{}{s.m.}{Abrev. do inglês \textit{water"-closet}; banheiro.}{W.C.}{0}
\verb{web}{}{Informát.}{}{}{s.f.}{Rede mundial de computadores, a internet, descentralizada e pública, que consiste num sistema de acesso a informações na forma de hipertexto, com \textit{links} entre documentos e outros objetos distribuídos em diversos pontos da rede.  }{\textit{web}}{0}
\verb{western}{}{}{}{}{s.m.}{Faroeste, bangue"-bangue.}{\textit{western}}{0}
\verb{Wh}{}{Fís.}{}{}{}{Símb. de \textit{watt"-hora}.}{Wh}{0}
\verb{wildiano}{uáil}{}{}{}{adj.}{Relativo ou pertencente ao escritor irlandês Oscar Wilde (1854--1900), ou próprio de seu estilo.}{wil.di.a.no}{0}
\verb{wildiano}{uáil}{}{}{}{s.m.}{Indivíduo estudioso ou apreciador da obra de Oscar Wilde.}{wil.di.a.no}{0}
\verb{winchester}{}{Informát.}{}{}{s.m.}{Disco magnético de alta capacidade de armazenamento de informações; disco rígido.}{\textit{winchester}}{0}
\verb{winchester}{}{}{}{}{s.f.}{Nome de uma antiga carabina norte"-americana de repetição.}{\textit{winchester}}{0}
\verb{windsurfe}{u}{Esport.}{}{}{s.m.}{Modalidade esportiva na qual o praticante se desloca na água, impulsionado pelo vento, de pé sobre uma prancha semelhante à de surfe, mas provida de uma vela.}{wind.sur.fe}{0}
\verb{windsurfista}{u}{Esport.}{}{}{s.2g.}{Indivíduo que pratica windsurfe.}{wind.sur.fis.ta}{0}
\verb{witoto}{}{}{}{}{adj.}{Relativo aos Witoto.}{wi.to.to}{0}
\verb{witoto}{}{}{}{}{s.2g.}{Indivíduo pertencente ao povo Witoto, família linguística Witoto.}{wi.to.to}{0}
\verb{WNE}{}{}{}{}{}{Abrev. de \textit{oés"-nordeste}.}{W.N.E.}{0}
\verb{WNW}{}{}{}{}{}{Abrev. do inglês de \textit{ west"-northwest}; \textsc{o.n.o}., oés"-noroeste.}{W.N.W.}{0}
\verb{workaholic}{}{}{}{}{adj.2g.}{Diz"-se do indivíduo que trabalha compulsivamente, preterindo outras atividades.}{\textit{workaholic}}{0}
\verb{workaholic}{}{}{}{}{s.2g.}{Esse indivíduo.}{\textit{workaholic}}{0}
\verb{workshop}{}{}{}{}{s.m.}{Oficina, seminário ou curso intensivo, de curta duração, em que é discutida, demonstrada e exercitada alguma técnica ou arte.}{\textit{workshop}}{0}
\verb{WSE}{}{}{}{}{}{Abrev. de \textit{oés"-sueste}.}{W.S.E.}{0}
\verb{WSW}{}{}{}{}{}{Abrev. do inglês \textit{west"-southwest}; \textsc{o.s.o}., oés"-sudoeste.}{W.S.W.}{0}
\verb{WWW}{}{Informát.}{}{}{s.f.}{Abrev. do inglês \textit{World Wide Web}, rede mundial; \textit{web}. }{WWW}{0}
