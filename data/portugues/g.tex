\verb{g}{}{}{}{}{s.m.}{Sétima letra do alfabeto português.}{g}{0}
\verb{G}{}{Mús.}{}{}{}{A nota ou o acorde referente ao \textit{sol} ou à quinta nota da escala de \textit{dó}.}{G}{0}
\verb{Ga}{}{Quím.}{}{}{}{Símb. do \textit{gálio}.}{Ga}{0}
\verb{gabação}{}{}{"-ões}{}{s.f.}{Ato ou efeito de gabar; elogio.}{ga.ba.ção}{0}
\verb{gabar}{}{}{}{}{v.t.}{Enaltecer as qualidades de alguém ou de algo; louvar, elogiar.}{ga.bar}{0}
\verb{gabar}{}{}{}{}{v.pron.}{Vangloriar"-se; jactar"-se.}{ga.bar}{\verboinum{1}}
\verb{gabardina}{}{}{}{}{s.f.}{Certo pano de lã, algodão ou seda, natural ou sintético, tecido em diagonal e próprio para roupas.}{ga.bar.di.na}{0}
\verb{gabardina}{}{}{}{}{}{Capa de chuva feita com esse tecido impermeabilizado.}{ga.bar.di.na}{0}
\verb{gabardine}{}{}{}{}{}{Var. de \textit{gabardina}.}{ga.bar.di.ne}{0}
\verb{gabaritado}{}{Pop.}{}{}{adj.}{Que apresenta qualificações para executar certo trabalho ou ocupar determinado cargo ou função; capaz, preparado.}{ga.ba.ri.ta.do}{0}
\verb{gabaritar}{}{Pop.}{}{}{v.t.}{Acertar todas as questões de uma prova.}{ga.ba.ri.tar}{\verboinum{1}}
\verb{gabarito}{}{}{}{}{s.m.}{Medida padrão.}{ga.ba.ri.to}{0}
\verb{gabarito}{}{}{}{}{}{Tabela de respostas corretas de uma prova.}{ga.ba.ri.to}{0}
\verb{gabarito}{}{Fig.}{}{}{}{Categoria, qualidade, classe.}{ga.ba.ri.to}{0}
\verb{gabarola}{ó}{}{}{}{adj.2g.}{Diz"-se de quem se gaba a si mesmo, dos próprios feitos; fanfarrão.}{ga.ba.ro.la}{0}
\verb{gabarolice}{}{}{}{}{s.f.}{Atitude própria de gabarola; fanfarrice.}{ga.ba.ro.li.ce}{0}
\verb{gabinete}{ê}{}{}{}{s.m.}{Sala de trabalho.}{ga.bi.ne.te}{0}
\verb{gabinete}{ê}{}{}{}{}{Sala reservada para funcionários superiores ou para certas funções.}{ga.bi.ne.te}{0}
\verb{gabinete}{ê}{}{}{}{}{Conselho de ministros do Estado.}{ga.bi.ne.te}{0}
\verb{gabiroba}{ó}{}{}{}{}{Var. de \textit{guabirola}.}{ga.bi.ro.ba}{0}
\verb{gabiru}{}{Pop.}{}{}{s.m.}{Indivíduo que age com esperteza; velhaco, malandro.}{ga.bi.ru}{0}
\verb{gabiru}{}{}{}{}{}{Indivíduo desajeitado, desengonçado.}{ga.bi.ru}{0}
\verb{gabiru}{}{Zool.}{}{}{}{Espécie de rato grande.}{ga.bi.ru}{0}
\verb{gabo}{}{}{}{}{s.m.}{Elogio.}{ga.bo}{0}
\verb{gabo}{}{}{}{}{}{Arrogância, presunção.}{ga.bo}{0}
\verb{gabola}{ó}{}{}{}{adj.2g.}{Gabarola.}{ga.bo.la}{0}
\verb{gabolice}{}{}{}{}{s.f.}{Gabarolice.}{ga.bo.li.ce}{0}
\verb{gabonês}{}{}{}{}{adj.}{Relativo a Gabão (África).}{ga.bo.nês}{0}
\verb{gabonês}{}{}{}{}{s.m.}{Indivíduo natural ou habitante desse país.}{ga.bo.nês}{0}
\verb{gadanhar}{}{}{}{}{v.t.}{Arrancar erva com gadanho.}{ga.da.nhar}{0}
\verb{gadanhar}{}{}{}{}{}{Arranhar com as unhas ou com o gadanho.}{ga.da.nhar}{0}
\verb{gadanhar}{}{}{}{}{}{Agarrar com firmeza.}{ga.da.nhar}{\verboinum{1}}
\verb{gadanho}{}{}{}{}{s.m.}{Garra de ave de rapina.}{ga.da.nho}{0}
\verb{gadanho}{}{Por ext.}{}{}{}{Unha.}{ga.da.nho}{0}
\verb{gadanho}{}{Pop.}{}{}{}{Os dedos da mão, ou a mão.}{ga.da.nho}{0}
\verb{gadanho}{}{}{}{}{}{Espécie de ancinho com grandes dentes de ferro para trabalhos agrícolas.}{ga.da.nho}{0}
\verb{gadídeo}{}{Zool.}{}{}{s.m.}{Espécime dos gadídeos, família de peixes encontrados nos mares frios e temperados, de grande valor comercial, e cujos representantes mais conhecidos são o bacalhau e o hadoque.}{ga.dí.deo}{0}
\verb{gado}{}{}{}{}{s.m.}{Conjunto de animais criados no campo para trabalhos agrícolas ou uso doméstico e industrial; rebanho.}{ga.do}{0}
\verb{gadolínio}{}{Quím.}{}{}{s.m.}{Elemento químico metálico, branco, maleável, dúctil, magnético, da família dos lantanídeos (terras"-raras); usado em reatores nucleares, materiais fluorescentes etc. \elemento{64}{157.25}{Gd}.}{ga.do.lí.nio}{0}
\verb{gaélico}{}{}{}{}{adj.}{Relativo aos primitivos habitantes da Gália, da Irlanda e da Escócia.}{ga.é.li.co}{0}
\verb{gaélico}{}{}{}{}{s.m.}{Língua falada na Irlanda e na Escócia.}{ga.é.li.co}{0}
\verb{gafanhoto}{ô}{Zool.}{}{}{s.m.}{Nome comum aos insetos que pertencem à mesma ordem dos grilos e das esperanças, de antenas curtas, com asas longas e estreitas, saltadores e muito nocivos à lavoura. }{ga.fa.nho.to}{0}
\verb{gafe}{}{}{}{}{s.f.}{Ato ou palavra impensada, inconveniente; mancada.}{ga.fe}{0}
\verb{gafieira}{ê}{}{}{}{s.f.}{Salão onde são realizados bailes populares.}{ga.fi.ei.ra}{0}
\verb{gafieira}{ê}{}{}{}{}{Baile popular; arrasta"-pé.}{ga.fi.ei.ra}{0}
%\verb{}{}{}{}{}{}{}{}{0}
\verb{gaforinha}{}{}{}{}{s.f.}{Cabelo eriçado, em desalinho.}{ga.fo.ri.nha}{0}
\verb{gagá}{}{}{}{}{adj.}{Diz"-se de indivíduo mentalmente incapaz, que voltou à infância; caduco.}{ga.gá}{0}
\verb{gago}{}{}{}{}{adj.}{Diz"-se daquele que gagueja, que fala com dificuldade. (\textit{O menino gago falava com dificuldade.})}{ga.go}{0}
\verb{gagueira}{ê}{}{}{}{s.f.}{Defeito na articulação da fala; embaraço fônico característico dos gagos.}{ga.guei.ra}{0}
\verb{gaguejar}{}{}{}{}{v.i.}{Pronunciar as palavras com dificuldade, sem clareza de sons, e repetindo as sílabas.}{ga.gue.jar}{0}
\verb{gaguejar}{}{}{}{}{}{Falar com embaraço em consequência de uma emoção ou de um estado anormal. (\textit{Nervosa com o acidente, a mulher gaguejava.})}{ga.gue.jar}{\verboinum{1}}
\verb{gaguejo}{ê}{}{}{}{s.m.}{Ato ou efeito de gaguejar.}{ga.gue.jo}{0}
\verb{gaguez}{ê}{}{}{}{s.f.}{Gagueira.}{ga.guez}{0}
\verb{gaguice}{}{}{}{}{s.f.}{Gagueira.}{ga.gui.ce}{0}
\verb{gaiacol}{ó}{}{}{}{}{Var. de \textit{guaiacol}.}{gai.a.col}{0}
\verb{gaiatice}{}{}{}{}{s.f.}{Atitude própria de gaiato.}{gai.a.ti.ce}{0}
\verb{gaiato}{}{}{}{}{adj.}{Diz"-se de indivíduo divertido, brincalhão.}{gai.a.to}{0}
\verb{gaio}{}{}{}{}{adj.}{Que revela alegria; jovial.}{gai.o}{0}
\verb{gaio}{}{Zool.}{}{}{s.m.}{Ave com tamanho aproximado de uma pomba, plumagem marrom avermelhada e asas e cauda negras.}{gai.o}{0}
\verb{gaiola}{ó}{}{}{}{s.f.}{Casinha portátil feita de arame ou construída com ripas finas, que se destina a aprisionar pequenos pássaros. (\textit{O pequeno pássaro azul foi aprisionado na gaiola.})}{gai.o.la}{0}
\verb{gaiola}{ó}{Fig.}{}{}{}{Cárcere, prisão.}{gai.o.la}{0}
\verb{gaiola}{ó}{}{}{}{s.m.}{Pequeno vapor de navegação fluvial.}{gai.o.la}{0}
\verb{gaita}{}{}{}{}{s.f.}{Instrumento de sopro composto de um canudo com vários orifícios.}{gai.ta}{0}
\verb{gaita}{}{Pop.}{}{}{}{Dinheiro, grana.}{gai.ta}{0}
\verb{gaitear}{}{}{}{}{v.i.}{Tocar gaita.}{gai.te.ar}{0}
\verb{gaitear}{}{}{}{}{}{Andar em folias; divertir"-se.}{gai.te.ar}{\verboinum{4}}
\verb{gaiteiro}{ê}{}{}{}{s.m.}{Tocador de gaita.}{gai.tei.ro}{0}
\verb{gaiteiro}{ê}{}{}{}{adj.}{Folião, festeiro.}{gai.tei.ro}{0}
\verb{gaivota}{ó}{Zool.}{}{}{s.f.}{Ave marinha de coloração branco acinzentada, com bico e pés avermelhados, que se alimenta de pequenos peixes e  de detritos do mar.}{gai.vo.ta}{0}
\verb{gajeiro}{ê}{}{}{}{s.m.}{Marinheiro que trabalha no alto do mastro observando o mar e o horizonte. (\textit{O gajeiro, do alto do mastro, observava os golfinhos.})}{ga.jei.ro}{0}
\verb{gajo}{}{}{}{}{s.m.}{Indivíduo de maneiras abrutalhadas; rude, grosseiro.}{ga.jo}{0}
\verb{gajo}{}{}{}{}{}{Indivíduo velhaco, esperto.}{ga.jo}{0}
\verb{gajo}{}{}{}{}{}{Tipo, sujeito, indivíduo.}{ga.jo}{0}
\verb{gala}{}{}{}{}{s.f.}{Traje próprio para ocasiões solenes ou dias festivos.}{ga.la}{0}
\verb{gala}{}{}{}{}{}{Festa nacional.}{ga.la}{0}
\verb{galã}{}{}{}{}{s.m.}{Homem belo e elegante.}{ga.lã}{0}
\verb{gala}{}{}{}{}{}{Mancha germinativa no ovo.}{ga.la}{0}
\verb{galã}{}{}{}{}{}{Ator ou personagem que, em cinema ou teatro, representa o papel principal em tramas amorosas.}{ga.lã}{0}
\verb{galáctico}{}{}{}{}{adj.}{Relativo a galáxia.}{ga.lác.ti.co}{0}
\verb{galadura}{}{}{}{}{s.f.}{Mancha germinativa no ovo; gala.}{ga.la.du.ra}{0}
\verb{galadura}{}{}{}{}{}{Fecundação.}{ga.la.du.ra}{0}
\verb{galalau}{}{Pop.}{}{}{s.m.}{Homem muito alto.}{ga.la.lau}{0}
\verb{galalite}{}{}{}{}{s.f.}{Material plástico obtido pela combinação da caseína com o formol.}{ga.la.li.te}{0}
\verb{galantaria}{}{}{}{}{}{Var. de \textit{galanteria}.}{ga.lan.ta.ri.a}{0}
\verb{galante}{}{}{}{}{adj.2g.}{Que se destaca pela elegância; distinto.}{ga.lan.te}{0}
\verb{galante}{}{}{}{}{}{Que é amável com as mulheres.}{ga.lan.te}{0}
\verb{galante}{}{}{}{}{s.2g.}{Indivíduo que faz galanteios.}{ga.lan.te}{0}
\verb{galanteador}{ô}{}{}{}{adj.}{Diz"-se de indivíduo que galanteia, que corteja mulheres.}{ga.lan.te.a.dor}{0}
\verb{galantear}{}{}{}{}{v.t.}{Dizer galanteios; cortejar.}{ga.lan.te.ar}{0}
\verb{galantear}{}{}{}{}{}{Colocar adornos; enfeitar.}{ga.lan.te.ar}{\verboinum{4}}
\verb{galanteio}{ê}{}{}{}{s.m.}{Gentileza ou elogio dirigido a uma mulher com propósito de sedução.}{ga.lan.tei.o}{0}
\verb{galanteria}{}{}{}{}{s.f.}{Qualidade de galante; gentileza.}{ga.lan.te.ri.a}{0}
\verb{galão}{}{}{"-ões}{}{s.m.}{Tira de tecido bordado usado como enfeite ou acabamento.}{ga.lão}{0}
\verb{galão}{}{}{"-ões}{}{}{Tira dourada usada com distintivo nas mangas da farda de certas categorias de militares.}{ga.lão}{0}
\verb{galão}{}{}{"-ões}{}{}{Recipiente grande para líquidos.}{ga.lão}{0}
\verb{galão}{}{}{"-ões}{}{}{Medida de capacidade de aproximadamente 4,5 litros.}{ga.lão}{0}
\verb{galar}{}{}{}{}{v.t.}{Fecundar a fêmea de galináceo.}{ga.lar}{\verboinum{1}}
\verb{galardão}{}{}{"-ões}{}{s.m.}{Prêmio, recompensa.}{ga.lar.dão}{0}
\verb{galardoar}{}{}{}{}{v.t.}{Conferir prêmio ou galardão por algum serviço ou merecimento; premiar, recompensar.}{ga.lar.do.ar}{\verboinum{7}}
\verb{galáxia}{cs}{Astron.}{}{}{s.f.}{Qualquer grande sistema estelar, isolado no espaço cósmico, que contém bilhões de estrelas, nebulosas, poeira e gás, que se mantêm agrupados por força da gravidade.}{ga.lá.xi.a}{0}
\verb{galáxia}{cs}{Astron.}{}{}{}{O sistema solar ao qual pertence a Terra, que contém o Sol e cerca de cem bilhões de estrelas; Via"-Láctea.}{ga.lá.xi.a}{0}
\verb{galé}{}{}{}{}{s.f.}{Antiga embarcação movida a vela e remos.}{ga.lé}{0}
\verb{galé}{}{}{}{}{s.m.}{Indivíduo condenado a trabalhos forçados.}{ga.lé}{0}
\verb{galeão}{}{}{"-ões}{}{s.m.}{Antigo navio mercante ou de guerra.}{ga.le.ão}{0}
\verb{galeão}{}{}{"-ões}{}{}{Aparelho de pesca de cerco.}{ga.le.ão}{0}
\verb{galego}{ê}{}{}{}{adj.}{Relativo à Galiza, região da Espanha.}{ga.le.go}{0}
\verb{galego}{ê}{}{}{}{s.m.}{Indivíduo natural ou habitante da Galiza.}{ga.le.go}{0}
\verb{galego}{ê}{}{}{}{}{A língua falada na Galiza.}{ga.le.go}{0}
\verb{galena}{}{}{}{}{s.f.}{O principal minério de chumbo.}{ga.le.na}{0}
\verb{galeota}{ó}{}{}{}{s.f.}{Pequena galé de três mastros. }{ga.le.o.ta}{0}
\verb{galera}{é}{}{}{}{s.f.}{Antiga embarcação movida a remos e a vela.}{ga.le.ra}{0}
\verb{galera}{é}{}{}{}{}{Carro para bombeiros.}{ga.le.ra}{0}
\verb{galera}{é}{}{}{}{}{Conjunto de amigos; turma. (\textit{A galera vai para um }show\textit{ de }rock\textit{ hoje à noite.})}{ga.le.ra}{0}
\verb{galeria}{}{}{}{}{s.f.}{Local para a exposição de objetos de arte.}{ga.le.ri.a}{0}
\verb{galeria}{}{}{}{}{}{Coleção de estátuas, quadros, retratos etc., organizada artisticamente. }{ga.le.ri.a}{0}
\verb{galeria}{}{}{}{}{}{Corredor subterrâneo.}{ga.le.ri.a}{0}
\verb{galês}{}{}{}{}{adj.}{Relativo à região do País de Gales (Grã"-Bretanha).}{ga.lês}{0}
\verb{galês}{}{}{}{}{s.m.}{Indivíduo natural ou habitante dessa região.}{ga.lês}{0}
\verb{galês}{}{}{}{}{}{A língua do País de Gales.}{ga.lês}{0}
\verb{galeto}{ê}{}{}{}{s.m.}{Frango ainda novo.}{ga.le.to}{0}
\verb{galeto}{ê}{Por ext.}{}{}{}{Frango assado.}{ga.le.to}{0}
\verb{galgar}{}{}{}{}{}{Percorrer. (\textit{Galgou a pé 20 quilômetros.})}{gal.gar}{0}
\verb{galgar}{}{}{}{}{v.t.}{Andar a passos largos.}{gal.gar}{0}
\verb{galgar}{}{}{}{}{}{Transpor, pular.}{gal.gar}{0}
\verb{galgar}{}{}{}{}{}{Atingir, alcançar.}{gal.gar}{\verboinum{5}}
\verb{galgo}{}{Zool.}{}{}{s.m.}{Raça de cão de pernas compridas, corpo alongado com abdômen muito estreito,  extremamente ágil e veloz, muito usado para caçar lebres.}{gal.go}{0}
\verb{galha}{}{Bot.}{}{}{s.f.}{Excrescência de alguns vegetais em virtude da ação de organismos como bactérias, fungos etc.}{ga.lha}{0}
\verb{galhada}{}{}{}{}{s.f.}{Chifre dos ruminantes.}{ga.lha.da}{0}
\verb{galhada}{}{}{}{}{}{Conjunto dos galhos de uma árvore ou arbusto; galharada, galharia.}{ga.lha.da}{0}
\verb{galharada}{}{}{}{}{s.f.}{Porção de galhos; galhada, galharia.}{ga.lha.ra.da}{0}
\verb{galhardete}{ê}{}{}{}{s.m.}{Bandeira triangular, estreita e comprida, que se coloca no alto dos mastros geralmente para sinalização. }{ga.lhar.de.te}{0}
\verb{galhardete}{ê}{}{}{}{}{Bandeirinha que se leva para ocasiões festivas.}{ga.lhar.de.te}{0}
\verb{galhardia}{}{}{}{}{s.f.}{Qualidade do que é garboso; elegância.  }{ga.lhar.di.a}{0}
\verb{galhardia}{}{}{}{}{}{Generosidade, magnanimidade.}{ga.lhar.di.a}{0}
\verb{galhardia}{}{}{}{}{}{Bravura, coragem, esforço.}{ga.lhar.di.a}{0}
\verb{galhardo}{}{}{}{}{adj.}{Que tem aparência garbosa, elegante. }{ga.lhar.do}{0}
\verb{galharia}{}{}{}{}{s.f.}{Galharada.}{ga.lha.ri.a}{0}
\verb{galheiro}{ê}{}{}{}{s.m.}{Veado de chifres grandes e ramificações.}{ga.lhei.ro}{0}
\verb{galheiro}{ê}{Pop.}{}{}{}{Homem traído pela mulher; corno.}{ga.lhei.ro}{0}
\verb{galheta}{ê}{}{}{}{s.f.}{Pequeno recipiente de vidro usado para servir azeite ou vinagre à mesa. }{ga.lhe.ta}{0}
\verb{galheta}{ê}{}{}{}{}{Cada um dos vasos em que se põem o vinho e a água para a missa.}{ga.lhe.ta}{0}
\verb{galheta}{ê}{}{}{}{}{Instrumento de vidro usado em laboratórios químicos.}{ga.lhe.ta}{0}
\verb{galheteiro}{ê}{}{}{}{s.m.}{Utensílio de mesa usado para sustentar as galhetas, o saleiro e o pimenteiro.}{ga.lhe.tei.ro}{0}
\verb{galho}{}{}{}{}{s.m.}{Cada uma das partes que brotam do tronco de uma planta.}{ga.lho}{0}
\verb{galho}{}{}{}{}{}{Chifre de veado ou de animal da mesma família.}{ga.lho}{0}
\verb{galho}{}{}{}{}{}{Situação difícil; complicação, dificuldade, encrenca.}{ga.lho}{0}
\verb{galhofa}{ó}{}{}{}{s.f.}{Manifestação alegre; brincadeira, gracejo.}{ga.lho.fa}{0}
\verb{galhofa}{ó}{}{}{}{}{Zombaria, escárnio, deboche.}{ga.lho.fa}{0}
\verb{galhofar}{}{}{}{}{v.i.}{Fazer confusão; divertir"-se ruidosamente.}{ga.lho.far}{0}
\verb{galhofar}{}{}{}{}{}{Dizer em tom de galhofa; debochar, zombar.}{ga.lho.far}{\verboinum{1}}
\verb{galhofeiro}{ê}{}{}{}{adj.}{Que é dado a galhofas; zombeteiro, brincalhão.}{ga.lho.fei.ro}{0}
\verb{galhudo}{}{}{}{}{adj.}{Cheio de galhos; com muitos chifres.}{ga.lhu.do}{0}
\verb{galhudo}{}{Pop.}{}{}{}{Homem traído pela mulher; corno, galheiro.}{ga.lhu.do}{0}
\verb{galicismo}{}{}{}{}{s.m.}{Palavra, expressão ou construção da língua francesa; francesismo.}{ga.li.cis.mo}{0}
\verb{galiforme}{ó}{Zool.}{}{}{s.m.}{Espécime dos galiformes, ordem de aves que inclui as galinhas, os perus e os faisões.}{ga.li.for.me}{0}
\verb{galiforme}{ó}{Zool.}{}{}{adj.}{Relativo aos galiformes.}{ga.li.for.me}{0}
\verb{galileu}{}{}{}{}{adj.}{Relativo à Galileia, região no norte da Palestina.}{ga.li.leu}{0}
\verb{galileu}{}{}{}{}{}{Indivíduo natural ou habitante dessa região.}{ga.li.leu}{0}
\verb{galileu}{}{Fís.}{}{}{s.m.}{Unidade de medida de aceleração igual a um centímetro por segundo quadrado, usada principalmente em prospecção geológica. Símb.: Gal.}{ga.li.leu}{0}
\verb{galináceo}{}{}{}{}{adj.}{Relativo a galiformes.}{ga.li.ná.ceo}{0}
\verb{galinha}{}{Zool.}{}{}{s.f.}{A fêmea adulta do galo.}{ga.li.nha}{0}
\verb{galinha}{}{Cul.}{}{}{}{Prato feito com essa ave.}{ga.li.nha}{0}
\verb{galinha}{}{Fig.}{}{}{s.2g.}{Pessoa muito medrosa ou fraca; covarde.}{ga.li.nha}{0}
\verb{galinha"-d'angola}{ó}{Zool.}{galinhas"-d'angola ⟨ó⟩}{}{s.f.}{Ave da família do galo, de plumagem cinzenta com pintas brancas, cuja cabeça nua é dotada de uma crista óssea.}{ga.li.nha"-d'an.go.la}{0}
\verb{galinhagem}{}{Pop.}{"-ens}{}{s.f.}{Agarramento com intenção de bolinação recíproca; libertinagem.}{ga.li.nha.gem}{0}
\verb{galinha"-morta}{ó}{Pop.}{galinhas"-mortas ⟨ó⟩}{}{s.f.}{Coisa muito barata; pechincha.}{ga.li.nha"-mor.ta}{0}
\verb{galinha"-morta}{ó}{Pop.}{galinhas"-mortas ⟨ó⟩}{}{s.2g.}{Indivíduo fraco, covarde, medroso.}{ga.li.nha"-mor.ta}{0}
\verb{galinheiro}{ê}{}{}{}{s.m.}{Cercado onde se criam ou se alojam galinhas e outras aves domésticas.}{ga.li.nhei.ro}{0}
\verb{galinheiro}{ê}{}{}{}{}{Vendedor de galinhas.}{ga.li.nhei.ro}{0}
\verb{galinicultura}{}{}{}{}{s.f.}{Criação de galináceos, tais como galinhas, galos etc.}{ga.li.ni.cul.tu.ra}{0}
\verb{gálio}{}{Quím.}{}{}{s.m.}{Elemento químico metálico, raro, utilizado em termômetros para altas temperaturas, em algumas ligas, transistores etc. \elemento{31}{69.723}{Ga}.}{gá.lio}{0}
\verb{galo}{}{Zool.}{}{}{s.m.}{Ave galiforme; macho adulto da galinha, de crista carnuda, asas curtas e largas, e dotado de esporões.}{ga.lo}{0}
\verb{galo}{}{Pop.}{}{}{}{Inchaço na testa ou na cabeça que resulta de pancada.}{ga.lo}{0}
\verb{galocha}{ó}{}{}{}{s.f.}{Calçado de borracha que se usa por cima dos sapatos para protegê"-los da umidade, da água e do barro.}{ga.lo.cha}{0}
\verb{galopada}{}{}{}{}{s.f.}{Ação de galopar; corrida a galope.}{ga.lo.pa.da}{0}
\verb{galopador}{ô}{}{}{}{adj.}{Que galopa bem e com perícia.}{ga.lo.pa.dor}{0}
\verb{galopante}{}{}{}{}{adj.2g.}{Que galopa rápido.}{ga.lo.pan.te}{0}
\verb{galopante}{}{Med.}{}{}{}{Diz"-se da tuberculose pulmonar aguda, de desenlace rápido.}{ga.lo.pan.te}{0}
\verb{galopar}{}{}{}{}{v.i.}{Andar ou cavalgar a galope.}{ga.lo.par}{0}
\verb{galopar}{}{}{}{}{}{Correr ou fugir às pressas.}{ga.lo.par}{\verboinum{1}}
\verb{galope}{ó}{}{}{}{s.m.}{O passo mais rápido do cavalo e de outros animais; galopada.}{ga.lo.pe}{0}
\verb{galope}{ó}{}{}{}{}{Espécie de dança a dois tempos.}{ga.lo.pe}{0}
\verb{galope}{ó}{Mús.}{}{}{}{Tipo de estrutura poética com estrofes de seis versos, de dez sílabas utilizada em música folclórica.}{ga.lo.pe}{0}
\verb{galopear}{}{}{}{}{v.i.}{Galopar.}{ga.lo.pe.ar}{\verboinum{4}}
\verb{galpão}{}{}{"-ões}{}{s.m.}{Espécie de construção coberta usada como depósito ou para fins industriais.}{gal.pão}{0}
\verb{galvanizar}{}{}{}{}{v.t.}{Sujeitar algo à ação de uma corrente elétrica, para fins de estimular fisiologicamente.}{gal.va.ni.zar}{0}
\verb{galvanizar}{}{}{}{}{}{Revestir ferro ou aço com uma tênue camada de zinco.}{gal.va.ni.zar}{\verboinum{1}}
\verb{gama}{}{}{}{}{s.m.}{Terceira letra do alfabeto grego.}{ga.ma}{0}
\verb{gama}{}{Mús.}{}{}{}{Sucessão de sons de uma oitava musical; escala.}{ga.ma}{0}
\verb{gama}{}{Fig.}{}{}{}{Série de ideias, teorias etc.}{ga.ma}{0}
\verb{gamação}{}{}{"-ões}{}{s.f.}{Estado de quem se encontra gamado; paixão.}{ga.ma.ção}{0}
\verb{gamado}{}{}{}{}{adj.}{Que se encontra apaixonado; enamorado.}{ga.ma.do}{0}
\verb{gamão}{}{}{"-ões}{}{s.m.}{Jogo de azar e cálculo praticado sobre um tabuleiro, usando"-se dados e tábulas.}{ga.mão}{0}
\verb{gamar}{}{}{}{}{v.i.}{Ficar fascinado; apaixonar"-se, encantar"-se.}{ga.mar}{\verboinum{1}}
\verb{gambá}{}{Zool.}{}{}{s.m.}{Mamífero carnívoro de coloração preta com uma faixa branca dorsal, que excreta, pela glândula anal, um líquido fétido e nauseante como defesa; cangambá, jaritataca.}{gam.bá}{0}
\verb{gambá}{}{Zool.}{}{}{}{\textit{Bras.} Designação comum a diversos marsupiais noturnos, que vivem em árvores e se alimentam de frutos, ovos, insetos e pequenos animais}{gam.bá}{0}
\verb{gambeta}{ê}{}{}{}{s.f.}{Movimento que se faz com o corpo e as pernas, fugindo de um lado para outro para escapar da pessoa que está à frente. }{gam.be.ta}{0}
\verb{gambiano}{}{}{}{}{adj.}{Relativo à Gâmbia (África Ocidental).}{gam.bi.a.no}{0}
\verb{gambiano}{}{}{}{}{s.m.}{Indivíduo natural ou habitante desse país.}{gam.bi.a.no}{0}
\verb{gambiarra}{}{}{}{}{s.f.}{Rampa de luzes na parte anterior e superior do palco.}{gam.bi.ar.ra}{0}
\verb{gambiarra}{}{}{}{}{}{Extensão puxada ilegalmente para furtar energia elétrica; gato.}{gam.bi.ar.ra}{0}
\verb{gambito}{}{}{}{}{s.m.}{Artimanha, ardil, para vencer o adversário.}{gam.bi.to}{0}
\verb{gambito}{}{Pop.}{}{}{}{Perna, de homem ou mulher, muito fina; cambito.  }{gam.bi.to}{0}
\verb{gamela}{é}{}{}{}{s.f.}{Vasilha de madeira ou de barro usada para dar de comer aos porcos.}{ga.me.la}{0}
\verb{gameleira}{ê}{}{}{}{s.f.}{Nome comum a várias espécies de árvores da família das moráceas, algumas com propriedades medicinais, outras de cuja madeira se confeccionam gamelas e certos objetos de uso doméstico.}{ga.me.lei.ra}{0}
\verb{gameta}{ê}{Biol.}{}{}{s.m.}{Célula sexual, masculina ou feminina, em que ocorre a fecundação. }{ga.me.ta}{0}
\verb{gamo}{}{Zool.}{}{}{s.m.}{Animal ruminante com chifres em forma de galhada plana, semelhante ao veado.}{ga.mo}{0}
\verb{gana}{}{}{}{}{s.f.}{Desejo agudo; grande apetite.}{ga.na}{0}
\verb{gana}{}{}{}{}{}{Desejo de fazer mal a alguém; vingança, ódio.}{ga.na}{0}
\verb{ganância}{}{}{}{}{s.f.}{Desejo exagerado de ganhar, lucrar; ambição desmedida.}{ga.nân.cia}{0}
\verb{ganância}{}{}{}{}{}{Ganho ilícito, ilegal.}{ga.nân.cia}{0}
\verb{ganancioso}{ô}{}{"-osos ⟨ó⟩}{"-osa ⟨ó⟩}{adj.}{Que tem desejo exagerado de ganhar, lucrar.}{ga.nan.ci.o.so}{0}
\verb{gancho}{}{}{}{}{s.m.}{Instrumento de metal ou madeira com ponta afiada, própria para pendurar ou suspender alguma coisa.}{gan.cho}{0}
\verb{gancho}{}{}{}{}{}{Suporte para o telefone.}{gan.cho}{0}
\verb{gandaia}{}{}{}{}{s.f.}{Condição de vadio; ociosidade, vadiagem.}{gan.dai.a}{0}
\verb{gandaia}{}{}{}{}{}{Farra, diversão.}{gan.dai.a}{0}
\verb{gandaiar}{}{}{}{}{v.i.}{Viver na gandaia; farrear, vadiar, vagabundear.}{gan.dai.ar}{\verboinum{1}}
\verb{gandula}{}{}{}{}{s.m.}{Pessoa encarregada de buscar as bolas que caem fora da quadra ou do campo.}{gan.du.la}{0}
\verb{ganense}{}{}{}{}{adj.2g.}{Relativo a Gana.}{ga.nen.se}{0}
\verb{ganense}{}{}{}{}{s.2g.}{Indivíduo natural ou habitante desse país.}{ga.nen.se}{0}
\verb{ganga}{}{}{}{}{s.f.}{Tecido ordinário fabricado na Índia.}{gan.ga}{0}
\verb{ganga}{}{}{}{}{s.f.}{Conjunto de resíduos ou restos não aproveitáveis; ninharia.}{gan.ga}{0}
\verb{gânglio}{}{Med.}{}{}{s.m.}{Massa de substância nervosa que contém células e fibras e se encontra ao longo de um nervo ou de um vaso linfático.}{gân.glio}{0}
\verb{ganglionar}{}{}{}{}{adj.2g.}{Relativo a gânglio.}{gan.gli.o.nar}{0}
\verb{gangorra}{ô}{}{}{}{s.f.}{Tábua comprida, apoiada e presa ao centro, própria para duas crianças, cada qual sentada numa de suas pontas, balançarem impulsionando para o alto pela pressão dos pés no solo. (\textit{As meninas estavam brincando na gangorra do parque.})}{gan.gor.ra}{0}
\verb{gangrena}{ê}{Med.}{}{}{s.f.}{Destruição ou morte dos tecidos de uma parte do corpo; necrose.}{gan.gre.na}{0}
\verb{gangrenar}{}{}{}{}{v.t.}{Produzir gangrena; necrosar.}{gan.gre.nar}{0}
\verb{gangrenar}{}{}{}{}{}{Perverter, corromper moralmente.}{gan.gre.nar}{\verboinum{1}}
\verb{gângster}{}{}{}{}{s.m.}{Membro de um bando organizado de malfeitores; bandido, criminoso.}{gângs.ter}{0}
\verb{gangsterismo}{}{}{}{}{s.m.}{Conjunto das atividades criminosas dos gângsteres.}{gangs.te.ris.mo}{0}
\verb{gangue}{}{}{}{}{s.f.}{Grupo organizado de bandidos; quadrilha.}{gan.gue}{0}
\verb{gangue}{}{}{}{}{}{Turma de jovens, grupo.}{gan.gue}{0}
\verb{ganhador}{ô}{}{}{}{adj.}{Que ganha.}{ga.nha.dor}{0}
\verb{ganha"-pão}{}{}{}{}{s.m.}{Trabalho, instrumento ou meio pelo qual se garante o mínimo necessário à sobrevivência.}{ga.nha"-pão}{0}
\verb{ganhar}{}{}{}{}{v.t.}{Adquirir, granjear.}{ga.nhar}{0}
\verb{ganhar}{}{}{}{}{}{Receber gratuitamente.}{ga.nhar}{0}
\verb{ganhar}{}{}{}{}{}{Conseguir, lograr, atingir.}{ga.nhar}{0}
\verb{ganhar}{}{}{}{}{}{Vencer.}{ga.nhar}{0}
\verb{ganhar}{}{}{}{}{}{Receber como remuneração.}{ga.nhar}{0}
\verb{ganhar}{}{}{}{}{v.i.}{Atingir a situação mais favorável em determinado contexto.}{ga.nhar}{\verboinum{1}}
\verb{ganho}{}{}{}{}{adj.}{Que se ganhou.}{ga.nho}{0}
\verb{ganho}{}{}{}{}{}{Aquilo que se ganha; proveito, lucro.}{ga.nho}{0}
\verb{ganido}{}{}{}{}{s.m.}{Grito agudo e lamentoso do cão; uivo.}{ga.ni.do}{0}
\verb{ganido}{}{}{}{}{}{Voz esganiçada.}{ga.ni.do}{0}
\verb{ganir}{}{}{}{}{v.i.}{Dar ganidos, gemer.}{ga.nir}{0}
\verb{ganir}{}{}{}{}{}{Gemer ou dar ganidos como os cães.}{ga.nir}{\verboinum{34}}
\verb{ganso}{}{Zool.}{}{}{s.m.}{Ave palmípede com pescoço comprido e plumagem branca ou cinza.}{gan.so}{0}
\verb{ganzá}{}{Mús.}{}{}{s.m.}{Instrumento de percussão de uso semelhante ao do chocalho, consistindo em um recipiente fechado com grãos ou pequenas pedras.}{gan.zá}{0}
\verb{garagem}{}{}{"-ens}{}{s.f.}{Lugar para abrigar veículos.}{ga.ra.gem}{0}
\verb{garagem}{}{}{"-ens}{}{}{Estabelecimento em que se guardam, consertam ou alugam automóveis.}{ga.ra.gem}{0}
\verb{garagista}{}{}{}{}{s.2g.}{Proprietário ou funcionário de garagem.}{ga.ra.gis.ta}{0}
\verb{garanhão}{}{}{"-ões}{}{adj.}{Diz"-se de cavalo destinado à reprodução.}{ga.ra.nhão}{0}
\verb{garanhão}{}{Pop.}{"-ões}{}{}{Diz"-se de homem conquistador.}{ga.ra.nhão}{0}
\verb{garantia}{}{}{}{}{s.f.}{Ato ou efeito de garantir.}{ga.ran.ti.a}{0}
\verb{garantia}{}{}{}{}{}{Ato, palavra ou documento com que se assegura a qualidade de um bem ou o cumprimento de uma obrigação.}{ga.ran.ti.a}{0}
\verb{garantia}{}{}{}{}{}{Fiança, abonação.}{ga.ran.ti.a}{0}
\verb{garantia}{}{}{}{}{}{Penhor, caução.}{ga.ran.ti.a}{0}
\verb{garantir}{}{}{}{}{v.t.}{Assegurar, certificar.}{ga.ran.tir}{0}
\verb{garantir}{}{}{}{}{}{Responsabilizar"-se, abonar.}{ga.ran.tir}{0}
\verb{garantir}{}{}{}{}{}{Livrar, defender.}{ga.ran.tir}{\verboinum{18}}
\verb{garapa}{}{}{}{}{s.f.}{Caldo extraído da cana"-de"-açúcar.}{ga.ra.pa}{0}
\verb{garatuja}{}{}{}{}{s.f.}{Letra ruim e pouco compreensível ou desenho malfeito.}{ga.ra.tu.ja}{0}
\verb{garatujar}{}{}{}{}{v.t.}{Fazer garatujas.}{ga.ra.tu.jar}{\verboinum{1}}
\verb{garbo}{}{}{}{}{s.m.}{Elegância, distinção, bizarria.}{gar.bo}{0}
\verb{garbosidade}{}{}{}{}{s.f.}{Qualidade ou comportamento de garboso.}{gar.bo.si.da.de}{0}
\verb{garboso}{ô}{}{"-osos ⟨ó⟩}{"-osa ⟨ó⟩}{adj.}{Que tem garbo; elegante, distinto.}{gar.bo.so}{0}
\verb{garça}{}{Zool.}{}{}{s.f.}{Ave de hábitos aquáticos, de pernas compridas, pescoço fino e bico longo e pontiagudo.}{gar.ça}{0}
\verb{garção}{}{}{}{}{}{Var. de \textit{garçom}.}{gar.ção}{0}
\verb{garçom}{}{}{"-ons}{}{s.m.}{Indivíduo que, em restaurantes e bares, serve as pessoas às mesas.}{gar.çom}{0}
\verb{garçonete}{é}{}{}{}{s.f.}{Mulher que, em restaurantes e bares, serve as pessoas às mesas.}{gar.ço.ne.te}{0}
\verb{gardênia}{}{Bot.}{}{}{s.f.}{Arbusto de flores grandes e brancas, cultivado como planta ornamental e também pelas propriedades medicinais de sua madeira; jasmim"-do"-cabo.}{gar.dê.nia}{0}
\verb{gare}{}{}{}{}{s.f.}{Lugar de embarque e desembarque nas estações de trem.}{ga.re}{0}
\verb{garfada}{}{}{}{}{s.f.}{Cada porção de comida que se pega com um garfo.}{gar.fa.da}{0}
\verb{garfada}{}{}{}{}{}{Golpe desferido com um garfo.}{gar.fa.da}{0}
\verb{garfar}{}{}{}{}{v.t.}{Mexer ou revolver usando um garfo.}{gar.far}{0}
\verb{garfar}{}{}{}{}{}{Ferir com um garfo.}{gar.far}{0}
\verb{garfar}{}{Pop.}{}{}{}{Lesar, roubar, prejudicar.}{gar.far}{\verboinum{1}}
\verb{garfo}{}{}{}{}{s.m.}{Utensílio de mesa com duas a quatro pontas em uma das extremidades do cabo, usado para pegar ou segurar alimentos.}{gar.fo}{0}
\verb{garfo}{}{}{}{}{}{Utensílio agrícola semelhante a um garfo de grande porte, usado para juntar ou revolver palha; forcado.}{gar.fo}{0}
\verb{gargalhada}{}{}{}{}{s.f.}{Risada ruidosa e prolongada.}{gar.ga.lha.da}{0}
\verb{gargalhar}{}{}{}{}{v.i.}{Dar gargalhadas.}{gar.ga.lhar}{\verboinum{1}}
\verb{gargalho}{}{}{}{}{s.m.}{Escarro grosso.}{gar.ga.lho}{0}
\verb{gargalo}{}{}{}{}{s.m.}{Abertura estreita localizada na extremidade do pescoço da garrafa.}{gar.ga.lo}{0}
\verb{garganta}{}{}{}{}{s.f.}{Cavidade localizada no fundo da boca, por onde passam os alimentos; laringe.}{gar.gan.ta}{0}
\verb{garganta}{}{}{}{}{}{Abertura ou passagem estreita.}{gar.gan.ta}{0}
\verb{garganta}{}{}{}{}{}{Local estreito entre duas montanhas; desfiladeiro.}{gar.gan.ta}{0}
\verb{gargantear}{}{}{}{}{v.i.}{Cantar, gorjear.}{gar.gan.te.ar}{\verboinum{4}}
\verb{garganteio}{ê}{}{}{}{s.m.}{Ato ou efeito de gargantear.}{gar.gan.tei.o}{0}
\verb{gargantilha}{}{}{}{}{s.f.}{Adorno usado no pescoço; colar.}{gar.gan.ti.lha}{0}
\verb{gargarejamento}{}{}{}{}{s.m.}{Ato ou efeito de gargarejar; gargarejo.}{gar.ga.re.ja.men.to}{0}
\verb{gargarejar}{}{}{}{}{v.i.}{Fazer circular substância líquida pela garganta expelindo o ar e geralmente soltando a voz.}{gar.ga.re.jar}{\verboinum{1}}
\verb{gargarejo}{ê}{}{}{}{s.m.}{Ato ou efeito de gargarejar.}{gar.ga.re.jo}{0}
\verb{gargarejo}{ê}{}{}{}{}{O líquido especialmente preparado para gargarejar.}{gar.ga.re.jo}{0}
\verb{gárgula}{}{}{}{}{s.f.}{Parte saliente das calhas de telhados por onde escoa a água da chuva.}{gár.gu.la}{0}
\verb{gárgula}{}{}{}{}{}{Escultura ou adorno com figuras monstruosas.}{gár.gu.la}{0}
\verb{gari}{}{}{}{}{s.2g.}{Indivíduo que varre as ruas da cidade.}{ga.ri}{0}
\verb{garimpagem}{}{Bras.}{"-ens}{}{s.f.}{Ato ou efeito de garimpar.}{ga.rim.pa.gem}{0}
\verb{garimpagem}{}{Fig.}{"-ens}{}{}{Pesquisa minuciosa em textos, arquivos.}{ga.rim.pa.gem}{0}
\verb{garimpar}{}{}{}{}{v.t.}{Trabalhar como garimpeiro, extraindo pedras e metais preciosos de rios e terras.}{ga.rim.par}{0}
\verb{garimpar}{}{Fig.}{}{}{}{Pesquisar minuciosamente.}{ga.rim.par}{\verboinum{1}}
\verb{garimpeiro}{ê}{}{}{}{s.m.}{Indivíduo que extrai pedras ou metais preciosos; trabalhador de garimpo.}{ga.rim.pei.ro}{0}
\verb{garimpo}{}{}{}{}{s.m.}{Local ou região onde se exploram pedras e metais preciosos.}{ga.rim.po}{0}
\verb{garnisé}{}{Zool.}{}{}{adj.}{Diz"-se de certas raças de galos de porte muito pequeno mesmo quando adultos.}{gar.ni.sé}{0}
\verb{garnisé}{}{Bras.}{}{}{adj.2g.}{Diz"-se de indivíduo de pequena estatura, geralmente de comportamento arrogante.}{gar.ni.sé}{0}
\verb{garoa}{ô}{}{}{}{s.f.}{Chuva miúda, geralmente de longa duração; chuvisco.}{ga.ro.a}{0}
\verb{garoar}{}{}{}{}{v.i.}{Cair garoa; chuviscar.}{ga.ro.ar}{\verboinum{7}}
\verb{garoento}{}{}{}{}{adj.}{Diz"-se de local ou clima em que cai ou costuma cair garoa.}{ga.ro.en.to}{0}
\verb{garota}{ô}{}{}{}{s.f.}{Mulher jovem.}{ga.ro.ta}{0}
\verb{garota}{ô}{Pop.}{}{}{}{Namorada.}{ga.ro.ta}{0}
\verb{garotada}{}{}{}{}{s.f.}{Grupo de garotos ou garotas.}{ga.ro.ta.da}{0}
%\verb{}{}{}{}{}{}{}{}{0}
\verb{garotice}{}{}{}{}{s.f.}{Ato ou dito próprio de garoto.}{ga.ro.ti.ce}{0}
\verb{garotice}{}{}{}{}{}{Época de garoto; juventude.}{ga.ro.ti.ce}{0}
\verb{garoto}{ô}{}{}{}{s.m.}{Rapaz jovem; menino.}{ga.ro.to}{0}
\verb{garoto}{ô}{}{}{}{adj.}{Diz"-se de menino que brinca pelas ruas.}{ga.ro.to}{0}
\verb{garoto"-propaganda}{ô}{Bras.}{garotos"-propaganda ⟨ô⟩}{}{s.m.}{Indivíduo que, nos meios impressos ou audiovisuais de comunicação, apresenta e geralmente fica associado à imagem de determinado produto.}{ga.ro.to"-pro.pa.gan.da}{0}
\verb{garoupa}{ô}{Zool.}{}{}{s.f.}{Peixe marinho encontrado em regiões costeiras.}{ga.rou.pa}{0}
\verb{garra}{}{}{}{}{s.f.}{Unhas em forma de gancho localizada nas extremidades dos membros de certos animais.}{gar.ra}{0}
\verb{garra}{}{Por ext.}{}{}{}{Unhas, dedos, mãos.}{gar.ra}{0}
\verb{garra}{}{Fig.}{}{}{}{Ímpeto, fibra, ardor.}{gar.ra}{0}
\verb{garra}{}{Fig.}{}{}{}{Poder, tirania.}{gar.ra}{0}
\verb{garra}{}{}{}{}{}{Qualquer utensílio em forma de garra, geralmente usado para prender ou segurar objetos.}{gar.ra}{0}
\verb{garrafa}{}{}{}{}{s.f.}{Recipiente com gargalo geralmente alto e estreito, utilizado para guardar líquidos.}{gar.ra.fa}{0}
\verb{garrafada}{}{}{}{}{s.f.}{Golpe desferido com uma garrafa.}{gar.ra.fa.da}{0}
\verb{garrafada}{}{}{}{}{}{Quantidade equivalente ao conteúdo de uma garrafa.}{gar.ra.fa.da}{0}
\verb{garrafal}{}{}{"-ais}{}{adj.2g.}{Diz"-se de letra grande e bastante visível.}{gar.ra.fal}{0}
\verb{garrafão}{}{}{"-ões}{}{s.m.}{Garrafa de grande capacidade, geralmente envolvida em acabamento de vime com alça.}{gar.ra.fão}{0}
\verb{garrafaria}{}{}{}{}{s.f.}{Local onde se guardam ou comercializam garrafas.}{gar.ra.fa.ri.a}{0}
\verb{garrafaria}{}{}{}{}{}{Conjunto de garrafas.}{gar.ra.fa.ri.a}{0}
\verb{garrafeira}{ê}{}{}{}{s.f.}{Conjunto ou depósito de garrafas.}{gar.ra.fei.ra}{0}
\verb{garrafeira}{ê}{}{}{}{}{Local onde se conservam e envelhecem vinhos engarrafados.}{gar.ra.fei.ra}{0}
\verb{garrafeiro}{ê}{}{}{}{s.m.}{Negociante de garrafas, geralmente ambulante.}{gar.ra.fei.ro}{0}
\verb{garranchento}{}{}{}{}{adj.}{Cheio de garranchos.}{gar.ran.chen.to}{0}
\verb{garrancho}{}{}{}{}{s.m.}{Galho tortuoso.}{gar.ran.cho}{0}
\verb{garrancho}{}{Bras.}{}{}{}{Letra manuscrita ruim, ininteligível.}{gar.ran.cho}{0}
\verb{garrancho}{}{Zool.}{}{}{}{Doença que dá no casco das cavalgaduras.}{gar.ran.cho}{0}
\verb{garrar}{}{}{}{}{v.i.}{Navegar à mercê da corrente.}{gar.rar}{\verboinum{1}}
\verb{garriça}{}{Zool.}{}{}{s.f.}{Pássaro pequeno com bico fino e curvo, de cor parda e movimentos ligeiros no solo, que se alimenta de insetos e larvas; garrincha, cambaxirra, corruíra.}{gar.ri.ça}{0}
\verb{garridice}{}{}{}{}{s.f.}{Requinte excessivo no vestuário.}{gar.ri.di.ce}{0}
\verb{garridice}{}{Fig.}{}{}{}{Elegância, brilho.}{gar.ri.di.ce}{0}
\verb{garrido}{}{}{}{}{adj.}{Elegante, vistoso, galante.}{gar.ri.do}{0}
\verb{garrido}{}{Fig.}{}{}{}{Diz"-se de cores vivas e alegres.}{gar.ri.do}{0}
\verb{garrincha}{}{Zool.}{}{}{s.f.}{Garriça.}{gar.rin.cha}{0}
\verb{garrote}{ó}{}{}{}{s.m.}{Pau curto com que se apertava a corda no estrangulamento dos condenados.}{gar.ro.te}{0}
\verb{garrote}{ó}{}{}{}{}{Faixa com que se aperta um membro para estancar sangramentos.}{gar.ro.te}{0}
\verb{garrote}{ó}{}{}{}{}{Bezerro entre dois e quatro anos de idade.}{gar.ro.te}{0}
\verb{garrotilho}{}{}{}{}{s.m.}{Angina aguda e sufocante.}{gar.ro.ti.lho}{0}
\verb{garrucha}{}{}{}{}{s.f.}{Antiga pistola de cano largo, cuja munição era carregada pela boca; bacamarte.}{gar.ru.cha}{0}
\verb{garrulice}{}{}{}{}{s.f.}{Qualidade de gárrulo; tagarelice.}{gar.ru.li.ce}{0}
\verb{gárrulo}{}{}{}{}{adj.}{Que canta muito.}{gár.ru.lo}{0}
\verb{gárrulo}{}{}{}{}{}{Que fala demais; tagarela.}{gár.ru.lo}{0}
\verb{garupa}{}{}{}{}{s.f.}{Parte de cima e posterior do corpo das cavalgaduras.}{ga.ru.pa}{0}
\verb{garupa}{}{}{}{}{}{Assento traseiro em motocicletas ou região atrás do selim em bicicletas.}{ga.ru.pa}{0}
\verb{gás}{}{}{}{}{s.m.}{Substância que se expande por todo o espaço do recipiente ou do ambiente que ocupa.}{gás}{0}
\verb{gás}{}{Fís.}{}{}{}{Um dos três estados da matéria.}{gás}{0}
\verb{gás}{}{Fig.}{}{}{}{Energia para determinada atividade; ânimo.}{gás}{0}
\verb{gaseificação}{}{}{"-ões}{}{s.f.}{Processo de gaseificar.}{ga.sei.fi.ca.ção}{0}
\verb{gaseificar}{}{}{}{}{v.t.}{Transformar em gás; vaporizar.}{ga.se.i.fi.car}{\verboinum{2}}
\verb{gases}{}{}{}{}{s.m.pl.}{Substâncias gasosas produzidas pela fermentação dos alimentos no estômago e no intestino.}{ga.ses}{0}
\verb{gasganete}{ê}{}{}{}{s.m.}{Garganta.}{gas.ga.ne.te}{0}
\verb{gasguita}{}{Bras.}{}{}{adj.2g.}{Diz"-se de mulher ou criança de voz estridente.}{gas.gui.ta}{0}
\verb{gasificar}{}{}{}{}{v.t.}{Gaseificar.}{ga.si.fi.car}{\verboinum{2}}
\verb{gasoduto}{}{}{}{}{s.m.}{Sistema de tubos para conduzir gases por grandes distâncias.}{ga.so.du.to}{0}
\verb{gasogênio}{}{}{}{}{s.m.}{Aparelho para produzir gás.}{ga.so.gê.nio}{0}
\verb{gasolina}{}{}{}{}{s.f.}{Substância líquida derivada do petróleo e altamente inflamável, utilizada como combustível em motores a combustão.}{ga.so.li.na}{0}
\verb{gasometria}{}{}{}{}{s.f.}{Técnica ou processo de medir o volume dos gases.}{ga.so.me.tri.a}{0}
\verb{gasômetro}{}{}{}{}{s.m.}{Fábrica ou reservatório de gás.}{ga.sô.me.tro}{0}
\verb{gasosa}{ó}{Bras.}{}{}{s.f.}{Bebida refrigerante que contém gás, feita à base de limão; soda limonada.}{ga.so.sa}{0}
\verb{gasoso}{ô}{}{"-osos ⟨ó⟩}{"-osa ⟨ó⟩}{adj.}{Da natureza do gás.}{ga.so.so}{0}
\verb{gasparinho}{}{Bras.}{}{}{s.m.}{Fração mínima de bilhete de loteria.}{gas.pa.ri.nho}{0}
\verb{gasparino}{}{Bras.}{}{}{s.m.}{Gasparinho.}{gas.pa.ri.no}{0}
\verb{gastador}{ô}{}{}{}{adj.}{Que gasta em excesso; perdulário, esbanjador.}{gas.ta.dor}{0}
\verb{gastar}{}{}{}{}{v.t.}{Diminuir pelo uso a quantidade de alguma coisa; consumir. (\textit{O rapaz gasta todo o seu salário comprando livros.})}{gas.tar}{0}
\verb{gastar}{}{}{}{}{}{Fazer alguma coisa ir perdendo a parte de fora; desgastar. (\textit{Andei tanto que gastei a sola do tênis.})}{gas.tar}{0}
\verb{gastar}{}{}{}{}{}{Deixar de ter alguma coisa pelo uso que se faz dela; despender. (\textit{Gastei muito tempo procurando um presente para minha irmã.})}{gas.tar}{\verboinum{1}}
\verb{gasto}{}{}{}{}{adj.}{Que se gastou.}{gas.to}{0}
\verb{gasto}{}{}{}{}{}{Danificado, estragado.}{gas.to}{0}
\verb{gasto}{}{}{}{}{}{Com sinais de muito uso; surrado.}{gas.to}{0}
\verb{gasto}{}{Fig.}{}{}{}{Abatido, consumido, abalado.}{gas.to}{0}
\verb{gasto}{}{}{}{}{s.m.}{Aquilo que se gastou; despesa.}{gas.to}{0}
\verb{gastralgia}{}{Med.}{}{}{s.f.}{Dor no estômago.}{gas.tral.gi.a}{0}
\verb{gastrenterite}{}{Med.}{}{}{s.f.}{Inflamação no estômago e nos intestinos.}{gas.tren.te.ri.te}{0}
\verb{gástrico}{}{Anat.}{}{}{adj.}{Relativo a estômago.}{gás.tri.co}{0}
\verb{gastrintestinal}{}{Anat.}{"-ais}{}{adj.2g.}{Relativo ao estômago, ao intestino e aos órgãos anexos.}{gas.trin.tes.ti.nal}{0}
\verb{gastrite}{}{Med.}{}{}{s.f.}{Inflamação no estômago.}{gas.tri.te}{0}
\verb{gastronomia}{}{}{}{}{s.f.}{Conjunto de técnicas e prática de conhecimentos culinários.}{gas.tro.no.mi.a}{0}
\verb{gastrônomo}{}{}{}{}{s.m.}{Conhecedor ou apreciador dos prazeres culinários.}{gas.trô.no.mo}{0}
\verb{gastrópode}{}{Zool.}{}{}{adj.}{Espécime dos gastrópodes, classe de moluscos que inclui lesmas, caramujos e caracóis.}{gas.tró.po.de}{0}
\verb{gastroscopia}{}{Med.}{}{}{s.f.}{Exame de observação do interior do estômago feito com aparelhos apropriados.}{gas.tros.co.pi.a}{0}
\verb{gastrovascular}{}{Anat.}{}{}{adj.2g.}{Que tem funções digestivas e circulatórias.}{gas.tro.vas.cu.lar}{0}
\verb{gastura}{}{Bras.}{}{}{s.f.}{Coceira, irritação, aflição.}{gas.tu.ra}{0}
\verb{gata}{}{}{}{}{s.f.}{Fêmea do gato.}{ga.ta}{0}
\verb{gata}{}{Bras.}{}{}{}{Mulher bonita e provocante, geralmente jovem.}{ga.ta}{0}
\verb{gata}{}{}{}{}{}{Antiga máquina de guerra semelhante à catapulta.}{ga.ta}{0}
\verb{gataria}{}{}{}{}{s.f.}{Bando de gatos.}{ga.ta.ri.a}{0}
\verb{gaticídio}{}{}{}{}{s.m.}{Ato de matar gato(s).}{ga.ti.cí.dio}{0}
\verb{gatil}{}{Bras.}{"-is}{}{s.m.}{Lugar onde se criam ou alojam gatos.}{ga.til}{0}
\verb{gatilho}{}{}{}{}{s.m.}{Peça de uma arma de fogo que a faz disparar.}{ga.ti.lho}{0}
\verb{gatimanhas}{}{}{}{}{s.f.}{Gatimanhos.}{ga.ti.ma.nhas}{0}
\verb{gatimanho}{}{}{}{}{s.m.}{Gatimanhos.}{ga.ti.ma.nho}{0}
\verb{gatimanhos}{}{}{}{}{s.m.pl.}{Sinais ou gesticulações ridículas; trejeitos.}{ga.ti.ma.nhos}{0}
\verb{gatimonha}{}{}{}{}{s.f.}{Gatimanhos.}{ga.ti.mo.nha}{0}
\verb{gatinha}{}{Bras.}{}{}{s.f.}{Mulher jovem e muito bonita.}{ga.ti.nha}{0}
\verb{gatinhas}{}{}{}{}{s.f.pl.}{Usado na locução \textit{de gatinhas}: modo de andar com as mãos e os joelhos no chão; engatinhar. (\textit{Andavam de gatinhas para não serem vistos.})}{ga.ti.nhas}{0}
\verb{gato}{}{Zool.}{}{}{s.m.}{Mamífero felino de pequeno porte, carnívoro, com unhas retráteis e grande habilidade.}{ga.to}{0}
\verb{gato}{}{Pop.}{}{}{}{Indivíduo que rouba; ladrão.}{ga.to}{0}
\verb{gato}{}{Bras.}{}{}{}{Homem muito atraente.}{ga.to}{0}
\verb{gato"-do"-mato}{}{Zool.}{gatos"-do"-mato}{}{s.m.}{Mamífero felino de médio porte que vive nas florestas.}{ga.to"-do"-ma.to}{0}
\verb{gato"-do"-mato"-grande}{}{Bras.}{gatos"-do"-mato"-grande}{}{s.m.}{Jaguatirica.}{ga.to"-do"-ma.to"-gran.de}{0}
\verb{gato"-pingado}{}{Pop.}{}{}{s.m.}{Cada uma das poucas pessoas que comparecem a uma reunião ou um espetáculo.}{ga.to"-pin.ga.do}{0}
\verb{gato"-sapato}{}{}{gatos"-sapatos}{}{s.m.}{Coisa desprezível.}{ga.to"-sa.pa.to}{0}
\verb{gato"-sapato}{}{}{gatos"-sapatos}{}{}{Usado na expressão \textit{fazer (de) gato"-sapato}: tratar alguém com desprezo e desrespeito.}{ga.to"-sa.pa.to}{0}
\verb{gatunagem}{}{}{"-ens}{}{s.f.}{Ato próprio de gatuno; roubo, furto.}{ga.tu.na.gem}{0}
\verb{gatunar}{}{}{}{}{v.t.}{Roubar, furtar.}{ga.tu.nar}{\verboinum{1}}
\verb{gatuno}{}{}{}{}{s.m.}{Indivíduo que furta; ladrão.}{ga.tu.no}{0}
\verb{gaturamo}{}{Bras.}{}{}{s.m.}{Pássaro de bico curto e grosso, cauda curta e dorso azul ou verde escuro, e que se alimenta de frutos.}{ga.tu.ra.mo}{0}
\verb{gauchada}{}{}{}{}{s.f.}{Grande quantidade de gaúchos.}{ga.u.cha.da}{0}
\verb{gauchada}{}{}{}{}{}{Ação própria de gaúcho; gaucharia.}{ga.u.cha.da}{0}
\verb{gaucharia}{}{}{}{}{s.f.}{Ação própria de gaúcho.}{ga.u.cha.ri.a}{0}
\verb{gauchesco}{ê}{Bras.}{}{}{adj.}{Relativo a gaúcho.}{ga.u.ches.co}{0}
\verb{gaúcho}{}{}{}{}{adj.}{Relativo ao Rio Grande do Sul; rio"-grandense"-do"-sul.}{ga.ú.cho}{0}
\verb{gaúcho}{}{}{}{}{s.m.}{Indivíduo natural ou habitante desse estado.}{ga.ú.cho}{0}
\verb{gaudério}{}{}{}{}{s.m.}{Vadio, malandro.}{gau.dé.rio}{0}
\verb{gaudério}{}{}{}{}{}{Folia, patuscada, folgança, brincadeira.}{gau.dé.rio}{0}
\verb{gáudio}{}{}{}{}{s.m.}{Alegria, regozijo, júbilo.}{gáu.dio}{0}
\verb{gáudio}{}{}{}{}{}{Folgança, brincadeira, folia.}{gáu.dio}{0}
\verb{gaulês}{}{}{}{}{adj.}{Relativo à Gália, país anteriormente localizado no atual território da França e que foi conquistado pelos romanos.}{gau.lês}{0}
\verb{gaulês}{}{}{}{}{s.m.}{Indivíduo natural ou habitante desse país.}{gau.lês}{0}
\verb{gaulês}{}{Por ext.}{}{}{}{Indivíduo natural ou habitante da França.}{gau.lês}{0}
\verb{gaulês}{}{Gram.}{}{}{}{Língua céltica falada pelo povo gaulês.}{gau.lês}{0}
\verb{gávea}{}{}{}{}{s.f.}{Guarita descoberta ou cesto no alto de um mastro em certas embarcações.}{gá.vea}{0}
\verb{gaveta}{ê}{}{}{}{s.f.}{Tipo de caixa descoberta e corrediça que fica embutida em móveis como escrivaninha, cômoda e armário, usada para guardar coisas.}{ga.ve.ta}{0}
\verb{gaveteiro}{ê}{}{}{}{s.m.}{Estrutura avulsa ou adaptada na parte interna de um móvel na qual se encaixam e correm as gavetas.}{ga.ve.tei.ro}{0}
\verb{gavial}{}{Zool.}{"-ais}{}{s.m.}{Crocodilo de grande porte e focinho longo e fino, que se encontra no rio Ganges, na Índia.}{ga.vi.al}{0}
\verb{gavião}{}{Zool.}{"-ões}{}{s.m.}{Nome comum a diversas aves de rapina.}{ga.vi.ão}{0}
\verb{gavião}{}{Pop.}{"-ões}{}{}{Indivíduo muito esperto ou conquistador.}{ga.vi.ão}{0}
\verb{gavinha}{}{Bot.}{}{}{s.f.}{Estrutura de fixação das plantas trepadeiras com a qual elas se prendem a paredes, estacas e outras plantas.}{ga.vi.nha}{0}
\verb{gay}{}{}{}{}{adj.}{Diz"-se de pessoa homossexual.}{\textit{gay}}{0}
\verb{gaze}{}{}{}{}{s.f.}{Tecido fino e transparente.}{ga.ze}{0}
\verb{gaze}{}{}{}{}{}{Tecido de algodão, muito poroso, usado em compressas, curativos etc.}{ga.ze}{0}
\verb{gazear}{}{}{}{}{v.i.}{Cantar (a garça, a andorinha etc.).}{ga.ze.ar}{\verboinum{4}}
\verb{gazeio}{ê}{}{}{}{s.m.}{Som produzido pela andorinha, a garça e outras aves.}{ga.zei.o}{0}
\verb{gazela}{é}{Zool.}{}{}{s.f.}{Designação comum aos ruminantes de pernas longas e chifres em forma de espiral.}{ga.ze.la}{0}
\verb{gazela}{é}{Fig.}{}{}{}{Moça bonita e elegante.}{ga.ze.la}{0}
\verb{gazeta}{ê}{}{}{}{s.f.}{Publicação periódica de notícias políticas, literárias, artísticas; jornal.}{ga.ze.ta}{0}
\verb{gazeta}{ê}{}{}{}{}{Falta à aula ou ao serviço para vadiar.}{ga.ze.ta}{0}
\verb{gazetear}{}{}{}{}{v.i.}{Faltar às aulas ou ao trabalho sem motivo relevante.}{ga.ze.te.ar}{\verboinum{4}}
\verb{gazeteiro}{ê}{}{}{}{s.m.}{Estudante que mata aula.}{ga.ze.tei.ro}{0}
\verb{gazeteiro}{ê}{}{}{}{}{Indivíduo que redige ou publica gazetas.}{ga.ze.tei.ro}{0}
\verb{gazetilha}{}{}{}{}{s.f.}{Seção com notícias literárias ou humorísticas de um periódico; folhetim.}{ga.ze.ti.lha}{0}
\verb{gazua}{}{}{}{}{s.f.}{Ferro curvo ou torto com que se podem abrir fechaduras; chave falsa.}{ga.zu.a}{0}
\verb{GB}{}{Informát.}{}{}{}{Símb. de \textit{gigabyte}.}{GB}{0}
\verb{Gd}{}{Quím.}{}{}{}{Símb. do \textit{gadolínio}. }{Gd}{0}
\verb{Ge}{}{Quím.}{}{}{}{Símb. do \textit{germânio}.}{Ge}{0}
\verb{gê}{}{}{}{}{s.m.}{Nome da letra \textit{g}.}{gê}{0}
\verb{geada}{}{}{}{}{s.f.}{Orvalho congelado que forma fina camada branca sobre as folhas, os solos, os telhados.}{ge.a.da}{0}
\verb{gear}{}{}{}{}{v.i.}{Formar"-se geada, cair geada.}{ge.ar}{\verboinum{4}}
\verb{geena}{}{}{}{}{s.f.}{O inferno.}{ge.e.na}{0}
\verb{geena}{}{}{}{}{}{Lugar de grande sofrimento.}{ge.e.na}{0}
\verb{gêiser}{}{}{}{}{s.m.}{Fonte natural da qual jorram jatos de água fervente.}{gêi.ser}{0}
\verb{gel}{é}{}{géis}{}{s.m.}{Substância gelatinosa utilizada em vários cosméticos, como o fixador de cabelo.}{gel}{0}
\verb{geladeira}{ê}{}{}{}{s.f.}{Eletrodoméstico dotado de máquina frigorífica para manter a temperatura baixa em seu interior, no qual são guardados alimentos perecíveis.}{ge.la.dei.ra}{0}
%\verb{}{}{}{}{}{}{}{}{0}
\verb{gelado}{}{}{}{}{adj.}{Muito frio.}{ge.la.do}{0}
\verb{gelado}{}{}{}{}{}{Sorvete.}{ge.la.do}{0}
\verb{gelado}{}{}{}{}{}{Qualquer bebida gelada.}{ge.la.do}{0}
\verb{geladura}{}{}{}{}{s.f.}{Queima ou seca produzida nas plantas pela geada.}{ge.la.du.ra}{0}
\verb{geladura}{}{Med.}{}{}{}{Lesão produzida nos tecidos, causada pela ação de frio intenso.}{ge.la.du.ra}{0}
\verb{gelar}{}{}{}{}{v.t.}{Tornar muito frio.}{ge.lar}{0}
\verb{gelar}{}{}{}{}{}{Causar medo.}{ge.lar}{0}
\verb{gelar}{}{}{}{}{v.i.}{Converter"-se em gelo.}{ge.lar}{0}
\verb{gelar}{}{}{}{}{}{Perder o entusiasmo.}{ge.lar}{0}
\verb{gelar}{}{}{}{}{}{Ficar amedrontado.}{ge.lar}{\verboinum{1}}
\verb{gelatina}{}{}{}{}{s.f.}{Substância transparente, incolor, inodora e insípida que se extrai dos ossos e dos tecidos fibrosos dos animais.}{ge.la.ti.na}{0}
\verb{gelatina}{}{}{}{}{}{Essa substância preparada para uso alimentar.}{ge.la.ti.na}{0}
\verb{gelatinoso}{ô}{}{"-osos ⟨ó⟩}{"-osa ⟨ó⟩}{adj.}{Que contém gelatina.}{ge.la.ti.no.so}{0}
\verb{gelatinoso}{ô}{}{"-osos ⟨ó⟩}{"-osa ⟨ó⟩}{}{Que tem a natureza ou o aspecto da geleia.}{ge.la.ti.no.so}{0}
\verb{gelatinoso}{ô}{Fig.}{"-osos ⟨ó⟩}{"-osa ⟨ó⟩}{}{Mole como gelatina.}{ge.la.ti.no.so}{0}
\verb{geleia}{é}{Cul.}{}{}{s.f.}{Alimento preparado com frutas cozidas em açúcar, e que, ao esfriar, toma consistência gelatinosa.}{ge.lei.a}{0}
\verb{geleira}{ê}{}{}{}{s.f.}{Grande acúmulo de gelo em regiões montanhosas.}{ge.lei.ra}{0}
\verb{geleira}{ê}{}{}{}{}{Montanha flutuante de gelo que é transportada pelas correntes marítimas.}{ge.lei.ra}{0}
\verb{gelha}{ê}{}{}{}{s.f.}{Grão de cereal malformado e cuja película se mostra enrugada.}{ge.lha}{0}
\verb{gelha}{ê}{Por ext.}{}{}{}{Ruga na pele, especialmente do rosto.}{ge.lha}{0}
\verb{gelidez}{ê}{}{}{}{s.f.}{Qualidade ou estado do que é gélido.}{ge.li.dez}{0}
\verb{gélido}{}{}{}{}{adj.}{Muito frio; congelado.}{gé.li.do}{0}
\verb{gelo}{ê}{}{}{}{s.m.}{Solidificação de água ou outro líquido pelo frio.}{ge.lo}{0}
\verb{gelo}{ê}{Fig.}{}{}{}{Frio excessivo.}{ge.lo}{0}
\verb{gelo}{ê}{Fig.}{}{}{}{Frieza, indiferença, insensibilidade.}{ge.lo}{0}
\verb{gelo"-seco}{ê\ldots{}ê}{}{gelos"-secos ⟨ê\ldots{}ê⟩}{}{s.m.}{Gás carbônico em estado sólido.}{ge.lo"-se.co}{0}
\verb{gelosia}{}{}{}{}{s.f.}{Grade feita de ripas de madeira em portas e janelas, que permite ver o interior sem ser visto.}{ge.lo.si.a}{0}
\verb{gema}{}{}{}{}{s.f.}{A parte central, amarela, do ovo das aves.}{ge.ma}{0}
\verb{gema}{}{}{}{}{}{Aquilo que, brotando de um tecido ou de um órgão, pode originar um novo indivíduo.}{ge.ma}{0}
\verb{gema}{}{}{}{}{}{Pedra preciosa.}{ge.ma}{0}
\verb{gemada}{}{}{}{}{s.f.}{Gema ou porção de gemas de ovo, batidas com açúcar, às quais se adiciona, às vezes, um líquido quente.}{ge.ma.da}{0}
\verb{gemedeira}{ê}{}{}{}{s.f.}{Vozearia de gemidos.}{ge.me.dei.ra}{0}
\verb{gemedeira}{ê}{Por ext.}{}{}{}{Lamúria, lamentação.}{ge.me.dei.ra}{0}
\verb{gêmeo}{}{}{}{}{adj.}{Diz"-se de indivíduo que nasceu do mesmo parto que outro.}{gê.meo}{0}
\verb{gêmeo}{}{}{}{}{}{Idêntico, igual.}{gê.meo}{0}
\verb{gêmeos}{}{Astron.}{}{}{s.m.pl.}{Terceira constelação do zodíaco.}{gê.me.os}{0}
\verb{gêmeos}{}{Astrol.}{}{}{}{O signo do zodíaco referente a essa constelação.}{gê.me.os}{0}
\verb{gemer}{ê}{}{}{}{v.i.}{Exprimir, por meio de gemidos, dor física ou moral.}{ge.mer}{0}
\verb{gemer}{ê}{}{}{}{}{Produzir som triste ou monótono.}{ge.mer}{\verboinum{12}}
\verb{gemido}{}{}{}{}{s.m.}{Ato de gemer.}{ge.mi.do}{0}
\verb{gemido}{}{}{}{}{}{Som lastimoso, que provém de dor física ou moral.}{ge.mi.do}{0}
\verb{geminação}{}{}{"-ões}{}{s.f.}{Disposição aos pares.}{ge.mi.na.ção}{0}
\verb{geminação}{}{Gram.}{"-ões}{}{}{Duplicação de consoantes.}{ge.mi.na.ção}{0}
\verb{geminado}{}{}{}{}{adj.}{Que se geminou, que se apresenta ligado; duplicado.}{ge.mi.na.do}{0}
\verb{geminado}{}{}{}{}{}{Diz"-se de órgãos vegetais dispostos dois a dois.}{ge.mi.na.do}{0}
\verb{geminar}{}{}{}{}{v.t.}{Duplicar, unindo.}{ge.mi.nar}{0}
\verb{geminar}{}{Gram.}{}{}{}{Duplicar (consoantes, vogais).}{ge.mi.nar}{\verboinum{1}}
\verb{geminiano}{}{Astrol.}{}{}{s.m.}{Indivíduo que nasceu sob o signo de gêmeos.}{ge.mi.ni.a.no}{0}
\verb{geminiano}{}{Astrol.}{}{}{adj.}{Relativo ou pertencente a esse signo.}{ge.mi.ni.a.no}{0}
\verb{genciana}{}{Bot.}{}{}{s.f.}{Planta de propriedades medicinais.   }{gen.ci.a.na}{0}
\verb{gene}{}{}{}{}{s.m.}{Cada uma das partículas do cromossomo que encerra os caracteres hereditários.   }{ge.ne}{0}
\verb{genealogia}{}{}{}{}{s.f.}{Exposição da origem e ramificações de uma família; linhagem, estirpe.   (\textit{Essa é a genealogia da minha família.})}{ge.ne.a.lo.gi.a}{0}
\verb{genealogia}{}{}{}{}{}{Procedência. (\textit{Ele é de genealogia francesa.})}{ge.ne.a.lo.gi.a}{0}
\verb{genealógico}{}{}{}{}{adj.}{Relativo a genealogia. (\textit{O estudo genealógico vincula"-se aos estudos históricos.})}{ge.ne.a.ló.gi.co}{0}
\verb{genealogista}{}{}{}{}{s.2g.}{Especialista em genealogia.   }{ge.ne.a.lo.gis.ta}{0}
\verb{genebra}{é}{}{}{}{s.f.}{Bebida alcoólica feita de aguardente e bagas de zimbro.}{ge.ne.bra}{0}
\verb{genebrês}{}{}{}{}{adj.}{Relativo a Genebra, cidade localizada na Suíça.}{ge.ne.brês}{0}
\verb{genebrês}{}{}{}{}{s.m.}{Indivíduo natural ou habitante dessa cidade.    }{ge.ne.brês}{0}
\verb{genebrino}{}{}{}{}{adj.}{Relativo a Genebra, cidade localizada na Suíça; genebrês. }{ge.ne.bri.no}{0}
\verb{genebrino}{}{}{}{}{s.m.}{Indivíduo natural ou habitante dessa cidade.  }{ge.ne.bri.no}{0}
\verb{general}{}{}{"-ais}{}{s.m.}{Oficial que pertence aos escalões mais elevados das Forças Armadas.}{ge.ne.ral}{0}
\verb{general}{}{Bot.}{"-ais}{}{}{Gardênia.  }{ge.ne.ral}{0}
\verb{generalado}{}{}{}{}{}{Var. de \textit{generalato}.    }{ge.ne.ra.la.do}{0}
\verb{generalato}{}{}{}{}{s.m.}{Graduação, dignidade de general.   }{ge.ne.ra.la.to}{0}
\verb{generalato}{}{}{}{}{}{Dignidade do geral de uma ordem religiosa.   }{ge.ne.ra.la.to}{0}
\verb{general"-de"-brigada}{}{}{generais"-de"-brigada}{}{s.m.}{Posto de hierarquia do Exército imediatamente superior ao de coronel e imediatamente inferior ao de general"-de"-divisão.   }{ge.ne.ral"-de"-bri.ga.da}{0}
\verb{general"-de"-brigada}{}{}{generais"-de"-brigada}{}{}{Militar que ocupa esse posto.}{ge.ne.ral"-de"-bri.ga.da}{0}
\verb{general"-de"-divisão}{}{}{generais"-de"-divisão}{}{s.m.}{Posto de hierarquia do Exército imediatamente superior ao de general"-de"-brigada e imediatamente inferior ao de general"-de"-exército.    }{ge.ne.ral"-de"-di.vi.são}{0}
\verb{general"-de"-divisão}{}{}{generais"-de"-divisão}{}{}{Militar que ocupa esse posto.}{ge.ne.ral"-de"-di.vi.são}{0}
\verb{general"-de"-exército}{z}{}{generais"-de"-exército}{}{s.m.}{Posto de hierarquia do Exército imediatamente superior ao de general"-de"-divisão e imediatamente inferior ao de marechal.   }{ge.ne.ral"-de"-e.xér.ci.to}{0}
\verb{general"-de"-exército}{z}{}{generais"-de"-exército}{}{}{Militar que ocupa esse posto.}{ge.ne.ral"-de"-e.xér.ci.to}{0}
\verb{generalidade}{}{}{}{}{s.f.}{Qualidade daquilo que é geral.}{ge.ne.ra.li.da.de}{0}
\verb{generalidade}{}{}{}{}{}{A quase totalidade.}{ge.ne.ra.li.da.de}{0}
\verb{generalidade}{}{}{}{}{}{Ideia ou princípio geral.   }{ge.ne.ra.li.da.de}{0}
\verb{generalidades}{}{}{}{}{s.f.pl.}{Rudimentos, princípios elementares gerais.   }{ge.ne.ra.li.da.des}{0}
\verb{generalíssimo}{}{}{}{}{adj.}{Superlativo de \textit{geral}. }{ge.ne.ra.lís.si.mo}{0}
\verb{generalíssimo}{}{}{}{}{s.m.}{O general"-chefe do Exército. }{ge.ne.ra.lís.si.mo}{0}
\verb{generalíssimo}{}{}{}{}{}{Título do soberano de uma nação, em relação ao Exército.  }{ge.ne.ra.lís.si.mo}{0}
\verb{generalização}{}{}{ões}{}{s.f.}{Ação ou efeito de generalizar; difusão geral; vulgarização.}{ge.ne.ra.li.za.ção}{0}
\verb{generalização}{}{}{ões}{}{}{Operação do conhecimento que consiste em comparar as qualidades comuns dos indivíduos, desprezando as diferenças para reuni"-los sob um mesmo rótulo.      }{ge.ne.ra.li.za.ção}{0}
\verb{generalizar}{}{}{}{}{v.t.}{Tornar geral, tornar comum. (\textit{O problema foi generalizar essa informação.})}{ge.ne.ra.li.zar}{0}
\verb{generalizar}{}{}{}{}{}{Difundir"-se, propagar"-se. }{ge.ne.ra.li.zar}{0}
\verb{generalizar}{}{}{}{}{}{Atribuir, precipitadamente, uma característica individual a todo um conjunto de fatos.   }{ge.ne.ra.li.zar}{\verboinum{1}}
\verb{generativo}{}{}{}{}{adj.}{Que gera ou tem a propriedade de geral; gerativo.}{ge.ne.ra.ti.vo}{0}
\verb{generativo}{}{}{}{}{}{Relativo a geração.  }{ge.ne.ra.ti.vo}{0}
\verb{genérico}{}{}{}{}{adj.}{Que pertence a gênero ou é relativo a ele.}{ge.né.ri.co}{0}
\verb{genérico}{}{}{}{}{}{Que tem caráter de generalidade. (\textit{Essa foi a proposta mais genérica que eu consegui fazer.})}{ge.né.ri.co}{0}
\verb{genérico}{}{}{}{}{}{Expresso em termos vagos. (\textit{Ele sempre fala com termos genéricos, sem dar nenhuma certeza.})}{ge.né.ri.co}{0}
\verb{genérico}{}{}{}{}{}{Diz"-se de medicamento comercializado com o nome técnico e não com o nome da marca comercial.   }{ge.né.ri.co}{0}
\verb{gênero}{}{}{}{}{s.m.}{Ideia geral de um grupo de seres ou de objetos que apresentam caracteres comuns.}{gê.ne.ro}{0}
\verb{gênero}{}{Gram.}{}{}{}{Grupo morfológico intermediário entre a família e a espécie.}{gê.ne.ro}{0}
\verb{gênero}{}{}{}{}{}{Maneira, modo, estilo. (\textit{Ela tem um gênero muito escandaloso no vestir"-se.})}{gê.ne.ro}{0}
\verb{gênero}{}{Gram.}{}{}{}{Categoria semântica que estabelece concordância obrigatória entre diversas classes gramaticais do português.  }{gê.ne.ro}{0}
\verb{gêneros}{}{}{}{}{s.m.pl.}{Produtos alimentícios, víveres.  }{gê.ne.ros}{0}
\verb{generosidade}{}{}{}{}{s.f.}{Qualidade de ser generoso; bondade. }{ge.ne.ro.si.da.de}{0}
\verb{generosidade}{}{}{}{}{}{Ação generosa, boa. (\textit{A sua atitude foi de uma generosidade incomparável.})}{ge.ne.ro.si.da.de}{0}
\verb{generoso}{o}{}{"-osos ⟨ó⟩}{"-osa ⟨ó⟩}{adj.}{Que tem prazer em ajudar. (\textit{Ela é uma senhora muito generosa.})}{ge.ne.ro.so}{0}
\verb{generoso}{o}{}{"-osos ⟨ó⟩}{"-osa ⟨ó⟩}{}{Que perdoa com facilidade.   }{ge.ne.ro.so}{0}
\verb{gênese}{}{}{}{}{s.f.}{Ponto de partida a que alguma coisa deve a sua existência; origem.   }{gê.ne.se}{0}
\verb{gênese}{}{}{}{}{s.m.}{Primeiro livro do Antigo Testamento, escrito por Moisés, em que se descreve a criação e os primeiros tempos do mundo.}{gê.ne.se}{0}
\verb{genesíaco}{}{}{}{}{adj.}{Genético.  }{ge.ne.sí.a.co}{0}
\verb{genésico}{}{}{}{}{adj.}{Genético.  }{ge.né.si.co}{0}
\verb{gênesis}{}{}{}{}{s.f.}{Gênese.  }{gê.ne.sis}{0}
\verb{genética}{}{}{}{}{s.f.}{Ciência que estuda a hereditariedade dos organismos por meio do conhecimento de suas unidades cromossômicas.  }{ge.né.ti.ca}{0}
\verb{geneticista}{}{}{}{}{s.2g.}{Indivíduo especializado em genética.  }{ge.ne.ti.cis.ta}{0}
\verb{genético}{}{}{}{}{adj.}{Relativo a gênese. }{ge.né.ti.co}{0}
\verb{genético}{}{}{}{}{}{Relativo a geração; genesíaco, genésico. }{ge.né.ti.co}{0}
\verb{genético}{}{}{}{}{}{Relativo a genética.   }{ge.né.ti.co}{0}
\verb{gengibre}{}{Bot.}{}{}{s.m.}{Erva cujo caule se aproveita no preparo de remédios e bebidas.}{gen.gi.bre}{0}
\verb{gengiva}{}{Anat.}{}{}{s.f.}{Tecido fibromuscular onde estão implantados os dentes.}{gen.gi.va}{0}
\verb{gengival}{}{}{"-ais}{}{adj.2g.}{Relativo a gengiva.}{gen.gi.val}{0}
\verb{gengivite}{}{Med.}{}{}{s.f.}{Inflamação da gengiva.}{gen.gi.vi.te}{0}
\verb{genial}{}{}{"-ais}{}{adj.2g.}{Que revela uma extraordinária inteligência.}{ge.ni.al}{0}
\verb{genial}{}{}{"-ais}{}{}{Formidável, ótimo, legal.}{ge.ni.al}{0}
\verb{genialidade}{}{}{}{}{s.f.}{Qualidade de genial.}{ge.ni.a.li.da.de}{0}
\verb{gênio}{}{}{}{}{s.m.}{Maneira de ser; índole, comportamento, caráter.}{gê.nio}{0}
\verb{gênio}{}{}{}{}{}{Indivíduo de inteligência fora do comum, de grande criatividade.}{gê.nio}{0}
\verb{genioso}{ô}{}{"-osos ⟨ó⟩}{"-osa ⟨ó⟩}{adj.}{De comportamento teimoso; irritável.}{ge.ni.o.so}{0}
\verb{genital}{}{}{"-ais}{}{adj.2g.}{Relativo a reprodução; que gera.}{ge.ni.tal}{0}
\verb{genitália}{}{}{}{}{s.f.}{O conjunto dos órgãos reprodutores, principalmente os externos.}{ge.ni.tá.lia}{0}
\verb{genitivo}{}{Gram.}{}{}{s.m.}{Um dos casos sintáticos, morfologicamente marcados, de algumas línguas, como o latim.}{ge.ni.ti.vo}{0}
\verb{genitor}{ô}{}{}{}{s.m.}{Indivíduo que gera ou gerou um descendente; pai, mãe.}{ge.ni.tor}{0}
%\verb{}{}{}{}{}{}{}{}{0}
\verb{geniturinário}{}{}{}{}{adj.}{Relativo a órgãos genitais e urinários.}{ge.ni.tu.ri.ná.rio}{0}
\verb{genocídio}{}{}{}{}{s.m.}{Extermínio de grupos humanos por motivos raciais, religiosos, políticos etc.}{ge.no.cí.dio}{0}
\verb{genoma}{}{Biol.}{}{}{s.m.}{Conjunto do material genético de uma espécie de ser vivo.}{ge.no.ma}{0}
\verb{genótipo}{}{}{}{}{s.m.}{Composição hereditária de um indivíduo, animal ou vegetal.}{ge.nó.ti.po}{0}
\verb{genovês}{}{}{}{}{adj.}{Relativo a Gênova, cidade da Itália.}{ge.no.vês}{0}
\verb{genovês}{}{}{}{}{}{Indivíduo natural ou habitante dessa cidade.}{ge.no.vês}{0}
\verb{genro}{}{}{}{}{s.m.}{Marido da filha em relação aos pais dela.}{gen.ro}{0}
\verb{gentalha}{}{Pop.}{}{}{s.f.}{Grupo de pessoas de condição humilde e pobre; ralé, plebe, populacho.}{gen.ta.lha}{0}
\verb{gente}{}{}{}{}{s.f.}{Multidão de pessoas; povo. (\textit{O parque de diversões estava cheio de gente.})}{gen.te}{0}
\verb{gente}{}{}{}{}{}{Conjunto de habitantes de algum lugar.}{gen.te}{0}
\verb{gente}{}{}{}{}{}{O gênero humano; a humanidade.}{gen.te}{0}
\verb{gente}{}{}{}{}{}{Na linguagem coloquial, \textit{a} \textit{gente} equivale a \textit{nós}.}{gen.te}{0}
\verb{gentil}{}{}{"-is}{}{adj.2g.}{Que revela bondade e carinho no trato com as pessoas; cortês, delicado, amável.}{gen.til}{0}
\verb{gentileza}{ê}{}{}{}{s.f.}{Qualidade de gentil; cortesia, delicadeza, amabilidade.}{gen.ti.le.za}{0}
\verb{gentileza}{ê}{}{}{}{}{Pequeno serviço que se presta por amizade.}{gen.ti.le.za}{0}
\verb{gentil"-homem}{}{}{gentis"-homens}{}{s.m.}{Homem nobre de nascimento; distinto, fidalgo.}{gen.til"-ho.mem}{0}
\verb{gentílico}{}{}{}{}{adj.}{Relativo aos gentios.}{gen.tí.li.co}{0}
\verb{gentílico}{}{Gram.}{}{}{}{Diz"-se da palavra que indica a nacionalidade.}{gen.tí.li.co}{0}
\verb{gentilidade}{}{}{}{}{s.f.}{O conjunto dos indivíduos que não foram batizados; paganismo.}{gen.ti.li.da.de}{0}
\verb{gentinha}{}{}{}{}{s.f.}{Grupo de pessoas de condição inferior; gentalha, ralé, plebe.}{gen.ti.nha}{0}
\verb{gentio}{}{}{}{}{adj.}{Que professa o paganismo; idólatra, pagão.}{gen.ti.o}{0}
\verb{gentio}{}{}{}{}{}{Que não é civilizado.}{gen.ti.o}{0}
\verb{genuflectir}{}{}{}{}{v.i.}{Dobrar a perna na altura do joelho; ajoelhar"-se.}{ge.nu.flec.tir}{\verboinum{29}}
\verb{genufletir}{}{}{}{}{}{Var. de \textit{genuflectir}.}{ge.nu.fle.tir}{0}
\verb{genuflexão}{cs}{}{"-ões}{}{s.f.}{Ação de dobrar o joelho ou ajoelhar"-se.}{ge.nu.fle.xão}{0}
\verb{genuflexo}{écs}{}{}{}{adj.}{Que ajoelhou; ajoelhado.}{ge.nu.fle.xo}{0}
\verb{genuflexório}{cs}{}{}{}{s.m.}{Nas igrejas e capelas, estrado para as pessoas se ajoelharem.}{ge.nu.fle.xó.rio}{0}
\verb{genuinidade}{}{}{}{}{s.f.}{Qualidade de ser genuíno; autenticidade, legitimidade.}{ge.nu.i.ni.da.de}{0}
\verb{genuíno}{}{}{}{}{adj.}{Que é exatamente o que se pensa ser; verdadeiro, autêntico, legítimo. (\textit{Este é um genuíno quadro de Portinari.})}{ge.nu.í.no}{0}
\verb{geocêntrico}{}{}{}{}{adj.}{Relativo ao centro da Terra, tomado como ponto de comparação.}{ge.o.cên.tri.co}{0}
\verb{geocentrismo}{}{Astron.}{}{}{s.m.}{Sistema em que se considerava a Terra como o centro do sistema solar e do Universo.}{ge.o.cen.tris.mo}{0}
\verb{geodesia}{}{}{}{}{s.f.}{Ramo da geofísica que se ocupa da forma e das dimensões da Terra ou de uma parte de sua superfície. }{ge.o.de.si.a}{0}
\verb{geodesia}{}{}{}{}{}{Arte de medir e dividir as terras.    }{ge.o.de.si.a}{0}
\verb{geodésia}{}{}{}{}{}{Var. de \textit{geodesia}.  }{ge.o.dé.sia}{0}
\verb{geodésico}{}{}{}{}{adj.}{Relativo a geodesia.}{ge.o.dé.si.co}{0}
\verb{geofagia}{}{}{}{}{s.f.}{Hábito ou mania de comer terra.}{ge.o.fa.gi.a}{0}
\verb{geófago}{}{}{}{}{adj.}{Que tem o hábito de comer terra.}{ge.ó.fa.go}{0}
\verb{geofísica}{}{Geol.}{}{}{s.f.}{Parte da geologia que estuda os fenômenos físicos e a estrutura do globo terrestre.}{ge.o.fí.si.ca}{0}
\verb{geofísico}{}{}{}{}{adj.}{Relativo a geofísica.}{ge.o.fí.si.co}{0}
\verb{geofísico}{}{}{}{}{s.m.}{Indivíduo especialista em geofísica.}{ge.o.fí.si.co}{0}
\verb{geografia}{}{}{}{}{s.f.}{Ciência que estuda as formas da superfície da Terra e a relação entre ela e seus habitantes.}{ge.o.gra.fi.a}{0}
\verb{geografia}{}{}{}{}{}{Conjunto das características geográficas de determinada região.}{ge.o.gra.fi.a}{0}
\verb{geográfico}{}{}{}{}{adj.}{Relativo a geografia.}{ge.o.grá.fi.co}{0}
\verb{geográfico}{}{}{}{}{}{Relativo a uma região particular.}{ge.o.grá.fi.co}{0}
\verb{geógrafo}{}{}{}{}{s.m.}{Indivíduo especialista em geografia.}{ge.ó.gra.fo}{0}
\verb{geoide}{}{}{}{}{s.m.}{Forma verdadeira da Terra, que não é esférica e sim achatada nos polos.}{ge.oi.de}{0}
\verb{geoide}{}{}{}{}{}{Corpo que tem forma semelhante à da Terra.}{ge.oi.de}{0}
\verb{geologia}{}{}{}{}{s.f.}{Ciência que estuda a formação, a origem e a estrutura da Terra. }{ge.o.lo.gi.a}{0}
\verb{geológico}{}{}{}{}{adj.}{Relativo a geologia.}{ge.o.ló.gi.co}{0}
\verb{geólogo}{}{}{}{}{s.m.}{Indivíduo especialista em geologia.}{ge.ó.lo.go}{0}
\verb{geômetra}{}{}{}{}{s.m.}{Indivíduo especialista em geometria.}{ge.ô.me.tra}{0}
\verb{geometria}{}{}{}{}{s.f.}{Ciência que estuda a forma e a medida das figuras e dos sólidos que se constroem num espaço.}{ge.o.me.tri.a}{0}
\verb{geométrico}{}{}{}{}{adj.}{Relativo a geometria.}{ge.o.mé.tri.co}{0}
\verb{geometrizar}{}{}{}{}{v.t.}{Dar forma geométrica; representar geometricamente.}{ge.o.me.tri.zar}{\verboinum{1}}
\verb{geopolítica}{}{Geogr.}{}{}{s.f.}{Estudo da influência do meio físico de uma nação sobre sua vida política.}{ge.o.po.lí.ti.ca}{0}
\verb{georgiano}{}{}{}{}{adj.}{Relativo a Geórgia.}{ge.or.gi.a.no}{0}
\verb{georgiano}{}{}{}{}{s.m.}{Indivíduo natural ou habitante desse país.}{ge.or.gi.a.no}{0}
\verb{geotermia}{}{Geogr.}{}{}{s.f.}{Parte da geografia que estuda o calor no interior da Terra e seus efeitos.}{ge.o.ter.mi.a}{0}
\verb{geotropismo}{}{}{}{}{s.m.}{Crescimento das raízes e dos caules dos vegetais orientado pela gravidade da Terra.}{ge.o.tro.pis.mo}{0}
\verb{geração}{}{}{"-ões}{}{s.f.}{Ato ou efeito de gerar; produção.}{ge.ra.ção}{0}
\verb{geração}{}{}{"-ões}{}{}{Sucessão de descendentes em linha reta.}{ge.ra.ção}{0}
\verb{geração}{}{}{"-ões}{}{}{Conjunto de pessoas que têm aproximadamente a mesma idade ou que vivem na mesma época.}{ge.ra.ção}{0}
\verb{gerador}{ô}{}{}{}{adj.}{Que gera; produtor.}{ge.ra.dor}{0}
\verb{gerador}{ô}{}{}{}{s.m.}{Máquina que transforma energia mecânica em elétrica; dínamo.}{ge.ra.dor}{0}
\verb{geral}{}{}{"-ais}{}{adj.2g.}{Que se refere a totalidade; universal.}{ge.ral}{0}
\verb{geral}{}{}{"-ais}{}{}{Pertencente a quase todos; mais comum.}{ge.ral}{0}
\verb{geral}{}{}{"-ais}{}{}{Que abrange uma grande quantidade de coisas diferentes; genérico.}{ge.ral}{0}
\verb{geral}{}{}{"-ais}{}{s.m.}{O comum, o normal.}{ge.ral}{0}
\verb{geral}{}{}{"-ais}{}{}{Local em estádios ou teatros no qual o preço do ingresso é mais baixo.}{ge.ral}{0}
\verb{gerânio}{}{Bot.}{}{}{s.m.}{Planta ornamental com folhas aromáticas e flores vermelhas ou brancas.}{ge.râ.nio}{0}
\verb{gerar}{}{}{}{}{v.t.}{Fazer um filho ou filhote passar a existir; procriar, reproduzir"-se.}{ge.rar}{0}
\verb{gerar}{}{}{}{}{}{Fazer aparecer; causar, produzir.}{ge.rar}{\verboinum{1}}
\verb{gerativo}{}{}{}{}{adj.}{Que gera, que causa; produtivo.}{ge.ra.ti.vo}{0}
\verb{geratriz}{}{Mat.}{}{}{s.f.}{Fração ordinária que gera uma dízima periódica.}{ge.ra.triz}{0}
\verb{gerbão}{}{Bot.}{"-ões}{}{s.m.}{Nome comum a várias ervas cultivadas como forragem para o rebanho.}{ger.bão}{0}
\verb{gerência}{}{}{}{}{s.f.}{Ato ou efeito de gerir; administração.}{ge.rên.cia}{0}
\verb{gerência}{}{}{}{}{}{Função ou exercício de gerente. (\textit{Compete à gerência decidir quais são as novas regras do contrato.})}{ge.rên.cia}{0}
\verb{gerência}{}{}{}{}{}{Lugar em que o gerente trabalha.}{ge.rên.cia}{0}
\verb{gerenciador}{ô}{Informát.}{}{}{s.m.}{Utilitário que organiza as funções de um computador.}{ge.ren.ci.a.dor}{0}
\verb{gerenciamento}{}{}{}{}{s.m.}{Ato ou efeito de gerenciar; gerência.}{ge.ren.ci.a.men.to}{0}
\verb{gerenciar}{}{}{}{}{v.t.}{Dirigir uma empresa na condição de gerente; administrar, gerir.}{ge.ren.ci.ar}{0}
\verb{gerenciar}{}{}{}{}{}{Organizar automaticamente um conjunto de operações.}{ge.ren.ci.ar}{\verboinum{1}}
\verb{gerente}{}{}{}{}{s.2g.}{Indivíduo que administra uma empresa ou um departamento dela.}{ge.ren.te}{0}
\verb{gergelim}{}{Bot.}{"-ins}{}{s.m.}{Planta medicinal com odor desagradável, muito cultivada pelas sementes. }{ger.ge.lim}{0}
\verb{gergelim}{}{}{"-ins}{}{}{Semente dessa planta usada em pães, bolos, doces, salgados, e da qual se extrai certo óleo utilizado na alimentação e na indústria de comésticos.}{ger.ge.lim}{0}
\verb{geriatra}{}{}{}{}{s.2g.}{Médico especialista no tratamento de pessoas idosas.}{ge.ri.a.tra}{0}
\verb{geriatria}{}{Med.}{}{}{s.f.}{Parte da medicina que se ocupa das doenças e das condições de vida das pessoas idosas.}{ge.ri.a.tri.a}{0}
\verb{geringonça}{}{}{}{}{s.f.}{Coisa feita sem cuidado e com estrutura frágil; engenhoca.}{ge.rin.gon.ça}{0}
\verb{gerir}{}{}{}{}{v.t.}{Dirigir algum negócio; administrar, gerenciar, governar.}{ge.rir}{\verboinum{29}}
\verb{germânico}{}{}{}{}{adj.}{Relativo à Germânia ou à Alemanha.}{ger.mâ.ni.co}{0}
\verb{germânico}{}{}{}{}{s.m.}{Conjunto de línguas faladas pelos povos germânicos, das quais se originaram o inglês, o holandês, o alemão, o escandinavo etc.}{ger.mâ.ni.co}{0}
\verb{germânio}{}{Quím.}{}{}{s.m.}{Elemento químico sólido, acinzentado, usado na fabricação de semicondutores. \elemento{32}{72.61}{Ge}.}{ger.mâ.nio}{0}
\verb{germanismo}{}{}{}{}{s.m.}{Palavra, expressão ou construção própria da língua alemã.}{ger.ma.nis.mo}{0}
\verb{germanismo}{}{}{}{}{}{Amor excessivo, predileção a tudo o que procede da Alemanha.}{ger.ma.nis.mo}{0}
\verb{germanista}{}{}{}{}{s.2g.}{Especialista no estudo da língua, da literatura e da cultura alemãs.}{ger.ma.nis.ta}{0}
\verb{germanização}{}{}{"-ões}{}{s.f.}{Ato ou efeito de germanizar.}{ger.ma.ni.za.ção}{0}
\verb{germanizar}{}{}{}{}{v.t.}{Tornar germânico; adaptar à maneira de ser, ao estilo dos alemães.}{ger.ma.ni.zar}{\verboinum{1}}
\verb{germano}{}{}{}{}{adj.}{Que procede do mesmo pai e da mesma mãe; irmão.}{ger.ma.no}{0}
\verb{germano}{}{}{}{}{adj.}{Diz"-se do povo que habitava a região denominada Germânia pelos romanos antigos.}{ger.ma.no}{0}
\verb{germanófilo}{}{}{}{}{adj.}{Que tem grande admiração pela Alemanha e por seus habitantes.}{ger.ma.nó.fi.lo}{0}
\verb{germe}{é}{Biol.}{}{}{s.m.}{Estágio inicial de desenvolvimento de um organismo.}{ger.me}{0}
\verb{germe}{é}{Biol.}{}{}{}{Microrganismo que causa doenças; micróbio.}{ger.me}{0}
\verb{germe}{é}{Bot.}{}{}{}{Parte da semente que reproduz o vegetal.}{ger.me}{0}
\verb{germe}{é}{}{}{}{}{Ponto de partida de um fato; causa, origem.}{ger.me}{0}
\verb{gérmen}{}{}{}{}{}{Var. de \textit{germe}.}{gér.men}{0}
\verb{germicida}{}{}{}{}{adj.}{Diz"-se da substância que mata os germes nocivos à saúde.}{ger.mi.ci.da}{0}
\verb{germinação}{}{}{"-ões}{}{s.f.}{Ato ou efeito de germinar.}{ger.mi.na.ção}{0}
\verb{germinação}{}{Bot.}{"-ões}{}{}{Desenvolvimento do germe contido numa semente.}{ger.mi.na.ção}{0}
\verb{germinação}{}{Fig.}{"-ões}{}{}{Expansão lenta; desenvolvimento, evolução.}{ger.mi.na.ção}{0}
\verb{germinal}{}{}{"-ais}{}{adj.2g.}{Relativo a germe.}{ger.mi.nal}{0}
\verb{germinante}{}{}{}{}{adj.2g.}{Que germina ou faz germinar.}{ger.mi.nan.te}{0}
\verb{germinar}{}{}{}{}{v.i.}{Começar a se desenvolver; brotar, nascer.}{ger.mi.nar}{0}
\verb{germinar}{}{}{}{}{}{Evoluir, desenvolver"-se, difundir"-se. }{ger.mi.nar}{0}
\verb{germinar}{}{}{}{}{}{Fazer acontecer; gerar, produzir.}{ger.mi.nar}{\verboinum{1}}
\verb{gerontocracia}{}{}{}{}{s.f.}{Predomínio de pessoas idosas em um governo ou um grupo social.}{ge.ron.to.cra.ci.a}{0}
\verb{gerontologia}{}{}{}{}{s.f.}{Estudo dos fenômenos fisiológicos, psicológicos e sociais relacionados ao processo de envelhecimento do ser humano. }{ge.ron.to.lo.gi.a}{0}
\verb{gerontológico}{}{}{}{}{adj.}{Relativo a gerontologia.}{ge.ron.to.ló.gi.co}{0}
\verb{gerontólogo}{}{}{}{}{s.m.}{Indivíduo especialista no estudo do processo de envelhecimento do ser humano.}{ge.ron.tó.lo.go}{0}
\verb{gerúndio}{}{Gram.}{}{}{s.m.}{Forma nominal de um verbo, invariável, que possui, geralmente, valor adverbial expressando uma circunstância.}{ge.rún.dio}{0}
\verb{gervão}{}{}{}{}{}{Var. de \textit{gerbão}.}{ger.vão}{0}
\verb{gessar}{}{}{}{}{v.t.}{Cobrir uma parte fraturada do corpo com gesso; engessar. (\textit{O enfermeiro gessou o braço do menino.})}{ges.sar}{\verboinum{1}}
\verb{gesso}{ê}{}{}{}{s.m.}{Sulfato de cálcio geralmente incolor ou branco; gipsita.}{ges.so}{0}
\verb{gesso}{ê}{}{}{}{}{Em ortopedia, objeto moldado usado para atar fraturas. }{ges.so}{0}
\verb{gesta}{é}{}{}{}{s.f.}{Ato heroico; façanha, proeza, feito.}{ges.ta}{0}
\verb{gesta}{é}{}{}{}{}{Canção que celebra e narra esses feitos.}{ges.ta}{0}
\verb{gestação}{}{}{"-ões}{}{s.f.}{Tempo, entre os mamíferos, em que se desenvolve o embrião no útero; gravidez. (\textit{A gestação do feto de um elefante dura aproximadamente dois anos.})}{ges.ta.ção}{0}
\verb{gestação}{}{Fig.}{"-ões}{}{}{Tempo de elaboração de um trabalho, de uma ideia.}{ges.ta.ção}{0}
\verb{gestante}{}{}{}{}{s.f.}{Entre os mamíferos, fêmea que tem o filho no ventre; grávida, prenhe. (\textit{Esse medicamento não deve ser usado por gestantes.})}{ges.tan.te}{0}
\verb{gestão}{}{}{"-ões}{}{s.f.}{Ato ou efeito de gerir; administração, direção, gerência. (\textit{A gestão desse governador tem sido polêmica por causa dos gastos excessivos com publicidade.})}{ges.tão}{0}
\verb{gestar}{}{}{}{}{v.t.}{Formar e sustentar um embrião em seu próprio organismo; gerar, conceber.}{ges.tar}{\verboinum{1}}
\verb{gestatório}{}{}{}{}{adj.}{Relativo a gestação.}{ges.ta.tó.rio}{0}
\verb{gesticulação}{}{}{"-ões}{}{s.f.}{Ato ou efeito de gesticular; gesto.}{ges.ti.cu.la.ção}{0}
\verb{gesticular}{}{}{}{}{v.i.}{Acompanhar a fala com gestos.}{ges.ti.cu.lar}{0}
\verb{gesticular}{}{}{}{}{}{Expressar"-se por mímica ou sinais.}{ges.ti.cu.lar}{\verboinum{1}}
\verb{gesto}{é}{}{}{}{s.m.}{Movimento do corpo, principalmente da cabeça, dos braços ou das mãos, para indicar alguma ideia ou sentimento.}{ges.to}{0}
\verb{gesto}{é}{}{}{}{}{Aceno, mímica, sinal.}{ges.to}{0}
\verb{gestor}{ô}{}{}{}{adj.}{Que gerencia negócios de outrem; administrador, gerente.}{ges.tor}{0}
\verb{giárdia}{}{Biol.}{}{}{s.f.}{Protozoário que vive no intestino humano e provoca, em geral, diarreias.}{gi.ár.dia}{0}
\verb{giba}{}{}{}{}{s.f.}{Saliência convexa nas costas ou no dorso de homem ou animal; corcunda, corcova, bossa.}{gi.ba}{0}
\verb{gibão}{}{}{"-ões}{}{s.m.}{Casaco ou colete de couro usado pelos vaqueiros nordestinos.}{gi.bão}{0}
\verb{gibão}{}{Zool.}{"-ões}{}{s.m.}{Nome comum dado a macacos grandes, de focinho alongado, braços grandes e sem cauda.}{gi.bão}{0}
\verb{gibi}{}{}{}{}{s.m.}{Publicação infantojuvenil de histórias em quadrinhos.}{gi.bi}{0}
\verb{gibi}{}{Pop.}{}{}{}{Menino magro; moleque.}{gi.bi}{0}
\verb{gibiteca}{é}{}{}{}{s.f.}{Seção ou coleção de gibis numa biblioteca.}{gi.bi.te.ca}{0}
\verb{gibiteca}{é}{}{}{}{}{Local onde ficam expostos esses gibis.}{gi.bi.te.ca}{0}
\verb{gibosidade}{}{}{}{}{s.f.}{Corcunda, corcova, giba.}{gi.bo.si.da.de}{0}
\verb{giboso}{ô}{}{"-osos ⟨ó⟩}{"-osa ⟨ó⟩}{adj.}{Que tem giba; corcunda.}{gi.bo.so}{0}
\verb{giesta}{é}{Bot.}{}{}{s.f.}{Planta da família das leguminosas, cultivada por suas propriedades medicinais e especialmente pela tintura amarela extraída das flores.}{gi.es.ta}{0}
\verb{giga}{}{Mús.}{}{}{s.f.}{Antiga dança de andamento vivaz que integra e finaliza as suítes instrumentais.}{gi.ga}{0}
\verb{gigabyte}{}{Informát.}{}{}{s.m.}{Quantidade de 1.024 \textit{megabytes}, pouco mais de um bilhão de \textit{bytes}.}{\textit{gigabyte}}{0}
\verb{gigante}{}{}{}{}{adj.2g.}{De tamanho muito grande; enorme, descomunal.}{gi.gan.te}{0}
\verb{gigante}{}{Mit.}{}{}{}{Na mitologia grega, ser fabuloso que lutava contra os deuses.}{gi.gan.te}{0}
\verb{gigante}{}{Fig.}{}{}{}{Que se distingue por seu valor, talento, inteligência; gênio.}{gi.gan.te}{0}
\verb{gigante}{}{Fig.}{}{}{}{Eminente, prodigioso, extraordinário.}{gi.gan.te}{0}
\verb{gigantesco}{ê}{}{}{}{adj.}{Relativo a gigante.}{gi.gan.tes.co}{0}
\verb{gigantesco}{ê}{}{}{}{}{De tamanho muito grande; colossal, descomunal, enorme.}{gi.gan.tes.co}{0}
\verb{gigantismo}{}{Med.}{}{}{s.m.}{Crescimento anormal, extraordinário, de um ser vivo ou de parte dele.}{gi.gan.tis.mo}{0}
\verb{gigolô}{}{}{}{}{s.m.}{Indivíduo que vive à custa de uma prostituta ou de mulher mantida por outro homem.}{gi.go.lô}{0}
\verb{gilete}{é}{}{}{}{s.f.}{Qualquer lâmina de barbear semelhante àquela produzida pelo fabricante \textit{Gillette}.}{gi.le.te}{0}
\verb{gilvaz}{}{}{}{}{s.m.}{Cicatriz ou ferimento no rosto provocado por golpe de instrumento cortante.}{gil.vaz}{0}
\verb{gim}{}{}{}{}{s.m.}{Aguardente de alto teor alcoólico, feita de cevada, trigo e aveia.}{gim}{0}
\verb{gimnosperma}{é}{Bot.}{}{}{s.f.}{Espécie de planta em que as sementes e os óvulos se apresentam nus, como o pinheiro e o cipreste.}{gim.nos.per.ma}{0}
\verb{gim"-tônica}{}{}{gins"-tônicas}{}{s.f.}{Mistura de gim com água tônica, servida gelada e com uma rodela de limão.}{gim"-tô.ni.ca}{0}
\verb{ginasial}{}{}{"-ais}{}{adj.2g.}{Relativo a ginásio.}{gi.na.si.al}{0}
\verb{ginasiano}{}{Desus.}{}{}{s.m.}{Aluno do antigo curso ginasial.}{gi.na.si.a.no}{0}
\verb{ginásio}{}{}{}{}{s.m.}{Lugar coberto onde se praticam esportes.}{gi.ná.sio}{0}
\verb{ginásio}{}{}{}{}{}{Nome que se dava às quatro últimas séries do atual ensino fundamental.}{gi.ná.sio}{0}
\verb{ginasta}{}{Esport.}{}{}{s.2g.}{Esportista que pratica ginástica olímpica ou rítmica.}{gi.nas.ta}{0}
\verb{ginasta}{}{}{}{}{}{Artista de circo que executa exercícios de força e destreza; acrobata.}{gi.nas.ta}{0}
\verb{ginástica}{}{}{}{}{s.f.}{Técnica de exercitar o corpo para mantê"-lo saudável e flexível.}{gi.nás.ti.ca}{0}
\verb{ginástica}{}{}{}{}{}{Conjunto de exercícios corporais executados, às vezes, com o auxílio de aparelhos, com objetivos competitivos, artísticos e terapêuticos.}{gi.nás.ti.ca}{0}
\verb{ginástica}{}{Fig.}{}{}{}{Força que se faz para se conseguir algo; esforço.}{gi.nás.ti.ca}{0}
\verb{ginástico}{}{}{}{}{adj.}{Relativo a ginástica.}{gi.nás.ti.co}{0}
\verb{gincana}{}{Esport.}{}{}{s.f.}{Competição entre equipes que devem realizar tarefas difíceis, na qual a destreza e a rapidez contribuem para a classificação.}{gin.ca.na}{0}
\verb{gineceu}{}{Bot.}{}{}{s.m.}{Conjunto dos órgãos reprodutores femininos de uma flor.}{gi.ne.ceu}{0}
\verb{ginecologia}{}{Med.}{}{}{s.f.}{Parte da medicina que trata da constituição e das doenças peculiares às mulheres.}{gi.ne.co.lo.gi.a}{0}
\verb{ginecológico}{}{}{}{}{adj.}{Relativo a ginecologia.}{gi.ne.co.ló.gi.co}{0}
\verb{ginecologista}{}{}{}{}{adj.2g.}{Médico especialista em ginecologia.}{gi.ne.co.lo.gis.ta}{0}
\verb{ginete}{ê}{}{}{}{s.m.}{Cavalo de boa raça, adestrado para montaria.}{gi.ne.te}{0}
\verb{ginete}{ê}{Por ext.}{}{}{}{Cavaleiro bom e experiente em equitação.}{gi.ne.te}{0}
\verb{ginetear}{}{}{}{}{v.i.}{Cavalgar bem, com elegância.}{gi.ne.te.ar}{\verboinum{4}}
\verb{ginga}{}{}{}{}{s.f.}{Movimento bamboleante que se faz ao andar ou ao dançar; requebro, gingado.}{gin.ga}{0}
\verb{ginga}{}{}{}{}{}{Movimento da capoeira, em que o lutador move o corpo para enganar o adversário.}{gin.ga}{0}
\verb{gingado}{}{}{}{}{s.m.}{Meneio de corpo; requebro, ginga. (\textit{O dançarino apresentava um bom gingado ao sambar.})}{gin.ga.do}{0}
\verb{gingar}{}{}{}{}{v.i.}{Curvar o corpo de um lado para outro enquanto anda ou dança; bambolear.}{gin.gar}{\verboinum{5}}
\verb{ginja}{}{}{}{}{s.f.}{Fruto da ginjeira semelhante à cereja, de cor vermelha ou amarelada, sabor agridoce e muito utilizado em doces e bebidas.}{gin.ja}{0}
\verb{ginja}{}{}{}{}{}{A bebida feita com esse fruto.}{gin.ja}{0}
\verb{ginja}{}{}{}{}{s.2g.}{Indivíduo apegado a costumes antigos.}{gin.ja}{0}
\verb{ginjeira}{ê}{Bot.}{}{}{s.f.}{Árvore de copa arredondada, flores brancas e frutos comestíveis semelhantes à cereja.}{gin.jei.ra}{0}
\verb{gípseo}{}{}{}{}{adj.}{Feito de gesso.}{gíp.seo}{0}
\verb{gipsita}{}{}{}{}{s.f.}{Sulfato de cálcio hidratado; gesso.}{gip.si.ta}{0}
\verb{gir}{}{}{}{}{s.m.}{Raça de gado zebu.}{gir}{0}
\verb{gira}{}{}{}{}{s.f.}{Ato ou efeito de girar.}{gi.ra}{0}
\verb{gira}{}{}{}{}{adj.}{Diz"-se de pessoa maluca, adoidada.}{gi.ra}{0}
\verb{girafa}{}{Zool.}{}{}{s.f.}{Mamífero herbívoro de pescoço longo e corpo amarelo com manchas castanhas, podendo atingir mais de cinco metros de altura.}{gi.ra.fa}{0}
\verb{girafa}{}{Pop.}{}{}{}{Indivíduo muito alto ou de pescoço comprido.}{gi.ra.fa}{0}
\verb{girândola}{}{}{}{}{s.f.}{Conjunto de fogos de artifício dispostos em roda ou travessão para serem lançados ao mesmo tempo.}{gi.rân.do.la}{0}
\verb{girar}{}{}{}{}{v.t.}{Fazer mover em sentido circular.}{gi.rar}{0}
\verb{girar}{}{}{}{}{v.i.}{Mover"-se em sentido circular.}{gi.rar}{0}
\verb{girar}{}{}{}{}{}{Andar de um lugar para outro; vaguear, circular.}{gi.rar}{\verboinum{1}}
\verb{girassol}{ó}{Bot.}{"-óis}{}{s.m.}{Planta ornamental de grandes flores amarelas que sempre se voltam para o Sol e de cujas sementes se extrai certo óleo comestível.}{gi.ras.sol}{0}
\verb{giratório}{}{}{}{}{adj.}{Que gira; rotatório.}{gi.ra.tó.rio}{0}
\verb{gíria}{}{}{}{}{s.f.}{Linguagem informal e característica de um grupo ou uma região.}{gí.ria}{0}
\verb{girino}{}{Zool.}{}{}{s.m.}{Larva dos anfíbios, de corpo arredondado, sem pernas e com cauda longa.}{gi.ri.no}{0}
\verb{giro}{}{}{}{}{s.m.}{Ato ou efeito de girar; rotação.}{gi.ro}{0}
\verb{giro}{}{Pop.}{}{}{}{Passeio, caminhada, volta.}{gi.ro}{0}
\verb{giroscópio}{}{Fís.}{}{}{s.m.}{Dispositivo composto de um disco giratório cuja direção permanece invariável, usado em aparelhos de navegação.}{gi.ros.có.pio}{0}
\verb{giz}{}{}{}{}{s.m.}{Bastonete de substância calcária usado para escrever em lousas. (\textit{A professora escreveu tanto que acabou o giz.})}{giz}{0}
\verb{gizar}{}{}{}{}{v.t.}{Escrever ou riscar com giz.}{gi.zar}{0}
\verb{gizar}{}{Fig.}{}{}{}{Delinear, conceber, calcular, determinar.}{gi.zar}{\verboinum{1}}
\verb{glabro}{}{}{}{}{adj.}{Sem pelos ou sem barba.}{gla.bro}{0}
\verb{glace}{}{}{}{}{s.f.}{Glacê.}{gla.ce}{0}
\verb{glacê}{}{}{}{}{s.m.}{Cobertura para doces feita à base de clara em neve, com açúcar e caldo de frutas.}{gla.cê}{0}
\verb{glaciação}{}{}{"-ões}{}{s.f.}{Transformação em gelo; congelamento.}{gla.ci.a.ção}{0}
\verb{glaciação}{}{}{"-ões}{}{}{Ação exercida pelas geleiras na superfície terrestre.}{gla.ci.a.ção}{0}
\verb{glacial}{}{}{"-ais}{}{adj.2g.}{Relativo ao gelo.}{gla.ci.al}{0}
\verb{glacial}{}{Geogr.}{"-ais}{}{}{Diz"-se de regiões ou épocas geológicas caracterizadas pela presença de vastas geleiras.}{gla.ci.al}{0}
\verb{glacial}{}{Fig.}{"-ais}{}{}{Diz"-se de pessoa ou gesto que denota frieza, insensibilidade ou indiferença.}{gla.ci.al}{0}
\verb{glaciário}{}{}{}{}{adj.}{Do gelo.}{gla.ci.á.rio}{0}
\verb{glaciário}{}{Paleo.}{}{}{}{Relativo à era glacial.}{gla.ci.á.rio}{0}
\verb{gladiador}{ô}{}{}{}{s.m.}{Na Roma Antiga, indivíduo que lutava com homens ou feras nas arenas para divertir o povo.}{gla.di.a.dor}{0}
\verb{gládio}{}{}{}{}{s.m.}{Tipo de espada.}{glá.dio}{0}
\verb{gládio}{}{Fig.}{}{}{}{Poder, força.}{glá.dio}{0}
\verb{glamour}{}{}{}{}{s.m.}{Encanto, charme.}{\textit{glamour}}{0}
\verb{glande}{}{Anat.}{}{}{s.f.}{Extremidade do pênis.}{glan.de}{0}
\verb{glande}{}{}{}{}{}{Fruto do carvalho.}{glan.de}{0}
\verb{glândula}{}{Anat.}{}{}{s.f.}{Órgão que produz hormônios e outras substâncias utilizadas pelo organismo ou excretadas.}{glân.du.la}{0}
\verb{glandular}{}{}{}{}{adj.2g.}{Relativo a glândula.}{glan.du.lar}{0}
\verb{glauco}{}{}{}{}{adj.}{De cor verde claro; esverdeado.}{glau.co}{0}
\verb{glaucoma}{}{Med.}{}{}{s.m.}{Doença da visão caracterizada pelo aumento da pressão intraocular, diminuindo a acuidade visual.}{glau.co.ma}{0}
\verb{glaucomatoso}{ô}{}{"-osos ⟨ó⟩}{"-osa ⟨ó⟩}{adj.}{Que sofre de glaucoma.}{glau.co.ma.to.so}{0}
\verb{gleba}{é}{}{}{}{s.f.}{Terreno para cultivo.}{gle.ba}{0}
\verb{gleba}{é}{}{}{}{}{Qualquer porção de terra.}{gle.ba}{0}
\verb{glicemia}{}{Med.}{}{}{s.f.}{Teor de glicose no sangue.}{gli.ce.mi.a}{0}
\verb{glicerina}{}{}{}{}{s.f.}{Substância líquida incolor usada na fabricação de óleos.}{gli.ce.ri.na}{0}
\verb{glicídio}{}{Quím.}{}{}{s.m.}{Designação geral de açúcares, carboidratos e substâncias análogas.}{gli.cí.dio}{0}
\verb{glicínia}{}{Bot.}{}{}{s.f.}{Planta com flores azuis e vagens, cultivada como ornamental.}{gli.cí.nia}{0}
\verb{glicose}{ó}{}{}{}{s.f.}{Substância orgânica que constitui a principal fonte de energia para os seres vivos.}{gli.co.se}{0}
\verb{global}{}{}{"-ais}{}{adj.2g.}{Relativo ao globo terrestre.}{glo.bal}{0}
\verb{global}{}{}{"-ais}{}{}{Integral, completo.}{glo.bal}{0}
\verb{globalização}{}{}{"-ões}{}{s.f.}{Ato ou efeito de globalizar.}{glo.ba.li.za.ção}{0}
\verb{globalização}{}{}{"-ões}{}{}{Processo pelo qual se eliminam as fronteiras comerciais, financeiras e culturais das nações do mundo.}{glo.ba.li.za.ção}{0}
\verb{globalizar}{}{}{}{}{v.t.}{Tornar global; universalizar, integralizar.}{glo.ba.li.zar}{\verboinum{1}}
\verb{globo}{ô}{}{}{}{s.m.}{Qualquer coisa de formato esférico.}{glo.bo}{0}
\verb{globo}{ô}{}{}{}{}{O planeta Terra.}{glo.bo}{0}
\verb{globular}{}{}{}{}{adj.2g.}{Que tem forma de globo.}{glo.bu.lar}{0}
\verb{globulina}{}{}{}{}{s.f.}{Proteína globular com função de transporte ou de anticorpo.}{glo.bu.li.na}{0}
\verb{glóbulo}{}{}{}{}{s.m.}{Pequeno corpo esférico.}{gló.bu.lo}{0}
\verb{glóbulo}{}{Anat.}{}{}{}{Célula ou elemento figurado do sangue.}{gló.bu.lo}{0}
\verb{glória}{}{}{}{}{s.f.}{Celebridade, fama.}{gló.ria}{0}
\verb{glória}{}{Relig.}{}{}{}{Bem"-aventurança do céu.}{gló.ria}{0}
\verb{gloriar}{}{}{}{}{v.t.}{Cobrir de glória; glorificar.}{glo.ri.ar}{\verboinum{6}}
\verb{glorificação}{}{}{"-ões}{}{s.f.}{Ato ou efeito de glorificar.}{glo.ri.fi.ca.ção}{0}
\verb{glorificação}{}{}{"-ões}{}{}{Elevação à bem"-aventurança; canonização.}{glo.ri.fi.ca.ção}{0}
\verb{glorificante}{}{}{}{}{adj.2g.}{Que glorifica.}{glo.ri.fi.can.te}{0}
\verb{glorificar}{}{}{}{}{v.t.}{Dar glória; honrar.}{glo.ri.fi.car}{0}
\verb{glorificar}{}{}{}{}{}{Canonizar.}{glo.ri.fi.car}{\verboinum{2}}
\verb{gloríola}{ó}{}{}{}{s.f.}{Glória falsa, vã.}{glo.rí.o.la}{0}
\verb{glorioso}{ô}{}{"-osos ⟨ó⟩}{"-osa ⟨ó⟩}{adj.}{Honroso, célebre, notável, ilustre.}{glo.ri.o.so}{0}
\verb{glosa}{ó}{}{}{}{s.f.}{Anotação feita na margem de um texto.}{glo.sa}{0}
\verb{glosa}{ó}{}{}{}{}{Composição poética que desenvolve um mote.}{glo.sa}{0}
\verb{glosa}{ó}{}{}{}{}{Recusa parcial ou total de orçamento, conta ou verba.}{glo.sa}{0}
\verb{glosador}{ô}{}{}{}{adj.}{Que glosa.}{glo.sa.dor}{0}
\verb{glosador}{ô}{}{}{}{s.m.}{Intérprete, comentarista, hermeneuta.}{glo.sa.dor}{0}
\verb{glosar}{}{}{}{}{v.t.}{Fazer glosas.}{glo.sar}{0}
\verb{glosar}{}{}{}{}{}{Interpretar, comentar, explicar, criticar.}{glo.sar}{0}
\verb{glosar}{}{}{}{}{}{Desenvolver um mote em versos.}{glo.sar}{\verboinum{1}}
\verb{glossário}{}{}{}{}{s.m.}{Conjunto de palavras explicadas; vocabulário.}{glos.sá.rio}{0}
\verb{glote}{ó}{Anat.}{}{}{s.f.}{Abertura na parte superior da laringe.}{glo.te}{0}
\verb{glótico}{}{}{}{}{adj.}{Relativo a glote.}{gló.ti.co}{0}
\verb{glucínio}{}{Quím.}{}{}{s.m.}{Elemento químico de número atômico 4; berílio.}{glu.cí.nio}{0}
\verb{gluglu}{}{}{}{}{s.m.}{Voz característica do peru.}{glu.glu}{0}
\verb{glutão}{}{}{"-ões}{glutona}{s.m.}{Indivíduo que come muito e com avidez.}{glu.tão}{0}
\verb{glute}{}{}{}{}{s.m.}{Glúten.}{glu.te}{0}
\verb{glúten}{}{}{}{}{s.m.}{Substância viscosa que fica na farinha dos cereais quando se separa o amido.}{glú.ten}{0}
\verb{glúteo}{}{}{}{}{adj.}{Relativo a nádega.}{glú.teo}{0}
\verb{glúteo}{}{}{}{}{s.m.}{O músculo da nádega.}{glú.teo}{0}
\verb{glutinoso}{ô}{}{"-osos ⟨ó⟩}{"-osa ⟨ó⟩}{adj.}{Que contém glúten.}{glu.ti.no.so}{0}
\verb{glutonaria}{}{}{}{}{s.f.}{Qualidade ou ato de glutão; gula, voracidade.}{glu.to.na.ri.a}{0}
\verb{glutoneria}{}{}{}{}{s.f.}{Glutonaria.}{glu.to.ne.ri.a}{0}
\verb{gnaisse}{}{Geol.}{}{}{s.m.}{Rocha composta de feldspato, mica e quartzo, dentre outros minerais.}{gnais.se}{0}
\verb{gnomo}{}{Mit.}{}{}{s.m.}{Ser fantástico, pequeno e atarracado, com poderes mágicos, que vive no interior da Terra e tem a guarda de seus tesouros em pedras e metais preciosos. }{gno.mo}{0}
\verb{gnose}{ó}{Relig.}{}{}{s.f.}{Conhecimento esotérico da verdade espiritual, que diversas seitas dos primeiros séculos da era cristã acreditavam ser essencial à salvação da alma; saber por excelência.}{gno.se}{0}
\verb{gnosiologia}{}{Filos.}{}{}{s.f.}{Teoria do conhecimento; epistemologia.}{gno.si.o.lo.gi.a}{0}
\verb{gnosticismo}{}{Relig.}{}{}{s.m.}{Movimento religioso desenvolvido nos primeiros séculos da era cristã, combinando misticismo e especulação filosófica.}{gnos.ti.cis.mo}{0}
\verb{gnóstico}{}{}{}{}{adj.}{Relativo a gnose ou a gnosticismo.}{gnós.ti.co}{0}
\verb{gnóstico}{}{}{}{}{s.m.}{Adepto do gnosticismo.}{gnós.ti.co}{0}
\verb{gnu}{}{Zool.}{}{}{s.m.}{Mamífero ruminante africano, de cabeça grande, cauda comprida e chifres curvados, cuja carne tenra é muito apreciada.}{gnu}{0}
\verb{GO}{}{}{}{}{}{Sigla do estado de Goiás.}{GO}{0}
\verb{godê}{}{}{}{}{s.m.}{Corte de tecido em viés, que se aplica em saias, mangas etc.; prega.}{go.dê}{0}
\verb{godo}{ô}{Hist.}{}{}{s.m.}{Indivíduo dos godos, povos germânicos que invadiram os impérios romanos do Ocidente e do Oriente, entre os séculos \textsc{iii} e \textsc{v}.}{go.do}{0}
\verb{goela}{é}{}{}{}{s.f.}{Parte do pescoço por onde passam o ar e a comida; garganta.}{go.e.la}{0}
\verb{gofrar}{}{}{}{}{v.t.}{Marcar por pressão, sem tinta, ouro ou outro material, ornatos e letras, nas lombadas e pastas de pano, couro etc.}{go.frar}{\verboinum{1}}
\verb{gogo}{ô}{}{}{}{s.m.}{Baba espessa que sai da boca de aves, especialmente galinhas.}{go.go}{0}
\verb{gogo}{ô}{Veter.}{}{}{}{Doença que a produz.}{go.go}{0}
\verb{gogó}{}{}{}{}{s.m.}{Parte saliente na frente do pescoço do homem.}{go.gó}{0}
\verb{gói}{}{}{}{}{s.2g.}{Entre os judeus, indivíduo ou povo que não é de origem judaica.}{gói}{0}
\verb{goiaba}{}{}{}{}{s.f.}{Fruto de forma arredondada, com uma camada estreita de polpa branca ou vermelha que envolve as sementes.}{goi.a.ba}{0}
\verb{goiabada}{}{Cul.}{}{}{s.f.}{Doce de goiada em pasta ou com consistência de corte.}{goi.a.ba.da}{0}
\verb{goiabal}{}{}{"-ais}{}{s.m.}{Coletivo de goiabeira.}{goi.a.bal}{0}
\verb{goiabeira}{ê}{Bot.}{}{}{s.f.}{Árvore de pequeno porte, de flores brancas, originária da América Tropical e cujo fruto comestível é a goiaba.}{goi.a.bei.ra}{0}
\verb{goiamum}{}{}{}{}{}{Var. de \textit{guaiamu}.}{goi.a.mum}{0}
\verb{goianiense}{}{}{}{}{adj.2g.}{Relativo a Goiânia, capital de Goiás.}{goi.a.ni.en.se}{0}
\verb{goianiense}{}{}{}{}{s.2g.}{Indivíduo natural ou habitante dessa cidade.}{goi.a.ni.en.se}{0}
\verb{goiano}{}{}{}{}{adj.}{Relativo a Goiás.}{goi.a.no}{0}
\verb{goiano}{}{}{}{}{s.m.}{Indivíduo natural ou habitante desse estado.}{goi.a.no}{0}
\verb{goitacá}{}{}{}{}{s.m.}{Indivíduo dos goitacás, tribo indígena que dominava o litoral brasileiro desde o Espírito Santo até o rio Paraíba do Sul.}{goi.ta.cá}{0}
\verb{goitacá}{}{}{}{}{adj.}{Relativo a essa tribo.}{goi.ta.cá}{0}
\verb{goiva}{}{}{}{}{s.f.}{Ferramenta que abre na madeira uma canaleta de fundo arredondado.}{goi.va}{0}
\verb{goivo}{}{Bot.}{}{}{s.m.}{Planta ornamental, de flores aromáticas vermelhas, rajadas de branco, ou amarelas.}{goi.vo}{0}
\verb{gol}{ô}{}{gols}{}{s.m.}{Conjunto de traves do campo de alguns esportes: arco, baliza. (\textit{O campo de futebol tem dois gols.})}{gol}{0}
\verb{gol}{ô}{}{gols}{}{}{Ponto que se faz quando a bola passa por essas traves. (\textit{Os jogadores de futebol sempre querem fazer gol.})}{gol}{0}
\verb{gola}{ó}{}{}{}{s.f.}{Parte da roupa em volta do pescoço.}{go.la}{0}
\verb{gole}{ó}{}{}{}{s.m.}{Porção de líquido que se bebe de uma vez. (\textit{Matei a sede com um só gole de água.})}{go.le}{0}
\verb{goleada}{}{}{}{}{s.f.}{Vitória com muitos gols. (\textit{Aquele time ganhou de goleada.})}{go.le.a.da}{0}
\verb{goleador}{ô}{}{}{}{s.m.}{Jogador que faz muitos gols.}{go.le.a.dor}{0}
\verb{golear}{}{}{}{}{v.t.}{Vencer o adversário por uma grande diferença de gols. (\textit{Aquele time goleou todos os adversários.})}{go.le.ar}{\verboinum{4}}
\verb{goleiro}{ê}{Esport.}{}{}{s.m.}{Jogador que defende o gol.}{go.lei.ro}{0}
\verb{golfada}{}{}{}{}{s.f.}{Quantidade de líquido que sai de repente; jato.}{gol.fa.da}{0}
\verb{golfar}{}{}{}{}{v.t.}{Expelir em golfadas; vomitar, jorrar.}{gol.far}{\verboinum{1}}
\verb{golfe}{ô}{Esport.}{}{}{s.m.}{Jogo em que uma bola pequena deve entrar numa porção de buracos que estão num campo muito grande.}{gol.fe}{0}
\verb{golfinho}{}{Bot.}{}{}{s.m.}{Mamífero marinho que não alcança grandes dimensões e possui o focinho alongado formando um bico. (\textit{Naquela ilha, há vários golfinhos.})}{gol.fi.nho}{0}
\verb{golfista}{}{}{}{}{s.2g.}{Jogador de golfe.}{gol.fis.ta}{0}
\verb{golfo}{ô}{}{}{}{s.m.}{Grande parte do mar que entra pela terra e tem uma abertura muito larga.}{gol.fo}{0}
\verb{golpe}{ó}{}{}{}{s.m.}{Toque violento que atinge pessoa ou coisa; pancada. (\textit{O homem deu um golpe contra a porta.})}{gol.pe}{0}
\verb{golpe}{ó}{}{}{}{}{Fato triste que acontece de repente.}{gol.pe}{0}
\verb{golpe}{ó}{}{}{}{}{Negócio desonesto que causa muito prejuízo a outra pessoa.}{gol.pe}{0}
\verb{golpear}{}{}{}{}{v.t.}{Dar uma pancada em pessoa ou coisa.}{gol.pe.ar}{\verboinum{4}}
\verb{golpista}{}{}{}{}{s.2g.}{Indivíduo que costuma dar golpes.}{gol.pis.ta}{0}
\verb{goma}{}{}{}{}{s.f.}{Matéria pegajosa e transparente que escorre de algumas plantas.}{go.ma}{0}
\verb{goma}{}{}{}{}{}{Massa feita com farinha de trigo e água para ligar uma coisa a outra; cola.}{go.ma}{0}
\verb{goma}{}{}{}{}{}{Preparado para endurecer ligeiramente a roupa que vai ser passada a ferro.}{go.ma}{0}
\verb{goma}{}{}{}{}{}{Farinha feita de grãos, preparada com água da mandioca espremida; polvilho, tapioca.}{go.ma}{0}
\verb{goma"-arábica}{}{}{}{}{s.f.}{Goma vermelha extraída de algumas espécies de acácia, empregada na indústria farmacêutica e alimentícia e na fabricação de uma cola do mesmo nome.}{go.ma"-a.rá.bi.ca}{0}
\verb{goma"-laca}{}{}{}{}{s.f.}{Resina vegetal usada para fabricar lacre, polir móveis etc.}{go.ma"-la.ca}{0}
\verb{gomar}{}{}{}{}{v.t.}{Colocar goma em uma roupa ou em parte dela e alisá"-la com o ferro; engomar.}{go.mar}{\verboinum{1}}
\verb{gomo}{}{}{}{}{s.m.}{Cada uma das partes de algumas frutas, recobertas por uma espécie de pele muito fina.}{go.mo}{0}
\verb{gomo}{}{}{}{}{}{Cada uma das partes de um vegetal, separadas por nós.}{go.mo}{0}
\verb{gomoso}{ô}{}{"-osos ⟨ó⟩}{"-osa ⟨ó⟩}{adj.}{Que produz goma; viscoso.}{go.mo.so}{0}
\verb{gomoso}{ô}{}{"-osos ⟨ó⟩}{"-osa ⟨ó⟩}{}{Disposto em gomos ou que tem forma de gomos.}{go.mo.so}{0}
\verb{gônada}{}{Anat.}{}{}{s.f.}{Glândula sexual que produz os gametas e segrega os hormônios.}{gô.na.da}{0}
\verb{gôndola}{}{}{}{}{s.f.}{Embarcação comprida, com a parte da frente e a de trás mais altas, movimentada por um remo.}{gôn.do.la}{0}
\verb{gôndola}{}{}{}{}{}{Estante ou conjunto de prateleiras em supermercados etc. }{gôn.do.la}{0}
\verb{gondoleiro}{ê}{}{}{}{s.m.}{Tripulante de gôndola.}{gon.do.lei.ro}{0}
\verb{gongo}{}{}{}{}{s.m.}{Instrumento musical de percussão formado de um disco de metal em que se bate com um bastão pequeno.}{gon.go}{0}
\verb{gongórico}{}{}{}{}{adj.}{Relativo ao gongorismo.}{gon.gó.ri.co}{0}
\verb{gongórico}{}{}{}{}{}{Rebuscado, precioso, afetado.}{gon.gó.ri.co}{0}
\verb{gongorismo}{}{}{}{}{s.m.}{Escola literária espanhola de estilo erudito, que usa muitas metáforas.}{gon.go.ris.mo}{0}
\verb{gonococo}{ô}{Biol.}{}{}{s.m.}{Bactéria causadora da gonorreia.}{go.no.co.co}{0}
\verb{gonorreia}{é}{Med.}{}{}{s.f.}{Doença sexualmente transmissível, caracterizada por fluxo purulento na uretra.}{go.nor.rei.a}{0}
\verb{gonzo}{}{}{}{}{s.m.}{Peça que tem dois anéis enfiados um no outro, cada um deles terminado por um cabo na forma de um prego, para se fixar um na parede e outro na porta ou na janela.}{gon.zo}{0}
\verb{gorar}{}{}{}{}{v.t.}{Fazer um ovo ficar estragado.}{go.rar}{0}
\verb{gorar}{}{}{}{}{}{Não deixar alguma coisa acontecer; anular, frustrar.}{go.rar}{0}
\verb{gorar}{}{}{}{}{v.i.}{Ficar sem chocar; estragar"-se.}{go.rar}{\verboinum{1}}
\verb{gordo}{ô}{}{}{}{adj.}{Que é pesado demais para o tamanho que tem.}{gor.do}{0}
\verb{gorducho}{}{}{}{}{adj.}{Diz"-se de indivíduo que é um tanto gordo. (\textit{Todos gostam de apertar as bochechas do menino gorducho.})}{gor.du.cho}{0}
\verb{gordura}{}{}{}{}{s.f.}{Matéria que se encontra no corpo e funciona como uma reserva de alimento.}{gor.du.ra}{0}
\verb{gordura}{}{}{}{}{}{Matéria pegajosa, de origem animal ou vegetal, que se derrete com facilidade.}{gor.du.ra}{0}
\verb{gorduroso}{ô}{}{"-osos ⟨ó⟩}{"-osa ⟨ó⟩}{adj.}{Relativo a gordura.}{gor.du.ro.so}{0}
\verb{gorduroso}{ô}{}{"-osos ⟨ó⟩}{"-osa ⟨ó⟩}{}{Que tem a consistência da gordura.}{gor.du.ro.so}{0}
\verb{gorgolar}{}{}{}{}{v.i.}{Entrar ou sair líquido em golfada.}{gor.go.lar}{0}
\verb{gorgolar}{}{}{}{}{v.t.}{Gorgolejar.}{gor.go.lar}{\verboinum{1}}
\verb{gorgolejar}{}{}{}{}{v.i.}{Produzir o ruído característico do gargarejo.}{gor.go.le.jar}{\verboinum{1}}
\verb{gorgomilo}{}{Pop.}{}{}{s.m.}{Entrada do esôfago e da laringe; goela, garganta.}{gor.go.mi.lo}{0}
%\verb{}{}{}{}{}{}{}{}{0}
\verb{gorgonzola}{ó}{}{}{}{s.m.}{Queijo italiano de gosto forte e picante.}{gor.gon.zo.la}{0}
\verb{gorgorão}{}{}{"-ões}{}{s.m.}{Tecido encorpado de seda ou lã.}{gor.go.rão}{0}
\verb{gorgulho}{}{}{}{}{s.m.}{Inseto que ataca os cereais; caruncho.}{gor.gu.lho}{0}
\verb{gorila}{}{Zool.}{}{}{s.m.}{O maior e o mais forte dos macacos, cujos membros apresentam proporções quase humanas.}{go.ri.la}{0}
\verb{gorja}{ó}{}{}{}{s.f.}{Parte de dentro do pescoço, por onde passam o ar e a comida; garganta, goela.}{gor.ja}{0}
\verb{gorjear}{}{}{}{}{v.i.}{Usar a gorja e o bico para soltar sons; cantar.}{gor.je.ar}{\verboinum{4}}
\verb{gorjeio}{ê}{}{}{}{s.m.}{Ato de gorjear; canto. (\textit{Gosto de ouvir o gorjeio das aves.})}{gor.jei.o}{0}
\verb{gorjeta}{ê}{}{}{}{s.f.}{Pequena gratificação em dinheiro a quem prestou algum serviço. (\textit{Deu uma gorjeta ao manobrista.})}{gor.je.ta}{0}
\verb{goro}{ô}{}{}{}{adj.}{Que se gorou; choco.}{go.ro}{0}
\verb{gororoba}{ó}{Pop.}{}{}{s.f.}{Comida.}{go.ro.ro.ba}{0}
\verb{gorro}{ô}{}{}{}{s.m.}{Cobertura feita de tecido macio e quente e que se usa na cabeça. (\textit{A criança usava um gorro colorido.})}{gor.ro}{0}
\verb{gosma}{ó}{}{}{}{s.f.}{Saliva grossa, misturada com outras matérias, que se expele pela boca.}{gos.ma}{0}
\verb{gosma}{ó}{}{}{}{}{Doença que ataca a língua das aves.}{gos.ma}{0}
\verb{gosmento}{}{}{}{}{adj.}{Que tem gosma ou consistência de gosma.}{gos.men.to}{0}
\verb{gostar}{}{}{}{}{v.t.}{Ter sentimento de carinho por pessoa ou coisa.}{gos.tar}{0}
\verb{gostar}{}{}{}{}{}{Achar bom ou belo.}{gos.tar}{0}
\verb{gostar}{}{}{}{}{}{Ter inclinação ou tendência para alguma coisa.}{gos.tar}{\verboinum{1}}
\verb{gosto}{ô}{}{}{}{s.m.}{Característica de alguma coisa que agrada ou desagrada quando se coloca na boca; sabor. (\textit{Esse doce tem um gosto bom.})}{gos.to}{0}
\verb{gosto}{ô}{}{}{}{}{Sentido pelo qual se percebem as diferenças dos alimentos que se colocam na boca; paladar.}{gos.to}{0}
\verb{gosto}{ô}{}{}{}{}{Estado de contentamento de quem gosta, agrado, prazer.}{gos.to}{0}
\verb{gosto}{ô}{}{}{}{}{Qualidade de quem aprecia ou escolhe alguma coisa.}{gos.to}{0}
\verb{gostosão}{}{Pop.}{"-ões}{}{s.m.}{Homem muito atraente, estimado pelas mulheres.}{gos.to.são}{0}
\verb{gostoso}{ô}{}{"-osos ⟨ó⟩}{"-osa ⟨ó⟩}{adj.}{Que é bom de comer ou beber; delicioso, saboroso. (\textit{A torta de morango estava muito gostosa.})}{gos.to.so}{0}
\verb{gostoso}{ô}{}{"-osos ⟨ó⟩}{"-osa ⟨ó⟩}{}{Que causa prazer; agradável. (\textit{O passeio pelo parque estava muito gostoso.})}{gos.to.so}{0}
\verb{gostosona}{}{Pop.}{}{}{s.f.}{Mulher atraente, estimada pelos homens.}{gos.to.so.na}{0}
\verb{gostosura}{}{}{}{}{s.f.}{Coisa muito gostosa. (\textit{O doce estava uma gostosura.})}{gos.to.su.ra}{0}
\verb{gota}{ô}{}{}{}{s.f.}{A menor quantidade de líquido que cai em forma arredondada ou alongada; pingo.}{go.ta}{0}
\verb{gota}{ô}{Med.}{}{}{}{Doença inflamatória das articulações, caracterizada pelo aumento de ácido úrico no sangue, podendo causar dor e dificuldade de movimentação.}{go.ta}{0}
\verb{goteira}{ê}{}{}{}{s.f.}{Buraco no telhado por onde entra água quando chove.}{go.tei.ra}{0}
\verb{goteira}{ê}{}{}{}{}{Canaleta ao longo da beira do telhado para recolher a água das chuvas; calha.}{go.tei.ra}{0}
\verb{gotejar}{}{}{}{}{v.i.}{Cair em gotas; pingar.}{go.te.jar}{\verboinum{1}}
\verb{gótico}{}{}{}{}{adj.}{Diz"-se de um estilo arquitetônico que floresceu na Europa do século \textsc{xiii} ao \textsc{xv}, e que se caracteriza sobretudo pelo uso de ogivas.}{gó.ti.co}{0}
\verb{gótico}{}{Fig.}{}{}{}{Diz"-se de gênero de prosa ficcional que envolve mistério e terror em ambientes lúgubres, como castelos arruinados com passagens secretas etc., frequentados por fantasmas e entidades sobrenaturais.  }{gó.ti.co}{0}
\verb{gotícula}{}{}{}{}{s.f.}{Gota muito pequena.}{go.tí.cu.la}{0}
\verb{goto}{ô}{Pop.}{}{}{s.m.}{Glote.}{go.to}{0}
\verb{goto}{ô}{}{}{}{}{Harpa ou lira japonesa.}{go.to}{0}
\verb{gourmet}{}{}{}{}{s.m.}{Indivíduo que conhece e aprecia bons pratos e vinhos.}{\textit{gourmet}}{0}
\verb{governador}{ô}{}{}{}{s.m.}{Indivíduo que governa, especialmente um Estado.}{go.ver.na.dor}{0}
\verb{governamental}{}{}{"-ais}{}{adj.2g.}{Relativo a governo.}{go.ver.na.men.tal}{0}
\verb{governanta}{}{}{}{}{s.f.}{Mulher encarregada de administrar uma casa alheia.}{go.ver.nan.ta}{0}
\verb{governanta}{}{}{}{}{}{Mulher contratada numa casa de família para cuidar da educação das crianças.}{go.ver.nan.ta}{0}
\verb{governante}{}{}{}{}{adj.2g.}{Diz"-se de indivíduo que governa.}{go.ver.nan.te}{0}
\verb{governar}{}{}{}{}{v.t.}{Fazer o que é necessário para o bom andamento de uma organização; administrar, dirigir.}{go.ver.nar}{0}
\verb{governar}{}{}{}{}{}{Fazer um veículo seguir um caminho; dirigir, guiar.}{go.ver.nar}{\verboinum{1}}
\verb{governismo}{}{}{}{}{s.m.}{Ideologia de governista.}{go.ver.nis.mo}{0}
\verb{governismo}{}{}{}{}{}{Exercício autoritário de poder.}{go.ver.nis.mo}{0}
\verb{governista}{}{}{}{}{adj.2g.}{Diz"-se de indivíduo que apoia o governo.}{go.ver.nis.ta}{0}
\verb{governo}{ê}{}{}{}{s.m.}{Ato de governar um país ou uma unidade administrativa.}{go.ver.no}{0}
\verb{governo}{ê}{}{}{}{}{Conjunto de pessoas que governam um país ou uma unidade administrativa.}{go.ver.no}{0}
\verb{governo}{ê}{}{}{}{}{Forma política de governar.}{go.ver.no}{0}
\verb{gozação}{}{}{"-ões}{}{s.f.}{Caçoada, zombaria, sarro.}{go.za.ção}{0}
\verb{gozado}{}{}{}{}{adj.}{Que faz rir; divertido, engraçado.}{go.za.do}{0}
\verb{gozado}{}{}{}{}{}{Que faz coisas que os outros estranham; esquisito, estranho.}{go.za.do}{0}
\verb{gozador}{ô}{}{}{}{adj.}{Diz"-se de indivíduo que faz gozação; brincalhão.}{go.za.dor}{0}
\verb{gozar}{}{}{}{}{v.t.}{Ter prazer com alguma coisa que faz bem; desfrutar.}{go.zar}{0}
\verb{gozar}{}{}{}{}{}{Achar"-se em um estado ou uma condição que faz bem; fruir, ter.}{go.zar}{0}
\verb{gozar}{}{}{}{}{}{Fazer gozação com pessoa ou coisa; debochar, zombar.}{go.zar}{\verboinum{1}}
\verb{gozo}{ô}{}{}{}{s.m.}{Ato de gozar; satisfação, prazer.}{go.zo}{0}
\verb{gozo}{ô}{}{}{}{}{Condição de poder usar alguma capacidade ou direito.}{go.zo}{0}
\verb{gozoso}{ô}{}{"-osos ⟨ó⟩}{"-osa ⟨ó⟩}{adj.}{Em que há gozo.}{go.zo.so}{0}
\verb{gozoso}{ô}{}{"-osos ⟨ó⟩}{"-osa ⟨ó⟩}{}{Que denota gozo; prazer, contente.}{go.zo.so}{0}
\verb{grã}{}{}{}{}{s.f.}{Lã tinta de escarlate.}{grã}{0}
\verb{grã}{}{Pop.}{}{}{}{O aspecto macroscópico do tecido das madeiras e do couro curtido.}{grã}{0}
\verb{grã}{}{}{}{}{adj.}{Grão.}{grã}{0}
\verb{graça}{}{}{}{}{s.f.}{Presente dado ou recebido; benefício, dádiva.}{gra.ça}{0}
\verb{graça}{}{}{}{}{}{Ato do governo, que anula ou diminui a pena de um condenado.}{gra.ça}{0}
\verb{graça}{}{}{}{}{}{Beleza de forma ou movimento.}{gra.ça}{0}
\verb{graça}{}{}{}{}{}{Coisa que causa riso.}{gra.ça}{0}
\verb{graça}{}{}{}{}{}{Nome de batismo.}{gra.ça}{0}
\verb{graças}{}{}{}{}{s.f.pl.}{Agradecimento.}{gra.ças}{0}
\verb{graças}{}{}{}{}{adv.}{Com ajuda de pessoa ou coisa. (\textit{Graças ao estudo, o aluno concluiu o curso.})}{gra.ças}{0}
\verb{gracejador}{ô}{}{}{}{s.m.}{Indivíduo que diz gracejos; caçoador.}{gra.ce.ja.dor}{0}
\verb{gracejar}{}{}{}{}{v.i.}{Dizer coisas engraçadas.}{gra.ce.jar}{0}
\verb{gracejar}{}{}{}{}{}{Dizer algo por brincadeira.}{gra.ce.jar}{\verboinum{1}}
\verb{gracejo}{ê}{}{}{}{s.m.}{Dito engraçado, espirituoso ou irônico sobre alguém.}{gra.ce.jo}{0}
\verb{grácil}{}{}{"-eis}{}{adj.2g.}{Fino, delgado.}{grá.cil}{0}
\verb{grácil}{}{}{"-eis}{}{}{Cheio de graça; encantador, gracioso.}{grá.cil}{0}
\verb{gracinha}{}{}{}{}{s.f.}{Gracejo, piada.}{gra.ci.nha}{0}
\verb{gracioso}{ô}{}{"-osos ⟨ó⟩}{"-osa ⟨ó⟩}{adj.}{Que tem graça, elegância, encanto.}{gra.ci.o.so}{0}
\verb{gracioso}{ô}{}{"-osos ⟨ó⟩}{"-osa ⟨ó⟩}{}{Que é dado sem que alguma coisa seja exigida em troca; gratuito.}{gra.ci.o.so}{0}
\verb{graçola}{ó}{}{}{}{s.f.}{Piada ou dito de mau gosto.}{gra.ço.la}{0}
\verb{grã"-cruz}{}{}{grã"-cruzes }{}{s.f.}{Cruz com que os governos condecoram militares e civis por serviços relevantes.}{grã"-cruz}{0}
\verb{gradação}{}{}{"-ões}{}{s.f.}{Ação de mudar, pouco a pouco, para mais ou para menos.}{gra.da.ção}{0}
\verb{gradativo}{}{}{}{}{adj.}{Em que há gradação, que aumenta ou diminui pouco a pouco; gradual.}{gra.da.ti.vo}{0}
\verb{grade}{}{}{}{}{s.f.}{Conjunto de varas de madeira, ferro ou outro metal, colocadas a espaços regulares e atravessadas por outras, próprio para servir de barreira ou proteção.}{gra.de}{0}
\verb{gradeado}{}{}{}{}{adj.}{Que tem grade; cercado.}{gra.de.a.do}{0}
\verb{gradear}{}{}{}{}{v.t.}{Colocar grade em alguma coisa.}{gra.de.ar}{\verboinum{4}}
\verb{gradiente}{}{}{}{}{s.m.}{Medida de inclinação de um terreno.}{gra.di.en.te}{0}
\verb{gradiente}{}{}{}{}{}{Medida da variação de determinada característica de um meio (temperatura, pressão atmosférica etc.), de um ponto para outro desse meio.}{gra.di.en.te}{0}
\verb{gradil}{}{}{"-is}{}{s.m.}{Grade baixa; cerca.}{gra.dil}{0}
\verb{grado}{}{}{}{}{s.m.}{Vontade. (\textit{Os alunos aceitaram de bom grado a proposta da professora.})}{gra.do}{0}
\verb{grado}{}{}{}{}{adj.}{Bem desenvolvido; graúdo.}{gra.do}{0}
\verb{grado}{}{Fig.}{}{}{}{Importante, ilustre.}{gra.do}{0}
\verb{graduação}{}{}{"-ões}{}{s.f.}{Disposição em graus.}{gra.du.a.ção}{0}
\verb{graduação}{}{}{"-ões}{}{}{Hierarquia.}{gra.du.a.ção}{0}
\verb{graduação}{}{}{"-ões}{}{}{Curso universitário.}{gra.du.a.ção}{0}
\verb{graduado}{}{}{}{}{adj.}{Dividido em graus.}{gra.du.a.do}{0}
\verb{graduado}{}{}{}{}{}{Diz"-se de indivíduo que tem grau universitário; diplomado.}{gra.du.a.do}{0}
\verb{graduado}{}{}{}{}{}{Que tem posto ou posição elevada.}{gra.du.a.do}{0}
\verb{gradual}{}{}{"-ais}{}{adj.2g.}{Que se faz passo a passo, sempre mais ou sempre menos.}{gra.du.al}{0}
\verb{graduando}{}{}{}{}{s.m.}{Indivíduo que está para receber um diploma de curso superior.}{gra.du.an.do}{0}
\verb{graduar}{}{}{}{}{v.t.}{Dispor em graus.}{gra.du.ar}{0}
\verb{graduar}{}{}{}{}{}{Conferir posto militar.}{gra.du.ar}{0}
\verb{graduar}{}{}{}{}{}{Dar diploma de curso superior a pessoa que terminou o curso; diplomar.}{gra.du.ar}{\verboinum{1}}
\verb{grã"-ducado}{}{}{grã"-ducados}{}{s.m.}{Grão"-ducado.}{grã"-du.ca.do}{0}
\verb{grã"-duque}{}{}{grã"-duques}{}{s.m.}{Grão"-duque.}{grã"-du.que}{0}
\verb{grafar}{}{}{}{}{v.t.}{Dar forma escrita a uma palavra; escrever. (\textit{O aluno grafou corretamente as palavras do ditado.})}{gra.far}{\verboinum{1}}
\verb{grafia}{}{}{}{}{s.f.}{Sistema de representação dos sons de uma língua por meio da escrita.}{gra.fi.a}{0}
\verb{grafia}{}{}{}{}{}{Maneira de escrever uma palavra.}{gra.fi.a}{0}
\verb{gráfica}{}{}{}{}{s.f.}{Estabelecimento onde se fazem trabalhos impressos.}{grá.fi.ca}{0}
\verb{gráfico}{}{}{}{}{adj.}{Relativo a grafia.}{grá.fi.co}{0}
\verb{gráfico}{}{}{}{}{}{Que se refere à produção de materiais impressos.}{grá.fi.co}{0}
\verb{gráfico}{}{}{}{}{s.m.}{Indivíduo que trabalha na indústria gráfica.}{grá.fi.co}{0}
\verb{gráfico}{}{}{}{}{}{Desenho que representa as quantidades de alguma coisa.}{grá.fi.co}{0}
\verb{grã"-fino}{}{}{grã"-finos}{grã"-fina}{adj.}{Diz"-se de indivíduo abastado, que vive uma vida de luxo.}{grã"-fi.no}{0}
\verb{grã"-fino}{}{}{grã"-finos}{grã"-fina}{}{De ótima qualidade e alto preço.}{grã"-fi.no}{0}
\verb{grafita}{}{}{}{}{s.f.}{Variedade de carbono preto empregado na fabricação de lápis.}{gra.fi.ta}{0}
\verb{grafitar}{}{}{}{}{v.t.}{Fazer grafite; pichar.}{gra.fi.tar}{\verboinum{1}}
\verb{grafite}{}{}{}{}{s.m.}{Bastão de carbono usado em lápis ou lapiseiras.}{gra.fi.te}{0}
\verb{grafite}{}{}{}{}{}{Lugar ou rabisco em local público.}{gra.fi.te}{0}
\verb{grafiteiro}{ê}{}{}{}{s.m.}{Indivíduo que faz grafites; pichador.}{gra.fi.tei.ro}{0}
\verb{grafologia}{}{}{}{}{s.f.}{Estudo geral da escrita e dos sistemas da escrita.}{gra.fo.lo.gi.a}{0}
\verb{grafologia}{}{}{}{}{}{Análise da personalidade de um indivíduo por meio do estudo dos traços de sua escrita.}{gra.fo.lo.gi.a}{0}
%\verb{}{}{}{}{}{}{}{}{0}
\verb{grafólogo}{}{}{}{}{s.m.}{Especialista em grafologia.}{gra.fó.lo.go}{0}
\verb{gralha}{}{Zool.}{}{}{s.f.}{Pássaro de voz forte e aguda, e de belas cores, que vive em bandos.}{gra.lha}{0}
\verb{gralha}{}{}{}{}{}{Erro na página impressa de uma publicação.}{gra.lha}{0}
\verb{gralha}{}{}{}{}{}{Indivíduo que fala demais; tagarela.}{gra.lha}{0}
\verb{gralhar}{}{}{}{}{v.i.}{Soltar gritos longos e agudos; grasnar.}{gra.lhar}{0}
\verb{gralhar}{}{}{}{}{}{Falar demais; tagarelar.}{gra.lhar}{\verboinum{1}}
\verb{grama}{}{Bot.}{}{}{s.f.}{Planta rasteira que cobre o chão.}{gra.ma}{0}
\verb{grama}{}{}{}{}{}{Lugar coberto por essa planta; gramado.}{gra.ma}{0}
\verb{grama}{}{}{}{}{s.m.}{Unidade de medida de massa equivalente à milésima parte de um quilo.}{gra.ma}{0}
\verb{gramado}{}{}{}{}{s.m.}{Terreno coberto de grama.}{gra.ma.do}{0}
\verb{gramado}{}{Por ext.}{}{}{}{Campo de futebol.}{gra.ma.do}{0}
\verb{gramar}{}{}{}{}{v.t.}{Recobrir de grama.}{gra.mar}{0}
\verb{gramar}{}{}{}{}{}{Aturar, aguentar.}{gra.mar}{\verboinum{1}}
\verb{gramática}{}{}{}{}{s.f.}{Estudo dos elementos de uma língua, tais como sons, formas, palavras, construções e recursos expressivos.}{gra.má.ti.ca}{0}
\verb{gramática}{}{}{}{}{}{Conjunto das regras e das normas para o uso de uma língua.}{gra.má.ti.ca}{0}
\verb{gramatical}{}{}{"-ais}{}{adj.2g.}{Relativo a gramática.}{gra.ma.ti.cal}{0}
\verb{gramático}{}{}{}{}{adj.}{Relativo a gramática; gramatical.}{gra.má.ti.co}{0}
\verb{gramático}{}{}{}{}{s.m.}{Especialista que se dedica a estudos gramaticais ou escreve a respeito de regras gramaticais.}{gra.má.ti.co}{0}
\verb{gramatura}{}{}{}{}{s.f.}{Peso do papel expresso em gramas, que serve como termo de comparação entre os tipos de papel.}{gra.ma.tu.ra}{0}
\verb{gramínea}{}{Bot.}{}{}{s.f.}{Planta de folhas geralmente longas e estreitas que saem do caule, como o arroz, o capim, o trigo, o milho, a cana"-de"-açúcar, o bambu etc.}{gra.mí.nea}{0}
\verb{gramíneo}{}{}{}{}{adj.}{Relativo a gramínea.}{gra.mí.neo}{0}
\verb{gramofone}{}{}{}{}{s.m.}{Antigo aparelho que reproduz o som gravado em disco; vitrola.}{gra.mo.fo.ne}{0}
\verb{grampeador}{ô}{}{}{}{s.m.}{Pequeno aparelho manual usado para grampear papéis.}{gram.pe.a.dor}{0}
\verb{grampeadora}{ô}{}{}{}{s.f.}{Máquina para grampear revistas ou folhetos.}{gram.pe.a.do.ra}{0}
\verb{grampear}{}{}{}{}{v.t.}{Prender com grampos.}{gram.pe.ar}{0}
\verb{grampear}{}{}{}{}{}{Interceptar ligações telefônicas para ouvir ou gravar conversações.}{gram.pe.ar}{\verboinum{4}}
\verb{grampo}{}{}{}{}{s.m.}{Peça fina de metal com dobras em cada ponta, usada para prender folhas de papel.}{gram.po}{0}
\verb{grampo}{}{}{}{}{}{Pequena peça de metal ou plástico usada para prender os cabelos. (\textit{A moça prendeu o coque com vários grampos.})}{gram.po}{0}
\verb{grampo}{}{}{}{}{}{Aparelho colocado numa linha telefônica para interceptar e gravar ligações.}{gram.po}{0}
\verb{grana}{}{Pop.}{}{}{s.f.}{Dinheiro.}{gra.na}{0}
\verb{granada}{}{}{}{}{s.f.}{Projétil explosivo de forma arredondada, que se atira com a mão.}{gra.na.da}{0}
\verb{granadeiro}{ê}{}{}{}{s.m.}{Soldado encarregado de lançar granadas.}{gra.na.dei.ro}{0}
\verb{granadino}{}{}{}{}{adj.}{Relativo a Granada (Espanha).}{gra.na.di.no}{0}
\verb{granadino}{}{}{}{}{s.m.}{Indivíduo natural ou habitante de Granada.}{gra.na.di.no}{0}
\verb{granar}{}{}{}{}{v.t.}{Dar aspecto ou forma de grão; granular.}{gra.nar}{\verboinum{1}}
\verb{grandalhão}{}{}{"-ões}{}{adj.}{Que é muito grande ou muito alto.}{gran.da.lhão}{0}
\verb{grandão}{}{}{"-ões}{}{adj.}{Grandalhão.}{gran.dão}{0}
\verb{grande}{}{}{}{}{}{Que ocupa muito espaço; vasto, extenso.}{gran.de}{0}
\verb{grande}{}{}{}{}{adj.2g.}{Que possui dimensões fora do padrão normal; imenso.}{gran.de}{0}
\verb{grande}{}{}{}{}{}{Notável, importante, poderoso.}{gran.de}{0}
\verb{grandeza}{ê}{}{}{}{s.f.}{Qualidade de grande; amplidão, vastidão.}{gran.de.za}{0}
\verb{grandeza}{ê}{}{}{}{}{Medida de quantidade; valor.}{gran.de.za}{0}
\verb{grandíloco}{}{}{}{}{}{Var. de \textit{grandíloquo}.}{gran.dí.lo.co}{0}
\verb{grandiloquência}{}{}{}{}{s.f.}{Modo afetado de se expressar por meio de palavras pomposas, rebuscadas.}{gran.di.lo.quên.cia}{0}
\verb{grandiloquente}{}{}{}{}{adj.2g.}{Que tem linguagem pomposa, elevada; grandíloquo.}{gran.di.lo.quen.te}{0}
\verb{grandíloquo}{}{}{}{}{adj.}{Grandiloquente.}{gran.dí.lo.quo}{0}
\verb{grandiosidade}{}{}{}{}{s.f.}{Qualidade de grandioso; magnificência, pompa.}{gran.di.o.si.da.de}{0}
\verb{grandioso}{ô}{}{"-osos ⟨ó⟩}{"-osa ⟨ó⟩}{adj.}{Muito grande; nobre, elevado.}{gran.di.o.so}{0}
\verb{grandioso}{ô}{}{"-osos ⟨ó⟩}{"-osa ⟨ó⟩}{}{Magnificente, pomposo, suntuoso.}{gran.di.o.so}{0}
\verb{granel}{é}{}{"-éis}{}{s.m.}{Local onde se guardam grãos; celeiro, depósito, tulha.}{gra.nel}{0}
\verb{granel}{é}{}{"-éis}{}{}{Usado na expressão \textit{a granel}: em grande quantidade; sem embalagem; solto.}{gra.nel}{0}
%\verb{granfino}{}{}{}{}{}{0}{gran.fi.no}{0}
\verb{granítico}{}{}{}{}{adj.}{Da natureza do granito; rígido, duro.}{gra.ní.ti.co}{0}
\verb{granito}{}{Geol.}{}{}{s.m.}{Rocha magmática mais comum, formada principalmente por quartzo e feldspato.  }{gra.ni.to}{0}
\verb{granívoro}{}{}{}{}{adj.}{Que se alimenta de grãos e sementes.}{gra.ní.vo.ro}{0}
\verb{granizo}{}{}{}{}{s.m.}{Gotas de água que se esfriam rapidamente e caem sob a forma de pedras de gelo; chuva de pedra; saraiva.}{gra.ni.zo}{0}
\verb{granja}{}{}{}{}{s.f.}{Pequena propriedade rural em que se criam aves e gado leiteiro e se fazem plantações.}{gran.ja}{0}
\verb{granjear}{}{}{}{}{v.t.}{Obter com trabalho ou esforço próprio; adquirir, conquistar.}{gran.je.ar}{0}
\verb{granjear}{}{}{}{}{}{Cultivar, amanhar.}{gran.je.ar}{\verboinum{4}}
\verb{granjeio}{ê}{}{}{}{s.m.}{Ato ou feito de granjear; lavoura, cultura.}{gran.jei.o}{0}
\verb{granjeio}{ê}{}{}{}{}{Ganho, proveito, lucro.}{gran.jei.o}{0}
\verb{granjeiro}{ê}{}{}{}{s.m.}{Proprietário ou trabalhador de granja.}{gran.jei.ro}{0}
\verb{granulação}{}{}{"-ões}{}{s.f.}{Ato ou efeito de granular.}{gra.nu.la.ção}{0}
\verb{granulação}{}{Med.}{"-ões}{}{}{Formação de massas carnosas na superfície de um órgão.}{gra.nu.la.ção}{0}
\verb{granulado}{}{}{}{}{adj.}{Formado de grãos ou que apresenta granulações. (\textit{O bolo foi coberto com creme e chocolate granulado.})}{gra.nu.la.do}{0}
\verb{granular}{}{}{}{}{v.t.}{Dar a forma de grãos.}{gra.nu.lar}{0}
\verb{granular}{}{}{}{}{}{Reduzir a pequenos grãos.}{gra.nu.lar}{\verboinum{1}}
\verb{granular}{}{}{}{}{adj.2g.}{Formado de grãos.}{gra.nu.lar}{0}
\verb{grânulo}{}{}{}{}{s.m.}{Pequeno grão; glóbulo. }{grâ.nu.lo}{0}
\verb{granulosidade}{}{}{}{}{adj.}{Qualidade de granuloso.}{gra.nu.lo.si.da.de}{0}
\verb{granuloso}{ô}{}{"-osos ⟨ó⟩}{"-osa ⟨ó⟩}{}{Formado de grânulos.}{gra.nu.lo.so}{0}
\verb{granuloso}{ô}{}{"-osos ⟨ó⟩}{"-osa ⟨ó⟩}{adj.}{Que tem a superfície áspera.}{gra.nu.lo.so}{0}
\verb{grão}{}{}{grãos}{}{s.m.}{Semente de cereais e de outras plantas.}{grão}{0}
\verb{grão}{}{}{grãos}{}{}{Partícula dura de uma substância.}{grão}{0}
\verb{grão}{}{}{grãos}{}{adj.}{Forma apocopada de \textit{grande}.}{grão}{0}
\verb{grão"-de"-bico}{}{Bot.}{grãos"-de"-bico}{}{s.m.}{Planta leguminosa de frutos comestíveis e folhas com propriedades medicinais.}{grão"-de"-bi.co}{0}
\verb{grão"-de"-bico}{}{}{grãos"-de"-bico}{}{}{O fruto dessa planta.}{grão"-de"-bi.co}{0}
\verb{grão"-ducado}{}{}{grão"-ducados}{}{s.m.}{Nação governada por um grão"-duque.}{grão"-du.ca.do}{0}
\verb{grão"-duque}{}{}{grão"-duques}{grã"-duquesa ⟨ê⟩}{s.m.}{Título de alguns príncipes soberanos.}{grão"-du.que}{0}
\verb{grão"-mestre}{é}{}{grão"-mestres ⟨é⟩}{}{s.m.}{A autoridade máxima da maçonaria.}{grão"-mes.tre}{0}
\verb{grão"-vizir}{}{}{grão"-vizires}{}{s.m.}{O primeiro"-ministro do antigo Império Otomano. }{grão"-vi.zir}{0}
\verb{grasnada}{}{}{}{}{s.f.}{Som rouco, produzido por aves como águias, gralhas, corvos; grasnido.}{gras.na.da}{0}
\verb{grasnar}{}{}{}{}{v.i.}{Soltar grasnadas; crocitar.}{gras.nar}{\verboinum{1}}
\verb{grasnido}{}{}{}{}{s.m.}{Grasnada.}{gras.ni.do}{0}
\verb{grasnir}{}{}{}{}{v.i.}{Grasnar.}{gras.nir}{\verboinum{18}}
\verb{grasno}{}{}{}{}{s.m.}{Grasnada.}{gras.no}{0}
\verb{grassar}{}{}{}{}{v.i.}{Multiplicar"-se, difundir"-se, propagar"-se.}{gras.sar}{\verboinum{1}}
\verb{gratidão}{}{}{"-ões}{}{s.f.}{Qualidade de grato; agradecimento, reconhecimento.}{gra.ti.dão}{0}
\verb{gratificação}{}{}{"-ões}{}{s.f.}{Ato de gratificar, recompensar; bonificação.}{gra.ti.fi.ca.ção}{0}
\verb{gratificação}{}{}{"-ões}{}{}{Gorjeta.}{gra.ti.fi.ca.ção}{0}
\verb{gratificante}{}{}{}{}{adj.2g.}{Que gratifica; compensador.}{gra.ti.fi.can.te}{0}
\verb{gratificar}{}{}{}{}{v.t.}{Conceder benefício ou bonificação; recompensar, premiar. }{gra.ti.fi.car}{0}
\verb{gratificar}{}{}{}{}{}{Dar gorjeta.}{gra.ti.fi.car}{\verboinum{2}}
\verb{gratinado}{}{Cul.}{}{}{adj.}{Que apresenta uma camada de queijo ralado ou farinha de rosca tostada ao forno.}{gra.ti.na.do}{0}
\verb{gratinar}{}{Cul.}{}{}{v.t.}{Cobrir de queijo ralado ou farinha de rosca e levar ao forno para tostar.}{gra.ti.nar}{\verboinum{1}}
\verb{grátis}{}{}{}{}{adj.2g.}{Sem pagamento; gratuito.}{grá.tis}{0}
\verb{grátis}{}{}{}{}{adv.}{De graça; gratuitamente.}{grá.tis}{0}
\verb{grato}{}{}{}{}{adj.}{Que revela sentimento de gratidão; agradecido, reconhecido.}{gra.to}{0}
\verb{grato}{}{}{}{}{}{Que causa prazer; agradável, bom.}{gra.to}{0}
\verb{gratuidade}{}{}{}{}{s.f.}{Qualidade daquilo que é gratuito.}{gra.tu.i.da.de}{0}
\verb{gratuito}{}{}{}{}{adj.}{Que não precisa pagar; de graça, grátis. (\textit{A rede de cinemas promoveu um dia de sessões gratuitas para alunos de escolas públicas.})}{gra.tui.to}{0}
\verb{gratuito}{}{}{}{}{}{Sem fundamento; sem razão.}{gra.tui.to}{0}
\verb{gratulação}{}{}{"-ões}{}{s.f.}{Ato ou efeito de gratular; felicitação.}{gra.tu.la.ção}{0}
\verb{gratular}{}{}{}{}{v.t.}{Dar parabéns; felicitar, congratular.}{gra.tu.lar}{\verboinum{1}}
\verb{gratulatório}{}{}{}{}{adj.}{Em que se manifesta gratidão, felicitação.}{gra.tu.la.tó.rio}{0}
\verb{grau}{}{}{}{}{s.m.}{Título conferido por uma escola superior ao se completar o curso.}{grau}{0}
\verb{grau}{}{}{}{}{}{Antiga divisão do currículo no ensino brasileiro.}{grau}{0}
\verb{grau}{}{}{}{}{}{Unidade de medida de temperatura.}{grau}{0}
\verb{grau}{}{}{}{}{}{Cada uma das divisões da escala de alguns instrumentos.}{grau}{0}
\verb{grau}{}{}{}{}{}{Gradação, nível.}{grau}{0}
\verb{grau}{}{Gram.}{}{}{}{Categoria que indica tamanho ou intensidade nos nomes.}{grau}{0}
\verb{graúdo}{}{}{}{}{adj.}{Muito crescido; desenvolvido, grande.}{gra.ú.do}{0}
\verb{graúdo}{}{}{}{}{}{Importante, poderoso, rico.}{gra.ú.do}{0}
\verb{graúna}{}{Zool.}{}{}{s.f.}{Pássaro de plumagem negra e de canto melodioso.}{gra.ú.na}{0}
\verb{gravação}{}{}{"-ões}{}{s.f.}{Ato ou efeito de gravar; inscrição.}{gra.va.ção}{0}
\verb{gravação}{}{}{"-ões}{}{}{Registro de ondas sonoras em disco, fita, \textsc{cd}"-\textsc{rom}.}{gra.va.ção}{0}
\verb{gravação}{}{}{"-ões}{}{}{O disco, a fita, o \textsc{cd}"-\textsc{rom} em que se gravou algo para ser reproduzido em aparelho adequado.}{gra.va.ção}{0}
\verb{gravador}{ô}{}{}{}{s.m.}{Aparelho que grava sons.}{gra.va.dor}{0}
\verb{gravador}{ô}{}{}{}{}{Indivíduo que faz gravuras.}{gra.va.dor}{0}
\verb{gravadora}{ô}{}{}{}{s.f.}{Empresa que grava e comercializa material fonográfico.}{gra.va.do.ra}{0}
\verb{gravame}{}{}{}{}{s.m.}{Imposto oneroso.}{gra.va.me}{0}
\verb{gravame}{}{}{}{}{}{Agravo, ofensa.}{gra.va.me}{0}
\verb{gravar}{}{}{}{}{v.t.}{Fazer gravura, escultura ou estampa.}{gra.var}{0}
\verb{gravar}{}{}{}{}{}{Registrar o som.}{gra.var}{0}
\verb{gravar}{}{}{}{}{}{Fixar na memória.}{gra.var}{\verboinum{1}}
\verb{gravata}{}{}{}{}{s.f.}{Adorno constituído de uma tira de pano colocada em volta do pescoço e que cai sobre o peito.}{gra.va.ta}{0}
\verb{gravatá}{}{Bot.}{}{}{s.m.}{Nome de várias plantas ornamentais; caraguatá.  }{gra.va.tá}{0}
\verb{gravata}{}{}{}{}{}{Golpe de estrangulamento aplicado no pescoço.}{gra.va.ta}{0}
\verb{gravateiro}{ê}{}{}{}{s.m.}{Indivíduo que fabrica ou vende gravatas.}{gra.va.tei.ro}{0}
\verb{grave}{}{}{}{}{adj.2g.}{Intenso, sério, profundo.}{gra.ve}{0}
\verb{grave}{}{}{}{}{}{Diz"-se de som de baixa frequência, grosso.}{gra.ve}{0}
\verb{grave}{}{}{}{}{}{Compenetrado de responsabilidade.}{gra.ve}{0}
\verb{graveto}{ê}{}{}{}{s.m.}{Galho seco e fino, geralmente usado como combustível.}{gra.ve.to}{0}
\verb{grávida}{}{}{}{}{adj.}{Diz"-se de mulher em gestação.}{grá.vi.da}{0}
\verb{gravidade}{}{}{}{}{s.f.}{Qualidade de grave.}{gra.vi.da.de}{0}
\verb{gravidade}{}{Fís.}{}{}{}{Força de atração que a Terra exerce sobre os corpos.}{gra.vi.da.de}{0}
\verb{gravidez}{ê}{}{}{}{s.f.}{Estado da fêmea durante a gestação.}{gra.vi.dez}{0}
\verb{grávido}{}{}{}{}{adj.}{Que está em processo de gestação.}{grá.vi.do}{0}
\verb{grávido}{}{}{}{}{}{Cheio, carregado, repleto.}{grá.vi.do}{0}
\verb{graviola}{ó}{Bot.}{}{}{s.f.}{Árvore com flores amareladas e fruto verde"-escuro comestível.}{gra.vi.o.la}{0}
\verb{gravitação}{}{}{"-ões}{}{s.f.}{Ato de gravitar.}{gra.vi.ta.ção}{0}
\verb{gravitação}{}{Fís.}{"-ões}{}{}{Atração mútua que os corpos exercem uns sobre os outros.}{gra.vi.ta.ção}{0}
\verb{gravitar}{}{}{}{}{v.i.}{Mover"-se sob efeito da gravitação.}{gra.vi.tar}{0}
\verb{gravitar}{}{Fig.}{}{}{}{Ter fortes relações. (\textit{Aqueles deputados gravitam em torno do vice"-presidente.})}{gra.vi.tar}{\verboinum{1}}
\verb{gravoso}{ô}{}{"-osos ⟨ó⟩}{"-osa ⟨ó⟩}{adj.}{Que oprime; pesado, vexatório.}{gra.vo.so}{0}
\verb{gravura}{}{}{}{}{s.f.}{Técnica ou processo de traçar figuras sobre material rígido.}{gra.vu.ra}{0}
\verb{gravura}{}{}{}{}{}{A ilustração obtida por esse processo.}{gra.vu.ra}{0}
\verb{graxa}{ch}{}{}{}{s.f.}{Pasta para lubrificar mecanismos.}{gra.xa}{0}
\verb{graxa}{ch}{}{}{}{}{Pasta para polir e conservar calçados e outras peças de couro.}{gra.xa}{0}
\verb{graxo}{ch}{}{}{}{adj.}{Que tem gordura; oleoso, gorduroso.}{gra.xo}{0}
\verb{greco"-latino}{ê}{}{greco"-latinos ⟨ê⟩}{greco"-latina ⟨ê⟩}{adj.}{Relativo às culturas grega e latina.}{gre.co"-la.ti.no}{0}
\verb{greco"-romano}{ê}{}{greco"-romanos ⟨ê⟩}{greco"-romana}{adj.}{Comum aos gregos e aos romanos.}{gre.co"-ro.ma.no}{0}
\verb{greda}{ê}{}{}{}{s.f.}{Argila clara e macia.}{gre.da}{0}
\verb{grega}{ê}{}{}{}{s.f.}{Tipo de ornato geométrico constituído de frisos com linhas quebradas em ângulo reto.}{gre.ga}{0}
\verb{gregário}{}{}{}{}{adj.}{Diz"-se de animal que faz parte de rebanho ou que vive em bando.}{gre.gá.rio}{0}
\verb{gregário}{}{Fig.}{}{}{}{Diz"-se de pessoa que gosta de estar na companhia de outras; sociável.}{gre.gá.rio}{0}
\verb{grego}{ê}{}{}{}{adj.}{Relativo à Grécia.}{gre.go}{0}
\verb{grego}{ê}{Fig.}{}{}{}{Obscuro, ininteligível.}{gre.go}{0}
\verb{grego}{ê}{}{}{}{s.m.}{Indivíduo natural ou habitante desse país.}{gre.go}{0}
\verb{grego}{ê}{}{}{}{}{A língua desse país.}{gre.go}{0}
\verb{gregoriano}{}{Relig.}{}{}{adj.}{Relativo aos papas Gregório \textsc{i} ou Gregório \textsc{xiii}.}{gre.go.ri.a.no}{0}
\verb{grei}{}{}{}{}{s.f.}{Rebanho de gado miúdo.}{grei}{0}
\verb{grei}{}{}{}{}{}{Partido, grupo, facção, sociedade.}{grei}{0}
\verb{grelar}{}{}{}{}{v.i.}{Formar grelo; brotar, germinar.}{gre.lar}{\verboinum{1}}
\verb{grelha}{ê}{}{}{}{s.f.}{Grade de metal para assar carne ou peixe.}{gre.lha}{0}
\verb{grelhado}{}{}{}{}{adj.}{Assado ou tostado na grelha.}{gre.lha.do}{0}
\verb{grelhar}{}{}{}{}{v.t.}{Assar ou tostar na grelha.}{gre.lhar}{\verboinum{1}}
\verb{grelo}{ê}{}{}{}{s.m.}{Broto, rebento.}{gre.lo}{0}
\verb{grêmio}{}{}{}{}{s.m.}{Associação de pessoas para defender interesses ou exercer atividades; sociedade, corporação.}{grê.mio}{0}
\verb{grená}{}{}{}{}{s.m.}{A cor avermelhada da granada (mineral).}{gre.ná}{0}
\verb{grenha}{}{}{}{}{s.f.}{Cabelo desalinhado.}{gre.nha}{0}
\verb{grés}{}{Desus.}{}{}{s.m.}{Certa rocha sedimentar; arenito.}{grés}{0}
\verb{greta}{ê}{}{}{}{s.f.}{Fenda na terra provocada pelo calor do sol.}{gre.ta}{0}
\verb{gretar}{}{}{}{}{v.t.}{Abrir greta, fenda.}{gre.tar}{\verboinum{1}}
\verb{gretar}{}{}{}{}{v.i.}{Rachar, fender.}{gre.tar}{0}
\verb{greve}{é}{}{}{}{s.f.}{Suspensão das atividades produtivas pelos trabalhadores como forma de pressão para que haja possibilidade de negociar sobre salários ou condições de trabalho.}{gre.ve}{0}
\verb{grevista}{}{}{}{}{adj.2g.}{Relativo a greve.}{gre.vis.ta}{0}
\verb{grevista}{}{}{}{}{s.2g.}{Indivíduo que participa de ou promove greve.}{gre.vis.ta}{0}
\verb{grifar}{}{}{}{}{v.t.}{Sublinhar palavra(s) em um texto.}{gri.far}{0}
\verb{grifar}{}{}{}{}{}{Destacar, salientar, frisar.}{gri.far}{\verboinum{1}}
\verb{grife}{}{}{}{}{s.f.}{Marca de fabricante de artigo de luxo.}{gri.fe}{0}
\verb{grifo}{}{Mit.}{}{}{s.m.}{Animal fabuloso, de cabeça de águia e garras de leão.}{gri.fo}{0}
\verb{grifo}{}{}{}{}{s.m.}{Letra inclinada; itálico.}{gri.fo}{0}
\verb{grifo}{}{Pop.}{}{}{}{Ferramenta para apertar e soltar roscas em encanamentos.}{gri.fo}{0}
\verb{grilado}{}{Pop.}{}{}{adj.}{Preocupado, cismado.}{gri.la.do}{0}
\verb{grilado}{}{}{}{}{}{Diz"-se de terreno tomado de forma ilícita.}{gri.la.do}{0}
\verb{grilagem}{}{}{"-ens}{}{s.f.}{Ato ou efeito de grilar terras.}{gri.la.gem}{0}
\verb{grilar}{}{Pop.}{}{}{v.t.}{Deixar cismado, preocupado.}{gri.lar}{0}
\verb{grilar}{}{}{}{}{}{Apossar"-se ilicitamente de terras falsificando escrituras.}{gri.lar}{\verboinum{1}}
\verb{grileiro}{ê}{Pop.}{}{}{s.m.}{Indivíduo que procura apossar"-se de terras alheias com escrituras falsas.}{gri.lei.ro}{0}
\verb{grilhão}{}{}{"-ões}{}{s.m.}{Corrente que prende condenados; cadeia, algema.}{gri.lhão}{0}
\verb{grilheta}{ê}{}{}{}{s.f.}{Anel de metal preso a corrente que fica atado ao tornozelo dos condenados.}{gri.lhe.ta}{0}
\verb{grilo}{}{Zool.}{}{}{s.m.}{Inseto saltador cujo macho produz ruído estridente.}{gri.lo}{0}
\verb{grilo}{}{Por ext.}{}{}{}{Ruído de atrito produzido por mecanismo mal"-ajustado.}{gri.lo}{0}
\verb{grilo}{}{Pop.}{}{}{}{Preocupação, cisma.}{gri.lo}{0}
\verb{grilo}{}{Pop.}{}{}{}{Terreno cuja escritura é falsa.}{gri.lo}{0}
\verb{grimpa}{}{}{}{}{s.f.}{Lâmina móvel do cata"-vento, a qual indica a direção do vento.}{grim.pa}{0}
\verb{grimpar}{}{}{}{}{v.i.}{Subir, trepar, escalar.}{grim.par}{\verboinum{1}}
\verb{grinalda}{}{}{}{}{s.f.}{Coroa de flores naturais ou artificiais.}{gri.nal.da}{0}
\verb{gringo}{}{Pop.}{}{}{s.m.}{Estrangeiro, especialmente o de pele bem clara e cabelo loiro ou ruivo.}{grin.go}{0}
\verb{gripado}{}{}{}{}{adj.}{Acometido por gripe.}{gri.pa.do}{0}
\verb{gripado}{}{}{}{}{}{Travado ou enguiçado por falta de lubrificação.}{gri.pa.do}{0}
\verb{gripal}{}{}{"-ais}{}{adj.2g.}{Relativo a gripe.}{gri.pal}{0}
\verb{gripar}{}{}{}{}{v.i.}{Pegar gripe.}{gri.par}{0}
\verb{gripar}{}{}{}{}{v.t.}{Provocar gripe.}{gri.par}{\verboinum{1}}
\verb{gripe}{}{}{}{}{s.f.}{Doença caracterizada por febre, tosse e congestão nasal.}{gri.pe}{0}
\verb{gris}{}{}{}{}{adj.2g.}{Acinzentado.}{gris}{0}
\verb{gris}{}{}{}{}{}{Pardo.}{gris}{0}
\verb{grisalho}{}{}{}{}{adj.}{Diz"-se de cabelos com fios pretos ou castanhos e brancos misturados.}{gri.sa.lho}{0}
\verb{grisalho}{}{}{}{}{}{Diz"-se de pessoa com cabelos grisalhos.}{gri.sa.lho}{0}
\verb{griseta}{ê}{}{}{}{s.f.}{Peça metálica onde se enfia a torcida das lâmpadas.}{gri.se.ta}{0}
\verb{grisu}{}{}{}{}{s.m.}{Gás inflamável encontrado em minas de carvão.}{gri.su}{0}
\verb{grita}{}{}{}{}{s.f.}{Gritaria.}{gri.ta}{0}
\verb{gritador}{ô}{}{}{}{adj.}{Que fala em voz muito alta ou que grita.}{gri.ta.dor}{0}
\verb{gritalhão}{}{}{"-ões}{}{s.m.}{Indivíduo que grita muito.}{gri.ta.lhão}{0}
\verb{gritante}{}{}{}{}{adj.2g.}{Que grita.}{gri.tan.te}{0}
\verb{gritante}{}{}{}{}{}{Muito evidente; que chama muito a atenção.}{gri.tan.te}{0}
\verb{gritar}{}{}{}{}{v.i.}{Falar em voz muito alta; berrar.}{gri.tar}{0}
\verb{gritar}{}{}{}{}{}{Soltar gritos.}{gri.tar}{\verboinum{1}}
\verb{gritaria}{}{}{}{}{s.f.}{Muitos gritos ocorrendo ao mesmo tempo ou em sucessão.}{gri.ta.ri.a}{0}
\verb{grito}{}{}{}{}{s.m.}{Voz muito alta e estridente.}{gri.to}{0}
\verb{grogue}{ó}{}{}{}{adj.}{Diz"-se de quem está ligeiramente tonto, especialmente sob efeito de bebida alcoólica.}{gro.gue}{0}
\verb{grosa}{ó}{}{}{}{s.f.}{Ferramenta abrasiva, como uma lima grossa, para desbastar madeira e outros materiais.}{gro.sa}{0}
\verb{grosa}{ó}{}{}{}{s.f.}{Doze dúzias.}{gro.sa}{0}
\verb{groselha}{é}{}{}{}{s.f.}{Fruto comestível da groselheira e do qual se extrai xarope.}{gro.se.lha}{0}
\verb{groselha}{é}{}{}{}{}{Refresco preparado com esse xarope.}{gro.se.lha}{0}
\verb{groselha}{é}{}{}{}{s.m.}{A cor vermelha da groselha.}{gro.se.lha}{0}
\verb{groselheira}{ê}{Bot.}{}{}{s.f.}{Certa planta que produz fruto comestível do qual se extrai xarope.}{gro.se.lhei.ra}{0}
\verb{grosseirão}{}{}{"-ões}{}{adj.}{De qualidade inferior; rude, tosco, grosseiro.}{gros.sei.rão}{0}
\verb{grosseiro}{ê}{}{}{}{adj.}{De qualidade inferior; rude, tosco.}{gros.sei.ro}{0}
\verb{grosseiro}{ê}{}{}{}{}{Mal"-educado, estúpido, áspero, rude.}{gros.sei.ro}{0}
\verb{grosseria}{}{}{}{}{s.f.}{Ato ou dito grosseiro; indelicadeza.}{gros.se.ri.a}{0}
\verb{grossista}{}{}{}{}{adj.2g.}{Relativo a comércio por atacado.}{gros.sis.ta}{0}
\verb{grossista}{}{}{}{}{s.2g.}{Negociante que atua nesse tipo de comércio.}{gros.sis.ta}{0}
\verb{grosso}{ô}{}{"-ossos ⟨ó⟩}{"-ossa ⟨ó⟩}{adj.}{De grande diâmetro.}{gros.so}{0}
\verb{grosso}{ô}{}{"-ossos ⟨ó⟩}{"-ossa ⟨ó⟩}{}{Áspero, rude, estúpido.}{gros.so}{0}
\verb{grosso}{ô}{}{"-ossos ⟨ó⟩}{"-ossa ⟨ó⟩}{}{Denso, viscoso, consistente.}{gros.so}{0}
\verb{grosso}{ô}{}{"-ossos ⟨ó⟩}{"-ossa ⟨ó⟩}{}{Diz"-se de voz grave.}{gros.so}{0}
\verb{grosso}{ô}{}{"-ossos ⟨ó⟩}{"-ossa ⟨ó⟩}{s.m.}{A maior parte; grande quantidade.}{gros.so}{0}
\verb{grossura}{}{}{}{}{s.f.}{Espessura ou diâmetro.}{gros.su.ra}{0}
\verb{grossura}{}{}{}{}{}{Grosseria.}{gros.su.ra}{0}
\verb{grossura}{}{}{}{}{}{Qualidade de grosso.}{gros.su.ra}{0}
\verb{grota}{ó}{}{}{}{}{Abertura produzida pela ação da água nas margens altas de um rio.}{gro.ta}{0}
\verb{grota}{ó}{}{}{}{s.f.}{Vale profundo.}{gro.ta}{0}
\verb{grotão}{}{}{"-ões}{}{s.m.}{Grota grande.}{gro.tão}{0}
\verb{grotesco}{ê}{}{}{}{adj.}{Que suscita riso ou escárnio; ridículo.}{gro.tes.co}{0}
\verb{grou}{ô}{Zool.}{}{}{s.m.}{Ave de pernas altas, pescoço longo e cabeça parcialmente nua.}{grou}{0}
\verb{grua}{}{}{}{}{s.f.}{Tipo de guindaste.}{gru.a}{0}
\verb{grudar}{}{}{}{}{v.t.}{Prender alguma coisa a outra com grude.}{gru.dar}{0}
\verb{grudar}{}{}{}{}{v.i.}{Prender"-se.}{gru.dar}{\verboinum{1}}
\verb{grude}{}{}{}{}{s.m.}{Tipo de cola.}{gru.de}{0}
\verb{grude}{}{Fig.}{}{}{}{Apego excessivo.}{gru.de}{0}
\verb{grude}{}{Pop.}{}{}{}{Comida, refeição.}{gru.de}{0}
\verb{grudento}{}{}{}{}{adj.}{Que tem as propriedades ou a consistência do grude.}{gru.den.to}{0}
\verb{grugrulhar}{}{Bras.}{}{}{v.i.}{Entrar em ebulição; ferver.}{gru.gru.lhar}{0}
\verb{grugrulhar}{}{}{}{}{}{Grugulejar.}{gru.gru.lhar}{\verboinum{8}}
\verb{grugulejar}{}{}{}{}{v.i.}{Soltar a voz (o peru).}{gru.gu.le.jar}{\verboinum{1}}
\verb{grumete}{ê}{}{}{}{s.m.}{Marinheiro iniciante.}{gru.me.te}{0}
\verb{grumixama}{ch}{Bras.}{}{}{s.f.}{Fruto da grumixameira.}{gru.mi.xa.ma}{0}
\verb{grumixameira}{ch}{Bot.}{}{}{s.f.}{Árvore de folhas elípticas e bagas roxas comestíveis.}{gru.mi.xa.mei.ra}{0}
\verb{grumo}{}{}{}{}{s.m.}{Aglomeração de partículas; grânulo.}{gru.mo}{0}
\verb{grumo}{}{}{}{}{}{Pequeno coágulo.}{gru.mo}{0}
\verb{grumoso}{ô}{}{"-osos ⟨ó⟩}{"-osa ⟨ó⟩}{adj.}{Que tem grumos; granuloso.}{gru.mo.so}{0}
\verb{grunhido}{}{}{}{}{s.m.}{Voz do porco ou do javali.}{gru.nhi.do}{0}
\verb{grunhir}{}{}{}{}{v.i.}{Soltar grunhidos.}{gru.nhir}{0}
\verb{grunhir}{}{}{}{}{}{Resmungar, gemer.}{gru.nhir}{\verboinum{33}}
\verb{grupamento}{}{}{}{}{s.m.}{Ato ou efeito de grupar.}{gru.pa.men.to}{0}
\verb{grupar}{}{}{}{}{v.t.}{Agrupar.}{gru.par}{\verboinum{1}}
\verb{grupelho}{ê}{}{}{}{s.m.}{Pequeno grupo ou facção sem importância.}{gru.pe.lho}{0}
\verb{grupiara}{}{Pop.}{}{}{s.f.}{Depósito de cascalho no alto de um morro.}{gru.pi.a.ra}{0}
\verb{grupo}{}{}{}{}{s.m.}{Conjunto de pessoas, animais ou coisas reunidas segundo um critério.}{gru.po}{0}
\verb{grupo}{}{}{}{}{}{Conjunto de pessoas reunidas para uma finalidade.}{gru.po}{0}
\verb{gruta}{}{}{}{}{s.f.}{Cavidade natural ou artificial encontrada na terra, na rocha ou na montanha.}{gru.ta}{0}
\verb{guabiraba}{}{Bot.}{}{}{s.f.}{Guabiroba.}{gua.bi.ra.ba}{0}
\verb{guabiroba}{ó}{Bot.}{}{}{s.f.}{Planta em forma de arbusto ou árvore, com casca e folhas medicinais e frutos comestíveis.}{gua.bi.ro.ba}{0}
\verb{guabiru}{}{Zool.}{}{}{s.m.}{Rato grande.}{gua.bi.ru}{0}
\verb{guabiru}{}{Fig.}{}{}{}{Larápio, gatuno, ladrão.}{gua.bi.ru}{0}
\verb{guache}{}{}{}{}{s.m.}{Tipo de tinta pastosa dissolvida em água para pintura artística sobre papel.}{gua.che}{0}
\verb{guacho}{}{}{}{}{s.m.}{Guache.}{gua.cho}{0}
\verb{guaiaca}{}{}{}{}{s.f.}{Cinturão largo de couro com bolsos para guardar miudezas, geralmente usado por campeiros.}{guai.a.ca}{0}
\verb{guaiaco}{}{Bot.}{}{}{s.m.}{Guáiaco.}{guai.a.co}{0}
\verb{guáiaco}{}{Bot.}{}{}{s.m.}{Árvore de madeira dura, resistente e impermeável, com diversas aplicações.}{guái.a.co}{0}
\verb{guaiacol}{ó}{}{"-óis}{}{s.m.}{Substância extraída da resina do guáiaco, com propriedades medicinais.}{guai.a.col}{0}
\verb{guaiamu}{}{Zool.}{}{}{s.m.}{Caranguejo grande que vive próximo ao mar em lugares lamacentos.}{guai.a.mu}{0}
\verb{guaiamum}{}{Bot.}{"-uns}{}{s.m.}{Guaiamu.}{guai.a.mum}{0}
\verb{guampa}{}{}{}{}{s.f.}{Chifre.}{guam.pa}{0}
\verb{guampa}{}{}{}{}{}{Tipo de vasilha feita com esse chifre.}{guam.pa}{0}
\verb{guanaco}{}{Zool.}{}{}{s.m.}{Mamífero ruminante com dorso de cor castanha e partes inferiores brancas, semelhante à lhama.}{gua.na.co}{0}
\verb{guando}{}{Bot.}{}{}{s.m.}{Arbusto com vagens de propriedades medicinais, usado também como alimento.}{guan.do}{0}
\verb{guandu}{}{Bot.}{}{}{s.m.}{Guando.}{guan.du}{0}
\verb{guano}{}{}{}{}{s.m.}{Adubo composto de excrementos e cadáveres de aves marinhas.}{gua.no}{0}
\verb{guapo}{}{}{}{}{adj.}{Ousado, corajoso, valente.}{gua.po}{0}
\verb{guará}{}{Zool.}{}{}{s.m.}{Ave com plumagem vermelha, bico recurvado e que vive geralmente em manguezais.}{gua.rá}{0}
\verb{guarajuba}{}{Zool.}{}{}{s.f.}{Peixe encontrado na costa atlântica do continente americano.}{gua.ra.ju.ba}{0}
\verb{guarajuba}{}{Zool.}{}{}{}{Arara amarela encontrada em matas do norte e nordeste do Brasil.}{gua.ra.ju.ba}{0}
\verb{guarajuba}{}{Bot.}{}{}{}{Árvore de grande porte, flores brancas e madeira nobre.}{gua.ra.ju.ba}{0}
\verb{guaraná}{}{Bot.}{}{}{s.m.}{Arbusto com flores pequenas aromáticas e sementes com várias propriedades medicinais.}{gua.ra.ná}{0}
\verb{guarânia}{}{}{}{}{s.f.}{Música de andamento lento e compasso ternário bastante comum no Paraguai.}{gua.râ.nia}{0}
\verb{guarda}{}{}{}{}{s.f.}{Ato de guardar; proteção, vigilância.}{guar.da}{0}
\verb{guarda}{}{}{}{}{s.2g.}{Indivíduo encarregado de tomar conta de coisas ou lugares; vigia.}{guar.da}{0}
\verb{guarda"-cancela}{é}{}{guarda"-cancelas ⟨é⟩}{}{s.m.}{Fiscal aduaneiro que fiscaliza o tráfego de bens em determinado ponto de uma estrada de ferro.}{guar.da"-can.ce.la}{0}
\verb{guarda"-chaves}{}{}{}{}{s.2g.}{Indivíduo responsável por manobrar as chaves nos entroncamentos ferroviários.}{guar.da"-cha.ves}{0}
\verb{guarda"-chuva}{}{}{guarda"-chuvas}{}{s.m.}{Armação de varetas cobertas por tecido impermeável e dotada de cabo, para proteger da chuva e do sol.}{guar.da"-chu.va}{0}
\verb{guarda"-civil}{}{}{guardas"-civis}{}{s.m.}{Membro de corporação policial não militarizada.}{guar.da"-ci.vil}{0}
\verb{guarda"-comida}{}{}{guarda"-comidas}{}{s.m.}{Móvel onde se guardam comidas, geralmente com porta que proporciona ventilação adequada.}{guar.da"-co.mi.da}{0}
\verb{guarda"-costas}{ó}{}{}{}{s.m.}{Indivíduo que acompanha outro para defendê"-lo de possíveis agressões.}{guar.da"-cos.tas}{0}
\verb{guardador}{ô}{}{}{}{adj.}{Que guarda, vigia.}{guar.da.dor}{0}
\verb{guarda"-florestal}{}{}{guardas"-florestais}{}{s.m.}{Indivíduo encarregado de vigiar florestas para evitar caça ilegal, derrubadas, incêndios.}{guar.da"-flo.res.tal}{0}
\verb{guarda"-fogo}{ô}{}{guarda"-fogos ⟨ó⟩}{}{s.m.}{Placa metálica usada em lareiras para evitar incêndios.}{guar.da"-fo.go}{0}
\verb{guarda"-freio}{ê}{}{guarda"-freios}{}{s.m.}{Indivíduo que manobra os freios do trem.}{guar.da"-frei.o}{0}
\verb{guarda"-joias}{}{}{}{}{s.m.}{Caixa para guardar joias e adornos miúdos; porta"-joias.}{guar.da"-joi.as}{0}
\verb{guarda"-livros}{}{}{}{}{s.m.}{Indivíduo responsável pelo registro nos livros contábeis de uma organização.}{guar.da"-li.vros}{0}
\verb{guarda"-louça}{}{}{guarda"-louças}{}{s.m.}{Prateleira ou armário com portas geralmente envidraçadas para guardar louças.}{guar.da"-lou.ça}{0}
\verb{guarda"-marinha}{}{}{guardas"-marinhas \textit{ou} guardas"-marinha}{}{}{Militar que ocupa esse posto.}{guar.da"-ma.ri.nha}{0}
\verb{guarda"-marinha}{}{}{guardas"-marinhas \textit{ou} guardas"-marinha}{}{s.m.}{Posto da Marinha imediatamente inferior ao de segundo"-tenente.}{guar.da"-ma.ri.nha}{0}
\verb{guarda"-mor}{ó}{}{guardas"-mores ⟨ó⟩}{}{s.m.}{Chefe da polícia aduaneira nos portos.}{guar.da"-mor}{0}
\verb{guardamoria}{}{}{}{}{s.f.}{Escritório ou repartição dirigida pelo guarda"-mor da alfândega.}{guar.da.mo.ri.a}{0}
\verb{guarda"-móveis}{}{}{}{}{s.m.}{Estabelecimento que recebe móveis em depósito, mediante pagamento.}{guar.da"-mó.veis}{0}
\verb{guardanapo}{}{}{}{}{s.m.}{Pequena toalha de pano ou papel que se usa para limpar os lábios ou os dedos e proteger a roupa.}{guar.da.na.po}{0}
\verb{guarda"-noturno}{}{}{guardas"-noturnos}{}{s.m.}{Profissional que guarda as casas durante a noite.}{guar.da"-no.tur.no}{0}
\verb{guarda"-pó}{}{}{guarda"-pós}{}{s.m.}{Espécie de avental, que se usa por cima da roupa para conservá"-la limpa.}{guar.da"-pó}{0}
\verb{guardar}{}{}{}{}{v.t.}{Estar perto de pessoa, animal ou coisa, pronto para sua defesa; vigiar, zelar.}{guar.dar}{0}
\verb{guardar}{}{}{}{}{}{Cuidar que pessoa ou coisa fique a salvo de algum perigo; proteger.}{guar.dar}{0}
\verb{guardar}{}{}{}{}{}{Colocar alguma coisa em lugar seguro.}{guar.dar}{0}
\verb{guardar}{}{}{}{}{}{Deixar alguma coisa em um lugar para alguma finalidade; reservar.}{guar.dar}{0}
\verb{guardar}{}{}{}{}{}{Obedecer a alguma obrigação religiosa ou social; observar, respeitar.}{guar.dar}{\verboinum{1}}
\verb{guarda"-roupa}{ô}{}{guarda"-roupas}{}{s.m.}{Armário onde se guarda a roupa.}{guar.da"-rou.pa}{0}
\verb{guarda"-roupa}{ô}{}{guarda"-roupas}{}{}{Conjunto das roupas de uma pessoa.}{guar.da"-rou.pa}{0}
\verb{guarda"-sol}{ó}{}{guarda"-sóis}{}{s.m.}{Guarda"-chuva.}{guar.da"-sol}{0}
\verb{guarda"-sol}{ó}{}{guarda"-sóis}{}{}{Grande armação de varetas coberta de pano, que se abre e fecha, com um cabo comprido para se fincar na terra, em geral usada na praia para proteger as pessoas do sol.}{guar.da"-sol}{0}
\verb{guarda"-vala}{}{Bras.}{guarda"-valas}{}{s.m.}{Goleiro.}{guar.da"-va.la}{0}
\verb{guardião}{}{}{"-ões}{}{s.m.}{Funcionário superior de alguns conventos.}{guar.di.ão}{0}
\verb{guardião}{}{Bras.}{"-ões}{}{}{Goleiro.}{guar.di.ão}{0}
\verb{guariba}{}{Zool.}{}{}{s.m.}{Macaco do norte do Brasil, de coloração escura, provido de barba.}{gua.ri.ba}{0}
\verb{guarida}{}{}{}{}{s.f.}{Lugar em que se fica protegido; abrigo, proteção.}{gua.ri.da}{0}
\verb{guarita}{}{}{}{}{s.f.}{Torre nos ângulos dos antigos baluartes onde as sentinelas se protegiam.}{gua.ri.ta}{0}
\verb{guarita}{}{}{}{}{}{Casinhola portátil, de madeira ou de outro material, que funciona como abrigo para sentinelas ou outros vigias.}{gua.ri.ta}{0}
%\verb{}{}{}{}{}{}{}{}{0}
\verb{guarnecer}{ê}{}{}{}{v.t.}{Prover do necessário; munir, abastecer.}{guar.ne.cer}{0}
\verb{guarnecer}{ê}{}{}{}{}{Colocar enfeite na parte de baixo das roupas; adornar.}{guar.ne.cer}{0}
\verb{guarnecer}{ê}{}{}{}{}{Dar maior poder de defesa a pessoa ou lugar; armar, munir; prover.}{guar.ne.cer}{\verboinum{15}}
\verb{guarnecido}{}{}{}{}{adj.}{Abastecido, munido.}{guar.ne.ci.do}{0}
\verb{guarnecimento}{}{}{}{}{s.m.}{Ato ou efeito de guarnecer; abastecimento.}{guar.ne.ci.men.to}{0}
\verb{guarnecimento}{}{}{}{}{}{Aquilo que guarnece; guarnição.}{guar.ne.ci.men.to}{0}
\verb{guarnição}{}{}{"-ões}{}{s.f.}{Coisa com que se guarnece algum lugar.}{guar.ni.ção}{0}
\verb{guarnição}{}{}{"-ões}{}{}{Tropas que defendem uma praça.}{guar.ni.ção}{0}
\verb{guarnição}{}{}{"-ões}{}{}{Enfeite colocado na parte de baixo de uma roupa.}{guar.ni.ção}{0}
\verb{guasca}{}{Bras.}{}{}{s.f.}{Tira ou correia de couro cru.}{guas.ca}{0}
\verb{guasca}{}{}{}{}{adj.2g. e s.2g.}{Gaúcho.}{guas.ca}{0}
\verb{guatemalense}{}{}{}{}{adj.2g. e s.2g.}{Guatemalteco.}{gua.te.ma.len.se}{0}
\verb{guatemalteco}{é}{}{}{}{adj.}{Relativo à Guatemala (América Central); guatemalense.}{gua.te.mal.te.co}{0}
\verb{guatemalteco}{é}{}{}{}{s.m.}{Indivíduo natural ou habitante desse país.}{gua.te.mal.te.co}{0}
\verb{guaxe}{ch}{Zool.}{}{}{s.m.}{Ave preta e vermelha, de bico amarelo.}{gua.xe}{0}
\verb{guaxima}{ch}{Bot.}{}{}{s.f.}{Planta herbácea de fibras têxteis e dotada de propriedades medicinais.}{gua.xi.ma}{0}
\verb{guaxinim}{ch}{Zool.}{"-ins}{}{s.m.}{Pequeno mamífero carnívoro que se alimenta sobretudo de caranguejos.}{gua.xi.nim}{0}
\verb{gude}{}{}{}{}{s.m.}{Jogo infantil em que se lançam bolinhas de vidro na direção de buracos cavados na terra.}{gu.de}{0}
\verb{guedelha}{ê}{}{}{}{s.f.}{Cabelo desgrenhado e longo.}{gue.de.lha}{0}
\verb{gueixa}{ch}{}{}{}{s.f.}{Japonesa treinada desde de jovem nas artes da dança, do canto e da conversação para entreter os fregueses das casas de chá, de banquetes etc.}{guei.xa}{0}
\verb{guelra}{é}{}{}{}{s.f.}{Aparelho respiratório dos animais que vivem ou podem viver na água e não respiram por pulmões; brânquia.}{guel.ra}{0}
\verb{guenzo}{}{}{}{}{adj.}{Muito magro; adoentado; fraco.}{guen.zo}{0}
\verb{guepardo}{}{Zool.}{}{}{s.m.}{Grande felino, encontrado na África e na Ásia, de pernas longas, cabeça pequena e pelagem amarelo claro com pintas pretas.}{gue.par.do}{0}
\verb{guerra}{é}{}{}{}{s.f.}{Luta armada entre nações ou partidos.}{guer.ra}{0}
\verb{guerra}{é}{}{}{}{}{Atividade em que aparecem muitas dificuldades; batalha, luta.}{guer.ra}{0}
\verb{guerrear}{}{}{}{}{v.t.}{Empregar forças militares contra outra nação.}{guer.re.ar}{0}
\verb{guerrear}{}{}{}{}{}{Fazer oposição a alguém ou a alguma coisa; hostilizar.}{guer.re.ar}{\verboinum{4}}
\verb{guerreiro}{ê}{}{}{}{adj.}{Relativo a guerra; bélico.}{guer.rei.ro}{0}
\verb{guerreiro}{ê}{}{}{}{}{Diz"-se de indivíduo que guerreia, que tem ânimo belicoso.}{guer.rei.ro}{0}
\verb{guerreiro}{ê}{}{}{}{s.m.}{Indivíduo que se empenha intensamente para conseguir o que quer; lutador, batalhador.}{guer.rei.ro}{0}
\verb{guerrilha}{}{}{}{}{s.f.}{Luta armada feita por pequenos grupos de pessoas que conhecem bem o lugar em que se escondem e atacam de surpresa.}{guer.ri.lha}{0}
\verb{guerrilhar}{}{}{}{}{v.i.}{Fazer guerrilha; viver como guerrilheiro.}{guer.ri.lhar}{\verboinum{1}}
\verb{guerrilheiro}{ê}{}{}{}{s.m.}{Indivíduo que participa de guerrilha.}{guer.ri.lhei.ro}{0}
\verb{guerrilheiro}{ê}{}{}{}{}{Relativo a guerrilha.}{guer.ri.lhei.ro}{0}
\verb{gueto}{ê}{}{}{}{s.m.}{Bairro onde os judeus eram forçados a morar em certas cidades europeias.}{gue.to}{0}
\verb{gueto}{ê}{Pop.}{}{}{}{Bairro habitado por minorias ou por grupos marginalizados pela sociedade.}{gue.to}{0}
\verb{guia}{}{}{}{}{s.f.}{Ato ou efeito de guiar.}{gui.a}{0}
\verb{guia}{}{}{}{}{}{Documento que acompanha correspondência oficial ou mercadoria para terem passagem livre.}{gui.a}{0}
\verb{guia}{}{}{}{}{}{Formulário para pagamento de impostos.}{gui.a}{0}
\verb{guia}{}{}{}{}{}{Fileira de pedras colocadas entre a calçada e a rua. (\textit{O carro encostou na guia.})}{gui.a}{0}
\verb{guia}{}{}{}{}{s.m.}{Publicação com informações turísticas e orientações sobre uma cidade, uma região ou um país; roteiro. (\textit{Precisamos olhar um guia para fazer a viagem.})}{gui.a}{0}
\verb{guia}{}{}{}{}{s.2g.}{Indivíduo que acompanha o outro para lhe indicar o caminho. (\textit{Os turistas seguem o guia.})}{gui.a}{0}
\verb{guianense}{}{}{}{}{adj.2g.}{Relativo às Guianas (Guiana Francesa, Guiana Inglesa e Guiana Holandesa).}{gu.i.a.nen.se}{0}
\verb{guianense}{}{}{}{}{adj.2g.}{Indivíduo natural ou habitante das Guianas; guianês.}{gu.i.a.nen.se}{0}
\verb{guianês}{}{}{}{}{adj. e s.m.  }{Guianense.}{gu.i.a.nês}{0}
\verb{guião}{}{}{"-ães \textit{ou} -ões}{}{s.m.}{Estandarte levado à frente de procissões, irmandades ou tropas.}{gui.ão}{0}
\verb{guiar}{}{}{}{}{v.t.}{Acompanhar uma pessoa para lhe mostrar o caminho; conduzir.}{gui.ar}{0}
\verb{guiar}{}{}{}{}{}{Indicar o caminho que alguém deve seguir; orientar.}{gui.ar}{0}
\verb{guiar}{}{}{}{}{}{Fazer animal ou veículo ir para algum lugar; conduzir, dirigir.}{gui.ar}{0}
\verb{guiar}{}{}{}{}{}{Acompanhar a vida de uma pessoa dando conselhos a ela; orientar.}{gui.ar}{\verboinum{6}}
\verb{guichê}{}{}{}{}{s.m.}{Pequena abertura em vidro ou parede pela qual um funcionário atende o público.}{gui.chê}{0}
\verb{guidão}{}{}{"-ões}{}{s.m.}{Barra de direção de bicicleta ou motocicleta.}{gui.dão}{0}
\verb{guidom}{}{}{}{}{}{Var. de \textit{guidão}.}{gui.dom}{0}
\verb{guilhotina}{}{}{}{}{s.f.}{Instrumento usado para decapitar condenados à morte.}{gui.lho.ti.na}{0}
\verb{guilhotina}{}{}{}{}{}{Aparelho para cortar papéis.}{gui.lho.ti.na}{0}
\verb{guilhotinar}{}{}{}{}{v.t.}{Decapitar, decepar a cabeça com a guilhotina.}{gui.lho.ti.nar}{\verboinum{1}}
\verb{guimba}{}{Pop.}{}{}{s.f.}{A parte final do cigarro ou do charuto, depois de fumados.}{guim.ba}{0}
\verb{guinada}{}{}{}{}{s.f.}{Golpe de direção que faz o carro ir de repente para outro lado.}{gui.na.da}{0}
\verb{guinada}{}{}{}{}{}{Mudança súbita e radical em atitude e situação.}{gui.na.da}{0}
\verb{guinar}{}{}{}{}{v.t.}{Mudar de repente a direção de alguma coisa.}{gui.nar}{0}
\verb{guinar}{}{Fig.}{}{}{}{Mudar brusca e inesperadamente de opinião, de atitude; tomar novos rumos.}{gui.nar}{\verboinum{1}}
\verb{guinchar}{}{}{}{}{v.i.}{Soltar guinchos; chiar.}{guin.char}{0}
\verb{guinchar}{}{Bras.}{}{}{v.t.}{Puxar ou arrastar um veículo com o guincho.}{guin.char}{\verboinum{1}}
\verb{guincho}{}{}{}{}{s.m.}{Grito muito agudo produzido pelo rato.}{guin.cho}{0}
\verb{guincho}{}{}{}{}{}{Aparelho para levantar cargas pesadas.}{guin.cho}{0}
\verb{guincho}{}{}{}{}{}{Carro que tem esse aparelho e serve para puxar outros veículos.}{guin.cho}{0}
\verb{guindar}{}{}{}{}{v.t.}{Fazer alguma coisa subir; alçar, erguer, levantar.}{guin.dar}{\verboinum{1}}
\verb{guindaste}{}{}{}{}{s.m.}{Aparelho apropriado para levantar grandes pesos.}{guin.das.te}{0}
\verb{guineano}{}{}{}{}{adj. e s.m.  }{Guineense.}{gui.ne.a.no}{0}
\verb{guineense}{}{}{}{}{adj.2g.}{Relativo à Guiné Bissau (costa oeste da África); guinéu, guineano.}{gui.ne.en.se}{0}
\verb{guineense}{}{}{}{}{s.2g.}{Indivíduo natural ou habitante desse país.}{gui.ne.en.se}{0}
\verb{guinéu}{}{}{}{}{adj.2g. e s.2g.}{Guineense.}{gui.néu}{0}
\verb{guirlanda}{}{}{}{}{s.f.}{Enfeite de flores, frutos ou folhagens; grinalda.}{guir.lan.da}{0}
\verb{guisa}{}{}{}{}{s.f.}{Maneira, modo.}{gui.sa}{0}
\verb{guisado}{}{Cul.}{}{}{s.m.}{Picadinho de carne.}{gui.sa.do}{0}
\verb{guisado}{}{Cul.}{}{}{}{Comida feita com uma mistura de vários temperos passados na gordura fervente.}{gui.sa.do}{0}
\verb{guisar}{}{}{}{}{v.t.}{Preparar com refogado.}{gui.sar}{\verboinum{1}}
\verb{guitarra}{}{}{}{}{s.f.}{Instrumento musical de cordas, com braço longo e caixa de ressonância de fundo chato.}{gui.tar.ra}{0}
\verb{guitarrista}{}{}{}{}{s.2g.}{Indivíduo que toca ou ensina a tocar guitarra.}{gui.tar.ris.ta}{0}
\verb{guizo}{}{}{}{}{s.m.}{Pequena esfera oca de metal, com pequenas aberturas ou furos, cheia de grãos, que produz som quando é agitada.}{gui.zo}{0}
\verb{gula}{}{}{}{}{s.f.}{Exagero quando se come ou bebe. (\textit{A gula é um dos sete pecados capitais.})}{gu.la}{0}
\verb{gulodice}{}{}{}{}{s.f.}{Gula.}{gu.lo.di.ce}{0}
\verb{gulodice}{}{}{}{}{}{Iguaria muito apetitosa; guloseima.}{gu.lo.di.ce}{0}
\verb{guloseima}{}{}{}{}{s.f.}{Doce ou iguaria qualquer, muito apetitosa; gulodice.}{gu.lo.sei.ma}{0}
\verb{guloso}{ô}{}{"-osos ⟨ó⟩}{"-osa ⟨ó⟩}{adj.}{Diz"-se de indivíduo que tem gula.}{gu.lo.so}{0}
\verb{guloso}{ô}{}{"-osos ⟨ó⟩}{"-osa ⟨ó⟩}{}{Diz"-se de indivíduo que gosta de comer gulodices.}{gu.lo.so}{0}
\verb{gume}{}{}{}{}{s.m.}{Lado afiado de instrumento cortante; corte, fio.}{gu.me}{0}
\verb{gupiara}{}{}{}{}{s.f.}{Depósito diamantífero no alto de morros.}{gu.pi.a.ra}{0}
\verb{guri}{}{}{}{}{s.m.}{Criança do sexo masculino; garoto.}{gu.ri}{0}
\verb{guria}{}{}{}{}{s.f.}{Criança do sexo feminino; menina.}{gu.ri.a}{0}
\verb{gurizada}{}{Bras.}{}{}{s.f.}{Grande número de guris; criançada.}{gu.ri.za.da}{0}
\verb{guru}{}{}{}{}{s.m.}{Na Índia, mestre espiritual ou líder de seita religiosa.}{gu.ru}{0}
\verb{guru}{}{}{}{}{}{Pessoa que aconselha outra; conselheiro, mentor.}{gu.ru}{0}
\verb{gurupés}{}{}{}{}{s.m.pl.}{Mastro que aponta para a parte da frente dos navios, colocado no bico de proa dos veleiros.}{gu.ru.pés}{0}
\verb{gusa}{}{}{}{}{s.f.}{Ferro que vem diretamente do alto"-forno, ainda com impurezas; forma reduzida de ferro"-gusa.}{gu.sa}{0}
\verb{gusano}{}{Zool.}{}{}{s.m.}{Verme que se desenvolve onde há matéria orgânica em decomposição.}{gu.sa.no}{0}
\verb{gustação}{}{}{"-ões}{}{s.f.}{Ato de provar um alimento ou uma bebida.}{gus.ta.ção}{0}
\verb{gustação}{}{}{"-ões}{}{}{Sentido pelo qual se distinguem os sabores; paladar.}{gus.ta.ção}{0}
\verb{gustativo}{}{}{}{}{adj.}{Relativo a gustação, a paladar.}{gus.ta.ti.vo}{0}
\verb{guta}{}{}{}{}{s.f.}{Substância opaca e resinosa, solúvel em água, extraída do látex das plantas gutíferas.}{gu.ta}{0}
\verb{guta"-percha}{é}{Bot.}{gutas"-perchas \textit{ou} guta"-perchas ⟨é⟩}{}{s.f.}{Planta da família das sapotáceas, cujo látex tem uso industrial.}{gu.ta"-per.cha}{0}
\verb{gutíferas}{}{Bot.}{}{}{s.f.pl.}{Família de árvores e arbustos com folhas geralmente opostas, flores unissexuais e seiva resinosa, de onde se extrai látex para uso industrial.}{gu.tí.fe.ras}{0}
\verb{gutífero}{}{}{}{}{adj.}{Relativo a guta ou a gutíferas.}{gu.tí.fe.ro}{0}
\verb{gutural}{}{}{"-ais}{}{adj.2g.}{Relativo a garganta.}{gu.tu.ral}{0}
\verb{gutural}{}{Gram.}{"-ais}{}{}{Diz"-se do som ou do fonema produzido na garganta.}{gu.tu.ral}{0}
\verb{guturalizar}{}{}{}{}{v.t.}{Pronunciar sons dando"-lhes inflexão gutural.}{gu.tu.ra.li.zar}{\verboinum{1}}
