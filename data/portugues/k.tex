\verb{k}{}{}{}{}{s.m.}{Décima primeira letra do alfabeto português.}{k}{0}
\verb{k}{}{}{}{}{}{Símb. de \textit{quilate}. }{k}{0}
\verb{K}{}{Quím.}{}{}{}{Símb. do \textit{potássio}.}{K}{0}
\verb{K}{}{Fís.}{}{}{}{Símb. de \textit{kelvin}.}{K}{0}
\verb{kafkiano}{}{}{}{}{adj.}{Relativo a Franz Kafka, escritor de língua alemã, que viveu no final do século \textsc{xix} e no início do século \textsc{xx}, ou a sua obra.}{kaf.ki.a.no}{0}
\verb{kafkiano}{}{}{}{}{s.m.}{Estudioso ou profundo conhecedor da obra de Franz Kafka.}{kaf.ki.a.no}{0}
\verb{kaingang}{}{}{}{}{adj.}{Relativo aos Kaingang.}{ka.in.gang}{0}
\verb{kaingang}{}{}{}{}{s.2g.}{Indivíduo pertencente ao povo kaingang, família linguística jê.}{ka.in.gang}{0}
\verb{kaixana}{ch}{}{}{}{adj.}{Relativo aos Kaixana.}{kai.xa.na}{0}
\verb{kaixana}{ch}{}{}{}{s.2g.}{Indivíduo pertencente ao povo kaixana.}{kai.xa.na}{0}
\verb{kalapalo}{}{}{}{}{adj.}{Relativo aos Kalapalo.}{ka.la.pa.lo}{0}
\verb{kalapalo}{}{}{}{}{s.2g.}{Indivíduo pertencente ao povo kalapalo, família linguística karib.}{ka.la.pa.lo}{0}
\verb{kamayurá}{}{}{}{}{adj.}{Relativo aos Kamayurá.}{ka.may.u.rá}{0}
\verb{kamayurá}{}{}{}{}{s.2g.}{Indivíduo pertencente ao povo kamayurá, família linguística tupi"-Guarani.}{ka.may.u.rá}{0}
\verb{kamba}{}{}{}{}{adj.}{Relativo aos Kamba.}{kam.ba}{0}
\verb{kamba}{}{}{}{}{s.2g.}{Indivíduo pertencente ao povo kamba.}{kam.ba}{0}
\verb{kambeba}{é}{}{}{}{adj.}{Relativo aos Kambeba.}{kam.be.ba}{0}
\verb{kambeba}{é}{}{}{}{s.2g.}{Indivíduo pertencente ao povo kambeba, família linguística gupi"-guarani.}{kam.be.ba}{0}
\verb{kambiwá}{}{}{}{}{adj.}{Relativo aos Kambiwá.}{kam.bi.wá}{0}
\verb{kambiwá}{}{}{}{}{s.2g.}{Indivíduo pertencente ao povo kambiwá.}{kam.bi.wá}{0}
\verb{kamikaze}{}{}{}{}{s.2g.}{Piloto da força aérea japonesa que, durante a \textsc{ii} Guerra Mundial, realizava ataques suicidas com aviões carregados de explosivos.}{\textit{kamikaze}}{0}
\verb{kamikaze}{}{Por ext.}{}{}{adj.2g.}{Que se arrisca muito ao agir, ignorando a sua segurança ou o seu bem"-estar. }{\textit{kamikaze}}{0}
\verb{kanamanti}{}{}{}{}{adj.}{Relativo aos Kanamanti.}{ka.na.man.ti}{0}
\verb{kanamanti}{}{}{}{}{s.2g.}{Indivíduo pertencente ao povo kanamanti, família linguística arawá.}{ka.na.man.ti}{0}
\verb{kanamari}{}{}{}{}{adj.}{Relativo aos Kanamari.}{ka.na.ma.ri}{0}
\verb{kanamari}{}{}{}{}{s.2g.}{Indivíduo pertencente ao povo kanamari, família linguística katukina.}{ka.na.ma.ri}{0}
\verb{kanela apaniekrá}{é}{}{}{}{adj.}{Relativo aos Kanela Apaniekrá.}{ka.ne.la a.pa.ni.e.krá}{0}
\verb{kanela apaniekrá}{é}{}{}{}{s.2g.}{Indivíduo pertencente ao povo kanela apaniekrá, família linguística jê.}{ka.ne.la a.pa.ni.e.krá}{0}
\verb{kanoê}{}{}{}{}{adj.}{Relativo aos Kanoê.}{ka.no.ê}{0}
\verb{kanoê}{}{}{}{}{s.2g.}{Indivíduo pertencente ao povo kanoê, família linguística kanoê.}{ka.no.ê}{0}
\verb{kantaruré}{}{}{}{}{adj.}{Relativo aos Kantaruré.}{kan.ta.ru.ré}{0}
\verb{kantaruré}{}{}{}{}{s.2g.}{Indivíduo pertencente ao povo kantaruré.}{kan.ta.ru.ré}{0}
\verb{kantiano}{}{}{}{}{adj.}{Relativo a Immanuel Kant, filósofo alemão que viveu no século \textsc{xviii}, ou a suas teorias.}{kan.ti.a.no}{0}
\verb{kantiano}{}{}{}{}{s.m.}{Estudioso ou admirador do kantismo.}{kan.ti.a.no}{0}
\verb{kantismo}{}{Filos.}{}{}{s.m.}{Doutrina filosófica de Immanuel Kant, pensador alemão do século \textsc{xviii}, que visa determinar os limites, o alcance e o valor da razão e do racionalismo crítico; criticismo.}{kan.tis.mo}{0}
\verb{kapinawá}{}{}{}{}{adj.}{Relativo aos Kapinawá.}{ka.pi.na.wá}{0}
\verb{kapinawá}{}{}{}{}{s.2g.}{Indivíduo pertencente ao povo kapinawá.}{ka.pi.na.wá}{0}
\verb{karajá}{}{}{}{}{adj.}{Relativo aos Karajá.}{ka.ra.já}{0}
\verb{karajá}{}{}{}{}{s.2g.}{Indivíduo pertencente ao povo karajá, família linguística karajá.}{ka.ra.já}{0}
\verb{karajá javaé}{}{}{}{}{adj.}{Relativo aos Karajá Javaé.}{ka.ra.já ja.va.é}{0}
\verb{karajá javaé}{}{}{}{}{s.2g.}{Indivíduo pertencente ao povo karajá javaé, família linguística karajá.}{ka.ra.já ja.va.é}{0}
\verb{karajá xambioá}{ch}{}{}{}{adj.}{Relativo aos Karajá Xambioá.}{ka.ra.já xam.bi.o.á}{0}
\verb{karajá xambioá}{ch}{}{}{}{s.2g.}{Indivíduo pertencente ao povo karajá xambioá, família linguística karajá.}{ka.ra.já xam.bi.o.á}{0}
\verb{karaokê}{}{}{}{}{s.m.}{Casa noturna onde qualquer cliente pode cantar ao microfone, acompanhado pelos músicos da casa ou por fundo instrumental já gravado.}{\textit{karaokê}}{0}
\verb{karaokê}{}{}{}{}{}{Dispositivo com o qual uma pessoa pode cantar ao microfone acompanhada de fundo musical.}{\textit{karaokê}}{0}
\verb{karapanã}{}{}{}{}{adj.}{Relativo aos Karapanã.}{ka.ra.pa.nã}{0}
\verb{karapanã}{}{}{}{}{s.2g.}{Indivíduo pertencente ao povo karapanã, família linguística tukano.}{ka.ra.pa.nã}{0}
\verb{karapotó}{}{}{}{}{adj.}{Relativo aos Karapotó.}{ka.ra.po.tó}{0}
\verb{karapotó}{}{}{}{}{s.2g.}{Indivíduo pertencente ao povo karapotó.}{ka.ra.po.tó}{0}
\verb{kardecismo}{}{Relig.}{}{}{s.m.}{Doutrina espírita codificada por Allan Kardec, pensador francês do século \textsc{xix}, que tem por princípio explicar a evolução do espírito por meio da reencarnação, segundo uma perspectiva cristã.}{kar.de.cis.mo}{0}
\verb{kardecista}{}{}{}{}{adj.2g.}{Relativo a Allan Kardec, pensador francês do século \textsc{xix}, ou ao kardecismo.}{kar.de.cis.ta}{0}
\verb{kardecista}{}{}{}{}{s.2g.}{Adepto ou praticante do kardecismo.}{kar.de.cis.ta}{0}
\verb{karipuna}{}{}{}{}{adj.}{Relativo aos Karipuna.}{ka.ri.pu.na}{0}
\verb{karipuna}{}{}{}{}{s.2g.}{Indivíduo pertencente ao povo karipuna, família linguística tupi"-guarani.}{ka.ri.pu.na}{0}
\verb{karipuna do Amapá}{}{}{}{}{adj.}{Relativo aos karipuna do Amapá.}{ka.ri.pu.na do A.ma.pá}{0}
\verb{karipuna do Amapá}{}{}{}{}{s.2g.}{Indivíduo pertencente ao povo karipuna do Amapá.}{ka.ri.pu.na do A.ma.pá}{0}
\verb{kariri}{}{}{}{}{adj.}{Relativo aos kariri.}{ka.ri.ri}{0}
\verb{kariri}{}{}{}{}{s.2g.}{Indivíduo pertencente ao povo kariri.}{ka.ri.ri}{0}
\verb{kariri"-xocó}{ch}{}{}{}{adj.}{Relativo aos kariri"-xocó.}{ka.ri.ri"-xo.có}{0}
\verb{kariri"-xocó}{ch}{}{}{}{s.2g.}{Indivíduo pertencente ao povo kariri"-xocó.}{ka.ri.ri"-xo.có}{0}
\verb{karitiana}{}{}{}{}{adj.}{Relativo aos Karitiana.}{ka.ri.ti.a.na}{0}
\verb{karitiana}{}{}{}{}{s.2g.}{Indivíduo pertencente ao povo karitiana, família linguística arikem.}{ka.ri.ti.a.na}{0}
\verb{kart}{}{}{}{}{s.m.}{Pequeno carro de corrida, com motor de até 125 cilindradas, sem carroceria nem marchas.}{\textit{kart}}{0}
\verb{kartódromo}{}{}{}{}{s.m.}{Pista própria para corridas de \textit{kart}. }{kar.tó.dro.mo}{0}
\verb{katuena}{ê}{}{}{}{adj.}{Relativo aos Katuena.}{ka.tu.e.na}{0}
\verb{katuena}{ê}{}{}{}{s.2g.}{Indivíduo pertencente ao povo katuena, família linguística karib.}{ka.tu.e.na}{0}
\verb{katukina}{}{}{}{}{adj.}{Relativo aos Katukina.}{ka.tu.ki.na}{0}
\verb{katukina}{}{}{}{}{s.2g.}{Indivíduo pertencente ao povo katukina, família linguística katukina.}{ka.tu.ki.na}{0}
\verb{katukina}{}{}{}{}{adj.}{Relativo aos Katukina.}{ka.tu.ki.na}{0}
\verb{katukina}{}{}{}{}{s.2g.}{Indivíduo pertencente ao povo katukina, família linguística pano.}{ka.tu.ki.na}{0}
\verb{kaxarari}{ch}{}{}{}{adj.}{Relativo aos Kaxarari.}{ka.xa.ra.ri}{0}
\verb{kaxarari}{ch}{}{}{}{s.2g.}{Indivíduo pertencente ao povo kaxarari, família linguística pano.}{ka.xa.ra.ri}{0}
\verb{kaxinawá}{ch}{}{}{}{adj.}{Relativo aos Kaxinawá.}{ka.xi.na.wá}{0}
\verb{kaxinawá}{ch}{}{}{}{s.2g.}{Indivíduo pertencente ao povo kaxinawá, família linguística pano.}{ka.xi.na.wá}{0}
\verb{kaxixó}{ch\ldots{}ch}{}{}{}{adj.}{Relativo aos Kaxixó.}{ka.xi.xó}{0}
\verb{kaxixó}{ch\ldots{}ch}{}{}{}{s.2g.}{Indivíduo pertencente ao povo kaxixó.}{ka.xi.xó}{0}
\verb{kaxuyana}{ch}{}{}{}{adj.}{Relativo aos Kaxuyana.}{ka.xu.ya.na}{0}
\verb{kaxuyana}{ch}{}{}{}{s.2g.}{Indivíduo pertencente ao povo kaxuyana, família linguística karib.}{ka.xu.ya.na}{0}
\verb{kayapó}{}{}{}{}{adj.}{Relativo aos Kayapó.}{kay.a.pó}{0}
\verb{kayapó}{}{}{}{}{s.2g.}{Indivíduo pertencente ao povo kayapó, família linguística jê.}{kay.a.pó}{0}
\verb{kb}{}{Fís.}{}{}{}{Símb. do \textit{quilobar}.}{kb}{0}
\verb{kB}{}{Informát.}{}{}{}{Símb. de \textit{quilobyte}.}{kB}{0}
\verb{kb}{}{Informát.}{}{}{}{Símb. de \textit{quilobit}.}{kb}{0}
\verb{kelvin}{}{Fís.}{}{}{s.m.}{Unidade de medida de temperatura no Sistema Internacional. Símb.: K.}{kel.vin}{0}
\verb{kepleriano}{}{}{}{}{adj.}{Relativo a Johannes Kepler, astrônomo alemão do final do século \textsc{xvi} e início do século \textsc{xvii}, ou a suas teorias e a seus estudos.}{ke.ple.ri.a.no}{0}
\verb{kepleriano}{}{}{}{}{}{Diz"-se do sistema que obedece às leis de Kepler.}{ke.ple.ri.a.no}{0}
\verb{ketchup}{}{}{}{}{s.m.}{Molho de tomate temperado com vinagre e outros condimentos, com sabor levemente adocicado; catchup.}{\textit{ketchup}}{0}
\verb{kg}{}{}{}{}{}{Abrev. de \textit{quilograma}.}{kg}{0}
\verb{kibutz}{}{}{}{}{s.m.}{Pequena fazenda comunitária em Israel, cuja organização se baseia na cooperação voluntária.}{\textit{kibutz}}{0}
\verb{kilt}{}{}{}{}{s.m.}{Saia pregueada, de lã xadrez, típica do vestuário da Escócia.}{\textit{kilt}}{0}
\verb{kiribatiano}{}{}{}{}{adj.}{}{ki.ri.ba.ti.a.no}{0}
\verb{kiribatiano}{}{}{}{}{s.m.}{}{ki.ri.ba.ti.a.no}{0}
\verb{kiriri}{}{}{}{}{adj.}{Relativo aos Kiriri.}{ki.ri.ri}{0}
\verb{kiriri}{}{}{}{}{s.2g.}{Indivíduo pertencente ao povo kiriri.}{ki.ri.ri}{0}
\verb{kirsch}{}{}{}{}{s.m.}{Aguardente semelhante ao conhaque, destilado de uma espécie de cereja silvestre.}{\textit{kirsch}}{0}
\verb{kit}{}{}{}{}{s.m.}{Caixa ou estojo contendo um conjunto de peças ou utensílios.}{\textit{kit}}{0}
\verb{kitchenette}{}{}{}{}{s.f.}{Pequeno apartamento, constituído de um único cômodo, com uma cozinha minúscula e um banheiro.}{\textit{kitchenette}}{0}
\verb{kitsch}{}{}{}{}{adj.2g.}{Diz"-se de tendência estética ou estilo artístico caracterizados pelo uso extravagante de elementos populares ou fora de moda, considerados de mau gosto pela cultura dominante.}{\textit{kitsch}}{0}
\verb{kiwi}{}{Bot.}{}{}{s.m.}{Trepadeira de folhagem densa, que produz frutos comestíveis, de casca marrom coberta de pelos e polpa verde"-amarelada; quiuí. }{\textit{kiwi}}{0}
\verb{kiwi}{}{}{}{}{}{O fruto dessa planta.}{\textit{kiwi}}{0}
\verb{km}{}{}{}{}{}{Abrev. de \textit{quilômetro}.}{km}{0}
\verb{know"-how}{}{}{}{}{s.m.}{Conjunto de conhecimentos necessários para a elaboração de um produto ou a execução de um serviço.}{\textit{know"-how}}{0}
\verb{kocama}{}{}{}{}{adj.}{Relativo aos Kocama.}{ko.ca.ma}{0}
\verb{kocama}{}{}{}{}{s.2g.}{Indivíduo pertencente ao povo kocama, família linguística tupi"-guarani.}{ko.ca.ma}{0}
\verb{kokuiregatejê}{}{}{}{}{adj.}{Relativo aos Kokuiregatejê.}{ko.kui.re.ga.te.jê}{0}
\verb{kokuiregatejê}{}{}{}{}{s.2g.}{Indivíduo pertencente ao povo kokuiregatejê, família linguística jê.}{ko.kui.re.ga.te.jê}{0}
\verb{korubo}{}{}{}{}{adj.}{Relativo aos Korubo.}{ko.ru.bo}{0}
\verb{korubo}{}{}{}{}{s.2g.}{Indivíduo pertencente ao povo korubo, família linguística pano.}{ko.ru.bo}{0}
\verb{Kr}{}{Quím.}{}{}{}{Símb. do \textit{criptônio}. }{Kr}{0}
\verb{krahô}{}{}{}{}{adj.}{Relativo aos Krahô.}{kra.hô}{0}
\verb{krahô}{}{}{}{}{s.2g.}{Indivíduo pertencente ao povo krahô, família linguística jê.}{kra.hô}{0}
\verb{krenak}{}{}{}{}{adj.}{Relativo aos Krenak.}{kre.nak}{0}
\verb{krenak}{}{}{}{}{s.2g.}{Indivíduo pertencente ao povo krenak, família linguística krenak.}{kre.nak}{0}
\verb{krinkati}{}{}{}{}{adj.}{Relativo aos Krinkati.}{krin.ka.ti}{0}
\verb{krinkati}{}{}{}{}{s.2g.}{Indivíduo pertencente ao povo krinkati, família linguística jê.}{krin.ka.ti}{0}
\verb{kubeo}{é}{}{}{}{adj.}{Relativo aos Kubeo.}{ku.be.o}{0}
\verb{kubeo}{é}{}{}{}{s.2g.}{Indivíduo pertencente ao povo kubeo, família linguística tukano.}{ku.be.o}{0}
\verb{kuikuro}{}{}{}{}{adj.}{Relativo aos Kuikuro.}{kui.ku.ro}{0}
\verb{kuikuro}{}{}{}{}{s.2g.}{Indivíduo pertencente ao povo kuikuro, família linguística karib.}{kui.ku.ro}{0}
\verb{kujubim}{}{}{"-ins}{}{adj.}{Relativo aos Kujubim.}{ku.ju.bim}{0}
\verb{kujubim}{}{}{"-ins}{}{s.2g.}{Indivíduo pertencente ao povo kujubim, família linguística txapakura.}{ku.ju.bim}{0}
\verb{ku klux klan}{cs}{}{}{}{s.m.}{Sociedade secreta, criada no sul dos Estados Unidos, que prega e pratica o racismo violento dos brancos contra os negros.}{ku klux klan}{0}
\verb{kulina pano}{}{}{}{}{adj.}{Relativo aos Kulina Pano.}{ku.li.na pa.no}{0}
\verb{kulina pano}{}{}{}{}{s.2g.}{Indivíduo pertencente ao povo kulina pano, família linguística pano.}{ku.li.na pa.no}{0}
\verb{kummel}{}{}{}{}{s.m.}{Licor alcoólico aromatizado com cominho, muito popular na Alemanha e na Rússia.}{\textit{kummel}}{0}
\verb{kung fu}{}{}{}{}{s.m.}{Arte marcial criada na China, baseada em exercícios de concentração, usada como instrumento de defesa pessoal.}{\textit{kung fu}}{0}
\verb{kuripako}{}{}{}{}{adj.}{Relativo aos Kuripako.}{ku.ri.pa.ko}{0}
\verb{kuripako}{}{}{}{}{s.2g.}{Indivíduo pertencente ao povo kuripako, família linguística aruak.}{ku.ri.pa.ko}{0}
\verb{kuruaia}{}{}{}{}{adj.}{Relativo aos Kuruaia.}{ku.ru.ai.a}{0}
\verb{kuruaia}{}{}{}{}{s.2g.}{Indivíduo pertencente ao povo kuruaiá, família linguística munduruku.}{ku.ru.ai.a}{0}
\verb{kuwaitiano}{}{}{}{}{adj.}{Relativo ao Kuwait; kuweitiano.}{ku.wai.ti.a.no}{0}
\verb{kuwaitiano}{}{}{}{}{s.m.}{Indivíduo natural ou habitante desse país. }{ku.wai.ti.a.no}{0}
\verb{kuweitiano}{}{}{}{}{adj. e s.m.  }{Kuwaitiano.}{ku.wei.ti.a.no}{0}
\verb{kw }{}{}{}{}{}{Abrev. de \textit{quilowatt.} }{kw }{0}
\verb{kwazá}{}{}{}{}{adj.}{Relativo aos Kwazá.}{kwa.zá}{0}
\verb{kwazá}{}{}{}{}{s.2g.}{Indivíduo pertencente ao povo kwazá.}{kwa.zá}{0}
\verb{kyrie}{}{Relig.}{}{}{s.m.}{Parte da missa em que se invoca três vezes a Deus: (\textit{Senhor, tende piedade de nós}).  }{\textit{kyrie}}{0}
