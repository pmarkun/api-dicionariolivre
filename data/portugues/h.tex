\verb{h}{}{}{}{}{s.m.}{Oitava letra do alfabeto português.}{h}{0}
\verb{h}{}{}{}{}{}{Abrev. de \textit{hora}.}{h}{0}
\verb{H}{}{Quím.}{}{}{}{Símb. do \textit{hidrogênio}.}{H}{0}
\verb{hã}{}{}{}{}{interj.}{Expressão que pode denotar  reflexão, meditação, elucidação, admiração.}{hã}{0}
\verb{ha}{}{}{}{}{}{Abrev. de \textit{hectare}.}{ha}{0}
\verb{habanera}{ê}{}{}{}{s.f.}{Dança de origem afro"-cubana, difundida na Espanha, e cuja forma rítmica influenciou o maxixe, o tango e a música popular de quase todos os países hispano"-americanos. }{ha.ba.ne.ra}{0}
\verb{habanera}{ê}{}{}{}{}{Música dessa dança.}{ha.ba.ne.ra}{0}
\verb{habeas"-corpus}{ó}{Jur.}{}{}{s.m.}{Medida jurídica que garante a liberdade de ir e vir de uma pessoa a ser interposta contra ato ilegal. }{\textit{habeas"-corpus}}{0}
\verb{hábil}{}{}{"-eis}{}{adj.2g.}{Que tem aptidão ou capacidade para fazer algo.}{há.bil}{0}
\verb{hábil}{}{}{"-eis}{}{}{Que executa algo com perfeição; competente, esperto.}{há.bil}{0}
\verb{hábil}{}{}{"-eis}{}{}{Que está de acordo com as exigências preestabelecidas.}{há.bil}{0}
\verb{habilidade}{}{}{}{}{s.f.}{Qualidade de quem é hábil. }{ha.bi.li.da.de}{0}
\verb{habilidade}{}{}{}{}{}{Destreza, agilidade, aptidão.}{ha.bi.li.da.de}{0}
\verb{habilidade}{}{}{}{}{}{Engenhosidade, astúcia.}{ha.bi.li.da.de}{0}
\verb{habilidoso}{ô}{}{"-osos ⟨ó⟩}{"-osa ⟨ó⟩}{adj.}{Que tem ou revela habilidade, destreza; que é bom naquilo que faz.}{ha.bi.li.do.so}{0}
\verb{habilidoso}{ô}{}{"-osos ⟨ó⟩}{"-osa ⟨ó⟩}{}{Capaz, hábil, jeitoso.}{ha.bi.li.do.so}{0}
\verb{habilitação}{}{}{"-ões}{}{s.f.}{Ato ou efeito de habilitar; capacidade, aptidão.}{ha.bi.li.ta.ção}{0}
\verb{habilitação}{}{}{"-ões}{}{}{Condições necessárias para exercício de uma função.}{ha.bi.li.ta.ção}{0}
\verb{habilitações}{}{}{}{}{s.f.pl.}{Conjunto de conhecimentos, documentos ou títulos de uma pessoa; qualificação. }{ha.bi.li.ta.ções}{0}
\verb{habilitado}{}{}{}{}{adj.}{Que se habilitou, que cumpriu as exigências ou pré"-requisitos para alguma coisa.}{ha.bi.li.ta.do}{0}
\verb{habilitado}{}{}{}{}{}{Que tem habilitação; apto, capaz.}{ha.bi.li.ta.do}{0}
\verb{habilitar}{}{}{}{}{v.t.}{Tornar hábil, apto, capacitado para alguma coisa. }{ha.bi.li.tar}{0}
\verb{habilitar}{}{}{}{}{}{Preparar, dispor.}{ha.bi.li.tar}{\verboinum{1}}
\verb{habitabilidade}{}{}{}{}{s.f.}{Qualidade de habitável; possibilidade de ser habitado.}{ha.bi.ta.bi.li.da.de}{0}
\verb{habitação}{}{}{"-ões}{}{s.f.}{Ato ou efeito de habitar.}{ha.bi.ta.ção}{0}
\verb{habitação}{}{}{"-ões}{}{}{Lugar para morar; moradia, residência.}{ha.bi.ta.ção}{0}
\verb{habitacional}{}{}{"-ais}{}{adj.2g.}{Relativo a habitação.}{ha.bi.ta.ci.o.nal}{0}
\verb{habitáculo}{}{}{}{}{s.m.}{Habitação pequena e modesta.}{ha.bi.tá.cu.lo}{0}
\verb{habitante}{}{}{}{}{adj.2g.}{Que mora habitualmente num lugar.}{ha.bi.tan.te}{0}
\verb{habitante}{}{}{}{}{s.m.}{Morador, residente.}{ha.bi.tan.te}{0}
\verb{habitar}{}{}{}{}{v.t.}{Ocupar como residência; residir, morar, viver, estar domiciliado em.}{ha.bi.tar}{0}
\verb{habitar}{}{}{}{}{}{Tornar habitado.}{ha.bi.tar}{\verboinum{1}}
\verb{habitat}{}{}{}{}{s.m.}{Lugar habitado por uma espécie ou grupo de espécies, que reúne um complexo de condições ambientais necessárias a sua sobrevivência. }{\textit{habitat}}{0}
\verb{habitat}{}{}{}{}{}{Lugar ou meio em que se desenvolve um ser vivo; ambiente natural.}{\textit{habitat}}{0}
\verb{habitável}{}{}{"-eis}{}{adj.2g.}{Que se pode habitar.}{ha.bi.tá.vel}{0}
\verb{habitável}{}{}{"-eis}{}{}{Próprio para se habitar.}{ha.bi.tá.vel}{0}
\verb{habite"-se}{}{}{}{}{s.m.}{Documento oficial que autoriza a ocupação e o uso de um prédio recém"-construído ou reformado, depois de sua inspeção.}{ha.bi.te"-se}{0}
\verb{hábito}{}{}{}{}{s.m.}{Disposição adquirida pela repetição frequente dos mesmos atos; maneira usual de ser. }{há.bi.to}{0}
\verb{hábito}{}{}{}{}{}{Costume, praxe.}{há.bi.to}{0}
\verb{hábito}{}{}{}{}{}{Vestimenta de frade ou freira.}{há.bi.to}{0}
%\verb{habituado}{}{}{}{}{}{0}{ha.bi.tu.a.do}{0}
\verb{habitual}{}{}{"-ais}{}{adj.2g.}{Que se faz ou que acontece por hábito, por costume.}{ha.bi.tu.al}{0}
\verb{habitual}{}{}{"-ais}{}{}{Usual, comum, frequente.}{ha.bi.tu.al}{0}
\verb{habituar}{}{}{}{}{v.t.}{Incutir um hábito; acostumar. }{ha.bi.tu.ar}{\verboinum{1}}
\verb{habitué}{}{}{}{habituée}{s.m.}{Frequentador assíduo.}{\textit{habitué}}{0}
\verb{hachura}{}{}{}{}{s.f.}{Traçado de linhas finas, paralelas, muito próximas umas das outras, que se emprega em desenhos, gravuras e trabalhos gráficos para produzir efeito de sombra ou meio"-tom.}{ha.chu.ra}{0}
\verb{hachurar}{}{}{}{}{v.t.}{Traçar hachuras em.}{ha.chu.rar}{\verboinum{1}}
%\verb{hachuras}{}{}{}{}{}{0}{ha.chu.ras}{0}
\verb{hacker}{}{Informát.}{}{}{s.m.}{Pessoa de grande habilidade no manuseio de computadores.}{\textit{hacker}}{0}
\verb{hacker}{}{Informát.}{}{}{}{Violador de sistemas de computação.}{\textit{hacker}}{0}
\verb{hadoque}{ó}{Zool.}{}{}{s.m.}{Peixe semelhante ao bacalhau, encontrado no Atlântico Norte, que possui uma linha lateral negra e uma mancha escura atrás das brânquias.}{ha.do.que}{0}
\verb{háfnio}{}{Quím.}{}{}{s.m.}{Elemento químico metálico, usado em reatores nucleares, ligas especiais, compostos refratários, lâmpadas de incandescência, entre outros. \elemento{72}{178.49}{Hf}.}{háf.nio}{0}
\verb{hagiografia}{}{}{}{}{s.f.}{Biografia ou escrito acerca dos santos.}{ha.gi.o.gra.fi.a}{0}
\verb{hagiográfico}{}{}{}{}{adj.}{Relativo à hagiografia.}{ha.gi.o.grá.fi.co}{0}
\verb{hagiógrafo}{}{}{}{}{adj.}{Diz"-se dos livros do Antigo Testamento, menos o Pentateuco e os Profetas.}{ha.gi.ó.gra.fo}{0}
\verb{hagiógrafo}{}{}{}{}{s.m.}{Cada um desses livros.}{ha.gi.ó.gra.fo}{0}
\verb{hagiógrafo}{}{}{}{}{}{Autor dos livros da Bíblia.}{ha.gi.ó.gra.fo}{0}
\verb{hagiógrafo}{}{}{}{}{}{Autor que conta a vida dos santos.}{ha.gi.ó.gra.fo}{0}
\verb{hagiológio}{}{}{}{}{s.m.}{Livro, tratado sobre santos.}{ha.gi.o.ló.gio}{0}
\verb{haicai}{}{}{}{}{s.m.}{Forma de poema japonês constituído de três versos, sendo o primeiro e o quinto com cinco sílabas e o segundo com sete sílabas.}{hai.cai}{0}
\verb{haitiano}{}{}{}{}{adj.}{Relativo ao Haiti.}{hai.ti.a.no}{0}
\verb{haitiano}{}{}{}{}{s.m.}{Indivíduo natural ou habitante desse país.}{hai.ti.a.no}{0}
\verb{haliêutica}{}{}{}{}{s.f.}{A arte da pesca.}{ha.li.êu.ti.ca}{0}
\verb{hálito}{}{}{}{}{s.m.}{O ar que sai pela boca durante a expiração; bafo.}{há.li.to}{0}
\verb{hálito}{}{}{}{}{}{Cheiro da boca.}{há.li.to}{0}
\verb{hálito}{}{Fig.}{}{}{}{Aragem, brisa.}{há.li.to}{0}
\verb{halitose}{ó}{Med.}{}{}{s.f.}{Mau hálito.}{ha.li.to.se}{0}
\verb{hall}{}{}{}{}{s.m.}{Sala de entrada de grandes dimensões; saguão, vestíbulo.}{\textit{hall}}{0}
\verb{halo}{}{}{}{}{s.m.}{Arco luminoso que, por vezes, aparece em volta do disco lunar ou solar.}{ha.lo}{0}
\verb{halo}{}{}{}{}{}{Círculo que envolve a cabeça de Cristo e dos santos nas imagens sacras.}{ha.lo}{0}
\verb{halo}{}{}{}{}{}{Círculo violáceo que rodeia o mamilo do seio.}{ha.lo}{0}
\verb{halo}{}{Fig.}{}{}{}{Glória, prestígio, honra.}{ha.lo}{0}
\verb{halogênio}{}{Quím.}{}{}{s.m.}{Denominação comum dada aos elementos químicos flúor, cloro, bromo, iodo e astatínio, por sua facilidade de formação de sais quando combinados com elementos eletropositivos.}{ha.lo.gê.nio}{0}
\verb{halter}{é}{}{}{}{}{Var. de \textit{haltere}.}{hal.ter}{0}
\verb{haltere}{é}{Esport.}{}{}{s.m.}{Instrumento formado por uma haste de metal com esferas ou discos idênticos nas extremidades, usado em exercícios físicos; peso.}{hal.te.re}{0}
\verb{halterofilismo}{}{Esport.}{}{}{s.m.}{Levantamento de peso como prática de ginástica ou competição.}{hal.te.ro.fi.lis.mo}{0}
\verb{halterofilista}{}{}{}{}{s.2g.}{Pessoa que pratica o halterofilismo.}{hal.te.ro.fi.lis.ta}{0}
\verb{hálux}{cs}{Anat.}{}{}{s.m.}{O dedo grande do pé humano ou o equivalente anatômico nas patas traseiras dos animais. }{há.lux}{0}
\verb{hambúrguer}{}{}{"-res}{}{s.m.}{Massa de carne moída, temperada e ligada com ovo, de forma chata e arredondada, que costuma ser consumida frita ou grelhada.}{ham.búr.guer}{0}
\verb{hamster}{}{Zool.}{}{}{s.m.}{Pequeno mamífero roedor, originário da África e da Ásia, dotado de grande bolsa facial e de cauda muito curta, criado como animal de estimação ou cobaia.}{hams.ter}{0}
\verb{handebol}{ó}{Esport.}{}{}{s.m.}{Jogo semelhante ao futebol, mas que se joga com as mãos.}{han.de.bol}{0}
\verb{handicap}{}{Esport.}{}{}{s.m.}{Deficiência física ou mental que dificulta as atividades de uma pessoa.}{\textit{handicap}}{0}
\verb{handicap}{}{Fig.}{}{}{}{Desvantagem.}{\textit{handicap}}{0}
\verb{handicap}{}{Por ext.}{}{}{}{Vantagem dada a um competidor para compensar possível inferioridade em relação aos adversários.}{\textit{handicap}}{0}
\verb{hangar}{}{}{}{}{s.m.}{Galpão ou abrigo para aeronaves.}{han.gar}{0}
\verb{hânio}{}{Fís.}{}{}{s.m.}{}{hânio}{0}
\verb{hanseniano}{}{}{}{}{adj.}{Relativo à hanseníase.}{han.se.ni.a.no}{0}
\verb{hanseniano}{}{}{}{}{s.m.}{Indivíduo que tem hanseníase; leproso, morfético, lazarento.}{han.se.ni.a.no}{0}
\verb{hanseníase}{}{Med.}{}{}{s.f.}{Doença infecciosa crônica e contagiosa, que afeta a pele e as mucosas, provocando lesões diversas; lepra, morfeia.}{han.se.ní.a.se}{0}
\verb{haplologia}{}{Gram.}{}{}{s.f.}{Mudança linguística em uma palavra que consiste na supressão de uma de duas sílabas vizinhas iguais ou semelhantes.}{ha.plo.lo.gi.a}{0}
\verb{happy hour}{}{}{happy hours}{}{s.m.}{Período do dia, após o encerramneto do trabalho, em que os amigos se reúnem para conversar, beber e comer.}{\textit{happy hour}}{0}
\verb{haraquiri}{}{}{}{}{s.m.}{Forma de suicídio praticada antigamente no Japão, que consiste em rasgar o ventre com faca ou sabre.}{ha.ra.qui.ri}{0}
\verb{haras}{}{}{}{}{s.m.}{Sítio ou fazenda de criação de cavalos de raça.}{ha.ras}{0}
\verb{hardware}{}{Informát.}{}{}{s.m.}{Nome comum dado aos componentes físicos de um computador ou aos seus periféricos.}{\textit{hardware}}{0}
\verb{harém}{}{}{"-éns}{}{s.m.}{Na cultura muçulmana, parte de um palácio ou de uma casa destinada à habitação das mulheres.}{ha.rém}{0}
\verb{harém}{}{}{"-éns}{}{}{O conjunto de mulheres que habitam um harém.}{ha.rém}{0}
\verb{harém}{}{Fig.}{"-éns}{}{}{Casa de prostituição.}{ha.rém}{0}
\verb{harmonia}{}{}{}{}{s.f.}{Disposição bem ordenada entre as partes de um todo.}{har.mo.ni.a}{0}
\verb{harmonia}{}{}{}{}{}{Ausência de conflitos; paz, concórdia.}{har.mo.ni.a}{0}
\verb{harmonia}{}{}{}{}{}{Conformidade entre coisas e pessoas; acordo, concordância.}{har.mo.ni.a}{0}
\verb{harmonia}{}{Mús.}{}{}{}{Combinação de sons que soam simultaneamente, produzindo acorde.}{har.mo.ni.a}{0}
\verb{harmonia}{}{Mús.}{}{}{}{Numa orquestra, conjunto de instrumentos de sopro.}{har.mo.ni.a}{0}
\verb{harmônica}{}{Mús.}{}{}{s.f.}{Instrumento musical de lâminas de vidro de comprimento desigual, que são vibradas com uma baqueta.}{har.mô.ni.ca}{0}
\verb{harmônica}{}{}{}{}{}{Sanfona.}{har.mô.ni.ca}{0}
\verb{harmônica}{}{}{}{}{}{Gaita de boca.}{har.mô.ni.ca}{0}
\verb{harmônico}{}{}{}{}{adj.}{Relativo a harmonia.}{har.mô.ni.co}{0}
\verb{harmônico}{}{}{}{}{}{Regular, coerente, proporcionado.}{har.mô.ni.co}{0}
\verb{harmônico}{}{Mús.}{}{}{}{Que se estrutura conforme os princípios da harmonia.}{har.mô.ni.co}{0}
\verb{harmônico}{}{Fís.}{}{}{}{Diz"-se de fenômenos periódicos cuja frequência é um múltiplo inteiro da frequência de outro.}{har.mô.ni.co}{0}
\verb{harmônico}{}{Mús.}{}{}{s.m.}{Sons parciais que integram a sonoridade de uma nota musical.}{har.mô.ni.co}{0}
\verb{harmônio}{}{Mús.}{}{}{s.m.}{Instrumento de teclados e foles acionados por pedais.}{har.mô.nio}{0}
\verb{harmonioso}{ô}{}{"-osos ⟨ó⟩}{"-osa ⟨ó⟩}{adj.}{Que tem harmonia ou está em harmonia.}{har.mo.ni.o.so}{0}
\verb{harmonioso}{ô}{}{"-osos ⟨ó⟩}{"-osa ⟨ó⟩}{}{Que é agradável ao ouvido ou à vista.}{har.mo.ni.o.so}{0}
\verb{harmonista}{}{Mús.}{}{}{s.2g.}{Indivíduo que sabe as regras da harmonia.}{har.mo.nis.ta}{0}
\verb{harmonista}{}{Mús.}{}{}{}{Instrumentista que toca harmônio.}{har.mo.nis.ta}{0}
\verb{harmonização}{}{}{"-ões}{}{s.f.}{Ato ou efeito de harmonizar, de pôr em harmonia.}{har.mo.ni.za.ção}{0}
\verb{harmonizar}{}{}{}{}{v.t.}{Pôr em harmonia, tornar harmônico; conciliar, congraçar.}{har.mo.ni.zar}{0}
\verb{harmonizar}{}{Mús.}{}{}{}{Escrever o acompanhamento musical para.}{har.mo.ni.zar}{0}
\verb{harmonizar}{}{}{}{}{v.i.}{Estar em harmonia, de acordo.}{har.mo.ni.zar}{\verboinum{1}}
\verb{harpa}{}{Mús.}{}{}{s.f.}{Instrumento musical de forma triangular, composto de cordas desiguais, que se tocam com os dedos.}{har.pa}{0}
\verb{harpar}{}{}{}{}{}{Var. de \textit{harpear}.}{har.par}{0}
\verb{harpear}{}{Mús.}{}{}{v.i.}{Tocar harpa; harpar, harpejar.}{har.pe.ar}{\verboinum{4}}
\verb{harpejar}{}{Mús.}{}{}{v.t.}{Harpear.}{har.pe.jar}{\verboinum{1}}
\verb{harpejo}{ê}{}{}{}{}{Var. de \textit{arpejo}.}{har.pe.jo}{0}
\verb{harpia}{}{Mit.}{}{}{s.f.}{Monstro fabuloso, com rosto de mulher e corpo de abutre.}{har.pi.a}{0}
\verb{harpia}{}{Fig.}{}{}{}{Pessoa ávida, rapace.}{har.pi.a}{0}
\verb{harpia}{}{Zool.}{}{}{}{Ave de rapina, com a parte superior do corpo cinza"-claro, parte inferior branca e cauda parda listrada de preto.}{har.pi.a}{0}
\verb{harpista}{}{Mús.}{}{}{s.2g.}{Indivíduo que toca harpa.}{har.pis.ta}{0}
\verb{hássio}{}{Quím.}{}{}{s.m.}{Elemento químico artificial, provavelmente metálico, sólido, de coloração cinza ou branco prateado, com propriedades comparáveis às do ósmio. \elemento{108}{(265)}{Hs}.}{hás.sio}{0}
\verb{hasta}{}{}{}{}{s.m.}{Arma ofensiva, com um longo cabo arrematado por peça pontiaguda de metal; lança, pique.}{has.ta}{0}
\verb{hasta}{}{}{}{}{}{Leilão, arrematação.}{has.ta}{0}
\verb{haste}{}{}{}{}{s.f.}{Pau ou ferro erguido, longo e retilíneo, no qual se encrava ou apoia alguma coisa.}{has.te}{0}
\verb{haste}{}{}{}{}{}{Pau de bandeira; mastro.}{has.te}{0}
\verb{haste}{}{Por ext.}{}{}{}{Parte de um vegetal a que estão seguros folhas, flores, frutos.}{has.te}{0}
\verb{haste}{}{}{}{}{}{Chifre, corno.}{has.te}{0}
\verb{hasteado}{}{}{}{}{adj.}{Que está disposto em haste; içado, erigido.}{has.te.a.do}{0}
\verb{hasteamento}{}{}{}{}{s.m.}{Ato de hastear, de fazer subir ou prender ao topo de uma haste.}{has.te.a.men.to}{0}
\verb{hastear}{}{}{}{}{v.t.}{Elevar ou prender ao topo de uma haste, vara ou mastro; içar, arvorar.}{has.te.ar}{0}
\verb{hastear}{}{Por ext.}{}{}{}{Erguer alto; levantar.}{has.te.ar}{\verboinum{4}}
\verb{hastil}{}{}{"-is}{}{s.m.}{Cabo de lança.}{has.til}{0}
\verb{hastil}{}{}{"-is}{}{}{Haste, pedúnculo.}{has.til}{0}
\verb{hastilha}{}{}{}{}{s.f.}{Pequena haste.}{has.ti.lha}{0}
\verb{haurir}{}{}{}{}{v.t.}{Retirar algo de lugar profundo; extrair, colher.}{hau.rir}{0}
\verb{haurir}{}{}{}{}{}{Consumir algo inteiramente; esgotar, esvaziar.}{hau.rir}{0}
\verb{haurir}{}{}{}{}{}{Absorver por aspiração; sorver.}{hau.rir}{\verboinum{34}}
\verb{haurível}{}{}{"-eis}{}{adj.2g.}{Que se pode haurir.}{hau.rí.vel}{0}
\verb{hausto}{}{}{}{}{s.m.}{Ato ou efeito de haurir.}{haus.to}{0}
\verb{hausto}{}{}{}{}{}{Aspiração longa, profunda; sorvo.}{haus.to}{0}
\verb{hausto}{}{}{}{}{}{Porção de bebida que se ingere de uma só vez; gole, trago.}{haus.to}{0}
\verb{hausto}{}{}{}{}{}{Medicamento que se bebe.}{haus.to}{0}
\verb{havaiano}{}{}{}{}{adj.}{Relativo ao arquipélago"-estado norte"-americano do Havaí.}{ha.vai.a.no}{0}
\verb{havaiano}{}{}{}{}{s.m.}{Indivíduo natural ou habitante desse estado. }{ha.vai.a.no}{0}
\verb{havana}{}{}{}{}{s.m.}{Charuto fabricado na cidade de Havana (Cuba).}{ha.va.na}{0}
\verb{havana}{}{}{}{}{}{A cor castanho claro do tabaco cubano.}{ha.va.na}{0}
\verb{havanês}{}{}{}{}{adj.}{Relativo a Havana, capital de Cuba.}{ha.va.nês}{0}
\verb{havanês}{}{}{}{}{s.m.}{Indivíduo natural ou habitante dessa cidade. }{ha.va.nês}{0}
\verb{haver}{ê}{}{}{}{v.i.}{Existir. (\textit{Há trinta alunos nesta sala.})}{ha.ver}{0}
\verb{haver}{ê}{}{}{}{}{Em perífrases verbais, denota intenção ou desejo. (\textit{Havemos de conseguir o prêmio.})}{ha.ver}{0}
\verb{haver}{ê}{}{}{}{v.pron.}{Comportar"-se, ajustar"-se, avir"-se, sair"-se. (\textit{O aluno se houve muito bem na apresentação do trabalho.})}{ha.ver}{\verboinum{40}}
\verb{haveres}{ê}{}{}{}{s.m.pl.}{Bens, posses, riqueza.}{ha.ve.res}{0}
\verb{haxixe}{ch\ldots{}ch}{}{}{}{s.m.}{Substância preparada com a resina do cânhamo, de efeito entorpecente e com propriedades medicinais.}{ha.xi.xe}{0}
\verb{He}{}{Quím.}{}{}{}{Símb. do \textit{hélio}. }{He}{0}
\verb{heavy metal}{}{}{}{}{s.m.}{Tipo de música \textit{rock} muito agressiva, que explora efeitos de batidas rítmicas e de vocalizações fortes. }{\textit{heavy metal}}{0}
\verb{hebdomadário}{}{}{}{}{adj.}{Relativo a semana; semanal.}{heb.do.ma.dá.rio}{0}
\verb{hebdomadário}{}{}{}{}{s.m.}{Publicação semanal; semanário.}{heb.do.ma.dá.rio}{0}
\verb{hebraico}{}{}{}{}{adj.}{Relativo aos hebreus, povo semita da Antiguidade. }{he.brai.co}{0}
\verb{hebraico}{}{}{}{}{s.m.}{Indivíduo desse povo.}{he.brai.co}{0}
\verb{hebraico}{}{}{}{}{}{Língua da família semítica falada pelos hebreus; hoje, língua oficial de Israel.}{he.brai.co}{0}
\verb{hebraísmo}{}{}{}{}{s.m.}{Palavra, expressão ou construção própria da língua hebraica.}{he.bra.ís.mo}{0}
\verb{hebreu}{}{}{}{hebreia}{adj.}{Relativo aos hebreus; hebraico.}{he.breu}{0}
\verb{hebreu}{}{}{}{hebreia}{s.m.}{Indíviduo dos hebreus, um dos povos semitas da Antiguidade}{he.breu}{0}
\verb{hebreu}{}{}{}{hebreia}{}{A língua dos hebreus; hebraico.}{he.breu}{0}
\verb{hecatombe}{}{}{}{}{s.f.}{Na Antiguidade, sacrifício de cem bois.}{he.ca.tom.be}{0}
\verb{hecatombe}{}{Por ext.}{}{}{}{Massacre, carnificina, desgraça; grande destruição.}{he.ca.tom.be}{0}
\verb{hectare}{}{}{}{}{s.m.}{Unidade de medida de superfície, correspondente a 100 ares, ou 10.000 metros quadrados. Símb.: ha.}{hec.ta.re}{0}
\verb{héctica}{}{Med.}{}{}{s.f.}{Estado febril prolongado, decorrente de doença crônica.}{héc.ti.ca}{0}
\verb{héctica}{}{Pop.}{}{}{}{Tuberculose pulmonar.}{héc.ti.ca}{0}
\verb{héctico}{}{}{}{}{adj.}{Relativo a héctica.}{héc.ti.co}{0}
\verb{héctico}{}{}{}{}{}{Que sofre de héctica.}{héc.ti.co}{0}
\verb{héctico}{}{}{}{}{s.m.}{Indivíduo que sofre de héctica. }{héc.ti.co}{0}
\verb{hectograma}{}{}{}{}{s.m.}{Unidade de medida de massa, correspondente a cem gramas. Símb.: hg.}{hec.to.gra.ma}{0}
\verb{hectolitro}{}{}{}{}{s.m.}{Unidade de medida de volume, correspondente a cem litros. Símb.: hl.}{hec.to.li.tro}{0}
\verb{hectômetro}{}{}{}{}{s.m.}{Unidade de medida de extensão, correspondente a cem metros. Símb.: hm.}{hec.tô.me.tro}{0}
\verb{hediondez}{ê}{}{}{}{s.f.}{Qualidade do que é hediondo, medonho.}{he.di.on.dez}{0}
\verb{hediondez}{ê}{Fig.}{}{}{}{Procedimento hediondo.}{he.di.on.dez}{0}
\verb{hediondo}{}{}{}{}{adj.}{Extremamente feio; horrível, deformado, repulsivo.}{he.di.on.do}{0}
\verb{hediondo}{}{Fig.}{}{}{}{Que causa grande indignação; ignóbil, sórdido, pavoroso.}{he.di.on.do}{0}
\verb{hedonismo}{}{}{}{}{s.m.}{Doutrina que considera o prazer individual como finalidade suprema da vida.}{he.do.nis.mo}{0}
\verb{hedonista}{}{}{}{}{adj.2g.}{Que é partidário do hedonismo.}{he.do.nis.ta}{0}
\verb{hedonista}{}{}{}{}{s.2g.}{Indivíduo partidário do hedonismo, ou que vive segundo seus preceitos. }{he.do.nis.ta}{0}
\verb{hegemonia}{}{}{}{}{s.f.}{Supremacia de uma cidade ou de um povo sobre outras cidades ou povos.}{he.ge.mo.ni.a}{0}
\verb{hegemônico}{}{}{}{}{adj.}{Relativo a hegemonia.}{he.ge.mô.ni.co}{0}
\verb{hegemônico}{}{}{}{}{}{Que detém hegemonia.}{he.ge.mô.ni.co}{0}
\verb{hégira}{}{Relig.}{}{}{s.f.}{Fuga atribuída a Maomé de Meca para Medina, no ano 622.}{hé.gi.ra}{0}
\verb{hégira}{}{Fig.}{}{}{}{Viagem longa, geralmente fugindo de um perigo; fuga.}{hé.gi.ra}{0}
\verb{hein}{}{}{}{}{interj.}{Expressão usada quando não se ouviu bem o que foi dito.}{hein}{0}
\verb{hein}{}{}{}{}{}{Expressa espanto ou indignação.}{hein}{0}
\verb{hein}{}{}{}{}{}{No final de comentários, equivale a  "não é verdade?".}{hein}{0}
\verb{helênico}{}{}{}{}{adj.}{Relativo à Hélade, ou Grécia Antiga.}{he.lê.ni.co}{0}
\verb{helênico}{}{}{}{}{s.m.}{Indivíduo natural ou habitante desse lugar.}{he.lê.ni.co}{0}
\verb{helênico}{}{}{}{}{}{Tronco linguístico de que faz parte o grego moderno.}{he.lê.ni.co}{0}
\verb{helenismo}{}{}{}{}{s.m.}{O conjunto de ideias e costumes da Grécia Antiga; a civilização grega da Antiguidade.}{he.le.nis.mo}{0}
\verb{helenismo}{}{}{}{}{}{Culto ou devoção às ideias e aos costumes da Grécia Antiga.}{he.le.nis.mo}{0}
\verb{helenismo}{}{Gram.}{}{}{}{Palavra, expressão ou construção própria da língua grega.}{he.le.nis.mo}{0}
\verb{helenista}{}{}{}{}{adj.2g.}{Que se dedica ao estudo da língua, literatura ou cultura da Grécia Antiga.}{he.le.nis.ta}{0}
\verb{helenístico}{}{Hist.}{}{}{adj.}{Relativo ao helenismo.}{he.le.nís.ti.co}{0}
\verb{helenístico}{}{}{}{}{}{Diz"-se do período da história da Grécia posterior às conquistas de Alexandre, o Grande.}{he.le.nís.ti.co}{0}
\verb{helenizar}{}{}{}{}{v.t.}{Dar características da cultura grega a.}{he.le.ni.zar}{\verboinum{1}}
\verb{heleno}{}{}{}{}{adj.}{Relativo ou pertencente aos helenos, povo que viveu na região do Epiro, no Noroeste da Grécia, e que deu origem ao povo grego.}{he.le.no}{0}
\verb{heleno}{}{}{}{}{}{Relativo à Grécia moderna.}{he.le.no}{0}
\verb{heleno}{}{}{}{}{s.m.}{Indivíduo do povo heleno. }{he.le.no}{0}
\verb{helianto}{}{Bot.}{}{}{s.m.}{Girassol.}{he.li.an.to}{0}
\verb{hélice}{}{}{}{}{s.f.}{Peça giratória formada por quatro pares de pás perpendiculares e opostas entre si, que têm a propriedade de sugar e expelir fortemente para trás o ar ou a água que está à frente, causando grande turbulência; peça usada, principalmente, como mecanismo propulsor de aviões, helicópteros, navios etc. }{hé.li.ce}{0}
\verb{hélice}{}{}{}{}{}{Figura espiralada ou em forma de caracol.}{hé.li.ce}{0}
\verb{hélice}{}{Zool.}{}{}{}{Nome comum a certos caracóis terrestres, com algumas espécies comestíveis, como o \textit{escargot}.}{hé.li.ce}{0}
\verb{helicoidal}{}{}{"-ais}{}{adj.2g.}{Helicoide.}{he.li.coi.dal}{0}
\verb{helicoide}{}{}{}{}{adj.2g.}{Que tem a forma de hélice; em caracol; helicoidal.}{he.li.coi.de}{0}
\verb{helicóptero}{}{}{}{}{s.m.}{Veículo aéreo dotado de uma hélice horizontal e um rotor, capaz de sustentar"-se no ar e deslocar"-se em qualquer sentido.}{he.li.cóp.te.ro}{0}
\verb{hélio}{}{Quím.}{}{}{s.m.}{Elemento químico da família dos gases nobres, incolor, muito leve, presente na atmosfera, usado para encher balões e em equipamentos de mergulho. \elemento{2}{4.002602}{He}.}{hé.lio}{0}
\verb{hélio}{}{}{}{}{}{Sol.}{hé.lio}{0}
\verb{heliocêntrico}{}{}{}{}{adj.}{Que tem o Sol como centro.}{he.li.o.cên.tri.co}{0}
\verb{heliocentrismo}{}{}{}{}{s.m.}{Sistema astronômico em que se considera o Sol como o centro do Universo.}{he.li.o.cen.tris.mo}{0}
\verb{heliografia}{}{Astron.}{}{}{s.f.}{Estudo descritivo do Sol.}{he.li.o.gra.fi.a}{0}
\verb{heliografia}{}{}{}{}{}{Processo de reprodução que utiliza a luz para produzir as cópias.}{he.li.o.gra.fi.a}{0}
\verb{heliogravura}{}{}{}{}{s.f.}{Processo fotomecânico de impressão que utiliza placas de cobre preparadas com auxílio de gelatina sensível à luz.}{he.li.o.gra.vu.ra}{0}
\verb{heliogravura}{}{Por ext.}{}{}{}{As placas preparadas por esse processo.}{he.li.o.gra.vu.ra}{0}
\verb{heliogravura}{}{Por ext.}{}{}{}{A estampa que se tira dessas placas.}{he.li.o.gra.vu.ra}{0}
\verb{heliômetro}{}{Astron.}{}{}{s.m.}{Instrumento de medição utilizado para determinar a posição relativa de dois astros, originalmente construído para medir o diâmetro aparente do Sol.}{he.li.ô.me.tro}{0}
\verb{helioscopia}{}{Astron.}{}{}{s.f.}{Observação do Sol através de helioscópio.}{he.li.os.co.pi.a}{0}
\verb{helioscópio}{}{Astron.}{}{}{s.m.}{Instrumento destinado à observação do Sol.}{he.li.os.có.pio}{0}
\verb{heliose}{ó}{Med.}{}{}{s.f.}{Insolação.}{he.li.o.se}{0}
\verb{heliostática}{}{}{}{}{s.f.}{Teoria dos movimentos planetários baseada na hipótese heliocêntrica.}{he.li.os.tá.ti.ca}{0}
\verb{helioterapia}{}{Med.}{}{}{s.f.}{Tratamento de certas doenças pela luz solar.}{he.li.o.te.ra.pi.a}{0}
\verb{helioterápico}{}{}{}{}{adj.}{Relativo à helioterapia.}{he.li.o.te.rá.pi.co}{0}
\verb{heliotermômetro}{}{}{}{}{s.m.}{Aparelho que mede a intensidade do calor solar.}{he.li.o.ter.mô.me.tro}{0}
\verb{heliotropia}{}{}{}{}{s.f.}{Heliotropismo.}{he.li.o.tro.pi.a}{0}
\verb{heliotrópico}{}{}{}{}{adj.}{Relativo ao heliotropismo.}{he.li.o.tró.pi.co}{0}
\verb{heliotrópio}{}{Bot.}{}{}{s.m.}{Nome comum a várias ervas tropicais ricas em alcaloides e com propriedades medicinais.}{he.li.o.tró.pio}{0}
\verb{heliotrópio}{}{Bot.}{}{}{}{Nome comum às plantas cujas flores se voltam para o Sol.}{he.li.o.tró.pio}{0}
\verb{heliotropismo}{}{Biol.}{}{}{s.m.}{Fenômeno da mudança da posição de certas partes dos organismos de acordo com a luz do Sol.}{he.li.o.tro.pis.mo}{0}
\verb{heliponto}{}{}{}{}{s.m.}{Área ou estrutura artificial para pouso e decolagem de helicópteros.}{he.li.pon.to}{0}
\verb{heliporto}{ô}{}{"-ortos ⟨ó⟩}{}{s.m.}{Área para pouso e decolagem de helicópteros, dotada de instalações para circulação de passageiros e transporte de carga.}{he.li.por.to}{0}
\verb{helmintíase}{}{Med.}{}{}{s.f.}{Infecção intestinal produzida por helmintos; verminose.}{hel.min.tí.a.se}{0}
\verb{helminto}{}{}{}{}{s.m.}{Nome comum a várias espécies de vermes endoparasitas.}{hel.min.to}{0}
\verb{helmintologia}{}{Biol.}{}{}{s.f.}{Estudo dos vermes em geral.}{hel.min.to.lo.gi.a}{0}
\verb{help}{}{Informát.}{}{}{s.m.}{Sistema de ajuda ao usuário de um programa, com explicações sobre seus recursos e comandos.}{\textit{help}}{0}
\verb{helvécio}{}{}{}{}{adj.}{Relativo à Helvécia ou à Suíça.}{hel.vé.cio}{0}
\verb{helvécio}{}{}{}{}{s.m.}{Indivíduo dos helvécios, povo que habitava na Helvécia. }{hel.vé.cio}{0}
\verb{hem}{}{}{}{}{interj.}{Hein.}{hem}{0}
\verb{hemácia}{}{Med.}{}{}{s.f.}{Tipo de célula sanguínea, glóbulo vermelho do sangue.}{he.má.cia}{0}
\verb{hematita}{}{}{}{}{s.f.}{Minério de que se extrai o ferro, sendo uma de suas principais fontes.}{he.ma.ti.ta}{0}
\verb{hematófago}{}{}{}{}{adj.}{Que se alimenta de sangue.}{he.ma.tó.fa.go}{0}
\verb{hematófilo}{}{}{}{}{adj.}{Que gosta de sangue.}{he.ma.tó.fi.lo}{0}
\verb{hematofobia}{}{}{}{}{s.f.}{Aversão, horror a sangue.}{he.ma.to.fo.bi.a}{0}
\verb{hematófobo}{}{}{}{}{adj.}{Que tem aversão a sangue.}{he.ma.tó.fo.bo}{0}
\verb{hematologia}{}{}{}{}{s.f.}{Estudo ou tratado a respeito do sangue e de suas propriedades físicas.}{he.ma.to.lo.gi.a}{0}
\verb{hematologia}{}{Med.}{}{}{}{Ramo da medicina que se ocupa das doenças do sangue, da medula e dos gânglios linfáticos.}{he.ma.to.lo.gi.a}{0}
\verb{hematológico}{}{}{}{}{adj.}{Relativo à hematologia, aos estudos acerca do sangue.}{he.ma.to.ló.gi.co}{0}
\verb{hematologista}{}{}{}{}{s.2g.}{Especialista em doenças do sangue.}{he.ma.to.lo.gis.ta}{0}
\verb{hematoma}{}{Med.}{}{}{s.m.}{Acúmulo de sangue em um órgão ou tecido ocasionado pela ruptura de vasos sanguíneos.}{he.ma.to.ma}{0}
\verb{hematopoese}{é}{Biol.}{}{}{s.f.}{Processo de formação e desenvolvimento das células sanguíneas.}{he.ma.to.po.e.se}{0}
\verb{hematose}{ó}{Med.}{}{}{s.f.}{Transformação do sangue venoso em sangue arterial ao nível dos alvéolos pulmonares.}{he.ma.to.se}{0}
\verb{hematozoário}{}{Biol.}{}{}{s.m.}{Protozoário que parasita o sangue, com algumas espécies responsáveis pela malária.}{he.ma.to.zo.á.rio}{0}
\verb{hematúria}{}{Med.}{}{}{s.f.}{Presença de sangue na urina.}{he.ma.tú.ria}{0}
\verb{hemeroteca}{é}{}{}{}{s.f.}{Setor onde se encontram coleções de revistas e jornais em uma biblioteca.}{he.me.ro.te.ca}{0}
\verb{hemialgia}{}{Med.}{}{}{s.f.}{Dor que ocorre somente em uma das metades do corpo ou de um órgão.}{he.mi.al.gi.a}{0}
\verb{hemiciclo}{}{}{}{}{s.m.}{Qualquer estrutura semicircular; semicírculo.}{he.mi.ci.clo}{0}
\verb{hemiciclo}{}{}{}{}{}{Espaço semicircular com assentos ou bancadas dispostos em filas que convergem para o centro; anfiteatro.}{he.mi.ci.clo}{0}
\verb{hemicrania}{}{Med.}{}{}{s.f.}{Dor que ocorre apenas em uma das metades da cabeça; enxaqueca.}{he.mi.cra.ni.a}{0}
\verb{hemiplegia}{}{Med.}{}{}{s.f.}{Paralisia total ou parcial de um dos lados do corpo.}{he.mi.ple.gi.a}{0}
\verb{hemiplégico}{}{}{}{}{adj.}{Relativo à hemiplegia; que sofre de hemiplegia.}{he.mi.plé.gi.co}{0}
\verb{hemiplégico}{}{}{}{}{s.m.}{Indivíduo que sofre de hemiplegia, que perdeu a mobilidade voluntária de uma das metades laterais do corpo.}{he.mi.plé.gi.co}{0}
\verb{hemíptero}{}{}{}{}{adj.}{Que possui asas ou barbatanas curtas.}{he.míp.te.ro}{0}
\verb{hemíptero}{}{}{}{}{}{Diz"-se do inseto que sofre metamorfose gradual e apresenta aparelho bicador ou sugador como os percevejos.}{he.míp.te.ro}{0}
\verb{hemisférico}{}{}{}{}{adj.}{Relativo a hemisfério.}{he.mis.fé.ri.co}{0}
\verb{hemisférico}{}{}{}{}{}{Que tem a forma da metade de uma esfera.}{he.mis.fé.ri.co}{0}
\verb{hemisfério}{}{}{}{}{s.m.}{Cada uma das metades de uma esfera.}{he.mis.fé.rio}{0}
\verb{hemisfério}{}{Geogr.}{}{}{}{Cada uma das metades do globo terrestre, dividido pelo Equador em norte e sul.}{he.mis.fé.rio}{0}
\verb{hemisfério}{}{Geogr.}{}{}{}{Cada uma das metades do globo terrestre, dividido em leste e oeste pelo meridiano de Greenwich e pela linha internacional de mudança de data.}{he.mis.fé.rio}{0}
\verb{hemisfério}{}{Astron.}{}{}{}{Cada uma das metades da esfera celeste, designadas como boreal e austral.}{he.mis.fé.rio}{0}
\verb{hemisfério}{}{Anat.}{}{}{}{Cada uma das metades, direita e esquerda, do cérebro. }{he.mis.fé.rio}{0}
\verb{hemistíquio}{}{Gram.}{}{}{s.m.}{Cada uma das metades de um verso, especialmente do verso alexandrino, marcadas por uma pausa ou um corte.}{he.mis.tí.quio}{0}
\verb{hemocentro}{}{}{}{}{s.m.}{Centro de análise e armazenamento de sangue; banco de sangue.}{he.mo.cen.tro}{0}
\verb{hemodiálise}{}{}{}{}{s.f.}{Procedimento de filtração do sangue com equipamento em caso de insuficiência renal.}{he.mo.di.á.li.se}{0}
\verb{hemofilia}{}{Med.}{}{}{s.f.}{Doença hemorrágica hereditária caracterizada por problemas de coagulação do sangue.}{he.mo.fi.li.a}{0}
\verb{hemofílico}{}{}{}{}{adj.}{Relativo à hemofilia.}{he.mo.fí.li.co}{0}
\verb{hemofílico}{}{}{}{}{}{Que sofre de hemofilia.}{he.mo.fí.li.co}{0}
\verb{hemoglobina}{}{Bioquím.}{}{}{s.f.}{Pigmento existente no interior das hemácias responsável pelo transporte de oxigênio.}{he.mo.glo.bi.na}{0}
\verb{hemograma}{}{Med.}{}{}{s.m.}{Exame laboratorial que fornece a contagem dos elementos do sangue, tais como plaquetas, glóbulos brancos e vermelhos, entre outros.}{he.mo.gra.ma}{0}
\verb{hemólise}{}{Med.}{}{}{s.f.}{Dissolução ou destruição dos glóbulos vermelhos do sangue com liberação de hemoglobina.}{he.mó.li.se}{0}
\verb{hemoptise}{}{Med.}{}{}{s.f.}{Expectoração de sangue observada principalmente na tuberculose pulmonar.}{he.mop.ti.se}{0}
\verb{hemorragia}{}{Med.}{}{}{s.f.}{Derramamento de sangue para fora dos vasos sanguíneos.}{he.mor.ra.gi.a}{0}
\verb{hemorrágico}{}{}{}{}{adj.}{Relativo a hemorragia; que sofre de hemorragia.}{he.mor.rá.gi.co}{0}
\verb{hemorroidal}{}{}{"-ais}{}{adj.2g.}{Relativo às hemorroidas.}{he.mor.roi.dal}{0}
\verb{hemorroidas}{}{Med.}{}{}{s.f.}{Dilatação das veias do ânus ou do reto. }{he.mor.roi.das}{0}
\verb{hemóstase}{}{Med.}{}{}{s.f.}{Ato ou efeito de interromper uma hemorragia.}{he.mós.ta.se}{0}
\verb{hemostático}{}{Med.}{}{}{adj.}{Diz"-se do medicamento capaz de interromper uma hemorragia. }{he.mos.tá.ti.co}{0}
\verb{hena}{}{Bot.}{}{}{s.f.}{Arbusto nativo do Norte da África, com casca e folhas de que se prepara tintura castanho avermelhado usada, normalmente, para tingir cabelos. }{he.na}{0}
\verb{hendecaedro}{é}{Geom.}{}{}{s.m.}{Poliedro que contém onze faces; undecaedro.}{hen.de.ca.e.dro}{0}
\verb{hendecágono}{}{Geom.}{}{}{s.m.}{Polígono que possui onze lados.}{hen.de.cá.go.no}{0}
\verb{hendecassílabo}{}{Gram.}{}{}{adj.}{Que tem onze sílabas.}{hen.de.cas.sí.la.bo}{0}
\verb{hepatalgia}{}{Med.}{}{}{s.f.}{Dor no fígado.}{he.pa.tal.gi.a}{0}
\verb{hepático}{}{}{}{}{adj.}{Relativo ao fígado.}{he.pá.ti.co}{0}
\verb{hepático}{}{}{}{}{s.m.}{Pessoa que sofre do fígado.}{he.pá.ti.co}{0}
\verb{hepatismo}{}{Med.}{}{}{s.m.}{Expressão que se usava para referir"-se à doença do fígado. }{he.pa.tis.mo}{0}
\verb{hepatite}{}{Med.}{}{}{s.f.}{Inflamação do fígado de origem tóxica ou infecciosa.}{he.pa.ti.te}{0}
\verb{hepatologia}{}{Med.}{}{}{s.f.}{Estudo da anatomia, fisiologia e patologia do fígado.}{he.pa.to.lo.gi.a}{0}
\verb{heptacampeão}{}{}{"-ões}{}{adj.}{Diz"-se de indivíduo, equipe ou clube campeão pela sétima vez.}{hep.ta.cam.pe.ão}{0}
\verb{heptaedro}{é}{Geom.}{}{}{s.m.}{Poliedro que contém sete faces.}{hep.ta.e.dro}{0}
\verb{heptagonal}{}{}{"-ais}{}{adj.2g.}{Relativo a heptágono.}{hep.ta.go.nal}{0}
\verb{heptagonal}{}{}{"-ais}{}{}{Que possui sete lados.}{hep.ta.go.nal}{0}
\verb{heptágono}{}{Geom.}{}{}{s.m.}{Polígono que possui sete lados.}{hep.tá.go.no}{0}
\verb{heptassílabo}{}{Gram.}{}{}{adj.}{Que tem sete sílabas; setissílabo.}{hep.tas.sí.la.bo}{0}
\verb{hera}{é}{Bot.}{}{}{s.f.}{Planta trepadeira, com raízes fixadoras, muito cultivada para revestir e ornamentar paredes e muros.}{he.ra}{0}
\verb{heráldica}{}{}{}{}{s.f.}{Arte ou ciência que estuda a origem, a composição e a significação dos brasões.}{he.rál.di.ca}{0}
\verb{heráldico}{}{}{}{}{adj.}{Relativo a brasões.}{he.rál.di.co}{0}
\verb{heráldico}{}{Fig.}{}{}{}{Que demonstra majestade, nobreza; aristocrático, distinto.}{he.rál.di.co}{0}
\verb{heraldista}{}{}{}{}{s.2g.}{Especialista no estudo dos brasões.}{he.ral.dis.ta}{0}
\verb{herança}{}{Jur.}{}{}{s.f.}{Aquilo que é transmitido por disposição testamentária ou por via de sucessão; legado.}{he.ran.ça}{0}
\verb{herança}{}{Biol.}{}{}{}{O que é transmitido geneticamente pelos seres vivos aos seus descendentes.}{he.ran.ça}{0}
\verb{herbáceo}{}{}{}{}{adj.}{Relativo a erva; herbóreo.}{her.bá.ceo}{0}
\verb{herbanário}{}{}{}{}{s.m.}{Lugar onde se vendem ervas medicinais.}{her.ba.ná.rio}{0}
\verb{herbanário}{}{}{}{}{}{Indivíduo que vende ou trabalha com ervas medicinais.}{her.ba.ná.rio}{0}
\verb{herbário}{}{}{}{}{s.m.}{Coleção de plantas, ou partes de plantas, secas e conservadas para estudo e classificação.}{her.bá.rio}{0}
\verb{herbário}{}{}{}{}{}{Lugar em que ficam guardadas essas plantas e coleções.}{her.bá.rio}{0}
\verb{herbicida}{}{}{}{}{s.m.}{Produto usado para eliminar ervas daninhas.  }{her.bi.ci.da}{0}
\verb{herbífero}{}{}{}{}{adj.}{Que produz ervas.}{her.bí.fe.ro}{0}
\verb{herbívoro}{}{Biol.}{}{}{adj.}{Que se alimenta de plantas.}{her.bí.vo.ro}{0}
\verb{herbolário}{}{}{}{}{s.m.}{Indivíduo que conhece, coleciona ou vende plantas medicinais; ervanário, herborista.}{her.bo.lá.rio}{0}
\verb{herbóreo}{}{}{}{}{adj.}{Relativo a erva; herbáceo.}{her.bó.re.o}{0}
\verb{herborista}{}{}{}{}{s.2g.}{Indivíduo que conhece, aprecia e vende plantas medicinais; ervanário, herbolário.}{her.bo.ris.ta}{0}
\verb{herborização}{}{}{"-ões}{}{s.f.}{Ato ou efeito de herborizar.}{her.bo.ri.za.ção}{0}
\verb{herborizar}{}{}{}{}{v.i.}{Colher ou colecionar plantas para estudo ou para aplicações medicinais.}{her.bo.ri.zar}{\verboinum{1}}
\verb{herboso}{ô}{}{"-osos ⟨ó⟩}{"-osa ⟨ó⟩}{adj.}{Que tem ervas; ervoso.}{her.bo.so}{0}
\verb{hercúleo}{}{}{}{}{adj.}{De ou digno de Hércules, semideus da mitologia grega.}{her.cú.leo}{0}
\verb{hercúleo}{}{Fig.}{}{}{}{Que é árduo ou difícil de se realizar.}{her.cú.leo}{0}
\verb{hercúleo}{}{}{}{}{}{Que tem força extraordinária; robusto, possante.}{her.cú.leo}{0}
\verb{hércules}{}{Fig.}{}{}{s.m.}{Indivíduo de força e valentia extraordinárias.}{hér.cu.les}{0}
\verb{herdade}{}{Lus.}{}{}{s.f.}{Grande propriedade rural, composta geralmente de plantações, gado e casa de moradia; quinta.}{her.da.de}{0}
\verb{herdar}{}{}{}{}{v.t.}{Receber por herança.}{her.dar}{0}
\verb{herdar}{}{}{}{}{}{Adquirir traços físicos ou morais por parentesco ou hereditariedade.}{her.dar}{0}
\verb{herdar}{}{Desus.}{}{}{}{Deixar por herança; legar.}{her.dar}{\verboinum{1}}
\verb{herdeiro}{ê}{}{}{}{s.m.}{Indivíduo que por direito de sucessão ou disposição testamentária herda bens, direitos ou obrigações; sucessor.}{her.dei.ro}{0}
\verb{herdeiro}{ê}{}{}{}{}{Indivíduo que herda, por parentesco ou consanguinidade, particularidades físicas ou morais.}{her.dei.ro}{0}
\verb{herdeiro}{ê}{Pop.}{}{}{}{Filho.}{her.dei.ro}{0}
\verb{hereditariedade}{}{}{}{}{s.f.}{Qualidade de hereditário; transmissão por herança.}{he.re.di.ta.ri.e.da.de}{0}
\verb{hereditário}{}{}{}{}{adj.}{Que se transmite por herança.}{he.re.di.tá.rio}{0}
\verb{herege}{é}{}{}{}{adj.2g.}{Que professa  ou sustenta heresia; heterodoxo.}{he.re.ge}{0}
\verb{herege}{é}{Por ext.}{}{}{}{Que sustenta ideias contrárias às geralmente admitidas.}{he.re.ge}{0}
\verb{herege}{é}{Pop.}{}{}{}{Que não pratica os deveres religiosos; ateu.}{he.re.ge}{0}
\verb{heresia}{}{}{}{}{s.f.}{Doutrina contrária à da Igreja; heterodoxia.}{he.re.si.a}{0}
\verb{heresia}{}{Por ext.}{}{}{}{Afirmação ou interpretação divergente dos princípios já estabelecidos por um sistema.}{he.re.si.a}{0}
\verb{heresia}{}{Pop.}{}{}{}{Opinião contrária ao senso comum; contrassenso, absurdo.}{he.re.si.a}{0}
\verb{heresiarca}{}{}{}{}{s.2g.}{Chefe ou fundador de seita herética. }{he.re.si.ar.ca}{0}
\verb{herético}{}{}{}{}{adj.}{Relativo a heresia; em que há heresia.}{he.ré.ti.co}{0}
\verb{herma}{é}{}{}{}{s.f.}{Escultura com cabeça, pescoço e parte do tronco do deus mitológico grego Hermes (Mercúrio na mitologia romana).}{her.ma}{0}
\verb{herma}{é}{Por ext.}{}{}{}{Busto com corte horizontal abaixo dos ombros.}{her.ma}{0}
\verb{hermafrodita}{}{Biol.}{}{}{adj.2g.}{Que tem órgãos reprodutores dos dois sexos; andrógino.}{her.ma.fro.di.ta}{0}
\verb{hermafroditismo}{}{Biol.}{}{}{s.m.}{Presença dos órgãos reprodutores dos dois sexos ou dos seus caracteres secundários num mesmo indivíduo, animal ou vegetal; androginia.}{her.ma.fro.di.tis.mo}{0}
\verb{hermafrodito}{}{}{}{}{adj.}{Hermafrodita.}{her.ma.fro.di.to}{0}
\verb{hermeneuta}{}{}{}{}{s.2g.}{Especialista na interpretação de leis e de textos antigos ou religiosos.}{her.me.neu.ta}{0}
\verb{hermenêutica}{}{}{}{}{s.f.}{Ciência ou arte da interpretação de textos antigos, religiosos ou filosóficos.}{her.me.nêu.ti.ca}{0}
\verb{hermenêutica}{}{}{}{}{}{Conjunto de princípios utilizados na interpretação de leis.}{her.me.nêu.ti.ca}{0}
\verb{hermenêutico}{}{}{}{}{adj.}{Relativo à hermenêutica.}{her.me.nêu.ti.co}{0}
\verb{hermético}{}{}{}{}{adj.}{Completamente fechado de modo a impedir a entrada e a saída de ar; selado.}{her.mé.ti.co}{0}
\verb{hermético}{}{}{}{}{}{Muito difícil de entender ou compreender; ininteligível, obscuro, impenetrável.}{her.mé.ti.co}{0}
\verb{hermetismo}{}{}{}{}{s.m.}{Qualidade do que é hermético; obscurantismo.}{her.me.tis.mo}{0}
\verb{hermetismo}{}{}{}{}{}{Estilo literário pouco acessível ao leitor comum.  }{her.me.tis.mo}{0}
\verb{hérnia}{}{Med.}{}{}{s.f.}{Passagem de um órgão ou de parte dele por um orifício em decorrência de um deslocamento anormal.}{hér.nia}{0}
\verb{hérnia}{}{Por ext.}{}{}{}{Excrescência, saliência, proeminência.}{hér.nia}{0}
\verb{hernial}{}{}{"-ais}{}{adj.2g.}{Relativo a hérnia.}{her.ni.al}{0}
\verb{hernioso}{ô}{}{"-osos ⟨ó⟩}{"-osa ⟨ó⟩}{adj.}{Que sofre de hérnia.}{her.ni.o.so}{0}
\verb{herniotomia}{}{Med.}{}{}{s.f.}{Intervenção cirúrgica de hérnia.}{her.ni.o.to.mi.a}{0}
\verb{herói}{}{}{}{heroína}{s.m.}{Indivíduo que se destaca por seus feitos ou por sua coragem.}{he.rói}{0}
\verb{herói}{}{}{}{heroína}{}{Indivíduo que suporta com ânimo elevado um grande sofrimento ou que arrisca sua vida em benefício de outrem.}{he.rói}{0}
\verb{herói}{}{}{}{heroína}{}{Personagem principal de uma obra literária, uma peça teatral, um filme ou uma novela.}{he.rói}{0}
\verb{herói}{}{}{}{heroína}{}{Indivíduo que atrai a atenção pública ou desperta admiração; ídolo.}{he.rói}{0}
\verb{herói}{}{}{}{heroína}{}{Na mitologia, filho de um deus ou uma deusa com um ser humano; semideus.}{he.rói}{0}
\verb{heroicidade}{}{}{}{}{s.f.}{Heroísmo.}{he.ro.i.ci.da.de}{0}
\verb{heroico}{ó}{}{}{}{adj.}{Relativo a ou próprio de herói; que denota heroísmo.}{he.roi.co}{0}
\verb{heroico}{ó}{}{}{}{}{Digno de um herói; que implica um grande esforço; estoico.}{he.roi.co}{0}
\verb{heroico}{ó}{Liter.}{}{}{}{Diz"-se do poema em que se celebram feitos heroicos; épico.}{he.roi.co}{0}
\verb{heroico}{ó}{Gram.}{}{}{}{Diz"-se de verso decassílabo comumente utilizado em poemas épicos.}{he.roi.co}{0}
\verb{herói"-cômico}{}{}{herói"-cômicos}{herói"-cômica}{adj.}{Que apresenta características simultaneamente heroicas e cômicas.}{he.rói"-cô.mi.co}{0}
\verb{herói"-cômico}{}{}{herói"-cômicos}{herói"-cômica}{}{Diz"-se do texto que aborda temas triviais e risíveis em tom de epopeia.}{he.rói"-cô.mi.co}{0}
\verb{heroificar}{}{}{}{}{v.t.}{Elevar à categoria de herói; glorificar, engrandecer.}{he.ro.i.fi.car}{\verboinum{2}}
\verb{heroína}{}{}{}{}{s.f.}{Mulher de valor excepcional, que se destaca por suas realizações.}{he.ro.í.na}{0}
\verb{heroína}{}{}{}{}{}{Principal personagem feminina de uma obra de literatura, uma peça teatral, um filme, uma novela.}{he.ro.í.na}{0}
\verb{heroína}{}{Quím.}{}{}{s.f.}{Substância tóxica derivada da morfina, com efeitos narcóticos e analgésicos mais potentes do que os dessa.}{he.ro.í.na}{0}
\verb{heroísmo}{}{}{}{}{s.m.}{Qualidade de herói ou do que é heroico.}{he.ro.ís.mo}{0}
\verb{heroísmo}{}{}{}{}{}{Grandeza da alma; magnanimidade, generosidade, heroicidade.}{he.ro.ís.mo}{0}
\verb{heroísmo}{}{}{}{}{}{Ato ou feito heroico.}{he.ro.ís.mo}{0}
\verb{herpes}{é}{Med.}{}{}{s.m.}{Afecções viróticas da pele caracterizadas pela formação de pequenas vesículas que, ao se romperem, provocam dor e coceira.}{her.pes}{0}
\verb{herpes}{é}{Fig.}{}{}{}{Mal contagioso; podridão, estrago. }{her.pes}{0}
\verb{herpes"-zóster}{é}{Med.}{herpes"-zósteres}{}{s.m.}{Inflamação virótica aguda que ataca os gânglios sensitivos da espinha dorsal e dos nervos cranianos e que apresenta sintomas como formação de vesículas na pele ou nas mucosas e dores nevrálgicas.}{her.pes"-zós.ter}{0}
\verb{herpetografia}{}{Zool.}{}{}{s.f.}{Descrição ou estudo dos répteis; herpetologia.}{her.pe.to.gra.fi.a}{0}
\verb{herpetologia}{}{}{}{}{s.f.}{Ramo da Zoologia que trata dos répteis.}{her.pe.to.lo.gi.a}{0}
\verb{herpetologia}{}{}{}{}{}{Estudo acerca dos répteis; herpetografia.}{her.pe.to.lo.gi.a}{0}
\verb{herpetologia}{}{Med.}{}{}{s.f.}{Estudo ou tratado acerca dos herpes.}{her.pe.to.lo.gi.a}{0}
\verb{hertz}{}{Fís.}{}{}{s.m.}{Unidade de medida de frequência, equivalente à frequência de um fenômeno cíclico cujo período dura um segundo. Símb.: Hz.}{hertz}{0}
\verb{hertziano}{}{}{}{}{adj.}{Relativo à faixa de frequência das ondas eletromagnéticas de rádio.}{hert.zi.a.no}{0}
\verb{hesitação}{}{}{"-ões}{}{s.f.}{Ato ou efeito de hesitar, vacilar.}{he.si.ta.ção}{0}
\verb{hesitação}{}{}{"-ões}{}{}{Indecisão, vacilação, dúvida.}{he.si.ta.ção}{0}
\verb{hesitante}{}{}{}{}{adj.2g.}{Que hesita; indeciso, vacilante. }{he.si.tan.te}{0}
\verb{hesitar}{}{}{}{}{v.t.}{Demonstrar dúvida ou insegurança sobre; vacilar em. }{he.si.tar}{0}
\verb{hesitar}{}{}{}{}{v.i.}{Estar ou ficar indeciso, inseguro, incerto.}{he.si.tar}{\verboinum{1}}
\verb{heteróclito}{}{}{}{}{adj.}{Que se afasta das normas ou dos princípios estabelecidos; extravagante, excêntrico.}{he.te.ró.cli.to}{0}
\verb{heteróclito}{}{Gram.}{}{}{}{Que se desvia dos padrões ou paradigmas gramaticais.}{he.te.ró.cli.to}{0}
\verb{heterocromia}{}{}{}{}{s.f.}{Coloração distinta de partes de uma estrutura que, normalmente, deveriam ter a mesma cor.}{he.te.ro.cro.mi.a}{0}
\verb{heterodoxia}{cs}{}{}{}{s.f.}{Qualidade ou condição do que se opõe a normas ou padrões estabelecidos; dissidência.}{he.te.ro.do.xi.a}{0}
\verb{heterodoxia}{cs}{Relig.}{}{}{}{Oposição aos princípios ou dogmas de uma religião; heresia.}{he.te.ro.do.xi.a}{0}
\verb{heterodoxo}{ócs}{}{}{}{adj.}{Que se opõe às normas ou aos padrões estabelecidos; dissidente.}{he.te.ro.do.xo}{0}
\verb{heterodoxo}{ócs}{}{}{}{}{Que se opõe aos princípios ou dogmas de uma religião.}{he.te.ro.do.xo}{0}
%\verb{heterofonia}{}{}{}{}{}{0}{he.te.ro.fo.ni.a}{0}
\verb{heterófono}{}{Gram.}{}{}{adj.}{Diz"-se de palavra que tem a mesma grafia que uma outra mas que tem pronúncia e significado diferentes.   }{he.te.ró.fo.no}{0}
\verb{heterogeneidade}{}{}{}{}{s.f.}{Qualidade ou condição do que é heterogêneo, do que não é uniforme ou não se pode fundir em um todo.}{he.te.ro.ge.nei.da.de}{0}
\verb{heterogêneo}{}{}{}{}{adj.}{De diferente natureza ou estrutura (em relação a outra coisa).}{he.te.ro.gê.neo}{0}
\verb{heterogêneo}{}{}{}{}{}{Constituído por fases ou elementos variados.}{he.te.ro.gê.neo}{0}
\verb{heterogêneo}{}{}{}{}{}{Que não é uniforme ou homogêneo.}{he.te.ro.gê.neo}{0}
\verb{heteromorfo}{ó}{}{}{}{adj.}{Que apresenta diversidade de formas ou modificação na forma comum.}{he.te.ro.mor.fo}{0}
\verb{heteromorfose}{ó}{Biol.}{}{}{s.f.}{Regeneração de um membro, órgão ou estrutura, com diferença na forma e no tamanho em relação à parte que foi perdida.}{he.te.ro.mor.fo.se}{0}
\verb{heteronímia}{}{Gram.}{}{}{s.f.}{Emprego de palavras com radicais diferentes para opor categorias gramaticais distintas como a indicação do gênero em \textit{boi} e \textit{vaca}.}{he.te.ro.ní.mia}{0}
\verb{heterônimo}{}{Gram.}{}{}{adj.}{Diz"-se da palavra que se distingue de outra por relação de heteronímia.}{he.te.rô.ni.mo}{0}
\verb{heterônimo}{}{Liter.}{}{}{s.m.}{Nome fictício com que um autor assina uma obra, criando qualidades próprias a esse autor imaginário, diferentes das suas.}{he.te.rô.ni.mo}{0}
\verb{heterossexual}{cs}{}{"-ais}{}{adj.2g.}{Relativo à afinidade ou ao comportamento sexual entre indivíduos de sexos diferentes.}{he.te.ros.se.xu.al}{0}
\verb{heterossexual}{cs}{}{"-ais}{}{s.2g.}{Indivíduo que se sente atraído física, emocional e espiritualmente por pessoas de sexo diferente do seu.}{he.te.ros.se.xu.al}{0}
\verb{heterossexualidade}{cs}{}{}{}{s.f.}{Caráter ou qualidade de heterossexual; heterossexualismo.}{he.te.ros.se.xu.a.li.da.de}{0}
\verb{heterossexualismo}{cs}{}{}{}{s.m.}{Desejo ou prática sexual entre indivíduos de sexo diferente; heterossexualidade.}{he.te.ros.se.xu.a.lis.mo}{0}
\verb{heterozigoto}{ô}{Biol.}{}{}{adj.}{Que apresenta, num determinado par de genes, dois alelos com determinação para características diferentes.}{he.te.ro.zi.go.to}{0}
\verb{heterozigoto}{ô}{}{}{}{s.m.}{Esse indivíduo.}{he.te.ro.zi.go.to}{0}
\verb{heureca}{é}{}{}{}{interj.}{Expressão que denota satisfação diante de uma descoberta ou da resolução de um problema difícil.}{heu.re.ca}{0}
\verb{heurística}{}{}{}{}{s.f.}{Conjunto de métodos e regras que levam à descoberta de fatos ou à resolução de problemas.}{heu.rís.ti.ca}{0}
\verb{hexacampeão}{cs/ ou /z}{}{"-ões}{}{adj.}{Diz"-se do indivíduo, equipe ou clube campeão pela sexta vez. }{he.xa.cam.pe.ão}{0}
\verb{hexaedro}{cs\ldots{}é}{Geom.}{}{}{s.m.}{Poliedro de seis faces.}{he.xa.e.dro}{0}
\verb{hexagonal}{cs/ ou /z}{}{"-ais}{}{adj.2g.}{Que tem seis ângulos. }{he.xa.go.nal}{0}
\verb{hexagonal}{cs/ ou /z}{}{"-ais}{}{}{Que tem forma de hexágono. }{he.xa.go.nal}{0}
\verb{hexágono}{cs/ ou /z}{Geom.}{}{}{s.m.}{Polígono que tem seis lados.}{he.xá.go.no}{0}
\verb{hexassílabo}{cs}{Gram.}{}{}{adj.}{Que tem seis sílabas.}{he.xas.sí.la.bo}{0}
\verb{Hf}{}{Quím.}{}{}{}{Símb. do \textit{háfnio}.}{Hf}{0}
\verb{Hg}{}{Quím.}{}{}{}{Símb. do \textit{mercúrio}.}{Hg}{0}
\verb{hialino}{}{}{}{}{adj.}{Relativo ao vidro.}{hi.a.li.no}{0}
\verb{hialino}{}{}{}{}{}{Que é semelhante ao vidro em aspecto ou transparência; translúcido, claro.  }{hi.a.li.no}{0}
\verb{hialografia}{}{}{}{}{s.f.}{Arte ou técnica de gravação sobre o vidro.}{hi.a.lo.gra.fi.a}{0}
\verb{hialoide}{}{}{}{}{adj.2g.}{Que é semelhante ao vidro; hialino.}{hi.a.loi.de}{0}
\verb{hialoide}{}{Anat.}{}{}{s.f.}{Membrana que contém o humor vítreo do olho.}{hi.a.loi.de}{0}
\verb{hiato}{}{Gram.}{}{}{s.m.}{Grupo de duas vogais contíguas pertencentes a sílabas diferentes.}{hi.a.to}{0}
\verb{hiato}{}{Anat.}{}{}{}{Abertura ou fenda no corpo humano.}{hi.a.to}{0}
\verb{hiato}{}{Fig.}{}{}{}{Abertura, lacuna, intervalo.}{hi.a.to}{0}
\verb{hibernação}{}{}{"-ões}{}{s.f.}{Estado de letargia, torpor, inatividade.}{hi.ber.na.ção}{0}
\verb{hibernação}{}{Biol.}{"-ões}{}{}{Estado de entorpecimento ou inatividade comum em muitos animais durante o inverno.}{hi.ber.na.ção}{0}
\verb{hibernal}{}{}{"-ais}{}{adj.2g.}{Que se refere ao ou é próprio do inverno.}{hi.ber.nal}{0}
\verb{hibernar}{}{}{}{}{v.i.}{Passar o inverno (certos animais, como o urso) dormindo ou em estado de letargia, para que o organismo gaste o mínimo possível de energia e economize o máximo de reservas alimentares até a chegada da primavera.}{hi.ber.nar}{\verboinum{1}}
\verb{hibisco}{}{Bot.}{}{}{s.m.}{Nome comum a diversas plantas, como a malva, cultivadas como ornamentais, para fins medicinais e para a extração de fibras têxteis.}{hi.bis.co}{0}
\verb{hibridação}{}{Biol.}{"-ões}{}{s.f.}{Cruzamento natural ou artificial de plantas ou animais de diferentes espécies.}{hi.bri.da.ção}{0}
\verb{hibridez}{ê}{}{}{}{s.f.}{Qualidade do que é híbrido.}{hi.bri.dez}{0}
\verb{hibridez}{ê}{}{}{}{}{Irregularidade, anormalidade, anomalia, hibridismo.}{hi.bri.dez}{0}
\verb{hibridismo}{}{}{}{}{s.m.}{Hibridez.}{hi.bri.dis.mo}{0}
\verb{hibridismo}{}{Gram.}{}{}{}{Formação de palavras com elementos ou radicais de línguas diferentes.}{hi.bri.dis.mo}{0}
\verb{híbrido}{}{Biol.}{}{}{adj.}{Diz"-se do animal ou vegetal originário do cruzamento de espécies diferentes.}{hí.bri.do}{0}
\verb{híbrido}{}{Gram.}{}{}{}{Diz"-se de palavra formada com elementos ou radicais de línguas diferentes.}{hí.bri.do}{0}
\verb{hidra}{}{Zool.}{}{}{s.f.}{Animal celenterado, solitário, que vive em água doce.}{hi.dra}{0}
\verb{hidra}{}{Astron.}{}{}{}{Constelação com forma de serpente que se estende pelos hemisférios norte e sul. (Nesta acepção, com letra maiúscula.)}{hi.dra}{0}
\verb{hidra}{}{Fig.}{}{}{}{Perigo incessantemente renovado que alude à figura da Hidra de Lerna, serpente mitológica que possuía sete cabeças que renasciam quando cortadas.}{hi.dra}{0}
\verb{hidrácido}{}{Quím.}{}{}{s.m.}{Substância ácida que não contém oxigênio.}{hi.drá.ci.do}{0}
\verb{hidramático}{}{}{}{}{adj.}{Diz"-se da mudança de marchas de um veículo que é acionada automaticamente por um sistema hidráulico.}{hi.dra.má.ti.co}{0}
\verb{hidrante}{}{}{}{}{s.m.}{Válvula ou torneira de saída de água a que se acopla uma mangueira utilizada para extinção de incêndios.}{hi.dran.te}{0}
\verb{hidrargirismo}{}{Med.}{}{}{s.m.}{Intoxicação causada pelo mercúrio.}{hi.drar.gi.ris.mo}{0}
\verb{hidratação}{}{}{"-ões}{}{s.f.}{Ato ou efeito de hidratar.}{hi.dra.ta.ção}{0}
\verb{hidratação}{}{Quím.}{"-ões}{}{}{Introdução de água em uma substância química.}{hi.dra.ta.ção}{0}
\verb{hidratação}{}{Quím.}{"-ões}{}{}{Transformação em hidrato.}{hi.dra.ta.ção}{0}
\verb{hidratado}{}{}{}{}{adj.}{Que foi tratado com água.}{hi.dra.ta.do}{0}
\verb{hidratado}{}{}{}{}{}{Misturado ou combinado com água.}{hi.dra.ta.do}{0}
\verb{hidratante}{}{}{}{}{adj.2g.}{Diz"-se da substância que hidrata.}{hi.dra.tan.te}{0}
\verb{hidratar}{}{}{}{}{v.t.}{Tratar por meio da água.}{hi.dra.tar}{0}
\verb{hidratar}{}{}{}{}{}{Impregnar de água.}{hi.dra.tar}{0}
\verb{hidratar}{}{}{}{}{}{Converter em hidrato.}{hi.dra.tar}{\verboinum{1}}
\verb{hidratável}{}{}{"-eis}{}{adj.2g.}{Que pode ser hidratado. }{hi.dra.tá.vel}{0}
\verb{hidrato}{}{Quím.}{}{}{s.m.}{Composto com moléculas de água.}{hi.dra.to}{0}
\verb{hidráulica}{}{Fís.}{}{}{s.f.}{Estudo das leis que regem o comportamento estático e dinâmico dos líquidos, principalmente da água, para sua aplicação em engenharia.}{hi.dráu.li.ca}{0}
\verb{hidráulico}{}{}{}{}{adj.}{Relativo à hidráulica.}{hi.dráu.li.co}{0}
\verb{hidráulico}{}{}{}{}{}{Que funciona ou se movimenta por meio de um líquido.}{hi.dráu.li.co}{0}
\verb{hidravião}{}{}{}{}{}{Var. de \textit{hidroavião}.}{hi.dra.vi.ão}{0}
\verb{hidrelétrica}{}{}{}{}{}{Var. de \textit{hidroelétrica}.}{hi.dre.lé.tri.ca}{0}
\verb{hidrelétrico}{}{}{}{}{}{Var. de \textit{hidroelétrico}.}{hi.dre.lé.tri.co}{0}
\verb{hídrico}{}{}{}{}{adj.}{Relativo à água.}{hí.dri.co}{0}
\verb{hidroavião}{}{}{"-ões}{}{s.m.}{Avião munido de flutuadores que lhe permitem decolar da água e pousar sobre ela; hidroplano.}{hi.dro.a.vi.ão}{0}
\verb{hidrocarbonato}{}{Desus.}{}{}{s.m.}{Bicarbonato.}{hi.dro.car.bo.na.to}{0}
\verb{hidrocarboneto}{ê}{Quím.}{}{}{s.m.}{Designação dos compostos formados por carbono e hidrogênio. }{hi.dro.car.bo.ne.to}{0}
\verb{hidrocarbônico}{}{Quím.}{}{}{adj.}{Relativo a hidrocarboneto.}{hi.dro.car.bô.ni.co}{0}
\verb{hidrocefalia}{}{Med.}{}{}{s.f.}{Acúmulo anormal de fluido cerebral no crânio, levando à dilatação dos ventrículos, aumento da cabeça e deficiência mental.}{hi.dro.ce.fa.li.a}{0}
\verb{hidrocéfalo}{}{}{}{}{adj.}{Que sofre de hidrocefalia.}{hi.dro.cé.fa.lo}{0}
\verb{hidrocele}{é}{Med.}{}{}{s.f.}{Acúmulo de líquido, especialmente na membrana que envolve os testículos ou ao longo do cordão espermático.}{hi.dro.ce.le}{0}
\verb{hidrodinâmica}{}{Fís.}{}{}{s.f.}{Estudo das leis que regem o movimento, o equilíbrio e o peso dos líquidos.}{hi.dro.di.nâ.mi.ca}{0}
\verb{hidrodinâmico}{}{}{}{}{adj.}{Relativo à hidrodinâmica.}{hi.dro.di.nâ.mi.co}{0}
\verb{hidrodinâmico}{}{}{}{}{}{Que diminui a resistência à água.}{hi.dro.di.nâ.mi.co}{0}
\verb{hidroelétrica}{}{}{}{}{s.f.}{Empresa que produz energia elétrica utilizando a força hidráulica; usina hidroelétrica.}{hi.dro.e.lé.tri.ca}{0}
\verb{hidroelétrico}{}{}{}{}{adj.}{Diz"-se da corrente elétrica gerada pela força dos cursos d'água. }{hi.dro.e.lé.tri.co}{0}
\verb{hidrófilo}{}{}{}{}{adj.}{Que absorve bem a água.}{hi.dró.fi.lo}{0}
\verb{hidrófilo}{}{}{}{}{}{Que gosta de água.}{hi.dró.fi.lo}{0}
\verb{hidrófito}{}{Bot.}{}{}{adj.}{Diz"-se do vegetal adaptado a viver na água, seja submerso ou flutuante.}{hi.dró.fi.to}{0}
\verb{hidrofobia}{}{}{}{}{s.f.}{Horror ou aversão aos líquidos.}{hi.dro.fo.bi.a}{0}
\verb{hidrofobia}{}{Med.}{}{}{}{Doença infecciosa causada por vírus, transmitida pela mordida de animais infectados, como cão, lobo e gato, e que acomete o sistema nervoso central, podendo provocar parada respiratória e convulsões; raiva.}{hi.dro.fo.bi.a}{0}
\verb{hidrófobo}{}{}{}{}{adj.}{Que tem horror a água ou a quaisquer líquidos.}{hi.dró.fo.bo}{0}
\verb{hidrófobo}{}{}{}{}{}{Que sofre de hidrofobia, raiva.}{hi.dró.fo.bo}{0}
\verb{hidrogenação}{}{Quím.}{"-ões}{}{s.f.}{Fixação de hidrogênio em substâncias, utilizada na petroquímica, no tratamento de óleos alimentícios, entre outros.}{hi.dro.ge.na.ção}{0}
\verb{hidrogenado}{}{}{}{}{adj.}{Que contém hidrogênio.}{hi.dro.ge.na.do}{0}
\verb{hidrogenado}{}{}{}{}{}{Combinado com hidrogênio.}{hi.dro.ge.na.do}{0}
\verb{hidrogenar}{}{}{}{}{v.t.}{Fixar hidrogênio em uma substância.}{hi.dro.ge.nar}{\verboinum{1}}
\verb{hidrogênio}{}{Quím.}{}{}{s.m.}{Elemento gasoso, pouco denso e incolor; é o mais leve, simples e abundante do universo; utilizado na fabricação do gás combustível para fins domésticos, na hidrogenação de óleos comestíveis, na fabricação de metanol, entre outros. \elemento{1}{1.00794}{H}.}{hi.dro.gê.nio}{0}
\verb{hidrogeologia}{}{Geol.}{}{}{s.f.}{Parte da geologia que estuda a ocorrência e a utlização de águas subterrâneas.}{hi.dro.ge.o.lo.gi.a}{0}
\verb{hidroginástica}{}{}{}{}{s.f.}{Ginástica que se pratica na água. }{hi.dro.gi.nás.ti.ca}{0}
\verb{hidrografia}{}{}{}{}{s.f.}{Conjunto das águas correntes e estáveis de uma região.}{hi.dro.gra.fi.a}{0}
\verb{hidrografia}{}{}{}{}{}{Parte da geografia que estuda as águas correntes, paradas, oceânicas e subterrâneas.}{hi.dro.gra.fi.a}{0}
\verb{hidrográfico}{}{}{}{}{adj.}{Relativo à hidrografia.}{hi.dro.grá.fi.co}{0}
\verb{hidrógrafo}{}{}{}{}{s.m.}{Especialista em hidrografia.}{hi.dró.gra.fo}{0}
\verb{hidrolisar}{}{}{}{}{v.t.}{Fazer a hidrólise de; decompor uma substância por processo de reação com a água.}{hi.dro.li.sar}{\verboinum{1}}
\verb{hidrólise}{}{Quím.}{}{}{s.f.}{Decomposição ou alteração de uma substância por reação com a água.}{hi.dró.li.se}{0}
\verb{hidrologia}{}{}{}{}{s.f.}{Ciência que estuda a ocorrência, a distribuição, a circulação e as propriedades mecânicas, físicas e químicas da água existente na natureza, em seus diferentes estados.}{hi.dro.lo.gi.a}{0}
\verb{hidrólogo}{}{}{}{}{s.m.}{Especialista em hidrologia.}{hi.dró.lo.go}{0}
\verb{hidromassagem}{}{}{"-ens}{}{s.f.}{Massagem feita por meio de jatos de  água.}{hi.dro.mas.sa.gem}{0}
\verb{hidromecânica}{}{}{}{}{s.f.}{Estudo do equilíbrio e dos movimentos dos líquidos em relação a si mesmos ou em relação a sólidos imersos. }{hi.dro.me.câ.ni.ca}{0}
\verb{hidromecânico}{}{}{}{}{adj.}{Que utiliza a água como força de movimento.}{hi.dro.me.câ.ni.co}{0}
\verb{hidromel}{é}{}{"-éis}{}{s.m.}{Mistura de água e mel, geralmente utilizada como laxante.}{hi.dro.mel}{0}
\verb{hidrometria}{}{}{}{}{s.f.}{Medição da velocidade ou da energia dos líquidos em movimento, particularmente da água.}{hi.dro.me.tri.a}{0}
\verb{hidrômetro}{}{}{}{}{s.m.}{Instrumento com que se mede a quantidade de água consumida nos imóveis.}{hi.drô.me.tro}{0}
\verb{hidromineral}{}{}{"-ais}{}{adj.2g.}{Relativo a águas minerais.}{hi.dro.mi.ne.ral}{0}
\verb{hidromotor}{ô}{}{}{}{s.m.}{Motor acionado pelo empuxo ou pelo peso da água.}{hi.dro.mo.tor}{0}
\verb{hidropatia}{}{}{}{}{s.f.}{Tratamento de doenças por meio do uso abundante da água.}{hi.dro.pa.ti.a}{0}
\verb{hidrópico}{}{}{}{}{adj.}{Que sofre de inchação ou retenção de líquidos.}{hi.dró.pi.co}{0}
\verb{hidropisia}{}{Med.}{}{}{s.f.}{Acúmulo anormal de líquido seroso em tecido celular ou em alguma cavidade do corpo.}{hi.dro.pi.si.a}{0}
\verb{hidroplano}{}{}{}{}{s.m.}{Hidroavião.}{hi.dro.pla.no}{0}
\verb{hidropônica}{}{}{}{}{s.f.}{Técnica de cultivo de vegetais que consiste em utilizar a água como meio de sustentação.  }{hi.dro.pô.ni.ca}{0}
\verb{hidrosfera}{é}{}{}{}{s.f.}{Conjunto das águas que cobrem a superfície terrestre, abrangendo oceanos, rios, lagos, mares, geleiras e calotas polares.}{hi.dros.fe.ra}{0}
\verb{hidrossolúvel}{}{}{"-eis}{}{adj.2g.}{Diz"-se de substância solúvel em água.}{hi.dros.so.lú.vel}{0}
\verb{hidrostática}{}{Fís.}{}{}{s.f.}{Estudo da pressão e do equilíbrio dos líquidos e dos gases sujeitos à gravidade.}{hi.dros.tá.ti.ca}{0}
\verb{hidróstato}{}{Fís.}{}{}{s.m.}{Instrumento de metal, flutuante, utilizado para medir o peso dos corpos.}{hi.drós.ta.to}{0}
\verb{hidrotecnia}{}{Fís.}{}{}{s.f.}{Parte da mecânica que trata da condução e distribuição das águas.}{hi.dro.tec.ni.a}{0}
\verb{hidroterapia}{}{}{}{}{s.f.}{Uso da água para tratamento de doenças por meio de banhos, duchas, compressas, entre outros.}{hi.dro.te.ra.pi.a}{0}
\verb{hidroterápico}{}{}{}{}{adj.}{Relativo à hidroterapia, ao uso da água com fins terapêuticos.}{hi.dro.te.rá.pi.co}{0}
\verb{hidrotropismo}{}{Biol.}{}{}{s.m.}{Mudança de orientação de um organismo vegetal ou de parte dele sob estímulo da água.}{hi.dro.tro.pis.mo}{0}
\verb{hidrovia}{}{}{}{}{s.f.}{Via marítima, fluvial ou lacustre utilizada para o transporte e as comunicações.}{hi.dro.vi.a}{0}
\verb{hidroviário}{}{}{}{}{adj.}{Relativo ao transporte feito por vias navegáveis.}{hi.dro.vi.á.rio}{0}
\verb{hidróxido}{cs}{Quím.}{}{}{s.m.}{Composto de um óxido metálico com a água.}{hi.dró.xi.do}{0}
\verb{hidroxila}{cs}{Quím.}{}{}{s.f.}{Grupamento monovalente negativo, composto por um átomo de oxigênio e um de hidrogênio, presente nos hidróxidos e nas bases.}{hi.dro.xi.la}{0}
\verb{hiena}{}{Zool.}{}{}{s.f.}{Mamífero carnívoro, noturno, encontrado na Ásia e na África, que se alimenta principalmente de animais indefesos ou caçados por outros animais.}{hi.e.na}{0}
\verb{hiena}{}{Fig.}{}{}{}{Indivíduo de índole vil, desprezível.}{hi.e.na}{0}
\verb{hierarquia}{}{}{}{}{s.f.}{Ordem e distribuição de poderes ou categorias eclesiásticas, militares e civis com subordinação de um indivíduo a outro.}{hi.e.rar.qui.a}{0}
\verb{hierarquia}{}{}{}{}{}{Qualquer classificação baseada nas relações entre superiores e dependentes.}{hi.e.rar.qui.a}{0}
\verb{hierarquia}{}{}{}{}{}{Ordenação, escala, graduação.}{hi.e.rar.qui.a}{0}
\verb{hierárquico}{}{}{}{}{adj.}{Relativo à hierarquia.}{hi.e.rár.qui.co}{0}
\verb{hierarquização}{}{}{"-ões}{}{s.f.}{Ato ou efeito de hierarquizar; graduação, ordenação.}{hi.e.rar.qui.za.ção}{0}
\verb{hierarquizar}{}{}{}{}{v.t.}{Organizar ou distribuir de acordo com uma ordem hierárquica.}{hi.e.rar.qui.zar}{\verboinum{1}}
\verb{hierático}{}{}{}{}{adj.}{Relativo às coisas sagradas, sacerdotais, religiosas.}{hi.e.rá.ti.co}{0}
\verb{hierático}{}{}{}{}{}{Diz"-se da escrita cursiva com que os antigos egípcios simplificavam a escritura hieroglífica, utilizada nos manuscritos.}{hi.e.rá.ti.co}{0}
\verb{hierofante}{}{}{}{}{s.m.}{Indivíduo que se inculca conhecedor de ciências ocultas ou de mistérios; adivinho.}{hi.e.ro.fan.te}{0}
\verb{hierofante}{}{Relig.}{}{}{}{Na antiga Grécia, sacerdote que presidia aos mistérios das religiões, instruindo os futuros iniciados e apresentando"-lhes os objetos sagrados.}{hi.e.ro.fan.te}{0}
\verb{hierofante}{}{Relig.}{}{}{}{Na Roma antiga, denominação dada ao chefe supremo de uma religião.}{hi.e.ro.fan.te}{0}
\verb{hieroglífico}{}{}{}{}{adj.}{Relativo aos hieróglifos.}{hi.e.ro.glí.fi.co}{0}
\verb{hieroglífico}{}{Fig.}{}{}{}{Ilegível, ininteligível, enigmático.}{hi.e.ro.glí.fi.co}{0}
\verb{hieroglifo}{}{}{}{}{}{Var. de \textit{hieróglifo}.}{hi.e.ro.gli.fo}{0}
\verb{hieróglifo}{}{}{}{}{s.m.}{Ideograma fundamental do sistema de escrita dos antigos egípcios.}{hi.e.ró.gli.fo}{0}
\verb{hieróglifo}{}{Fig.}{}{}{}{Escrita ou figura enigmática, indecifrável, ininteligível.}{hi.e.ró.gli.fo}{0}
\verb{hierosolimita}{}{}{}{}{adj.2g.}{Relativo a Jerusalém.}{hi.e.ro.so.li.mi.ta}{0}
\verb{hierosolimita}{}{}{}{}{s.2g.}{Indivíduo natural ou habitante dessa cidade. }{hi.e.ro.so.li.mi.ta}{0}
\verb{hierosolimitano}{}{}{}{}{adj. e s.m.  }{Hierosolimita.}{hi.e.ro.so.li.mi.ta.no}{0}
\verb{hífen}{}{Gram.}{}{}{s.m.}{Sinal gráfico usado para separar sílabas em final de linha, ligar os elementos de uma palavra composta ou pronomes átonos a verbos; traço"-de"-união.}{hí.fen}{0}
\verb{hifenização}{}{Gram.}{"-ões}{}{s.f.}{Ato ou efeito de hifenizar; ligação ou separação de palavras por meio de hífen.}{hi.fe.ni.za.ção}{0}
\verb{hifenizar}{}{Gram.}{}{}{v.t.}{Unir ou separar palavras por meio de hífen.}{hi.fe.ni.zar}{\verboinum{1}}
\verb{higidez}{ê}{}{}{}{s.f.}{Estado de perfeita saúde.}{hi.gi.dez}{0}
\verb{hígido}{}{}{}{}{adj.}{Relativo à saúde.}{hí.gi.do}{0}
\verb{hígido}{}{}{}{}{}{Que goza de perfeita saúde; sadio, saudável.}{hí.gi.do}{0}
\verb{higiene}{ê}{}{}{}{s.f.}{Conjunto de práticas ou hábitos que conduzem ao bem"-estar e à preservação da saúde; asseio, limpeza.}{hi.gi.e.ne}{0}
\verb{higiene}{ê}{Med.}{}{}{}{Parte da medicina que visa à preservação da saúde e à prevenção de doenças.}{hi.gi.e.ne}{0}
\verb{higiênico}{}{}{}{}{adj.}{Relativo à higiene.}{hi.gi.ê.ni.co}{0}
\verb{higiênico}{}{}{}{}{}{Que revela limpeza; asseado, limpo.}{hi.gi.ê.ni.co}{0}
\verb{higienista}{}{}{}{}{s.2g.}{Especialista em assuntos de higiene; sanitarista.}{hi.gi.e.nis.ta}{0}
\verb{higienizar}{}{}{}{}{v.t.}{Tornar saudável, higiênico, limpo.}{hi.gi.e.ni.zar}{\verboinum{1}}
\verb{higrógrafo}{}{}{}{}{s.m.}{Instrumento que mede a umidade da atmosfera.}{hi.gró.gra.fo}{0}
\verb{higrologia}{}{}{}{}{s.f.}{Estudo da umidade atmosférica.}{hi.gro.lo.gi.a}{0}
\verb{higrometria}{}{}{}{}{s.f.}{Estudo da umidade atmosférica e dos processos e das técnicas de sua medição.}{hi.gro.me.tri.a}{0}
\verb{higrométrico}{}{}{}{}{adj.}{Relativo à higrometria.}{hi.gro.mé.tri.co}{0}
\verb{higrômetro}{}{}{}{}{s.m.}{Denominação genérica dos instrumentos que medem a umidade do ar ou de um gás.}{hi.grô.me.tro}{0}
\verb{higroscópio}{}{}{}{}{s.m.}{Tipo de higrômetro que mede a umidade do ar sem muita precisão.}{hi.gros.có.pio}{0}
\verb{hilariante}{}{}{}{}{adj.2g.}{Que provoca riso, alegria.}{hi.la.ri.an.te}{0}
\verb{hilaridade}{}{}{}{}{s.f.}{Alegria; vontade de rir.}{hi.la.ri.da.de}{0}
\verb{hilário}{}{}{}{}{adj.}{Risonho, alegre; hilariante.}{hi.lá.rio}{0}
\verb{hilarizar}{}{}{}{}{v.t.}{Alegrar, contentar.}{hi.la.ri.zar}{\verboinum{1}}
\verb{hileia}{é}{}{}{}{s.f.}{A floresta amazônica.}{hi.lei.a}{0}
\verb{hilota}{ó}{}{}{}{s.2g.}{Pessoa reduzida à miséria e à ínfima condição social.}{hi.lo.ta}{0}
\verb{hímen}{}{}{himens \textit{ou} hímenes}{}{s.m.}{Prega membranosa que fecha total ou parcialmente o orifício vaginal.}{hí.men}{0}
\verb{himeneu}{}{}{}{}{s.m.}{Casamento, matrimônio.}{hi.me.neu}{0}
\verb{himeneu}{}{}{}{}{}{Festa de núpcias.}{hi.me.neu}{0}
\verb{himenóptero}{}{Zool.}{}{}{s.m.}{Espécime dos himenópteros, ordem de insetos com quatro asas membranosas e aparelho bucal mastigador em que se incluem as abelhas, as formigas, as vespas, entre outros.}{hi.me.nóp.te.ro}{0}
\verb{himenóptero}{}{Zool.}{}{}{adj.}{Relativo aos himenópteros.}{hi.me.nóp.te.ro}{0}
\verb{hinário}{}{}{}{}{s.m.}{Coleção de hinos.}{hi.ná.rio}{0}
\verb{hinário}{}{}{}{}{}{Livro que contém hinos religiosos.}{hi.ná.rio}{0}
\verb{hindi}{}{}{}{}{s.m.}{Língua falada no norte da Índia, que se destaca como uma das línguas vernáculas mais importantes. }{hin.di}{0}
\verb{hindu}{}{}{}{}{adj.2g.}{Que é próprio da Índia; indiano.}{hin.du}{0}
\verb{hindu}{}{}{}{}{}{Relativo ao hinduísmo; hinduísta.}{hin.du}{0}
\verb{hindu}{}{}{}{}{}{Indivíduo que segue o hinduísmo; hinduísta.}{hin.du}{0}
\verb{hindu}{}{}{}{}{s.2g.}{Indivíduo natural ou habitante da Índia; indiano.}{hin.du}{0}
\verb{hinduísmo}{}{}{}{}{s.m.}{Religião originada a partir do vedismo e do bramanismo, seguida pela maioria dos povos indianos.}{hin.du.ís.mo}{0}
\verb{hinduísta}{}{}{}{}{adj.2g.}{Relativo ou pertencente ao hinduísmo; hindu.}{hin.du.ís.ta}{0}
\verb{hindustani}{}{}{}{}{s.m.}{Uma das línguas nacionais da Índia.}{hin.dus.ta.ni}{0}
\verb{hino}{}{}{}{}{s.m.}{Poema, música ou cântico de veneração, geralmente de tema sacro, marcial ou patriótico.}{hi.no}{0}
\verb{hioide}{}{Anat.}{}{}{s.m.}{Pequeno osso da parte anterior do pescoço.}{hi.oi.de}{0}
%\verb{hip}{}{}{}{}{}{0}{hip}{0}
\verb{hiperacidez}{ê}{}{}{}{s.f.}{Qualidade ou estado do que contém teor de acidez acima do normal. }{hi.pe.ra.ci.dez}{0}
\verb{hiperácido}{}{}{}{}{adj.}{Que é muito ácido.}{hi.pe.rá.ci.do}{0}
\verb{hiperatividade}{}{}{}{}{s.f.}{Excesso de atividade.}{hi.pe.ra.ti.vi.da.de}{0}
\verb{hiperatividade}{}{Med.}{}{}{}{Aumento anormal da atividade.}{hi.pe.ra.ti.vi.da.de}{0}
\verb{hiperativo}{}{}{}{}{adj.}{Que manifesta ou sofre de hiperatividade.}{hi.pe.ra.ti.vo}{0}
\verb{hiperbárico}{}{}{}{}{adj.}{Que tem pressão maior do que uma atmosfera.}{hi.per.bá.ri.co}{0}
\verb{hipérbato}{}{Gram.}{}{}{s.m.}{Figura de linguagem que consiste na inversão da ordem natural das palavras ou das orações.}{hi.pér.ba.to}{0}
%\verb{hiperbáton}{}{}{}{}{}{0}{hiperbáton}{0}
\verb{hipérbole}{}{Gram.}{}{}{s.f.}{Figura de linguagem caracterizada por aumento ou diminuição excessivos dos fatos.}{hi.pér.bo.le}{0}
\verb{hipérbole}{}{Geom.}{}{}{}{Curva formada por pontos que têm a mesma distância a outros dois pontos fixos, chamados focos.}{hi.pér.bo.le}{0}
\verb{hiperbólico}{}{Gram.}{}{}{adj.}{Em que há hipérbole; exagerado.}{hi.per.bó.li.co}{0}
\verb{hiperbólico}{}{Geom.}{}{}{}{Relativo a hipérbole; que tem forma de hipérbole.}{hi.per.bó.li.co}{0}
\verb{hipercalórico}{}{}{}{}{adj.}{Diz"-se de alimentos que têm muitas calorias.}{hi.per.ca.ló.ri.co}{0}
\verb{hipercorreção}{}{Gram.}{"-ões}{}{s.f.}{Troca de uma forma correta da língua por uma forma incorreta, considerada equivocadamente, como mais culta, na tentativa de dissimular seu dialeto socialmente estigmatizado; hiperurbanismo, ultracorreção.}{hi.per.cor.re.ção}{0}
%\verb{hipercrítico}{}{}{}{}{}{0}{hi.per.crí.ti.co}{0}
\verb{hiperglicemia}{}{Med.}{}{}{s.f.}{Excesso de glicose no sangue, característico entre diabéticos.}{hi.per.gli.ce.mi.a}{0}
\verb{hiperinflação}{}{Econ.}{"-ões}{}{s.f.}{Inflação muito elevada, eventualmente fora de controle.}{hi.pe.rin.fla.ção}{0}
\verb{hipermercado}{}{}{}{}{s.m.}{Supermercado de grande porte em que geralmente se vende grande variedade de mercadorias.}{hi.per.mer.ca.do}{0}
\verb{hipermetropia}{}{Med.}{}{}{s.f.}{Distúrbio da visão que resulta em dificuldade de enxergar de perto.}{hi.per.me.tro.pi.a}{0}
\verb{hipermídia}{}{}{}{}{s.f.}{Registro e apresentação de informações em vários formatos (texto, áudio, fotografia e vídeo, entre outros) de forma conjugada.}{hi.per.mí.dia}{0}
\verb{hipersensibilidade}{}{}{}{}{s.f.}{Qualidade ou estado de hipersensível.}{hi.per.sen.si.bi.li.da.de}{0}
\verb{hipersensibilidade}{}{Med.}{}{}{}{Reação excessiva a um agente estranho ao organismo.}{hi.per.sen.si.bi.li.da.de}{0}
\verb{hipersensível}{}{}{"-eis}{}{adj.2g.}{Que é muito sensível.}{hi.per.sen.sí.vel}{0}
\verb{hipersensível}{}{}{"-eis}{}{}{Que se choca facilmente; suscetível.}{hi.per.sen.sí.vel}{0}
\verb{hipertensão}{}{Med.}{"-ões}{}{s.f.}{Pressão sanguínea acima dos níveis normais.}{hi.per.ten.são}{0}
\verb{hipertensão}{}{Por ext.}{"-ões}{}{}{Qualquer estado de tensão acima da normalidade.}{hi.per.ten.são}{0}
\verb{hipertenso}{}{}{}{}{adj.}{Que apresenta hipertensão.}{hi.per.ten.so}{0}
\verb{hipertexto}{ês}{Informát.}{}{}{s.m.}{Texto que apresenta vínculos e remissões a outras de suas partes ou a outros textos, permitindo ao leitor uma consulta não necessariamente linear de seu conteúdo.}{hi.per.tex.to}{0}
\verb{hipertireoidismo}{}{Med.}{}{}{s.m.}{Doença decorrente da disfunção da glândula tireoide.}{hi.per.ti.re.oi.dis.mo}{0}
\verb{hipertrofia}{}{Med.}{}{}{s.f.}{Crescimento excessivo de um órgão ou tecido.}{hi.per.tro.fi.a}{0}
\verb{hipertrofiar}{}{}{}{}{v.t.}{Provocar hipertrofia em.}{hi.per.tro.fi.ar}{0}
\verb{hipertrofiar}{}{}{}{}{v.pron.}{Sofrer hipertrofia.}{hi.per.tro.fi.ar}{\verboinum{1}}
\verb{hipertrófico}{}{Med.}{}{}{adj.}{Relativo a hipertrofia.}{hi.per.tró.fi.co}{0}
\verb{hiperurbanismo}{}{Gram.}{}{}{s.m.}{Hipercorreção.}{hi.pe.rur.ba.nis.mo}{0}
\verb{hip hop}{}{}{}{}{s.m.}{Movimento cultural originado nos Estados Unidos, que pode ser caracterizado pela música \textit{rap}, pela dança \textit{break} e pela pintura do grafite.}{\textit{hip hop}}{0}
\verb{hípico}{}{}{}{}{adj.}{Relativo a cavalos.}{hí.pi.co}{0}
\verb{hípico}{}{Esport.}{}{}{}{Relativo ao hipismo.}{hí.pi.co}{0}
\verb{hipismo}{}{}{}{}{s.m.}{Conjunto de conhecimentos relativos a criação e adestramento de cavalos.}{hi.pis.mo}{0}
\verb{hipismo}{}{Esport.}{}{}{}{Qualquer um dos esportes praticados com cavalo.}{hi.pis.mo}{0}
\verb{hipnose}{ó}{}{}{}{s.f.}{Estado semelhante ao sono, artificialmente produzido, em que o indivíduo fica extremamente receptivo a sugestões feitas pelo hipnotizador.}{hip.no.se}{0}
\verb{hipnose}{ó}{Por ext.}{}{}{}{Sonolência, torpor.}{hip.no.se}{0}
\verb{hipnótico}{}{}{}{}{adj.}{Relativo à hipnose ou ao hipnotismo.}{hip.nó.ti.co}{0}
\verb{hipnótico}{}{}{}{}{}{Que produz sono artificial.}{hip.nó.ti.co}{0}
\verb{hipnótico}{}{Por ext.}{}{}{}{Que fascina; cativante, magnético.}{hip.nó.ti.co}{0}
\verb{hipnótico}{}{}{}{}{s.m.}{Medicamento que provoca sono; narcótico.}{hip.nó.ti.co}{0}
\verb{hipnotismo}{}{}{}{}{s.m.}{Conjunto de técnicas destinadas a provocar a hipnose por mecanismos de sugestão.}{hip.no.tis.mo}{0}
\verb{hipnotismo}{}{}{}{}{}{Ciência que estuda o conjunto dos fenômenos da hipnose.}{hip.no.tis.mo}{0}
\verb{hipnotizador}{ô}{}{}{}{s.m.}{Indivíduo que hipnotiza.}{hip.no.ti.za.dor}{0}
\verb{hipnotizar}{}{}{}{}{v.t.}{Provocar hipnose em.}{hip.no.ti.zar}{0}
\verb{hipnotizar}{}{Por ext.}{}{}{}{Provocar encanto; fascinar; magnetizar.}{hip.no.ti.zar}{\verboinum{1}}
\verb{hipocalórico}{}{}{}{}{adj.}{Diz"-se do alimento que tem muito poucas calorias.}{hi.po.ca.ló.ri.co}{0}
\verb{hipocampo}{}{Zool.}{}{}{s.m.}{Cavalo"-marinho.}{hi.po.cam.po}{0}
\verb{hipocampo}{}{Anat.}{}{}{}{Área do cérebro, especialmente relacionada com as atividades de memorização e de aprendizagem.}{hi.po.cam.po}{0}
\verb{hipoclorito}{}{Quím.}{}{}{s.m.}{Sal ou ânion derivado de ácido hipocloroso.}{hi.po.clo.ri.to}{0}
\verb{hipocloroso}{ô}{Quím.}{"-osos ⟨ó⟩}{"-osa ⟨ó⟩}{adj.}{Diz"-se de ácido formado de cloro, oxigênio e hidrogênio.}{hi.po.clo.ro.so}{0}
\verb{hipocondria}{}{Med.}{}{}{s.f.}{Preocupação obsessiva com o próprio estado de saúde, que leva à criação ou distorção de sintomas mórbidos e à busca de tratamentos desnecessários e eventualmente perigosos.}{hi.po.con.dri.a}{0}
\verb{hipocondríaco}{}{Med.}{}{}{adj.}{Relativo a hipocondria.}{hi.po.con.drí.a.co}{0}
\verb{hipocondríaco}{}{}{}{}{}{Que sofre de hipocondria.}{hi.po.con.drí.a.co}{0}
\verb{hipocondríaco}{}{}{}{}{s.m.}{Indivíduo que sofre de hipocondria.}{hi.po.con.drí.a.co}{0}
\verb{hipocôndrio}{}{Anat.}{}{}{s.m.}{Cada uma das duas partes laterais e superiores do abdômen.}{hi.po.côn.drio}{0}
\verb{hipocorístico}{}{Gram.}{}{}{adj.}{Diz"-se da palavra usada no trato familiar, com conotação carinhosa, inclusive certos diminutivos e termos oriundos da linguagem infantil.}{hi.po.co.rís.ti.co}{0}
\verb{hipocrisia}{}{}{}{}{s.f.}{Qualidade de hipócrita; falsidade.}{hi.po.cri.si.a}{0}
\verb{hipócrita}{}{}{}{}{adj.2g.}{Que dissimula sua verdadeira personalidade, seus sentimentos e suas opiniões, geralmente em razão de interesses próprios; fingido, falso.}{hi.pó.cri.ta}{0}
\verb{hipoderme}{é}{Biol.}{}{}{s.f.}{Camada de tecido situada abaixo da epiderme.}{hi.po.der.me}{0}
\verb{hipodérmico}{}{Biol.}{}{}{adj.}{Relativo à hipoderme.}{hi.po.dér.mi.co}{0}
\verb{hipodérmico}{}{}{}{}{}{Que está por baixo da pele.}{hi.po.dér.mi.co}{0}
\verb{hipodérmico}{}{}{}{}{}{Diz"-se de injeção que se aplica por baixo da pele.}{hi.po.dér.mi.co}{0}
\verb{hipódromo}{}{}{}{}{s.m.}{Local onde se realizam corridas de cavalo.}{hi.pó.dro.mo}{0}
\verb{hipófise}{}{Anat.}{}{}{s.f.}{Glândula de secreção interna, situada no crânio, sob a face inferior do cérebro, e que desempenha funções múltiplas em numerosos fenômenos fisiológicos; glândula pituitária.}{hi.pó.fi.se}{0}
\verb{hipogástrico}{}{}{}{}{adj.}{Relativo ao hipogástrio.}{hi.po.gás.tri.co}{0}
\verb{hipogástrio}{}{Anat.}{}{}{s.m.}{Parte inferior do abdômen, abaixo da região umbilical.}{hi.po.gás.trio}{0}
\verb{hipogeu}{}{}{}{}{s.m.}{Escavação subterrânea onde os povos antigos enterravam seus mortos; túmulo subterrâneo.}{hi.po.geu}{0}
\verb{hipoglicemia}{}{Med.}{}{}{s.f.}{Diminuição anormal da taxa de glicose no sangue.}{hi.po.gli.ce.mi.a}{0}
\verb{hipoglosso}{ô}{Anat.}{}{}{adj.}{Que fica embaixo da língua.}{hi.po.glos.so}{0}
\verb{hipoglosso}{ô}{Anat.}{}{}{s.m.}{Cada um dos nervos cranianos motores que inervam os músculos da língua.}{hi.po.glos.so}{0}
\verb{hipopótamo}{}{Zool.}{}{}{s.m.}{Mamífero herbívoro, de grande porte, com corpo e perna muito robustos, focinho largo, boca grande e pele grossa com poucos pelos, que habita as margens dos rios africanos.}{hi.po.pó.ta.mo}{0}
\verb{hipopótamo}{}{Fig.}{}{}{}{Indivíduo gordo.}{hi.po.pó.ta.mo}{0}
\verb{hipotálamo}{}{Anat.}{}{}{s.m.}{Região do cérebro, abaixo do tálamo, onde se encontram numerosos centros do sistema nervoso, que regula o sono, o apetite, a temperatura corporal, entre outros.}{hi.po.tá.la.mo}{0}
\verb{hipoteca}{é}{}{}{}{s.f.}{Ato ou efeito de hipotecar.}{hi.po.te.ca}{0}
\verb{hipoteca}{é}{Jur.}{}{}{}{Direito real que onera bens imobiliários para garantir o pagamento de uma dívida.}{hi.po.te.ca}{0}
\verb{hipoteca}{é}{}{}{}{}{A dívida contraída por hipoteca. }{hi.po.te.ca}{0}
\verb{hipotecar}{}{}{}{}{v.t.}{Dar em garantia a um credor sem que haja transferência de título ou de posse.}{hi.po.te.car}{0}
\verb{hipotecar}{}{Fig.}{}{}{}{Garantir, assegurar.}{hi.po.te.car}{\verboinum{2}}
\verb{hipotecário}{}{}{}{}{adj.}{Relativo a hipoteca.}{hi.po.te.cá.rio}{0}
\verb{hipotecário}{}{}{}{}{}{Que concede empréstimo mediante hipoteca.}{hi.po.te.cá.rio}{0}
\verb{hipotecável}{}{}{"-eis}{}{adj.2g.}{Que pode ser hipotecado.}{hi.po.te.cá.vel}{0}
\verb{hipotensão}{}{Med.}{"-ões}{}{s.f.}{Pressão abaixo do normal, no interior de um órgão do corpo ou num sistema orgânico.}{hi.po.ten.são}{0}
\verb{hipotenso}{}{Med.}{}{}{adj.}{Que tem ou sofre de hipotensão.}{hi.po.ten.so}{0}
\verb{hipotenso}{}{}{}{}{s.m.}{Indivíduo que sofre de hipotensão. }{hi.po.ten.so}{0}
\verb{hipotenusa}{}{Geom.}{}{}{s.f.}{Lado de um triângulo retângulo, oposto ao ângulo reto.}{hi.po.te.nu.sa}{0}
\verb{hipotermia}{}{Med.}{}{}{s.f.}{Diminuição excessiva da temperatura normal do corpo.}{hi.po.ter.mi.a}{0}
\verb{hipotermia}{}{}{}{}{}{Método de cura de diversos males por meio do frio.}{hi.po.ter.mi.a}{0}
\verb{hipótese}{}{}{}{}{s.f.}{Crença baseada em indícios, suposição.}{hi.pó.te.se}{0}
\verb{hipótese}{}{}{}{}{}{Afirmação sem compromisso com a realidade, que se usa como ponto de partida para um raciocínio ou explicação.}{hi.pó.te.se}{0}
\verb{hipotético}{}{}{}{}{adj.}{Que contém hipótese; conjectural, suposto.}{hi.po.té.ti.co}{0}
\verb{hipotético}{}{}{}{}{}{Baseado em suposição; duvidoso, incerto.}{hi.po.té.ti.co}{0}
\verb{hipotonia}{}{Med.}{}{}{s.f.}{Diminuição acentuada ou perda completa da tensão muscular ou em outras partes do corpo.}{hi.po.to.ni.a}{0}
\verb{hipotônico}{}{}{}{}{adj.}{Que apresenta hipotonia.}{hi.po.tô.ni.co}{0}
\verb{hipotrofia}{}{Med.}{}{}{s.f.}{Nutrição insuficiente; subalimentação.}{hi.po.tro.fi.a}{0}
\verb{hippie}{}{}{}{}{adj.2g.}{Relativo a \textit{hippies} ou a seu movimento.}{\textit{hippie}}{0}
\verb{hippie}{}{}{}{}{s.2g.}{Pessoa que, nas décadas de 1960 e 1970, rejeitava a sociedade de consumo, caracterizava"-se pelo rompimento com a sociedade tradicional, pregava a não violência, a liberdade sexual e a liberação das drogas.}{\textit{hippie}}{0}
\verb{hipsilo}{}{}{}{}{s.m.}{Nome da vigésima letra do alfabeto grego, que se representa por \textit{y}.}{hip.si.lo}{0}
\verb{hircino}{}{}{}{}{adj.}{Relativo ao bode (macho da cabra).}{hir.ci.no}{0}
\verb{hirsuto}{}{}{}{}{adj.}{Provido de pelos ou cabelos longos, duros e grossos.}{hir.su.to}{0}
\verb{hirsuto}{}{}{}{}{}{Diz"-se de pelos ou cabelos com aspecto malcuidado; espetado, arrepiado.}{hir.su.to}{0}
\verb{hirsuto}{}{Fig.}{}{}{}{Que não tem afabilidade para tratar as pessoas; intratável.}{hir.su.to}{0}
\verb{hirteza}{ê}{}{}{}{s.f.}{Qualidade ou estado de hirto; rigidez.}{hir.te.za}{0}
\verb{hirto}{}{}{}{}{adj.}{Que não tem flexibilidade; duro, teso.}{hir.to}{0}
\verb{hirto}{}{}{}{}{}{Completamente imóvel; estacado.}{hir.to}{0}
\verb{hirto}{}{}{}{}{}{Diz"-se de cabelos ou pelos duros, eriçados.}{hir.to}{0}
\verb{hirto}{}{Fig.}{}{}{}{Que não trata bem as pessoas; áspero.}{hir.to}{0}
\verb{hirundíneo}{}{}{}{}{adj.}{Relativo a andorinha.}{hi.run.dí.neo}{0}
\verb{hirundino}{}{}{}{}{}{Var. de \textit{hirundíneo}.}{hi.run.di.no}{0}
\verb{hispânico}{}{}{}{}{adj.}{Relativo à Espanha. }{his.pâ.ni.co}{0}
\verb{hispânico}{}{}{}{}{}{Relativo à Hispânia, província do Império Romano na Antiguidade (correspondente à Península Ibérica). }{his.pâ.ni.co}{0}
\verb{hispânico}{}{}{}{}{s.m.}{Indivíduo natural ou habitante da Espanha.}{his.pâ.ni.co}{0}
\verb{hispânico}{}{}{}{}{}{Indivíduo natural ou habitante da Hispânia.}{his.pâ.ni.co}{0}
\verb{hispânico}{}{Por ext.}{}{}{adj.}{Diz"-se dos nativos de países latino"-americanos, cuja língua é o espanhol ou o português, que vivem nos Estados Unidos.}{his.pâ.ni.co}{0}
\verb{hispanismo}{}{}{}{}{s.m.}{Palavra, locução ou construção da língua espanhola tomada de empréstimo por outra língua.}{his.pa.nis.mo}{0}
\verb{hispanista}{}{}{}{}{s.2g.}{Indivíduo que se dedica a estudar a língua e a cultura espanholas.}{his.pa.nis.ta}{0}
\verb{hispano"-americano}{}{}{hispano"-americanos}{hispano"-americana}{adj.}{Relativo à América de língua espanhola.}{his.pa.no"-a.me.ri.ca.no}{0}
\verb{hispano"-americano}{}{}{hispano"-americanos}{hispano"-americana}{s.m.}{Indivíduo de origem espanhola e americana.}{his.pa.no"-a.me.ri.ca.no}{0}
\verb{hispidez}{ê}{}{}{}{s.f.}{Qualidade de híspido; eriçamento, hirteza.}{his.pi.dez}{0}
\verb{híspido}{}{}{}{}{adj.}{Diz"-se de pelos, cabelos ou cerdas rigidamente em pé; eriçado, hirto.}{hís.pi.do}{0}
\verb{hissope}{ó}{Relig.}{}{}{s.m.}{Instrumento que serve para aspergir a água benta; aspersório.}{his.so.pe}{0}
\verb{histeralgia}{}{Med.}{}{}{s.f.}{Sensação de dor no útero.}{his.te.ral.gi.a}{0}
\verb{histerectomia}{}{Med.}{}{}{s.f.}{Remoção cirúrgica de parte ou da totalidade do útero, por via abdominal ou vaginal.}{his.te.rec.to.mi.a}{0}
\verb{histeria}{}{Med.}{}{}{s.f.}{Neurose caracterizada por ataques convulsivos, excessiva emotividade ou por pânico; exprime"-se por manifestações de ordem corporal, sem que haja qualquer problema orgânico funcional. }{his.te.ri.a}{0}
\verb{histérico}{}{}{}{}{adj.}{Relativo à histeria.}{his.té.ri.co}{0}
\verb{histérico}{}{}{}{}{}{Que tem ou mostra histeria.}{his.té.ri.co}{0}
\verb{histérico}{}{Pop.}{}{}{}{Irritadiço, nervoso.}{his.té.ri.co}{0}
\verb{histérico}{}{}{}{}{s.m.}{Indivíduo que sofre de histeria.}{his.té.ri.co}{0}
\verb{histerismo}{}{}{}{}{s.m.}{Predisposição à histeria.}{his.te.ris.mo}{0}
\verb{histerismo}{}{}{}{}{}{Forte manifestação de descontrole emocional; irritabilidade ou nervosismo excessivo.}{his.te.ris.mo}{0}
\verb{histerografia}{}{Med.}{}{}{s.f.}{Registro gráfico da força das contrações uterinas durante o parto.}{his.te.ro.gra.fi.a}{0}
\verb{histerografia}{}{Med.}{}{}{}{Radiografia contrastada do útero.}{his.te.ro.gra.fi.a}{0}
\verb{histerômetro}{}{}{}{}{s.m.}{Instrumento com que se mede a cavidade do útero.}{his.te.rô.me.tro}{0}
\verb{histeroscopia}{}{Med.}{}{}{s.f.}{Exame endoscópico do interior do útero.}{his.te.ros.co.pi.a}{0}
\verb{histeroscópio}{}{Med.}{}{}{s.m.}{Endoscópio, que, introduzido no canal cervical do útero, permite o exame visual direto da cavidade uterina; espéculo.}{his.te.ros.có.pio}{0}
\verb{histologia}{}{Biol.}{}{}{s.f.}{Ramo da biologia que estuda a estrutura microscópica, a composição e a função dos tecidos vivos.}{his.to.lo.gi.a}{0}
\verb{histológico}{}{}{}{}{adj.}{Relativo à histologia.}{his.to.ló.gi.co}{0}
\verb{história}{}{}{}{}{s.f.}{Ciência que estuda os acontecimentos ocorridos na vida dos povos e da humanidade. (\textit{Faremos uma viagem de estudos com o professor de História.})}{his.tó.ria}{0}
\verb{história}{}{}{}{}{}{Sequência de acontecimentos, reais ou fictícios; narrativa. (\textit{As crianças gostam de histórias de terror.})}{his.tó.ria}{0}
\verb{história}{}{}{}{}{}{Conjunto dos fatos relativos a determinado assunto.  }{his.tó.ria}{0}
\verb{historiador}{ô}{}{}{}{s.m.}{Pessoa que é especialista em história.}{his.to.ri.a.dor}{0}
\verb{historiador}{ô}{}{}{}{}{Indivíduo que historia ou narra um fato ou acontecimento.}{his.to.ri.a.dor}{0}
\verb{historiar}{}{}{}{}{v.t.}{Fazer o relato histórico de.}{his.to.ri.ar}{0}
\verb{historiar}{}{}{}{}{}{Narrar, contar.}{his.to.ri.ar}{0}
\verb{historiar}{}{}{}{}{}{Embelezar com ornatos; enfeitar.}{his.to.ri.ar}{\verboinum{1}}
\verb{historicidade}{}{}{}{}{s.f.}{Caráter do que é histórico.}{his.to.ri.ci.da.de}{0}
\verb{historicismo}{}{}{}{}{s.m.}{Qualidade do que é histórico.}{his.to.ri.cis.mo}{0}
\verb{historicismo}{}{Filos.}{}{}{}{Doutrina segundo a qual a compreensão dos valores e fatos de uma sociedade está relacionada à compreensão de sua história.}{his.to.ri.cis.mo}{0}
\verb{histórico}{}{}{}{}{adj.}{Relativo à história.}{his.tó.ri.co}{0}
\verb{histórico}{}{}{}{}{}{Digno de ser consagrado pela história; célebre.}{his.tó.ri.co}{0}
\verb{histórico}{}{}{}{}{}{Que existiu; real.}{his.tó.ri.co}{0}
\verb{histórico}{}{}{}{}{s.m.}{Exposição cronológica de fatos.}{his.tó.ri.co}{0}
\verb{historieta}{ê}{}{}{}{s.f.}{História pequena, curta.}{his.to.ri.e.ta}{0}
\verb{historieta}{ê}{}{}{}{}{Narrativa sobre fato de pouca ou nenhuma importância.}{his.to.ri.e.ta}{0}
\verb{historieta}{ê}{}{}{}{}{Relato sucinto de um fato jocoso ou curioso; anedota.}{his.to.ri.e.ta}{0}
\verb{historiografia}{}{}{}{}{s.f.}{A arte e o trabalho do historiógrafo; estudo e descrição da história.}{his.to.ri.o.gra.fi.a}{0}
\verb{historiográfico}{}{}{}{}{adj.}{Relativo a historiografia.}{his.to.ri.o.grá.fi.co}{0}
\verb{historiógrafo}{}{}{}{}{s.m.}{Escritor designado para escrever a história de uma época, de um povo, de uma nação, entre outros.}{his.to.ri.ó.gra.fo}{0}
\verb{historiógrafo}{}{}{}{}{}{Autor de trabalhos históricos; historiador.}{his.to.ri.ó.gra.fo}{0}
\verb{histrião}{}{}{"-ões}{}{s.m.}{No antigo teatro romano, jogral ou comediante que representava as farsas populares da época.}{his.tri.ão}{0}
\verb{histrião}{}{Por ext.}{"-ões}{}{}{Farsista, comediante, cômico.}{his.tri.ão}{0}
\verb{histrião}{}{Fig.}{"-ões}{}{}{Palhaço, bufão.}{his.tri.ão}{0}
\verb{histrião}{}{}{"-ões}{}{}{Pessoa ridícula, vil, abjeta.}{his.tri.ão}{0}
\verb{hitita}{}{}{}{}{adj.2g.}{Relativo aos Hititas, povo que habitava a Ásia Menor, na Antiguidade, e que falava uma língua indo"-europeia.}{hi.ti.ta}{0}
\verb{hitita}{}{}{}{}{s.m.}{A língua desse povo.}{hi.ti.ta}{0}
\verb{hitlerismo}{}{}{}{}{s.m.}{Doutrina política e social de Adolf Hitler, governante da Alemanha de 1933 até o final da Segunda Guerra Mundial; nazismo.}{hi.tle.ris.mo}{0}
\verb{HIV}{}{}{}{}{s.m.}{Abrev. do inglês \textit{Human Immunodeficience Virus}, vírus da imunodeficiência humana; vírus da Aids.}{HIV}{0}
\verb{Ho}{}{Quím.}{}{}{}{Símb. do \textit{hólmio}. }{Ho}{0}
\verb{hobby}{}{}{}{}{s.m.}{Atividade praticada como forma de lazer, de distração; passatempo.}{\textit{hobby}}{0}
\verb{hodierno}{é}{}{}{}{adj.}{Relativo aos dias de hoje; atual.}{ho.di.er.no}{0}
\verb{hodômetro}{}{}{}{}{s.m.}{Instrumento com que se medem distâncias percorridas.}{ho.dô.me.tro}{0}
\verb{hoje}{ô}{}{}{}{adv.}{No presente dia; neste dia; no dia em que estamos.}{ho.je}{0}
\verb{hoje}{ô}{}{}{}{}{No tempo presente; atualmente.}{ho.je}{0}
\verb{holandês}{}{}{}{}{adj.}{Relativo à Holanda (Países Baixos).}{ho.lan.dês}{0}
\verb{holandês}{}{}{}{}{s.m.}{Indivíduo natural ou habitante desse país.}{ho.lan.dês}{0}
\verb{holandês}{}{}{}{}{}{Língua oficial da Holanda. }{ho.lan.dês}{0}
\verb{holding}{}{}{}{}{s.f.}{Empresa que controla outras empresas por meio da posse da maioria de suas ações.}{\textit{holding}}{0}
\verb{holerite}{}{}{}{}{s.m.}{Documento fornecido pelo empregador a cada empregado, com discriminação de salário bruto, deduções e eventuais acréscimos; contracheque.}{ho.le.ri.te}{0}
\verb{holismo}{}{Filos.}{}{}{s.m.}{Doutrina, no campo das ciências humanas e naturais, que compreende a realidade em totalidades integradas, em oposição ao procedimento analítico em que seus componentes são tomados isoladamente.}{ho.lis.mo}{0}
\verb{holística}{}{}{}{}{s.f.}{Holismo.}{ho.lís.ti.ca}{0}
\verb{holístico}{}{}{}{}{adj.}{Relativo a holismo.}{ho.lís.ti.co}{0}
\verb{hólmio}{}{Quím.}{}{}{s.m.}{Elemento químico metálico, sólido, prateado, estável ao ar, da família dos lantanídeos (terras"-raras); usado em espectroscopia, tubos de alto vácuo, entre outros. \elemento{67}{164.93031}{Ho}.}{hól.mi.o}{0}
\verb{holocausto}{}{}{}{}{s.m.}{Massacre maciço de pessoas; genocídio.}{ho.lo.caus.to}{0}
\verb{holocausto}{}{Hist.}{}{}{}{Massacre de judeus e outras minorias em campos de concentração durante o nazismo. (Nesta acepção, com maiúscula.)}{ho.lo.caus.to}{0}
\verb{holocausto}{}{Fig.}{}{}{}{Sacrifício, imolação.}{ho.lo.caus.to}{0}
\verb{holoceno}{ê}{Geol.}{}{}{s.m.}{Período da Era Cenozoica, posterior ao Pleistoceno; período mais recente da escala do tempo geológico.}{ho.lo.ce.no}{0}
\verb{holoceno}{ê}{Geol.}{}{}{adj.}{Relativo ou pertencente a esse período.}{ho.lo.ce.no}{0}
\verb{holofote}{ó}{}{}{}{s.m.}{Aparelho que projeta luz com grande intensidade, usado para iluminar objetos a distância ou fazer sinais.}{ho.lo.fo.te}{0}
\verb{holografia}{}{}{}{}{s.f.}{Processo fotográfico para a obtenção de imagens tridimensionais, mediante utilização de \textit{laser}.}{ho.lo.gra.fi.a}{0}
\verb{holograma}{}{}{}{}{s.m.}{Imagem fotográfica tridimensional obtida por raio \textit{laser}.}{ho.lo.gra.ma}{0}
\verb{hombridade}{}{}{}{}{s.f.}{Aspecto varonil, másculo.}{hom.bri.da.de}{0}
\verb{hombridade}{}{Fig.}{}{}{}{Nobreza de caráter; dignidade, brio.}{hom.bri.da.de}{0}
\verb{home banking}{}{}{}{}{s.m.}{Forma de acesso a serviços bancários por meio da rede de computadores.}{\textit{home banking}}{0}
\verb{home care}{}{}{}{}{s.f.}{Assistência médica domiciliar.}{\textit{home care}}{0}
\verb{homem}{}{}{"-ens}{}{s.m.}{O ser humano; pessoa; qualquer indivíduo da espécie animal que apresenta o maior grau de complexidade na escala evolutiva.}{ho.mem}{0}
\verb{homem}{}{}{"-ens}{}{}{A espécie humana; a humanidade.}{ho.mem}{0}
\verb{homem}{}{}{"-ens}{}{}{Ser humano do sexo masculino.}{ho.mem}{0}
\verb{homem}{}{Pop.}{"-ens}{}{}{Marido; amante.}{ho.mem}{0}
\verb{homem"-hora}{ó}{}{homens"-horas ⟨ó⟩}{}{s.m.}{Unidade de medida que corresponde ao trabalho efetuado por um homem durante uma  hora.}{ho.mem"-ho.ra}{0}
\verb{homem"-rã}{}{}{homens"-rã \textit{ou} homens"-rãs}{}{s.m.}{Nome dado aos mergulhadores profissionais, convenientemente equipados e treinados para trabalhos de resgate, salvamento, explorações científicas ou militares, entre outros.}{ho.mem"-rã}{0}
\verb{homem"-sanduíche}{}{}{homens"-sanduíches \textit{ou}  homens"-sanduíche}{}{s.m.}{Indivíduo que carrega dois cartazes de propaganda pendurados, um nas costas e outro no peito.}{ho.mem"-san.du.í.che}{0}
\verb{homenageado}{}{}{}{}{adj.}{Que recebeu homenagem.}{ho.me.na.ge.a.do}{0}
\verb{homenagear}{}{}{}{}{v.t.}{Prestar homenagem.}{ho.me.na.ge.ar}{\verboinum{4}}
\verb{homenagem}{}{}{"-ens}{}{s.f.}{Demonstração, ato ou prova de respeito, admiração ou consideração por algo ou alguém.}{ho.me.na.gem}{0}
\verb{homenagem}{}{}{"-ens}{}{}{Juramento de fidelidade, subordinação e respeito; preito.}{ho.me.na.gem}{0}
\verb{homenagem}{}{Hist.}{"-ens}{}{}{Juramento pelo qual um vassalo prometia fidelidade a seu suserano.}{ho.me.na.gem}{0}
\verb{homenzarrão}{}{}{"-ões}{}{s.m.}{Homem grande, corpulento.}{ho.men.zar.rão}{0}
\verb{homeopata}{}{Med.}{}{}{adj.2g.}{Diz"-se de profissional que segue ou exerce a homeopatia.}{ho.me.o.pa.ta}{0}
\verb{homeopatia}{}{Med.}{}{}{s.f.}{Método terapêutico baseado na lei da similitude (o semelhante curado pelo semelhante), que consiste num tratamento com doses mínimas de substâncias que provocam no paciente sintomas parecidos com os da doença a ser curada, fazendo com que ele se recupere.}{ho.me.o.pa.ti.a}{0}
\verb{homeopático}{}{}{}{}{adj.}{Relativo à homeopatia.}{ho.me.o.pá.ti.co}{0}
\verb{homeopático}{}{Fig.}{}{}{}{Muito pequeno.}{ho.me.o.pá.ti.co}{0}
\verb{homeostase}{}{}{}{}{s.f.}{Estado de equilíbrio autorregulado entre órgãos ou sistemas diferentes.}{ho.me.os.ta.se}{0}
\verb{homeostasia}{}{}{}{}{s.f.}{Homeostase.}{ho.me.os.ta.si.a}{0}
\verb{homeotermia}{}{}{}{}{}{Var. de \textit{homotermia}.}{ho.me.o.ter.mi.a}{0}
\verb{homeotérmico}{}{}{}{}{}{Var. de \textit{homotérmico}.}{ho.me.o.tér.mi.co}{0}
\verb{home page}{}{Informát.}{home pages}{}{s.f.}{Página principal de um \textit{site} da \textit{web}, que geralmente contém uma apresentação e remissões a outras seções do \textit{site}.}{\textit{home page}}{0}
\verb{home page}{}{Informát.}{home pages}{}{}{\textit{Site} da \textit{web}.}{\textit{home page}}{0}
\verb{homérico}{}{}{}{}{adj.}{Relativo a Homero, poeta grego, às suas obras ou ao seu estilo.}{ho.mé.ri.co}{0}
\verb{homérico}{}{Fig.}{}{}{}{Grande, extraordinário, fora do comum.}{ho.mé.ri.co}{0}
\verb{homessa}{é}{}{}{}{interj.}{Expressão que denota surpresa, espanto; ora essa; essa não.}{ho.mes.sa}{0}
\verb{home theater}{}{}{}{}{s.m.}{Conjunto de equipamentos eletroeletrônicos destinados à reprodução caseira de gravações audiovisuais.}{\textit{home theater}}{0}
\verb{homicida}{}{}{}{}{adj.2g.}{Que causa a morte de outra pessoa; que mata alguém.}{ho.mi.ci.da}{0}
\verb{homicida}{}{}{}{}{s.2g.}{Indivíduo que mata uma pessoa; que pratica o homicídio.  }{ho.mi.ci.da}{0}
\verb{homicídio}{}{}{}{}{s.m.}{Ação de matar alguém, voluntária ou involuntariamente; assassínio, assassinato.}{ho.mi.cí.dio}{0}
\verb{homilia}{}{Relig.}{}{}{s.f.}{Comentário feito na missa pelo sacerdote, em estilo familiar e quase coloquial, sobre o trecho que foi lido do Evangelho.}{ho.mi.li.a}{0}
\verb{homília}{}{}{}{}{}{Var. de \textit{homilia}.}{ho.mí.lia}{0}
\verb{hominídeo}{}{Zool.}{}{}{adj.}{Relativo aos hominídeos.}{ho.mi.ní.deo}{0}
\verb{hominídeo}{}{}{}{}{}{Espécime dos hominídeos, família de mamíferos primatas, que inclui o homem atual e seus ancestrais fósseis.}{ho.mi.ní.deo}{0}
\verb{homiziar}{}{}{}{}{v.t.}{Dar abrigo a; esconder, encobrir.}{ho.mi.zi.ar}{0}
\verb{homiziar}{}{}{}{}{}{Dar asilo a alguém contra a ação da justiça; acobertar.}{ho.mi.zi.ar}{0}
\verb{homiziar}{}{}{}{}{}{Criar inimizade; indispor, intrigar.}{ho.mi.zi.ar}{\verboinum{1}}
\verb{homizio}{}{}{}{}{s.m.}{Ato ou efeito de homiziar, abrigar; esconderijo, refúgio.}{ho.mi.zi.o}{0}
\verb{homofonia}{}{Gram.}{}{}{s.f.}{Semelhança de sons ou de pronúncia.}{ho.mo.fo.ni.a}{0}
\verb{homófono}{}{Gram.}{}{}{adj.}{Diz"-se de cada uma das palavras que têm a mesma pronúncia, mas que diferem no significado.  }{ho.mó.fo.no}{0}
\verb{homogeneidade}{}{}{}{}{s.f.}{Qualidade ou condição do que apresenta natureza homogênea, uniforme.}{ho.mo.ge.nei.da.de}{0}
\verb{homogeneização}{}{}{"-ões}{}{s.f.}{Ato ou efeito de homogeneizar; uniformização.}{ho.mo.ge.nei.za.ção}{0}
\verb{homogeneização}{}{}{"-ões}{}{}{Tratamento dado ao leite para evitar a decantação dos elementos que o compõem.}{ho.mo.ge.nei.za.ção}{0}
\verb{homogeneizar}{}{}{}{}{v.t.}{Tornar homogêneo, uniforme; uniformizar, igualar.}{ho.mo.ge.nei.zar}{\verboinum{1}}
\verb{homogêneo}{}{}{}{}{adj.}{Que possui a mesma natureza ou estrutura em relação a outra coisa.}{ho.mo.gê.neo}{0}
\verb{homogêneo}{}{}{}{}{}{Que possui composição uniforme, não permitindo a distinção de seus elementos ou de suas fases.}{ho.mo.gê.neo}{0}
\verb{homogêneo}{}{}{}{}{}{Que apresenta unidade ou correspondência entre seus elementos.}{ho.mo.gê.neo}{0}
\verb{homogênese}{}{}{}{}{s.f.}{Reprodução contínua e idêntica entre gerações sucessivas.}{ho.mo.gê.ne.se}{0}
\verb{homógrafo}{}{Gram.}{}{}{adj.}{Diz"-se de cada uma das palavras que têm a mesma grafia, mas que diferem no significado.}{ho.mó.gra.fo}{0}
\verb{homologação}{}{}{"-ões}{}{s.f.}{Ato ou efeito de homologar; aprovação, ratificação.}{ho.mo.lo.ga.ção}{0}
\verb{homologação}{}{Jur.}{"-ões}{}{}{Aprovação ou confirmação de um ato oficial pela autoridade judiciária ou administrativa competente para que esse ato possa apresentar validade jurídica.}{ho.mo.lo.ga.ção}{0}
\verb{homologar}{}{}{}{}{v.t.}{Reconhecer ou aceitar oficialmente; legitimar.}{ho.mo.lo.gar}{0}
\verb{homologar}{}{}{}{}{}{Ratificar ou aprovar por autoridade judiciária ou administrativa.}{ho.mo.lo.gar}{\verboinum{5}}
\verb{homologia}{}{}{}{}{s.f.}{Qualidade ou condição do que é homólogo; semelhança, similaridade.}{ho.mo.lo.gi.a}{0}
\verb{homologia}{}{Biol.}{}{}{}{Semelhança de origem e estrutura entre partes de organismos diferentes.}{ho.mo.lo.gi.a}{0}
\verb{homologia}{}{Gram.}{}{}{}{Uso repetido das mesmas palavras ou dos mesmos conceitos em um discurso.}{ho.mo.lo.gi.a}{0}
\verb{homólogo}{}{}{}{}{adj.}{Que mantém relação de correspondência com outra coisa; semelhante, equivalente.}{ho.mó.lo.go}{0}
\verb{homólogo}{}{Biol.}{}{}{}{Diz"-se de partes de organismos diferentes que possuem a mesma estrutura e origem.}{ho.mó.lo.go}{0}
\verb{homomorfo}{ó}{}{}{}{adj.}{Que apresenta uniformidade de formas.}{ho.mo.mor.fo}{0}
\verb{homonímia}{}{Gram.}{}{}{s.f.}{Relação de identidade de pronúncia ou grafia entre palavras que diferem quanto ao significado.}{ho.mo.ní.mia}{0}
\verb{homônimo}{}{}{}{}{adj.}{Que possui o mesmo nome.}{ho.mô.ni.mo}{0}
\verb{homônimo}{}{Gram.}{}{}{}{Diz"-se de cada uma das palavras que têm a mesma grafia ou a mesma pronúncia, mas que diferem no significado.}{ho.mô.ni.mo}{0}
\verb{homóptero}{}{Zool.}{}{}{adj.}{Espécime dos homópteros, ordem de insetos que apresentam metamorfose gradual, aparelho bucal sugador e textura unifome nas asas, como a cigarra e o pulgão.}{ho.móp.te.ro}{0}
\verb{homóptero}{}{Zool.}{}{}{adj.}{Relativo aos homópteros.}{ho.móp.te.ro}{0}
\verb{homossexual}{cs}{}{"-ais}{}{adj.2g.}{Relativo à afinidade ou ao comportamento sexual entre indivíduos do mesmo sexo. }{ho.mos.se.xu.al}{0}
\verb{homossexual}{cs}{}{"-ais}{}{s.2g.}{Indivíduo que se sente atraído física, emocional e espiritualmente por uma pessoa do mesmo sexo. }{ho.mos.se.xu.al}{0}
\verb{homossexualidade}{cs}{}{}{}{s.f.}{Caráter ou qualidade do que ou de quem é homossexual; homossexualismo.}{ho.mos.se.xu.a.li.da.de}{0}
\verb{homossexualismo}{cs}{}{}{}{s.m.}{Desejo ou prática sexual entre indivíduos de mesmo sexo; homossexualidade.}{ho.mos.se.xu.a.lis.mo}{0}
\verb{homotermia}{}{}{}{}{s.f.}{Qualidade de um corpo que mantém fixa a sua temperatura.}{ho.mo.ter.mi.a}{0}
\verb{homotérmico}{}{}{}{}{adj.}{Que mantém temperatura constante, com independência do meio em que se insere.}{ho.mo.tér.mi.co}{0}
\verb{homozigoto}{ô}{}{}{}{s.m.}{Indivíduo que tem alelos idênticos de um ou mais genes.}{ho.mo.zi.go.to}{0}
\verb{homúnculo}{}{}{}{}{s.m.}{Homem pequeno, homenzinho.}{ho.mún.cu.lo}{0}
\verb{homúnculo}{}{Fig.}{}{}{}{Indivíduo insignificante, desprezível, vil.}{ho.mún.cu.lo}{0}
\verb{homus}{}{Cul.}{}{}{s.m.}{Pasta comestível de sementes de grão"-de"-bico.}{ho.mus}{0}
\verb{hondurenho}{}{}{}{}{adj.}{Relativo a Honduras.}{hon.du.re.nho}{0}
\verb{hondurenho}{}{}{}{}{s.m.}{Indivíduo natural ou habitante desse país. }{hon.du.re.nho}{0}
\verb{honestar}{}{}{}{}{v.t.}{Tornar honesto, decente; honrar.}{ho.nes.tar}{0}
\verb{honestar}{}{}{}{}{}{Embelezar, adornar, enfeitar.}{ho.nes.tar}{\verboinum{1}}
\verb{honestidade}{}{}{}{}{s.f.}{Qualidade ou caráter do que ou de quem é honesto; probidade, honradez, dignidade.}{ho.nes.ti.da.de}{0}
\verb{honestidade}{}{}{}{}{}{Decência, decoro, castidade.}{ho.nes.ti.da.de}{0}
\verb{honesto}{é}{}{}{}{adj.}{Que revela honradez; digno, íntegro.}{ho.nes.to}{0}
\verb{honesto}{é}{}{}{}{}{Que se comporta com decência; casto, virtuoso.}{ho.nes.to}{0}
\verb{honorabilidade}{}{}{}{}{s.f.}{Qualidade ou caráter do que é honorável; respeitabilidade, honradez.}{ho.no.ra.bi.li.da.de}{0}
\verb{honorário}{}{}{}{}{adj.}{Que dá honras e glórias sem nenhum proveito material; honorífico.}{ho.no.rá.rio}{0}
\verb{honorário}{}{}{}{}{}{Que conserva o título e as honras mesmo sem exercer mais o cargo ou a função.}{ho.no.rá.rio}{0}
\verb{honorários}{}{}{}{}{s.m.pl.}{Remuneração ou vencimentos pagos a profissional liberal como médicos, advogados etc. }{ho.no.rá.ri.os}{0}
\verb{honorável}{}{}{"-eis}{}{adj.2g.}{Que é digno de consideração, respeito, honra.}{ho.no.rá.vel}{0}
\verb{honorífico}{}{}{}{}{adj.}{Que honra e distingue; honroso, respeitoso.}{ho.no.rí.fi.co}{0}
\verb{honoris causa}{ó}{}{}{}{}{Expressão latina usada para designar título honorífico concedido a alguém sem exame ou concurso, a título de homenagem.}{\textit{honoris causa}}{0}
\verb{honra}{}{}{}{}{s.f.}{Sentimento de dignidade moral que leva um indivíduo a agir para conservar a própria estima e merecer a consideração de outros; brio.}{hon.ra}{0}
\verb{honra}{}{}{}{}{}{Atitude de consideração, estima ou reconhecimento.}{hon.ra}{0}
\verb{honra}{}{}{}{}{}{Dignidade, retidão, probidade.}{hon.ra}{0}
\verb{honra}{}{}{}{}{}{Homenagem, distinção ou graça que se concede a alguém.}{hon.ra}{0}
\verb{honra}{}{}{}{}{}{Castidade, pureza, inocência.}{hon.ra}{0}
\verb{honradez}{ê}{}{}{}{s.f.}{Qualidade ou condição de honrado; dignidade, honestidade, integridade.}{hon.ra.dez}{0}
\verb{honrado}{}{}{}{}{adj.}{Que tem honra; digno, íntegro, honesto.}{hon.ra.do}{0}
\verb{honrado}{}{}{}{}{}{Casto, puro, decente.}{hon.ra.do}{0}
\verb{honrar}{}{}{}{}{v.t.}{Demonstrar respeito por; conferir merecimento a.}{hon.rar}{0}
\verb{honrar}{}{}{}{}{}{Distinguir com honraria; dignificar, exaltar.}{hon.rar}{0}
\verb{honrar}{}{}{}{}{}{Cumprir (compromissos ou tratos).}{hon.rar}{0}
\verb{honrar}{}{}{}{}{v.pron.}{Lisonjear"-se; ufanar"-se, vangloriar"-se.}{hon.rar}{\verboinum{1}}
\verb{honraria}{}{}{}{}{s.f.}{Grandeza de algum cargo ou função; distinção, graça, honra.}{hon.ra.ri.a}{0}
\verb{honraria}{}{}{}{}{}{Manifestação honrosa.}{hon.ra.ri.a}{0}
\verb{honroso}{ô}{}{"-osos ⟨ó⟩}{"-osa ⟨ó⟩}{adj.}{Que enobrece, dignifica.}{hon.ro.so}{0}
\verb{honroso}{ô}{}{"-osos ⟨ó⟩}{"-osa ⟨ó⟩}{}{Honesto, honrado. }{hon.ro.so}{0}
\verb{hóquei}{}{Esport.}{}{}{s.m.}{Jogo disputado entre duas equipes sobre o gelo ou sobre a grama, com bastões recurvados em uma das extremidades, usados para impelir uma bola maciça ou um disco através do arco adversário.}{hó.quei}{0}
\verb{hora}{ó}{}{}{}{s.f.}{Unidade de medida de tempo que equivale à 24ª parte do dia solar, a 60 minutos ou a 3600 segundos. Símb.: h.}{ho.ra}{0}
\verb{hora}{ó}{}{}{}{}{Indicação precisa desse período de tempo, incluindo"-se minutos e segundos.}{ho.ra}{0}
\verb{hora}{ó}{}{}{}{}{Momento conveniente; oportunidade, ocasião.}{ho.ra}{0}
\verb{hora}{ó}{}{}{}{}{Momento fixado para a realização de algo; horário.}{ho.ra}{0}
\verb{hora}{ó}{}{}{}{}{Carga horária semanal que um funcionário deve cumprir no estabelecimento em que está empregado.}{ho.ra}{0}
\verb{hora"-luz}{ó}{}{horas"-luz ⟨ó⟩}{}{s.f.}{Distância que a luz percorre em uma hora.}{ho.ra"-luz}{0}
\verb{horário}{}{}{}{}{adj.}{Relativo a hora.}{ho.rá.rio}{0}
\verb{horário}{}{}{}{}{s.m.}{Indicação das horas ou do período de tempo em que se executa certa atividade.}{ho.rá.rio}{0}
\verb{horário}{}{}{}{}{}{Hora prefixada de chegada ou partida de um meio de transporte; hora.}{ho.rá.rio}{0}
\verb{horda}{ó}{}{}{}{s.f.}{Tribo nômade.}{hor.da}{0}
\verb{horda}{ó}{Por ext.}{}{}{}{Bando indisciplinado que provoca brigas e desordem.}{hor.da}{0}
\verb{hordéolo}{}{Med.}{}{}{s.m.}{Pequeno abscesso que cresce na borda da pálpebra; terçol.}{hor.dé.o.lo}{0}
\verb{horista}{}{}{}{}{adj.2g.}{Diz"-se do empregado que recebe remuneração por hora trabalhada.}{ho.ris.ta}{0}
\verb{horizontal}{}{}{"-ais}{}{adj.2g.}{Que é paralelo ao horizonte.}{ho.ri.zon.tal}{0}
\verb{horizontal}{}{}{"-ais}{}{}{Perpendicular à vertical.}{ho.ri.zon.tal}{0}
\verb{horizontal}{}{}{"-ais}{}{}{Diz"-se da posição de quem está deitado, estendido horizontalmente. }{ho.ri.zon.tal}{0}
\verb{horizontal}{}{}{"-ais}{}{s.f.}{Linha paralela ao horizonte.}{ho.ri.zon.tal}{0}
\verb{horizontalidade}{}{}{}{}{s.f.}{Qualidade ou condição do que está na posição horizontal.}{ho.ri.zon.ta.li.da.de}{0}
\verb{horizonte}{}{}{}{}{s.m.}{Linha circular em que o céu e a terra ou o mar parecem se unir limitando o campo visual de uma pessoa que não tenha obstáculo à vista.}{ho.ri.zon.te}{0}
\verb{horizonte}{}{}{}{}{}{Espaço visível a uma pessoa ao ar livre.}{ho.ri.zon.te}{0}
\verb{horizonte}{}{Fig.}{}{}{}{Perspectiva; possibilidade de progresso, de melhoria.}{ho.ri.zon.te}{0}
\verb{horizonte}{}{Fig.}{}{}{}{Representação dos limites da consciência, da memória.}{ho.ri.zon.te}{0}
\verb{hormonal}{}{}{"-ais}{}{adj.2g.}{Relativo a hormônio.}{hor.mo.nal}{0}
\verb{hormônio}{}{Biol.}{}{}{s.m.}{Substância produzida por uma glândula endócrina ou por certos tipos de tecido e que, quando secretada na corrente sanguínea, tem efeito específico sobre a atividade de um órgão ou de uma estrutura.}{hor.mô.nio}{0}
\verb{horóscopo}{}{}{}{}{s.m.}{Predição ou prognóstico feito geralmente por um astrólogo a partir da interpretação da posição dos astros no momento do nascimento de uma pessoa; previsão astrológica; mapa astral.}{ho.rós.co.po}{0}
\verb{horrendo}{}{}{}{}{adj.}{Que causa horror; pavoroso, horrível, medonho, monstruoso.}{hor.ren.do}{0}
\verb{horrendo}{}{}{}{}{}{Marcado pela maldade; cruel, hediondo, abominável.}{hor.ren.do}{0}
\verb{horripilação}{}{}{"-ões}{}{s.f.}{Ato ou efeito de horripilar; arrepio de horror.}{hor.ri.pi.la.ção}{0}
\verb{horripilação}{}{}{"-ões}{}{}{Eriçamento dos pelos; calafrio.}{hor.ri.pi.la.ção}{0}
\verb{horripilante}{}{}{}{}{adj.2g.}{Que provoca arrepio de medo, calafrio; arrepiante, medonho, assustador.}{hor.ri.pi.lan.te}{0}
\verb{horripilar}{}{}{}{}{v.t.}{Provocar arrepios ou calafrios em; arrepiar.}{hor.ri.pi.lar}{0}
\verb{horripilar}{}{}{}{}{v.i.}{Causar horror.}{hor.ri.pi.lar}{\verboinum{1}}
\verb{horrível}{}{}{"-eis}{}{adj.2g.}{Que causa horror; terrível, pavoroso, medonho, horrendo.}{hor.rí.vel}{0}
\verb{horrível}{}{}{"-eis}{}{}{Muito ruim; péssimo.}{hor.rí.vel}{0}
\verb{horror}{ô}{}{}{}{s.m.}{Medo, pavor.}{hor.ror}{0}
\verb{horror}{ô}{}{}{}{}{Sentimento de aversão, nojo, repulsa.}{hor.ror}{0}
\verb{horror}{ô}{}{}{}{}{Crime bárbaro, hediondo.}{hor.ror}{0}
\verb{horror}{ô}{}{}{}{}{Aspecto muito feio, desagradável.}{hor.ror}{0}
\verb{horror}{ô}{Pop.}{}{}{}{Sofrimento atroz; crueldade.}{hor.ror}{0}
\verb{horrorizado}{}{}{}{}{adj.}{Que se horrorizou; assustado, apavorado.}{hor.ro.ri.za.do}{0}
\verb{horrorizar}{}{}{}{}{v.t.}{Causar horror, pânico a; amedrontar.}{hor.ro.ri.zar}{\verboinum{1}}
\verb{horroroso}{ô}{}{"-osos ⟨ó⟩}{"-osa ⟨ó⟩}{adj.}{Extremamente ruim; medonho, horrível, pavoroso.}{hor.ro.ro.so}{0}
\verb{hors"-concours}{}{}{}{}{adj.2g.}{Diz"-se do que se apresenta e é aceito em concurso, mas não pode concorrer a prêmio por estar fora das regras.}{\textit{hors"-concours}}{0}
\verb{hors"-d'oeuvre}{}{}{}{}{s.m.}{Prato leve e frio, servido antes da entrada ou do prato principal.}{\textit{hors"-d'oeuvre}}{0}
\verb{horta}{ó}{}{}{}{s.f.}{Terreno onde se cultivam especialmente hortaliças e legumes.}{hor.ta}{0}
\verb{hortaliça}{}{}{}{}{s.f.}{Designação genérica de plantas herbáceas ou leguminosas, empregadas na alimentação humana; verdura, legume.}{hor.ta.li.ça}{0}
\verb{hortelã}{}{Bot.}{}{}{s.f.}{Nome de uma planta herbácea, de folhas aromáticas, usada para fins medicinais e como condimento; menta.}{hor.te.lã}{0}
\verb{hortelão}{}{}{"-ãos \textit{ou} -ões}{horteloa}{s.m.}{Indivíduo que cuida de horta.}{hor.te.lão}{0}
\verb{hortelã"-pimenta}{}{}{hortelãs"-pimentas \textit{ou} hortelãs"-pimenta}{}{s.f.}{Erva aromática, de folhas moles, flores pequenas e sabor picante, da qual se extrai um óleo rico em mentol.}{hor.te.lã"-pi.men.ta}{0}
\verb{horteloa}{ô}{}{}{}{s.f.}{Feminino de hortelão.}{hor.te.lo.a}{0}
\verb{hortense}{}{}{}{}{adj.}{Relativo a horta.}{hor.ten.se}{0}
\verb{hortênsia}{}{Bot.}{}{}{s.f.}{Nome de um arbusto ornamental, nativo da China e do Japão, de flores brancas, azuis ou róseas.}{hor.tên.sia}{0}
\verb{hortícola}{}{}{}{}{adj.}{Relativo a horta ou a horticultura; hortense.}{hor.tí.co.la}{0}
\verb{horticultor}{ô}{}{}{}{s.m.}{Indivíduo que se dedica à horticultura.}{hor.ti.cul.tor}{0}
\verb{horticultura}{}{}{}{}{s.f.}{Técnica de cultivar hortas e jardins.}{hor.ti.cul.tu.ra}{0}
\verb{hortifrutigranjeiro}{ê}{Bras.}{}{}{adj.}{Diz"-se dos produtos de hortas, pomares e granjas.}{hor.ti.fru.ti.gran.jei.ro}{0}
\verb{hortifrutigranjeiro}{ê}{}{}{}{s.m.}{Produto de hortas, pomares e granjas. }{hor.ti.fru.ti.gran.jei.ro}{0}
\verb{hortigranjeiro}{ê}{Bras.}{}{}{adj.}{Diz"-se dos produtos de hortas e granjas.  }{hor.ti.gran.jei.ro}{0}
\verb{hortigranjeiro}{ê}{}{}{}{s.m.}{Produto de hortas e granjas. }{hor.ti.gran.jei.ro}{0}
\verb{horto}{ô}{}{}{}{s.m.}{Pequena horta.}{hor.to}{0}
\verb{horto}{ô}{}{}{}{}{Terreno não muito extenso onde se cultivam plantas ornamentais.}{hor.to}{0}
\verb{horto}{ô}{}{}{}{}{Estabelecimento de horticultura.}{hor.to}{0}
\verb{hosana}{}{Relig.}{}{}{s.f.}{Hino eclesiástico que se canta em domingo de Ramos.}{ho.sa.na}{0}
\verb{hosana}{}{Fig.}{}{}{}{Saudação, louvor.}{ho.sa.na}{0}
\verb{hosana}{}{}{}{}{interj.}{Expressão que denota aclamação, alegria; salve.}{ho.sa.na}{0}
\verb{hóspeda}{}{}{}{}{s.f.}{Mulher a quem se oferece hospedagem.}{hós.pe.da}{0}
\verb{hóspeda}{}{Desus.}{}{}{}{Mulher que hospeda.}{hós.pe.da}{0}
\verb{hospedagem}{}{}{"-ens}{}{s.f.}{Ato ou efeito de hospedar, de receber como hóspede.}{hos.pe.da.gem}{0}
\verb{hospedagem}{}{}{"-ens}{}{}{Hospedaria.}{hos.pe.da.gem}{0}
\verb{hospedagem}{}{Fig.}{"-ens}{}{}{Hospitalidade, bom acolhimento.}{hos.pe.da.gem}{0}
\verb{hospedar}{}{}{}{}{v.t.}{Dar hospedagem a, receber como hóspede.}{hos.pe.dar}{0}
\verb{hospedar}{}{}{}{}{}{Oferecer abrigo a, alojar.}{hos.pe.dar}{\verboinum{1}}
\verb{hospedaria}{}{}{}{}{s.f.}{Casa onde se recebem hóspedes, especialmente mediante pagamento; hospedagem, pousada.}{hos.pe.da.ri.a}{0}
\verb{hóspede}{}{}{}{}{adj.}{Estranho, alheio.}{hós.pe.de}{0}
\verb{hóspede}{}{Fig.}{}{}{}{Ignorante de alguma coisa; leigo.}{hós.pe.de}{0}
\verb{hóspede}{}{}{}{}{s.m.}{Indivíduo que se acomoda por tempo provisório em casa alheia.}{hós.pe.de}{0}
\verb{hóspede}{}{Desus.}{}{}{}{Hospedeiro.}{hós.pe.de}{0}
\verb{hóspede}{}{}{}{}{}{Indivíduo estranho, alheio; peregrino.}{hós.pe.de}{0}
\verb{hospedeiro}{ê}{}{}{}{adj.}{Que hospeda.}{hos.pe.dei.ro}{0}
\verb{hospedeiro}{ê}{Por ext.}{}{}{}{Afável, acolhedor.}{hos.pe.dei.ro}{0}
\verb{hospedeiro}{ê}{}{}{}{s.m.}{Indivíduo que dá hospedagem.}{hos.pe.dei.ro}{0}
\verb{hospedeiro}{ê}{}{}{}{}{Dono de hospedaria.}{hos.pe.dei.ro}{0}
\verb{hospedeiro}{ê}{Biol.}{}{}{}{Qualquer animal ou planta que abriga ou nutre outro organismo.}{hos.pe.dei.ro}{0}
\verb{hospício}{}{}{}{}{s.m.}{Asilo de loucos, hospital de alienados; manicômio.}{hos.pí.cio}{0}
\verb{hospício}{}{}{}{}{}{Estabelecimento onde se dá hospedagem ou tratamento gratuitos a pessoas pobres ou doentes; asilo.}{hos.pí.cio}{0}
\verb{hospício}{}{}{}{}{}{Lugar onde se recolhem e tratam animais abandonados.}{hos.pí.cio}{0}
\verb{hospital}{}{}{"-ais}{}{s.m.}{Estabelecimento onde se internam e tratam doentes e feridos.}{hos.pi.tal}{0}
\verb{hospital}{}{Desus.}{"-ais}{}{adj.2g.}{Que pratica a hospitalidade; caridoso, benévolo.}{hos.pi.tal}{0}
\verb{hospitalar}{}{}{}{}{adj.2g.}{Relativo a hospital ou a hospício.}{hos.pi.ta.lar}{0}
\verb{hospitaleiro}{ê}{}{}{}{adj.}{Que dá hospedagem por bondade ou caridade.}{hos.pi.ta.lei.ro}{0}
\verb{hospitaleiro}{ê}{}{}{}{}{Que dá boa hospitalidade, que acolhe francamente.}{hos.pi.ta.lei.ro}{0}
\verb{hospitaleiro}{ê}{}{}{}{s.m.}{Indivíduo que dá hospedagem. }{hos.pi.ta.lei.ro}{0}
\verb{hospitalidade}{}{}{}{}{s.f.}{Ato ou efeito de hospedar; acolhida de hóspedes.}{hos.pi.ta.li.da.de}{0}
\verb{hospitalidade}{}{}{}{}{}{Qualidade de hospitaleiro.}{hos.pi.ta.li.da.de}{0}
\verb{hospitalidade}{}{Por ext.}{}{}{}{Acolhimento afetuoso.}{hos.pi.ta.li.da.de}{0}
\verb{hospitalização}{}{}{"-ões}{}{s.f.}{Ato ou efeito de hospitalizar.}{hos.pi.ta.li.za.ção}{0}
\verb{hospitalizar}{}{}{}{}{v.t.}{Converter em hospital provisório.}{hos.pi.ta.li.zar}{0}
\verb{hospitalizar}{}{}{}{}{}{Internar em hospital.}{hos.pi.ta.li.zar}{\verboinum{1}}
\verb{hoste}{ó}{Desus.}{}{}{s.m.}{Inimigo, adversário.}{hos.te}{0}
\verb{hoste}{ó}{}{}{}{s.f.}{Força armada; tropa, exército.}{hos.te}{0}
\verb{hoste}{ó}{Fig.}{}{}{}{Bando, multidão.}{hos.te}{0}
\verb{hóstia}{}{Relig.}{}{}{s.f.}{Partícula circular de massa de pão ázimo, que é consagrada na missa.}{hós.tia}{0}
\verb{hóstia}{}{Relig.}{}{}{}{Vítima oferecida em sacrifício à divindade.}{hós.tia}{0}
\verb{hóstia}{}{Relig.}{}{}{}{Massa de pão ázimo que envolve certos alimentos e medicamentos.}{hós.tia}{0}
\verb{hostiário}{}{Relig.}{}{}{s.m.}{Caixa onde se guardam hóstias ainda não consagradas.}{hos.ti.á.rio}{0}
\verb{hostil}{}{}{"-is}{}{adj.2g.}{Que se opõe a; que manifesta inimizade; contrário, adverso.}{hos.til}{0}
\verb{hostil}{}{}{"-is}{}{}{Que revela agressividade; provocante.}{hos.til}{0}
\verb{hostilidade}{}{}{}{}{s.f.}{Qualidade do que é hostil.}{hos.ti.li.da.de}{0}
\verb{hostilidade}{}{}{}{}{}{Ato ou efeito de hostilizar; manifestação de agressividade, de rivalidade.}{hos.ti.li.da.de}{0}
\verb{hostilização}{}{}{"-ões}{}{s.f.}{Ato ou efeito de hostilizar; provocação.}{hos.ti.li.za.ção}{0}
\verb{hostilizar}{}{}{}{}{v.t.}{Tratar com agressividade ou inimizade.}{hos.ti.li.zar}{0}
\verb{hostilizar}{}{}{}{}{}{Ter sentimento hostil contra, acolher mal.}{hos.ti.li.zar}{0}
\verb{hostilizar}{}{}{}{}{}{Fazer guerra contra; provocar dano em.}{hos.ti.li.zar}{\verboinum{1}}
\verb{hot dog}{ó\ldots{}ó}{}{}{}{s.m.}{Sanduíche de pão com salsicha; cachorro"-quente. }{\textit{hot dog}}{0}
\verb{hotel}{é}{}{"-éis}{}{s.m.}{Estabelecimento que provê alojamento em quartos ou em apartamentos mobiliados e, habitualmente, refeições, entretenimentos e outros serviços para o público. }{ho.tel}{0}
\verb{hotelaria}{}{}{}{}{s.f.}{Rede de hotéis de uma região, cidade ou país.}{ho.te.la.ri.a}{0}
\verb{hotelaria}{}{}{}{}{}{Arte e técnica de dirigir ou administrar hotéis.}{ho.te.la.ri.a}{0}
\verb{hoteleiro}{ê}{}{}{}{s.m.}{Dono ou administrador de hotel.}{ho.te.lei.ro}{0}
\verb{hoteleiro}{ê}{}{}{}{adj.}{Relativo a hotéis.}{ho.te.lei.ro}{0}
\verb{huguenote}{ó}{}{}{}{adj.2g.}{Relativo à corrente religiosa dos huguenotes.}{hu.gue.no.te}{0}
\verb{huguenote}{ó}{}{}{}{s.2g.}{Designação depreciativa dada pelos católicos franceses aos seguidores do protestantismo, especialmente aos de orientação calvinista.}{hu.gue.no.te}{0}
\verb{hulha}{}{}{}{}{s.f.}{Espécie de carvão mineral que serve como combustível.}{hu.lha}{0}
\verb{hulha"-branca}{}{}{hulhas"-brancas}{}{s.f.}{Designação comum às cachoeiras ou quedas d'água como potenciais hidráulicos para produção de energia elétrica.  }{hu.lha"-bran.ca}{0}
\verb{hulheira}{ê}{}{}{}{s.f.}{Jazida ou mina de hulha.}{hu.lhei.ra}{0}
\verb{hulhífero}{}{}{}{}{adj.}{Que tem ou produz hulha.}{hu.lhí.fe.ro}{0}
\verb{hum}{}{}{}{}{interj.}{Expressão que denota dúvida, receio, inquietação, reticência, aprovação.}{hum}{0}
\verb{humanal}{}{}{"-ais}{}{adj.2g.}{Relativo a homem; humano.}{hu.ma.nal}{0}
\verb{humanar}{}{}{}{}{v.t.}{Tornar humano ou benévolo; humanizar.}{hu.ma.nar}{\verboinum{1}}
\verb{humanidade}{}{}{}{}{s.f.}{Conjunto de características específicas à natureza humana.}{hu.ma.ni.da.de}{0}
\verb{humanidade}{}{}{}{}{}{O conjunto dos seres humanos.}{hu.ma.ni.da.de}{0}
\verb{humanidade}{}{}{}{}{}{Sentimento de benevolência, bondade, compaixão.}{hu.ma.ni.da.de}{0}
\verb{humanidades}{}{}{}{}{s.f.pl.}{O estudo das letras clássicas.}{hu.ma.ni.da.des}{0}
\verb{humanismo}{}{Filos.}{}{}{s.m.}{Corrente filosófica interessada no desenvolvimento das potencialidades humanas para a constituição de uma sociedade mais harmônica.}{hu.ma.nis.mo}{0}
\verb{humanismo}{}{}{}{}{}{Doutrina e movimento dos humanistas da Renascença, que ressuscitaram o estudo das línguas e literaturas greco"-latinas.}{hu.ma.nis.mo}{0}
\verb{humanismo}{}{}{}{}{}{Formação do espírito humano pela cultura literária ou científica.}{hu.ma.nis.mo}{0}
\verb{humanista}{}{}{}{}{adj.2g.}{Relativo a humanismo.}{hu.ma.nis.ta}{0}
\verb{humanista}{}{}{}{}{s.2g.}{Indivíduo versado no estudo das humanidades.}{hu.ma.nis.ta}{0}
\verb{humanista}{}{}{}{}{}{Partidário do humanismo filosófico.}{hu.ma.nis.ta}{0}
\verb{humanitário}{}{}{}{}{adj.}{Que se dedica a promover o bem"-estar do homem; benfeitor.}{hu.ma.ni.tá.rio}{0}
\verb{humanitário}{}{}{}{}{}{Que ama os seus semelhantes; bondoso.}{hu.ma.ni.tá.rio}{0}
\verb{humanitário}{}{}{}{}{s.m.}{Indivíduo que trabalha para o bem geral da humanidade; filantropo.}{hu.ma.ni.tá.rio}{0}
\verb{humanitarismo}{}{}{}{}{s.m.}{Amor à humanidade; filantropia.}{hu.ma.ni.ta.ris.mo}{0}
\verb{humanização}{}{}{"-ões}{}{s.f.}{Ato ou efeito de humanizar, de tornar humano.}{hu.ma.ni.za.ção}{0}
\verb{humanizar}{}{}{}{}{v.t.}{Tornar humano, dar condição humana a; humanar.}{hu.ma.ni.zar}{0}
\verb{humanizar}{}{}{}{}{}{Tornar benévolo, afável.}{hu.ma.ni.zar}{0}
\verb{humanizar}{}{}{}{}{}{Tornar mais sociável, mais tratável; civilizar.}{hu.ma.ni.zar}{\verboinum{1}}
\verb{humano}{}{}{}{}{adj.}{Relativo ao homem.}{hu.ma.no}{0}
\verb{humano}{}{}{}{}{}{Que mostra piedade, compreensão para com outras pessoas; bondoso, humanitário.}{hu.ma.no}{0}
\verb{humano}{}{}{}{}{s.m.}{O ser humano; o homem. }{hu.ma.no}{0}
\verb{humanoide}{}{}{}{}{adj.2g.}{Que se assemelha a seres humanos. (\textit{É um robô humanoide que eles acabaram de inventar.})}{hu.ma.noi.de}{0}
\verb{humificação}{}{}{"-ões}{}{s.f.}{Transformação em humo.}{hu.mi.fi.ca.ção}{0}
\verb{humildade}{}{}{}{}{s.f.}{Qualidade de humilde.}{hu.mil.da.de}{0}
\verb{humildade}{}{}{}{}{}{Virtude caracterizada pela consciência das próprias limitações; modéstia, simplicidade, pobreza.}{hu.mil.da.de}{0}
\verb{humildade}{}{}{}{}{}{Sentimento de fraqueza, inferioridade em relação a alguém ou a alguma coisa.}{hu.mil.da.de}{0}
\verb{humildade}{}{}{}{}{}{Respeito para com superiores; reverência, submissão.}{hu.mil.da.de}{0}
\verb{humilde}{}{}{}{}{adj.2g.}{Que tem ou manifesta a virtude de conhecer suas próprias limitações.}{hu.mil.de}{0}
\verb{humilde}{}{}{}{}{}{Que manifesta sentimento de fraqueza, de modéstia; singelo.}{hu.mil.de}{0}
\verb{humilde}{}{}{}{}{}{Que expressa ou reflete submissão; respeitoso, acatador.}{hu.mil.de}{0}
\verb{humilde}{}{}{}{}{s.2g.}{Pessoa pobre, de condição modesta.}{hu.mil.de}{0}
\verb{humilhação}{}{}{"-ões}{}{s.f.}{Ato ou efeito de humilhar.}{hu.mi.lha.ção}{0}
\verb{humilhação}{}{}{"-ões}{}{}{Submissão, abatimento.}{hu.mi.lha.ção}{0}
\verb{humilhação}{}{}{"-ões}{}{}{Rebaixamento moral.}{hu.mi.lha.ção}{0}
\verb{humilhante}{}{}{}{}{adj.2g.}{Que humilha, rebaixa; vexatório, vergonhoso.}{hu.mi.lhan.te}{0}
\verb{humilhar}{}{}{}{}{v.t.}{Tornar humilde.}{hu.mi.lhar}{0}
\verb{humilhar}{}{}{}{}{}{Tornar desacreditado; vexar, rebaixar.}{hu.mi.lhar}{0}
\verb{humilhar}{}{}{}{}{}{Tratar com desdém, com soberba.}{hu.mi.lhar}{0}
\verb{humilhar}{}{}{}{}{}{Submeter, sujeitar.}{hu.mi.lhar}{\verboinum{1}}
\verb{humo}{}{}{}{}{s.m.}{Matéria resultante da decomposição dos restos vegetais e animais acumulada na parte superficial do solo; contribui para a nutrição vegetal.}{hu.mo}{0}
\verb{humor}{ô}{}{}{}{s.m.}{Estado psíquico que revela a disposição afetiva de um sujeito em certo momento; estado de espírito; temperamento.}{hu.mor}{0}
\verb{humor}{ô}{}{}{}{}{Faculdade de perceber e expressar o que é cômico ou divertido.}{hu.mor}{0}
\verb{humor}{ô}{}{}{}{}{Graça, jocosidade, espírito.}{hu.mor}{0}
\verb{humor}{ô}{Biol.}{}{}{}{Designação comum a certas substâncias orgânicas líquidas ou semilíquidas presentes no organismo.}{hu.mor}{0}
\verb{humoral}{}{}{"-ais}{}{adj.2g.}{Relativo ao conjunto de líquidos do organismo.}{hu.mo.ral}{0}
\verb{humoral}{}{}{"-ais}{}{}{Relativo ao humor.}{hu.mo.ral}{0}
\verb{humorismo}{}{}{}{}{s.m.}{Capacidade de perceber e expressar a comicidade de uma situação.}{hu.mo.ris.mo}{0}
\verb{humorismo}{}{}{}{}{}{Atividade de humorista.}{hu.mo.ris.mo}{0}
\verb{humorismo}{}{}{}{}{}{Qualidade do que manifesta humor, graça.}{hu.mo.ris.mo}{0}
\verb{humorista}{}{}{}{}{s.2g.}{Pessoa que manifesta humor com seus ditos, atos, desenhos.}{hu.mo.ris.ta}{0}
\verb{humorístico}{}{}{}{}{adj.}{Que revela ou tem humor, graça, feição irônica.}{hu.mo.rís.ti.co}{0}
\verb{húmus}{}{}{}{}{}{Var. de \textit{humo}.}{hú.mus}{0}
\verb{húngaro}{}{}{}{}{adj.}{Relativo à Hungria; magiar.}{hún.ga.ro}{0}
\verb{húngaro}{}{}{}{}{s.m.}{Indivíduo natural ou habitante desse país.}{hún.ga.ro}{0}
\verb{húngaro}{}{}{}{}{}{O idioma falado na Hungria. }{hún.ga.ro}{0}
\verb{huno}{}{}{}{}{adj.}{Relativo aos hunos.}{hu.no}{0}
\verb{huno}{}{}{}{}{s.m.}{Indivíduo pertencente aos hunos, antigo povo nômade, habitante da Ásia central.  }{hu.no}{0}
\verb{hurra}{}{}{}{}{interj.}{Expressão que denota alegria, aprovação, entusiasmo.}{hur.ra}{0}
\verb{hurra}{}{}{}{}{s.m.}{Grito ou exclamação de guerra entre os russos.}{hur.ra}{0}
\verb{husky}{}{}{}{}{s.m.}{Raça de cachorro, originária das regiões árticas da América do Norte.}{\textit{husky}}{0}
\verb{hussardo}{}{}{}{}{s.m.}{Cavaleiro húngaro.}{hus.sar.do}{0}
\verb{hussardo}{}{}{}{}{}{Soldado pertencente à cavalaria ligeira, na França e na Alemanha.}{hus.sar.do}{0}
\verb{hussita}{}{Relig.}{}{}{adj.2g.}{Que segue a doutrina de Jan Hus, reformista tcheco do fim do século \textsc{xiv} que pregava que as boas obras eram indiferentes para a salvação eterna.}{hus.si.ta}{0}
\verb{Hz}{}{}{}{}{}{Símb. de \textit{hertz}.}{Hz}{0}
