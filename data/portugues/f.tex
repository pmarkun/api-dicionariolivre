\verb{f}{éfe}{}{}{}{s.m.}{Sexta letra do alfabeto português.}{f}{0}
\verb{F}{}{Mat.}{}{}{}{No sistema hexadecimal, representa o décimo sexto algarismo, equivalente ao número decimal 15.}{F}{0}
\verb{F}{}{Mús.}{}{}{}{A nota ou o acorde referente ao \textit{fá}, ou à quarta nota da escala de \textit{dó}.}{F}{0}
\verb{F}{}{Quím.}{}{}{}{Símb. do \textit{flúor}. }{F}{0}
\verb{fá}{}{Mús.}{}{}{s.m.}{A quarta nota musical na escala de \textit{dó}.}{fá}{0}
\verb{fã}{}{}{}{}{s.2g.}{Indivíduo que tem e demonstra grande admiração por artistas de cinema, televisão, rádio, esportistas, políticos etc.}{fã}{0}
\verb{fábrica}{}{}{}{}{s.f.}{Estabelecimento industrial onde se processa a transformação de matéria"-prima em produtos destinados ao consumo.}{fá.bri.ca}{0}
\verb{fábrica}{}{}{}{}{}{O pessoal que trabalha nesse estabelecimento.}{fá.bri.ca}{0}
\verb{fábrica}{}{}{}{}{}{O processo completo de industrialização; fabrico, fabricação.}{fá.bri.ca}{0}
\verb{fabricação}{}{}{"-ões}{}{s.f.}{Ato ou efeito de fabricar, manufaturar; fábrica, fabrico. }{fa.bri.ca.ção}{0}
\verb{fabricante}{}{}{}{}{s.2g.}{Proprietário ou dirigente do estabelecimento, da maquinaria e dos produtos relacionados à atividade fabril.}{fa.bri.can.te}{0}
\verb{fabricante}{}{}{}{}{}{Operário que trabalha na fabricação.}{fa.bri.can.te}{0}
\verb{fabricar}{}{}{}{}{v.t.}{Transformar matérias"-primas em produtos de consumo; manufaturar, produzir.}{fa.bri.car}{0}
\verb{fabricar}{}{}{}{}{}{Executar a construção; edificar.}{fa.bri.car}{0}
\verb{fabricar}{}{}{}{}{}{Inventar, idear, maquinar.}{fa.bri.car}{\verboinum{2}}
\verb{fabrico}{}{}{}{}{s.m.}{Ato ou efeito de fabricar; fabricação, manufatura, produção.}{fa.bri.co}{0}
\verb{fabril}{}{}{"-is}{}{adj.2g.}{Relativo a fábrica, fabrico ou fabricante.}{fa.bril}{0}
\verb{fábula}{}{Liter.}{}{}{s.f.}{Narrativa curta que tem, geralmente, como personagens animais agindo como seres humanos, e que ilustra um preceito moral.}{fá.bu.la}{0}
\verb{fábula}{}{}{}{}{}{Mito, ficção, lenda.}{fá.bu.la}{0}
\verb{fábula}{}{Fig.}{}{}{}{Grande quantia de dinheiro.}{fá.bu.la}{0}
\verb{fabulação}{}{}{"-ões}{}{s.f.}{Versão romanceada de um acontecimento.}{fa.bu.la.ção}{0}
\verb{fabulação}{}{}{"-ões}{}{}{Invenção, fantasia, mentira.}{fa.bu.la.ção}{0}
\verb{fabular}{}{}{}{}{v.t.}{Narrar em forma de fábula.}{fa.bu.lar}{0}
\verb{fabular}{}{}{}{}{}{Fingir, fantasiar, inventar.}{fa.bu.lar}{0}
\verb{fabular}{}{}{}{}{}{Falar sem fundamento; mentir.}{fa.bu.lar}{\verboinum{1}}
\verb{fabulário}{}{Liter.}{}{}{s.m.}{Coleção de fábulas.}{fa.bu.lá.rio}{0}
\verb{fabulista}{}{Liter.}{}{}{s.2g.}{Autor de fábulas.}{fa.bu.lis.ta}{0}
\verb{fabuloso}{ô}{}{"-osos ⟨ó⟩}{"-osa ⟨ó⟩}{adj.}{Relativo a fábula, ficção; inventado, imaginado.}{fa.bu.lo.so}{0}
\verb{fabuloso}{ô}{}{"-osos ⟨ó⟩}{"-osa ⟨ó⟩}{}{Que tem caráter admirável; incrível, espantoso.}{fa.bu.lo.so}{0}
\verb{fabuloso}{ô}{}{"-osos ⟨ó⟩}{"-osa ⟨ó⟩}{}{Excelente, ótimo, fantástico.}{fa.bu.lo.so}{0}
\verb{faca}{}{}{}{}{s.f.}{Instrumento cortante constituído de lâmina de um só gume presa a um cabo.}{fa.ca}{0}
\verb{faca}{}{Fig.}{}{}{}{Bisturi.}{fa.ca}{0}
\verb{facada}{}{}{}{}{s.f.}{Golpe ou ferida por faca.}{fa.ca.da}{0}
\verb{facada}{}{Fig.}{}{}{}{Surpresa dolorosa; desgosto inesperado.}{fa.ca.da}{0}
\verb{facada}{}{Fig.}{}{}{}{Ato de pedir dinheiro a alguém.}{fa.ca.da}{0}
\verb{façanha}{}{}{}{}{s.f.}{Feito heroico; ato excepcional.}{fa.ça.nha}{0}
\verb{façanha}{}{}{}{}{}{Ação escandalosa, imprudente.}{fa.ça.nha}{0}
\verb{facão}{}{}{"-ões}{}{s.m.}{Grande faca usada para abrir caminho no mato.}{fa.cão}{0}
\verb{facão}{}{}{"-ões}{}{}{Espada, sabre.}{fa.cão}{0}
\verb{facção}{}{}{"-ões}{}{s.f.}{Grupo dissidente de um partido político.}{fac.ção}{0}
\verb{facção}{}{}{"-ões}{}{}{Grupo de indivíduos partidários de uma mesma causa em oposição à de outros grupos.}{fac.ção}{0}
\verb{facção}{}{}{"-ões}{}{}{Bando ou partido revoltoso, sedicioso.}{fac.ção}{0}
\verb{facciosismo}{}{}{}{}{s.m.}{Qualidade de quem é faccioso; sectarismo.}{fac.ci.o.sis.mo}{0}
\verb{facciosismo}{}{}{}{}{}{Paixão cega e exacerbada.}{fac.ci.o.sis.mo}{0}
\verb{faccioso}{ô}{}{"-osos ⟨ó⟩}{"-osa ⟨ó⟩}{adj.}{Que tem espírito sectarista; parcial.}{fac.ci.o.so}{0}
\verb{faccioso}{ô}{}{"-osos ⟨ó⟩}{"-osa ⟨ó⟩}{}{Que exerce ação apaixonada e violenta; sedicioso.}{fac.ci.o.so}{0}
\verb{face}{}{}{}{}{s.f.}{Parte da cabeça entre a testa e o queixo; cara, rosto, semblante.}{fa.ce}{0}
\verb{face}{}{}{}{}{}{Cada um dos lados do rosto.}{fa.ce}{0}
\verb{face}{}{}{}{}{}{Cada um dos lados planos de alguma coisa. (\textit{As duas faces da moeda.})}{fa.ce}{0}
\verb{facear}{}{}{}{}{v.t.}{Fazer faces ou lados.}{fa.ce.ar}{0}
\verb{facear}{}{}{}{}{}{Estar à face; mostrar"-se à frente.}{fa.ce.ar}{\verboinum{4}}
\verb{facécia}{}{}{}{}{s.f.}{Qualidade ou modos de faceto, cômico; brincadeira.}{fa.cé.cia}{0}
\verb{facécia}{}{}{}{}{}{Chacota, gracejo, pilhéria.}{fa.cé.cia}{0}
\verb{faceirice}{}{}{}{}{s.f.}{Exibição de elegância; ar pretensioso; afetação.}{fa.cei.ri.ce}{0}
\verb{faceirice}{}{}{}{}{}{Aparência brincalhona; ar risonho.}{fa.cei.ri.ce}{0}
\verb{faceiro}{ê}{}{}{}{adj.}{Que gosta de se enfeitar; elegante, garboso, janota.}{fa.cei.ro}{0}
\verb{faceiro}{ê}{}{}{}{}{Alegre, contente, risonho.}{fa.cei.ro}{0}
\verb{faceta}{ê}{}{}{}{s.f.}{Pequena superfície plana de um objeto; lado, face.}{fa.ce.ta}{0}
\verb{faceta}{ê}{Fig.}{}{}{}{Aspecto peculiar pelo qual se considera uma questão, um objeto, uma pessoa.}{fa.ce.ta}{0}
\verb{facetar}{}{}{}{}{v.t.}{Fazer facetas; talhar, lapidar.}{fa.ce.tar}{0}
\verb{facetar}{}{Fig.}{}{}{}{Aprimorar, aperfeiçoar.}{fa.ce.tar}{\verboinum{1}}
\verb{faceto}{ê}{}{}{}{adj.}{Que tem caráter brincalhão; cômico, alegre.}{fa.ce.to}{0}
\verb{fachada}{}{}{}{}{s.f.}{Face externa de um prédio voltada para a rua; frente.}{fa.cha.da}{0}
\verb{fachada}{}{Fig.}{}{}{}{Aparência exterior; cara, semblante.}{fa.cha.da}{0}
\verb{facho}{}{}{}{}{s.m.}{Matéria inflamável cuja luz serve de sinal; archote, farol, luzeiro.}{fa.cho}{0}
\verb{facho}{}{Fig.}{}{}{}{Aquilo que esclarece ou serve de luz espiritual; guia.}{fa.cho}{0}
\verb{facial}{}{}{"-ais}{}{adj.2g.}{Relativo a face.}{fa.ci.al}{0}
\verb{fácil}{}{}{"-eis}{}{adj.2g.}{Que se faz ou se obtém sem dificuldade.}{fá.cil}{0}
\verb{fácil}{}{}{"-eis}{}{}{Simples, claro, compreensível.}{fá.cil}{0}
\verb{fácil}{}{}{"-eis}{}{}{De temperamento brando; complacente, tolerante.}{fá.cil}{0}
\verb{fácil}{}{Pop.}{"-eis}{}{}{Diz"-se do indivíduo de honestidade e moral duvidosa.}{fá.cil}{0}
\verb{facilidade}{}{}{}{}{s.f.}{Característica do que se faz ou se obtém sem dificuldade.}{fa.ci.li.da.de}{0}
\verb{facilidade}{}{}{}{}{}{Pendor, aptidão, dom.}{fa.ci.li.da.de}{0}
\verb{facilidade}{}{}{}{}{}{Situação oportuna; possibilidade, ocasião.}{fa.ci.li.da.de}{0}
\verb{facilidades}{}{}{}{}{s.f.pl.}{Meios cômodos e disponíveis para se realizar ou obter algo.}{fa.ci.li.da.des}{0}
\verb{facilidades}{}{}{}{}{}{Condescendência, complacência.}{fa.ci.li.da.des}{0}
\verb{facilitar}{}{}{}{}{v.t.}{Tornar fácil.}{fa.ci.li.tar}{0}
\verb{facilitar}{}{}{}{}{}{Pôr à disposição; facultar, prontificar.}{fa.ci.li.tar}{0}
\verb{facilitar}{}{}{}{}{v.i.}{Agir com imprudência; expor"-se ao perigo.}{fa.ci.li.tar}{\verboinum{1}}
\verb{facínora}{}{}{}{}{adj.2g.}{Que cometeu um crime com crueldade ou perversidade. }{fa.cí.no.ra}{0}
\verb{facínora}{}{}{}{}{s.2g.}{Indivíduo perverso e criminoso.}{fa.cí.no.ra}{0}
\verb{fã"-clube}{}{}{fãs"-clubes \textit{ou } fãs"-clube}{}{s.m.}{Associação de admiradores de um artista ou de um grupo de artistas.}{fã"-clu.be}{0}
\verb{fac"-similado}{}{}{}{}{adj.}{Reproduzido em fac"-símile.}{fac"-si.mi.la.do}{0}
\verb{fac"-similar}{}{}{fac"-similares}{}{adj.2g.}{Relativo a fac"-símile.}{fac"-si.mi.lar}{0}
\verb{fac"-similar}{}{}{fac"-similares}{}{v.t.}{Imprimir em fac"-símile.}{fac"-si.mi.lar}{\verboinum{1}}
\verb{fac"-símile}{}{}{fac"-símiles}{}{s.m.}{Reprodução exata, por meios fotomecânicos, de textos ou imagens.}{fac"-sí.mi.le}{0}
\verb{factício}{}{}{}{}{adj.}{Que não demonstra naturalidade; artificial, convencional.}{fac.tí.cio}{0}
\verb{factível}{}{}{"-eis}{}{adj.2g.}{Que pode ser feito; realizável.}{fac.tí.vel}{0}
\verb{factótum}{}{}{"-uns}{}{s.m.}{Pessoa que se encarrega de todos os negócios e afazeres de outrem.}{fac.tó.tum}{0}
\verb{factótum}{}{}{"-uns}{}{}{Indivíduo imprescindível.}{fac.tó.tum}{0}
\verb{factual}{}{}{"-ais}{}{adj.2g.}{Relativo a fatos; que se baseia em fatos; palpável, real.}{fac.tu.al}{0}
\verb{faculdade}{}{}{}{}{s.f.}{Poder de realizar algo; capacidade.}{fa.cul.da.de}{0}
\verb{faculdade}{}{}{}{}{}{Livre escolha; opção.}{fa.cul.da.de}{0}
\verb{faculdade}{}{}{}{}{}{Escola de ensino superior. (\textit{No próximo ano, meu irmão prestará vestibular para entrar na faculdade.})}{fa.cul.da.de}{0}
\verb{facultar}{}{}{}{}{v.t.}{Dar permissão; conceder, facilitar.}{fa.cul.tar}{0}
\verb{facultar}{}{}{}{}{}{Proporcionar, oferecer, ensejar.}{fa.cul.tar}{\verboinum{1}}
\verb{facultativo}{}{}{}{}{adj.}{Que pode ser feito ou não, que não é obrigatório; opcional.}{fa.cul.ta.ti.vo}{0}
\verb{facultativo}{}{}{}{}{}{Que concede um direito ou poder.}{fa.cul.ta.ti.vo}{0}
\verb{facúndia}{}{}{}{}{s.f.}{Facilidade para discursar; eloquência. }{fa.cún.dia}{0}
\verb{facundo}{}{}{}{}{adj.}{Que tem facilidade para discursar; eloquente, fluente.}{fa.cun.do}{0}
\verb{fada}{}{}{}{}{s.f.}{Ser fabuloso a que se atribui o poder mágico de influenciar no destino das pessoas.}{fa.da}{0}
\verb{fada}{}{Fig.}{}{}{}{Mulher notável pela beleza, bondade e encanto.}{fa.da}{0}
\verb{fadado}{}{}{}{}{adj.}{Que tem o destino traçado; predestinado.}{fa.da.do}{0}
\verb{fadar}{}{}{}{}{v.t.}{Determinar com antecipação; predestinar, vaticinar.}{fa.dar}{0}
\verb{fadar}{}{}{}{}{}{Conceder dons excepcionais; favorecer, dotar.}{fa.dar}{\verboinum{1}}
\verb{fadário}{}{}{}{}{s.m.}{Destino imposto por um poder sobrenatural e do qual não se pode fugir; fado.}{fa.dá.rio}{0}
\verb{fadário}{}{}{}{}{}{Vida trabalhosa e difícil; desgostos.}{fa.dá.rio}{0}
\verb{fadiga}{}{}{}{}{s.f.}{Sensação de esgotamento físico; cansaço, estafa.}{fa.di.ga}{0}
\verb{fadiga}{}{}{}{}{}{Trabalho cansativo; lida, faina.}{fa.di.ga}{0}
\verb{fadigar}{}{}{}{}{v.t.}{Causar fadiga; cansar, fatigar.}{fa.di.gar}{\verboinum{5}}
\verb{fadista}{}{}{}{}{s.2g.}{Músico que toca ou canta fados.}{fa.dis.ta}{0}
\verb{fado}{}{}{}{}{s.m.}{Aquilo que inevitavelmente acontecerá; destino, sorte.}{fa.do}{0}
\verb{fado}{}{Mús.}{}{}{}{Canção e dança populares de Portugal, sempre acompanhadas pela guitarra portuguesa.}{fa.do}{0}
\verb{fagácea}{}{Bot.}{}{}{s.f.}{Espécime das fagáceas, família de árvores dos climas temperados que inclui o carvalho, o castanheiro e a faia.}{fa.gá.cea}{0}
\verb{fagócito}{}{Biol.}{}{}{s.m.}{Célula capaz de englobar e digerir micróbios e partículas orgânicas e inorgânicas.}{fa.gó.ci.to}{0}
\verb{fagocitose}{ó}{Biol.}{}{}{s.f.}{Processo de captura e digestão de  micróbios ou substâncias orgânicas e inorgânicas  pelos fagócitos.}{fa.go.ci.to.se}{0}
\verb{fagote}{ó}{Mús.}{}{}{s.m.}{Instrumento de sopro de som grave, geralmente de madeira e com palheta dupla.}{fa.go.te}{0}
\verb{fagotista}{}{}{}{}{s.2g.}{Músico que toca o fagote.}{fa.go.tis.ta}{0}
\verb{fagueiro}{ê}{}{}{}{adj.}{Que afaga; meigo, carinhoso.}{fa.guei.ro}{0}
\verb{fagueiro}{ê}{}{}{}{}{Agradável, prazeroso, afável.}{fa.guei.ro}{0}
\verb{fagueiro}{ê}{}{}{}{}{Alegre, satisfeito, contente.}{fa.guei.ro}{0}
\verb{fagulha}{}{}{}{}{s.f.}{Faísca que se solta de um corpo em brasa ou do atrito entre dois corpos; centelha, chispa.}{fa.gu.lha}{0}
\verb{fagulhar}{}{}{}{}{v.i.}{Emitir fagulhas; faiscar, cintilar.}{fa.gu.lhar}{\verboinum{1}}
\verb{faia}{}{Bot.}{}{}{s.f.}{Árvore ornamental de casca lisa e cinzenta.}{fai.a}{0}
\verb{faiança}{}{}{}{}{s.f.}{Louça de argila ou de pó de pedra, recoberta com um verniz impermeável e opaco e ricamente decorada.}{fai.an.ça}{0}
\verb{faina}{}{}{}{}{s.f.}{Trabalho árduo e extenso; lida, azáfama.}{fai.na}{0}
\verb{faina}{}{}{}{}{}{Trabalho de bordo, do qual participa  a tripulação de um navio.}{fai.na}{0}
\verb{faisão}{}{Zool.}{"-ões \textit{ou} -ães}{}{s.m.}{Ave da família do galo, de penas coloridas e cauda comprida.}{fai.são}{0}
\verb{faísca}{}{}{}{}{s.f.}{Partícula incandescente que se desprende de um corpo em brasa ou do atrito entre dois corpos; fagulha, chispa, centelha.}{fa.ís.ca}{0}
\verb{faísca}{}{}{}{}{}{Cintilação que acompanha uma descarga elétrica; raio.}{fa.ís.ca}{0}
\verb{faísca}{}{}{}{}{}{Palheta de ouro perdida na terra ou na areia das minas.}{fa.ís.ca}{0}
\verb{faísca}{}{Fig.}{}{}{}{Brilho espiritual; graça, vivacidade.}{fa.ís.ca}{0}
\verb{faiscador}{ô}{}{}{}{s.m.}{Indivíduo que procura faíscas de ouro perdidas no solo; garimpeiro.}{fa.is.ca.dor}{0}
\verb{faiscante}{}{}{}{}{adj.2g.}{Que produz faíscas luminosas; brilhante, cintilante.}{fa.is.can.te}{0}
\verb{faiscar}{}{}{}{}{v.t.}{Emitir faíscas; brilhar, cintilar.}{fa.is.car}{0}
\verb{faiscar}{}{Fig.}{}{}{v.i.}{Sobressair"-se, distinguir"-se, brilhar.}{fa.is.car}{0}
\verb{faiscar}{}{}{}{}{}{Procurar faíscas de ouro no solo das minas garimpadas.}{fa.is.car}{\verboinum{2}}
\verb{faixa}{ch}{}{}{}{s.f.}{Qualquer tira de pano ou de outro material.}{fai.xa}{0}
\verb{faixa}{ch}{}{}{}{}{Aquilo que se apresenta em forma de uma tira ou listra.}{fai.xa}{0}
\verb{faixa}{ch}{}{}{}{}{Intervalo entre dois limites dados. (\textit{Grande parte da população mundial vive abaixo da faixa de pobreza.})}{fai.xa}{0}
\verb{fajuto}{}{}{}{}{adj.}{De má qualidade; malfeito, ruim.}{fa.ju.to}{0}
\verb{fajuto}{}{}{}{}{}{Adulterado, falsificado.}{fa.ju.to}{0}
\verb{fala}{}{}{}{}{s.f.}{Ato ou efeito de falar.}{fa.la}{0}
\verb{fala}{}{}{}{}{}{Faculdade de exprimir o pensamento através de palavras.}{fa.la}{0}
\verb{falação}{}{Pop.}{"-ões}{}{s.f.}{Conversa desconexa e sem importância; falatório, palavreado, palração.}{fa.la.ção}{0}
\verb{falácia}{}{}{}{}{s.f.}{Qualidade do que é falaz; ilusão, quimera, engano.}{fa.lá.cia}{0}
\verb{falácia}{}{}{}{}{}{Argumento capcioso que induz ao erro; sofisma.}{fa.lá.cia}{0}
\verb{falacioso}{ô}{}{"-osos ⟨ó⟩}{"-osa ⟨ó⟩}{adj.}{Que contém falácia; enganoso, ilusório.}{fa.la.ci.o.so}{0}
\verb{faladeira}{ê}{}{}{}{adj.}{Diz"-se da mulher que fala muito; tagarela, mexeriqueira.}{fa.la.dei.ra}{0}
\verb{falado}{}{}{}{}{adj.}{Que se exprime pela palavra; dito, proferido.}{fa.la.do}{0}
\verb{falado}{}{}{}{}{}{Sobre o que se falou; citado, referido, mencionado.}{fa.la.do}{0}
\verb{falado}{}{}{}{}{}{De que muito se fala; afamado, celebrado.}{fa.la.do}{0}
\verb{falado}{}{}{}{}{}{Combinado, ajustado, acordado.}{fa.la.do}{0}
\verb{falado}{}{}{}{}{}{Que tem má fama; de que se fala mal.}{fa.la.do}{0}
\verb{falador}{ô}{}{}{}{adj.}{Que fala muito; loquaz, tagarela.}{fa.la.dor}{0}
\verb{falador}{ô}{}{}{}{}{Que faz intrigas; mexeriqueiro, maledicente.}{fa.la.dor}{0}
\verb{falange}{}{}{}{}{s.f.}{Corpo de tropas, de infantaria.}{fa.lan.ge}{0}
\verb{falange}{}{}{}{}{}{Grande agrupamento de pessoas; legião, multidão. }{fa.lan.ge}{0}
\verb{falange}{}{}{}{}{}{Grupo marginal que atua de forma organizada na sociedade para fins ilícitos.}{fa.lan.ge}{0}
\verb{falange}{}{Anat.}{}{}{}{Cada um dos ossos que formam os dedos das mãos e dos pés.}{fa.lan.ge}{0}
\verb{falangeta}{ê}{Anat.}{}{}{s.f.}{A última falange dos dedos, onde crescem as unhas; denominação substituída por \textit{falange distal}.}{fa.lan.ge.ta}{0}
\verb{falanginha}{}{Anat.}{}{}{s.f.}{A falange do meio, nos dedos que possuem três; denominação substituída por \textit{falange medial}.}{fa.lan.gi.nha}{0}
\verb{falante}{}{}{}{}{adj.2g.}{Que fala.}{fa.lan.te}{0}
\verb{falante}{}{}{}{}{}{Diz"-se daquele que fala muito e com desembaraço.}{fa.lan.te}{0}
\verb{falante}{}{}{}{}{s.2g.}{Indivíduo capaz de se expressar em um idioma.}{fa.lan.te}{0}
\verb{falante}{}{}{}{}{s.m.}{No discurso, pessoa que enuncia, por oposição a \textit{ouvinte}; emissor, locutor.}{fa.lan.te}{0}
\verb{falar}{}{}{}{}{v.t.}{Exprimir o pensamento por meio de palavras; dizer.}{fa.lar}{0}
\verb{falar}{}{}{}{}{}{Saber usar outro idioma.}{fa.lar}{\verboinum{1}}
\verb{falastrão}{}{}{"-ões}{}{adj.}{Que fala muito, de maneira empolada e afetada; falador.}{fa.las.trão}{0}
\verb{falatório}{}{}{}{}{s.m.}{Ruído de muitas vozes simultâneas; burburinho.}{fa.la.tó.rio}{0}
\verb{falatório}{}{}{}{}{}{Conversa sobre coisa sem importância; palavreado, falação.}{fa.la.tó.rio}{0}
\verb{falatório}{}{}{}{}{}{Maledicência, diz"-que"-diz, mexerico.}{fa.la.tó.rio}{0}
\verb{falaz}{}{}{}{}{adj.2g.}{Que ilude; vão, ilusório, quimérico.}{fa.laz}{0}
\verb{falaz}{}{}{}{}{}{Que engana; ardiloso, fraudulento, capcioso.}{fa.laz}{0}
\verb{falcão}{}{Zool.}{"-ões}{}{s.m.}{Ave de bico curvo e garras muito fortes, que vive da caça.}{fal.cão}{0}
\verb{falcatrua}{}{}{}{}{s.f.}{Artifício, artimanha para ludibriar; logro, fraude, ardil.}{fal.ca.tru.a}{0}
\verb{falcoaria}{}{}{}{}{s.f.}{Técnica de adestrar falcões para caça; caçada com falcões.}{fal.co.a.ri.a}{0}
\verb{falcoaria}{}{}{}{}{}{Local onde se criam falcões.}{fal.co.a.ri.a}{0}
\verb{falconídeo}{}{Zool.}{}{}{s.m.}{Espécime dos falconídeos, família de aves de rapina semelhantes ao falcão, e que inclui os carcarás, ximangos etc. }{fal.co.ní.deo}{0}
\verb{falconídeo}{}{}{}{}{adj.}{Relativo a falcão.}{fal.co.ní.deo}{0}
\verb{falda}{}{}{}{}{s.f.}{Base de montanha, serra ou monte; sopé, encosta.}{fal.da}{0}
\verb{falecer}{ê}{}{}{}{v.i.}{Deixar de existir; morrer, expirar.}{fa.le.cer}{0}
\verb{falecer}{ê}{}{}{}{}{Faltar, escassear, carecer.}{fa.le.cer}{0}
\verb{falecer}{ê}{}{}{}{}{Ser insuficiente.}{fa.le.cer}{\verboinum{15}}
\verb{falecido}{}{}{}{}{adj.}{Que faleceu; morto, finado.}{fa.le.ci.do}{0}
\verb{falecido}{}{}{}{}{}{Que carece de algo; necessitado.}{fa.le.ci.do}{0}
\verb{falecimento}{}{}{}{}{s.m.}{Ato ou efeito de falecer; morte, óbito, passamento.}{fa.le.ci.men.to}{0}
\verb{falecimento}{}{}{}{}{}{Falta, carência, privação.}{fa.le.ci.men.to}{0}
\verb{falência}{}{}{}{}{s.f.}{Ato ou efeito de falir; quebra, bancarrota.}{fa.lên.cia}{0}
\verb{falência}{}{}{}{}{}{Fracasso, ruína, malogro.}{fa.lên.cia}{0}
\verb{falência}{}{}{}{}{}{Falha, falta, omissão.}{fa.lên.cia}{0}
\verb{falésia}{}{Geol.}{}{}{s.f.}{Forma de relevo na costa litorânea caracterizada por escarpas íngremes e abruptas.}{fa.lé.sia}{0}
\verb{falha}{}{}{}{}{s.f.}{Fenda em uma superfície; lasca, fratura.}{fa.lha}{0}
\verb{falha}{}{}{}{}{}{Defeito físico ou moral; deficiência.}{fa.lha}{0}
\verb{falha}{}{}{}{}{}{Espaço vazio; lacuna, falta.}{fa.lha}{0}
\verb{falha}{}{}{}{}{}{Deslize, descuido, omissão.}{fa.lha}{0}
\verb{falha}{}{Geol.}{}{}{}{Fratura ao longo do qual houve deslocamento de blocos rochosos contíguos, um contra o outro. }{fa.lha}{0}
\verb{falhar}{}{}{}{}{v.t.}{Fazer falha; fender, rachar.}{fa.lhar}{0}
\verb{falhar}{}{}{}{}{v.i.}{Cometer falta; errar.}{fa.lhar}{0}
\verb{falhar}{}{}{}{}{}{Faltar a obrigação ou promessa; deixar de cumprir.}{fa.lhar}{0}
\verb{falhar}{}{}{}{}{}{Não funcionar ou funcionar mal.}{fa.lhar}{0}
\verb{falhar}{}{}{}{}{}{Não suceder como se esperava; malograr, frustrar.}{fa.lhar}{\verboinum{1}}
\verb{falho}{}{}{}{}{adj.}{Que apresenta falha; trincado, lascado.}{fa.lho}{0}
\verb{falho}{}{}{}{}{}{A que falta algo; carente, desprovido.}{fa.lho}{0}
\verb{falho}{}{}{}{}{}{Frustrado, malogrado.}{fa.lho}{0}
\verb{falibilidade}{}{}{}{}{s.f.}{Qualidade do que é falível; possibilidade de engano, erro.}{fa.li.bi.li.da.de}{0}
\verb{fálico}{}{}{}{}{adj.}{Relativo a falo ou a seu culto.}{fá.li.co}{0}
\verb{fálico}{}{}{}{}{}{Semelhante ao falo.}{fá.li.co}{0}
\verb{falido}{}{}{}{}{adj.}{Que abriu falência; quebrado.}{fa.li.do}{0}
\verb{falido}{}{}{}{}{}{Arruinado, fracassado.}{fa.li.do}{0}
\verb{falir}{}{}{}{}{v.i.}{Suspender o pagamento aos credores; abrir falência; quebrar.}{fa.lir}{0}
\verb{falir}{}{}{}{}{}{Ser malsucedido; fracassar, malograr.}{fa.lir}{0}
\verb{falir}{}{}{}{}{}{Faltar, minguar, escassear.}{fa.lir}{\verboinum{35}\verboirregular{\emph{def.} falimos, falis}}
\verb{falível}{}{}{"-eis}{}{adj.2g.}{Que pode falhar ou faltar.}{fa.lí.vel}{0}
\verb{falível}{}{}{"-eis}{}{}{Sujeito a erro, a engano.}{fa.lí.vel}{0}
\verb{falo}{}{}{}{}{s.m.}{Representação do pênis em ereção como símbolo da fecundidade e da virilidade.}{fa.lo}{0}
\verb{falo}{}{}{}{}{}{O próprio pênis.}{fa.lo}{0}
\verb{falripas}{}{}{}{}{s.f.pl.}{Cabelos curtos e ralos; farripas.}{fal.ri.pas}{0}
\verb{falsário}{}{}{}{}{adj.}{Que falsifica qualquer coisa; falsificador.}{fal.sá.rio}{0}
\verb{falsário}{}{}{}{}{}{Que jura falso; perjuro.}{fal.sá.rio}{0}
\verb{falseamento}{}{}{}{}{s.m.}{Ato ou efeito de falsear; falsificação, deturpação.}{fal.se.a.men.to}{0}
\verb{falsear}{}{}{}{}{v.t.}{Tornar falso; falsificar.}{fal.se.ar}{0}
\verb{falsear}{}{}{}{}{}{Fazer alteração; deturpar, desvirtuar.}{fal.se.ar}{0}
\verb{falsear}{}{}{}{}{}{Enganar, atraiçoar, trair.}{fal.se.ar}{0}
\verb{falsear}{}{}{}{}{v.i.}{Pisar em falso.}{fal.se.ar}{\verboinum{4}}
\verb{falseta}{ê}{}{}{}{s.f.}{Ato desleal; falsidade.}{fal.se.ta}{0}
\verb{falsete}{ê}{Mús.}{}{}{s.m.}{Registro agudo produzido por voz masculina.}{fal.se.te}{0}
\verb{falsidade}{}{}{}{}{s.f.}{Qualidade do que é falso, do que é contrário à verdade.}{fal.si.da.de}{0}
\verb{falsidade}{}{}{}{}{}{Fraude, mentira, calúnia.}{fal.si.da.de}{0}
\verb{falsidade}{}{}{}{}{}{Hipocrisia, deslealdade, perfídia.}{fal.si.da.de}{0}
\verb{falsificação}{}{}{"-ões}{}{s.f.}{Ato ou efeito de falsificar; adulteração.}{fal.si.fi.ca.ção}{0}
\verb{falsificador}{ô}{}{}{}{adj.}{Que realiza falsificações; falsário. }{fal.si.fi.ca.dor}{0}
\verb{falsificar}{}{}{}{}{v.t.}{Dar aparência enganosa com o fim de fraudar ou iludir.}{fal.si.fi.car}{0}
\verb{falsificar}{}{}{}{}{}{Adulterar, desvirtuar, contrafazer.}{fal.si.fi.car}{\verboinum{2}}
\verb{falso}{}{}{}{}{adj.}{Contrário à verdade; mentiroso, fingido.}{fal.so}{0}
\verb{falso}{}{}{}{}{}{Que aparenta ser o que não é; ilegítimo, adulterado.}{fal.so}{0}
\verb{falta}{}{}{}{}{s.f.}{Ato ou efeito de faltar; ausência, privação.}{fal.ta}{0}
\verb{falta}{}{}{}{}{}{Erro, engano, imperfeição.}{fal.ta}{0}
\verb{falta}{}{}{}{}{}{Culpa, pecado, ofensa.}{fal.ta}{0}
\verb{falta}{}{}{}{}{}{Não comparecimento às aulas, ao trabalho etc.}{fal.ta}{0}
\verb{falta}{}{Esport.}{}{}{}{Transgressão das regras de um jogo ou esporte; infração.}{fal.ta}{0}
\verb{faltar}{}{}{}{}{v.i.}{Não existir; deixar de haver; carecer, escassear.}{fal.tar}{0}
\verb{faltar}{}{}{}{}{}{Não ter em quantidade suficiente; haver menos do que o necessário.}{fal.tar}{0}
\verb{faltar}{}{}{}{}{}{Não comparecer; ausentar"-se.}{fal.tar}{0}
\verb{faltar}{}{}{}{}{}{Desaparecer, falecer, morrer.}{fal.tar}{0}
\verb{faltar}{}{}{}{}{}{Deixar de fazer, de cumprir; omitir"-se, falhar.}{fal.tar}{\verboinum{1}}
\verb{falto}{}{}{}{}{adj.}{Que nada possui; carente, necessitado.}{fal.to}{0}
\verb{falto}{}{}{}{}{}{Desprovido, falho.}{fal.to}{0}
\verb{faltoso}{ô}{}{"-osos ⟨ó⟩}{"-osa ⟨ó⟩}{adj.}{Que cometeu falta; culpado.}{fal.to.so}{0}
\verb{faltoso}{ô}{}{"-osos ⟨ó⟩}{"-osa ⟨ó⟩}{}{Que costuma faltar, que não comparece.}{fal.to.so}{0}
\verb{falua}{}{}{}{}{s.f.}{Embarcação à vela, maior que um bote, usada na descarga de navios.}{fa.lu.a}{0}
\verb{fama}{}{}{}{}{s.f.}{Conceito que se tem de alguém ou de algo; reputação.}{fa.ma}{0}
\verb{fama}{}{}{}{}{}{Renome, celebridade, notoriedade.}{fa.ma}{0}
\verb{famélico}{}{}{}{}{adj.}{Que tem muita fome; faminto, esfaimado.}{fa.mé.li.co}{0}
\verb{famigerado}{}{}{}{}{adj.}{Diz"-se daquele que tem muita fama, geralmente má; célebre, famoso.}{fa.mi.ge.ra.do}{0}
\verb{família}{}{}{}{}{s.f.}{Conjunto social formado de pai, mãe e filhos.}{fa.mí.lia}{0}
\verb{família}{}{}{}{}{}{Conjunto de todos os parentes de uma pessoa.}{fa.mí.lia}{0}
\verb{família}{}{}{}{}{}{Grupo de vegetais, animais e minerais que apresentam características comuns.}{fa.mí.lia}{0}
\verb{família}{}{}{}{}{}{Conjunto de pessoas que têm interesses em comum.}{fa.mí.lia}{0}
\verb{familiar}{}{}{}{}{adj.2g.}{Relativo a família; doméstico, caseiro.}{fa.mi.li.ar}{0}
\verb{familiar}{}{}{}{}{}{Aqulele que é da família ou goza do convívio de alguém; íntimo.}{fa.mi.li.ar}{0}
\verb{familiar}{}{}{}{}{}{Conhecido, trivial, comum.}{fa.mi.li.ar}{0}
\verb{familiar}{}{}{}{}{}{Sem afetação; simples, singelo.}{fa.mi.li.ar}{0}
\verb{familiar}{}{}{}{}{}{Diz"-se de palavra, construção ou expressão usada correntemente em situação de maior informalidade, em casa e na presença de pessoas mais ou menos íntimas.}{fa.mi.li.ar}{0}
\verb{familiar}{}{}{}{}{s.m.}{Pessoa da família, que tem o mesmo sangue.}{fa.mi.li.ar}{0}
\verb{familiaridade}{}{}{}{}{s.f.}{Qualidade do que é familiar; intimidade.}{fa.mi.li.a.ri.da.de}{0}
\verb{familiaridade}{}{}{}{}{}{Franqueza, confiança, sinceridade.}{fa.mi.li.a.ri.da.de}{0}
\verb{familiarizar}{}{}{}{}{v.t.}{Tornar familiar, íntimo.}{fa.mi.li.a.ri.zar}{0}
\verb{familiarizar}{}{}{}{}{v.pron.}{Habituar"-se, acostumar"-se.}{fa.mi.li.a.ri.zar}{\verboinum{1}}
\verb{faminto}{}{}{}{}{adj.}{Que tem muita fome; esfomeado, famélico.}{fa.min.to}{0}
\verb{faminto}{}{Fig.}{}{}{}{Que deseja com ardor; ávido, sôfrego.}{fa.min.to}{0}
\verb{famoso}{ô}{}{"-osos ⟨ó⟩}{"-osa ⟨ó⟩}{adj.}{Que tem fama; célebre, renomado.}{fa.mo.so}{0}
\verb{famoso}{ô}{}{"-osos ⟨ó⟩}{"-osa ⟨ó⟩}{}{Superior, invulgar, excepcional.}{fa.mo.so}{0}
\verb{fâmulo}{}{}{}{}{s.m.}{Indivíduo que presta serviços domésticos; criado, serviçal.}{fâ.mu.lo}{0}
\verb{fâmulo}{}{Fig.}{}{}{}{Indivíduo subserviente, servil, bajulador.}{fâ.mu.lo}{0}
\verb{fanal}{}{}{"-ais}{}{s.m.}{Sinal luminoso; facho de luz; farol.}{fa.nal}{0}
\verb{fanal}{}{Fig.}{"-ais}{}{}{Guia, luz espiritual.}{fa.nal}{0}
\verb{fanar}{}{}{}{}{v.t.}{Fazer amputação; truncar, cortar.}{fa.nar}{0}
\verb{fanar}{}{}{}{}{v.pron.}{Perder o frescor; murchar, secar.}{fa.nar}{\verboinum{1}}
\verb{fanático}{}{}{}{}{adj.}{Que revela entusiasmo excessivo; apreciador apaixonado; exaltado.}{fa.ná.ti.co}{0}
\verb{fanático}{}{}{}{}{}{Diz"-se do indivíduo que se acredita inspirado por uma divindade; iluminado.}{fa.ná.ti.co}{0}
\verb{fanático}{}{}{}{}{}{Que tem zelo excessivo pela religião; intolerante. }{fa.ná.ti.co}{0}
\verb{fanatismo}{}{}{}{}{s.m.}{Dedicação ou devoção excessiva a alguém ou algo; paixão.}{fa.na.tis.mo}{0}
\verb{fanatismo}{}{}{}{}{}{Faccionismo partidário; adesão cega a um sistema ou a uma doutrina.}{fa.na.tis.mo}{0}
\verb{fanatismo}{}{}{}{}{}{Zelo obsessivo por uma religião, que pode levar a extremos de intolerância.}{fa.na.tis.mo}{0}
\verb{fanatizar}{}{}{}{}{v.t.}{Tornar fanático; inspirar fanatismo.}{fa.na.ti.zar}{\verboinum{1}}
\verb{fancaria}{}{}{}{}{s.f.}{Trabalho grosseiro, ordinário, mal"-acabado.}{fan.ca.ri.a}{0}
\verb{fandango}{}{}{}{}{s.m.}{Dança e música populares espanholas, em compasso ternário, acompanhadas por guitarra ibérica e castanholas.}{fan.dan.go}{0}
\verb{fandango}{}{}{}{}{}{Dança folclórica brasileira com predominância de sapateado e acompanhada de músicas em que se alternam estrofes com refrão.}{fan.dan.go}{0}
\verb{fandango}{}{}{}{}{}{Qualquer baile popular ou folia.}{fan.dan.go}{0}
\verb{fanerógamo}{}{Bot.}{}{}{s.m.}{Espécime das fanerógamas, divisão do reino vegetal que reúne as plantas que se reproduzem por sementes, e que compreende as gimnospermas e as angiospermas.}{fa.ne.ró.ga.mo}{0}
\verb{fanerógamo}{}{}{}{}{adj.}{Diz"-se de planta cujos órgãos reprodutores podem ser vistos, como as que possuem flores, ao contrário daquelas que se reproduzem por esporos ou gametas (criptógamo).}{fa.ne.ró.ga.mo}{0}
\verb{fanfarra}{}{Mús.}{}{}{s.f.}{Banda de música composta por instrumentos de sopro (metais) e percussão.}{fan.far.ra}{0}
\verb{fanfarra}{}{}{}{}{}{Música executada por essa banda.}{fan.far.ra}{0}
\verb{fanfarrão}{}{}{"-ões}{}{adj.}{Que ostenta valentia, mas é covarde; valentão, gabola.}{fan.far.rão}{0}
\verb{fanfarrear}{}{}{}{}{v.i.}{Dizer fanfarronices, bravatas; fanfarronar, bazofiar, jactar"-se.}{fan.far.re.ar}{\verboinum{4}}
\verb{fanfarrice}{}{}{}{}{s.f.}{Fanfarronice.}{fan.far.ri.ce}{0}
\verb{fanfarronada}{}{}{}{}{s.f.}{Fanfarronice.}{fan.far.ro.na.da}{0}
\verb{fanfarronar}{}{}{}{}{v.i.}{Fanfarrear.}{fan.far.ro.nar}{\verboinum{1}}
\verb{fanfarronice}{}{}{}{}{s.f.}{Qualidade, ato ou dito de fanfarrão; bazófia, jactância, fanfarrice, fanfarronada.}{fan.far.ro.ni.ce}{0}
\verb{fanho}{}{}{}{}{adj.}{Fanhoso.}{fa.nho}{0}
\verb{fanhoso}{ô}{}{"-osos ⟨ó⟩}{"-osa ⟨ó⟩}{adj.}{Diz"-se daquele que possui voz anasalada, que fala expelindo parte do ar pelo nariz, deixando de pronunciar certos fonemas; fanho.}{fa.nho.so}{0}
\verb{faniquito}{}{}{}{}{s.m.}{Ligeiro ataque nervoso; chilique, fricote.}{fa.ni.qui.to}{0}
\verb{fantasia}{}{}{}{}{s.f.}{Obra da imaginação, que não corresponde à realidade.}{fan.ta.si.a}{0}
\verb{fantasia}{}{}{}{}{}{Devaneio, sonho, ficção.}{fan.ta.si.a}{0}
\verb{fantasia}{}{}{}{}{}{Roupa usada no carnaval ou em festas para disfarce.}{fan.ta.si.a}{0}
\verb{fantasiar}{}{}{}{}{v.t.}{Criar fantasia; imaginar, idealizar, devanear.}{fan.ta.si.ar}{0}
\verb{fantasiar}{}{}{}{}{}{Vestir com fantasia de carnaval; mascarar.}{fan.ta.si.ar}{\verboinum{1}}
\verb{fantasioso}{ô}{}{"-osos ⟨ó⟩}{"-osa ⟨ó⟩}{adj.}{Cheio de fantasia; imaginoso, irreal.}{fan.ta.si.o.so}{0}
\verb{fantasioso}{ô}{}{"-osos ⟨ó⟩}{"-osa ⟨ó⟩}{}{Que revela imaginação; fantástico.}{fan.ta.si.o.so}{0}
\verb{fantasista}{}{}{}{}{adj.2g.}{Que fantasia, que se deixa levar pela imaginação.}{fan.ta.sis.ta}{0}
\verb{fantasma}{}{}{}{}{s.m.}{Suposta aparição de pessoa falecida; assombração, espectro, alma do outro mundo.}{fan.tas.ma}{0}
\verb{fantasma}{}{}{}{}{}{Imagem falsa, ilusória, que infunde terror e pavor.}{fan.tas.ma}{0}
\verb{fantasma}{}{Fig.}{}{}{}{Indivíduo muito magro e pálido.}{fan.tas.ma}{0}
\verb{fantasma}{}{}{}{}{}{Pessoa que não tem mais a aparência que antes apresentava; sombra, simulacro.}{fan.tas.ma}{0}
\verb{fantasmagoria}{}{}{}{}{s.f.}{Arte de fazer surgir imagens luminosas em meio à escuridão por efeitos de ilusão de óptica.}{fan.tas.ma.go.ri.a}{0}
\verb{fantasmagoria}{}{}{}{}{}{Conjunto de imagens ou visões fantásticas, irreais.}{fan.tas.ma.go.ri.a}{0}
\verb{fantasmagoria}{}{Fig.}{}{}{}{Falsa aparência; coisa imaginária; ilusão.}{fan.tas.ma.go.ri.a}{0}
\verb{fantasmagórico}{}{}{}{}{adj.}{Relativo a fantasmagoria ou a fantasma.}{fan.tas.ma.gó.ri.co}{0}
\verb{fantasmagórico}{}{Fig.}{}{}{}{Ilusório, irreal, fantástico.}{fan.tas.ma.gó.ri.co}{0}
\verb{fantástico}{}{}{}{}{adj.}{Que é criado pela fantasia; que só existe na imaginação.}{fan.tás.ti.co}{0}
\verb{fantástico}{}{}{}{}{}{Extraordinário, incrível, excepcional.}{fan.tás.ti.co}{0}
\verb{fantástico}{}{Art.}{}{}{}{Diz"-se de obra literária, artística ou cinematográfica que transcende o real.}{fan.tás.ti.co}{0}
\verb{fantoche}{ó}{}{}{}{s.m.}{Boneco que se faz movimentar por meio de arames ou com as mãos; bonifrate, mamulengo, marionete.}{fan.to.che}{0}
\verb{fantoche}{ó}{Fig.}{}{}{}{Pessoa sem personalidade, que fala ou procede conforme a vontade de outrem.}{fan.to.che}{0}
\verb{faqueiro}{ê}{}{}{}{s.m.}{Jogo completo de talheres do mesmo material e marca.}{fa.quei.ro}{0}
\verb{faqueiro}{ê}{}{}{}{}{Lugar onde se guardam os talheres, especialmente as facas.}{fa.quei.ro}{0}
\verb{faqueiro}{ê}{}{}{}{}{Indivíduo que fabrica ou vende facas.}{fa.quei.ro}{0}
\verb{faquir}{}{}{}{}{s.m.}{Monge muçulmano ou hindu que pratica a mendicância e submete"-se a uma vida de privações.}{fa.quir}{0}
\verb{faquir}{}{}{}{}{}{Artista que faz exibições de jejum prolongado e se submete a suplícios para dar provas de sua insensibilidade à dor.}{fa.quir}{0}
\verb{farândola}{}{}{}{}{s.f.}{Música e dança de origem provençal em que os pares formam uma longa fila que se movimenta de maneira agitada.}{fa.rân.do.la}{0}
\verb{farândola}{}{}{}{}{}{Bando de maltrapilhos.}{fa.rân.do.la}{0}
\verb{faraó}{}{}{}{}{s.m.}{Título dos soberanos do antigo Egito.}{fa.ra.ó}{0}
\verb{faraônico}{}{}{}{}{adj.}{Relativo aos faraós ou ao seu tempo.}{fa.ra.ô.ni.co}{0}
\verb{faraônico}{}{Fig.}{}{}{}{Grandioso, monumental, extravagante.}{fa.ra.ô.ni.co}{0}
\verb{farda}{}{}{}{}{s.f.}{Vestimenta padronizada usada pelos militares, escolares etc.; uniforme, fardamento.}{far.da}{0}
\verb{farda}{}{Por ext.}{}{}{}{A vida militar.}{far.da}{0}
\verb{farda}{}{}{}{}{}{Uniforme usado pelos criados; libré.}{far.da}{0}
\verb{fardamento}{}{}{}{}{s.m.}{Ato ou efeito de fardar.}{far.da.men.to}{0}
\verb{fardamento}{}{}{}{}{}{Fardas de uma corporação militar. }{far.da.men.to}{0}
\verb{fardamento}{}{}{}{}{}{Uniforme completo.}{far.da.men.to}{0}
\verb{fardão}{}{}{"-ões}{}{s.m.}{Uniforme de gala dos militares.}{far.dão}{0}
\verb{fardão}{}{}{"-ões}{}{}{Veste dos membros da Academia Brasileira de Letras.}{far.dão}{0}
\verb{fardar}{}{}{}{}{v.t.}{Vestir"-se com farda.}{far.dar}{\verboinum{1}}
\verb{fardo}{}{}{}{}{s.m.}{Conjunto de objetos mais ou menos volumosos e pesados que se destinam ao transporte; carga. }{far.do}{0}
\verb{fardo}{}{}{}{}{}{Embrulho, pacote, peso.}{far.do}{0}
\verb{fardo}{}{Fig.}{}{}{}{Algo difícil ou duro de suportar; encargo, atribulação.}{far.do}{0}
\verb{farejador}{ô}{}{}{}{adj.}{Que fareja, que segue pelo faro.}{fa.re.ja.dor}{0}
\verb{farejar}{}{}{}{}{v.t.}{Seguir, buscar pelo faro ou olfato.}{fa.re.jar}{0}
\verb{farejar}{}{}{}{}{}{Aspirar o cheiro; cheirar.}{fa.re.jar}{0}
\verb{farejar}{}{}{}{}{}{Andar em busca; tentar descobrir; procurar.}{fa.re.jar}{0}
\verb{farejar}{}{}{}{}{}{Adivinhar, descobrir, prever.}{fa.re.jar}{\verboinum{1}}
\verb{farelo}{é}{}{}{}{s.m.}{Resíduo da farinha de trigo, que permanece depois da peneirada; migalha.}{fa.re.lo}{0}
\verb{farelo}{é}{}{}{}{}{Serragem de madeira.}{fa.re.lo}{0}
\verb{farelo}{é}{Fig.}{}{}{}{Insignificância, ninharia, bagatela.}{fa.re.lo}{0}
\verb{farfalhante}{}{}{}{}{adj.2g.}{Que farfalha, rumoreja; sussurrante.}{far.fa.lhan.te}{0}
\verb{farfalhar}{}{}{}{}{v.i.}{Produzir sons rápidos e indistintos; rumorejar, sussurrar.}{far.fa.lhar}{0}
\verb{farfalhar}{}{Fig.}{}{}{}{Falar muito e rapidamente.}{far.fa.lhar}{\verboinum{1}}
\verb{farináceo}{}{}{}{}{adj.}{Que tem a natureza, o aspecto ou a consistência da farinha.}{fa.ri.ná.ceo}{0}
\verb{farináceo}{}{}{}{}{}{Que contém fécula ou amido.}{fa.ri.ná.ceo}{0}
\verb{farináceo}{}{}{}{}{}{Diz"-se do alimento constituído de algum tipo de farinha.}{fa.ri.ná.ceo}{0}
\verb{faringe}{}{Anat.}{}{}{s.f.}{Conduto muscular e membranoso situado entre a boca e a parte superior do esôfago.}{fa.rin.ge}{0}
\verb{faringite}{}{Med.}{}{}{s.f.}{Inflamação da mucosa da faringe.}{fa.rin.gi.te}{0}
\verb{farinha}{}{}{}{}{s.f.}{Pó obtido pela moagem de um cereal, de um legume seco ou de certas raízes.}{fa.ri.nha}{0}
\verb{farinha}{}{}{}{}{}{Pó obtido pela trituração de grãos de cereais, com que se fazem pães, bolos, massas etc.}{fa.ri.nha}{0}
\verb{farinhada}{}{}{}{}{s.f.}{Fabricação de farinha de mandioca.}{fa.ri.nha.da}{0}
\verb{farinheira}{ê}{}{}{}{s.f.}{Mulher que vende farinha.}{fa.ri.nhei.ra}{0}
\verb{farinheira}{ê}{}{}{}{}{Vasilha em que, às refeições, se serve a farinha de milho ou de mandioca.}{fa.ri.nhei.ra}{0}
\verb{farinhento}{}{}{}{}{adj.}{Que se assemelha à farinha.}{fa.ri.nhen.to}{0}
\verb{farinhento}{}{}{}{}{}{Que contém muita farinha.}{fa.ri.nhen.to}{0}
\verb{farinhento}{}{}{}{}{}{Que se esfarinha com facilidade.}{fa.ri.nhen.to}{0}
\verb{farisaico}{}{}{}{}{adj.}{Relativo a fariseu.}{fa.ri.sai.co}{0}
\verb{farisaico}{}{Fig.}{}{}{}{Hipócrita, falso.}{fa.ri.sai.co}{0}
\verb{farisaísmo}{}{}{}{}{s.m.}{Caráter ou atitude de fariseu.}{fa.ri.sa.ís.mo}{0}
\verb{farisaísmo}{}{Fig.}{}{}{}{Hipocrisia, falsidade.}{fa.ri.sa.ís.mo}{0}
\verb{fariseu}{}{Relig.}{}{}{s.m.}{Membro de uma seita religiosa judaica, surgida no século \textsc{ii} a.C., que vivia na estrita observância das escrituras religiosas e da tradição oral, ostentando grande santidade exterior.}{fa.ri.seu}{0}
\verb{fariseu}{}{Fig.}{}{}{}{Indivíduo hipócrita, orgulhoso, falso.}{fa.ri.seu}{0}
\verb{farmacêutico}{}{}{}{}{adj.}{Relativo a farmácia.}{far.ma.cêu.ti.co}{0}
\verb{farmacêutico}{}{}{}{}{s.m.}{Especialista em farmácia, habilitado em nível superior.}{far.ma.cêu.ti.co}{0}
\verb{farmacêutico}{}{}{}{}{}{Profissional de farmácia; boticário.}{far.ma.cêu.ti.co}{0}
\verb{farmácia}{}{}{}{}{s.f.}{Ramo da farmacologia que trata das propriedades químicas de substâncias e suas respectivas classificações, visando à concepção e à preparação de  medicamentos.}{far.má.cia}{0}
\verb{farmácia}{}{}{}{}{}{Profissão de farmacêutico.}{far.má.cia}{0}
\verb{farmácia}{}{}{}{}{}{Estabelecimento onde são preparados ou conservados e vendidos medicamentos; botica.}{far.má.cia}{0}
\verb{farmácia}{}{}{}{}{}{Provisão de produtos farmacêuticos para primeiros socorros, que se tem em casa, colégios, empresas etc.}{far.má.cia}{0}
\verb{fármaco}{}{}{}{}{s.m.}{Substância química usada como medicamento.}{fár.ma.co}{0}
\verb{farmacologia}{}{}{}{}{s.f.}{Ciência que trata do preparo, do emprego, da posologia e da conservação dos medicamentos.}{far.ma.co.lo.gi.a}{0}
\verb{farmacológico}{}{}{}{}{adj.}{Relativo a farmacologia.}{far.ma.co.ló.gi.co}{0}
\verb{farmacologista}{}{}{}{}{s.2g.}{Indivíduo especialista em farmacologia.}{far.ma.co.lo.gis.ta}{0}
\verb{farmacopeia}{é}{}{}{}{s.f.}{Compilação, coleção ou catálogo de receitas e fórmulas de drogas e medicamentos.}{far.ma.co.pei.a}{0}
\verb{farnel}{é}{}{"-éis}{}{s.m.}{Conjunto de provisões para uma pequena viagem; merenda, lanche.}{far.nel}{0}
\verb{farnel}{é}{}{"-éis}{}{}{Saco ou bolsa onde se colocam essas provisões.}{far.nel}{0}
\verb{faro}{}{}{}{}{s.m.}{O olfato do cão e de outros animais.}{fa.ro}{0}
\verb{faro}{}{Fig.}{}{}{}{Instinto, perspicácia, intuição.}{fa.ro}{0}
\verb{faro}{}{Fig.}{}{}{}{Indício, sinal, vislumbre.}{fa.ro}{0}
\verb{faroeste}{é}{}{}{}{s.m.}{Filme ou livro que retrata cenas da conquista do Oeste norte"-americano, especialmente na segunda metade do século \textsc{xix}, e que geralmente envolve lutas e tiroteios.}{fa.ro.es.te}{0}
\verb{faroeste}{é}{Bras.}{}{}{}{Região assolada por alto índice de assaltos, tiroteios, crimes.}{fa.ro.es.te}{0}
\verb{farofa}{ó}{}{}{}{s.f.}{Iguaria feita de farinha de mandioca ou milho, torrada ou frita na manteiga ou na gordura, e às vezes acrescida com ovos, carne, torresmo etc.}{fa.ro.fa}{0}
\verb{farofa}{ó}{Fig.}{}{}{}{Conversa sem importância; papo"-furado; lero"-lero.}{fa.ro.fa}{0}
\verb{farofa}{ó}{Fig.}{}{}{}{Bravata, bazófia, pretensão.}{fa.ro.fa}{0}
\verb{farofeiro}{ê}{}{}{}{adj.}{Que conta bravatas; fanfarrão, parlapatão.}{fa.ro.fei.ro}{0}
\verb{farofeiro}{ê}{Pop.}{}{}{}{Diz"-se do indivíduo que mora longe das praias e, ao frequentá"-las, leva comidas e bebidas para passar o dia.}{fa.ro.fei.ro}{0}
\verb{farol}{ó}{}{"-óis}{}{s.m.}{Construção erguida junto ao mar ou numa ilha, em forma de torre com um foco de luz, para guiar os navegantes durante a noite.}{fa.rol}{0}
\verb{farol}{ó}{}{"-óis}{}{}{Aparelho instalado nas ruas ou nos cruzamentos para dar, manual ou automaticamente, sinais luminosos reguladores do trânsito; semáforo, sinal, sinaleiro, sinaleira.}{fa.rol}{0}
\verb{farol}{ó}{}{"-óis}{}{}{Projetor de luz colocado na parte da frente de um veículo.}{fa.rol}{0}
\verb{farol}{ó}{Fig.}{"-óis}{}{}{Fanfarronice, bazófia, ostentação.}{fa.rol}{0}
\verb{farolagem}{}{}{"-ens}{}{s.f.}{Conjunto de faróis instalados em certo trecho do litoral para auxiliar a navegação.}{fa.ro.la.gem}{0}
\verb{farolagem}{}{}{"-ens}{}{}{Fanfarronice, bazófia, ostentação.}{fa.ro.la.gem}{0}
\verb{faroleiro}{ê}{}{}{}{s.m.}{Indivíduo encarregado da manutenção e do funcionamento de um farol marítimo.}{fa.ro.lei.ro}{0}
\verb{faroleiro}{ê}{Pop.}{}{}{}{Indivíduo fanfarrão, farofeiro, parlapatão.}{fa.ro.lei.ro}{0}
\verb{farolete}{ê}{}{}{}{s.m.}{Cada um dos pequenos faróis ou lanternas, dianteiros e traseiros, que num veículo indicam a sua presença à noite e em situações de pouca visibilidade.}{fa.ro.le.te}{0}
\verb{farpa}{}{}{}{}{s.f.}{Ponta metálica aguda e penetrante que se adapta a setas, arame etc. }{far.pa}{0}
\verb{farpa}{}{}{}{}{}{Pequena lasca de madeira que acidentalmente se introduz na pele ou na carne de homem ou animal.}{far.pa}{0}
\verb{farpa}{}{Fig.}{}{}{}{Crítica mordaz; sarcasmo.}{far.pa}{0}
\verb{farpado}{}{}{}{}{adj.}{Provido de farpas, dentes ou pontas.}{far.pa.do}{0}
\verb{farpado}{}{}{}{}{}{Em forma de farpa.}{far.pa.do}{0}
\verb{farpar}{}{}{}{}{v.t.}{Introduzir, colocar farpas.}{far.par}{0}
\verb{farpar}{}{}{}{}{}{Munir ou armar de farpas ou pontas.}{far.par}{0}
\verb{farpar}{}{}{}{}{}{Rasgar, estraçalhar, romper.}{far.par}{\verboinum{1}}
\verb{farpear}{}{}{}{}{v.t.}{Ferir, picar com farpas.}{far.pe.ar}{0}
\verb{farpear}{}{}{}{}{}{Dirigir farpas; criticar com sarcasmo.}{far.pe.ar}{\verboinum{4}}
\verb{farpela}{é}{}{}{}{s.f.}{Gancho agudo na extremidade de um dos lados da agulha de crochê.}{far.pe.la}{0}
\verb{farpela}{é}{}{}{}{s.f.}{Traje pobre e malfeito.}{far.pe.la}{0}
\verb{farra}{}{}{}{}{s.f.}{Festa licenciosa; folia, orgia, diversão.}{far.ra}{0}
\verb{farra}{}{}{}{}{}{Brincadeira, troça, caçoada.}{far.ra}{0}
\verb{farrancho}{}{}{}{}{s.m.}{Grupo de pessoas que vão para romaria ou diversão.}{far.ran.cho}{0}
\verb{farrapo}{}{}{}{}{s.m.}{Pedaço de pano muito usado ou rasgado; trapo cortado; andrajo. }{far.ra.po}{0}
\verb{farrapo}{}{}{}{}{}{Indivíduo maltrapilho.}{far.ra.po}{0}
\verb{farrapo}{}{}{}{}{}{Farroupilha.}{far.ra.po}{0}
\verb{farrear}{}{}{}{}{v.i.}{Fazer farra, pândega; foliar.}{far.re.ar}{\verboinum{4}}
\verb{farripas}{}{}{}{}{s.f.pl.}{Cabelos curtos e ralos; falripas.}{far.ri.pas}{0}
\verb{farrista}{}{}{}{}{adj.2g.}{Que é dado a farras; pândego.}{far.ris.ta}{0}
\verb{farroupilha}{}{}{}{}{s.2g.}{Indivíduo maltrapilho, miserável, desprezível.}{far.rou.pi.lha}{0}
\verb{farroupilha}{}{}{}{}{}{Revolucionário da Guerra dos Farrapos ou Revolução Farroupilha que irrompeu em 1835 no Rio Grande do Sul.}{far.rou.pi.lha}{0}
\verb{farrusco}{}{}{}{}{adj.}{Sujo de fuligem ou de carvão.}{far.rus.co}{0}
\verb{farrusco}{}{}{}{}{}{Negro, escuro.}{far.rus.co}{0}
\verb{farsa}{}{}{}{}{s.f.}{Gênero teatral exageradamente cômico e burlesco.}{far.sa}{0}
\verb{farsa}{}{}{}{}{}{Ato ridículo, grotesco.}{far.sa}{0}
\verb{farsa}{}{}{}{}{}{Fingimento, embuste, impostura.}{far.sa}{0}
\verb{farsa}{}{}{}{}{}{Ilusão, mentira, burla.}{far.sa}{0}
\verb{farsante}{}{}{}{}{s.2g.}{Artista que faz rir com suas representações burlescas.}{far.san.te}{0}
\verb{farsante}{}{}{}{}{adj.2g.}{Que não procede com seriedade, que não se pode levar a sério.}{far.san.te}{0}
\verb{farsante}{}{}{}{}{}{Fingido, embusteiro, farsista.}{far.san.te}{0}
\verb{farsista}{}{}{}{}{adj.2g.}{Que graceja muito, geralmente de modo chulo.}{far.sis.ta}{0}
\verb{farsista}{}{}{}{}{}{Embusteiro, fingido, farsante.}{far.sis.ta}{0}
\verb{fartação}{}{}{"-ões}{}{s.f.}{Enfarte, inchação, enfartação.}{far.ta.ção}{0}
\verb{fartar}{}{}{}{}{v.t.}{Tornar cheio; abarrotar, atulhar.}{far.tar}{0}
\verb{fartar}{}{}{}{}{}{Saciar a fome, a sede.}{far.tar}{0}
\verb{fartar}{}{}{}{}{}{Satisfazer os desejos, instintos.}{far.tar}{0}
\verb{fartar}{}{}{}{}{}{Causar aborrecimento; cansar, enfastiar.}{far.tar}{0}
\verb{fartar}{}{}{}{}{v.i.}{Ser bastante, suficiente.}{far.tar}{\verboinum{1}}
\verb{farto}{}{}{}{}{adj.}{Que se saciou; satisfeito, empanturrado.}{far.to}{0}
\verb{farto}{}{}{}{}{}{Em que há fartura; abundante, copioso.}{far.to}{0}
\verb{farto}{}{}{}{}{}{Aborrecido, enfastiado.}{far.to}{0}
\verb{fartum}{}{}{"-uns}{}{s.m.}{Cheiro desagradável emitido por substâncias rançosas ou por alguns animais.}{far.tum}{0}
\verb{fartura}{}{}{}{}{s.f.}{Estado de farto, cheio.}{far.tu.ra}{0}
\verb{fartura}{}{}{}{}{}{Grande quantidade de víveres; abundância.}{far.tu.ra}{0}
\verb{fascículo}{}{}{}{}{s.m.}{Cada um dos cadernos ou dos folhetos de uma obra que se publica por partes.}{fas.cí.cu.lo}{0}
\verb{fascículo}{}{}{}{}{}{Pequeno feixe de espigas; gavela.}{fas.cí.cu.lo}{0}
\verb{fascículo}{}{Anat.}{}{}{}{Pequeno feixe de fibras nervosas, tendinosas ou musculares.}{fas.cí.cu.lo}{0}
\verb{fascículo}{}{Bot.}{}{}{}{Inflorescência cujas flores se inserem, apertadas no mesmo caule.}{fas.cí.cu.lo}{0}
\verb{fascinação}{}{}{"-ões}{}{s.f.}{Ato ou efeito de fascinar; atração irresistível; encantamento, feitiço.}{fas.ci.na.ção}{0}
\verb{fascinação}{}{}{"-ões}{}{}{Deslumbramento, ilusão.}{fas.ci.na.ção}{0}
\verb{fascinador}{ô}{}{}{}{adj.}{Que fascina, enfeitiça; fascinante.}{fas.ci.na.dor}{0}
\verb{fascinante}{}{}{}{}{adj.2g.}{Que fascina; cativante, encantador, sedutor.}{fas.ci.nan.te}{0}
\verb{fascinar}{}{}{}{}{v.t.}{Atrair, seduzir, dominar de forma irresistível.}{fas.ci.nar}{0}
\verb{fascinar}{}{}{}{}{}{Dominar sob feitiço; encantar, enfeitiçar.}{fas.ci.nar}{0}
\verb{fascinar}{}{}{}{}{}{Deslumbrar, iludir.}{fas.ci.nar}{\verboinum{1}}
\verb{fascínio}{}{}{}{}{s.m.}{Qualidade ou poder de fascinar; encanto, sedução, fascinação.}{fas.cí.nio}{0}
\verb{fascínio}{}{}{}{}{}{Encantamento, feitiço, quebranto.}{fas.cí.nio}{0}
\verb{fascismo}{}{Hist.}{}{}{s.m.}{Regime autoritário estabelecido por Benito Mussolini na Itália, em 1922, fundado sob os conceitos de nação e raça acima dos valores individuais e representado por um governo ditatorial.}{fas.cis.mo}{0}
\verb{fascismo}{}{}{}{}{}{Doutrina e prática que visa estabelecer um regime hierarquizado e totalitário.}{fas.cis.mo}{0}
\verb{fascista}{}{}{}{}{adj.2g.}{Relativo a fascismo.}{fas.cis.ta}{0}
\verb{fascista}{}{}{}{}{s.2g.}{Partidário ou simpatizante do fascismo.}{fas.cis.ta}{0}
\verb{fase}{}{}{}{}{s.f.}{Cada uma das etapas de um fenômeno em evolução; estágio.}{fa.se}{0}
\verb{fase}{}{}{}{}{}{Cada uma das mudanças de aspecto da Lua. (\textit{A Lua está entrando em sua fase cheia.})}{fa.se}{0}
\verb{faseolar}{}{}{}{}{adj.2g.}{Que tem forma semelhante ao feijão.}{fa.se.o.lar}{0}
\verb{fasianídeo}{}{Zool.}{}{}{s.m.}{Espécime dos fasianídeos, família de aves de bico geralmente curto e forte, e pernas adaptadas para a corrida e para ciscar o chão em busca de alimento e que inclui os galos e os faisões.}{fa.si.a.ní.deo}{0}
\verb{fasianídeo}{}{Zool.}{}{}{adj.}{Relativo aos fasianídeos.}{fa.si.a.ní.deo}{0}
\verb{fasquia}{}{}{}{}{s.f.}{Lasca fina e alongada tirada de um tronco.}{fas.qui.a}{0}
\verb{fasquia}{}{}{}{}{}{Ripa de madeira serrada, comprida e estreita.}{fas.qui.a}{0}
\verb{fastidioso}{ô}{}{"-osos ⟨ó⟩}{"-osa ⟨ó⟩}{adj.}{Que causa fastio; enfadonho, maçante, tedioso.}{fas.ti.di.o.so}{0}
\verb{fastígio}{}{}{}{}{s.m.}{Ponto mais elevado; cume, pico.}{fas.tí.gio}{0}
\verb{fastígio}{}{Fig.}{}{}{}{Plenitude, esplendor, auge.}{fas.tí.gio}{0}
\verb{fastigioso}{ô}{}{"-osos ⟨ó⟩}{"-osa ⟨ó⟩}{adj.}{Que está no auge; eminente.}{fas.ti.gi.o.so}{0}
\verb{fastio}{}{}{}{}{s.m.}{Ausência de fome, de apetite.}{fas.ti.o}{0}
\verb{fastio}{}{}{}{}{}{Sentimento de tédio; aborrecimento, enfado.}{fas.ti.o}{0}
\verb{fastio}{}{}{}{}{}{Repugnância, aversão, enjoo.}{fas.ti.o}{0}
\verb{fastos}{}{Hist.}{}{}{s.m.pl.}{Na Roma antiga, tábuas cronológicas que indicavam os dias de pompa e luxo e os dias de luto ou de tragédias.}{fas.tos}{0}
\verb{fastuoso}{ô}{}{"-osos ⟨ó⟩}{"-osa ⟨ó⟩}{adj.}{Que contém fausto; luxuoso, pomposo, magnificente.}{fas.tu.o.so}{0}
\verb{fatacaz}{}{}{}{}{s.m.}{Fatia grossa de pão, bolo; naco.}{fa.ta.caz}{0}
\verb{fatal}{}{}{"-ais}{}{adj.2g.}{Determinado pelo destino; inevitável, inexorável.}{fa.tal}{0}
\verb{fatal}{}{}{"-ais}{}{}{Mortal, derradeiro, final.}{fa.tal}{0}
\verb{fatal}{}{}{"-ais}{}{}{Desastroso, funesto, nocivo.}{fa.tal}{0}
\verb{fatalidade}{}{}{}{}{s.f.}{Destino inevitável; fado.}{fa.ta.li.da.de}{0}
\verb{fatalidade}{}{}{}{}{}{Acontecimento funesto; acaso infeliz.}{fa.ta.li.da.de}{0}
\verb{fatalidade}{}{}{}{}{}{Desgraça, infortúnio, adversidade.}{fa.ta.li.da.de}{0}
\verb{fatalismo}{}{}{}{}{s.m.}{Atitude de quem se abandona passivamente aos acontecimentos; pessimismo.}{fa.ta.lis.mo}{0}
\verb{fatalismo}{}{}{}{}{}{Doutrina dos que negam o livre"-arbítrio e atribuem tudo à fatalidade.}{fa.ta.lis.mo}{0}
\verb{fatalista}{}{}{}{}{adj.2g.}{Relativo a fatalismo.}{fa.ta.lis.ta}{0}
\verb{fatalista}{}{}{}{}{}{Que acredita no fatalismo.}{fa.ta.lis.ta}{0}
\verb{fatalista}{}{}{}{}{}{Que se abandona sem reação aos fatos e aos acontecimentos; pessimista.}{fa.ta.lis.ta}{0}
\verb{fatia}{}{}{}{}{s.f.}{Pedaço de alimento sólido cortado fino. }{fa.ti.a}{0}
\verb{fatia}{}{}{}{}{}{Porção, parte, parcela.}{fa.ti.a}{0}
\verb{fatiar}{}{}{}{}{v.t.}{Cortar em fatias; fazer em pedaços.}{fa.ti.ar}{\verboinum{1}}
\verb{fatídico}{}{}{}{}{adj.}{Que revela as decisões ou as imposições do destino; previsto, profetizado.}{fa.tí.di.co}{0}
\verb{fatídico}{}{}{}{}{}{Sinistro, funesto, trágico.}{fa.tí.di.co}{0}
\verb{fatigante}{}{}{}{}{adj.2g.}{Que fatiga; cansativo, entediante.}{fa.ti.gan.te}{0}
\verb{fatigar}{}{}{}{}{v.t.}{Provocar fadiga; cansar, fadigar.}{fa.ti.gar}{0}
\verb{fatigar}{}{}{}{}{}{Aborrecer, enfastiar, enfadar.}{fa.ti.gar}{\verboinum{5}}
\verb{fatiota}{ó}{}{}{}{s.f.}{Vestimenta em geral; traje, roupa, fato.}{fa.ti.o.ta}{0}
\verb{fato}{}{}{}{}{s.m.}{Aquilo que aconteceu; evento, acontecimento. (\textit{Ela lamentou muito o fato ocorrido em sua ausência. })}{fa.to}{0}
\verb{fato}{}{}{}{}{}{Aquilo que é real; realidade.}{fa.to}{0}
\verb{fato}{}{}{}{}{}{Roupa, vestuário.}{fa.to}{0}
\verb{fator}{ô}{}{}{}{s.m.}{Aquilo que determina ou executa alguma coisa.}{fa.tor}{0}
\verb{fator}{ô}{}{}{}{}{Qualquer elemento que concorre para um resultado.}{fa.tor}{0}
\verb{fator}{ô}{Mat.}{}{}{}{Cada um dos elementos que participam da multiplicação.}{fa.tor}{0}
\verb{fatoração}{}{Mat.}{"-ões}{}{s.f.}{Operação que consiste em decompor um número ou um polinômio através de divisões sucessivas por outros até a unidade.}{fa.to.ra.ção}{0}
\verb{fatorar}{}{Mat.}{}{}{v.t.}{Decompor um número ou um polinômio em todos os seus fatores.}{fa.to.rar}{\verboinum{1}}
\verb{fatorial}{}{Mat.}{"-ais}{}{s.m.}{Produto dos números inteiros de uma progressão aritmética.}{fa.to.ri.al}{0}
\verb{fatual}{}{}{"-ais}{}{adj.2g.}{Relativo a um ou mais fatos; factual.}{fa.tu.al}{0}
\verb{fatual}{}{}{"-ais}{}{}{Que se atém aos fatos; real, verdadeiro.}{fa.tu.al}{0}
\verb{fatuidade}{}{}{}{}{s.f.}{Qualidade de quem é fátuo; presunção, vaidade. }{fa.tu.i.da.de}{0}
\verb{fatuidade}{}{}{}{}{}{Qualidade do que é passageiro; transitoriedade, fugacidade. }{fa.tu.i.da.de}{0}
\verb{fátuo}{}{}{}{}{adj.}{Que é vaidoso, presunçoso, pretensioso.}{fá.tu.o}{0}
\verb{fátuo}{}{}{}{}{}{Que não dura muito; passageiro, fugaz, transitório.}{fá.tu.o}{0}
\verb{fatura}{}{}{}{}{s.f.}{Ato ou efeito de fazer; obra feita; feitura.}{fa.tu.ra}{0}
\verb{fatura}{}{}{}{}{}{Nota que discrimina as mercadorias vendidas e os respectivos preços.}{fa.tu.ra}{0}
\verb{faturamento}{}{}{}{}{s.m.}{Ato ou efeito de faturar.}{fa.tu.ra.men.to}{0}
\verb{faturamento}{}{}{}{}{}{Valor total das vendas de uma empresa em um dado período. }{fa.tu.ra.men.to}{0}
\verb{faturar}{}{}{}{}{v.t.}{Fazer a fatura das mercadorias vendidas.}{fa.tu.rar}{0}
\verb{faturar}{}{}{}{}{}{Incluir uma mercadoria na fatura.}{fa.tu.rar}{0}
\verb{faturar}{}{}{}{}{}{Ganhar muito dinheiro; lucrar.}{fa.tu.rar}{0}
\verb{faturar}{}{Pop.}{}{}{}{Tirar proveito; obter vantagem.}{fa.tu.rar}{\verboinum{1}}
\verb{faturista}{}{}{}{}{s.2g.}{Funcionário encarregado de fazer faturas em um estabelecimento comercial.}{fa.tu.ris.ta}{0}
\verb{fauce}{}{Zool.}{}{}{s.f.}{Parte superior e interior da garganta do leão, do tigre e de outros animais; goela.}{fau.ce}{0}
\verb{faúlha}{}{}{}{}{s.f.}{A parte mais fina que se levanta da farinha quando peneirada.}{fa.ú.lha}{0}
\verb{faúlha}{}{}{}{}{}{Fagulha.}{fa.ú.lha}{0}
\verb{fauna}{}{Zool.}{}{}{s.f.}{Conjunto das espécies animais próprias de determinada área, época geológica ou meio ambiente específico.}{fau.na}{0}
\verb{fauno}{}{Mit.}{}{}{s.m.}{Na mitologia romana, divindade campestre com pés de cabra, pelos abundantes e chifres, que vivia nos bosques e protegia os rebanhos.}{fau.no}{0}
\verb{fausto}{}{}{}{}{adj.}{Que é feliz, próspero, ditoso.}{faus.to}{0}
\verb{fausto}{}{}{}{}{s.m.}{Ostentação de grandeza; luxo, pompa.}{faus.to}{0}
\verb{faustoso}{ô}{}{"-osos ⟨ó⟩}{"-osa ⟨ó⟩}{adj.}{Que ostenta fausto; pomposo, luxuoso, aparatoso.}{faus.to.so}{0}
\verb{fautor}{ô}{}{}{fautriz}{s.m.}{Que favorece, promove, estimula algo.}{fau.tor}{0}
\verb{fava}{}{Bot.}{}{}{s.f.}{Planta leguminosa, hortense, de grandes sementes e vagens comestíveis.}{fa.va}{0}
\verb{fava}{}{}{}{}{}{A vagem ou as sementes dessa planta.}{fa.va}{0}
\verb{favela}{é}{}{}{}{s.f.}{Aglomeração de casas pobres, mal construídas e em geral sem condições de higiene.}{fa.ve.la}{0}
\verb{favela}{é}{Bot.}{}{}{}{Arbusto grande, de flores alvas.}{fa.ve.la}{0}
\verb{favelado}{}{}{}{}{adj.}{Que habita em favela.}{fa.ve.la.do}{0}
\verb{favo}{}{}{}{}{s.m.}{Conjunto de alvéolos em que as abelhas depositam o mel.}{fa.vo}{0}
\verb{favo}{}{Fig.}{}{}{}{Coisa agradável, doce.}{fa.vo}{0}
\verb{favônio}{}{}{}{}{s.m.}{Vento brando, suave, que sopra do poente.}{fa.vô.nio}{0}
\verb{favor}{ô}{}{}{}{s.m.}{Serviço gratuito prestado ou recebido; graça, obséquio.}{fa.vor}{0}
\verb{favor}{ô}{}{}{}{}{Benefício, interesse, bem.}{fa.vor}{0}
\verb{favor}{ô}{}{}{}{}{Defesa, proteção.}{fa.vor}{0}
\verb{favor}{ô}{}{}{}{}{Parcialidade, indulgência.}{fa.vor}{0}
\verb{favor}{ô}{}{}{}{}{Poder, prestígio, influência.}{fa.vor}{0}
\verb{favorável}{}{}{"-eis}{}{adj.2g.}{Que favorece ou auxilia; conveniente, propício.}{fa.vo.rá.vel}{0}
\verb{favorável}{}{}{"-eis}{}{}{Indulgente, benévolo.}{fa.vo.rá.vel}{0}
\verb{favorável}{}{}{"-eis}{}{}{Propenso, inclinado.}{fa.vo.rá.vel}{0}
\verb{favorecer}{ê}{}{}{}{v.t.}{Dar auxílio; proteger, apoiar.}{fa.vo.re.cer}{0}
\verb{favorecer}{ê}{}{}{}{}{Dotar de boas qualidades.}{fa.vo.re.cer}{0}
\verb{favorecer}{ê}{}{}{}{}{Oferecer condições propícias; beneficiar.}{fa.vo.re.cer}{0}
\verb{favorecer}{ê}{}{}{}{}{Proteger com parcialidade.}{fa.vo.re.cer}{\verboinum{15}}
\verb{favorita}{}{}{}{}{s.f.}{A mais querida.}{fa.vo.ri.ta}{0}
\verb{favorita}{}{}{}{}{}{Mulher predileta do senhor de um harém.}{fa.vo.ri.ta}{0}
\verb{favoritismo}{}{}{}{}{s.m.}{Preferência, proteção que se dá ao favorito.}{fa.vo.ri.tis.mo}{0}
\verb{favoritismo}{}{}{}{}{}{Sistema em que se acolhem ou amparam favores injustos, ilegais.}{fa.vo.ri.tis.mo}{0}
\verb{favorito}{}{}{}{}{adj.}{O mais querido; predileto.}{fa.vo.ri.to}{0}
\verb{favorito}{}{}{}{}{s.m.}{Concorrente que apresenta maiores possibilidades de vitória em uma competição.}{fa.vo.ri.to}{0}
\verb{fax}{cs}{}{}{}{s.m.}{Sistema de transmissão de impressos e desenhos a distância.}{fax}{0}
\verb{fax}{cs}{}{}{}{}{O documento enviado por esse sistema.}{fax}{0}
\verb{faxina}{ch}{}{}{}{s.f.}{Serviço de limpeza.}{fa.xi.na}{0}
\verb{faxinar}{ch}{}{}{}{v.t.}{Fazer serviço de limpeza; limpar.}{fa.xi.nar}{\verboinum{1}}
\verb{faxineiro}{ch}{}{}{}{s.m.}{Funcionário encarregado dos serviços de limpeza de um lugar.}{fa.xi.nei.ro}{0}
\verb{faz"-de"-conta}{}{}{}{}{s.m.}{O mundo da fantasia, do imaginário.}{faz"-de"-con.ta}{0}
\verb{fazedor}{ô}{}{}{}{adj.}{Que faz ou executa algo.}{fa.ze.dor}{0}
\verb{fazenda}{}{}{}{}{s.f.}{Grande propriedade rural destinada à lavoura ou à criação de gado.}{fa.zen.da}{0}
\verb{fazenda}{}{}{}{}{}{Pano, tecido.}{fa.zen.da}{0}
\verb{fazenda}{}{}{}{}{}{Bens, haveres.}{fa.zen.da}{0}
\verb{fazenda}{}{}{}{}{}{O tesouro público; as finanças do Estado.}{fa.zen.da}{0}
\verb{fazendário}{}{}{}{}{adj.}{Relativo à fazenda pública; financeiro.}{fa.zen.dá.rio}{0}
\verb{fazendeiro}{ê}{}{}{}{s.m.}{Dono de grande propriedade rural.}{fa.zen.dei.ro}{0}
\verb{fazer}{ê}{}{}{}{v.t.}{Produzir através de alguma ação; realizar, praticar.}{fa.zer}{0}
\verb{fazer}{ê}{}{}{}{}{Fabricar, manufaturar, confeccionar. (\textit{Minha mãe fazia todas as nossas roupas quando éramos pequenos.})}{fa.zer}{0}
\verb{fazer}{ê}{}{}{}{}{Executar a construção; edificar, erguer.}{fa.zer}{0}
\verb{fazer}{ê}{}{}{}{}{Estar, existir, haver. (\textit{A previsão do tempo indica que fará sol amanhã.})}{fa.zer}{0}
\verb{fazer}{ê}{}{}{}{}{Ter decorrido um período de tempo; haver, completar"-se. (\textit{Já faz dez anos que eu moro nessa casa.})}{fa.zer}{\verboinum{42}}
\verb{fazimento}{}{}{}{}{s.m.}{Ato ou efeito de fazer; feitura.}{fa.zi.men.to}{0}
\verb{faz"-tudo}{}{}{}{}{s.2g.}{Indivíduo que exerce várias funções ou atividades.}{faz"-tu.do}{0}
\verb{Fe}{}{Quím.}{}{}{}{Símb. do \textit{ferro}.}{Fe}{0}
\verb{fé}{}{}{}{}{s.f.}{Adesão absoluta do espírito àquilo que considera verdadeiro.}{fé}{0}
\verb{fé}{}{}{}{}{}{Fidelidade em honrar compromissos; garantia, lealdade.}{fé}{0}
\verb{fé}{}{}{}{}{}{Confiança absoluta em algo ou alguém; crédito.}{fé}{0}
\verb{fé}{}{Relig.}{}{}{}{Crença nos dogmas de uma religião.}{fé}{0}
\verb{fé}{}{Jur.}{}{}{}{Testemunho autêntico dado por escrito pelo tabelião sobre determinados atos e que tem força em juízo.}{fé}{0}
\verb{fealdade}{}{}{}{}{s.f.}{Qualidade de feio; feiura.}{fe.al.da.de}{0}
\verb{febre}{é}{}{}{}{s.f.}{Elevação da temperatura corporal acima de 37° C.}{fe.bre}{0}
\verb{febre}{é}{Fig.}{}{}{}{Paixão viva e desenfreada; agitação.}{fe.bre}{0}
\verb{febre}{é}{Fig.}{}{}{}{Ânsia de possuir; mania.}{fe.bre}{0}
\verb{febricitante}{}{}{}{}{adj.2g.}{Que está com febre; febril.}{fe.bri.ci.tan.te}{0}
\verb{febrícula}{}{}{}{}{s.f.}{Febre ligeira e branda.}{fe.brí.cu.la}{0}
\verb{febrífugo}{}{}{}{}{adj.}{Diz"-se do medicamento que combate a febre, diminuindo a temperatura do corpo; antipirético.}{fe.brí.fu.go}{0}
\verb{febril}{}{}{"-is}{}{adj.2g.}{Relativo a febre; pirético.}{fe.bril}{0}
\verb{febril}{}{}{"-is}{}{}{Cheio de paixão; inflamado, exaltado.}{fe.bril}{0}
\verb{fecal}{}{}{"-ais}{}{adj.2g.}{Relativo a fezes, excrementos.}{fe.cal}{0}
\verb{fechadura}{}{}{}{}{s.f.}{Peça metálica que, por meio de uma lingueta acionada por chave, serve para fechamento de portas, gavetas, caixas etc.}{fe.cha.du.ra}{0}
\verb{fecha"-fecha}{é\ldots{}é}{}{fecha"-fechas ⟨é\ldots{}é⟩}{}{s.m.}{Pânico por motivo de desordem(s) pública(s), que provoca o fechamento de estabelecimentos comerciais, bancos etc.}{fe.cha"-fe.cha}{0}
\verb{fechamento}{}{}{}{}{s.m.}{Ato ou efeito de fechar; encerramento.}{fe.cha.men.to}{0}
\verb{fechamento}{}{}{}{}{}{Finalização de um negócio, da redação de um jornal etc.}{fe.cha.men.to}{0}
\verb{fechar}{}{}{}{}{v.t.}{Impedir a abertura através de tranca, chave etc.}{fe.char}{0}
\verb{fechar}{}{}{}{}{}{Deixar algo sem entrada nem saída; trancar.}{fe.char}{0}
\verb{fechar}{}{}{}{}{}{Interromper a passagem; obstruir, bloquear.}{fe.char}{0}
\verb{fechar}{}{}{}{}{}{Concluir, terminar, acabar.}{fe.char}{0}
\verb{fechar}{}{}{}{}{}{Fazer o cerco; assediar.}{fe.char}{\verboinum{1}}
\verb{fecho}{ê}{}{}{}{s.m.}{Qualquer peça com que se fecha um objeto; tranca, ferrolho, aldrava.}{fe.cho}{0}
\verb{fecho}{ê}{}{}{}{}{Acabamento, fim, remate.}{fe.cho}{0}
\verb{fecho"-ecler}{ê\ldots{}é}{}{fechos"-ecler ⟨ê\ldots{}é⟩}{}{s.m.}{Fecho articulado de largo uso em bolsas, roupas etc; zíper.}{fecho"-ecler}{0}
\verb{fécula}{}{}{}{}{s.f.}{Substância farinácea muito fina, produzida por certos vegetais, que a armazenam em tubérculos ou raízes.}{fé.cu.la}{0}
\verb{feculento}{}{}{}{}{adj.}{Que contém fécula.}{fe.cu.len.to}{0}
\verb{fecundação}{}{}{"-ões}{}{s.f.}{Ato ou efeito de fecundar; fertilização.}{fe.cun.da.ção}{0}
\verb{fecundação}{}{Biol.}{"-ões}{}{}{União do gameta masculino com o gameta feminino para formar um ovo ou zigoto.}{fe.cun.da.ção}{0}
\verb{fecundante}{}{}{}{}{adj.2g.}{Que fecunda, fertiliza.}{fe.cun.dan.te}{0}
\verb{fecundar}{}{}{}{}{v.t.}{Dar origem; procriar, multiplicar.}{fe.cun.dar}{0}
\verb{fecundar}{}{}{}{}{}{Tornar fecundo; fertilizar, produzir, gerar.}{fe.cun.dar}{0}
\verb{fecundar}{}{}{}{}{}{Fazer desenvolver; fomentar, fortalecer.}{fe.cun.dar}{\verboinum{1}}
\verb{fecundidade}{}{}{}{}{s.f.}{Qualidade ou condição de fecundo; fertilidade.}{fe.cun.di.da.de}{0}
\verb{fecundidade}{}{Biol.}{}{}{}{Capacidade de reproduzir"-se muitas vezes.}{fe.cun.di.da.de}{0}
\verb{fecundidade}{}{Fig.}{}{}{}{Facilidade de imaginação ou criatividade.}{fe.cun.di.da.de}{0}
\verb{fecundo}{}{}{}{}{adj.}{Que pode reproduzir"-se; fértil.}{fe.cun.do}{0}
\verb{fecundo}{}{}{}{}{}{Rico, copioso, abundante.}{fe.cun.do}{0}
\verb{fecundo}{}{Fig.}{}{}{}{Inventivo, criativo, imaginativo.}{fe.cun.do}{0}
\verb{fedegoso}{ô}{}{"-osos ⟨ó⟩}{"-osa ⟨ó⟩}{adj.}{Que exala mau cheiro; fétido, fedorento.}{fe.de.go.so}{0}
\verb{fedegoso}{ô}{Bot.}{"-osos ⟨ó⟩}{"-osa ⟨ó⟩}{s.m.}{Nome comum a várias plantas leguminosas, algumas forrageiras com propriedades medicinais, outras ornamentais.}{fe.de.go.so}{0}
\verb{fedelho}{ê}{}{}{}{s.m.}{Criança muito nova, que ainda cheira a cueiros.}{fe.de.lho}{0}
\verb{fedelho}{ê}{}{}{}{}{Rapaz com atitudes infantis; criançola.}{fe.de.lho}{0}
\verb{fedentina}{}{}{}{}{s.f.}{Cheiro repugnante; fedor.}{fe.den.ti.na}{0}
\verb{feder}{ê}{}{}{}{v.i.}{Cheirar mal.}{fe.der}{\verboinum{12}}
\verb{federação}{}{}{"-ões}{}{s.f.}{União política entre estados que gozam de certa autonomia em relação a um governo central.}{fe.de.ra.ção}{0}
\verb{federação}{}{}{"-ões}{}{}{Aliança, organização, agrupamento.}{fe.de.ra.ção}{0}
\verb{federação}{}{}{"-ões}{}{}{Associação de clubes esportivos, sindicatos ou corporações.}{fe.de.ra.ção}{0}
\verb{federal}{}{}{"-ais}{}{adj.2g.}{Relativo a federação.}{fe.de.ral}{0}
\verb{federal}{}{}{"-ais}{}{}{Relativo ao governo central de um país.}{fe.de.ral}{0}
\verb{federalismo}{}{}{}{}{s.m.}{Sistema de governo em que várias províncias ou estados se reúnem para formar uma nação, cada um conservando sua autonomia.}{fe.de.ra.lis.mo}{0}
\verb{federalista}{}{}{}{}{adj.2g.}{Relativo a federalismo.}{fe.de.ra.lis.ta}{0}
\verb{federalista}{}{}{}{}{s.2g.}{Indivíduo partidário do federalismo.}{fe.de.ra.lis.ta}{0}
\verb{federalizar}{}{}{}{}{v.t.}{Tornar federal; transformar algo em bem ou serviço do Estado.}{fe.de.ra.li.zar}{\verboinum{1}}
\verb{federar}{}{}{}{}{v.t.}{Constituir um Estado em regime de federação.}{fe.de.rar}{\verboinum{1}}
\verb{federativo}{}{}{}{}{adj.}{Que é constituído em federação.}{fe.de.ra.ti.vo}{0}
\verb{fedido}{}{}{}{}{adj.}{Que cheira mal; fétido, fedorento.}{fe.di.do}{0}
\verb{fedor}{ô}{}{}{}{s.m.}{Cheiro ruim, repugnante; fedentina.}{fe.dor}{0}
\verb{fedorento}{}{}{}{}{adj.}{Que tem mau cheiro; fedido, fétido.}{fe.do.ren.to}{0}
\verb{feedback}{}{}{}{}{s.m.}{Ação de resposta a um estímulo; retorno.}{\textit{feedback}}{0}
\verb{feérico}{}{}{}{}{adj.}{Relativo ao mundo das fadas; mágico, fantástico, maravilhoso.}{fe.é.ri.co}{0}
\verb{feição}{}{}{"-ões}{}{s.f.}{Aparência de alguma coisa; forma, feitio, aspecto.}{fei.ção}{0}
\verb{feição}{}{}{"-ões}{}{}{Comportamento, índole, natureza.}{fei.ção}{0}
\verb{feição}{}{}{"-ões}{}{}{Traços do rosto; fisionomia, figura.}{fei.ção}{0}
\verb{feições}{}{}{}{}{s.f.pl.}{Conjunto dos traços do rosto; fisionomia.}{fei.ções}{0}
\verb{feijão}{}{}{"-ões}{}{s.m.}{Semente do feijoeiro.}{fei.jão}{0}
\verb{feijão}{}{Cul.}{"-ões}{}{}{Essa semente cozida, temperada ou não, e misturada com carnes, legumes etc., usada como base da alimentação de vários povos.}{fei.jão}{0}
\verb{feijão}{}{}{"-ões}{}{}{O feijoeiro.}{fei.jão}{0}
\verb{feijão"-soja}{ó}{}{feijões"-soja ⟨ó⟩}{}{s.m.}{Tipo de feijão de origem asiática usado em muitos pratos e, principalmente, do qual se extrai um tipo de óleo de cozinha; soja.}{fei.jão"-so.ja}{0}
\verb{feijoada}{}{Cul.}{}{}{s.f.}{Prato típico brasileiro, composto de feijão, carne"-seca, linguiça, toicinho e de várias outras partes do porco. }{fei.jo.a.da}{0}
\verb{feijoal}{}{}{"-ais}{}{s.m.}{Plantação de feijão.}{fei.jo.al}{0}
\verb{feijoeiro}{ê}{Bot.}{}{}{s.m.}{Nome comum a várias plantas leguminosas que produzem vagens cheias de grãos as quais se comem; feijão. }{fei.jo.ei.ro}{0}
\verb{feio}{ê}{}{}{}{adj.}{Que tem uma aparência desagradável, desproporcionada.}{fei.o}{0}
\verb{feio}{ê}{Fig.}{}{}{}{Ofensivo, indecoroso, desonesto.}{fei.o}{0}
\verb{feioso}{ô}{}{"-osos ⟨ó⟩}{"-osa ⟨ó⟩}{adj.}{Um pouco ou um tanto feio.}{fei.o.so}{0}
\verb{feira}{ê}{}{}{}{s.f.}{Conjunto de barracas, armadas na via pública, onde se vendem frutas, legumes, verduras e outras mercadorias.}{fei.ra}{0}
\verb{feira}{ê}{}{}{}{}{As compras que se fazem na feira.}{fei.ra}{0}
\verb{feira}{ê}{}{}{}{}{Lugar onde fabricantes se reúnem para expor e vender seus produtos; exposição.}{fei.ra}{0}
\verb{feirante}{}{}{}{}{s.2g.}{Indivíduo que possui barraca de feira ou que trabalha na feira.}{fei.ran.te}{0}
\verb{feita}{ê}{}{}{}{s.f.}{Ato, obra, ação. }{fei.ta}{0}
\verb{feita}{ê}{}{}{}{}{Ocasião, vez.   }{fei.ta}{0}
\verb{feitiçaria}{}{Pop.}{}{}{}{Sedução, encanto, enlevo.}{fei.ti.ça.ri.a}{0}
\verb{feitiçaria}{}{}{}{}{adj.}{Atividade de feiticeiro; bruxaria, encantamento.}{fei.ti.ça.ri.a}{0}
\verb{feiticeira}{ê}{}{}{}{s.f.}{Mulher que faz feitiçaria; bruxa.}{fei.ti.cei.ra}{0}
\verb{feiticeira}{ê}{}{}{}{}{Mulher muito atraente, fascinante, encantadora.}{fei.ti.cei.ra}{0}
\verb{feiticeiro}{ê}{}{}{}{}{Homem que atrai, encanta, fascina.}{fei.ti.cei.ro}{0}
\verb{feiticeiro}{ê}{}{}{}{s.m.}{Indivíduo que faz feitiço; bruxo, mágico.}{fei.ti.cei.ro}{0}
\verb{feitiço}{}{}{}{}{s.m.}{Ato ou efeito de enfeitiçar; bruxaria, sortilégio.}{fei.ti.ço}{0}
\verb{feitiço}{}{Fig.}{}{}{}{Encanto, fascínio, sedução.}{fei.ti.ço}{0}
\verb{feitiço}{}{}{}{}{}{Objeto a que se atribui poder sobrenatural; amuleto.}{fei.ti.ço}{0}
\verb{feitio}{}{}{}{}{s.m.}{Configuração física; forma, aparência.}{fei.ti.o}{0}
\verb{feitio}{}{}{}{}{}{Disposição de espírito; caráter, índole.}{fei.ti.o}{0}
\verb{feitio}{}{}{}{}{}{Trabalho de costura; talho de vestido.}{fei.ti.o}{0}
\verb{feito}{ê}{}{}{}{adj.}{Realizado, consumado, constituído.}{fei.to}{0}
\verb{feito}{ê}{}{}{}{s.m.}{Ato ou efeito de fazer; façanha, empresa.}{fei.to}{0}
\verb{feito}{ê}{}{}{}{}{Acabado, completo, concluído.}{fei.to}{0}
\verb{feito}{ê}{}{}{}{}{Adulto, desenvolvido, amadurecido.}{fei.to}{0}
\verb{feito}{ê}{Jur.}{}{}{}{Processo judicial.}{fei.to}{0}
\verb{feitor}{ô}{}{}{}{s.m.}{Administrador dos bens alheios.}{fei.tor}{0}
\verb{feitor}{ô}{}{}{}{}{Indivíduo encarregado dos trabalhadores escravos; capataz.}{fei.tor}{0}
\verb{feitoria}{}{}{}{}{s.f.}{Administração ou cargo do feitor.}{fei.to.ri.a}{0}
\verb{feitoria}{}{Hist.}{}{}{}{Estabelecimentos fundados pelos portugueses no litoral do Brasil e do continente africano para negociações como o tráfico de escravos.}{fei.to.ri.a}{0}
\verb{feitura}{}{}{}{}{s.f.}{Ato ou efeito de fazer; fazimento, trabalho, obra.}{fei.tu.ra}{0}
\verb{feiura}{}{}{}{}{s.f.}{Qualidade do que é feio; fealdade.}{fei.u.ra}{0}
\verb{feixe}{ch}{}{}{}{s.m.}{Agrupamento de vários objetos unidos no sentido do comprimento; molho, braçada, gavela.}{fei.xe}{0}
\verb{feixe}{ch}{Fig.}{}{}{}{Grande porção; acervo.}{fei.xe}{0}
\verb{fel}{é}{}{féis \textit{ou} feles ⟨é⟩}{}{s.m.}{Líquido muito amargo excretado pelo fígado de animais e do homem; bílis.}{fel}{0}
\verb{fel}{é}{Fig.}{féis \textit{ou} feles ⟨é⟩}{}{}{Mau humor; azedume, amargura.}{fel}{0}
\verb{felá}{}{}{}{}{s.m.}{Camponês egípcio e de outras regiões árabes.}{fe.lá}{0}
\verb{felação}{}{}{"-ões}{}{s.f.}{Prática sexual que consiste na estimulação do pênis com a boca.}{fe.la.ção}{0}
\verb{feldspato}{}{Geol.}{}{}{s.m.}{Silicato de alumínio de consistência dura, composto de sílica e potassa, dentre outros minerais.}{felds.pa.to}{0}
\verb{felicidade}{}{}{}{}{s.f.}{Estado de quem é feliz; ventura, contentamento. (\textit{O ser humano sempre buscou a felicidade.})}{fe.li.ci.da.de}{0}
\verb{felicidade}{}{}{}{}{}{Boa sorte; circunstância favorável. }{fe.li.ci.da.de}{0}
\verb{felicidade}{}{}{}{}{}{Sucesso, acerto.}{fe.li.ci.da.de}{0}
\verb{felicitação}{}{}{"-ões}{}{s.f.}{Ato ou efeito de felicitar; cumprimento, congratulação, parabéns.}{fe.li.ci.ta.ção}{0}
\verb{felicitar}{}{}{}{}{v.t.}{Tornar feliz; contentar, alegrar.}{fe.li.ci.tar}{0}
\verb{felicitar}{}{}{}{}{}{Dar parabéns; congratular, cumprimentar.}{fe.li.ci.tar}{\verboinum{1}}
\verb{felídeo}{}{Zool.}{}{}{adj.}{Diz"-se da família de mamíferos carnívoros de unhas retráteis e molares cortantes, como o leão, o gato, o lince, o tigre, a onça etc.}{fe.lí.deo}{0}
\verb{felino}{}{}{}{}{adj.}{Relativo a gato.}{fe.li.no}{0}
\verb{felino}{}{Fig.}{}{}{}{Ágil, atraente, sensual.}{fe.li.no}{0}
\verb{felino}{}{Zool.}{}{}{}{Felídeo.}{fe.li.no}{0}
\verb{feliz}{}{}{}{}{adj.2g.}{Favorecido pela sorte; afortunado, ditoso.}{fe.liz}{0}
\verb{feliz}{}{}{}{}{}{Contente, alegre, satisfeito.}{fe.liz}{0}
\verb{feliz}{}{}{}{}{}{Próspero, bem"-sucedido, rico.}{fe.liz}{0}
\verb{felizardo}{}{}{}{}{adj.}{Que é muito feliz, que tem muita sorte.}{fe.li.zar.do}{0}
\verb{felonia}{}{}{}{}{s.f.}{Rebelião do vassalo contra o senhor feudal.}{fe.lo.ni.a}{0}
\verb{felonia}{}{}{}{}{}{Ato de traição; deslealdade.}{fe.lo.ni.a}{0}
\verb{felpa}{ê}{}{}{}{s.f.}{Pelo saliente nos tecidos.}{fel.pa}{0}
\verb{felpa}{ê}{}{}{}{}{Penugem, lanugem.}{fel.pa}{0}
\verb{felpo}{ê}{}{}{}{adj.}{Felpudo.}{fel.po}{0}
\verb{felpudo}{}{}{}{}{adj.}{Que tem muita felpa; peludo, felpo.}{fel.pu.do}{0}
\verb{feltro}{ê}{}{}{}{s.m.}{Estofo de lã ou de pelo usado principalmente na fabricação de chapéus e pantufas.}{fel.tro}{0}
\verb{fêmea}{}{}{}{}{s.f.}{Animal ou vegetal do sexo feminino.}{fê.mea}{0}
\verb{fêmea}{}{Pop.}{}{}{}{Mulher.}{fê.mea}{0}
\verb{fêmea}{}{}{}{}{}{Peça que se encaixa em outra chamada macho.}{fê.mea}{0}
\verb{fêmeo}{}{}{}{}{adj.}{Relativo a fêmea; feminil.}{fê.meo}{0}
\verb{feminil}{}{}{"-is}{}{adj.2g.}{Relativo a mulheres; feminino.}{fe.mi.nil}{0}
\verb{feminilidade}{}{}{}{}{s.f.}{Qualidade, caráter, modo de ser próprio da mulher.}{fe.mi.ni.li.da.de}{0}
\verb{feminino}{}{}{}{}{adj.}{Que se refere ou pertence a mulher; fêmeo, feminil.}{fe.mi.ni.no}{0}
\verb{feminino}{}{Biol.}{}{}{}{Diz"-se do indivíduo ou de órgão vegetal ou animal portador de célula reprodutora mais volumosa.}{fe.mi.ni.no}{0}
\verb{feminino}{}{Gram.}{}{}{}{Diz"-se do gênero gramatical que se opõe ao masculino e ao neutro.}{fe.mi.ni.no}{0}
\verb{feminismo}{}{}{}{}{s.m.}{Movimento pela ampliação legal dos direitos civis e políticos da mulher na sociedade.}{fe.mi.nis.mo}{0}
\verb{feminista}{}{}{}{}{adj.2g.}{Relativo a feminismo.}{fe.mi.nis.ta}{0}
\verb{feminista}{}{}{}{}{s.2g.}{Partidário do feminismo.}{fe.mi.nis.ta}{0}
\verb{feminizar}{}{}{}{}{v.t.}{Dar feição ou caráter feminino.}{fe.mi.ni.zar}{\verboinum{1}}
\verb{fêmur}{}{Anat.}{}{}{s.m.}{Osso longo que constitui a coxa humana.}{fê.mur}{0}
\verb{fenda}{}{}{}{}{s.f.}{Abertura longa e estreita; greta, rachadura, fendimento.}{fen.da}{0}
\verb{fender}{ê}{}{}{}{v.t.}{Abrir uma fenda; rachar, sulcar.}{fen.der}{0}
\verb{fender}{ê}{}{}{}{}{Atravessar, cortar, cruzar.}{fen.der}{\verboinum{12}}
\verb{fendimento}{}{}{}{}{s.m.}{Ato ou efeito de fender; rachadura.}{fen.di.men.to}{0}
\verb{fenecer}{ê}{}{}{}{v.i.}{Tornar"-se extinto; findar, morrer.}{fe.ne.cer}{0}
\verb{fenecer}{ê}{}{}{}{}{Perder o frescor; murchar, secar.}{fe.ne.cer}{\verboinum{15}}
\verb{fenecimento}{}{}{}{}{s.m.}{Ato ou efeito de fenecer; fim, falecimento.}{fe.ne.ci.men.to}{0}
\verb{fenício}{}{}{}{}{adj.}{Relativo à Fenícia, região do Oriente Médio, atual Líbano.}{fe.ní.cio}{0}
\verb{fenício}{}{}{}{}{s.m.}{Indivíduo natural ou habitante dessa região.}{fe.ní.cio}{0}
\verb{fenício}{}{}{}{}{}{Língua semítica extinta, cujo alfabeto é considerado o mais antigo das escritas alfabéticas.}{fe.ní.cio}{0}
\verb{fênico}{}{Quím.}{}{}{adj.}{Relativo ao fenol.}{fê.ni.co}{0}
\verb{fênix}{cs}{Mit.}{}{}{s.f.}{Na mitologia egípcia, ave fabulosa que vivia cerca de 500 anos, morria queimada e renascia das próprias cinzas.}{fê.nix}{0}
\verb{fênix}{cs}{Fig.}{}{}{}{Ser único, raro, superior.}{fê.nix}{0}
\verb{feno}{ê}{}{}{}{s.m.}{Erva ceifada e seca usada para alimento do gado bovino.}{fe.no}{0}
\verb{fenol}{ó}{Quím.}{}{}{s.m.}{Classe de compostos orgânicos formados por uma ou mais hidroxilas ligadas a anéis aromáticos.}{fe.nol}{0}
\verb{fenomenal}{}{}{"-ais}{}{adj.2g.}{Relativo a fenômeno.}{fe.no.me.nal}{0}
\verb{fenomenal}{}{}{"-ais}{}{}{Surpreendente, formidável, espantoso.}{fe.no.me.nal}{0}
\verb{fenômeno}{}{}{}{}{s.m.}{Fato natural passível de observação científica.}{fe.nô.me.no}{0}
\verb{fenômeno}{}{}{}{}{}{Acontecimento raro e surpreendente; prodígio, maravilha.}{fe.nô.me.no}{0}
\verb{fenômeno}{}{Fig.}{}{}{}{Pessoa ou coisa original, notável.}{fe.nô.me.no}{0}
\verb{fenomenologia}{}{Filos.}{}{}{s.f.}{Doutrina filosófica que estuda os fenômenos em si mesmos tal como eles se apresentam à consciência.}{fe.no.me.no.lo.gi.a}{0}
\verb{fenótipo}{}{Biol.}{}{}{s.m.}{Conjunto dos caracteres visíveis de um indivíduo, que exprimem as reações de seu genótipo em face de seu meio ambiente.}{fe.nó.ti.po}{0}
\verb{fera}{é}{}{}{}{s.f.}{Nome comum aos mamíferos carnívoros, bravios e selvagens.}{fe.ra}{0}
\verb{fera}{é}{Fig.}{}{}{}{Pessoa cruel e má.}{fe.ra}{0}
\verb{fera}{é}{Pop.}{}{}{}{Pessoa muito habilidosa em alguma atividade.}{fe.ra}{0}
\verb{feracidade}{}{}{}{}{s.f.}{Qualidade ou estado de feraz; fertilidade, fecundidade.}{fe.ra.ci.da.de}{0}
\verb{feraz}{}{}{}{}{adj.2g.}{De grande produtividade; fértil, fecundo.}{fe.raz}{0}
\verb{féretro}{}{}{}{}{s.m.}{Caixão mortuário; ataúde, esquife.}{fé.re.tro}{0}
\verb{fereza}{ê}{}{}{}{s.f.}{Qualidade ou estado de feroz; ferocidade, braveza.}{fe.re.za}{0}
\verb{féria}{}{}{}{}{s.f.}{Dia da semana.}{fé.ria}{0}
\verb{féria}{}{}{}{}{}{Salário diário de operário.}{fé.ria}{0}
\verb{féria}{}{}{}{}{}{Dinheiro apurado nas vendas diárias de um estabelecimento comercial.}{fé.ria}{0}
\verb{feriado}{}{}{}{}{s.m.}{Dia em que se suspendem as atividades em empresas, escolas, repartições etc. por motivo religioso ou de comemoração civil.}{fe.ri.a.do}{0}
\verb{férias}{}{}{}{}{s.f.pl.}{Período de dias consecutivos destinados ao descanso dos trabalhadores e dos estudantes.}{fé.ri.as}{0}
\verb{ferida}{}{}{}{}{s.f.}{Ato ou efeito de ferir; ferimento, lesão, chaga.}{fe.ri.da}{0}
\verb{ferida}{}{}{}{}{}{Ofensa, injúria, mágoa.}{fe.ri.da}{0}
\verb{ferido}{}{}{}{}{adj.}{Que recebeu ferimento; machucado.}{fe.ri.do}{0}
\verb{ferido}{}{}{}{}{}{Maltratado, magoado, injuriado.}{fe.ri.do}{0}
\verb{ferimento}{}{}{}{}{s.m.}{Ato ou efeito de ferir; ferida, lesão.}{fe.ri.men.to}{0}
\verb{ferino}{}{}{}{}{adj.}{Semelhante a fera; feroz.}{fe.ri.no}{0}
\verb{ferino}{}{Fig.}{}{}{}{Penetrante, agudo, mordaz.}{fe.ri.no}{0}
\verb{ferir}{}{}{}{}{v.t.}{Causar ferimento; machucar.}{fe.rir}{0}
\verb{ferir}{}{}{}{}{}{Provocar sofrimento; magoar, ofender.}{fe.rir}{0}
\verb{ferir}{}{}{}{}{}{Violar, transgredir, contrariar.}{fe.rir}{\verboinum{29}}
\verb{fermentação}{}{Bioquím.}{"-ões}{}{s.f.}{Reação de um composto orgânico a um fermento que o decompõe.}{fer.men.ta.ção}{0}
\verb{fermentar}{}{}{}{}{v.t.}{Decompor pela fermentação; fazer crescer, levedar. (\textit{O açúcar ajuda a fermentar a massa do bolo.})}{fer.men.tar}{0}
\verb{fermentar}{}{Fig.}{}{}{}{Agitar, fomentar, excitar.}{fer.men.tar}{\verboinum{1}}
\verb{fermento}{}{}{}{}{s.m.}{Substância capaz de provocar a fermentação em outra; enzima.}{fer.men.to}{0}
\verb{fermento}{}{}{}{}{}{Massa de farinha que, tendo azedado, provoca a fermentação em outra massa de pão quando misturada a esta.}{fer.men.to}{0}
\verb{férmio}{}{Quím.}{}{}{s.m.}{Elemento químico radioativo, do grupo dos actinídeos, obtido artificialmente. \elemento{100}{(257)}{Fm}.}{fér.mio}{0}
\verb{fernando"-noronhense}{}{}{fernando"-noronhenses}{}{adj.2g.}{Relativo a Fernando de Noronha, ilha do estado de Pernambuco.}{fer.nan.do"-no.ro.nhen.se}{0}
\verb{fernando"-noronhense}{}{}{fernando"-noronhenses}{}{s.2g.}{Indivíduo natural ou habitante dessa ilha.}{fer.nan.do"-no.ro.nhen.se}{0}
\verb{fero}{é}{}{}{}{adj.}{Feroz, bravio, selvagem.}{fe.ro}{0}
\verb{ferocidade}{}{}{}{}{s.f.}{Qualidade de fero ou feroz; fereza, crueldade.}{fe.ro.ci.da.de}{0}
\verb{feroz}{ó}{}{}{}{adj.2g.}{Que se porta como uma fera; bravio, selvagem, fero.}{fe.roz}{0}
\verb{feroz}{ó}{Fig.}{}{}{}{Cruel, desumano, violento, perverso.}{fe.roz}{0}
\verb{ferra}{é}{}{}{}{s.f.}{Ato ou efeito de ferrar o gado.}{fer.ra}{0}
\verb{ferra}{é}{}{}{}{}{Época em que se ferra o gado.}{fer.ra}{0}
\verb{ferrabrás}{}{}{}{}{adj.2g.}{Que conta bravatas; fanfarrão, valentão.}{fer.ra.brás}{0}
\verb{ferrado}{}{}{}{}{adj.}{Diz"-se do cavalo em que foi colocada ferradura.}{fer.ra.do}{0}
\verb{ferrado}{}{Fig.}{}{}{}{Muito apegado; obstinado, teimoso.}{fer.ra.do}{0}
\verb{ferrado}{}{Pop.}{}{}{}{Que se encontra em péssima situação; encrencado, atrapalhado.}{fer.ra.do}{0}
\verb{ferrador}{ô}{}{}{}{s.m.}{Operário encarregado de ferrar cavalos.}{fer.ra.dor}{0}
\verb{ferradura}{}{}{}{}{s.f.}{Peça de ferro que se coloca no casco dos cavalos.}{fer.ra.du.ra}{0}
\verb{ferrageiro}{ê}{}{}{}{s.m.}{Indivíduo que comercializa ferragem ou obras de ferro; ferragista.}{fer.ra.gei.ro}{0}
\verb{ferragem}{}{}{"-ens}{}{s.f.}{Colocação das ferraduras em animais de sela; ferra.}{fer.ra.gem}{0}
\verb{ferragem}{}{}{"-ens}{}{}{Conjunto de peças de ferro utilizadas em um trabalho, uma obra de arte etc.}{fer.ra.gem}{0}
\verb{ferragista}{}{}{}{}{s.2g.}{Ferrageiro.}{fer.ra.gis.ta}{0}
\verb{ferramenta}{}{}{}{}{s.f.}{Instrumento utilizado na execução de um ofício ou de uma arte.}{fer.ra.men.ta}{0}
\verb{ferramenta}{}{}{}{}{}{Utensílio de ferro de um trabalhador.}{fer.ra.men.ta}{0}
\verb{ferramenteiro}{ê}{}{}{}{s.m.}{Operário que trabalha na fabricação de ferramentas.}{fer.ra.men.tei.ro}{0}
\verb{ferrão}{}{}{"-ões}{}{s.m.}{Ponta aguçada de ferro que se acrescenta a uma vara; aguilhão.}{fer.rão}{0}
\verb{ferrão}{}{}{"-ões}{}{}{Órgão pontiagudo de certos insetos, como da abelha, do marimbondo etc. }{fer.rão}{0}
\verb{ferrão}{}{Por ext.}{"-ões}{}{}{O ferimento feito pelo ferrão, ferroada.}{fer.rão}{0}
\verb{ferrar}{}{}{}{}{v.t.}{Colocar ferradura em animal de sela.}{fer.rar}{0}
\verb{ferrar}{}{}{}{}{}{Marcar o animal com ferro em brasa.}{fer.rar}{0}
\verb{ferrar}{}{}{}{}{}{Ornar ou guarnecer de ferro.}{fer.rar}{0}
\verb{ferrar}{}{Pop.}{}{}{pron.}{Sair"-se mal; encrencar"-se.}{fer.rar}{\verboinum{1}}
\verb{ferraria}{}{}{}{}{s.f.}{Local onde se fabricam grandes peças de ferro.}{fer.ra.ri.a}{0}
\verb{ferreiro}{ê}{}{}{}{s.m.}{Operário que trabalha em ferro; artífice.}{fer.rei.ro}{0}
\verb{ferrenho}{ê}{}{}{}{adj.}{Da natureza do ferro; duro, inflexível.}{fer.re.nho}{0}
\verb{ferrenho}{ê}{Fig.}{}{}{}{Persistente, obstinado, pertinaz.}{fer.re.nho}{0}
\verb{férreo}{}{}{}{}{adj.}{Que contém ferro ou sais de ferro; ferruginoso.}{fér.re.o}{0}
\verb{férreo}{}{}{}{}{}{Que não cede; inflexível, duro, imbatível.}{fér.re.o}{0}
\verb{ferrete}{ê}{}{}{}{s.m.}{Ferro com que se marcavam os escravos e os criminosos.}{fer.re.te}{0}
\verb{ferretear}{}{}{}{}{v.t.}{Marcar com ferrete.}{fer.re.te.ar}{0}
\verb{ferretear}{}{Fig.}{}{}{}{Desonrar, estigmatizar, macular.}{fer.re.te.ar}{\verboinum{4}}
\verb{ferretoar}{}{}{}{}{v.t.}{Ferroar.}{fer.re.to.ar}{\verboinum{7}}
\verb{ferrífero}{}{}{}{}{adj.}{Que é composto de ferro ou de sais de ferro.}{fer.rí.fe.ro}{0}
\verb{ferro}{é}{Quím.}{}{}{s.m.}{Elemento químico metálico, acinzentado, abundante na crosta terrestre de onde é extraído industrialmente; utilizado em muitas ligas especiais e importantes, como o aço e o ferro fundido. \elemento{26}{55.845}{Fe}.}{fer.ro}{0}
\verb{ferro}{é}{}{}{}{}{Qualquer instrumento ou utensílio feito de ferro.}{fer.ro}{0}
\verb{ferroada}{}{}{}{}{s.f.}{Picada com ferrão; aguilhoada.}{fer.ro.a.da}{0}
\verb{ferroada}{}{Fig.}{}{}{}{Censura sarcástica e picante.}{fer.ro.a.da}{0}
\verb{ferroar}{}{}{}{}{v.t.}{Dar ferroadas; aguilhoar; ferretoar.}{fer.ro.ar}{0}
\verb{ferroar}{}{Fig.}{}{}{}{Censurar, criticar.}{fer.ro.ar}{\verboinum{7}}
\verb{ferro"-gusa}{é}{}{ferros"-gusas \textit{ou} ferros"-gusa ⟨é⟩}{}{s.m.}{O ferro que se obtém diretamente de um alto"-forno, com alto teor de carbono e várias impurezas; gusa.}{fer.ro"-gu.sa}{0}
\verb{ferrolho}{ô}{}{}{}{s.m.}{Peça de ferro corrediço com que se trancam janelas, portas etc.; tranca, trinco.}{fer.ro.lho}{0}
\verb{ferroso}{ô}{}{"-osos ⟨ó⟩}{"-osa ⟨ó⟩}{adj.}{Que contém ferro; ferrífero, ferruginoso.}{fer.ro.so}{0}
\verb{ferro"-velho}{é\ldots{}é}{}{ferros"-velhos ⟨é\ldots{}é⟩}{}{s.m.}{Objeto metálico velho refundível.}{fer.ro"-ve.lho}{0}
\verb{ferro"-velho}{é\ldots{}é}{}{ferros"-velhos ⟨é\ldots{}é⟩}{}{}{Tudo o que está em más condições de conservação e que tem pouco valor; sucata.}{fer.ro"-ve.lho}{0}
\verb{ferro"-velho}{é\ldots{}é}{}{ferros"-velhos ⟨é\ldots{}é⟩}{}{}{Depósito que negocia esse material.}{fer.ro"-ve.lho}{0}
\verb{ferrovia}{}{}{}{}{s.f.}{Sistema de transporte sobre trilhos; estrada de ferro; via férrea.}{fer.ro.vi.a}{0}
\verb{ferroviário}{}{}{}{}{adj.}{Relativo a ferrovia.}{fer.ro.vi.á.rio}{0}
\verb{ferroviário}{}{}{}{}{s.m.}{Trabalhador de ferrovia.}{fer.ro.vi.á.rio}{0}
\verb{ferrugem}{}{}{"-ens}{}{s.f.}{Produto da corrosão do ferro provocada pela exposição à umidade.}{fer.ru.gem}{0}
\verb{ferrugem}{}{Bot.}{"-ens}{}{}{Doença que ataca certas plantas gramíneas, principalmente o trigo.}{fer.ru.gem}{0}
\verb{ferrugento}{}{}{}{}{adj.}{Que tem ferrugem; enferrujado.}{fer.ru.gen.to}{0}
\verb{ferrugento}{}{Fig.}{}{}{}{Antiquado, desusado, emperrado.}{fer.ru.gen.to}{0}
\verb{ferruginoso}{ô}{}{"-osos ⟨ó⟩}{"-osa ⟨ó⟩}{adj.}{Que contém ferro ou sais de ferro; ferroso.}{fer.ru.gi.no.so}{0}
\verb{ferruginoso}{ô}{}{"-osos ⟨ó⟩}{"-osa ⟨ó⟩}{}{Da cor do ferro.}{fer.ru.gi.no.so}{0}
\verb{fértil}{}{}{"-eis}{}{adj.2g.}{Que produz ou reproduz com facilidade e abundância; fecundo.}{fér.til}{0}
\verb{fértil}{}{}{"-eis}{}{}{De muita criatividade; inventivo.}{fér.til}{0}
\verb{fertilidade}{}{}{}{}{s.f.}{Qualidade de fértil; fecundidade.}{fer.ti.li.da.de}{0}
\verb{fertilização}{}{}{"-ões}{}{s.f.}{Ato ou efeito de fertilizar; fecundação.}{fer.ti.li.za.ção}{0}
\verb{fertilizante}{}{}{}{}{adj.2g.}{Que fertiliza; adubo.}{fer.ti.li.zan.te}{0}
\verb{fertilizar}{}{}{}{}{v.t.}{Tornar fértil; fecundar, adubar.}{fer.ti.li.zar}{\verboinum{1}}
\verb{férula}{}{}{}{}{s.f.}{Instrumento de castigo; palmatória.}{fé.ru.la}{0}
\verb{fervedoiro}{ô}{}{}{}{}{Var. de \textit{fervedouro}.}{fer.ve.doi.ro}{0}
\verb{fervedouro}{ô}{}{}{}{s.m.}{Movimento como de um líquido em ebulição; agitação.}{fer.ve.dou.ro}{0}
\verb{fervente}{}{}{}{}{adj.2g.}{Que ferve; que está em ebulição.}{fer.ven.te}{0}
\verb{fervente}{}{Fig.}{}{}{}{Veemente, fervoroso, ardente.}{fer.ven.te}{0}
\verb{ferver}{ê}{}{}{}{v.i.}{Entrar ou estar em ebulição.}{fer.ver}{0}
\verb{ferver}{ê}{}{}{}{}{Sentir grande calor; queimar, arder.}{fer.ver}{0}
\verb{ferver}{ê}{}{}{}{v.t.}{Limpar ou lavar algo em água fervente; esterilizar. (\textit{A mulher ferveu a roupa para eliminar todos os micróbios.})}{fer.ver}{\verboinum{12}}
\verb{férvido}{}{}{}{}{adj.}{Muito quente; abrasador, tórrido.}{fér.vi.do}{0}
\verb{fervilhar}{}{}{}{}{v.i.}{Ferver continuamente.}{fer.vi.lhar}{0}
\verb{fervilhar}{}{Fig.}{}{}{}{Mexer"-se muito; agitar"-se; excitar"-se. (\textit{O público fervilhava com o })show\textit{.}}{fer.vi.lhar}{\verboinum{1}}
\verb{fervor}{ô}{Fig.}{}{}{}{Devoção ardente; zelo, paixão.}{fer.vor}{0}
\verb{fervor}{ô}{}{}{}{s.m.}{Ato de ferver; fervura.}{fer.vor}{0}
\verb{fervoroso}{ô}{}{"-osos ⟨ó⟩}{"-osa ⟨ó⟩}{adj.}{Cheio de fervor; ardoroso.}{fer.vo.ro.so}{0}
\verb{fervura}{}{}{}{}{s.f.}{Estado de um líquido em ebulição.}{fer.vu.ra}{0}
\verb{fervura}{}{Fig.}{}{}{}{Alvoroço, excitação, agitação.}{fer.vu.ra}{0}
\verb{fescenino}{}{}{}{}{adj.}{De caráter obsceno; licencioso, libertino, devasso.}{fes.ce.ni.no}{0}
\verb{festa}{é}{}{}{}{s.f.}{Reunião alegre para fins de divertimento ou de comemoração de algo.}{fes.ta}{0}
\verb{festa}{é}{}{}{}{}{Solenidade, cerimônia em regozijo por algum evento ou data.}{fes.ta}{0}
\verb{festa}{é}{}{}{}{}{Demonstração de alegria; júbilo.}{fes.ta}{0}
\verb{festança}{}{}{}{}{s.f.}{Grande festa com muita animação; festejo.}{fes.tan.ça}{0}
\verb{festão}{}{}{"-ões}{}{s.m.}{Ornato que se assemelha a uma grinalda com folhagens e flores.}{fes.tão}{0}
\verb{festas}{é}{}{}{}{s.f.pl.}{O Natal e o Ano"-Novo.}{fes.tas}{0}
\verb{festeiro}{ê}{}{}{}{adj.}{Que faz ou organiza festas.}{fes.tei.ro}{0}
\verb{festeiro}{ê}{}{}{}{}{Que gosta de brincar e de fazer carinho; meigo.}{fes.tei.ro}{0}
\verb{festejar}{}{}{}{}{v.i.}{Fazer festa por algum evento; comemorar, celebrar}{fes.te.jar}{\verboinum{1}}
\verb{festejo}{ê}{}{}{}{s.m.}{Ato ou efeito de festejar; comemoração, solenidade.}{fes.te.jo}{0}
\verb{festim}{}{}{"-ins}{}{s.m.}{Banquete em que se reúne um grupo de pessoas para alguma solenidade.}{fes.tim}{0}
\verb{festim}{}{}{"-ins}{}{}{Cartucho sem projétil para simulação de tiro.}{fes.tim}{0}
\verb{festival}{}{}{"-ais}{}{s.m.}{Série de espetáculos artísticos, geralmente periódicos, em que há apresentação de filmes, músicas etc.; mostra.}{fes.ti.val}{0}
\verb{festival}{}{}{"-ais}{}{}{Grande quantidade; série. (\textit{Ele disse um festival de asneiras em seu discurso de posse.})}{fes.ti.val}{0}
\verb{festividade}{}{}{}{}{s.f.}{Festa religiosa ou cívica; comemoração, solenidade.}{fes.ti.vi.da.de}{0}
\verb{festivo}{}{}{}{}{adj.}{Relativo a festa; alegre, comemorativo. (\textit{No começo de dezembro, a cidade já apresentava um ar festivo.})}{fes.ti.vo}{0}
\verb{festo}{ê}{}{}{}{s.m.}{Dobra ao meio em toda a extensão de um tecido; vinco.}{fes.to}{0}
\verb{festoar}{}{}{}{}{v.t.}{Ornar com festões.}{fes.to.ar}{\verboinum{7}}
\verb{fetal}{}{}{}{}{adj.2g.}{Relativo a feto.}{fe.tal}{0}
\verb{fetiche}{}{}{}{}{s.m.}{Objeto a que se atribui muito poder e que é cultuado como ídolo por alguns povos.}{fe.ti.che}{0}
\verb{fetichismo}{}{}{}{}{s.m.}{Adoração, veneração a fetiches.}{fe.ti.chis.mo}{0}
\verb{fetidez}{ê}{}{}{}{s.f.}{Qualidade ou estado de fétido; fedor, mau cheiro.}{fe.ti.dez}{0}
\verb{fétido}{}{}{}{}{adj.}{Que cheira muito mal; fedorento, fedido.}{fé.ti.do}{0}
\verb{feto}{é}{Biol.}{}{}{s.m.}{Ser vivo já formado dentro do ventre materno, depois que adquire a aparência do adulto de sua espécie.}{fe.to}{0}
\verb{feudal}{}{}{"-ais}{}{adj.2g.}{Relativo a feudo ou a feudalismo.}{feu.dal}{0}
\verb{feudalismo}{}{}{}{}{s.m.}{Sistema socioeconômico vigente na Europa na Idade Média que se baseava nas relações entre vassalos e senhores feudais.}{feu.da.lis.mo}{0}
\verb{feudatário}{}{}{}{}{adj.}{Proprietário do feudo; senhor feudal.}{feu.da.tá.rio}{0}
\verb{feudatário}{}{}{}{}{}{Indivíduo subordinado a ele; vassalo, súdito.}{feu.da.tá.rio}{0}
\verb{feudo}{}{}{}{}{s.m.}{Propriedade rural que um vassalo recebia de um senhor feudal em troca de serviços ou pagamento de rendas.}{feu.do}{0}
\verb{fevereiro}{ê}{}{}{}{s.m.}{O segundo mês do ano civil.}{fe.ve.rei.ro}{0}
\verb{fez}{ê}{}{}{}{s.m.}{Gorro cônico, geralmente vermelho, usado especialmente pelos turcos.}{fez}{0}
\verb{fezes}{é}{}{}{}{s.f.pl.}{Material alimentar que não foi absorvido pelo organismo e é excretado; excremento, borra.}{fe.zes}{0}
\verb{fezes}{é}{Pop.}{}{}{}{Cocô, bosta, merda.}{fe.zes}{0}
\verb{fi}{}{}{}{}{s.m.}{Vigésima primeira letra do alfabeto grego.}{fi}{0}
\verb{fiação}{}{}{"-ões}{}{s.f.}{Ato ou efeito de fiar.}{fi.a.ção}{0}
\verb{fiação}{}{}{"-ões}{}{s.f.}{Instalação elétrica de uma casa.}{fi.a.ção}{0}
\verb{fiação}{}{}{"-ões}{}{}{Fábrica de tecidos têxteis.}{fi.a.ção}{0}
\verb{fiacre}{}{}{}{}{s.m.}{Antiga carruagem de aluguel, puxada por um só cavalo.}{fi.a.cre}{0}
\verb{fiada}{}{}{}{}{s.f.}{Alinhamento, sequência, fileira.}{fi.a.da}{0}
\verb{fiado}{}{}{}{}{adj.}{Que se fiou; reduzido a fio.}{fi.a.do}{0}
\verb{fiado}{}{}{}{}{adj.}{Que tem fé ou confiança; confiado.}{fi.a.do}{0}
\verb{fiado}{}{}{}{}{}{Negociado sem pagamento no ato.}{fi.a.do}{0}
\verb{fiado}{}{}{}{}{adv.}{A crédito, a prazo. (\textit{Aquele armazém não vende fiado.})}{fi.a.do}{0}
\verb{fiador}{ô}{}{}{}{s.m.}{Indivíduo que garante o pagamento de uma dívida; avalista. (\textit{Para se alugar uma casa, é necessário um fiador.})}{fi.a.dor}{0}
\verb{fiambre}{}{}{}{}{s.m.}{Presunto ou carne preparada para ser comida fria; apresuntado.}{fi.am.bre}{0}
\verb{fiança}{}{}{}{}{s.f.}{Obrigação de um fiador; responsabilidade.}{fi.an.ça}{0}
\verb{fiança}{}{}{}{}{}{Quantia paga por um acusado para responder a um processo criminal em liberdade.}{fi.an.ça}{0}
\verb{fiandeira}{ê}{}{}{}{s.f.}{Mulher que se ocupa em fiar.}{fi.an.dei.ra}{0}
\verb{fiandeira}{ê}{Zool.}{}{}{}{Cavidade abdominal nas aranhas, por onde é excretado o fio com que fazem a teia.}{fi.an.dei.ra}{0}
\verb{fiandeiro}{ê}{}{}{}{s.m.}{Indivíduo que fia por ofício.}{fi.an.dei.ro}{0}
\verb{fiapo}{}{}{}{}{s.m.}{Fio fino e curto; fiozinho. (\textit{A toalha estava cheia de fiapos de linha.})}{fi.a.po}{0}
\verb{fiar}{}{}{}{}{v.t.}{Vender a crédito, sem pagamento à vista.}{fi.ar}{0}
\verb{fiar}{}{}{}{}{v.t.}{Reduzir a fios.}{fi.ar}{\verboinum{1}}
\verb{fiar}{}{}{}{}{}{Confiar, acreditar.}{fi.ar}{\verboinum{1}}
\verb{fiasco}{}{}{}{}{s.m.}{Resultado desastroso de uma tentativa; malogro.}{fi.as.co}{0}
\verb{fiasco}{}{}{}{}{}{Insucesso, fracasso.}{fi.as.co}{0}
\verb{fibra}{}{Biol.}{}{}{s.f.}{Estrutura filamentosa disposta em forma de feixe, encontrada nos tecidos animais e vegetais ou em algumas substâncias minerais. }{fi.bra}{0}
\verb{fibra}{}{Fig.}{}{}{}{Força de vontade; firmeza de caráter; energia. (\textit{Esse esportista é um homem de fibra.})}{fi.bra}{0}
\verb{fibroide}{}{}{}{}{adj.2g.}{Relativo a fibra.}{fi.broi.de}{0}
\verb{fibroide}{}{}{}{}{}{Que se assemelha a fibra.}{fi.broi.de}{0}
\verb{fibroma}{}{Med.}{}{}{s.m.}{Tumor benigno de tecido fibroso.}{fi.bro.ma}{0}
\verb{fibromuscular}{}{}{}{}{adj.2g.}{Constituído de fibras e músculos.}{fi.bro.mus.cu.lar}{0}
\verb{fibroso}{ô}{}{"-osos ⟨ó⟩}{"-osa ⟨ó⟩}{adj.}{Que contém fibras.}{fi.bro.so}{0}
\verb{fíbula}{}{Anat.}{}{}{s.f.}{Osso longo situado na face externa da perna, também chamado de perônio.}{fí.bu.la}{0}
\verb{ficar}{}{}{}{}{v.t.}{Manter em um lugar; continuar, permanecer.}{fi.car}{0}
\verb{ficar}{}{}{}{}{v.i.}{Estar localizado; situar"-se.}{fi.car}{0}
\verb{ficar}{}{}{}{}{}{Passar de um estado para outro; tornar"-se.}{fi.car}{0}
\verb{ficar}{}{}{}{}{}{Durar, perdurar, persistir. }{fi.car}{0}
\verb{ficar}{}{}{}{}{}{Sobrar, remanescer, restar. }{fi.car}{\verboinum{2}}
\verb{ficção}{}{}{"-ões}{}{s.f.}{Ato ou efeito de fingir; fingimento, simulação.}{fic.ção}{0}
\verb{ficção}{}{}{"-ões}{}{}{Resultado da imaginação; criação da fantasia; artifício.}{fic.ção}{0}
\verb{ficção}{}{Liter.}{"-ões}{}{}{Designação genérica para romances, novelas ou contos que narram histórias criadas pela imaginação de seus autores.}{fic.ção}{0}
\verb{ficcionista}{}{}{}{}{s.2g.}{Autor de obras de ficção.}{fic.ci.o.nis.ta}{0}
\verb{ficha}{}{}{}{}{s.f.}{Folha ou cartão solto onde são feitas anotações para posterior consulta ou classificação.}{fi.cha}{0}
\verb{ficha}{}{}{}{}{}{Peça pequena de metal ou de plástico que é prova de algum pagamento.}{fi.cha}{0}
\verb{ficha}{}{}{}{}{}{Informação confidencial sobre alguém ou algo.}{fi.cha}{0}
\verb{fichamento}{}{}{}{}{s.m.}{Ato ou efeito de fichar.}{fi.cha.men.to}{0}
\verb{fichamento}{}{}{}{}{}{Conjunto de dados ou informações anotado para posterior consulta. (\textit{Os alunos podiam consultar os fichamentos dos livros durante a prova.})}{fi.cha.men.to}{0}
\verb{fichar}{}{}{}{}{v.t.}{Registrar em fichas; anotar, cadastrar.}{fi.char}{\verboinum{1}}
\verb{fichário}{}{}{}{}{s.m.}{Conjunto de fichas de anotação; catálogo.}{fi.chá.rio}{0}
\verb{fichário}{}{}{}{}{}{Móvel onde se guardam as fichas catalogadas.}{fi.chá.rio}{0}
\verb{ficologia}{}{Biol.}{}{}{s.f.}{Ramo da Biologia que estuda as algas.}{fi.co.lo.gi.a}{0}
\verb{fictício}{}{}{}{}{adj.}{Que se imagina; irreal, ilusório, simulado.}{fic.tí.cio}{0}
\verb{fícus}{}{Bot.}{}{}{s.m.}{Árvore ornamental de flores pequenas, muito usada na arborização de ruas.}{fí.cus}{0}
\verb{fidalgo}{}{}{}{}{s.m.}{Indivíduo que possui título de nobreza; nobre.}{fi.dal.go}{0}
\verb{fidalgo}{}{}{}{}{}{Indivíduo que procede como nobre; cortês, distinto, generoso.}{fi.dal.go}{0}
\verb{fidalguia}{}{}{}{}{s.f.}{Qualidade ou modos de fidalgo; nobreza, generosidade.}{fi.dal.gui.a}{0}
\verb{fidedignidade}{}{}{}{}{s.f.}{Qualidade de fidedigno; autenticidade, confiança.}{fi.de.dig.ni.da.de}{0}
\verb{fidedigno}{}{}{}{}{adj.}{Em que se pode acreditar; que é digno de confiança e fé.}{fi.de.dig.no}{0}
\verb{fidelidade}{}{}{}{}{s.f.}{Qualidade do que é fiel; lealdade.}{fi.de.li.da.de}{0}
\verb{fidelidade}{}{}{}{}{}{Exatidão entre o original e a cópia.}{fi.de.li.da.de}{0}
\verb{fidúcia}{}{}{}{}{s.f.}{Gesto ou comportamento confiante, ousado; atrevimento, segurança.}{fi.dú.cia}{0}
\verb{fiducial}{}{}{"-ais}{}{adj.2g.}{Relativo a fidúcia.}{fi.du.ci.al}{0}
\verb{fiduciário}{}{}{}{}{adj.}{Que depende de confiança.}{fi.du.ci.á.rio}{0}
\verb{fieira}{ê}{}{}{}{s.f.}{Barbante com que se faz girar o pião.}{fi.ei.ra}{0}
\verb{fieira}{ê}{}{}{}{}{Fileira, sequência.}{fi.ei.ra}{0}
\verb{fieira}{ê}{}{}{}{}{Conjunto de coisas presas por uma linha; enfiada.}{fi.ei.ra}{0}
\verb{fiel}{é}{}{"-éis}{}{adj.2g.}{Que cumpre o que prometeu.}{fi.el}{0}
\verb{fiel}{é}{}{"-éis}{}{}{Em que se pode confiar; leal.}{fi.el}{0}
\verb{fiel}{é}{}{"-éis}{}{}{Que reproduz exatamente um fato ou um documento; verdadeiro.}{fi.el}{0}
\verb{fiel}{é}{}{"-éis}{}{s.m.}{Seguidor de uma religião; crente.}{fi.el}{0}
\verb{figa}{}{}{}{}{s.f.}{Objeto em forma de mão fechada, com o polegar entre o dedo indicador e o médio, usado como amuleto.}{fi.ga}{0}
\verb{figa}{}{}{}{}{}{Sinal que se faz com os dedos postos como nesse objeto, para esconjurar ou se proteger.}{fi.ga}{0}
\verb{figadal}{}{}{"-ais}{}{adj.2g.}{Relativo a fígado; hepático.}{fi.ga.dal}{0}
\verb{figadal}{}{Fig.}{"-ais}{}{}{Sentimento hostil muito profundo que se tem contra alguém ou algo.}{fi.ga.dal}{0}
\verb{fígado}{}{Anat.}{}{}{s.m.}{Órgão situado na parte superior do abdômen, à direita, que apresenta numerosas funções, dentre elas a secreção da bílis.}{fí.ga.do}{0}
\verb{figo}{}{}{}{}{s.m.}{Fruto da figueira, de polpa macia e rica em açúcar.}{fi.go}{0}
\verb{figueira}{ê}{Bot.}{}{}{s.f.}{Árvore pequena que produz um fruto de polpa macia e sabor doce muito apreciado.}{fi.guei.ra}{0}
\verb{figura}{}{}{}{}{s.f.}{Forma exterior de um corpo; imagem, delineamento.}{fi.gu.ra}{0}
\verb{figura}{}{}{}{}{}{Personalidade que se destaca, que marca; vulto.}{fi.gu.ra}{0}
\verb{figura}{}{}{}{}{}{Cada uma das cartas do baralho que representam o rei, o valete ou a dama.}{fi.gu.ra}{0}
\verb{figura}{}{Geom.}{}{}{}{Conjunto de linhas, pontos ou superfícies que determinam uma direção ou limitam um espaço.}{fi.gu.ra}{0}
\verb{figura}{}{Gram.}{}{}{}{Recurso linguístico que, afastado de uma norma linguística, cria efeitos de expressividade.}{fi.gu.ra}{0}
\verb{figuração}{}{}{"-ões}{}{s.f.}{Ato ou efeito de figurar; representação.}{fi.gu.ra.ção}{0}
\verb{figuração}{}{}{"-ões}{}{}{Mera presença decorativa em uma reunião, assembleia ou gravação de novela, cinema etc.}{fi.gu.ra.ção}{0}
\verb{figurado}{}{Gram.}{}{}{adj.}{Em que há figura de linguagem; metafórico, alegórico.}{fi.gu.ra.do}{0}
\verb{figurado}{}{}{}{}{}{Hipotético, suposto.}{fi.gu.ra.do}{0}
\verb{figurante}{}{}{}{}{s.2g.}{Personagem que participa de uma cena de teatro, novela ou cinema, sozinho ou misturado com um grupo de pessoas, ocupando um papel secundário, geralmente sem fala.}{fi.gu.ran.te}{0}
\verb{figurão}{}{}{"-ões}{}{s.m.}{Indivíduo muito importante em algum setor da sociedade. (\textit{Aquele industrial é um figurão na política local.})}{fi.gu.rão}{0}
\verb{figurar}{}{}{}{}{v.i.}{Representar por pintura, escultura ou desenho.}{fi.gu.rar}{0}
\verb{figurar}{}{}{}{}{}{Parecer, semelhar, aparentar.}{fi.gu.rar}{0}
\verb{figurar}{}{}{}{}{}{Fazer parte de um conjunto; estar incluído. (\textit{Essa cidade nem figura no mapa!})}{fi.gu.rar}{\verboinum{1}}
\verb{figurativo}{}{}{}{}{adj.}{Que figura; simbólico, representativo.}{fi.gu.ra.ti.vo}{0}
\verb{figurinha}{}{}{}{}{s.f.}{Pequena estampa, com uma figura estampada, com que se preenchem álbuns, formando uma história ou uma coleção.}{fi.gu.ri.nha}{0}
\verb{figurinista}{}{}{}{}{s.2g.}{Profissional que desenha figurinos; modelista.}{fi.gu.ri.nis.ta}{0}
\verb{figurino}{}{}{}{}{s.m.}{Figura que representa o traje da moda.}{fi.gu.ri.no}{0}
\verb{figurino}{}{}{}{}{}{Revista de modas.}{fi.gu.ri.no}{0}
\verb{fijiano}{}{}{}{}{adj.}{Relativo às Ilhas Fiji, no sudoeste do Pacífico.}{fi.ji.a.no}{0}
\verb{fijiano}{}{}{}{}{s.m.}{Indivíduo natural ou habitante dessas ilhas.}{fi.ji.a.no}{0}
\verb{fila}{}{}{}{}{s.f.}{Grupo de pessoas ou coisas dispostas em linha, umas após as outras, pela ordem de chegada. (\textit{A fila para o recadastramento era imensa.})}{fi.la}{0}
\verb{fila}{}{}{}{}{s.f.}{Ato de filar, de agarrar à força.}{fi.la}{0}
\verb{fila}{}{Zool.}{}{}{s.m.}{Raça de cão grande e forte usado para guardar casas; cão de fila.}{fi.la}{0}
\verb{filaça}{}{}{}{}{s.f.}{Filamento grosseiro de matéria têxtil.}{fi.la.ça}{0}
\verb{filamento}{}{}{}{}{s.m.}{Fio tênue, de diâmetro muito pequeno.}{fi.la.men.to}{0}
\verb{filamentoso}{ô}{}{"-osos ⟨ó⟩}{"-osa ⟨ó⟩}{adj.}{Constituído de filamentos.}{fi.la.men.to.so}{0}
\verb{filante}{}{}{}{}{adj.2g.}{Que costuma filar refeições, cigarros etc. de outrem.}{fi.lan.te}{0}
\verb{filantropia}{}{}{}{}{s.f.}{Sentimento que leva os seres humanos a ajudar os outros; humanitarismo, altruísmo, caridade.}{fi.lan.tro.pi.a}{0}
\verb{filantrópico}{}{}{}{}{adj.}{Relativo a filantropia ou inspirado nela.}{fi.lan.tró.pi.co}{0}
\verb{filantropo}{ô}{}{}{}{adj. e s.m.  }{Que gosta de ajudar os outros; humanitário, caridoso, altruísta.}{fi.lan.tro.po}{0}
\verb{filão}{}{}{"-ões}{}{s.m.}{Região de uma mina onde se encontra o minério; veio.}{fi.lão}{0}
\verb{filão}{}{Fig.}{"-ões}{}{}{Situação ou acontecimento que se explora proveitosamente.}{fi.lão}{0}
\verb{filar}{}{}{}{}{}{Agarrar à força; capturar, abocanhar.}{fi.lar}{\verboinum{1}}
\verb{filar}{}{}{}{}{v.t.}{Conseguir algo de graça de outrem.}{fi.lar}{0}
\verb{filária}{}{Zool.}{}{}{s.f.}{Verme longo e fino responsável por causar a filariose, doença conhecida também por elefantíase.}{fi.lá.ria}{0}
\verb{filariose}{ó}{Med.}{}{}{s.f.}{Doença inflamatória que obstrui os vasos linfáticos, causada por filária e transmitida por mosquitos; elefantíase.}{fi.la.ri.o.se}{0}
\verb{filarmônica}{}{}{}{}{s.f.}{Sociedade musical.}{fi.lar.mô.ni.ca}{0}
\verb{filarmônica}{}{}{}{}{}{Orquestra sinfônica; banda de música.}{fi.lar.mô.ni.ca}{0}
\verb{filatelia}{}{}{}{}{s.f.}{Estudo dos selos do correio usados em diversos países.}{fi.la.te.li.a}{0}
\verb{filatelia}{}{}{}{}{}{Prática de colecionar selos do correio.}{fi.la.te.li.a}{0}
\verb{filatélico}{}{}{}{}{adj.}{Relativo a filatelia.}{fi.la.té.li.co}{0}
\verb{filatelista}{}{}{}{}{s.2g.}{Indivíduo que coleciona selos do correio.}{fi.la.te.lis.ta}{0}
\verb{filáucia}{}{}{}{}{s.f.}{Amor"-próprio, egoísmo.}{fi.láu.cia}{0}
\verb{filáucia}{}{}{}{}{}{Gesto vaidoso; presunção.}{fi.láu.cia}{0}
\verb{filé}{}{}{}{}{s.m.}{Carne de boi ou outro animal tirada de perto das costelas.}{fi.lé}{0}
\verb{filé}{}{}{}{}{}{Fatia fina de peixe, sem espinhas nem ossos.}{fi.lé}{0}
\verb{fileira}{ê}{}{}{}{s.f.}{Sequência de coisas, animais ou pessoas em linha reta; fila, ala, série.}{fi.lei.ra}{0}
\verb{filete}{ê}{}{}{}{s.m.}{Pequeno fio; fiozinho. (\textit{Da face do rapaz escorria um filete de sangue por causa da pancada na janela.})}{fi.le.te}{0}
\verb{filha}{}{}{}{}{s.f.}{Pessoa do sexo feminino em relação a seus pais.}{fi.lha}{0}
\verb{filharada}{}{}{}{}{s.f.}{Grande quantidade de filhos.}{fi.lha.ra.da}{0}
\verb{filho}{}{}{}{}{s.m.}{Pessoa do sexo masculino em relação a seus pais.}{fi.lho}{0}
\verb{filho}{}{}{}{}{}{Qualquer pessoa em relação à sua família ou ao lugar onde nasceu; descendente. (\textit{Itabira tem Carlos Drummond de Andrade como seu mais ilustre filho.})}{fi.lho}{0}
\verb{filhó}{}{Cul.}{}{}{s.m.}{Massa de farinha e ovos, frita em azeite e passada em calda de açúcar.}{fi.lhó}{0}
\verb{filho"-família}{}{}{filhos"-famílias \textit{ou} filhos"-família}{}{s.m.}{Filho menor de idade, sujeito ao pátrio poder.}{fi.lho"-fa.mí.lia}{0}
\verb{filhós}{}{}{}{}{s.m.pl.}{Filhó.}{fi.lhós}{0}
\verb{filhote}{ó}{}{}{}{s.m.}{Cria de animal.}{fi.lho.te}{0}
\verb{filhote}{ó}{Pop.}{}{}{}{Filho pequeno; filhinho.}{fi.lho.te}{0}
\verb{filhotismo}{}{}{}{}{s.m.}{Proteção ou preferência que se dá ao favorito; favoritismo, nepotismo.}{fi.lho.tis.mo}{0}
\verb{filiação}{}{}{"-ões}{}{s.f.}{Ato ou efeito de filiar, perfilhar.}{fi.li.a.ção}{0}
\verb{filiação}{}{}{"-ões}{}{}{Relação de parentesco entre os pais e seus filhos.}{fi.li.a.ção}{0}
\verb{filiação}{}{}{"-ões}{}{}{Admissão em uma comunidade, associação etc.}{fi.li.a.ção}{0}
\verb{filial}{}{}{"-ais}{}{adj.2g.}{Relativo a filho ou filha.}{fi.li.al}{0}
\verb{filial}{}{}{"-ais}{}{s.f.}{Estabelecimento sucursal ou dependente de outro.}{fi.li.al}{0}
\verb{filiar}{}{}{}{}{}{Admitir numa comunidade ou associação.}{fi.li.ar}{\verboinum{1}}
\verb{filiar}{}{}{}{}{v.t.}{Adotar como filho; perfilhar.}{fi.li.ar}{0}
\verb{filiforme}{ó}{}{}{}{adj.2g.}{Que tem forma de fio.}{fi.li.for.me}{0}
\verb{filigrana}{}{}{}{}{s.f.}{Obra de ourivesaria, em forma de renda, entrelaçada com fios de ouro e prata.}{fi.li.gra.na}{0}
\verb{filigrana}{}{}{}{}{}{Letras, desenhos ou linhas em papel de escrever, visíveis apenas por transparência.}{fi.li.gra.na}{0}
\verb{filipino}{}{}{}{}{adj.}{Relativo às ilhas Filipinas.}{fi.li.pi.no}{0}
\verb{filipino}{}{}{}{}{s.m.}{Indivíduo natural ou habitante dessas ilhas.}{fi.li.pi.no}{0}
\verb{filisteu}{}{}{}{}{s.m.}{Habitante da antiga Filisteia, hoje Palestina.}{fi.lis.teu}{0}
\verb{filmadora}{ô}{}{}{}{s.f.}{Máquina utilizada em filmagens; câmera cinematográfica.}{fil.ma.do.ra}{0}
\verb{filmagem}{}{}{"-ens}{}{s.f.}{Ato ou efeito de filmar; gravação, registro.}{fil.ma.gem}{0}
\verb{filmar}{}{}{}{}{v.t.}{Registrar um fato ou evento em filme cinematográfico.}{fil.mar}{\verboinum{1}}
\verb{filme}{}{}{}{}{s.m.}{Sequência de cenas ou imagens projetadas numa tela; fita, película.}{fil.me}{0}
\verb{filme}{}{}{}{}{}{Fita de celuloide utilizada para gravar imagens; película.}{fil.me}{0}
\verb{filme}{}{}{}{}{}{Fina folha de plástico transparente, com substância adesiva, usada para recobrir alimentos.}{fil.me}{0}
\verb{filmoteca}{é}{}{}{}{s.f.}{Coleção de filmes.}{fil.mo.te.ca}{0}
\verb{filmoteca}{é}{}{}{}{}{Seção de microfilmes de documentos em uma biblioteca.}{fil.mo.te.ca}{0}
\verb{filo}{}{Biol.}{}{}{s.m.}{Cada uma das grandes divisões dos reinos vegetal e animal, imediatamente superior a classe e inferior a sub"-reino.}{fi.lo}{0}
\verb{filó}{}{}{}{}{s.m.}{Tecido transparente de malha fina.}{fi.ló}{0}
\verb{filogenia}{}{Biol.}{}{}{s.f.}{História evolutiva de uma espécie orgânica ou qualquer outro grupo taxonômico.}{fi.lo.ge.ni.a}{0}
\verb{filologia}{}{}{}{}{s.f.}{Ciência que estuda os documentos escritos visando à fixação dos textos e à compreensão dos fatos culturais a que a língua serve de veículo.}{fi.lo.lo.gi.a}{0}
\verb{filológico}{}{}{}{}{adj.}{Relativo a filologia.}{fi.lo.ló.gi.co}{0}
\verb{filólogo}{}{}{}{}{s.m.}{Especialista versado em filologia.}{fi.ló.lo.go}{0}
\verb{filoneísmo}{}{}{}{}{s.m.}{Gosto excessivo pelas coisas novas.}{fi.lo.ne.ís.mo}{0}
\verb{filosofal}{}{}{"-ais}{}{adj.2g.}{Relativo a filósofo.}{fi.lo.so.fal}{0}
\verb{filosofal}{}{}{"-ais}{}{}{Diz"-se da pedra, segundo os alquimistas, que devia transformar qualquer metal em ouro.}{fi.lo.so.fal}{0}
\verb{filosofar}{}{}{}{}{v.i.}{Raciocinar sobre um assunto de interesse filosófico.}{fi.lo.so.far}{0}
\verb{filosofar}{}{}{}{}{}{Discorrer sobre qualquer assunto; meditar, refletir.}{fi.lo.so.far}{\verboinum{1}}
\verb{filosofia}{}{}{}{}{s.f.}{Ciência que busca o conhecimento, especialmente da origem e do sentido da existência. }{fi.lo.so.fi.a}{0}
\verb{filosofia}{}{}{}{}{}{Maneira própria de pensar sobre as coisas. (\textit{Sua filosofia de vida era o respeito ao próximo acima de tudo.})}{fi.lo.so.fi.a}{0}
\verb{filosófico}{}{}{}{}{adj.}{Relativo a filósofo ou a filosofia.}{fi.lo.só.fi.co}{0}
\verb{filósofo}{}{}{}{}{s.m.}{Indivíduo que se dedica à filosofia.}{fi.ló.so.fo}{0}
\verb{filósofo}{}{Fig.}{}{}{}{Indivíduo idealista, reflexivo.}{fi.ló.so.fo}{0}
\verb{filtração}{}{}{"-ões}{}{s.f.}{Ato ou efeito de filtrar.}{fil.tra.ção}{0}
\verb{filtrar}{}{}{}{}{v.t.}{Fazer passar por filtro, retendo as partículas sólidas; coar.}{fil.trar}{0}
\verb{filtrar}{}{}{}{}{}{Impedir a passagem; reter, obstruir. (\textit{Esse produto filtra os raios solares nocivos à saúde.})}{fil.trar}{0}
\verb{filtrar}{}{}{}{}{}{Submeter a controle; selecionar, separar.}{fil.trar}{0}
\verb{filtrar}{}{}{}{}{}{Deixar passar aos poucos; infiltrar, instilar.}{fil.trar}{\verboinum{1}}
\verb{filtro}{}{}{}{}{s.m.}{Material usado para reter ou impedir a passagem de algo.}{fil.tro}{0}
\verb{fim}{}{}{fins}{}{s.m.}{A última parte de algo; final, termo, fecho.}{fim}{0}
\verb{fim}{}{}{fins}{}{}{Parte em que algo termina; extremidade, limite.}{fim}{0}
\verb{fim}{}{}{fins}{}{}{Término, conclusão, remate.}{fim}{0}
\verb{fim}{}{}{fins}{}{}{Objetivo, finalidade, meta.}{fim}{0}
\verb{fímbria}{}{}{}{}{s.f.}{Orla inferior da veste; franja.}{fím.bria}{0}
\verb{fimose}{ó}{Med.}{}{}{s.f.}{Estreitamento da pele que cobre o pênis, dificultando a saída da urina e a limpeza da glande.}{fi.mo.se}{0}
\verb{finado}{}{}{}{}{s.m.}{Indivíduo que faleceu; defundo.}{fi.na.do}{0}
\verb{finado}{}{}{}{}{adj.}{Que faleceu; falecido, morto.}{fi.na.do}{0}
\verb{finado}{}{}{}{}{}{}{fi.na.do}{0}
\verb{final}{}{}{"-ais}{}{adj.2g.}{Que chega ao fim; último, derradeiro.}{fi.nal}{0}
\verb{final}{}{Gram.}{"-ais}{}{}{Diz"-se da conjunção e da oração que exprimem finalidade, intenção.}{fi.nal}{0}
\verb{final}{}{Esport.}{"-ais}{}{s.f.}{Prova ou partida decisiva de um campeonato.}{fi.nal}{0}
\verb{finalidade}{}{}{}{}{s.f.}{Fim determinado; meta, objetivo.}{fi.na.li.da.de}{0}
\verb{finalidade}{}{}{}{}{}{Serventia, utilidade.}{fi.na.li.da.de}{0}
\verb{finalíssima}{}{Esport.}{}{}{s.f.}{Partida decisiva de um campeonato; final.}{fi.na.lís.si.ma}{0}
\verb{finalista}{}{Esport.}{}{}{s.2g.}{Esportista ou equipe que disputa prova ou partida final.}{fi.na.lis.ta}{0}
\verb{finalizar}{}{}{}{}{v.t.}{Pôr fim; concluir, terminar, acabar.}{fi.na.li.zar}{\verboinum{1}}
\verb{finanças}{}{}{}{}{s.f.pl.}{Dinheiro que se possui; recursos.}{fi.nan.ças}{0}
\verb{finanças}{}{}{}{}{}{Ciência ou profissão que trata do manejo e da administração do dinheiro.}{fi.nan.ças}{0}
\verb{financeira}{ê}{}{}{}{s.f.}{Sociedade de crédito e financiamento.}{fi.nan.cei.ra}{0}
\verb{financeiro}{ê}{}{}{}{adj.}{Relativo a finanças.}{fi.nan.cei.ro}{0}
\verb{financiamento}{}{}{}{}{s.m.}{Ato ou efeito de financiar; empréstimo, custeio.}{fi.nan.ci.a.men.to}{0}
\verb{financiar}{}{}{}{}{v.t.}{Emprestar dinheiro; custear.}{fi.nan.ci.ar}{\verboinum{1}}
\verb{financista}{}{}{}{}{s.2g.}{Especialista versado em finanças.}{fi.nan.cis.ta}{0}
\verb{finar}{}{}{}{}{v.pron.}{Definhar, falecer, morrer.}{fi.nar}{\verboinum{1}}
\verb{finar}{}{}{}{}{v.i.}{Acabar, findar, terminar.}{fi.nar}{0}
\verb{finca"-pé}{}{}{}{}{s.m.}{Teimosia, persistência, insistência.}{fin.ca"-pé}{0}
\verb{fincar}{}{}{}{}{v.t.}{Fazer chegar até o fim de uma superfície; cravar, enterrar.}{fin.car}{\verboinum{2}}
\verb{findar}{}{}{}{}{v.t.}{Pôr fim; terminar, acabar, concluir.}{fin.dar}{\verboinum{1}}
\verb{findo}{}{}{}{}{adj.}{Chegado ao fim; concluído, terminado.}{fin.do}{0}
\verb{finês}{}{}{}{}{adj. e s.m.  }{Finlandês.}{fi.nês}{0}
\verb{fineza}{ê}{}{}{}{s.f.}{Qualidade de fino; delicadeza, amabilidade.}{fi.ne.za}{0}
\verb{fineza}{ê}{}{}{}{}{Favor, obséquio, mercê.}{fi.ne.za}{0}
\verb{fingido}{}{}{}{}{adj.}{Que finge; dissimulado, falso, hipócrita.}{fin.gi.do}{0}
\verb{fingidor}{ô}{}{}{}{adj.}{Que finge; enganador.}{fin.gi.dor}{0}
\verb{fingimento}{}{}{}{}{s.m.}{Ato ou efeito de fingir; dissimulação, falsidade, hipocrisia.}{fin.gi.men.to}{0}
\verb{fingir}{}{}{}{}{v.t.}{Ocultar intenção, sentimento; dissimular, aparentar.}{fin.gir}{0}
\verb{fingir}{}{}{}{}{}{Supor, fantasiar, imaginar.}{fin.gir}{\verboinum{22}}
\verb{finidade}{}{}{}{}{s.f.}{Propriedade do que é finito.}{fi.ni.da.de}{0}
\verb{finito}{}{}{}{}{adj.}{Que tem fim; limitado.}{fi.ni.to}{0}
\verb{finito}{}{Mat.}{}{}{}{Diz"-se do número cujo valor pode ser determinado com precisão.}{fi.ni.to}{0}
\verb{finlandês}{}{}{}{}{adj.}{Relativo à Finlândia; finês.}{fin.lan.dês}{0}
\verb{finlandês}{}{}{}{}{s.m.}{Indivíduo natural ou habitante desse país.}{fin.lan.dês}{0}
\verb{fino}{}{}{}{}{adj.}{De pouca espessura; delgado.}{fi.no}{0}
\verb{fino}{}{}{}{}{}{Aguçado, apurado.}{fi.no}{0}
\verb{fino}{}{}{}{}{}{De qualidade superior; excelente, precioso.}{fi.no}{0}
\verb{fino}{}{}{}{}{}{De boas maneiras; educado, delicado.}{fi.no}{0}
\verb{finório}{}{}{}{}{adj.}{Espertalhão; sagaz, astuto.}{fi.nó.rio}{0}
\verb{finta}{}{Esport.}{}{}{s.f.}{Jogada ou golpe para desnortear o adversário; drible.}{fin.ta}{0}
\verb{finta}{}{}{}{}{}{Calote, logro.}{fin.ta}{0}
\verb{fintar}{}{}{}{}{v.t.}{Enganar o adversário com movimento de corpo; driblar.}{fin.tar}{0}
\verb{fintar}{}{}{}{}{}{Passar calote; lograr.}{fin.tar}{\verboinum{1}}
\verb{finura}{}{}{}{}{s.f.}{Qualidade de fino; pouca espessura.}{fi.nu.ra}{0}
\verb{finura}{}{Fig.}{}{}{}{Argucia, perspicácia, sutileza.}{fi.nu.ra}{0}
\verb{fio}{}{}{}{}{s.m.}{Fibra longa e torcida extraída de plantas têxteis e que serve para tecer; linha.}{fi.o}{0}
\verb{fio}{}{}{}{}{}{Peça de metal flexível, alongada e de diâmetro muito pequeno.}{fi.o}{0}
\verb{fio}{}{}{}{}{}{Corrente tênue de um líquido.}{fi.o}{0}
\verb{fio}{}{}{}{}{}{Parte cortante de um instrumento; gume, corte. (\textit{Esta faca está sem fio.})}{fi.o}{0}
\verb{fio"-dental}{}{}{fios"-dentais}{}{s.m.}{Fio de \textit{nylon} usado na higiene bucal.}{fi.o"-den.tal}{0}
\verb{fiorde}{ó}{Geol.}{}{}{s.m.}{Golfo profundo, estreito e alongado, escavado pela ação das geleiras.}{fi.or.de}{0}
\verb{firma}{}{}{}{}{s.f.}{Assinatura de uma pessoa como garantia de um documento.}{fir.ma}{0}
\verb{firma}{}{}{}{}{}{Estabelecimento comercial ou industrial.}{fir.ma}{0}
\verb{firmamento}{}{}{}{}{s.m.}{Abóbada celeste; céu.}{fir.ma.men.to}{0}
\verb{firmamento}{}{}{}{}{}{Fundamento, base, alicerce.}{fir.ma.men.to}{0}
\verb{firmar}{}{}{}{}{v.t.}{Tornar firme, seguro; fixar.}{fir.mar}{0}
\verb{firmar}{}{}{}{}{}{Assinar um documento como garantia.}{fir.mar}{0}
\verb{firmar}{}{}{}{}{}{Fazer um acordo; pactuar.}{fir.mar}{\verboinum{1}}
\verb{firme}{}{}{}{}{adj.2g.}{Que está fixo, seguro, estável.}{fir.me}{0}
\verb{firme}{}{}{}{}{}{Resoluto, decidido, inalterável.}{fir.me}{0}
\verb{firmeza}{ê}{}{}{}{s.f.}{Qualidade do que é firme; solidez, estabilidade, constância.}{fir.me.za}{0}
\verb{firula}{}{Esport.}{}{}{s.f.}{Demonstração de domínio da bola no futebol; virtuosismo.}{fi.ru.la}{0}
\verb{firula}{}{Pop.}{}{}{}{Floreio, circunlóquio.}{fi.ru.la}{0}
\verb{fiscal}{}{}{"-ais}{}{adj.2g.}{Relativo a fisco.}{fis.cal}{0}
\verb{fiscal}{}{}{"-ais}{}{s.m.}{Encarregado da fiscalização de certos atos para que não ocorram irregularidades.}{fis.cal}{0}
\verb{fiscal}{}{}{"-ais}{}{}{Empregado aduaneiro; empregado do fisco.}{fis.cal}{0}
\verb{fiscalização}{}{}{"-ões}{}{s.f.}{Ato ou efeito de fiscalizar; verificação, exame.}{fis.ca.li.za.ção}{0}
\verb{fiscalizar}{}{}{}{}{v.t.}{Submeter à vigilância; observar atentamente; examinar.}{fis.ca.li.zar}{\verboinum{1}}
\verb{fisco}{}{}{}{}{s.m.}{Ramo da administração pública encarregado de calcular e cobrar os impostos.}{fis.co}{0}
\verb{fisga}{}{}{}{}{s.f.}{Arpão de pesca; anzol.}{fis.ga}{0}
\verb{fisgada}{}{}{}{}{s.f.}{Ato ou efeito de fisgar.}{fis.ga.da}{0}
\verb{fisgada}{}{}{}{}{}{Dor violenta e rápida; pontada. (\textit{O maratonista sentiu uma fisgada na perna enquanto corria.})}{fis.ga.da}{0}
\verb{fisgar}{}{}{}{}{v.t.}{Pegar um peixe com anzol, arpão ou outro apetrecho.}{fis.gar}{0}
\verb{fisgar}{}{}{}{}{}{Apanhar ou perceber com rapidez.}{fis.gar}{\verboinum{5}}
\verb{física}{}{}{}{}{s.f.}{Ciência que estuda as propriedades da matéria e as leis que tendem a modificar"-lhe o estado e o movimento sem alterar"-lhe a natureza.}{fí.si.ca}{0}
\verb{físico}{}{}{}{}{adj.}{Relativo à Física.}{fí.si.co}{0}
\verb{físico}{}{}{}{}{}{Que pertence à matéria; corporal, material.}{fí.si.co}{0}
\verb{físico}{}{}{}{}{s.m.}{Cientista especializado em Física.}{fí.si.co}{0}
\verb{físico"-química}{}{}{}{}{s.f.}{Ciência que estuda a Física e a Química em seus domínios comuns.}{fí.si.co"-quí.mi.ca}{0}
\verb{físico"-químico}{}{}{físico"-químicos}{}{adj.}{Relativo à físico"-química ou ao cientista versado nessa ciência.}{fí.si.co"-quí.mi.co}{0}
\verb{fisiografia}{}{Geogr.}{}{}{s.f.}{Parte da Geografia que descreve a natureza e os fenômenos naturais; geografia física.}{fi.si.o.gra.fi.a}{0}
\verb{fisiologia}{}{Biol.}{}{}{s.f.}{Parte da Biologia que estuda as funções dos órgãos dos seres vivos, animais e vegetais.}{fi.si.o.lo.gi.a}{0}
\verb{fisiológico}{}{}{}{}{adj.}{Relativo a fisiologia.}{fi.si.o.ló.gi.co}{0}
\verb{fisiologismo}{}{}{}{}{s.m.}{Atitude de certos representantes e servidores públicos que se preocupam com a satisfação de interesses ou vantagens pessoais ou partidários em detrimento do bem comum.}{fi.si.o.lo.gis.mo}{0}
\verb{fisionomia}{}{}{}{}{s.f.}{Conjunto dos traços do rosto; feições, semblante.}{fi.si.o.no.mi.a}{0}
\verb{fisionômico}{}{}{}{}{adj.}{Relativo a fisionomia.}{fi.si.o.nô.mi.co}{0}
\verb{fisionomista}{}{}{}{}{s.2g.}{Indivíduo que tem boa memória para gravar fisionomias.}{fi.si.o.no.mis.ta}{0}
\verb{fisioterapeuta}{}{}{}{}{s.2g.}{Profissional que se dedica ao estudo e à prática da fisioterapia.}{fi.si.o.te.ra.peu.ta}{0}
\verb{fisioterapia}{}{}{}{}{s.f.}{Especialidade que emprega agentes físicos como água, luz, calor etc., massagens e exercícios no tratamento de doenças.}{fi.si.o.te.ra.pi.a}{0}
\verb{fisioterápico}{}{}{}{}{adj.}{Relativo a fisioterapia.}{fi.si.o.te.rá.pi.co}{0}
\verb{fissão}{}{}{"-ões}{}{s.f.}{Ato de fender; racha, quebra.}{fis.são}{0}
\verb{fissão}{}{Fís.}{"-ões}{}{}{Ruptura de um núcleo atômico pelo bombardeio com nêutrons, seguida da liberação de grande quantidade de energia.}{fis.são}{0}
\verb{físsil}{}{}{"-eis}{}{adj.2g.}{Que se pode dividir, fender.}{fís.sil}{0}
\verb{fissura}{}{}{}{}{s.f.}{Pequena abertura; rachadura, fenda, cissura.}{fis.su.ra}{0}
\verb{fissura}{}{Fig.}{}{}{}{Forte inclinação; loucura, paixão.}{fis.su.ra}{0}
\verb{fissurado}{}{}{}{}{adj.}{Que se fendeu; rachado, partido.}{fis.su.ra.do}{0}
\verb{fissurado}{}{Fig.}{}{}{}{Louco de paixão; gamado.}{fis.su.ra.do}{0}
\verb{fissurar}{}{}{}{}{v.t.}{Produzir fissuras; rachar, partir.}{fis.su.rar}{0}
\verb{fissurar}{}{Fig.}{}{}{}{Apaixonar"-se; gamar.}{fis.su.rar}{\verboinum{1}}
\verb{fístula}{}{Med.}{}{}{s.f.}{Canal estreito e profundo por onde fluem secreções diversas.}{fís.tu.la}{0}
\verb{fita}{}{}{}{}{s.f.}{Banda estreita de tecido, de pouco comprimento.}{fi.ta}{0}
\verb{fita}{}{}{}{}{}{Conjunto de imagens gravadas em uma película de celuloide; filme.}{fi.ta}{0}
\verb{fita}{}{}{}{}{}{Atitude vistosa para impressionar; manha, fingimento.}{fi.ta}{0}
\verb{fitar}{}{}{}{}{v.t.}{Fixar a vista em uma pessoa, um animal ou uma coisa; olhar.}{fi.tar}{\verboinum{1}}
\verb{fiteiro}{ê}{}{}{}{adj.}{Que faz coisas para impressionar; fingido, manhoso.}{fi.tei.ro}{0}
\verb{fitilho}{}{}{}{}{s.m.}{Fita muito estreita, de plástico ou tecido, usada como enfeite; nastro. }{fi.ti.lho}{0}
\verb{fito}{}{}{}{}{s.m.}{Objeto de desejo; intenção, objetivo, alvo.}{fi.to}{0}
\verb{fito}{}{}{}{}{adj.}{Que se fitou; cravado, fixo.}{fi.to}{0}
\verb{fitogenia}{}{Bot.}{}{}{s.f.}{Ramo da botânica que estuda a origem, a evolução e o desenvolvimento das plantas.}{fi.to.ge.ni.a}{0}
\verb{fitogeografia}{}{Bot.}{}{}{s.f.}{Estudo das relações entre os vegetais e o meio ambiente.}{fi.to.ge.o.gra.fi.a}{0}
\verb{fitografia}{}{Bot.}{}{}{s.f.}{Descrição dos diversos tipos de vegetais em relação a sua classificação.}{fi.to.gra.fi.a}{0}
\verb{fitologia}{}{Biol.}{}{}{s.f.}{Ramo da Biologia que tem por objeto o reino vegetal e que se divide em grandes áreas de estudo; botânica.}{fi.to.lo.gi.a}{0}
\verb{fitoplâncton}{}{Biol.}{}{}{s.m.}{Conjunto do plâncton vegetal.}{fi.to.plânc.ton}{0}
\verb{fitoterapia}{}{Med.}{}{}{s.f.}{Tratamento ou prevenção de doenças por meio de plantas.}{fi.to.te.ra.pi.a}{0}
\verb{fiúza}{}{}{}{}{s.f.}{Segurança, confiança, fé.}{fi.ú.za}{0}
\verb{fivela}{é}{}{}{}{s.f.}{Peça metálica com um pino usada para prender cintos, calçados, arreios etc.}{fi.ve.la}{0}
\verb{fivela}{é}{}{}{}{}{Peça de metal, plástico ou madeira usada para prender os cabelos.}{fi.ve.la}{0}
\verb{fixação}{cs}{}{"-ões}{}{s.f.}{Ato ou efeito de fixar, firmar.}{fi.xa.ção}{0}
\verb{fixação}{cs}{}{"-ões}{}{}{Ideia fixa; apego, obsessão. (\textit{A moça tinha verdadeira fixação por sapatos.})}{fi.xa.ção}{0}
\verb{fixador}{cs\ldots{}ô}{}{}{}{adj.}{Que fixa, firma.}{fi.xa.dor}{0}
\verb{fixador}{cs\ldots{}ô}{}{}{}{s.m.}{Loção para fixar o penteado.}{fi.xa.dor}{0}
\verb{fixador}{cs\ldots{}ô}{}{}{}{}{Substância que se adiciona ao perfume para que ele não se dissipe.}{fi.xa.dor}{0}
\verb{fixar}{cs}{}{}{}{v.t.}{Tornar firme; prender, pregar.}{fi.xar}{0}
\verb{fixar}{cs}{}{}{}{}{Estabelecer residência para outrem ou para si mesmo; assentar. (\textit{O governo fixou os sem"-teto em uma área desapropriada.})}{fi.xar}{0}
\verb{fixar}{cs}{}{}{}{}{Decidir, determinar, designar.}{fi.xar}{0}
\verb{fixar}{cs}{}{}{}{}{Ficar concentrado; deter a atenção. }{fi.xar}{\verboinum{1}}
\verb{fixo}{cs}{}{}{}{adj.}{Que não se move; preso, firme.}{fi.xo}{0}
\verb{fixo}{cs}{}{}{}{}{Que não se altera; estável, determinado.}{fi.xo}{0}
\verb{flã}{}{Cul.}{}{}{s.m.}{Flan.}{flã}{0}
\verb{flacidez}{ê}{}{}{}{s.f.}{Qualidade ou estado de flácido; frouxidão, relaxamento.}{fla.ci.dez}{0}
\verb{flácido}{}{}{}{}{adj.}{Que perdeu a firmeza; sem elasticidade; frouxo, mole.}{flá.ci.do}{0}
\verb{flagelação}{}{}{"-ões}{}{s.f.}{Ato ou efeito de flagelar; sofrimento, suplício, tormento.}{fla.ge.la.ção}{0}
\verb{flagelado}{}{}{}{}{adj.}{Que foi torturado, atormentado, açoitado.}{fla.ge.la.do}{0}
\verb{flagelado}{}{}{}{}{}{Que passa necessidade por causa de calamidades como secas, enchentes, terremotos etc. (\textit{A campanha daquela organização visa arrecadar fundos para os flagelados da seca.})}{fla.ge.la.do}{0}
\verb{flagelar}{}{}{}{}{v.t.}{Bater com flagelo; açoitar, chicotear.}{fla.ge.lar}{0}
\verb{flagelar}{}{}{}{}{}{Causar muito sofrimento; afligir, atormentar.}{fla.ge.lar}{\verboinum{1}}
\verb{flagelo}{é}{}{}{}{s.m.}{Instrumento de tiras de couro usado para açoitar; chicote, azorrague, chibata.}{fla.ge.lo}{0}
\verb{flagelo}{é}{}{}{}{}{Calamidade pública; praga, epidemia.}{fla.ge.lo}{0}
\verb{flagelo}{é}{Biol.}{}{}{}{Filamento alongado usado na locomoção de vários organismos unicelulares. }{fla.ge.lo}{0}
\verb{flagra}{}{Pop.}{}{}{s.m.}{Flagrante.}{fla.gra}{0}
\verb{flagrância}{}{}{}{}{s.f.}{Condição do que é flagrante; momento em que se registra o flagrante.}{fla.grân.cia}{0}
\verb{flagrante}{}{}{}{}{adj.2g.}{Que se reconhece facilmente; manifesto, evidente, incontestável. (\textit{Na sua atuação, ele mostrou uma perícia flagrante a todos nós.})}{fla.gran.te}{0}
\verb{flagrante}{}{}{}{}{s.m.}{Instante, momento, ensejo. (\textit{A foto grava um flagrante da viagem.})}{fla.gran.te}{0}
\verb{flagrante}{}{}{}{}{}{Usado na expressão \textit{em flagrante}: no momento de praticar o ato. (\textit{O ladrão foi preso em flagrante pela polícia.})}{fla.gran.te}{0}
\verb{flagrar}{}{}{}{}{v.t.}{Surpreender alguém cometendo um delito.}{fla.grar}{\verboinum{1}}
\verb{flama}{}{}{}{}{s.f.}{Brilho avermelhado de gases que estão sendo queimados; chama, labareda.}{fla.ma}{0}
\verb{flamante}{}{}{}{}{adj.2g.}{Que arde em chama; flamejante.}{fla.man.te}{0}
\verb{flamante}{}{}{}{}{}{Brilhante, vistoso.}{fla.man.te}{0}
\verb{flamar}{}{}{}{}{v.t.}{Embeber um alimento em bebida alcoólica e atear"-lhe fogo em seguida para servi"-lo assim que as chamas se apagarem; flambar. }{fla.mar}{\verboinum{1}}
\verb{flambar}{}{}{}{}{v.t.}{Embeber um alimento em bebida alcoólica e atear"-lhe fogo em seguida para servi"-lo assim que as chamas se apagarem; flamar. }{flam.bar}{\verboinum{1}}
\verb{flamboaiã}{}{Bot.}{}{}{s.m.}{Forma aportuguesada de \textit{flamboyant}.}{flam.bo.ai.ã}{0}
\verb{flamboyant}{}{Bot.}{}{}{s.m.}{Árvore de grande porte, cultivada como ornamental por suas belas flores vermelhas como chamas.}{\textit{flamboyant}}{0}
\verb{flamejante}{}{}{}{}{adj.2g.}{Que lança flamas; chamejante, flamante, resplandecente.}{fla.me.jan.te}{0}
\verb{flamejar}{}{}{}{}{v.i.}{Lançar flamas; arder, chamejar.}{fla.me.jar}{\verboinum{1}}
\verb{flamenco}{}{Mús.}{}{}{s.m.}{Música, dança e canto populares da Andaluzia, na Espanha, normalmente acompanhados de guitarras e palmas.}{fla.men.co}{0}
\verb{flamengo}{}{}{}{}{adj.}{Relativo a Flandres, região da Bélgica e da França.}{fla.men.go}{0}
\verb{flamengo}{}{}{}{}{s.m.}{Indivíduo natural e habitante dessa região.}{fla.men.go}{0}
\verb{flamingo}{}{Zool.}{}{}{s.m.}{Ave de pernas muito longas e de plumagem rosa.}{fla.min.go}{0}
\verb{flâmula}{}{}{}{}{s.f.}{Bandeira triangular, estreita, com emblema ou divisa de clube, escola etc; bandeirola.}{flâ.mu.la}{0}
\verb{flâmula}{}{}{}{}{}{Bandeira, pendão, lábaro.}{flâ.mu.la}{0}
\verb{flan}{}{Cul.}{}{}{s.m.}{Pudim cremoso, à base de leite e ovos, servido com calda de caramelo; flã.  }{flan}{0}
\verb{flanar}{}{}{}{}{v.i.}{Andar sem pressa e sem rumo certo; vaguear, perambular.}{fla.nar}{\verboinum{1}}
\verb{flanco}{}{}{}{}{s.m.}{Cada parte, direita ou esquerda, de alguma coisa.}{flan.co}{0}
\verb{flanco}{}{Anat.}{}{}{}{Parte lateral do tórax do homem e dos animais; ilharga.}{flan.co}{0}
\verb{flandres}{}{}{}{}{s.m.}{Lâmina de ferro recoberta de estanho.}{flan.dres}{0}
\verb{flanela}{é}{}{}{}{s.f.}{Tecido macio de algodão ou lã. (\textit{Meu avô adora andar pela casa com seu pijama de flanela.})}{fla.ne.la}{0}
\verb{flanelinha}{}{}{}{}{s.m.}{Guardador de carros, encontrado nas ruas das grandes cidades.}{fla.ne.li.nha}{0}
\verb{flanquear}{}{}{}{}{v.t.}{Atacar pelos flancos; ladear.}{flan.que.ar}{\verboinum{4}}
\verb{flape}{}{}{}{}{s.m.}{Freio aerodinâmico situado na parte inferior da asa do avião.}{fla.pe}{0}
\verb{flash}{}{}{}{}{s.m.}{Clarão instantâneo produzido para fotografar em lugares de pouca luz.}{\textit{flash}}{0}
\verb{flash}{}{}{}{}{}{Dispositivo que produz esse clarão.}{\textit{flash}}{0}
\verb{flash}{}{}{}{}{}{Imagem ou cena muito breve.}{\textit{flash}}{0}
\verb{flash}{}{}{}{}{}{Em televisão ou rádio, informação ou notícia curta dada com prioridade, interrompendo a programação normal.}{\textit{flash}}{0}
\verb{flashback}{}{}{}{}{s.m.}{Cena de filme que mostra fato já ocorrido durante a fita.}{\textit{flashback}}{0}
\verb{flato}{}{}{}{}{s.m.}{Gás que se forma nos intestinos.}{fla.to}{0}
\verb{flato}{}{}{}{}{}{Saída desses gases; peido, pum.}{fla.to}{0}
\verb{flatulência}{}{}{}{}{s.f.}{Acumulação de gases no tubo digestivo.}{fla.tu.lên.cia}{0}
\verb{flatulento}{}{}{}{}{adj.}{Que produz flatulência.}{fla.tu.len.to}{0}
\verb{flatulento}{}{}{}{}{}{Que sofre de flatulência.}{fla.tu.len.to}{0}
\verb{flauta}{}{Mús.}{}{}{s.f.}{Instrumento musical de sopro, formado por um tubo oco com chaves e orifícios.}{flau.ta}{0}
\verb{flautear}{}{}{}{}{v.i.}{Tocar flauta.}{flau.te.ar}{0}
\verb{flautear}{}{}{}{}{}{Levar a vida sem trabalhar; vadiar.}{flau.te.ar}{\verboinum{4}}
\verb{flautim}{}{Mús.}{"-ins}{}{s.m.}{Instrumento musical de sopro semelhante à flauta, porém menor, e que produz som mais agudo.}{flau.tim}{0}
\verb{flautista}{}{Mús.}{}{}{s.2g.}{Músico que toca flauta.}{flau.tis.ta}{0}
\verb{flébil}{}{}{"-eis}{}{adj.}{Que chora; lacrimoso, choroso, plangente.}{flé.bil}{0}
\verb{flebite}{}{Med.}{}{}{s.f.}{Inflamação das paredes de uma veia.}{fle.bi.te}{0}
\verb{flecha}{é}{}{}{}{s.f.}{Haste comprida com uma ponta afiada e a outra contendo penas e que é atirada por meio de um arco; seta.}{fle.cha}{0}
\verb{flechada}{}{}{}{}{s.f.}{Disparo ou ferimento de flecha.}{fle.cha.da}{0}
\verb{flechar}{}{}{}{}{v.t.}{Atingir pessoa, animal ou coisa com flecha.}{fle.char}{\verboinum{1}}
\verb{flecheiro}{ê}{}{}{}{s.m.}{Indivíduo que atira flechas ou setas.}{fle.chei.ro}{0}
\verb{flectir}{}{}{}{}{v.t.}{Fazer flexão; dobrar, curvar, flexionar.}{flec.tir}{\verboinum{29}}
\verb{flegma}{}{}{}{}{}{Var. de \textit{fleuma}.}{fleg.ma}{0}
\verb{flegmão}{}{Med.}{}{}{s.m.}{Inflamação do tecido conjuntivo.}{fleg.mão}{0}
\verb{flegmático}{}{}{}{}{}{Var. de \textit{fleumático}.}{fleg.má.ti.co}{0}
\verb{fleimão}{}{}{}{}{}{Var. de \textit{flegmão}.}{flei.mão}{0}
\verb{fleróvio}{}{Quím.}{}{}{s.m.}{Elemento químico sintético, reproduzido apenas duas vezes em 1999, cujas propriedades químicas talvez sejam semelhantes às do chumbo. \elemento{114}{(289)}{Fl}.}{fle.ró.vio}{0}
\verb{flertar}{}{}{}{}{v.t.}{Namorar por divertimento, sem intenção séria.}{fler.tar}{\verboinum{1}}
\verb{flerte}{é/ ou /ê}{}{}{}{s.m.}{Namoro ligeiro, como passatempo.}{fler.te}{0}
\verb{fletir}{}{}{}{}{}{Var. de \textit{flectir}.}{fle.tir}{0}
\verb{fleuma}{}{}{}{}{s.m.}{Frieza de ânimo; impassibilidade, serenidade.}{fleu.ma}{0}
\verb{fleumático}{}{}{}{}{adj.}{Que tem ânimo frio; impassível, imperturbável.}{fleu.má.ti.co}{0}
\verb{flexão}{cs}{}{"-ões}{}{s.f.}{Ato ou efeito de flectir, dobrar, curvatura.}{fle.xão}{0}
\verb{flexão}{cs}{Gram.}{"-ões}{}{}{Variação da forma de uma palavra, compreendendo, no Português, as categorias de gênero e número para os nomes e de modo, tempo, número e pessoa para os verbos.}{fle.xão}{0}
\verb{flexibilidade}{cs}{}{}{}{s.f.}{Qualidade do que é flexível, dobrável; maleabilidade.}{fle.xi.bi.li.da.de}{0}
\verb{flexibilidade}{cs}{}{}{}{}{Docilidade, submissão.}{fle.xi.bi.li.da.de}{0}
\verb{flexibilizar}{cs}{}{}{}{v.t.}{Tornar flexível, maleável.}{fle.xi.bi.li.zar}{\verboinum{1}}
\verb{flexionar}{cs}{}{}{}{v.t.}{Dobrar, vergar, flectir.}{fle.xi.o.nar}{0}
\verb{flexionar}{cs}{Gram.}{}{}{}{Fazer a flexão de uma palavra.}{fle.xi.o.nar}{\verboinum{1}}
\verb{flexível}{cs}{}{"-eis}{}{adj.2g.}{Que se dobra com facilidade sem quebrar; maleável. (\textit{O material é bastante flexível.})}{fle.xí.vel}{0}
\verb{flexível}{cs}{}{"-eis}{}{}{Que pode ser levado a mudar de opinião; complacente, submisso. (\textit{Ele é muito flexível nas suas decisões.})}{fle.xí.vel}{0}
\verb{flexor}{csô}{}{}{}{adj.}{Que faz dobrar.}{fle.xor}{0}
\verb{flexor}{csô}{Anat.}{}{}{}{Músculo que se destina a flexionar diversas partes do corpo.}{fle.xor}{0}
\verb{flexuoso}{cs\ldots{}ô}{}{"-osos ⟨ó⟩}{"-osa ⟨ó⟩}{adj.}{Que não é reto; sinuoso, tortuoso.}{fle.xu.o.so}{0}
\verb{fliperama}{}{}{}{}{s.m.}{Máquina de jogo eletrônico acionado com movimentos bruscos dos dedos, principalmente os polegares. (\textit{Meu irmão adora ir ao shopping para jogar fliperama.})}{fli.pe.ra.ma}{0}
\verb{fliperama}{}{}{}{}{}{Estabelecimento onde há vários tipos dessa máquina.}{fli.pe.ra.ma}{0}
\verb{floco}{ó}{}{}{}{s.m.}{Partícula de neve que cai lentamente. (\textit{As crianças olhavam pela janela os primeiros flocos de neve daquele inverno.})}{flo.co}{0}
\verb{floco}{ó}{}{}{}{}{Conjunto de pequenos fios de lã, felpas e penugem que voam pelo ar, movidos pelo vento.}{flo.co}{0}
\verb{flóculo}{}{}{}{}{s.m.}{Floco muito fino e leve.}{fló.cu.lo}{0}
\verb{flor}{ô}{}{}{}{s.f.}{Órgão reprodutor das plantas, formado por um conjunto de folhas coloridas. (\textit{A flor da laranjeira é muito utilizada em buquês de noivas.})}{flor}{0}
\verb{flor}{ô}{Fig.}{}{}{}{Pessoa ou coisa bela, delicada.}{flor}{0}
\verb{flora}{ó}{}{}{}{s.f.}{Conjunto das espécimes vegetais de uma região. (\textit{A flora brasileira é riquíssima em espécimes, principalmente na região do cerrado.})}{flo.ra}{0}
\verb{floração}{}{}{"-ões}{}{s.f.}{Ato ou efeito de florir, florescer.}{flo.ra.ção}{0}
\verb{floração}{}{}{"-ões}{}{}{Condição de uma planta coberta de flores.}{flo.ra.ção}{0}
\verb{florada}{}{}{}{}{s.f.}{Desabrochamento das flores de uma ou várias plantas; floração.}{flo.ra.da}{0}
\verb{floral}{}{}{"-ais}{}{adj.2g.}{Relativo a flor.}{flo.ral}{0}
\verb{floral}{}{}{"-ais}{}{s.m.}{Substância extraída de flores silvestres, usada como tratamento alternativo na harmonização de problemas e desequilíbrios emocionais.}{flo.ral}{0}
\verb{florão}{}{}{"-ões}{}{s.m.}{Enfeite em forma de flor.}{flo.rão}{0}
\verb{florar}{}{}{}{}{v.i.}{Cobrir de flores; florir, florescer.}{flo.rar}{\verboinum{1}}
\verb{flor"-de"-lis}{ô}{Bot.}{flores"-de"-lis ⟨ô⟩}{}{s.f.}{Planta de flores grandes e vistosas.}{flor"-de"-lis}{0}
\verb{flor"-de"-lis}{ô}{}{flores"-de"-lis ⟨ô⟩}{}{}{Símbolo dos antigos reis da França, com o formato de um lírio.}{flor"-de"-lis}{0}
\verb{floreado}{}{}{}{}{adj.}{Coberto ou ornamentado com flores.}{flo.re.a.do}{0}
\verb{floreado}{}{}{}{}{}{Diz"-se do estilo rebuscado, exagerado.}{flo.re.a.do}{0}
\verb{florear}{}{}{}{}{v.t.}{Cobrir ou ornar com flores; enfeitar.}{flo.re.ar}{0}
\verb{florear}{}{Mús.}{}{}{v.i.}{Fazer variações adicionais numa execução musical.}{flo.re.ar}{\verboinum{4}}
\verb{floreio}{ê}{}{}{}{s.m.}{Ato ou efeito de florear, de ornar com flores.}{flo.rei.o}{0}
\verb{floreio}{ê}{Mús.}{}{}{}{Variação musical caprichosa e graciosa.}{flo.rei.o}{0}
\verb{floreira}{ê}{}{}{}{s.f.}{Recipiente em que se plantam flores; vaso.}{flo.rei.ra}{0}
\verb{florentino}{}{}{}{}{adj.}{Relativo a Florença, cidade da Itália.}{flo.ren.ti.no}{0}
\verb{florentino}{}{}{}{}{s.m.}{Indivíduo natural ou habitante dessa cidade.}{flo.ren.ti.no}{0}
\verb{flóreo}{}{}{}{}{adj.}{Relativo a flor; floral.}{fló.re.o}{0}
\verb{florescência}{}{}{}{}{s.f.}{Ato ou efeito de florescer; floração.}{flo.res.cên.cia}{0}
\verb{florescente}{}{}{}{}{adj.2g.}{Que floresce, viceja; viçoso, próspero.}{flo.res.cen.te}{0}
\verb{florescer}{ê}{}{}{}{v.i.}{Produzir flores; florir, florar.}{flo.res.cer}{0}
\verb{florescer}{ê}{Fig.}{}{}{}{Desenvolver"-se muito bem; prosperar, frutificar.}{flo.res.cer}{\verboinum{15}}
\verb{floresta}{é}{}{}{}{s.f.}{Grande extensão de terra formada de árvores de grande porte; mata espessa; selva.}{flo.res.ta}{0}
\verb{florestal}{}{}{"-ais}{}{adj.2g.}{Relativo a floresta.}{flo.res.tal}{0}
\verb{florete}{ê}{}{}{}{s.m.}{Espada fina e pontiaguda usada em esgrima.}{flo.re.te}{0}
\verb{florianopolitano}{}{}{}{}{adj.}{Relativo a Florianópolis, capital de Santa Catarina.}{flo.ri.a.no.po.li.ta.no}{0}
\verb{florianopolitano}{}{}{}{}{s.m.}{Indivíduo natural ou habitante dessa cidade.}{flo.ri.a.no.po.li.ta.no}{0}
\verb{floricultor}{ô}{}{}{}{s.m.}{Indivíduo que se dedica ao cultivo e à venda de flores.}{flo.ri.cul.tor}{0}
\verb{floricultura}{}{}{}{}{s.f.}{Cultivo de flores.}{flo.ri.cul.tu.ra}{0}
\verb{floricultura}{}{}{}{}{}{Loja onde se vendem flores e arranjos florais.}{flo.ri.cul.tu.ra}{0}
\verb{florido}{}{}{}{}{adj.}{Coberto de flores; florescente. (\textit{Mal chegou a primavera e o jardim já está todo florido.})}{flo.ri.do}{0}
\verb{flórido}{}{}{}{}{adj.}{Que possui a beleza das flores; esplendoroso, notável.}{fló.ri.do}{0}
\verb{florífero}{}{}{}{}{adj.}{Diz"-se do vegetal que produz flores.}{flo.rí.fe.ro}{0}
\verb{florilégio}{}{}{}{}{s.m.}{Coleção de flores.}{flo.ri.lé.gio}{0}
\verb{florilégio}{}{Liter.}{}{}{}{Coleção de trechos literários; antologia, compilação.}{flo.ri.lé.gio}{0}
\verb{florim}{}{}{"-ins}{}{s.m.}{Unidade monetária do Suriname. }{flo.rim}{0}
\verb{florim}{}{}{"-ins}{}{}{Antiga moeda utilizada na Holanda antes da adoção do euro.}{flo.rim}{0}
\verb{florir}{}{}{}{}{v.i.}{Cobrir de flores; florescer, florar.}{flo.rir}{0}
\verb{florir}{}{Fig.}{}{}{}{Desenvolver"-se muito bem; progredir.}{flo.rir}{\verboinum{34}\verboirregular{\emph{def.}}}
\verb{florista}{}{}{}{}{s.2g.}{Indivíduo que vende flores ou fabrica flores artificiais.}{flo.ris.ta}{0}
\verb{flotilha}{}{}{}{}{s.f.}{Pequena frota de barcos da marinha de guerra; esquadrilha.}{flo.ti.lha}{0}
\verb{fluência}{}{}{}{}{s.f.}{Qualidade de fluente; fluidez.}{flu.ên.cia}{0}
\verb{fluência}{}{}{}{}{}{Naturalidade, espontaneidade.}{flu.ên.cia}{0}
\verb{fluente}{ê}{}{}{}{adj.2g.}{Que flui; corrente, abundante.}{flu.en.te}{0}
\verb{fluente}{ê}{}{}{}{}{Natural, espontâneo, fácil. (\textit{Meu amigo é fluente em árabe e suahíli.})}{flu.en.te}{0}
\verb{fluidez}{ê}{}{}{}{s.f.}{Qualidade do que é fluido; fluência.}{flu.i.dez}{0}
\verb{fluidificar}{}{}{}{}{v.t.}{Reduzir a fluido; diluir.}{flu.i.di.fi.car}{\verboinum{2}}
\verb{fluido}{}{}{}{}{adj.}{Que flui fácil e rapidamente como um líquido.}{flu.i.do}{0}
\verb{fluido}{}{}{}{}{s.m.}{Substância que toma a forma do lugar em que se encontra.}{flu.i.do}{0}
\verb{fluir}{}{}{}{}{v.i.}{Correr em estado líquido; manar.}{flu.ir}{0}
\verb{fluir}{}{}{}{}{}{Percorrer distâncias com rapidez; correr.}{flu.ir}{\verboinum{26}}
\verb{fluminense}{}{}{}{}{adj.}{Referente a rio; fluvial.}{flu.mi.nen.se}{0}
\verb{fluminense}{}{}{}{}{}{Relativo a Rio de Janeiro.}{flu.mi.nen.se}{0}
\verb{fluminense}{}{}{}{}{s.m.}{Indivíduo natural ou habitante do estado do Rio de Janeiro.}{flu.mi.nen.se}{0}
\verb{flúor}{}{Quím.}{}{}{s.m.}{Elemento químico do grupo dos halogênios, gasoso, inflamável, corrosivo, usado na fabricação de plásticos, gases de refrigeração etc. \elemento{9}{18.998404}{F}.}{flú.or}{0}
\verb{fluorar}{}{}{}{}{v.t.}{Tratar com flúor.}{flu.o.rar}{0}
\verb{fluorar}{}{}{}{}{}{Adicionar flúor à água do abastecimento urbano.}{flu.o.rar}{\verboinum{1}}
\verb{fluorescência}{}{}{}{}{s.f.}{Luminosidade que certas substâncias apresentam quando recai sobre elas um foco de luz.}{flu.o.res.cên.cia}{0}
\verb{fluorescente}{}{}{}{}{adj.2g.}{Que apresenta a propriedade da fluorescência. (\textit{Há peixes fluorescentes nas regiões abissais do oceano.})}{flu.o.res.cen.te}{0}
\verb{fluoreto}{ê}{Quím.}{}{}{s.m.}{Designação geral dos sais ou ânions do ácido fluorídrico. }{flu.o.re.to}{0}
\verb{flutuação}{}{}{"-ões}{}{s.f.}{Ato ou efeito de flutuar, boiar; oscilação.}{flu.tu.a.ção}{0}
\verb{flutuador}{ô}{}{}{}{s.m.}{Cada uma das partes do hidroavião sobre as quais ele pousa ou flutua na água.}{flu.tu.a.dor}{0}
\verb{flutuante}{}{}{}{}{adj.2g.}{Que flutua, boia; ondulante, oscilante.}{flu.tu.an.te}{0}
\verb{flutuante}{}{Fig.}{}{}{}{Indeciso, hesitante, vacilante.}{flu.tu.an.te}{0}
\verb{flutuar}{}{}{}{}{v.i.}{Ficar à tona de um líquido; vagar sobre as ondas; boiar.}{flu.tu.ar}{0}
\verb{flutuar}{}{}{}{}{}{Agitar"-se ao vento; ondular.}{flu.tu.ar}{0}
\verb{flutuar}{}{}{}{}{}{Variar de valor. (\textit{O valor de câmbio do dólar costuma flutuar muito nessa época do ano.})}{flu.tu.ar}{\verboinum{1}}
\verb{fluvial}{}{}{"-ais}{}{adj.2g.}{Relativo a rio.}{flu.vi.al}{0}
\verb{fluvial}{}{}{"-ais}{}{}{Que vive nos rios.}{flu.vi.al}{0}
\verb{fluviômetro}{}{}{}{}{s.m.}{Instrumento usado para medir a altura das enchentes fluviais.}{flu.vi.ô.me.tro}{0}
\verb{flux}{cs}{}{}{}{s.m.}{Fluxo.}{flux}{0}
\verb{flux}{cs}{}{}{}{}{Usado na expressão \textit{a flux}: em abundância.}{flux}{0}
\verb{fluxo}{cs}{}{}{}{s.m.}{Ato ou efeito de fluir; curso.}{flu.xo}{0}
\verb{fluxo}{cs}{}{}{}{}{Grande quantidade de pessoas, animais ou coisas em movimento contínuo. (\textit{No final da tarde, o fluxo de veículos nas ruas é intenso.})}{flu.xo}{0}
\verb{fluxograma}{cs}{}{}{}{s.m.}{Diagrama para representação de um algoritmo; diagrama de fluxo.}{flu.xo.gra.ma}{0}
\verb{Fm}{}{Quím.}{}{}{}{Símb. do \textit{férmio}.}{Fm}{0}
\verb{FM}{}{}{}{}{}{Abrev. de \textit{frequência modulada}.}{FM}{0}
\verb{fobia}{}{}{}{}{s.f.}{Medo doentio, angustiante.}{fo.bi.a}{0}
\verb{foca}{ó}{Zool.}{}{}{s.f.}{Animal mamífero sem orelhas, de pernas curtas e achatadas em forma de nadadeiras, que vive nos mares frios e é caçado para o aproveitamento da pele, da carne e da gordura.}{fo.ca}{0}
\verb{focal}{}{}{"-ais}{}{adj.2g.}{Relativo a foco.}{fo.cal}{0}
\verb{focalizar}{}{}{}{}{v.t.}{Ajustar as lentes de instrumento óptico para ter a imagem clara de uma pessoa, um animal etc.; enfocar.}{fo.ca.li.zar}{0}
\verb{focalizar}{}{}{}{}{}{Pôr algum assunto em destaque.}{fo.ca.li.zar}{\verboinum{1}}
\verb{focar}{}{}{}{}{v.t.}{Focalizar.}{fo.car}{\verboinum{2}}
\verb{focinhar}{}{}{}{}{v.t.}{Cavar a terra com o focinho; fossar, fuçar.}{fo.ci.nhar}{0}
\verb{focinhar}{}{}{}{}{v.i.}{Cair com o rosto no chão.}{fo.ci.nhar}{\verboinum{1}}
\verb{focinheira}{ê}{}{}{}{s.f.}{Correia que tapa a boca do animal e o impede de morder.}{fo.ci.nhei.ra}{0}
\verb{focinho}{}{}{}{}{s.m.}{Parte da cabeça do animal em que se encontra o nariz, a boca e o queixo.}{fo.ci.nho}{0}
\verb{focinhudo}{}{}{}{}{adj.}{Que tem o focinho saliente.}{fo.ci.nhu.do}{0}
\verb{focinhudo}{}{Fig.}{}{}{}{Mal"-humorado, carrancudo.}{fo.ci.nhu.do}{0}
\verb{foco}{ó}{}{}{}{s.m.}{Ponto para onde se dirigem os raios de luz.}{fo.co}{0}
\verb{foco}{ó}{}{}{}{}{Ponto de saem raios de luz; facho, farol.}{fo.co}{0}
\verb{foco}{ó}{}{}{}{}{Ponto para onde a atenção de todos se dirige.}{fo.co}{0}
\verb{foco}{ó}{}{}{}{}{Ponto ou fonte de doenças.}{fo.co}{0}
\verb{fofo}{ô}{}{}{}{adj.}{Que afunda facilmente com algum peso ou pressão. (\textit{O bolo que fiz não ficou muito fofo.})}{fo.fo}{0}
\verb{fofo}{ô}{}{}{}{}{Bonito e gracioso. (\textit{Veja que bebê fofinho!})}{fo.fo}{0}
\verb{fofoca}{ó}{}{}{}{s.f.}{Ato ou efeito de fofocar; mexerico, intriga.}{fo.fo.ca}{0}
\verb{fofoca}{ó}{}{}{}{}{Aquilo que é comentado em segredo sobre outrem.}{fo.fo.ca}{0}
\verb{fofocar}{}{}{}{}{v.t.}{Fazer fofocas; divulgar segredos alheios; mexericar, bisbilhotar.}{fo.fo.car}{\verboinum{2}}
\verb{fofoqueiro}{ê}{}{}{}{adj.}{Diz"-se de indivíduo que faz fofocas, que se intromete em assuntos alheios.}{fo.fo.quei.ro}{0}
\verb{fofura}{}{}{}{}{s.f.}{Qualidade do que é fofo.}{fo.fu.ra}{0}
\verb{fofura}{}{}{}{}{}{Pessoa, animal ou coisa que desperta muita simpatia, afeição.}{fo.fu.ra}{0}
\verb{fogaça}{}{Cul.}{}{}{s.f.}{Grande bolo ou pão cozido.}{fo.ga.ça}{0}
\verb{fogacho}{}{}{}{}{s.m.}{Pequena labareda.}{fo.ga.cho}{0}
\verb{fogacho}{}{Fig.}{}{}{}{Sensação de quentura que vem à face em decorrência de fortes emoções ou de males físicos.}{fo.ga.cho}{0}
\verb{fogagem}{}{}{"-ens}{}{s.f.}{Designação popular a diversos distúrbios cutâneos e mucosos.}{fo.ga.gem}{0}
\verb{fogão}{}{}{"-ões}{}{s.m.}{Aparelho de ferro ou outro metal, geralmente provido de forno, cujo fogo se acende por meio de gás, eletricidade, carvão ou lenha, e que é usado para cozinhar.}{fo.gão}{0}
\verb{fogão}{}{}{"-ões}{}{}{Lareira.}{fo.gão}{0}
\verb{fogareiro}{ê}{}{}{}{s.m.}{Pequeno fogão portátil, geralmente de metal, onde se acende fogo para cozinhar ou para aquecer.}{fo.ga.rei.ro}{0}
\verb{fogaréu}{}{}{}{}{s.m.}{Fogo que se expande em labaredas.}{fo.ga.réu}{0}
\verb{fogo}{ô}{}{}{}{s.m.}{Produção de calor e claridade por meio da queima de material inflamável. (\textit{No acampamento, acendemos o fogo para que pudéssemos enxergar alguma coisa.})}{fo.go}{0}
\verb{fogo}{ô}{}{}{}{}{Descarga de arma de fogo. (\textit{Fugimos rapidamente do fogo inimigo.})}{fo.go}{0}
\verb{fogo}{ô}{}{}{}{}{Resultado de tomar muita bebida alcoólica; bebedeira. (\textit{É só ele beber um pouco que já fica de fogo.})}{fo.go}{0}
\verb{fogo"-apagou}{ô}{Zool.}{}{}{s.f.}{Rolinha cujo canto ou gemido lembra seu nome.}{fo.go"-a.pa.gou}{0}
\verb{fogo"-fátuo}{ô}{}{fogos"-fátuos ⟨ó⟩}{}{s.m.}{Luz gerada pela combustão de gases expelidos de matéria orgânica decomposta.}{fo.go"-fá.tu.o}{0}
\verb{fogos}{ó}{}{}{}{s.m.pl.}{Foguetes, rojões.}{fo.gos}{0}
\verb{fogo"-selvagem}{ô}{Med.}{fogos"-selvagens ⟨ó⟩}{}{s.m.}{Doença que enche a pele de bolhas.}{fo.go"-sel.va.gem}{0}
\verb{fogoso}{ô}{}{"-osos ⟨ó⟩}{"-osa ⟨ó⟩}{adj.}{Que tem fogo ou calor; abrasado.}{fo.go.so}{0}
\verb{fogoso}{ô}{Fig.}{"-osos ⟨ó⟩}{"-osa ⟨ó⟩}{}{Que impressiona pelo vigor; veemente, animado.}{fo.go.so}{0}
\verb{fogoso}{ô}{}{"-osos ⟨ó⟩}{"-osa ⟨ó⟩}{}{Irriquieto, impetuoso.}{fo.go.so}{0}
\verb{fogueira}{ê}{}{}{}{s.f.}{Monte de lenha ou de outro combustível em chamas.}{fo.guei.ra}{0}
\verb{fogueira}{ê}{Fig.}{}{}{}{Ardor, exaltação.}{fo.guei.ra}{0}
\verb{foguetada}{}{}{}{}{s.f.}{Lançamento simultâneo ou consecutivo de foguetes.}{fo.gue.ta.da}{0}
\verb{foguetada}{}{Pop.}{}{}{}{Repreensão.}{fo.gue.ta.da}{0}
\verb{foguete}{ê}{}{}{}{s.m.}{Peça de fogos de artifício que estoura no ar ou se abre em jogos de luzes vistosas.}{fo.gue.te}{0}
\verb{foguete}{ê}{Astron.}{}{}{}{Veículo com propulsão a jato, destinado a viagens na atmosfera ou no espaço, com força propulsora proveniente da ejeção de gases altamente aquecidos.}{fo.gue.te}{0}
\verb{fogueteiro}{ê}{}{}{}{s.m.}{Fabricante de foguetes e de outras peças de fogos de artifício.}{fo.gue.tei.ro}{0}
\verb{fogueteiro}{ê}{}{}{}{}{Indivíduo que se encarrega de queimar foguetes por ocasião de um festejo.}{fo.gue.tei.ro}{0}
\verb{fogueteiro}{ê}{Fig.}{}{}{}{Indivíduo dado a exageros ou mentiras.}{fo.gue.tei.ro}{0}
\verb{foguetório}{}{}{}{}{s.m.}{Grande porção de foguetes que estouram ao mesmo tempo.}{fo.gue.tó.rio}{0}
\verb{foguista}{}{Bras.}{}{}{s.2g.}{Indivíduo que se encarrega das fornalhas ou das caldeiras nas máquinas a vapor.}{fo.guis.ta}{0}
\verb{foiçada}{}{}{}{}{s.f.}{Golpe de foice.}{foi.ça.da}{0}
\verb{foiçar}{}{}{}{}{v.t.}{Cortar com foice; foiçar ou segar.}{foi.çar}{0}
\verb{foiçar}{}{}{}{}{v.i.}{Dar golpes com foice.}{foi.çar}{\verboinum{3}}
\verb{foice}{}{}{}{}{s.f.}{Instrumento formado de uma lâmina de aço curva, com o qual se ceifa ou sega.}{foi.ce}{0}
\verb{fojo}{ô}{}{}{}{s.m.}{Cova tapada por galhos, usada como armadilha.}{fo.jo}{0}
\verb{folclore}{ó}{}{}{}{s.m.}{Conjunto de tradições, artes, conhecimentos e crenças populares transmitidas oralmente.}{fol.clo.re}{0}
\verb{folclore}{ó}{Pop.}{}{}{}{Algo pitoresco ou fantasioso.}{fol.clo.re}{0}
\verb{folclórico}{}{}{}{}{adj.}{Relativo a folclore.}{fol.cló.ri.co}{0}
\verb{folclórico}{}{Por ext.}{}{}{}{Fantasioso.}{fol.cló.ri.co}{0}
\verb{folclórico}{}{Por ext.}{}{}{}{Antiquado, ultrapassado.}{fol.cló.ri.co}{0}
\verb{folclorista}{}{}{}{}{s.2g.}{Especialista em folclore.}{fol.clo.ris.ta}{0}
\verb{fôlder}{}{}{}{}{s.m.}{Folheto informativo, constituído de uma folha impressa e dobrada uma ou mais vezes.}{fôl.der}{0}
\verb{fole}{ó}{}{}{}{s.m.}{Artefato que faz vento ao ser contraído e expandido, destinado a ativar uma combustão ou limpar cavidades.}{fo.le}{0}
\verb{fole}{ó}{}{}{}{}{Sanfona.}{fo.le}{0}
\verb{fôlego}{}{}{}{}{s.m.}{Respiração.}{fô.le.go}{0}
\verb{fôlego}{}{}{}{}{}{Capacidade de reter ar nos pulmões.}{fô.le.go}{0}
\verb{fôlego}{}{}{}{}{}{Ânimo, disposição.}{fô.le.go}{0}
\verb{folga}{ó}{}{}{}{s.f.}{Espaço de tempo durante o qual se interrompe uma atividade ou trabalho; descanso, pausa.}{fol.ga}{0}
\verb{folga}{ó}{}{}{}{}{Desaperto, largueza.}{fol.ga}{0}
\verb{folga}{ó}{Pop.}{}{}{}{Atrevimento, abuso.}{fol.ga}{0}
\verb{folgado}{}{}{}{}{adj.}{Que tem folga.}{fol.ga.do}{0}
\verb{folgado}{}{}{}{}{}{Descansado.}{fol.ga.do}{0}
\verb{folgado}{}{}{}{}{}{Largo, amplo, não apertado.}{fol.ga.do}{0}
\verb{folgado}{}{Pop.}{}{}{}{Diz"-se de quem vive alegre, despreocupado, livre de cuidados.}{fol.ga.do}{0}
\verb{folgado}{}{Pop.}{}{}{}{Diz"-se de quem se esquiva ao trabalho, às obrigações ou aos deveres.}{fol.ga.do}{0}
\verb{folgança}{}{}{}{}{s.f.}{Folga, descanso.}{fol.gan.ça}{0}
\verb{folgança}{}{}{}{}{}{Divertimento, recreação.}{fol.gan.ça}{0}
\verb{folgar}{}{}{}{}{v.t.}{Dar descanso a uma pessoa, animal ou coisa.}{fol.gar}{0}
\verb{folgar}{}{}{}{}{}{Diminuir o aperto de alguma coisa; desapertar.}{fol.gar}{0}
\verb{folgar}{}{}{}{}{}{Estar sem trabalho.}{fol.gar}{0}
\verb{folgar}{}{}{}{}{v.i.}{Ficar alegre; alegrar"-se.}{fol.gar}{\verboinum{5}}
\verb{folgazão}{}{}{"-ões}{}{adj.}{Que gosta de folgar, brincar, divertir"-se.}{fol.ga.zão}{0}
\verb{folguedo}{ê}{}{}{}{s.m.}{Coisa que se faz por prazer; brincadeira, diversão.}{fol.gue.do}{0}
\verb{folha}{ô}{Bot.}{}{}{s.f.}{Órgão de forma chata e quase sempre verde que nasce nos galhos e nos ramos das plantas.}{fo.lha}{0}
\verb{folha}{ô}{}{}{}{}{Pedaço de papel de determinada forma e tamanho.}{fo.lha}{0}
\verb{folha}{ô}{}{}{}{}{Peça achatada e fina de algum material; chapa.}{fo.lha}{0}
\verb{folha"-corrida}{ô}{}{folhas"-corridas ⟨ô⟩}{}{s.f.}{Certidão passada por autoridades policiais, que atesta a ausência ou a presença de registros criminais.}{fo.lha"-cor.ri.da}{0}
\verb{folha"-de"-flandres}{ô}{}{folhas"-de"-flandres ⟨ô⟩}{}{s.f.}{Fina chapa de ferro laminado coberta com uma camada de estanho, de diversas aplicações como a fabricação de latas.}{fo.lha"-de"-flan.dres}{0}
\verb{folhado}{}{}{}{}{adj.}{Que tem muitas folhas.}{fo.lha.do}{0}
\verb{folhado}{}{}{}{}{}{Com a forma de folha.}{fo.lha.do}{0}
\verb{folhado}{}{Cul.}{}{}{}{Diz"-se de massa, torta, pastel etc. formado de inúmeras lâminas finas de massa de farinha de trigo.}{fo.lha.do}{0}
\verb{folhagem}{}{}{"-ens}{}{s.f.}{Conjunto das folhas de uma ou várias plantas.}{fo.lha.gem}{0}
\verb{folhagem}{}{}{"-ens}{}{}{Planta de muitas folhas e sem flores.}{fo.lha.gem}{0}
\verb{folhar}{}{}{}{}{v.t.}{Cobrir de folhas.}{fo.lhar}{\verboinum{1}}
\verb{folharada}{}{}{}{}{s.f.}{Grande porção de folhas.}{fo.lha.ra.da}{0}
\verb{folheado}{}{}{}{}{adj.}{Composto de folhas.}{fo.lhe.a.do}{0}
\verb{folheado}{}{}{}{}{}{Que se folheou; revestido.}{fo.lhe.a.do}{0}
\verb{folheado}{}{}{}{}{s.m.}{Lâmina de madeira ou de metal usada em revestimentos.}{fo.lhe.a.do}{0}
\verb{folhear}{}{}{}{}{v.t.}{Virar as folhas de livro, caderno, jornal, revista.}{fo.lhe.ar}{0}
\verb{folhear}{}{}{}{}{}{Ler rapidamente as folhas de caderno, jornal, livro, revista.}{fo.lhe.ar}{0}
\verb{folhear}{}{}{}{}{}{Revestir de lâminas.}{fo.lhe.ar}{\verboinum{4}}
\verb{folhetim}{}{}{"-ins}{}{s.m.}{Pequena parte de romance publicado dia a dia num jornal.}{fo.lhe.tim}{0}
\verb{folhetinesco}{ê}{}{}{}{adj.}{Relativo ou semelhante a folhetim.}{fo.lhe.ti.nes.co}{0}
\verb{folhetinesco}{ê}{}{}{}{}{Que tem pouco valor; de qualidade inferior.}{fo.lhe.ti.nes.co}{0}
\verb{folhetinista}{}{}{}{}{s.2g.}{Indivíduo que escreve folhetins.}{fo.lhe.ti.nis.ta}{0}
\verb{folheto}{ê}{}{}{}{s.m.}{Papel impresso para fazer propaganda.}{fo.lhe.to}{0}
\verb{folheto}{ê}{}{}{}{}{Caderno impresso com poucas páginas.}{fo.lhe.to}{0}
\verb{folhinha}{}{}{}{}{s.f.}{Pequena folha.}{fo.lhi.nha}{0}
\verb{folhinha}{}{}{}{}{}{Folha com o calendário impresso.}{fo.lhi.nha}{0}
\verb{folhoso}{ô}{}{"-osos ⟨ó⟩}{"-osa ⟨ó⟩}{adj.}{Cheio ou coberto de folhas.}{fo.lho.so}{0}
\verb{folhoso}{ô}{Zool.}{"-osos ⟨ó⟩}{"-osa ⟨ó⟩}{s.m.}{O terceiro estômago dos animais ruminantes.}{fo.lho.so}{0}
\verb{folhudo}{}{}{}{}{adj.}{Cheio ou coberto de folhas.}{fo.lhu.do}{0}
\verb{folhudo}{}{}{}{}{}{Com muitas folhas juntas.}{fo.lhu.do}{0}
\verb{folia}{}{}{}{}{s.f.}{Divertimento com muito barulho; farra.}{fo.li.a}{0}
\verb{foliação}{}{}{"-ões}{}{s.f.}{Época de brotamento das folhas.}{fo.li.a.ção}{0}
\verb{foliáceo}{}{}{}{}{adj.}{Relativo ou semelhante a folha.}{fo.li.á.ceo}{0}
\verb{foliáceo}{}{}{}{}{}{Que é feito de folhas.}{fo.li.á.ceo}{0}
\verb{folião}{}{}{"-ões}{}{s.m.}{Indivíduo que se diverte na folia.}{fo.li.ão}{0}
\verb{foliar}{}{}{}{}{v.i.}{Fazer folia; divertir"-se.}{fo.li.ar}{\verboinum{1}}
\verb{folículo}{}{}{}{}{s.m.}{Pequena folha ou lâmina.}{fo.lí.cu.lo}{0}
\verb{folículo}{}{}{}{}{}{Película, membrana.}{fo.lí.cu.lo}{0}
\verb{folículo}{}{Anat.}{}{}{}{Designação comum a várias estruturas em forma de saco.}{fo.lí.cu.lo}{0}
\verb{fólio}{}{}{}{}{s.m.}{Número que indica a paginação de uma publicação impressa.}{fó.lio}{0}
\verb{folíolo}{}{}{}{}{s.m.}{Pequena folha.}{fo.lí.o.lo}{0}
\verb{folíolo}{}{Bot.}{}{}{}{Nome dado a cada divisão do limbo de uma folha composta.}{fo.lí.o.lo}{0}
\verb{fome}{}{}{}{}{s.f.}{Desejo de comer.}{fo.me}{0}
\verb{fome}{}{}{}{}{}{Escassez generalizada de alimentos ou de condições para que as pessoas tenham acesso aos gêneros.}{fo.me}{0}
\verb{fome}{}{Fig.}{}{}{}{Necessidade de algo; desejo, avidez.}{fo.me}{0}
\verb{fomentação}{}{}{"-ões}{}{s.f.}{Ato ou efeito de fomentar; fomento.}{fo.men.ta.ção}{0}
\verb{fomentador}{ô}{}{}{}{adj.}{Que fomenta.}{fo.men.ta.dor}{0}
\verb{fomentar}{}{}{}{}{v.t.}{Promover o desenvolvimento.}{fo.men.tar}{0}
\verb{fomentar}{}{}{}{}{}{Estimular, excitar.}{fo.men.tar}{0}
\verb{fomentar}{}{}{}{}{}{Esfregar com medicamento líquido ou cremoso.}{fo.men.tar}{\verboinum{1}}
\verb{fomento}{}{}{}{}{s.m.}{Ato ou efeito de formentar.}{fo.men.to}{0}
\verb{fomento}{}{}{}{}{}{Medicamento líquido ou cremoso que se fricciona sobre a pele.}{fo.men.to}{0}
\verb{fominha}{}{Bras.}{}{}{adj.}{Sovina, avarento.}{fo.mi.nha}{0}
\verb{fonação}{}{Gram.}{"-ões}{}{s.f.}{Produção dos sons da fala.}{fo.na.ção}{0}
\verb{fonado}{}{}{}{}{adj.}{Diz"-se de telegrama ditado à agência telegráfica por telefone.}{fo.na.do}{0}
\verb{fonador}{ô}{}{}{}{adj.}{Que produz voz.}{fo.na.dor}{0}
\verb{fone}{}{}{}{}{s.m.}{Redução de \textit{telefone}.}{fo.ne}{0}
\verb{fone}{}{}{}{}{}{A peça do telefone que se leva ao ouvido, dotada de um alto"-falante e um microfone.}{fo.ne}{0}
\verb{fonema}{}{Gram.}{}{}{s.m.}{A menor unidade sonora distintiva da língua.}{fo.ne.ma}{0}
\verb{fonética}{}{Gram.}{}{}{s.f.}{Ciência que estuda os sons da fala, sua produção e percepção.}{fo.né.ti.ca}{0}
\verb{foneticista}{}{}{}{}{s.2g.}{Especialista em fonética.}{fo.ne.ti.cis.ta}{0}
\verb{fonético}{}{}{}{}{adj.}{Relativo à fonética ou aos sons da fala.}{fo.né.ti.co}{0}
\verb{foniatria}{}{Med.}{}{}{s.f.}{Ramo da medicina que estuda as perturbações da fonação decorrentes de anomalia fisiológica ou morfológica do aparelho fonador.}{fo.ni.a.tri.a}{0}
\verb{fônico}{}{}{}{}{adj.}{Relativo a som.}{fô.ni.co}{0}
\verb{fônico}{}{Gram.}{}{}{}{Relativo aos sons da fala.}{fô.ni.co}{0}
\verb{fonoaudiologia}{}{Med.}{}{}{s.f.}{Especialidade da medicina que estuda e trata os distúrbios da fala e da audição.}{fo.no.au.di.o.lo.gi.a}{0}
\verb{fonoaudiólogo}{}{}{}{}{s.m.}{Especialista em fonoaudiologia.}{fo.no.au.di.ó.lo.go}{0}
\verb{fonografia}{}{Fís.}{}{}{s.f.}{Representação gráfica das ondas sonoras.}{fo.no.gra.fi.a}{0}
\verb{fonógrafo}{}{}{}{}{s.m.}{Aparelho que reproduz os sons; gramofone.}{fo.nó.gra.fo}{0}
\verb{fonograma}{}{}{}{}{s.m.}{Telegrama transmitido por telefone.}{fo.no.gra.ma}{0}
\verb{fonologia}{}{Gram.}{}{}{s.f.}{Estudo dos fonemas e dos sons da língua, e das relações entre eles.}{fo.no.lo.gi.a}{0}
\verb{fonoteca}{é}{}{}{}{s.f.}{Coleção ou conjunto de registros e documentos sonoros.}{fo.no.te.ca}{0}
\verb{fontanela}{é}{Anat.}{}{}{s.f.}{Espaço membranoso no crânio dos fetos e dos bebês; moleira.}{fon.ta.ne.la}{0}
\verb{fonte}{}{}{}{}{s.f.}{Lugar de onde a água sai da terra; nascente. (\textit{A fonte em que pegamos água fica bem ao pé do morro.})}{fon.te}{0}
\verb{fonte}{}{}{}{}{}{Lugar de onde vem alguma coisa; origem. (\textit{A principal fonte de calor é o Sol.})}{fon.te}{0}
\verb{fonte}{}{}{}{}{}{Cada um dos lados da cabeça, entre os olhos e as orelhas. (\textit{Acertou"-lhe um soco na fonte.})}{fon.te}{0}
\verb{fora}{ó}{}{}{}{adv.}{No ambiente externo, em relação a algum outro. (\textit{Ele ficou fora de casa toda a noite.})}{fo.ra}{0}
\verb{fora}{ó}{}{}{}{}{Em lugar diferente daquele em que se mora. (\textit{Saímos para jantar fora.})}{fo.ra}{0}
\verb{fora}{ó}{Pop.}{}{}{s.m.}{Erro grosseiro; gafe. (\textit{Ele deu um fora na frente de todos.})}{fo.ra}{0}
\verb{fora}{ó}{Pop.}{}{}{}{Término de namoro. (\textit{Ele levou um fora da namorada.})}{fo.ra}{0}
\verb{fora}{ó}{}{}{}{prep.}{Exceto. (\textit{Todos nós fomos viajar, fora você. Mas na próxima você vai também.})}{fo.ra}{0}
\verb{fora"-da"-lei}{ó}{}{foras"-da"-lei ⟨ó⟩}{}{s.2g.}{Indivíduo marginal, delinquente.}{fo.ra"-da"-lei}{0}
\verb{foragido}{}{}{}{}{adj.}{Que está fora de sua terra; emigrado.}{fo.ra.gi.do}{0}
\verb{foragido}{}{}{}{}{}{Que está desaparecido para escapar de perseguidores, da polícia ou da justiça.}{fo.ra.gi.do}{0}
\verb{foragir"-se}{}{}{}{}{v.pron.}{Ir para fora de sua terra; emigrar.}{fo.ra.gir"-se}{0}
\verb{foragirse}{}{}{}{}{}{Fugir ou esconder"-se para escapar de perseguidores, da polícia ou da justiça.}{fo.ra.gir"-se}{\verboinum{22}}
\verb{forasteiro}{ê}{}{}{}{adj.}{Que é de fora, de outro lugar; estrangeiro, alheio, peregrino.}{fo.ras.tei.ro}{0}
\verb{forca}{ô}{}{}{}{s.f.}{Instrumento para executar por estrangulação os condenados.}{for.ca}{0}
\verb{força}{ô}{}{}{}{s.f.}{Qualidade do que é forte; vigor, energia, robustez.}{for.ça}{0}
\verb{força}{ô}{Fís.}{}{}{}{Agente físico capaz de produzir, alterar ou cessar o movimento.}{for.ça}{0}
\verb{força}{ô}{}{}{}{}{Energia elétrica.}{for.ça}{0}
\verb{força}{ô}{}{}{}{}{Aquilo que influencia; poder, influência, domínio, poderio.}{for.ça}{0}
\verb{forcado}{}{}{}{}{s.m.}{Ferramenta de uso na lavoura constituída de uma haste com duas ou três pontas; garfo.}{for.ca.do}{0}
\verb{forcado}{}{}{}{}{}{Porção de palha que se apanha de uma vez com o forcado.}{for.ca.do}{0}
\verb{forçado}{}{}{}{}{adj.}{Coagido, pressionado, compelido, obrigado.}{for.ça.do}{0}
\verb{forçado}{}{}{}{}{}{Feito sem naturalidade; artificial, fingido, contrafeito, não espontâneo.}{for.ça.do}{0}
\verb{forçado}{}{}{}{}{}{Diz"-se de trabalho feito como forma de pena, resultante de condenação.}{for.ça.do}{0}
\verb{forçar}{}{}{}{}{v.t.}{Obter ou entrar à força; arrombar.}{for.çar}{0}
\verb{forçar}{}{}{}{}{}{Constranger, violentar.}{for.çar}{0}
\verb{forçar}{}{}{}{}{}{Oferecer interpretação descabida e carente de fundamentos.}{for.çar}{\verboinum{3}}
\verb{forcejar}{}{}{}{}{v.t.}{Fazer diligência; esforçar"-se, empenhar"-se.}{for.ce.jar}{0}
\verb{forcejar}{}{}{}{}{}{Resistir, lutar, pelejar.}{for.ce.jar}{\verboinum{1}}
\verb{forcejo}{ê}{}{}{}{s.m.}{Ato ou efeito de forcejar.}{for.ce.jo}{0}
\verb{fórceps}{}{}{}{}{s.m.}{Espécie de pinça cirúrgica.}{fór.ceps}{0}
\verb{fórceps}{}{}{}{}{}{Instrumento para extrair a criança do útero em caso de problemas de parto.}{fór.ceps}{0}
\verb{forçoso}{ô}{}{"-osos ⟨ó⟩}{"-osa ⟨ó⟩}{adj.}{Que tem força; vigoroso.}{for.ço.so}{0}
\verb{forçoso}{ô}{}{"-osos ⟨ó⟩}{"-osa ⟨ó⟩}{}{Violento.}{for.ço.so}{0}
\verb{forçoso}{ô}{}{"-osos ⟨ó⟩}{"-osa ⟨ó⟩}{}{Inevitável, fatal, imprescindível.}{for.ço.so}{0}
\verb{forçudo}{}{Pop.}{}{}{adj.}{Forte, robusto, musculoso.}{for.çu.do}{0}
\verb{foreiro}{ê}{}{}{}{adj.}{Referente a foro.}{fo.rei.ro}{0}
\verb{foreiro}{ê}{}{}{}{}{Obrigado por benefício; sujeito, constrangido.}{fo.rei.ro}{0}
\verb{forense}{}{}{}{}{adj.2g.}{Relativo a foro judicial ou a tribunal.}{fo.ren.se}{0}
\verb{forja}{ó}{}{}{}{s.f.}{Oficina de fundição e modelagem de metais.}{for.ja}{0}
\verb{forjador}{ô}{Fig.}{}{}{adj.}{Que inventa ou elabora algo artificialmente; mentiroso, falsário.}{for.ja.dor}{0}
\verb{forjador}{ô}{}{}{}{s.m.}{Indivíduo que forja; ferreiro.}{for.ja.dor}{0}
\verb{forjar}{}{}{}{}{v.t.}{Fabricar ou modelar na forja.}{for.jar}{0}
\verb{forjar}{}{Fig.}{}{}{}{Inventar ou elaborar artificialmente; mentir, criar.}{for.jar}{\verboinum{1}}
\verb{forjicar}{}{}{}{}{v.t.}{Forjar algo que dê muito trabalho.}{for.ji.car}{0}
\verb{forjicar}{}{}{}{}{}{Forjar.}{for.ji.car}{\verboinum{2}}
\verb{forma}{ô}{}{}{}{s.f.}{Molde, modelo de qualquer coisa.}{for.ma}{0}
\verb{forma}{ó}{}{}{}{s.f.}{Feitio, aparência, configuração das coisas.}{for.ma}{0}
\verb{forma}{ó}{}{}{}{}{Disposição exterior das partes de um corpo e que constitui a diferença entre um e outro indivíduo, uma e outra espécie.}{for.ma}{0}
\verb{forma}{ó}{}{}{}{}{Maneira, modo.}{for.ma}{0}
\verb{forma}{ô}{}{}{}{}{Peça que imita o pé, usada no fabrico de calçados.}{for.ma}{0}
\verb{forma}{ô}{}{}{}{}{Vasilha em que se assam bolos, tortas etc.}{for.ma}{0}
\verb{formação}{}{}{"-ões}{}{s.f.}{Ato, efeito ou modo de formar; constituição, criação.}{for.ma.ção}{0}
\verb{formação}{}{}{"-ões}{}{}{Constituição de um ser, um objeto ou um conjunto deles.}{for.ma.ção}{0}
\verb{formação}{}{}{"-ões}{}{}{Constituição de uma personalidade ou um caráter; educação.}{for.ma.ção}{0}
\verb{formação}{}{}{"-ões}{}{}{Disposição dos elementos de uma tropa de soldados, navios, aeronaves.}{for.ma.ção}{0}
\verb{formado}{}{}{}{}{adj.}{Que se formou; pronto, completo.}{for.ma.do}{0}
\verb{formado}{}{}{}{}{}{Que concluiu etapa do percurso acadêmico, geralmente curso superior.}{for.ma.do}{0}
\verb{formador}{ô}{}{}{}{adj.}{Que constitui ou dá forma.}{for.ma.dor}{0}
\verb{formal}{}{}{"-ais}{}{adj.2g.}{Relativo a forma.}{for.mal}{0}
\verb{formal}{}{}{"-ais}{}{}{Relativo à aparência em detrimento da essência.}{for.mal}{0}
\verb{formal}{}{}{"-ais}{}{}{Oficial, textual, solene.}{for.mal}{0}
\verb{formal}{}{}{"-ais}{}{}{Não espontâneo; cerimonioso, protocolar, sério, sisudo.}{for.mal}{0}
\verb{formaldeído}{}{Quím.}{}{}{s.m.}{Substância usada na produção de desinfetantes, desodorantes, fungicidas.}{for.mal.de.í.do}{0}
\verb{formalidade}{}{}{}{}{s.f.}{Ato oficial cumprido de maneira previamente estabelecida; praxe.}{for.ma.li.da.de}{0}
\verb{formalidade}{}{}{}{}{}{Comprotamento formal; cerimônia.}{for.ma.li.da.de}{0}
\verb{formalismo}{}{}{}{}{s.m.}{Característica do que é formal.}{for.ma.lis.mo}{0}
\verb{formalismo}{}{}{}{}{}{Apego a normas de comportamento; rigidez.}{for.ma.lis.mo}{0}
\verb{formalista}{}{}{}{}{adj.2g.}{Adepto do formalismo.}{for.ma.lis.ta}{0}
\verb{formalização}{}{}{"-ões}{}{s.f.}{Ato ou efeito de formalizar.}{for.ma.li.za.ção}{0}
\verb{formalizar}{}{}{}{}{v.t.}{Dar forma.}{for.ma.li.zar}{0}
\verb{formalizar}{}{}{}{}{}{Executar observando as formalidades.}{for.ma.li.zar}{\verboinum{1}}
\verb{formando}{}{Bras.}{}{}{s.m.}{Indivíduo que está prestes a formar"-se, a concluir um curso.}{for.man.do}{0}
\verb{formão}{}{}{"-ões}{}{s.m.}{Ferramenta com cabo de madeira e ponta cortante para desbastar ou fazer cavidades na madeira.}{for.mão}{0}
\verb{formar}{}{}{}{}{v.t.}{Dar forma.}{for.mar}{0}
\verb{formar}{}{}{}{}{}{Constituir, compor.}{for.mar}{0}
\verb{formar}{}{}{}{}{}{Educar, orientar, instruir.}{for.mar}{0}
\verb{formar}{}{}{}{}{v.pron.}{Concluir etapa do percurso acadêmico, especialmente o ensino superior.}{for.mar}{\verboinum{1}}
\verb{formatação}{}{Informát.}{"-ões}{}{s.f.}{Ato ou efeito de formatar.}{for.ma.ta.ção}{0}
\verb{formatar}{}{Informát.}{}{}{v.t.}{Criar ou adaptar a estrutura de um conjunto de dados a um padrão determinado.}{for.ma.tar}{\verboinum{1}}
\verb{formativo}{}{}{}{}{adj.}{Que forma ou serve para formar; formador.}{for.ma.ti.vo}{0}
\verb{formato}{}{}{}{}{s.m.}{Configuração física das formas de algo.}{for.ma.to}{0}
\verb{formato}{}{}{}{}{}{Tamanho, dimensão.}{for.ma.to}{0}
\verb{formatura}{}{}{}{}{s.f.}{Ato ou efeito de formar.}{for.ma.tu.ra}{0}
\verb{formatura}{}{}{}{}{}{Graduação em escola superior ou em outros cursos.}{for.ma.tu.ra}{0}
\verb{formiato}{}{Quím.}{}{}{s.m.}{Qualquer sal, éster ou ânion derivado do ácido fórmico.}{for.mi.a.to}{0}
\verb{fórmica}{}{}{}{}{s.f.}{Nome comercial de material sintético laminado, duro e liso, usado geralmente para revestir móveis e paredes.}{fór.mi.ca}{0}
\verb{formicida}{}{}{}{}{s.m.}{Preparado químico para matar formigas.}{for.mi.ci.da}{0}
\verb{fórmico}{}{Quím.}{}{}{adj.}{Diz"-se de um ácido e de um aldeído.}{fór.mi.co}{0}
\verb{formidável}{}{}{"-eis}{}{adj.2g.}{Que ultrapassa as dimensões normais; colossal, gigantesco.}{for.mi.dá.vel}{0}
\verb{formidável}{}{}{"-eis}{}{}{Excelente, fantástico, ótimo.}{for.mi.dá.vel}{0}
\verb{formiga}{}{Zool.}{}{}{s.f.}{Nome comum dado a insetos que vivem em sociedade debaixo da terra, em ninhos sobre árvores, no oco de paus etc.}{for.mi.ga}{0}
\verb{formigamento}{}{Med.}{}{}{s.m.}{Sensação desagradável como uma comichão na superfície da pele, acompanhada de dormência. }{for.mi.ga.men.to}{0}
\verb{formigão}{}{}{"-ões}{}{s.m.}{Mistura dosada de terra, pedregulho, água e cal, própria para construção de paredes de taipa.}{for.mi.gão}{0}
\verb{formigão}{}{}{"-ões}{}{s.m.}{Formiga grande.}{for.mi.gão}{0}
\verb{formigar}{}{}{}{}{v.i.}{Ter a sensação de formigamento. (\textit{Minha perna está formigando.})}{for.mi.gar}{0}
\verb{formigar}{}{}{}{}{}{Existir em abundância; pulular, fervilhar.}{for.mi.gar}{\verboinum{5}}
\verb{formigueiro}{ê}{}{}{}{s.m.}{Ninho ou toca de formigas. }{for.mi.guei.ro}{0}
\verb{formigueiro}{ê}{Fig.}{}{}{}{Grande multidão; grande quantidade. (\textit{Em época de Natal, o centro da cidade fica parecendo um formigueiro.})}{for.mi.guei.ro}{0}
\verb{forminha}{}{}{}{}{s.f.}{Pequena forma circular usada para assar empadinhas, bolinhos, brevidades etc. }{for.mi.nha}{0}
\verb{formol}{ó}{}{}{}{s.m.}{Solução aquosa usada como antisséptico, bactericida ou para conservar cadáveres.}{for.mol}{0}
\verb{formosino}{}{}{}{}{adj.}{Relativo a Formosa ou República Nacional da China (Ásia), atual Taiwan; taiuanês.}{for.mo.si.no}{0}
\verb{formosino}{}{}{}{}{s.m.}{Indivíduo natural ou habitante desse país.  }{for.mo.si.no}{0}
\verb{formoso}{ô}{}{"-osos ⟨ó⟩}{"-osa ⟨ó⟩}{adj.}{De belas formas; bonito, perfeito, harmonioso.}{for.mo.so}{0}
\verb{formosura}{}{}{}{}{s.f.}{Qualidade de formoso; beleza, perfeição, harmonia.}{for.mo.su.ra}{0}
\verb{fórmula}{}{}{}{}{s.f.}{Lista dos componentes de um medicamento.}{fór.mu.la}{0}
\verb{fórmula}{}{}{}{}{}{Receita para se fazer algo.}{fór.mu.la}{0}
\verb{fórmula}{}{}{}{}{}{Expressão de um preceito, uma regra ou um princípio.}{fór.mu.la}{0}
\verb{fórmula}{}{Mat.}{}{}{}{Expressão genérica para resolver problemas semelhantes.}{fór.mu.la}{0}
\verb{formulação}{}{}{"-ões}{}{s.f.}{Ato ou efeito de formular.}{for.mu.la.ção}{0}
\verb{formulação}{}{}{"-ões}{}{}{Enunciação clara; exposição, redação.}{for.mu.la.ção}{0}
\verb{formular}{}{}{}{}{v.t.}{Receitar medicamentos; aviar receitas.}{for.mu.lar}{0}
\verb{formular}{}{}{}{}{}{Expor com precisão; articular, enunciar.}{for.mu.lar}{\verboinum{1}}
\verb{formulário}{}{}{}{}{s.m.}{Modelo impresso em que se preenchem dados particulares para fazer pedidos, prestar declarações ou outras finalidades.}{for.mu.lá.rio}{0}
\verb{formulário}{}{}{}{}{}{Coleção de fórmulas.}{for.mu.lá.rio}{0}
\verb{fornada}{}{}{}{}{s.f.}{Conjunto de pães, biscoitos etc. que são cozidos de uma vez no mesmo forno.}{for.na.da}{0}
\verb{fornada}{}{}{}{}{}{Quantidade de coisas que se fazem de uma vez.}{for.na.da}{0}
\verb{fornalha}{}{}{}{}{s.f.}{Parte da caldeira a vapor onde se queima o combustível.}{for.na.lha}{0}
\verb{fornalha}{}{Fig.}{}{}{}{Lugar muito quente. (\textit{Esse quarto está uma fornalha de tão quente!})}{for.na.lha}{0}
\verb{fornecedor}{ô}{}{}{}{adj.}{Que fornece; abastecedor.}{for.ne.ce.dor}{0}
\verb{fornecer}{ê}{}{}{}{v.t.}{Prover do necessário; abastecer.}{for.ne.cer}{0}
\verb{fornecer}{ê}{}{}{}{}{Produzir, gerar.}{for.ne.cer}{\verboinum{15}}
\verb{fornecimento}{}{}{}{}{s.m.}{Ato ou efeito de fornecer; abastecimento, provisão.}{for.ne.ci.men.to}{0}
\verb{forneiro}{ê}{}{}{}{adj.}{Relativo a forno.}{for.nei.ro}{0}
\verb{forneiro}{ê}{}{}{}{s.m.}{Indivíduo que se encarrega ou é dono de um forno.}{for.nei.ro}{0}
\verb{fornicar}{}{}{}{}{v.t.}{Ter relações sexuais; copular.}{for.ni.car}{\verboinum{2}}
\verb{fornido}{}{}{}{}{adj.}{Que se forniu; abastecido, provido.}{for.ni.do}{0}
\verb{fornido}{}{}{}{}{}{Forte, robusto, encorpado.}{for.ni.do}{0}
\verb{fornilho}{}{}{}{}{s.m.}{Pequeno forno; fogareiro.}{for.ni.lho}{0}
\verb{fornilho}{}{}{}{}{}{Parte do cachimbo onde arde o fumo.}{for.ni.lho}{0}
\verb{fornir}{}{}{}{}{v.t.}{Prover do necessário; abastecer, fornecer.}{for.nir}{0}
\verb{fornir}{}{}{}{}{}{Tornar nutrido, robusto.}{for.nir}{\verboinum{18}\verboirregular{\emph{def.} fornimos, fornis}}
\verb{forno}{ô}{}{}{}{s.m.}{Compartimento do fogão onde se fazem os assados.}{for.no}{0}
\verb{forno}{ô}{Fig.}{}{}{}{Lugar muito quente.}{for.no}{0}
\verb{foro}{ô}{}{}{}{s.m.}{Local em que o poder judiciário desempenha suas funções; fórum.}{fo.ro}{0}
\verb{foro}{ô}{}{}{}{}{Alçada, jurisdição.}{fo.ro}{0}
\verb{foro}{ô}{}{}{}{}{Pensão anual de um prédio que se paga ao senhorio. }{fo.ro}{0}
\verb{foro}{ó}{}{}{}{s.m.}{Local para debates, ou reunião para o mesmo fim.    }{fo.ro}{0}
\verb{foro}{ó}{}{}{}{}{Praça pública na antiga Roma.  }{fo.ro}{0}
\verb{foros}{ó}{}{}{}{s.m.pl.}{Prerrogativas que a lei concede a alguém; privilégios, imunidades.}{fo.ros}{0}
\verb{forqueta}{ê}{}{}{}{s.f.}{Pau bifurcado na ponta; forquilha.}{for.que.ta}{0}
\verb{forquilha}{}{}{}{}{s.f.}{Pequeno forcado de três pontas.}{for.qui.lha}{0}
\verb{forquilha}{}{}{}{}{}{Vara bifurcada, usada geralmente para se fazer estilingue.}{for.qui.lha}{0}
\verb{forra}{ó}{}{}{}{s.f.}{Reparação de ofensa; desforra, represália, vingança.}{for.ra}{0}
\verb{forrado}{}{}{}{}{adj.}{Que se forrou; coberto, revestido.}{for.ra.do}{0}
\verb{forrageiro}{ê}{}{}{}{adj.}{Diz"-se da planta ou do grão que serve como forragem.}{for.ra.gei.ro}{0}
\verb{forragem}{}{}{"-ens}{}{s.f.}{Planta ou grão utilizado para alimentação do gado.}{for.ra.gem}{0}
\verb{forrar}{}{}{}{}{v.t.}{Pôr forro; cobrir, revestir.}{for.rar}{0}
\verb{forrar}{}{Pop.}{}{}{}{Alimentar"-se. (\textit{Antes de sair, preciso forrar o estômago.})}{for.rar}{\verboinum{1}}
\verb{forreta}{ê}{}{}{}{adj.}{Avarento, muquirana.}{for.re.ta}{0}
\verb{forro}{ô}{}{}{}{s.m.}{Material que serve para encher ou revestir a superfície interna de alguma coisa, como sofás, almofadas, peças de vestuário etc.}{for.ro}{0}
\verb{forró}{}{}{}{}{s.m.}{Baile popular de origem nordestina; arrasta"-pé.}{for.ró}{0}
\verb{forrobodó}{}{}{}{}{s.m.}{Baile popular; arrasta"-pé.}{for.ro.bo.dó}{0}
\verb{forrobodó}{}{}{}{}{}{Confusão, tumulto, briga.}{for.ro.bo.dó}{0}
\verb{fortalecedor}{ô}{}{}{}{adj.}{Que fortalece; que refaz as forças.}{for.ta.le.ce.dor}{0}
\verb{fortalecer}{ê}{}{}{}{v.t.}{Tornar forte, robusto; fortificar.}{for.ta.le.cer}{\verboinum{15}}
\verb{fortalecimento}{}{}{}{}{s.m.}{Ato ou efeito de fortalecer, de refazer as forças.}{for.ta.le.ci.men.to}{0}
\verb{fortaleza}{ê}{}{}{}{}{Qualidade de forte; força, solidez, firmeza.}{for.ta.le.za}{0}
\verb{fortaleza}{ê}{}{}{}{s.f.}{Fortificação, castelo, forte.}{for.ta.le.za}{0}
\verb{fortalezense}{}{}{}{}{adj.2g.}{Relativo a Fortaleza, capital do Ceará.}{for.ta.le.zen.se}{0}
\verb{fortalezense}{}{}{}{}{s.2g.}{Indivíduo natural ou habitante dessa cidade.}{for.ta.le.zen.se}{0}
\verb{forte}{ó}{}{}{}{adj.2g.}{Que tem grande força física; vigoroso, robusto, resistente.}{for.te}{0}
\verb{forte}{ó}{}{}{}{}{Intenso, concentrado.}{for.te}{0}
\verb{forte}{ó}{}{}{}{s.m.}{Obra de fortificação, fortaleza.}{for.te}{0}
\verb{fortidão}{}{}{"-ões}{}{s.f.}{Qualidade do que é forte, consistente; solidez.}{for.ti.dão}{0}
\verb{fortificação}{}{}{"-ões}{}{s.f.}{Ato ou efeito de fortificar.}{for.ti.fi.ca.ção}{0}
\verb{fortificação}{}{}{"-ões}{}{}{Fortaleza, baluarte, forte.}{for.ti.fi.ca.ção}{0}
\verb{fortificador}{ô}{}{}{}{adj.}{Que fortifica, robustece; fortificante.}{for.ti.fi.ca.dor}{0}
\verb{fortificante}{}{}{}{}{adj.2g.}{Que fortifica.}{for.ti.fi.can.te}{0}
\verb{fortificante}{}{}{}{}{s.m.}{Medicamento que fortifica o organismo.}{for.ti.fi.can.te}{0}
\verb{fortificar}{}{}{}{}{v.t.}{Tornar forte; fortalecer.}{for.ti.fi.car}{0}
\verb{fortificar}{}{}{}{}{}{Cercar de fortificações.}{for.ti.fi.car}{\verboinum{2}}
\verb{fortim}{}{}{"-ins}{}{s.m.}{Pequeno forte.}{for.tim}{0}
\verb{fortuito}{}{}{}{}{adj.}{Que ocorre por acaso; casual, imprevisto.}{for.tui.to}{0}
\verb{fortuna}{}{}{}{}{s.f.}{Conjunto de bens, haveres; riqueza.}{for.tu.na}{0}
\verb{fortuna}{}{}{}{}{}{Acontecimento imprevisto; casualidade, eventualidade.}{for.tu.na}{0}
\verb{fortuna}{}{}{}{}{}{Destino, fado, sina.}{for.tu.na}{0}
\verb{fórum}{}{}{"-uns}{}{s.m.}{Tribunal de justiça; foro.}{fó.rum}{0}
\verb{fosco}{ô}{}{}{}{adj.}{Sem brilho; de pouca transparência; embaciado. (\textit{Esse remédio precisa ser acondicionado em recipiente fosco porque ele não pode ter contato com a luz.})}{fos.co}{0}
\verb{fosfatado}{}{}{}{}{adj.}{Que contém fosfato.}{fos.fa.ta.do}{0}
\verb{fosfato}{}{Quím.}{}{}{s.m.}{Qualquer sal, éster ou ânion derivado do ácido fosfórico.}{fos.fa.to}{0}
\verb{fosforado}{}{Quím.}{}{}{adj.}{Combinado ou misturado com fósforo.}{fos.fo.ra.do}{0}
\verb{fosforescência}{}{}{}{}{s.f.}{Propriedade que certos corpos possuem de brilhar no escuro sem emitir calor.}{fos.fo.res.cên.cia}{0}
\verb{fosforescente}{}{}{}{}{adj.2g.}{Diz"-se do corpo que brilha no escuro. (\textit{Nas regiões mais profundas do oceano, existem peixes fosforescentes.})}{fos.fo.res.cen.te}{0}
\verb{fosforescer}{ê}{}{}{}{v.i.}{Brilhar no escuro sem emitir calor.}{fos.fo.res.cer}{\verboinum{15}}
\verb{fosfórico}{}{}{}{}{adj.}{Relativo a fósforo.}{fos.fó.ri.co}{0}
\verb{fosfórico}{}{}{}{}{}{Que brilha como o fósforo.}{fos.fó.ri.co}{0}
\verb{fósforo}{}{Quím.}{}{}{s.m.}{Elemento químico do grupo dos não metais, reativo, insolúvel em água, inflamável no ar, usado na agricultura como adubo ou pesticida, na fabricação de fósforos etc. \elemento{15}{30.97376}{P}.}{fós.fo.ro}{0}
\verb{fósforo}{}{Por ext.}{}{}{}{Palito de madeira ou cartão com uma das extremidades coberta com substâncias que se inflamam quando esfregadas em uma superfície áspera.}{fós.fo.ro}{0}
\verb{fosforoscópio}{}{Fís.}{}{}{s.m.}{Instrumento usado para medir a fosforescência de uma substância.}{fos.fo.ros.có.pio}{0}
\verb{fossa}{ó}{}{}{}{s.f.}{Buraco cavado na terra e ligado ao encanamento de esgoto para receber os dejetos de uma casa.}{fos.sa}{0}
\verb{fossa}{ó}{Fig.}{}{}{}{Estado de grande tristeza; depressão.}{fos.sa}{0}
\verb{fossar}{}{}{}{}{v.t.}{Cavar buracos com o focinho; fuçar, escavar.}{fos.sar}{\verboinum{1}}
\verb{fóssil}{}{}{"-eis}{}{s.m.}{Restos de seres vivos muito antigos que foram soterrados e ficaram endurecidos.}{fós.sil}{0}
\verb{fóssil}{}{}{"-eis}{}{adj.2g.}{Retrógrado, antiquado.}{fós.sil}{0}
\verb{fossilização}{}{}{"-ões}{}{s.f.}{Processo ou efeito de fossilizar.}{fos.si.li.za.ção}{0}
\verb{fossilizar}{}{}{}{}{v.t.}{Tornar fóssil; petrificar.}{fos.si.li.zar}{0}
\verb{fossilizar}{}{Fig.}{}{}{}{Tornar retrógrado, antiquado.}{fos.si.li.zar}{\verboinum{1}}
\verb{fosso}{ô}{}{}{}{s.m.}{Grande abertura cavada na terra; vala, buraco.}{fos.so}{0}
\verb{fotelétrico}{}{Fís.}{}{}{adj.}{Fotoelétrico.}{fo.te.lé.tri.co}{0}
\verb{foto}{ó}{}{}{}{s.f.}{Redução de \textit{fotografia}.}{fo.to}{0}
\verb{fotocomposição}{}{}{"-ões}{}{s.f.}{Processo de composição que utiliza técnicas fotográficas no preparo de fotolitos para impressão.}{fo.to.com.po.si.ção}{0}
\verb{fotocompositora}{ô}{}{}{}{s.f.}{Equipamento utilizado no processo de fotocomposição. }{fo.to.com.po.si.to.ra}{0}
\verb{fotocópia}{}{}{}{}{s.f.}{Cópia de material impresso feita por processo fotográfico.}{fo.to.có.pia}{0}
\verb{fotocopiar}{}{}{}{}{v.t.}{Reproduzir por meio de fotocópia.}{fo.to.co.pi.ar}{\verboinum{1}}
\verb{fotoelétrico}{}{Fís.}{}{}{adj.}{Que transforma energia luminosa em elétrica.}{fo.to.e.lé.tri.co}{0}
\verb{fotofobia}{}{}{}{}{s.f.}{Aversão ou intolerância à luz, geralmente associada a doenças oculares ou neurológicas.}{fo.to.fo.bi.a}{0}
\verb{fotogênico}{}{}{}{}{adj.}{Que aparece bem em fotografia. (\textit{Minha filha é fotogênica.})}{fo.to.gê.ni.co}{0}
\verb{fotografar}{}{}{}{}{v.t.}{Reproduzir por processo de fotografia.}{fo.to.gra.far}{\verboinum{1}}
\verb{fotografia}{}{}{}{}{s.f.}{Técnica e processo de fixação de imagens em materiais sensíveis à luz.}{fo.to.gra.fi.a}{0}
\verb{fotografia}{}{}{}{}{}{A imagem obtida por esse processo.}{fo.to.gra.fi.a}{0}
\verb{fotográfico}{}{}{}{}{adj.}{Relativo a fotografia.}{fo.to.grá.fi.co}{0}
\verb{fotógrafo}{}{}{}{}{s.m.}{Indivíduo que se dedica à fotografia.}{fo.tó.gra.fo}{0}
\verb{fotogravura}{}{}{}{}{s.f.}{Processo fotográfico de gravação sobre pranchas metálicas.}{fo.to.gra.vu.ra}{0}
\verb{fotolito}{}{}{}{}{s.m.}{Filme fotográfico com imagens, com o qual se prepara a matriz da impressão.}{fo.to.li.to}{0}
\verb{fotometria}{}{Fís.}{}{}{s.f.}{Medição das propriedades da luz.}{fo.to.me.tri.a}{0}
\verb{fotômetro}{}{}{}{}{s.m.}{Instrumento que mede a intensidade da luz.}{fo.tô.me.tro}{0}
\verb{fotomontagem}{}{}{"-ens}{}{s.f.}{Técnica que permite criar uma composição pela reunião de duas ou mais imagens distintas.}{fo.to.mon.ta.gem}{0}
\verb{fotonovela}{é}{}{}{}{s.f.}{Narrativa apresentada por meio de quadrinhos fotográficos, com os textos dos personagens em balões ou legendas como nas histórias em quadrinhos.}{fo.to.no.ve.la}{0}
\verb{fotosfera}{é}{}{}{}{s.f.}{Camada externa do Sol que emana luz e calor.}{fo.tos.fe.ra}{0}
\verb{fotossensibilidade}{}{}{}{}{s.f.}{Sensibilidade excessiva da pele aos raios solares.}{fo.tos.sen.si.bi.li.da.de}{0}
\verb{fotossensível}{}{Fís.}{"-eis}{}{adj.2g.}{Sensível à luz.}{fo.tos.sen.sí.vel}{0}
\verb{fotossensível}{}{Med.}{"-eis}{}{}{Que apresenta fotossensibilidade.}{fo.tos.sen.sí.vel}{0}
\verb{fotossíntese}{}{Bot.}{}{}{s.f.}{Síntese de matéria orgânica que utiliza a luz como fonte de energia.}{fo.tos.sín.te.se}{0}
\verb{fototerapia}{}{Med.}{}{}{s.f.}{Método terapêutico no qual o paciente é submetido a banhos de luz natural ou artificial.}{fo.to.te.ra.pi.a}{0}
\verb{fototropismo}{}{Bot.}{}{}{s.m.}{Inclinação das plantas na direção da fonte de luz.}{fo.to.tro.pis.mo}{0}
\verb{foz}{ó}{}{}{}{s.f.}{Local em que um rio deságua em outro rio ou no mar.}{foz}{0}
\verb{Fr}{}{Quím.}{}{}{}{Símb. do \textit{frâncio}.}{Fr}{0}
\verb{fracalhão}{}{}{"-ões}{}{s.m.}{Indivíduo muito fraco ou covarde, medroso.}{fra.ca.lhão}{0}
\verb{fração}{}{}{"-ões}{}{s.f.}{Parte de um todo dividido ou quebrado.}{fra.ção}{0}
\verb{fração}{}{Mat.}{"-ões}{}{}{Expressão do quociente entre dois números.}{fra.ção}{0}
\verb{fracassar}{}{}{}{}{v.i.}{Não atingir os objetivos; falhar, malograr.}{fra.cas.sar}{\verboinum{1}}
\verb{fracasso}{}{}{}{}{s.m.}{Insucesso, falha, malogro.}{fra.cas.so}{0}
\verb{fracionamento}{}{}{}{}{s.m.}{Ato ou efeito de fracionar.}{fra.ci.o.na.men.to}{0}
\verb{fracionar}{}{}{}{}{v.t.}{Dividir um todo em frações, partes; fragmentar.}{fra.ci.o.nar}{\verboinum{1}}
\verb{fracionário}{}{}{}{}{adj.}{Relativo a fração.}{fra.ci.o.ná.rio}{0}
\verb{fraco}{}{}{}{}{adj.}{Sem força ou intensidade.}{fra.co}{0}
\verb{fraco}{}{}{}{}{}{Sem capacidade; incompetente, insuficiente.}{fra.co}{0}
\verb{fractal}{}{Mat.}{"-ais}{}{s.m.}{Estrutura geométrica cujas propriedades se repetem em qualquer escala.}{frac.tal}{0}
\verb{frade}{}{}{}{freira}{s.m.}{Membro de ordem religiosa; monge, frei.}{fra.de}{0}
\verb{fraga}{}{}{}{}{s.f.}{Rocha escarpada; penhasco.}{fra.ga}{0}
\verb{fragata}{}{}{}{}{s.f.}{Tipo de navio de guerra.}{fra.ga.ta}{0}
\verb{frágil}{}{}{"-eis}{}{adj.2g.}{Pouco durável; quebradiço, delicado.}{frá.gil}{0}
\verb{frágil}{}{}{"-eis}{}{}{Fraco, débil.}{frá.gil}{0}
\verb{frágil}{}{}{"-eis}{}{}{Sujeito a erros.}{frá.gil}{0}
\verb{fragilidade}{}{}{}{}{s.f.}{Qualidade de frágil; fraqueza.}{fra.gi.li.da.de}{0}
\verb{fragilizar}{}{}{}{}{v.t.}{Tornar frágil.}{fra.gi.li.zar}{\verboinum{1}}
\verb{fragmentação}{}{}{"-ões}{}{s.f.}{Ato ou efeito de fragmentar.}{frag.men.ta.ção}{0}
\verb{fragmentar}{}{}{}{}{v.t.}{Fazer em fragmentos; fracionar, quebrar.}{frag.men.tar}{\verboinum{1}}
\verb{fragmentário}{}{}{}{}{adj.}{Relativo a fragmento.}{frag.men.tá.rio}{0}
\verb{fragmento}{}{}{}{}{s.m.}{Parte do todo que foi dividido ou se quebrou; fração, pedaço.}{frag.men.to}{0}
\verb{fragmento}{}{}{}{}{}{Trecho extraído de uma obra.}{frag.men.to}{0}
\verb{fragor}{ô}{}{}{}{s.m.}{Barulho semelhante ao de algo se quebrando.}{fra.gor}{0}
\verb{fragoroso}{ô}{}{"-osos ⟨ó⟩}{"-osa ⟨ó⟩}{adj.}{Que produz fragor.}{fra.go.ro.so}{0}
\verb{fragoroso}{ô}{Fig.}{"-osos ⟨ó⟩}{"-osa ⟨ó⟩}{}{Que causa estardalhaço, que tem forte repercussão.}{fra.go.ro.so}{0}
\verb{fragoso}{ô}{}{"-osos ⟨ó⟩}{"-osa ⟨ó⟩}{adj.}{Cheio de fragas; rochoso.}{fra.go.so}{0}
\verb{fragrância}{}{}{}{}{s.f.}{Aroma, perfume.}{fra.grân.cia}{0}
\verb{fragrante}{}{}{}{}{adj.2g.}{Que exala cheiro agradável; perfumado, aromático.}{fra.gran.te}{0}
\verb{frajola}{ó}{Pop.}{}{}{adj.}{Diz"-se de indivíduo muito bem vestido, elegante.}{fra.jo.la}{0}
\verb{fralda}{}{}{}{}{s.f.}{Peça de pano macio ou material descartável que se coloca no bebê para reter urina e fezes.}{fral.da}{0}
\verb{fralda}{}{}{}{}{}{A parte inferior da camisa.}{fral.da}{0}
\verb{framboesa}{ê}{}{}{}{s.f.}{Fruto da framboeseira.}{fram.bo.e.sa}{0}
\verb{framboeseira}{ê}{Bot.}{}{}{s.f.}{Arbusto com flores brancas e frutos aromáticos e comestíveis geralmente vermelhos.}{fram.bo.e.sei.ra}{0}
\verb{framboeseiro}{ê}{Bot.}{}{}{s.m.}{Framboeseira.}{fram.bo.e.sei.ro}{0}
\verb{franças}{}{}{}{}{s.f.pl.}{Os ramos mais altos das árvores.}{fran.ças}{0}
\verb{francês}{}{}{}{}{adj.}{Relativo à França.}{fran.cês}{0}
\verb{francês}{}{}{}{}{s.m.}{Indivíduo natural ou habitante desse país.}{fran.cês}{0}
\verb{francesismo}{}{}{}{}{s.m.}{Palavra, expressão ou construção da língua francesa, adotada em outra língua.}{fran.ce.sis.mo}{0}
\verb{francesismo}{}{}{}{}{}{Imitação dos costumes ou das coisas francesas.}{fran.ce.sis.mo}{0}
\verb{frâncio}{}{Quím.}{}{}{s.m.}{Elemento químico radioativo, do grupo dos metais alcalinos. \elemento{87}{(223)}{Fr}.}{frân.cio}{\verboinum{1}}
\verb{franciscano}{}{}{}{}{adj.}{Relativo à ordem de São Francisco de Assis.}{fran.cis.ca.no}{0}
\verb{franciscano}{}{}{}{}{}{Diz"-se de pobreza ou miséria extrema.}{fran.cis.ca.no}{0}
\verb{franciscano}{}{}{}{}{s.m.}{Religioso que pertence à ordem de São Francisco de Assis; frade franciscano.}{fran.cis.ca.no}{0}
\verb{franco}{}{}{}{}{adj.}{Que diz o que pensa; sincero.}{fran.co}{0}
\verb{franco}{}{}{}{}{}{Livre de impostos.}{fran.co}{0}
\verb{franco}{}{}{}{}{}{Sem pagamento; gratuito.}{fran.co}{0}
\verb{franco}{}{}{}{}{s.m.}{Moeda da França, da Bélgica, da Suíça e de outros países europeus.}{fran.co}{0}
\verb{franco"-atirador}{ô}{}{francos"-atiradores ⟨ô⟩}{}{s.m.}{Diz"-se de soldado que não faz parte de uma tropa regular; guerrilheiro.}{fran.co"-a.ti.ra.dor}{0}
%\verb{}{}{}{}{}{}{}{}{0}
\verb{franga}{}{}{}{}{s.f.}{Galinha nova, que ainda não põe ovo.}{fran.ga}{0}
\verb{frangalho}{}{}{}{}{s.m.}{Pedaço de roupa rasgada; farrapo, trapo.}{fran.ga.lho}{0}
\verb{frango}{}{Zool.}{}{}{s.m.}{Galo ainda novo.}{fran.go}{0}
\verb{frango}{}{Pop.}{}{}{}{Bola que, embora fácil de defender, o goleiro deixa passar.}{fran.go}{0}
\verb{frangote}{ó}{}{}{}{s.m.}{Frango pequeno.}{fran.go.te}{0}
\verb{frangote}{ó}{Fig.}{}{}{}{Rapazinho, adolescente.}{fran.go.te}{0}
\verb{frangueiro}{ê}{Pop.}{}{}{adj.}{Diz"-se de goleiro que falha em gols defensáveis.}{fran.guei.ro}{0}
\verb{franja}{}{}{}{}{s.f.}{Cabelo que cobre a testa.}{fran.ja}{0}
\verb{franja}{}{}{}{}{}{Conjunto de fios pendentes que enfeitam toalhas, cortinas, roupas e outras peças.}{fran.ja}{0}
\verb{franjar}{}{}{}{}{v.t.}{Pôr franja em uma roupa.}{fran.jar}{\verboinum{1}}
\verb{franquear}{}{}{}{}{v.t.}{Deixar alguma coisa livre para o proveito de alguém; liberar.}{fran.que.ar}{\verboinum{4}}
\verb{franqueza}{ê}{}{}{}{s.f.}{Qualidade de quem é franco; sinceridade.}{fran.que.za}{0}
\verb{franquia}{}{}{}{}{s.f.}{Licença para fabricar um produto, dada pela empresa proprietária.}{fran.qui.a}{0}
\verb{franzido}{}{}{}{}{adj.}{Que se franziu.}{fran.zi.do}{0}
\verb{franzido}{}{}{}{}{s.m.}{Coisa franzida.}{fran.zi.do}{0}
\verb{franzino}{}{}{}{}{adj.}{Que é fraco e pouco desenvolvido.}{fran.zi.no}{0}
\verb{franzir}{}{}{}{}{v.t.}{Fazer dobras pequenas e seguidas numa roupa, reduzindo sua largura.}{fran.zir}{0}
\verb{franzir}{}{}{}{}{}{Fazer parte do corpo ficar cheia de rugas; enrugar, frisar.}{fran.zir}{\verboinum{18}}
\verb{fraque}{}{}{}{}{s.m.}{Traje de cerimônia masculino, bem ajustado ao tronco, curto na frente e com longas abas atrás.}{fra.que}{0}
\verb{fraquear}{}{}{}{}{v.i.}{Fraquejar.}{fra.que.ar}{\verboinum{4}}
\verb{fraquejar}{}{}{}{}{v.i.}{Mostrar"-se fraco; perder as forças.}{fra.que.jar}{0}
\verb{fraquejar}{}{}{}{}{}{Perder a coragem; desencorajar"-se.}{fra.que.jar}{\verboinum{1}}
\verb{fraqueza}{ê}{}{}{}{s.f.}{Falta de força, energia ou resistência; debilidade, fragilidade.}{fra.que.za}{0}
\verb{frasal}{}{Gram.}{"-ais}{}{adj.2g.}{Relativo a frase.}{fra.sal}{0}
\verb{frascaria}{}{}{}{}{s.f.}{Grande quantidade de frascos.}{fras.ca.ri.a}{0}
\verb{frascário}{}{}{}{}{adj.}{Diz"-se de indivíduo que é libertino, leviano, devasso.}{fras.cá.rio}{0}
\verb{frasco}{}{}{}{}{s.m.}{Recipiente para perfumes ou remédios; vidro.}{fras.co}{0}
\verb{frase}{}{}{}{}{s.f.}{Reunião de palavras com sentido completo.}{fra.se}{0}
\verb{fraseado}{}{}{}{}{adj.}{Disposto em frases.}{fra.se.a.do}{0}
\verb{fraseado}{}{}{}{}{s.m.}{Modo próprio de dizer ou de escrever.}{fra.se.a.do}{0}
\verb{frasear}{}{}{}{}{v.i.}{Dispor as ideias em frases.}{fra.se.ar}{\verboinum{4}}
\verb{fraseologia}{}{Gram.}{}{}{s.f.}{Parte da gramática que estuda a construção da frase.}{fra.se.o.lo.gi.a}{0}
\verb{fraseologia}{}{Gram.}{}{}{}{Construção de frase peculiar a uma língua ou a um escritor.}{fra.se.o.lo.gi.a}{0}
\verb{frasqueira}{ê}{}{}{}{s.f.}{Lugar onde se guardam frascos.}{fras.quei.ra}{0}
\verb{frasqueira}{ê}{}{}{}{}{Maleta para transporte de objetos de toalete e miudezas.}{fras.quei.ra}{0}
\verb{fraternal}{}{}{"-ais}{}{adj.2g.}{Próprio de irmão; fraterno.}{fra.ter.nal}{0}
\verb{fraternal}{}{}{"-ais}{}{}{Que demonstra afeição, caridade.}{fra.ter.nal}{0}
\verb{fraternidade}{}{}{}{}{s.f.}{Parentesco de irmãos.}{fra.ter.ni.da.de}{0}
\verb{fraternidade}{}{}{}{}{}{Amor ao próximo.}{fra.ter.ni.da.de}{0}
\verb{fraternizar}{}{}{}{}{v.t.}{Unir com amizade fraterna.}{fra.ter.ni.zar}{\verboinum{1}}
\verb{fraterno}{é}{}{}{}{adj.}{Próprio de irmãos; afetuoso, fraternal.}{fra.ter.no}{0}
\verb{fratricida}{}{}{}{}{adj.2g.}{Diz"-se de quem mata o próprio irmão ou irmã.}{fra.tri.ci.da}{0}
\verb{fratricídio}{}{}{}{}{s.m.}{Assassínio de irmão.}{fra.tri.cí.dio}{0}
\verb{fratura}{}{}{}{}{s.f.}{Ato ou efeito de fraturar; rompimento; quebra.}{fra.tu.ra}{0}
\verb{fraturar}{}{}{}{}{v.t.}{Partir osso, cartilagem dura ou dente.}{fra.tu.rar}{0}
\verb{fraturar}{}{}{}{}{}{Produzir rachadura; fender.}{fra.tu.rar}{\verboinum{1}}
\verb{fraudar}{}{}{}{}{v.t.}{Realizar fraude; iludir, enganar.}{frau.dar}{\verboinum{1}}
\verb{fraudatório}{}{}{}{}{adj.}{Em que há fraude; impostor.}{frau.da.tó.rio}{0}
\verb{fraude}{}{}{}{}{s.f.}{Artifício para enganar.}{frau.de}{0}
\verb{fraude}{}{}{}{}{}{Ato de má"-fé.}{frau.de}{0}
\verb{fraudulento}{}{}{}{}{adj.}{Propenso à fraude. (\textit{Aquela é uma empresa fraudulenta.})}{frau.du.len.to}{0}
\verb{fraudulento}{}{}{}{}{}{Em que há fraude; impostor.}{frau.du.len.to}{0}
\verb{frauta}{}{}{}{}{}{Var. de \textit{flauta}.}{frau.ta}{0}
\verb{freada}{}{}{}{}{s.f.}{Ato de frear, de apertar o freio de um veículo.}{fre.a.da}{0}
\verb{frear}{}{}{}{}{v.t.}{Reprimir, conter.}{fre.ar}{0}
\verb{frear}{}{}{}{}{}{Fazer um veículo parar ou diminuir a velocidade.}{fre.ar}{\verboinum{4}}
\verb{freático}{}{}{}{}{adj.}{Diz"-se de lençol subterrâneo situado em nível pouco profundo e explorado por poços.}{fre.á.ti.co}{0}
\verb{frecha}{é}{}{}{}{}{Var. de \textit{flecha}.}{fre.cha}{0}
\verb{frechada}{}{}{}{}{}{Var. de \textit{flechada}.}{fre.cha.da}{0}
\verb{frechal}{}{}{"-ais}{}{s.m.}{Viga na qual se pregam os caibros à beira do telhado.}{fre.chal}{0}
\verb{frechar}{}{}{}{}{}{Var. de \textit{flechar}.}{fre.char}{0}
\verb{frecheiro}{ê}{}{}{}{}{Var. de \textit{flecheiro}.}{fre.chei.ro}{0}
\verb{free lance}{}{}{}{}{s.m.}{Profissional sem vínculo empregatício.}{\textit{free lance}}{0}
\verb{freeware}{}{Informát.}{}{}{s.m.}{Programa que se pode ter sem pagamento nenhum.}{\textit{freeware}}{0}
\verb{freezer}{}{}{freezers}{}{s.m.}{Eletrodoméstico para o congelamento de alimentos. (\textit{A carne está congelada no }freezer\textit{.})}{\textit{freezer}}{0}
\verb{frege}{é}{}{}{}{s.m.}{Desordem, briga, bagunça.}{fre.ge}{0}
\verb{freguês}{}{}{}{}{s.m.}{Pessoa que tem o costume de comprar em determinada casa comercial; cliente. (\textit{Meu pai é freguês daquele empório.})}{fre.guês}{0}
\verb{freguesia}{}{}{}{}{s.f.}{Conjunto de fregueses de uma casa comercial; clientela.}{fre.gue.si.a}{0}
\verb{frei}{}{Relig.}{}{sóror}{s.m.}{Forma abreviada da antiga palavra \textit{freire}, sempre usada antes do nome; irmão em ordem religiosa. (\textit{Ontem, assistimos à missa celebrada pelo Frei Betto.})}{frei}{0}
\verb{freio}{ê}{}{}{}{s.m.}{Dispositivo com que se faz o veículo parar; breque.}{frei.o}{0}
\verb{freio}{ê}{}{}{}{}{Peça de metal que se põe na boca do animal para o dirigir ou o fazer parar.}{frei.o}{0}
\verb{freira}{ê}{}{}{}{s.f.}{Mulher que pertence a uma ordem religiosa.}{frei.ra}{0}
\verb{freire}{}{Relig.}{}{}{s.m.}{Religioso, membro das antigas ordens religiosas e militares; frei.}{frei.re}{0}
\verb{freixo}{ch}{Bot.}{}{}{s.m.}{Planta própria dos climas temperados, de madeira clara, macia e resistente.}{frei.xo}{0}
\verb{fremente}{}{}{}{}{adj.2g.}{Que freme, que agita; agitado, trêmulo.}{fre.men.te}{0}
\verb{fremente}{}{Fig.}{}{}{}{Provido de emoção; apaixonado, vibrante.}{fre.men.te}{0}
\verb{fremir}{}{}{}{}{v.i.}{Fazer ruído surdo e áspero.}{fre.mir}{0}
\verb{fremir}{}{}{}{}{}{Vibrar, tremer.}{fre.mir}{0}
\verb{fremir}{}{Fig.}{}{}{}{Estremecer de alegria.}{fre.mir}{\verboinum{34}\verboirregular{\emph{def.} fremimos, fremir}}
\verb{frêmito}{}{}{}{}{s.m.}{Ruído, rumor.}{frê.mi.to}{0}
\verb{frêmito}{}{}{}{}{}{Estremecimento, agitação.}{frê.mi.to}{0}
\verb{frenar}{}{}{}{}{v.t.}{Frear, conter.}{fre.nar}{0}
\verb{frenar}{}{Fig.}{}{}{}{Moderar, reprimir.}{fre.nar}{\verboinum{1}}
\verb{frenesi}{}{}{}{}{s.m.}{Delírio, desvario.}{fre.ne.si}{0}
\verb{frenesi}{}{}{}{}{}{Excitação, arrebatamento.}{fre.ne.si}{0}
\verb{frenesim}{}{}{}{}{s.m.}{Frenesi.}{fre.ne.sim}{0}
\verb{frenético}{}{}{}{}{adj.}{Que tem frenesi; delirante.}{fre.né.ti.co}{0}
\verb{frenético}{}{}{}{}{}{Exaltado, agitado.}{fre.né.ti.co}{0}
\verb{frente}{}{}{}{}{s.f.}{Parte dianteira de alguma coisa.}{fren.te}{0}
\verb{frente}{}{}{}{}{}{Face, rosto, cara.}{fren.te}{0}
\verb{frente}{}{}{}{}{}{Linha de frente; vanguarda.}{fren.te}{0}
\verb{frentista}{}{}{}{}{s.2g.}{Indivíduo que trabalha em posto de gasolina como atendente.}{fren.tis.ta}{0}
\verb{frequência}{}{}{}{}{s.f.}{Ato ou efeito de frequentar; frequentação; assiduidade.}{fre.quên.cia}{0}
\verb{frequência}{}{}{}{}{}{Número de vezes em que algo se repete em um intervalo de tempo; periodicidade.}{fre.quên.cia}{0}
\verb{frequentação}{}{}{"-ões}{}{s.f.}{Ato ou efeito de frequentar; assiduidade, frequência.}{fre.quen.ta.ção}{0}
\verb{frequentador}{ô}{}{}{}{adj.}{Que frequenta habitualmente um lugar. }{fre.quen.ta.dor}{0}
\verb{frequentar}{}{}{}{}{v.t.}{Ir muitas vezes a algum lugar.}{fre.quen.tar}{0}
\verb{frequentar}{}{}{}{}{}{Ter familiaridade, intimidade; conviver.}{fre.quen.tar}{0}
\verb{frequentar}{}{}{}{}{}{Cursar regularmente; seguir aula, disciplina.}{fre.quen.tar}{\verboinum{1}}
\verb{frequente}{}{}{}{}{adj.2g.}{Que se repete muito; continuado, assíduo.}{fre.quen.te}{0}
\verb{fresa}{é}{}{}{}{s.f.}{Ferramenta empregada para desbastar ou cortar madeira ou metais.}{fre.sa}{0}
\verb{fresar}{}{}{}{}{v.t.}{Desbastar ou cortar madeira ou metais.}{fre.sar}{\verboinum{1}}
\verb{fresca}{ê}{}{}{}{s.f.}{Brisa amena e agradável que sopra de manhã ou ao entardecer.}{fres.ca}{0}
\verb{fresco}{ê}{}{}{}{adj.}{Ligeiramente frio; entre morno e frio.}{fres.co}{0}
\verb{fresco}{ê}{}{}{}{}{Que não está murcho; viçoso, verdejante.}{fres.co}{0}
\verb{fresco}{ê}{}{}{}{}{Que não está estragado; conservado.}{fres.co}{0}
\verb{fresco}{ê}{}{}{}{}{Que foi feito há pouco tempo; recente.}{fres.co}{0}
\verb{frescobol}{ó}{Esport.}{}{}{s.m.}{Modalidade esportiva semelhante ao tênis, praticada ao ar livre, geralmente nas praias. }{fres.co.bol}{0}
\verb{frescor}{ô}{}{}{}{s.m.}{Qualidade do que é fresco; frescura, brisa amena.}{fres.cor}{0}
\verb{frescura}{}{}{}{}{s.f.}{Qualidade do que é fresco; frescor.}{fres.cu.ra}{0}
\verb{frescura}{}{Pop.}{}{}{}{Sentimentalismo demasiado; pieguice.}{fres.cu.ra}{0}
\verb{fressura}{}{}{}{}{s.f.}{Conjunto de vísceras de um animal; bofes.}{fres.su.ra}{0}
\verb{fresta}{é}{}{}{}{s.f.}{Pequena abertura em uma superfície; fenda, greta.}{fres.ta}{0}
\verb{fretamento}{}{}{}{}{s.m.}{Ato ou efeito de fretar, alugar.}{fre.ta.men.to}{0}
\verb{fretar}{}{}{}{}{v.t.}{Alugar um veículo ou embarcação para transporte de carga ou de passageiros.}{fre.tar}{\verboinum{1}}
\verb{frete}{é}{}{}{}{s.m.}{Quantia que se paga pelo transporte de algo.}{fre.te}{0}
\verb{frete}{é}{}{}{}{}{Transporte de mercadoria por via marítima ou fluvial.}{fre.te}{0}
\verb{frete}{é}{}{}{}{}{A carga transportada.}{fre.te}{0}
\verb{freudiano}{}{}{}{}{adj.}{Relativo a Freud, médico austríaco fundador da psicanálise.}{freu.di.a.no}{0}
\verb{freudiano}{}{}{}{}{}{Que segue os métodos e os conceitos de Freud.}{freu.di.a.no}{0}
\verb{frevo}{ê}{Mús.}{}{}{s.m.}{Dança e música popular do Nordeste, especialmente do Recife, de ritmo rápido e coreografia individual.}{fre.vo}{0}
\verb{fria}{}{}{}{}{s.f.}{Situação difícil, embaraçosa.}{fri.a}{0}
\verb{friagem}{}{}{"-ens}{}{s.f.}{Queda repentina da temperatura; vento frio.}{fri.a.gem}{0}
\verb{frialdade}{}{}{}{}{s.f.}{Qualidade do que apresenta baixa temperatura; friagem, frieza.}{fri.al.da.de}{0}
\verb{friável}{}{}{"-eis}{}{adj.2g.}{Que se fragmenta ou se esfacela facilmente.}{fri.á.vel}{0}
\verb{fricassê}{}{Cul.}{}{}{s.m.}{Guisado de frango ou peixe partido em pequenos pedaços, cozidos em fogo brando com cebola, salsa e outros temperos.}{fri.cas.sê}{0}
\verb{fricção}{}{}{"-ões}{}{s.f.}{Ato ou efeito de friccionar; atrito, esfregação.}{fric.ção}{0}
\verb{friccionar}{}{}{}{}{v.t.}{Fazer fricção com líquido ou pasta; esfregar. (\textit{O massagista friccionou com óleo as costas do cliente.})}{fric.ci.o.nar}{0}
\verb{friccionar}{}{}{}{}{}{Atritar, roçar, esfregar.}{fric.ci.o.nar}{\verboinum{1}}
\verb{fricote}{ó}{}{}{}{s.m.}{Chilique nervoso; manha, fingimento, dengo.}{fri.co.te}{0}
\verb{fricoteiro}{ê}{}{}{}{adj.}{Que tem fricotes; manhoso, dengoso.}{fri.co.tei.ro}{0}
\verb{frieira}{ê}{Med.}{}{}{s.f.}{Inflamação cutânea, caracterizada de bolhas e rachaduras, localizada nos pés, especialmente entre os dedos.}{fri.ei.ra}{0}
\verb{frieza}{ê}{}{}{}{s.f.}{Qualidade do que é frio; frialdade, friagem.}{fri.e.za}{0}
\verb{frieza}{ê}{}{}{}{}{Falta de amabilidade; indiferença.}{fri.e.za}{0}
\verb{frigideira}{ê}{}{}{}{s.f.}{Panela larga e rasa, com cabo comprido, própria para fritar, frigir.}{fri.gi.dei.ra}{0}
\verb{frigidez}{ê}{}{}{}{s.f.}{Qualidade de frígido; frieza.}{fri.gi.dez}{0}
\verb{frigidez}{ê}{}{}{}{}{Impotência sexual feminina; ausência de prazer.}{fri.gi.dez}{0}
\verb{frígido}{}{}{}{}{adj.}{Muito frio; gélido.}{frí.gi.do}{0}
\verb{frígido}{}{}{}{}{}{Sem desejo sexual; insensível.}{frí.gi.do}{0}
\verb{frígio}{}{}{}{}{adj.}{Relativo à Frígia, antiga região da Ásia Menor.}{frí.gio}{0}
\verb{frígio}{}{}{}{}{s.m.}{Indivíduo natural ou habitante dessa região.}{frí.gio}{0}
\verb{frígio}{}{}{}{}{}{Barrete em forma de capacete, usado na França durante a primeira república.}{frí.gio}{0}
\verb{frigir}{}{}{}{}{v.t.}{Cozinhar em óleo, gordura ou manteiga muito quente.}{fri.gir}{\verboinum{59}}
\verb{frigobar}{}{}{}{}{s.m.}{Pequena geladeira usada em quartos de hotéis.}{fri.go.bar}{0}
\verb{frigorífico}{}{}{}{}{adj.}{Que produz ou conserva o frio.}{fri.go.rí.fi.co}{0}
\verb{frigorífico}{}{}{}{}{s.m.}{Aparelho para conservar os alimentos gelados.}{fri.go.rí.fi.co}{0}
\verb{frila}{}{Pop.}{}{}{s.2g.}{Forma reduzida de \textit{free lance}.}{fri.la}{0}
\verb{frincha}{}{}{}{}{s.f.}{Pequena abertura; greta, fenda, fresta.}{frin.cha}{0}
\verb{frio}{}{}{}{}{adj.}{Que não tem ou que perdeu calor.}{fri.o}{0}
\verb{frio}{}{}{}{}{}{Isento de ânimo, de paixão; insensível, impassível.}{fri.o}{0}
\verb{frioleira}{ê}{}{}{}{s.f.}{Acontecimento sem importância; futilidade, ninharia, insignificância.}{fri.o.lei.ra}{0}
\verb{friorento}{}{}{}{}{adj.}{Que é muito sensível ao frio.}{fri.o.ren.to}{0}
\verb{frios}{}{}{}{}{s.m.pl.}{Produtos feitos geralmente à base de carne de porco, como a mortadela, o presunto etc., e que são consumidos frios. (\textit{Meu amigo serviu uma tábua de frios em sua festa.})}{fri.os}{0}
\verb{frisa}{}{}{}{}{s.f.}{Nos teatros e nas salas de espetáculo, camarote que se situa um pouco acima do nível da plateia.}{fri.sa}{0}
\verb{frisado}{}{}{}{}{adj.}{Diz"-se do cabelo artificialmente encrespado.}{fri.sa.do}{0}
\verb{frisador}{ô}{}{}{}{adj.}{Diz"-se do aparelho usado para frisar, encaracolar o cabelo.}{fri.sa.dor}{0}
\verb{frisador}{ô}{}{}{}{}{Diz"-se do instrumento usado para frisar o tecido.}{fri.sa.dor}{0}
\verb{frisante}{}{}{}{}{adj.2g.}{Que torna crespo, frisa.}{fri.san.te}{0}
\verb{frisante}{}{}{}{}{}{Diz"-se da bebida capaz de produzir borbulhas; borbulhante. }{fri.san.te}{0}
\verb{frisar}{}{}{}{}{v.t.}{Encrespar, anelar o cabelo de forma natural ou artificial.}{fri.sar}{\verboinum{1}}
\verb{frisar}{}{}{}{}{v.t.}{Colocar friso. (\textit{Pedi à costureira que frisasse meu vestido.})}{fri.sar}{0}
\verb{frisar}{}{}{}{}{}{Salientar, enfatizar, sublinhar.}{fri.sar}{\verboinum{1}}
\verb{friso}{}{}{}{}{s.m.}{Faixa ou tira ao longo de uma parede usada como ornamento. (\textit{O banheiro tinha um friso de azulejos amarelos.})}{fri.so}{0}
\verb{fritada}{}{}{}{}{s.f.}{Quantidade de algo que se frita de uma vez. (\textit{Minha tia preparou uma fritada de camarão.})}{fri.ta.da}{0}
\verb{fritar}{}{}{}{}{v.t.}{Cozinhar na manteiga, no azeite ou no óleo; frigir.}{fri.tar}{\verboinum{1}}
\verb{fritas}{}{Cul.}{}{}{s.f.pl.}{Batatas fritas. (\textit{Eu pedi ao garçom um filé com fritas.})}{fri.tas}{0}
\verb{frito}{}{}{}{}{adj.}{Que se fritou ou frigiu.}{fri.to}{0}
\verb{frito}{}{Pop.}{}{}{}{Em situação difícil; em apuros; em maus lençóis.}{fri.to}{0}
\verb{fritura}{}{}{}{}{s.f.}{Qualquer alimento frito; fritada. (\textit{O médico recomendou à minha mãe que ela evitasse comer fritura.})}{fri.tu.ra}{0}
\verb{friúra}{}{}{}{}{s.f.}{Estado ou qualidade do que é ou está frio; frialdade, frieza. }{fri.ú.ra}{0}
\verb{frivolidade}{}{}{}{}{s.f.}{Qualidade ou ato de frívolo; futilidade, inconsistência.}{fri.vo.li.da.de}{0}
\verb{frívolo}{}{}{}{}{adj.}{Que não tem importância; sem valor, inútil, superficial.}{frí.vo.lo}{0}
\verb{frívolo}{}{}{}{}{}{Fútil, leviano, volúvel.}{frí.vo.lo}{0}
%\verb{}{}{}{}{}{}{}{}{0}
\verb{fronde}{}{}{}{}{s.f.}{Conjunto de galhos e folhas na parte superior das árvores; folhagem, ramagem.}{fron.de}{0}
\verb{frondoso}{ô}{}{"-osos ⟨ó⟩}{"-osa ⟨ó⟩}{adj.}{Que tem muitas folhas; copado, abundante. (\textit{A mangueira é uma árvore frondosa.})}{fron.do.so}{0}
\verb{fronha}{ô}{}{}{}{s.f.}{Capa que envolve o travesseiro. (\textit{Um jogo de cama de casal geralmente contém dois lençóis e duas fronhas.})}{fro.nha}{0}
\verb{front}{}{}{}{}{s.m.}{Frente de batalha.}{\textit{front}}{0}
\verb{frontal}{}{}{"-ais}{}{adj.2g.}{Relativo à fronte ou a frente.}{fron.tal}{0}
\verb{frontal}{}{Anat.}{"-ais}{}{}{Diz"-se do osso da testa.}{fron.tal}{0}
\verb{frontão}{}{}{"-ões}{}{s.m.}{Enfeite em forma de triângulo ou meio círculo em cima de porta ou janela ou que coroa a entrada principal de um edifício.}{fron.tão}{0}
\verb{frontaria}{}{}{}{}{s.f.}{Fachada de um edifício; frente, frontispício.}{fron.ta.ri.a}{0}
\verb{fronte}{}{}{}{}{s.f.}{Parte da frente da cabeça; testa. (\textit{O rapaz inclinou a fronte para receber a bênção do pai.})}{fron.te}{0}
\verb{fronteira}{ê}{}{}{}{s.f.}{Linha divisória entre dois países, estados, municípios, regiões; divisa, limite.}{fron.tei.ra}{0}
\verb{fronteira}{ê}{}{}{}{}{Região que fica perto dessa linha. (\textit{Meus tios moravam na fronteira com o Uruguai.})}{fron.tei.ra}{0}
\verb{fronteiriço}{}{}{}{}{adj.}{Que vive ou se situa na fronteira.}{fron.tei.ri.ço}{0}
\verb{fronteiro}{ê}{}{}{}{adj.}{Situado à frente; em face.}{fron.tei.ro}{0}
\verb{frontispício}{}{}{}{}{s.m.}{Nos livros, a folha de rosto.}{fron.tis.pí.cio}{0}
\verb{frontispício}{}{}{}{}{}{Fachada, frontaria.}{fron.tis.pí.cio}{0}
\verb{frota}{ó}{}{}{}{s.f.}{Conjunto de navios de guerra; esquadra.}{fro.ta}{0}
\verb{frota}{ó}{}{}{}{}{Conjunto de veículos de uma empresa ou corporação. (\textit{A prefeitura aumentou em 50\% a frota de ônibus municipais nos últimos meses.})}{fro.ta}{0}
\verb{frouxidão}{ch}{}{"-ões}{}{s.f.}{Qualidade de frouxo; falta de atividade, moleza.}{frou.xi.dão}{0}
\verb{frouxo}{ch}{}{}{}{adj.}{Que não está muito apertado; folgado, solto.}{frou.xo}{0}
\verb{frouxo}{ch}{}{}{}{}{Que não tem coragem; sem energia moral, desanimado, fraco.}{frou.xo}{0}
\verb{frouxo}{ch}{}{}{}{}{Medroso, covarde.}{frou.xo}{0}
\verb{frufru}{}{}{}{}{s.m.}{Rumor de folhas, vestidos ou asas em voo.}{fru.fru}{0}
\verb{frugal}{}{}{"-ais}{}{adj.2g.}{Que se contenta com pouco alimento; sóbrio, moderado.}{fru.gal}{0}
\verb{frugalidade}{}{}{}{}{s.f.}{Qualidade de frugal; moderação, sobriedade.}{fru.ga.li.da.de}{0}
\verb{frugívoro}{}{}{}{}{adj.}{Que se alimenta de frutos. (\textit{Os pássaros são animais frugívoros.})}{fru.gí.vo.ro}{0}
\verb{fruição}{}{}{"-ões}{}{s.f.}{Ato ou efeito de fruir; gozo, desfrutamento.}{fru.i.ção}{0}
\verb{fruir}{}{}{}{}{v.t.}{Tirar proveito; gozar, desfrutar.}{fru.ir}{\verboinum{26}}
\verb{frumento}{}{}{}{}{s.m.}{Trigo selecionado.}{fru.men.to}{0}
\verb{frustração}{}{}{"-ões}{}{s.f.}{Ato ou efeito de frustrar; decepção. (\textit{O filme daquele diretor famoso foi uma frustração.})}{frus.tra.ção}{0}
\verb{frustração}{}{}{"-ões}{}{}{Estado de quem sofre por deixar de ter ou de fazer o que queria.}{frus.tra.ção}{0}
\verb{frustrante}{}{}{}{}{adj.2g.}{Que frustra; decepcionante. (\textit{O final da novela foi frustrante.})}{frus.tran.te}{0}
\verb{frustrar}{}{}{}{}{v.t.}{Enganar a expectativa; decepcionar, desiludir.}{frus.trar}{0}
\verb{frustrar}{}{}{}{}{}{Falhar, inutilizar, baldar.}{frus.trar}{\verboinum{1}}
\verb{fruta}{}{Bot.}{}{}{s.f.}{O fruto comestível dos vegetais. (\textit{Minha mãe ganhou uma cesta de frutas do mercado em que ela faz compras regularmente.})}{fru.ta}{0}
%\verb{}{}{}{}{}{}{}{}{0}
\verb{fruta"-do"-conde}{}{}{frutas"-do"-conde}{}{s.f.}{Fruta de casca rugosa e com sementes recobertas por uma polpa doce e macia; pinha, ata.}{fru.ta"-do"-con.de}{0}
\verb{fruta"-pão}{}{Bot.}{frutas"-pães \textit{ou} frutas"-pão}{}{s.f.}{Árvore da família da jaqueira cultivada por seus múltiplos usos medicinais, para extração de fibras da casca e por seus frutos comestíveis. }{fru.ta"-pão}{0}
\verb{fruta"-pão}{}{}{frutas"-pães \textit{ou} frutas"-pão}{}{}{Fruto dessa árvore, comestível e que, depois de cozido, apresenta uma massa cuja consistência se assemelha ao pão.}{fru.ta"-pão}{0}
\verb{fruteira}{ê}{}{}{}{s.f.}{Árvore que dá frutos.}{fru.tei.ra}{0}
\verb{fruteira}{ê}{}{}{}{}{Vaso ou cesto em que se colocam frutas à mesa.}{fru.tei.ra}{0}
\verb{fruteira}{ê}{}{}{}{}{Vendedora de frutas.}{fru.tei.ra}{0}
\verb{fruteiro}{ê}{}{}{}{adj.}{Que gosta de frutas.}{fru.tei.ro}{0}
\verb{fruteiro}{ê}{}{}{}{s.m.}{Vendedor de frutas.}{fru.tei.ro}{0}
\verb{fruteiro}{ê}{}{}{}{}{Recipiente onde se guardam frutas; fruteira.}{fru.tei.ro}{0}
\verb{fruticultor}{ô}{}{}{}{s.m.}{Indivíduo que cultiva árvores frutíferas.}{fru.ti.cul.tor}{0}
\verb{fruticultura}{}{}{}{}{s.f.}{Cultivo de árvores frutíferas.}{fru.ti.cul.tu.ra}{0}
\verb{frutífero}{}{}{}{}{adj.}{Que produz frutos.}{fru.tí.fe.ro}{0}
\verb{frutificação}{}{}{"-ões}{}{s.f.}{Ato ou efeito de frutificar; formação dos frutos.}{fru.ti.fi.ca.ção}{0}
\verb{frutificar}{}{}{}{}{v.i.}{Produzir frutos.}{fru.ti.fi.car}{0}
\verb{frutificar}{}{Fig.}{}{}{}{Dar bons resultados.}{fru.ti.fi.car}{\verboinum{2}}
\verb{frutívoro}{}{}{}{}{adj.}{Que se alimenta de frutos; frugívoro.}{fru.tí.vo.ro}{0}
\verb{fruto}{}{Bot.}{}{}{s.m.}{Parte da planta que resulta do ovário fecundado e desenvolvido da flor e que contém a semente.}{fru.to}{0}
\verb{fruto}{}{}{}{}{}{O que a terra produz para benefício do ser humano.}{fru.to}{0}
\verb{fruto}{}{Fig.}{}{}{}{Resultado, proveito, vantagem.}{fru.to}{0}
\verb{frutose}{ó}{}{}{}{s.f.}{Açúcar encontrado no mel e nas frutas.}{fru.to.se}{0}
\verb{frutuoso}{ô}{}{"-osos ⟨ó⟩}{"-osa ⟨ó⟩}{adj.}{Que dá muitos frutos.}{fru.tu.o.so}{0}
\verb{frutuoso}{ô}{Fig.}{"-osos ⟨ó⟩}{"-osa ⟨ó⟩}{}{Que dá bons resultados; vantajoso, proveitoso.}{fru.tu.o.so}{0}
\verb{fuá}{}{}{}{}{s.m.}{Comentário maldoso; fuxico, intriga.}{fu.á}{0}
\verb{fuá}{}{}{}{}{}{Baderna, confusão, desordem.}{fu.á}{0}
\verb{fubá}{}{}{}{}{s.m.}{Farinha de milho.}{fu.bá}{0}
\verb{fubeca}{é}{Pop.}{}{}{s.f.}{Fubecada.}{fu.be.ca}{0}
\verb{fubecada}{}{Pop.}{}{}{s.f.}{Ato de bater em uma pessoa; surra, pancada.}{fu.be.ca.da}{0}
\verb{fubecada}{}{Pop.}{}{}{}{Reprimenda, descompostura.}{fu.be.ca.da}{0}
\verb{fuça}{}{}{}{}{s.f.}{A parte anterior da cabeça de determinados animais; focinho. }{fu.ça}{0}
\verb{fuça}{}{}{}{}{}{Cada uma das ventas de um animal.}{fu.ça}{0}
\verb{fuça}{}{}{}{}{}{Fisionomia de alguém; rosto, cara. }{fu.ça}{0}
%\verb{}{}{}{}{}{}{}{}{0}
\verb{fuçar}{}{}{}{}{v.t.}{Fazer buracos com o focinho; fossar.}{fu.çar}{0}
\verb{fuçar}{}{}{}{}{}{Bisbilhotar, remexer, fuxicar.}{fu.çar}{\verboinum{3}}
%\verb{}{}{}{}{}{}{}{}{0}
\verb{fúcsia}{}{Bot.}{}{}{s.f.}{Planta ornamental cuja flor apresenta o cálice maior que as pétalas e longos estames; brinco"-de"-princesa. }{fúc.si.a}{0}
\verb{fúcsia}{}{}{}{}{}{A cor presente no cálice da flor dessa planta, com tom de cor"-de"-rosa forte, próximo ao magenta.}{fúc.si.a}{0}
\verb{fueiro}{ê}{}{}{}{s.m.}{Estaca para amparar a carga no carro de bois.}{fu.ei.ro}{0}
\verb{fuga}{}{}{}{}{s.f.}{Ato ou efeito de fugir; escapada, retirada.}{fu.ga}{0}
\verb{fugacidade}{}{}{}{}{s.f.}{Qualidade do que é fugaz; transitoriedade, efemeridade.}{fu.ga.ci.da.de}{0}
\verb{fugaz}{}{}{}{}{adj.2g.}{Que dura pouco tempo; transitório, efêmero, passageiro.}{fu.gaz}{0}
\verb{fugida}{}{}{}{}{s.f.}{Ato ou efeito de sair e voltar rapidamente; escapada, fuga.}{fu.gi.da}{0}
\verb{fugidio}{}{}{}{}{adj.}{Que some rapidamente; esquivo, arisco.}{fu.gi.di.o}{0}
\verb{fugir}{}{}{}{}{v.i.}{Retirar"-se rapidamente para se livrar de um perigo ou de alguém.}{fu.gir}{0}
\verb{fugir}{}{}{}{}{}{Passar rapidamente.}{fu.gir}{0}
\verb{fugir}{}{}{}{}{}{Desviar"-se, afastar"-se.}{fu.gir}{0}
\verb{fugir}{}{}{}{}{}{Sair de onde se encontrava preso; escapar.}{fu.gir}{\verboinum{33}}
\verb{fugitivo}{}{}{}{}{adj.}{Que foge; desertor.}{fu.gi.ti.vo}{0}
\verb{fuinha}{}{Zool.}{}{}{s.f.}{Animal mamífero carnívoro de focinho muito fino, que vive da caça.}{fu.i.nha}{0}
\verb{fujão}{}{}{"-ões}{}{adj.}{Diz"-se do indivíduo que foge com frequência ou que já fugiu.}{fu.jão}{0}
\verb{fulano}{}{}{}{}{s.m.}{Nome vago dado para se referir a um indivíduo que não se quer nomear.}{fu.la.no}{0}
\verb{fulano}{}{}{}{}{}{Indivíduo desconhecido.}{fu.la.no}{0}
\verb{fulcro}{}{}{}{}{s.m.}{Ponto de apoio; base, suporte.}{ful.cro}{0}
\verb{fulcro}{}{Fís.}{}{}{}{Ponto de apoio da alavanca.}{ful.cro}{0}
\verb{fuleiro}{ê}{Pop.}{}{}{adj.}{Que não tem valor; medíocre, reles. (\textit{O aluno fez um trabalho de ciências bem fuleiro.})}{fu.lei.ro}{0}
\verb{fulgente}{}{}{}{}{adj.2g.}{Que tem fulgor; que brilha muito; fúlgido, fulgurante.}{ful.gen.te}{0}
\verb{fúlgido}{}{}{}{}{adj.}{Fulgente.}{fúl.gi.do}{0}
\verb{fulgir}{}{}{}{}{v.i.}{Brilhar muito; fulgurar, resplandecer.}{ful.gir}{\verboinum{34}}
\verb{fulgor}{ô}{}{}{}{s.m.}{Brilho intenso; clarão, esplendor.}{ful.gor}{0}
\verb{fulguração}{}{}{"-ões}{}{s.f.}{Clarão intenso resultante da eletricidade que se manifesta na atmosfera, diferente do relâmpago por não vir acompanhado de trovão.}{ful.gu.ra.ção}{0}
\verb{fulgurante}{}{}{}{}{adj.2g.}{Que fulgura; cintilante, resplandecente.}{ful.gu.ran.te}{0}
\verb{fulgurar}{}{}{}{}{v.i.}{Brilhar rápida e intensamente como um relâmpago; relampejar, cintilar.}{ful.gu.rar}{0}
\verb{fulgurar}{}{}{}{}{}{Brilhar muito; resplandecer, fulgir.}{ful.gu.rar}{\verboinum{1}}
\verb{fuligem}{}{}{"-ens}{}{s.f.}{Camada de substância preta deixada pela fumaça de chaminés.}{fu.li.gem}{0}
\verb{fuliginoso}{ô}{}{"-osos ⟨ó⟩}{"-osa ⟨ó⟩}{adj.}{Que está coberto pela fuligem; enegrecido, escuro.}{fu.li.gi.no.so}{0}
\verb{fulminante}{}{}{}{}{adj.2g.}{Que fulmina; que mata instantaneamente.}{ful.mi.nan.te}{0}
\verb{fulminar}{}{}{}{}{v.t.}{Matar à maneira de um raio.}{ful.mi.nar}{0}
\verb{fulminar}{}{}{}{}{}{Matar instantaneamente.}{ful.mi.nar}{0}
\verb{fulminar}{}{}{}{}{}{Deixar sem ação; aniquilar.}{ful.mi.nar}{\verboinum{1}}
\verb{fulo}{}{}{}{}{adj.}{Que muda de cor por causa da raiva; irritado, raivoso.}{fu.lo}{0}
\verb{fulvo}{}{}{}{}{adj.}{De cor amarelo escuro.}{ful.vo}{0}
\verb{fumaça}{}{}{}{}{s.f.}{Grande massa de gases que sai de alguma coisa que queima. (\textit{Sai muita fumaça da chaminé da fábrica.})}{fu.ma.ça}{0}
\verb{fumaçar}{}{}{}{}{v.i.}{Produzir fumaça.}{fu.ma.çar}{0}
\verb{fumaçar}{}{}{}{}{}{Dificultar a visão.}{fu.ma.çar}{\verboinum{3}}
\verb{fumaceira}{ê}{}{}{}{s.f.}{Grande quantidade de fumaça.}{fu.ma.cei.ra}{0}
\verb{fumacento}{}{}{}{}{adj.}{Que solta grande quantidade de fumo ou fumaça.}{fu.ma.cen.to}{0}
\verb{fumada}{}{}{}{}{s.f.}{Tragada que o fumante tira de uma vez do cigarro, do cachimbo ou do charuto.}{fu.ma.da}{0}
\verb{fumada}{}{}{}{}{}{Fumo utilizado como meio de sinalização; fumaça.}{fu.ma.da}{0}
\verb{fumante}{}{}{}{}{adj.2g.}{Diz"-se de indivíduo que tem o costume de fumar. (\textit{Meu pai é fumante.})}{fu.man.te}{0}
\verb{fumar}{}{}{}{}{v.t.}{Puxar para dentro dos pulmões e soltar em seguida a fumaça do cigarro.}{fu.mar}{0}
\verb{fumar}{}{}{}{}{v.i.}{Soltar fumaça; fumegar.}{fu.mar}{\verboinum{1}}
\verb{fumarento}{}{}{}{}{adj.}{Que lança fumo ou fumaça.}{fu.ma.ren.to}{0}
\verb{fumê}{}{}{}{}{adj.}{Diz"-se de cor próxima do cinza"-escuro, como se resultasse de esfumaceamento.}{fu.mê}{0}
\verb{fumegante}{}{}{}{}{adj.2g.}{Que fumega, solta fumo; fumante.}{fu.me.gan.te}{0}
\verb{fumegante}{}{Por ext.}{}{}{}{Muito quente.}{fu.me.gan.te}{0}
\verb{fumegar}{}{}{}{}{v.i.}{Soltar fumaça.}{fu.me.gar}{0}
\verb{fumegar}{}{}{}{}{}{Soltar vapor por estar fervendo.}{fu.me.gar}{\verboinum{5}}
\verb{fumeiro}{ê}{}{}{}{s.m.}{Chaminé.}{fu.mei.ro}{0}
\verb{fumeiro}{ê}{}{}{}{}{Espaço entre a lareira ou o fogão e o telhado onde se põem alimentos para defumar.}{fu.mei.ro}{0}
\verb{fumicultor}{ô}{}{}{}{s.m.}{Indivíduo que cultiva fumo ou tabaco.}{fu.mi.cul.tor}{0}
\verb{fumicultura}{}{}{}{}{s.f.}{Cultura do fumo ou do tabaco.}{fu.mi.cul.tu.ra}{0}
\verb{fumigação}{}{}{"-ões}{}{s.f.}{Ato ou efeito de fumigar.}{fu.mi.ga.ção}{0}
\verb{fumigar}{}{}{}{}{v.t.}{Expôr a fumaça, a vapores ou gases; defumar.}{fu.mi.gar}{0}
\verb{fumigar}{}{}{}{}{}{Desinfetar por meio de fumo ou fumaça.}{fu.mi.gar}{\verboinum{5}}
\verb{fumo}{}{Bot.}{}{}{s.m.}{Planta de folhas longas e macias usadas para fazer charutos; tabaco.}{fu.mo}{0}
\verb{fumo}{}{}{}{}{}{Folhas secas e esmigalhadas dessa planta; tabaco.}{fu.mo}{0}
\verb{fumo}{}{}{}{}{}{Massa de gases que sai de alguma coisa que queima.}{fu.mo}{0}
\verb{funambulesco}{ê}{}{}{}{adj.}{Relativo a funâmbulo.}{fu.nam.bu.les.co}{0}
\verb{funambulesco}{ê}{Fig.}{}{}{}{Que é extravagante, excêntrico.}{fu.nam.bu.les.co}{0}
\verb{funambulismo}{}{}{}{}{s.m.}{Arte ou ofício de funâmbulo.}{fu.nam.bu.lis.mo}{0}
\verb{funâmbulo}{}{}{}{}{s.m.}{Equilibrista que anda na corda bamba ou no arame; aramista.}{fu.nâm.bu.lo}{0}
\verb{função}{}{}{"-ões}{}{s.f.}{Atividade que é própria de uma pessoa ou uma coisa.}{fun.ção}{0}
\verb{função}{}{}{"-ões}{}{}{Cada um dos trabalhos que uma pessoa deve fazer; encargo.}{fun.ção}{0}
\verb{funchal}{}{}{"-ais}{}{s.m.}{Local onde crescem funchos.}{fun.chal}{0}
\verb{funcho}{}{Bot.}{}{}{s.m.}{Planta aromática com ramos em abundância; erva"-doce.}{fun.cho}{0}
\verb{funcional}{}{}{"-ais}{}{adj.2g.}{Próprio de uma função.}{fun.ci.o.nal}{0}
\verb{funcional}{}{}{"-ais}{}{}{Que é adequado ao uso; prático.}{fun.ci.o.nal}{0}
\verb{funcionalismo}{}{}{}{}{s.m.}{Conjunto dos funcionários públicos.}{fun.ci.o.na.lis.mo}{0}
\verb{funcionamento}{}{}{}{}{s.m.}{Ato ou efeito de funcionar; atividade; desempenho.}{fun.ci.o.na.men.to}{0}
\verb{funcionar}{}{}{}{}{v.i.}{Exercer as funções que lhe são próprias.}{fun.ci.o.nar}{0}
\verb{funcionar}{}{}{}{}{}{Estar em atividade. (\textit{As lojas da galeria funcionam o dia todo.})}{fun.ci.o.nar}{0}
\verb{funcionar}{}{}{}{}{}{Dar bom resultado.}{fun.ci.o.nar}{\verboinum{1}}
\verb{funcionário}{}{}{}{}{s.m.}{Indivíduo que tem uma função em uma empresa e recebe pagamento por ela.}{fun.ci.o.ná.rio}{0}
\verb{funda}{}{}{}{}{s.f.}{Laçada de couro ou de corda para arremessar pedras, ou outros projéteis, ao longe.}{fun.da}{0}
\verb{funda}{}{}{}{}{}{Dispositivo empregado para deter o progresso de certas hérnias.}{fun.da}{0}
\verb{fundação}{}{}{"-ões}{}{s.f.}{Ato de fundar.}{fun.da.ção}{0}
\verb{fundação}{}{}{"-ões}{}{}{Instituição que emprega os recursos que recebe numa atividade de interesse público.}{fun.da.ção}{0}
\verb{fundação}{}{}{"-ões}{}{}{A base de uma construção; alicerce, fundamento.}{fun.da.ção}{0}
\verb{fundado}{}{}{}{}{adj.}{Que se apoia ou funda na razão, ou em boas razões.}{fun.da.do}{0}
\verb{fundador}{ô}{}{}{}{adj.}{Diz"-se de indivíduo que funda; instituidor, iniciador.}{fun.da.dor}{0}
\verb{fundamentação}{}{}{"-ões}{}{s.f.}{Ato ou efeito de fundamentar, apoiar, documentar; fundação.}{fun.da.men.ta.ção}{0}
\verb{fundamental}{}{}{"-ais}{}{adj.2g.}{Que serve de fundamento; básico, essencial, principal.}{fun.da.men.tal}{0}
\verb{fundamentalismo}{}{}{}{}{s.m.}{Movimento religioso conservador que interpreta literalmente as doutrinas antigas.}{fun.da.men.ta.lis.mo}{0}
\verb{fundamentalismo}{}{}{}{}{}{Movimento, doutrina ou atitude muito conservadora.}{fun.da.men.ta.lis.mo}{0}
\verb{fundamentalista}{}{}{}{}{adj.2g.}{Relativo a fundamentalismo.}{fun.da.men.ta.lis.ta}{0}
\verb{fundamentalista}{}{}{}{}{s.2g.}{Seguidor do fundamentalismo.}{fun.da.men.ta.lis.ta}{0}
\verb{fundamentar}{}{}{}{}{v.t.}{Fazer alguma coisa com base em outra; alicerçar, apoiar, fundar.}{fun.da.men.tar}{0}
\verb{fundamentar}{}{}{}{}{}{Dar fundamento ao que se diz; justificar.}{fun.da.men.tar}{\verboinum{1}}
\verb{fundamento}{}{}{}{}{s.m.}{Base, alicerce.}{fun.da.men.to}{0}
\verb{fundamento}{}{}{}{}{}{Cada uma das causas que se têm para afirmar alguma coisa; motivo, razão.}{fun.da.men.to}{0}
\verb{fundão}{}{}{"-ões}{}{s.m.}{Lugar afastado, isolado, ermo.}{fun.dão}{0}
\verb{fundar}{}{}{}{}{v.t.}{Dar início a uma organização; constituir, criar.}{fun.dar}{0}
\verb{fundar}{}{}{}{}{}{Fazer as bases de uma construção; alicerçar, fundamentar.}{fun.dar}{0}
\verb{fundar}{}{}{}{}{}{Fazer alguma coisa com base em outra; apoiar, basear.}{fun.dar}{\verboinum{1}}
\verb{fundeadouro}{ô}{}{}{}{s.m.}{Lugar próprio para a ancoragem; ancoradouro.}{fun.de.a.dou.ro}{0}
\verb{fundear}{}{}{}{}{v.i.}{Deitar ferro ou âncora; ancorar.}{fun.de.ar}{0}
\verb{fundear}{}{}{}{}{}{Ir ao fundo.}{fun.de.ar}{\verboinum{4}}
\verb{fundente}{}{}{}{}{adj.2g.}{Que está em fusão.}{fun.den.te}{0}
\verb{fundente}{}{}{}{}{}{Diz"-se de substância que facilita a fusão de metais.}{fun.den.te}{0}
\verb{fundiário}{}{}{}{}{adj.}{Relativo a terrenos.}{fun.di.á.rio}{0}
\verb{fundição}{}{}{"-ões}{}{s.f.}{Ato de fundir um metal.}{fun.di.ção}{0}
\verb{fundição}{}{}{"-ões}{}{}{Lugar onde se funde metal.}{fun.di.ção}{0}
\verb{fundidor}{ô}{}{}{}{s.m.}{Trabalhador que funde.}{fun.di.dor}{0}
\verb{fundilho}{}{}{}{}{s.m.}{Parte das calças e das cuecas que corresponde ao assento.}{fun.di.lho}{0}
\verb{fundilhos}{}{}{}{}{s.m.pl.}{Fundilho.}{fun.di.lhos}{0}
\verb{fundir}{}{}{}{}{v.t.}{Fazer um material sólido ficar líquido; derreter.}{fun.dir}{0}
\verb{fundir}{}{}{}{}{}{Incorporar várias coisas em uma só; conciliar.}{fun.dir}{\verboinum{18}}
\verb{fundista}{}{}{}{}{s.2g.}{Corredor que disputa provas de longa distância.}{fun.dis.ta}{0}
\verb{fundo}{}{}{}{}{adj.}{Que se estende muito para dentro de alguma coisa; profundo.}{fun.do}{0}
\verb{fundo}{}{}{}{}{}{Que ficou mais para dentro; cavado, reentrante. (\textit{Amanheci com os olhos fundos por ter dormido mal durante a noite.})}{fun.do}{0}
\verb{fundo}{}{}{}{}{s.m.}{A parte mais afastada de alguma coisa.}{fun.do}{0}
\verb{fundo}{}{}{}{}{}{A parte mais baixa ou mais interior de um lugar, de uma região.}{fun.do}{0}
\verb{fundos}{}{}{}{}{s.m.pl.}{Lugar atrás de alguma coisa. (\textit{A churrasqueira fica nos fundos da casa.})}{fun.dos}{0}
\verb{fundos}{}{}{}{}{}{Dinheiro que uma empresa tem; capital.}{fun.dos}{0}
\verb{fundos}{}{}{}{}{}{Dinheiro deixado numa conta bancária.}{fun.dos}{0}
\verb{fundura}{}{}{}{}{s.f.}{Distância que vai da entrada ao fundo de alguma coisa.}{fun.du.ra}{0}
\verb{fúnebre}{}{}{}{}{adj.2g.}{Relativo à morte ou aos mortos; funeral.}{fú.ne.bre}{0}
\verb{fúnebre}{}{Fig.}{}{}{}{Sombrio, lúgubre.}{fú.ne.bre}{0}
\verb{funeral}{}{}{"-ais}{}{s.m.}{Cerimônia de enterro.}{fu.ne.ral}{0}
\verb{funerária}{}{}{}{}{s.f.}{Empresa que trata de enterros.}{fu.ne.rá.ria}{0}
\verb{funerário}{}{}{}{}{adj.}{Que se refere a enterro.}{fu.ne.rá.rio}{0}
\verb{funéreo}{}{}{}{}{adj.}{Fúnebre.}{fu.né.re.o}{0}
\verb{funesto}{é/ ou /ê}{}{}{}{adj.}{Que provoca a morte, a desgraça.}{fu.nes.to}{0}
\verb{funesto}{é/ ou /ê}{}{}{}{}{Que prognostica desgraça, desventura.}{fu.nes.to}{0}
\verb{funesto}{é/ ou /ê}{}{}{}{}{Que é danoso, prejudicial, nocivo.}{fu.nes.to}{0}
\verb{fungação}{}{}{"-ões}{}{s.f.}{Ato de fungar repetidamente.}{fun.ga.ção}{0}
\verb{fungar}{}{}{}{}{v.i.}{Aspirar fazendo ruído pelo nariz.}{fun.gar}{0}
\verb{fungar}{}{}{}{}{}{Choramingar, resmungar.}{fun.gar}{\verboinum{5}}
\verb{fungicida}{}{}{}{}{adj.2g.}{Diz"-se de substância empregada no combate aos fungos.}{fun.gi.ci.da}{0}
\verb{fungível}{}{}{"-eis}{}{adj.2g.}{Que se gasta com o uso.}{fun.gí.vel}{0}
\verb{fungo}{}{Bot.}{}{}{s.m.}{Organismo vegetal, sem clorofila nem flores, incapaz de produzir seu próprio alimento.}{fun.go}{0}
\verb{fungo}{}{Med.}{}{}{}{Excrescência carnosa ou esponjosa na pele, principalmente em volta de uma ferida.}{fun.go}{0}
\verb{funicular}{}{}{}{}{adj.2g.}{Composto de cordas.}{fu.ni.cu.lar}{0}
\verb{funicular}{}{}{}{}{}{Que funciona por meio de cordas.}{fu.ni.cu.lar}{0}
\verb{funicular}{}{}{}{}{s.m.}{Transporte tracionado por cabos.}{fu.ni.cu.lar}{0}
\verb{funil}{}{}{"-is}{}{s.m.}{Utensílio que tem uma boca larga, provido de um tubo estreito, próprio para passar líquidos de uma vasilha para outra}{fu.nil}{0}
\verb{funilaria}{}{}{}{}{s.f.}{Estabelecimento ou loja de funileiro.}{fu.ni.la.ri.a}{0}
\verb{funileiro}{ê}{}{}{}{s.m.}{Fabricante de funis.}{fu.ni.lei.ro}{0}
\verb{funileiro}{ê}{}{}{}{}{Profissional que trabalha com objetos de folhas de metal.}{fu.ni.lei.ro}{0}
\verb{funileiro}{ê}{}{}{}{}{Indivíduo que trabalha com o conserto das partes amassadas da lataria de um carro.}{fu.ni.lei.ro}{0}
\verb{funk}{}{}{}{}{s.m.}{Gênero musical originário dos Estados Unidos, caracterizado pela simplicidade melódica e pelo ritmo vigoroso e repetitivo.}{\textit{funk}}{0}
\verb{fura"-bolo}{ô}{Pop.}{fura"-bolos ⟨ô⟩}{}{s.m.}{Dedo que fica ao lado do polegar; indicador.}{fu.ra"-bo.lo}{0}
\verb{fura"-bolos}{ô}{}{}{}{s.m.pl.}{Fura"-bolo.}{fu.ra"-bo.los}{0}
\verb{furacão}{}{}{"-ões}{}{s.m.}{Vento de grande velocidade e poder destrutivo.}{fu.ra.cão}{0}
\verb{furadeira}{ê}{}{}{}{s.f.}{Ferramenta manual ou elétrica para perfurar madeira, metal e alvenaria.}{fu.ra.dei.ra}{0}
\verb{furado}{}{}{}{}{adj.}{Que se furou ou que apresenta furos ou buracos.}{fu.ra.do}{0}
\verb{furado}{}{}{}{}{}{Que apresenta vazamento.}{fu.ra.do}{0}
\verb{furado}{}{Pop.}{}{}{}{Que não deu certo; malogrado.}{fu.ra.do}{0}
\verb{furador}{ô}{}{}{}{s.m.}{Utensílio usado para fazer furos.}{fu.ra.dor}{0}
\verb{furão}{}{Zool.}{"-ões}{}{s.m.}{Mamífero carnívoro de corpo longo e delgado e patas curtas.}{fu.rão}{0}
\verb{furar}{}{}{}{}{v.t.}{Fazer um furo.}{fu.rar}{0}
\verb{furar}{}{}{}{}{v.i.}{Adquirir um furo. (\textit{O pneu do carro furou.})}{fu.rar}{0}
\verb{furar}{}{Pop.}{}{}{}{Deixar de cumprir uma promessa ou algo combinado.}{fu.rar}{\verboinum{1}}
\verb{furgão}{}{}{"-ões}{}{s.m.}{Veículo utilitário pequeno e fechado, para transporte de pequenas cargas ou de um pequeno grupo de pessoas.}{fur.gão}{0}
\verb{fúria}{}{}{}{}{s.f.}{Sentimento de raiva; cólera.}{fú.ria}{0}
\verb{fúria}{}{}{}{}{}{A grande energia empregada em uma ação; ímpeto, furor.}{fú.ria}{0}
\verb{furibundo}{}{}{}{}{adj.}{Cheio de raiva; furioso.}{fu.ri.bun.do}{0}
\verb{furioso}{ô}{}{"-osos ⟨ó⟩}{"-osa ⟨ó⟩}{adj.}{Cheio de raiva; raivoso, irado.}{fu.ri.o.so}{0}
\verb{furna}{}{}{}{}{s.f.}{Cavidade profunda aberta em uma rocha; caverna.}{fur.na}{0}
\verb{furo}{}{}{}{}{s.m.}{Abertura em uma superfície; orifício, buraco.}{fu.ro}{0}
\verb{furor}{ô}{}{}{}{s.m.}{Fúria, raiva, ira.}{fu.ror}{0}
\verb{furor}{ô}{}{}{}{}{Exaltação violenta; frenesi, delírio.}{fu.ror}{0}
\verb{furor}{ô}{}{}{}{}{Entusiasmo, impetuosidade.}{fu.ror}{0}
\verb{furriel}{é}{Desus.}{"-éis}{}{s.m.}{No exército do Brasil colonial, graduação militar acima do cabo e abaixo do sargento.}{fur.ri.el}{0}
\verb{furta"-cor}{ô}{}{}{}{adj.2g.}{Que muda de cor conforme a luz incidente.}{fur.ta"-cor}{0}
\verb{furtadela}{é}{}{}{}{s.f.}{Ato ou efeito de furtar.}{fur.ta.de.la}{0}
\verb{furtar}{}{}{}{}{v.t.}{Tirar um bem alheio sem ser visto.}{fur.tar}{0}
\verb{furtar}{}{}{}{}{v.pron.}{Negar"-se a fazer algo.}{fur.tar}{\verboinum{1}}
\verb{furtivo}{}{}{}{}{adj.}{Que age às escondidas; dissimulado.}{fur.ti.vo}{0}
\verb{furto}{}{}{}{}{s.m.}{Ato ou efeito de furtar.}{fur.to}{0}
\verb{furúnculo}{}{Med.}{}{}{s.m.}{Pequena infecção da pele, geralmente de forma circular e com pus no centro.}{fu.rún.cu.lo}{0}
\verb{furunculose}{ó}{Med.}{}{}{s.f.}{Erupção de vários furúnculos.}{fu.run.cu.lo.se}{0}
\verb{fusa}{}{Mús.}{}{}{s.f.}{Figura rítmica da notação musical equivalente a metade do valor de uma semicolcheia.}{fu.sa}{0}
\verb{fusão}{}{}{"-ões}{}{s.f.}{Ato ou efeito de fundir.}{fu.são}{0}
\verb{fusão}{}{}{"-ões}{}{}{União, aliança.}{fu.são}{0}
\verb{fusão}{}{Fís.}{"-ões}{}{}{Passagem de uma substância do estado sólido para o estado líquido.}{fu.são}{0}
\verb{fusca}{}{Pop.}{}{}{s.m.}{Certo modelo de automóvel mundialmente conhecido.}{fus.ca}{0}
\verb{fusco}{}{}{}{}{adj.}{Escuro, pardo, sem brilho.}{fus.co}{0}
\verb{fuselagem}{}{}{"-ens}{}{s.f.}{Parte principal e central do corpo das aeronaves, na qual se fixam as asas.}{fu.se.la.gem}{0}
\verb{fusibilidade}{}{}{}{}{s.f.}{Qualidade do que é fusível.}{fu.si.bi.li.da.de}{0}
\verb{fusiforme}{ó}{}{}{}{adj.2g.}{Que tem forma de fuso.}{fu.si.for.me}{0}
\verb{fúsil}{}{}{"-eis}{}{adj.2g.}{Que se pode fundir; fundível, fusível.}{fú.sil}{0}
\verb{fusível}{}{}{"-eis}{}{adj.2g.}{Que se pode fundir.}{fu.sí.vel}{0}
\verb{fusível}{}{}{"-eis}{}{s.m.}{Dispositivo de proteção de circuitos elétricos constituído de um fio metálico que se funde quando há sobrecarga ou curto"-circuito.}{fu.sí.vel}{0}
\verb{fuso}{}{}{}{}{s.m.}{Carretel em que se enrola o fio quando se tece.}{fu.so}{0}
\verb{fusô}{}{}{}{}{s.m.}{Calça justa para uso esportivo ou casual, geralmente feita de malha e, algumas vezes, presa por uma alça debaixo dos pés.}{fu.sô}{0}
\verb{fusologia}{}{}{}{}{s.f.}{Conjunto de conhecimentos relacionados com a construção de foguetes e mísseis balísticos.}{fu.so.lo.gi.a}{0}
\verb{fusquinha}{}{Pop.}{}{}{s.m.}{Fusca.}{fus.qui.nha}{0}
\verb{fustão}{}{}{"-ões}{}{s.m.}{Tecido de algodão, linho, lã ou seda em relevo.}{fus.tão}{0}
\verb{fuste}{}{}{}{}{s.m.}{Haste de madeira.}{fus.te}{0}
\verb{fuste}{}{}{}{}{}{Peça com que se esteiam os mastros dos navios.}{fus.te}{0}
\verb{fustigar}{}{}{}{}{v.t.}{Bater com vara; açoitar.}{fus.ti.gar}{0}
\verb{fustigar}{}{}{}{}{}{Castigar, maltratar.}{fus.ti.gar}{\verboinum{5}}
\verb{futebol}{ó}{Esport.}{}{}{s.m.}{Jogo em que a bola é direcionada apenas com os pés, disputado entre dois times de 11 jogadores cada, cujo objetivo é fazer entrar a bola no gol do time adversário.}{fu.te.bol}{0}
\verb{futebolista}{}{}{}{}{s.2g.}{Jogador, especialista ou apreciador de futebol.}{fu.te.bo.lis.ta}{0}
\verb{futevôlei}{}{Esport.}{}{}{s.m.}{Variedade de voleibol em que só se pode tocar a bola com os pés, a cabeça ou o peito.}{fu.te.vô.lei}{0}
\verb{fútil}{}{}{"-eis}{}{adj.2g.}{Sem importância; inútil, superficial.}{fú.til}{0}
\verb{futilidade}{}{}{}{}{s.f.}{Qualidade de fútil.}{fu.ti.li.da.de}{0}
\verb{futilizar}{}{}{}{}{v.t.}{Tornar fútil.}{fu.ti.li.zar}{\verboinum{1}}
\verb{futrica}{}{}{}{}{s.f.}{Intriga, mexerico, provocação.}{fu.tri.ca}{0}
\verb{futricar}{}{}{}{}{v.t.}{Intrometer"-se para atrapalhar; arruinar, importunar.}{fu.tri.car}{0}
\verb{futricar}{}{}{}{}{v.i.}{Fazer intriga; mexericar.}{fu.tri.car}{\verboinum{2}}
\verb{futriqueiro}{ê}{}{}{}{adj.}{Que faz intrigas; fuxiqueiro.}{fu.tri.quei.ro}{0}
\verb{futsal}{}{Esport.}{}{}{s.m.}{Modalidade semelhante ao futebol, mas com times de sete jogadores cada e praticada em quadra fechada; futebol de salão.}{fut.sal}{0}
\verb{futucar}{}{}{}{}{v.t.}{Espetar, cutucar, furar.}{fu.tu.car}{0}
\verb{futucar}{}{}{}{}{}{Aborrecer, importunar.}{fu.tu.car}{\verboinum{2}}
\verb{futurar}{}{}{}{}{v.i.}{Predizer o futuro; prognosticar, prenunciar.}{fu.tu.rar}{\verboinum{1}}
\verb{futurismo}{}{}{}{}{s.m.}{Movimento artístico do início do século \textsc{xx} inspirado em uma concepção dinâmica da vida e cultuador da velocidade, da máquina.}{fu.tu.ris.mo}{0}
\verb{futurista}{}{}{}{}{adj.2g.}{Relativo a futurismo.}{fu.tu.ris.ta}{0}
\verb{futuro}{}{}{}{}{adj.}{Que há de ser ou acontecer; vindouro.}{fu.tu.ro}{0}
\verb{futuro}{}{}{}{}{s.m.}{O tempo que ainda vai chegar; posteridade.}{fu.tu.ro}{0}
\verb{futuro}{}{}{}{}{}{O destino.}{fu.tu.ro}{0}
\verb{futuro}{}{Gram.}{}{}{}{Tempo verbal que indica ação posterior ao momento da enunciação.}{fu.tu.ro}{0}
\verb{futurologia}{}{}{}{}{s.f.}{Conjunto de estudos sobre os possíveis rumos a serem tomados pela sociedade, economia, ciencia.}{fu.tu.ro.lo.gi.a}{0}
\verb{futuroso}{ô}{}{"-osos ⟨ó⟩}{"-osa ⟨ó⟩}{adj.}{Que tem bom futuro; promissor.}{fu.tu.ro.so}{0}
\verb{fuxicar}{ch}{}{}{}{v.i.}{Futricar, intrigar, mexericar.}{fu.xi.car}{0}
\verb{fuxicar}{ch}{}{}{}{v.t.}{Alinhavar, coser.}{fu.xi.car}{\verboinum{2}}
\verb{fuxico}{ch}{}{}{}{s.m.}{Intriga, futrica, mexerico.}{fu.xi.co}{0}
\verb{fuxiqueiro}{ch}{}{}{}{adj.}{Que faz intrigas.}{fu.xi.quei.ro}{0}
\verb{fuzarca}{}{}{}{}{s.f.}{Festividade grande, agitada e numerosa; folia, farra.}{fu.zar.ca}{0}
\verb{fuzil}{}{}{"-is}{}{s.m.}{Arma portátil de cano comprido e de longo alcance.}{fu.zil}{0}
\verb{fuzilamento}{}{}{}{}{s.m.}{Execução de um condenado à morte por um pelotão militar armado com fuzis.}{fu.zi.la.men.to}{0}
\verb{fuzilar}{}{}{}{}{v.t.}{Matar com arma de fogo.}{fu.zi.lar}{0}
\verb{fuzilar}{}{}{}{}{}{Brilhar muito forte; cintilar, relampejar.}{fu.zi.lar}{\verboinum{1}}
\verb{fuzilaria}{}{}{}{}{s.f.}{Muitos tiros simultâneos; tiroteio.}{fu.zi.la.ri.a}{0}
\verb{fuzileiro}{ê}{}{}{}{s.m.}{Soldado equipado com fuzil.}{fu.zi.lei.ro}{0}
\verb{fuzuê}{}{}{}{}{s.m.}{Festa ruidosa e animada; folia.}{fu.zu.ê}{0}
\verb{fuzuê}{}{}{}{}{}{Confusão, rolo, briga.}{fu.zu.ê}{0}
