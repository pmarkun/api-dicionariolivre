\verb{s}{}{}{}{}{s.m.}{Décima nona letra do alfabeto português.}{s}{0}
\verb{S}{}{}{}{}{}{Com ponto, abrev. de \textit{sul}.}{S}{0}
\verb{S}{}{Quím.}{}{}{}{Símb. do \textit{enxofre}.}{S}{0}
\verb{sã}{}{}{}{}{adj.}{Feminino de \textit{são}; sadia, ilesa.}{sã}{0}
\verb{saariano}{}{}{}{}{adj.}{Relativo ao deserto do Saara (África).}{sa.a.ri.a.no}{0}
\verb{sabá}{}{}{}{}{s.m.}{Descanso religioso dos judeus que acontece no sétimo dia da semana.}{sa.bá}{0}
\verb{sabá}{}{}{}{}{}{Assembleia noturna de feiticeiros e feiticeiras que, segundo a superstição medieval, se reunia no sábado, à meia"-noite, sob a direção de Satanás.}{sa.bá}{0}
\verb{sábado}{}{}{}{}{s.m.}{O sétimo dia da semana.}{sá.ba.do}{0}
\verb{sabão}{}{}{"-ões}{}{s.m.}{Produto de limpeza que se vende em pedaços semelhantes a pequenos tijolos ou em pó. (\textit{Comprei muito sabão de coco.})}{sa.bão}{0}
\verb{sabão}{}{Pop.}{"-ões}{}{}{Censura, repreensão. (\textit{A mãe passou um sabão na filha, pelas mentiras que tinha contado.})}{sa.bão}{0}
\verb{sabático}{}{}{}{}{adj.}{Relativo ao sábado.}{sa.bá.ti.co}{0}
\verb{sabático}{}{}{}{}{}{Referente ao sabá.}{sa.bá.ti.co}{0}
\verb{sabatina}{}{}{}{}{s.f.}{Repetição, no sábado, das lições estudadas durante a semana.}{sa.ba.ti.na}{0}
\verb{sabatina}{}{}{}{}{}{Recapitulação de lições por meio de perguntas e respostas.}{sa.ba.ti.na}{0}
\verb{sabatina}{}{}{}{}{}{Reza do ofício divino própria para o sábado.}{sa.ba.ti.na}{0}
\verb{sabatina}{}{Fig.}{}{}{}{Discussão, debate.}{sa.ba.ti.na}{0}
\verb{sabatinar}{}{}{}{}{v.t.}{Recordar, recapitular em sabatina.}{sa.ba.ti.nar}{0}
\verb{sabatinar}{}{}{}{}{}{Fazer resumo; condensar.}{sa.ba.ti.nar}{0}
\verb{sabatinar}{}{}{}{}{v.i.}{Discutir miudamente e usando de sofismas.}{sa.ba.ti.nar}{\verboinum{1}}
\verb{sabedor}{ô}{}{}{}{adj.}{Que sabe, que tem conhecimento de algo; ciente.}{sa.be.dor}{0}
\verb{sabedor}{ô}{}{}{}{}{Que tem sabedoria; erudito.}{sa.be.dor}{0}
\verb{sabedor}{ô}{}{}{}{s.m.}{Indivíduo que sabe alguma coisa; inteirado.}{sa.be.dor}{0}
\verb{sabedor}{ô}{}{}{}{}{Indivíduo que tem profundo conhecimento de algo; sábio.}{sa.be.dor}{0}
\verb{sabedoria}{}{}{}{}{s.f.}{Qualidade de sábio.}{sa.be.do.ri.a}{0}
\verb{sabedoria}{}{}{}{}{}{Acúmulo de muitos conhecimentos; grande instrução; erudição, ciência, saber.}{sa.be.do.ri.a}{0}
\verb{sabedoria}{}{}{}{}{}{Prudência e moderação no modo de agir; temperança, reflexão.}{sa.be.do.ri.a}{0}
\verb{sabedoria}{}{}{}{}{}{Conhecimento justo das coisas; conhecimento da verdade; razão.}{sa.be.do.ri.a}{0}
\verb{sabedoria}{}{}{}{}{}{Massa dos conhecimentos adquiridos; ciência.}{sa.be.do.ri.a}{0}
\verb{sabença}{}{Pop.}{}{}{s.f.}{Soma de muitos conhecimentos; saber, erudição, sabedoria.}{sa.ben.ça}{0}
\verb{saber}{ê}{}{}{}{s.m.}{Soma de conhecimentos; erudição, sabedoria. (\textit{Ele tem o saber da medicina.})}{sa.ber}{0}
\verb{saber}{ê}{}{}{}{v.t.}{Ter o conhecimento de alguma coisa. (\textit{Ele sabe muito da língua portuguesa.})}{sa.ber}{0}
\verb{saber}{ê}{}{}{}{}{Ter conhecimentos técnicos ou especiais que permitem fazer alguma coisa. (\textit{Eu sei jogar futebol.})}{sa.ber}{\verboinum{45}}
\verb{sabe"-tudo}{}{Pop.}{}{}{s.2g.}{Indivíduo que ostenta erudição; sabichão.}{sa.be"-tu.do}{0}
\verb{sabiá}{}{Zool.}{}{}{s.2g.}{Designação comum às aves onívoras, que possuem plumagem de colorido simples, geralmente marrom, preta ou cinza, com as partes inferiores lisas ou manchadas.}{sa.bi.á}{0}
\verb{sabichão}{}{Pop.}{"-ões}{"-ona}{adj.}{Que é grande sábio ou alardeia sabedoria.}{sa.bi.chão}{0}
\verb{sabichão}{}{}{"-ões}{"-ona}{s.m.}{Indivíduo que é grande sábio ou julga saber muito.}{sa.bi.chão}{0}
\verb{sabido}{}{}{}{}{adj.}{Que se sabe; conhecido.}{sa.bi.do}{0}
\verb{sabido}{}{}{}{}{}{Que tem saber; erudito.}{sa.bi.do}{0}
\verb{sabido}{}{}{}{}{}{Diz"-se de pessoa prudente, cautelosa.}{sa.bi.do}{0}
\verb{sabido}{}{}{}{}{}{Diz"-se de pessoa velhaca, esperta.}{sa.bi.do}{0}
\verb{sabido}{}{}{}{}{s.m.}{Indivíduo que é versado; conhecedor.}{sa.bi.do}{0}
\verb{sabido}{}{}{}{}{}{Indivíduo que é prudente, cauto.}{sa.bi.do}{0}
\verb{sabido}{}{}{}{}{}{Indivíduo trapaceiro, astuto.}{sa.bi.do}{0}
\verb{Sabinada}{}{Hist.}{}{}{s.f.}{Revolução separatista ocorrida na Bahia, cujo objetivo era desligar a província do governo regencial, instaurando ali uma república provisória.}{Sa.bi.na.da}{0}
\verb{sabino}{}{}{}{}{adj.}{Relativo aos sabinos, antigo povo da Itália central, vizinho dos latinos.}{sa.bi.no}{0}
\verb{sabino}{}{}{}{}{s.m.}{Indivíduo desse povo.}{sa.bi.no}{0}
\verb{sabino}{}{}{}{}{}{A língua itálica desse povo.}{sa.bi.no}{0}
\verb{sabino}{}{}{}{}{}{Integrante da facção política que lutou pela Sabinada.}{sa.bi.no}{0}
\verb{sábio}{}{}{}{}{adj.}{Que tem muita sabedoria, que sabe muito; erudito.}{sá.bio}{0}
\verb{sábio}{}{}{}{}{}{Que tem muitos conhecimentos numa dada matéria ou especialidade; conhecedor, versado.}{sá.bio}{0}
\verb{sábio}{}{}{}{}{}{Sensato, prudente.}{sá.bio}{0}
\verb{sábio}{}{}{}{}{s.m.}{Indivíduo que sabe muito; erudito.}{sá.bio}{0}
\verb{sábio}{}{}{}{}{}{Filósofo, pensador.}{sá.bio}{0}
\verb{sábio}{}{}{}{}{}{Indivíduo prudente, equilibrado.}{sá.bio}{0}
\verb{sabível}{}{}{"-eis}{}{adj.2g.}{Que se pode saber.}{sa.bí.vel}{0}
\verb{saboaria}{}{}{}{}{s.f.}{Local onde se fabrica, se vende ou se guarda sabão. }{sa.bo.a.ri.a}{0}
\verb{saboeiro}{ê}{}{}{}{s.m.}{Indivíduo que fabrica ou vende sabão.}{sa.bo.ei.ro}{0}
\verb{sabonete}{ê}{}{}{}{s.m.}{Sabão fino, geralmente aromatizado, de vários formatos, próprio para a higiene corporal.}{sa.bo.ne.te}{0}
\verb{saboneteira}{ê}{}{}{}{s.f.}{Recipiente próprio para colocar o sabonete.}{sa.bo.ne.tei.ra}{0}
\verb{sabor}{ô}{}{}{}{s.m.}{Impressão causada na boca, pela comida ou pela bebida; gosto.  (\textit{O sabor dessa comida é muito bom.})}{sa.bor}{0}
\verb{sabor}{ô}{Fig.}{}{}{}{Qualidade comparável a qualquer coisa. }{sa.bor}{0}
\verb{saborear}{}{}{}{}{v.t.}{Comer ou beber alguma coisa sentindo e apreciando o gosto bom que ela tem. (\textit{Ele saboreava cada pedaço de bolo que comia.})}{sa.bo.re.ar}{0}
\verb{saborear}{}{Fig.}{}{}{}{Ter prazer com alguma coisa; deleitar"-se.   (\textit{Todas as tardes de domingo, ele saboreia jogos de futebol.})}{sa.bo.re.ar}{\verboinum{4}}
\verb{saboroso}{ô}{}{"-osos ⟨ó⟩}{"-osa ⟨ó⟩}{adj.}{Que tem bom sabor ou gosto; gostoso.}{sa.bo.ro.so}{0}
\verb{saboroso}{ô}{Fig.}{"-osos ⟨ó⟩}{"-osa ⟨ó⟩}{}{Que proporciona prazer; agradável.}{sa.bo.ro.so}{0}
\verb{sabotagem}{}{}{"-ens}{}{s.f.}{Ato ou efeito de sabotar.}{sa.bo.ta.gem}{0}
\verb{sabotagem}{}{}{"-ens}{}{}{Destruição, danificação propositada de estradas, meios de transporte, instalações industriais, militares etc., para a interrupção de serviços.}{sa.bo.ta.gem}{0}
\verb{sabotar}{}{}{}{}{v.t.}{Danificar ou destruir propositada e criminosamente instalações ferroviárias, industriais, militares etc. para impedir, retardar ou interromper seu funcionamento.}{sa.bo.tar}{0}
\verb{sabotar}{}{}{}{}{}{Dificultar ou prejudicar uma atividade por meio de resistência passiva; minar.}{sa.bo.tar}{\verboinum{1}}
\verb{sabre}{}{}{}{}{s.m.}{Arma branca de lâmina reta ou curva, pontuda e afiada de um só lado.}{sa.bre}{0}
\verb{sabugal}{}{}{"-ais}{}{s.m.}{Coletivo de sabugueiro.}{sa.bu.gal}{0}
\verb{sabugo}{}{}{}{}{s.m.}{Espiga de milho a que se retiraram os grãos.}{sa.bu.go}{0}
\verb{sabugo}{}{}{}{}{}{Miolo do sabugueiro.}{sa.bu.go}{0}
\verb{sabugo}{}{}{}{}{}{Parte do dedo a que está aderida a unha.}{sa.bu.go}{0}
\verb{sabugo}{}{}{}{}{}{Parte interna e pouco resistente dos chifres dos animais.}{sa.bu.go}{0}
\verb{sabugueiro}{ê}{}{}{}{s.m.}{Arbusto originário da Europa e cultivado em jardins, de frutos comestíveis e flores que podem ser usadas como remédio e aromatizante de bebidas.}{sa.bu.guei.ro}{0}
\verb{sabujar}{}{}{}{}{v.t.}{Lisonjear alguém de modo excessivo; bajular, adular.}{sa.bu.jar}{\verboinum{1}}
\verb{sabujice}{}{}{}{}{s.f.}{Bajulação, servilismo.}{sa.bu.ji.ce}{0}
\verb{sabujo}{}{}{}{}{s.m.}{Grande cão de caça.}{sa.bu.jo}{0}
\verb{sabujo}{}{Fig.}{}{}{}{Indivíduo capacho, bajulador.}{sa.bu.jo}{0}
\verb{sabujo}{}{}{}{}{adj.}{Diz"-se daquele que é bajulador, servil.}{sa.bu.jo}{0}
\verb{saburra}{}{}{}{}{s.f.}{Matérias mucosas que se acreditava acumularem"-se no estômago em consequência de más digestões.}{sa.bur.ra}{0}
\verb{saburra}{}{}{}{}{}{Crosta esbranquiçada que recobre a língua, em certas doenças.}{sa.bur.ra}{0}
\verb{saburra}{}{Por ext.}{}{}{}{Casca, camada, revestimento.}{sa.bur.ra}{0}
\verb{saburra}{}{Por ext.}{}{}{}{Areia grossa que serve de lastro nos navios.}{sa.bur.ra}{0}
\verb{saburrento}{}{}{}{}{adj.}{Que tem saburra; saburroso.}{sa.bur.ren.to}{0}
\verb{saburroso}{ô}{}{"-osos ⟨ó⟩}{"-osa ⟨ó⟩}{adj.}{Saburrento.}{sa.bur.ro.so}{0}
\verb{saca}{}{}{}{}{s.f.}{Grande saco.}{sa.ca}{0}
\verb{saca}{}{}{}{}{}{Conteúdo de uma saca, de peso equivalente a 60 Kg.}{sa.ca}{0}
\verb{sacada}{}{}{}{}{s.f.}{Conteúdo de um saco.                                                                                                                     }{sa.ca.da}{0}
\verb{sacada}{}{}{}{}{}{Construção que avança da fachada de uma parede ou do nível de outra construção.}{sa.ca.da}{0}
\verb{sacada}{}{}{}{}{}{Sacadela.}{sa.ca.da}{0}
\verb{sacado}{}{}{}{}{adj.}{Que se sacou, que foi extraído, tirado para fora.}{sa.ca.do}{0}
\verb{sacado}{}{}{}{}{s.m.}{Aquele contra quem se emitiu um título de crédito.}{sa.ca.do}{0}
\verb{sacador}{ô}{}{}{}{adj.}{Que saca.}{sa.ca.dor}{0}
\verb{sacador}{ô}{}{}{}{}{Indivíduo que emite título de crédito.}{sa.ca.dor}{0}
\verb{sacal}{}{Pop.}{"-ais}{}{adj.2g.}{Enfadonho, aborrecido, chato.}{sa.cal}{0}
\verb{sacana}{}{}{}{}{adj.2g.}{Que não tem caráter; canalha.}{sa.ca.na}{0}
\verb{sacana}{}{Bras.}{}{}{}{Diz"-se de pessoa sem"-vergonha.}{sa.ca.na}{0}
\verb{sacana}{}{Bras.}{}{}{}{Diz"-se de pessoa brincalhona.}{sa.ca.na}{0}
\verb{sacana}{}{}{}{}{s.2g.}{Indivíduo sem caráter; patife.}{sa.ca.na}{0}
\verb{sacana}{}{}{}{}{}{Indivíduo libidinoso, libertino.}{sa.ca.na}{0}
\verb{sacana}{}{}{}{}{}{Indivíduo zombeteiro, gozador.}{sa.ca.na}{0}
\verb{sacanagem}{}{}{"-ens}{}{s.f.}{Ato, procedimento ou dito de sacana; safadeza, libertinagem, devassidão.}{sa.ca.na.gem}{0}
\verb{sacanear}{}{Pop.}{}{}{v.i.}{Agir como sacana.}{sa.ca.ne.ar}{0}
\verb{sacanear}{}{}{}{}{v.t.}{Irritar, aborrecer, apoquentar.}{sa.ca.ne.ar}{\verboinum{4}}
\verb{sacar}{}{}{}{}{v.t.}{Tirar para fora à força; arrancar.}{sa.car}{0}
\verb{sacar}{}{}{}{}{}{Retirar bruscamente; puxar em ameaça.}{sa.car}{0}
\verb{sacar}{}{}{}{}{}{Dar um saque (no tênis, voleibol etc.)}{sa.car}{0}
\verb{sacar}{}{}{}{}{}{Emitir (contra alguém) um título de crédito.}{sa.car}{0}
\verb{sacar}{}{}{}{}{}{Tirar, colher, obter.}{sa.car}{0}
\verb{sacar}{}{Pop.}{}{}{}{Entender, compreender.}{sa.car}{\verboinum{2}}
\verb{sacaria}{}{}{}{}{s.f.}{Grande porção de sacos ou sacas.}{sa.ca.ri.a}{0}
\verb{sacaria}{}{}{}{}{}{Indústria de sacos.}{sa.ca.ri.a}{0}
\verb{sacarídeo}{}{}{}{}{adj.}{Semelhante ao açúcar.}{sa.ca.rí.deo}{0}
\verb{sacarídeo}{}{Quím.}{}{}{}{Glicídio.}{sa.ca.rí.deo}{0}
\verb{sacarífero}{}{}{}{}{adj.}{Que produz ou contém açúcar.}{sa.ca.rí.fe.ro}{0}
\verb{sacarificar}{}{Quím.}{}{}{v.t.}{Converter industrialmente o amido em açúcar.}{sa.ca.ri.fi.car}{\verboinum{2}}
\verb{sacarímetro}{}{Quím.}{}{}{s.m.}{Instrumento empregado para medir a quantidade de açúcar em dissolução em um líquido.}{sa.ca.rí.me.tro}{0}
\verb{sacarina}{}{Quím.}{}{}{s.f.}{Substância cristalina, branca, muito doce, usada como adoçante especialmente na substituição do açúcar; açúcar dos diabéticos.}{sa.ca.ri.na}{0}
\verb{sacarino}{}{}{}{}{adj.}{Relativo a açúcar.}{sa.ca.ri.no}{0}
\verb{sacarino}{}{}{}{}{}{Em que há açúcar ou que é doce como açúcar.}{sa.ca.ri.no}{0}
\verb{sacarino}{}{Biol.}{}{}{}{Sacarívoro.}{sa.ca.ri.no}{0}
\verb{sacarívoro}{}{Biol.}{}{}{adj.}{Diz"-se de animal que se alimenta de açúcar; sacarino.}{sa.ca.rí.vo.ro}{0}
\verb{sacaroide}{ó}{}{}{}{adj.2g.}{Diz"-se da textura granular semelhante ao açúcar cristalizado.}{sa.ca.roi.de}{0}
\verb{saca"-rolhas}{ô}{}{}{}{s.m.}{Instrumento usado para perfurar e retirar a rolha de bebidas engarrafadas.}{sa.ca"-ro.lhas}{0}
\verb{sacarose}{ó}{Quím.}{}{}{s.f.}{Açúcar da cana e da beterraba, cristalino, incolor, doce, usado em produtos farmacêuticos e na alimentação humana.}{sa.ca.ro.se}{0}
\verb{sacerdócio}{}{}{}{}{s.m.}{O ofício do sacerdote.}{sa.cer.dó.cio}{0}
\verb{sacerdócio}{}{}{}{}{}{A carreira eclesiástica.}{sa.cer.dó.cio}{0}
\verb{sacerdócio}{}{Fig.}{}{}{}{Função de caráter nobre e que exige devotamento.}{sa.cer.dó.cio}{0}
\verb{sacerdotal}{}{}{"-ais}{}{adj.2g.}{Relativo a sacerdote ou a sacerdócio.}{sa.cer.do.tal}{0}
\verb{sacerdote}{ó}{}{}{sacerdotisa}{s.m.}{Indivíduo que distribui os dons divinos; ministro do culto divino; padre.}{sa.cer.do.te}{0}
\verb{sacerdote}{ó}{Fig.}{}{sacerdotisa}{}{Indivíduo que exerce profissão muito honrosa.}{sa.cer.do.te}{0}
\verb{sacerdotisa}{}{}{}{}{s.f.}{Mulher consagrada ao culto divino.}{sa.cer.do.ti.sa}{0}
\verb{sachador}{ô}{}{}{}{s.m.}{Indivíduo que sacha; capinador.}{sa.cha.dor}{0}
\verb{sachador}{ô}{}{}{}{}{Aparelho de tração animal ou automotriz, que consiste em uma armação provida de várias fileiras de lâminas, usado para tirar as ervas daninhas dos intervalos das plantas cultivadas.}{sa.cha.dor}{0}
\verb{sachadura}{}{}{}{}{s.f.}{Ato ou efeito de sachar, de escavar com o sacho.}{sa.cha.du.ra}{0}
\verb{sachar}{}{}{}{}{v.t.}{Afofar, cavar a terra com o sacho.}{sa.char}{0}
\verb{sachar}{}{}{}{}{}{Tirar com o sacho as ervas daninhas, o excesso de ramos ou os ramos secos; mondar com o sacho.}{sa.char}{\verboinum{1}}
\verb{sachê}{}{}{}{}{s.m.}{Pequeno saco ou almofada de pano com substâncias aromáticas, usado para perfumar roupas, armários, gavetas, ambientes.}{sa.chê}{0}
\verb{sacho}{}{}{}{}{s.m.}{Pequena enxada estreita, longa e pontiaguda, às vezes bifurcada na parte superior, usada para afofar ou escavar a terra.}{sa.cho}{0}
\verb{sachola}{ó}{}{}{}{s.f.}{Espécie de enxada com uma lâmina mais estreita.}{sa.cho.la}{0}
\verb{saci}{}{Bras.}{}{}{s.m.}{Personagem e entidade mitológica, menino negro de uma só perna, de cachimbo e carapuça vermelha, que persegue e prepara ciladas para os viajantes; saci"-pererê.}{sa.ci}{0}
\verb{saciar}{}{}{}{}{v.t.}{Aplacar, satisfazer a fome, a sede ou a vontade.}{sa.ci.ar}{\verboinum{1}}
\verb{saciedade}{}{}{}{}{s.f.}{Estado de quem se saciou; satisfação, fartura.}{sa.ci.e.da.de}{0}
\verb{saciedade}{}{}{}{}{}{Aborrecimento decorrente de já ter havido satisfação de uma vontade; fastio, tédio.}{sa.ci.e.da.de}{0}
\verb{saciforme}{ó}{}{}{}{adj.2g.}{Que tem forma de saco.}{sa.ci.for.me}{0}
\verb{saci"-pererê}{}{Bras.}{sacis"-pererês \textit{ou }saci"-pererês}{}{s.m.}{Saci.}{sa.ci"-pe.re.rê}{0}
\verb{saco}{}{}{}{}{s.m.}{Receptáculo de pano, papel, plástico, couro etc., fechado no fundo e nos lados e aberto em cima.}{sa.co}{0}
\verb{saco}{}{Biol.}{}{}{}{Designação comum a certas cavidades orgânicas que têm forma de saco.}{sa.co}{0}
\verb{saco}{}{Pop.}{}{}{}{A bolsa escrotal; testículos.}{sa.co}{0}
\verb{saco}{}{Bras.}{}{}{}{Paciência.}{sa.co}{0}
\verb{saco}{}{Bras.}{}{}{}{Chateação, enfado, tédio.}{sa.co}{0}
\verb{sacola}{ó}{}{}{}{s.f.}{Tipo de saco com alças, usado para carregar objetos.}{sa.co.la}{0}
\verb{sacoleiro}{ê}{}{}{}{s.m.}{Indivíduo que se dedica à venda ou compra de diversos tipos de mercadorias sem ter estabelecimento fixo.}{sa.co.lei.ro}{0}
\verb{sacolejar}{}{}{}{}{v.t.}{Sacudir repetidamente.}{sa.co.le.jar}{\verboinum{1}}
\verb{sacolejo}{ê}{}{}{}{s.m.}{Ato ou efeito de sacolejar.}{sa.co.le.jo}{0}
\verb{sacral}{}{Anat.}{"-ais}{}{adj.2g.}{Relativo ao osso sacro ou à sua localização.}{sa.cral}{0}
\verb{sacralizar}{}{}{}{}{v.t.}{Atribuir caráter sagrado.}{sa.cra.li.zar}{\verboinum{1}}
\verb{sacramentado}{}{Relig.}{}{}{adj.}{Que recebeu um ou mais sacramentos.}{sa.cra.men.ta.do}{0}
\verb{sacramentado}{}{Bras.}{}{}{}{Diz"-se de um documento legalizado ou ato formalizado.}{sa.cra.men.ta.do}{0}
\verb{sacramentado}{}{Bras.}{}{}{}{Diz"-se de um compromisso assumido.}{sa.cra.men.ta.do}{0}
\verb{sacramental}{}{}{"-ais}{}{adj.2g.}{Relativo a sacramento.}{sa.cra.men.tal}{0}
\verb{sacramental}{}{Fig.}{"-ais}{}{}{Imposto pelo hábito; costumeiro, obrigatório, habitual.}{sa.cra.men.tal}{0}
\verb{sacramentar}{}{Relig.}{}{}{v.t.}{Ministrar os sacramentos, especialmente os da confissão, da comunhão e da extrema"-unção.}{sa.cra.men.tar}{0}
\verb{sacramentar}{}{}{}{}{}{Dar caráter sagrado.}{sa.cra.men.tar}{0}
\verb{sacramentar}{}{Bras.}{}{}{}{Formalizar todos os requisitos de um contrato.}{sa.cra.men.tar}{\verboinum{1}}
\verb{sacramento}{}{Relig.}{}{}{s.m.}{Na religião católica, cada um dos rituais sagrados (batismo, eucaristia, crisma, penitência, ordem, matrimônio e extrema"-unção) para a salvação da alma.}{sa.cra.men.to}{0}
\verb{sacrário}{}{}{}{}{s.m.}{Lugar onde se guardam objetos sagrados.}{sa.crá.rio}{0}
\verb{sacrário}{}{Fig.}{}{}{}{Lugar onde se tem privacidade.}{sa.crá.rio}{0}
%\verb{}{}{}{}{}{}{}{}{0}
\verb{sacrifical}{}{}{"-ais}{}{adj.2g.}{Relativo a sacrifício.}{sa.cri.fi.cal}{0}
\verb{sacrificante}{}{}{}{}{s.m.}{O padre que celebra a missa.}{sa.cri.fi.can.te}{0}
\verb{sacrificar}{}{}{}{}{v.t.}{Oferecer em sacrifício a uma divindade; imolar.}{sa.cri.fi.car}{0}
\verb{sacrificar}{}{}{}{}{}{Desprezar em favor de outro. (\textit{É perigoso sacrificar a qualidade para aumentar a quantidade.})}{sa.cri.fi.car}{\verboinum{2}}
\verb{sacrifício}{}{}{}{}{s.m.}{Ato ou efeito de sacrificar.}{sa.cri.fí.cio}{0}
\verb{sacrifício}{}{}{}{}{}{Oferenda ritual e solene de produtos, animais ou pessoas a uma divindade, simbolizada em sua destruição ou morte.}{sa.cri.fí.cio}{0}
\verb{sacrifício}{}{}{}{}{}{Renúncia, privação.}{sa.cri.fí.cio}{0}
\verb{sacrilégio}{}{}{}{}{s.m.}{Uso profano de lugar, objeto, pessoa ou nome sagrado.}{sa.cri.lé.gio}{0}
\verb{sacrílego}{}{}{}{}{adj.}{Em que há sacrilégio.}{sa.crí.le.go}{0}
\verb{sacrílego}{}{}{}{}{}{Que cometeu sacrilégio.}{sa.crí.le.go}{0}
\verb{sacripanta}{}{}{}{}{adj.2g.}{Desprezível, indigno, patife.}{sa.cri.pan.ta}{0}
\verb{sacripanta}{}{}{}{}{s.2g.}{Falso beato.}{sa.cri.pan.ta}{0}
\verb{sacristania}{}{}{}{}{s.f.}{Função de sacristão ou sacristã.}{sa.cris.ta.ni.a}{0}
\verb{sacristão}{}{}{"-ãos}{sacristã}{s.m.}{Indivíduo encarregado da limpeza e da ordem de uma igreja e da sacristia, e que também ajuda o sacerdote na missa e em seus afazeres.}{sa.cris.tão}{0}
\verb{sacristão}{}{Fig.}{"-ãos}{sacristã}{}{Falso beato; hipócrita.}{sa.cris.tão}{0}
\verb{sacristia}{}{}{}{}{s.f.}{Casa ou compartimento anexo à igreja onde se guardam os paramentos e objetos diversos.}{sa.cris.ti.a}{0}
\verb{sacristia}{}{Fig.}{}{}{}{Os rendimentos de uma igreja.}{sa.cris.ti.a}{0}
\verb{sacro}{}{Anat.}{}{}{s.m.}{Osso localizado na parte posterior da bacia.}{sa.cro}{0}
\verb{sacro}{}{}{}{}{adj.}{Sagrado, divino, santo.}{sa.cro}{0}
\verb{sacro}{}{Fig.}{}{}{}{Digno de respeito; venerável.}{sa.cro}{0}
\verb{sacrossanto}{}{}{}{}{adj.}{Duplamente sagrado; inviolável.}{sa.cros.san.to}{0}
\verb{sacudida}{}{}{}{}{s.f.}{Ato ou efeito de sacudir.}{sa.cu.di.da}{0}
\verb{sacudidela}{é}{}{}{}{s.f.}{Sacudida leve.}{sa.cu.di.de.la}{0}
\verb{sacudidela}{é}{Pop.}{}{}{}{Pequena sova.}{sa.cu.di.de.la}{0}
\verb{sacudido}{}{}{}{}{adj.}{Que se sacudiu.}{sa.cu.di.do}{0}
\verb{sacudido}{}{Bras.}{}{}{}{Saudável, disposto, robusto, ágil.}{sa.cu.di.do}{0}
\verb{sacudido}{}{Bras.}{}{}{}{Valente, intrépido.}{sa.cu.di.do}{0}
\verb{sacudir}{}{}{}{}{v.t.}{Agitar forte e repetidamente; balançar, chacoalhar.}{sa.cu.dir}{0}
\verb{sacudir}{}{Fig.}{}{}{}{Impressionar bastante; comover, abalar.}{sa.cu.dir}{\verboinum{33}}
\verb{sáculo}{}{}{}{}{s.m.}{Pequeno saco.}{sá.cu.lo}{0}
\verb{sáculo}{}{Anat.}{}{}{}{Pequena estrutura do ouvido humano capaz de detectar a posição da cabeça e controlar o equilíbrio do corpo.}{sá.cu.lo}{0}
\verb{sádico}{}{}{}{}{adj.}{Relativo a sadismo.}{sá.di.co}{0}
\verb{sádico}{}{}{}{}{}{Que pratica sadismo.}{sá.di.co}{0}
\verb{sádico}{}{Por ext.}{}{}{}{Que se deleita em fazer mal aos outros.}{sá.di.co}{0}
\verb{sadio}{}{}{}{}{adj.}{Que tem boa saúde.}{sa.di.o}{0}
\verb{sadio}{}{}{}{}{}{Que é bom para a saúde física ou mental; salubre.}{sa.di.o}{0}
\verb{sadismo}{}{}{}{}{s.m.}{Perversão em que se obtém prazer sexual por meio da violência física ou moral infligida ao parceiro.}{sa.dis.mo}{0}
\verb{sadismo}{}{Por ext.}{}{}{}{Prazer com o sofrimento alheio.}{sa.dis.mo}{0}
\verb{sadista}{}{}{}{}{adj.2g. e s.2g.}{Sádico.}{sa.dis.ta}{0}
\verb{sadomasoquismo}{}{}{}{}{s.m.}{Relação caracterizada pela combinação de sadismo e masoquismo.}{sa.do.ma.so.quis.mo}{0}
\verb{sadomasoquista}{}{}{}{}{adj.2g.}{Relativo ao sadomasoquismo.}{sa.do.ma.so.quis.ta}{0}
\verb{sadomasoquista}{}{}{}{}{}{Que pratica o sadomasoquismo.}{sa.do.ma.so.quis.ta}{0}
\verb{sadomasoquista}{}{}{}{}{s.2g.}{Indivíduo que pratica o sadomasoquismo.}{sa.do.ma.so.quis.ta}{0}
\verb{safa}{}{}{}{}{interj.}{Expressão que denota admiração ou repugnância. \textit{abon?}}{sa.fa}{0}
\verb{safadeza}{ê}{}{}{}{s.f.}{Qualidade ou procedimento de indivíduo safado.}{sa.fa.de.za}{0}
\verb{safadeza}{ê}{Pop.}{}{}{}{Ato ou expressão imoral; devassidão.}{sa.fa.de.za}{0}
\verb{safadeza}{ê}{Pop.}{}{}{}{Travessura, traquinagem.}{sa.fa.de.za}{0}
\verb{safadismo}{}{Bras.}{}{}{s.m.}{Safadeza.}{sa.fa.dis.mo}{0}
\verb{safado}{}{Pop.}{}{}{adj.}{Que não tem vergonha de prejudicar os outros; canalha, desavergonhado, desonesto. (\textit{Era um sujeito safado, que desapareceu com o dinheiro de muita gente.})}{sa.fa.do}{0}
\verb{safanão}{}{}{"-ões}{}{s.m.}{Puxão ou empurrão forte e brusco.}{sa.fa.não}{0}
\verb{safanão}{}{Pop.}{"-ões}{}{}{Tapa, bofetada.}{sa.fa.não}{0}
\verb{safar}{}{}{}{}{v.t.}{Tirar, sacar, extrair, surrupiar.}{sa.far}{0}
\verb{safar}{}{}{}{}{}{Livrar, salvar.}{sa.far}{0}
\verb{safar}{}{}{}{}{v.pron.}{Escapar, fugir, esquivar"-se.}{sa.far}{\verboinum{1}}
\verb{safardana}{}{}{}{}{s.m.}{Indivíduo sem escrúpulos; pulha, canalha.}{sa.far.da.na}{0}
\verb{safári}{}{}{}{}{s.m.}{Expedição para fins de caça ou exploração.}{sa.fá.ri}{0}
\verb{safári}{}{Por ext.}{}{}{}{O conjunto dos indivíduos e equipamentos envolvidos na expedição.}{sa.fá.ri}{0}
\verb{sáfaro}{}{}{}{}{adj.}{Que não permite cultivo; improdutivo, estéril, agreste (diz"-se de solo).}{sá.fa.ro}{0}
\verb{sáfaro}{}{}{}{}{}{Difícil de amansar; bravio (diz"-se de animal).}{sá.fa.ro}{0}
\verb{sáfaro}{}{Fig.}{}{}{}{De trato difícil; indelicado, estranho, esquivo.}{sá.fa.ro}{0}
\verb{safena}{}{Anat.}{}{}{s.f.}{Cada uma das veias superficiais que drenam o sangue dos membros inferiores.}{sa.fe.na}{0}
\verb{safenado}{}{}{}{}{adj.}{Diz"-se de indivíduo que foi submetido a cirurgia de ponte de safena.}{sa.fe.na.do}{0}
\verb{sáfico}{}{}{}{}{adj.}{Relativo a Safo, poetisa grega.}{sá.fi.co}{0}
\verb{sáfico}{}{}{}{}{}{Relativo a safismo; lésbico.}{sá.fi.co}{0}
\verb{sáfico}{}{Gram.}{}{}{}{Tipo de verso decassílabo.}{sá.fi.co}{0}
\verb{safira}{}{}{}{}{s.f.}{Pedra preciosa de cor azul.}{sa.fi.ra}{0}
\verb{safira}{}{Por ext.}{}{}{}{A cor azul.}{sa.fi.ra}{0}
\verb{safismo}{}{}{}{}{s.m.}{Homossexualismo entre mulheres; lesbianismo.}{sa.fis.mo}{0}
\verb{safo}{}{}{}{}{adj.}{Que se safou; livre.}{sa.fo}{0}
\verb{safo}{}{Bras.}{}{}{}{Esperto, vivo, desembaraçado.}{sa.fo}{0}
\verb{safra}{}{}{}{}{s.f.}{A produção agrícola de determinado período; colheita.}{sa.fra}{0}
\verb{saga}{}{}{}{}{s.f.}{Designação comum das narrativas e lendas escandinavas escritas principalmente nos séculos \textsc{xiii} e \textsc{xiv}.}{sa.ga}{0}
\verb{saga}{}{Por ext.}{}{}{}{História rica em incidentes.}{sa.ga}{0}
\verb{sagacidade}{}{}{}{}{s.f.}{Qualidade ou procedimento de sagaz; perspicácia.}{sa.ga.ci.da.de}{0}
\verb{sagaz}{}{}{}{}{adj.}{Perspicaz, arguto, astucioso, esperto.}{sa.gaz}{0}
\verb{sagital}{}{}{"-ais}{}{adj.2g.}{Que tem forma de seta.}{sa.gi.tal}{0}
\verb{sagital}{}{Anat.}{"-ais}{}{}{Diz"-se do plano que se define a partir de um corte vertical reto da parte anterior para a posterior do corpo humano.}{sa.gi.tal}{0}
\verb{sagitariano}{}{Astrol.}{}{}{s.m.}{Indivíduo que nasceu sob o signo de sagitário.}{sa.gi.ta.ri.a.no}{0}
\verb{sagitariano}{}{Astrol.}{}{}{adj.}{Relativo ou pertencente a esse signo.}{sa.gi.ta.ri.a.no}{0}
\verb{sagitário}{}{Astron.}{}{}{s.m.}{Nona constelação zodiacal.}{sa.gi.tá.rio}{0}
\verb{sagitário}{}{Astrol.}{}{}{}{O signo do zodíaco referente a essa constelação.}{sa.gi.tá.rio}{0}
\verb{sagitífero}{}{}{}{}{adj.}{Munido ou armado de setas.}{sa.gi.tí.fe.ro}{0}
\verb{sagração}{}{}{"-ões}{}{s.f.}{Ato ou efeito de sagrar em cerimônia religiosa.}{sa.gra.ção}{0}
\verb{sagração}{}{}{"-ões}{}{}{A própria cerimônia.}{sa.gra.ção}{0}
\verb{sagrado}{}{}{}{}{adj.}{Relativo às coisas divinas; santo, sacro.}{sa.gra.do}{0}
\verb{sagrado}{}{}{}{}{}{Que se sagrou ou foi consagrado.}{sa.gra.do}{0}
\verb{sagrado}{}{}{}{}{}{Que não se deve infringir ou que não se pode deixar de cumprir; inviolável, intocável.}{sa.gra.do}{0}
\verb{sagrar}{}{}{}{}{v.t.}{Dedicar ao serviço divino; consagrar.}{sa.grar}{0}
\verb{sagrar}{}{}{}{}{}{Santificar, benzer, consagrar.}{sa.grar}{0}
\verb{sagrar}{}{}{}{}{}{Investir numa dignidade por meio de cerimônia religiosa.}{sa.grar}{0}
\verb{sagrar}{}{}{}{}{}{Tornar venerado ou respeitado.}{sa.grar}{\verboinum{1}}
\verb{sagu}{}{}{}{}{s.m.}{Amido que se extrai da parte central do sagueiro.}{sa.gu}{0}
\verb{saguão}{}{}{"-ões}{}{s.m.}{Área coberta à entrada de uma edificação e que conduz aos corredores e escadas.}{sa.guão}{0}
\verb{saguão}{}{}{"-ões}{}{}{Área descoberta no interior de uma edificação; pátio.}{sa.guão}{0}
\verb{sagueiro}{ê}{}{}{}{s.m.}{Designação comum a diversas palmeiras de cujo caule se extrai o sagu.}{sa.guei.ro}{0}
\verb{sagui}{}{Zool.}{}{}{s.m.}{Designação comum a diversos primatas pequenos, de cauda longa, dedo polegar não oponível e que se alimentam de insetos e frutas.}{sa.gui}{0}
\verb{saguim}{}{}{}{}{}{Var. de \textit{sagui}.}{sa.guim}{0}
\verb{saí}{}{Zool.}{}{}{s.f.}{Designação comum a vários pássaros de cores brilhantes esverdeadas ou azuladas.}{sa.í}{0}
\verb{saí}{}{}{}{}{s.f.}{Monge budista; bonzo.}{sa.í}{0}
\verb{saia}{}{}{}{}{s.f.}{Parte do vestuário feminino que é presa à cintura e desce, com diferentes comprimentos, sobre as pernas.}{sai.a}{0}
\verb{saia"-calça}{}{}{}{}{s.f.}{Calça feminina que, apresentando corte largo, assemelha"-se a uma saia.}{sai.a"-cal.ça}{0}
\verb{saião}{}{Bot.}{"-ões}{}{s.m.}{Planta com folhas carnosas em roseta e flores amarelas, cultivada como ornamental.}{sai.ão}{0}
\verb{saibo}{}{}{}{}{s.m.}{Sabor.}{sai.bo}{0}
\verb{saibrar}{}{}{}{}{v.t.}{Cobrir com saibro.}{sai.brar}{0}
\verb{saibrar}{}{}{}{}{}{Escavar a terra para o plantio de mudas de estaca.}{sai.brar}{\verboinum{1}}
\verb{saibro}{}{}{}{}{s.m.}{Areia grossa de rio.}{sai.bro}{0}
\verb{saibroso}{ô}{}{"-osos ⟨ó⟩}{"-osa ⟨ó⟩}{adj.}{Que tem saibro.}{sai.bro.so}{0}
\verb{saída}{}{}{}{}{s.f.}{Ato ou efeito de sair.}{sa.í.da}{0}
\verb{saída}{}{}{}{}{}{Lugar por onde se sai.}{sa.í.da}{0}
\verb{saída}{}{}{}{}{}{Momento em que se sai.}{sa.í.da}{0}
\verb{saída}{}{}{}{}{}{Modo de superar um obstáculo; solução, recurso, artifício.}{sa.í.da}{0}
\verb{saída}{}{}{}{}{}{Peça de vestuário que se usa para proteger do frio no momento de sair de certos lugares.}{sa.í.da}{0}
\verb{saída"-de"-banho}{}{}{saídas"-de"-banho}{}{s.f.}{Roupão, em geral atoalhado, que se usa sobre maiô, biquíni etc. para sair de piscina ou praia.}{sa.í.da"-de"-ba.nho}{0}
\verb{saída"-de"-praia}{}{}{saídas"-de"-praia}{}{s.f.}{Ver \textit{saída"-de"-banho}.}{sa.í.da"-de"-prai.a}{0}
\verb{saideira}{ê}{Bras.}{}{}{s.f.}{Última dose de bebida alcoólica que se bebe antes de sair de um bar ou festa.}{sa.i.dei.ra}{0}
\verb{saideira}{ê}{}{}{}{}{Última dança de um baile.}{sa.i.dei.ra}{0}
\verb{saído}{}{}{}{}{adj.}{Que saiu; ausente, afastado.}{sa.í.do}{0}
\verb{saído}{}{}{}{}{}{Que sobressai; saliente.}{sa.í.do}{0}
\verb{saído}{}{Pop.}{}{}{}{Intrometido, atrevido, abelhudo.}{sa.í.do}{0}
\verb{saimento}{}{}{}{}{s.m.}{Funeral, enterro.}{sa.i.men.to}{0}
\verb{saimento}{}{Pop.}{}{}{}{Qualidade ou procedimento de indivíduo saído; atrevimento.}{sa.i.men.to}{0}
\verb{sainete}{ê}{}{}{}{s.m.}{Gosto especial; sabor.}{sai.ne.te}{0}
\verb{sainete}{ê}{}{}{}{}{Tipo de peça teatral cômica curta, com dois ou três personagens.}{sai.ne.te}{0}
\verb{saiote}{ó}{}{}{}{s.m.}{Roupa íntima feminina que se usa por baixo do vestido ou da saia; anágua.}{sai.o.te}{0}
\verb{sair}{}{}{}{}{v.i.}{Passar de dentro de um lugar para outro. (\textit{Saímos da sala e fomos para a cozinha.})}{sa.ir}{0}
\verb{sair}{}{}{}{}{}{Aparecer publicado. (\textit{A notícia saiu em todos os jornais.})}{sa.ir}{0}
\verb{sair}{}{}{}{}{}{Deixar o emprego; demitir"-se. (\textit{Eu saí do emprego depois do ocorrido.})}{sa.ir}{0}
\verb{sair}{}{}{}{}{v.pron.}{Conseguir um resultado, bom ou mau. (\textit{Não sei como ele se saiu nas provas de fim de ano.})}{sa.ir}{\verboinum{19}}
\verb{saíra}{}{Zool.}{}{}{s.f.}{Ver \textit{saí}.}{sa.í.ra}{0}
\verb{sais}{}{}{}{}{s.m.pl.}{Substâncias voláteis que têm a propriedade de reanimar pessoas desfalecidas.}{sais}{0}
%\verb{}{}{}{}{}{}{}{}{0}
%\verb{}{}{}{}{}{}{}{}{0}
\verb{sal}{}{Quím.}{sais}{}{s.m.}{Substância formada pela reação entre um ácido e uma base.}{sal}{0}
\verb{sal}{}{}{sais}{}{}{Cloreto de sódio, ou sal de cozinha, usado como tempero na culinária.}{sal}{0}
\verb{sal}{}{Fig.}{sais}{}{}{Graça, charme, vivacidade. \textit{abon}}{sal}{0}
\verb{sala}{}{}{}{}{s.f.}{Dependência geralmente ampla de uma residência, destinada ao uso social.}{sa.la}{0}
\verb{sala}{}{}{}{}{}{Local onde aulas são ministradas; sala de aula, classe.}{sa.la}{0}
\verb{sala}{}{}{}{}{}{Local para realização de audições de música, projeções de cinema, espetáculos.}{sa.la}{0}
\verb{sala}{}{}{}{}{}{Qualquer dependência ampla de uma edificação.}{sa.la}{0}
\verb{salacidade}{}{}{}{}{s.f.}{Qualidade de salaz; devassidão.}{sa.la.ci.da.de}{0}
\verb{salada}{}{}{}{}{s.f.}{Prato que se serve frio, preparado com verduras, legumes, ovos, carne, peixe etc.}{sa.la.da}{0}
\verb{salada}{}{Pop.}{}{}{}{Confusão, trapalhada.}{sa.la.da}{0}
\verb{saladeira}{ê}{}{}{}{s.f.}{Recipiente para servir a salada à mesa.}{sa.la.dei.ra}{0}
\verb{salafrário}{}{Pop.}{}{}{s.m.}{Homem desonesto, ordinário; patife, safardana.}{sa.la.frá.rio}{0}
\verb{salamaleque}{é}{}{}{}{s.m.}{Saudação cerimoniosa entre os muçulmanos.}{sa.la.ma.le.que}{0}
\verb{salamaleque}{é}{Fig.}{}{}{}{Cortesia ou polidez exagerada.}{sa.la.ma.le.que}{0}
\verb{salamandra}{}{Zool.}{}{}{s.f.}{Animal anfíbio de corpo alongado e cauda longa, encontrado em regiões temperadas.}{sa.la.man.dra}{0}
\verb{salamandra}{}{}{}{}{s.m.}{Operário que entra em caldeiras ou locais quentes para fazer manutenção.}{sa.la.man.dra}{0}
\verb{sal"-amargo}{}{Quím.}{sais"-amargos}{}{s.m.}{Sulfato de magnésio.}{sal"-a.mar.go}{0}
\verb{salame}{}{Cul.}{}{}{s.m.}{Embutido apimentado, de carne de porco ou de boi, geralmente defumado, e que se come frio.}{sa.la.me}{0}
\verb{salaminho}{}{}{}{}{s.m.}{Tipo de salame feito em tripa fina e curta.}{sa.la.mi.nho}{0}
\verb{salão}{}{}{"-ões}{}{s.m.}{Grande sala, geralmente destinada a reuniões, festas, visitas etc.}{sa.lão}{0}
\verb{salão}{}{}{"-ões}{}{}{Estabelecimento comercial de barbeiro ou cabeleireiro; barbearia.}{sa.lão}{0}
\verb{salão}{}{}{"-ões}{}{}{Exposição periódica de livros, artes, produtos industriais etc.}{sa.lão}{0}
\verb{salariado}{}{}{}{}{adj.}{Relativo a salário; assalariado.}{sa.la.ri.a.do}{0}
\verb{salariado}{}{}{}{}{s.m.}{Empregado ou operário assalariado.}{sa.la.ri.a.do}{0}
\verb{salariado}{}{}{}{}{}{O conjunto dos trabalhadores assalariados.}{sa.la.ri.a.do}{0}
\verb{salarial}{}{}{"-ais}{}{adj.2g.}{Relativo a salário.}{sa.la.ri.al}{0}
\verb{salário}{}{}{}{}{s.m.}{Remuneração paga pelo empregador por serviços prestados, geralmente mediante contrato de trabalho.}{sa.lá.rio}{0}
\verb{salário}{}{}{}{}{}{Recompensa de serviços.}{sa.lá.rio}{0}
\verb{salaz}{}{}{}{}{adj.2g.}{Libertino, devasso, luxurioso.}{sa.laz}{0}
\verb{salchicha}{}{}{}{}{}{Var de \textit{salsicha}.}{sal.chi.cha}{0}
\verb{salchicharia}{}{}{}{}{}{Var de \textit{salsicharia}.}{sal.chi.cha.ri.a}{0}
\verb{salchicheiro}{ê}{}{}{}{}{Var de \textit{salsicheiro}.}{sal.chi.chei.ro}{0}
\verb{saldar}{}{}{}{}{v.t.}{Pagar o saldo; liquidar (contas).}{sal.dar}{0}
\verb{saldar}{}{}{}{}{}{Liquidar uma dívida moral; tirar satisfação; vingar.}{sal.dar}{\verboinum{1}}
\verb{saldo}{}{}{}{}{s.m.}{Diferença entre débito e crédito.}{sal.do}{0}
\verb{saldo}{}{}{}{}{}{Quantia que resta pagar ou receber; sobra, resto.}{sal.do}{0}
\verb{saldo}{}{}{}{}{}{Parte do estoque de certas mercadorias, vendida com desconto.}{sal.do}{0}
\verb{saleiro}{ê}{}{}{}{adj.}{Relativo a sal.}{sa.lei.ro}{0}
\verb{saleiro}{ê}{}{}{}{s.m.}{Pequeno utensílio doméstico em que se põe sal.}{sa.lei.ro}{0}
\verb{saleiro}{ê}{}{}{}{}{Indivíduo que fabrica ou vende sal.}{sa.lei.ro}{0}
\verb{salesiano}{}{}{}{}{adj.}{Relativo à Congregação de São Francisco de Sales, fundada por São João Bosco, e que se destina à educação de jovens.}{sa.le.si.a.no}{0}
\verb{salesiano}{}{}{}{}{s.m.}{Membro dessa congregação.}{sa.le.si.a.no}{0}
\verb{saleta}{ê}{}{}{}{s.f.}{Pequena sala.}{sa.le.ta}{0}
\verb{salga}{}{}{}{}{s.f.}{Ato ou efeito de salgar, de temperar com sal.}{sal.ga}{0}
\verb{salgadinho}{}{}{}{}{s.m.}{Iguaria de tamanho pequeno, salgada, tal como amendoim, canapés, empada etc., geralmente servida como aperitivo.}{sal.ga.di.nho}{0}
%\verb{}{}{}{}{}{}{}{}{0}
\verb{salgado}{}{}{}{}{adj.}{Que tem sal.}{sal.ga.do}{0}
\verb{salgado}{}{}{}{}{}{Que tem excesso de sal.}{sal.ga.do}{0}
\verb{salgado}{}{}{}{}{}{Conservado em sal.}{sal.ga.do}{0}
\verb{salgado}{}{Fig.}{}{}{}{Que tem o tom malicioso; picante.}{sal.ga.do}{0}
\verb{salgado}{}{Fig.}{}{}{}{Muito caro, de preço exagerado.}{sal.ga.do}{0}
\verb{salgadura}{}{}{}{}{s.f.}{Ver \textit{salga}.}{sal.ga.du.ra}{0}
\verb{salgar}{}{}{}{}{v.t.}{Temperar com sal.}{sal.gar}{0}
\verb{salgar}{}{}{}{}{}{Conservar em sal.}{sal.gar}{0}
\verb{salgar}{}{}{}{}{}{Tornar salgado, impregnar de sal.}{sal.gar}{\verboinum{5}}
\verb{sal"-gema}{ê}{Quím.}{sais"-gemas}{}{s.m.}{Cloreto de sódio (sal de cozinha) extraído das minas, usado na obtenção de compostos de cloro e de sódio.}{sal"-ge.ma}{0}
\verb{salgueiro}{ê}{Bot.}{}{}{s.m.}{Árvore ou arbusto, de folhas delgadas, longos ramos pendentes, que cresce nos terrenos úmidos ou à beira dos rios, cultivada como ornamental ou pelas madeiras.}{sal.guei.ro}{0}
\verb{salicácea}{}{Bot.}{}{}{s.f.}{Espécime das salicáceas, árvores e arbustos, com pequenas flores sem pétalas, cultivados como medicinais, ornamentais ou pelas madeiras. }{sa.li.cá.cea}{0}
\verb{salicílico}{}{Quím.}{}{}{adj.}{Diz"-se de ácido usado em medicina, na indústria de fármacos e também de corantes.}{sa.li.cí.li.co}{0}
\verb{salicultura}{}{}{}{}{s.f.}{Produção do sal em salinas.}{sa.li.cul.tu.ra}{0}
\verb{saliência}{}{}{}{}{s.f.}{Qualidade de saliente.}{sa.li.ên.cia}{0}
\verb{saliência}{}{}{}{}{}{Parte em relevo numa superfície lisa; proeminência, ressalto.}{sa.li.ên.cia}{0}
\verb{salientar}{}{}{}{}{v.t.}{Tornar saliente ou notável; distinguir, evidenciar.}{sa.li.en.tar}{0}
\verb{salientar}{}{}{}{}{}{Tornar saliente, bem visível ou distinto; destacar.}{sa.li.en.tar}{\verboinum{1}}
\verb{saliente}{}{}{}{}{adj.2g.}{Que avança, que se sobressai do plano em que se assenta.}{sa.li.en.te}{0}
\verb{saliente}{}{Fig.}{}{}{}{Que chama atenção; notável, distinto, evidente.}{sa.li.en.te}{0}
\verb{saliente}{}{Fig.}{}{}{}{Digno de mérito; importante, valioso.}{sa.li.en.te}{0}
\verb{salificar}{}{Quím.}{}{}{v.t.}{Tratar um ácido por uma base.}{sa.li.fi.car}{0}
\verb{salificar}{}{}{}{}{}{Transformar uma substância em sal.}{sa.li.fi.car}{\verboinum{2}}
\verb{salina}{}{}{}{}{s.f.}{Local onde se produz sal por evaporação da água do mar, ou de lago de água salgada.}{sa.li.na}{0}
\verb{salina}{}{Por ext.}{}{}{}{Empresa que explora esse tipo de atividade.}{sa.li.na}{0}
\verb{salina}{}{}{}{}{}{Monte de sal.}{sa.li.na}{0}
\verb{salina}{}{Fig.}{}{}{}{Coisa salgada demais.}{sa.li.na}{0}
\verb{salineiro}{ê}{}{}{}{adj.}{Relativo a salina.}{sa.li.nei.ro}{0}
\verb{salineiro}{ê}{}{}{}{s.m.}{Indivíduo que fabrica, empilha ou vende sal.}{sa.li.nei.ro}{0}
\verb{salinidade}{}{}{}{}{s.f.}{Qualidade de salino.}{sa.li.ni.da.de}{0}
\verb{salinidade}{}{}{}{}{}{Teor de sal num determinado meio.}{sa.li.ni.da.de}{0}
\verb{salinidade}{}{}{}{}{}{Concentração de sais minerais nas águas do mar.}{sa.li.ni.da.de}{0}
\verb{salino}{}{}{}{}{adj.}{Que contém sal, é da natureza do sal ou é próprio dele. (\textit{O ar salino da praia enferruja o automóvel.})}{sa.li.no}{0}
\verb{salino}{}{}{}{}{}{Que nasceu à beira"-mar.}{sa.li.no}{0}
\verb{salitrado}{}{}{}{}{adj.}{Que contém ou foi impregnado de salitre.}{sa.li.tra.do}{0}
\verb{salitre}{}{Quím.}{}{}{s.m.}{Designação vulgar do nitrato de potássio.}{sa.li.tre}{0}
\verb{saliva}{}{}{}{}{s.f.}{Líquido transparente e insípido segregado pelas glândulas salivares, que atua sobre os alimentos para facilitar sua digestão.}{sa.li.va}{0}
\verb{salivação}{}{}{"-ões}{}{s.f.}{Ato ou efeito de salivar, de produzir saliva.}{sa.li.va.ção}{0}
\verb{salivar}{}{}{}{}{adj.2g.}{Relativo à saliva ou às glândulas salivares.}{sa.li.var}{0}
\verb{salivar}{}{}{}{}{v.i.}{Expelir saliva.}{sa.li.var}{0}
\verb{salivar}{}{}{}{}{v.t.}{Molhar com saliva.}{sa.li.var}{\verboinum{1}}
\verb{salmão}{}{Zool.}{"-ões}{}{s.m.}{Peixe marinho do Atlântico Norte, de carne rosada e muito saborosa.}{sal.mão}{0}
\verb{salmão}{}{}{"-ões}{}{}{A cor da carne do salmão.}{sal.mão}{0}
\verb{salmista}{}{}{}{}{s.2g.}{Indivíduo que compõe salmos.}{sal.mis.ta}{0}
\verb{salmo}{}{}{}{}{s.m.}{Cântico sagrado dos hebreus, acompanhado por instrumento de cordas ou de sopro.}{sal.mo}{0}
\verb{salmo}{}{Relig.}{}{}{}{Cada um dos 150 poemas líricos do Livro dos Salmos no Antigo Testamento, atribuídos, em sua maior parte, ao rei Davi, e que foram musicados para uso religioso.}{sal.mo}{0}
\verb{salmo}{}{Relig.}{}{}{}{Oração em gênero poético, cuja característica é o duplo ritmo, o da palavra e o das ideias, para ser acompanhada pelo saltério.}{sal.mo}{0}
\verb{salmodiar}{}{}{}{}{v.i.}{Cantar ou recitar salmos.}{sal.mo.di.ar}{\verboinum{6}}
\verb{salmoira}{ô}{}{}{}{s.f.}{Ver \textit{salmoura}.}{sal.moi.ra}{0}
\verb{salmonela}{é}{Biol.}{}{}{s.f.}{Gênero das bactérias anaeróbias, flageladas, em forma de bastonetes, comuns na natureza, em animais e em alimentos.}{sal.mo.ne.la}{0}
\verb{salmoura}{ô}{}{}{}{s.f.}{Água saturada de sal para conservar alimentos.}{sal.mou.ra}{0}
\verb{salmoura}{ô}{}{}{}{}{Vasilha onde se faz essa conservação.}{sal.mou.ra}{0}
\verb{salmoura}{ô}{}{}{}{}{Líquido que escorre da carne ou do peixe salgados.}{sal.mou.ra}{0}
\verb{salobre}{ô}{}{}{}{}{Var. de \textit{salobro}.}{sa.lo.bre}{0}
\verb{salobro}{ô}{}{}{}{adj.}{Que tem certo sabor de sal.}{sa.lo.bro}{0}
\verb{salobro}{ô}{}{}{}{}{Diz"-se de água que tem em dissolução alguns sais ou substâncias que a tornam desagradável ao paladar.}{sa.lo.bro}{0}
\verb{saloio}{ô}{}{}{}{adj.}{Diz"-se do camponês dos arredores de Lisboa, capital de Portugal.}{sa.loi.o}{0}
\verb{saloio}{ô}{Fig.}{}{}{}{Diz"-se de camponês, indivíduo rústico.}{sa.loi.o}{0}
\verb{saloio}{ô}{Fig.}{}{}{s.m.}{Camponês, aldeão rústico.}{sa.loi.o}{0}
\verb{salomônico}{}{}{}{}{adj.}{Relativo a Salomão, filho de Davi, terceiro rei dos judeus, considerado sábio e criterioso.}{sa.lo.mô.ni.co}{0}
\verb{salomônico}{}{}{}{}{}{Relativo às Ilhas Salomão (arquipélago do Pacífico Sul).}{sa.lo.mô.ni.co}{0}
\verb{salomônico}{}{}{}{}{s.m.}{Indivíduo natural ou habitante dessas ilhas.}{sa.lo.mô.ni.co}{0}
\verb{salpicão}{}{Bras.}{"-ões}{}{s.m.}{Espécie de salada à base de galinha desfiada, peixe, crustáceos ou carne, com batatas, pimentões etc., bastante tempero, em geral servida com maionese ou creme de leite.}{sal.pi.cão}{0}
\verb{salpicar}{}{}{}{}{v.t.}{Temperar, espalhando gotas salgadas ou pedras de sal.}{sal.pi.car}{0}
\verb{salpicar}{}{}{}{}{}{Polvilhar com pitadas.}{sal.pi.car}{0}
\verb{salpicar}{}{}{}{}{}{Espalhar com pingos ou partículas.}{sal.pi.car}{0}
\verb{salpicar}{}{}{}{}{}{Espalhar pequenas manchas.}{sal.pi.car}{\verboinum{1}}
\verb{salpico}{}{}{}{}{s.m.}{Ato ou efeito de salpicar.}{sal.pi.co}{0}
\verb{salpico}{}{}{}{}{}{Pedra de sal com que se salga o peixe ou a carne.}{sal.pi.co}{0}
\verb{salpico}{}{}{}{}{}{Vestígio em forma de pingo deixado por um líquido noutro corpo.}{sal.pi.co}{0}
\verb{salsa}{}{}{}{}{s.f.}{Erva aromática rica em vitamina \textsc{c} e utilizada na culinária como condimento.}{sal.sa}{0}
\verb{salsa}{}{}{}{}{s.f.}{Gênero de música surgido em Cuba na década de 1940.}{sal.sa}{0}
\verb{salsada}{}{Pop.}{}{}{s.f.}{Confusão, trapalhada, embrulhada.}{sal.sa.da}{0}
\verb{salsaparrilha}{}{}{}{}{s.f.}{Planta de cuja raiz se extrai uma substância utilizada como depurativo.}{sal.sa.par.ri.lha}{0}
\verb{salseira}{ê}{}{}{}{s.f.}{Recipiente para servir molhos à mesa; molheira.}{sal.sei.ra}{0}
\verb{salseiro}{ê}{}{}{}{s.m.}{Chuva forte, localizada, repentina e de curta duração.}{sal.sei.ro}{0}
\verb{salseiro}{ê}{Bras.}{}{}{}{Confusão, motim, briga, desordem.}{sal.sei.ro}{0}
\verb{salseiro}{ê}{Lus.}{}{}{}{Vento baixo e forte.}{sal.sei.ro}{0}
\verb{salsicha}{}{}{}{}{s.f.}{Embutido de pequeno diâmetro feito de carne de porco moída com sal e temperos.}{sal.si.cha}{0}
\verb{salsicha}{}{Desus.}{}{}{}{Estopim para atear fogo às minas.}{sal.si.cha}{0}
\verb{salsicha}{}{Bras.}{}{}{}{Ver \textit{bassê}.}{sal.si.cha}{0}
\verb{salsichão}{}{}{"-ões}{}{s.m.}{Embutido semelhante à salsicha com diâmetro maior.}{sal.si.chão}{0}
\verb{salsicharia}{}{}{}{}{s.f.}{Técnica de preparar embutidos.}{sal.si.cha.ri.a}{0}
\verb{salsicharia}{}{}{}{}{}{Estabelecimento onde se vende esse tipo de alimento.}{sal.si.cha.ri.a}{0}
\verb{salsicheiro}{ê}{}{}{}{s.m.}{Indivíduo que fabrica ou vende produtos de salsicharia.}{sal.si.chei.ro}{0}
\verb{salsinha}{}{}{}{}{s.f.}{Erva aromática rica em vitamina \textsc{c} e utilizada na culinária como condimento; salsa.}{sal.si.nha}{0}
\verb{salsinha}{}{Pop.}{}{}{}{Homem afeminado; maricas.}{sal.si.nha}{0}
\verb{salso}{}{Liter.}{}{}{adj.}{Salgado.}{sal.so}{0}
\verb{salsugem}{}{}{"-ens}{}{s.f.}{Qualidade do que é salso.}{sal.su.gem}{0}
\verb{salsugem}{}{}{"-ens}{}{}{Detritos que flutuam nas proximidades de praias e portos.}{sal.su.gem}{0}
\verb{salsuginoso}{ô}{}{"-osos ⟨ó⟩}{"-osa ⟨ó⟩}{adj.}{Repleto de salsugem.}{sal.su.gi.no.so}{0}
\verb{saltado}{}{}{}{}{adj.}{Que está fora do nível ou do alinhamento; saliente, ressaltado.}{sal.ta.do}{0}
\verb{saltão}{}{}{"-ões}{}{adj.}{Que dá muitos ou grandes saltos.}{sal.tão}{0}
\verb{saltão}{}{Bras.}{"-ões}{}{s.m.}{Gafanhoto jovem, que ainda não criou asas e se locomove aos saltos.}{sal.tão}{0}
\verb{saltar}{}{}{}{}{v.t.}{Atravessar ou passar por cima pulando.}{sal.tar}{0}
\verb{saltar}{}{Fig.}{}{}{}{Deixar de considerar; ignorar, omitir. \textit{abon}}{sal.tar}{0}
\verb{saltar}{}{}{}{}{}{Descer ou apear de meio de transporte. \textit{abon?}}{sal.tar}{0}
\verb{saltar}{}{}{}{}{}{Passar de determinada posição ou estado a outro sem atravessar as etapas intermediárias. \textit{abon?}}{sal.tar}{0}
\verb{saltar}{}{}{}{}{v.i.}{Dar pulos.}{sal.tar}{\verboinum{1}}
\verb{salteado}{}{}{}{}{adj.}{Atacado de maneira imprevista; assaltado.}{sal.te.a.do}{0}
\verb{salteado}{}{}{}{}{}{Não sucessivo; alternado.}{sal.te.a.do}{0}
\verb{salteador}{ô}{}{}{}{adj.}{Que salteia.}{sal.te.a.dor}{0}
\verb{salteador}{ô}{}{}{}{s.m.}{Ladrão de estrada.}{sal.te.a.dor}{0}
\verb{saltear}{}{}{}{}{v.t.}{Atacar de maneira imprevista; assaltar.}{sal.te.ar}{\verboinum{4}}
\verb{saltério}{}{Mús.}{}{}{s.m.}{Instrumento de forma triangular com treze ordens de cordas.}{sal.té.rio}{0}
\verb{saltério}{}{Mús.}{}{}{}{Instrumento de cordas dedilháveis mencionado no Velho Testamento.}{sal.té.rio}{0}
\verb{saltimbanco}{}{}{}{}{s.m.}{Artista popular intinerante, que se apresenta em grupo ou por conta própria.}{sal.tim.ban.co}{0}
\verb{saltitante}{}{}{}{}{adj.2g.}{Que saltita.}{sal.ti.tan.te}{0}
\verb{saltitante}{}{Fig.}{}{}{}{Irrequieto, agitado.}{sal.ti.tan.te}{0}
\verb{saltitante}{}{Fig.}{}{}{}{Ostensivamente feliz; radiante.}{sal.ti.tan.te}{0}
\verb{saltitar}{}{}{}{}{v.i.}{Dar saltos pequenos e de maneira repetida.}{sal.ti.tar}{\verboinum{1}}
\verb{salto}{}{}{}{}{s.m.}{Pulo.}{sal.to}{0}
\verb{salto}{}{}{}{}{}{Transição rápida e brusca.}{sal.to}{0}
\verb{salto}{}{}{}{}{}{Parte saliente da sola de um calçado na direção do calcanhar.}{sal.to}{0}
\verb{salto}{}{}{}{}{}{Queda"-d'água, cachoeira, cascata. }{sal.to}{0}
\verb{salto"-mortal}{}{}{saltos"-mortais}{}{s.m.}{Acrobacia em que se dá uma volta completa no ar.}{sal.to"-mor.tal}{0}
\verb{salubre}{}{}{}{}{adj.}{Que faz bem à saúde; saudável, sadio.}{sa.lu.bre}{0}
\verb{salubre}{}{}{}{}{}{Facilmente curável.}{sa.lu.bre}{0}
\verb{salubridade}{}{}{}{}{s.f.}{Qualidade de salubre.}{sa.lu.bri.da.de}{0}
\verb{salubridade}{}{}{}{}{}{Conjunto de condições propícias à saúde pública.}{sa.lu.bri.da.de}{0}
\verb{salutar}{}{}{}{}{adj.2g.}{Que faz bem para a saúde; salubre.}{sa.lu.tar}{0}
\verb{salutar}{}{Fig.}{}{}{}{Edificante, construtivo.}{sa.lu.tar}{0}
\verb{salva}{}{}{}{}{s.f.}{Descarga de tiros de fuzil ou canhão em honra de alguém.}{sal.va}{0}
\verb{salva}{}{}{}{}{}{Reserva, ressalva, restrição.}{sal.va}{0}
\verb{salva}{}{}{}{}{}{Tipo de bandeja pequena.}{sal.va}{0}
\verb{salvação}{}{}{"-ões}{}{s.f.}{Ato ou efeito de salvar.}{sal.va.ção}{0}
\verb{salvação}{}{}{"-ões}{}{}{Ato ou efeito de saudar; saudação.}{sal.va.ção}{0}
\verb{salvação}{}{}{"-ões}{}{}{Saída de uma situação adversa; triunfo, redenção, resgate.}{sal.va.ção}{0}
\verb{salvação}{}{Relig.}{"-ões}{}{}{Felicidade eterna obtida após a morte.}{sal.va.ção}{0}
\verb{salvador}{ô}{}{}{}{adj.}{Que salva, ampara, protege.}{sal.va.dor}{0}
\verb{salvador}{ô}{Relig.}{}{}{s.m.}{Na religião católica, epíteto dado a Jesus Cristo, que veio ao mundo para salvar os homens. (Usa"-se inicial maiúscula, nesta acepção.)}{sal.va.dor}{0}
\verb{salvadorenho}{}{}{}{}{adj.}{Relativo a El Salvador.}{sal.va.do.re.nho}{0}
\verb{salvadorenho}{}{}{}{}{s.m.}{Indivíduo natural ou habitante desse país.}{sal.va.do.re.nho}{0}
\verb{salvadorense}{}{}{}{}{adj.2g.}{Relativo a Salvador, capital da Bahia; soteropolitano.}{sal.va.do.ren.se}{0}
\verb{salvadorense}{}{}{}{}{s.2g.}{Indivíduo natural ou habitante dessa cidade.}{sal.va.do.ren.se}{0}
\verb{salvados}{}{}{}{}{s.m.pl.}{Os restos que escaparam de uma catástrofe.}{sal.va.dos}{0}
\verb{salvaguarda}{}{}{}{}{s.f.}{Coisa ou pessoa que protege; proteção, amparo, imunidade.}{sal.va.guar.da}{0}
\verb{salvaguardar}{}{}{}{}{v.t.}{Pôr ou manter fora de perigo; proteger, defender.}{sal.va.guar.dar}{\verboinum{1}}
\verb{salvamento}{}{}{}{}{s.m.}{Ato ou efeito de salvar; operação de resgate.}{sal.va.men.to}{0}
\verb{salvamento}{}{}{}{}{}{Lugar seguro.}{sal.va.men.to}{0}
\verb{salvamento}{}{}{}{}{}{Sucesso, êxito.}{sal.va.men.to}{0}
\verb{salvar}{}{}{}{}{v.t.}{Tirar de perigo ou dificuldade.}{sal.var}{0}
\verb{salvar}{}{}{}{}{}{Defender, proteger, preservar.}{sal.var}{0}
\verb{salvar}{}{}{}{}{}{Saudar (com salvas).}{sal.var}{0}
\verb{salvar}{}{Informát.}{}{}{}{Gravar os dados para posterior consulta.}{sal.var}{0}
\verb{salvar}{}{}{}{}{v.i.}{Dar salvas de artilharia.}{sal.var}{0}
\verb{salvar}{}{Relig.}{}{}{v.pron.}{Obter a salvação eterna.}{sal.var}{\verboinum{1}}
\verb{salva"-vidas}{}{}{}{}{s.2g.}{Embarcação, boia ou equipamento destinado ao salvamento de náufragos.}{sal.va"-vi.das}{0}
\verb{salva"-vidas}{}{}{}{}{}{Profissional que, nas praias de banho, está a serviço dos postos de salvamento.}{sal.va"-vi.das}{0}
\verb{salve}{}{}{}{}{interj.}{Expressão usada para saudar alguém.}{sal.ve}{0}
\verb{salve"-rainha}{}{Relig.}{salve"-rainhas}{}{s.f.}{Oração católica dedicada à Virgem Maria.}{sal.ve"-ra.i.nha}{0}
\verb{sálvia}{}{Bot.}{}{}{s.f.}{Erva ornamental com propriedades medicinais utilizada na culinária e na fabricação de bebidas.}{sál.vi.a}{0}
\verb{salvo}{}{}{}{}{adj.}{Que se salvou.}{sal.vo}{0}
\verb{salvo}{}{}{}{}{}{Livre de perigo; resguardado, protegido.}{sal.vo}{0}
\verb{salvo}{}{Relig.}{}{}{}{Que obteve a salvação eterna.}{sal.vo}{0}
\verb{salvo}{}{}{}{}{prep.}{Exceto, afora, tirante; com exclusão de.}{sal.vo}{0}
\verb{salvo"-conduto}{}{Jur.}{salvo"-condutos \textit{e} salvos"-condutos}{}{s.m.}{Licença por escrito para alguém viajar ou transitar livremente.}{sal.vo"-con.du.to}{0}
\verb{salvo"-conduto}{}{Fig.}{salvo"-condutos \textit{e} salvos"-condutos}{}{}{Privilégio, imunidade, salvaguarda, isenção.}{sal.vo"-con.du.to}{0}
\verb{samambaia}{}{Bot.}{}{}{s.f.}{Planta ornamental formada de muitas folhas finas e compridas. (\textit{Ele adornou sua casa com samambaias.})}{sa.mam.bai.a}{0}
\verb{samarinês}{}{}{}{}{adj.}{Relativo à República de San Marino.}{sa.ma.ri.nês}{0}
\verb{samarinês}{}{}{}{}{s.m.}{Indivíduo natural ou habitante desse país.}{sa.ma.ri.nês}{0}
\verb{samário}{}{Quím.}{}{}{s.m.}{Elemento químico metálico, da família dos lantanídeos (terras"-raras); usado em reatores nucleares, ímãs permanentes, certos tipos de cerâmicas etc. \elemento{62}{150.36}{Sm}.}{sa.má.rio}{0}
\verb{samaritano}{}{}{}{}{adj.}{Relativo a Samaria, cidade e região da Palestina.}{sa.ma.ri.ta.no}{0}
\verb{samaritano}{}{}{}{}{s.m.}{Indivíduo natural ou habitante dessa cidade ou região.}{sa.ma.ri.ta.no}{0}
\verb{samaritano}{}{}{}{}{}{A língua dos samaritanos.}{sa.ma.ri.ta.no}{0}
\verb{samaritano}{}{Fig.}{}{}{}{Indivíduo bom e caridoso, disposto a socorrer os doentes e desamparados.}{sa.ma.ri.ta.no}{0}
\verb{samba}{}{Mús.}{}{}{s.m.}{Gênero musical em compasso binário e de origem africana.}{sam.ba}{0}
\verb{samba}{}{Bras.}{}{}{}{Baile popular onde se dança, ouve e canta samba.}{sam.ba}{0}
\verb{samba"-canção}{}{}{sambas"-canção \textit{ou} sambas"-canções}{}{s.m.}{Tipo de samba, mais lento e geralmente com letra muito sentimental.}{sam.ba"-can.ção}{0}
\verb{samba"-canção}{}{}{sambas"-canção \textit{ou} sambas"-canções}{}{adj.}{Diz"-se de tipo de cueca semelhante a uma bermuda.}{sam.ba"-can.ção}{0}
\verb{samba"-enredo}{ê}{}{sambas"-enredo \textit{ou} sambas"-enredos}{}{s.m.}{Samba composto especialmente para um desfile de carnaval, de acordo com o tema escolhido como enredo.}{sam.ba"-en.re.do}{0}
\verb{sambambaia}{}{}{}{}{}{Var. de \textit{samambaia}.}{sam.bam.bai.a}{0}
\verb{sambaqui}{}{Bras.}{}{}{s.m.}{Depósito pré"-histórico formado por grande quantidade de conchas, em que geralmente se encontram objetos de interesse arqueológico (ossos, utensílios), sendo, portanto, evidência de presença humana.}{sam.ba.qui}{0}
\verb{sambar}{}{}{}{}{v.i.}{Dançar samba.}{sam.bar}{0}
\verb{sambar}{}{Pop.}{}{}{}{Dar"-se mal; ser posto de lado; ser demitido; ser preso.}{sam.bar}{\verboinum{1}}
\verb{sambista}{}{Bras.}{}{}{s.2g.}{Compositor de sambas.}{sam.bis.ta}{0}
\verb{sambista}{}{}{}{}{}{Integrante ou frequentador de escola de samba.}{sam.bis.ta}{0}
\verb{sambista}{}{}{}{}{}{Exímio dançarino de samba.}{sam.bis.ta}{0}
\verb{sambódromo}{}{}{}{}{s.m.}{Construção, geralmente com arquibancadas, feita para abrigar desfiles de escolas de samba e o público de tal evento.}{sam.bó.dro.mo}{0}
\verb{samburá}{}{}{}{}{s.m.}{Cesto de boca estreita, feito de cipó ou taquara, usado por pescadores para recolher o produto da pesca ou seus utensílios.}{sam.bu.rá}{0}
\verb{samoano}{}{}{}{}{adj.}{Relativo às ilhas de Samoa, arquipélago do Centro"-Sul do Oceano Pacífico.}{sa.mo.a.no}{0}
\verb{samoano}{}{}{}{}{s.m.}{Indivíduo natural ou habitante desse arquipélago.}{sa.mo.a.no}{0}
\verb{samovar}{}{}{}{}{s.m.}{Utensílio doméstico de origem russa, usado para colocar brasa e aquecer e manter quente a água para o chá.}{sa.mo.var}{0}
\verb{samovar}{}{Bras.}{}{}{}{Tipo de bule de metal nobre sobre armação com fogareiro para aquecer água.}{sa.mo.var}{0}
\verb{samurai}{}{}{}{}{s.m.}{Guerreiro japonês, membro da casta militar, a serviço de um nobre.}{sa.mu.rai}{0}
\verb{sanar}{}{}{}{}{v.t.}{Tornar são; curar, sarar, remediar.}{sa.nar}{\verboinum{1}}
\verb{sanativo}{}{}{}{}{adj.}{Que sana; próprio para sanar.}{sa.na.ti.vo}{0}
\verb{sanatório}{}{}{}{}{s.m.}{Estabelecimento para tratamento ou convalescença de doentes, particularmente psicopatas. }{sa.na.tó.rio}{0}
\verb{sanção}{}{}{"-ões}{}{s.f.}{Aprovação dada pelo chefe do poder executivo a uma lei votada pelo órgão legislativo; homologação.}{san.ção}{0}
\verb{sanção}{}{}{"-ões}{}{}{Pena ou recompensa correspondente à violação ou cumprimento de lei, acordo etc.}{san.ção}{0}
\verb{sancionar}{}{}{}{}{v.t.}{Dar sanção; aprovar, ratificar. (\textit{O presidente sancionou a lei.})}{san.ci.o.nar}{\verboinum{1}}
\verb{sandália}{}{}{}{}{s.f.}{Calçado constituído de uma sola com tiras ou cordões que a prendem ao pé.}{san.dá.lia}{0}
\verb{sândalo}{}{Bot.}{}{}{s.m.}{Árvore de madeira resistente da qual se extrai um óleo essencial muito utilizado em perfumaria.}{sân.da.lo}{0}
\verb{sândalo}{}{}{}{}{}{O perfume extraído dessa árvore.}{sân.da.lo}{0}
\verb{sandeu}{}{}{}{sandia}{adj.}{Que diz ou pratica tolices; estúpido, parvo, pateta, tonto.}{san.deu}{0}
\verb{sandice}{}{}{}{}{s.f.}{Ato, pensamento ou afirmação que denota  estupidez, ignorância, tolice.}{san.di.ce}{0}
\verb{sanduíche}{}{Cul.}{}{}{s.m.}{Alimento feito com duas fatias de pão, entre as quais se colocam carne, queijo, presunto, peixe, ovos, conserva etc.}{san.du.í.che}{0}
\verb{saneamento}{}{}{}{}{s.m.}{Ato ou efeito de sanear, limpar; asseio, limpeza.}{sa.ne.a.men.to}{0}
\verb{saneamento}{}{}{}{}{}{Conjunto de técnicas ou medidas que tornam um ambiente limpo, sadio, habitável.}{sa.ne.a.men.to}{0}
\verb{saneamento}{}{}{}{}{}{Reparação, emenda.}{sa.ne.a.men.to}{0}
\verb{sanear}{}{}{}{}{v.t.}{Tornar habitável, saudável; limpar.}{sa.ne.ar}{0}
\verb{sanear}{}{}{}{}{}{Preparar o terreno para agricultura.}{sa.ne.ar}{0}
\verb{sanear}{}{Fig.}{}{}{}{Eliminar falhas ou excessos; reparar, remediar.}{sa.ne.ar}{\verboinum{4}}
\verb{sanefa}{é}{}{}{}{s.f.}{Larga faixa de pano que se estende sobre a parte superior de uma cortina ou reposteiro.}{sa.ne.fa}{0}
\verb{sanfona}{ô}{Mús.}{}{}{s.f.}{Instrumento de sopro, dotado de fole que se retrai e expande, teclado e botões, semelhante ao acordeão e à harmônica; rabeca.}{san.fo.na}{0}
\verb{sanfoneiro}{ê}{}{}{}{s.m.}{Indivíduo que toca sanfona ou acordeão.}{san.fo.nei.ro}{0}
\verb{sangradoiro}{ô}{}{}{}{}{Var. de \textit{sangradouro}.  }{san.gra.doi.ro}{0}
\verb{sangrador}{ô}{}{}{}{s.m.}{Ver \textit{sangradouro}.}{san.gra.dor}{0}
\verb{sangradouro}{ô}{}{}{}{s.m.}{Sulco ou canal por onde se desvia parte da água de um rio, fonte, açude, represa. }{san.gra.dou.ro}{0}
\verb{sangradouro}{ô}{}{}{}{}{Lugar, no pescoço ou no peito dos animais, onde se golpeia para abatê"-los.}{san.gra.dou.ro}{0}
\verb{sangradouro}{ô}{Med.}{}{}{}{Parte interna do braço, oposta ao cotovelo, onde antigamente se fazia a sangria.}{san.gra.dou.ro}{0}
\verb{sangradura}{}{}{}{}{s.f.}{Ver \textit{sangria}.}{san.gra.du.ra}{0}
\verb{sangrar}{}{}{}{}{v.i.}{Verter sangue; gotejar.}{san.grar}{0}
\verb{sangrar}{}{}{}{}{v.t.}{Ferir com derramamento de sangue.}{san.grar}{0}
\verb{sangrar}{}{}{}{}{}{Extrair algum líquido; drenar.}{san.grar}{0}
\verb{sangrar}{}{}{}{}{}{Tirar sangue, abrindo uma veia.}{san.grar}{0}
\verb{sangrar}{}{Fig.}{}{}{}{Extorquir bens, dinheiro, valores.}{san.grar}{\verboinum{1}}
\verb{sangrento}{}{}{}{}{adj.}{Que está coberto ou manchado de sangue; ensanguentado.}{san.gren.to}{0}
\verb{sangrento}{}{}{}{}{}{Que envolve derramamento de sangue; cruento, sanguinolento.}{san.gren.to}{0}
\verb{sangria}{}{}{}{}{s.f.}{Ato ou efeito de sangrar; sangradura.}{san.gri.a}{0}
\verb{sangria}{}{Med.}{}{}{}{Abertura de uma veia para a retirada de sangue.}{san.gri.a}{0}
\verb{sangria}{}{}{}{}{}{Abertura ou corte feito na casca de certas árvores para a extração de resinas.}{san.gri.a}{0}
\verb{sangria}{}{}{}{}{}{Bebida preparada com vinho, água, limão e açúcar.}{san.gri.a}{0}
\verb{sangue}{}{}{}{}{s.m.}{Líquido vermelho, viscoso, que circula pelo corpo, impulsionado pelos movimentos do coração, e que tem como uma das funções principais conduzir oxigênio e substâncias nutritivas às células do organismo.}{san.gue}{0}
\verb{sangue}{}{Fig.}{}{}{}{A vida, a existência.}{san.gue}{0}
\verb{sangue}{}{Fig.}{}{}{}{Família, estirpe, linhagem, geração.}{san.gue}{0}
\verb{sangue"-frio}{}{}{sangues"-frios}{}{s.m.}{Controle emocional, calma, tranquilidade diante de situações difíceis ou perigosas.}{san.gue"-fri.o}{0}
\verb{sangueira}{ê}{}{}{}{s.f.}{Sangue derramado em grande quantidade.}{san.guei.ra}{0}
\verb{sanguessuga}{}{Zool.}{}{}{s.f.}{Verme anelídeo parasita, dotado de ventosas com as quais ele se fixa em animais para sugar"-lhes o sangue.}{san.gues.su.ga}{0}
\verb{sanguessuga}{}{Fig.}{}{}{}{Indivíduo que explora outros, pedindo dinheiro ou favores.}{san.gues.su.ga}{0}
\verb{sanguinário}{}{}{}{}{adj.}{Que gosta de ver ou de derramar sangue; sanguinolento, sanguissedento.}{san.gui.ná.rio}{0}
\verb{sanguinário}{}{}{}{}{}{Cruel, feroz, desumano.}{san.gui.ná.rio}{0}
\verb{sanguinário}{}{}{}{}{}{Var. de \textit{sanguinário}.}{san.gui.ná.rio}{0}
\verb{sanguíneo}{}{}{}{}{adj.}{Relativo ao sangue.}{san.guí.neo}{0}
\verb{sanguíneo}{}{}{}{}{}{Que tem a cor de sangue.}{san.guí.neo}{0}
\verb{sanguíneo}{}{Fig.}{}{}{}{Aquele em que predomina o sangue; de temperamento impulsivo.}{san.guí.neo}{0}
\verb{sanguíneo}{}{}{}{}{}{Var. de \textit{sanguíneo}. }{san.guí.neo}{0}
\verb{sanguinolência}{}{}{}{}{s.f.}{Derramamento de sangue; crueldade, ferocidade, desumanidade.}{san.gui.no.lên.cia}{0}
\verb{sanguinolento}{}{}{}{}{adj.}{Misturado ou tinto de sangue; ensanguentado.}{san.gui.no.len.to}{0}
\verb{sanguinolento}{}{}{}{}{}{Cruento, sangrento.}{san.gui.no.len.to}{0}
\verb{sanguinolento}{}{}{}{}{}{Que se compraz em ver ou derramar sangue; sanguinário.}{san.gui.no.len.to}{0}
\verb{sanguissedento}{}{}{}{}{adj.}{Que tem sede de sangue; sanguinário.}{san.guis.se.den.to}{0}
\verb{sanha}{}{}{}{}{s.f.}{Rancor, fúria, ira.}{sa.nha}{0}
\verb{sanha}{}{}{}{}{}{Vontade incontrolável.}{sa.nha}{0}
\verb{sanhaço}{}{Zool.}{}{}{s.m.}{Designação comum de várias aves de plumagem geral cinza"-azulada ou esverdeada e asas com enfeites variados, que se alimentam sobretudo de frutas.}{sa.nha.ço}{0}
\verb{sanhaçu}{}{Zool.}{}{}{s.m.}{Ver \textit{sanhaço}.}{sa.nha.çu}{0}
\verb{sanidade}{}{}{}{}{s.f.}{Qualidade ou estado de são.}{sa.ni.da.de}{0}
\verb{sanidade}{}{}{}{}{}{Conjunto de condições que conduzem ao bem"-estar e à saúde; higiene, salubridade.}{sa.ni.da.de}{0}
\verb{sanidade}{}{}{}{}{}{Normalidade física ou psíquica.}{sa.ni.da.de}{0}
\verb{sânie}{}{Med.}{}{}{s.f.}{Pus ou matéria purulenta gerada pelas úlceras e chagas não tratadas.}{sâ.nie}{0}
\verb{sânie}{}{Por ext.}{}{}{}{Estado de podre; podridão.}{sâ.nie}{0}
\verb{sanioso}{ô}{}{"-osos ⟨ó⟩}{"-osa ⟨ó⟩}{adj.}{Que tem sânie.}{sa.ni.o.so}{0}
\verb{sanitário}{}{}{}{}{adj.}{Relativo a saúde ou a higiene.}{sa.ni.tá.rio}{0}
\verb{sanitário}{}{}{}{}{}{Relativo a banheiro, aposento com todo aparelhamento para higiene corporal.}{sa.ni.tá.rio}{0}
\verb{sanitário}{}{}{}{}{s.m.}{Vaso sanitário.}{sa.ni.tá.rio}{0}
\verb{sanitário}{}{}{}{}{}{Local público ou privado equipado com vaso sanitário; toalete, mictório.}{sa.ni.tá.rio}{0}
\verb{sanitarista}{}{}{}{}{s.2g.}{Especialista em saúde pública; higienista.}{sa.ni.ta.ris.ta}{0}
\verb{san"-marinense}{}{}{}{}{adj.2g. e s.2g.}{Samarinês.}{san"-ma.ri.nen.se}{0}
\verb{sânscrito}{}{}{}{}{s.m.}{Antiga língua clássica da Índia.}{sâns.cri.to}{0}
\verb{sansei}{}{}{}{}{s.2g.}{Cidadão americano neto de emigrantes japoneses.}{san.sei}{0}
\verb{santa}{}{}{}{}{s.f.}{Mulher que foi canonizada.}{san.ta}{0}
\verb{santa}{}{}{}{}{}{Imagem dessa mulher.}{san.ta}{0}
\verb{santa}{}{Fig.}{}{}{}{Mulher virtuosa, bondosa.}{san.ta}{0}
\verb{santantônio}{}{Bras.}{}{}{s.m.}{Saliência à frente da sela, a que pode agarrar"-se o montador; cabeçote de sela.}{san.tan.tô.nio}{0}
\verb{santarrão}{}{}{"-ões}{}{adj.}{Que simula pureza, santidade.}{san.tar.rão}{0}
\verb{santarrão}{}{}{"-ões}{}{s.m.}{Indivíduo que finge santidade; falso devoto.}{san.tar.rão}{0}
\verb{santeiro}{ê}{}{}{}{adj.}{Que demonstra grande devoção religiosa; devoto, beato.}{san.tei.ro}{0}
\verb{santeiro}{ê}{}{}{}{s.m.}{Indivíduo que esculpe ou vende imagens ou gravuras de santos.}{san.tei.ro}{0}
\verb{santelmo}{é}{}{}{}{s.m.}{Chama azulada que, sobretudo por ocasião de tempestade, surge no topo dos mastros dos navios, produzida pela eletricidade.}{san.tel.mo}{0}
\verb{santidade}{}{}{}{}{s.f.}{Qualidade ou estado de santo.}{san.ti.da.de}{0}
\verb{santidade}{}{}{}{}{}{Estado de santificação; virtude, pureza, religiosidade.}{san.ti.da.de}{0}
\verb{santificação}{}{}{"-ões}{}{s.f.}{Ato ou efeito de santificar, de tornar santo.}{san.ti.fi.ca.ção}{0}
\verb{santificação}{}{}{"-ões}{}{}{Engrandecimento e valorização de alguém ou algo; elevação, exaltação.}{san.ti.fi.ca.ção}{0}
\verb{santificação}{}{}{"-ões}{}{}{Processo de canonização.}{san.ti.fi.ca.ção}{0}
\verb{santificação}{}{}{"-ões}{}{}{Celebração de acordo com os ritos religiosos.}{san.ti.fi.ca.ção}{0}
\verb{santificador}{ô}{}{}{}{adj.}{Que santifica, que torna santo.}{san.ti.fi.ca.dor}{0}
\verb{santificador}{ô}{}{}{}{s.m.}{Indivíduo que santifica; canonizador.}{san.ti.fi.ca.dor}{0}
\verb{santificante}{}{}{}{}{adj.2g.}{Santificador.}{san.ti.fi.can.te}{0}
\verb{santificar}{}{}{}{}{v.t.}{Tornar santo; sagrar.}{san.ti.fi.car}{0}
\verb{santificar}{}{}{}{}{}{Inscrever no rol dos santos; canonizar.}{san.ti.fi.car}{0}
\verb{santificar}{}{}{}{}{}{Conduzir pelo caminho da salvação.}{san.ti.fi.car}{0}
\verb{santificar}{}{}{}{}{}{Celebrar conforme os princípios da religião.}{san.ti.fi.car}{\verboinum{2}}
\verb{santinho}{}{}{}{}{s.m.}{Pequena imagem religiosa.}{san.ti.nho}{0}
\verb{santinho}{}{Por ext.}{}{}{}{Imagem de candidato a cargo eletivo em foto pequena como de imagem de santo.}{san.ti.nho}{0}
\verb{santíssimo}{}{}{}{}{adj.}{Superlativo absoluto sintético de \textit{santo}.}{san.tís.si.mo}{0}
\verb{santíssimo}{}{Relig.}{}{}{s.m.}{O sacramento da Eucaristia; a hóstia consagrada.}{san.tís.si.mo}{0}
\verb{santista}{}{}{}{}{adj.2g.}{Relativo a Santos (\textsc{sp}).}{san.tis.ta}{0}
\verb{santista}{}{}{}{}{s.2g.}{Indivíduo natural ou habitante dessa cidade.}{san.tis.ta}{0}
\verb{santo}{}{}{}{}{adj.}{Que vive segundo os preceitos religiosos, a lei divina.}{san.to}{0}
\verb{santo}{}{}{}{}{}{Diz"-se daquele que a igreja canonizou.}{san.to}{0}
\verb{santo}{}{}{}{}{}{Puro, imaculado, inocente.}{san.to}{0}
\verb{santo}{}{}{}{}{}{Respeitável, venerável.}{san.to}{0}
\verb{santo}{}{Por ext.}{}{}{}{Que tem bom coração, bondoso em extremo.}{san.to}{0}
\verb{santo}{}{}{}{}{s.m.}{Indivíduo que morreu em estado de santidade e foi canonizado.}{san.to}{0}
\verb{santo}{}{}{}{}{}{Imagem desse indivíduo.}{san.to}{0}
\verb{santo}{}{Por ext.}{}{}{}{Pessoa muito austera ou de bondade extraordinária.}{san.to}{0}
\verb{santo"-antônio}{}{}{santo"-antônios}{}{s.m.}{Santantônio.}{san.to"-an.tô.nio}{0}
\verb{santuário}{}{}{}{}{s.m.}{Lugar consagrado pela religião; lugar santo.}{san.tu.á.rio}{0}
\verb{santuário}{}{}{}{}{}{A parte interior, reservada de um templo.}{san.tu.á.rio}{0}
\verb{santuário}{}{}{}{}{}{Templo, igreja, basílica.}{san.tu.á.rio}{0}
\verb{santuário}{}{}{}{}{}{Sacrário, relicário.}{san.tu.á.rio}{0}
\verb{sanzala}{}{}{}{}{}{Var. de \textit{senzala}.}{san.za.la}{0}
\verb{são}{}{}{"-ãos}{sã}{adj.}{Livre da doença; sadio. (\textit{A criança está sã.})}{são}{0}
\verb{são}{}{}{}{}{adj.}{Que a Igreja declarou merecedor da devoção dos católicos. (\textit{São Judas Tadeu é o nosso protetor.})}{são}{0}
\verb{são"-bernardo}{}{}{são"-bernardos}{}{s.m.}{Raça de cães de origem suíça, de pelagem muito densa, macia e ondulada, orelhas caídas lateralmente, de grande porte, famoso por socorrer vítimas de tempestades de neve e avalanches.}{são"-ber.nar.do}{0}
\verb{são"-bernardo}{}{}{são"-bernardos}{}{}{Cão dessa raça.}{são"-ber.nar.do}{0}
\verb{são"-luisense}{}{}{são"-luisenses}{}{adj.2g.}{Relativo a São Luís, capital do Maranhão.}{são"-lu.i.sen.se}{0}
\verb{são"-luisense}{}{}{são"-luisenses}{}{s.2g.}{Indivíduo natural ou habitante dessa cidade.}{são"-lu.i.sen.se}{0}
\verb{são"-tomense}{}{}{são"-tomenses}{}{adj.2g.}{Relativo à ilha de São Tomé e Príncipe (África).}{são"-to.men.se}{0}
\verb{são"-tomense}{}{}{são"-tomenses}{}{s.2g.}{Indivíduo natural ou habitante dessa ilha.}{são"-to.men.se}{0}
\verb{sapa}{}{}{}{}{s.f.}{Pá com que se cavam trincheiras, fossos etc.}{sa.pa}{0}
\verb{sapa}{}{}{}{}{}{A fêmea do sapo.}{sa.pa}{0}
\verb{sapador}{ô}{}{}{}{s.m.}{Indivíduo que faz sapa ou outros trabalhos ligeiros de engenharia militar.}{sa.pa.dor}{0}
\verb{sapar}{}{}{}{}{v.i.}{Levantar a terra com sapa.}{sa.par}{0}
\verb{sapar}{}{}{}{}{}{Fazer trabalhos de sapa.}{sa.par}{\verboinum{1}}
\verb{saparia}{}{}{}{}{s.f.}{Certa quantidade de sapos.}{sa.pa.ri.a}{0}
\verb{saparia}{}{Pop.}{}{}{}{Grupo de indivíduos de má índole; corja, cambada.}{sa.pa.ri.a}{0}
\verb{sapata}{}{}{}{}{s.f.}{Peça ou construção que se coloca para outras se sustentarem sobre ela. (\textit{A fundação desta mureta será uma sapata corrida colocada a 0,3 m de profundidade.})}{sa.pa.ta}{0}
\verb{sapatada}{}{}{}{}{s.f.}{Pancada dada com sapato ou sapata.}{sa.pa.ta.da}{0}
%\verb{}{}{}{}{}{}{}{}{0}
\verb{sapataria}{}{}{}{}{s.f.}{Ofício de sapateiro.}{sa.pa.ta.ri.a}{0}
\verb{sapataria}{}{}{}{}{}{Fábrica, loja ou oficina de sapatos.}{sa.pa.ta.ri.a}{0}
\verb{sapateado}{}{}{}{}{s.m.}{Dança popular espanhola, em geral sem acompanhamento musical, caracterizada pelo martelar rítmico dos tacões dos sapatos no chão.}{sa.pa.te.a.do}{0}
\verb{sapateado}{}{}{}{}{}{Dança de origem norte"-americana, executada com sapatos especiais dotados de chapa metálica na sola, para produzir um ruído característico.}{sa.pa.te.a.do}{0}
\verb{sapateado}{}{}{}{}{adj.}{Batido no chão com os saltos do sapato.}{sa.pa.te.a.do}{0}
\verb{sapateador}{ô}{}{}{}{adj.}{Que sapateia.}{sa.pa.te.a.dor}{0}
\verb{sapateador}{ô}{}{}{}{s.m.}{Dançarino de sapateado.}{sa.pa.te.a.dor}{0}
\verb{sapatear}{}{}{}{}{v.i.}{Executar uma dança, batendo ruidosamente com o salto ou a sola do calçado no chão. (\textit{Ele sapateou sobre todas as mesas.})}{sa.pa.te.ar}{0}
\verb{sapatear}{}{}{}{}{}{Bater repetida e vivamente com os pés no chão; enfurecer"-se. (\textit{A criança sapateou de raiva, mas acabou cedendo.})}{sa.pa.te.ar}{\verboinum{4}}
\verb{sapateira}{ê}{}{}{}{s.f.}{Móvel onde se guardam sapatos.}{sa.pa.tei.ra}{0}
\verb{sapateiro}{ê}{}{}{}{s.m.}{Indivíduo que fabrica, vende ou conserta calçados.}{sa.pa.tei.ro}{0}
\verb{sapatilha}{}{}{}{}{s.f.}{Sapato próprio para bailarinos, leve e flexível, com ponta reforçada.}{sa.pa.ti.lha}{0}
\verb{sapato}{}{}{}{}{s.m.}{Calçado, em geral de capa dura, que cobre o pé parcial ou completamente.}{sa.pa.to}{0}
\verb{sapé}{}{Bot.}{}{}{s.m.}{Capim da família das gramíneas, cujos caules secos são muito utilizados na cobertura de choças e choupanas.}{sa.pé}{0}
\verb{sapê}{}{}{}{}{}{Var. de \textit{sapé}.}{sa.pê}{0}
\verb{sapear}{}{Pop.}{}{}{v.t.}{Ficar olhando de fora, às escondidas.}{sa.pe.ar}{0}
\verb{sapear}{}{Pop.}{}{}{}{Observar sem participar.}{sa.pe.ar}{\verboinum{4}}
\verb{sapeca}{é}{}{}{}{s.f.}{Ato ou efeito de sapecar, chamuscar.}{sa.pe.ca}{0}
\verb{sapeca}{é}{}{}{}{adj.2g.}{Diz"-se da pessoa irrequieta, assanhada, namoradeira.}{sa.pe.ca}{0}
\verb{sapeca}{é}{}{}{}{}{Surra, sova.}{sa.pe.ca}{0}
\verb{sapecar}{}{}{}{}{v.t.}{Moquear para conservar; chamuscar.}{sa.pe.car}{0}
\verb{sapecar}{}{Pop.}{}{}{v.i.}{Namorar muito; divertir"-se, farrear.}{sa.pe.car}{\verboinum{2}}
\verb{sapecar}{}{}{}{}{}{Bater ou dar tapas.}{sa.pe.car}{\verboinum{2}}
\verb{sápido}{}{}{}{}{adj.}{Que tem sabor; saboroso, gostoso.}{sá.pi.do}{0}
\verb{sapiência}{}{}{}{}{s.f.}{Grande sabedoria; erudição.}{sa.pi.ên.cia}{0}
\verb{sapiência}{}{}{}{}{}{Sabedoria divina.}{sa.pi.ên.cia}{0}
\verb{sapiente}{}{}{}{}{adj.2g.}{Que sabe muito; sábio, erudito.}{sa.pi.en.te}{0}
\verb{sapiente}{}{}{}{}{}{Que conhece as coisas divinas.}{sa.pi.en.te}{0}
\verb{sapinho}{}{Med.}{}{}{s.m.}{Infecção das mucosas da boca, produzida por fungos, e que aparece na forma de placas brancas ou amareladas; candidíase.}{sa.pi.nho}{0}
\verb{sapinho}{}{}{}{}{}{Pequeno sapo.}{sa.pi.nho}{0}
\verb{sapo}{}{Zool.}{}{}{s.m.}{Nome dado a anfíbios anuros com forma larval aquática e forma adulta terrestre, que possuem a pele seca e rugosa e são peçonhentos.}{sa.po}{0}
\verb{sapo}{}{Veter.}{}{}{}{Inflamação dos cascos dos cavalos.}{sa.po}{0}
\verb{sapo}{}{Pop.}{}{}{}{Indivíduo que sapeia, que assiste a um jogo sem participar dele.}{sa.po}{0}
\verb{sapo"-cururu}{}{Zool.}{sapos"-cururus}{}{s.m.}{Certo sapo de grande porte; cururu.}{sa.po"-cu.ru.ru}{0}
\verb{sapo"-cururu}{}{}{sapos"-cururus}{}{}{Cantiga de ninar muito popular e tradicional.}{sa.po"-cu.ru.ru}{0}
\verb{sapo"-jururu}{}{Zool.}{}{}{s.m.}{Sapo"-cururu.}{sa.po"-ju.ru.ru}{0}
\verb{sapólio}{}{}{}{}{s.m.}{Saponáceo em forma de tijolinhos ou em pó, usado para limpeza úmida de alumínio, louças, azulejo etc.}{sa.pó.lio}{0}
\verb{saponáceo}{}{}{}{}{adj.}{Que tem as mesmas características do sabão.}{sa.po.ná.ceo}{0}
\verb{saponáceo}{}{}{}{}{}{Que pode ser usado como sabão.}{sa.po.ná.ceo}{0}
\verb{sapota}{ó}{Bot.}{}{}{s.f.}{Árvore da família das sapotáceas, da qual se extrai látex para fabricação de chiclete, e cujo fruto, comestível, é uma baga doce e carnosa; sapotizeiro.}{sa.po.ta}{0}
\verb{sapotácea}{}{Bot.}{}{}{adj.}{Diz"-se de família de árvores e arbustos, de seiva leitosa, flores solitárias e hermafroditas e frutos carnosos, geralmente comestíveis, e que tem como principais representantes a sapota e o abieiro.}{sa.po.tá.cea}{0}
\verb{sapoti}{}{}{}{}{s.m.}{Fruto comestível da sapota, do tamanho de uma laranja, com casca fina castanho"-escura, e baga muito doce, sendo apreciado tanto ao natural quanto na forma de refrescos.}{sa.po.ti}{0}
\verb{sapotizeiro}{ê}{Bot.}{}{}{s.m.}{Ver \textit{sapota}.}{sa.po.ti.zei.ro}{0}
\verb{saprófago}{}{}{}{}{adj.}{Diz"-se do animal que se alimenta de restos orgânicos em estado de decomposição.}{sa.pró.fa.go}{0}
\verb{saprofitismo}{}{Biol.}{}{}{s.m.}{Modo de certos organismos obterem seus nutrientes por meio de matéria orgânica putrefata.}{sa.pro.fi.tis.mo}{0}
\verb{saprófito}{}{}{}{}{adj.}{Diz"-se do organismo, geralmente vegetal, que obtém os nutrientes por meio de matéria orgânica em decomposição.}{sa.pró.fi.to}{0}
\verb{sapucaia}{}{Bot.}{}{}{s.f.}{Nome dado a diversas árvores, cujos frutos apresentam forma de cuia e as sementes, oleaginosas, são doces e comestíveis.}{sa.pu.cai.a}{0}
\verb{sapucaia}{}{Pop.}{}{}{}{Depósito de lixo; monturo, lixeira.}{sa.pu.cai.a}{0}
\verb{saque}{}{}{}{}{s.m.}{Ato ou efeito de sacar, tirar.}{sa.que}{0}
\verb{saque}{}{}{}{}{s.m.}{Ato ou efeito de saquear, roubar.}{sa.que}{0}
\verb{saque}{}{}{}{}{}{Emissão de ordem de pagamento ou título de crédito. }{sa.que}{0}
\verb{saque}{}{Esport.}{}{}{}{Jogada inicial em certos jogos como tênis, voleibol, pingue"-pongue etc.}{sa.que}{0}
\verb{saque}{}{}{}{}{}{Pilhagem ou rapina, geralmente feita por soldados numa cidade invadida ou por vândalos em situações de desordem.}{sa.que}{0}
\verb{saquê}{}{Cul.}{}{}{s.m.}{Bebida alcoólica japonesa, obtida pela fermentação artificial de arroz.}{sa.quê}{0}
\verb{saqueador}{ô}{}{}{}{adj.}{Que saqueia, faz pilhagem.}{sa.que.a.dor}{0}
\verb{saquear}{}{}{}{}{v.t.}{Pilhar, roubar com violência.}{sa.que.ar}{0}
\verb{saquear}{}{}{}{}{}{Devastar, assolar, destruir.}{sa.que.ar}{\verboinum{4}}
\verb{sarabanda}{}{}{}{}{s.f.}{Grande agitação; tumulto, roda"-viva.}{sa.ra.ban.da}{0}
\verb{sarabanda}{}{}{}{}{}{Censura, repreensão, advertência.}{sa.ra.ban.da}{0}
\verb{sarabanda}{}{Mús.}{}{}{}{Dança nobre e majestosa acompanhada de composição musical própria, em voga nos séculos \textsc{xvii} e \textsc{xviii}.}{sa.ra.ban.da}{0}
\verb{sarabatana}{}{}{}{}{}{Var. de \textit{zarabatana}.}{sa.ra.ba.ta.na}{0}
\verb{sarabulhento}{}{}{}{}{adj.}{Cheio de sarabulhos; áspero.}{sa.ra.bu.lhen.to}{0}
\verb{sarabulho}{}{}{}{}{s.m.}{Aspereza encontrada na superfície das louças, causada por grãos de areia ou por vidro mal fundido.}{sa.ra.bu.lho}{0}
\verb{sarabulho}{}{Fig.}{}{}{}{Pequena ferida com crosta; pústula.}{sa.ra.bu.lho}{0}
\verb{saracoteamento}{}{}{}{}{s.m.}{Ver \textit{saracoteio}.}{sa.ra.co.te.a.men.to}{0}
\verb{saracotear}{}{}{}{}{v.t.}{Agitar o corpo ou os quadris graciosamente; rebolar.}{sa.ra.co.te.ar}{0}
\verb{saracotear}{}{}{}{}{v.i.}{Andar de um lado para outro; estar irrequieto, buliçoso.}{sa.ra.co.te.ar}{\verboinum{4}}
\verb{saracoteio}{ê}{}{}{}{s.m.}{Ato ou efeito de saracotear; rebolado, molejo.}{sa.ra.co.tei.o}{0}
\verb{saracura}{}{Zool.}{}{}{s.f.}{Ave com pernas e dedos longos que vive em ambiente aquático e se alimenta de insetos, crustáceos e pequenos peixes.}{sa.ra.cu.ra}{0}
\verb{sarado}{}{}{}{}{adj.}{Que sarou; curado.}{sa.ra.do}{0}
\verb{sarado}{}{Pop.}{}{}{}{Forte, resistente, musculoso.}{sa.ra.do}{0}
\verb{sarado}{}{Pop.}{}{}{}{Valentão, decidido.}{sa.ra.do}{0}
\verb{sarado}{}{Pop.}{}{}{}{Guloso, comilão.}{sa.ra.do}{0}
\verb{sarado}{}{Pop.}{}{}{}{Esperto, sabido.}{sa.ra.do}{0}
\verb{saraiva}{}{}{}{}{s.f.}{Ver \textit{granizo}.}{sa.rai.va}{0}
\verb{saraivada}{}{}{}{}{s.f.}{Precipitação repentina e violenta de granizo.}{sa.rai.va.da}{0}
\verb{saraivada}{}{Por ext.}{}{}{}{Grande quantidade de alguma coisa arremessada.}{sa.rai.va.da}{0}
\verb{saraivar}{}{}{}{}{v.i.}{Cair saraiva.}{sa.rai.var}{0}
\verb{saraivar}{}{}{}{}{}{Lançar"-se em saraivada. \textit{abon}}{sa.rai.var}{0}
\verb{saraivar}{}{}{}{}{v.t.}{Castigar, flagelar com saraiva. \textit{abon}}{sa.rai.var}{\verboinum{1}}
\verb{sarampento}{}{}{}{}{adj.}{Que está atacado de sarampo.}{sa.ram.pen.to}{0}
\verb{sarampo}{}{Med.}{}{}{s.m.}{Doença infecciosa e contagiosa causada por vírus, caracterizada por febre, inflamação das mucosas respiratórias e erupções na pele.}{sa.ram.po}{0}
\verb{sarapantar}{}{}{}{}{v.t.}{Espantar, assustar, atordoar.}{sa.ra.pan.tar}{0}
\verb{sarapantar}{}{}{}{}{v.i.}{Causar espanto, admiração.}{sa.ra.pan.tar}{\verboinum{1}}
\verb{sarapatel}{é}{Cul.}{"-éis}{}{s.m.}{Prato preparado com miúdos, tripas e sangue de porco ou carneiro e bem condimentada.}{sa.ra.pa.tel}{0}
\verb{sarapintado}{}{}{}{}{adj.}{Que tem pintas variadas.}{sa.ra.pin.ta.do}{0}
\verb{sarapintar}{}{}{}{}{v.t.}{Fazer pintas; salpicar, pintalgar.}{sa.ra.pin.tar}{\verboinum{1}}
\verb{sarar}{}{}{}{}{v.t.}{Restituir a saúde; curar.}{sa.rar}{0}
\verb{sarar}{}{}{}{}{}{Corrigir, emendar.}{sa.rar}{0}
\verb{sarar}{}{}{}{}{v.i.}{Recuperar a saúde; cicatrizar"-se.}{sa.rar}{\verboinum{1}}
\verb{sarará}{}{}{}{}{adj.}{Diz"-se do cabelo crespo e arruivado de alguns indivíduos mulatos.}{sa.ra.rá}{0}
\verb{sarará}{}{Por ext.}{}{}{}{Diz"-se do indivíduo mulato com esse tipo de cabelo.}{sa.ra.rá}{0}
\verb{sarará}{}{Bras.}{}{}{}{Ver \textit{albino}.}{sa.ra.rá}{0}
\verb{sarará}{}{Bras.}{}{}{s.f.}{Formiga que vive em cupinzeiros abandonados e ataca alimentos armazenados.}{sa.ra.rá}{0}
\verb{sarassará}{}{}{}{}{s.m.}{Ver \textit{sarará}.}{sa.ras.sa.rá}{0}
\verb{sarau}{}{}{}{}{s.m.}{Festa noturna para apreciar literatura, ouvir música, conversar.}{sa.rau}{0}
\verb{sarau}{}{}{}{}{}{Concerto musical noturno.}{sa.rau}{0}
\verb{sarça}{}{}{}{}{s.f.}{Arbusto espinhoso com propriedades medicinais.}{sar.ça}{0}
\verb{sarcasmo}{}{}{}{}{s.m.}{Manifestação irônica ou maliciosa que visa insultar, humilhar, ridicularizar.}{sar.cas.mo}{0}
\verb{sarcástico}{}{}{}{}{adj.}{Que envolve sarcasmo.}{sar.cás.ti.co}{0}
\verb{sarcófago}{}{}{}{}{s.m.}{Túmulo em que os antigos colocavam os cadáveres que não queriam incinerar. (\textit{Reis e imperadores eram colocados em sarcófagos.})}{sar.có.fa.go}{0}
\verb{sarcófago}{}{Por ext.}{}{}{}{Túmulo, tumba. }{sar.có.fa.go}{0}
\verb{sarcófago}{}{}{}{}{adj.}{Que corrói ou devora carne. (\textit{Alguns trabalhos foram feitos sobre os insetos sarcófagos.})}{sar.có.fa.go}{0}
\verb{sarcoma}{}{Med.}{}{}{s.m.}{Tumor maligno originado no tecido conjuntivo.}{sar.co.ma}{0}
\verb{sarcomatoso}{ô}{}{"-osos ⟨ó⟩}{"-osa ⟨ó⟩}{adj.}{Semelhante a um sarcoma.}{sar.co.ma.to.so}{0}
\verb{sarcomatoso}{ô}{}{"-osos ⟨ó⟩}{"-osa ⟨ó⟩}{}{Que tem sarcoma.}{sar.co.ma.to.so}{0}
\verb{sarda}{}{}{}{}{s.f.}{Mancha amarelo"-escura que aparece na pele, especialmente em pessoas muito claras.}{sar.da}{0}
\verb{sardento}{}{}{}{}{adj.}{Que tem sardas.}{sar.den.to}{0}
\verb{sardinha}{}{}{}{}{s.f.}{Pequeno peixe marinho comestível e de grande valor comercial. (\textit{Gosto muito de comer sardinha frita.})}{sar.di.nha}{0}
\verb{sardinha}{}{Bras.}{}{}{}{Brincadeira na qual se colocam as mãos sobre as do oponente e se tenta tirá"-las antes que o outro consiga bater"-lhe. (\textit{Eles ficaram jogando sardinha a tarde inteira.})}{sar.di.nha}{0}
\verb{sardinha}{}{Pop.}{}{}{}{Navalha.}{sar.di.nha}{0}
\verb{sardônico}{}{}{}{}{adj.}{Diz"-se de riso forçado e sarcástico.}{sar.dô.ni.co}{0}
\verb{sargaço}{}{}{}{}{s.m.}{Alga de grandes dimensões, que vive fixa a rochas ou flutuante em alto"-mar.}{sar.ga.ço}{0}
\verb{sargento}{}{}{}{}{s.2g.}{Posto de praça imediatamente superior ao de cabo.}{sar.gen.to}{0}
\verb{sargento}{}{}{}{}{s.m.}{Tipo de prensa usada por marceneiros e carpinteiros.}{sar.gen.to}{0}
\verb{sargento}{}{}{}{}{}{O militar que ocupa esse posto.}{sar.gen.to}{0}
\verb{sargento}{}{Zool.}{}{}{}{Peixe de águas salgadas tropicais e subtropicais, com dorso verde"-azulado e faixas verticais negras.}{sar.gen.to}{0}
\verb{sariguê}{}{}{}{}{s.m.}{Gambá.}{sa.ri.guê}{0}
\verb{sarilho}{}{}{}{}{s.m.}{Cilindro horizontal dotado de manivela no qual se enrola corda ou cabo para levantar grandes pesos.}{sa.ri.lho}{0}
\verb{sarilho}{}{Pop.}{}{}{}{Confusão, barafunda, rolo.}{sa.ri.lho}{0}
\verb{sarja}{}{}{}{}{s.f.}{Tecido entrançado de algodão, lã ou seda usado na confecção de roupas.}{sar.ja}{0}
\verb{sarja}{}{Med.}{}{}{s.f.}{Incisão superficial na pele para extrair sangue ou pus.}{sar.ja}{0}
\verb{sarjar}{}{}{}{}{v.t.}{Fazer sarja, incisão.}{sar.jar}{\verboinum{1}}
\verb{sarjeta}{ê}{}{}{}{s.f.}{Escoadouro de águas nas vias e praças públicas.}{sar.je.ta}{0}
\verb{sarjeta}{ê}{Fig.}{}{}{}{Estado de indigência; decadência.}{sar.je.ta}{0}
\verb{sarmento}{}{Bot.}{}{}{s.m.}{Ramo da videira.}{sar.men.to}{0}
\verb{sarmento}{}{Por ext.}{}{}{}{Qualquer ramo lenhoso, longo e delgado com nós bem marcados.}{sar.men.to}{0}
\verb{sarna}{}{}{}{}{s.f.}{Moléstia cutânea parasitária e contagiosa; escabiose.}{sar.na}{0}
\verb{sarna}{}{Fig.}{}{}{}{Pessoa importuna, rabugenta, maçante.}{sar.na}{0}
\verb{sarnento}{}{}{}{}{adj.}{Que tem sarna.}{sar.nen.to}{0}
\verb{sarnento}{}{}{}{}{}{Rançoso; ligeiramente apodrecido.}{sar.nen.to}{0}
\verb{sarnento}{}{Fig.}{}{}{}{Abatido, combalido.}{sar.nen.to}{0}
\verb{sarongue}{}{}{}{}{s.m.}{Pedaço de tecido estampado usado como vestuário amarrado à cintura na região da Oceania.}{sa.ron.gue}{0}
\verb{sarrabulhada}{}{}{}{}{s.f.}{Grande porção de sarrabulho.}{sar.ra.bu.lha.da}{0}
\verb{sarrabulhada}{}{Fig.}{}{}{}{Grande quantidade de sangue derramado; matança.}{sar.ra.bu.lha.da}{0}
\verb{sarrabulhada}{}{Fig.}{}{}{}{Confusão, desordem, baderna.}{sar.ra.bu.lha.da}{0}
\verb{sarrabulho}{}{}{}{}{s.m.}{Sangue coagulado de porco.}{sar.ra.bu.lho}{0}
\verb{sarrabulho}{}{}{}{}{}{Ver \textit{sarapatel}.}{sar.ra.bu.lho}{0}
\verb{sarrabulho}{}{Fig.}{}{}{}{Confusão, desordem, bate"-boca.}{sar.ra.bu.lho}{0}
\verb{sarraceno}{ê}{}{}{}{adj.}{Relativo aos sarracenos, povo nômade que habitava os desertos da Síria e da Arábia.}{sar.ra.ce.no}{0}
\verb{sarraceno}{ê}{}{}{}{s.m.}{Na Idade Média, nome dado aos muçulmanos pelos ocidentais europeus; mouro, árabe.}{sar.ra.ce.no}{0}
\verb{sarrafada}{}{}{}{}{s.f.}{Pancada ou golpe desferido com pedaço de pau; paulada, cacetada.}{sar.ra.fa.da}{0}
\verb{sarrafo}{}{}{}{}{s.m.}{Tira de madeira comprida e estreita.}{sar.ra.fo}{0}
\verb{sarrafo}{}{Bras.}{}{}{}{Pedaço de pau; cacete.}{sar.ra.fo}{0}
\verb{sarrento}{}{}{}{}{adj.}{Que está coberto de sarros, crostas, borras.}{sar.ren.to}{0}
\verb{sarrido}{}{}{}{}{s.m.}{Dificuldade de respirar, geralmente dos moribundos.}{sar.ri.do}{0}
\verb{sarro}{}{}{}{}{s.m.}{Borra que se forma no vinho ou em outros líquidos e que adere no fundo do recipiente.}{sar.ro}{0}
\verb{sarro}{}{}{}{}{}{Resíduo de fumo ou nicotina que fica no fundo dos cachimbos, piteiras e nos dentes dos fumantes.}{sar.ro}{0}
\verb{sarro}{}{Med.}{}{}{}{Camada esbranquiçada que se forma na língua decorrente de algumas doenças.}{sar.ro}{0}
\verb{sarro}{}{Pop.}{}{}{}{Pessoa ou coisa engraçada, divertida.}{sar.ro}{0}
\verb{sartório}{}{Anat.}{}{}{adj.}{Diz"-se do músculo da coxa.}{sar.tó.rio}{0}
\verb{sashimi}{}{Cul.}{}{}{s.m.}{Prato japonês preparado com carne crua de peixes, cortada em fatias finas e regada com molho de soja e uma pasta apimentada.}{\textit{sashimi}}{0}
\verb{sassaricar}{}{}{}{}{v.i.}{Andar, sacudindo o corpo; rebolar, dançar, folgar, brincar.}{sas.sa.ri.car}{\verboinum{2}}
\verb{satã}{}{}{}{}{s.m.}{Ver \textit{satanás}.}{sa.tã}{0}
\verb{satanás}{}{Relig.}{}{}{s.m.}{Na tradição judaico"-islâmico"-cristã, o chefe dos anjos rebeldes contra Deus.}{sa.ta.nás}{0}
\verb{satanás}{}{}{}{}{}{Diabo, satã, demônio.}{sa.ta.nás}{0}
\verb{satânico}{}{}{}{}{adj.}{Relativo a Satanás; diabólico, infernal.}{sa.tâ.ni.co}{0}
\verb{satanismo}{}{}{}{}{s.m.}{Qualidade ou condição do que é satânico.}{sa.ta.nis.mo}{0}
\verb{satanismo}{}{}{}{}{}{Culto ou adoração a Satanás.}{sa.ta.nis.mo}{0}
\verb{satélite}{}{}{}{}{s.m.}{Corpo celeste que gravita em torno de um astro, geralmente de um planeta como a Lua em relação à Terra.}{sa.té.li.te}{0}
\verb{satélite}{}{Fig.}{}{}{}{Pessoa que vive sob a dependência ou a proteção de outra.}{sa.té.li.te}{0}
\verb{satélite}{}{}{}{}{adj.}{Diz"-se da cidade formada ao redor de grandes metrópoles.}{sa.té.li.te}{0}
%\verb{}{}{}{}{}{}{}{}{0}
%\verb{}{}{}{}{}{}{}{}{0}
\verb{sátira}{}{Liter.}{}{}{s.f.}{Composição poética em que se critica pessoas e instituições.}{sá.ti.ra}{0}
\verb{sátira}{}{}{}{}{}{Qualquer escrito ou discurso maledicente ou picante em que se criticam os costumes sociais.}{sá.ti.ra}{0}
\verb{sátira}{}{}{}{}{}{Zombaria, ironia, troça.}{sá.ti.ra}{0}
\verb{satírico}{}{}{}{}{adj.}{Relativo à sátira; mordaz, sarcástico.}{sa.tí.ri.co}{0}
\verb{satirista}{}{}{}{}{s.2g.}{Indivíduo que escreve sátiras, discursos mordazes e sarcásticos.}{sa.ti.ris.ta}{0}
\verb{satirizar}{}{}{}{}{v.t.}{Expôr ao ridículo; escarnecer.}{sa.ti.ri.zar}{0}
\verb{satirizar}{}{}{}{}{v.i.}{Fazer críticas mordazes, maledicentes.}{sa.ti.ri.zar}{\verboinum{1}}
\verb{sátiro}{}{}{}{}{s.m.}{Homem devasso, libidinoso, luxurioso.}{sá.ti.ro}{0}
\verb{sátiro}{}{Mit.}{}{}{}{Na antiga Grécia, semideus habitante das florestas, com chifres curtos e pernas de bode.}{sá.ti.ro}{0}
\verb{satisfação}{}{}{"-ões}{}{s.f.}{Ato ou efeito de satisfazer; contentamento, aprazimento.}{sa.tis.fa.ção}{0}
\verb{satisfação}{}{}{"-ões}{}{}{Pagamento do que se deve; reparação, compensação, indenização.}{sa.tis.fa.ção}{0}
\verb{satisfação}{}{}{"-ões}{}{}{Explicação, justificativa, desculpa.}{sa.tis.fa.ção}{0}
\verb{satisfatório}{}{}{}{}{adj.}{Que é capaz de satisfazer.}{sa.tis.fa.tó.rio}{0}
\verb{satisfatório}{}{}{}{}{}{Aceitável, suficiente, convincente, regular.}{sa.tis.fa.tó.rio}{0}
\verb{satisfazer}{ê}{}{}{}{v.t.}{Causar prazer; contentar, agradar.}{sa.tis.fa.zer}{0}
\verb{satisfazer}{ê}{}{}{}{}{Dar execução; desempenhar, cumprir, realizar.}{sa.tis.fa.zer}{0}
\verb{satisfazer}{ê}{}{}{}{}{Saciar, fartar, mitigar.}{sa.tis.fa.zer}{0}
\verb{satisfazer}{ê}{}{}{}{}{Corresponder à expectativa; completar, preencher. }{sa.tis.fa.zer}{0}
\verb{satisfazer}{ê}{}{}{}{}{Pagar o que se deve; indenizar, reparar.}{sa.tis.fa.zer}{0}
\verb{satisfazer}{ê}{}{}{}{v.i.}{Ser suficiente; bastar, atender.}{sa.tis.fa.zer}{\verboinum{42}}
\verb{satisfeito}{ê}{}{}{}{adj.}{Que se satisfez, se contentou.}{sa.tis.fei.to}{0}
\verb{satisfeito}{ê}{}{}{}{}{Contente, alegre de ter feito ou dito algo.}{sa.tis.fei.to}{0}
\verb{satisfeito}{ê}{}{}{}{}{Que se realizou; executado, cumprido.}{sa.tis.fei.to}{0}
\verb{satisfeito}{ê}{}{}{}{}{Repleto, saciado, farto.}{sa.tis.fei.to}{0}
\verb{saturação}{}{}{"-ões}{}{s.f.}{Ato ou efeito de saturar, saciar.}{sa.tu.ra.ção}{0}
\verb{saturação}{}{Fís.}{"-ões}{}{}{Estado em que o vapor se encontra em equilíbrio com seu líquido.}{sa.tu.ra.ção}{0}
\verb{saturação}{}{Quím.}{"-ões}{}{}{Operação que consiste em dissolver ao máximo uma substância em uma solução  numa dada temperatura.}{sa.tu.ra.ção}{0}
\verb{saturado}{}{}{}{}{adj.}{Que se saturou; impregnado, embebido ao máximo.}{sa.tu.ra.do}{0}
\verb{saturado}{}{}{}{}{}{Plenamente saciado; farto, cheio.}{sa.tu.ra.do}{0}
\verb{saturado}{}{}{}{}{}{Enfastiado, aborrecido.}{sa.tu.ra.do}{0}
\verb{saturar}{}{}{}{}{v.t.}{Encher inteiramente; impregnar, embeber.}{sa.tu.rar}{0}
\verb{saturar}{}{}{}{}{}{Fartar, saciar.}{sa.tu.rar}{0}
\verb{saturar}{}{}{}{}{}{Incomodar, enfastiar, aborrecer.}{sa.tu.rar}{\verboinum{1}}
\verb{saturnino}{}{}{}{}{adj.}{Relativo ao chumbo e a seus compostos.}{sa.tur.ni.no}{0}
\verb{saturnino}{}{}{}{}{}{Diz"-se de doença provocada pelo chumbo.}{sa.tur.ni.no}{0}
\verb{saturnino}{}{Fig.}{}{}{}{Sombrio, triste, melancólico.}{sa.tur.ni.no}{0}
\verb{saturnismo}{}{Med.}{}{}{s.m.}{Envenenamento agudo ou crônico causado por chumbo ou por um de seus compostos.}{sa.tur.nis.mo}{0}
\verb{saturno}{}{}{}{}{s.m.}{Em ordem crescente e em relação ao Sol, o segundo e maior planeta do sistema solar. (Nessa acepção, com inicial maiúscula.)}{sa.tur.no}{0}
\verb{saturno}{}{}{}{}{}{Ver \textit{chumbo}.}{sa.tur.no}{0}
\verb{saudação}{}{}{"-ões}{}{s.f.}{Ato ou efeito de saudar.}{sau.da.ção}{0}
\verb{saudação}{}{}{"-ões}{}{}{Gesto ou palavra de cumprimento; demonstração de respeito.}{sau.da.ção}{0}
\verb{saudade}{}{}{}{}{s.f.}{Sentimento de incompletude causado pela ausência de pessoa, coisa ou situação.}{sau.da.de}{0}
\verb{saudade}{}{Bras.}{}{}{}{Tipo de cantiga entoada por marinheiros em alto"-mar.}{sau.da.de}{0}
\verb{saudade}{}{}{}{}{}{Cumprimentos afetuosos enviados a pessoa ausente. (Usa"-se no plural nesta acepção.)}{sau.da.de}{0}
\verb{saudar}{}{}{}{}{v.t.}{Cumprimentar, cortejar, louvar.}{sau.dar}{0}
\verb{saudar}{}{}{}{}{}{Aclamar.}{sau.dar}{\verboinum{1}}
\verb{saudável}{}{}{"-eis}{}{adj.2g.}{Que é bom para a saúde; salutar, salubre.}{sau.dá.vel}{0}
\verb{saudável}{}{}{"-eis}{}{}{Que tem ou demonstra boa saúde física ou mental.}{sau.dá.vel}{0}
\verb{saudável}{}{Por ext.}{"-eis}{}{}{Proveitoso, útil, benéfico.}{sau.dá.vel}{0}
\verb{saúde}{}{}{}{}{s.f.}{Estado de equilíbrio das funções fisiológicas ou psíquicas de um indivíduo.}{sa.ú.de}{0}
\verb{saúde}{}{}{}{}{}{Força, vigor, energia.}{sa.ú.de}{0}
\verb{saúde}{}{}{}{}{interj.}{Expressão utilizada quando alguém espirra.}{sa.ú.de}{0}
\verb{saúde}{}{}{}{}{}{Expressão utilizada quando se faz um brinde.}{sa.ú.de}{0}
\verb{saudita}{}{}{}{}{adj.}{Relativo a Arábia Saudita.}{sau.di.ta}{0}
\verb{saudita}{}{}{}{}{s.m.}{Indivíduo natural ou habitante desse país; árabe"-saudita.}{sau.di.ta}{0}
\verb{saudosismo}{}{}{}{}{s.m.}{Tendência a valorizar o passado, superestimando"-o.}{sau.do.sis.mo}{0}
\verb{saudosista}{}{}{}{}{adj.2g.}{Relativo a saudosismo.}{sau.do.sis.ta}{0}
\verb{saudosista}{}{}{}{}{s.2g.}{Indivíduo que cultiva o saudosismo.}{sau.do.sis.ta}{0}
\verb{saudoso}{ô}{}{"-osos ⟨ó⟩}{"-osa ⟨ó⟩}{adj.}{Que inspira saudades.}{sau.do.so}{0}
\verb{saudoso}{ô}{Por ext.}{"-osos ⟨ó⟩}{"-osa ⟨ó⟩}{}{Falecido.}{sau.do.so}{0}
\verb{saudoso}{ô}{}{"-osos ⟨ó⟩}{"-osa ⟨ó⟩}{}{Que sente saudades.}{sau.do.so}{0}
\verb{saudoso}{ô}{}{"-osos ⟨ó⟩}{"-osa ⟨ó⟩}{}{Que expressa saudades. (\textit{Recebi uma carta saudosa de uma grande amiga.})}{sau.do.so}{0}
\verb{sauna}{}{}{}{}{s.f.}{Banho em ambiente quente, entre 60 e 80 graus centígrados, repleto de vapor.}{sau.na}{0}
\verb{sauna}{}{}{}{}{}{Estabelecimento onde se toma esse tipo de banho.}{sau.na}{0}
\verb{sauna}{}{Fig.}{}{}{}{Qualquer recinto desagradavelmente quente.}{sau.na}{0}
\verb{sáurio}{}{}{}{}{s.m.}{Animal réptil com pele revestida de escamas, encontrado em regiões tropicais e temperadas de todo o mundo; lagarto.}{sáu.ri.o}{0}
\verb{saúva}{}{Zool.}{}{}{s.f.}{Formiga cortadeira que utiliza folhas cortadas para cultivar seu alimento, sendo uma relevante praga agrícola.}{sa.ú.va}{0}
\verb{sauveiro}{ê}{Bras.}{}{}{s.m.}{Toca de saúva; formigueiro.}{sa.u.vei.ro}{0}
\verb{savana}{}{}{}{}{s.f.}{Tipo de paisagem vegetal com gramíneas, pequenos arbustos e árvores esparsas, característica de regiões tropicais com longa estação seca.}{sa.va.na}{0}
\verb{saveiro}{ê}{}{}{}{s.m.}{Barco de pequeno porte, utilizado na pesca ou na travessia de rios.}{sa.vei.ro}{0}
\verb{saxão}{cs}{}{"-ões}{saxã}{adj.}{Relativo aos saxões, antigo povo germânico que invadiu a Inglaterra entre os séculos \textsc{v} e \textsc{vi}, e lá se fixou.}{sa.xão}{0}
\verb{saxão}{cs}{}{"-ões}{saxã}{s.m.}{Indivíduo natural ou habitante da moderna Saxônia (Alemanha).}{sa.xão}{0}
\verb{saxão}{cs}{Por ext.}{"-ões}{saxã}{}{Inglês.}{sa.xão}{0}
\verb{sáxeo}{cs}{}{}{}{adj.}{De pedra; pétreo.}{sá.xe.o}{0}
\verb{saxifragácea}{cs}{Bot.}{}{}{s.f.}{Planta cujas flores formam inflorescências compactas, como a hortênsia.}{sa.xi.fra.gá.cea}{0}
\verb{saxofone}{cs}{Mús.}{}{}{s.m.}{Instrumento de sopro feito de metal, com sistema de chaves e embocadura de palheta simples.}{sa.xo.fo.ne}{0}
\verb{saxofonista}{cs}{Mús.}{}{}{s.2g.}{Músico que toca saxofone.}{sa.xo.fo.nis.ta}{0}
\verb{saxônico}{cs}{}{}{}{adj. e s.m.  }{Ver \textit{saxão}.}{sa.xô.ni.co}{0}
\verb{saxônio}{cs}{}{}{}{adj. e s.m.  }{Ver \textit{saxão}.}{sa.xô.nio}{0}
\verb{saxorne}{csó}{Mús.}{}{}{s.m.}{Instrumento de sopro feito de metal com tubo alongado e cônico e dotado de pistões.}{sa.xor.ne}{0}
\verb{sazão}{}{}{"-ões}{}{s.f.}{Cada uma das estações do ano.}{sa.zão}{0}
\verb{sazão}{}{Fig.}{"-ões}{}{}{Ocasião propícia; ensejo.}{sa.zão}{0}
\verb{sazonado}{}{}{}{}{adj.}{Pronto para colher; maduro.}{sa.zo.na.do}{0}
\verb{sazonal}{}{}{"-ais}{}{adj.2g.}{Próprio de uma estação ou época do ano.}{sa.zo.nal}{0}
\verb{sazonal}{}{}{"-ais}{}{}{Relativo a sazão.}{sa.zo.nal}{0}
\verb{sazonar}{}{}{}{}{v.t.}{Amadurecer.}{sa.zo.nar}{0}
\verb{sazonar}{}{Fig.}{}{}{}{Tornar saboroso, agradável, interessante.}{sa.zo.nar}{\verboinum{1}}
\verb{Sb}{}{Quím.}{}{}{}{Símb. do \textit{antimônio}.}{Sb}{0}
\verb{Sc}{}{Quím.}{}{}{}{Símb. do \textit{escândio}.}{Sc}{0}
\verb{SC}{}{}{}{}{}{Sigla do estado de Santa Catarina.}{SC}{0}
\verb{scanner}{}{}{}{}{s.m.}{Dispositivo que digitaliza uma imagem, transformando informação óptica em dados que podem ser armazenados em um computador.}{\textit{scanner}}{0}
\verb{script}{}{}{}{}{s.m.}{Texto com os diálogos e indicações cênicas de peça teatral, filme ou programa de televisão.}{\textit{script}}{0}
\verb{script}{}{Informát.}{}{}{}{Conjunto de instruções executadas em uma ordem predefinida em determinado aplicativo.}{\textit{script}}{0}
\verb{Se}{}{Quím.}{}{}{}{Símb. do \textit{selênio}.}{Se}{0}
\verb{SE}{}{}{}{}{}{Sigla do estado de Sergipe.}{SE}{0}
\verb{se}{}{}{}{}{pron.}{A si mesmo. (\textit{Ele se viu no espelho.})}{se}{0}
\verb{sé}{}{}{}{}{s.f.}{Principal igreja de uma diocese; igreja episcopal.}{sé}{0}
\verb{sé}{}{}{}{}{}{A jurisdição episcopal.}{sé}{0}
\verb{se}{}{}{}{}{}{Um ao outro. (\textit{Discutiram e acabaram se batendo.})}{se}{0}
\verb{se}{}{}{}{}{}{Uma ou mais pessoas. (\textit{Precisa"-se de empregados.})}{se}{0}
\verb{se}{}{}{}{}{conj.}{Palavra que indica alguma dúvida ou condição. (\textit{Ele vem se quiser.})}{se}{0}
\verb{seabórgio}{}{Quím.}{}{}{s.m.}{Elemento químico artificial, radioativo, metálico, sólido, de aspecto prateado, com nenhum uso específico aleḿ da pesquisa científico. \elemento{106}{(263)}{Sg}.}{sea.bór.gio}{0}
\verb{seara}{}{}{}{}{s.f.}{Campo de cereais.}{se.a.ra}{0}
\verb{seara}{}{}{}{}{}{Extensão de terra cultivada.}{se.a.ra}{0}
\verb{seara}{}{}{}{}{}{Conjunto de pessoas de uma associação, partido, agremiação.}{se.a.ra}{0}
\verb{sebáceo}{}{}{}{}{adj.}{Que tem sebo; semelhante a sebo; sebento, seboso.}{se.bá.ceo}{0}
\verb{sebáceo}{}{}{}{}{}{Que contém ou produz sebo.}{se.bá.ceo}{0}
\verb{sebe}{é}{}{}{}{s.f.}{Cerca de plantas ou arbustos para proteger terrenos e quintais.}{se.be}{0}
\verb{sebento}{}{}{}{}{adj.}{Que é semelhante ao sebo; sebáceo.}{se.ben.to}{0}
\verb{sebento}{}{}{}{}{}{Sujo, seboso.}{se.ben.to}{0}
\verb{sebo}{ê}{}{}{}{s.m.}{Produto das glândulas sebáceas, com função de proteger a pele.}{se.bo}{0}
\verb{sebo}{ê}{}{}{}{}{Substância gordurosa e consistente extraída das vísceras de alguns animais.}{se.bo}{0}
\verb{seborreia}{é}{}{}{}{s.f.}{Secreção excessiva de sebo pelas glândulas sebáceas.}{se.bor.rei.a}{0}
\verb{seborreico}{é}{}{}{}{adj.}{Relativo a seborreia.}{se.bor.rei.co}{0}
\verb{seboso}{ô}{}{"-osos ⟨ó⟩}{"-osa ⟨ó⟩}{adj.}{Semelhante ao sebo; sebáceo.}{se.bo.so}{0}
\verb{seboso}{ô}{}{"-osos ⟨ó⟩}{"-osa ⟨ó⟩}{}{Sujo de sebo ou material gorduroso.}{se.bo.so}{0}
\verb{seboso}{ô}{Pop.}{"-osos ⟨ó⟩}{"-osa ⟨ó⟩}{}{Pedante, imodesto; metido a sebo.}{se.bo.so}{0}
\verb{seboso}{ô}{Bras.}{"-osos ⟨ó⟩}{"-osa ⟨ó⟩}{s.m.}{Indivíduo sujo, porcalhão.}{se.bo.so}{0}
\verb{seca}{ê}{}{}{}{s.f.}{Período sem chuvas; estiagem.}{se.ca}{0}
\verb{seca}{ê}{Pop.}{}{}{}{Conversa longa.}{se.ca}{0}
\verb{seca}{ê}{Pop.}{}{}{}{Má sorte; azar.}{se.ca}{0}
\verb{secador}{ô}{}{}{}{adj.}{Diz"-se de vários aparelhos destinados a retirar a umidade de diversos materiais como tecidos, grãos de café, louças, cabelos, roupas etc.}{se.ca.dor}{0}
\verb{secagem}{}{}{"-ens}{}{s.f.}{Ato ou efeito de secar.}{se.ca.gem}{0}
\verb{secagem}{}{}{"-ens}{}{}{Processo que deixa amargos os grãos de cevada para a produção de cerveja.}{se.ca.gem}{0}
\verb{secante}{}{}{}{}{adj.2g.}{Que seca.}{se.can.te}{0}
\verb{secante}{}{}{}{}{s.m.}{Substância empregada pelos pintores para fazer secar mais rapidamente as tintas.  }{se.can.te}{0}
\verb{secante}{}{Geom.}{}{}{adj.2g.}{Diz"-se de reta que intercepta outra reta ou curva. }{se.can.te}{0}
\verb{secante}{}{Mat.}{}{}{}{Diz"-se de função igual ao inverso do cosseno. Símb.: sec.}{se.can.te}{0}
\verb{seção}{}{}{}{}{}{Var. de \textit{secção}.}{se.ção}{0}
\verb{secar}{}{}{}{}{v.t.}{Retirar a água; fazer ressequir. (\textit{O sol secou a roupa.})}{se.car}{0}
\verb{secar}{}{}{}{}{v.i.}{Perder a água. (\textit{O poço secou.})}{se.car}{0}
\verb{secar}{}{Pop.}{}{}{v.t.}{Cobiçar. (\textit{A menina vive secando o namorado da outra.})}{se.car}{\verboinum{2}}
\verb{secarrão}{}{}{"-ões}{}{adj.}{Que é muito seco.}{se.car.rão}{0}
\verb{secarrão}{}{}{"-ões}{}{s.m.}{Indivíduo muito seco, rude, que não manifesta carinho ou ternura.}{se.car.rão}{0}
\verb{secativo}{}{}{}{}{adj.}{Diz"-se de substância de ação adstringente ou cicatrizante.}{se.ca.ti.vo}{0}
\verb{secção}{}{}{"-ões}{}{s.f.}{Ato ou efeito de seccionar.}{sec.ção}{0}
\verb{secção}{}{}{"-ões}{}{}{Parte de um todo; parcela, porção, segmento.}{sec.ção}{0}
\verb{secção}{}{}{"-ões}{}{}{Cada uma das subdivisões de uma repartição pública, empresa, estabelecimento comercial etc.}{sec.ção}{0}
\verb{seccional}{}{}{"-ais}{}{adj.2g.}{Relativo a secção.}{sec.ci.o.nal}{0}
\verb{seccionar}{}{}{}{}{v.t.}{Dividir em partes, pedaços, porções.}{sec.ci.o.nar}{\verboinum{1}}
\verb{secessão}{}{}{"-ões}{}{s.f.}{Ato de separar; separação.}{se.ces.são}{0}
\verb{secional}{}{}{}{}{}{Var. de \textit{seccional}.}{se.ci.o.nal}{0}
\verb{secionar}{}{}{}{}{}{Var. de \textit{seccionar}.}{se.ci.o.nar}{0}
\verb{seco}{ê}{}{}{}{adj.}{Sem umidade; enxuto.}{se.co}{0}
\verb{seco}{ê}{}{}{}{}{Murcho (diz"-se de planta).}{se.co}{0}
\verb{seco}{ê}{}{}{}{}{Muito magro; descarnado.}{se.co}{0}
\verb{seco}{ê}{Fig.}{}{}{}{Sem cortesia ou ternura; ríspido, grosseiro.}{se.co}{0}
\verb{secos}{ê}{}{}{}{s.m.pl.}{Gêneros alimentícios sólidos.}{se.cos}{0}
\verb{secreção}{}{}{"-ões}{}{s.f.}{Ato ou efeito de segregar.}{se.cre.ção}{0}
\verb{secreção}{}{}{"-ões}{}{}{Operação pela qual as células, especialmente as dos tecidos glandulares, elaboram materiais que são evacuados, por um canal excretor, para um outro órgão ou para o exterior, ou diretamente para o sangue.}{se.cre.ção}{0}
\verb{secreção}{}{}{"-ões}{}{}{O produto dessa operação.  }{se.cre.ção}{0}
\verb{secreta}{é}{Relig.}{}{}{s.f.}{Oração dita em voz baixa pelo padre antes da missa.}{se.cre.ta}{0}
\verb{secreta}{é}{Pop.}{}{}{}{Sanitário, privada, latrina.}{se.cre.ta}{0}
\verb{secretar}{}{}{}{}{v.t.}{Produzir secreção; expelir.}{se.cre.tar}{\verboinum{1}}
\verb{secretaria}{}{}{}{}{s.f.}{Local de uma empresa ou repartição onde se fazem serviços de expediente e se arquivam documentos.}{se.cre.ta.ri.a}{0}
\verb{secretaria}{}{}{}{}{}{Órgão de um governo que cuida de determinado setor da administração.}{se.cre.ta.ri.a}{0}
\verb{secretária}{}{}{}{}{s.f.}{Mulher que trabalha numa secretaria. (\textit{Ela é secretária no Ministério de Minas e Energia.})}{se.cre.tá.ria}{0}
\verb{secretária}{}{Desus.}{}{}{}{Mesa de escritório; escrivaninha. (\textit{Deixou todos os papéis sobre a secretária e saiu.})}{se.cre.tá.ria}{0}
\verb{secretariado}{}{}{}{}{s.m.}{Cargo ou função de secretário.}{se.cre.ta.ri.a.do}{0}
\verb{secretariado}{}{}{}{}{}{Conjunto dos secretários de Estado.}{se.cre.ta.ri.a.do}{0}
\verb{secretariar}{}{}{}{}{v.t.}{Exercer funções em uma secretaria. }{se.cre.ta.ri.ar}{\verboinum{1}}
\verb{secretário}{}{}{}{}{s.m.}{Indivíduo responsável por anotar e escrever a ata de uma reunião.}{se.cre.tá.rio}{0}
\verb{secretário}{}{}{}{}{}{Indivíduo responsável pela correspondência de uma pessoa ou organização.}{se.cre.tá.rio}{0}
\verb{secretário}{}{}{}{}{}{Indivíduo responsável por uma secretaria de Estado.}{se.cre.tá.rio}{0}
\verb{secreto}{é}{}{}{}{adj.}{Que está em segredo; encoberto; não revelado.}{se.cre.to}{0}
\verb{secreto}{é}{}{}{}{}{Que não é visível; não aparente.}{se.cre.to}{0}
\verb{secreto}{é}{}{}{}{}{Que se oculta, que se impede de manifestar; íntimo, particular.}{se.cre.to}{0}
\verb{secretor}{ô}{}{}{}{adj.}{Que segrega, que produz secreção. (\textit{Algumas células são secretoras de hormônios.})}{se.cre.tor}{0}
\verb{sectário}{}{}{}{}{adj.}{Relativo a seita.}{sec.tá.rio}{0}
\verb{sectário}{}{}{}{}{s.m.}{Membro de uma seita.}{sec.tá.rio}{0}
\verb{sectário}{}{}{}{}{}{Partidário ferrenho, apaixonado.}{sec.tá.rio}{0}
\verb{sectarismo}{}{}{}{}{s.m.}{Espírito limitado, estreito, de seita.}{sec.ta.ris.mo}{0}
\verb{sectarismo}{}{}{}{}{}{Partidarismo faccioso.}{sec.ta.ris.mo}{0}
\verb{sector}{ô}{}{}{}{}{Var. de \textit{setor}.}{sec.tor}{0}
\verb{secular}{}{}{}{}{adj.2g.}{Relativo a século.}{se.cu.lar}{0}
\verb{secular}{}{}{}{}{}{Que se faz de século a século.}{se.cu.lar}{0}
\verb{secular}{}{Por ext.}{}{}{}{Que é muito antigo, que dura há muitos anos.}{se.cu.lar}{0}
\verb{secular}{}{Fig.}{}{}{}{Tão longo, tão demorado que parece durar um século.}{se.cu.lar}{0}
\verb{secular}{}{}{}{}{}{Próprio do século (mundo), que não cabe à Igreja; profano, temporal, mundano.}{se.cu.lar}{0}
\verb{secular}{}{}{}{}{}{Diz"-se de eclesiástico que participa do século, da vida civil, em oposição àquele que pertence a uma ordem religiosa.}{se.cu.lar}{0}
\verb{secularizar}{}{}{}{}{v.t.}{Fazer voltar ao século, à vida leiga; dispensar dos votos monásticos; deixar de pertencer a uma ordem ou à vida religiosa.}{se.cu.la.ri.zar}{0}
\verb{secularizar}{}{}{}{}{}{Sujeitar, o que era do direito canônico, às leis civis.}{se.cu.la.ri.zar}{0}
\verb{secularizar}{}{}{}{}{}{Tomar terras ou bens de Igreja.}{se.cu.la.ri.zar}{\verboinum{1}}
\verb{século}{}{}{}{}{s.m.}{Período de cem anos. (\textit{Foi durante o século retrasado que ocorreram a Independência e a Proclamação da República.})}{sé.cu.lo}{0}
\verb{secundar}{}{}{}{}{v.t.}{Auxiliar, ajudar em funções; coadjuvar.}{se.cun.dar}{0}
\verb{secundar}{}{}{}{}{}{Fazer ou tentar fazer pela segunda vez.}{se.cun.dar}{0}
\verb{secundar}{}{Bras.}{}{}{}{Responder, replicar.}{se.cun.dar}{0}
\verb{secundar}{}{Bras.}{}{}{}{Tornar a fazer ou a dizer; repetir, reforçar.}{se.cun.dar}{\verboinum{1}}
\verb{secundário}{}{}{}{}{adj.}{Que vem ou está em segundo lugar ou ordem.}{se.cun.dá.rio}{0}
\verb{secundário}{}{}{}{}{}{De pouco valor, de menor importância; insignificante.}{se.cun.dá.rio}{0}
\verb{secundário}{}{}{}{}{}{Antigo designativo do ensino ou instrução de grau intermediário, antes do superior.}{se.cun.dá.rio}{0}
\verb{secundinas}{}{Med.}{}{}{s.f.pl.}{Conjunto de placenta e membranas expulsas após o parto.}{se.cun.di.nas}{0}
\verb{secura}{}{}{}{}{s.f.}{Qualidade de seco.}{se.cu.ra}{0}
\verb{secura}{}{}{}{}{}{Ausência de umidade; seca.}{se.cu.ra}{0}
\verb{secura}{}{}{}{}{}{Sede.}{se.cu.ra}{0}
\verb{secura}{}{}{}{}{}{Frieza, dureza, aspereza.}{se.cu.ra}{0}
\verb{secura}{}{Pop.}{}{}{}{Desejo ardente, sobretudo de natureza sexual.}{se.cu.ra}{0}
\verb{securitário}{}{}{}{}{adj.}{Relativo a seguros.}{se.cu.ri.tá.rio}{0}
\verb{securitário}{}{}{}{}{s.m.}{Funcionário de companhia de seguros.}{se.cu.ri.tá.rio}{0}
\verb{seda}{ê}{}{}{}{s.f.}{Substância filamentosa e brilhante que constitui o casulo do bicho"-da"-seda, ou o fio feito com tal substância.}{se.da}{0}
\verb{seda}{ê}{Por ext.}{}{}{}{Tecido feito com esse fio.}{se.da}{0}
\verb{seda}{ê}{Fig.}{}{}{}{Pessoa muito amável, delicada.}{se.da}{0}
\verb{sedar}{}{}{}{}{v.t.}{Dopar, drogar, tranquilizar por meio de sedativo.}{se.dar}{\verboinum{1}}
\verb{sedativo}{}{}{}{}{adj.}{Que seda ou acalma; calmante.}{se.da.ti.vo}{0}
\verb{sedativo}{}{}{}{}{s.m.}{Medicamento ou qualquer substância que acalma ou faz ceder dor, ansiedade; calmante, tranquilizante.}{se.da.ti.vo}{0}
\verb{sede}{é}{}{}{}{s.f.}{Lugar onde funciona um tribunal, um governo, uma administração, ou onde uma empresa tem seu principal estabelecimento.}{se.de}{0}
\verb{sede}{é}{}{}{}{}{Ponto em que se concentram certos fatos ou fenômenos.}{se.de}{0}
\verb{sede}{ê}{Fig.}{}{}{}{Desejo imoderado; cobiça, avidez.}{se.de}{0}
\verb{sede}{ê}{}{}{}{s.f.}{Sensação produzida pela necessidade de beber.}{se.de}{0}
\verb{sede}{ê}{}{}{}{}{Falta de umidade; secura.  }{se.de}{0}
\verb{sedentariedade}{}{}{}{}{s.f.}{Qualidade de sedentário.}{se.den.ta.ri.e.da.de}{0}
\verb{sedentariedade}{}{}{}{}{}{Vida de sedentário; inatividade.}{se.den.ta.ri.e.da.de}{0}
\verb{sedentário}{}{}{}{}{adj.}{Que está comumente sentado, que anda e se exercita pouco; inativo.}{se.den.tá.rio}{0}
\verb{sedentário}{}{}{}{}{}{Que tem habitação fixa.}{se.den.tá.rio}{0}
\verb{sedento}{}{}{}{}{adj.}{Que tem muita sede.}{se.den.to}{0}
\verb{sedento}{}{Fig.}{}{}{}{Muito desejoso ou ávido.}{se.den.to}{0}
\verb{sedestre}{é}{}{}{}{adj.}{Diz"-se de estátua de pessoa sentada.}{se.des.tre}{0}
\verb{sediado}{}{}{}{}{adj.}{Que tem abrigo, sede. (\textit{Ficamos sediados na casa da fazenda.})}{se.di.a.do}{0}
\verb{sediar}{}{}{}{}{v.t.}{Servir de sede; acolher.}{se.di.ar}{\verboinum{6}}
\verb{sedição}{}{}{"-ões}{}{s.f.}{Sublevação contra qualquer autoridade constituída; revolta, motim.}{se.di.ção}{0}
\verb{sedicioso}{ô}{}{"-osos ⟨ó⟩}{"-osa ⟨ó⟩}{adj.}{Que incita à sedição ou dela participa; revoltoso.}{se.di.ci.o.so}{0}
\verb{sedicioso}{ô}{}{"-osos ⟨ó⟩}{"-osa ⟨ó⟩}{s.m.}{Indivíduo que provoca ou incita à sedição, ou que nela se envolve; sublevado, rebelde.}{se.di.ci.o.so}{0}
\verb{sedimentação}{}{}{"-ões}{}{s.f.}{Ato ou efeito de sedimentar; formação de sedimentos.}{se.di.men.ta.ção}{0}
\verb{sedimentação}{}{}{"-ões}{}{}{Processo pelo qual substâncias minerais ou rochosas, ou substâncias de origem orgânica, se depositam em ambiente aquoso ou aéreo.}{se.di.men.ta.ção}{0}
\verb{sedimentar}{}{}{}{}{adj.2g.}{Formado pela deposição de sedimento ou a partir de um processo de sedimentação.}{se.di.men.tar}{0}
\verb{sedimentar}{}{}{}{}{v.i.}{Formar sedimentos.}{se.di.men.tar}{\verboinum{1}}
\verb{sedimento}{}{}{}{}{s.m.}{Material sólido desagregado, originado da alteração de rochas preexistentes e transportado ou depositado pelo ar, água ou gelo.}{se.di.men.to}{0}
\verb{sedimento}{}{}{}{}{}{Qualquer depósito material insolúvel, especialmente se assentado por gravitação.}{se.di.men.to}{0}
\verb{sedoso}{ô}{}{"-osos ⟨ó⟩}{"-osa ⟨ó⟩}{adj.}{Que contém seda.}{se.do.so}{0}
\verb{sedoso}{ô}{}{"-osos ⟨ó⟩}{"-osa ⟨ó⟩}{}{Semelhante à seda; fino, macio.}{se.do.so}{0}
\verb{sedução}{}{}{"-ões}{}{s.f.}{Ato ou efeito de seduzir ou ser seduzido.}{se.du.ção}{0}
\verb{sedução}{}{}{"-ões}{}{}{Qualidade de sedutor.}{se.du.ção}{0}
\verb{sedução}{}{}{"-ões}{}{}{Atração, encanto, fascínio.}{se.du.ção}{0}
\verb{sedutor}{ô}{}{}{}{adj.}{Que seduz; atraente, encantador.}{se.du.tor}{0}
\verb{sedutor}{ô}{}{}{}{s.m.}{Indivíduo que seduz.}{se.du.tor}{0}
\verb{seduzir}{}{}{}{}{v.t.}{Convencer, persuadir com astúcia, sob promessa de vantagens. (\textit{O vendedor o seduziu com toda aquela conversa e ele acabou comprando.})}{se.du.zir}{0}
\verb{seduzir}{}{}{}{}{}{Desonrar mulher menor de idade, recorrendo a promessas, encantos ou amavios.}{se.du.zir}{\verboinum{21}}
\verb{sega}{é}{}{}{}{s.f.}{Ato ou efeito de segar; ceifa.}{se.ga}{0}
\verb{sega}{é}{}{}{}{}{O tempo que dura a ceifa.}{se.ga}{0}
\verb{segadeira}{ê}{Lus.}{}{}{s.f.}{Espécie de foice de grande porte.}{se.ga.dei.ra}{0}
\verb{segadeira}{ê}{}{}{}{}{Ver \textit{ceifadeira}.}{se.ga.dei.ra}{0}
\verb{segador}{ô}{}{}{}{adj.}{Que sega, que ceifa.}{se.ga.dor}{0}
\verb{segador}{ô}{}{}{}{s.m.}{Indivíduo que sega; ceifador.}{se.ga.dor}{0}
\verb{segadura}{}{}{}{}{s.f.}{Ver \textit{sega}.}{se.ga.du.ra}{0}
\verb{segar}{}{}{}{}{v.t.}{Cortar ou abater cereais, ervas etc. com foice ou instrumento apropriado; ceifar.}{se.gar}{0}
\verb{segar}{}{}{}{}{}{Pôr fim; acabar.}{se.gar}{\verboinum{5}}
\verb{sege}{é}{}{}{}{s.f.}{Antiga carruagem fechada, com duas rodas e um só assento, fechada com cortinas na parte dianteira.}{se.ge}{0}
\verb{sege}{é}{Por ext.}{}{}{}{Qualquer carruagem.}{se.ge}{0}
\verb{segmentação}{}{}{"-ões}{}{s.f.}{Ato ou efeito de segmentar; divisão por segmentos; fracionamento.}{seg.men.ta.ção}{0}
\verb{segmentar}{}{}{}{}{adj.2g.}{Relativo a segmento. }{seg.men.tar}{0}
\verb{segmentar}{}{}{}{}{}{Formado por segmentos.}{seg.men.tar}{0}
\verb{segmentar}{}{}{}{}{v.t.}{Dividir em segmentos.}{seg.men.tar}{\verboinum{1}}
\verb{segmentário}{}{}{}{}{adj.}{Ver \textit{segmentar}.}{seg.men.tá.rio}{0}
\verb{segmento}{}{}{}{}{s.m.}{Porção de um todo; seção.}{seg.men.to}{0}
\verb{segmento}{}{}{}{}{}{Porção bem delimitada, destacada de um conjunto.}{seg.men.to}{0}
\verb{segredar}{}{}{}{}{v.t.}{Dizer em segredo ou em voz baixa; cochichar, murmurar.}{se.gre.dar}{\verboinum{1}}
\verb{segredo}{ê}{}{}{}{s.m.}{Coisa que não deve ser contada aos outros.}{se.gre.do}{0}
\verb{segredo}{ê}{}{}{}{}{Coisa que se conta a uma pessoa e se esconde dos outros; confidência.}{se.gre.do}{0}
\verb{segredo}{ê}{}{}{}{}{Sequência de números que servem para abrir uma fechadura. (\textit{Cada cofre tem um segredo que só o dono conhece.})}{se.gre.do}{0}
\verb{segregação}{}{}{"-ões}{}{s.f.}{Ato ou efeito de segregar; afastamento, separação.}{se.gre.ga.ção}{0}
\verb{segregar}{}{}{}{}{v.t.}{Separar com o objetivo de isolar, de evitar contato; desligar, desunir.}{se.gre.gar}{0}
\verb{segregar}{}{}{}{}{}{Pôr de lado; separar, apartar.}{se.gre.gar}{0}
\verb{segregar}{}{}{}{}{}{Expelir qualquer secreção.}{se.gre.gar}{\verboinum{5}}
\verb{seguida}{}{}{}{}{s.f.}{Ato ou efeito de seguir; seguimento, continuação.}{se.gui.da}{0}
\verb{seguido}{}{}{}{}{}{Que se segue ou imita, posto em prática; adotado, usado.}{se.gui.do}{0}
\verb{seguido}{}{}{}{}{adj.}{Que está ou vem logo depois; imediato, contínuo.}{se.gui.do}{0}
\verb{seguido}{}{}{}{}{}{Que persiste continuamente; sem interrupção.}{se.gui.do}{0}
\verb{seguido}{}{}{}{}{adv.}{Seguidamente, ininterruptamente.}{se.gui.do}{0}
\verb{seguidor}{ô}{}{}{}{adj.}{Que segue; continuador.}{se.gui.dor}{0}
\verb{seguidor}{ô}{}{}{}{}{Que é simpatizante de um partido; partidário.}{se.gui.dor}{0}
\verb{seguidor}{ô}{}{}{}{}{Que segue; perseguidor.}{se.gui.dor}{0}
\verb{seguidor}{ô}{}{}{}{s.m.}{Indivíduo continuador, prosseguidor. }{se.gui.dor}{0}
\verb{seguidor}{ô}{}{}{}{}{Indivíduo partidário, prosélito, sequaz.}{se.gui.dor}{0}
\verb{seguidor}{ô}{}{}{}{}{Indivíduo perseguidor, acossador.}{se.gui.dor}{0}
\verb{seguimento}{}{}{}{}{s.m.}{Ato ou efeito de seguir; seguida, continuação.}{se.gui.men.to}{0}
\verb{seguinte}{}{}{}{}{adj.2g.}{Que segue ou se segue; que vem ou ocorre logo depois; imediato.}{se.guin.te}{0}
\verb{seguinte}{}{}{}{}{s.m.}{Aquilo que se segue a outro.}{se.guin.te}{0}
\verb{seguir}{}{}{}{}{v.t.}{Ir atrás de pessoa ou coisa; acompanhar. (\textit{Os fiéis seguem a procissão.})}{se.guir}{0}
\verb{seguir}{}{}{}{}{}{Não perder coisa ou pessoa de vista; observar, acompanhar. (\textit{O namorado seguia o rival com os olhos.})}{se.guir}{0}
\verb{seguir}{}{}{}{}{}{Ligar"-se a pessoa ou doutrina; abraçar, aceitar, aderir.}{se.guir}{0}
\verb{seguir}{}{}{}{}{}{Fazer o que outra pessoa faz ou manda. (\textit{O filho seguia atentamente as regras da casa.})}{se.guir}{0}
\verb{seguir}{}{}{}{}{}{Ir em alguma direção. (\textit{O diretor seguiu para o auditório para ver a apresentação dos alunos.})}{se.guir}{0}
\verb{seguir}{}{}{}{}{v.pron.}{Vir depois de alguma coisa. (\textit{Ao concerto, seguiu"-se o coquetel.})}{se.guir}{\verboinum{24}}
\verb{segunda}{}{}{}{}{s.f.}{De segunda classe, de qualidade inferior.}{se.gun.da}{0}
\verb{segunda}{}{}{}{}{}{Forma reduzida de \textit{segunda"-feira}.}{se.gun.da}{0}
\verb{segunda}{}{}{}{}{}{Uma das marchas de velocidade dos automóveis.}{se.gun.da}{0}
\verb{segunda}{}{}{}{}{}{Número ordinal, feminino de segundo.}{se.gun.da}{0}
\verb{segunda"-feira}{ê}{}{segundas"-feiras}{}{s.f.}{O segundo dia da semana.}{se.gun.da"-fei.ra}{0}
\verb{segundanista}{}{}{}{}{s.2g.}{Estudante que frequenta o segundo ano de um curso ou faculdade.}{se.gun.da.nis.ta}{0}
\verb{segundo}{}{}{}{}{num.}{Ordinal que corresponde a dois. (\textit{Fui o segundo a chegar à festa.})}{se.gun.do}{0}
\verb{segundo}{}{}{}{}{adj.}{Com todas as características do primeiro; novo, outro. (\textit{Essa cantora é uma segunda Elis Regina.})}{se.gun.do}{0}
\verb{segundo}{}{}{}{}{s.m.}{Cada uma das sessenta partes do minuto.}{se.gun.do}{0}
\verb{segundo}{}{}{}{}{}{Espaço de tempo muito curto; instante, momento.}{se.gun.do}{0}
\verb{segundo}{}{}{}{}{prep.}{De acordo com pessoa ou coisa; conforme.}{se.gun.do}{0}
\verb{segundo}{}{}{}{}{conj.}{Do modo que; conforme, consoante.}{se.gun.do}{0}
\verb{segundo"-sargento}{}{}{segundos"-sargentos}{}{s.m.}{Posto militar entre primeiro"-sargento e terceiro"-sargento.}{se.gun.do"-sar.gen.to}{0}
\verb{segundo"-tenente}{}{}{segundos"-tenentes}{}{s.m.}{Posto militar entre aspirante e primeiro"-tenente.}{se.gun.do"-te.nen.te}{0}
\verb{segundo"-tenente}{}{}{segundos"-tenentes}{}{}{}{se.gun.do"-te.nen.te}{0}
\verb{segurado}{}{}{}{}{adj.}{Que está no seguro; que tem seguro.}{se.gu.ra.do}{0}
\verb{segurado}{}{Jur.}{}{}{s.m.}{Pessoa que paga o prêmio do seguro, obtendo assim a garantia estipulada no contrato.}{se.gu.ra.do}{0}
\verb{segurador}{ô}{}{}{}{adj.}{Que segura.}{se.gu.ra.dor}{0}
\verb{segurador}{ô}{}{}{}{}{Aquele que, num contrato de seguro, se obriga a indenizar prejuízos eventuais.}{se.gu.ra.dor}{0}
\verb{segurador}{ô}{}{}{}{s.m.}{Indivíduo ou coisa que segura.}{se.gu.ra.dor}{0}
\verb{segurador}{ô}{}{}{}{}{Indivíduo que, num contrato de seguro, se obriga a indenizar prejuízos determinados.}{se.gu.ra.dor}{0}
\verb{seguradora}{ô}{}{}{}{s.f.}{Companhia de seguros.}{se.gu.ra.do.ra}{0}
\verb{segurança}{}{}{}{}{s.f.}{Ato ou efeito de tornar seguro; estabilidade, firmeza.}{se.gu.ran.ça}{0}
\verb{segurança}{}{}{}{}{}{Ato ou efeito de assegurar ou garantir alguma coisa; garantia, caução.}{se.gu.ran.ça}{0}
\verb{segurança}{}{}{}{}{}{Condição daquele ou daquilo em que se pode confiar.}{se.gu.ran.ça}{0}
\verb{segurança}{}{}{}{}{}{Certeza, convicção.}{se.gu.ran.ça}{0}
\verb{segurança}{}{}{}{}{}{Protesto, afirmação.}{se.gu.ran.ça}{0}
\verb{segurança}{}{}{}{}{s.2g.}{Indivíduo encarregado de segurança pessoal de alguém, de empresa, de um condomínio etc.}{se.gu.ran.ça}{0}
\verb{segurar}{}{}{}{}{}{Amparar, impedindo que caia ou se arruíne.}{se.gu.rar}{0}
\verb{segurar}{}{}{}{}{}{Agarrar, conter, prender.}{se.gu.rar}{0}
\verb{segurar}{}{}{}{}{v.t.}{Tornar seguro; firmar, fixar.}{se.gu.rar}{0}
\verb{segurar}{}{}{}{}{}{Garantir, afirmar, assegurar.}{se.gu.rar}{0}
\verb{segurar}{}{}{}{}{}{Pôr no seguro; firmar contrato de seguro.}{se.gu.rar}{0}
\verb{segurar}{}{}{}{}{}{Tranquilizar, serenar, sossegar.}{se.gu.rar}{0}
\verb{segurar}{}{}{}{}{v.pron.}{Prevenir"-se, precaver"-se.}{se.gu.rar}{0}
\verb{segurar}{}{}{}{}{}{Conter"-se, controlar"-se.}{se.gu.rar}{\verboinum{1}}
\verb{seguridade}{}{}{}{}{s.f.}{Ver \textit{segurança}.}{se.gu.ri.da.de}{0}
\verb{seguro}{}{}{}{}{adj.}{Sem perigo de cair ou se soltar; firme.}{se.gu.ro}{0}
\verb{seguro}{}{}{}{}{}{Em que não existe perigo. (\textit{A casa possui escadas seguras.})}{se.gu.ro}{0}
\verb{seguro}{}{}{}{}{}{Que tem certeza do que afirma.}{se.gu.ro}{0}
\verb{seguro}{}{}{}{}{}{Que não gosta de gastar dinheiro; avarento, avaro, pão"-duro, sovina.}{se.gu.ro}{0}
\verb{seguro}{}{}{}{}{s.m.}{Contrato pelo qual uma das partes se obriga a pagar algum prejuízo de outra. (\textit{Fizemos o seguro contra roubos da casa.})}{se.gu.ro}{0}
\verb{seguro"-desemprego}{ê}{}{seguros"-desempregos \textit{ou} seguros"-desemprego ⟨ê⟩}{}{s.m.}{Benefício concedido pelo poder público ao trabalhador desempregado, com o intuito de lhe garantir assistência temporária, em razão de dispensa sem justa causa ou de paralisação das atividades do empregador.}{se.gu.ro"-de.sem.pre.go}{0}
\verb{seichelense}{}{}{}{}{adj.2g.}{Relativo às Ilhas Seicheles (arquipélago do Oceano Índico).}{sei.che.len.se}{0}
\verb{seichelense}{}{}{}{}{s.2g.}{Indivíduo natural ou habitante dessas ilhas.}{sei.che.len.se}{0}
\verb{seio}{ê}{}{}{}{s.m.}{Dobra ou prega sinuosa; curvatura, volta.}{sei.o}{0}
\verb{seio}{ê}{Anat.}{}{}{}{Parte do corpo humano onde se situam as glândulas mamárias; peito.}{sei.o}{0}
\verb{seio}{ê}{}{}{}{}{Centro, interior, âmago.}{sei.o}{0}
\verb{seio}{ê}{}{}{}{}{Ambiente, meio.}{sei.o}{0}
\verb{seio}{ê}{}{}{}{}{Ventre, útero.}{sei.o}{0}
\verb{seis}{}{}{}{}{num.}{Nome dado à quantidade expressa pelo número 6.  }{seis}{0}
\verb{seiscentos}{}{}{}{}{num.}{Nome dado à quantidade expressa pelo número 600.  }{seis.cen.tos}{0}
\verb{seita}{ê}{}{}{}{s.f.}{Doutrina ou sistema que se afasta da crença ou opinião geral.}{sei.ta}{0}
\verb{seita}{ê}{}{}{}{}{Conjunto de indivíduos que professam a mesma doutrina.}{sei.ta}{0}
\verb{seita}{ê}{}{}{}{}{Comunidade fechada, de cunho radical.}{sei.ta}{0}
\verb{seita}{ê}{Pop.}{}{}{}{Facção, partido.}{sei.ta}{0}
\verb{seiva}{ê}{}{}{}{s.f.}{Líquido que contém princípios nutritivos e que circula no interior dos vegetais, por meio de um sistema vascular.}{sei.va}{0}
\verb{seiva}{ê}{Por ext.}{}{}{}{Qualquer fluido orgânico aquoso; sangue, linfa.}{sei.va}{0}
\verb{seiva}{ê}{Fig.}{}{}{}{Energia física ou mental; força, vigor.}{sei.va}{0}
\verb{seixo}{ê\ldots{}ch}{}{}{}{s.m.}{Fragmento natural de rocha dura, de diâmetro variável, transportado pela água, que lhe arredonda as arestas.}{sei.xo}{0}
\verb{sela}{é}{}{}{}{s.f.}{Peça de couro posta sobre o lombo da cavalgadura, sobre a qual senta o cavaleiro e na qual são presos outros apetrechos dos arreios.}{se.la}{0}
\verb{selagem}{}{}{"-ens}{}{s.f.}{Ato ou efeito de selar; aposição de selo, carimbo, estampilha.}{se.la.gem}{0}
\verb{selagem}{}{}{"-ens}{}{}{Ato ou efeito de selar; colocação de sela.}{se.la.gem}{0}
\verb{selar}{}{}{}{}{v.t.}{Colocar estampilha, selo, carimbo.}{se.lar}{0}
\verb{selar}{}{}{}{}{}{Confirmar, validar.}{se.lar}{0}
\verb{selar}{}{}{}{}{}{Cerrar, fechar.}{se.lar}{0}
\verb{selar}{}{}{}{}{}{Pôr sela.}{se.lar}{\verboinum{1}}
\verb{selaria}{}{}{}{}{s.f.}{Arte ou ofício de seleiro.}{se.la.ri.a}{0}
\verb{selaria}{}{}{}{}{}{Estabelecimento de seleiro.}{se.la.ri.a}{0}
\verb{selaria}{}{}{}{}{}{Conjunto de selas e demais arreios.}{se.la.ri.a}{0}
\verb{selaria}{}{}{}{}{}{Local onde se guardam selas e arreios.}{se.la.ri.a}{0}
\verb{seleção}{}{}{"-ões}{}{s.f.}{Ato ou efeito de selecionar; escolha fundamentada.}{se.le.ção}{0}
\verb{seleção}{}{}{"-ões}{}{}{Antologia literária.}{se.le.ção}{0}
\verb{seleção}{}{}{"-ões}{}{}{Equipe constituída pelos melhores atletas.}{se.le.ção}{0}
\verb{selecionado}{}{}{}{}{adj.}{Que foi escolhido, eleito.}{se.le.ci.o.na.do}{0}
\verb{selecionado}{}{Por ext.}{}{}{}{Apurado, distinto, especial.}{se.le.ci.o.na.do}{0}
\verb{selecionado}{}{}{}{}{s.m.}{Ver \textit{seleto}.}{se.le.ci.o.na.do}{0}
\verb{selecionar}{}{}{}{}{v.t.}{Fazer seleção; escolher.}{se.le.ci.o.nar}{\verboinum{1}}
\verb{seleiro}{ê}{}{}{}{s.m.}{Fabricante ou vendedor de selas.}{se.lei.ro}{0}
\verb{seleiro}{ê}{}{}{}{adj.}{Que é bom cavaleiro ou se firma bem na sela.}{se.lei.ro}{0}
\verb{seleiro}{ê}{}{}{}{}{Diz"-se do cavalo que já experimentou sela.}{se.lei.ro}{0}
\verb{selênico}{}{}{}{}{adj.}{Relativo à Lua; lunar.}{se.lê.ni.co}{0}
\verb{selênico}{}{}{}{}{}{Relativo ao elemento químico selênio.}{se.lê.ni.co}{0}
\verb{selênio}{}{Quím.}{}{}{s.m.}{Elemento químico do grupo dos não metais, semicondutor positivo, fotossensível, usado em transistores, fotômetros, xerografia, câmaras de televisão, na indústria de vidro e de cerâmicas. \elemento{34}{78.96}{Se}.}{se.lê.nio}{0}
\verb{selenita}{}{}{}{}{s.f.}{Variedade incolor de gipsita, que ocorre em cristais monoclínicos transparentes.}{se.le.ni.ta}{0}
\verb{selenita}{}{}{}{}{s.2g.}{Suposto habitante da Lua.}{se.le.ni.ta}{0}
\verb{seleta}{é}{}{}{}{s.f.}{Conjunto de trechos literários selecionados; antologia.}{se.le.ta}{0}
\verb{seleta}{é}{}{}{}{}{Certa variedade de laranja e de pera.}{se.le.ta}{0}
\verb{seletivo}{}{}{}{}{adj.}{Relativo a seleção.}{se.le.ti.vo}{0}
\verb{seletivo}{}{}{}{}{}{Aquilo que faz seleção.}{se.le.ti.vo}{0}
\verb{seleto}{é}{}{}{}{adj.}{Que foi objeto de seleção; escolhido, selecionado.}{se.le.to}{0}
\verb{seleto}{é}{}{}{}{}{Que ressalta dentre os melhores; excelente, distinto.}{se.le.to}{0}
\verb{seletor}{ô}{}{}{}{adj.}{Que seleciona; selecionador.}{se.le.tor}{0}
\verb{seletor}{ô}{}{}{}{s.m.}{Dispositivo que efetua uma operação de seleção.}{se.le.tor}{0}
\verb{self"-service}{}{}{}{}{s.m.}{Sistema comercial em que o cliente se serve sozinho.}{\textit{self"-service}}{0}
\verb{self"-service}{}{}{}{}{}{Estabelecimento comercial que usa esse sistema.}{\textit{self"-service}}{0}
\verb{selim}{}{}{"-ins}{}{s.m.}{Pequena sela rasa.}{se.lim}{0}
\verb{selim}{}{}{"-ins}{}{}{Assento de bicicletas.}{se.lim}{0}
\verb{selo}{ê}{}{}{}{s.m.}{Pequeno pedaço de papel com estampa, em geral quadrado ou retangular, que se cola sobre o que se manda pelo correio. (\textit{Colei o selo na carta e a coloquei na caixa do correio.})}{se.lo}{0}
\verb{selo}{ê}{}{}{}{}{Marca que o fabricante coloca sobre o produto; etiqueta.}{se.lo}{0}
\verb{selva}{é}{}{}{}{s.f.}{Lugar naturalmente arborizado; floresta, bosque.}{sel.va}{0}
\verb{selva}{é}{Fig.}{}{}{}{Grande quantidade de algo.}{sel.va}{0}
\verb{selva}{é}{Fig.}{}{}{}{Ambiente em que predominam as rivalidades hostis, as disputas acirradas.}{sel.va}{0}
\verb{selvagem}{}{}{"-ens}{}{adj.}{Que nasce, cresce e vive na selva. (\textit{Animais selvagens podem ser muito perigosos.})}{sel.va.gem}{0}
\verb{selvagem}{}{Fig.}{"-ens}{}{}{Em que se nota comportamento rude. (\textit{Aquela pessoa tinha hábitos selvagens.})}{sel.va.gem}{0}
\verb{selvagem}{}{Pejor.}{"-ens}{}{}{Dizia"-se das pessoas que habitavam nas matas.}{sel.va.gem}{0}
\verb{selvageria}{}{}{}{}{s.f.}{Qualidade, comportamento de selvagem.}{sel.va.ge.ri.a}{0}
\verb{selvageria}{}{}{}{}{}{Estado da sociedade entre os selvagens; rusticidade, incivilidade, grosseria.}{sel.va.ge.ri.a}{0}
\verb{selvajaria}{}{}{}{}{}{Var. de \textit{selvageria}.}{sel.va.ja.ri.a}{0}
\verb{selvático}{}{}{}{}{adj.}{Ver \textit{selvagem}.}{sel.vá.ti.co}{0}
\verb{selvícola}{}{}{}{}{}{Var. de \textit{silvícola}.}{sel.ví.co.la}{0}
\verb{sem}{}{}{}{}{prep.}{Indica ausência, falta.}{sem}{0}
\verb{semáforo}{}{}{}{}{s.m.}{Aparelho de sinalização rodoviária ou ferroviária que orienta o tráfego por meio de luzes coloridas; sinal de trânsito,farol, sinaleiro, sinaleira.}{se.má.fo.ro}{0}
\verb{semana}{}{}{}{}{s.f.}{Período de sete dias, contado do domingo ao sábado.}{se.ma.na}{0}
\verb{semana}{}{}{}{}{}{Qualquer período de sete dias.}{se.ma.na}{0}
\verb{semanal}{}{}{"-ais}{}{adj.2g.}{Relativo a semana.}{se.ma.nal}{0}
\verb{semanal}{}{}{"-ais}{}{}{Que ocorre uma vez por semana.}{se.ma.nal}{0}
\verb{semanário}{}{}{}{}{adj.}{Ver \textit{semanal}.}{se.ma.ná.rio}{0}
\verb{semanário}{}{}{}{}{s.m.}{Publicação semanal.}{se.ma.ná.rio}{0}
\verb{semantema}{}{Gram.}{}{}{s.m.}{Parte do vocábulo que carrega o núcleo do sentido, o significado lexical, e não gramatical.}{se.man.te.ma}{0}
\verb{semântica}{}{}{}{}{s.f.}{Área da linguística que estuda as relações de significação e a representação do sentido.}{se.mân.ti.ca}{0}
\verb{semântico}{}{}{}{}{adj.}{Relativo ao significado.}{se.mân.ti.co}{0}
\verb{semântico}{}{Gram.}{}{}{}{Relativo à semântica.}{se.mân.ti.co}{0}
\verb{semasiologia}{}{}{}{}{s.f.}{Estudo da significação que parte das formas linguísticas em direção aos significados possíveis.}{se.ma.si.o.lo.gi.a}{0}
\verb{semblante}{}{}{}{}{s.m.}{Rosto, cara.}{sem.blan.te}{0}
\verb{semblante}{}{}{}{}{}{Aparência, aspecto exterior.}{sem.blan.te}{0}
\verb{sem"-cerimônia}{}{}{sem"-cerimônias}{}{s.f.}{Liberdade nos gestos, nas ações; naturalidade, informalidade.}{sem"-ce.ri.mô.nia}{0}
\verb{sem"-cerimônia}{}{}{sem"-cerimônias}{}{}{Desprezo pelos preceitos da etiqueta, das convenções sociais.}{sem"-ce.ri.mô.nia}{0}
\verb{sem"-cerimônia}{}{}{sem"-cerimônias}{}{}{Falta de educação, de refinamento; grosseria.}{sem"-ce.ri.mô.nia}{0}
\verb{sêmea}{}{}{}{}{s.f.}{A parte mais fina da farinha de trigo.}{sê.mea}{0}
\verb{semeador}{ô}{}{}{}{adj.}{Que semeia; sementeiro.}{se.me.a.dor}{0}
\verb{semeador}{ô}{Fig.}{}{}{}{Diz"-se daquele que divulga ideias, comportamentos.}{se.me.a.dor}{0}
\verb{semeadura}{}{}{}{}{s.f.}{Ato ou efeito de semear; semeação.}{se.me.a.du.ra}{0}
\verb{semeadura}{}{}{}{}{}{Porção de terreno semeado.}{se.me.a.du.ra}{0}
\verb{semeadura}{}{}{}{}{}{Quantidade de grãos suficiente para semear um determinado terreno.}{se.me.a.du.ra}{0}
\verb{semear}{}{}{}{}{v.t.}{Espalhar sementes para que germinem.}{se.me.ar}{0}
\verb{semear}{}{Fig.}{}{}{}{Divulgar, espalhar, disseminar, propalar.}{se.me.ar}{\verboinum{4}}
\verb{semelhança}{}{}{}{}{s.f.}{Qualidade de semelhante; conformidade física ou moral entre pessoas ou coisas.}{se.me.lhan.ça}{0}
\verb{semelhante}{}{}{}{}{adj.2g.}{Parecido, análogo.}{se.me.lhan.te}{0}
\verb{semelhante}{}{}{}{}{}{De mesma natureza, qualidade, função, forma.}{se.me.lhan.te}{0}
\verb{semelhante}{}{}{}{}{s.m.}{Pessoa ou objeto de mesma natureza. (\textit{Devemos dar atenção aos nossos semelhantes.})}{se.me.lhan.te}{0}
\verb{semelhar}{}{}{}{}{v.t.}{Ser ou tornar parecido.}{se.me.lhar}{0}
\verb{semelhar}{}{}{}{}{v.pron.}{Parecer"-se mutuamente. \textit{abon}}{se.me.lhar}{\verboinum{1}}
\verb{sêmen}{}{}{}{}{s.m.}{Ver \textit{esperma}.}{sê.men}{0}
\verb{semente}{}{Bot.}{}{}{s.f.}{Parte do fruto responsável pela reprodução vegetal.}{se.men.te}{0}
\verb{semente}{}{Por ext.}{}{}{}{Qualquer grão que pode ser semeado.}{se.men.te}{0}
\verb{semente}{}{Fig.}{}{}{}{Causa, origem, germe.}{se.men.te}{0}
\verb{sementeira}{ê}{}{}{}{s.f.}{Terra semeada.}{se.men.tei.ra}{0}
\verb{sementeira}{ê}{}{}{}{}{Canteiro para posterior transplante.}{se.men.tei.ra}{0}
\verb{semestral}{}{}{"-ais}{}{adj.2g.}{Relativo a semestre.}{se.mes.tral}{0}
\verb{semestral}{}{}{"-ais}{}{}{Que ocorre uma vez por semestre.}{se.mes.tral}{0}
\verb{semestral}{}{}{"-ais}{}{}{Que tem a duração de um semestre.}{se.mes.tral}{0}
\verb{semestralidade}{}{}{}{}{s.f.}{Qualidade de semestral.}{se.mes.tra.li.da.de}{0}
\verb{semestralidade}{}{}{}{}{}{Pagamento que se faz uma vez por semestre.}{se.mes.tra.li.da.de}{0}
\verb{semestre}{é}{}{}{}{s.m.}{Período de seis meses.}{se.mes.tre}{0}
\verb{sem"-fim}{}{}{sem"-fins}{}{s.m.}{Quantidade inúmera; número indeterminado.}{sem"-fim}{0}
\verb{sem"-fim}{}{}{sem"-fins}{}{}{Espaço indefinido, ilimitado; vastidão.}{sem"-fim}{0}
\verb{semianalfabeto}{é}{}{semianalfabetos ⟨é⟩}{}{adj.}{Que foi mal ou parcialmente alfabetizado.}{se.mi.a.nal.fa.be.to}{0}
\verb{semianalfabeto}{é}{Por ext.}{semianalfabetos ⟨é⟩}{}{}{Que demonstra conhecimento precário de determinado assunto.}{se.mi.a.nal.fa.be.to}{0}
\verb{semiaquático}{}{Zool.}{semiaquáticos}{}{adj.}{Diz"-se de animal que vive tanto em \textit{habitat} terrestre como em aquático.}{se.mi.a.quá.ti.co}{0}
\verb{semiárido}{}{}{semiáridos}{}{adj.}{Diz"-se do clima e outras características de zonas periféricas às regiões áridas.}{se.mi.á.ri.do}{0}
\verb{semibárbaro}{}{}{}{}{adj.}{Que é meio bárbaro; quase selvagem.}{se.mi.bár.ba.ro}{0}
\verb{semibreve}{é}{Mús.}{}{}{s.f.}{Elemento de ritmo da notação musical, com metade da duração da \textit{breve} (não usada atualmente) e o dobro da duração da \textit{mínima}.}{se.mi.bre.ve}{0}
\verb{semicircular}{}{}{}{}{adj.2g.}{Que tem forma de semicírculo.}{se.mi.cir.cu.lar}{0}
\verb{semicircular}{}{}{}{}{}{Relativo a semicírculo.}{se.mi.cir.cu.lar}{0}
\verb{semicírculo}{}{Geom.}{}{}{s.m.}{Metade de um círculo.}{se.mi.cír.cu.lo}{0}
\verb{semicircunferência}{}{Geom.}{}{}{s.f.}{Metade de uma circunferência.}{se.mi.cir.cun.fe.rên.cia}{0}
\verb{semicolcheia}{ê}{Mús.}{}{}{s.f.}{Elemento de ritmo da notação musical, com metade da duração da \textit{colcheia} e o dobro da duração da \textit{fusa}.}{se.mi.col.chei.a}{0}
\verb{semicondutor}{ô}{Fís.}{}{}{s.m.}{Elemento sólido não metálico, de fraca condução elétrica. }{se.mi.con.du.tor}{0}
\verb{semiconsciência}{}{}{}{}{s.f.}{Estado intermediário entre o estado de consciência e o de inconsciência.}{se.mi.cons.ci.ên.cia}{0}
\verb{semiconsciente}{}{}{}{}{adj.2g.}{Que está parcialmente consciente.}{se.mi.cons.ci.en.te}{0}
\verb{semicúpio}{}{}{}{}{s.m.}{Banho de imersão da parte inferior do corpo; banho de assento.}{se.mi.cú.pio}{0}
\verb{semicúpio}{}{}{}{}{}{Tipo de banheira ou bacia apropriada para esse tipo de banho.}{se.mi.cú.pio}{0}
\verb{semideus}{}{}{}{}{s.m.}{Na mitologia, um ser que participa da divindade, sendo filho de um ser divino com um mortal.}{se.mi.deus}{0}
\verb{semifinal}{}{Esport.}{"-ais}{}{s.f.}{Em uma competição, a prova ou partida que antecede imediatamente a final, definindo a classificação para esta.}{se.mi.fi.nal}{0}
\verb{semifinalista}{}{Esport.}{}{}{adj.2g.}{Diz"-se de pessoa ou equipe que se classificou para disputar a final.}{se.mi.fi.na.lis.ta}{0}
\verb{semifusa}{}{Mús.}{}{}{s.f.}{Elemento de ritmo da notação musical, com metade do valor do tempo de uma \textit{fusa}.}{se.mi.fu.sa}{0}
\verb{semi"-internato}{}{}{semi"-internatos}{}{s.m.}{Estado ou condição de semi"-interno.}{se.mi"-in.ter.na.to}{0}
\verb{semi"-internato}{}{}{semi"-internatos}{}{}{Escola em que os alunos vivem na condição de semi"-internos.}{se.mi"-in.ter.na.to}{0}
\verb{semi"-interno}{é}{}{semi"-internos ⟨é⟩}{}{adj.}{Diz"-se de aluno que permanece o dia todo no colégio, sem nele residir.}{se.mi"-in.ter.no}{0}
\verb{semilíquido}{}{}{}{}{adj.}{Que não é nem líquido, nem sólido; pastoso.}{se.mi.lí.qui.do}{0}
\verb{semimetal}{}{Quím.}{"-ais}{}{s.m.}{Elemento químico que possui propriedades intermediárias entre as dos metais e as dos não metais, como o silício, o boro etc.}{se.mi.me.tal}{0}
\verb{semimorto}{ô}{}{"-s ⟨ó⟩}{"-as ⟨ó⟩}{adj.}{Que está quase morto; desfalecido, exânime.}{se.mi.mor.to}{0}
\verb{semimorto}{ô}{Fig.}{"-s ⟨ó⟩}{"-as ⟨ó⟩}{}{Que se extenuou; fatigado, esgotado, enfraquecido.}{se.mi.mor.to}{0}
\verb{seminal}{}{}{"-ais}{}{adj.2g.}{Relativo a sêmen ou a semente.}{se.mi.nal}{0}
\verb{seminal}{}{Fig.}{"-ais}{}{}{Que produz novas ideias; criativo, inspirador.}{se.mi.nal}{0}
\verb{seminário}{}{}{}{}{s.m.}{Instituição onde se formam os que se dedicam à vida eclesiástica.}{se.mi.ná.rio}{0}
\verb{seminário}{}{}{}{}{}{Congresso científico ou cultural, com exposições e debates.}{se.mi.ná.rio}{0}
\verb{seminário}{}{}{}{}{}{Mesa"-redonda, slmpósio, colóquio.}{se.mi.ná.rio}{0}
\verb{seminário}{}{}{}{}{}{Aula em que um grupo de alunos expõe e debate temas específicos.}{se.mi.ná.rio}{0}
\verb{seminarista}{}{}{}{}{s.2g.}{Aluno que estuda como interno em um seminário para se dedicar à vida eclesiástica.}{se.mi.na.ris.ta}{0}
\verb{semínima}{}{Mús.}{}{}{s.f.}{Elemento de ritmo da notação musical, com metade do valor do tempo de uma \textit{mínima}. }{se.mí.ni.ma}{0}
\verb{seminu}{}{}{}{}{adj.}{Quase nu, meio nu.}{se.mi.nu}{0}
\verb{seminu}{}{}{}{}{}{Vestido com andrajos; maltrapilho, esfarrapado.}{se.mi.nu}{0}
\verb{semiologia}{}{}{}{}{s.f.}{Ciência que estuda qualquer sistema de signos, como imagens, vestuário, gestos, ritos etc.}{se.mi.o.lo.gi.a}{0}
\verb{semiologia}{}{Med.}{}{}{}{Parte da medicina que estuda os sinais e sintomas das doenças. }{se.mi.o.lo.gi.a}{0}
\verb{semiológico}{}{}{}{}{adj.}{Relativo a semiologia.}{se.mi.o.ló.gi.co}{0}
\verb{semiótica}{}{}{}{}{s.f.}{Teoria geral dos signos e símbolos, tanto nas línguas naturais quanto nas artificialmente criadas, principalmente quanto às suas funções e suas propriedades.}{se.mi.ó.ti.ca}{0}
\verb{semiplano}{}{Geom.}{}{}{s.m.}{Região ou parte de um plano limitado por uma reta.}{se.mi.pla.no}{0}
\verb{semiprecioso}{ô}{}{"-osos ⟨ó⟩}{"-osa ⟨ó⟩}{adj.}{Diz"-se de gema de valor comercial menor que uma gema preciosa como a ametista, a água"-marinha, o topázio etc.}{se.mi.pre.ci.o.so}{0}
\verb{semirreta}{é}{Geom.}{semirretas ⟨é⟩}{}{s.f.}{Cada uma das duas partes em que uma reta fica dividida por qualquer ponto tomado sobre ela.}{se.mir.re.ta}{0}
\verb{semirroto}{ô}{}{semirrotos ⟨ô⟩}{}{adj.}{Meio roto.}{se.mir.ro.to}{0}
\verb{semis"-selvagem}{}{}{semisselvagens}{}{adj.}{Pouco civilizado; rude.}{se.mis"-sel.va.gem}{0}
\verb{semita}{}{}{}{}{adj.}{Relativo aos semitas.}{se.mi.ta}{0}
\verb{semita}{}{}{}{}{}{Indivíduo pertencente a grupo étnico e linguístico oriental que compreende os árabes, hebreus, fenícios, aramaicos etc.}{se.mi.ta}{0}
\verb{semita}{}{Por ext.}{}{}{}{Judeu.}{se.mi.ta}{0}
\verb{semítico}{}{}{}{}{adj.}{Relativo aos semitas.}{se.mí.ti.co}{0}
\verb{semítico}{}{}{}{}{}{Relativo aos judeus.}{se.mí.ti.co}{0}
\verb{semítico}{}{}{}{}{}{Diz"-se de um grupo de línguas que se estende do norte da África até a Ásia médio"-oriental e compreende línguas como o árabe, o aramaico, o hebraico, o assírio etc.}{se.mí.ti.co}{0}
\verb{semitismo}{}{}{}{}{s.m.}{Conjunto das características próprias do que é semítico, como civilização e influência.}{se.mi.tis.mo}{0}
\verb{semitismo}{}{}{}{}{}{Qualidade ou caráter do que é judeu ou judaico.}{se.mi.tis.mo}{0}
\verb{semitom}{}{Mús.}{"-ons}{}{s.m.}{O menor intervalo usado na música ocidental, que equivale à metade de um tom; meio"-tom.}{se.mi.tom}{0}
\verb{semitransparente}{}{}{}{}{adj.2g.}{Que é um pouco transparente.}{se.mi.trans.pa.ren.te}{0}
\verb{semivivo}{}{}{}{}{adj.}{Quase sem vida; semimorto.}{se.mi.vi.vo}{0}
\verb{semivogal}{}{Gram.}{"-ais}{}{s.f.}{Segmento sonoro que apresenta características tanto vocálicas quanto consonantais e que, ao juntar"-se a uma vogal, forma um ditongo.}{se.mi.vo.gal}{0}
\verb{sem"-modos}{ó}{}{}{}{s.m.}{Indivíduo mal"-educado.}{sem"-mo.dos}{0}
\verb{sem"-número}{}{}{}{}{s.m.}{Grande número, número indeterminado.}{sem"-nú.me.ro}{0}
\verb{sêmola}{}{}{}{}{s.f.}{Farinha granulada, seca e peneirada, extraída do grão de trigo ou de outros cereais e que é utilizada no preparo de massas, sopas, mingaus etc.}{sê.mo.la}{0}
\verb{sêmola}{}{}{}{}{}{Semolina.}{sê.mo.la}{0}
\verb{semolina}{}{}{}{}{s.f.}{Fécula de farinha de trigo ou de arroz, usada para engrossar caldos, pudins etc.; sêmola.}{se.mo.li.na}{0}
\verb{semovente}{}{}{}{}{adj.2g.}{Que se move ou anda por si próprio.}{se.mo.ven.te}{0}
\verb{sem"-par}{}{}{}{}{adj.2g.}{Que é único, sem igual; singular.}{sem"-par}{0}
\verb{sempiterno}{é}{}{}{}{adj.}{Que não teve princípio nem terá fim; que vive sempre; eterno, infinito.}{sem.pi.ter.no}{0}
\verb{sempiterno}{é}{}{}{}{}{Que é muito velho, muito antigo.}{sem.pi.ter.no}{0}
\verb{sempre}{}{}{}{}{}{Em todo o tempo; em qualquer ocasião.}{sem.pre}{0}
\verb{sempre}{}{}{}{}{adv.}{Por todo o tempo; continuamente, constantemente.}{sem.pre}{0}
\verb{sempre}{}{}{}{}{}{Em todo caso; em toda vez.}{sem.pre}{0}
\verb{sempre}{}{}{}{}{s.m.}{Todo o tempo passado ou futuro; a eternidade.}{sem.pre}{0}
\verb{sempre"-viva}{}{Bot.}{sempre"-vivas}{}{s.f.}{Erva largamente cultivada como ornamental devido às propriedades de suas inflorescências que, mesmo secas, não murcham nem perdem a cor.}{sem.pre"-vi.va}{0}
\verb{sem"-sal}{}{}{}{}{adj.2g.}{Insosso.}{sem"-sal}{0}
\verb{sem"-sal}{}{Pop.}{}{}{}{Diz"-se de pessoa sem graça, sem brilho.}{sem"-sal}{0}
\verb{sem"-terra}{é}{Bras.}{}{}{s.2g.}{Trabalhador rural que não tem posse legal da terra em que trabalha e por isso serve de mão de obra agrícola.}{sem"-ter.ra}{0}
\verb{sem"-teto}{é}{}{}{}{s.2g.}{Indivíduo que, por falta de recursos, vive na rua ou em abrigos improvisados, geralmente em cidades.}{sem"-te.to}{0}
\verb{sem"-vergonha}{}{}{}{}{adj.}{Diz"-se de pessoa sem pudor.}{sem"-ver.go.nha}{0}
\verb{sem"-vergonhice}{}{}{}{}{s.f.}{Qualidade, ato ou dito de indivíduo sem"-vergonha.}{sem"-ver.go.nhi.ce}{0}
\verb{sem"-vergonhice}{}{}{}{}{}{Falta de pudor.}{sem"-ver.go.nhi.ce}{0}
\verb{sem"-vergonhismo}{}{}{}{}{s.m.}{Ver \textit{sem"-vergonhice}.}{sem"-ver.go.nhis.mo}{0}
\verb{sena}{}{}{}{}{s.f.}{Carta de jogo ou pedra de dominó marcada com seis pintas ou seis pontos.}{se.na}{0}
\verb{sena}{}{}{}{}{}{Jogo de loteria com prêmio em dinheiro para quem acertar os seis números sorteados.}{se.na}{0}
\verb{senado}{}{}{}{}{s.m.}{Corpo de legisladores que constitui uma das duas assembleias parlamentares de alguns regimes políticos.}{se.na.do}{0}
\verb{senado}{}{}{}{}{}{Local onde se reúnem esses legisladores.}{se.na.do}{0}
\verb{senado}{}{Hist.}{}{}{}{Na antiga Roma, assembleia de patrícios que constituía o Conselho Supremo do governo.}{se.na.do}{0}
\verb{senador}{ô}{}{}{}{s.m.}{Indivíduo que exerce as funções legislativas em um senado.}{se.na.dor}{0}
\verb{senão}{}{}{\textit{do s.m.:} -ões}{}{conj.}{Mas sim; mas, porém.}{se.não}{0}
\verb{senão}{}{}{\textit{do s.m.:} -ões}{}{}{De outro modo; do contrário; aliás.}{se.não}{0}
\verb{senão}{}{}{\textit{do s.m.:} -ões}{}{prep.}{A não ser; com exceção de; salvo. }{se.não}{0}
\verb{senão}{}{}{\textit{do s.m.:} -ões}{}{s.m.}{Falha, defeito, imperfeição.}{se.não}{0}
\verb{senário}{}{}{}{}{s.m.}{Que é formado de seis unidades.}{se.ná.rio}{0}
\verb{senatoria}{}{}{}{}{s.f.}{Período de exercício ou mandato de senador.}{se.na.to.ri.a}{0}
\verb{senatória}{}{}{}{}{}{Var. de \textit{senatoria}.}{se.na.tó.ria}{0}
\verb{senatorial}{}{}{"-ais}{}{adj.2g.}{Relativo ao senado ou aos senadores; senatório.}{se.na.to.ri.al}{0}
\verb{senatório}{}{}{}{}{adj.}{Ver \textit{senatorial}.}{se.na.tó.rio}{0}
\verb{senda}{}{}{}{}{s.f.}{Caminho estreito; atalho, vereda.}{sen.da}{0}
\verb{senda}{}{}{}{}{}{Aquilo que se faz habitualmente; rotina, praxe.}{sen.da}{0}
\verb{sendeiro}{ê}{}{}{}{adj.}{Diz"-se de cavalo velho e ruim.}{sen.dei.ro}{0}
\verb{senectude}{}{}{}{}{s.f.}{Idade senil; velhice, senilidade. }{se.nec.tu.de}{0}
\verb{senegalês}{}{}{}{}{adj.}{Relativo ao Senegal (África); senegalesco.}{se.ne.ga.lês}{0}
\verb{senegalês}{}{}{}{}{s.m.}{Indivíduo natural ou habitante desse país.}{se.ne.ga.lês}{0}
\verb{senegalesco}{ê}{}{}{}{adj.}{Ver \textit{senegalês}.}{se.ne.ga.les.co}{0}
\verb{senha}{}{}{}{}{s.f.}{Gesto combinado entre pessoas; sinal, indício.}{se.nha}{0}
\verb{senha}{}{}{}{}{}{Palavra ou ação secreta convencionada entre pessoas para ser usada como sinal de reconhecimento.}{se.nha}{0}
\verb{senha}{}{}{}{}{}{Papel ou bilhete que possibilita a entrada ou o retorno de alguém a reuniões, assembleias, espetáculos etc.}{se.nha}{0}
\verb{senha}{}{Informát.}{}{}{}{Código secreto que serve para a identificação do usuário permitindo o acesso a dados, programas ou sistemas  não disponíveis ao público.  }{se.nha}{0}
\verb{senhor}{ô}{}{}{}{s.m.}{Forma de tratamento respeitoso dispensado aos homens.}{se.nhor}{0}
\verb{senhor}{ô}{}{}{}{}{Homem de meia"-idade ou idoso.}{se.nhor}{0}
\verb{senhor}{ô}{}{}{}{}{Indivíduo nobre, distinto, importante.}{se.nhor}{0}
\verb{senhor}{ô}{}{}{}{}{Dono da casa; amo, patrão.}{se.nhor}{0}
\verb{senhor}{ô}{Relig.}{}{}{}{O Ser Supremo; Deus; Jesus Cristo. (Nesta acepção, com inicial maiúscula.)}{se.nhor}{0}
\verb{senhora}{ó}{}{}{}{s.f.}{Feminino de \textit{senhor}.}{se.nho.ra}{0}
\verb{senhora}{ó}{}{}{}{}{Forma de tratamento respeitosa dispensada às mulheres casadas ou às não muito jovens.}{se.nho.ra}{0}
\verb{senhora}{ó}{}{}{}{}{Esposa, mulher.}{se.nho.ra}{0}
\verb{senhorear}{}{}{}{}{v.t.}{Tornar"-se senhor; tomar posse; assenhorear.}{se.nho.re.ar}{0}
\verb{senhorear}{}{}{}{}{}{Ter influência moral; conquistar, cativar.}{se.nho.re.ar}{0}
\verb{senhorear}{}{}{}{}{}{Refrear, reprimir, conter.}{se.nho.re.ar}{\verboinum{4}}
\verb{senhoria}{}{}{}{}{s.f.}{Autoridade ou condição de senhor ou de senhora.}{se.nho.ri.a}{0}
\verb{senhoria}{}{}{}{}{}{Proprietária de bens imóveis; senhora.}{se.nho.ri.a}{0}
\verb{senhorial}{}{}{"-ais}{}{adj.2g.}{Relativo ao senhor ou ao senhorio.}{se.nho.ri.al}{0}
\verb{senhorial}{}{}{"-ais}{}{}{Requintado, distinto, elegante.}{se.nho.ri.al}{0}
\verb{senhoril}{}{}{"-is}{}{adj.2g.}{Próprio de senhor, de nobre.}{se.nho.ril}{0}
\verb{senhoril}{}{}{"-is}{}{}{Que revela modos distintos, majestosos, graves.}{se.nho.ril}{0}
\verb{senhorinha}{}{}{}{}{s.f.}{Ver \textit{senhorita}.}{se.nho.ri.nha}{0}
\verb{senhorio}{}{}{}{}{s.m.}{Direito de senhor sobre algo; mando, autoridade.}{se.nho.ri.o}{0}
\verb{senhorio}{}{}{}{}{}{Posse, domínio, propriedade.}{se.nho.ri.o}{0}
\verb{senhorio}{}{}{}{}{}{Proprietário de bens imóveis; senhor.}{se.nho.ri.o}{0}
\verb{senhorita}{}{}{}{}{s.f.}{Moça solteira.}{se.nho.ri.ta}{0}
\verb{senhorita}{}{}{}{}{}{Forma de tratamento respeitosa dispensada às moças solteiras; senhorinha.}{se.nho.ri.ta}{0}
\verb{senil}{}{}{"-is}{}{adj.2g.}{Relativo à velhice ou aos velhos.}{se.nil}{0}
\verb{senil}{}{}{"-is}{}{}{Que resulta ou provém da velhice.}{se.nil}{0}
\verb{senilidade}{}{}{}{}{s.f.}{Condição ou estado de senil.}{se.ni.li.da.de}{0}
\verb{senilidade}{}{}{}{}{}{Debilidade física e mental causada pela velhice.}{se.ni.li.da.de}{0}
\verb{sênior}{}{}{}{}{adj.2g.}{Que é o mais velho relativamente a outro.}{sê.nior}{0}
\verb{sênior}{}{Esport.}{}{}{s.m.}{Desportista que já obteve vários prêmios. }{sê.nior}{0}
\verb{sênior}{}{Esport.}{}{}{}{Desportista que já passou da idade de júnior, mas ainda não é veterano.}{sê.nior}{0}
\verb{seno}{}{Mat.}{}{}{s.m.}{Razão entre o cateto oposto a um ângulo de um triângulo retângulo e a hipotenusa.}{se.no}{0}
\verb{senoidal}{}{}{"-ais}{}{adj.2g.}{Relativo a seno ou a senoide.}{se.noi.dal}{0}
\verb{senoide}{ó}{Geom.}{}{}{s.f.}{Em um sistema cartesiano, curva que representa a função seno.}{se.noi.de}{0}
\verb{sensabor}{ô}{}{}{}{adj.2g.}{Que não tem gosto ou sabor; insípido.}{sen.sa.bor}{0}
\verb{sensabor}{ô}{}{}{}{}{Que não desperta interesse; sem graça, desinteressante.}{sen.sa.bor}{0}
\verb{sensaborão}{}{}{"-ões}{}{adj.}{Que é muito sem graça; desinteressante.}{sen.sa.bo.rão}{0}
\verb{sensaboria}{}{}{}{}{s.f.}{Condição ou estado do que é sensabor; insipidez, monotonia.}{sen.sa.bo.ri.a}{0}
\verb{sensaboria}{}{}{}{}{}{Incidente desagradável; aborrecimento, contratempo.}{sen.sa.bo.ri.a}{0}
\verb{sensação}{}{}{"-ões}{}{s.f.}{Percepção causada por uma reação específica a um estímulo externo ou interno enviada ao sistema nervoso central. }{sen.sa.ção}{0}
\verb{sensação}{}{}{"-ões}{}{}{Qualquer impressão física percebida pelos sentidos.}{sen.sa.ção}{0}
\verb{sensação}{}{}{"-ões}{}{}{Surpresa, espanto ou grande impressão ante a um acontecimento raro.}{sen.sa.ção}{0}
\verb{sensação}{}{}{"-ões}{}{}{Comoção moral, emoção forte.}{sen.sa.ção}{0}
\verb{sensacional}{}{}{"-ais}{}{adj.2g.}{Que produz sensação intensa, grande emoção.}{sen.sa.ci.o.nal}{0}
\verb{sensacional}{}{}{"-ais}{}{}{Surpreendente, espetacular, formidável.}{sen.sa.ci.o.nal}{0}
\verb{sensacionalismo}{}{}{}{}{s.m.}{Divulgação e exploração de matérias que provocam grande comoção ou escândalo.}{sen.sa.ci.o.na.lis.mo}{0}
\verb{sensacionalismo}{}{}{}{}{}{Prática de atitudes chocantes, escândalos ou hábitos exóticos.}{sen.sa.ci.o.na.lis.mo}{0}
\verb{sensacionalista}{}{}{}{}{adj.2g.}{Que faz uso de sensacionalismo; escandaloso, chocante.}{sen.sa.ci.o.na.lis.ta}{0}
\verb{sensatez}{ê}{}{}{}{s.f.}{Qualidade de sensato; bom"-senso, juízo, equilíbrio.}{sen.sa.tez}{0}
\verb{sensatez}{ê}{}{}{}{}{Prudência, cautela, discrição.}{sen.sa.tez}{0}
\verb{sensato}{}{}{}{}{adj.}{Que revela bom"-senso; ajuizado, equilibrado.}{sen.sa.to}{0}
\verb{sensato}{}{}{}{}{}{Que demonstra cautela; prudente, discreto.}{sen.sa.to}{0}
\verb{sensibilidade}{}{}{}{}{s.f.}{Faculdade de um organismo de receber estímulos.}{sen.si.bi.li.da.de}{0}
\verb{sensibilidade}{}{}{}{}{}{Tendência do ser humano de captar e transmitir impressões capazes de causar emoção.}{sen.si.bi.li.da.de}{0}
\verb{sensibilidade}{}{}{}{}{}{Delicadeza de sentimentos; simpatia, solidariedade.}{sen.si.bi.li.da.de}{0}
\verb{sensibilizador}{ô}{}{}{}{adj.}{Que sensibiliza, que torna sensível a um agente qualquer; sensibilizante.}{sen.si.bi.li.za.dor}{0}
\verb{sensibilizante}{}{}{}{}{adj.2g.}{Sensibilizador.}{sen.si.bi.li.zan.te}{0}
\verb{sensibilizar}{}{}{}{}{v.t.}{Tornar sensível; comover, entristecer.}{sen.si.bi.li.zar}{0}
\verb{sensibilizar}{}{}{}{}{}{Abrandar o coração; enternecer, emocionar.}{sen.si.bi.li.zar}{0}
\verb{sensibilizar}{}{}{}{}{}{Tornar sensível a ação de qualquer agente.}{sen.si.bi.li.zar}{\verboinum{1}}
\verb{sensitiva}{}{Bot.}{}{}{s.f.}{Planta da família das leguminosas, cujas folhas têm a propriedade de se retraírem quando tocadas; dormideira.}{sen.si.ti.va}{0}
\verb{sensitivo}{}{}{}{}{adj.}{Relativo aos sentidos ou às sensações.}{sen.si.ti.vo}{0}
\verb{sensitivo}{}{}{}{}{}{Que tem a faculdade de sentir.}{sen.si.ti.vo}{0}
\verb{sensitivo}{}{}{}{}{}{Que é muito suscetível a emoções.}{sen.si.ti.vo}{0}
\verb{sensitivo}{}{}{}{}{}{Diz"-se de pessoa dotada de percepção além dos sentidos.}{sen.si.ti.vo}{0}
\verb{sensível}{}{}{"-eis}{}{adj.2g.}{Que pode ser percebido pelos sentidos; perceptível.}{sen.sí.vel}{0}
\verb{sensível}{}{}{"-eis}{}{}{Dotado de sensibilidade; que tem sentidos.}{sen.sí.vel}{0}
\verb{sensível}{}{}{"-eis}{}{}{Que se deixa comover com facilidade; impressionável.}{sen.sí.vel}{0}
\verb{sensível}{}{}{"-eis}{}{}{Que se faz perceber claramente; evidente, visível.}{sen.sí.vel}{0}
\verb{senso}{}{}{}{}{s.m.}{Faculdade de raciocinar; juízo, entendimento.}{sen.so}{0}
\verb{senso}{}{}{}{}{}{Prudência, cautela, circunspecção.}{sen.so}{0}
\verb{senso}{}{}{}{}{}{Sensatez, equilíbrio.}{sen.so}{0}
\verb{sensor}{ô}{}{}{}{s.m.}{Nome dado a qualquer dispositivo que capta estímulos físicos como calor, luz, som e os transforma em impulsos elétricos mensuráveis.}{sen.sor}{0}
\verb{sensorial}{}{}{"-ais}{}{adj.2g.}{Relativo aos órgãos dos sentidos.}{sen.so.ri.al}{0}
\verb{sensorial}{}{}{"-ais}{}{}{Sensível, palpável.}{sen.so.ri.al}{0}
\verb{sensório}{}{}{}{}{adj.}{Relativo à sensibilidade.}{sen.só.rio}{0}
\verb{sensório}{}{}{}{}{}{Que transmite sensações.}{sen.só.rio}{0}
\verb{sensual}{}{}{"-ais}{}{adj.2g.}{Relativo aos sentidos ou aos órgãos do sentido.}{sen.su.al}{0}
\verb{sensual}{}{}{"-ais}{}{}{Que satisfaz os sentidos; voluptuoso, lascivo, lúbrico.}{sen.su.al}{0}
\verb{sensualidade}{}{}{}{}{s.f.}{Qualidade ou caráter do que é sensual; volúpia, luxúria. }{sen.su.a.li.da.de}{0}
\verb{sensualidade}{}{}{}{}{}{Inclinação exagerada aos prazeres sexuais; voluptuosidade.}{sen.su.a.li.da.de}{0}
\verb{sensualizar}{}{}{}{}{v.t.}{Tornar sensual; excitar aos prazeres sexuais.}{sen.su.a.li.zar}{\verboinum{1}}
\verb{sentar}{}{}{}{}{v.t.}{Flexionar as pernas até apoiar as nádegas em assento; assentar.}{sen.tar}{0}
\verb{sentar}{}{}{}{}{}{Fazer tomar assento.}{sen.tar}{0}
\verb{sentar}{}{}{}{}{v.pron.}{Estabelecer"-se, fixar"-se, colocar"-se.}{sen.tar}{\verboinum{1}}
\verb{sentença}{}{}{}{}{s.f.}{Expressão que encerra um valor moral ou uma ordem geral; máxima, provérbio.}{sen.ten.ça}{0}
\verb{sentença}{}{Jur.}{}{}{}{Decisão ou resolução do juiz que põe termo a um processo.}{sen.ten.ça}{0}
\verb{sentença}{}{Gram.}{}{}{}{Construção que encerra um sentido completo; oração.}{sen.ten.ça}{0}
\verb{sentenciado}{}{}{}{}{adj.}{Que recebeu uma sentença judicial.}{sen.ten.ci.a.do}{0}
\verb{sentenciar}{}{}{}{}{v.t.}{Julgar, condenar por meio de sentença.}{sen.ten.ci.ar}{0}
\verb{sentenciar}{}{}{}{}{}{Proferir ou emitir sentença; dar opinião ou voto.}{sen.ten.ci.ar}{\verboinum{1}}
\verb{sentencioso}{ô}{}{"-osos ⟨ó⟩}{"-osa ⟨ó⟩}{adj.}{Que tem forma ou caráter de sentença.}{sen.ten.ci.o.so}{0}
\verb{sentencioso}{ô}{}{"-osos ⟨ó⟩}{"-osa ⟨ó⟩}{}{Que encerra sentença, conceito.}{sen.ten.ci.o.so}{0}
\verb{sentido}{}{}{}{}{s.m.}{Cada uma das formas de perceber o que acontece fora do corpo. (\textit{A visão é um dos sentidos do corpo humano.})}{sen.ti.do}{0}
\verb{sentido}{}{}{}{}{}{Cada uma das duas direções de uma reta. (\textit{Pela manhã, o trânsito das ruas é sempre no sentido bairro"-centro.})}{sen.ti.do}{0}
\verb{sentido}{}{}{}{}{}{O Conjunto de informações que uma palavra ou texto apresenta; significado. (\textit{Não consegui entender bem o sentido do texto que eu li.})}{sen.ti.do}{0}
\verb{sentido}{}{}{}{}{}{Conjunto de ideias e valores que um fato apresenta. (\textit{As festas de fim de ano sempre têm um sentido especial para todos nós.})}{sen.ti.do}{0}
\verb{sentido}{}{}{}{}{}{Motivo que justifica alguma ação; cabimento, lógica, razão de ser. (\textit{Não tem sentido ir todos os dias à escola e não estudar.})}{sen.ti.do}{0}
\verb{sentido}{}{}{}{}{}{Reconhecimento do que se deve fazer; noção, senso. (\textit{Ele tem muito sentido de suas responsabilidades.})}{sen.ti.do}{0}
\verb{sentido}{}{}{}{}{adj.}{Que se sentiu. (\textit{Foi um tremor de terra sentido a quilômetros de distância.})}{sen.ti.do}{0}
\verb{sentido}{}{}{}{}{}{Cheio de tristeza e sofrimento; ressentido. (\textit{Ele ficou muito sentido com a morte do amigo.})}{sen.ti.do}{0}
\verb{sentimental}{}{}{"-ais}{}{adj.2g.}{Relativo a sentimento; afetivo.}{sen.ti.men.tal}{0}
\verb{sentimental}{}{}{"-ais}{}{}{Que é capaz de manifestar sentimentos; compassivo.}{sen.ti.men.tal}{0}
\verb{sentimental}{}{}{"-ais}{}{}{Que revela sensibilidade excessiva, romanesca.}{sen.ti.men.tal}{0}
\verb{sentimentalismo}{}{}{}{}{s.m.}{Tendência à exaltação do sentimento ou da sensibilidade emocional.}{sen.ti.men.ta.lis.mo}{0}
\verb{sentimentalista}{}{}{}{}{adj.2g.}{Relativo ao sentimentalismo.}{sen.ti.men.ta.lis.ta}{0}
\verb{sentimentalista}{}{}{}{}{}{Diz"-se do indivíduo que coloca os sentimentos acima da razão.}{sen.ti.men.ta.lis.ta}{0}
\verb{sentimento}{}{}{}{}{s.m.}{Aptidão ou capacidade de sentir, de receber impressões mentais.}{sen.ti.men.to}{0}
\verb{sentimento}{}{}{}{}{}{Sensação psíquica de desprazer; pesar, angústia, desgosto.}{sen.ti.men.to}{0}
\verb{sentimento}{}{}{}{}{}{Emoção terna; afeto, amor, amizade.}{sen.ti.men.to}{0}
\verb{sentina}{}{}{}{}{s.f.}{Vaso sanitário; latrina.}{sen.ti.na}{0}
\verb{sentina}{}{Fig.}{}{}{}{Lugar muito sujo, imundo.}{sen.ti.na}{0}
\verb{sentina}{}{Fig.}{}{}{}{Pessoa cheia de vícios.}{sen.ti.na}{0}
\verb{sentinela}{é}{}{}{}{s.m.}{Soldado armado que guarda um posto.}{sen.ti.ne.la}{0}
\verb{sentinela}{é}{}{}{}{}{Pessoa que vigia algo ou vela por algo.}{sen.ti.ne.la}{0}
\verb{sentir}{}{}{}{}{v.t.}{Tomar conhecimento de alguma coisa por meio do contato. (\textit{Somente ao se encostarem, sentiu que havia alguém por perto.})}{sen.tir}{0}
\verb{sentir}{}{}{}{}{}{Notar que tem determinado estado físico ou mental. (\textit{Ao sentir medo, a criança chorou.})}{sen.tir}{0}
\verb{sentir}{}{}{}{}{}{Notar algum fato sem poder explicá"-lo. (\textit{Eu logo senti que havia alguma coisa errada.})}{sen.tir}{0}
\verb{sentir}{}{}{}{}{}{Ficar triste com alguma situação. (\textit{Eu senti muito quando ele partiu.})}{sen.tir}{0}
\verb{sentir}{}{}{}{}{v.pron.}{Achar"-se em determinado estado. (\textit{Ele se sentia bem após a cirurgia.})}{sen.tir}{\verboinum{29}}
\verb{senzala}{}{}{}{}{s.f.}{Grupo de habitações que se destinavam à moradia dos escravos dos engenhos ou das fazendas.}{sen.za.la}{0}
\verb{sépala}{}{Bot.}{}{}{s.f.}{Cada uma das partes, geralmente verdes, que constituem o cálice da flor.}{sé.pa.la}{0}
\verb{separação}{}{}{"-ões}{}{s.f.}{Ato ou efeito de separar; partição, divisão.}{se.pa.ra.ção}{0}
\verb{separação}{}{}{"-ões}{}{}{Aquilo que separa ou veda, como muro, parede, cerca etc.}{se.pa.ra.ção}{0}
\verb{separação}{}{}{"-ões}{}{}{Ruptura do casamento.}{se.pa.ra.ção}{0}
\verb{separado}{}{}{}{}{adj.}{Que foi posto de lado; afastado, apartado.}{se.pa.ra.do}{0}
\verb{separado}{}{}{}{}{}{Diz"-se daquele que desfez uma união matrimonial.}{se.pa.ra.do}{0}
\verb{separador}{ô}{}{}{}{adj.}{Que separa; divisor.}{se.pa.ra.dor}{0}
\verb{separador}{ô}{}{}{}{s.m.}{Espécie de desnatadeira que serve para separar líquidos de densidade diferente.}{se.pa.ra.dor}{0}
\verb{separar}{}{}{}{}{v.t.}{Pôr de lado; afastar, apartar.}{se.pa.rar}{0}
\verb{separar}{}{}{}{}{}{Repartir por meio de divisória; dividir, isolar.}{se.pa.rar}{0}
\verb{separar}{}{}{}{}{}{Distinguir, diferenciar, classificar.}{se.pa.rar}{0}
\verb{separar}{}{}{}{}{}{Lançar a discórdia; desunir.}{se.pa.rar}{0}
\verb{separar}{}{}{}{}{v.pron.}{Romper o casamento; divorciar"-se.}{se.pa.rar}{\verboinum{1}}
\verb{separata}{}{}{}{}{s.f.}{Publicação, em volume ou livreto, de artigos já editados em jornal ou revista, na qual se mantém a mesma composição tipográfica.  }{se.pa.ra.ta}{0}
\verb{separatismo}{}{}{}{}{s.m.}{Tendência dos habitantes de um território ou região de separar"-se do Estado a que pertencem com o intuito de constituírem um Estado independente.}{se.pa.ra.tis.mo}{0}
\verb{separatista}{}{}{}{}{adj.2g.}{Relativo ao separatismo.}{se.pa.ra.tis.ta}{0}
\verb{separatista}{}{}{}{}{}{Que é favorável ao separatismo.}{se.pa.ra.tis.ta}{0}
\verb{separável}{}{}{"-eis}{}{adj.2g.}{Que se pode separar, apartar, desunir.}{se.pa.rá.vel}{0}
\verb{sépia}{}{}{}{}{s.f.}{Tinta castanho"-escura extraída do molusco desse nome, também conhecido por siba.}{sé.pia}{0}
\verb{sépia}{}{}{}{}{}{Pintura ou desenho feito com essa cor.}{sé.pia}{0}
\verb{sépia}{}{}{}{}{}{Ver \textit{siba}.}{sé.pia}{0}
\verb{sépia}{}{}{}{}{adj.}{Que tem a cor dessa substância.}{sé.pia}{0}
\verb{sepsia}{}{Med.}{}{}{s.f.}{Intoxicação causada pela presença de substâncias putrefatas na corrente sanguínea ou nos tecidos; infecção.}{sep.si.a}{0}
\verb{septicemia}{}{Med.}{}{}{s.f.}{Processo infeccioso generalizado em que os micro"-organismos infectores se instalam e se disseminam através da corrente sanguínea.}{sep.ti.ce.mi.a}{0}
\verb{septicêmico}{}{Med.}{}{}{adj.}{Relativo a septicemia.}{sep.ti.cê.mi.co}{0}
\verb{septicêmico}{}{Med.}{}{}{}{Que apresenta septicemia.}{sep.ti.cê.mi.co}{0}
\verb{séptico}{}{}{}{}{adj.}{Que causa infecção.}{sép.ti.co}{0}
\verb{séptico}{}{}{}{}{}{Que contém germes patogênicos.}{sép.ti.co}{0}
\verb{septo}{é}{Anat.}{}{}{s.m.}{Estrutura que divide cavidades, órgãos ou tecidos.}{sep.to}{0}
\verb{septuagenário}{}{}{}{}{adj.}{Diz"-se de indivíduo que está na faixa dos setenta anos de idade.}{sep.tu.a.ge.ná.rio}{0}
\verb{septuagenário}{}{}{}{}{s.m.}{Indivíduo que está nessa faixa de idade.  }{sep.tu.a.ge.ná.rio}{0}
\verb{septuagésimo}{}{}{}{}{num.}{Ordinal e fracionário correspondente a 70.}{sep.tu.a.gé.si.mo}{0}
\verb{sepulcral}{}{}{"-ais}{}{adj.2g.}{Relativo a sepulcro.}{se.pul.cral}{0}
\verb{sepulcral}{}{}{"-ais}{}{}{Relativo a morte.}{se.pul.cral}{0}
\verb{sepulcral}{}{Fig.}{"-ais}{}{}{Que evoca o sepulcro ou a morte; sombrio, medonho.}{se.pul.cral}{0}
\verb{sepulcro}{}{}{}{}{s.m.}{Sepultura, túmulo, monumento fúnebre.}{se.pul.cro}{0}
\verb{sepulcro}{}{}{}{}{}{Cavidade localizada no altar de uma igreja, onde se guardam relíquias de santos.}{se.pul.cro}{0}
\verb{sepultamento}{}{}{}{}{s.m.}{Ato de sepultar.}{se.pul.ta.men.to}{0}
\verb{sepultar}{}{}{}{}{v.t.}{Enterrar, inumar.}{se.pul.tar}{0}
\verb{sepultar}{}{}{}{}{}{Soterrar.}{se.pul.tar}{0}
\verb{sepultar}{}{Fig.}{}{}{}{Afastar"-se do convívio social; recolher"-se.}{se.pul.tar}{\verboinum{1}}
\verb{sepulto}{}{}{}{}{adj.}{Sepultado, enterrado.}{se.pul.to}{0}
\verb{sepulto}{}{Fig.}{}{}{}{Acabado, extinto.}{se.pul.to}{0}
\verb{sepultura}{}{}{}{}{s.f.}{Cova em que se enterram os cadáveres.}{se.pul.tu.ra}{0}
\verb{sequaz}{}{}{}{}{adj.}{Que segue ou professa as ideias de um filósofo ou crenças de uma religião.}{se.quaz}{0}
\verb{sequaz}{}{}{}{}{}{Que faz parte de um partido, seita, agremiação.}{se.quaz}{0}
\verb{sequaz}{}{}{}{}{s.m.}{Comparsa, capanga.}{se.quaz}{0}
\verb{sequela}{é}{}{}{}{s.f.}{Sequência ou consequência.}{se.que.la}{0}
\verb{sequela}{é}{}{}{}{}{Bando, súcia.}{se.que.la}{0}
\verb{sequela}{é}{Med.}{}{}{}{Lesão anatômica ou funcional que permanece após curada uma doença.}{se.que.la}{0}
\verb{sequência}{}{}{}{}{s.f.}{Ato ou efeito de seguir.}{se.quên.cia}{0}
\verb{sequência}{}{}{}{}{}{Série, sucessão de quaisquer elementos.}{se.quên.cia}{0}
\verb{sequência}{}{}{}{}{}{Continuação, seguimento.}{se.quên.cia}{0}
\verb{sequência}{}{}{}{}{}{Em cinema ou televisão, o conjunto de cenas que formam uma unidade da ação.}{se.quên.cia}{0}
\verb{sequencial}{}{}{"-ais}{}{adj.2g.}{Em que há sequência.}{se.quen.ci.al}{0}
\verb{sequente}{}{}{}{}{adj.2g.}{Que vem logo depois; seguinte, sucessor.}{se.quen.te}{0}
\verb{sequer}{é}{}{}{}{adv.}{Pelo menos; ao menos. \textit{abon}}{se.quer}{0}
\verb{sequer}{é}{}{}{}{}{Nem mesmo. \textit{abon}}{se.quer}{0}
\verb{sequestração}{}{}{"-ões}{}{s.f.}{Ato ou efeito de sequestrar; sequestro.}{se.ques.tra.ção}{0}
\verb{sequestrador}{ô}{}{}{}{s.m.}{Indivíduo que realiza um sequestro.}{se.ques.tra.dor}{0}
\verb{sequestrar}{}{}{}{}{v.t.}{Tomar e manter algo à força.}{se.ques.trar}{0}
\verb{sequestrar}{}{Jur.}{}{}{}{Apreender por decisão judicial.}{se.ques.trar}{0}
\verb{sequestrar}{}{}{}{}{}{Pôr à parte; isolar.}{se.ques.trar}{0}
\verb{sequestrar}{}{}{}{}{}{Desviar (geralmente um meio de transporte) da rota mediante violência.}{se.ques.trar}{\verboinum{1}}
\verb{sequestro}{é}{Jur.}{}{}{s.m.}{Apreensão de um bem por decisão judicial.}{se.ques.tro}{0}
\verb{sequestro}{é}{}{}{}{}{Ato de privar alguém da liberdade mediante ameaça ou cativeiro.}{se.ques.tro}{0}
\verb{sequestro}{é}{}{}{}{}{O objeto sequestrado.}{se.ques.tro}{0}
\verb{sequestro}{é}{}{}{}{}{Ato ou efeito de sequestrar.}{se.ques.tro}{0}
\verb{sequidão}{}{}{"-ões}{}{s.f.}{Secura.}{se.qui.dão}{0}
\verb{sequidão}{}{}{"-ões}{}{}{Frieza, desinteresse.}{se.qui.dão}{0}
\verb{sequidão}{}{}{"-ões}{}{}{Magreza.}{se.qui.dão}{0}
\verb{sequilho}{}{}{}{}{s.m.}{Biscoito farináceo, feito geralmente de amido de milho ou de polvilho de araruta.}{se.qui.lho}{0}
\verb{sequioso}{ô}{}{"-osos ⟨ó⟩}{"-osa ⟨ó⟩}{}{Extremamente seco.}{se.qui.o.so}{0}
\verb{sequioso}{ô}{}{"-osos ⟨ó⟩}{"-osa ⟨ó⟩}{adj.}{Sedento.}{se.qui.o.so}{0}
\verb{sequioso}{ô}{Fig.}{"-osos ⟨ó⟩}{"-osa ⟨ó⟩}{}{Extremamente desejoso, cobiçoso, ávido.}{se.qui.o.so}{0}
\verb{séquito}{}{}{}{}{s.m.}{Conjunto das pessoas que acompanham outra(s); comitiva, cortejo.}{sé.qui.to}{0}
\verb{séquito}{}{}{}{}{}{Var. de \textit{séquito}.}{sé.qui.to}{0}
\verb{sequoia}{ó}{Bot.}{}{}{s.f.}{Árvore de grande porte e que chega a viver mil anos, cultivada como ornamental.}{se.quoi.a}{0}
\verb{ser}{ê}{}{}{}{v.i.}{Usado impessoalmente, refere"-se a tempo. (\textit{É muito cedo para sairmos.})}{ser}{0}
\verb{ser}{ê}{}{}{}{}{Estabelecer relação entre as coisas, os fatos e seus atributos. (\textit{A casa é alta. Os móveis são de madeira. A viagem foi longa.})}{ser}{0}
\verb{ser}{ê}{}{}{}{v.pred.}{Estabelecer relação de igualdade entre dois fatos ou entre um fato e uma qualidade. (\textit{O menino é meu colega de classe. O ouro é um metal precioso.})}{ser}{0}
\verb{ser}{ê}{}{}{}{}{Ocorrer, acontecer. (\textit{A comemoração será após os discursos.})}{ser}{0}
\verb{ser}{ê}{}{}{}{}{Existir. (\textit{Somos muitos nesta sala. Era uma vez um príncipe e uma princesa. })}{ser}{0}
\verb{ser}{ê}{}{}{}{s.m.}{Qualquer uma das criaturas que existem. (\textit{Mesmo os menores animais são seres que devem ser considerados.})}{ser}{0}
\verb{seráfico}{}{}{}{}{adj.}{Referente a serafim.}{se.rá.fi.co}{0}
\verb{seráfico}{}{Fig.}{}{}{}{Belo, puro, místico, sublime.}{se.rá.fi.co}{0}
\verb{serafim}{}{Relig.}{"-ins}{}{s.m.}{Na teologia cristã, anjo da primeira hierarquia; anjo, querubim.}{se.ra.fim}{0}
\verb{serafim}{}{Fig.}{"-ins}{}{}{Pessoa de beleza extraordinária.}{se.ra.fim}{0}
\verb{serão}{}{}{"-ões}{}{s.m.}{Trabalho extraordinário feito à noite.}{se.rão}{0}
\verb{sereia}{ê}{}{}{}{s.f.}{Ser mitológico cujo corpo é metade mulher e metade peixe, que com seu canto atrai os marinheiros fazendo"-os perecer no mar.}{se.rei.a}{0}
\verb{sereia}{ê}{Fig.}{}{}{}{Mulher sedutora, fisicamente muito bela.}{se.rei.a}{0}
\verb{serelepe}{é}{Zool.}{}{}{s.m.}{Espécie de esquilo florestal, de cauda longa e cores marrom"-esverdeado e laranja, encontrado em regiões do norte e nordeste do Brasil; caxinguelê.}{se.re.le.pe}{0}
\verb{serelepe}{é}{Fig.}{}{}{adj.}{Vivo, ligeiro, esperto.}{se.re.le.pe}{0}
\verb{serelepe}{é}{Fig.}{}{}{}{Gracioso, atraente.}{se.re.le.pe}{0}
\verb{serenar}{}{}{}{}{v.t.}{Tornar sereno; acalmar, tranquilizar.}{se.re.nar}{0}
\verb{serenar}{}{}{}{}{}{Apaziguar, pacificar.}{se.re.nar}{0}
\verb{serenar}{}{}{}{}{}{Ficar ou deixar exposto ao sereno.}{se.re.nar}{\verboinum{1}}
\verb{serenata}{}{}{}{}{s.f.}{Música simples e melodiosa, semelhante às trovas, cantada ao ar livre por um conjunto musical, geralmente à janela de alguém.}{se.re.na.ta}{0}
\verb{serenidade}{}{}{}{}{s.f.}{Qualidade ou estado de sereno.}{se.re.ni.da.de}{0}
\verb{sereno}{}{}{}{}{adj.}{Calmo, sossegado, tranquilo.}{se.re.no}{0}
\verb{sereno}{}{}{}{}{}{Límpido, claro, sem nuvens (diz"-se do céu).}{se.re.no}{0}
\verb{sereno}{}{}{}{}{s.m.}{Vapor atmosférico noturno; relento.}{se.re.no}{0}
\verb{seresta}{é}{}{}{}{s.f.}{Serenata.}{se.res.ta}{0}
\verb{seresteiro}{ê}{}{}{}{adj.}{Que canta ou compõe serestas.}{se.res.tei.ro}{0}
\verb{seresteiro}{ê}{}{}{}{}{Relativo a seresta.}{se.res.tei.ro}{0}
\verb{sergipano}{}{}{}{}{adj.}{Relativo a Sergipe.}{ser.gi.pa.no}{0}
\verb{sergipano}{}{}{}{}{s.m.}{Indivíduo natural ou habitante desse estado.}{ser.gi.pa.no}{0}
\verb{sergipense}{}{}{}{}{adj.2g. e s.2g.}{Sergipano.}{ser.gi.pen.se}{0}
\verb{seriação}{}{}{"-ões}{}{s.f.}{Ato ou efeito de seriar.}{se.ri.a.ção}{0}
\verb{seriado}{}{}{}{}{s.m.}{Filme cinematográfico ou televisivo que se exibe por partes, geralmente com histórias independentes.}{se.ri.a.do}{0}
\verb{seriado}{}{}{}{}{adj.}{Disposto em série; sequenciado.}{se.ri.a.do}{0}
\verb{serial}{}{}{"-ais}{}{adj.2g.}{Relativo a série.}{se.ri.al}{0}
\verb{serial}{}{}{"-ais}{}{}{Organizado em série; que forma série; seriado.}{se.ri.al}{0}
\verb{serial}{}{}{"-ais}{}{}{Que executa atos similares periodicamente.}{se.ri.al}{0}
\verb{seriar}{}{}{}{}{v.t.}{Dispor em série.}{se.ri.ar}{0}
\verb{seriar}{}{}{}{}{}{Ordenar, classificar.}{se.ri.ar}{\verboinum{1}}
\verb{sericicultor}{ô}{}{}{}{adj.}{Que se dedica à sericicultura.}{se.ri.ci.cul.tor}{0}
\verb{sericicultura}{}{}{}{}{}{Fabricação de seda.}{se.ri.ci.cul.tu.ra}{0}
\verb{sericicultura}{}{}{}{}{s.f.}{Criação de bicho"-da"-seda.}{se.ri.ci.cul.tu.ra}{0}
\verb{sericultor}{ô}{}{}{}{s.m.}{Ver \textit{sericicultor}.}{se.ri.cul.tor}{0}
\verb{sericultura}{}{}{}{}{s.f.}{Ver \textit{sericicultura}.}{se.ri.cul.tu.ra}{0}
\verb{série}{}{}{}{}{s.f.}{Ordem de fatos similares; sequência, sucessão. (\textit{As coisas têm de estar dispostas em séries organizadas do maior para o menor.})}{sé.rie}{0}
\verb{série}{}{}{}{}{}{Sequência ininterrupta.}{sé.rie}{0}
\verb{série}{}{}{}{}{}{Grande quantidade. (\textit{Como eu tive uma série de tarefas para fazer, não pude sair com ela.})}{sé.rie}{0}
\verb{série}{}{}{}{}{}{Cada uma das etapas, com um ano de duração, da educação escolar; ano, classe.}{sé.rie}{0}
\verb{série}{}{}{}{}{}{Conjunto de obras publicadas em conjunto, geralmente com volumes numerados.}{sé.rie}{0}
\verb{seriedade}{}{}{}{}{s.f.}{Qualidade de sério.}{se.ri.e.da.de}{0}
\verb{seriedade}{}{}{}{}{}{Inteireza de caráter; rigor no modo de proceder; austeridade.}{se.ri.e.da.de}{0}
\verb{seriema}{}{Zool.}{}{}{s.f.}{Ave de cor cinza com finas riscas escuras, bico e pernas vermelhas e canto característico.}{se.ri.e.ma}{0}
\verb{serifa}{}{}{}{}{s.f.}{Traço característico que dá acabamento às extremidades de certos tipos de letras.}{se.ri.fa}{0}
\verb{serigrafia}{}{}{}{}{s.f.}{Técnica de impressão que utiliza uma moldura com tela de seda em que se coloca a máscara que permite ou não a passagem da tinta, que é espalhada com a ajuda de um rodo.}{se.ri.gra.fi.a}{0}
\verb{serigrafia}{}{}{}{}{}{A estampa obtida por esse processo.}{se.ri.gra.fi.a}{0}
%\verb{}{}{}{}{}{}{}{}{0}
\verb{seriguela}{é}{Bot.}{}{}{s.f.}{Árvore que pertence à mesma família do caju, do cajá e da manga, nativa da América tropical e aclimatada no nordeste do Brasil; umbuzeiro.}{se.ri.gue.la}{0}
\verb{seriguela}{é}{}{}{}{}{O fruto dessa árvore, amarelo"-avermelhado, doce e sumarento; umbu. }{se.ri.gue.la}{0}
\verb{seringa}{}{}{}{}{s.f.}{Bomba portátil, de vidro ou de plástico, para aplicação de injeções ou para retirar líquidos do organismo.}{se.rin.ga}{0}
\verb{seringa}{}{}{}{}{}{Denominação da borracha bruta.}{se.rin.ga}{0}
\verb{seringal}{}{}{"-ais}{}{s.m.}{Coletivo de seringueira.}{se.rin.gal}{0}
\verb{seringal}{}{}{"-ais}{}{}{Propriedade, fazenda, geralmente à margem de rios.}{se.rin.gal}{0}
\verb{seringalista}{}{}{}{}{s.2g.}{Proprietário de seringal.}{se.rin.ga.lis.ta}{0}
\verb{seringueira}{ê}{Bot.}{}{}{s.f.}{Árvore de grande porte, nativa da Amazônia, com madeira branca, sementes oleaginosas e largamente explorada para produção de borracha natural.}{se.rin.guei.ra}{0}
\verb{seringueiro}{ê}{}{}{}{s.m.}{Trabalhador que extrai o látex da seringueira e com ele prepara a borracha.}{se.rin.guei.ro}{0}
\verb{sério}{}{}{}{}{adj.}{Que tem um aspecto sisudo, circunspecto, grave. (\textit{É um homem sério em todas as situações.})}{sé.rio}{0}
\verb{sério}{}{}{}{}{}{Que tenta fazer todas as atividades da melhor maneira possível. (\textit{É um jogador muito sério na equipe.})}{sé.rio}{0}
\verb{sério}{}{}{}{}{}{Que não permite erros. (\textit{A medicina é uma atividade séria.})}{sé.rio}{0}
\verb{sério}{}{}{}{}{}{Que apresenta riscos. (\textit{O acidente foi muito sério. Essa é uma doença séria.})}{sé.rio}{0}
\verb{sermão}{}{}{"-ões}{}{s.m.}{Discurso religioso geralmente pregado no púlpito por um predicador.}{ser.mão}{0}
\verb{sermão}{}{}{"-ões}{}{}{Discurso moralizador, geralmente longo e enfadonho.}{ser.mão}{0}
\verb{sermão}{}{}{"-ões}{}{}{Qualquer fala com objetivo de convencer alguém de algo.}{ser.mão}{0}
\verb{sermão}{}{}{"-ões}{}{}{Admoestação em tom severo; repreensão; descompostura.}{ser.mão}{0}
\verb{serôdio}{}{}{}{}{adj.}{Que vem tarde, fora do tempo; tardio.}{se.rô.dio}{0}
\verb{serôdio}{}{}{}{}{}{Antiquado, ultrapassado.}{se.rô.dio}{0}
\verb{serosa}{ó}{Anat.}{}{}{s.f.}{Membrana cuja forma lembra um saco, que segrega serosidade em sua face interna e reveste algumas cavidades orgânicas.}{se.ro.sa}{0}
\verb{serosidade}{}{}{}{}{s.f.}{Qualidade de seroso.}{se.ro.si.da.de}{0}
\verb{serosidade}{}{}{}{}{}{Líquido orgânico transparente, segregado e contido nas cavidades serosas.}{se.ro.si.da.de}{0}
\verb{serosidade}{}{}{}{}{}{Líquido semelhante ao soro sanguíneo.}{se.ro.si.da.de}{0}
\verb{seroso}{ô}{}{"-osos ⟨ó⟩}{"-osa ⟨ó⟩}{adj.}{Relativo a soro.}{se.ro.so}{0}
\verb{seroso}{ô}{}{"-osos ⟨ó⟩}{"-osa ⟨ó⟩}{}{Que contém soro.}{se.ro.so}{0}
\verb{seroso}{ô}{}{"-osos ⟨ó⟩}{"-osa ⟨ó⟩}{}{Que secreta serosidade.}{se.ro.so}{0}
\verb{serpe}{é}{Zool.}{}{}{s.f.}{Serpente.}{ser.pe}{0}
\verb{serpe}{é}{}{}{}{}{Antigo artefato de artilharia, longo e fino, semelhante à colubrina; serpentina.}{ser.pe}{0}
\verb{serpe}{é}{}{}{}{}{Linha de ornato em forma de serpente.}{ser.pe}{0}
\verb{serpear}{}{}{}{}{v.i.}{Arrastar"-se pelo chão em ziguezagues, ou mover"-se sinuosamente, como a serpente; ondular.}{ser.pe.ar}{0}
\verb{serpear}{}{}{}{}{}{Ter curso sinuoso ou tortuoso; colear.}{ser.pe.ar}{\verboinum{4}}
\verb{serpejar}{}{}{}{}{v.i.}{Ver \textit{serpear}.}{ser.pe.jar}{\verboinum{1}}
\verb{serpentário}{}{Bras.}{}{}{s.m.}{Lugar onde se criam cobras para estudos.}{ser.pen.tá.rio}{0}
\verb{serpentário}{}{Zool.}{}{}{}{Ave de rapina, que se nutre sobretudo de serpentes.}{ser.pen.tá.rio}{0}
\verb{serpente}{}{}{}{}{s.f.}{Designação geral dos ofídios, sobretudo das espécies peçonhentas; cobra.}{ser.pen.te}{0}
\verb{serpente}{}{Pop.}{}{}{}{Pessoa má ou traiçoeira.}{ser.pen.te}{0}
\verb{serpenteante}{}{}{}{}{adj.2g.}{Que serpenteia; ondulante.}{ser.pen.te.an.te}{0}
\verb{serpentear}{}{}{}{}{v.i.}{Ver \textit{serpear}.}{ser.pen.te.ar}{\verboinum{4}}
\verb{serpentiforme}{ó}{}{}{}{adj.}{Que tem forma de serpente.}{ser.pen.ti.for.me}{0}
\verb{serpentina}{}{}{}{}{s.f.}{Tira estreita e comprida de papel colorido que se desenrola quando se atira. (\textit{Na festa, todos atiram serpentinas.})}{ser.pen.ti.na}{0}
\verb{serpentina}{}{}{}{}{}{Cano estreito e longo, geralmente espiralado, usado para resfriar líquido que passa por ele. (\textit{A serpentina do radiador serve para resfriar a água que resfria o motor.})}{ser.pen.ti.na}{0}
\verb{serpentina}{}{}{}{}{}{Castiçal de três braços e três luzes que se acende no Sábado de Aleluia.}{ser.pen.ti.na}{0}
\verb{serpentino}{}{}{}{}{adj.}{Relativo a, ou que tem forma de serpente.}{ser.pen.ti.no}{0}
\verb{serpentino}{}{}{}{}{}{Diz"-se de certos mármores que apresentam em sua constituição veios de serpentina.}{ser.pen.ti.no}{0}
\verb{serra}{é}{}{}{}{s.f.}{Instrumento cortante, que tem como peça principal uma lâmina ou um disco dentado de aço.}{ser.ra}{0}
\verb{serra}{é}{}{}{}{}{A própria lâmina ou disco cortante desse instrumento.}{ser.ra}{0}
\verb{serra}{é}{}{}{}{}{Cadeia de montanhas e montes com muitos picos e quebradas.}{ser.ra}{0}
\verb{serração}{}{}{"-ões}{}{s.f.}{Ato ou efeito de serrar, de cortar com serra ou serrote.}{ser.ra.ção}{0}
\verb{serrado}{}{}{}{}{adj.}{Que se serrou; cortado com serra ou serrote.}{ser.ra.do}{0}
\verb{serrado}{}{}{}{}{}{Que tem o aspecto dentado da serra.}{ser.ra.do}{0}
\verb{serrador}{ô}{}{}{}{adj.}{Que serra.}{ser.ra.dor}{0}
\verb{serrador}{ô}{}{}{}{s.m.}{Indivíduo que serra.}{ser.ra.dor}{0}
\verb{serrador}{ô}{}{}{}{}{Serrote grande e curvado, com o qual se corta palha.}{ser.ra.dor}{0}
\verb{serragem}{}{}{"-ens}{}{s.f.}{Ver \textit{serração}.}{ser.ra.gem}{0}
\verb{serragem}{}{}{"-ens}{}{}{Pó de fragmentos ou lascas que saem da madeira que é serrada.}{ser.ra.gem}{0}
\verb{serralha}{}{Bot.}{}{}{s.f.}{Verdura cultivada geralmente em hortas domésticas e de pouco alcance comercial.  }{ser.ra.lha}{0}
\verb{serralharia}{}{}{}{}{s.f.}{Arte de trabalhar o ferro, de fabricar ou consertar objetos de ferro.}{ser.ra.lha.ri.a}{0}
\verb{serralharia}{}{}{}{}{}{Fábrica ou oficina para trabalhos em ferro batido ou forjado.}{ser.ra.lha.ri.a}{0}
\verb{serralheiro}{ê}{}{}{}{s.m.}{Indivíduo que fabrica ou conserta objetos de ferro.}{ser.ra.lhei.ro}{0}
\verb{serralheria}{}{}{}{}{}{Var. de \textit{serralharia}.}{ser.ra.lhe.ri.a}{0}
\verb{serralho}{}{}{}{}{}{Espaço desse palácio destinado às mulheres desses personagens; harém.}{ser.ra.lho}{0}
\verb{serralho}{}{}{}{}{}{Local destinado à prostituição; prostíbulo.}{ser.ra.lho}{0}
\verb{serralho}{}{}{}{}{s.m.}{Antigo palácio do sultão, dos príncipes ou dos dignitários do Estado turco maometano.}{ser.ra.lho}{0}
\verb{serrania}{}{}{}{}{s.f.}{Aglomeração de serras ou montanhas; cordilheira.}{ser.ra.ni.a}{0}
\verb{serrano}{}{}{}{}{adj.}{Relativo a serras.}{ser.ra.no}{0}
\verb{serrano}{}{}{}{}{s.m.}{Indivíduo que nasceu ou vive nas serras.}{ser.ra.no}{0}
\verb{serrano}{}{Por ext.}{}{}{}{Pessoa rústica; camponês.}{ser.ra.no}{0}
\verb{serrar}{}{}{}{}{v.t.}{Cortar com serra ou serrote.}{ser.rar}{0}
\verb{serrar}{}{Pop.}{}{}{}{Conseguir gratuitamente, por meios hábeis; filar.}{ser.rar}{0}
\verb{serrar}{}{}{}{}{v.i.}{Produzir um som semelhante ao do serrote ou ao da serra em ação.}{ser.rar}{\verboinum{1}}
\verb{serraria}{}{}{}{}{s.f.}{Estabelecimento ou oficina em que se serram madeiras.}{ser.ra.ri.a}{0}
\verb{serraria}{}{}{}{}{}{Armação de madeira sobre a qual se apoia o objeto que se quer serrar.}{ser.ra.ri.a}{0}
\verb{serrear}{}{}{}{}{v.t.}{Dar aspecto de serra.}{ser.re.ar}{0}
\verb{serrear}{}{}{}{}{}{Recortar ou dentear em forma de serra.}{ser.re.ar}{\verboinum{4}}
\verb{serrilha}{}{}{}{}{s.f.}{Trabalho ornamental feito em forma de dentes de serra.}{ser.ri.lha}{0}
\verb{serrilha}{}{}{}{}{}{Bordo denteado na periferia das moedas.}{ser.ri.lha}{0}
\verb{serrilha}{}{}{}{}{}{Bordo denteado de qualquer objeto.}{ser.ri.lha}{0}
\verb{serrilhar}{}{}{}{}{v.t.}{Fazer serrilha; dentear.}{ser.ri.lhar}{\verboinum{1}}
\verb{serro}{ê}{}{}{}{s.m.}{Aresta de montanha; espinhaço.}{ser.ro}{0}
\verb{serrote}{ó}{}{}{}{s.m.}{Serra manual portátil, operável com uma mão, composta de lâmina de aço serrilhada e presa a um cabo por uma de suas extremidades.}{ser.ro.te}{0}
\verb{serrote}{ó}{Bras.}{}{}{}{Pequena serra ou monte.}{ser.ro.te}{0}
\verb{sertanejo}{ê}{}{}{}{adj.}{Relativo a sertão.}{ser.ta.ne.jo}{0}
\verb{sertanejo}{ê}{}{}{}{}{Que habita o sertão.}{ser.ta.ne.jo}{0}
\verb{sertanejo}{ê}{}{}{}{}{Rústico, agreste, rude.}{ser.ta.ne.jo}{0}
\verb{sertanejo}{ê}{}{}{}{s.m.}{Indivíduo que habita o sertão.}{ser.ta.ne.jo}{0}
\verb{sertanejo}{ê}{}{}{}{}{Indivíduo que vive nas regiões interiores, em especial os de pouca instrução e de convívio e hábitos rústicos; caipira.}{ser.ta.ne.jo}{0}
\verb{sertanista}{}{}{}{}{s.m.}{Indivíduo que se embrenhava nos sertões à cata de riquezas; bandeirante.}{ser.ta.nis.ta}{0}
\verb{sertanista}{}{}{}{}{s.2g.}{Indivíduo que frequenta e conhece bem o sertão e os hábitos sertanejos; especialista em assuntos do sertão.}{ser.ta.nis.ta}{0}
\verb{sertão}{}{}{"-ões}{}{s.m.}{Região agreste, afastada dos núcleos  urbanos e das terras cultivadas.}{ser.tão}{0}
\verb{sertão}{}{}{"-ões}{}{}{A terra e a povoação do interior; o interior do país.}{ser.tão}{0}
\verb{sertão}{}{Bras.}{"-ões}{}{}{Toda região pouco povoada do interior, em especial a zona mais seca que a caatinga, ligada ao ciclo do gado e onde permanecem tradições e costumes antigos.}{ser.tão}{0}
\verb{serva}{é}{}{}{}{s.f.}{Pessoa que vive em estado de servidão.}{ser.va}{0}
\verb{servente}{}{}{}{}{adj.2g.}{Que serve; servidor. (\textit{Ele trabalha como servente na construção da casa.})}{ser.ven.te}{0}
\verb{servente}{}{}{}{}{s.2g.}{Indivíduo que executa serviços auxiliares, notadamente de limpeza ou de conservação das coisas, em repartição ou escritório. (\textit{Ela conseguiu trabalhar como servente na empresa.})}{ser.ven.te}{0}
\verb{serventia}{}{}{}{}{s.f.}{Qualidade do que serve; utilidade, préstimo.}{ser.ven.ti.a}{0}
\verb{serventia}{}{}{}{}{}{Condição de servo; servidão.}{ser.ven.ti.a}{0}
\verb{serventia}{}{}{}{}{}{Passagem, lugar por onde se passa.}{ser.ven.ti.a}{0}
\verb{serventia}{}{}{}{}{}{Trabalho do serventuário.}{ser.ven.ti.a}{0}
\verb{serventuário}{}{}{}{}{s.m.}{Indivíduo que serve em emprego ou ofício.}{ser.ven.tu.á.rio}{0}
\verb{serventuário}{}{}{}{}{}{Indivíduo que é funcionário da Justiça.}{ser.ven.tu.á.rio}{0}
\verb{serviçal}{}{}{"-ais}{}{adj.2g.}{Que presta serviços.}{ser.vi.çal}{0}
\verb{serviçal}{}{}{"-ais}{}{s.m.}{Empregado encarregado dos serviços domésticos; criado.}{ser.vi.çal}{0}
\verb{serviço}{}{}{}{}{s.m.}{Ato ou efeito de servir.}{ser.vi.ço}{0}
\verb{serviço}{}{}{}{}{}{Trabalho, ofício, atividade.}{ser.vi.ço}{0}
\verb{serviço}{}{}{}{}{}{Conjunto de peças usadas à mesa (talheres, baixela, guardanapo).}{ser.vi.ço}{0}
\verb{serviço}{}{}{}{}{}{Acréscimo na conta de restaurantes ou hotéis destinado à gratificação dos funcionários.}{ser.vi.ço}{0}
\verb{servidão}{}{}{"-ões}{}{s.f.}{Condição ou estado de servo, escravo; escravidão.}{ser.vi.dão}{0}
\verb{servidão}{}{}{"-ões}{}{}{Dependência, sujeição, submissão.}{ser.vi.dão}{0}
\verb{servidor}{ô}{}{}{}{adj.}{Que serve.}{ser.vi.dor}{0}
\verb{servidor}{ô}{}{}{}{s.m.}{Serviçal, servente; funcionário, empregado.}{ser.vi.dor}{0}
\verb{servidor}{ô}{Informát.}{}{}{}{Computador, ligado a uma rede, cuja função é compartilhar arquivos, programas e periféricos com os demais computadores da rede.}{ser.vi.dor}{0}
\verb{servil}{}{}{"-is}{}{adj.2g.}{Relativo a ou próprio do servo ou da sua condição.}{ser.vil}{0}
\verb{servil}{}{}{"-is}{}{}{Subserviente, adulador. }{ser.vil}{0}
\verb{servilismo}{}{}{}{}{s.m.}{Qualidade de servil; subserviência, adulação. }{ser.vi.lis.mo}{0}
\verb{sérvio}{}{}{}{}{adj.}{Relativo à Sérvia, uma das repúblicas da Iugoslávia, ou ao dialeto servo"-croata.}{sér.vio}{0}
\verb{sérvio}{}{}{}{}{s.m.}{Indivíduo natural ou habitante desse país. }{sér.vio}{0}
\verb{servir}{}{}{}{}{v.t.}{Trabalhar como empregado, funcionário.}{ser.vir}{0}
\verb{servir}{}{}{}{}{}{Encarregar"-se; servir, ajudar, auxiliar.}{ser.vir}{0}
\verb{servir}{}{}{}{}{}{Prestar o serviço militar.}{ser.vir}{0}
\verb{servir}{}{}{}{}{v.pron.}{Fazer uso; utilizar, recorrer.}{ser.vir}{\verboinum{29}}
\verb{servo}{é}{}{}{}{s.m.}{Indivíduo que serve; criado, serviçal.}{ser.vo}{0}
\verb{servo}{é}{}{}{}{}{Pessoa que não é livre; escravo, súdito.}{ser.vo}{0}
\verb{servo}{é}{Hist.}{}{}{}{No sistema feudal, indivíduo que devia vassalagem ao suserano; vassalo.}{ser.vo}{0}
\verb{sésamo}{}{Bot.}{}{}{s.m.}{Ver \textit{gergelim}. }{sé.sa.mo}{0}
\verb{sesmaria}{}{}{}{}{s.f.}{Lote de terra não cultivado ou abandonado.}{ses.ma.ri.a}{0}
\verb{sesmaria}{}{}{}{}{}{Antiga medida agrária, ainda em uso no estado do Rio Grande do Sul, cuja légua equivale a 3.000 braças ou 6.600 metros.}{ses.ma.ri.a}{0}
\verb{sesmeiro}{ê}{}{}{}{s.m.}{Indivíduo a quem se concedia a sesmaria.}{ses.mei.ro}{0}
\verb{sesquicentenário}{}{}{}{}{adj.}{Que possui 150 anos.}{ses.qui.cen.te.ná.rio}{0}
\verb{sesquicentenário}{}{}{}{}{s.m.}{Festa ou comemoração do centésimo quinquagésimo aniversário de evento, instituição etc.}{ses.qui.cen.te.ná.rio}{0}
\verb{sessão}{}{}{"-ões}{}{s.f.}{Espaço de tempo em que se realiza certa atividade. }{ses.são}{0}
\verb{sessão}{}{}{"-ões}{}{}{Encontro com determinada finalidade; reunião, assembleia.}{ses.são}{0}
\verb{sessão}{}{}{"-ões}{}{}{Cada uma das repetições de uma apresentação, espetáculo etc. }{ses.são}{0}
\verb{sessenta}{}{}{}{}{num.}{Nome dado à quantidade expressa pelo número 60.}{ses.sen.ta}{0}
\verb{sessentão}{}{Pop.}{"-ões}{sessentona}{adj.}{Diz"-se do indivíduo que completou sessenta anos de idade; sexagenário.}{ses.sen.tão}{0}
\verb{sessentão}{}{Pop.}{"-ões}{sessentona}{s.m.}{Indivíduo que está nessa faixa etária.}{ses.sen.tão}{0}
\verb{séssil}{}{}{"-eis}{}{adj.2g.}{Que não apresenta suporte.}{sés.sil}{0}
\verb{séssil}{}{Biol.}{"-eis}{}{}{Imóvel, fixo.}{sés.sil}{0}
\verb{séssil}{}{Bot.}{"-eis}{}{}{Diz"-se de órgão ou estrutura, como folha ou flor, que está diretamente inserida no eixo principal de um organismo, sem apresentar suporte ou haste de sustentação.}{sés.sil}{0}
\verb{sesta}{é}{}{}{}{s.f.}{Hora de repouso após o almoço.}{ses.ta}{0}
\verb{sesta}{é}{}{}{}{}{Momento do dia em que o calor é mais intenso.}{ses.ta}{0}
\verb{sestear}{}{}{}{}{v.i.}{Fazer a sesta.}{ses.te.ar}{0}
\verb{sestear}{}{}{}{}{v.t.}{Proteger o gado do calor.}{ses.te.ar}{\verboinum{4}}
\verb{sestro}{é}{}{}{}{adj.}{Que está à esquerda; esquerdo.}{ses.tro}{0}
\verb{sestro}{é}{}{}{}{}{Agourento, sinistro.}{ses.tro}{0}
\verb{sestro}{é}{}{}{}{s.m.}{Trejeito, mania, cacoete.}{ses.tro}{0}
\verb{sestro}{é}{}{}{}{}{Vício, mau hábito.}{ses.tro}{0}
\verb{sestroso}{ô}{}{"-osos ⟨"-ó⟩}{"-osa ⟨ó⟩}{adj.}{Que tem sestro; manhoso, malandro, teimoso. }{ses.tro.so}{0}
\verb{sestroso}{ô}{}{"-osos ⟨"-ó⟩}{"-osa ⟨ó⟩}{}{Esperto, vivo, sagaz.}{ses.tro.so}{0}
\verb{set}{}{}{}{}{s.m.}{Cenário, para representação teatral ou cinematográfica.}{\textit{set}}{0}
\verb{set}{}{}{}{}{}{Cada segmento ou etapa em que uma partida é dividida, em alguns esportes.}{\textit{set}}{0}
\verb{seta}{é}{}{}{}{s.f.}{Haste de madeira ou metal, com uma extremidade pontiaguda, que se arremessa por meio de um arco.}{se.ta}{0}
\verb{seta}{é}{}{}{}{}{Sinal em forma de flecha que serve para apontar rumo ou direção.}{se.ta}{0}
\verb{sete}{é}{}{}{}{num.}{Nome dado à quantidade expressa pelo número 7.}{se.te}{0}
\verb{setecentos}{}{}{}{}{num.}{Nome dado à quantidade expressa pelo número 700.}{se.te.cen.tos}{0}
%\verb{}{}{}{}{}{}{}{}{0}
\verb{seteira}{ê}{}{}{}{s.f.}{Abertura estreita pela qual se atiram flechas ao inimigo.}{se.tei.ra}{0}
\verb{seteira}{ê}{}{}{}{}{Fresta para a passagem de luz num aposento.}{se.tei.ra}{0}
\verb{setembro}{}{}{}{}{s.m.}{O nono mês do ano civil.}{se.tem.bro}{0}
\verb{setemesinho}{}{}{}{}{adj.}{Que nasce de sete meses.}{se.te.me.si.nho}{0}
\verb{setena}{}{}{}{}{s.f.}{Conjunto de sete seres ou objetos de natureza idêntica.}{se.te.na}{0}
\verb{setenário}{}{}{}{}{adj.}{Que tem ou vale sete.}{se.te.ná.rio}{0}
\verb{setenário}{}{}{}{}{s.m.}{Duração de sete dias de certas festas religiosas.}{se.te.ná.rio}{0}
\verb{setênio}{}{}{}{}{s.m.}{Período de sete anos.}{se.tê.nio}{0}
\verb{setenta}{}{}{}{}{num.}{Nome dado à quantidade expressa pelo número 70. }{se.ten.ta}{0}
\verb{setentão}{}{Pop.}{"-ões}{setentona}{adj.}{Diz"-se do indivíduo que completou setenta anos de idade; septuagenário. }{se.ten.tão}{0}
\verb{setentão}{}{Pop.}{"-ões}{setentona}{s.m.}{Indivíduo que está nessa faixa de idade.}{se.ten.tão}{0}
\verb{setentrião}{}{}{"-ões}{}{s.m.}{O polo norte.}{se.ten.tri.ão}{0}
\verb{setentrião}{}{}{"-ões}{}{}{Vento norte.}{se.ten.tri.ão}{0}
\verb{setentrional}{}{}{"-ais}{}{adj.2g.}{Diz"-se do ou situado no norte; boreal.}{se.ten.tri.o.nal}{0}
\verb{setentrional}{}{}{"-ais}{}{s.2g.}{Indivíduo natural ou habitante do hemisfério norte.}{se.ten.tri.o.nal}{0}
\verb{setilha}{}{}{}{}{s.f.}{Poesia ou estrofe de sete versos.}{se.ti.lha}{0}
\verb{setilhão}{}{}{"-ões}{}{num.}{Mil sextilhões.}{se.ti.lhão}{0}
\verb{setilião}{}{}{}{}{}{Var. de \textit{setilhão}. }{se.ti.li.ão}{0}
\verb{sétima}{}{Mús.}{}{}{s.f.}{Intervalo de sete notas na escala diatônica.}{sé.ti.ma}{0}
\verb{sétima}{}{Mús.}{}{}{}{Nota acrescentada à distância de sete notas da fundamental de um acorde.}{sé.ti.ma}{0}
\verb{sétima}{}{Gram.}{}{}{}{Estrofe de sete versos; hepteto.}{sé.ti.ma}{0}
\verb{sétimo}{}{}{}{}{num.}{Numa sequência, o que ocupa a posição de número sete. }{sé.ti.mo}{0}
\verb{setingentésimo}{}{}{}{}{num.}{Numa sequência, o que ocupa a posição do número 700.}{se.tin.gen.té.si.mo}{0}
\verb{setissílabo}{}{Gram.}{}{}{adj.}{Diz"-se do verso ou da palavra que tem sete sílabas; heptassílabo.}{se.tis.sí.la.bo}{0}
\verb{setor}{ô}{}{}{}{s.m.}{Divisão ou subdivisão de um estabelecimento; seção.}{se.tor}{0}
\verb{setor}{ô}{}{}{}{}{Área ou esfera de atividade; âmbito.}{se.tor}{0}
\verb{setorial}{}{}{"-ais}{}{adj.2g.}{Relativo a setor.}{se.to.ri.al}{0}
%\verb{}{}{}{}{}{}{}{}{0}
\verb{setorizar}{}{}{}{}{v.t.}{Dividir em setores ou seções.}{se.to.ri.zar}{\verboinum{1}}
\verb{setuagenário}{}{}{}{}{}{Var. de \textit{septuagenário}.}{se.tu.a.ge.ná.rio}{0}
\verb{setuagésimo}{}{}{}{}{}{Var. de \textit{septuagésimo}.}{se.tu.a.gé.si.mo}{0}
\verb{setuplicar}{}{}{}{}{v.t.}{Multiplicar por sete.}{se.tu.pli.car}{\verboinum{2}}
\verb{sétuplo}{}{}{}{}{num.}{Que contém sete vezes o mesmo elemento ou quantidade.}{sé.tu.plo}{0}
\verb{seu}{}{}{}{}{pron.}{Possessivo que indica a terceira pessoa. (\textit{José comprou seus próprios livros.})}{seu}{0}
\verb{seu}{}{}{}{}{pron.}{Forma de tratamento usada para demonstrar respeito. (\textit{Ontem, encontrei Seu João andando pela rua.})}{seu}{0}
\verb{seu}{}{Pop.}{}{}{}{Possessivo que indica a segunda pessoa. (\textit{Assim que vi seu recado, telefonei para você.})}{seu}{0}
\verb{seu}{}{}{}{}{}{Em exclamações vocativas, indica intensificação valorativa. (\textit{Seu felizardo, ganhou na loteria! Seu mentiroso, não confio mais em você!})}{seu}{0}
\verb{seu"-vizinho}{}{Pop.}{}{}{s.m.}{O dedo anular.}{seu"-vizinho}{0}
\verb{sevandija}{}{}{}{}{s.2g.}{Indivíduo que vive à custa dos outros; parasito.}{se.van.di.ja}{0}
\verb{sevandija}{}{}{}{}{}{Indivíduo vergonhosamente servil.}{se.van.di.ja}{0}
\verb{sevandija}{}{}{}{}{s.f.}{Nome comum aos vermes e parasitas.}{se.van.di.ja}{0}
\verb{severidade}{}{}{}{}{s.f.}{Qualidade do que é severo; austeridade, rigor.}{se.ve.ri.da.de}{0}
\verb{severidade}{}{}{}{}{}{Falta de flexibilidade ao julgar, disciplinar, castigar.}{se.ve.ri.da.de}{0}
\verb{severidade}{}{}{}{}{}{Qualidade de estilo severo; sobriedade.}{se.ve.ri.da.de}{0}
\verb{severo}{é}{}{}{}{adj.}{Que impõe as condições com todo o rigor; rígido, rigoroso.}{se.ve.ro}{0}
\verb{severo}{é}{}{}{}{}{Inflexível nas decisões ou na disciplina.}{se.ve.ro}{0}
\verb{severo}{é}{}{}{}{}{Extremamente sério; grave.}{se.ve.ro}{0}
%\verb{}{}{}{}{}{}{}{}{0}
\verb{seviciar}{}{}{}{}{v.t.}{Praticar sevícias; maltratar fisicamente.}{se.vi.ci.ar}{\verboinum{6}}
\verb{sevícias}{}{}{}{}{s.f.pl.}{Maus"-tratos; atos de crueldade, de tortura física ou mental.}{se.ví.cias}{0}
\verb{sevo}{é}{}{}{}{adj.}{Que aplica sevícias; cruel, desumano.}{se.vo}{0}
\verb{sexagenário}{cs}{}{}{}{adj.}{Diz"-se do indivíduo que está na casa dos sessenta anos de idade.}{se.xa.ge.ná.rio}{0}
\verb{sexagenário}{cs}{}{}{}{s.m.}{Indivíduo que está nessa faixa etária.}{se.xa.ge.ná.rio}{0}
\verb{sexagésimo}{cs}{}{}{}{num.}{Ordinal e fracionário correspondente a 60.}{se.xa.gé.si.mo}{0}
\verb{sexangular}{cs}{Geom.}{}{}{adj.2g.}{Que tem seis ângulos.}{se.xan.gu.lar}{0}
\verb{sex"-appeal}{}{}{}{}{s.m.}{Encanto físico que provoca o desejo sexual.}{\textit{sex"-appeal}}{0}
\verb{sexcentésimo}{cs}{}{}{}{num.}{Ordinal e fracionário correspondente a 600.}{sex.cen.té.si.mo}{0}
\verb{sexênio}{cs}{}{}{}{s.m.}{Período de seis anos.}{se.xê.nio}{0}
\verb{sexismo}{cs}{}{}{}{s.m.}{Tratamento diferenciado e preconceituoso dado ao sexo oposto, sobretudo ao feminino, nas relações sociais e profissionais.}{se.xis.mo}{0}
\verb{sexo}{écs}{}{}{}{s.m.}{Aspecto particular que distingue o macho da fêmea, nos animais e nos vegetais, atribuindo"-lhes um papel determinado na geração e conferindo"-lhes certas características distintivas.}{se.xo}{0}
\verb{sexo}{écs}{}{}{}{}{O conjunto das pessoas que possuem o mesmo sexo.}{se.xo}{0}
\verb{sexo}{écs}{Por ext.}{}{}{}{Volúpia, sensualidade, sexualidade.}{se.xo}{0}
\verb{sexo}{écs}{Bras.}{}{}{}{Os órgãos sexuais externos.}{se.xo}{0}
\verb{sexologia}{cs}{}{}{}{s.f.}{Ciência que tem por objeto o estudo da sexualidade e dos problemas fisiológicos ou psíquicos com ela relacionados.}{se.xo.lo.gi.a}{0}
\verb{sexologista}{cs}{}{}{}{s.2g.}{Indivíduo que se especializa em sexologia; sexólogo.}{se.xo.lo.gis.ta}{0}
\verb{sexólogo}{cs}{}{}{}{s.m.}{Sexologista.}{se.xó.lo.go}{0}
\verb{sexta}{ês}{}{}{}{s.f.}{Forma reduzida de \textit{sexta"-feira}.}{sex.ta}{0}
\verb{sexta"-feira}{ês}{}{sextas"-feiras ⟨ês⟩}{}{s.f.}{O sexto dia da semana.}{sex.ta"-fei.ra}{0}
\verb{sextante}{s}{}{}{}{s.m.}{A sexta parte de um círculo; arco de sessenta graus.}{sex.tan.te}{0}
\verb{sextante}{s}{}{}{}{}{Instrumento ótico, constituído de dois espelhos e uma luneta astronômica, destinado a medir a altura de um astro acima do horizonte.}{sex.tan.te}{0}
\verb{sextavado}{s}{}{}{}{adj.}{Que tem seis lados.}{sex.ta.va.do}{0}
\verb{sextavar}{s}{}{}{}{v.t.}{Cortar em forma sexangular, formando seis faces ou lados. }{sex.ta.var}{\verboinum{1}}
\verb{sexteto}{s\ldots{}ê}{Mús.}{}{}{s.m.}{Composição musical para seis vozes ou instrumentos.}{sex.te.to}{0}
\verb{sexteto}{s\ldots{}ê}{Mús.}{}{}{}{Conjunto dos músicos ou cantores que executam ou cantam essa composição.}{sex.te.to}{0}
\verb{sextilha}{s}{Gram.}{}{}{s.f.}{Estrofe de seis versos.}{sex.ti.lha}{0}
\verb{sextilhão}{s}{}{"-ões}{}{num.}{Número cardinal equivalente a mil quintilhões.}{sex.ti.lhão}{0}
\verb{sextilião}{s}{}{}{}{}{Var. de \textit{sextilhão}.}{sex.ti.li.ão}{0}
\verb{sextina}{s}{}{}{}{s.f.}{Ver \textit{sextilha}.}{sex.ti.na}{0}
\verb{sexto}{s}{}{}{}{num.}{Ordinal e fracionário correspondente a 6.}{sex.to}{0}
\verb{sextuplicar}{s}{}{}{}{v.t.}{Multiplicar por seis; tornar seis vezes maior.}{sex.tu.pli.car}{\verboinum{1}}
\verb{sêxtuplo}{s}{}{}{}{num.}{Que é seis vezes maior que outro.}{sêx.tu.plo}{0}
\verb{sêxtuplo}{s}{}{}{}{s.m.}{Quantidade seis vezes maior que outra.}{sêx.tu.plo}{0}
\verb{sexuado}{cs}{}{}{}{adj.}{Que tem sexo; que é provido de orgãos de reprodução.}{se.xu.a.do}{0}
\verb{sexual}{cs}{}{"-ais}{}{adj.2g.}{Relativo ao sexo, à diferença biológica entre macho e fêmea.}{se.xu.al}{0}
\verb{sexual}{cs}{}{"-ais}{}{}{Referente à cópula, ao ato sexual.}{se.xu.al}{0}
\verb{sexual}{cs}{}{"-ais}{}{}{Que caracteriza o sexo das plantas e dos animais.}{se.xu.al}{0}
\verb{sexualidade}{cs}{}{}{}{s.f.}{Qualidade do que é sexual; maneira de ser, própria do que tem o sexo.}{se.xu.a.li.da.de}{0}
\verb{sexualidade}{cs}{}{}{}{}{O conjunto dos fenômenos da vida sexual.}{se.xu.a.li.da.de}{0}
\verb{sexualidade}{cs}{}{}{}{}{Sensualidade, volúpia.}{se.xu.a.li.da.de}{0}
\verb{sexualismo}{cs}{}{}{}{s.m.}{Condição do que tem sexo.}{se.xu.a.lis.mo}{0}
\verb{sexualismo}{cs}{}{}{}{}{A vida sexual; as funções sexuais.}{se.xu.a.lis.mo}{0}
\verb{sexy}{}{}{}{}{}{Diz"-se de coisa sexualmente sugestiva ou estimulante; erótico.}{\textit{sexy}}{0}
\verb{sexy}{}{}{}{}{adj.2g.}{Diz"-se de pessoa sexualmente atraente.}{\textit{sexy}}{0}
\verb{sezão}{}{}{"-ões}{}{s.f.}{Febre intermitente e cíclica; febre da malária.}{se.zão}{0}
%\verb{}{}{}{}{}{}{}{}{0}
%\verb{}{}{}{}{}{}{}{}{0}
\verb{shopping center}{}{}{}{}{s.m.}{Centro comercial de arquitetura específica, reunindo lojas de produtos muito variados, além de restaurantes, cinemas, teatros, boates etc.}{\textit{shopping center}}{0}
\verb{short}{}{}{}{}{s.m.}{Calça curta, geralmente menos comprida que a bermuda, usada para esporte ou para passeio.}{\textit{short}}{0}
\verb{short}{}{}{}{}{}{Filme breve, geralmente de atualidade ou de caráter documentário.}{\textit{short}}{0}
\verb{show}{}{}{}{}{s.m.}{Espetáculo apresentado em teatro, televisão, rádio, casas noturnas ou mesmo ao ar livre, geralmente montado em torno de um cantor ou animador.}{\textit{show}}{0}
\verb{si}{}{Mús.}{}{}{s.m.}{A sétima nota musical na escala de \textit{dó}.}{si}{0}
\verb{si}{}{Gram.}{}{}{pron.}{Pronome oblíquo reflexivo tônico de terceira pessoa.}{si}{0}
\verb{Si}{}{Quím.}{}{}{}{Símb. do \textit{silício}.}{Si}{0}
\verb{siá}{}{Bras.}{}{}{s.f.}{Ver \textit{sinhá}.}{si.á}{0}
\verb{siamês}{}{}{}{}{adj.}{Relativo ao Sião (atual Tailândia).}{si.a.mês}{0}
\verb{siamês}{}{}{}{}{}{Diz"-se de uma raça de gatos de olhos azuis, pelo curto, de cor creme no corpo e castanho"-escuro na face, orelhas, patas e cauda.}{si.a.mês}{0}
\verb{siamês}{}{}{}{}{}{Diz"-se de cada gêmeo que nasce ligado a outro, por uma parte do corpo.}{si.a.mês}{0}
\verb{siamês}{}{}{}{}{s.m.}{Indivíduo natural ou habitante do Sião.}{si.a.mês}{0}
\verb{siamês}{}{}{}{}{}{A língua falada na Tailândia.}{si.a.mês}{0}
\verb{siba}{}{Zool.}{}{}{s.f.}{Gênero de moluscos, de corpo curto, largo e achatado, e concha interna, reduzida, dotada de câmaras preenchidas por gás, o que contribui para a flutuação; produzem um líquido negro, chamado sépia, usado para defesa, e utilizado também como pigmento na confecção de tintas.}{si.ba}{0}
\verb{sibarita}{}{}{}{}{adj.}{Relativo à antiga cidade grega de Síbaris.}{si.ba.ri.ta}{0}
\verb{sibarita}{}{}{}{}{}{Diz"-se de pessoa dada à indolência ou à vida de prazeres, por alusão aos antigos habitantes de Síbaris, famosos por sua riqueza e voluptuosidade.}{si.ba.ri.ta}{0}
\verb{sibarita}{}{}{}{}{s.m.}{Indivíduo natural ou habitante dessa cidade.}{si.ba.ri.ta}{0}
\verb{sibarita}{}{}{}{}{}{Indivíduo dado aos prazeres físicos, à voluptuosidade e à indolência, a exemplo dos antigos habitantes de Síbaris, que muito ricos, tinham fama de cultivar esses hábitos.}{si.ba.ri.ta}{0}
\verb{siberiano}{}{}{}{}{adj.}{Relativo à Sibéria (Rússia).}{si.be.ri.a.no}{0}
\verb{siberiano}{}{}{}{}{s.m.}{Indivíduo natural ou habitante desse país.}{si.be.ri.a.no}{0}
\verb{sibila}{}{}{}{}{s.f.}{Entre os antigos, mulher a quem se atribuíam o dom da profecia e o conhecimento do futuro.}{si.bi.la}{0}
\verb{sibila}{}{Por ext.}{}{}{}{Bruxa, feiticeira.}{si.bi.la}{0}
\verb{sibilante}{}{}{}{}{adj.2g.}{Que sibila, que produz som agudo e contínuo.}{si.bi.lan.te}{0}
\verb{sibilar}{}{}{}{}{v.i.}{Produzir som agudo e contínuo, assoprando; assobiar, silvar.}{si.bi.lar}{\verboinum{1}}
\verb{sibilino}{}{}{}{}{adj.}{Relativo a sibila.}{si.bi.li.no}{0}
\verb{sibilino}{}{Fig.}{}{}{}{Que é difícil de entender; obscuro, enigmático.}{si.bi.li.no}{0}
\verb{sibilo}{}{}{}{}{s.m.}{Ato ou efeito de sibilar; silvo, assobio.}{si.bi.lo}{0}
\verb{sicário}{}{}{}{}{s.m.}{Assassino profissional; facínora.}{si.cá.rio}{0}
\verb{sicário}{}{}{}{}{adj.}{Sanguinário, cruel.}{si.cá.rio}{0}
\verb{siciliano}{}{}{}{}{adj.}{Relativo à Sicília, ilha no sul da Itália; sículo, siciliense, siciliota.}{si.ci.li.a.no}{0}
\verb{siciliano}{}{}{}{}{s.m.}{Indivíduo natural ou habitante dessa ilha.}{si.ci.li.a.no}{0}
\verb{sicômoro}{}{Bot.}{}{}{s.m.}{Espécie de figueira nativa da África, cultivada pelos figos comestíveis e pela madeira.}{si.cô.mo.ro}{0}
\verb{sicômoro}{}{}{}{}{}{A madeira dessa árvore.}{si.cô.mo.ro}{0}
\verb{sicrano}{}{}{}{}{s.m.}{Indivíduo indeterminado, cujo nome não se sabe ou não se quer mencionar.}{si.cra.no}{0}
\verb{SIDA}{}{}{}{}{s.f.}{Sigla de \textit{Síndrome de Imunodeficiência Adquirida}; aids.}{SIDA}{0}
\verb{sideral}{}{}{"-ais}{}{adj.2g.}{Relativo aos astros ou ao céu.}{si.de.ral}{0}
\verb{siderar}{}{}{}{}{v.t.}{Estarrecer, paralisar, fulminar.}{si.de.rar}{0}
\verb{siderar}{}{Fig.}{}{}{}{Atordoar, perturbar.}{si.de.rar}{\verboinum{1}}
\verb{sidéreo}{}{Liter.}{}{}{adj.}{Ver \textit{sideral}.}{si.dé.re.o}{0}
\verb{siderurgia}{}{}{}{}{s.f.}{Técnica de fundição e preparação de ferro e aço.}{si.de.rur.gi.a}{0}
\verb{siderurgia}{}{}{}{}{}{Arte do ferrador.}{si.de.rur.gi.a}{0}
\verb{siderúrgica}{}{}{}{}{s.f.}{Empresa ou usina siderúrgica.}{si.de.rúr.gi.ca}{0}
\verb{siderúrgico}{}{}{}{}{adj.}{Relativo a siderurgia.}{si.de.rúr.gi.co}{0}
\verb{siderúrgico}{}{}{}{}{s.m.}{Operário de siderurgia.}{si.de.rúr.gi.co}{0}
\verb{sidra}{}{}{}{}{s.f.}{Bebida fermentada e espumante preparada com suco de maçã.}{si.dra}{0}
\verb{sidra}{}{Bras.}{}{}{}{Vinho de mandioca.}{si.dra}{0}
\verb{sifão}{}{}{"-ões}{}{s.m.}{Tubo recurvado em forma de \textsc{u} utilizado para retirar líquido de recipientes sem incliná"-los.}{si.fão}{0}
\verb{sifão}{}{}{"-ões}{}{}{Dispositivo instalado no ralo de pias, ralos, esgotos etc. que impede o retorno de mau cheiro.}{si.fão}{0}
\verb{sifão}{}{}{"-ões}{}{}{Garrafa com dispositivo que faz jorrar água gasosa sob pressão.}{si.fão}{0}
\verb{sífilis}{}{Med.}{}{}{s.f.}{Doença infecciosa, transmitida por contato sexual e transmissível à descendência, causada por bactéria e caracterizada por lesões na pele e mucosas.}{sí.fi.lis}{0}
\verb{sifilítico}{}{}{}{}{adj.}{Relativo a sífilis.}{si.fi.lí.ti.co}{0}
\verb{sifilítico}{}{}{}{}{}{Que está doente de sífilis.}{si.fi.lí.ti.co}{0}
\verb{sifilítico}{}{}{}{}{s.m.}{Indivíduo doente de sífilis.}{si.fi.lí.ti.co}{0}
\verb{sigilo}{}{}{}{}{s.m.}{Informação que não pode ser revelada; segredo.}{si.gi.lo}{0}
\verb{sigilo}{}{}{}{}{}{Discrição, reserva. \textit{abon}}{si.gi.lo}{0}
\verb{sigiloso}{ô}{}{"-osos ⟨ó⟩}{"-osa ⟨ó⟩}{adj.}{Que envolve sigilo.}{si.gi.lo.so}{0}
\verb{sigla}{}{}{}{}{s.f.}{Conjunto de uma ou mais letras utilizado como abreviação de um nome próprio ou expressão.}{si.gla}{0}
\verb{sigla}{}{}{}{}{}{Vocábulo formado a partir de uma abreviação.}{si.gla}{0}
\verb{sigma}{}{}{}{}{s.m.}{Décima oitava letra do alfabeto grego, correspondente ao \textit{s} do latim e das línguas neolatinas.}{sig.ma}{0}
\verb{signatário}{}{}{}{}{adj.}{Que assina um texto ou documento.}{sig.na.tá.rio}{0}
\verb{significação}{}{}{"-ões}{}{s.f.}{Ato ou efeito de significar.}{sig.ni.fi.ca.ção}{0}
\verb{significação}{}{}{"-ões}{}{}{Aquilo que alguma coisa ou palavra quer dizer; significado, acepção, valor.}{sig.ni.fi.ca.ção}{0}
\verb{significado}{}{}{}{}{s.m.}{Conjunto de informações que uma palavra ou texto apresenta; sentido. (\textit{É preciso apreender bem o significado das palavras.})}{sig.ni.fi.ca.do}{0}
\verb{significado}{}{}{}{}{}{Conjunto de ideias e valores que um fato apresenta. (\textit{O Dia das Mães acabou tendo um significado especial para ela.})}{sig.ni.fi.ca.do}{0}
\verb{significante}{}{}{}{}{adj.2g.}{Significativo. (\textit{Foi uma atividade significante para a vida dela.})}{sig.ni.fi.can.te}{0}
\verb{significante}{}{Gram.}{}{}{s.m.}{A parte sonora de um signo.}{sig.ni.fi.can.te}{0}
\verb{significar}{}{}{}{}{v.i.}{Ter determinada importância para alguém. (\textit{Aquelas flores significaram muito para ela.})}{sig.ni.fi.car}{0}
\verb{significar}{}{}{}{}{v.t.}{Produzir a ideia de alguma coisa; denotar, indicar, representar. (\textit{A troca de olhares significava que ambos pretendiam conhecer"-se.})}{sig.ni.fi.car}{\verboinum{2}}
\verb{significativo}{}{}{}{}{adj.}{Que significa com clareza.}{sig.ni.fi.ca.ti.vo}{0}
\verb{significativo}{}{}{}{}{}{Que contém revelação interessante ou relevante; expressivo.}{sig.ni.fi.ca.ti.vo}{0}
\verb{signo}{}{}{}{}{s.m.}{O que representa ou evoca uma outra coisa; símbolo, sinal.}{sig.no}{0}
\verb{signo}{}{}{}{}{}{Cada uma das constelações que formam o zodíaco.}{sig.no}{0}
\verb{signo"-de"-salomão}{}{Relig.}{signos"-de"-salomão}{}{s.m.}{Ver \textit{estrela"-de"-davi}.}{sig.no"-de"-sa.lo.mão}{0}
\verb{signo"-de"-salomão}{}{Por ext.}{signos"-de"-salomão}{}{}{Amuleto em forma de estrela, para proteger contra magia ou forças malignas.}{sig.no"-de"-sa.lo.mão}{0}
\verb{sílaba}{}{Gram.}{}{}{s.f.}{Grupo de sons ou fonemas caracterizados por um só ápice de sonoridade (em português, uma vogal).}{sí.la.ba}{0}
\verb{silabação}{}{Gram.}{"-ões}{}{s.f.}{Ato ou efeito de silabar, de ler ou pronunciar por sílaba.}{si.la.ba.ção}{0}
\verb{silabada}{}{Gram.}{}{}{s.f.}{Erro de pronúncia caracterizado pelo deslocamento do acento tônico da palavra; por ex. \textit{púdica} em vez de \textit{pudica}.}{si.la.ba.da}{0}
\verb{silabar}{}{}{}{}{v.t.}{Ler ou pronunciar separando as sílabas.}{si.la.bar}{0}
\verb{silabar}{}{}{}{}{}{Fazer a separação silábica na escrita.}{si.la.bar}{\verboinum{1}}
\verb{silábico}{}{}{}{}{adj.}{Relativo a sílaba.}{si.lá.bi.co}{0}
\verb{silagem}{}{}{"-ens}{}{s.f.}{Ver \textit{ensilagem}.}{si.la.gem}{0}
\verb{silagem}{}{}{"-ens}{}{}{A porção de forragem retirada dos silos destinada à alimentação dos animais.}{si.la.gem}{0}
\verb{silenciador}{ô}{}{}{}{adj.}{Que silencia.}{si.len.ci.a.dor}{0}
\verb{silenciador}{ô}{}{}{}{s.m.}{Dispositivo que se adapta ao cano de uma arma de fogo para abafar o som do tiro.}{si.len.ci.a.dor}{0}
\verb{silenciador}{ô}{}{}{}{}{Parte do sistema de escapamento de um automóvel que diminui o ruído emitido; silencioso.}{si.len.ci.a.dor}{0}
\verb{silenciar}{}{}{}{}{v.t.}{Guardar silêncio. }{si.len.ci.ar}{0}
\verb{silenciar}{}{}{}{}{}{Impor silêncio a algo ou alguém. }{si.len.ci.ar}{\verboinum{1}}
\verb{silêncio}{}{}{}{}{}{Sigilo, segredo.}{si.lên.cio}{0}
\verb{silêncio}{}{}{}{}{s.m.}{Ausência de ruído.}{si.lên.cio}{0}
\verb{silêncio}{}{}{}{}{}{Estado de quem se cala.}{si.lên.cio}{0}
\verb{silêncio}{}{}{}{}{}{Calma, sossego.}{si.lên.cio}{0}
\verb{silêncio}{}{}{}{}{interj.}{Expressão usada para pedir silêncio ou para fazer calar.}{si.lên.cio}{0}
\verb{silencioso}{ô}{}{"-osos ⟨ó⟩}{"-osa ⟨ó⟩}{adj.}{Que está em silêncio.}{si.len.ci.o.so}{0}
\verb{silencioso}{ô}{}{"-osos ⟨ó⟩}{"-osa ⟨ó⟩}{}{Que não fala.}{si.len.ci.o.so}{0}
\verb{silencioso}{ô}{}{"-osos ⟨ó⟩}{"-osa ⟨ó⟩}{s.m.}{Parte do sistema de escapamento de um automóvel que diminui o ruído emitido; silenciador.}{si.len.ci.o.so}{0}
\verb{silente}{}{Liter.}{}{}{adj.2g.}{Silencioso.}{si.len.te}{0}
\verb{silepse}{é}{Gram.}{}{}{s.f.}{Figura em que a concordância é feita pelo referente e não pela forma da palavra. (\textit{A turma estavam barulhentos.})}{si.lep.se}{0}
\verb{silepse}{é}{Gram.}{}{}{}{Emprego de uma palavra com o sentido original e figurado ao mesmo tempo.}{si.lep.se}{0}
\verb{sílex}{cs}{Geol.}{}{}{s.m.}{Rocha muito dura, composta de calcedônia e opala, de cor cinza, amarela ou preta; pederneira.}{sí.lex}{0}
\verb{sílex}{cs}{}{}{}{}{Diz"-se de artefato ou arma pré"-histórica feita desse material.}{sí.lex}{0}
\verb{sílfide}{}{}{}{}{s.f.}{Mulher esbelta e graciosa.}{síl.fi.de}{0}
\verb{sílfide}{}{}{}{}{}{Imagem diáfana, vaporosa.}{síl.fi.de}{0}
\verb{sílfide}{}{Mit.}{}{}{}{Espírito feminino do ar na mitologia céltica e germânica.}{síl.fi.de}{0}
\verb{silhueta}{ê}{}{}{}{s.f.}{Desenho que representa o perfil de uma pessoa ou de um objeto apenas pelos contornos de sua sombra.}{si.lhu.e.ta}{0}
\verb{silhueta}{ê}{}{}{}{}{As linhas, o contorno do corpo humano.}{si.lhu.e.ta}{0}
\verb{sílica}{}{Geol.}{}{}{s.f.}{Dióxido de silício, substância branca e sólida encontrada em minerais, areias e silicatos, usada principalmente na fabricação de vidro.}{sí.li.ca}{0}
\verb{silicato}{}{Quím.}{}{}{s.m.}{Sal ou éster produzido pela combinação do ácido silícico com uma base e que constitui parte importante da composição das rochas da crosta terrestre.}{si.li.ca.to}{0}
\verb{silícico}{}{Quím.}{}{}{adj.}{Diz"-se do ácido derivado do silício ou de compostos em que entra o silício.}{si.lí.ci.co}{0}
\verb{silício}{}{Quím.}{}{}{s.m.}{Elemento químico com aspecto metálico, cinza"-azulado, muito quebradiço e leve, utilizado como semicondutor em eletrônica. \elemento{14}{28.0855}{Si}.}{si.lí.cio}{0}
\verb{silicone}{ô}{Quím.}{}{}{s.m.}{Nome genérico de polímeros que contêm silício e oxigênio usados como lubrificantes, fluidos hidráulicos, adesivos, e também na cosmetologia.}{si.li.co.ne}{0}
\verb{silk"-screen}{}{}{}{}{s.m.}{Ver \textit{serigrafia}.}{\textit{silk"-screen}}{0}
\verb{silo}{}{}{}{}{s.m.}{Depósito ou reservatório fechado para armazenamento de cereais ou forragens.}{si.lo}{0}
\verb{silogismo}{}{Filos.}{}{}{s.m.}{Segundo o aristotelismo, raciocínio dedutivo formado por três proposições, das quais a terceira é a consequência das duas primeiras, chamadas de premissas.}{si.lo.gis.mo}{0}
\verb{siluriano}{}{Geol.}{}{}{s.m.}{Período da era paleozoica, compreendido entre o ordoviciano e o devoniano, há cerca de 400 milhões de anos, no qual surgiram invertebrados marinhos, grandes crustáceos e recifes de corais.}{si.lu.ri.a.no}{0}
\verb{silva}{}{Bot.}{}{}{s.f.}{Nome dado a diversos arbustos da família das rosáceas, tais como a sarça, o espinheiro etc.}{sil.va}{0}
\verb{silvar}{}{}{}{}{v.i.}{Produzir som agudo e prolongado; assobiar, sibilar.}{sil.var}{0}
\verb{silvar}{}{}{}{}{v.t.}{Aspirar, produzindo silvo ou zumbido.}{sil.var}{\verboinum{1}}
\verb{silvestre}{é}{}{}{}{adj.}{Próprio das selvas; selvagem.}{sil.ves.tre}{0}
\verb{silvestre}{é}{}{}{}{}{Diz"-se da vegetação que nasce espontaneamente, sem precisar ser cultivada pelo homem.}{sil.ves.tre}{0}
\verb{silvestre}{é}{}{}{}{}{Agreste, bravio.}{sil.ves.tre}{0}
\verb{silvícola}{}{}{}{}{adj.}{Que nasce ou vive na selva; selvagem.}{sil.ví.co.la}{0}
\verb{silvicultor}{ô}{}{}{}{s.m.}{Engenheiro agrônomo especializado na formação e conservação de florestas.}{sil.vi.cul.tor}{0}
\verb{silvicultura}{}{}{}{}{s.f.}{Ciência que se dedica ao estudo, ao cultivo e à exploração das florestas.}{sil.vi.cul.tu.ra}{0}
\verb{silvo}{}{}{}{}{s.m.}{Qualquer som agudo e prolongado produzido pelo homem ou por alguns animais, como serpentes e pássaros.}{sil.vo}{0}
\verb{silvo}{}{}{}{}{}{Assobio, sibilo, zumbido.}{sil.vo}{0}
\verb{sim}{}{}{}{}{adv.}{Designa afirmação, permissão, consentimento, aprovação.}{sim}{0}
\verb{sim}{}{}{}{}{s.m.}{Ato de consentir.}{sim}{0}
\verb{simbiose}{ó}{Biol.}{}{}{s.f.}{Associação entre seres vivos diferentes, na qual ambos recebem benefícios.}{sim.bi.o.se}{0}
\verb{simbiose}{ó}{Fig.}{}{}{}{Associação íntima entre duas pessoas.}{sim.bi.o.se}{0}
\verb{simbólico}{}{}{}{}{adj.}{Relativo a símbolo.}{sim.bó.li.co}{0}
\verb{simbólico}{}{}{}{}{}{Que se expressa por meio de símbolos; alegórico, metafórico.}{sim.bó.li.co}{0}
\verb{simbolismo}{}{}{}{}{s.m.}{Expressão ou interpretação por meio de símbolos.}{sim.bo.lis.mo}{0}
\verb{simbolismo}{}{Liter.}{}{}{}{Movimento literário e artístico que surgiu no fim do século \textsc{xix} como reação ao Realismo e ao Parnasianismo e que se caracterizava pela expressão subjetiva e espiritual por meio da combinação de símbolos, imagens e sons.}{sim.bo.lis.mo}{0}
\verb{simbolizar}{}{}{}{}{v.t.}{Expressar ou representar por meio de símbolos.}{sim.bo.li.zar}{0}
\verb{simbolizar}{}{}{}{}{}{Servir de símbolo; significar.}{sim.bo.li.zar}{0}
\verb{simbolizar}{}{}{}{}{}{Falar ou escrever simbolicamente.}{sim.bo.li.zar}{\verboinum{1}}
\verb{símbolo}{}{}{}{}{s.m.}{Sinal, figura, imagem que representa um conceito.}{sím.bo.lo}{0}
\verb{símbolo}{}{}{}{}{}{Pessoa, animal ou personagem que representa um determinado comportamento ou atividade.}{sím.bo.lo}{0}
\verb{símbolo}{}{}{}{}{}{Alegoria, comparação, metáfora.}{sím.bo.lo}{0}
\verb{simbologia}{}{}{}{}{s.f.}{Estudo ou interpretação dos símbolos.}{sim.bo.lo.gi.a}{0}
\verb{simbologia}{}{}{}{}{}{Sistema ou conjunto de símbolos.}{sim.bo.lo.gi.a}{0}
\verb{simetria}{}{}{}{}{s.f.}{Correspondência em tamanho de partes em lados opostos de um corpo ou de uma figura.}{si.me.tri.a}{0}
\verb{simétrico}{}{}{}{}{adj.}{Relativo a simetria.}{si.mé.tri.co}{0}
\verb{simétrico}{}{}{}{}{}{Que tem simetria; regular, harmonioso.}{si.mé.tri.co}{0}
\verb{simétrico}{}{Geom.}{}{}{}{Diz"-se de figuras cujos pontos estão todos, dois a dois, a igual distância de um ponto, de uma linha ou de um plano.}{si.mé.tri.co}{0}
\verb{simiesco}{ê}{}{}{}{adj.}{Relativo ou semelhante a símio ou macaco.}{si.mi.es.co}{0}
\verb{símil}{}{}{"-eis}{}{adj.2g.}{Que se assemelha; semelhante, similar.}{sí.mil}{0}
\verb{similar}{}{}{}{}{s.m.}{Artigo ou produto semelhante a outro.}{si.mi.lar}{0}
\verb{símile}{}{}{}{}{s.m.}{Qualidade do que é semelhante, similar.}{sí.mi.le}{0}
\verb{símile}{}{}{}{}{}{Comparação de coisas que são semelhantes entre si; similitude.}{sí.mi.le}{0}
\verb{similitude}{}{}{}{}{s.f.}{Ver \textit{semelhança}.}{si.mi.li.tu.de}{0}
\verb{símio}{}{}{}{}{s.m.}{Ver \textit{macaco}.}{sí.mio}{0}
\verb{simonia}{}{Relig.}{}{}{s.f.}{Compra ou venda ilícita de bens espirituais ou objetos sagrados.}{si.mo.ni.a}{0}
\verb{simpatia}{}{}{}{}{s.f.}{Simpatia de agrado que pessoa ou coisa desperta em alguém.}{sim.pa.ti.a}{0}
\verb{simpatia}{}{}{}{}{}{Ritual para prevenir ou curar enfermidade ou mal"-estar.  }{sim.pa.ti.a}{0}
\verb{simpático}{}{}{}{}{adj.}{Relativo a simpatia.}{sim.pá.ti.co}{0}
\verb{simpático}{}{}{}{}{}{Que inspira simpatia; agradável, aprazível.}{sim.pá.ti.co}{0}
\verb{simpático}{}{Anat.}{}{}{}{Diz"-se de um dos setores do sistema nervoso autônomo responsável pela regulação da vida vegetativa dos órgãos.}{sim.pá.ti.co}{0}
\verb{simpatizante}{}{}{}{}{adj.2g.}{Que manifesta simpatia por um partido, uma doutrina, uma causa etc.}{sim.pa.ti.zan.te}{0}
\verb{simpatizar}{}{}{}{}{v.t.}{Sentir afeição, interesse ou inclinação por algo ou alguém.}{sim.pa.ti.zar}{\verboinum{1}}
\verb{simples}{}{}{}{}{adj.}{Que é composto de apenas uma substância ou elemento; puro, sem mistura.}{sim.ples}{0}
\verb{simples}{}{}{}{}{}{Sem ornamentação, nem enfeites; singelo.}{sim.ples}{0}
\verb{simples}{}{}{}{}{}{Que se compreende com facilidade.}{sim.ples}{0}
\verb{simples}{}{}{}{}{}{Sem maldade nem malícia; inocente, puro.}{sim.ples}{0}
\verb{simples}{}{}{}{}{}{Modesto, pobre, humilde.}{sim.ples}{0}
\verb{simplicidade}{}{}{}{}{s.f.}{Qualidade do que é simples, que não apresenta dificuldade ou complicação.}{sim.pli.ci.da.de}{0}
\verb{simplicidade}{}{}{}{}{}{Comportamento natural; espontaneidade.}{sim.pli.ci.da.de}{0}
\verb{simplicidade}{}{}{}{}{}{Qualidade do que é ingênuo; pureza, singeleza.}{sim.pli.ci.da.de}{0}
\verb{simplicidade}{}{}{}{}{}{Ausência de luxo, sofisticação.}{sim.pli.ci.da.de}{0}
\verb{simplificar}{}{}{}{}{v.t.}{Tornar simples, menos complexo.}{sim.pli.fi.car}{0}
\verb{simplificar}{}{}{}{}{}{Tornar mais claro, de fácil compreensão.}{sim.pli.fi.car}{0}
\verb{simplificar}{}{}{}{}{}{Reduzir a termos menores ou mais precisos.}{sim.pli.fi.car}{\verboinum{2}}
\verb{simplismo}{}{}{}{}{s.m.}{Prática que consiste em desprezar elementos necessários à solução.}{sim.plis.mo}{0}
\verb{simplismo}{}{}{}{}{}{Emprego de meios ou processos exageradamente simples.}{sim.plis.mo}{0}
\verb{simplório}{}{}{}{}{adj.}{Que acredita nas coisas com muita facilidade; tolo, ingênuo, crédulo.}{sim.pló.rio}{0}
\verb{simpósio}{}{}{}{}{}{Na antiga Grécia, segunda parte de um banquete, durante a qual os convidados bebiam e participavam de jogos.}{sim.pó.sio}{0}
\verb{simpósio}{}{}{}{}{s.m.}{Reunião técnica ou científica na qual os participantes expõem e debatem temas específicos; congresso, seminário.}{sim.pó.sio}{0}
\verb{simulação}{}{}{"-ões}{}{s.f.}{Ato ou efeito de simular; fingimento, simulacro, disfarce.}{si.mu.la.ção}{0}
\verb{simulação}{}{}{"-ões}{}{}{Experiência ou ensaio do funcionamento de um processo por meio do funcionamento de outro similar. }{si.mu.la.ção}{0}
\verb{simulacro}{}{}{}{}{s.m.}{Representação ou imagem de pessoa ou divindade pagã.}{si.mu.la.cro}{0}
\verb{simulacro}{}{}{}{}{}{Reprodução imperfeita; falsificação, imitação.}{si.mu.la.cro}{0}
\verb{simulacro}{}{}{}{}{}{Fingimento, disfarce, simulação.}{si.mu.la.cro}{0}
\verb{simulado}{}{}{}{}{adj.}{Que não é verdadeiro; fingido, suposto.}{si.mu.la.do}{0}
\verb{simulado}{}{}{}{}{}{Feito à imitação de algo real.}{si.mu.la.do}{0}
\verb{simulador}{ô}{}{}{}{adj.}{Que simula, finge, imita.}{si.mu.la.dor}{0}
\verb{simulador}{ô}{}{}{}{}{Diz"-se de aparelho capaz de reproduzir o funcionamento de outro aparelho para que se possa estudá"-lo.}{si.mu.la.dor}{0}
\verb{simular}{}{}{}{}{v.t.}{Fazer parecer real o que não é; fingir, representar.}{si.mu.lar}{0}
\verb{simular}{}{}{}{}{}{Disfarçar, dissimular, aparentar.}{si.mu.lar}{\verboinum{1}}
\verb{simultaneidade}{}{}{}{}{s.f.}{Qualidade de simultâneo.}{si.mul.ta.nei.da.de}{0}
\verb{simultâneo}{}{}{}{}{adj.}{Diz"-se de uma coisa que acontece ao mesmo tempo que outra; concomitante.}{si.mul.tâ.neo}{0}
\verb{sina}{}{}{}{}{s.f.}{Destino, sorte, fado.}{si.na}{0}
\verb{sinagoga}{ó}{}{}{}{s.f.}{Templo judaico.}{si.na.go.ga}{0}
\verb{sinal}{}{}{"-ais}{}{s.m.}{Aquilo que serve de advertência, ou que possibilita conhecer, reconhecer ou prever alguma coisa.}{si.nal}{0}
\verb{sinal}{}{}{"-ais}{}{}{Marca deixada em algum lugar; indício, rastro.}{si.nal}{0}
\verb{sinal}{}{}{"-ais}{}{}{Marca de uma operação matemática; símbolo.}{si.nal}{0}
\verb{sinal}{}{}{"-ais}{}{}{Aparelho instalado nas ruas ou cruzamentos para dar, manual ou automaticamente, sinais luminosos reguladores do trânsito; semáforo, farol, sinaleira, sinaleiro.}{si.nal}{0}
\verb{sinal"-da"-cruz}{}{Relig.}{sinais"-da"-cruz}{}{s.m.}{Gesto ou ato litúrgico em que se traça com os dedos uma cruz, tocando a testa, o peito e cada um dos ombros, pronunciando as palavras "em nome do Pai, do Filho, do Espírito Santo". }{si.nal"-da"-cruz}{0}
\verb{sinaleira}{ê}{}{}{}{s.f.}{Aparelho instalado nas ruas ou cruzamentos para dar, manual ou automaticamente, sinais luminosos reguladores do trânsito; semáforo, farol, sinal, sinaleiro.}{si.na.lei.ra}{0}
\verb{sinaleiro}{ê}{}{}{}{s.m.}{Indivíduo incumbido de dar sinais a bordo.}{si.na.lei.ro}{0}
\verb{sinaleiro}{ê}{}{}{}{}{Indivíduo que nas estações de estradas de ferro faz sinais aos trens para avisar que a linha se encontra desimpedida.}{si.na.lei.ro}{0}
\verb{sinaleiro}{ê}{}{}{}{}{Aparelho instalado nas ruas ou cruzamentos para dar, manual ou automaticamente, sinais luminosos reguladores do trânsito; semáforo, farol, sinal, sinaleira.}{si.na.lei.ro}{0}
\verb{sinalização}{}{}{"-ões}{}{s.f.}{Ato ou efeito de sinalizar.}{si.na.li.za.ção}{0}
\verb{sinalização}{}{}{"-ões}{}{}{Sistema de sinais de tráfego usado nas cidades, nas estradas de ferro e de rodagem etc.}{si.na.li.za.ção}{0}
\verb{sinalização}{}{}{"-ões}{}{}{Indicação ou advertência destinada a orientar motoristas.}{si.na.li.za.ção}{0}
\verb{sinalizar}{}{}{}{}{v.i.}{Exercer as funções de sinaleiro.}{si.na.li.zar}{0}
\verb{sinalizar}{}{}{}{}{v.t.}{Marcar com sinais.}{si.na.li.zar}{0}
\verb{sinalizar}{}{}{}{}{}{Pôr sinalização.}{si.na.li.zar}{\verboinum{1}}
\verb{sinapismo}{}{}{}{}{s.m.}{Emplastro, cataplasma que se coloca sobre afecções cutâneas.}{si.na.pis.mo}{0}
\verb{sinceridade}{}{}{}{}{s.f.}{Qualidade daquele que não mente quanto às suas intenções, pensamentos e sentimentos; franqueza.}{sin.ce.ri.da.de}{0}
\verb{sincero}{é}{}{}{}{adj.}{Diz"-se daquele que não dissimula seus pensamentos, sentimentos ou intenções; franco, verdadeiro.}{sin.ce.ro}{0}
\verb{sincopado}{}{Gram.}{}{}{adj.}{Diz"-se do vocábulo em que houve síncope.}{sin.co.pa.do}{0}
\verb{sincopado}{}{Mús.}{}{}{}{Diz"-se do som que vem no contratempo; acentuado. }{sin.co.pa.do}{0}
\verb{sincopar}{}{Gram.}{}{}{v.t.}{Suprimir fonema(s) no interior de um vocábulo.}{sin.co.par}{0}
\verb{sincopar}{}{Mús.}{}{}{}{Introduzir síncopes, contratempos na pulsação de uma música.}{sin.co.par}{\verboinum{1}}
\verb{síncope}{}{Gram.}{}{}{s.f.}{Supressão de fonema(s) no interior de um vocábulo.}{sín.co.pe}{0}
\verb{síncope}{}{Med.}{}{}{}{Perda dos sentidos; desmaio, delíquio.}{sín.co.pe}{0}
\verb{síncope}{}{Mús.}{}{}{}{Som acentuado na parte fraca do tempo ou compasso; contratempo.}{sín.co.pe}{0}
\verb{sincrético}{}{}{}{}{adj.}{Diz"-se do que é resultado da fusão de várias religiões, ideias, filosofias ou cosmovisões.}{sin.cré.ti.co}{0}
\verb{sincretismo}{}{}{}{}{s.m.}{Fusão de cultos ou religiões em que seus elementos são reinterpretados ou transformados.}{sin.cre.tis.mo}{0}
\verb{sincretismo}{}{Por ext.}{}{}{}{Fusão de elementos culturais ou de culturas distintas.}{sin.cre.tis.mo}{0}
%\verb{}{}{}{}{}{}{}{}{0}
\verb{sincronia}{}{}{}{}{s.f.}{Ato ou efeito de sincronizar.}{sin.cro.ni.a}{0}
\verb{sincronia}{}{}{}{}{}{Ocorrência simultânea de um evento qualquer.}{sin.cro.ni.a}{0}
\verb{sincrônico}{}{}{}{}{adj.}{Que acontece ao mesmo tempo; simultâneo.}{sin.crô.ni.co}{0}
\verb{sincronismo}{}{}{}{}{s.m.}{Estado ou condição em que dois ou mais acontecimentos se dão dentro do mesmo limite de tempo; sincronização.}{sin.cro.nis.mo}{0}
\verb{sincronização}{}{}{"-ões}{}{s.f.}{Ver \textit{sincronismo}. }{sin.cro.ni.za.ção}{0}
\verb{sincronizar}{}{}{}{}{v.t.}{Tornar sincrônico, simultâneo.}{sin.cro.ni.zar}{0}
\verb{sincronizar}{}{}{}{}{}{Fazer a sincronização de uma peça audiovisual, combinando som e movimento com precisão.}{sin.cro.ni.zar}{\verboinum{1}}
\verb{síncrono}{}{}{}{}{adj.}{Diz"-se dos eventos simultâneos; sincrônico.}{sín.cro.no}{0}
\verb{sindético}{}{Gram.}{}{}{adj.}{Relativo a síndeto; em que há conjunção coordenativa.}{sin.dé.ti.co}{0}
\verb{síndeto}{}{Gram.}{}{}{s.m.}{Ocorrência de conjunção coordenativa entre palavras, termos da oração ou em orações coordenadas. }{sín.de.to}{0}
\verb{sindical}{}{}{"-ais}{}{adj.2g.}{Relativo a sindicato.}{sin.di.cal}{0}
\verb{sindicalismo}{}{}{}{}{s.m.}{Movimento cujo objetivo é congregar as classes profissionais para discutir e defender seus interesses.}{sin.di.ca.lis.mo}{0}
\verb{sindicalismo}{}{}{}{}{}{Ação reivindicatória, política ou social de sindicatos.}{sin.di.ca.lis.mo}{0}
\verb{sindicalista}{}{}{}{}{adj.2g.}{Relativo a sindicato ou sindicalismo.}{sin.di.ca.lis.ta}{0}
\verb{sindicalista}{}{}{}{}{s.2g.}{Indivíduo que é partidário do sindicalismo ou pertence a um sindicato.}{sin.di.ca.lis.ta}{0}
\verb{sindicalizar}{}{}{}{}{v.t.}{Organizar, reunir em sindicato.}{sin.di.ca.li.zar}{0}
\verb{sindicalizar}{}{}{}{}{v.pron.}{Afiliar"-se a um sindicato.}{sin.di.ca.li.zar}{\verboinum{1}}
\verb{sindicância}{}{}{}{}{s.f.}{Conjunto de medidas para a avaliação e verificação de fatos, procedimentos, serviços etc., a fim de apurar a verdade; investigação.}{sin.di.cân.cia}{0}
\verb{sindicar}{}{}{}{}{v.t.}{Realizar, promover sindicância; inquirir, investigar.}{sin.di.car}{0}
\verb{sindicar}{}{}{}{}{}{Organizar ou reunir em sindicato; sindicalizar.}{sin.di.car}{\verboinum{1}}
\verb{sindicato}{}{}{}{}{s.m.}{Associação, congregação de uma classe de profissionais para a discussão e defesa de seus interesses, sobretudo os trabalhistas.}{sin.di.ca.to}{0}
\verb{síndico}{}{}{}{}{s.m.}{Indivíduo que faz sindicância; sindicante.}{sín.di.co}{0}
\verb{síndico}{}{Jur.}{}{}{}{Indivíduo que administra a falência, escolhido pelo juiz dentre os credores do falido.}{sín.di.co}{0}
\verb{síndico}{}{Bras.}{}{}{}{Morador encarregado da administração de um prédio de apartamentos, eleito pelos condôminos. }{sín.di.co}{0}
\verb{síndrome}{}{Med.}{}{}{s.f.}{Conjunto de sintomas de uma patologia cujas causas podem ser diversas ou desconhecidas.}{sín.dro.me}{0}
\verb{síndrome}{}{Fig.}{}{}{}{Conjunto de sinais ou características passíveis de despertar medo, insegurança, pânico etc., manifestado em situações anormais ou extremas. \textit{abon}}{sín.dro.me}{0}
\verb{sinecura}{}{}{}{}{s.f.}{Emprego cuja remuneração é muito alta com relação ao trabalho exercido; prebenda.}{si.ne.cu.ra}{0}
\verb{sine die}{}{}{}{}{loc. adv.}{Expressão latina que significa \textit{sem dia marcado}.}{\textit{sine die}}{0}
\verb{sinédoque}{}{Gram.}{}{}{s.f.}{Tipo de metonímia baseada na relação quantitativa entre o significado original da palavra e o seu conteúdo. }{si.né.do.que}{0}
\verb{sinédrio}{}{}{}{}{s.m.}{Tribunal, assembleia dos antigos judeus formada pelos anciãos da classe dominante.}{si.né.drio}{0}
\verb{sineiro}{ê}{}{}{}{adj.}{Que possui sino(s).}{si.nei.ro}{0}
\verb{sineiro}{ê}{}{}{}{s.m.}{Indivíduo encarregado de tocar o sino.}{si.nei.ro}{0}
\verb{sinergia}{}{Med.}{}{}{s.f.}{Associação de vários órgãos para a execução de movimento ou função orgânica.}{si.ner.gi.a}{0}
\verb{sinergia}{}{Quím.}{}{}{}{Potencialização da ação ou efeito de uma ou mais substâncias químicas ou farmacológicas pela associação de princípios ativos distintos. }{si.ner.gi.a}{0}
\verb{sineta}{ê}{}{}{}{s.f.}{Sino pequeno; campainha.}{si.ne.ta}{0}
\verb{sinete}{ê}{}{}{}{s.m.}{Utensílio para gravar em baixo ou alto"-relevo a assinatura, monograma ou divisa de quem o usa; chancela, carimbo.}{si.ne.te}{0}
\verb{sinfonia}{}{Mús.}{}{}{s.f.}{Tipo de composição para grande orquestra, em forma de sonata, caracterizada pela multiplicidade de executantes e pela diversidade de instrumentos musicais. }{sin.fo.ni.a}{0}
\verb{sinfônica}{}{}{}{}{s.f.}{A orquestra sinfônica.}{sin.fô.ni.ca}{0}
\verb{sinfônico}{}{}{}{}{}{Diz"-se de peça musical cuja execução é destinada a vários instrumentos. }{sin.fô.ni.co}{0}
\verb{sinfônico}{}{}{}{}{adj.}{Relativo a ou próprio de sinfonia.}{sin.fô.ni.co}{0}
\verb{singeleza}{ê}{}{}{}{s.f.}{Qualidade ou condição do que é singelo, simples, natural; simplicidade.}{sin.ge.le.za}{0}
\verb{singelo}{é}{}{}{}{adj.}{Que é puro, não corrompido, inocente.}{sin.ge.lo}{0}
\verb{singelo}{é}{}{}{}{}{Simples, sem ornatos, enfeites ou sofisticação.}{sin.ge.lo}{0}
\verb{singrar}{}{}{}{}{v.i.}{Navegar, velejar.}{sin.grar}{0}
\verb{singrar}{}{}{}{}{v.t.}{Percorrer navegando, velejando.}{sin.grar}{\verboinum{1}}
\verb{singular}{}{}{}{}{adj.2g.}{Relativo a um apenas; único, individual.}{sin.gu.lar}{0}
\verb{singular}{}{}{}{}{}{Invulgar, extraordinário, raro, excêntrico.}{sin.gu.lar}{0}
\verb{singular}{}{Gram.}{}{}{s.m.}{Diz"-se do número que indica uma pessoa ou coisa apenas.}{sin.gu.lar}{0}
\verb{singularidade}{}{}{}{}{s.f.}{Qualidade ou condição do que é singular.}{sin.gu.la.ri.da.de}{0}
\verb{singularidade}{}{}{}{}{}{Particularidade, peculiaridade, excentricidade.}{sin.gu.la.ri.da.de}{0}
\verb{singularizar}{}{}{}{}{v.t.}{Tornar singular, único; distinguir, particularizar.}{sin.gu.la.ri.zar}{\verboinum{1}}
\verb{sinhá}{}{Bras.}{}{}{s.f.}{Forma de tratamento com que os escravos designavam a senhora ou patroa. }{si.nhá}{0}
\verb{sinhá"-moça}{ô}{Bras.}{sinhás"-moças ⟨ô⟩}{}{s.f.}{Forma de tratamento com que os escravos se referiam à filha do senhor ou patrão, ou a uma donzela; sinhazinha. }{si.nhá"-mo.ça}{0}
\verb{sinhazinha}{}{}{}{}{s.f.}{Ver \textit{sinhá"-moça}.}{si.nha.zi.nha}{0}
\verb{sinhô}{}{Bras.}{}{}{s.m.}{Forma de tratamento que os escravos usavam para o senhor ou patrão.}{si.nhô}{0}
\verb{sinhô"-moço}{ô}{Bras.}{sinhôs"-moços ⟨ô⟩}{}{s.m.}{Forma de tratamento com que os escravos se referiam ao filho do senhor ou patrão.}{si.nhô"-mo.ço}{0}
\verb{sinhozinho}{}{}{}{}{s.m.}{Ver \textit{sinhô"-moço}.}{si.nho.zi.nho}{0}
\verb{sinistra}{}{}{}{}{s.f.}{Mão esquerda; canhota, sestra.}{si.nis.tra}{0}
\verb{sinistrado}{}{}{}{}{adj.}{Que sofreu sinistro.}{si.nis.tra.do}{0}
\verb{sinistrar}{}{}{}{}{v.i.}{Sofrer desastre ou sinistro de algo assegurado. }{si.nis.trar}{\verboinum{1}}
\verb{sinistro}{}{}{}{}{adj.}{Esquerdo; canhoto.}{si.nis.tro}{0}
\verb{sinistro}{}{}{}{}{}{Agourento; funesto.}{si.nis.tro}{0}
\verb{sinistro}{}{}{}{}{s.m.}{Qualquer acontecimento danoso; acidente, desastre, morte.}{si.nis.tro}{0}
\verb{sinistro}{}{}{}{}{}{Dano de um bem assegurado.}{si.nis.tro}{0}
\verb{sino}{}{}{}{}{s.m.}{Instrumento em forma de campânula, oco, geralmente de bronze, que se percute com badalo ou martelo.}{si.no}{0}
%\verb{}{}{}{}{}{}{}{}{0}
\verb{sinódico}{}{}{}{}{adj.}{Relativo a sínodo.}{si.nó.di.co}{0}
\verb{sínodo}{}{Relig.}{}{}{s.m.}{Assembleia de bispos de todo o mundo, presidida pelo papa, para tratar de assuntos referentes à Igreja universal. }{sí.no.do}{0}
\verb{sinologia}{}{}{}{}{s.f.}{Ciência que estuda todos os aspectos da cultura e sociedade chinesas.}{si.no.lo.gi.a}{0}
\verb{sinólogo}{}{}{}{}{s.m.}{Indivíduo que estuda a cultura e a sociedade chinesas.}{si.nó.lo.go}{0}
\verb{sinonímia}{}{Gram.}{}{}{s.f.}{Equivalência de significado entre vocábulos.}{si.no.ní.mia}{0}
\verb{sinônimo}{}{Gram.}{}{}{s.m.}{Vocábulo cujo significado é idêntico ou muito próximo do de outro.}{si.nô.ni.mo}{0}
\verb{sinopse}{ó}{}{}{}{s.f.}{Relato breve; resumo, síntese, sumário.}{si.nop.se}{0}
\verb{sinóptico}{}{}{}{}{adj.}{Relativo a sinopse.}{si.nóp.ti.co}{0}
\verb{sinótico}{}{}{}{}{}{Var. de \textit{sinóptico}.}{si.nó.ti.co}{0}
%\verb{}{}{}{}{}{}{}{}{0}
\verb{sintático}{}{Gram.}{}{}{adj.}{Relativo ou pertencente à sintaxe.}{sin.tá.ti.co}{0}
\verb{sintaxe}{s}{Gram.}{}{}{s.f.}{Parte da gramática que estuda a construção frasal, determinando as relações de concordância, de subordinação e de ordem na organização do discurso. }{sin.ta.xe}{0}
\verb{sintáxico}{s}{}{}{}{}{Var. de \textit{sintático}.}{sin.tá.xi.co}{0}
\verb{sinteco}{é}{}{}{}{s.m.}{Verniz transparente usado para revestir assoalhos de madeira.}{sin.te.co}{0}
\verb{síntese}{}{}{}{}{s.f.}{Composição de um todo pela reunião de suas partes.}{sín.te.se}{0}
\verb{síntese}{}{}{}{}{}{Exposição resumida; sumário, resumo.}{sín.te.se}{0}
\verb{síntese}{}{Quím.}{}{}{}{Processo de obtenção de um composto complexo combinando"-se elementos simples. }{sín.te.se}{0}
\verb{sintético}{}{}{}{}{adj.}{Relativo ou pertencente a síntese.}{sin.té.ti.co}{0}
\verb{sintético}{}{Quím.}{}{}{}{Diz"-se dos produtos obtidos por síntese química, que reproduzem a composição e as propriedades de substâncias ou compostos encontrados na natureza.}{sin.té.ti.co}{0}
\verb{sintetizador}{ô}{}{}{}{adj.}{Relativo a síntese.}{sin.te.ti.za.dor}{0}
\verb{sintetizador}{ô}{Mús.}{}{}{s.m.}{Instrumento ou dispositivo eletrônico que sintetiza as características sonoras de instrumento ou voz. }{sin.te.ti.za.dor}{0}
\verb{sintetizar}{}{}{}{}{v.t.}{Tornar sintético; resumir, sumarizar.}{sin.te.ti.zar}{0}
\verb{sintetizar}{}{Bioquím.}{}{}{}{Produzir substância ou composto por síntese natural ou artificial.}{sin.te.ti.zar}{0}
\verb{sintetizar}{}{Mús.}{}{}{}{Produzir um som por meio de síntese.}{sin.te.ti.zar}{\verboinum{1}}
\verb{sintoma}{}{Med.}{}{}{s.m.}{Fenômeno, manifestação a partir da qual se pode diagnosticar uma enfermidade.}{sin.to.ma}{0}
\verb{sintoma}{}{Por ext.}{}{}{}{Sinal, indício revelador da existência de algo. }{sin.to.ma}{0}
\verb{sintomático}{}{}{}{}{adj.}{Relativo a sintoma.}{sin.to.má.ti.co}{0}
\verb{sintomático}{}{Por ext.}{}{}{}{Diz"-se do que indica um determinado estado de coisas; revelador, significativo.}{sin.to.má.ti.co}{0}
\verb{sintomatologia}{}{Med.}{}{}{s.f.}{Estudo e interpretação do conjunto de sintomas das patologias.}{sin.to.ma.to.lo.gi.a}{0}
\verb{sintonia}{}{Fís.}{}{}{s.f.}{Estado de um sistema receptor de rádio em ressonância.}{sin.to.ni.a}{0}
\verb{sintonia}{}{Fig.}{}{}{}{Acordo no pensar e no sentir; harmonia, reciprocidade, sintonização. }{sin.to.ni.a}{0}
\verb{sintonizar}{}{}{}{}{v.t.}{Ajustar um aparelho receptor ao comprimento da onda do aparelho transmissor.}{sin.to.ni.zar}{0}
\verb{sintonizar}{}{Fig.}{}{}{}{Acordar as maneiras de pensar e agir; afinar, combinar, ajustar.}{sin.to.ni.zar}{\verboinum{1}}
\verb{sinuca}{}{}{}{}{s.f.}{Modalidade de bilhar que se joga com oito bolas de cores distintas numa mesa de seis caçapas.}{si.nu.ca}{0}
\verb{sinuca}{}{Fig.}{}{}{}{Situação embaraçosa, de difícil resolução; impasse.}{si.nu.ca}{0}
\verb{sinuosidade}{}{}{}{}{s.f.}{Qualidade ou condição do que é sinuoso; tortuosidade, rodeio, ziguezague.}{si.nu.o.si.da.de}{0}
\verb{sinuoso}{ô}{}{"-osos ⟨ó⟩}{"-osa ⟨ó⟩}{adj.}{Diz"-se do que não segue linha ou caminho reto; cheio de curvas, tortuoso, ondulante.}{si.nu.o.so}{0}
\verb{sinusite}{}{Med.}{}{}{s.f.}{Inflamação dos seios da face.}{si.nu.si.te}{0}
\verb{sionismo}{}{}{}{}{s.m.}{Conjunto de estudos e conhecimentos sobre Jerusalém.}{si.o.nis.mo}{0}
\verb{sionismo}{}{}{}{}{}{Movimento político que visa ao estabelecimento e consolidação de um Estado judeu na Palestina.}{si.o.nis.mo}{0}
\verb{sionista}{}{}{}{}{adj.2g.}{Relativo ao sionismo.}{si.o.nis.ta}{0}
\verb{sionista}{}{}{}{}{s.2g.}{Partidário ou simpatizante do sionismo.}{si.o.nis.ta}{0}
\verb{sirena}{}{}{}{}{}{Var. de \textit{sereia}.}{si.re.na}{0}
\verb{sirena}{}{}{}{}{}{Var. de \textit{sirene}.}{si.re.na}{0}
\verb{sirene}{}{}{}{}{s.f.}{Aparelho sonoro usado para dar alarmes, avisos etc.}{si.re.ne}{0}
\verb{sirene}{}{}{}{}{}{Var. de \textit{sereia}.}{si.re.ne}{0}
\verb{siri}{}{Zool.}{}{}{s.m.}{Nome comum aos crustáceos, marinhos, que se distinguem dos caranguejos pelo último par de patas, em forma de remo, adaptadas para nadar.}{si.ri}{0}
%\verb{}{}{}{}{}{}{}{}{0}
%\verb{}{}{}{}{}{}{}{}{0}
\verb{sirigaita}{}{Pop.}{}{}{s.f.}{Mulher buliçosa, que se vale de trejeitos para seduzir ou atrair.}{si.ri.gai.ta}{0}
\verb{sirigaita}{}{}{}{}{}{Mulher muito desembaraçada, com resposta para tudo.}{si.ri.gai.ta}{0}
\verb{sírio}{}{}{}{}{adj.}{Relativo à Síria (Ásia).}{sí.rio}{0}
\verb{sírio}{}{}{}{}{s.m.}{Indivíduo natural ou habitante desse país.}{sí.rio}{0}
\verb{sírio}{}{}{}{}{}{O dialeto árabe falado na Síria.}{sí.rio}{0}
\verb{sírio}{}{Astron.}{}{}{}{A estrela mais brilhante vista da Terra e a maior da constelação Cão Maior.}{sí.rio}{0}
\verb{siriri}{}{Zool.}{}{}{s.m.}{Denominação comum à forma alada de diversos insetos, como o cupim, quando saem do ninho para acasalar; aleluia.}{si.ri.ri}{0}
\verb{siriri}{}{Zool.}{}{}{s.m.}{Molusco bivalve comestível; sururu.}{si.ri.ri}{0}
\verb{siroco}{ô}{}{}{}{s.m.}{Vento quente e seco que sopra sobre o mar Mediterrâneo, de origem saariana.}{si.ro.co}{0}
\verb{sisal}{}{}{"-ais}{}{s.m.}{Nome comum a algumas plantas de cujas folhas se obtém fibra têxtil.}{si.sal}{0}
\verb{sisal}{}{}{"-ais}{}{}{A fibra extraída dessas plantas, usada na confecção de cordas, tapetes etc.}{si.sal}{0}
\verb{sísmico}{}{}{}{}{adj.}{Relativo a sismo ou terremoto.  }{sís.mi.co}{0}
\verb{sismo}{}{Geol.}{}{}{s.m.}{Movimento, tremor de terra; abalo sísmico; terremoto.  }{sis.mo}{0}
\verb{sismografia}{}{Geol.}{}{}{s.f.}{Registro gráfico de abalo sísmico, com intensidade, duração e direção dos tremores. }{sis.mo.gra.fi.a}{0}
\verb{sismógrafo}{}{Geol.}{}{}{s.m.}{Aparelho usado para detectar e registrar as vibrações da Terra.}{sis.mó.gra.fo}{0}
\verb{siso}{}{}{}{}{s.m.}{Qualidade de sensato; prudência, tino, juízo. }{si.so}{0}
\verb{siso}{}{Anat.}{}{}{s.m.}{Cada um dos terceiros dentes molares.}{si.so}{0}
\verb{sistema}{}{}{}{}{s.m.}{Conjunto de procedimentos que devem ser seguidos para se fazer alguma coisa. (\textit{O técnico experimentou um sistema de treinamento para os jogadores.})}{sis.te.ma}{0}
\verb{sistema}{}{}{}{}{}{Conjunto de elementos que possuem relações uns com os outros. (\textit{Ainda não se tem certeza da existência de outros sistemas como o sistema solar.})}{sis.te.ma}{0}
\verb{sistema}{}{Anat.}{}{}{}{Conjunto de órgãos que têm uma função essencial à vida. (\textit{O sistema nervoso do corpo humano.})}{sis.te.ma}{0}
\verb{sistemática}{}{}{}{}{s.f.}{Conjunto de elementos classificados e organizados segundo determinados critérios.}{sis.te.má.ti.ca}{0}
\verb{sistemática}{}{Biol.}{}{}{}{Ciência que classifica os seres vivos; taxionomia.}{sis.te.má.ti.ca}{0}
\verb{sistemático}{}{}{}{}{adj.}{Relativo a sistema.}{sis.te.má.ti.co}{0}
\verb{sistemático}{}{Por ext.}{}{}{}{Ordenado; metódico; organizado.}{sis.te.má.ti.co}{0}
\verb{sistemático}{}{Bras.}{}{}{}{Diz"-se do indivíduo cujas atitudes são demasiado meticulosas; excêntrico.}{sis.te.má.ti.co}{0}
\verb{sistematização}{}{}{ões}{}{s.f.}{Ato ou efeito de sistematizar.}{sis.te.ma.ti.za.ção}{0}
\verb{sistematizar}{}{}{}{}{v.t.}{Dispor ou organizar em um sistema.}{sis.te.ma.ti.zar}{0}
\verb{sistematizar}{}{}{}{}{}{Tornar sistemático, metódico, organizado.}{sis.te.ma.ti.zar}{\verboinum{1}}
\verb{sistina}{}{}{}{}{adj.}{Diz"-se da capela do Vaticano construída sob a égide do papa Sisto \textsc{iv}.}{sis.ti.na}{0}
\verb{sístole}{}{Med.}{}{}{s.f.}{Parte do ciclo cardíaco caracterizada pela contração rítmica, por meio da qual o sangue é transportado para a aorta e para a artéria pulmonar. }{sís.to.le}{0}
\verb{sístole}{}{Gram.}{}{}{}{Deslocamento do acento tônico de uma palavra para a sílaba anterior.}{sís.to.le}{0}
\verb{sisudez}{ê}{}{}{}{s.f.}{Qualidade ou caráter de sisudo;        seriedade, circunspecção.}{si.su.dez}{0}
\verb{sisudo}{}{}{}{}{adj.}{Que tem siso, bom"-senso, juízo. }{si.su.do}{0}
\verb{sisudo}{}{}{}{}{}{Diz"-se daquele que é muito sério; carrancudo.}{si.su.do}{0}
\verb{site}{}{Informát.}{}{}{s.m.}{Conjunto de documentos apresentados por um indivíduo, instituição etc., publicado na \textit{internet}. }{\textit{site}}{0}
\verb{sitiante}{}{}{}{}{adj.2g.}{Que sitia praça, tropa etc.; assediador.}{si.ti.an.te}{0}
\verb{sitiante}{}{}{}{}{}{}{si.ti.an.te}{0}
\verb{sitiar}{}{}{}{}{v.t.}{Cercar, rodear de tropas para atacar; assediar, coagir.}{si.ti.ar}{0}
\verb{sitiar}{}{Fig.}{}{}{}{Cercar, pressionar. }{si.ti.ar}{\verboinum{1}}
\verb{sítio}{}{}{}{}{s.m.}{Ato ou efeito de sitiar; assédio, cerco.}{sí.tio}{0}
\verb{sítio}{}{}{}{}{}{Qualquer lugar, localidade ou povoação.}{sí.tio}{0}
\verb{sítio}{}{Bras.}{}{}{}{Propriedade rural nas redondezas de uma cidade.}{sí.tio}{0}
\verb{sito}{}{}{}{}{adj.}{Situado; localizado.}{si.to}{0}
\verb{situação}{}{}{"-ões}{}{s.f.}{Ato ou efeito de situar.}{si.tu.a.ção}{0}
\verb{situação}{}{}{"-ões}{}{}{O modo como alguma coisa ou pessoa está situada em relação a determinado ambiente; posição, localização.}{si.tu.a.ção}{0}
\verb{situação}{}{Fig.}{"-ões}{}{}{Combinação ou concorrência de acontecimentos ou circunstâncias num dado momento; conjuntura.}{si.tu.a.ção}{0}
\verb{situação}{}{Fig.}{"-ões}{}{}{Estado ou condição de caráter econômico, profissional, social, afetivo etc.; posição.}{si.tu.a.ção}{0}
\verb{situação}{}{Fig.}{"-ões}{}{}{Cada um dos momentos de uma ação real ou fictícia que provocam interesse ou emoção, ou concorrem para um determinado desenlace.}{si.tu.a.ção}{0}
\verb{situação}{}{}{"-ões}{}{}{Conjunto das forças ou outros elementos de caráter político ou social que se encontram no poder.}{si.tu.a.ção}{0}
\verb{situacionismo}{}{}{}{}{s.m.}{Partido político dos que se encontram no poder.}{si.tu.a.ci.o.nis.mo}{0}
\verb{situacionismo}{}{}{}{}{}{Permanência de uma situação política.}{si.tu.a.ci.o.nis.mo}{0}
\verb{situacionista}{}{}{}{}{adj.2g.}{Relativo ao situacionismo.}{si.tu.a.ci.o.nis.ta}{0}
\verb{situacionista}{}{}{}{}{s.2g.}{Indivíduo pertencente ao situacionismo.}{si.tu.a.ci.o.nis.ta}{0}
\verb{situar}{}{}{}{}{v.t.}{Colocar, pôr em determinado lugar; estabelecer.}{si.tu.ar}{0}
\verb{situar}{}{}{}{}{}{Estabelecer de forma permanente; construir, edificar.}{si.tu.ar}{0}
\verb{situar}{}{}{}{}{}{Determinar lugar certo; localizar.}{si.tu.ar}{0}
\verb{situar}{}{}{}{}{}{Tomar uma posição.}{si.tu.ar}{\verboinum{8}}
\verb{sizígia}{}{Astron.}{}{}{s.f.}{Conjunção ou oposição de um planeta, especialmente a Lua, com o Sol.}{si.zí.gia}{0}
\verb{skate}{}{}{}{}{s.m.}{Prancha de madeira sobre quatro rodinhas, na qual o esportista se equilibra, impulsionando"-a e direcionando"-a com os próprios pés.}{\textit{skate}}{0}
\verb{slide}{}{}{}{}{s.m.}{Foto positiva e transparente constituída por um quadro isolado de filme, em geral montado em moldura de plástico ou papelão e destinado a projeção fixa; diapositivo.}{\textit{slide}}{0}
\verb{slogan}{}{}{}{}{s.m.}{Expressão concisa, fácil de lembrar, utilizada em campanhas políticas, de publicidade, de propaganda, para lançar um produto, marca etc.}{\textit{slogan}}{0}
\verb{Sm}{}{Quím.}{}{}{}{Símb. do \textit{samário}. }{Sm}{0}
\verb{smoking}{}{}{}{}{s.m.}{Roupa masculina, com paletó geralmente preto, de lapelas de cetim, usada como traje de cerimônia à noite.}{\textit{smoking}}{0}
\verb{Sn}{}{Quím.}{}{}{}{Símb. do \textit{estanho}.}{Sn}{0}
\verb{SO}{}{}{}{}{s.m.}{Símbolo da região sudoeste.}{S.O.}{0}
\verb{só}{}{}{}{}{adj.}{Desacompanhado, solitário.}{só}{0}
\verb{só}{}{}{}{}{}{Que é apenas um; único.}{só}{0}
\verb{só}{}{}{}{}{}{Diz"-se daquele que se encontra afastado da convivência; isolado.}{só}{0}
\verb{só}{}{}{}{}{}{Que é ermo, deserto.}{só}{0}
\verb{só}{}{}{}{}{}{Diz"-se daquele que está desamparado; solitário.}{só}{0}
\verb{só}{}{}{}{}{s.m.}{Indivíduo que vive sem companhia.}{só}{0}
\verb{só}{}{}{}{}{adv.}{Apenas, somente, unicamente.}{só}{0}
\verb{soabrir}{}{}{}{}{v.t.}{Abrir um pouco, ligeiramente; entreabrir.}{so.a.brir}{\verboinum{18}}
\verb{soalhar}{}{}{}{}{}{Var. de \textit{assoalhar}.}{so.a.lhar}{0}
\verb{soalheira}{ê}{}{}{}{s.f.}{A hora de calor mais intenso; calor.}{so.a.lhei.ra}{0}
\verb{soalheira}{ê}{}{}{}{}{Exposição aos raios solares.}{so.a.lhei.ra}{0}
\verb{soalho}{}{}{}{}{s.m.}{Piso de madeira; assoalho.}{so.a.lho}{0}
\verb{soante}{}{}{}{}{adj.2g.}{Que soa, que emite ou produz som.}{so.an.te}{0}
\verb{soar}{}{}{}{}{v.t.}{Emitir, produzir som. (\textit{A campainha soava insistentemente.})}{so.ar}{0}
\verb{soar}{}{}{}{}{}{Ter determinada aceitação; cair, ficar, repercutir.  (\textit{A fala do deputado soou muito bem na ocasião.})}{so.ar}{\verboinum{7}}
\verb{sob}{ô}{}{}{}{prep.}{Debaixo de; por baixo de.}{sob}{0}
\verb{sob}{ô}{}{}{}{}{No tempo de.}{sob}{0}
\verb{sob}{ô}{}{}{}{}{Debaixo de autoridade de, de orientação de.}{sob}{0}
\verb{soba}{ó}{}{}{}{s.m.}{Chefe de povo ou de pequeno Estado africano, especialmente na costa ocidental, ao sul de Angola.}{so.ba}{0}
\verb{sobejar}{}{}{}{}{v.i.}{Ser demais; exceder os limites do necessário; sobrar.}{so.be.jar}{\verboinum{1}}
\verb{sobejo}{ê}{}{}{}{adj.}{Que sobeja; demasiado, excessivo.}{so.be.jo}{0}
\verb{sobejo}{ê}{}{}{}{}{Que é enorme, inumerável, imenso.}{so.be.jo}{0}
\verb{sobejo}{ê}{}{}{}{s.m.}{Sobra, resto.}{so.be.jo}{0}
\verb{soberana}{}{}{}{}{s.f.}{Mulher que exerce o poder soberano sobre um Estado; rainha, imperatriz.}{so.be.ra.na}{0}
\verb{soberana}{}{Fig.}{}{}{}{Mulher que, entre outras, ocupa o primeiro lugar.}{so.be.ra.na}{0}
\verb{soberania}{}{}{}{}{s.f.}{Qualidade de soberano.}{so.be.ra.ni.a}{0}
\verb{soberania}{}{}{}{}{}{Poder ou autoridade suprema de soberano.}{so.be.ra.ni.a}{0}
\verb{soberania}{}{}{}{}{}{Autoridade moral tida como suprema.}{so.be.ra.ni.a}{0}
\verb{soberania}{}{}{}{}{}{Propriedade que tem um Estado de ser uma ordem suprema que não deve a sua validade a nenhuma outra ordem superior.}{so.be.ra.ni.a}{0}
\verb{soberania}{}{Fig.}{}{}{}{Altivez, arrogância.}{so.be.ra.ni.a}{0}
\verb{soberano}{}{}{}{}{adj.}{Diz"-se daquele que exerce o poder supremo, sem restrição nem neutralização.}{so.be.ra.no}{0}
\verb{soberano}{}{Fig.}{}{}{}{Que é absoluto, supremo.}{so.be.ra.no}{0}
\verb{soberano}{}{}{}{}{}{Que é altivo, arrogante.}{so.be.ra.no}{0}
\verb{soberano}{}{Fig.}{}{}{s.m.}{Chefe do Estado monárquico; monarca.}{so.be.ra.no}{0}
\verb{soberano}{}{Fig.}{}{}{}{Indivíduo que influi poderosamente.}{so.be.ra.no}{0}
\verb{soberano}{}{}{}{}{}{A libra esterlina.}{so.be.ra.no}{0}
\verb{soberba}{ê}{}{}{}{s.f.}{Elevação ou altura de uma coisa em relação a outra.}{so.ber.ba}{0}
\verb{soberba}{ê}{}{}{}{}{Orgulho excessivo; altivez, arrogância, presunção.}{so.ber.ba}{0}
\verb{soberbia}{}{}{}{}{s.f.}{Qualidade do que é soberbo.}{so.ber.bi.a}{0}
\verb{soberbia}{}{}{}{}{}{Soberba exagerada.}{so.ber.bi.a}{0}
\verb{soberbo}{ê}{}{}{}{adj.}{Que é mais alto ou está mais elevado que outro.}{so.ber.bo}{0}
\verb{soberbo}{ê}{}{}{}{}{Que tem soberba; arrogante, orgulhoso.}{so.ber.bo}{0}
\verb{soberbo}{ê}{}{}{}{}{Que impressiona pelo aspecto grandioso, sublime, magnífico.}{so.ber.bo}{0}
\verb{soberbo}{ê}{}{}{}{}{Que é de grande e raro valor; luxuoso, valioso.}{so.ber.bo}{0}
\verb{soberbo}{ê}{}{}{}{s.m.}{Indivíduo que tem soberba; altivo, arrogante.}{so.ber.bo}{0}
\verb{sobpor}{}{}{}{}{v.t.}{Pôr de baixo ou por baixo.}{sob.por}{0}
\verb{sobpor}{}{Fig.}{}{}{}{Diminuir o valor, a qualidade; depreciar, menosprezar.}{sob.por}{\verboinum{60}}
\verb{sobra}{ó}{}{}{}{s.f.}{Ato ou efeito de sobrar.}{so.bra}{0}
\verb{sobra}{ó}{}{}{}{}{Resto; sobejo.}{so.bra}{0}
\verb{sobraçar}{}{}{}{}{v.t.}{Pôr debaixo do braço, mantendo seguro ou preso.}{so.bra.çar}{0}
\verb{sobraçar}{}{}{}{}{}{Firmar nos braços ao caminhar; sustentar, apoiar.}{so.bra.çar}{0}
\verb{sobraçar}{}{Fig.}{}{}{}{Servir de apoio moral; amparar.}{so.bra.çar}{0}
\verb{sobraçar}{}{}{}{}{v.pron.}{Abraçar"-se.}{so.bra.çar}{\verboinum{1}}
\verb{sobrado}{}{}{}{}{s.m.}{Pavimento superior ao térreo de um edifício.}{so.bra.do}{0}
\verb{sobrado}{}{}{}{}{}{Casa de dois ou mais pavimentos.}{so.bra.do}{0}
\verb{sobranceiro}{ê}{}{}{}{adj.}{Que está no alto, acima; dominante, proeminente.}{so.bran.cei.ro}{0}
\verb{sobranceiro}{ê}{}{}{}{}{Arrogante, orgulhoso, altivo.}{so.bran.cei.ro}{0}
\verb{sobranceiro}{ê}{}{}{}{}{Que tem ânimo superior, especialmente nos reveses da vida.}{so.bran.cei.ro}{0}
\verb{sobrancelha}{ê}{}{}{}{s.f.}{Saliência arqueada, guarnecida de pelos, que se dispõe acima da órbita ocular.}{so.bran.ce.lha}{0}
\verb{sobrancelha}{ê}{Por ext.}{}{}{}{O conjunto de pelos que revestem essa saliência.}{so.bran.ce.lha}{0}
\verb{sobrancelhas}{ê}{}{}{}{}{Var. de \textit{sobrancelha}.}{so.bran.ce.lhas}{0}
\verb{sobrar}{}{}{}{}{v.t.}{Haver em excesso; ser demasiado; sobejar.}{so.brar}{0}
\verb{sobrar}{}{}{}{}{}{Ficar, restar.}{so.brar}{0}
\verb{sobrar}{}{}{}{}{v.i.}{Ser deixado de lado.}{so.brar}{\verboinum{1}}
\verb{sobras}{ó}{}{}{}{s.f.}{Restos, sobejos.}{so.bras}{0}
\verb{sobre}{ô}{}{}{}{prep.}{Em cima de; na parte superior de; por cima de.}{so.bre}{0}
\verb{sobre}{ô}{}{}{}{}{Acima de.}{so.bre}{0}
\verb{sobre}{ô}{}{}{}{}{A respeito de.}{so.bre}{0}
\verb{sobreaviso}{}{}{}{}{s.m.}{Aviso prévio; precaução, prevenção.}{so.bre.a.vi.so}{0}
\verb{sobreaviso}{}{}{}{}{adj.}{Diz"-se daquele que é prevenido, acautelado.}{so.bre.a.vi.so}{0}
\verb{sobrecapa}{}{}{}{}{s.f.}{Cobertura de papel com que se envolve e protege a capa de um livro e na qual vêm impressos o título, o nome do autor, informações sobre a obra etc., por vezes com ilustrações.}{so.bre.ca.pa}{0}
\verb{sobrecarga}{}{}{}{}{s.f.}{Carga excessiva.}{so.bre.car.ga}{0}
\verb{sobrecarga}{}{}{}{}{}{Aquilo que se acresce à carga.}{so.bre.car.ga}{0}
\verb{sobrecarga}{}{}{}{}{}{Aquilo que transtorna o equilíbrio da carga.}{so.bre.car.ga}{0}
\verb{sobrecarregar}{}{}{}{}{v.t.}{Pôr excesso de carga; carregar demais.}{so.bre.car.re.gar}{0}
\verb{sobrecarregar}{}{}{}{}{}{Aumentar excessivamente.}{so.bre.car.re.gar}{0}
\verb{sobrecarregar}{}{}{}{}{}{Causar vexame; oprimir, humilhar.}{so.bre.car.re.gar}{0}
\verb{sobrecarregar}{}{}{}{}{}{Aumentar encargos.}{so.bre.car.re.gar}{\verboinum{5}}
\verb{sobrecarta}{}{}{}{}{s.f.}{Carta seguida a outra com que tem relação.}{so.bre.car.ta}{0}
\verb{sobrecarta}{}{}{}{}{}{Envoltório usado para enviar ou conservar carta; envelope.}{so.bre.car.ta}{0}
\verb{sobrecasaca}{}{}{}{}{s.f.}{Peça do vestuário masculino, atualmente em desuso, que atingia a altura dos joelhos e tinha abas que rodeavam o corpo.}{so.bre.ca.sa.ca}{0}
\verb{sobrecenho}{}{}{}{}{s.m.}{O conjunto das sobrancelhas.}{so.bre.ce.nho}{0}
\verb{sobrecenho}{}{}{}{}{}{Semblante severo, carrancudo.}{so.bre.ce.nho}{0}
\verb{sobrecéu}{}{}{}{}{s.m.}{Cobertura levantada por cima de um leito ou pavilhão.}{so.bre.céu}{0}
\verb{sobrecomum}{}{Gram.}{"-uns}{}{adj.}{Diz"-se do substantivo que possui apenas um gênero gramatical, podendo ser empregado tanto para o sexo masculino como para o feminino.}{so.bre.co.mum}{0}
\verb{sobrecoser}{ê}{}{}{}{v.t.}{Fazer sobrecostura.}{so.bre.co.ser}{\verboinum{12}}
\verb{sobrecostura}{}{}{}{}{s.f.}{Costura feita sobre uma outra costura ou sobre duas peças já cosidas uma à outra.  }{so.bre.cos.tu.ra}{0}
\verb{sobrecoxa}{ôch}{}{}{}{s.f.}{Parte das aves que une a coxa à parte central do corpo.}{so.bre.co.xa}{0}
\verb{sobrecu}{}{Pop.}{}{}{s.m.}{Parte das aves, localizada sobre as últimas vértebras, onde se prendem as penas da cauda; uropígio.}{so.bre.cu}{0}
\verb{sobredito}{}{}{}{}{adj.}{Dito acima ou atrás; já referido; supracitado; supradito.}{so.bre.di.to}{0}
\verb{sobreeminente}{}{}{}{}{adj.2g.}{Muito elevado; magnífico.}{so.bre.e.mi.nen.te}{0}
\verb{sobreestar}{}{}{}{}{}{Var. de \textit{sobrestar}.}{so.bre.es.tar}{0}
\verb{sobreestimar}{}{}{}{}{}{Var. de \textit{sobrestimar}.}{so.bre.es.ti.mar}{0}
\verb{sobreexceder}{s}{}{}{}{v.t.}{Exceder muito; ultrapassar, ir muito além.}{so.bre.ex.ce.der}{\verboinum{12}}
\verb{sobreexcelente}{s}{}{}{}{adj.2g.}{Mais que excelente; sublime.}{so.bre.ex.ce.len.te}{0}
\verb{sobreexcitar}{s}{}{}{}{v.t.}{Excitar intensamente; estimular ou impressionar muito os ânimos de algo ou alguém.}{so.bre.ex.ci.tar}{\verboinum{1}}
\verb{sobre"-humano}{}{}{sobre"-humanos}{}{adj.}{Que supera as forças humanas ou a natureza do homem.}{so.bre"-hu.ma.no}{0}
\verb{sobre"-humano}{}{Fig.}{sobre"-humanos}{}{}{Elevado, sublime, excelso.}{so.bre"-hu.ma.no}{0}
\verb{sobreiro}{ê}{Bot.}{}{}{s.m.}{Árvore com frutos comestíveis, cultivada como ornamental e de cuja casca se extrai a cortiça.}{so.brei.ro}{0}
\verb{sobrejacente}{}{}{}{}{adj.2g.}{Que está por cima.}{so.bre.ja.cen.te}{0}
\verb{sobrelevar}{}{}{}{}{v.t.}{Exceder em altura; ultrapassar, sobrepujar, crescer.}{so.bre.le.var}{0}
\verb{sobrelevar}{}{}{}{}{}{Superar em importância; destacar"-se.}{so.bre.le.var}{0}
\verb{sobrelevar}{}{Fig.}{}{}{}{Obter vantagem; vencer, suplantar.}{so.bre.le.var}{\verboinum{1}}
\verb{sobreloja}{ó}{}{}{}{s.f.}{Pavimento de edifícios comerciais localizado entre o andar térreo e o primeiro andar, geralmente de pé"-direito baixo.}{so.bre.lo.ja}{0}
\verb{sobreloja}{ó}{}{}{}{}{Cada uma das lojas situadas nesse pavimento.}{so.bre.lo.ja}{0}
\verb{sobremaneira}{ê}{}{}{}{adv.}{Muito, excessivamente.}{so.bre.ma.nei.ra}{0}
\verb{sobremesa}{ê}{}{}{}{s.f.}{Iguaria leve, geralmente fruta ou doce, que se serve ao final de uma refeição.}{so.bre.me.sa}{0}
\verb{sobremodo}{ó}{}{}{}{adv.}{Sobremaneira.}{so.bre.mo.do}{0}
\verb{sobrenadar}{}{}{}{}{v.i.}{Nadar à superfície; boiar, flutuar.}{so.bre.na.dar}{\verboinum{1}}
\verb{sobrenatural}{}{}{"-ais}{}{adj.2g.}{Que não se explica pelas leis naturais conhecidas.}{so.bre.na.tu.ral}{0}
\verb{sobrenatural}{}{Fig.}{"-ais}{}{}{Muito intenso; excessivo.}{so.bre.na.tu.ral}{0}
\verb{sobrenatural}{}{}{"-ais}{}{s.m.}{Aquilo que não se explica pelas leis naturais conhecidas.}{so.bre.na.tu.ral}{0}
\verb{sobrenatural}{}{Por ext.}{"-ais}{}{}{Aquilo que é muito extraordinário.}{so.bre.na.tu.ral}{0}
\verb{sobrenome}{}{}{}{}{s.m.}{Nome de família, que vem após o primeiro nome.}{so.bre.no.me}{0}
\verb{sobreolhar}{}{}{}{}{v.t.}{Olhar com desprezo, com ar de superioridade.}{so.bre.o.lhar}{\verboinum{1}}
\verb{sobrepaga}{}{}{}{}{s.f.}{Parte do pagamento que excede o combinado; gratificação.}{so.bre.pa.ga}{0}
\verb{sobrepairar}{}{}{}{}{v.t.}{Pairar acima.}{so.bre.pai.rar}{\verboinum{1}}
\verb{sobrepeliz}{}{}{}{}{s.f.}{Veste branca usada pelos clérigos sobre a batina.}{so.bre.pe.liz}{0}
\verb{sobrepeso}{ê}{}{}{}{s.m.}{O peso que excede um determinado limite; excesso de peso; sobrecarga.}{so.bre.pe.so}{0}
\verb{sobrepor}{ô}{}{}{}{v.t.}{Pôr em cima.}{so.bre.por}{0}
\verb{sobrepor}{ô}{}{}{}{}{Juntar, acrescentar.}{so.bre.por}{0}
\verb{sobrepor}{ô}{}{}{}{}{Ocupar o lugar de alguma coisa recebendo prioridade. \textit{abon}}{so.bre.por}{0}
\verb{sobrepor}{ô}{}{}{}{}{Suceder, sobrevir. \textit{abon}}{so.bre.por}{\verboinum{60}}
\verb{sobreposição}{}{}{"-ões}{}{s.f.}{Ato ou efeito de sobrepor.}{so.bre.po.si.ção}{0}
\verb{sobreposição}{}{}{"-ões}{}{}{Acréscimo, junção.}{so.bre.po.si.ção}{0}
\verb{sobreposto}{ô}{}{"-s ⟨ó⟩}{"-a ⟨ó⟩}{adj.}{Posto por cima; superposto.}{so.bre.pos.to}{0}
\verb{sobreposto}{ô}{}{"-s ⟨ó⟩}{"-a ⟨ó⟩}{s.m.}{Adorno que se usa sobre as vestimentas.}{so.bre.pos.to}{0}
\verb{sobrepujar}{}{}{}{}{v.t.}{Exceder, ultrapassar, superar.}{so.bre.pu.jar}{0}
\verb{sobrepujar}{}{}{}{}{}{Vencer, dominar; mostrar"-se superior; ter vantagem.}{so.bre.pu.jar}{0}
\verb{sobrepujar}{}{Fig.}{}{}{}{Ir além; ultrapassar. \textit{abon}}{so.bre.pu.jar}{\verboinum{1}}
\verb{sobre"-restar}{}{}{}{}{v.i.}{Ficar de resto; sobrar.}{so.bre"-res.tar}{0}
\verb{sobre"-restar}{}{}{}{}{v.t.}{Restar a outro(s); sobreviver.}{so.bre"-res.tar}{\verboinum{1}}
\verb{sobrescrever}{ê}{}{}{}{v.t.}{Escrever por cima, substituindo o que estava anteriormente.}{so.bres.cre.ver}{0}
\verb{sobrescrever}{ê}{}{}{}{}{Sobrescritar.}{so.bres.cre.ver}{\verboinum{12}}
\verb{sobrescritar}{}{}{}{}{v.t.}{Pôr nome e endereço do destinatário; endereçar.}{so.bres.cri.tar}{0}
\verb{sobrescritar}{}{}{}{}{}{Destinar correspondência.}{so.bres.cri.tar}{\verboinum{1}}
\verb{sobrescrito}{}{}{}{}{s.m.}{Nome e endereço do destinatário.}{so.bres.cri.to}{0}
\verb{sobrescrito}{}{Fig.}{}{}{}{O destinatário.}{so.bres.cri.to}{0}
\verb{sobrescrito}{}{}{}{}{adj.}{Diz"-se do caractere, menor que os demais, que se grafa acima do alinhamento do texto.}{so.bres.cri.to}{0}
\verb{sobressair}{}{}{}{}{v.t.}{Estar saliente; realçar.}{so.bres.sa.ir}{0}
\verb{sobressair}{}{}{}{}{}{Dar na vista; estar muito evidente; atrair a atenção.}{so.bres.sa.ir}{\verboinum{19}}
\verb{sobressair}{}{}{}{}{}{Estar visível; distinguir, ressaltar.}{so.bres.sa.ir}{0}
\verb{sobressalente}{}{}{}{}{adj.2g.}{Diz"-se de peça ou coisa que serve para substituir uma outra, em caso de falta, avaria ou desgaste natural.}{so.bres.sa.len.te}{0}
\verb{sobressalente}{}{}{}{}{s.m.}{A peça ou coisa com esse fim.}{so.bres.sa.len.te}{0}
\verb{sobressalente}{}{Bras.}{}{}{}{Pneu sobressalente; estepe.}{so.bres.sa.len.te}{0}
\verb{sobressaltar}{}{}{}{}{v.t.}{Tomar ou ser tomado de assalto.}{so.bres.sal.tar}{0}
\verb{sobressaltar}{}{}{}{}{}{Assustar, atemorizar.}{so.bres.sal.tar}{\verboinum{1}}
\verb{sobressalto}{}{}{}{}{s.m.}{Susto, surpresa.}{so.bres.sal.to}{0}
\verb{sobressalto}{}{}{}{}{}{Preocupação, apreensão, inquietação.}{so.bres.sal.to}{0}
\verb{sobressalto}{}{}{}{}{}{Tumulto, confusão, desordem.}{so.bres.sal.to}{0}
\verb{sobresselente}{}{}{}{}{}{Var. de \textit{sobressalente}.}{so.bres.se.len.te}{0}
\verb{sobrestar}{}{}{}{}{v.i.}{Parar, deter"-se.}{so.bres.tar}{0}
\verb{sobrestar}{}{}{}{}{}{Abster"-se.}{so.bres.tar}{0}
\verb{sobrestar}{}{}{}{}{}{Estar iminente; ameaçar.}{so.bres.tar}{0}
\verb{sobrestar}{}{}{}{}{}{Sustar, suspender.}{so.bres.tar}{\verboinum{36}}
\verb{sobrestimar}{}{}{}{}{v.t.}{Ver \textit{superestimar}.}{so.bres.ti.mar}{\verboinum{1}}
\verb{sobretaxa}{ch}{}{}{}{s.f.}{Taxa ou tarifa adicional, que incide, geralmente, sobre um serviço público.}{so.bre.ta.xa}{0}
\verb{sobretaxa}{ch}{Econ.}{}{}{}{Taxa suplementar sobre algo já tributado.}{so.bre.ta.xa}{0}
\verb{sobretaxar}{ch}{}{}{}{v.t.}{Aplicar uma sobretaxa.}{so.bre.ta.xar}{\verboinum{1}}
\verb{sobretudo}{}{}{}{}{adv.}{Especialmente, principalmente.}{so.bre.tu.do}{0}
\verb{sobretudo}{}{}{}{}{s.m.}{Espécie de casaco comprido usado sobre as outras vestes para proteger do frio.}{so.bre.tu.do}{0}
\verb{sobrevida}{}{}{}{}{s.f.}{Prolongamento da existência além da morte. \textit{abon}}{so.bre.vi.da}{0}
\verb{sobrevida}{}{Med.}{}{}{}{Tempo de vida que excede um limite; prolongamento da vida. \textit{abon}}{so.bre.vi.da}{0}
\verb{sobrevindo}{}{}{}{}{adj.}{Que sobreveio.}{so.bre.vin.do}{0}
\verb{sobrevindo}{}{}{}{}{}{Que chegou inesperadamente.}{so.bre.vin.do}{0}
\verb{sobrevir}{}{}{}{}{v.i.}{Vir ou acontecer depois de alguma outra coisa.}{so.bre.vir}{0}
\verb{sobrevir}{}{}{}{}{}{Chegar ou ocorrer inesperadamente.}{so.bre.vir}{\verboinum{56}}
\verb{sobrevivência}{}{}{}{}{s.f.}{Ato de continuar a viver ou existir.}{so.bre.vi.vên.cia}{0}
\verb{sobrevivência}{}{}{}{}{}{Capacidade daquele ou daquilo que subsiste a algum outro.}{so.bre.vi.vên.cia}{0}
\verb{sobrevivência}{}{}{}{}{}{Continuidade, persistência, subsistência.}{so.bre.vi.vên.cia}{0}
\verb{sobrevivente}{}{}{}{}{adj.2g.}{Que permanece vivo.}{so.bre.vi.ven.te}{0}
\verb{sobrevivente}{}{}{}{}{s.2g.}{Indivíduo que permaneceu vivo, geralmente após acidente ou catástrofe.}{so.bre.vi.ven.te}{0}
\verb{sobreviver}{ê}{}{}{}{v.i.}{Continuar a viver além do que outros.}{so.bre.vi.ver}{0}
\verb{sobreviver}{ê}{}{}{}{v.t.}{Permanecer vivo.}{so.bre.vi.ver}{0}
\verb{sobreviver}{ê}{}{}{}{}{Escapar, resistir.}{so.bre.vi.ver}{\verboinum{12}}
\verb{sobrevoar}{}{}{}{}{v.t.}{Voar por cima.}{so.bre.vo.ar}{\verboinum{7}}
\verb{sobriedade}{}{}{}{}{s.f.}{Qualidade ou condição de sóbrio; equilíbrio, temperança.}{so.bri.e.da.de}{0}
\verb{sobriedade}{}{}{}{}{}{Moderação, comedimento, por exemplo, ao comer ou beber.}{so.bri.e.da.de}{0}
\verb{sobrinho}{}{}{}{}{s.m.}{Filho da irmã ou do irmão.}{so.bri.nho}{0}
\verb{sobrinho}{}{}{}{}{}{Esposo da sobrinha.}{so.bri.nho}{0}
\verb{sóbrio}{}{}{}{}{adj.}{Que é moderado, comedido no ato de comer ou beber.}{só.brio}{0}
\verb{sóbrio}{}{}{}{}{}{Que não se encontra alcoolizado, bêbado.}{só.brio}{0}
\verb{sóbrio}{}{}{}{}{}{Discreto, sereno, recatado.}{só.brio}{0}
\verb{sobrolho}{ô}{}{}{}{s.m.}{Ver \textit{sobrancelha}.}{so.bro.lho}{0}
\verb{soca}{ó}{}{}{}{s.f.}{A segunda colheita de arroz, cana, fumo etc., após o corte da primeira.}{so.ca}{0}
\verb{soca}{ó}{Pop.}{}{}{}{Caule subterrâneo ou rizoma.}{so.ca}{0}
\verb{socado}{}{}{}{}{adj.}{Que levou socos; soqueado.}{so.ca.do}{0}
\verb{socado}{}{}{}{}{}{Que foi amassado com socador ou com pilão.}{so.ca.do}{0}
\verb{socado}{}{Pop.}{}{}{}{Diz"-se de indivíduo baixo e gordo.}{so.ca.do}{0}
\verb{soçaite}{}{Pop.}{}{}{s.f.}{A alta sociedade; a elite econômica; a alta"-roda.}{so.çai.te}{0}
\verb{socalco}{}{}{}{}{s.m.}{Porção plana de terreno em uma encosta; plataforma.}{so.cal.co}{0}
\verb{socapa}{}{}{}{}{s.f.}{Qualquer coisa usada para disfarçar; máscara, disfarce.}{so.ca.pa}{0}
\verb{socapa}{}{}{}{}{}{Usada na expressão \textit{à socapa}: de forma disfarçada; sorrateiramente. }{so.ca.pa}{0}
\verb{socar}{}{}{}{}{v.t.}{Dar socos; esmurrar, soquear.}{so.car}{0}
\verb{socar}{}{}{}{}{}{Moer no pilão; esmigalhar, esmagar.}{so.car}{0}
\verb{socar}{}{}{}{}{}{Amassar muito; sovar, bater.}{so.car}{0}
\verb{socar}{}{}{}{}{}{Colocar em lugar oculto; esconder.}{so.car}{\verboinum{2}}
\verb{socavão}{}{}{"-ões}{}{s.m.}{Grande cavidade em rochedo; gruta, lapa.}{so.ca.vão}{0}
\verb{socavão}{}{}{"-ões}{}{}{Local usado para se abrigar do mau tempo ou de um perigo; abrigo, esconderijo.}{so.ca.vão}{0}
\verb{socavar}{}{}{}{}{v.t.}{Escavar por baixo, por sob o chão.}{so.ca.var}{0}
\verb{socavar}{}{}{}{}{v.i.}{Fazer escavações; cavar.}{so.ca.var}{\verboinum{1}}
\verb{sociabilidade}{}{}{}{}{s.f.}{Qualidade ou estado de quem é sociável; tendência ou inclinação para a vida em sociedade.}{so.ci.a.bi.li.da.de}{0}
\verb{sociabilidade}{}{}{}{}{}{Maneiras ou modos de quem vive em sociedade; civilidade, urbanidade.}{so.ci.a.bi.li.da.de}{0}
\verb{sociabilizar}{}{}{}{}{v.t.}{Reunir em sociedade; tornar social.}{so.ci.a.bi.li.zar}{0}
\verb{sociabilizar}{}{}{}{}{}{Tornar sociável; civilizar, socializar.}{so.ci.a.bi.li.zar}{\verboinum{1}}
\verb{social}{}{}{"-ais}{}{adj.2g.}{Relativo à sociedade.}{so.ci.al}{0}
\verb{social}{}{}{"-ais}{}{}{Que vive em grupos, em sociedade; sociável.}{so.ci.al}{0}
\verb{social}{}{}{"-ais}{}{}{Que diz respeito às relações entre os indivíduos que vivem em sociedade.}{so.ci.al}{0}
\verb{social}{}{}{"-ais}{}{}{Relativo aos sócios de uma entidade ou agremiação.}{so.ci.al}{0}
\verb{socialismo}{}{Filos.}{}{}{s.m.}{Conjunto de doutrinas que propõem a transformação das relações de propriedade visando à organização de uma sociedade igualitária, sem classes.}{so.ci.a.lis.mo}{0}
\verb{socialismo}{}{}{}{}{}{}{so.ci.a.lis.mo}{0}
\verb{socialista}{}{}{}{}{adj.2g.}{Relativo ao socialismo.}{so.ci.a.lis.ta}{0}
\verb{socialista}{}{}{}{}{}{Diz"-se daquele que é partidário ou militante do socialismo.}{so.ci.a.lis.ta}{0}
\verb{socialite}{}{}{}{}{s.2g.}{Pessoa que se destaca nas camadas mais altas da sociedade.}{so.ci.a.li.te}{0}
\verb{socialização}{}{}{"-ões}{}{s.f.}{Ato ou efeito de socializar.}{so.ci.a.li.za.ção}{0}
\verb{socialização}{}{}{"-ões}{}{}{Extensão, por meio de leis, das vantagens de um grupo para toda a sociedade; estatização.}{so.ci.a.li.za.ção}{0}
\verb{socialização}{}{}{"-ões}{}{}{Desenvolvimento do espírito de coletividade e cooperação social nos indivíduos de uma sociedade.}{so.ci.a.li.za.ção}{0}
\verb{socializar}{}{}{}{}{v.t.}{Tornar social; reunir em sociedade.}{so.ci.a.li.zar}{0}
\verb{socializar}{}{}{}{}{}{Tornar propriedade coletiva ou governamental; estatizar.}{so.ci.a.li.zar}{0}
\verb{socializar}{}{}{}{}{}{Adotar o socialismo; tornar socialista.}{so.ci.a.li.zar}{\verboinum{1}}
\verb{sociável}{}{}{"-eis}{}{adj.2g.}{De convívio agradável; comunicativo, civilizado, amistoso.}{so.ci.á.vel}{0}
\verb{sociável}{}{}{"-eis}{}{}{Que tende para a vida em sociedade; social.}{so.ci.á.vel}{0}
\verb{socieconômico}{}{}{}{}{}{Var. de \textit{socioeconômico}.}{so.ci.e.co.nô.mi.co}{0}
\verb{sociedade}{}{}{}{}{s.f.}{Conjunto de pessoas ou de animais que vivem em grupos organizados por regras conhecidas por todos.}{so.ci.e.da.de}{0}
\verb{sociedade}{}{}{}{}{}{União de várias pessoas que se organizam legalmente para obedecer a um conjunto de regras escritas, os estatutos.}{so.ci.e.da.de}{0}
\verb{sociedade}{}{}{}{}{}{Parceria, acordo entre duas ou mais pessoas, para o uso comum de um determinado benefício.}{so.ci.e.da.de}{0}
\verb{societário}{}{}{}{}{adj.}{Relativo a uma sociedade.}{so.ci.e.tá.rio}{0}
\verb{societário}{}{}{}{}{}{Que faz parte de uma sociedade; sócio.}{so.ci.e.tá.rio}{0}
\verb{sócio}{}{}{}{}{s.m.}{Membro de uma sociedade civil ou comercial; societário.}{só.cio}{0}
\verb{sócio}{}{}{}{}{}{INdivíduo que se associa com outro em um empreendimento ou empresa.}{só.cio}{0}
\verb{sócio}{}{}{}{}{}{Companheiro, parceiro, cúmplice.}{só.cio}{0}
\verb{sociocultural}{}{}{"-ais}{}{adj.2g.}{Relativo a fatores sociais e culturais simultaneamente.}{so.ci.o.cul.tu.ral}{0}
\verb{socioeconômico}{}{}{}{}{adj.}{Relativo a fatores sociais e econômicos simultaneamente.}{so.ci.o.e.co.nô.mi.co}{0}
\verb{sociologia}{}{}{}{}{s.f.}{Ciência que tem por objeto de estudo a organização das sociedades humanas, das relações entre os homens e dos fenômenos sociais. }{so.ci.o.lo.gi.a}{0}
\verb{sociólogo}{}{}{}{}{s.m.}{Especialista no estudo das relações humanas em sociedade.}{so.ci.ó.lo.go}{0}
\verb{sociopolítico}{}{}{}{}{adj.}{Relativo a fatores sociais e políticos simultaneamente.}{so.ci.o.po.lí.ti.co}{0}
\verb{soco}{ô}{}{}{}{s.m.}{Golpe dado com a mão fechada; murro.}{so.co}{0}
\verb{socó}{}{Zool.}{}{}{s.m.}{Ave, semelhante à garça, que vive à margem dos rios e em manguezais, alimentando"-se de peixes.}{so.có}{0}
\verb{soçobrar}{}{}{}{}{v.t.}{Submergir na água; afundar, naufragar.}{so.ço.brar}{0}
\verb{soçobrar}{}{}{}{}{}{Reduzir a nada; aniquilar, anular.}{so.ço.brar}{0}
\verb{soçobrar}{}{}{}{}{}{Perturbar, agitar, desvairar.}{so.ço.brar}{\verboinum{1}}
\verb{soçobro}{ô}{}{}{}{s.m.}{Ato ou efeito de soçobrar, afundar; naufrágio.}{so.ço.bro}{0}
\verb{soçobro}{ô}{Fig.}{}{}{}{Estado de quem se encontra sem forças; desânimo, desalento.}{so.ço.bro}{0}
\verb{soco"-inglês}{ô}{}{socos"-ingleses ⟨ô\ldots{}ê⟩}{}{s.m.}{Peça inteiriça de metal adaptável à mão, com orifícios para os dedos, usada como arma para aumentar a força do golpe.}{so.co"-in.glês}{0}
\verb{socorrer}{}{}{}{}{v.t.}{Trazer auxílio, remédio ou esmola.}{so.cor.rer}{0}
\verb{socorrer}{}{}{}{}{}{Prestar ajuda; defender, proteger.}{so.cor.rer}{0}
\verb{socorrer}{}{}{}{}{v.pron.}{Lançar mão; valer"-se, empregar.}{so.cor.rer}{\verboinum{12}}
\verb{socorro}{ô}{}{}{}{s.m.}{Ato ou efeito de socorrer; ajuda, auxílio, proteção.}{so.cor.ro}{0}
\verb{socorro}{ô}{}{}{}{}{Atendimento que se presta a pessoa acidentada ou adoentada.}{so.cor.ro}{0}
\verb{socorro}{ô}{}{}{}{interj.}{Expressão usada para pedir auxílio ou proteção.}{so.cor.ro}{0}
\verb{socrático}{}{Filos.}{}{}{adj.}{Relativo a Sócrates, filósofo grego do século \textsc{iv} a.C., ou à sua filosofia.}{so.crá.ti.co}{0}
\verb{soda}{ó}{}{}{}{s.f.}{Soda cáustica; hidróxido de sódio.}{so.da}{0}
\verb{soda}{ó}{}{}{}{}{Carbonato de sódio, também conhecido por \textit{soda do comércio}.}{so.da}{0}
\verb{soda}{ó}{}{}{}{}{Água carregada de gás carbônico, na qual se adiciona xarope de frutas e serve"-se como refrigerante ou na qual se diluem bebidas alcoólicas.}{so.da}{0}
\verb{sodalício}{}{}{}{}{s.m.}{Associação ou sociedade literária, filantrópica etc.}{so.da.lí.cio}{0}
\verb{sódio}{}{Quím.}{}{}{s.m.}{Elemento químico radioativo, branco, mole, moldável, dúctil, do grupo dos metais alcalinos, utilizado na indústria  metalúrgica, de sabões, tecidos, vidros, alimentos etc. Símb.: \elemento{11}{22.98977}{Na}.}{só.dio}{0}
\verb{sodomia}{}{}{}{}{s.f.}{Cópula anal entre dois homens ou entre um homem e uma mulher.}{so.do.mi.a}{0}
\verb{soer}{ê}{}{}{}{v.i.}{Ocorrer com frequência; ser hábito; costumar.}{so.er}{\verboinum{17}}
\verb{soerguer}{}{}{}{}{v.t.}{Erguer um pouco; levantar a pequena altura.}{so.er.guer}{0}
\verb{soerguer}{}{}{}{}{v.pron.}{Erguer"-se a custo, com dificuldade.}{so.er.guer}{\verboinum{53}}
\verb{soerguimento}{}{}{}{}{s.m.}{Ato ou efeito de soerguer; levantamento.}{so.er.gui.men.to}{0}
\verb{soez}{ê}{}{}{}{adj.}{De caráter vil; baixo, ordinário.}{so.ez}{0}
\verb{soez}{ê}{}{}{}{}{De pouca importância; reles, insignificante.}{so.ez}{0}
\verb{sofá}{}{}{}{}{s.m.}{Assento estofado, dotado de braços e encosto, para duas ou mais pessoas.}{so.fá}{0}
\verb{sofá"-cama}{}{}{sofás"-cama \textit{ou} sofás"-camas}{}{s.m.}{Sofá que pode se transformar em uma cama.}{so.fá"-ca.ma}{0}
\verb{sofisma}{}{Filos.}{}{}{s.m.}{Raciocínio ou argumento falso com aparência de verdadeiro, feito ou não com intenção deliberada de enganar ou iludir.}{so.fis.ma}{0}
\verb{sofismar}{}{}{}{}{v.t.}{Raciocinar através de sofismas; iludir.}{so.fis.mar}{\verboinum{1}}
\verb{sofista}{}{}{}{}{adj.2g.}{Que se utiliza de argumentos falsos ou inconsistentes para convencer alguém.}{so.fis.ta}{0}
\verb{sofisticação}{}{}{"-ões}{}{s.f.}{Ato ou efeito de sofisticar, de expressar com rebuscamentos.}{so.fis.ti.ca.ção}{0}
\verb{sofisticação}{}{}{"-ões}{}{}{Qualidade de pessoa sofisticada, requintada.}{so.fis.ti.ca.ção}{0}
\verb{sofisticado}{}{}{}{}{adj.}{Que tem requinte; refinado.}{so.fis.ti.ca.do}{0}
\verb{sofisticado}{}{}{}{}{}{Que foi alterado, falsificado, adulterado.}{so.fis.ti.ca.do}{0}
\verb{sofisticar}{}{}{}{}{v.t.}{Tornar sofisticado, requintado; refinar.}{so.fis.ti.car}{0}
\verb{sofisticar}{}{}{}{}{}{Expressar com afetação, com exagero.}{so.fis.ti.car}{0}
\verb{sofisticar}{}{}{}{}{}{Alterar composição de produto; adulterar, falsificar.}{so.fis.ti.car}{\verboinum{2}}
\verb{sofístico}{}{}{}{}{adj.}{Que se expressa por meio de sofisma.}{so.fís.ti.co}{0}
\verb{sofrê}{}{Zool.}{}{}{s.m.}{Pássaro canoro, de cor preta, com o ventre alaranjado e uma mancha branca nas asas, notável pela capacidade de imitar sons; corrupião.}{so.frê}{0}
\verb{sofrear}{}{}{}{}{v.t.}{Manter a calma; conter, moderar, reprimir.}{so.fre.ar}{0}
\verb{sofrear}{}{}{}{}{}{Puxar as rédeas de uma montaria para mudar a marcha ou parar.}{so.fre.ar}{\verboinum{4}}
\verb{sofredor}{ô}{}{}{}{adj.}{Que sofre muito; infeliz. }{so.fre.dor}{0}
\verb{sôfrego}{}{}{}{}{adj.}{Que tem pressa ou avidez no ato de comer e beber.}{sô.fre.go}{0}
\verb{sôfrego}{}{}{}{}{}{Impaciente, desejoso, ávido.}{sô.fre.go}{0}
\verb{sofreguidão}{}{}{"-ões}{}{s.f.}{Maneira apressada e ávida de comer e beber; gulodice.}{so.fre.gui.dão}{0}
\verb{sofreguidão}{}{}{"-ões}{}{}{Desejo, ambição, avidez, impaciência.}{so.fre.gui.dão}{0}
\verb{sofrer}{ê}{}{}{}{v.t.}{Sentir dores; padecer.}{so.frer}{0}
\verb{sofrer}{ê}{}{}{}{}{Experimentar com resignação; suportar, tolerar.}{so.frer}{\verboinum{12}}
\verb{sofrido}{}{}{}{}{}{Feito com muito custo, com dificuldade.}{so.fri.do}{0}
\verb{sofrido}{}{}{}{}{adj.}{Que revela sofrimento; padecido, castigado.}{so.fri.do}{0}
\verb{sofrimento}{}{}{}{}{s.m.}{Ato ou efeito de sofrer; padecimento.}{so.fri.men.to}{0}
\verb{sofrimento}{}{}{}{}{}{Dor física ou moral; ansiedade, angústia.}{so.fri.men.to}{0}
\verb{sofrível}{}{}{"-eis}{}{adj.2g.}{Que se pode sofrer; suportável, tolerável.}{so.frí.vel}{0}
\verb{sofrível}{}{}{"-eis}{}{}{Acima de medíocre; razoável, passável.}{so.frí.vel}{0}
\verb{software}{}{Informát.}{}{}{s.m.}{Qualquer programa, rotina ou conjunto de instruções que controlam o funcionamento de um computador ou a maneira como ele deve executar uma tarefa.}{\textit{software}}{0}
\verb{soga}{ó}{}{}{}{s.f.}{Corda grossa e forte.}{so.ga}{0}
\verb{soga}{ó}{}{}{}{}{Sulco ou vala no terreno para condução ou escoamento de águas.}{so.ga}{0}
\verb{sogra}{ó}{}{}{}{s.f.}{Mãe do marido, em relação à nora, ou mãe da mulher, em relação ao genro.}{so.gra}{0}
\verb{sogro}{ô}{}{}{}{s.m.}{Pai do marido, em relação à nora; pai da mulher, em relação ao genro.}{so.gro}{0}
\verb{soja}{ó}{}{}{}{s.f.}{Erva da família das leguminosas, de caule ramificado e flores brancas ou amarelas, cujas sementes são ricas em proteínas e seus subprodutos são utilizados como substitutos a alimentos de origem animal.}{so.ja}{0}
\verb{soja}{ó}{}{}{}{}{A semente dessa erva.}{so.ja}{0}
\verb{sol}{ó}{Mús.}{}{}{s.m.}{A quinta nota musical na escala de \textit{dó}.}{sol}{0}
\verb{sol}{ó}{}{}{}{s.m.}{Estrela em torno da qual giram a Terra, os planetas e os cometas. (Nesta acepção, com inicial maiúscula.)}{sol}{0}
\verb{sol}{ó}{Fig.}{}{}{}{Brilho, luz.}{sol}{0}
\verb{sola}{ó}{}{}{}{s.f.}{Parte inferior do calçado, mais dura e resistente, que mantém contato com o chão; solado.}{so.la}{0}
\verb{sola}{ó}{}{}{}{}{Couro grosso curtido, usado para fabricação de calçados.}{so.la}{0}
\verb{sola}{ó}{Por ext.}{}{}{}{A planta do pé.}{so.la}{0}
\verb{solado}{}{}{}{}{s.m.}{A sola do calçado.}{so.la.do}{0}
\verb{solado}{}{}{}{}{adj.}{Diz"-se da massa que endureceu, que não assou uniformemente.}{so.la.do}{0}
\verb{solanácea}{}{Bot.}{}{}{s.f.}{Espécime das solanáceas, família de plantas dicotiledôneas, de larga distribuição na América do Sul, com muitas espécies de grande importância alimentar e na indústria farmacêutica, e cujos representantes mais conhecidos são o tomate, a batata, o tabaco, a berinjela, a pimenta"-malagueta, a beladona, o pimentão etc.}{so.la.ná.cea}{0}
\verb{solapar}{}{}{}{}{v.t.}{Fazer cova; escavar.}{so.la.par}{0}
\verb{solapar}{}{}{}{}{}{Destruir a base de algo; arruinar, enfraquecer.}{so.la.par}{0}
\verb{solapar}{}{}{}{}{}{Encobrir, disfarçar, ocultar.}{so.la.par}{\verboinum{1}}
\verb{solar}{}{}{}{}{adj.2g.}{Relativo ao sol.}{so.lar}{0}
\verb{solar}{}{}{}{}{s.m.}{Antiga casa nobre; mansão, palácio.}{so.lar}{0}
\verb{solar}{}{}{}{}{v.t.}{Pôr sola em calçado.}{so.lar}{0}
\verb{solar}{}{}{}{}{}{Cantar ou tocar sozinho uma música.}{so.lar}{\verboinum{1}}
\verb{solarengo}{}{}{}{}{adj.}{Relativo a solar, a casa antiga. }{so.la.ren.go}{0}
\verb{solarengo}{}{}{}{}{s.m.}{Proprietário de solar.}{so.la.ren.go}{0}
\verb{solário}{}{}{}{}{s.m.}{Terraço reservado aos banhos de sol para fins terapêuticos ou estéticos.}{so.lá.rio}{0}
\verb{solário}{}{}{}{}{}{Antigo relógio de sol dos romanos.}{so.lá.rio}{0}
\verb{solavanco}{}{}{}{}{s.m.}{Movimento brusco e violento de um veículo ou das pessoas que ele transporta.}{so.la.van.co}{0}
\verb{solavanco}{}{}{}{}{}{Sacudidela brusca; tranco.}{so.la.van.co}{0}
\verb{solda}{ó}{}{}{}{s.f.}{Ato ou efeito de soldar, unir.}{sol.da}{0}
\verb{solda}{ó}{}{}{}{}{Composição metálica, possível de ser fundida para ser utilizada na ligação de peças também metálicas. }{sol.da}{0}
\verb{soldada}{}{}{}{}{s.f.}{Quantia com que se paga o trabalho de criados, empregados etc.; soldo, salário.}{sol.da.da}{0}
\verb{soldada}{}{Fig.}{}{}{}{Recompensa, prêmio.}{sol.da.da}{0}
\verb{soldadesca}{ê}{}{}{}{s.f.}{Conjunto de soldados; tropa militar.}{sol.da.des.ca}{0}
\verb{soldadesca}{ê}{Pejor.}{}{}{}{Grupo de soldados insubordinados.}{sol.da.des.ca}{0}
\verb{soldado}{}{}{}{}{adj.}{Que foi ligado ou vedado com solda; unido.}{sol.da.do}{0}
\verb{soldado}{}{}{}{}{s.m.}{Indivíduo alistado nas Forças Armadas, voluntariamente ou por obrigação.}{sol.da.do}{0}
\verb{soldado}{}{}{}{}{}{Qualquer militar terrestre.}{sol.da.do}{0}
\verb{soldado}{}{Fig.}{}{}{}{Partidário ou participante ativo de uma causa; defensor, paladino.}{sol.da.do}{0}
\verb{soldadura}{}{}{}{}{s.f.}{União de duas partes metálicas por meio de solda.}{sol.da.du.ra}{0}
\verb{soldar}{}{}{}{}{v.t.}{Unir ou vedar, por meio de aquecimento ou pressão, duas peças metálicas.}{sol.dar}{0}
\verb{soldar}{}{}{}{}{}{Ajustar ou ligar estreitamente.}{sol.dar}{\verboinum{1}}
\verb{soldo}{ô}{}{}{}{s.m.}{Quantia paga pelo governo aos militares.}{sol.do}{0}
\verb{soldo}{ô}{}{}{}{}{Recompensa, salário.}{sol.do}{0}
\verb{solecismo}{}{Gram.}{}{}{s.m.}{Qualquer desvio de sintaxe, como problemas de concordância, regência, colocação pronominal etc.}{so.le.cis.mo}{0}
\verb{soledade}{}{}{}{}{s.f.}{Característica de quem está solitário, melancólico.}{so.le.da.de}{0}
\verb{soledade}{}{}{}{}{}{Lugar deserto, retirado, ermo.}{so.le.da.de}{0}
\verb{soleira}{ê}{}{}{}{s.f.}{Revestimento de pedra ou madeira que fica embaixo da porta, rasante com o piso.}{so.lei.ra}{0}
\verb{soleira}{ê}{}{}{}{}{Lugar da sela que o cavaleiro utiliza para apoiar o pé.}{so.lei.ra}{0}
\verb{solene}{ê}{}{}{}{adj.}{Acompanhado de pompas e cerimônias públicas.}{so.le.ne}{0}
\verb{solene}{ê}{}{}{}{}{Que denota importância, seriedade; grave, imponente.}{so.le.ne}{0}
\verb{solene}{ê}{}{}{}{}{Que se celebra com formalidades; cerimonioso, afetado.}{so.le.ne}{0}
\verb{solenidade}{}{}{}{}{s.f.}{Qualidade ou caráter do que é solene, pomposo.}{so.le.ni.da.de}{0}
\verb{solenidade}{}{}{}{}{}{Cerimônia formal, oficial.}{so.le.ni.da.de}{0}
\verb{solenidade}{}{}{}{}{}{Conjunto de formalidades que validam certos atos.}{so.le.ni.da.de}{0}
\verb{solenizar}{}{}{}{}{v.t.}{Tornar solene.}{so.le.ni.zar}{0}
\verb{solenizar}{}{}{}{}{}{Celebrar publicamente e com pompa.}{so.le.ni.zar}{\verboinum{1}}
\verb{solerte}{é}{}{}{}{adj.}{Que é sagaz ou velhaco.}{so.ler.te}{0}
\verb{soletrar}{}{}{}{}{v.t.}{Ler, em voz alta, pronunciando separadamente as letras de uma palavra e, em seguida, juntando"-as em sílabas.}{so.le.trar}{0}
\verb{soletrar}{}{}{}{}{}{Ler mal, ler por alto.}{so.le.trar}{0}
\verb{soletrar}{}{}{}{}{}{Ler com calma e atenção.}{so.le.trar}{\verboinum{1}}
\verb{solfejar}{}{Mús.}{}{}{v.t.}{Ler ou entoar uma peça musical, pronunciando o nome das notas e respeitando os sinais de notação.}{sol.fe.jar}{0}
\verb{solfejo}{ê}{}{}{}{s.m.}{Ato ou efeito de solfejar.}{sol.fe.jo}{0}
\verb{solfejo}{ê}{Mús.}{}{}{}{Exercício musical para se aprender a ler notas, marcando o compasso com as mãos.}{sol.fe.jo}{0}
\verb{solferino}{}{}{}{}{s.m.}{Cor escarlate ou roxa, com que se tingem tecidos, principalmente as vestes e solidéus dos bispos.}{sol.fe.ri.no}{0}
\verb{solicitação}{}{}{"-ões}{}{}{Pedido feito com insistência; rogativa, súplica.}{so.li.ci.ta.ção}{0}
\verb{solicitação}{}{}{"-ões}{}{s.f.}{Ato ou efeito de solicitar; pedido.}{so.li.ci.ta.ção}{0}
\verb{solicitação}{}{}{"-ões}{}{}{Convite, tentação, apelo.}{so.li.ci.ta.ção}{0}
\verb{solicitador}{ô}{}{}{}{adj.}{Que solicita, pede; solicitante.}{so.li.ci.ta.dor}{0}
\verb{solicitador}{ô}{Jur.}{}{}{s.m.}{Auxiliar de advogado, legalmente habilitado para acompanhar o andamento das ações, assinar os termos dos recursos etc.}{so.li.ci.ta.dor}{0}
\verb{solicitar}{}{}{}{}{v.t.}{Pedir ou rogar com insistência.}{so.li.ci.tar}{0}
\verb{solicitar}{}{}{}{}{}{Requisitar, requerer com deferência.}{so.li.ci.tar}{0}
\verb{solicitar}{}{}{}{}{}{Convidar, convocar, desafiar.}{so.li.ci.tar}{\verboinum{1}}
\verb{solícito}{}{}{}{}{adj.}{Que se esforça para ser útil, para ajudar; prestimoso, atencioso.}{so.lí.ci.to}{0}
\verb{solícito}{}{}{}{}{}{Que tem cuidado; zeloso, diligente.}{so.lí.ci.to}{0}
\verb{solícito}{}{}{}{}{}{Apreensivo, receoso, inquieto.}{so.lí.ci.to}{0}
\verb{solicitude}{}{}{}{}{s.f.}{Desejo de atender a algum pedido da melhor maneira possível; boa vontade.}{so.li.ci.tu.de}{0}
\verb{solicitude}{}{}{}{}{}{Cuidado atencioso; zelo, desvelo.}{so.li.ci.tu.de}{0}
\verb{solidão}{}{}{"-ões}{}{s.f.}{Estado de quem se encontra só; isolamento.}{so.li.dão}{0}
\verb{solidão}{}{}{"-ões}{}{}{Sensação de viver ou encontrar"-se nessas condições.}{so.li.dão}{0}
\verb{solidão}{}{}{"-ões}{}{}{Qualidade dos locais despovoados, ermos.}{so.li.dão}{0}
\verb{solidariedade}{}{}{}{}{s.f.}{Qualidade de solidário.}{so.li.da.ri.e.da.de}{0}
\verb{solidário}{}{}{}{}{adj.}{Em que há interesse comum.}{so.li.dá.rio}{0}
\verb{solidário}{}{}{}{}{}{Que se dispõe a auxiliar, apoiar ou defender alguém.}{so.li.dá.rio}{0}
\verb{solidário}{}{}{}{}{}{Que compartilha sentimentos, opiniões, interesses.}{so.li.dá.rio}{0}
\verb{solidarizar}{}{}{}{}{v.t.}{Tornar solidário.}{so.li.da.ri.zar}{0}
\verb{solidarizar}{}{}{}{}{v.pron.}{Demonstrar solidariedade. \textit{abon}}{so.li.da.ri.zar}{\verboinum{1}}
\verb{solidéu}{}{}{}{}{s.m.}{Espécie de cobertura em forma de uma pequena calota, usada pelos bispos e pelos judeus em determinadas ocasiões.}{so.li.déu}{0}
\verb{solidez}{ê}{}{}{}{s.f.}{Qualidade de sólido.}{so.li.dez}{0}
\verb{solidez}{ê}{}{}{}{}{Resistência, durabilidade, força.}{so.li.dez}{0}
\verb{solidez}{ê}{Fig.}{}{}{}{Estabilidade, segurança, firmeza.}{so.li.dez}{0}
\verb{solidificação}{}{}{"-ões}{}{s.f.}{Ato ou efeito de solidificar.}{so.li.di.fi.ca.ção}{0}
\verb{solidificação}{}{Fig.}{"-ões}{}{}{Passagem a um estado de estabilidade; consolidação.}{so.li.di.fi.ca.ção}{0}
\verb{solidificar}{}{}{}{}{v.t.}{Tornar sólido ou estável.}{so.li.di.fi.car}{0}
\verb{solidificar}{}{Fig.}{}{}{}{Fortalecer, estabilizar, consolidar.}{so.li.di.fi.car}{\verboinum{2}}
\verb{sólido}{}{}{}{}{adj.}{Diz"-se do estado físico da matéria em que ela apresenta forma e volume definidos.}{só.li.do}{0}
\verb{sólido}{}{}{}{}{}{Consistente, compacto.}{só.li.do}{0}
\verb{sólido}{}{}{}{}{}{Forte, resistente, robusto.}{só.li.do}{0}
\verb{sólido}{}{Fig.}{}{}{}{Fundamentado, sério, estável.}{só.li.do}{0}
\verb{sólido}{}{}{}{}{s.m.}{Qualquer corpo sólido.}{só.li.do}{0}
\verb{solilóquio}{}{}{}{}{s.m.}{Ato de conversar consigo mesmo; monólogo.}{so.li.ló.quio}{0}
\verb{sólio}{}{}{}{}{s.m.}{A cadeira do rei; trono.}{só.lio}{0}
\verb{sólio}{}{}{}{}{}{A cadeira do papa.}{só.lio}{0}
\verb{sólio}{}{Fig.}{}{}{}{O poder real ou papal.}{só.lio}{0}
\verb{solípede}{}{Zool.}{}{}{adj.}{Diz"-se de animal que tem um único casco em cada pata.}{so.lí.pe.de}{0}
\verb{solista}{}{}{}{}{s.2g.}{Instrumentista, cantor ou dançarino que executa um solo.}{so.lis.ta}{0}
\verb{solista}{}{}{}{}{adj.2g.}{Que toca, canta ou dança solos.}{so.lis.ta}{0}
\verb{solitária}{}{Zool.}{}{}{s.f.}{Verme parasita do intestino dos vertebrados, com corpo em forma de fita e cabeça com ventosas ou ganchos; tênia.}{so.li.tá.ria}{0}
\verb{solitária}{}{Bras.}{}{}{}{Cela penitenciária onde um detento é isolado como castigo ou precaução.}{so.li.tá.ria}{0}
\verb{solitário}{}{}{}{}{adj.}{Que vive ou se encontra sem companhia; desacompanhado.}{so.li.tá.rio}{0}
\verb{solitário}{}{}{}{}{}{Que acontece em solidão.}{so.li.tá.rio}{0}
\verb{solitário}{}{}{}{}{}{Que gosta de estar só, não convive em sociedade.}{so.li.tá.rio}{0}
\verb{solitário}{}{}{}{}{s.m.}{Indivíduo que vive na solidão.}{so.li.tá.rio}{0}
\verb{solitário}{}{}{}{}{}{Monge que vive isolado; eremita, anacoreta.}{so.li.tá.rio}{0}
\verb{solitude}{}{Liter.}{}{}{s.f.}{Solidão.}{so.li.tu.de}{0}
\verb{solo}{ó}{}{}{}{s.m.}{Superfície de terra onde se pisa; chão, terra.}{so.lo}{0}
\verb{solo}{ó}{Mús.}{}{}{s.m.}{Obra ou trecho musical executado por um só instrumento ou voz.}{so.lo}{0}
\verb{solo}{ó}{}{}{}{}{O conjunto dos materiais sólidos que constituem a crosta terrestre.}{so.lo}{0}
\verb{solo}{ó}{}{}{}{}{A terra, considerada em suas propriedades produtivas.}{so.lo}{0}
\verb{solo}{ó}{}{}{}{}{Dança executada por um só dançarino.}{so.lo}{0}
\verb{solstício}{}{Astron.}{}{}{s.m.}{Cada um dos dois dias do ano (22 ou 23 de junho e 22 ou 23 de dezembro) em que a aparente trajetória do Sol no céu atinge seu maior ângulo em relação ao equador, fazendo com que a duração do dia e da noite sejam muito diferentes.}{sols.tí.cio}{0}
\verb{solta}{ô}{}{}{}{s.f.}{Peia para amarrar cavalgadura.}{sol.ta}{0}
\verb{solta}{ô}{}{}{}{}{Usada na expressão \textit{à solta}: livremente.}{sol.ta}{0}
\verb{soltar}{}{}{}{}{v.t.}{Tornar livre; libertar, livrar.}{sol.tar}{0}
\verb{soltar}{}{}{}{}{}{Afrouxar, desprender, desamarrar, desatar, desligar.}{sol.tar}{0}
\verb{soltar}{}{}{}{}{}{Emitir, exalar, liberar, expelir.}{sol.tar}{0}
\verb{soltar}{}{}{}{}{}{Proferir, dizer.}{sol.tar}{0}
\verb{soltar}{}{}{}{}{}{Aplicar, desferir, dar.}{sol.tar}{\verboinum{1}}
\verb{solteirão}{}{}{"-ões}{}{adj.}{Diz"-se de homem de meia"-idade ou mais velho que não se casou.}{sol.tei.rão}{0}
\verb{solteiro}{ê}{}{}{}{adj.}{Que ainda não se casou.}{sol.tei.ro}{0}
\verb{solteiro}{ê}{}{}{}{}{Que não está casado.}{sol.tei.ro}{0}
\verb{solteiro}{ê}{Fig.}{}{}{}{Diz"-se de pessoa cujo cônjuge ou namorado(a) está ausente.}{sol.tei.ro}{0}
\verb{solteiro}{ê}{}{}{}{s.m.}{Indivíduo que não se casou.}{sol.tei.ro}{0}
\verb{solto}{ô}{}{}{}{adj.}{Que tem partes que não estão corretamente fixadas.}{sol.to}{0}
\verb{solto}{ô}{}{}{}{}{Que não está fixado, que está livre.}{sol.to}{0}
\verb{solto}{ô}{}{}{}{}{Largo, folgado.}{sol.to}{0}
\verb{solto}{ô}{}{}{}{}{Que não está preso; posto em liberdade; livre.}{sol.to}{0}
\verb{soltura}{}{}{}{}{s.f.}{Ato ou efeito de soltar.}{sol.tu.ra}{0}
\verb{soltura}{}{Pop.}{}{}{}{Diarreia.}{sol.tu.ra}{0}
\verb{solubilidade}{}{}{}{}{s.f.}{Qualidade de solúvel.}{so.lu.bi.li.da.de}{0}
\verb{solubilidade}{}{Quím.}{}{}{}{Medida da capacidade de uma substância (soluto) dissolver"-se em outra (solvente).}{so.lu.bi.li.da.de}{0}
\verb{solubilizar}{}{}{}{}{v.t.}{Tornar solúvel.}{so.lu.bi.li.zar}{0}
\verb{solubilizar}{}{Quím.}{}{}{}{Dissolver.}{so.lu.bi.li.zar}{\verboinum{1}}
\verb{solução}{}{}{"-ões}{}{s.f.}{Ato ou efeito de solver.}{so.lu.ção}{0}
\verb{solução}{}{}{"-ões}{}{}{Meio pelo qual se resolve um problema ou dificuldade; saída.}{so.lu.ção}{0}
\verb{solução}{}{}{"-ões}{}{}{Resposta correta ou esperada a uma questão de prova.}{so.lu.ção}{0}
\verb{solução}{}{}{"-ões}{}{}{Resposta ou raciocínio que gera o resultado de um problema de qualquer natureza; enigma.}{so.lu.ção}{0}
\verb{solução}{}{}{"-ões}{}{}{Separação das partes de um todo.}{so.lu.ção}{0}
\verb{solução}{}{Quím.}{"-ões}{}{}{Sistema homogêneo com mais de uma substância.}{so.lu.ção}{0}
\verb{soluçar}{}{}{}{}{v.i.}{Produzir soluços.}{so.lu.çar}{0}
\verb{soluçar}{}{}{}{}{}{Chorar, eventualmente soltando soluços.}{so.lu.çar}{\verboinum{3}}
\verb{solucionar}{}{}{}{}{v.t.}{Dar solução.}{so.lu.ci.o.nar}{\verboinum{1}}
\verb{soluço}{}{}{}{}{s.m.}{Contração involuntária e espasmódica do diafragma, em que se inicia o movimento de inspiração, o qual é logo detido pelo fechamento da glote, produzindo ruído característico.}{so.lu.ço}{0}
\verb{soluço}{}{}{}{}{}{Pranto acompanhado de inspirações fortes e ruidosas.}{so.lu.ço}{0}
\verb{soluto}{}{Quím.}{}{}{s.m.}{Substância que é dissolvida numa outra em maior quantidade e chamada solvente.}{so.lu.to}{0}
\verb{solúvel}{}{Quím.}{"-eis}{}{adj.2g.}{Que pode ser dissolvido.}{so.lú.vel}{0}
\verb{solúvel}{}{}{"-eis}{}{}{Que pode ser resolvido.}{so.lú.vel}{0}
\verb{solvável}{}{}{"-eis}{}{adj.2g.}{Solvível.}{sol.vá.vel}{0}
\verb{solvente}{}{Quím.}{}{}{s.m.}{Substância capaz de dissolver outra substância.}{sol.ven.te}{0}
\verb{solvente}{}{}{}{}{}{Devedor que tem condições de pagar o débito.}{sol.ven.te}{0}
\verb{solver}{ê}{}{}{}{v.t.}{Solucionar, resolver.}{sol.ver}{0}
\verb{solver}{ê}{}{}{}{}{Pagar, quitar, saldar.}{sol.ver}{0}
\verb{solver}{ê}{}{}{}{}{Dissolver, diluir.}{sol.ver}{0}
\verb{solver}{ê}{}{}{}{}{Desatar, separar.}{sol.ver}{\verboinum{12}}
\verb{solvível}{}{}{"-eis}{}{adj.2g.}{Que se pode solver, pagar  o que deve. }{sol.ví.vel}{0}
\verb{som}{}{}{}{}{s.m.}{Fenômeno acústico caracterizado pela propagação de ondas mecânicas produzidas pela vibração de um corpo e que pode ser percebido pelo aparelho auditivo humano.}{som}{0}
\verb{som}{}{}{}{}{}{A sensação auditiva causada por essas ondas.}{som}{0}
\verb{som}{}{}{}{}{}{Emissão de voz; os sons da fala.}{som}{0}
\verb{som}{}{Pop.}{}{}{}{Música.}{som}{0}
\verb{som}{}{Pop.}{}{}{}{Equipamento sonoro eletrônico.}{som}{0}
\verb{soma}{}{}{}{}{s.f.}{Operação aritmética de adição.}{so.ma}{0}
\verb{soma}{}{Biol.}{}{}{s.m.}{O conjunto das células de um organismo, com exceção dos gametas.}{so.ma}{0}
\verb{soma}{}{}{}{}{}{O resultado de uma operação de adição.}{so.ma}{0}
\verb{soma}{}{}{}{}{}{Quantia.}{so.ma}{0}
\verb{soma}{}{Fig.}{}{}{}{Grande porção; abundância.}{so.ma}{0}
\verb{soma}{}{Fig.}{}{}{}{Totalidade, conjunto.}{so.ma}{0}
\verb{soma}{}{Med.}{}{}{}{O organismo físico, por oposição a psíquico.}{so.ma}{0}
\verb{somali}{}{}{}{}{adj.2g.}{Relativo à Somália (nordeste da África).}{so.ma.li}{0}
\verb{somali}{}{}{}{}{s.2g.}{Indivíduo natural ou habitante desse país.}{so.ma.li}{0}
\verb{somali}{}{}{}{}{s.m.}{Língua falada na Somália, Etiópia, Quênia e Djibuti.}{so.ma.li}{0}
\verb{somar}{}{}{}{}{v.i.}{Realizar a operação de adição.}{so.mar}{0}
\verb{somar}{}{}{}{}{v.t.}{Fazer a soma; adicionar.}{so.mar}{0}
\verb{somar}{}{}{}{}{}{Juntar a determinado conjunto.}{so.mar}{\verboinum{1}}
\verb{somático}{}{}{}{}{adj.}{Relativo ao corpo (em oposição a \textit{psíquico}).}{so.má.ti.co}{0}
\verb{somatizar}{}{}{}{}{v.t.}{Manifestar doença ou sintomas físicos provocados por problemas psíquicos.}{so.ma.ti.zar}{\verboinum{1}}
\verb{somatório}{}{}{}{}{adj.}{Relativo a soma matemática.}{so.ma.tó.rio}{0}
\verb{somatório}{}{}{}{}{s.m.}{Soma de todos os termos de uma sequência.}{so.ma.tó.rio}{0}
\verb{sombra}{}{}{}{}{s.f.}{Falta de luz produzida por um corpo opaco.}{som.bra}{0}
\verb{sombra}{}{}{}{}{}{Lugar onde não bate luz direta; escuridão, obscuridade.}{som.bra}{0}
\verb{sombra}{}{}{}{}{}{Em um desenho ou pintura, as partes onde existem tons mais escuros.}{som.bra}{0}
\verb{sombra}{}{Fig.}{}{}{}{Ausência de instrução, cultura, justiça; ignorância, despotismo.}{som.bra}{0}
\verb{sombra}{}{Fig.}{}{}{}{Indício, traço, sinal.}{som.bra}{0}
\verb{sombra}{}{Fig.}{}{}{}{Isolamento, solidão.}{som.bra}{0}
\verb{sombra}{}{}{}{}{}{Maquiagem utilizada na área dos olhos para realçar traços ou dar expressividade.}{som.bra}{0}
\verb{sombreado}{}{}{}{}{adj.}{Em que há sombra.}{som.bre.a.do}{0}
\verb{sombreado}{}{}{}{}{s.m.}{Uso de tons escuros em desenho ou pintura, para dar efeito de volume.}{som.bre.a.do}{0}
\verb{sombrear}{}{}{}{}{v.t.}{Produzir sombras.}{som.bre.ar}{0}
\verb{sombrear}{}{}{}{}{}{Criar efeito de sombreado.}{som.bre.ar}{\verboinum{4}}
\verb{sombreiro}{ê}{}{}{}{s.m.}{Qualquer objeto que produz sombra.}{som.brei.ro}{0}
\verb{sombreiro}{ê}{}{}{}{}{Chapéu de abas bastante largas.}{som.brei.ro}{0}
\verb{sombrinha}{}{}{}{}{s.f.}{Pequeno guarda"-chuva colorido ou bordado utilizado para proteger do sol.}{som.bri.nha}{0}
\verb{sombrio}{}{}{}{}{adj.}{Em que há pouca luz.}{som.bri.o}{0}
\verb{sombrio}{}{Fig.}{}{}{}{Triste, melancólico, taciturno, solitário, ermo.}{som.bri.o}{0}
\verb{sombroso}{ô}{}{"-osos ⟨ó⟩}{"-osa ⟨ó⟩}{adj.}{Em que há muita sombra.}{som.bro.so}{0}
\verb{sombroso}{ô}{Fig.}{"-osos ⟨ó⟩}{"-osa ⟨ó⟩}{}{Triste, sombrio.}{som.bro.so}{0}
\verb{somenos}{}{}{}{}{adj.}{Que é de menor valor que outro; inferior.}{so.me.nos}{0}
\verb{somente}{ó}{}{}{}{adv.}{Não mais que, unicamente, só, exclusivamente.}{so.men.te}{0}
\verb{somiticaria}{}{}{}{}{s.f.}{Qualidade ou ação de somítico; mesquinharia, avareza.}{so.mi.ti.ca.ri.a}{0}
\verb{somítico}{}{}{}{}{adj.}{Diz"-se daquele que é avarento.}{so.mí.ti.co}{0}
\verb{somítico}{}{}{}{}{s.m.}{Indivíduo mesquinho.}{so.mí.ti.co}{0}
\verb{sonambulismo}{}{}{}{}{s.m.}{O estado ou doença do sonâmbulo.}{so.nam.bu.lis.mo}{0}
\verb{sonâmbulo}{}{}{}{}{adj.}{Que anda, fala e executa movimentos durante o sono.}{so.nâm.bu.lo}{0}
\verb{sonâmbulo}{}{}{}{}{s.m.}{Indivíduo que se levanta, anda e fala durante o sono. }{so.nâm.bu.lo}{0}
\verb{sonante}{}{}{}{}{adj.2g.}{Que produz som; que soa.}{so.nan.te}{0}
\verb{sonar}{}{}{}{}{s.m.}{Radar submarino guiado por ondas acústicas, usado para detectar objetos imersos em água e determinar"-lhes a posição e a velocidade. }{so.nar}{0}
\verb{sonar}{}{}{}{}{adj.2g.}{Relativo a esse radar.}{so.nar}{0}
\verb{sonata}{}{Mús.}{}{}{s.f.}{Composição musical em um ou diversos movimentos, para solista ou conjunto instrumental.}{so.na.ta}{0}
\verb{sonata}{}{}{}{}{}{Melodia agradável ao ouvido.}{so.na.ta}{0}
\verb{sonda}{}{}{}{}{s.f.}{Peça de chumbo presa a uma linha, usada para medir a profundidade de água ou buraco.}{son.da}{0}
\verb{sonda}{}{}{}{}{}{Broca que perfura a terra em busca de água, petróleo etc.}{son.da}{0}
\verb{sonda}{}{}{}{}{}{Aparelho colocado numa nave espacial para mandar informações sobre o espaço à Terra.}{son.da}{0}
\verb{sonda}{}{}{}{}{}{Tubo introduzido no corpo para examinar órgãos.}{son.da}{0}
\verb{sondagem}{}{}{"-ens}{}{s.f.}{Ato ou efeito de sondar.}{son.da.gem}{0}
\verb{sondagem}{}{}{"-ens}{}{}{Investigação, pesquisa.}{son.da.gem}{0}
\verb{sondagem}{}{}{"-ens}{}{}{Exploração demorada e metódica de um meio, pela utilização de aparelhos e processos científicos.}{son.da.gem}{0}
\verb{sondar}{}{}{}{}{v.t.}{Examinar uma coisa por meio de uma sonda.}{son.dar}{0}
\verb{sondar}{}{}{}{}{}{Fazer perguntas para descobrir a opinião de alguém; inquerir.}{son.dar}{\verboinum{1}}
\verb{soneca}{é}{Pop.}{}{}{s.f.}{Breve espaço de tempo que se passa dormindo; pequeno sono; cochilo.}{so.ne.ca}{0}
\verb{sonegação}{}{}{"-ões}{}{s.f.}{Ato ou efeito de sonegar, de ocultar, deixando de mencionar ou descrever.}{so.ne.ga.ção}{0}
\verb{sonegação}{}{}{"-ões}{}{}{Falta deliberada e fraudulenta de pagamento de algum imposto.}{so.ne.ga.ção}{0}
%\verb{}{}{}{}{}{}{}{}{0}
\verb{sonegar}{}{}{}{}{v.t.}{Ocultar, deixando de escrever ou mencionar nos casos em que a lei exige descrição ou a menção.}{so.ne.gar}{\verboinum{5}}
\verb{soneira}{ê}{}{}{}{s.f.}{Ver \textit{sonolência}.}{so.nei.ra}{0}
\verb{sonetista}{}{}{}{}{adj.2g.}{Que faz sonetos.}{so.ne.tis.ta}{0}
\verb{sonetista}{}{}{}{}{s.2g.}{Indivíduo que escreve sonetos.}{so.ne.tis.ta}{0}
\verb{soneto}{ê}{Liter.}{}{}{s.m.}{Poema de forma fixa composto de catorze versos, dois quartetos e dois tercetos.}{so.ne.to}{0}
\verb{songamonga}{}{Pop.}{}{}{s.2g.}{Indivíduo sonso, fingido, hipócrita.}{son.ga.mon.ga}{0}
\verb{sonhador}{ô}{}{}{}{adj.}{Que sonha, que devaneia.}{so.nha.dor}{0}
\verb{sonhador}{ô}{}{}{}{s.m.}{Indivíduo que sonha; devaneador.}{so.nha.dor}{0}
\verb{sonhar}{}{}{}{}{v.i.}{Ter sonho enquanto dorme.}{so.nhar}{0}
\verb{sonhar}{}{}{}{}{}{Entregar"-se a fantasias e devaneios.}{so.nhar}{0}
\verb{sonhar}{}{}{}{}{v.t.}{Desejar com insistência; ter a ideia fixa.}{so.nhar}{0}
\verb{sonhar}{}{}{}{}{}{Ver em sonho.}{so.nhar}{\verboinum{1}}
\verb{sonho}{}{}{}{}{s.m.}{Sequência de fenômenos psíquicos, conjunto de imagens e cenas, que involuntariamente ocorrem durante o sono.}{so.nho}{0}
\verb{sonho}{}{}{}{}{}{Coisa que se deseja muito.}{so.nho}{0}
\verb{sonho}{}{Cul.}{}{}{}{Pequeno bolo esférico, preparado com farinha de trigo cozida, leite e ovos, frito em gordura quente, e passado em açúcar e canela, ou servido com calda rala, podendo também ser recheado.}{so.nho}{0}
\verb{sônico}{}{}{}{}{adj.}{Relativo ao som.}{sô.ni.co}{0}
\verb{sônico}{}{}{}{}{}{Relativo à velocidade do som.}{sô.ni.co}{0}
\verb{sonido}{}{}{}{}{s.m.}{Qualquer som; rumor, ruído.}{so.ni.do}{0}
\verb{sonífero}{}{}{}{}{adj.}{Que produz ou provoca som.}{so.ní.fe.ro}{0}
\verb{sonífero}{}{}{}{}{}{Diz"-se de substância que possui a propriedade de provocar sono.}{so.ní.fe.ro}{0}
\verb{sonífero}{}{}{}{}{s.m.}{Substância ou medicamento que provoca o sono.}{so.ní.fe.ro}{0}
\verb{sono}{}{}{}{}{s.m.}{Estado fisiológico temporário, que ocorre periodicamente, caracterizado pela supressão da vigilância, desaceleração do metabolismo, relaxamento muscular, diminuição da atividade sensorial, suspensão das experiências conscientes.}{so.no}{0}
\verb{sono}{}{}{}{}{}{Vontade, desejo ou necessidade de dormir.}{so.no}{0}
\verb{sono}{}{}{}{}{}{Estado de pessoa ou animal que dorme.}{so.no}{0}
\verb{sono}{}{Fig.}{}{}{}{Inércia, inatividade.}{so.no}{0}
\verb{sonolência}{}{}{}{}{s.f.}{Grande vontade de dormir; sono.}{so.no.lên.cia}{0}
\verb{sonolência}{}{}{}{}{}{Sono imperfeito, incompleto.}{so.no.lên.cia}{0}
\verb{sonolência}{}{}{}{}{}{Momento de transição entre o estado de quem dorme e o acordar.}{so.no.lên.cia}{0}
\verb{sonolência}{}{}{}{}{}{Estado de torpor, falta de vigor físico e de ânimo.}{so.no.lên.cia}{0}
\verb{sonolento}{}{}{}{}{adj.}{Que sente sonolência.}{so.no.len.to}{0}
\verb{sonolento}{}{}{}{}{}{Que causa sono.}{so.no.len.to}{0}
\verb{sonolento}{}{Fig.}{}{}{}{Que se move devagar; mole, vagaroso.}{so.no.len.to}{0}
\verb{sonoplasta}{}{}{}{}{s.2g.}{Indivíduo que trabalha em sonoplastia, ou se dedica ao estudo dela.}{so.no.plas.ta}{0}
\verb{sonoplastia}{}{}{}{}{s.f.}{Técnica de compor e fazer funcionar os ruídos e efeitos acústicos e musicais que constituem o elemento sonoro dos espetáculos teatrais, filmes, programas de rádio e televisão etc.}{so.no.plas.ti.a}{0}
\verb{sonoridade}{}{}{}{}{s.f.}{Qualidade do que é sonoro.}{so.no.ri.da.de}{0}
\verb{sonoridade}{}{}{}{}{}{Propriedade que têm certos corpos de emitir sons intensos ou sons de frequência regular.}{so.no.ri.da.de}{0}
\verb{sonoridade}{}{}{}{}{}{Propriedade, que apresentam certos corpos ou ambientes, de reforçar os sons.}{so.no.ri.da.de}{0}
\verb{sonorização}{}{}{"-ões}{}{s.f.}{Ato ou efeito de sonorizar, de tornar sonoro, de produzir som.}{so.no.ri.za.ção}{0}
\verb{sonorizador}{ô}{}{}{}{adj.}{Que produz som.}{so.no.ri.za.dor}{0}
\verb{sonorizador}{ô}{}{}{}{s.m.}{Redutor de velocidade instalado em ruas ou estradas que, por apresentar relevos e concavidades alternados e sucessivos, faz o carro trepidar ruidosamente.}{so.no.ri.za.dor}{0}
\verb{sonorizar}{}{}{}{}{v.t.}{Tornar sonoro.}{so.no.ri.zar}{0}
\verb{sonorizar}{}{}{}{}{v.i.}{Produzir som.}{so.no.ri.zar}{\verboinum{1}}
\verb{sonoro}{ó}{}{}{}{adj.}{Que produz ou reforça o som.}{so.no.ro}{0}
\verb{sonoro}{ó}{}{}{}{}{Que emite som intenso.}{so.no.ro}{0}
\verb{sonoro}{ó}{}{}{}{}{Que é agradável ao ouvido; harmonioso, melodioso.}{so.no.ro}{0}
\verb{sonoroso}{ô}{}{"-osos ⟨ó⟩}{"-osa ⟨ó⟩}{adj.}{Que dá som agradável; melodioso.}{so.no.ro.so}{0}
\verb{sonoroso}{ô}{}{"-osos ⟨ó⟩}{"-osa ⟨ó⟩}{}{Que é estrondoso; estripitoso.}{so.no.ro.so}{0}
\verb{sonoroso}{ô}{}{"-osos ⟨ó⟩}{"-osa ⟨ó⟩}{}{Diz"-se daquele que fala ou escreve em tom sonoro.}{so.no.ro.so}{0}
\verb{sonoterapia}{}{Med.}{}{}{s.f.}{Tratamento de certas doenças mentais que consiste em provocar no paciente, mediante o emprego de hipnóticos, um sono artificial profundo e prolongado.}{so.no.te.ra.pi.a}{0}
\verb{sonso}{}{}{}{}{adj.}{Esperto, manhoso, dissimulado, fingido.}{son.so}{0}
\verb{sopa}{ô}{Cul.}{}{}{s.f.}{Caldo com carne, legumes, massas ou outras substâncias sólidas, servido, normalmente, como o primeiro prato do jantar. }{so.pa}{0}
\verb{sopa}{ô}{Pop.}{}{}{}{Coisa fácil de ser feita, vencida ou resolvida.}{so.pa}{0}
\verb{sopapear}{}{}{}{}{v.t.}{Atingir alguém com sopapos; esbofetear.}{so.pa.pe.ar}{\verboinum{4}}
\verb{sopapo}{}{}{}{}{s.m.}{Murro, soco.}{so.pa.po}{0}
\verb{sopapo}{}{}{}{}{}{Bofetão, tapa.}{so.pa.po}{0}
\verb{sopé}{}{}{}{}{s.m.}{Parte inferior, mais próxima do plano; base.}{so.pé}{0}
\verb{sopeira}{ê}{}{}{}{s.f.}{Vasilha em que se leva a sopa para a mesa, geralmente larga e bojuda, com tampa.}{so.pei.ra}{0}
\verb{sopesar}{}{}{}{}{v.t.}{Calcular com a mão o peso de algo.}{so.pe.sar}{0}
\verb{sopesar}{}{}{}{}{}{Aguentar o peso de algo.}{so.pe.sar}{0}
\verb{sopesar}{}{}{}{}{}{Repartir metódica ou parcimoniosamente.}{so.pe.sar}{0}
\verb{sopesar}{}{}{}{}{v.pron.}{Manter"-se em equilíbrio; equilibrar"-se.}{so.pe.sar}{\verboinum{1}}
\verb{sopitado}{}{}{}{}{adj.}{Que caiu em sonolência; sonolento.}{so.pi.ta.do}{0}
\verb{sopitado}{}{Por ext.}{}{}{}{Que se acalmou; serenado, abrandado.}{so.pi.ta.do}{0}
\verb{sopitar}{}{}{}{}{v.t.}{Fazer dormir; adormecer.}{so.pi.tar}{0}
\verb{sopitar}{}{}{}{}{}{Abrandar, acalmar.}{so.pi.tar}{0}
\verb{sopitar}{}{}{}{}{}{Enfraquecer, debilitar.}{so.pi.tar}{0}
\verb{sopitar}{}{}{}{}{}{Dominar, vencer.}{so.pi.tar}{\verboinum{1}}
\verb{sopor}{ô}{}{}{}{s.m.}{Sono mórbido; torpor, modorra.}{so.por}{0}
\verb{sopor}{ô}{}{}{}{}{Disposição para dormir; sonolência.}{so.por}{0}
\verb{sopor}{ô}{}{}{}{}{Estado de quem está em coma.}{so.por}{0}
\verb{soporífero}{}{}{}{}{adj.}{Que produz sono ou sopor.}{so.po.rí.fe.ro}{0}
\verb{soporífero}{}{Por ext.}{}{}{}{Que é enfadonho, cansativo.}{so.po.rí.fe.ro}{0}
\verb{soporífero}{}{}{}{}{s.m.}{Substância que sopita, que faz dormir.}{so.po.rí.fe.ro}{0}
\verb{soporífero}{}{Por ext.}{}{}{}{Coisa fastidiosa, maçante.}{so.po.rí.fe.ro}{0}
\verb{soporífico}{}{}{}{}{adj. e s.m.  }{Ver \textit{soporífero}.}{so.po.rí.fi.co}{0}
\verb{soprano}{}{Mús.}{}{}{s.m.}{Voz mais aguda de mulher ou menino.}{so.pra.no}{0}
\verb{soprano}{}{}{}{}{s.2g.}{Cantor que possui essa voz.}{so.pra.no}{0}
\verb{soprar}{}{}{}{}{v.t.}{Expelir o ar pela boca ou nariz; assoprar.}{so.prar}{0}
\verb{soprar}{}{}{}{}{}{Falar em voz baixa; cochichar, segredar, sussurrar.}{so.prar}{0}
\verb{soprar}{}{}{}{}{v.i.}{Movimentar"-se o ar, o vento.}{so.prar}{\verboinum{1}}
\verb{sopro}{ô}{}{}{}{s.m.}{Ato ou efeito de soprar; exalação, expiração, assopro.}{so.pro}{0}
\verb{sopro}{ô}{}{}{}{}{Vento; brisa; aragem.}{so.pro}{0}
\verb{soquear}{}{}{}{}{v.t.}{Dar murros; socar, esmurrar.}{so.que.ar}{\verboinum{4}}
\verb{soquete}{ê}{}{}{}{s.m.}{Suporte para lâmpadas elétricas; bocal.}{so.que.te}{0}
\verb{soquete}{é}{}{}{}{s.f.}{Meia curta feminina que vai até o tornozelo.}{so.que.te}{0}
\verb{sordidez}{ê}{}{}{}{}{Avareza exacerbada; mesquinharia.}{sor.di.dez}{0}
\verb{sordidez}{ê}{}{}{}{s.f.}{Qualidade ou condição do que é sórdido; torpeza, vileza, indignidade, infâmia.}{sor.di.dez}{0}
\verb{sórdido}{}{}{}{}{adj.}{Que está sujo; imundo, asqueroso, nojento.}{sór.di.do}{0}
\verb{sórdido}{}{}{}{}{}{Indecente, indigno, vergonhoso, torpe, ignóbil. }{sór.di.do}{0}
\verb{sórdido}{}{}{}{}{}{Avaro; mesquinho; miserável.}{sór.di.do}{0}
\verb{sorgo}{ô}{}{}{}{s.m.}{Planta da família das gramíneas, cujos grãos arredondados, amarelos, brancos ou vermelhos, são usados na alimentação humana e como ração.}{sor.go}{0}
\verb{sorgo}{ô}{}{}{}{}{Os grãos desse cereal.}{sor.go}{0}
\verb{sorna}{ô}{}{}{}{adj.}{Que aborrece; chato, maçante.}{sor.na}{0}
\verb{sorna}{ô}{}{}{}{s.f.}{Lerdeza, preguiça, moleza.}{sor.na}{0}
\verb{sorna}{ô}{}{}{}{}{Soneca, sono.}{sor.na}{0}
\verb{soro}{ô}{}{}{}{s.m.}{Líquido amarelo claro, remanescente da coagulação do leite.}{so.ro}{0}
\verb{soro}{ô}{Med.}{}{}{}{Solução de substância orgânica ou mineral usada na hidratação ou alimentação de pacientes.}{so.ro}{0}
%\verb{}{}{}{}{}{}{}{}{0}
\verb{sorologia}{}{Med.}{}{}{s.f.}{Ramo da imunologia que estuda, \textit{in vitro}, as reações que ocorrem entre antígeno e anticorpo no soro.}{so.ro.lo.gi.a}{0}
\verb{soropositivo}{}{Med.}{}{}{adj.}{Diz"-se daquele cujo soro sanguíneo contém os anticorpos específicos de um determinado antígeno.}{so.ro.po.si.ti.vo}{0}
\verb{soropositivo}{}{Bras.}{}{}{adj.}{Aquele cuja análise sanguínea revela a presença do vírus da aids.}{so.ro.po.si.ti.vo}{0}
\verb{soror}{}{}{}{}{}{Var. de \textit{sóror}.}{so.ror}{0}
\verb{sóror}{}{}{}{}{s.f.}{Forma de tratamento dada às freiras.}{só.ror}{0}
\verb{sororoca}{ó}{}{}{}{s.f.}{Respiração arquejante dos moribundos; estertor.}{so.ro.ro.ca}{0}
\verb{sorrateiro}{ê}{}{}{}{adj.}{Que age secretamente, às escondidas. }{sor.ra.tei.ro}{0}
\verb{sorrateiro}{ê}{}{}{}{}{Dissimulado, matreiro, manhoso.}{sor.ra.tei.ro}{0}
\verb{sorrelfa}{é}{}{}{}{s.f.}{Disfarce para enganar, iludir.}{sor.rel.fa}{0}
\verb{sorrelfa}{é}{}{}{}{adj.}{Diz"-se daquele que é dissimulado, matreiro, manhoso.}{sor.rel.fa}{0}
\verb{sorridente}{}{}{}{}{adj.2g.}{Que sorri, demonstra alegria; risonho, alegre.}{sor.ri.den.te}{0}
\verb{sorridente}{}{Fig.}{}{}{}{Que é promissor, favorável, propício.}{sor.ri.den.te}{0}
\verb{sorrir}{}{}{}{}{v.i.}{Contrair de leve os músculos do rosto, sem fazer barulho, por estar alegre, ou diante de uma coisa engraçada; rir de leve.}{sor.rir}{\verboinum{57}}
\verb{sorriso}{}{}{}{}{s.m.}{Movimento da boca e dos olhos que pode exprimir contentamento, alegria, amabilidade, ironia, desdém, malícia etc.; riso.}{sor.ri.so}{0}
\verb{sorte}{ó}{}{}{}{s.f.}{Força que determina todos os acontecimentos da vida; destino, fado, sina, fatalidade.}{sor.te}{0}
\verb{sorte}{ó}{}{}{}{}{Fortuna, felicidade, ventura, boa estrela, boa sorte.}{sor.te}{0}
\verb{sorte}{ó}{}{}{}{}{Acontecimento casual; acaso, casualidade, eventualidade.}{sor.te}{0}
\verb{sorte}{ó}{}{}{}{}{Série de coincidências ou acasos felizes que acontecem a uma pessoa.}{sor.te}{0}
\verb{sorte}{ó}{}{}{}{}{O prêmio em loteria ou sorteio.}{sor.te}{0}
\verb{sorte}{ó}{}{}{}{}{Bilhete de rifa ou loteria.}{sor.te}{0}
\verb{sorteado}{}{}{}{}{adj.}{Diz"-se do escolhido por sorteio ou sorte.}{sor.te.a.do}{0}
\verb{sortear}{}{}{}{}{v.t.}{Fazer sorteio, tirar ao acaso; rifar.}{sor.te.ar}{\verboinum{4}}
\verb{sorteio}{ê}{}{}{}{s.m.}{Ato ou efeito de sortear; escolher ao acaso.}{sor.tei.o}{0}
\verb{sortido}{ô}{}{}{}{adj.}{Abastecido; aprovisionado; provido.}{sor.ti.do}{0}
\verb{sortido}{ô}{}{}{}{}{De muitos tipos; variado.}{sor.ti.do}{0}
\verb{sortilégio}{}{}{}{}{s.m.}{Encantamento; feitiço; magia; bruxaria.}{sor.ti.lé.gio}{0}
\verb{sortilégio}{}{}{}{}{}{Encanto ou sedução produzida por dons naturais ou mágicos.}{sor.ti.lé.gio}{0}
\verb{sortimento}{}{}{}{}{s.m.}{Ato ou efeito de sortir, prover de quaisquer produtos; variedade, sortido.}{sor.ti.men.to}{0}
\verb{sortimento}{}{}{}{}{}{Estoque de mercadorias; provisão, abastecimento.}{sor.ti.men.to}{0}
\verb{sortir}{}{}{}{}{v.t.}{Aprovisionar; abastecer; prover.}{sor.tir}{0}
\verb{sortir}{}{}{}{}{}{Dispor coisas diferentes; combinar, mesclar, misturar.}{sor.tir}{\verboinum{18}}
\verb{sortudo}{}{}{}{}{adj.}{Diz"-se daquele que tem muita sorte ou sorte constante; felizardo.}{sor.tu.do}{0}
\verb{sorumbático}{}{}{}{}{adj.}{Diz"-se daquele de ar triste, sombrio, melancólico; macambúzio, jururu.}{so.rum.bá.ti.co}{0}
\verb{sorva}{ô}{}{}{}{s.f.}{O fruto comestível da sorveira.}{sor.va}{0}
\verb{sorvedoiro}{ô}{}{}{}{}{Var. de \textit{sorvedouro}.}{sor.ve.doi.ro}{0}
\verb{sorvedouro}{ô}{}{}{}{s.m.}{Redemoinho de águas que arrasta para o fundo; turbilhão, abismo, voragem, sumidouro.}{sor.ve.dou.ro}{0}
\verb{sorvedouro}{ô}{}{}{}{}{Aquilo que gera desperdício, ruína, morte.}{sor.ve.dou.ro}{0}
\verb{sorveira}{ê}{Bot.}{}{}{s.f.}{Árvore nativa da Amazônia, de flores róseas, que alcança até 30 metros de altura e cujo fruto comestível é a sorva.}{sor.vei.ra}{0}
\verb{sorver}{ê}{}{}{}{v.t.}{Beber aos poucos, aos sorvos; sugar.}{sor.ver}{0}
\verb{sorver}{ê}{}{}{}{}{Engolir, tragar.}{sor.ver}{0}
\verb{sorver}{ê}{}{}{}{}{Inspirar, haurir, aspirar, inalar.}{sor.ver}{0}
\verb{sorver}{ê}{}{}{}{}{Desperdiçar, aniquilar, destruir.}{sor.ver}{\verboinum{12}}
\verb{sorvete}{ê}{}{}{}{s.m.}{Iguaria, geralmente adocicada, feita de creme ou suco de frutas congelado.}{sor.ve.te}{0}
\verb{sorveteira}{ê}{}{}{}{s.f.}{Máquina, utensílio para fazer sorvetes.}{sor.ve.tei.ra}{0}
\verb{sorveteiro}{ê}{}{}{}{s.m.}{Indivíduo que faz ou vende sorvetes.}{sor.ve.tei.ro}{0}
\verb{sorveteria}{}{Bras.}{}{}{s.f.}{Estabelecimento que fabrica ou vende sorvetes.}{sor.ve.te.ri.a}{0}
\verb{sorvo}{ô}{}{}{}{s.m.}{Ato ou efeito de sorver.}{sor.vo}{0}
\verb{sorvo}{ô}{}{}{}{}{Trago, gole.}{sor.vo}{0}
\verb{SOS}{}{}{}{}{}{Código internacional para pedido de socorro; abrev. de \textit{save our souls}.}{S.O.S.}{0}
\verb{sósia}{}{}{}{}{s.2g.}{Indivíduo muito semelhante a outro, a ponto de ser confundido.}{só.sia}{0}
\verb{soslaio}{}{}{}{}{s.m.}{Esguelha, obliquidade, viés.}{sos.lai.o}{0}
\verb{soslaio}{}{}{}{}{}{Usado na expressão \textit{de soslaio}: obliquamente, de esguelha.}{sos.lai.o}{0}
\verb{sossegar}{}{}{}{}{v.t.}{Pôr em sossego; aquietar, acalmar, tranquilizar.}{sos.se.gar}{0}
\verb{sossegar}{}{}{}{}{}{Descansar por fim, ou morrer, depois de sofrimento prolongado.}{sos.se.gar}{\verboinum{5}}
\verb{sossego}{ê}{}{}{}{s.m.}{Ato ou efeito de sossegar.}{sos.se.go}{0}
\verb{sossego}{ê}{}{}{}{}{Tranquilidade, calma, descanso, paz.}{sos.se.go}{0}
\verb{sotaina}{}{}{}{}{s.f.}{Batina usada por sacerdote católico.}{so.tai.na}{0}
\verb{sótão}{}{}{"-ãos}{}{s.m.}{Parte de uma habitação entre o forro e o telhado; desvão.}{só.tão}{0}
\verb{sótão}{}{}{"-ãos}{}{}{Compartimento entre o chão e o pavimento de uma habitação, geralmente usado como depósito.}{só.tão}{0}
\verb{sotaque}{}{}{}{}{s.m.}{Pronúncia, inflexão característica de um país, região, cidade, indivíduo etc.; acento.}{so.ta.que}{0}
\verb{sotaque}{}{}{}{}{}{Pronúncia imprecisa de um indivíduo ao falar uma língua estrangeira.}{so.ta.que}{0}
\verb{sotavento}{}{}{}{}{s.m.}{Lado de uma embarcação oposto àquele de onde sopra o vento.}{so.ta.ven.to}{0}
\verb{soteropolitano}{}{}{}{}{adj. e s.m.  }{Ver \textit{salvadorense}.}{so.te.ro.po.li.ta.no}{0}
\verb{soterramento}{}{}{}{}{s.m.}{Ato ou efeito de soterrar, cobrir de terra; soterração.}{so.ter.ra.men.to}{0}
\verb{soterrar}{}{}{}{}{v.t.}{Cobrir de terra, escombros etc.}{so.ter.rar}{\verboinum{1}}
\verb{sotopor}{}{}{}{}{v.t.}{Pôr debaixo; subpor.}{so.to.por}{0}
\verb{sotopor}{}{}{}{}{}{Adiar, postergar, procrastinar.}{so.to.por}{\verboinum{60}}
\verb{sotoposto}{ô}{}{"-s ⟨ó⟩}{"-a ⟨ó⟩}{adj.}{Colocado por baixo.}{so.to.pos.to}{0}
\verb{sotoposto}{ô}{}{"-s ⟨ó⟩}{"-a ⟨ó⟩}{}{Deixado de lado; adiado.}{so.to.pos.to}{0}
\verb{soturno}{}{}{}{}{adj.}{Diz"-se do que é tristonho, taciturno, melancólico.}{so.tur.no}{0}
\verb{soturno}{}{}{}{}{}{Sombrio, escuro, lúgubre.}{so.tur.no}{0}
\verb{souto}{ô}{}{}{}{s.m.}{Mata ou bosque cerrado.}{sou.to}{0}
\verb{souvenir}{}{}{}{}{s.m.}{Suvenir.}{\textit{souvenir}}{0}
\verb{sova}{ó}{}{}{}{s.f.}{Ato ou efeito de sovar; surra, tunda.}{so.va}{0}
\verb{sova}{ó}{Fig.}{}{}{}{Derrota esmagadora, humilhante.}{so.va}{0}
\verb{sovaco}{}{}{}{}{s.m.}{Cavidade na parte inferior na junção do braço com o tronco; axila.}{so.va.co}{0}
\verb{sovar}{}{}{}{}{v.t.}{Amassar ou misturar a massa do pão, batendo"-a. }{so.var}{0}
\verb{sovar}{}{}{}{}{}{Dar uma sova; esmurrar, socar.}{so.var}{0}
\verb{sovar}{}{Fig.}{}{}{}{Usar muito (roupa, sapato etc.); surrar.}{so.var}{\verboinum{1}}
\verb{sovela}{é}{}{}{}{s.f.}{Instrumento usado pelos sapateiros para perfurar o couro e costurá"-lo. }{so.ve.la}{0}
\verb{soverter}{ê}{Bras.}{}{}{v.i.}{Desaparecer; sumir.}{so.ver.ter}{0}
\verb{soverter}{ê}{}{}{}{v.t.}{Cobrir de terra ou água; soterrar.}{so.ver.ter}{\verboinum{12}}
\verb{soviete}{é}{}{}{}{s.m.}{Conselho composto de representantes eleitos de operários, camponeses e soldados, na antiga União Soviética (\textsc{urss}), ao qual competia a eleição dos dirigentes políticos e membros das assembleias.}{so.vi.e.te}{0}
\verb{soviete}{é}{}{}{}{}{Estado organizado segundo o sistema comunista.}{so.vi.e.te}{0}
\verb{soviético}{}{}{}{}{adj.}{Relativo aos sovietes ou à antiga União Soviética (Europa e Ásia).}{so.vi.é.ti.co}{0}
\verb{soviético}{}{}{}{}{s.m.}{Indivíduo natural ou habitante da antiga União Soviética.}{so.vi.é.ti.co}{0}
\verb{sovina}{}{}{}{}{s.f.}{Torno de madeira.}{so.vi.na}{0}
\verb{sovina}{}{Fig.}{}{}{adj.}{Que tem aversão a gastar, geralmente dinheiro, mesmo quando é preciso, e se compraz em acumular; avaro; avarento; mesquinho.}{so.vi.na}{0}
\verb{sovinice}{}{}{}{}{s.f.}{Qualidade do que é sovina; avareza, mesquinharia.}{so.vi.ni.ce}{0}
\verb{sozinho}{}{}{}{}{adj.}{Sem companhia; só, isolado, único.}{so.zi.nho}{0}
\verb{SP}{}{}{}{}{}{Sigla do estado de São Paulo.}{SP}{0}
\verb{speaker}{}{Desus.}{}{}{s.m.}{Palavra inglesa que designa o locutor ou apresentador de programa de rádio.}{\textit{speaker}}{0}
\verb{spot}{}{}{}{}{s.m.}{Pequena luminária direcionável que concentra a luz num feixe estreito.}{\textit{spot}}{0}
\verb{spray}{}{}{}{}{s.m.}{Recipiente fechado com jato de aerossol.}{\textit{spray}}{0}
\verb{Sr}{}{Quím.}{}{}{}{Símb. do \textit{estrôncio}.}{Sr}{0}
\verb{SSE}{}{}{}{}{}{Abrev. de \textit{su"-sueste}.  }{S.S.E.}{0}
\verb{SSO}{}{}{}{}{}{Abrev. de \textit{su"-sudoeste}.  }{S.S.O.}{0}
\verb{SSW}{}{}{}{}{}{Abrev. de \textit{su"-sudoeste}.  }{S.S.W.}{0}
\verb{status}{}{}{}{}{s.m.}{Posição social privilegiada.}{\textit{status}}{0}
\verb{status}{}{}{}{}{}{Prestígio, renome.}{\textit{status}}{0}
\verb{status quo}{ó}{}{}{}{}{Expressão latina que significa \textit{estado atual das coisas, situação inalterada}.}{\textit{status quo}}{0}
\verb{stricto sensu}{ê}{}{}{}{}{Expressão latina que significa \textit{em sentido restrito}.}{\textit{stricto sensu}}{0}
\verb{strip"-tease}{}{}{}{}{s.m.}{Espetáculo em que o ator ou atriz vai tirando a roupa de forma provocante, ao ritmo lento e sugestivo de uma música de fundo.}{\textit{strip"-tease}}{0}
\verb{sua}{}{Gram.}{}{}{pron.}{Pronome possessivo que determina pessoa ou coisa do gênero feminino que pertence a alguém (inclusive ao interlocutor) ou a algo de que se fala; dele(s), dela(s). (\textit{Ele esteve verificando seus papéis até a hora de sair.})}{su.a}{0}
\verb{suã}{}{}{}{}{s.f.}{Parte inferior do lombo do porco, no final da espinha.}{su.ã}{0}
\verb{suã}{}{Bras.}{}{}{}{Espinha ou vértebra de qualquer animal.}{su.ã}{0}
\verb{sua}{}{}{}{}{}{(\textit{Sua sortuda, ganhou outra vez!})}{su.a}{0}
\verb{sua}{}{}{}{}{}{(\textit{Eu vi quando você chegou com sua mãe.})}{su.a}{0}
\verb{suado}{}{}{}{}{adj.}{Molhado de suor.}{su.a.do}{0}
\verb{suado}{}{Fig.}{}{}{}{Conseguido à custa de muito esforço ou trabalho.}{su.a.do}{0}
\verb{suador}{ô}{}{}{}{adj.}{Que sua.}{su.a.dor}{0}
\verb{suador}{ô}{}{}{}{s.m.}{Remédio, bebida ou chá que faz suar.}{su.a.dor}{0}
\verb{suador}{ô}{Fig.}{}{}{}{Situação complicada, penosa ou difícil; dificuldade.}{su.a.dor}{0}
\verb{suadouro}{ô}{}{}{}{s.m.}{Ato ou efeito de suar.}{su.a.dou.ro}{0}
\verb{suadouro}{ô}{}{}{}{}{Bebida ou remédio que faz suar.}{su.a.dou.ro}{0}
\verb{suadouro}{ô}{}{}{}{}{Lugar muito quente; sauna.}{su.a.dou.ro}{0}
\verb{suar}{}{}{}{}{v.i.}{Eliminar suor pelos poros da pele; transpirar.}{su.ar}{0}
\verb{suar}{}{}{}{}{v.t.}{Esforçar"-se até a exaustão para conseguir algo.}{su.ar}{\verboinum{1}}
\verb{suarabácti}{}{Gram.}{}{}{s.m.}{Tipo de epêntese em que se intercala um grupo consonantal com uma vogal; anaptixe.  }{su.a.ra.bác.ti}{0}
\verb{suarento}{}{}{}{}{adj.}{Que está coberto, empapado de suor.}{su.a.ren.to}{0}
\verb{suarento}{}{}{}{}{}{Que faz suar.}{su.a.ren.to}{0}
\verb{suasivo}{}{}{}{}{adj.}{Próprio para persuadir; convincente, persuasivo.}{su.a.si.vo}{0}
\verb{suasório}{}{}{}{}{adj.}{Suasivo.}{su.a.só.rio}{0}
\verb{suástica}{}{}{}{}{s.f.}{Símbolo em forma de cruz, com as hastes curvas formando quatro ângulos retos, que representava a felicidade, a saudação e a salvação entre brâmanes e budistas.}{su.ás.ti.ca}{0}
\verb{suástica}{}{}{}{}{}{Essa cruz, com os braços voltados para o lado direito, adotada como o emblema do hitlerismo, e que simboliza o nazismo.}{su.ás.ti.ca}{0}
\verb{suave}{}{}{}{}{adj.}{Que é agradável aos sentidos; delicado, ameno, leve, aprazível.}{su.a.ve}{0}
\verb{suave}{}{}{}{}{}{Terno, meigo, afetuoso, doce.}{su.a.ve}{0}
\verb{suave}{}{}{}{}{}{Que não é intenso nem exagerado; moderado, brando.}{su.a.ve}{0}
\verb{suavidade}{}{}{}{}{s.f.}{Qualidade ou condição do que é suave; brandura, delicadeza, meiguice, doçura.}{su.a.vi.da.de}{0}
\verb{suavizar}{}{}{}{}{v.t.}{Tornar suave, brando; amenizar.}{su.a.vi.zar}{\verboinum{1}}
\verb{suazi}{}{}{}{}{adj.}{Relativo ao Reino da Suazilândia (nordeste da África do Sul).}{su.a.zi}{0}
\verb{suazi}{}{}{}{}{s.2g.}{Indivíduo natural ou habitante desse reino.}{su.a.zi}{0}
\verb{suazi}{}{}{}{}{s.m.}{Língua falada na Suazilândia e na África do Sul.}{su.a.zi}{0}
%\verb{}{}{}{}{}{}{}{}{0}
\verb{subafluente}{}{Geogr.}{}{}{s.m.}{Afluente de um afluente de um curso de água.}{su.ba.flu.en.te}{0}
\verb{subalimentação}{}{}{"-ões}{}{s.f.}{Estado ou condição de quem está insuficientemente alimentado e deficiente de vitaminas, sais minerais etc.; subnutrição.}{su.ba.li.men.ta.ção}{0}
\verb{subalimentação}{}{}{"-ões}{}{}{Alimentação deficiente em calorias.}{su.ba.li.men.ta.ção}{0}
\verb{subalimentado}{}{}{}{}{adj.}{Diz"-se daquele que se encontra em estado de subalimentação; subnutrido, desnutrido.}{su.ba.li.men.ta.do}{0}
\verb{subalterno}{é}{}{}{}{adj.}{Que está sob as ordens de outrem, subordinado ou inferior em graduação ou autoridade.}{su.bal.ter.no}{0}
\verb{subalterno}{é}{Por ext.}{}{}{}{Que se coloca na condição de dever obediência a outrem; submisso.}{su.bal.ter.no}{0}
\verb{subalugar}{}{}{}{}{v.t.}{Sublocar.}{su.ba.lu.gar}{\verboinum{5}}
\verb{subaquático}{}{}{}{}{adj.}{Que está ou vive debaixo da água.}{su.ba.quá.ti.co}{0}
\verb{subarbusto}{}{Bot.}{}{}{s.m.}{Planta de porte menor que o de um arbusto, de caule lenhoso na base, da qual são lançadas as ramificações que perecem após cada período de crescimento.}{su.bar.bus.to}{0}
\verb{subarrendar}{}{}{}{}{v.t.}{Arrendar a terceiro (a coisa arrendada), transferindo as obrigações anteriormente assumidas; sublocar.}{su.bar.ren.dar}{\verboinum{1}}
\verb{subchefe}{é}{}{}{}{s.m.}{Imediato ao chefe ou substituto dele.}{sub.che.fe}{0}
\verb{subclasse}{}{}{}{}{s.f.}{Divisão de classe.}{sub.clas.se}{0}
\verb{subclasse}{}{Bot.}{}{}{}{Categoria taxonômica situada abaixo da classe e acima da ordem.}{sub.clas.se}{0}
\verb{subclasse}{}{Mat.}{}{}{}{Subconjunto.}{sub.clas.se}{0}
\verb{subcomissão}{}{}{"-ões}{}{s.f.}{Comissão menor, formada por parte dos membros de uma comissão maior, para se ocuparem de assunto específico.  }{sub.co.mis.são}{0}
\verb{subcomissário}{}{}{}{}{s.m.}{Indivíduo que ocupa cargo imediatamente abaixo do de comissário, ou que o substitui.}{sub.co.mis.sá.rio}{0}
\verb{subconjunto}{}{}{}{}{s.m.}{Parte de um conjunto, com características próprias.	 }{sub.con.jun.to}{0}
\verb{subconjunto}{}{Mat.}{}{}{}{Conjunto cujos elementos pertencem a outro; conjunto que está contido em outro; subclasse.}{sub.con.jun.to}{0}
\verb{subconsciência}{}{}{}{}{s.f.}{Obscurecimento da consciência; semiconsciência.}{sub.cons.ci.ên.cia}{0}
\verb{subconsciente}{}{}{}{}{adj.2g.}{Pertencente ou relativo ao subconsciente ou à subconsciência.  }{sub.cons.ci.en.te}{0}
\verb{subconsciente}{}{Psicol.}{}{}{s.m.}{O conjunto de processos e fatos psíquicos que estão latentes no indivíduo, como lembranças, hábitos, tendências etc., mas passíveis de lhe influenciar a conduta e que podem vir à tona na consciência; subconsciência, subliminar.}{sub.cons.ci.en.te}{0}
\verb{subcutâneo}{}{Anat.}{}{}{adj.}{Diz"-se do que está situado debaixo da pele; hipodérmico.}{sub.cu.tâ.neo}{0}
\verb{subdelegado}{}{}{}{}{adj.}{Diz"-se daquele a quem se subdelegou algo.	}{sub.de.le.ga.do}{0}
\verb{subdelegado}{}{}{}{}{s.m.}{Imediato ou substituto de um delegado.  }{sub.de.le.ga.do}{0}
\verb{subdelegar}{}{}{}{}{v.t.}{Transmitir um encargo a quem o assumiu como delegado. }{sub.de.le.gar}{\verboinum{5}}
\verb{subdesenvolvido}{}{}{}{}{adj.}{Diz"-se de povo, sociedade, indivíduo ou economia que se encontra em estado de subdesenvolvimento; pouco desenvolvido, atrasado.}{sub.de.sen.vol.vi.do}{0}
\verb{subdesenvolvido}{}{Bras.}{}{}{s.m.}{Indivíduo que não tem educação,   ignorante, malcomportado.}{sub.de.sen.vol.vi.do}{0}
\verb{subdesenvolvimento}{}{}{}{}{s.m.}{Desenvolvimento incompleto ou imperfeito, abaixo do normal.}{sub.de.sen.vol.vi.men.to}{0}
\verb{subdesenvolvimento}{}{Econ.}{}{}{}{Condição de economias que apresentam níveis baixos de produtividade, escolaridade, renda \textit{per capita}, desenvolvimento científico e tecnológico, que têm dependência externa etc.; atraso.}{sub.de.sen.vol.vi.men.to}{0}
\verb{subdesenvolvimento}{}{Bras.}{}{}{}{Pobreza, fome, miséria.}{sub.de.sen.vol.vi.men.to}{0}
\verb{subdiácono}{}{Relig.}{}{}{s.m.}{Clérigo, abaixo de diácono, que recebeu a primeira das três ordens maiores da Igreja ocidental.}{sub.di.á.co.no}{0}
\verb{subdiretor}{ô}{}{}{}{s.m.}{Imediato ao diretor ou que o substitui; vice"-diretor.}{sub.di.re.tor}{0}
\verb{súbdito}{}{}{}{}{}{Var. de \textit{súdito}.}{súb.di.to}{0}
\verb{subdividir}{}{}{}{}{v.t.}{Dividir novamente.}{sub.di.vi.dir}{\verboinum{34}}
\verb{subdivisão}{}{}{"-ões}{}{s.f.}{Divisão de algo que já foi dividido.}{sub.di.vi.são}{0}
\verb{subdivisão}{}{Bot.}{"-ões}{}{}{Categoria taxonômica inferior à divisão e superior à classe.}{sub.di.vi.são}{0}
\verb{subemprego}{ê}{}{}{}{s.m.}{Emprego informal, geralmente mal remunerado.}{su.bem.pre.go}{0}
\verb{subentender}{ê}{}{}{}{v.t.}{Entender o que não foi expresso, não está claro ou explícito.}{su.ben.ten.der}{\verboinum{12}}
\verb{subentendido}{}{}{}{}{adj.}{Que se subentendeu.}{su.ben.ten.di.do}{0}
\verb{subentendido}{}{}{}{}{s.m.}{Aquilo que se pensa, mas que não foi expresso ou escrito.}{su.ben.ten.di.do}{0}
\verb{súber}{}{Bot.}{}{}{s.m.}{Tecido vegetal, espesso, leve e impermeável, formado pelas células mortas cheias de ar e a partir do qual se obtém a cortiça; cortiça.}{sú.ber}{0}
\verb{subespécie}{}{Biol.}{}{}{s.f.}{Divisão de uma espécie que difere segundo determinados critérios de outra.}{su.bes.pé.cie}{0}
\verb{subespécie}{}{}{}{}{}{Categoria taxonômica em que se divide a espécie quando há mais de um tipo bem caracterizado, identificada pelo acréscimo de um terceiro nome latino; variedade.}{su.bes.pé.cie}{0}
\verb{subestação}{}{}{"-ões}{}{s.f.}{Estação de trens secundária.}{su.bes.ta.ção}{0}
\verb{subestação}{}{}{"-ões}{}{}{Estação secundária de uma rede elétrica, que transforma e distribui a corrente gerada pela estação central.}{su.bes.ta.ção}{0}
\verb{subestimar}{}{}{}{}{v.t.}{Não dar o devido valor, estima ou apreço; não ter ou não levar em grande consideração; desdenhar, menosprezar.}{su.bes.ti.mar}{\verboinum{1}}
\verb{subfaturamento}{}{}{}{}{s.m.}{Ato ou efeito de subfaturar.}{sub.fa.tu.ra.men.to}{0}
\verb{subfaturar}{}{}{}{}{v.t.}{Emitir fatura com preço inferior ao cobrado, para burlar o fisco.  }{sub.fa.tu.rar}{\verboinum{1}}
\verb{subgerente}{}{}{}{}{s.m.}{Indivíduo que é imediato ao gerente ou que o substitui.}{sub.ge.ren.te}{0}
\verb{subgrupo}{}{}{}{}{s.m.}{Grupo que faz parte de outro grupo. }{sub.gru.po}{0}
\verb{subgrupo}{}{Mat.}{}{}{}{Parte de um grupo que apresenta em si mesma uma estrutura de grupo.}{sub.gru.po}{0}
\verb{subhumano}{}{}{}{}{adj.}{Que está abaixo do nível humano.}{sub.hu.ma.no}{0}
\verb{subhumano}{}{}{}{}{}{Desumano, inumano.}{sub.hu.ma.no}{0}
\verb{subida}{}{}{}{}{s.f.}{Ato ou efeito de subir.}{su.bi.da}{0}
\verb{subida}{}{}{}{}{}{Elevação ou inclinação de terreno; aclive, ascensão.}{su.bi.da}{0}
\verb{subida}{}{}{}{}{}{Acréscimo, aumento.}{su.bi.da}{0}
\verb{subido}{}{}{}{}{adj.}{Que subiu.}{su.bi.do}{0}
\verb{subido}{}{Fig.}{}{}{}{Em posição elevada; alto, eminente.}{su.bi.do}{0}
\verb{subido}{}{Fig.}{}{}{}{Nobre, ilustre, luminar.}{su.bi.do}{0}
\verb{subir}{}{}{}{}{v.i.}{Ir para o alto; alçar"-se, ascender, erguer"-se.}{su.bir}{0}
\verb{subir}{}{}{}{}{v.t.}{Pôr"-se sobre alguma coisa ou alguém; trepar, montar.}{su.bir}{0}
\verb{subir}{}{}{}{}{}{Ficar mais alto (temperatura).}{su.bir}{0}
\verb{subir}{}{}{}{}{v.t.}{Entrar num veículo; tomar, pegar.}{su.bir}{0}
\verb{subir}{}{}{}{}{v.i.}{Ficar mais caro (produto, preço); aumentar.}{su.bir}{0}
\verb{subir}{}{Fig.}{}{}{}{Melhorar de vida, de situação; progredir, prosperar.}{su.bir}{\verboinum{33}}
\verb{subitâneo}{}{}{}{}{adj.}{Súbito.}{su.bi.tâ.neo}{0}
\verb{súbito}{}{}{}{}{adj.}{Que chega ou aparece de repente; inesperado, repentino, inopinado,  imprevisto.}{sú.bi.to}{0}
\verb{súbito}{}{}{}{}{}{Usado na locução \textit{de súbito}: de forma imprevista, subitamente.}{sú.bi.to}{0}
\verb{subjacente}{}{}{}{}{adj.2g.}{Que jaz ou está debaixo. }{sub.ja.cen.te}{0}
\verb{subjacente}{}{Fig.}{}{}{}{Encoberto, subentendido, escondido, implícito.}{sub.ja.cen.te}{0}
\verb{subjetivar}{}{}{}{}{v.t.}{Tornar ou considerar subjetivo; subjetivizar.}{sub.je.ti.var}{\verboinum{1}}
\verb{subjetivismo}{}{}{}{}{s.m.}{Tendência em considerar tudo como subjetivo.}{sub.je.ti.vis.mo}{0}
\verb{subjetivismo}{}{Filos.}{}{}{}{Doutrina filosófica que considera a realidade do mundo objetivo apenas uma projeção da subjetividade humana.}{sub.je.ti.vis.mo}{0}
\verb{subjetivo}{}{}{}{}{adj.}{Relativo a sujeito.}{sub.je.ti.vo}{0}
\verb{subjetivo}{}{}{}{}{}{Que existe apenas na mente do sujeito.}{sub.je.ti.vo}{0}
\verb{subjetivo}{}{}{}{}{}{Que é próprio ou exclusivo de uma pessoa; particular, pessoal, individual.}{sub.je.ti.vo}{0}
\verb{subjetivo}{}{Gram.}{}{}{}{Que tem função de sujeito.}{sub.je.ti.vo}{0}
\verb{subjugar}{}{}{}{}{}{Reprimir, conter, refrear.}{sub.ju.gar}{0}
\verb{subjugar}{}{}{}{}{v.t.}{Dominar, submeter, sujeitar, apoderar"-se.}{sub.ju.gar}{0}
\verb{subjugar}{}{Fig.}{}{}{}{Exercer influência, impressionar.}{sub.ju.gar}{\verboinum{5}}
\verb{subjuntivo}{}{}{}{}{adj.}{Que está em relação de dependência; subordinado.}{sub.jun.ti.vo}{0}
\verb{subjuntivo}{}{Gram.}{}{}{}{Diz"-se do modo verbal no qual se exprime um fato duvidoso, incerto, eventual.}{sub.jun.ti.vo}{0}
\verb{sublevação}{}{}{"-ões}{}{s.f.}{Ato ou efeito de sublevar, amotinar.}{sub.le.va.ção}{0}
\verb{sublevação}{}{}{"-ões}{}{}{Rebelião individual ou em massa; levante, revolta, insurreição.}{sub.le.va.ção}{0}
\verb{sublevar}{}{}{}{}{v.t.}{Mover de baixo para cima; levantar, erguer.}{sub.le.var}{0}
\verb{sublevar}{}{}{}{}{}{Provocar a revolta; rebelar, amotinar.}{sub.le.var}{\verboinum{1}}
\verb{sublimação}{}{}{"-ões}{}{s.f.}{Ato ou efeito de sublimar, de engrandecer; exaltação, enaltecimento.}{su.bli.ma.ção}{0}
\verb{sublimação}{}{Fís.}{"-ões}{}{}{Passagem direta da fase sólida para a gasosa, e vice"-versa.}{su.bli.ma.ção}{0}
\verb{sublimação}{}{Psicol.}{"-ões}{}{}{Processo inconsciente que consiste na transformação da energia da libido para projetos elevados, úteis.}{su.bli.ma.ção}{0}
\verb{sublimado}{}{}{}{}{adj.}{Elevado à maior altura; engrandecido, exaltado.}{su.bli.ma.do}{0}
\verb{sublimado}{}{}{}{}{}{Que passou do estado sólido para o gasoso.}{su.bli.ma.do}{0}
\verb{sublimar}{}{}{}{}{v.t.}{Tornar sublime; exaltar, engrandecer.}{su.bli.mar}{0}
\verb{sublimar}{}{}{}{}{}{Fazer passar uma substância do estado sólido para o gasoso.}{su.bli.mar}{0}
\verb{sublimar}{}{Fig.}{}{}{}{Retirar as imperfeições ou impurezas; purificar.}{su.bli.mar}{\verboinum{1}}
\verb{sublime}{}{}{}{}{adj.}{Muito agradável, encantador, divino.}{su.bli.me}{0}
\verb{subliminar}{}{}{}{}{adj.2g.}{Que é inferior, ou não ultrapassa o limiar.}{sub.li.mi.nar}{0}
\verb{subliminar}{}{}{}{}{}{Que está subentendido nas entrelinhas ou se faz associação por ideias.}{sub.li.mi.nar}{0}
\verb{sublingual}{}{Anat.}{"-ais}{}{s.f.}{Região situada abaixo da língua.}{sub.lin.gual}{0}
\verb{sublingual}{}{}{"-ais}{}{adj.2g.}{O que é feito ou colocado sob a língua.}{sub.lin.gual}{0}
\verb{sublinha}{}{}{}{}{s.f.}{Linha que se traça por baixo de uma palavra.}{sub.li.nha}{0}
\verb{sublinhar}{}{}{}{}{v.t.}{Traçar uma linha embaixo de palavra ou frase para chamar a atenção do leitor; grifar.}{su.bli.nhar}{0}
\verb{sublinhar}{}{}{}{}{}{Enfatizar, marcar, destacar.}{su.bli.nhar}{\verboinum{1}}
\verb{subliteratura}{}{}{}{}{s.f.}{Literatura de qualidade inferior, sem mérito literário ou artístico, medíocre.}{sub.li.te.ra.tu.ra}{0}
\verb{sublocação}{}{}{"-ões}{}{s.f.}{Ato ou efeito de sublocar; subaluguel.}{sub.lo.ca.ção}{0}
\verb{sublocação}{}{}{"-ões}{}{}{Contrato por meio do qual se subaluga um imóvel a um terceiro.}{sub.lo.ca.ção}{0}
\verb{sublocador}{ô}{}{}{}{s.m.}{Locatário que subloca um imóvel ou a parte de um imóvel a um terceiro.}{sub.lo.ca.dor}{0}
\verb{sublocar}{}{}{}{}{v.t.}{Alugar a um terceiro um imóvel ou a parte de um imóvel; subalugar.}{sub.lo.car}{\verboinum{2}}
\verb{sublocatário}{}{}{}{}{s.m.}{Indivíduo que recebe um imóvel ou parte dele, por sublocação, das mãos de um locatário.}{sub.lo.ca.tá.rio}{0}
\verb{sublunar}{}{}{}{}{adj.2g.}{Que se situa abaixo da Lua ou entre a Terra e a Lua.}{sub.lu.nar}{0}
\verb{submarino}{}{}{}{}{adj.}{Que existe ou está debaixo das águas do mar.}{sub.ma.ri.no}{0}
\verb{submarino}{}{}{}{}{}{Que se realiza por baixo do mar.}{sub.ma.ri.no}{0}
\verb{submarino}{}{}{}{}{s.m.}{Navio de guerra totalmente fechado, destinado a submergir e operar sob a água.}{sub.ma.ri.no}{0}
\verb{submergir}{}{}{}{}{v.t.}{Cobrir de água; inundar, alagar.}{sub.mer.gir}{0}
\verb{submergir}{}{}{}{}{}{Fazer sumir na água; mergulhar, afundar.}{sub.mer.gir}{0}
\verb{submergir}{}{}{}{}{}{Fazer desaparecer; ocultar, encobrir.}{sub.mer.gir}{\verboinum{34}\verboirregular{\emph{def.} submerges, submerge, submergimos, submergis, submergem}}
\verb{submergível}{}{}{"-eis}{}{adj.2g.}{Que pode ser submergido, afundado; submersível.}{sub.mer.gí.vel}{0}
\verb{submersão}{}{}{"-ões}{}{s.f.}{Ato ou efeito de submergir; emersão.}{sub.mer.são}{0}
\verb{submersão}{}{Veter.}{"-ões}{}{}{Enfraquecimento do casco de uma cavalgadura devido a pancada.}{sub.mer.são}{0}
\verb{submersível}{}{}{"-eis}{}{adj.2g.}{Diz"-se do aparelho ou instrumento que pode funcionar debaixo da água.}{sub.mer.sí.vel}{0}
\verb{submersível}{}{}{"-eis}{}{}{Submergível.}{sub.mer.sí.vel}{0}
\verb{submerso}{é}{}{}{}{adj.}{Que está coberto pelas águas; mergulhado, inundado.}{sub.mer.so}{0}
\verb{submerso}{é}{}{}{}{}{Que não se consegue ver; oculto pelas águas; escondido.}{sub.mer.so}{0}
\verb{submerso}{é}{}{}{}{}{Absorvido, absorto.}{sub.mer.so}{0}
\verb{submeter}{ê}{}{}{}{v.t.}{Tirar a liberdade e a independência; sujeitar, subjugar.}{sub.me.ter}{0}
\verb{submeter}{ê}{}{}{}{}{Reduzir a dependência; dominar, controlar.}{sub.me.ter}{0}
\verb{submeter}{ê}{}{}{}{}{Tornar objeto de exame; subordinar.}{sub.me.ter}{\verboinum{12}}
\verb{subministrar}{}{}{}{}{v.t.}{Prover do necessário; dar, fornecer, ministrar.}{sub.mi.nis.trar}{\verboinum{1}}
\verb{submissão}{}{}{"-ões}{}{s.f.}{Ato ou efeito de submeter; dominação, controle.}{sub.mis.são}{0}
\verb{submissão}{}{}{"-ões}{}{}{Disposição para obedecer; docilidade, aceitação.}{sub.mis.são}{0}
\verb{submissão}{}{}{"-ões}{}{}{Humildade excessiva; servilismo.}{sub.mis.são}{0}
\verb{submisso}{}{}{}{}{adj.}{Que está em posição inferior.}{sub.mis.so}{0}
\verb{submisso}{}{}{}{}{}{Disposto à obediência; dócil, respeitoso.}{sub.mis.so}{0}
\verb{submisso}{}{}{}{}{}{Que serve sem reclamar; subserviente, servil.}{sub.mis.so}{0}
\verb{submúltiplo}{}{Mat.}{}{}{s.m.}{Número inteiro que divide outro inteiro exatamente.}{sub.múl.ti.plo}{0}
\verb{submundo}{}{}{}{}{s.m.}{O mundo, a vida dos marginais, cujas atividades são ligadas à delinquência, ao crime organizado, ao tráfico de drogas etc.}{sub.mun.do}{0}
\verb{subnutrição}{}{}{"-ões}{}{s.f.}{Insuficiência de quantidade e qualidade de alimentação; subalimentação.}{sub.nu.tri.ção}{0}
\verb{subnutrição}{}{}{"-ões}{}{}{Alimentação deficiente em calorias.}{sub.nu.tri.ção}{0}
\verb{subnutrido}{}{}{}{}{adj.}{Que se encontra em estado de insuficiência alimentar que, prolongado, pode comprometer a saúde, ocasionando a morte.}{sub.nu.tri.do}{0}
\verb{subnutrir}{}{}{}{}{v.t.}{Ingerir quantidade ou qualidade insuficiente de alimento ou de nutrientes; subalimentar.}{sub.nu.trir}{\verboinum{18}}
\verb{suboficial}{}{}{"-ais}{}{s.m.}{Patente militar na Aeronáutica e na Marinha, abaixo da de aspirante, e que corresponde à de subtenente no Exército.}{su.bo.fi.ci.al}{0}
\verb{suboficial}{}{}{"-ais}{}{}{Militar que ocupa essa patente.}{su.bo.fi.ci.al}{0}
\verb{subordem}{ó}{Biol.}{"-ens}{}{s.f.}{Nas classificações vegetais e animais, categoria situada abaixo da ordem e acima da família.}{su.bor.dem}{0}
\verb{subordinação}{}{}{"-ões}{}{s.f.}{Ato ou efeito de subordinar, sujeitar; obediência, submissão.}{su.bor.di.na.ção}{0}
\verb{subordinação}{}{Gram.}{"-ões}{}{}{Processo sintático que consiste em relações de dependência entre unidades linguísticas que apresentam funções diferentes, formando sintagmas. }{su.bor.di.na.ção}{0}
\verb{subordinada}{}{Gram.}{}{}{adj.}{Diz"-se da oração que exerce uma função sintática dentro de outra.}{su.bor.di.na.da}{0}
\verb{subordinado}{}{}{}{}{adj.}{Que é dependente de algo ou alguém; subalterno.}{su.bor.di.na.do}{0}
\verb{subordinado}{}{}{}{}{}{Que, em relação a outro, ocupa posição inferior, secundária.}{su.bor.di.na.do}{0}
\verb{subordinado}{}{Gram.}{}{}{}{Diz"-se do termo que exerce uma função sintática dentro de uma oração principal.}{su.bor.di.na.do}{0}
\verb{subordinar}{}{}{}{}{v.t.}{Estabelecer uma ordem de dependência entre dois seres vinculados, considerando um inferior e outro superior.}{su.bor.di.nar}{0}
\verb{subordinar}{}{}{}{}{}{Tornar dependente, secundário.}{su.bor.di.nar}{0}
\verb{subordinar}{}{}{}{}{}{Pôr sob dependência; sujeitar, submeter.}{su.bor.di.nar}{\verboinum{1}}
\verb{subordinativo}{}{}{}{}{adj.}{Que estabelece ou demonstra subordinação.}{su.bor.di.na.ti.vo}{0}
\verb{subordinativo}{}{Gram.}{}{}{}{Diz"-se da conjunção que liga uma oração subordinada a uma principal.}{su.bor.di.na.ti.vo}{0}
\verb{subornar}{}{}{}{}{v.t.}{Dar ou oferecer dinheiro ou benefícios, induzindo a práticas de atos ilegais em proveito próprio.}{su.bor.nar}{0}
\verb{subornar}{}{}{}{}{}{Atrair com promessas enganosas para a execução de práticas imorais; corromper.}{su.bor.nar}{\verboinum{1}}
\verb{suborno}{ô}{}{}{}{s.m.}{Ato ou efeito de subornar, corromper; peita.}{su.bor.no}{0}
\verb{subprefeito}{ê}{}{}{}{s.m.}{O funcionário imediato ao prefeito; vice"-prefeito.}{sub.pre.fei.to}{0}
\verb{subprefeito}{ê}{}{}{}{}{O administrador de uma subprefeitura.}{sub.pre.fei.to}{0}
\verb{subprefeitura}{}{}{}{}{s.f.}{Cada uma das subdivisões administrativas de uma prefeitura, que são dirigidas por subprefeitos.}{sub.pre.fei.tu.ra}{0}
\verb{subproduto}{}{}{}{}{s.m.}{Aquilo que resulta secundariamente de outra coisa; consequência, resultado.}{sub.pro.du.to}{0}
\verb{subproduto}{}{}{}{}{}{Produto que se obtém acessoriamente no curso da fabricação de outra substância; resíduo.}{sub.pro.du.to}{0}
\verb{sub"-raça}{}{}{sub"-raças}{}{s.f.}{Raça considerada, por preconceito, inferior, seja do ponto de vista étnico, seja do ponto de vista social e econômico.}{sub"-ra.ça}{0}
\verb{sub"-raça}{}{Pejor.}{sub"-raças}{}{}{Grupo ou classe social desprezível.}{sub"-ra.ça}{0}
\verb{sub"-reitor}{ô}{}{sub"-reitores ⟨ô⟩}{}{s.m.}{Indivíduo que auxilia o reitor ou o que o substitui; vice"-reitor.}{sub"-rei.tor}{0}
\verb{sub"-repção}{}{}{sub"-repções}{}{s.f.}{Ato ou efeito de obter graça ou  benefício por meios sub"-reptícios ou se valendo de mentiras.}{sub"-rep.ção}{0}
\verb{sub"-reptício}{}{}{sub"-reptícios}{}{adj.}{Que é feito às ocultas;  dissimulado, furtivo, clandestino. }{sub"-rep.tí.cio}{0}
\verb{sub"-reptício}{}{}{sub"-reptícios}{}{}{Em que há fraude; doloso, ilícito.}{sub"-rep.tí.cio}{0}
\verb{sub"-rogação}{}{}{sub"-rogações}{}{s.f.}{Ato ou efeito de sub"-rogar; substituir.}{sub"-ro.ga.ção}{0}
\verb{sub"-rogação}{}{Jur.}{sub"-rogações}{}{}{Substituição ou transferência judicial de uma pessoa ou coisa por outra, respeitando"-se a mesma relação jurídica.  }{sub"-ro.ga.ção}{0}
\verb{sub"-rogar}{}{}{}{}{v.t.}{Colocar em lugar de outrem; substituir.}{sub"-ro.gar}{0}
\verb{sub"-rogar}{}{Jur.}{}{}{}{Transferir encargo ou direito para outrem; substabelecer.}{sub"-ro.gar}{0}
\verb{sub"-rogar}{}{}{}{}{v.pron.}{Tomar o lugar de outrem; assumir.}{sub"-ro.gar}{\verboinum{5}}
\verb{subscrever}{ê}{}{}{}{v.t.}{Escrever por baixo; assinar, subscritar.}{subs.cre.ver}{0}
\verb{subscrever}{ê}{}{}{}{}{Estar de acordo; aceitar, aprovar.}{subs.cre.ver}{0}
\verb{subscrever}{ê}{}{}{}{}{Comprometer"-se a contribuir com certa quantia de dinheiro a uma obra de caridade.}{subs.cre.ver}{\verboinum{12}}
\verb{subscrição}{}{}{"-ões}{}{s.f.}{Ato ou efeito de subscrever; assinatura.}{subs.cri.ção}{0}
\verb{subscrição}{}{}{"-ões}{}{}{Compromisso assumido de contribuir com certa quantia em dinheiro para uma obra de caridade, uma fundação etc.}{subs.cri.ção}{0}
\verb{subscritar}{}{}{}{}{v.t.}{Assinar embaixo; firmar com assinatura; subscrever.}{subs.cri.tar}{\verboinum{1}}
\verb{subscrito}{}{}{}{}{adj.}{Que está escrito por baixo; assinado.}{subs.cri.to}{0}
\verb{subscritor}{ô}{}{}{}{adj.}{Que subscreve; assinante.}{subs.cri.tor}{0}
\verb{subsequência}{}{}{}{}{s.f.}{Aquilo que se segue; seguimento, continuação.}{sub.se.quên.cia}{0}
\verb{subsequente}{}{}{}{}{adj.2g.}{Que segue logo depois; imediato, seguinte.}{sub.se.quen.te}{0}
\verb{subserviência}{}{}{}{}{s.f.}{Condição do que é subserviente; submissão.}{sub.ser.vi.ên.cia}{0}
\verb{subserviência}{}{}{}{}{}{Adulação, bajulação.}{sub.ser.vi.ên.cia}{0}
\verb{subserviente}{}{}{}{}{adj.2g.}{Que consente em servir outro de maneira humilhante.}{sub.ser.vi.en.te}{0}
\verb{subserviente}{}{}{}{}{}{Que atende às vontades alheias com facilidade; bajulador.}{sub.ser.vi.en.te}{0}
\verb{subsidiar}{ss}{}{}{}{v.t.}{Dar subsídio; financiar.}{sub.si.di.ar}{0}
\verb{subsidiar}{ss}{}{}{}{}{Contribuir com subsídio; auxiliar, ajudar.}{sub.si.di.ar}{\verboinum{6}}
\verb{subsidiária}{ss}{}{}{}{s.f.}{Empresa controlada por outra, a qual detém a maioria ou o total de suas ações.}{sub.si.di.á.ria}{0}
\verb{subsidiário}{}{}{}{}{adj.}{Relativo a subsídio.}{sub.si.di.á.rio}{0}
\verb{subsidiário}{}{}{}{}{}{Que concede subsídio; que ajuda.}{sub.si.di.á.rio}{0}
\verb{subsidiário}{}{}{}{}{}{Diz"-se de um elemento secundário que reforça outro de maior importância ou para este converge.}{sub.si.di.á.rio}{0}
\verb{subsidiário}{}{}{}{}{}{Que reforça ou dá apoio a algo anteriormente apresentado.}{sub.si.di.á.rio}{0}
\verb{subsídio}{}{}{}{}{s.m.}{Ajuda financeira, intelectual etc. dada por uma organização a outra organização ou a uma pessoa.}{sub.sí.dio}{0}
\verb{subsídio}{}{}{}{}{}{Auxílio, colaboração.}{sub.sí.dio}{0}
\verb{subsídio}{}{}{}{}{}{Quantia destinada pelo Estado a obras e projetos de interesse público.}{sub.sí.dio}{0}
\verb{subsistência}{z}{}{}{}{s.f.}{Estado das pessoas ou coisas que subsistem, que se mantêm; existência, permanência.}{sub.sis.tên.cia}{0}
\verb{subsistência}{z}{}{}{}{}{Conjunto das coisas essenciais à manutenção da vida; sustento.}{sub.sis.tên.cia}{0}
\verb{subsistir}{z}{}{}{}{v.i.}{Ser, existir.}{sub.sis.tir}{0}
\verb{subsistir}{z}{}{}{}{}{Manter"-se vivo, continuar a existir; conservar"-se, sobreviver.}{sub.sis.tir}{0}
\verb{subsistir}{z}{}{}{}{}{Estar em vigor; manter"-se.}{sub.sis.tir}{0}
\verb{subsistir}{z}{}{}{}{}{Conservar a sua força e ação.}{sub.sis.tir}{\verboinum{18}}
\verb{subsolo}{ssó}{}{}{}{s.m.}{Camada mais profunda do solo, que fica imediatamente abaixo da camada arável.}{sub.so.lo}{0}
\verb{subsolo}{ssó}{}{}{}{}{Parte de uma construção localizada abaixo do rés"-do"-chão.}{sub.so.lo}{0}
\verb{substabelecer}{ê}{}{}{}{v.t.}{Transferir para outrem; sub"-rogar.}{subs.ta.be.le.cer}{0}
\verb{substabelecer}{ê}{}{}{}{}{Nomear alguém como substituto.}{subs.ta.be.le.cer}{\verboinum{12}}
\verb{substância}{}{}{}{}{s.f.}{A parte real ou essencial de alguma coisa.}{subs.tân.cia}{0}
\verb{substância}{}{}{}{}{}{A natureza de um corpo; aquilo que lhe define as qualidades materiais; matéria.}{subs.tân.cia}{0}
\verb{substância}{}{}{}{}{}{O que é necessário à vida, o que alimenta.}{subs.tân.cia}{0}
\verb{substância}{}{}{}{}{}{O que constitui a base, o ponto fundamental de uma questão, de um assunto; o essencial; o substancial.}{subs.tân.cia}{0}
\verb{substancial}{}{}{"-ais}{}{adj.2g.}{Relativo a substância.}{subs.tan.ci.al}{0}
\verb{substancial}{}{}{"-ais}{}{}{Que é considerado grande; considerável, vultoso.}{subs.tan.ci.al}{0}
\verb{substancial}{}{}{"-ais}{}{}{Que é nutritivo, alimentício.}{subs.tan.ci.al}{0}
\verb{substancial}{}{}{"-ais}{}{}{Que é essencial, fundamental.}{subs.tan.ci.al}{0}
\verb{substancial}{}{}{"-ais}{}{}{Que encerra muitos ensinamentos, que possui muito conteúdo.}{subs.tan.ci.al}{0}
\verb{substanciar}{}{}{}{}{v.t.}{Dar alimento substancial; nutrir.}{subs.tan.ci.ar}{0}
\verb{substanciar}{}{}{}{}{}{Expor em substância, nos aspectos essenciais; resumir.}{subs.tan.ci.ar}{0}
\verb{substanciar}{}{}{}{}{}{Tornar mais eficaz, mais poderoso; fortalecer, reforçar.}{subs.tan.ci.ar}{\verboinum{6}}
\verb{substancioso}{ô}{}{"-osos ⟨ó⟩}{"-osa ⟨ó⟩}{adj.}{Em que há muita substância.}{subs.tan.ci.o.so}{0}
\verb{substancioso}{ô}{}{"-osos ⟨ó⟩}{"-osa ⟨ó⟩}{}{Que dá força, energia.}{subs.tan.ci.o.so}{0}
\verb{substantivar}{}{Gram.}{}{}{v.t.}{Empregar como substantivo.}{subs.tan.ti.var}{0}
\verb{substantivar}{}{Gram.}{}{}{}{Dar caráter de substantivo; tornar substantivo.}{subs.tan.ti.var}{\verboinum{1}}
\verb{substantivo}{}{}{}{}{adj.}{Que é fundamental, substancial, essencial.}{subs.tan.ti.vo}{0}
\verb{substantivo}{}{Gram.}{}{}{s.m.}{Classe de palavra usada para referir nominalmente seres, objetos, sentimentos, estados.}{subs.tan.ti.vo}{0}
\verb{substituição}{}{}{"-ões}{}{s.f.}{Ato ou efeito de substituir.}{subs.ti.tu.i.ção}{0}
\verb{substituição}{}{}{"-ões}{}{}{Colocação de coisa ou pessoa no lugar de outra; troca.}{subs.ti.tu.i.ção}{0}
\verb{substituir}{}{}{}{}{v.t.}{Retirar pessoa ou coisa para colocar outra em seu lugar; trocar.}{subs.ti.tu.ir}{\verboinum{26}}
\verb{substitutivo}{}{}{}{}{adj.}{Que substitui, que faz as vezes ou toma o lugar de alguém ou alguma coisa.}{subs.ti.tu.ti.vo}{0}
\verb{substitutivo}{}{}{}{}{}{Emenda, substituição.}{subs.ti.tu.ti.vo}{0}
\verb{substituto}{}{}{}{}{s.m.}{Indivíduo que substitui outro ou lhe faz as vezes.}{subs.ti.tu.to}{0}
\verb{substrato}{}{}{}{}{s.m.}{O que constitui a parte essencial do ser; a essência.}{subs.tra.to}{0}
\verb{substrato}{}{}{}{}{}{O que apoia, sustenta; base.}{subs.tra.to}{0}
\verb{subtender}{ê}{}{}{}{v.t.}{Estender por baixo.}{sub.ten.der}{\verboinum{12}}
\verb{subtenente}{}{}{}{}{s.m.}{Militar de graduação inferior à de tenente.}{sub.te.nen.te}{0}
\verb{subterfúgio}{}{}{}{}{s.m.}{Manobra ou pretexto para evitar dificuldades; pretexto, evasiva.}{sub.ter.fú.gio}{0}
\verb{subterfúgio}{}{}{}{}{}{Ardil para se conseguir algo.}{sub.ter.fú.gio}{0}
\verb{subterrâneo}{}{}{}{}{adj.}{Que fica debaixo da terra.}{sub.ter.râ.neo}{0}
\verb{subterrâneo}{}{}{}{}{}{Que ocorre debaixo da terra.}{sub.ter.râ.neo}{0}
\verb{subterrâneo}{}{Fig.}{}{}{}{Feito clandestinamente; secreto.}{sub.ter.râ.neo}{0}
\verb{subterrâneo}{}{}{}{}{s.m.}{Passagem, galeria ou compartimento construído no subsolo de uma edificação.}{sub.ter.râ.neo}{0}
\verb{subtil}{}{}{}{}{}{Var. de \textit{sutil}.}{sub.til}{0}
\verb{subtileza}{ê}{}{}{}{}{Var. de \textit{sutileza}.}{sub.ti.le.za}{0}
\verb{subtilizar}{}{}{}{}{}{Var. de \textit{sutilizar}.}{sub.ti.li.zar}{0}
\verb{subtítulo}{}{}{}{}{s.m.}{Expressão ou palavra que se segue ao título principal, complementando"-o.}{sub.tí.tu.lo}{0}
\verb{subtotal}{}{}{"-ais}{}{adj.2g.}{Que constitui ou abrange parte do total.}{sub.to.tal}{0}
\verb{subtotal}{}{}{"-ais}{}{s.m.}{Total parcial; a soma de uma parte das parcelas de uma soma maior.}{sub.to.tal}{0}
\verb{subtração}{}{}{"-ões}{}{s.f.}{Ato ou efeito de subtrair.}{sub.tra.ção}{0}
\verb{subtração}{}{}{"-ões}{}{}{Roubo fraudulento; furto.}{sub.tra.ção}{0}
\verb{subtração}{}{Mat.}{"-ões}{}{}{Operação inversa à da adição; diminuição.}{sub.tra.ção}{0}
\verb{subtraendo}{}{Mat.}{}{}{s.m.}{Número que se tira do outro numa subtração.}{sub.tra.en.do}{0}
\verb{subtrair}{}{}{}{}{v.t.}{Tirar às escondidas, ou fraudulentamente; furtar, roubar.}{sub.tra.ir}{0}
\verb{subtrair}{}{Mat.}{}{}{}{Efetuar uma subtração; diminuir.}{sub.tra.ir}{0}
\verb{subtrair}{}{}{}{}{v.pron.}{Esquivar"-se, fugir, escapar.}{sub.tra.ir}{\verboinum{19}}
\verb{subtropical}{}{}{"-ais}{}{adj.2g.}{Que se situa perto dos trópicos, até 40 graus de latitude.}{sub.tro.pi.cal}{0}
\verb{suburbano}{}{}{}{}{adj.}{Relativo a subúrbio.}{su.bur.ba.no}{0}
\verb{suburbano}{}{}{}{}{}{Que mora em subúrbio.}{su.bur.ba.no}{0}
\verb{suburbano}{}{}{}{}{}{Diz"-se do que tem ou revela mau gosto.}{su.bur.ba.no}{0}
\verb{suburbano}{}{}{}{}{s.m.}{Indivíduo que mora em subúrbio.}{su.bur.ba.no}{0}
\verb{suburbano}{}{}{}{}{}{Indivíduo que tem mau gosto; cafona, brega.}{su.bur.ba.no}{0}
\verb{subúrbio}{}{}{}{}{s.m.}{Cercanias de cidade ou de outra povoação.}{su.búr.bio}{0}
\verb{subúrbio}{}{}{}{}{}{Bairro localizado longe do centro, fora ou nas adjacências dos seus limites.}{su.búr.bio}{0}
\verb{subvenção}{}{}{"-ões}{}{s.f.}{Ato de socorrer; ajuda.}{sub.ven.ção}{0}
\verb{subvenção}{}{}{"-ões}{}{}{Auxílio representado em dinheiro, por via de regra concedido pelos poderes públicos.}{sub.ven.ção}{0}
\verb{subvencionar}{}{}{}{}{v.t.}{Dar auxílio; ajudar.}{sub.ven.ci.o.nar}{0}
\verb{subvencionar}{}{}{}{}{}{Dar subvenção; subsidiar.}{sub.ven.ci.o.nar}{\verboinum{1}}
\verb{subversão}{}{}{"-ões}{}{s.f.}{Ato ou efeito de subverter.}{sub.ver.são}{0}
\verb{subversão}{}{}{"-ões}{}{}{Revolta contra a ordem estabelecida, o poder constituído.}{sub.ver.são}{0}
\verb{subversivo}{}{}{}{}{adj.}{Que subverte ou pode subverter.}{sub.ver.si.vo}{0}
\verb{subversivo}{}{}{}{}{s.m.}{Militante da subversão; revolucionário.}{sub.ver.si.vo}{0}
\verb{subverter}{ê}{}{}{}{v.t.}{Voltar de baixo para cima; revolver.}{sub.ver.ter}{0}
\verb{subverter}{ê}{}{}{}{}{Destruir, aniquilar.}{sub.ver.ter}{0}
\verb{subverter}{ê}{}{}{}{}{Perturbar completamente; transtornar.}{sub.ver.ter}{0}
\verb{subverter}{ê}{}{}{}{}{Agitar, revolucionar.}{sub.ver.ter}{0}
\verb{subverter}{ê}{}{}{}{}{Submergir, afundar.}{sub.ver.ter}{\verboinum{12}}
\verb{sução}{}{}{}{}{}{Var. de \textit{sucção}.}{su.ção}{0}
\verb{sucata}{}{}{}{}{s.f.}{Estrutura, objeto ou peça metálica inutilizada pelo uso ou pela oxidação, e que pode ser refundida para utilização posterior.}{su.ca.ta}{0}
\verb{sucata}{}{Por ext.}{}{}{}{Qualquer peça metálica inutilizada.}{su.ca.ta}{0}
\verb{sucata}{}{}{}{}{}{Depósito de ferro velho.}{su.ca.ta}{0}
\verb{sucatear}{}{}{}{}{v.t.}{Transformar em sucata.}{su.ca.te.ar}{0}
\verb{sucatear}{}{}{}{}{}{Arruinar por falta de cuidado.}{su.ca.te.ar}{\verboinum{4}}
\verb{sucção}{}{}{"-ões}{}{s.f.}{Ato ou efeito de sugar, chupar, sorver.}{suc.ção}{0}
\verb{sucedâneo}{}{}{}{}{adj.}{Diz"-se do medicamento que pode substituir outro, por ter mais ou menos as mesmas propriedades.   }{su.ce.dâ.neo}{0}
\verb{sucedâneo}{}{}{}{}{s.m.}{Esse medicamento.}{su.ce.dâ.neo}{0}
\verb{suceder}{ê}{}{}{}{v.i.}{Acontecer, ocorrer.}{su.ce.der}{0}
\verb{suceder}{ê}{}{}{}{v.t.}{Acontecer depois; seguir"-se.}{su.ce.der}{0}
\verb{suceder}{ê}{}{}{}{}{Ocupar o lugar de outrem ou de outra coisa.}{su.ce.der}{\verboinum{12}}
\verb{sucedido}{}{}{}{}{adj.}{Que sucedeu; ocorrido.}{su.ce.di.do}{0}
\verb{sucessão}{}{}{"-ões}{}{s.f.}{Ato ou efeito de suceder.}{su.ces.são}{0}
\verb{sucessão}{}{}{"-ões}{}{}{Série de pessoas, coisas ou fatos.}{su.ces.são}{0}
\verb{sucessão}{}{}{"-ões}{}{}{Conjunto dos descendentes; prole.}{su.ces.são}{0}
\verb{sucessão}{}{}{"-ões}{}{}{Transmissão de direitos, bens, cargos ou encargos segundo regras definidas.}{su.ces.são}{0}
\verb{sucessivo}{}{}{}{}{adj.}{Relativo a sucessão; hereditário.}{su.ces.si.vo}{0}
\verb{sucessivo}{}{}{}{}{}{Que vem em seguida; consecutivo.}{su.ces.si.vo}{0}
\verb{sucessivo}{}{}{}{}{}{Que se repete.}{su.ces.si.vo}{0}
\verb{sucesso}{é}{}{}{}{s.m.}{Bom êxito; triunfo.}{su.ces.so}{0}
\verb{sucesso}{é}{Bras.}{}{}{}{Obra ou artista de grande popularidade ou êxito mercadológico.}{su.ces.so}{0}
\verb{sucesso}{é}{}{}{}{}{Aquilo que sucede; acontecimento.}{su.ces.so}{0}
\verb{sucesso}{é}{}{}{}{interj.}{Expressão utilizada para desejar êxito a alguém que vai enfrentar um desafio, ou para despedir"-se em determinados contextos.}{su.ces.so}{0}
\verb{sucessor}{ô}{}{}{}{adj.}{Que sucede alguém em um cargo ou função.}{su.ces.sor}{0}
\verb{sucessor}{ô}{}{}{}{s.m.}{Herdeiro.}{su.ces.sor}{0}
\verb{sucessor}{ô}{}{}{}{}{Os descendentes. (Usa"-se no plural nesta acepção.)}{su.ces.sor}{0}
\verb{sucessório}{}{}{}{}{adj.}{Relativo a sucessão.}{su.ces.só.rio}{0}
\verb{súcia}{}{}{}{}{s.f.}{Grupo de pessoas desonestas ou ladrões que obedecem a um chefe; quadrilha, corja.}{sú.cia}{0}
\verb{sucinto}{}{}{}{}{adj.}{Resumido, conciso, condensado.}{su.cin.to}{0}
\verb{suco}{}{}{}{}{s.m.}{Caldo nutritivo extraído de frutas, legumes, carne; sumo.}{su.co}{0}
\verb{suco}{}{Por ext.}{}{}{}{Qualquer líquido bebível, geralmente com sabor natural ou artificial.}{su.co}{0}
\verb{suçuarana}{}{Bras.}{}{}{s.f.}{Animal felino de cor amarelo"-avermelhada encontrado em toda a América; puma, jaguaruna.}{su.çu.a.ra.na}{0}
\verb{suculência}{}{}{}{}{s.f.}{Qualidade de suculento.}{su.cu.lên.cia}{0}
\verb{suculência}{}{}{}{}{}{Abundância de suco.}{su.cu.lên.cia}{0}
\verb{suculento}{}{}{}{}{adj.}{Que tem bastante suco.}{su.cu.len.to}{0}
\verb{suculento}{}{}{}{}{}{Gordo.}{su.cu.len.to}{0}
\verb{suculento}{}{}{}{}{}{Nutritivo, substancial.}{su.cu.len.to}{0}
\verb{sucumbir}{}{}{}{}{v.t.}{Cair sob a força de algo; vergar, abater"-se.}{su.cum.bir}{0}
\verb{sucumbir}{}{}{}{}{}{Não resistir; ceder.}{su.cum.bir}{0}
\verb{sucumbir}{}{}{}{}{v.i.}{Perder a disposição; abater"-se. \textit{abon}}{su.cum.bir}{0}
\verb{sucumbir}{}{}{}{}{}{Ser derrotado.}{su.cum.bir}{0}
\verb{sucumbir}{}{}{}{}{}{Morrer.}{su.cum.bir}{\verboinum{18}}
\verb{sucupira}{}{Bras.}{}{}{s.f.}{Árvore da família das leguminosas com frutos do tipo vagem e cultivada como ornamental e por sua madeira de qualidade.}{su.cu.pi.ra}{0}
\verb{sucuri}{}{Zool.}{}{}{s.f.}{Serpente de cor esverdeada que vive perto de rios e pântanos, não venenosa e que engole sua presa após triturá"-la por compressão.}{su.cu.ri}{0}
\verb{sucursal}{}{}{"-ais}{}{adj.2g.}{Diz"-se de estabelecimento comercial subordinado a uma matriz; filial.}{su.cur.sal}{0}
\verb{sucursal}{}{}{"-ais}{}{s.f.}{Escritório de representação de uma empresa comercial ou jornalística situado em outra cidade ou país.}{su.cur.sal}{0}
\verb{sudação}{}{Med.}{"-ões}{}{s.f.}{Ato ou efeito de suar.}{su.da.ção}{0}
\verb{sudanês}{}{}{}{}{adj.}{Relativo ao Sudão (África centro"-oriental).}{su.da.nês}{0}
\verb{sudanês}{}{}{}{}{s.m.}{Indivíduo natural ou habitante desse país.}{su.da.nês}{0}
\verb{sudanês}{}{}{}{}{}{A língua da Guiné.}{su.da.nês}{0}
\verb{sudário}{}{}{}{}{s.m.}{Pano usado antigamente para enxugar o suor.}{su.dá.rio}{0}
\verb{sudário}{}{}{}{}{}{Tipo de lençol para envolver um cadáver; mortalha.}{su.dá.rio}{0}
\verb{sudário}{}{Relig.}{}{}{}{(\textit{Santo Sudário}) A mortalha de Cristo.}{su.dá.rio}{0}
\verb{sudeste}{é}{}{}{}{adj.}{Sueste.}{su.des.te}{0}
\verb{sudeste}{é}{}{}{}{s.m.}{Região geográfica e administrativa do Brasil que inclui os estados de São Paulo, Espírito Santo, Minas Gerais e Rio de Janeiro.}{su.des.te}{0}
\verb{súdito}{}{}{}{}{adj.}{Submetido à autoridade de um soberano.}{sú.di.to}{0}
\verb{súdito}{}{}{}{}{s.m.}{Indivíduo submetido à autoridade de um soberano; vassalo.}{sú.di.to}{0}
\verb{sudoeste}{é}{}{}{}{s.m.}{Ponto cardeal situado entre o sul e o oeste.}{su.do.es.te}{0}
\verb{sudoeste}{é}{}{}{}{adj.}{Diz"-se do vento que sopra dessa direção.}{su.do.es.te}{0}
\verb{sudorese}{é}{Med.}{}{}{s.f.}{Transpiração.}{su.do.re.se}{0}
\verb{sudorífero}{}{}{}{}{adj.}{Que faz suar; suador.}{su.do.rí.fe.ro}{0}
\verb{sudorífico}{}{}{}{}{adj.}{Que faz suar; sudorífero.}{su.do.rí.fi.co}{0}
\verb{sudoríparo}{}{}{}{}{adj.}{Que produz suor. \textit{Glândula sudorípara.}}{su.do.rí.pa.ro}{0}
\verb{sudoríparo}{}{}{}{}{}{Relativo a suor.}{su.do.rí.pa.ro}{0}
\verb{sueca}{é}{}{}{}{s.f.}{Modalidade de ginástica.}{su.e.ca}{0}
\verb{sueca}{é}{}{}{}{}{Jogo de cartas, espécie de bisca.}{su.e.ca}{0}
\verb{sueca}{é}{Mús.}{}{}{}{Espécie de quadrilha, de andamento ligeiro.}{su.e.ca}{0}
\verb{sueco}{é}{}{}{}{adj.}{Relativo à Suécia.}{su.e.co}{0}
\verb{sueco}{é}{}{}{}{s.m.}{Indivíduo natural ou habitante desse país.}{su.e.co}{0}
\verb{sueco}{é}{}{}{}{}{A língua oficial da Suécia.}{su.e.co}{0}
\verb{suelto}{ê}{}{}{}{s.m.}{Nota curta escrita em jornal, geralmente sobre um assunto do dia; tópico, vária.}{su.el.to}{0}
\verb{suelto}{ê}{}{}{}{}{Boato.}{su.el.to}{0}
\verb{sueste}{é}{}{}{}{s.m.}{Ponto do horizonte a igual distância do sul e do este. Abrev. \textsc{s.e}.}{su.es.te}{0}
\verb{sueste}{é}{}{}{}{}{O vento que sopra dessa direção.}{su.es.te}{0}
\verb{sueste}{é}{}{}{}{}{Região ou regiões que ficam nessa direção.}{su.es.te}{0}
\verb{suéter}{}{Bras.}{}{}{s.2g.}{Agasalho de lã fechado; pulôver.}{su.é.ter}{0}
\verb{sueto}{ê}{}{}{}{s.m.}{Feriado escolar.}{su.e.to}{0}
\verb{sueto}{ê}{}{}{}{}{Folga no trabalho; descanso, ócio.}{su.e.to}{0}
\verb{suficiência}{}{}{}{}{s.f.}{Qualidade de suficiente.}{su.fi.ci.ên.cia}{0}
\verb{suficiência}{}{}{}{}{}{Conjunto de habilidades específicas requeridas para o desempenho de determinada função; qualificação, habilidade.}{su.fi.ci.ên.cia}{0}
\verb{suficiência}{}{}{}{}{}{Presunção.}{su.fi.ci.ên.cia}{0}
\verb{suficiente}{}{}{}{}{adj.2g.}{Que basta para o que se quer; bastante. }{su.fi.ci.en.te}{0}
\verb{suficiente}{}{}{}{}{s.m.}{O que basta; o bastante.  }{su.fi.ci.en.te}{0}
\verb{sufixo}{cs}{Gram.}{}{}{s.m.}{Afixo que se coloca após o radical, formando palavras flexionadas (cant\textit{ei}) ou derivadas (laranj\textit{al}).}{su.fi.xo}{0}
\verb{suflê}{}{Cul.}{}{}{s.m.}{Prato de forno preparado à base de farinha de trigo, gema de ovo e clara em neve.}{su.flê}{0}
\verb{sufocação}{}{}{"-ões}{}{s.f.}{Ato ou efeito de sufocar.}{su.fo.ca.ção}{0}
\verb{sufocação}{}{}{"-ões}{}{}{Dificuldade em respirar.}{su.fo.ca.ção}{0}
\verb{sufocação}{}{}{"-ões}{}{}{Asfixia, estrangulamento.}{su.fo.ca.ção}{0}
\verb{sufocação}{}{Fig.}{"-ões}{}{}{Repressão das manifestações do pensamento; censura.}{su.fo.ca.ção}{0}
\verb{sufocante}{}{}{}{}{adj.2g.}{Que sufoca; sufocador, asfixiante.}{su.fo.can.te}{0}
\verb{sufocante}{}{}{}{}{}{Diz"-se do clima muito quente ou abafado.}{su.fo.can.te}{0}
\verb{sufocante}{}{Fig.}{}{}{}{Desagradável, desconfortável, que causa mal"-estar.}{su.fo.can.te}{0}
\verb{sufocar}{}{}{}{}{v.t.}{Tirar o ar.}{su.fo.car}{0}
\verb{sufocar}{}{}{}{}{}{Matar por falta de ar; asfixiar.}{su.fo.car}{0}
\verb{sufocar}{}{Fig.}{}{}{}{Impedir de se manifestar; reprimir, abafar.}{su.fo.car}{\verboinum{1}}
\verb{sufoco}{ô}{}{}{}{s.m.}{Ato ou efeito de sufocar.}{su.fo.co}{0}
\verb{sufoco}{ô}{Bras.}{}{}{}{Situação muito difícil; dificuldade, aperto.}{su.fo.co}{0}
\verb{sufoco}{ô}{Bras.}{}{}{}{Angústia, ansiedade.}{su.fo.co}{0}
\verb{sufragâneo}{}{}{}{}{adj.}{Diz"-se de bispo ou bispado sem autonomia, subordinado a outro.}{su.fra.gâ.neo}{0}
\verb{sufragar}{}{}{}{}{v.t.}{Apoiar com sufrágio; votar.}{su.fra.gar}{0}
\verb{sufragar}{}{}{}{}{}{Eleger.}{su.fra.gar}{0}
\verb{sufragar}{}{}{}{}{}{Rezar em intenção.}{su.fra.gar}{\verboinum{5}}
\verb{sufrágio}{}{}{}{}{s.m.}{Eleição.}{su.frá.gio}{0}
\verb{sufrágio}{}{}{}{}{}{O voto em uma eleição.}{su.frá.gio}{0}
\verb{sufrágio}{}{}{}{}{}{Opinião favorável; aprovação.}{su.frá.gio}{0}
\verb{sufrágio}{}{}{}{}{}{Oração ou gesto pela alma de um morto.}{su.frá.gio}{0}
\verb{sugador}{ô}{Bot.}{}{}{adj.}{Diz"-se de planta que suga nutrientes de outra; parasita.}{su.ga.dor}{0}
\verb{sugador}{ô}{Zool.}{}{}{s.m.}{Sugadouro.}{su.ga.dor}{0}
\verb{sugadouro}{ô}{Zool.}{}{}{s.m.}{Aparelho bucal longo de alguns insetos; sugador.}{su.ga.dou.ro}{0}
\verb{sugar}{}{}{}{}{v.t.}{Chupar, sorver.}{su.gar}{0}
\verb{sugar}{}{}{}{}{}{Tirar ou extrair por sucção.}{su.gar}{\verboinum{5}}
\verb{sugerir}{}{}{}{}{v.t.}{Apresentar sutil e indiretamente uma ideia; insinuar.}{su.ge.rir}{0}
\verb{sugerir}{}{}{}{}{}{Propor, lembrar.}{su.ge.rir}{0}
\verb{sugerir}{}{}{}{}{}{Provocar um pensamento ou sensação por associação de ideias. (\textit{A casa vazia sugeria tristeza.})}{su.ge.rir}{0}
\verb{sugerir}{}{}{}{}{}{Proporcionar, ocasionar.}{su.ge.rir}{\verboinum{29}}
\verb{sugestão}{}{}{"-ões}{}{s.f.}{Ato ou efeito de sugerir; aquilo que se sugere.}{su.ges.tão}{0}
\verb{sugestão}{}{}{"-ões}{}{}{Proposta, ideia, conselho.}{su.ges.tão}{0}
\verb{sugestão}{}{}{"-ões}{}{}{Estímulo, inspiração.}{su.ges.tão}{0}
\verb{sugestionar}{}{}{}{}{v.t.}{Produzir sugestão; sugerir.}{su.ges.ti.o.nar}{\verboinum{1}}
\verb{sugestivo}{}{}{}{}{adj.}{Que contém sugestão; que inspira, instiga.}{su.ges.ti.vo}{0}
\verb{sugestivo}{}{}{}{}{}{Atraente, insinuante, sedutor.}{su.ges.ti.vo}{0}
\verb{suíça}{}{}{}{}{s.f.}{Suíças.}{suíça}{0}
\verb{suíças}{}{}{}{}{s.f.pl.}{Estilo de corte de barba em que esta fica nas laterais do rosto, como uma costeleta que vai até próximo da boca.}{su.í.ças}{0}
\verb{suicida}{}{}{}{}{adj.2g.}{Relativo a suicídio.}{su.i.ci.da}{0}
\verb{suicida}{}{}{}{}{}{Que implica em dano próprio.}{su.i.ci.da}{0}
\verb{suicida}{}{}{}{}{s.2g.}{Indivíduo que cometeu ou tende a cometer suicídio.}{su.i.ci.da}{0}
\verb{suicidar"-se}{}{}{}{}{v.pron.}{Pôr fim à própria vida; matar"-se.}{su.i.ci.dar"-se}{0}
\verb{suicidar"-se}{}{Fig.}{}{}{}{Arruinar"-se, prejudicar"-se, desgraçar"-se. (\textit{Suicidou"-se financeiramente.})}{su.i.ci.dar"-se}{\verboinum{1}}
\verb{suicídio}{}{}{}{}{s.m.}{Ato ou efeito de suicidar"-se.}{su.i.cí.dio}{0}
\verb{suicídio}{}{Fig.}{}{}{}{Desgraça ou prejuízo causado a si mesmo por gesto voluntário.}{su.i.cí.dio}{0}
\verb{suíço}{}{}{}{}{adj.}{Relativo à Suíça.}{su.í.ço}{0}
\verb{suíço}{}{}{}{}{s.m.}{Indivíduo natural ou habitante desse país.}{su.í.ço}{0}
\verb{suídeo}{}{Zool.}{}{}{adj.}{Relativo aos suídeos, família de mamíferos que compreende os porcos domésticos.}{su.í.deo}{0}
\verb{suídeo}{}{}{}{}{s.m.}{Animal dessa família.}{su.í.deo}{0}
\verb{sui generis}{}{}{}{}{adj.}{Sem semelhante; único, singular, peculiar.}{\textit{sui generis}}{0}
\verb{suindara}{}{Zool.}{}{}{s.f.}{Coruja de plumagem branca com manchas marrons.}{su.in.da.ra}{0}
\verb{suingue}{}{}{}{}{s.m.}{Estilo de \textit{jazz} dançante e arranjado para grande orquestra.}{su.in.gue}{0}
\verb{suingue}{}{Por ext.}{}{}{}{O ritmo vivaz desse estilo musical ou semelhante a ele.}{su.in.gue}{0}
\verb{suingue}{}{}{}{}{}{A dança ao som dessa música.}{su.in.gue}{0}
\verb{suingue}{}{Por ext.}{}{}{}{Movimentos corporais característicos feitos ao som de ritmos musicais bem marcados; balanço.}{su.in.gue}{0}
\verb{suingue}{}{Bras.}{}{}{}{Prática sexual envolvendo mais de um casal.}{su.in.gue}{0}
\verb{suíno}{}{Zool.}{}{}{adj.}{Relativo aos suínos, grupo de mamíferos que compreende os porcos.}{su.í.no}{0}
\verb{suíno}{}{}{}{}{s.m.}{Animal desse grupo; porco.}{su.í.no}{0}
\verb{suinocultor}{ô}{}{}{}{adj.}{Relativo à suinocultura.}{su.i.no.cul.tor}{0}
\verb{suinocultor}{ô}{}{}{}{s.m.}{Indivíduo que se dedica à suinocultura.}{su.i.no.cul.tor}{0}
\verb{suinocultura}{}{}{}{}{s.f.}{Técnica e atividade de criar porcos.}{su.i.no.cul.tu.ra}{0}
\verb{suíte}{}{Bras.}{}{}{s.f.}{Em hotéis, acomodação com quarto, banheiro e saleta.}{su.í.te}{0}
\verb{suíte}{}{Bras.}{}{}{}{Em residências, quarto com banheiro contíguo.}{su.í.te}{0}
\verb{suíte}{}{Mús.}{}{}{}{Sequência de composições ou danças, alternando andamentos vivos e lentos e geralmente num mesmo tom.}{su.í.te}{0}
\verb{sujar}{}{}{}{}{v.t.}{Tornar sujo; manchar, poluir, emporcalhar.}{su.jar}{0}
\verb{sujar}{}{Fig.}{}{}{}{Tornar suja a imagem de uma pessoa; macular, manchar.}{su.jar}{0}
\verb{sujar}{}{}{}{}{}{Perverter, depravar, corromper.}{su.jar}{0}
\verb{sujar}{}{Bras.}{}{}{v.pron.}{Tornar"-se objeto de incriminação.}{su.jar}{0}
\verb{sujar}{}{Pop.}{}{}{v.i.}{Acontecer a chegada repentina de alguém de quem se esconde.}{su.jar}{0}
\verb{sujeição}{}{}{"-ões}{}{s.f.}{Ato ou efeito de sujeitar.}{su.jei.ção}{0}
\verb{sujeição}{}{}{"-ões}{}{}{Submissão, dependência, vassalagem.}{su.jei.ção}{0}
\verb{sujeição}{}{}{"-ões}{}{}{Obediência, resignação, submissão.}{su.jei.ção}{0}
\verb{sujeira}{ê}{}{}{}{s.f.}{Material indesejado que prejudica a aparência ou a higiene de algo; imundície, porcaria, mancha, poeira.}{su.jei.ra}{0}
\verb{sujeira}{ê}{}{}{}{}{Acúmulo de dejetos, poeira ou qualquer material que atrapalhe o bom funcionamento de algo.}{su.jei.ra}{0}
\verb{sujeira}{ê}{Fig.}{}{}{}{Procedimento vil, ilícito ou condenável.}{su.jei.ra}{0}
\verb{sujeira}{ê}{}{}{}{}{Fezes.}{su.jei.ra}{0}
\verb{sujeira}{ê}{Fig.}{}{}{}{Palavras obscenas ou inadequadas.}{su.jei.ra}{0}
\verb{sujeitar}{}{}{}{}{v.t.}{Subjugar, dominar, submeter.}{su.jei.tar}{0}
\verb{sujeitar}{}{}{}{}{}{Tornar dependente.}{su.jei.tar}{\verboinum{1}}
\verb{sujeitar}{}{}{}{}{v.pron.}{Submeter"-se, render"-se.}{su.jei.tar}{\verboinum{1}}
\verb{sujeito}{ê}{}{}{}{adj.}{Que se sujeitou; submetido.}{su.jei.to}{0}
\verb{sujeito}{ê}{}{}{}{}{Em que pode acontecer algo. (\textit{Esta área é sujeita a neblina.})}{su.jei.to}{0}
\verb{sujeito}{ê}{}{}{}{s.m.}{Indivíduo indeterminado cujo nome é desconhecido ou se quer omitir.}{su.jei.to}{0}
\verb{sujeito}{ê}{Filos.}{}{}{}{Em uma proposição, termo do qual se diz algo.}{su.jei.to}{0}
\verb{sujeito}{ê}{Gram.}{}{}{}{Termo da oração com o qual o verbo concorda.}{su.jei.to}{0}
\verb{sujidade}{}{}{}{}{s.f.}{Qualidade ou condição de sujo; sujeira.}{su.ji.da.de}{0}
\verb{sujidade}{}{}{}{}{}{Excrementos, fezes.}{su.ji.da.de}{0}
\verb{sujidade}{}{Fig.}{}{}{}{Devassidão.}{su.ji.da.de}{0}
\verb{sujo}{}{}{}{}{adj.}{Que não está limpo; manchado, imundo, porco.}{su.jo}{0}
\verb{sujo}{}{Fig.}{}{}{}{Moralmente condenável.}{su.jo}{0}
\verb{sujo}{}{}{}{}{}{Feito sem capricho; malfeito.}{su.jo}{0}
\verb{sujo}{}{Bras.}{}{}{}{Que está sem moral, respeito ou crédito; desmoralizado.}{su.jo}{0}
\verb{sul}{}{}{}{}{s.m.}{Ponto cardeal que se opõe diretamente ao norte. Abrev. \textsc{s}.}{sul}{0}
\verb{sul}{}{}{}{}{}{O vento que sopra dessa direção.}{sul}{0}
\verb{sul}{}{}{}{}{}{Região ou regiões situadas nessa direção.}{sul}{0}
\verb{sul}{}{}{}{}{}{Região geográfica e administrativa do Brasil que inclui os estados do Rio Grande do Sul, Santa Catarina e Paraná.}{sul}{0}
\verb{sul}{}{}{}{}{}{O polo sul.}{sul}{0}
\verb{sul"-africano}{}{}{sul"-africanos}{}{adj.}{Relativo à África do Sul.}{sul"-a.fri.ca.no}{0}
\verb{sul"-africano}{}{}{sul"-africanos}{}{s.m.}{Indivíduo natural ou habitante desse país.}{sul"-a.fri.ca.no}{0}
\verb{sul"-americano}{}{}{sul"-americanos}{}{adj.}{Relativo à América do Sul.}{sul"-a.me.ri.ca.no}{0}
\verb{sul"-americano}{}{}{sul"-americanos}{}{s.m.}{Indivíduo natural ou habitante de qualquer um dos países desse continente.}{sul"-a.me.ri.ca.no}{0}
\verb{sulcar}{}{}{}{}{v.t.}{Fazer sulcos.}{sul.car}{0}
\verb{sulcar}{}{}{}{}{}{Enrugar; fazer pregas ou fendas.}{sul.car}{0}
\verb{sulcar}{}{Fig.}{}{}{}{Navegar; cortar as águas.}{sul.car}{\verboinum{2}}
\verb{sulco}{}{}{}{}{s.m.}{Depressão aberta na terra pelo arado.}{sul.co}{0}
\verb{sulco}{}{}{}{}{}{Ruga, prega.}{sul.co}{0}
\verb{sul"-coreano}{}{}{sul"-coreanos}{}{adj.}{Relativo à Coreia do Sul.}{sul"-co.re.a.no}{0}
\verb{sul"-coreano}{}{}{sul"-coreanos}{}{s.m.}{Indivíduo natural ou habitante desse país.}{sul"-co.re.a.no}{0}
\verb{sulfa}{}{Farm.}{}{}{s.f.}{Forma reduzida de \textit{sulfanilamida}.}{sul.fa}{0}
\verb{sulfanilamida}{}{Med.}{}{}{s.f.}{Droga utilizada no tratamento de infecções bacterianas; sulfa.}{sul.fa.ni.la.mi.da}{0}
\verb{sulfatar}{}{Quím.}{}{}{v.t.}{Embeber de sulfato de cobre ou de ferro.}{sul.fa.tar}{0}
\verb{sulfatar}{}{}{}{}{}{Borrifar solução de sulfatos (em plantas).}{sul.fa.tar}{\verboinum{1}}
\verb{sulfato}{}{Quím.}{}{}{s.m.}{Qualquer sal obtido a partir do ácido sulfúrico.}{sul.fa.to}{0}
\verb{sulfite}{}{}{}{}{adj.}{Diz"-se de um tipo de papel sem pauta, usado em escritórios e escolas.}{sul.fi.te}{0}
\verb{sulfito}{}{Quím.}{}{}{s.m.}{Sal obtido a partir do ácido sulfuroso.}{sul.fi.to}{0}
\verb{sulfonamida}{}{Farm.}{}{}{s.f.}{Droga utilizada como agente anti"-infeccioso.}{sul.fo.na.mi.da}{0}
\verb{súlfur}{}{Quím.}{}{}{s.m.}{Enxofre.}{súl.fur}{0}
\verb{sulfúreo}{}{}{}{}{adj.}{Relativo a enxofre; sulfuroso.}{sul.fú.re.o}{0}
\verb{sulfúrico}{}{Quím.}{}{}{adj.}{Relativo a enxofre; sulfúreo.}{sul.fú.ri.co}{0}
\verb{sulfurino}{}{}{}{}{adj.}{Que tem a cor do enxofre.}{sul.fu.ri.no}{0}
\verb{sulfuroso}{ô}{Quím.}{"-osos ⟨ó⟩}{"-osa ⟨ó⟩}{adj.}{Relativo a enxofre; sulfúreo.}{sul.fu.ro.so}{0}
\verb{sulino}{}{}{}{}{adj.}{Relativo à região sul do Brasil; sulista.}{su.li.no}{0}
\verb{sulino}{}{}{}{}{s.m.}{Indivíduo natural ou habitante dessa região.}{su.li.no}{0}
\verb{sulista}{}{}{}{}{adj.2g.}{Relativo à região sul do Brasil; sulino.}{su.lis.ta}{0}
\verb{sulista}{}{}{}{}{}{Relativo ao sul de um país ou região.}{su.lis.ta}{0}
\verb{sulista}{}{}{}{}{s.2g.}{Indivíduo natural ou habitante dessa região.}{su.lis.ta}{0}
\verb{sul"-mato"-grossense}{}{}{sul"-matogrossenses}{}{adj.2g.}{Relativo ao estado de Mato Grosso do Sul; mato"-grossense"-do"-sul.}{sul"-ma.to"-gros.sen.se}{0}
\verb{sul"-mato"-grossense}{}{}{sul"-matogrossenses}{}{s.2g.}{Indivíduo natural ou habitante desse estado.}{sul"-ma.to"-gros.sen.se}{0}
\verb{sul"-rio"-grandense}{}{}{sul"-rio"-grandenses}{}{adj.2g.}{Relativo ao Rio Grande do Sul; rio"-grandense"-do"-sul.}{sul"-ri.o"-gran.den.se}{0}
\verb{sul"-rio"-grandense}{}{}{sul"-rio"-grandenses}{}{s.2g.}{Indivíduo natural ou habitante desse estado.  }{sul"-ri.o"-gran.den.se}{0}
\verb{sultana}{}{}{}{}{s.f.}{Cada uma das mulheres de um sultão.}{sul.ta.na}{0}
\verb{sultanato}{}{}{}{}{s.m.}{Dignidade de sultão.}{sul.ta.na.to}{0}
\verb{sultanato}{}{}{}{}{}{País ou região governada por um sultão.}{sul.ta.na.to}{0}
\verb{sultão}{}{}{"-ões}{sultana}{s.m.}{Príncipe maometano.}{sul.tão}{0}
\verb{sultão}{}{Fig.}{"-ões}{sultana}{}{Indivíduo muito poderoso.}{sul.tão}{0}
\verb{sultão}{}{Fig.}{"-ões}{sultana}{}{Indivíduo que tem muitas amantes; paxá.}{sul.tão}{0}
\verb{suma}{}{}{}{}{s.f.}{Sumário, resumo.}{su.ma}{0}
\verb{suma}{}{}{}{}{}{Soma, somatória.}{su.ma}{0}
\verb{suma}{}{}{}{}{}{Essência, sumo.}{su.ma}{0}
\verb{suma}{}{Relig.}{}{}{}{Tratado teológico medieval com um resumo da teologia.}{su.ma}{0}
\verb{sumagre}{}{Bot.}{}{}{s.m.}{Planta em forma de arbusto com flores esbranquiçadas com utilidade na medicina, culinária, tinturaria.}{su.ma.gre}{0}
\verb{sumagre}{}{}{}{}{}{O pó grosso resultante da trituração das folhas, flores e casca dessa planta.}{su.ma.gre}{0}
\verb{sumaré}{}{Bot.}{}{}{s.m.}{Planta orquidácea com flores amarelas cultivada como ornamental.}{sumaré}{0}
\verb{sumarento}{}{}{}{}{adj.}{Que tem muito sumo; sumoso.}{su.ma.ren.to}{0}
\verb{sumariar}{}{}{}{}{v.t.}{Fazer sumário; resumir.}{su.ma.ri.ar}{\verboinum{1}}
\verb{sumário}{}{}{}{}{adj.}{Breve, resumido.}{su.má.rio}{0}
\verb{sumário}{}{}{}{}{}{Realizado sem demora ou sem formalidades; rápido, simples, decidido.}{su.má.rio}{0}
\verb{sumário}{}{}{}{}{}{De pequena extensão. \textit{abon}}{su.má.rio}{0}
\verb{sumário}{}{}{}{}{s.m.}{Resumo com os pontos principais de livro, discurso; suma.}{su.má.rio}{0}
\verb{sumário}{}{}{}{}{}{Enumeração das principais divisões de uma obra ou documento, com indicação dos respectivos números de página.}{su.má.rio}{0}
\verb{sumaúma}{}{Bot.}{}{}{s.f.}{Árvore de grande porte, nativa de florestas inundáveis, que possui cápsulas filamentosas das quais se extrai uma paina largamente utilizada em isolamento acústico e térmico, assim como na fabricação de boias e salva"-vidas.}{su.ma.ú.ma}{0}
\verb{sumeriano}{}{}{}{}{adj. e s.m.  }{Sumério.}{su.me.ri.a.no}{0}
\verb{sumério}{}{}{}{}{adj.}{Relativo à Suméria, antiga região da Mesopotâmia, na Ásia; sumeriano.}{su.mé.rio}{0}
\verb{sumério}{}{}{}{}{s.m.}{Indivíduo natural ou habitante dessa região.}{su.mé.rio}{0}
\verb{sumiço}{}{}{}{}{s.m.}{Ato ou efeito de sumir; desaparecimento, extravio.}{su.mi.ço}{0}
\verb{sumidade}{}{}{}{}{s.f.}{Estado ou qualidade do que é alto; eminência.}{su.mi.da.de}{0}
\verb{sumidade}{}{}{}{}{}{A extremidade ou o ponto mais alto; cume, topo.}{su.mi.da.de}{0}
\verb{sumidade}{}{}{}{}{}{Indivíduo que se destaca por seu talento ou saber.}{su.mi.da.de}{0}
\verb{sumidiço}{}{}{}{}{adj.}{Que some ou desaparece com facilidade; efêmero, transitório.}{su.mi.di.ço}{0}
\verb{sumido}{}{}{}{}{adj.}{Que sumiu, que não se viu; desaparecido.}{su.mi.do}{0}
\verb{sumido}{}{}{}{}{}{Com aparência de magro; abatido, definhado.}{su.mi.do}{0}
\verb{sumido}{}{}{}{}{}{Que mal se ouve; fraco, distante.}{su.mi.do}{0}
\verb{sumidoiro}{ô}{}{}{}{}{Var. de \textit{sumidouro}.}{su.mi.doi.ro}{0}
\verb{sumidouro}{ô}{}{}{}{s.m.}{Abertura por onde a água escoa, some; escoadouro; sumidoiro.}{su.mi.dou.ro}{0}
\verb{sumidouro}{ô}{}{}{}{}{Lugar onde se desperdiça ou se consome muito dinheiro; sorvedouro.}{su.mi.dou.ro}{0}
\verb{sumir}{}{}{}{}{v.t.}{Fazer desaparecer; perder.}{su.mir}{0}
\verb{sumir}{}{}{}{}{}{Gastar muito rápido; consumir.}{su.mir}{0}
\verb{sumir}{}{}{}{}{}{Pôr a pique; afundar, submergir.}{su.mir}{0}
\verb{sumir}{}{}{}{}{v.i.}{Ausentar"-se por longo tempo; esconder"-se, retirar"-se.}{su.mir}{\verboinum{33}}
\verb{sumo}{}{}{}{}{s.m.}{Suco extraído de algumas plantas ou frutas.}{su.mo}{0}
\verb{sumo}{}{}{}{}{adj.}{Que se situa no lugar mais elevado; supremo, máximo.}{su.mo}{0}
\verb{sumo}{}{}{}{}{}{Poderoso, excelso.}{su.mo}{0}
\verb{sumo}{}{}{}{}{}{Extraordinário, grandioso.}{su.mo}{0}
\verb{sumô}{}{Esport.}{}{}{s.m.}{Espécie de luta japonesa, na qual os competidores são grandes e corpulentos, e que tem por objetivo projetar o adversário para o chão ou para fora do ringue.}{su.mô}{0}
\verb{sumptuário}{}{}{}{}{}{Var. de \textit{suntuário}.}{sump.tu.á.rio}{0}
\verb{sumptuoso}{ô}{}{}{}{}{Var. de \textit{suntuoso}.}{sump.tu.o.so}{0}
\verb{súmula}{}{}{}{}{s.f.}{Breve resumo; sinopse, condensação.}{sú.mu.la}{0}
\verb{súmula}{}{Esport.}{}{}{}{Relatório curto de uma competição esportiva.}{sú.mu.la}{0}
\verb{sundae}{}{Cul.}{}{}{s.m.}{Sorvete servido em taça, coberto com calda cremosa, castanha, frutas ou confeitos.}{\textit{sundae}}{0}
\verb{sunga}{}{}{}{}{s.f.}{Calção curto, cavado e baixo, para banho ou natação; calção de banho.}{sun.ga}{0}
\verb{sunga}{}{}{}{}{}{Cueca estreita, semelhante a esse calção.}{sun.ga}{0}
\verb{sungar}{}{}{}{}{v.t.}{Puxar para cima; levantar, erguer.}{sun.gar}{0}
\verb{sungar}{}{}{}{}{}{Suspender o cós da calça ou da saia.}{sun.gar}{\verboinum{5}}
\verb{suntuário}{}{}{}{}{adj.}{Relativo a luxo ou a gastos; magnificente.}{sun.tu.á.rio}{0}
\verb{suntuosidade}{}{}{}{}{s.f.}{Característica do que é suntuoso; pompa, ostentação, luxo, grandiosidade.}{sun.tu.o.si.da.de}{0}
\verb{suntuoso}{ô}{}{"-osos ⟨ó⟩}{"-osa ⟨ó⟩}{adj.}{Que exige muita despesa, muito gasto de dinheiro.}{sun.tu.o.so}{0}
\verb{suntuoso}{ô}{}{"-osos ⟨ó⟩}{"-osa ⟨ó⟩}{}{Em que há grande luxo; pomposo, magnificente.}{sun.tu.o.so}{0}
\verb{suor}{ô/ ou /ó}{}{}{}{s.m.}{Líquido aquoso, salgado, de odor particular, que é excretado pelos poros da pele.}{su.or}{0}
\verb{suor}{ô/ ou /ó}{}{}{}{}{Ato ou efeito de suar; transpiração.}{su.or}{0}
\verb{suor}{ô/ ou /ó}{Fig.}{}{}{}{Trabalho penoso, árduo.}{su.or}{0}
\verb{suor}{ô/ ou /ó}{Fig.}{}{}{}{Fruto de trabalho intenso, de grande fadiga.}{su.or}{0}
\verb{supedâneo}{}{}{}{}{s.m.}{Pequeno banco, para descanso dos pés; escabelo.}{su.pe.dâ.neo}{0}
\verb{supedâneo}{}{}{}{}{}{Estrado de madeira em que o padre põe os pés enquanto reza a missa.}{su.pe.dâ.neo}{0}
\verb{supedâneo}{}{}{}{}{}{Suporte, base, pedestal.}{su.pe.dâ.neo}{0}
\verb{supeditar}{}{}{}{}{v.t.}{Fornecer algo para alguém fazer uso; prover, ministrar.}{su.pe.di.tar}{\verboinum{1}}
\verb{superabundância}{}{}{}{}{s.f.}{Abundância excessiva; fartura, exuberância.}{su.pe.ra.bun.dân.cia}{0}
\verb{superabundante}{}{}{}{}{adj.2g.}{Muito abundante; excessivo, demasiado, farto.}{su.pe.ra.bun.dan.te}{0}
\verb{superabundar}{}{}{}{}{v.i.}{Existir em excessiva quantidade; sobejar.}{su.pe.ra.bun.dar}{0}
\verb{superabundar}{}{}{}{}{}{Ser mais do que necessário; sobrar, exceder.}{su.pe.ra.bun.dar}{\verboinum{1}}
\verb{superação}{}{}{"-ões}{}{s.f.}{Ato ou efeito de superar; sobrepujamento.}{su.pe.ra.ção}{0}
\verb{superalimentação}{}{}{"-ões}{}{s.f.}{Ato ou efeito de superalimentar, ingerir alimentos numa quantidade superior ao normal.}{su.pe.ra.li.men.ta.ção}{0}
\verb{superalimentação}{}{Med.}{"-ões}{}{}{Processo terapêutico que consiste no aumento anormal da quantidade de alimento absorvida por um paciente. }{su.pe.ra.li.men.ta.ção}{0}
\verb{superalimentar}{}{}{}{}{v.t.}{Alimentar em quantidade excessiva.}{su.pe.ra.li.men.tar}{\verboinum{1}}
\verb{superaquecer}{ê}{}{}{}{v.t.}{Submeter a uma temperatura elevada; aquecer em excesso.}{su.pe.ra.que.cer}{\verboinum{15}}
\verb{superaquecimento}{}{}{}{}{s.m.}{Ato ou efeito de superaquecer; elevação excessiva da temperatura.}{su.pe.ra.que.ci.men.to}{0}
\verb{superar}{}{}{}{}{v.t.}{Alcançar vitória; vencer, dominar.}{su.pe.rar}{0}
\verb{superar}{}{}{}{}{}{Ir além; ultrapassar, exceder.}{su.pe.rar}{0}
\verb{superar}{}{}{}{}{}{Passar por cima; devastar, aniquilar.}{su.pe.rar}{0}
\verb{superar}{}{}{}{}{}{Resolver, solucionar.}{su.pe.rar}{\verboinum{1}}
\verb{superável}{}{}{"-eis}{}{adj.2g.}{Passível de ser superado.}{su.pe.rá.vel}{0}
\verb{superávit}{}{}{}{}{s.m.}{Em um orçamento, diferença a mais entre a receita e a despesa.}{su.pe.rá.vit}{0}
\verb{supercílio}{}{Anat.}{}{}{s.m.}{Região acima dos olhos, que limita a borda  óssea da órbita ocular e é coberta de pelos em forma de arco; sobrancelha.}{su.per.cí.lio}{0}
\verb{supercomputador}{ô}{Informát.}{}{}{s.m.}{Computador de grande velocidade e potência, utilizado geralmente em simulações, cálculos numéricos e pesquisas científicas.}{su.per.com.pu.ta.dor}{0}
\verb{supercondutividade}{}{Fís.}{}{}{s.f.}{Ausência ou desaparecimento da resistência elétrica apresentada por certas substâncias quando em temperaturas baixas.}{su.per.con.du.ti.vi.da.de}{0}
\verb{supercondutor}{ô}{Fís.}{}{}{adj.}{Diz"-se do metal, composto ou liga que apresenta ausência de resistência elétrica quando em baixa temperatura.}{su.per.con.du.tor}{0}
\verb{superdotado}{}{}{}{}{adj.}{Diz"-se de indivíduo que possui inteligência superior à média.}{su.per.do.ta.do}{0}
\verb{superego}{é}{}{}{}{s.m.}{Indivíduo que é um exemplo, um modelo para  alguém.}{su.pe.re.go}{0}
\verb{superego}{é}{}{}{}{}{Em psicanálise, a instância intermediária entre o id e o ego, que exerce a função de censura e repressão diante do instintivo.}{su.pe.re.go}{0}
\verb{superestimar}{}{}{}{}{v.t.}{Estimar muito alguém; adorar.}{su.pe.res.ti.mar}{0}
\verb{superestimar}{}{}{}{}{}{Atribuir valor acima do razoável.}{su.pe.res.ti.mar}{\verboinum{1}}
\verb{superestrutura}{}{}{}{}{s.f.}{Conjunto de ideologias religiosas, jurídicas, políticas e culturais presentes numa sociedade.}{su.pe.res.tru.tu.ra}{0}
\verb{superestrutura}{}{}{}{}{}{Conjunto de construções situadas acima do convés de um navio.}{su.pe.res.tru.tu.ra}{0}
\verb{superexcitação}{s}{}{"-ões}{}{s.f.}{Excitação intensa.}{su.pe.rex.ci.ta.ção}{0}
\verb{superexcitar}{s}{}{}{}{v.t.}{Excitar além dos limites; despertar, avivar o ânimo.}{su.pe.rex.ci.tar}{\verboinum{1}}
\verb{superfaturamento}{}{}{}{}{s.m.}{Emissão de fatura com preço muito superior ao efetivamente cobrado.}{su.per.fa.tu.ra.men.to}{0}
\verb{superfaturar}{}{}{}{}{v.t.}{Expedir, frauduladamente, fatura com preço muito maior ao que foi realmente cobrado.}{su.per.fa.tu.rar}{\verboinum{1}}
\verb{superfetação}{}{Med.}{"-ões}{}{s.f.}{Fecundação de um segundo óvulo, quando um feto já se encontra em gestação.}{su.per.fe.ta.ção}{0}
\verb{superficial}{}{}{"-ais}{}{adj.2g.}{Relativo à superfície de um corpo.}{su.per.fi.ci.al}{0}
\verb{superficial}{}{}{"-ais}{}{}{Que se limita à superfície; pouco profundo.}{su.per.fi.ci.al}{0}
\verb{superficial}{}{Fig.}{"-ais}{}{}{Sem seriedade, sem fundamento, leviano.}{su.per.fi.ci.al}{0}
\verb{superficialidade}{}{}{}{}{s.f.}{Caráter do que é superficial; leviandade.}{su.per.fi.ci.a.li.da.de}{0}
\verb{superfície}{}{}{}{}{s.f.}{A parte externa dos corpos; face, exterior.}{su.per.fí.cie}{0}
\verb{superfície}{}{}{}{}{}{Extensão bidimensional de uma área.}{su.per.fí.cie}{0}
\verb{superfície}{}{}{}{}{}{Pouco aprofundamento em conhecimento, estudo.}{su.per.fí.cie}{0}
\verb{superfino}{}{}{}{}{adj.}{Que é extremamente fino.}{su.per.fi.no}{0}
\verb{superfino}{}{}{}{}{}{Muito fino; de qualidade superior.}{su.per.fi.no}{0}
\verb{supérfluo}{}{}{}{}{adj.}{Que é demais, inútil por excesso; desnecessário.}{su.pér.flu.o}{0}
\verb{supérfluo}{}{}{}{}{s.m.}{Aquilo que é supérfluo, desnecessário.}{su.pér.flu.o}{0}
\verb{super"-herói}{}{}{super"-heróis}{}{s.m.}{Personagem de ficção, dotado de poderes sobre"-humanos, que defende o bem da sociedade e combate o mal, personificado em criminosos e vilões.}{su.per"-he.rói}{0}
\verb{super"-homem}{}{}{super"-homens}{}{s.m.}{Indivíduo dotado de qualidades especiais, superiores às humanas, geralmente destinado a uma missão extraordinária.}{su.per"-ho.mem}{0}
\verb{super"-humano}{}{}{super"-humanos}{}{adj.}{Sobre"-humano.}{su.per"-hu.ma.no}{0}
\verb{superintendência}{}{}{}{}{s.f.}{Ato ou efeito de superintender, de coordenar, administrar.}{su.pe.rin.ten.dên.cia}{0}
\verb{superintendência}{}{}{}{}{}{Cargo ou funções de superintendente.}{su.pe.rin.ten.dên.cia}{0}
\verb{superintendência}{}{}{}{}{}{Casa ou repartição onde se exercem essas funções.}{su.pe.rin.ten.dên.cia}{0}
\verb{superintendente}{}{}{}{}{adj.2g.}{Que superintende; que, como chefe, supervisiona empresa, repartição, obra etc.}{su.pe.rin.ten.den.te}{0}
\verb{superintendente}{}{}{}{}{s.2g.}{Indivíduo que exerce a função de supervisionar empresa, obra, repartição etc.}{su.pe.rin.ten.den.te}{0}
\verb{superintender}{ê}{}{}{}{v.t.}{Dirigir na qualidade de chefe; coordenar, administrar.}{su.pe.rin.ten.der}{0}
\verb{superintender}{ê}{}{}{}{}{Inspecionar em nível superior; supervisionar.}{su.pe.rin.ten.der}{\verboinum{12}}
\verb{superior}{ô}{}{}{}{adj.2g.}{Que está acima ou por cima.}{su.pe.ri.or}{0}
\verb{superior}{ô}{}{}{}{}{De maior valor ou qualidade.}{su.pe.ri.or}{0}
\verb{superior}{ô}{}{}{}{s.m.}{Indivíduo que dirige um convento; abade.}{su.pe.ri.or}{0}
\verb{superior}{ô}{}{}{}{s.2g.}{Indivíduo que exerce autoridade sobre outrem.}{su.pe.ri.or}{0}
\verb{superiora}{ô}{}{}{}{s.f.}{Freira que dirige um convento.}{su.pe.ri.o.ra}{0}
\verb{superioridade}{}{}{}{}{s.f.}{Qualidade do que é superior.}{su.pe.ri.o.ri.da.de}{0}
\verb{superioridade}{}{}{}{}{}{Preeminência, vantagem, primazia.}{su.pe.ri.o.ri.da.de}{0}
\verb{superlativo}{}{}{}{}{adj.}{Que exprime uma qualidade em grau muito alto, ou no mais alto grau.}{su.per.la.ti.vo}{0}
\verb{superlativo}{}{Gram.}{}{}{s.m.}{Grau do adjetivo ou advérbio que indica qualidade ou modalidade marcadamente superior ou inferior, quer numa comparação, quer independente de qualquer referência.}{su.per.la.ti.vo}{0}
\verb{superlotação}{}{}{"-ões}{}{s.f.}{Ato ou efeito de superlotar; lotação excessiva.}{su.per.lo.ta.ção}{0}
\verb{superlotar}{}{Bras.}{}{}{v.t.}{Exceder a lotação; lotar excessivamente.}{su.per.lo.tar}{\verboinum{1}}
\verb{supermercado}{}{}{}{}{s.m.}{Amplo estabelecimento comercial de autosserviço, onde se expõe à venda grande variedade de mercadorias.}{su.per.mer.ca.do}{0}
\verb{supernova}{ó}{Astron.}{}{}{s.f.}{Estrela que, subitamente, adquire brilho intenso, para depois enfraquecer lentamente.}{su.per.no.va}{0}
\verb{superpopulação}{}{}{"-ões}{}{s.f.}{Excesso de população.}{su.per.po.pu.la.ção}{0}
\verb{superpor}{}{}{}{}{v.t.}{Pôr em cima; sobrepor.}{su.per.por}{\verboinum{60}}
\verb{superposto}{ô}{}{"-s ⟨ó⟩}{"-a ⟨ó⟩}{adj.}{Que se superpõe; posto em cima; sobreposto.}{su.per.pos.to}{0}
\verb{superpotência}{}{}{}{}{s.f.}{País que se destaca das demais potências por seu poderio econômico e militar, em especial no que diz respeito a armas atômicas.}{su.per.po.tên.cia}{0}
\verb{superpovoado}{}{}{}{}{adj.}{Que é povoado em excesso, em que a densidade populacional é muito alta.}{su.per.po.vo.a.do}{0}
\verb{superpovoar}{}{}{}{}{v.t.}{Povoar em excesso; existir (uma população) em uma determinada região, em quantidade superior à que esta suporta.}{su.per.po.vo.ar}{\verboinum{7}}
\verb{superprodução}{}{Econ.}{"-ões}{}{s.f.}{Produção de mercadorias acima do que é viável comercializar, que supera a demanda; excesso de produção.}{su.per.pro.du.ção}{0}
\verb{superprodução}{}{Art.}{"-ões}{}{}{Produção de cinema, teatro, televisão etc., em geral de custo elevado e que se caracteriza por grandiosidade de cenários, elenco numeroso, amplos recursos técnicos etc.}{su.per.pro.du.ção}{0}
\verb{superquadra}{}{}{}{}{s.f.}{Área residencial urbana, aberta, constituída por blocos de apartamentos, escolas, zonas ajardinadas etc., e na qual o tráfego dos veículos é separado do trânsito de pedestres. }{su.per.qua.dra}{0}
\verb{supersena}{}{}{}{}{s.f.}{Modalidade de loteria que substitui a antiga sena, e premia seis dezenas de quarenta e oito.}{su.per.se.na}{0}
\verb{supersensível}{}{}{"-eis}{}{adj.2g.}{Que é superior à ação dos sentidos.}{su.per.sen.sí.vel}{0}
\verb{supersensível}{}{}{"-eis}{}{}{Muito, extremamente sensível.}{su.per.sen.sí.vel}{0}
\verb{supersônico}{}{}{}{}{adj.}{Diz"-se de velocidade superior à do som que se propaga no ar.}{su.per.sô.ni.co}{0}
\verb{supersônico}{}{}{}{}{}{Que tem ou alcança essa velocidade.}{su.per.sô.ni.co}{0}
\verb{superstição}{}{}{"-ões}{}{s.f.}{Desvio do sentimento religioso que se funda no temor ou na ignorância, e que leva ao cumprimento de falsos deveres, a quimeras, ou à confiança em coisas fantásticas e ineficazes.}{su.pers.ti.ção}{0}
\verb{superstição}{}{}{"-ões}{}{}{Presságio infundado; crendice, credulidade, fanatismo.}{su.pers.ti.ção}{0}
\verb{supersticioso}{ô}{}{"-osos ⟨ó⟩}{"-osa ⟨ó⟩}{adj.}{Em que há, ou que envolve superstição, que é fruto de superstição.}{su.pers.ti.ci.o.so}{0}
\verb{supersticioso}{ô}{}{"-osos ⟨ó⟩}{"-osa ⟨ó⟩}{}{Que tem superstição.}{su.pers.ti.ci.o.so}{0}
\verb{supersticioso}{ô}{}{"-osos ⟨ó⟩}{"-osa ⟨ó⟩}{s.m.}{Indivíduo que tem superstição ou que é dominado por ela; crédulo.}{su.pers.ti.ci.o.so}{0}
\verb{supérstite}{}{}{}{}{adj.}{Que sobrevive; sobrevivente.}{su.pérs.ti.te}{0}
\verb{superveniência}{}{}{}{}{s.f.}{Qualidade de superveniente.}{su.per.ve.ni.ên.cia}{0}
\verb{superveniência}{}{}{}{}{}{Ato ou efeito de sobrevir, de vir sobre ou depois de alguma coisa.}{su.per.ve.ni.ên.cia}{0}
\verb{superveniente}{}{}{}{}{adj.2g.}{Que sobrevém, que vem, acontece ou surge depois.}{su.per.ve.ni.en.te}{0}
\verb{supervisão}{}{}{"-ões}{}{s.f.}{Ato ou efeito de supervisionar, de controlar, supervisar.}{su.per.vi.são}{0}
\verb{supervisão}{}{}{"-ões}{}{}{Atribuição ou função de supervisor.}{su.per.vi.são}{0}
\verb{supervisar}{}{}{}{}{v.t.}{Dirigir um trabalho, inspecionando; controlar, supervisionar.}{su.per.vi.sar}{\verboinum{1}}
\verb{supervisionar}{}{}{}{}{v.t.}{Dirigir, orientar ou inspecionar em plano superior.}{su.per.vi.si.o.nar}{\verboinum{1}}
\verb{supervisor}{ô}{}{}{}{s.m.}{Indivíduo que supervisiona ou supervisa; diretor.}{su.per.vi.sor}{0}
\verb{supetão}{}{}{"-ões}{}{s.m.}{Usado na locução \textit{de supetão}: movimento rápido e inesperado; impulso.}{su.pe.tão}{0}
\verb{supimpa}{}{Pop.}{}{}{adj.}{Muito bom; ótimo, excelente.}{su.pim.pa}{0}
\verb{supino}{}{}{}{}{adj.}{Alto, elevado, superior.}{su.pi.no}{0}
\verb{supino}{}{}{}{}{}{Em estado de supinação, voltado para cima.}{su.pi.no}{0}
\verb{supino}{}{}{}{}{}{Deitado de costas.}{su.pi.no}{0}
\verb{supino}{}{Esport.}{}{}{s.m.}{Exercício que se faz com o corpo em repouso em um plano horizontal, esticando e recolhendo sobre o peito os braços, com halteres ou barra.}{su.pi.no}{0}
\verb{suplantar}{}{}{}{}{v.t.}{Pôr debaixo dos pés; calcar, pisar.}{su.plan.tar}{0}
\verb{suplantar}{}{}{}{}{}{Abater, prostrar, derrubar.}{su.plan.tar}{0}
\verb{suplantar}{}{}{}{}{}{Levar vantagem; superar.}{su.plan.tar}{0}
\verb{suplantar}{}{}{}{}{}{Ser superior; exceder, sobrepujar.}{su.plan.tar}{0}
\verb{suplantar}{}{Por ext.}{}{}{}{Humilhar, vexar, rebaixar.}{su.plan.tar}{\verboinum{1}}
\verb{suplementar}{}{}{}{}{adj.2g.}{Relativo a suplemento.}{su.ple.men.tar}{0}
\verb{suplementar}{}{}{}{}{v.t.}{Fornecer suplemento; acrescer alguma coisa.}{su.ple.men.tar}{0}
\verb{suplementar}{}{}{}{}{}{Que serve de suplemento.}{su.ple.men.tar}{0}
\verb{suplementar}{}{}{}{}{}{Que amplia, adicional.}{su.ple.men.tar}{0}
\verb{suplementar}{}{}{}{}{}{Servir de suplemento ou aditamento.}{su.ple.men.tar}{0}
\verb{suplementar}{}{}{}{}{}{Suprir ou compensar alguma deficiência.}{su.ple.men.tar}{\verboinum{1}}
\verb{suplemento}{}{}{}{}{s.m.}{Aquilo que supre; o que serve para suprir alguma falta.}{su.ple.men.to}{0}
\verb{suplemento}{}{}{}{}{}{Acréscimo, aditamento.}{su.ple.men.to}{0}
\verb{suplemento}{}{}{}{}{}{Páginas com matéria especial, geralmente ilustrada, que em determinados números de jornal ou revista se acrescenta à matéria ordinária.}{su.ple.men.to}{0}
\verb{suplência}{}{}{}{}{s.f.}{Ato de suprir, de completar, preencher.}{su.plên.cia}{0}
\verb{suplência}{}{}{}{}{}{Cargo de suplente.}{su.plên.cia}{0}
\verb{suplência}{}{}{}{}{}{Tempo de exercício desse cargo.}{su.plên.cia}{0}
\verb{suplente}{}{}{}{}{}{Indivíduo que pode ser chamado a exercer certas funções, na falta daquele a quem elas cabiam efetivamente. }{su.plen.te}{0}
\verb{suplente}{}{}{}{}{adj.2g.}{Que supre, que substitui.}{su.plen.te}{0}
\verb{suplente}{}{}{}{}{s.2g.}{Indivíduo que supre; substituto.}{su.plen.te}{0}
\verb{suplente}{}{Esport.}{}{}{}{Reserva.}{su.plen.te}{0}
\verb{supletivo}{}{}{}{}{adj.}{Que completa ou que serve de suplemento.}{su.ple.ti.vo}{0}
\verb{supletivo}{}{Bras.}{}{}{s.m.}{Ensino que se destina a suprir a escolarização regular de adolescentes e adultos que não a concluíram na idade própria.}{su.ple.ti.vo}{0}
\verb{súplica}{}{}{}{}{s.f.}{Ato ou efeito de suplicar, de rogar, implorar.}{sú.pli.ca}{0}
\verb{súplica}{}{}{}{}{}{Pedido ou oração insistente e humilde; prece.}{sú.pli.ca}{0}
\verb{suplicante}{}{}{}{}{adj.2g.}{Que suplica, que roga.}{su.pli.can.te}{0}
\verb{suplicante}{}{}{}{}{s.2g.}{Indivíduo que suplica, que pede.}{su.pli.can.te}{0}
\verb{suplicante}{}{}{}{}{}{Indivíduo que requer; impetrante.}{su.pli.can.te}{0}
\verb{suplicar}{}{}{}{}{v.t.}{Pedir com instância e humildade; rogar, implorar, pedir.}{su.pli.car}{\verboinum{2}}
\verb{súplice}{}{}{}{}{adj.}{Que se prostra, pedindo.}{sú.pli.ce}{0}
\verb{súplice}{}{}{}{}{}{Que exprime súplica.}{sú.pli.ce}{0}
\verb{supliciado}{}{}{}{}{adj.}{Que sofreu suplício; torturado.}{su.pli.ci.a.do}{0}
\verb{supliciado}{}{}{}{}{s.m.}{Indivíduo que sofreu suplício; martirizado.}{su.pli.ci.a.do}{0}
\verb{supliciar}{}{}{}{}{v.t.}{Aplicar suplício; torturar.}{su.pli.ci.ar}{0}
\verb{supliciar}{}{}{}{}{}{Punir com pena de morte.}{su.pli.ci.ar}{0}
\verb{supliciar}{}{}{}{}{}{Fazer sofrer; afligir.}{su.pli.ci.ar}{\verboinum{6}}
\verb{suplício}{}{}{}{}{s.m.}{Dura punição corporal, imposta por sentença.}{su.plí.cio}{0}
\verb{suplício}{}{}{}{}{}{Pena de morte.}{su.plí.cio}{0}
\verb{suplício}{}{}{}{}{}{Execução dessa pena.}{su.plí.cio}{0}
\verb{suplício}{}{Fig.}{}{}{}{Pessoa ou coisa que aflige muito.}{su.plí.cio}{0}
\verb{supor}{}{}{}{}{v.t.}{Estabelecer por hipótese.}{su.por}{0}
\verb{supor}{}{}{}{}{}{Conjeturar, imaginar, presumir, considerar, julgar.}{su.por}{\verboinum{60}}
\verb{suportar}{}{}{}{}{v.t.}{Ter algo sobre ou contra si; aguentar, resistir, sustentar.}{su.por.tar}{0}
\verb{suportar}{}{}{}{}{}{Tolerar ou sofrer pacientemente; aturar.}{su.por.tar}{\verboinum{1}}
\verb{suporte}{ó}{}{}{}{s.m.}{Aquilo que sustenta algo; apoio, escora, base, sustentáculo.}{su.por.te}{0}
\verb{suporte}{ó}{}{}{}{}{Qualquer material sobre o qual se pode inscrever ou imprimir um texto, uma imagem etc.}{su.por.te}{0}
\verb{suporte}{ó}{Informát.}{}{}{}{Setor ou indivíduo que assiste tecnicamente o usuário de um \textit{software, hardware} etc.}{su.por.te}{0}
\verb{suposição}{}{}{"-ões}{}{s.f.}{Ato ou efeito de supor.}{su.po.si.ção}{0}
\verb{suposição}{}{}{"-ões}{}{}{Pressuposição, hipótese, conjetura, suposição.}{su.po.si.ção}{0}
\verb{supositório}{}{Farm.}{}{}{s.m.}{Medicamento sólido, cilíndrico, cuja introdução e absorção são feitas por algum orifício do corpo (ânus, vagina, uretra).}{su.po.si.tó.rio}{0}
\verb{supositório}{}{}{}{}{adj.}{Que se supõe; hipotético, suposto.}{su.po.si.tó.rio}{0}
\verb{suposto}{ô}{}{"-s ⟨ó⟩}{"-a ⟨ó⟩}{adj.}{Que se admite por hipótese; conjeturado, fictício.}{su.pos.to}{0}
\verb{supra}{}{}{}{}{adv.}{Termo latino que serve para indicar referência, trecho, palavra  etc. citado ou mencionado acima ou anteriormente. }{\textit{supra}}{0}
\verb{supracitado}{}{}{}{}{adj.}{Que foi citado acima ou anteriormente; supramencionado.}{su.pra.ci.ta.do}{0}
\verb{suprapartidário}{}{}{}{}{adj.}{Que está acima de partidos. }{su.pra.par.ti.dá.rio}{0}
\verb{suprapartidário}{}{}{}{}{}{Que reúne vários partidos, mas não está subordinado ao interesse de apenas um deles.}{su.pra.par.ti.dá.rio}{0}
\verb{suprarrenal}{}{}{suprarrenais}{}{adj.2g.}{Que está localizado acima dos rins.}{su.prar.re.nal}{0}
\verb{suprarrenal}{}{Anat.}{suprarrenais}{}{s.f.}{Glândula situada na posição superior dos rins.}{su.prar.re.nal}{0}
\verb{suprassumo}{}{}{suprassumos}{}{s.m.}{O grau mais elevado; ápice, auge, máximo, quinta"-essência.}{su.pras.su.mo}{0}
\verb{supremacia}{}{}{}{}{s.f.}{Superioridade incontestável; hegemonia, primazia, preponderância, preeminência. }{su.pre.ma.ci.a}{0}
\verb{supremo}{}{}{}{}{adj.}{Que está acima de tudo; sumo, máximo, extremo.}{su.pre.mo}{0}
\verb{supremo}{}{}{}{}{}{Relativo a Deus.}{su.pre.mo}{0}
\verb{supressão}{}{}{"-ões}{}{s.f.}{Ato ou efeito de suprimir.}{su.pres.são}{0}
\verb{supressão}{}{}{"-ões}{}{}{Eliminação, corte, retirada, extinção, cancelamento.}{su.pres.são}{0}
\verb{supressivo}{}{}{}{}{adj.}{Que suprime, elimina; supressor.}{su.pres.si.vo}{0}
\verb{supresso}{é}{}{}{}{adj.}{Que foi suprimido, cortado.}{su.pres.so}{0}
\verb{suprimento}{}{}{}{}{s.m.}{Ato ou efeito de suprir.}{su.pri.men.to}{0}
\verb{suprimento}{}{Bras.}{}{}{}{Provisão, fornecimento, aprovisionamento.}{su.pri.men.to}{0}
\verb{suprimento}{}{}{}{}{}{Suplemento, acréscimo, adição.}{su.pri.men.to}{0}
\verb{suprimir}{}{}{}{}{v.t.}{Fazer desaparecer; eliminar, extinguir, cancelar, abolir, anular.}{su.pri.mir}{0}
\verb{suprimir}{}{}{}{}{}{Impedir que uma publicação ou difusão  qualquer apareça.}{su.pri.mir}{0}
\verb{suprimir}{}{}{}{}{}{Não mencionar; omitir.}{su.pri.mir}{\verboinum{18}}
\verb{suprir}{}{}{}{}{v.t.}{Ajuntar para completar; inteirar, preencher.}{su.prir}{0}
\verb{suprir}{}{}{}{}{}{Abastecer, prover, aprovisionar.}{su.prir}{\verboinum{18}}
\verb{supurar}{}{}{}{}{v.t.}{Formar ou expelir pus.}{su.pu.rar}{\verboinum{8}}
\verb{supurativo}{}{}{}{}{adj.}{Relativo a supuração.}{su.pu.ra.ti.vo}{0}
\verb{supurativo}{}{}{}{}{s.m.}{Medicamento que produz ou facilita a supuração; supuratório.}{su.pu.ra.ti.vo}{0}
\verb{sura}{}{Anat.}{}{}{s.f.}{Barriga (batata) da perna; panturrilha.}{su.ra}{0}
\verb{sura}{}{Relig.}{}{}{s.f.}{Versículo, seção ou capítulo do Corão; surata.}{su.ra}{0}
\verb{surdez}{ê}{Med.}{}{}{s.f.}{Característica ou condição do que é surdo, apresenta ausência, perda ou diminuição considerável da audição; ensurdecência.}{sur.dez}{0}
\verb{surdina}{}{Mús.}{}{}{s.f.}{Dispositivo que serve para abafar o som ou alterar o timbre de certos instrumentos.}{sur.di.na}{0}
\verb{surdina}{}{}{}{}{}{Usado nas locuções \textit{à surdina }e\textit{ na surdina}: em silêncio, em segredo; à socapa.}{sur.di.na}{0}
\verb{surdir}{}{}{}{}{v.i.}{Sair de dentro; aparecer, surgir, irromper.}{sur.dir}{\verboinum{18}}
\verb{surdo}{}{}{}{}{adj.}{Que não ouve, ou ouve muito pouco; mouco.}{sur.do}{0}
\verb{surdo}{}{Por ext.}{}{}{}{Que não obedece; que não atende o que se pede.}{sur.do}{0}
\verb{surdo}{}{}{}{}{s.m.}{Pessoa que não ouve, que perdeu o sentido da audição.}{sur.do}{0}
\verb{surdo"-mudo}{}{}{surdos"-mudos}{surda"-muda}{adj.}{Diz"-se de indivíduo que é, ao mesmo tempo, surdo e mudo. }{sur.do"-mu.do}{0}
\verb{surfar}{}{Bras.}{}{}{v.i.}{Praticar, fazer surfe.}{sur.far}{\verboinum{1}}
\verb{surfe}{}{Esport.}{}{}{s.m.}{Modalidade esportiva, marítima, em que o praticante, equilibrando"-se em pé sobre uma prancha, desliza na crista de uma onda, executando manobras.}{sur.fe}{0}
\verb{surfista}{}{Bras.}{}{}{s.2g.}{Indivíduo que faz ou pratica surfe.}{sur.fis.ta}{0}
\verb{surgimento}{}{}{}{}{s.m.}{Ato ou efeito de surgir.  }{sur.gi.men.to}{0}
\verb{surgir}{}{}{}{}{v.i.}{Aparecer de repente ou de surpresa; sobrevir.}{sur.gir}{0}
\verb{surgir}{}{}{}{}{}{Nascer; manifestar"-se; aparecer.}{sur.gir}{\verboinum{22}}
\verb{surinamês}{}{}{}{}{adj.}{Relativo ao Suriname (antiga Guiana Holandesa, América do Sul).}{su.ri.na.mês}{0}
\verb{surinamês}{}{}{}{}{s.m.}{Indivíduo natural ou habitante desse país.}{su.ri.na.mês}{0}
\verb{suro}{}{Bras.}{}{}{adj.}{Diz"-se de animal sem cauda ou que apresenta apenas um coto de cauda. }{su.ro}{0}
\verb{surpreendente}{}{}{}{}{adj.2g.}{Que causa surpresa, espanto, admiração; inesperado.}{sur.pre.en.den.te}{0}
\verb{surpreendente}{}{}{}{}{}{Magnífico, espantoso, admirável, maravilhoso.}{sur.pre.en.den.te}{0}
\verb{surpreender}{ê}{}{}{}{v.t.}{Aparecer de repente diante de uma pessoa no momento em que ela faz alguma coisa; apanhar.  }{sur.pre.en.der}{0}
\verb{surpreender}{ê}{}{}{}{}{Deixar uma pessoa sem saber o que dizer ou fazer por causa de alguma coisa que ela não esperava; espantar, impressionar.}{sur.pre.en.der}{\verboinum{12}}
\verb{surpresa}{ê}{Por ext.}{}{}{}{Presente.}{sur.pre.sa}{0}
\verb{surpresa}{ê}{}{}{}{s.f.}{Ato ou efeito de surpreender.}{sur.pre.sa}{0}
\verb{surpresa}{ê}{}{}{}{}{Acontecimento ou fato inesperado, imprevisto, repentino.}{sur.pre.sa}{0}
\verb{surpresa}{ê}{}{}{}{}{Prazer inesperado.}{sur.pre.sa}{0}
\verb{surpreso}{ê}{}{}{}{adj.}{Que se surpreendeu; surpreendido, atônito, perplexo, pasmo, admirado.}{sur.pre.so}{0}
\verb{surra}{}{}{}{}{s.f.}{Ato ou efeito de surrar; tunda, sova.}{sur.ra}{0}
\verb{surra}{}{Bras.}{}{}{}{Derrota expressiva e humilhante imposta ao adversário.}{sur.ra}{0}
\verb{surrado}{}{}{}{}{adj.}{Que se surrou; espancado, socado.}{sur.ra.do}{0}
\verb{surrado}{}{Bras.}{}{}{}{Gasto pelo uso frequente; batido, puído.}{sur.ra.do}{0}
\verb{surrado}{}{}{}{}{}{Curtido, pisado.}{sur.ra.do}{0}
\verb{surrão}{}{}{"-ões}{}{s.m.}{Saco de couro usado como farnel pelos pastores; sarrão.}{sur.rão}{0}
\verb{surrão}{}{Por ext.}{"-ões}{}{}{Indivíduo imundo, porco.}{sur.rão}{0}
\verb{surrar}{}{}{}{}{v.t.}{Bater em pessoa ou animal.}{sur.rar}{0}
\verb{surrar}{}{}{}{}{}{Gastar alguma peça do vestuário, de tanto usar.}{sur.rar}{0}
\verb{surrar}{}{}{}{}{}{Vencer o adversário por muitos pontos de diferença. (\textit{Ontem, surraram nosso time de futebol.})}{sur.rar}{\verboinum{1}}
\verb{surrealismo}{}{Art.}{}{}{s.m.}{Movimento literário e artístico caracterizado pela expressão e primazia absoluta do inconsciente, pela espontaneidade e automatismo do pensamento, que exalta o sonho, o instinto e o desejo, e prega a renovação dos valores morais, políticos, científicos e filosóficos; suprarrealismo, super"-realismo.}{sur.re.a.lis.mo}{0}
\verb{surrealista}{}{}{}{}{adj.2g.}{Relativo a ou próprio do surrealismo. }{sur.re.a.lis.ta}{0}
\verb{surrealista}{}{}{}{}{s.2g.}{Indivíduo que adota o surrealismo ou é partidário dele.}{sur.re.a.lis.ta}{0}
\verb{surriada}{}{}{}{}{s.f.}{Descarga de artilharia ou de arma de fogo; tiroteio.}{sur.ri.a.da}{0}
\verb{surriada}{}{Fig.}{}{}{}{Troça, vaia, zombaria.}{sur.ri.a.da}{0}
\verb{surripiar}{}{}{}{}{}{Var. de \textit{surrupiar}.}{sur.ri.pi.ar}{0}
\verb{surrupiar}{}{Pop.}{}{}{v.t.}{Furtar, tirar sorrateiramente, às escondidas.}{sur.ru.pi.ar}{\verboinum{1}}
\verb{sursis}{}{Jur.}{}{}{s.m.}{Dispensa do cumprimento de uma pena; \textit{sursi}.}{\textit{sursis}}{0}
\verb{surtida}{}{}{}{}{s.f.}{Ataque, investida, arremetida, assalto (ao inimigo).}{sur.ti.da}{0}
\verb{surtir}{}{}{}{}{v.t.}{Produzir efeito; ter resultado; ter como consequência.}{sur.tir}{\verboinum{18}}
\verb{surto}{}{}{}{}{adj.}{Ancorado, fundeado.}{sur.to}{0}
\verb{surto}{}{}{}{}{s.m.}{Impulso, arrancada.}{sur.to}{0}
\verb{surto}{}{}{}{}{}{Surgimento repentino de vários casos de uma doença num mesmo local; irrupção, epidemia.}{sur.to}{0}
\verb{surto}{}{}{}{}{}{Deflagração de crise psicótica.}{sur.to}{0}
\verb{surto}{}{}{}{}{}{Ataque brusco, repentino; acesso.}{sur.to}{0}
\verb{suru}{}{}{}{}{adj.}{Suro.}{su.ru}{0}
\verb{surubi}{}{}{}{}{}{Var. de \textit{surubim}.}{su.ru.bi}{0}
\verb{surubim}{}{Zool.}{"-ins}{}{s.m.}{Peixe de água doce, de grande porte, amarelo com faixas ou pintas escuras transversais, tem a cabeça grande e achatada e pode atingir 3 m.   }{su.ru.bim}{0}
\verb{surucucu}{}{Zool.}{}{}{s.f.}{Serpente venenosa, da família dos viperídeos, que pode alcançar mais de 2 m de comprimento, encontrada nas matas tropicais do Brasil e da América Central; é a maior serpente peçonhenta das Américas.}{su.ru.cu.cu}{0}
\verb{surucutinga}{}{Zool.}{}{}{s.f.}{Surucucu.}{su.ru.cu.tin.ga}{0}
%\verb{}{}{}{}{}{}{}{}{0}
%\verb{}{}{}{}{}{}{}{}{0}
%\verb{}{}{}{}{}{}{}{}{0}
%\verb{}{}{}{}{}{}{}{}{0}
\verb{sururu}{}{Zool.}{}{}{s.m.}{Molusco bivalve comestível, de alto valor nutritivo, encontrado no litoral nordeste e sudeste do Brasil, especialmente no estado de Alagoas, onde é tradicionalmente consumido; siriri.}{su.ru.ru}{0}
\verb{sururu}{}{Pop.}{}{}{}{Confusão ou briga entre muitas pessoas; alarido, fuzuê, rolo.}{su.ru.ru}{0}
\verb{sus}{}{}{}{}{interj.}{Expressão para infundir coragem, ânimo; eia!, avante!}{sus}{0}
\verb{susceptibilidade}{}{}{}{}{}{Var. de \textit{suscetibilidade}.}{sus.cep.ti.bi.li.da.de}{0}
\verb{susceptibilizar}{}{}{}{}{}{Var. de \textit{suscetibilizar}.}{sus.cep.ti.bi.li.zar}{0}
\verb{susceptível}{}{}{}{}{}{Var. de \textit{suscetível}.}{sus.cep.tí.vel}{0}
\verb{suscetibilidade}{}{}{}{}{}{Predisposição para contrair enfermidades.}{sus.ce.ti.bi.li.da.de}{0}
\verb{suscetibilidade}{}{}{}{}{s.f.}{Qualidade ou condição do que é suscetível.}{sus.ce.ti.bi.li.da.de}{0}
\verb{suscetibilidade}{}{}{}{}{}{Predisposição para se ressentir ou se ofender com facilidade; melindre.}{sus.ce.ti.bi.li.da.de}{0}
\verb{suscetibilizar}{}{}{}{}{v.t.}{Melindrar ou ofender ligeiramente.}{sus.ce.ti.bi.li.zar}{\verboinum{1}}
\verb{suscetível}{}{}{"-eis}{}{adj.2g.}{Diz"-se do que tem a capacidade de receber impressões, sofrer alterações ou adquirir qualidades; capaz.}{sus.ce.tí.vel}{0}
\verb{suscetível}{}{}{"-eis}{}{}{Que se ressente ou se ofende com facilidade; melindroso.}{sus.ce.tí.vel}{0}
\verb{suscetível}{}{}{"-eis}{}{}{Que tem tendência a contrair doenças.}{sus.ce.tí.vel}{0}
\verb{suscitar}{}{}{}{}{v.t.}{Dar origem, fazer aparecer, causar, provocar.}{sus.ci.tar}{0}
\verb{suscitar}{}{}{}{}{}{Lembrar, sugerir.}{sus.ci.tar}{\verboinum{1}}
\verb{suserania}{}{}{}{}{s.f.}{Qualidade, condição, poder ou território de suserano.}{su.se.ra.ni.a}{0}
\verb{suserania}{}{}{}{}{}{Conjuntos de poderes e atribuições de suserano.}{su.se.ra.ni.a}{0}
\verb{suserano}{}{}{}{}{adj.}{Relativo ou pertencente a suserania.}{su.se.ra.no}{0}
\verb{suserano}{}{}{}{}{s.m.}{No feudalismo, indivíduo que dividia seu território entre aliados e subalternos (vassalos) sob juramento de fidelidade à defesa de seus domínios e prestação de serviços agrícolas e militares; senhor feudal.}{su.se.ra.no}{0}
\verb{sushi}{}{Cul.}{}{}{s.m.}{Iguaria japonesa que consiste num bolinho de arroz temperado com saquê e vinagre, envolvido em alga, com uma fatia de peixe cru, ou fruto do mar, que se tempera com molho de soja e pasta de raiz"-forte.}{\textit{sushi}}{0}
\verb{suspeição}{}{}{"-ões}{}{s.f.}{Ato ou efeito de suspeitar; desconfiança, dúvida, suspeita.}{sus.pei.ção}{0}
\verb{suspeição}{}{Jur.}{"-ões}{}{}{Situação que impede juízes, representantes do Ministério Público, advogados etc. de atuarem em certos processos em decorrência da dúvida de que não possam exercer suas funções de forma imparcial e independente.}{sus.pei.ção}{0}
\verb{suspeita}{ê}{}{}{}{s.f.}{Ato ou efeito de suspeitar; desconfiança, suspeição.}{sus.pei.ta}{0}
\verb{suspeita}{ê}{}{}{}{}{Desconfiança, convicção ou opinião, fundamentada em poucos indícios, mas não provada, a respeito de algo ou de alguém; suposição.}{sus.pei.ta}{0}
\verb{suspeitar}{}{}{}{}{v.t.}{Desconfiar, sem provas, de alguém ou de alguma coisa. }{sus.pei.tar}{0}
\verb{suspeitar}{}{}{}{}{}{Achar, acreditar, ou supor sem ter certeza; conjeturar.}{sus.pei.tar}{\verboinum{1}}
\verb{suspeito}{ê}{}{}{}{adj.}{Diz"-se do que provoca suspeita, dúvida, desconfiança, inquietação.}{sus.pei.to}{0}
\verb{suspeito}{ê}{}{}{}{s.m.}{Indivíduo que gera suspeita.}{sus.pei.to}{0}
\verb{suspeitoso}{ô}{}{"-osos ⟨ó⟩}{"-osa ⟨ó⟩}{adj.}{Que levanta suspeitas, desconfianças, dúvidas; receoso, desconfiado, apreensivo.}{sus.pei.to.so}{0}
\verb{suspender}{ê}{}{}{}{v.t.}{Levantar alguma coisa acima do solo; erguer, içar.}{sus.pen.der}{0}
\verb{suspender}{ê}{}{}{}{}{Interromper alguma coisa temporariamente.}{sus.pen.der}{0}
\verb{suspender}{ê}{}{}{}{}{Impedir alguém de exercer sua função por algum tempo ou indefinidamente como punição por falta cometida; privar, despojar.}{sus.pen.der}{\verboinum{12}}
\verb{suspensão}{}{}{"-ões}{}{s.f.}{Ato ou efeito de suspender.}{sus.pen.são}{0}
\verb{suspensão}{}{}{"-ões}{}{}{Adiamento temporário ou definitivo de algo; interrupção.}{sus.pen.são}{0}
\verb{suspensão}{}{Esport.}{"-ões}{}{}{No futebol, medida punitiva que proíbe um clube ou jogador de atuar em decorrência de infração disciplinar ou regulamentar.}{sus.pen.são}{0}
\verb{suspensão}{}{Fís. e Quím.}{"-ões}{}{}{Num sistema heterogêneo, incorporação de partículas sólidas a um meio líquido de maneira que a substância suspensa não se dissolva.}{sus.pen.são}{0}
\verb{suspense}{}{Art.}{}{}{s.m.}{Momento de tensão, ansiedade ou expectativa quanto ao desenrolar dos fatos de um enredo de um filme, livro, peça de teatro etc.}{sus.pen.se}{0}
\verb{suspense}{}{Por ext.}{}{}{}{Situação cuja resolução ou desfecho é aguardado ansiosamente.}{sus.pen.se}{0}
\verb{suspensivo}{}{}{}{}{adj.}{Que suspende ou tem capacidade de suspender.}{sus.pen.si.vo}{0}
\verb{suspenso}{}{}{}{}{adj.}{Que está pendurado; pendente.}{sus.pen.so}{0}
\verb{suspenso}{}{}{}{}{}{Que foi interrompido, cessado temporariamente.}{sus.pen.so}{0}
\verb{suspensório}{}{}{}{}{adj.}{Que suspende; próprio para suspender.}{sus.pen.só.rio}{0}
\verb{suspensório}{}{}{}{}{s.m.}{Suspensórios.}{sus.pen.só.rio}{0}
\verb{suspensórios}{}{}{}{}{s.m.pl.}{Peça de vestuário composta de duas tiras, geralmente elásticas, que, passadas sobre os ombros, seguram a calça ou a saia pelo cós; suspensório.}{sus.pen.só.ri.os}{0}
\verb{suspicaz}{}{}{}{}{adj.2g.}{Que gera suspeita; suspeito, estranho.}{sus.pi.caz}{0}
\verb{suspicaz}{}{}{}{}{}{Diz"-se daquele que é desconfiado; suspeitoso, matreiro.}{sus.pi.caz}{0}
\verb{suspirar}{}{}{}{}{v.i.}{Respirar dando suspiros.}{sus.pi.rar}{0}
\verb{suspirar}{}{}{}{}{v.t.}{Desejar alguma coisa com grande intensidade.}{sus.pi.rar}{\verboinum{1}}
\verb{suspiro}{}{}{}{}{s.m.}{Respiração ou expiração entrecortada, mais ou menos audível, que exprime ou manifesta desgosto, cansaço, alívio etc.  }{sus.pi.ro}{0}
\verb{suspiro}{}{Fig.}{}{}{}{Desejo, anseio.}{sus.pi.ro}{0}
\verb{suspiro}{}{Fig.}{}{}{}{Murmúrio ou gemido amoroso.}{sus.pi.ro}{0}
\verb{suspiro}{}{Fig.}{}{}{}{Lamentação, queixume, gemido, ai. }{sus.pi.ro}{0}
\verb{suspiro}{}{Cul.}{}{}{}{Doce de claras de ovos batidas em neve com açúcar; merengue.}{sus.pi.ro}{0}
\verb{suspiroso}{ô}{}{"-osos ⟨ó⟩}{"-osa ⟨ó⟩}{adj.}{Que suspira; lamentoso, choroso, queixoso. }{sus.pi.ro.so}{0}
\verb{sussurrante}{}{}{}{}{adj.2g.}{Que sussurra; murmurante, rumorejante.}{sus.sur.ran.te}{0}
\verb{sussurrar}{}{}{}{}{v.i.}{Produzir sussurros; murmurar, murmurejar, rumorejar.}{sus.sur.rar}{0}
\verb{sussurrar}{}{}{}{}{v.t.}{Falar em voz baixa ao ouvido; cochichar, segredar.}{sus.sur.rar}{\verboinum{1}}
\verb{sussurro}{}{}{}{}{s.m.}{Ato ou efeito de sussurrar; cochicho, murmúrio, cicio, zumbido, rumorejo.}{sus.sur.ro}{0}
\verb{sustância}{}{Pop.}{}{}{s.f.}{Vigor físico, força, robustez; sustança.}{sus.tân.cia}{0}
\verb{sustância}{}{}{}{}{}{Substância.}{sus.tân.cia}{0}
\verb{sustar}{}{}{}{}{v.t.}{Fazer parar, impedir de continuar; suspender, interromper.}{sus.tar}{\verboinum{1}}
\verb{sustável}{}{}{"-eis}{}{adj.2g.}{Que se pode sustar.}{sus.tá.vel}{0}
\verb{sustenido}{}{Mús.}{}{}{adj.}{Diz"-se da nota alterada pelo sinal de sustenido.}{sus.te.ni.do}{0}
\verb{sustenido}{}{Mús.}{}{}{s.m.}{Na notação musical, sinal que indica que a nota à sua direita deve ser elevada um semitom. Símb.: \#.}{sus.te.ni.do}{0}
\verb{sustentação}{}{}{"-ões}{}{s.f.}{Ato ou efeito de sustentar; apoio, sustentáculo.}{sus.ten.ta.ção}{0}
\verb{sustentação}{}{}{"-ões}{}{}{Alimento, sustento, abastecimento, nutrição.}{sus.ten.ta.ção}{0}
\verb{sustentação}{}{}{"-ões}{}{}{Manutenção, conservação. }{sus.ten.ta.ção}{0}
\verb{sustentação}{}{}{"-ões}{}{}{Confirmação, ratificação, validação.}{sus.ten.ta.ção}{0}
\verb{sustentáculo}{}{}{}{}{s.m.}{Que sustenta, sustém; apoio, escora, base, suporte.}{sus.ten.tá.cu.lo}{0}
\verb{sustentáculo}{}{Fig.}{}{}{}{Amparo, proteção, defesa, arrimo.}{sus.ten.tá.cu.lo}{0}
\verb{sustentar}{}{}{}{}{v.t.}{Segurar por baixo, evitar que caia; suster, apoiar.}{sus.ten.tar}{0}
\verb{sustentar}{}{}{}{}{}{Prover com o necessário; amparar, arrimar.}{sus.ten.tar}{0}
\verb{sustentar}{}{}{}{}{}{Aguentar, suportar.}{sus.ten.tar}{\verboinum{1}}
\verb{sustento}{}{}{}{}{s.m.}{Ato ou efeito de sustentar; alimentação, nutrição.}{sus.ten.to}{0}
\verb{sustento}{}{}{}{}{}{Aquilo que é necessário para garantir a vida; alimento, mantimento.}{sus.ten.to}{0}
\verb{suster}{ê}{}{}{}{v.t.}{Segurar para evitar a queda de algo; sustentar, firmar.}{sus.ter}{\verboinum{12}}
\verb{susto}{}{}{}{}{s.m.}{Medo repentino ou sobressalto causado por acontecimento súbito e inesperado; choque, abalo.}{sus.to}{0}
\verb{su"-sudeste}{é}{}{su"-sudestes ⟨é⟩}{}{s.m.}{Su"-sueste.}{su"-su.des.te}{0}
\verb{su"-sudoeste}{é}{}{su"-sudoestes ⟨é⟩}{}{s.m.}{Ponto do horizonte a meia distância angular do sul e do sudoeste. Abrev.: \textsc{s.s.o}. ou \textsc{s.s.w}.}{su"-su.do.es.te}{0}
\verb{su"-sueste}{é}{}{su"-suestes ⟨é⟩}{}{s.m.}{Ponto do horizonte a meia distância angular do sul e do sudeste. Abrev.: \textsc{s.s.e}.}{su"-su.es.te}{0}
\verb{sutache}{}{}{}{}{s.f.}{Galão, trança ou cadarço de seda, lã ou algodão, usado como enfeite de peças de vestuário ou para cobrir as costuras destas.}{su.ta.che}{0}
\verb{sutiã}{}{}{}{}{s.m.}{Peça íntima do vestuário feminino usada para suster ou modelar os seios; porta"-seios.}{su.ti.ã}{0}
\verb{sutiã}{}{Bras.}{}{}{}{No jornalismo, palavra ou frase que antecede o título; antetítulo.}{su.ti.ã}{0}
\verb{sutil}{}{}{"-is}{}{adj.2g.}{Demasiado tênue, fino, delgado, grácil.}{su.til}{0}
\verb{sutil}{}{}{"-is}{}{}{Que se infiltra, se insinua, penetra facilmente.}{su.til}{0}
\verb{sutil}{}{Fig.}{"-is}{}{}{Que tem sensibilidade apurada, agudeza de espírito; penetrante, perspicaz, arguto.}{su.til}{0}
\verb{sútil}{}{}{"-is}{}{adj.2g.}{Que se costurou; cosido.}{sú.til}{0}
\verb{sútil}{}{}{"-is}{}{}{Feito de pedaços cosidos uns aos outros.}{sú.til}{0}
\verb{sutileza}{ê}{}{}{}{s.f.}{Qualidade ou caráter do que é sutil.}{su.ti.le.za}{0}
\verb{sutileza}{ê}{}{}{}{}{Dito ou argumento com o objetivo de embaraçar alguém ou fazê"-lo contradizer"-se.}{su.ti.le.za}{0}
\verb{sutileza}{ê}{}{}{}{}{Sensibilidade apurada, agudeza de espírito; sagacidade, perspicácia, argúcia.}{su.ti.le.za}{0}
\verb{sutileza}{ê}{}{}{}{}{Detalhe quase imperceptível; discrição, particularidade, minúcia.}{su.ti.le.za}{0}
\verb{sutilizar}{}{}{}{}{v.t.}{Aguçar, aprimorar, apurar.}{su.ti.li.zar}{0}
\verb{sutilizar}{}{}{}{}{v.i.}{Falar, argumentar, discorrer com sutileza.}{su.ti.li.zar}{0}
\verb{sutilizar}{}{}{}{}{v.pron.}{Evolar"-se; volatilizar"-se, vaporizar"-se. }{su.ti.li.zar}{\verboinum{1}}
\verb{sutura}{}{}{}{}{s.f.}{Ato ou efeito de suturar; suturação, cosedura, costura.}{su.tu.ra}{0}
\verb{sutura}{}{Med.}{}{}{}{Operação de juntar os bordos de um corte, ferida ou incisão, com agulha e linha cirúrgica, para melhorar e apressar a cicatrização.}{su.tu.ra}{0}
\verb{suturar}{}{}{}{}{v.t.}{Fazer sutura; costurar.}{su.tu.rar}{\verboinum{1}}
\verb{suvenir}{}{}{}{}{s.m.}{Objeto característico de um lugar, que se adquire, geralmente, em viagens turísticas, como recordação; lembrança.}{su.ve.nir}{0}
\verb{SW}{}{}{}{}{}{Abrev. de \textit{sudoeste}.  }{S.W.}{0}
