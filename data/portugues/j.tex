\verb{j}{}{}{}{}{s.m.}{Décima letra do alfabeto português.}{j}{0}
\verb{J}{}{Fís.}{}{}{}{Símb. de \textit{joule}.}{J}{0}
\verb{já}{}{}{}{}{adv.}{Termo que refere um momento pontual que antecede algum outro em que uma ação ou uma ideia irá se desenvolver.}{já}{0}
\verb{jabá}{}{}{}{}{s.m.}{Carne de vaca, salgada e seca ao sol; charque.}{ja.bá}{0}
\verb{jabá}{}{Pop.}{}{}{}{Propina, suborno, jabaculê.}{ja.bá}{0}
\verb{jabaculê}{}{}{}{}{s.m.}{Dinheiro com que se corrompem pessoas influentes para se obter um benefício; suborno, propina.}{ja.ba.cu.lê}{0}
\verb{jabaculê}{}{Desus.}{}{}{}{Gorjeta, gratificação.}{ja.ba.cu.lê}{0}
\verb{jaborandi}{}{Bot.}{}{}{s.m.}{Nome comum dado a várias plantas que fornecem folhas medicinais.}{ja.bo.ran.di}{0}
\verb{jabota}{ó}{Zool.}{}{}{s.f.}{A fêmea do jabuti.}{ja.bo.ta}{0}
\verb{jaboti}{}{}{}{}{}{Var. de \textit{jabuti}.}{ja.bo.ti}{0}
\verb{jaburu}{}{Zool.}{}{}{s.m.}{Ave pernalta de grande porte, com bico longo e negro, que vive às margens dos rios, lagos e pântanos.}{ja.bu.ru}{0}
\verb{jaburu}{}{Fig.}{}{}{}{Indivíduo feio, muito magro e alto.}{ja.bu.ru}{0}
\verb{jabuti}{}{Zool.}{}{}{s.m.}{Nome comum dado a répteis quelônios, terrestres e herbívoros, que apresentam carapaça alta, cobertas por escudos poligonais amarelos e negros.}{ja.bu.ti}{0}
\verb{jabuticaba}{}{}{}{}{s.f.}{Fruto da jabuticabeira, redondo, de casca roxo"-escura e polpa branca de sabor muito doce.}{ja.bu.ti.ca.ba}{0}
\verb{jabuticaba}{}{}{}{}{}{O pé dessa fruta.}{ja.bu.ti.ca.ba}{0}
\verb{jabuticabal}{}{}{"-ais}{}{s.m.}{Coletivo de jabuticaba.}{ja.bu.ti.ca.bal}{0}
\verb{jabuticabeira}{ê}{Bot.}{}{}{s.f.}{Árvore de porte médio, de flores alvas e folhas pequenas, e cujos frutos, comestíveis, crescem diretamente do tronco liso.}{ja.bu.ti.ca.bei.ra}{0}
\verb{jaca}{}{}{}{}{s.f.}{Fruto da jaqueira, grande e pesado, de casca cheia de picos, gomos amarelos, viscosos e doces, e sementes grandes, também comestíveis quando assadas.}{ja.ca}{0}
\verb{jacá}{}{}{}{}{s.m.}{Grande cesto trançado de cipó ou taquara, de forma variável, preso ao lombo dos animais, usado para transportar alimentos.}{ja.cá}{0}
\verb{jaça}{}{}{}{}{s.f.}{Mancha ou falha na estrutura de uma pedra preciosa.}{ja.ça}{0}
\verb{jaça}{}{Fig.}{}{}{}{Mácula, defeito, imperfeição. }{ja.ça}{0}
\verb{jacamim}{}{Zool.}{"-ins}{}{}{Ave semelhante à galinha, de cabeça pequena, pescoço curto, bico forte e curvo e pernas avantajadas.}{ja.ca.mim}{0}
\verb{jacamim}{}{Bot.}{"-ins}{}{s.m.}{Árvore encontrada nas áreas inundáveis da Amazônia, que fornece madeira dura, amarela e compacta.}{ja.ca.mim}{0}
\verb{jaçanã}{}{Zool.}{}{}{s.f.}{Ave de pernas finas e compridas, pelos de cor castanha, bico alongado e amarelo, comum nos açudes e brejos brasileiros.}{ja.ça.nã}{0}
\verb{jacarandá}{}{Bot.}{}{}{s.m.}{Nome comum a diversas espécies de árvores de porte regular que fornecem preciosa madeira de lei.}{ja.ca.ran.dá}{0}
\verb{jacaré}{}{Zool.}{}{}{s.m.}{Nome comum a répteis de focinho largo e chato, pernas curtas, e dedos providos de garras e membranas natatórias, encontrados nos rios e pântanos da América do Norte e do Sul.}{ja.ca.ré}{0}
\verb{jacaré}{}{}{}{}{}{Colher de pedreiro, utilizado para introduzir argamassa nas juntas de alvenaria.}{ja.ca.ré}{0}
\verb{jacente}{}{}{}{}{}{Que está situado em determinado lugar; posicionado.}{ja.cen.te}{0}
\verb{jacente}{}{}{}{}{adj.2g.}{Que jaz; que está deitado; em posição horizontal.}{ja.cen.te}{0}
\verb{jacente}{}{}{}{}{}{Diz"-se da estátua ou figura representada na tampa dos túmulos.}{ja.cen.te}{0}
\verb{jacente}{}{Jur.}{}{}{}{Diz"-se da herança que não foi reclamada por nenhum herdeiro.}{ja.cen.te}{0}
\verb{jaci}{}{Bot.}{}{}{s.m.}{Palmeira de folhas muito compridas e frutos de polpa mole e com até três sementes, dos quais se extrai um óleo comestível.}{ja.ci}{0}
\verb{jacinto}{}{Bot.}{}{}{s.m.}{Erva bulbosa, cultivada como ornamental e de cujas flores, azuis, rosas ou brancas, extrai"-se uma essência utilizada na fabricação de perfumes. }{ja.cin.to}{0}
\verb{jacobinismo}{}{}{}{}{s.m.}{Doutrina política revolucionária fundada em Paris, em 1789, que defendia uma democracia igualitária de forma exacerbada e intransigente.}{ja.co.bi.nis.mo}{0}
\verb{jacobinismo}{}{Por ext.}{}{}{}{Ideologia revolucionária exaltada; radicalismo.}{ja.co.bi.nis.mo}{0}
\verb{jacobino}{}{}{}{}{s.m.}{Membro da ordem dos jacobinos ou partidário do jacobinismo.}{ja.co.bi.no}{0}
\verb{jacobino}{}{}{}{}{}{Partidário exaltado da democracia e da centralização do poder do Estado.}{ja.co.bi.no}{0}
\verb{jacobino}{}{Por ext.}{}{}{}{Nacionalista radical; xenófobo.}{ja.co.bi.no}{0}
\verb{já"-começa}{é}{Pop.}{}{}{s.2g.}{Coceira causada por uma afecção; sarna.}{já"-co.me.ça}{0}
\verb{já"-começa}{é}{Pop.}{}{}{}{Cachaça, aguardente.}{já"-co.me.ça}{0}
\verb{jactância}{}{}{}{}{s.f.}{Comportamento de alguém que revela arrogância; vaidade, orgulho.}{jac.tân.cia}{0}
\verb{jactância}{}{}{}{}{}{Atitude de quem conta vantagens, bravatas; fanfarronice.}{jac.tân.cia}{0}
\verb{jactancioso}{ô}{}{"-osos ⟨ó⟩}{"-osa ⟨ó⟩}{adj.}{Que manifesta jactância; orgulhoso, vaidoso, arrogante.}{jac.tan.ci.o.so}{0}
\verb{jactancioso}{ô}{}{"-osos ⟨ó⟩}{"-osa ⟨ó⟩}{}{Que conta vantagens, bravatas; fanfarrão, pretencioso.}{jac.tan.ci.o.so}{0}
\verb{jactar"-se}{}{}{}{}{v.pron.}{Expressar orgulho exagerado; vangloriar"-se, ufanar"-se.}{jac.tar"-se}{\verboinum{1}}
\verb{jacto}{}{}{}{}{}{Var. de \textit{jato}.}{jac.to}{0}
\verb{jacu}{}{Zool.}{}{}{s.m.}{Nome comum a várias aves galiformes, que se alimentam de folhas e frutas e habitam as matas primárias do Brasil.}{ja.cu}{0}
\verb{jacuba}{}{}{}{}{s.f.}{Refresco feito com água, farinha de mandioca, açúcar ou mel e, às vezes, um pouco de cachaça ou leite.}{ja.cu.ba}{0}
\verb{jacuba}{}{}{}{}{}{Café engrossado com farinha de mandioca.}{ja.cu.ba}{0}
\verb{jacular}{}{}{}{}{v.t.}{Lançar longe; arremessar.}{ja.cu.lar}{0}
\verb{jacular}{}{}{}{}{}{Ferir com arma de arremesso. }{ja.cu.lar}{0}
\verb{jacular}{}{}{}{}{}{Ejacular.}{ja.cu.lar}{\verboinum{1}}
\verb{jaculatória}{}{Relig.}{}{}{s.f.}{Oração católica breve e fervorosa.}{ja.cu.la.tó.ria}{0}
\verb{jacumã}{}{}{}{}{}{Indivíduo que maneja ou pilota a canoa.}{ja.cu.mã}{0}
\verb{jacumã}{}{}{}{}{s.m.}{Espécie de remo indígena em forma de pá.}{ja.cu.mã}{0}
\verb{jacundá}{}{Zool.}{}{}{s.m.}{Nome comum dado a peixes de água doce, semelhantes à traíra, que possuem o corpo alongado e uma nadadeira dorsal contínua, dividida em uma parte espinhosa e outra ramosa.}{ja.cun.dá}{0}
\verb{jacutinga}{}{Zool.}{}{}{s.f.}{Nome comum dado a aves galiformes, de coloração azulada e parte superior da cabeça branca, que ocorre nas matas virgens desde a Bahia até o Rio Grande do Sul.}{ja.cu.tin.ga}{0}
\verb{jade}{}{Geol.}{}{}{s.m.}{Rocha muito dura, de coloração variando do branco ao verde"-escuro, usada como gema semipreciosa ou na escultura de estatuetas e objetos decorativos.}{ja.de}{0}
\verb{jaez}{ê}{}{}{}{s.m.}{Conjunto de traços ou características; qualidade fundamental; tipo específico.}{ja.ez}{0}
\verb{jaez}{ê}{}{}{}{}{Conjunto das peças que aparelham ou adornam as cavalgaduras.}{ja.ez}{0}
\verb{jaguar}{}{Zool.}{}{}{s.m.}{Animal felino de grande porte, de coloração amarelo"-avermelhada, com manchas pretas arredondadas, encontrado em quase todo o continente americano; onça"-pintada.}{ja.guar}{0}
\verb{jaguaré}{}{Zool.}{}{}{s.m.}{Mamífero carnívoro, assemelhado à jaritataca, encontrado nas regiões campestres do sul da América do Sul; zorrilho. }{ja.gua.ré}{0}
\verb{jaguatirica}{}{Zool.}{}{}{s.f.}{Animal felino de médio porte, de coloração amarela com manchas pretas, encontrado principalmente na América do Sul.}{ja.gua.ti.ri.ca}{0}
\verb{jagunço}{}{}{}{}{s.m.}{No Nordeste, criminoso contratado para ser guarda"-costas de indivíduo influente; cangaceiro, capanga.}{ja.gun.ço}{0}
\verb{jagunço}{}{}{}{}{}{Nome que se dava aos seguidores de Antônio Conselheiro na campanha de Canudos, no final do século \textsc{xix}.}{ja.gun.ço}{0}
\verb{jaleco}{é}{}{}{}{s.m.}{Casaco de tecido leve usado por médicos, dentistas etc. para a proteção da roupa durante o trabalho.}{ja.le.co}{0}
\verb{jamaicano}{}{}{}{}{adj.}{Relativo à Jamaica (Antilhas).}{ja.mai.ca.no}{0}
\verb{jamaicano}{}{}{}{}{s.m.}{Indivíduo natural ou habitante desse país.}{ja.mai.ca.no}{0}
\verb{jamais}{}{}{}{}{adv.}{Em tempo algum; nunca.}{ja.mais}{0}
\verb{jamais}{}{}{}{}{}{Em nenhuma condição; de forma alguma.}{ja.mais}{0}
\verb{jamanta}{}{Zool.}{}{}{s.f.}{Grande peixe cartilaginoso, semelhante à raia, de cabeça achatada e nadadeiras peitorais, com dois lóbulos em forma de chifre na cabeça.}{ja.man.ta}{0}
\verb{jamanta}{}{Fig.}{}{}{}{Pessoa de volume avantajado, desengonçada, grandalhona.}{ja.man.ta}{0}
\verb{jamanta}{}{}{}{}{}{Veículo de grandes dimensões, usado para transporte de carga pesada.}{ja.man.ta}{0}
\verb{jambeiro}{ê}{Bot.}{}{}{s.m.}{Árvore de folhas amplas e brilhantes e flores vermelhas, de fruto comestível e casca adstringente.}{jam.bei.ro}{0}
\verb{jambo}{}{Gram.}{}{}{s.m.}{Pé métrico composto de duas sílabas, uma breve e outra longa.}{jam.bo}{0}
\verb{jambo}{}{}{}{}{s.m.}{Fruto comestível do jambeiro, constituído de uma baga amarelo rosado, aromática e suculenta.}{jam.bo}{0}
\verb{jambolão}{}{}{"-ões}{}{s.m.}{Jamelão.}{jam.bo.lão}{0}
\verb{jamegão}{}{Pop.}{"-ões}{}{s.m.}{Nome firmado na parte inferior de um documento; assinatura, rubrica, firma.}{ja.me.gão}{0}
\verb{jamelão}{}{Bot.}{"-ões}{}{s.m.}{Árvore com folhas oblongas e flores com vários estames, cujo fruto é uma baga roxa, comestível e carnosa, e expele um corante violáceo; jambolão.}{ja.me.lão}{0}
\verb{jamelão}{}{}{"-ões}{}{}{O fruto dessa árvore.}{ja.me.lão}{0}
\verb{jandaia}{}{Zool.}{}{}{s.f.}{Nome comum a três espécies de periquitos de coloração distinta, uma de alaranjado intenso, outra de dorso verde e uma terceira verde"-escura com a cabeça vermelha.}{jan.dai.a}{0}
\verb{janeiro}{ê}{}{}{}{s.m.}{O primeiro mês do ano civil.}{ja.nei.ro}{0}
\verb{janela}{é}{}{}{}{s.f.}{Abertura ou vão na parede externa de uma construção, para deixar passar a luz e o ar.}{ja.ne.la}{0}
\verb{janela}{é}{}{}{}{}{Abertura nos veículos, em geral com vidro.}{ja.ne.la}{0}
\verb{janela}{é}{}{}{}{}{Qualquer abertura, buraco ou rasgão.}{ja.ne.la}{0}
\verb{janela}{é}{Fig.}{}{}{}{Em um texto, espaço em branco onde falta uma palavra; lacuna.}{ja.ne.la}{0}
\verb{jangada}{}{}{}{}{s.f.}{Embarcação de pescadores do Norte e Nordeste do Brasil, feita de paus roliços muito leves, amarrados entre si, com um mastro ao qual se iça uma vela.}{jan.ga.da}{0}
\verb{jangada}{}{}{}{}{}{Balsa ou prancha flutuante feita de peças de madeira usadas para transportar pessoas, mercadorias etc.}{jan.ga.da}{0}
\verb{jangadeiro}{ê}{}{}{}{s.m.}{Proprietário ou pescador tripulante de jangada.}{jan.ga.dei.ro}{0}
\verb{jângal}{}{}{jângales}{}{s.m.}{Nome comum dado às florestas densas e selvagens do sul da Ásia.}{jân.gal}{0}
\verb{jângal}{}{Por ext.}{jângales}{}{}{Mata virgem, floresta, selva.}{jân.gal}{0}
\verb{janota}{ó}{}{}{}{s.m.}{Indivíduo que se veste com apuro exagerado; muito enfeitado, almofadinha.}{ja.no.ta}{0}
\verb{janta}{}{Pop.}{}{}{s.f.}{Jantar comum em família.}{jan.ta}{0}
\verb{jantar}{}{}{}{}{s.m.}{Refeição tomada à noite.}{jan.tar}{0}
\verb{jantar}{}{}{}{}{}{O conjunto dos pratos que compõem essa refeição.}{jan.tar}{0}
\verb{jantar}{}{}{}{}{v.i.}{Comer a refeição noturna; cear.}{jan.tar}{\verboinum{1}}
\verb{jaó}{}{Zool.}{}{}{s.m.}{Ave de plumagem estriada, vivamente colorida, pescoço branco e pernas esverdeadas, muito comum na Amazônia e no Pantanal.}{ja.ó}{0}
\verb{japão}{}{Pop.}{"-ões}{}{adj. e s.m.  }{Japonês.}{ja.pão}{0}
\verb{japi}{}{Zool.}{}{}{s.m.}{Japim.}{ja.pi}{0}
\verb{japim}{}{Zool.}{"-ins}{}{s.m.}{Ave de coloração negra, cauda e bico amarelo forte, que ocorre em grande parte do Brasil; xexéu.}{ja.pim}{0}
\verb{japona}{ô}{}{}{}{s.f.}{Casaco esportivo de lã grossa ou tecido sintético, com feitio semelhante ao do jaquetão.}{ja.po.na}{0}
\verb{japonês}{}{}{}{}{adj.}{Relativo ao Japão.}{ja.po.nês}{0}
\verb{japonês}{}{}{}{}{s.m.}{Indivíduo natural ou habitante desse país; japônico.}{ja.po.nês}{0}
\verb{japonês}{}{}{}{}{}{Língua falada no Japão.}{ja.po.nês}{0}
\verb{japônico}{}{}{}{}{adj. e s.m.  }{Japonês.}{ja.pô.ni.co}{0}
\verb{jaqueira}{ê}{Bot.}{}{}{s.f.}{Árvore de grande porte, de madeira amarela, cultivada pelos grandes frutos com polpa comestível, amarela e viscosa, a jaca.}{ja.quei.ra}{0}
\verb{jaqueta}{ê}{}{}{}{s.f.}{Casaco sem abas, justo e curto, que chega só até a cintura.}{ja.que.ta}{0}
\verb{jaqueta}{ê}{}{}{}{}{Espécie de prótese dentária de cerâmica ou resina sintética que reveste um dente lesado.}{ja.que.ta}{0}
\verb{jaquetão}{}{}{"-ões}{}{s.m.}{Espécie de jaqueta ou casaco largo de tecido grosso, que desce até abaixo da cintura, usado principalmente no inverno.}{ja.que.tão}{0}
\verb{jaraguá}{}{}{}{}{s.m.}{Capim alto com inflorescências de cor vermelho escuro, muito comum nos pastos do Brasil, utilizado como forragem para o gado bovino.}{ja.ra.guá}{0}
\verb{jararaca}{}{Zool.}{}{}{s.f.}{Serpente venenosa, que atinge até 1,5 m de comprimento, com cabeça triangular, corpo revestido de escamas e sem guizo, muito comum nos cerrados e campos cultivados do Brasil.}{ja.ra.ra.ca}{0}
\verb{jararaca}{}{Pop.}{}{}{}{Indivíduo de mau gênio, traiçoeiro.}{ja.ra.ra.ca}{0}
\verb{jararacuçu}{}{Zool.}{}{}{s.2g.}{Serpente venenosa, de coloração variável, com manchas triangulares marrom"-escuras, que atinge até 2,20 m de comprimento, encontrada principalmente nas áreas alagadas e baixas do Brasil.}{ja.ra.ra.cu.çu}{0}
\verb{jarda}{}{}{}{}{s.f.}{Unidade de medida de comprimento, utilizada nos países anglo"-saxões, que equivale a 91,44 cm ou a 3 pés. Símb.: yd.}{jar.da}{0}
\verb{jardim}{}{}{"-ins}{}{s.m.}{Pedaço de terreno, geralmente fechado, onde se cultivam flores, plantas e árvores ornamentais.}{jar.dim}{0}
\verb{jardim}{}{}{"-ins}{}{}{Local ou região que apresenta vegetação rica, abundante e harmoniosa.}{jar.dim}{0}
\verb{jardim"-de"-infância}{}{}{jardins"-de"-infância}{}{s.m.}{Estabelecimento de educação pré"-escolar, destinado a crianças menores de seis anos.}{jar.dim"-de"-in.fân.cia}{0}
\verb{jardim"-de"-inverno}{é}{}{jardins"-de"-inverno ⟨é⟩}{}{s.m.}{Área envidraçada, geralmente anexa a uma sala, onde se cultivam plantas e flores e que também costuma ser utilizada como sala de estar e de recreação.}{jar.dim"-de"-in.ver.no}{0}
\verb{jardinagem}{}{}{"-ens}{}{s.f.}{Técnica de cultivar e manter jardins.}{jar.di.na.gem}{0}
\verb{jardinar}{}{}{}{}{v.t.}{Cultivar jardins; praticar a jardinagem.}{jar.di.nar}{\verboinum{1}}
\verb{jardineira}{ê}{}{}{}{}{Mulher que trata de jardins.}{jar.di.nei.ra}{0}
\verb{jardineira}{ê}{}{}{}{s.f.}{Recipiente ou canteiro de forma alongada onde se cultivam flores e plantas ornamentais.}{jar.di.nei.ra}{0}
\verb{jardineira}{ê}{}{}{}{}{Modelo de calça com peitilho costurado à cintura, preso com alças que se abotoam atrás no cós.}{jar.di.nei.ra}{0}
\verb{jardineira}{ê}{}{}{}{}{Tipo de ônibus com bancos dispostos lateralmente.}{jar.di.nei.ra}{0}
\verb{jardineiro}{ê}{}{}{}{s.m.}{Indivíduo que trata de jardins ou pratica a jardinagem.}{jar.di.nei.ro}{0}
\verb{jargão}{}{}{"-ões}{}{s.m.}{Variedade linguística própria de uma profissão ou atividade, com vocabulário técnico e especializado da área.}{jar.gão}{0}
\verb{jarina}{}{Bot.}{}{}{s.f.}{Palmeira baixa, cujas folhas são usadas como cobertura de casas, e de cujas sementes, grandes e duras, fabricam"-se botões, dados e outros pequenos objetos.}{ja.ri.na}{0}
\verb{jaritataca}{}{Zool.}{}{}{s.f.}{Mamífero carnívoro, de coloração preta, com uma faixa branca dorsal, que excreta, pela glândula anal, um líquido fétido e nauseante, como defesa; cangambá, gambá.}{ja.ri.ta.ta.ca}{0}
\verb{jarra}{}{}{}{}{s.f.}{Recipiente utilizado para servir água, refrescos ou outras bebidas à mesa.}{jar.ra}{0}
\verb{jarra}{}{}{}{}{}{Vaso usado para conter flores; jarro.}{jar.ra}{0}
\verb{jarrete}{ê}{Anat.}{}{}{s.m.}{A parte da perna situada atrás do joelho, oposta à rótula.}{jar.re.te}{0}
\verb{jarrete}{ê}{Veter.}{}{}{}{Tendão ou nervo da perna de bois e cavalos.}{jar.re.te}{0}
\verb{jarro}{}{}{}{}{s.m.}{Vaso com asa e bico, usado tanto para conter flores, quanto para conter água; jarra.}{jar.ro}{0}
\verb{jasmim}{}{Bot.}{"-ins}{}{s.m.}{Nome comum dado a vários arbustos e trepadeiras, cultivados pelas flores brancas aromáticas, das quais se extrai um óleo utilizado na fabricação de perfumes.}{jas.mim}{0}
\verb{jasmim}{}{}{"-ins}{}{}{Perfume extraído da flor dessa planta.}{jas.mim}{0}
\verb{jasmim"-do"-cabo}{}{Bot.}{jasmins"-do"-cabo}{}{s.m.}{Arbusto de flores grandes e brancas, cultivado como ornamental e pelas propriedades medicinais de sua madeira; gardênia.}{jas.mim"-do"-ca.bo}{0}
\verb{jasmineiro}{ê}{Bot.}{}{}{s.m.}{Jasmim.}{jas.mi.nei.ro}{0}
\verb{jaspe}{}{Geol.}{}{}{s.m.}{Variedade semicristalina de quartzo opaco, de cores diversas, sendo a cor mais comum a vermelha, devido a inclusões de óxido de ferro, muito utilizada como pedra ornamental.}{jas.pe}{0}
\verb{jataí}{}{Bot.}{}{}{s.m.}{Árvore de grande porte, de madeira avermelhada e dura, e de cujo tronco se extrai uma resina usada na fabricação de vernizes; jatobá.}{ja.ta.í}{0}
\verb{jataí}{}{Zool.}{}{}{s.f.}{Espécie de abelha brasileira que faz o ninho em troncos ocos ou em vãos de parede e cujo mel, claro e de aroma suave, é muito apreciado.}{ja.ta.í}{0}
\verb{jato}{}{}{}{}{s.m.}{Movimento de algo que sai impetuosamente de um orifício ou abertura; jacto.}{ja.to}{0}
\verb{jato}{}{}{}{}{}{Emissão súbita; lançamento, arremesso.}{ja.to}{0}
\verb{jato}{}{}{}{}{}{Manifestação repentina de um sentimento; arroubo, impulso.}{ja.to}{0}
\verb{jatobá}{}{Bot.}{}{}{s.m.}{Árvore de grande porte, de madeira avermelhada e dura, de cujo tronco se extrai uma resina usada na fabricação de vernizes; jataí.}{ja.to.bá}{0}
\verb{jau}{}{}{}{}{adj. e s.m.  }{Javanês.}{jau}{0}
\verb{jaú}{}{Zool.}{}{}{s.m.}{Peixe de água doce, de coloração parda com manchas escuras, que chega a atingir 1,5 m de comprimento e pesa até 120 kg.}{ja.ú}{0}
\verb{jaula}{}{}{}{}{s.f.}{Prisão com grades de ferro, utilizada para  guardar animais ferozes.}{jau.la}{0}
\verb{javali}{}{Zool.}{}{}{s.m.}{Mamífero semelhante ao porco selvagem, de cabeça grande, focinho alongado e provido de dentes caninos salientes.}{ja.va.li}{0}
\verb{javanês}{}{}{}{}{adj.}{Relativo a Java, ilha da Polinésia; jau.}{ja.va.nês}{0}
\verb{javanês}{}{}{}{}{s.m.}{Indivíduo natural ou habitante dessa ilha.}{ja.va.nês}{0}
\verb{jazer}{ê}{}{}{}{v.i.}{Estar deitado, estendido em posição horizontal.}{ja.zer}{0}
\verb{jazer}{ê}{}{}{}{}{Estar morto ou parecer morto.}{ja.zer}{0}
\verb{jazer}{ê}{}{}{}{}{Estar quieto, imóvel.}{ja.zer}{0}
\verb{jazer}{ê}{}{}{}{}{Estar sepultado, enterrado.}{ja.zer}{\verboinum{54}}
\verb{jazida}{}{}{}{}{s.f.}{Local onde alguém jaz, repousa.}{ja.zi.da}{0}
\verb{jazida}{}{Geol.}{}{}{}{Depósito natural de minérios que podem ser explorados economicamente.}{ja.zi.da}{0}
\verb{jazigo}{}{}{}{}{s.m.}{Buraco na terra, onde se enterram os cadáveres; sepultura, cova.}{ja.zi.go}{0}
\verb{jazigo}{}{}{}{}{}{Monumento funerário.}{ja.zi.go}{0}
\verb{jazz}{}{Mús.}{}{}{s.m.}{Música de origem negro"-americana, caracterizada pela improvisação e pelos ritmos e sonoridades sincopados.}{\textit{jazz}}{0}
\verb{jê}{}{}{}{}{s.m.}{Família linguística filiada ao tronco macro"-jê, que reúne diversas línguas indígenas de tribos brasileiras.}{jê}{0}
\verb{jê}{}{}{}{}{adj.}{Relativo a uma dessas línguas ou a seus falantes.}{jê}{0}
\verb{jeans}{}{}{}{}{s.m.}{Tecido de algodão durável, de trama grossa, geralmente tingido de índigo, e usado na confecção de calças e outras peças de vestuário; brim.}{\textit{jeans}}{0}
\verb{jeans}{}{}{}{}{}{Calças feitas desse tecido.}{\textit{jeans}}{0}
\verb{jeca}{é}{}{}{}{adj.}{Diz"-se do indivíduo que vive no interior; que mora no campo ou na roça; caipira.}{je.ca}{0}
\verb{jeca}{é}{Fig.}{}{}{}{Que revela mau gosto; cafona. }{je.ca}{0}
\verb{jeca"-tatu}{é}{}{jecas"-tatus ⟨é⟩}{}{s.m.}{Nome e símbolo do habitante da roça, quando doente e desanimado, inspirado no personagem homônimo da obra "Urupês", de Monteiro Lobato.}{je.ca"-ta.tu}{0}
\verb{jegue}{é}{}{}{}{s.m.}{No Nordeste, nome dado ao jumento.}{je.gue}{0}
\verb{jeira}{ê}{}{}{}{s.f.}{Medida agrária que varia, conforme o país, de 19 a 36 hectares.}{jei.ra}{0}
\verb{jeitinho}{}{}{}{}{s.m.}{Modo delicado de agir ou se comportar.}{jei.ti.nho}{0}
\verb{jeitinho}{}{Bras.}{}{}{}{Maneira hábil e astuciosa de se conseguir algo, principalmente em situações difíceis.}{jei.ti.nho}{0}
\verb{jeito}{ê}{}{}{}{s.m.}{Maneira; modo. (\textit{Esse é o jeito certo de fazer as coisas.})}{jei.to}{0}
\verb{jeito}{ê}{}{}{}{}{Feição; gesto.}{jei.to}{0}
\verb{jeito}{ê}{}{}{}{}{Habilidade; queda; propensão. (\textit{Ele tem muito jeito para a música.})}{jei.to}{0}
\verb{jeitoso}{ô}{}{"-osos ⟨ó⟩}{"-osa ⟨"-ó⟩}{adj.}{Que tem jeito, habilidade, para realizar algo; capaz, apto.}{jei.to.so}{0}
\verb{jeitoso}{ô}{}{"-osos ⟨ó⟩}{"-osa ⟨"-ó⟩}{}{Que possui boa aparência; atraente, gracioso.}{jei.to.so}{0}
\verb{jeitoso}{ô}{}{"-osos ⟨ó⟩}{"-osa ⟨"-ó⟩}{}{Próprio, adequado, conveniente.}{jei.to.so}{0}
\verb{jejuar}{}{}{}{}{v.i.}{Abster"-se ou privar"-se de comer algo.}{je.ju.ar}{0}
\verb{jejuar}{}{}{}{}{}{Observar o preceito religioso do jejum.}{je.ju.ar}{\verboinum{1}}
\verb{jejum}{}{}{"-uns}{}{s.m.}{Privação ou abstinência de algo durante certo tempo.}{je.jum}{0}
\verb{jejum}{}{}{"-uns}{}{}{Período em que ocorre essa privação.}{je.jum}{0}
\verb{jejum}{}{}{"-uns}{}{}{Preceito religioso que consiste na privação completa ou parcial de alimentos.}{je.jum}{0}
\verb{jejuno}{}{}{}{}{adj.}{Diz"-se daquele que está praticando o jejum.}{je.ju.no}{0}
\verb{jejuno}{}{}{}{}{}{Ignorante, insipiente, inexperiente.}{je.ju.no}{0}
\verb{jejuno}{}{Anat.}{}{}{s.m.}{Porção do intestino delgado situada entre o duodeno e o íleo.}{je.ju.no}{0}
\verb{jenipapeiro}{ê}{Bot.}{}{}{s.m.}{Árvore baixa e grossa, que fornece madeira de boa qualidade e cujo fruto, comestível, é utilizado tanto na produção de licores quanto para extração de uma tinta azul"-escura com que os índios pintam o corpo.}{je.ni.pa.pei.ro}{0}
\verb{jenipapo}{}{}{}{}{s.m.}{Fruto do jenipapeiro, de baga comestível, e do qual se extrai uma tintura utilizada pelos índios na pintura corporal.}{je.ni.pa.po}{0}
\verb{jeová}{}{Relig.}{}{}{s.m.}{Designação de Deus no Antigo Testamento.}{je.o.vá}{0}
\verb{jequi}{}{}{}{}{s.m.}{Cesto oblongo e afunilado, usado para pesca.}{je.qui}{0}
\verb{jequiá}{}{}{}{}{s.m.}{Cesto grande para pesca em lugares rasos.}{je.qui.á}{0}
\verb{jequice}{}{}{}{}{s.f.}{Modo ou atitude de jeca, caipira.}{je.qui.ce}{0}
\verb{jequitibá}{}{Bot.}{}{}{s.m.}{Árvore de grande porte, com um tronco grosso que chega a medir 1 m de diâmetro, cuja madeira era usada para a construção civil e para a fabricação de pequenos objetos.}{je.qui.ti.bá}{0}
\verb{jequitiranaboia}{ó}{Zool.}{}{}{s.f.}{Nome dado a um inseto cuja cabeça lembra a de um lagarto, o que o torna temido pelo povo, embora seja inofensivo.}{je.qui.ti.ra.na.boi.a}{0}
\verb{jereré}{}{}{}{}{s.m.}{Espécie de rede em forma de saco usada para pescar camarões, siris e peixes miúdos.}{je.re.ré}{0}
\verb{jererê}{}{Pop.}{}{}{s.m.}{Cigarro de maconha; baseado.}{je.re.rê}{0}
\verb{jerico}{}{}{}{}{s.m.}{Asno, burrico, jumento.}{je.ri.co}{0}
\verb{jerimum}{}{}{"-uns}{}{s.m.}{No Nordeste, nome dado à abóbora.}{je.ri.mum}{0}
\verb{jerimunzeiro}{ê}{}{}{}{s.f.}{No Nordeste, nome dado à aboboreira.}{je.ri.mun.zei.ro}{0}
\verb{jeropiga}{}{}{}{}{s.f.}{Bebida alcoólica feita de suco de fruta, fermentado ou não, aguardente e açúcar.}{je.ro.pi.ga}{0}
\verb{jeropiga}{}{}{}{}{}{Vinho ou aguardente de baixa qualidade.}{je.ro.pi.ga}{0}
\verb{jérsei}{}{}{}{}{s.m.}{Tecido de malha muito fina, fabricada em lã, linho, algodão ou seda, usado na confecção de roupas.}{jér.sei}{0}
\verb{jesuíta}{}{}{}{}{s.m.}{Membro da Companhia de Jesus, ordem religiosa católica fundada por Inácio de Loyola, no século \textsc{xvi}.}{je.su.í.ta}{0}
\verb{jesuítico}{}{}{}{}{adj.}{Relativo aos jesuítas. }{je.su.í.ti.co}{0}
\verb{jesuítico}{}{Fig.}{}{}{}{Hipócrita, dissimulado, fingido.}{je.su.í.ti.co}{0}
\verb{jesuitismo}{}{}{}{}{s.m.}{Sistema ou doutrina moral dos jesuítas.}{je.su.i.tis.mo}{0}
\verb{jesuitismo}{}{Fig.}{}{}{}{Hipocrisia, falsidade, dissimulação.}{je.su.i.tis.mo}{0}
\verb{jetom}{}{}{"-ons}{}{s.m.}{Pequena ficha que vale como símbolo de comparecimento a uma reunião entregue aos membros participantes e que lhes servem para cobrarem uma remuneração.}{je.tom}{0}
\verb{jetom}{}{}{"-ons}{}{}{Essa remuneração.}{je.tom}{0}
\verb{jia}{}{Bras.}{}{}{s.f.}{Nome popular dado às rãs.}{ji.a}{0}
\verb{jiboia}{ó}{Zool.}{}{}{s.f.}{Serpente arborícola não venenosa, com dorso amarelo ou cinza e manchas avermelhadas, que se alimenta de aves e roedores e pode atingir até 4 m de comprimento.}{ji.boi.a}{0}
\verb{jiboia}{ó}{Bot.}{}{}{}{Planta ornamental de caule ramoso e folhas oblongas verdes com manchas amarelas.}{ji.boi.a}{0}
\verb{jiboiar}{}{Fig.}{}{}{v.t.}{Digerir uma refeição farta em repouso.}{ji.boi.ar}{\verboinum{1}}
\verb{jiçara}{}{}{}{}{}{}{ji.ça.ra}{0}
\verb{jiló}{}{}{}{}{s.m.}{Fruto do jiloeiro, de sabor amargo e propriedades tônicas.}{ji.ló}{0}
\verb{jiloeiro}{ê}{Bot.}{}{}{s.m.}{Planta herbácea de ramos tomentosos, flores brancas e fruto comestível de sabor amargo e propriedades tônicas.}{ji.lo.ei.ro}{0}
\verb{jingle}{}{}{}{}{s.m.}{Tema musical que consiste em estribilho simples e de curta duração, composto para mensagem publicitária veiculada em rádio ou televisão.}{\textit{jingle}}{0}
\verb{jinjibirra}{}{}{}{}{s.f.}{Bebida fermentada feita de gengibre, frutos, açúcar, fermento de pão e água.}{jin.ji.bir.ra}{0}
\verb{jinriquixá}{ch}{}{}{}{s.m.}{Veículo pequeno e leve, para um passageiro, com duas rodas, puxado por um homem a pé, utilizado em vários locais do Oriente.}{jin.ri.qui.xá}{0}
\verb{jipe}{}{}{}{}{s.m.}{Automóvel de tração nas quatro rodas, de grande facilidade de manobras em terrenos acidentados.}{ji.pe}{0}
\verb{jirau}{}{}{}{}{s.m.}{Armação feita de varas e troncos, semelhante a um estrado, usada para dormir no mato ou esperar a caça.}{ji.rau}{0}
\verb{jirau}{}{}{}{}{}{Cama de varas.}{ji.rau}{0}
\verb{jirau}{}{}{}{}{}{Estrado ou palanque que serve de assento aos passageiros de uma jangada.}{ji.rau}{0}
\verb{jiu"-jítsu}{}{Esport.}{}{}{s.m.}{Sistema de luta corporal, de origem japonesa, que combina golpes de destreza e defesa sem armas.}{jiu"-jít.su}{0}
\verb{joalharia}{}{}{}{}{}{}{jo.a.lha.ri.a}{0}
\verb{joalheiro}{ê}{}{}{}{s.m.}{Indivíduo que fabrica ou vende joias.}{jo.a.lhei.ro}{0}
\verb{joalheiro}{ê}{}{}{}{}{Engastador de pedras preciosas.}{jo.a.lhei.ro}{0}
\verb{joalheria}{}{}{}{}{s.f.}{Arte ou técnica de trabalhar pedras preciosas ou metais para fabricação de joias.}{jo.a.lhe.ri.a}{0}
\verb{joalheria}{}{}{}{}{}{Estabelecimento onde se vendem joias.}{jo.a.lhe.ri.a}{0}
\verb{joanete}{ê/ ou /é}{}{}{}{s.m.}{Saliência crônica na articulação do primeiro osso do dedo do pé, causada por inflamação crônica da bolsa membranosa.}{jo.a.ne.te}{0}
\verb{joaninha}{}{Zool.}{}{}{s.f.}{Inseto da ordem dos besouros, de pequeno tamanho, de corpo oval e asas coloridas com pintas brancas ou pretas.}{jo.a.ni.nha}{0}
\verb{joanino}{}{}{}{}{adj.}{Relativo aos reis portugueses de nome João, à época em que reinaram ou às realizações de seus reinados.}{jo.a.ni.no}{0}
\verb{joanino}{}{}{}{}{}{Relativo aos festejos de São João.}{jo.a.ni.no}{0}
\verb{joão"-de"-barro}{}{Zool.}{joões"-de"-barro}{}{s.m.}{Nome comum dado a alguns tipos de pássaros, de plumagem que varia do vermelho ao branco, que constroem seus ninhos de barro em forma de forno.}{jo.ão"-de"-bar.ro}{0}
\verb{joão"-ninguém}{}{}{joões"-ninguém}{}{s.m.}{Indivíduo sem importância, sem peso social, insignificante.}{jo.ão"-nin.guém}{0}
\verb{joão"-pestana}{}{Pop.}{joões"-pestanas \textit{ou}  joão"-pestanas}{}{s.m.}{Vontade de dormir, sono.}{jo.ão"-pes.ta.na}{0}
\verb{joça}{ó}{}{}{}{s.f.}{Coisa qualquer, sem importância, insignificante.}{jo.ça}{0}
\verb{jocoso}{ô}{}{"-osos ⟨ó⟩}{"-osa ⟨ó⟩}{adj.}{Que provoca riso; alegre, cômico, chistoso.}{jo.co.so}{0}
\verb{joeira}{ê}{}{}{}{s.f.}{Peneira para separar o trigo do joio e de outras sementes.}{jo.ei.ra}{0}
\verb{joeirar}{}{}{}{}{v.t.}{Passar o trigo pela joeira para separá"-lo do joio e de outras sementes.}{jo.ei.rar}{0}
\verb{joeirar}{}{}{}{}{}{Separar, examinar, averiguar.}{jo.ei.rar}{\verboinum{1}}
\verb{joeireiro}{ê}{}{}{}{adj.}{Que peneira, que separa o trigo do joio.}{jo.ei.rei.ro}{0}
\verb{joelhada}{}{}{}{}{s.f.}{Pancada com o joelho.}{jo.e.lha.da}{0}
\verb{joelhada}{}{}{}{}{}{Na capoeira, golpe em que o jogador se aproveita de um descuido do adversário e o atinge com o joelho.}{jo.e.lha.da}{0}
\verb{joelheira}{ê}{}{}{}{s.f.}{Peça de malha acolchoada ou curativo para proteger o joelho do atleta na prática de esportes.}{jo.e.lhei.ra}{0}
\verb{joelheira}{ê}{}{}{}{}{Parte da bota de montar ou da armadura que protege o joelho.}{jo.e.lhei.ra}{0}
\verb{joelho}{ê}{Anat.}{}{}{s.m.}{Articulação que liga a coxa com a perna, compreendendo o fêmur, a tíbia e a rótula.}{jo.e.lho}{0}
\verb{joelho}{ê}{Anat.}{}{}{}{Parte exterior da perna que corresponde à articulação.}{jo.e.lho}{0}
\verb{jogada}{}{}{}{}{s.f.}{Ato ou efeito de jogar, lançar.}{jo.ga.da}{0}
\verb{jogada}{}{}{}{}{}{Movimento ou manobra feita em um jogo; lance.}{jo.ga.da}{0}
\verb{jogada}{}{}{}{}{}{Esquema de negócio armado com o intuito de se obter lucro.}{jo.ga.da}{0}
\verb{jogado}{}{}{}{}{adj.}{Que se jogou; arremessado, lançado.}{jo.ga.do}{0}
\verb{jogado}{}{}{}{}{}{Que está caído, prostrado, inerte.}{jo.ga.do}{0}
\verb{jogado}{}{}{}{}{}{Sem assistência; desamparado, abandonado.}{jo.ga.do}{0}
\verb{jogador}{ô}{}{}{}{adj.}{Diz"-se da pessoa que joga, que pratica esportes em equipe.}{jo.ga.dor}{0}
\verb{jogador}{ô}{}{}{}{}{Que joga por hábito ou vício.}{jo.ga.dor}{0}
\verb{jogar}{}{}{}{}{v.t.}{Disputar simuladamente contra um adversário, verdadeiro ou fictício, valendo"-se de habilidades pessoais ou apenas contando com a sorte.}{jo.gar}{0}
\verb{jogar}{}{}{}{}{}{Atirar um objeto com as mãos.}{jo.gar}{0}
\verb{jogar}{}{}{}{}{v.i.}{Balanço que ocorre devido à folga no encaixe de uma peça.}{jo.gar}{0}
\verb{jogar}{}{Desus.}{}{}{}{Diz"-se especialmente de armas de artilharia; utilizar, pôr em funcionamento, valer"-se.}{jo.gar}{\verboinum{5}}
\verb{jogatina}{}{}{}{}{s.f.}{Hábito ou vício de jogo, principalmente o de azar.}{jo.ga.ti.na}{0}
\verb{jogging}{}{}{}{}{s.m.}{Ação de correr ou andar em ritmo moderado e ao ar livre sem fins competitivos.}{\textit{jogging}}{0}
\verb{jogging}{}{}{}{}{}{Vestimenta esportiva composta de calça e blusão de malha ou moletom usado para andar ou correr.}{\textit{jogging}}{0}
\verb{jogo}{ô}{}{"-s ⟨ó⟩}{}{s.m.}{Ação ou efeito de jogar. }{jo.go}{0}
\verb{jogo}{ô}{}{"-s ⟨ó⟩}{}{}{Balanço; oscilação.}{jo.go}{0}
\verb{jogo}{ô}{}{"-s ⟨ó⟩}{}{}{Passatempo; brinquedo; divertimento; esporte. (\textit{Ele se diverte muito com esse jogo.})}{jo.go}{0}
\verb{jogo}{ô}{}{"-s ⟨ó⟩}{}{}{Vício de jogar.}{jo.go}{0}
\verb{jogo}{ô}{}{"-s ⟨ó⟩}{}{}{Série de coisas que constituem um todo. (\textit{Eu comprei um jogo de panelas.})}{jo.go}{0}
\verb{jogral}{}{Mús.}{"-ais}{}{s.m.}{Coro de várias vozes entremeado de declamações de textos literários que anima solenidades sociais ou representações cênicas.}{jo.gral}{0}
\verb{jogral}{}{}{"-ais}{}{}{Na Idade Média, músico que ganhava a vida tocando em festas populares, fazendo mágicas ou recitando poesias.}{jo.gral}{0}
\verb{jogralesco}{ê}{}{}{}{adj.}{Relativo a jogral.}{jo.gra.les.co}{0}
\verb{joguete}{ê}{}{}{}{s.m.}{Pessoa ou coisa que é motivo de zombaria, de brincadeiras.}{jo.gue.te}{0}
\verb{joguete}{ê}{}{}{}{}{Indivíduo submetido às forças do destino, que não tem controle sobre sua vida.}{jo.gue.te}{0}
\verb{joia}{ó}{Fig.}{}{}{adj.}{Muito bom, excelente.}{joi.a}{0}
\verb{joia}{ó}{}{}{}{s.f.}{Objeto de adorno feito com pedras e metais preciosos.}{joi.a}{0}
\verb{joia}{ó}{Fig.}{}{}{}{Pessoa ou objeto de grande valor ou muito bom.}{joi.a}{0}
\verb{joia}{ó}{}{}{}{}{Quantia paga no ato de admissão ao quadro de sócios de certa associação ou agremiação.}{joi.a}{0}
\verb{joint venture}{}{}{}{}{s.f.}{Associação de sociedades ou empresas, não definitiva, para realizar determinados negócios, sem que nenhuma perca sua personalidade jurídica.}{\textit{joint venture}}{0}
\verb{joio}{ô}{Bot.}{}{}{s.m.}{Planta gramínea disposta em espigas alongadas que nasce junto a plantações de trigo.}{joi.o}{0}
\verb{joio}{ô}{Fig.}{}{}{}{Qualquer coisa ruim ou prejudicial que se mistura às boas, corrompendo"-as.}{joi.o}{0}
\verb{jojoba}{ó}{Bot.}{}{}{s.f.}{Planta arbustiva cuja semente fornece um tipo de óleo rico em proteína utilizado na fabricação de lubrificantes e cosméticos.}{jo.jo.ba}{0}
\verb{joldra}{ô}{}{}{}{}{Var. de \textit{choldra}.}{jol.dra}{0}
\verb{jongo}{}{Mús.}{}{}{s.m.}{Dança de roda de origem africana, com acompanhamento de tambores, semelhante ao samba; caxambu.}{jon.go}{0}
\verb{jônico}{}{}{}{}{adj.}{Relativo à Jônia, antiga colônia grega, ou aos jônios, seus habitantes.}{jô.ni.co}{0}
\verb{jônico}{}{}{}{}{}{Relativo a uma das ordens arquitetônicas clássicas, caracterizada pelas colunas com capitéis ornamentados com espirais.}{jô.ni.co}{0}
\verb{jônio}{}{}{}{}{adj.}{Relativo à Jônia, antiga colônia grega.}{jô.nio}{0}
\verb{jônio}{}{}{}{}{s.m.}{Indivíduo natural ou habitante dessa colônia.}{jô.nio}{0}
\verb{jóquei}{}{}{}{}{s.m.}{Cavaleiro profissional que participa de corridas de competição; ginete.}{jó.quei}{0}
\verb{jordaniano}{}{}{}{}{adj.}{Relativo à Jordânia (Ásia).}{jor.da.ni.a.no}{0}
\verb{jordaniano}{}{}{}{}{s.m.}{Indivíduo natural ou habitante desse país.}{jor.da.ni.a.no}{0}
\verb{jornada}{}{}{}{}{s.f.}{Marcha, caminhada ou percurso feito no período de um dia.}{jor.na.da}{0}
\verb{jornada}{}{Por ext.}{}{}{}{Duração do trabalho diário.}{jor.na.da}{0}
\verb{jornada}{}{}{}{}{}{Expedição militar, batalha.}{jor.na.da}{0}
\verb{jornadear}{}{}{}{}{v.i.}{Fazer jornada; ir de um ponto a outro; caminhar.}{jor.na.de.ar}{\verboinum{4}}
\verb{jornal}{}{}{"-ais}{}{s.m.}{Publicação diária com notícias do cenário político, econômico, cultural, policial etc.}{jor.nal}{0}
\verb{jornal}{}{Por ext.}{"-ais}{}{}{Noticiário transmitido pelo rádio, televisão ou cinema.}{jor.nal}{0}
\verb{jornaleco}{é}{}{}{}{s.m.}{Jornal de pouca importância, mal redigido.}{jor.na.le.co}{0}
\verb{jornaleiro}{ê}{}{}{}{s.m.}{Vendedor ou entregador de jornais.}{jor.na.lei.ro}{0}
\verb{jornaleiro}{ê}{Desus.}{}{}{}{Trabalhador a que se paga pelo trabalho diário; diarista.}{jor.na.lei.ro}{0}
\verb{jornalismo}{}{}{}{}{s.m.}{Conjunto das atividades relativas à redação de um jornal ou de qualquer outro órgão de imprensa.}{jor.na.lis.mo}{0}
\verb{jornalismo}{}{}{}{}{}{Profissão do jornalista.}{jor.na.lis.mo}{0}
\verb{jornalista}{}{}{}{}{s.2g.}{Indivíduo que trabalha em órgão de imprensa como redator, repórter, colunista ou diretor.}{jor.na.lis.ta}{0}
\verb{jornalístico}{}{}{}{}{adj.}{Relativo a jornal, jornalista ou jornalismo.}{jor.na.lís.ti.co}{0}
\verb{jorrar}{}{}{}{}{v.t.}{Sair em jato forte; brotar com ímpeto; emanar.}{jor.rar}{0}
\verb{jorrar}{}{}{}{}{}{Expelir, lançar, irromper.}{jor.rar}{0}
\verb{jorrar}{}{}{}{}{}{Formar barriga ou saliência convexa.}{jor.rar}{\verboinum{1}}
\verb{jorro}{ô}{}{}{}{s.m.}{Ato ou efeito de jorrar.}{jor.ro}{0}
\verb{jorro}{ô}{}{}{}{}{Jato forte; saída impetuosa de um líquido; golfada.}{jor.ro}{0}
\verb{jota}{ó}{}{}{}{s.m.}{Nome da letra \textit{j}.}{jo.ta}{0}
\verb{joule}{}{Fís.}{}{}{s.m.}{Unidade de medida de energia no Sistema Internacional; a energia transportada por segundo em um condutor percorrido por uma corrente elétrica invariável de um \textit{ampère}, sob uma diferença de potencial constante igual a um volt. Símb.: J.  }{jou.le}{0}
\verb{jovem}{ó}{}{"-ens ⟨ó⟩}{}{adj.2g.}{De pouca idade; novo, moço.}{jo.vem}{0}
\verb{jovem}{ó}{}{"-ens ⟨ó⟩}{}{}{Que possui o vigor, a energia e a flexibilidade próprios da juventude.}{jo.vem}{0}
\verb{jovem}{ó}{}{"-ens ⟨ó⟩}{}{}{Que ainda não tem maturidade; ingênuo.}{jo.vem}{0}
\verb{jovial}{}{}{"-ais}{}{adj.2g.}{Que revela alegria; contente, expansivo.}{jo.vi.al}{0}
\verb{jovial}{}{}{"-ais}{}{}{Que tem graça; divertido, espirituoso.}{jo.vi.al}{0}
\verb{jovialidade}{}{}{}{}{s.f.}{Qualidade de jovial; alegria, bom humor.}{jo.vi.a.li.da.de}{0}
\verb{jovialidade}{}{}{}{}{}{Dito alegre, divertido, espirituoso.}{jo.vi.a.li.da.de}{0}
\verb{joystick}{}{}{}{}{s.m.}{Dispositivo de entrada, utilizado em jogos de computador ou vídeo, dotado de alavancas e botões, usado para comandar certas ações e interagir com cenas que aparecem na tela da televisão ou do computador.}{\textit{joystick}}{0}
\verb{juá}{}{}{}{}{s.m.}{O fruto do juazeiro, de coloração amarela ou vermelha e de formato arredondado.}{ju.á}{0}
\verb{juazeiro}{ê}{Bot.}{}{}{s.m.}{Árvore frutífera alta e copada, nativa do sertão nordestino, cuja casca é amarga, com propriedades medicinais adstringentes.}{ju.a.zei.ro}{0}
\verb{juba}{}{}{}{}{s.f.}{Pelo flexível e resistente do alto da cabeça do leão; crina.}{ju.ba}{0}
\verb{juba}{}{Pop.}{}{}{}{Cabeleira abundante e despenteada.}{ju.ba}{0}
\verb{jubilação}{}{}{"-ões}{}{s.f.}{Ato ou efeito de jubilar.}{ju.bi.la.ção}{0}
\verb{jubilação}{}{}{"-ões}{}{}{Aposentadoria honrosa no magistério, após certo tempo de atividade.}{ju.bi.la.ção}{0}
\verb{jubilação}{}{}{"-ões}{}{}{Afastamento de um aluno de uma instituição de ensino por motivo de reprovações, faltas etc.}{ju.bi.la.ção}{0}
\verb{jubilado}{}{}{}{}{adj.}{Diz"-se do professor que recebeu aposentadoria honrosa.}{ju.bi.la.do}{0}
\verb{jubilado}{}{}{}{}{}{Diz"-se do aluno que foi afastado de instituição de ensino.}{ju.bi.la.do}{0}
\verb{jubilar}{}{}{}{}{v.t.}{Encher de júbilo, contentamento; regozijar.}{ju.bi.lar}{0}
\verb{jubilar}{}{}{}{}{adj.2g.}{Relativo a jubileu, a solenidade celebrada a cada cinquenta anos.}{ju.bi.lar}{0}
\verb{jubilar}{}{}{}{}{}{Conceder aposentadoria honrosa a professor.}{ju.bi.lar}{0}
\verb{jubilar}{}{}{}{}{}{Impor o afastamento de aluno de uma instituição de ensino.}{ju.bi.lar}{\verboinum{1}}
\verb{jubileu}{}{}{}{}{s.m.}{Cinquentenário de um casamento, do exercício de um cargo, da existência de uma instituição etc.}{ju.bi.leu}{0}
\verb{jubileu}{}{Relig.}{}{}{}{Na Igreja Católica, indulgência coletiva que o papa concede em certas ocasiões importantes.}{ju.bi.leu}{0}
\verb{jubileu}{}{}{}{}{}{Solenidade em que é concedida essa indulgência.}{ju.bi.leu}{0}
\verb{júbilo}{}{}{}{}{s.m.}{Grande contentamento ou alegria; exultação, regozijo.}{jú.bi.lo}{0}
\verb{jubiloso}{ô}{}{"-osos ⟨ó⟩}{"-osa ⟨ó⟩}{adj.}{Tomado de intensa alegria; exultante.}{ju.bi.lo.so}{0}
\verb{jucá}{}{Bot.}{}{}{s.m.}{Árvore de tronco liso, madeira resistente e muito dura e flores amarelas e vistosas; pau"-ferro.}{ju.cá}{0}
\verb{juçara}{}{Bot.}{}{}{s.f.}{Palmeira alta e delgada, cujas folhas, longas e segmentadas, são usadas como forragem, e de cujo caule é extraído o palmito; açaizeiro.}{ju.ça.ra}{0}
\verb{jucundidade}{}{}{}{}{s.f.}{Qualidade de jucundo; alegria, prazer.}{ju.cun.di.da.de}{0}
\verb{jucundo}{}{}{}{}{adj.}{Que denota alegria; agradável, prazenteiro.}{ju.cun.do}{0}
\verb{judaico}{}{}{}{}{adj.}{Relativo aos judeus ou ao judaísmo.}{ju.dai.co}{0}
\verb{judaísmo}{}{Relig.}{}{}{s.m.}{Uma das três principais religiões de tradição escrita; baseia"-se na torá, a lei mosaica.}{ju.da.ís.mo}{0}
\verb{judaísmo}{}{}{}{}{}{Conjunto da civilização e cultura judaicas.}{ju.da.ís.mo}{0}
\verb{judaizante}{}{}{}{}{adj.2g.}{Que observa as leis e os ritos do judaísmo.}{ju.da.i.zan.te}{0}
\verb{judaizar}{}{}{}{}{v.i.}{Observar as leis e os ritos judaicos.}{ju.da.i.zar}{0}
\verb{judaizar}{}{}{}{}{v.t.}{Converter ao judaísmo.}{ju.da.i.zar}{\verboinum{1}}
\verb{judas}{}{Relig.}{}{}{s.m.}{Nome do apóstolo Judas Iscariotes, que traiu Jesus Cristo, entregando"-o aos fariseus.}{ju.das}{0}
\verb{judas}{}{Por ext.}{}{}{}{Amigo falso; traidor.}{ju.das}{0}
\verb{judas}{}{}{}{}{}{Boneco feito à semelhança do apóstolo, que é surrado e queimado no sábado de Aleluia.}{ju.das}{0}
\verb{judeu}{}{}{}{judia}{s.m.}{Indivíduo que segue a religião judaica; israelita.}{ju.deu}{0}
\verb{judeu}{}{}{}{judia}{}{Indivíduo nascido de mãe judia ou de pai e mãe judeus.}{ju.deu}{0}
\verb{judeu}{}{}{}{judia}{adj.}{Relativo à antiga Judeia, região meridional da Palestina, ou a seus habitantes.}{ju.deu}{0}
\verb{judeu}{}{}{}{judia}{}{Relativo à antiga tribo de Judá ou a seus descendentes.}{ju.deu}{0}
\verb{judia}{}{}{}{}{s.f.}{Feminino de \textit{judeu}.}{ju.di.a}{0}
\verb{judiação}{}{}{"-ões}{}{s.f.}{Ato ou efeito de judiar; maus"-tratos, escárnio; judiaria.}{ju.di.a.ção}{0}
\verb{judiar}{}{}{}{}{v.t.}{Tratar com escárnio ou zombaria; maltratar, atormentar.}{ju.di.ar}{\verboinum{1}}
\verb{judiaria}{}{}{}{}{}{Antigo bairro dos judeus; gueto.}{ju.di.a.ri.a}{0}
\verb{judiaria}{}{}{}{}{s.f.}{Judiação.}{ju.di.a.ri.a}{0}
\verb{judicante}{}{}{}{}{adj.2g.}{Que julga.}{ju.di.can.te}{0}
\verb{judicante}{}{}{}{}{}{Que exerce as funções de juiz.}{ju.di.can.te}{0}
\verb{judicativo}{}{}{}{}{adj.}{Qualidade do que julga.}{ju.di.ca.ti.vo}{0}
\verb{judicatório}{}{}{}{}{adj.}{Relativo a julgamento.}{ju.di.ca.tó.rio}{0}
\verb{judicatório}{}{}{}{}{}{Próprio para julgar.}{ju.di.ca.tó.rio}{0}
\verb{judicatura}{}{}{}{}{s.f.}{Poder de julgar.}{ju.di.ca.tu.ra}{0}
\verb{judicatura}{}{}{}{}{}{Cargo ou dignidade de juiz; magistratura.}{ju.di.ca.tu.ra}{0}
\verb{judicatura}{}{}{}{}{}{O poder judiciário de um estado.}{ju.di.ca.tu.ra}{0}
\verb{judicial}{}{}{"-ais}{}{adj.2g.}{Relativo a juiz, a tribunais ou à justiça.}{ju.di.ci.al}{0}
\verb{judiciário}{}{}{}{}{adj.}{Relativo à organização da justiça ou a juiz.}{ju.di.ci.á.rio}{0}
\verb{judiciário}{}{}{}{}{s.m.}{O poder judiciário, um dos três poderes do estado, ao qual cabe zelar pelo respeito à Constituição e aplicar a justiça.}{ju.di.ci.á.rio}{0}
\verb{judicioso}{ô}{}{"-osos ⟨ó⟩}{"-osa ⟨ó⟩}{adj.}{Que é perspicaz e justo em seus julgamentos.}{ju.di.ci.o.so}{0}
\verb{judicioso}{ô}{}{"-osos ⟨ó⟩}{"-osa ⟨ó⟩}{}{Que revela acerto, juízo; acertado.}{ju.di.ci.o.so}{0}
\verb{judicioso}{ô}{Fig.}{"-osos ⟨ó⟩}{"-osa ⟨ó⟩}{}{Que é sentencioso, crítico.}{ju.di.ci.o.so}{0}
\verb{judô}{}{Esport.}{}{}{s.m.}{Sistema de luta corporal, de combate e defesa, que se baseia na agilidade e flexibilidade do jogador, e constitui uma modalidade do antigo jiu"-jítsu.}{ju.dô}{0}
\verb{judoca}{ó}{Esport.}{}{}{s.2g.}{Indivíduo que pratica o judô.}{ju.do.ca}{0}
\verb{jugo}{}{}{}{}{s.m.}{Peça de madeira assentada sobre a cabeça dos bois para puxar o arado ou o carro.}{ju.go}{0}
\verb{jugo}{}{Fig.}{}{}{}{Opressão, sujeição.}{ju.go}{0}
\verb{jugular}{}{}{}{}{adj.2g.}{Relativo à garganta ou ao pescoço.}{ju.gu.lar}{0}
\verb{jugular}{}{Anat.}{}{}{s.f.}{Cada uma das quatro veias grossas, localizadas em cada face lateral do pescoço.}{ju.gu.lar}{0}
\verb{jugular}{}{}{}{}{v.t.}{Impedir o desenvolvimento ou as manifestações de algo; extinguir, combater.}{ju.gu.lar}{0}
\verb{jugular}{}{}{}{}{}{Decapitar, degolar.}{ju.gu.lar}{\verboinum{1}}
\verb{juiz}{}{}{}{}{s.m.}{Indivíduo que, investido de autoridade pública, tem poder para julgar, na qualidade de administrador da Justiça do Estado.}{ju.iz}{0}
\verb{juiz}{}{}{}{}{}{Membro de um júri.}{ju.iz}{0}
\verb{juiz}{}{}{}{}{}{Árbitro, julgador.}{ju.iz}{0}
\verb{juíza}{}{}{}{}{s.f.}{Mulher que exerce as funções de juiz.}{ju.í.za}{0}
\verb{juíza}{}{}{}{}{}{Mulher que dirige certas festividades de igreja.}{ju.í.za}{0}
\verb{juizado}{}{}{}{}{s.m.}{Cargo ou função de juiz.}{ju.i.za.do}{0}
\verb{juizado}{}{}{}{}{}{Repartição onde o juiz exerce suas funções.}{ju.i.za.do}{0}
\verb{juízo}{}{}{}{}{s.m.}{Ação ou efeito de julgar; julgamento. (\textit{O juízo foi de que o réu era culpado.})}{ju.í.zo}{0}
\verb{juízo}{}{}{}{}{}{Tribunal; foro. (\textit{Os criminosos terão de responder em juízo.})}{ju.í.zo}{0}
\verb{juízo}{}{}{}{}{}{Opinião; parecer. (\textit{O juízo de todos era de que não deveríamos sair tão cedo.})}{ju.í.zo}{0}
\verb{juízo}{}{}{}{}{}{Tino; sino; bom"-senso. (\textit{As pessoas têm de ter muito juízo para não cometerem faltas.})}{ju.í.zo}{0}
\verb{jujuba}{}{}{}{}{s.f.}{Árvore originária da Índia, semelhante ao juazeiro, porém menor.}{ju.ju.ba}{0}
\verb{jujuba}{}{}{}{}{}{O fruto dessa árvore.}{ju.ju.ba}{0}
\verb{jujuba}{}{}{}{}{}{O suco ou a massa desse fruto.}{ju.ju.ba}{0}
\verb{jujuba}{}{}{}{}{}{A bala feita de jujuba; bala de goma.}{ju.ju.ba}{0}
\verb{julgado}{}{}{}{}{adj.}{Que foi objeto de julgamento.}{jul.ga.do}{0}
\verb{julgado}{}{Jur.}{}{}{s.m.}{Matéria decidida em sentença ou acórdão.}{jul.ga.do}{0}
\verb{julgador}{ô}{}{}{}{adj.}{Que julga; que avalia, sentencia, decide.}{jul.ga.dor}{0}
\verb{julgador}{ô}{}{}{}{s.m.}{Indivíduo que julga; juiz, árbitro.}{jul.ga.dor}{0}
\verb{julgamento}{}{}{}{}{s.m.}{Ação ou efeito de julgar.}{jul.ga.men.to}{0}
\verb{julgamento}{}{}{}{}{}{Decisão; sentença. (\textit{O julgamento do juiz foi de que o criminoso deveria ir para a cadeia.})}{jul.ga.men.to}{0}
\verb{julgar}{}{}{}{}{v.t.}{Decidir como juiz ou árbitro.}{jul.gar}{0}
\verb{julgar}{}{}{}{}{}{Pronunciar uma sentença; sentenciar.}{jul.gar}{0}
\verb{julgar}{}{}{}{}{}{Supor, imaginar, conjeturar.}{jul.gar}{0}
\verb{julgar}{}{}{}{}{}{Formar opinião; avaliar.}{jul.gar}{0}
\verb{julgar}{}{}{}{}{}{Dar, adjudicar.}{jul.gar}{0}
\verb{julgar}{}{}{}{}{}{Reputar, considerar.}{jul.gar}{\verboinum{5}}
\verb{julho}{}{}{}{}{s.m.}{O sétimo mês do ano civil.}{ju.lho}{0}
\verb{juliano}{}{}{}{}{adj.}{Relativo à reforma cronológica ordenada por Júlio César, estadista e general romano.}{ju.li.a.no}{0}
\verb{jumento}{}{Zool.}{}{}{s.m.}{Mamífero semelhante ao cavalo, mas de menor tamanho e orelhas mais longas, utilizado como animal de tração e carga.}{ju.men.to}{0}
\verb{jumento}{}{Pop.}{}{}{}{Indivíduo com pouca inteligência ou grosseiro.}{ju.men.to}{0}
\verb{junção}{}{}{"-ões}{}{s.f.}{Ação ou efeito de juntar. (\textit{Eu vou fazer a junção das cordas com um nó.})}{jun.ção}{0}
\verb{junção}{}{}{"-ões}{}{}{Ponto em que duas coisas se juntam. (\textit{O joelho fica na junção da coxa com a canela.})}{jun.ção}{0}
\verb{juncar}{}{}{}{}{v.t.}{Cobrir de juncos.}{jun.car}{0}
\verb{juncar}{}{}{}{}{}{Cobrir de flores, de folhas, de ramos.}{jun.car}{0}
\verb{juncar}{}{}{}{}{}{Espalhar em quantidade; encher, cobrir.}{jun.car}{\verboinum{1}}
\verb{junco}{}{Bot.}{}{}{s.m.}{Ervas de haste longa, desprovida de folhas, que habitam lugares úmidos.}{jun.co}{0}
\verb{junco}{}{}{}{}{}{Pequeno navio, de fundo chato, com dois ou três mastros e velas, utilizado para transporte ou para pesca no Oriente.}{jun.co}{0}
\verb{jungir}{}{}{}{}{v.t.}{Juntar, unir, ligar.}{jun.gir}{0}
\verb{jungir}{}{}{}{}{}{Emparelhar animais por meio do jugo.}{jun.gir}{0}
\verb{jungir}{}{}{}{}{}{Submeter através da força; sujeitar.}{jun.gir}{\verboinum{34}}
\verb{junho}{}{}{}{}{s.m.}{O sexto mês do ano civil.}{ju.nho}{0}
\verb{junino}{}{}{}{}{adj.}{Relativo ao mês de junho, ou que se realiza nesse mês.}{ju.ni.no}{0}
\verb{júnior}{}{}{"-ores ⟨ô⟩}{}{adj.}{Mais moço, mais jovem.}{jú.nior}{0}
\verb{júnior}{}{}{"-ores ⟨ô⟩}{}{}{Qualificativo anexado ao nome do filho homônimo do pai. (\textit{Esse é o meu marido, Antônio, e esse é nosso filho, Antônio, o júnior.})}{jú.nior}{0}
\verb{júnior}{}{}{"-ores ⟨ô⟩}{}{s.m.}{Atleta concorrente de menos idade. (\textit{Ele compete como atleta júnior no futebol.})}{jú.nior}{0}
\verb{junípero}{}{Bot.}{}{}{s.m.}{Planta nativa do hemisfério norte cujos frutos são usados na preparação do gim e da genebra e na aromatização de defumados e conservas; zimbro.}{ju.ní.pe.ro}{0}
\verb{junquilho}{}{Bot.}{}{}{s.m.}{Erva ornamental, nativa da Europa, de flores douradas e perfumadas e folhas estreitas.}{jun.qui.lho}{0}
\verb{junquilho}{}{Por ext.}{}{}{}{A flor dessa planta.}{jun.qui.lho}{0}
\verb{junta}{}{}{}{}{s.f.}{Ponto de junção e reunião.}{jun.ta}{0}
\verb{junta}{}{Anat.}{}{}{}{O conjunto das superfícies e dos ligamentos que articulam dois ossos entre si; articulação, juntura.}{jun.ta}{0}
\verb{junta}{}{}{}{}{}{Reunião de pessoas convocadas para determinado fim.}{jun.ta}{0}
\verb{junta}{}{}{}{}{}{Conferência de médicos junto a um enfermo.}{jun.ta}{0}
\verb{junta}{}{}{}{}{}{Intervalo entre dois tijolos, ou duas pedras justapostas, em uma alvenaria.}{jun.ta}{0}
\verb{juntada}{}{Jur.}{}{}{s.f.}{Ato judicial pelo qual são anexados ao processo petições, laudos, provas ou qualquer outra peça processual.}{jun.ta.da}{0}
\verb{juntar}{}{}{}{}{v.t.}{Ajuntar. (\textit{Vou juntar as peças quebradas para ver se consigo colá"-las.})}{jun.tar}{0}
\verb{juntar}{}{}{}{}{}{Pôr junto; aproximar. (\textit{Vamos juntar os brinquedos para guardá"-los.})}{jun.tar}{0}
\verb{juntar}{}{}{}{}{v.pron.}{Associar"-se; unir"-se. (\textit{Os dois amigos se juntaram para criar uma empresa.})}{jun.tar}{\verboinum{1}}
\verb{junto}{}{}{}{}{adj.}{Que se encontra unido, anexo.}{jun.to}{0}
\verb{junto}{}{}{}{}{}{Que está próximo, contíguo, chegado.}{jun.to}{0}
\verb{junto}{}{}{}{}{adv.}{Juntamente.}{jun.to}{0}
\verb{junto}{}{}{}{}{}{Ao lado, perto.}{jun.to}{0}
\verb{juntura}{}{}{}{}{s.f.}{Junção, reunião.}{jun.tu.ra}{0}
\verb{juntura}{}{}{}{}{}{Articulação, junta.}{jun.tu.ra}{0}
\verb{juntura}{}{}{}{}{}{Ponto onde duas peças se juntam.}{jun.tu.ra}{0}
\verb{Júpiter}{}{Astron.}{jupíteres}{}{s.m.}{O maior planeta do Sistema Solar.}{Jú.pi.ter}{0}
\verb{Júpiter}{}{Mit.}{jupíteres}{}{}{O pai dos deuses entre os romanos, correspondente ao Zeus dos gregos.}{Jú.pi.ter}{0}
\verb{jupiteriano}{}{Fig.}{}{}{}{Que tem caráter dominador.}{ju.pi.te.ri.a.no}{0}
\verb{jupiteriano}{}{}{}{}{adj.}{Relativo ao planeta Júpiter.}{ju.pi.te.ri.a.no}{0}
\verb{juquiri}{}{Bot.}{}{}{s.m.}{Árvore de flores pequenas, madeira dura e escura.}{ju.qui.ri}{0}
\verb{jura}{}{}{}{}{s.f.}{Ato de jurar; juramento.}{ju.ra}{0}
\verb{jura}{}{}{}{}{}{Ato de amaldiçoar; praga.}{ju.ra}{0}
\verb{jura}{}{Pop.}{}{}{}{Aguardente de cana; cachaça.}{ju.ra}{0}
\verb{jurado}{}{}{}{}{adj.}{Solenemente declarado.}{ju.ra.do}{0}
\verb{jurado}{}{}{}{}{}{Protestado com juramento.}{ju.ra.do}{0}
\verb{jurado}{}{}{}{}{}{Que foi ameaçado de agressão ou de morte.}{ju.ra.do}{0}
\verb{jurado}{}{}{}{}{s.m.}{Cada um dos cidadãos que compõem o tribunal do júri.}{ju.ra.do}{0}
\verb{juramentado}{}{}{}{}{adj.}{Que prestou juramento.}{ju.ra.men.ta.do}{0}
\verb{juramentar}{}{}{}{}{v.t.}{Fazer juramento.}{ju.ra.men.tar}{0}
\verb{juramentar}{}{}{}{}{v.pron.}{Prometer sob juramento; obrigar"-se por juramento}{ju.ra.men.tar}{\verboinum{1}}
\verb{juramento}{}{}{}{}{s.m.}{Ato ou efeito de jurar; jura.}{ju.ra.men.to}{0}
\verb{juramento}{}{}{}{}{}{Afirmação ou promessa solene de fidelidade, em que se toma por testemunha uma coisa que se tem como sagrada.}{ju.ra.men.to}{0}
\verb{jurar}{}{}{}{}{v.t.}{Declarar ou prometer solenemente; afirmar sob juramento.}{ju.rar}{0}
\verb{jurar}{}{}{}{}{}{Invocar, chamar.}{ju.rar}{0}
\verb{jurar}{}{}{}{}{}{Afirmar categoricamente; afiançar.}{ju.rar}{0}
\verb{jurar}{}{}{}{}{v.i.}{Prestar juramento.}{ju.rar}{\verboinum{1}}
\verb{jurássico}{}{}{}{}{s.m.}{Período geológico da era mesozoica entre o triásico e o cretácio, com duração aproximada de sessenta milhões de anos, caracterizado pelo aparecimento das aves.}{ju.rás.si.co}{0}
\verb{jurássico}{}{}{}{}{adj.}{Relativo a esse período.}{ju.rás.si.co}{0}
\verb{jurema}{}{Bot.}{}{}{s.f.}{Arbusto de troncos e galhos providos de espinhos, de flores brancas ou esverdeadas, e madeira dura e pouco utilizável.}{ju.re.ma}{0}
\verb{jurema}{}{}{}{}{}{Bebida de propriedades alucinógenas, feita dessa planta.}{ju.re.ma}{0}
\verb{júri}{}{}{}{}{s.m.}{Tribunal judiciário constituído por um juiz de direito, que é o seu presidente, e certo número de cidadãos, que, como jurados, julgam uma causa. }{jú.ri}{0}
\verb{júri}{}{}{}{}{}{Comissão encarregada de examinar ou de avaliar o mérito de pessoas ou coisas.}{jú.ri}{0}
\verb{jurídico}{}{}{}{}{adj.}{Relativo ao direito.}{ju.rí.di.co}{0}
\verb{jurídico}{}{}{}{}{}{Que se faz por via da justiça; conforme aos princípios do direito; lícito, legal.}{ju.rí.di.co}{0}
\verb{jurisconsulto}{}{Jur.}{}{}{s.m.}{Indivíduo de grande conhecimento jurídico e que faz profissão de dar pareceres acerca de questões do direito; jurisperito.}{ju.ris.con.sul.to}{0}
\verb{jurisdição}{}{}{"-ões}{}{s.f.}{Poder atribuído a uma autoridade para aplicar as leis em determinada área.}{ju.ris.di.ção}{0}
\verb{jurisdição}{}{}{"-ões}{}{}{Área dentro da qual se exerce esse poder.}{ju.ris.di.ção}{0}
\verb{jurisperito}{}{}{}{}{s.m.}{Indivíduo versado na ciência do Direito; jurisconsulto.}{ju.ris.pe.ri.to}{0}
\verb{jurisprudência}{}{Jur.}{}{}{s.f.}{Ciência do direito e das leis.}{ju.ris.pru.dên.cia}{0}
\verb{jurisprudência}{}{}{}{}{}{Conjunto de soluções dadas às questões de direito pelos tribunais superiores. }{ju.ris.pru.dên.cia}{0}
\verb{jurisprudente}{}{}{}{}{s.2g.}{Indivíduo versado na ciência do Direito; jurisconsulto, jurisperito.}{ju.ris.pru.den.te}{0}
\verb{jurista}{}{}{}{}{s.2g.}{Jurisconsulto.}{ju.ris.ta}{0}
\verb{jurista}{}{}{}{}{}{Indivíduo que empresta dinheiro a juros.}{ju.ris.ta}{0}
\verb{jurista}{}{}{}{}{}{Indivíduo que possui títulos da dívida pública.}{ju.ris.ta}{0}
\verb{juriti}{}{Zool.}{}{}{s.m.}{Espécie de pomba, de coloração geral parda, com tons avermelhados, que vive em locais quentes, matas e cerrados.}{ju.ri.ti}{0}
\verb{juro}{}{}{}{}{s.m.}{Lucro, calculado sobre determinada taxa, de dinheiro emprestado, ou de capital empregado; rendimento.}{ju.ro}{0}
\verb{jurubeba}{é}{Bot.}{}{}{s.f.}{Planta de folhas moles, flores vistosas e alvas, e frutos que são bagas, muito utilizados em medicina popular e no preparo de aperitivos.}{ju.ru.be.ba}{0}
\verb{jurujuba}{}{Bot.}{}{}{s.f.}{Erva largamente distribuída no país, cultivada como ornamental e de uso medicinal.}{ju.ru.ju.ba}{0}
\verb{jurupari}{}{}{}{}{s.m.}{O demônio entre os índios tupis.}{ju.ru.pa.ri}{0}
\verb{jurupari}{}{Bot.}{}{}{}{Planta que fornece madeira dura, utilizada na confecção de instrumentos musicais africanos.}{ju.ru.pa.ri}{0}
\verb{jururu}{}{}{}{}{adj.2g.}{Diz"-se daquele que está triste, cabisbaixo, melancólico.}{ju.ru.ru}{0}
\verb{juruti}{}{}{}{}{}{Var. de \textit{juriti}.}{ju.ru.ti}{0}
\verb{jus}{}{}{}{}{s.m.}{Prerrogativa legal; direito, merecimento.}{jus}{0}
\verb{jusante}{}{Desus.}{}{}{s.f.}{Refluxo da maré.}{ju.san.te}{0}
\verb{jusante}{}{}{}{}{}{O sentido da correnteza num curso de água.}{ju.san.te}{0}
\verb{justa}{}{}{}{}{s.f.}{Combate entre dois cavaleiros armados de lança, na Idade Média.}{jus.ta}{0}
\verb{justa}{}{Por ext.}{}{}{}{Torneio, luta, combate.}{jus.ta}{0}
\verb{justa}{}{Por ext.}{}{}{}{Questão, pendência.}{jus.ta}{0}
\verb{justa}{}{Pop.}{}{}{}{A polícia.}{jus.ta}{0}
\verb{justafluvial}{}{}{"-ais}{}{adj.2g.}{Que está nas margens de um rio; marginal, ribeirinho.}{jus.ta.flu.vi.al}{0}
\verb{justalinear}{}{}{}{}{adj.2g.}{Diz"-se da tradução em que o original e a versão situam"-se lado a lado, linha a linha.}{jus.ta.li.ne.ar}{0}
\verb{justapor}{}{}{}{}{v.t.}{Pôr junto, em contiguidade.}{jus.ta.por}{\verboinum{60}}
\verb{justaposição}{}{}{"-ões}{}{s.f.}{Ato ou efeito de justapor.}{jus.ta.po.si.ção}{0}
\verb{justaposição}{}{}{"-ões}{}{}{Situação de contiguidade; aposição.}{jus.ta.po.si.ção}{0}
\verb{justaposição}{}{Gram.}{"-ões}{}{}{Formação de palavras compostas por simples junção de uma a outra, conservando a integridade fonética de cada uma.}{jus.ta.po.si.ção}{0}
\verb{justaposto}{ô}{}{"-s ⟨ó⟩}{"-a ⟨ó⟩}{adj.}{Que está junto ou em contiguidade; sobreposto.}{jus.ta.pos.to}{0}
\verb{justar}{}{}{}{}{v.i.}{Participar de justa, lutar ou combater com outra pessoa, a lança e a cavalo, em torneio.}{jus.tar}{0}
\verb{justar}{}{}{}{}{v.t.}{Esgrimir, jogar.}{jus.tar}{0}
\verb{justar}{}{}{}{}{}{Ajustar.}{jus.tar}{0}
\verb{justar}{}{}{}{}{}{Contratar por soldo; assalariar.}{jus.tar}{\verboinum{1}}
\verb{justeza}{ê}{}{}{}{s.f.}{Qualidade daquilo que é justo, conforme à justiça ou à razão.}{jus.te.za}{0}
\verb{justeza}{ê}{}{}{}{}{Absoluta precisão na determinação de medida, valor, peso etc.; exatidão.}{jus.te.za}{0}
\verb{justiça}{}{}{}{}{s.f.}{Conformidade com o direito; a virtude de dar a cada um aquilo que é seu.}{jus.ti.ça}{0}
\verb{justiça}{}{}{}{}{}{A faculdade de julgar com equilíbrio e imparcialidade.}{jus.ti.ça}{0}
\verb{justiça}{}{}{}{}{}{Conjunto de magistrados judiciais e pessoas que servem junto deles.}{jus.ti.ça}{0}
\verb{justiça}{}{}{}{}{}{O pessoal de um tribunal.}{jus.ti.ça}{0}
\verb{justiça}{}{Por ext.}{}{}{}{O poder judiciário.}{jus.ti.ça}{0}
\verb{justiçado}{}{}{}{}{adj.}{Que recebeu como pena um suplício corporal ou a condenação à morte.}{jus.ti.ça.do}{0}
\verb{justiçado}{}{}{}{}{s.m.}{Indivíduo que foi supliciado ou punido com a morte.}{jus.ti.ça.do}{0}
\verb{justiçar}{}{}{}{}{v.t.}{Punir com a morte ou com suplício.}{jus.ti.çar}{\verboinum{3}}
\verb{justiceiro}{ê}{}{}{}{adj.}{Que se arroga o direito de fazer justiça com as próprias mãos.}{jus.ti.cei.ro}{0}
\verb{justiceiro}{ê}{}{}{}{}{Que é rigoroso na aplicação da lei; imparcial, severo.}{jus.ti.cei.ro}{0}
\verb{justificação}{}{}{"-ões}{}{s.f.}{Ato ou efeito de justificar.}{jus.ti.fi.ca.ção}{0}
\verb{justificação}{}{}{"-ões}{}{}{Razão, explicação, fundamento, justificativa.}{jus.ti.fi.ca.ção}{0}
\verb{justificação}{}{}{"-ões}{}{}{Prova judicial de um ato alegado.}{jus.ti.fi.ca.ção}{0}
\verb{justificar}{}{}{}{}{v.t.}{Demonstrar que algo está certo; fornecer argumentos; legitimar.}{jus.ti.fi.car}{0}
\verb{justificar}{}{}{}{}{}{Dar razão; demonstrar.}{jus.ti.fi.car}{0}
\verb{justificar}{}{}{}{}{}{Provar a inocência; isentar de culpa.}{jus.ti.fi.car}{0}
\verb{justificar}{}{}{}{}{}{Provar judicialmente.}{jus.ti.fi.car}{0}
\verb{justificar}{}{}{}{}{}{Dispor um texto de forma que haja alinhamento em ambas as extremidades das linhas.}{jus.ti.fi.car}{0}
\verb{justificar}{}{}{}{}{v.pron.}{Apresentar razões para os próprios atos.}{jus.ti.fi.car}{\verboinum{2}}
\verb{justificativa}{}{}{}{}{s.f.}{Afirmação, fato ou documento que confirma a veracidade de uma proposição ou a justiça de uma ação.}{jus.ti.fi.ca.ti.va}{0}
\verb{justificativo}{}{}{}{}{adj.}{Que serve para justificar.}{jus.ti.fi.ca.ti.vo}{0}
%\verb{justificável}{}{}{}{}{}{0}{jus.ti.fi.cá.vel}{0}
\verb{justo}{}{}{}{}{adj.}{Conforme à lei ou ao direito; legal; legítimo. (\textit{Essa é uma atividade justa no Brasil.})}{jus.to}{0}
\verb{justo}{}{}{}{}{}{Imparcial; equitativo. (\textit{Ele sempre faz uma divisão justa entre os filhos.})}{jus.to}{0}
\verb{justo}{}{}{}{}{}{Exato; preciso. (\textit{As contas deram um resultado justo.})}{jus.to}{0}
\verb{justo}{}{}{}{}{}{Apertado, cingido. (\textit{Roupas muito justas podem causar problemas.})}{jus.to}{0}
\verb{justo}{}{}{}{}{s.m.}{Homem virtuoso, santo. (\textit{Espera"-se que os justos sejam recompensados.})}{jus.to}{0}
\verb{juta}{}{}{}{}{s.f.}{Planta originária da Índia e cultivada na Amazônia e da qual se extrai uma fibra têxtil.}{ju.ta}{0}
\verb{juta}{}{Por ext.}{}{}{}{A fibra extraída dessa planta.}{ju.ta}{0}
\verb{juta}{}{Por ext.}{}{}{}{O tecido feito com essa fibra.}{ju.ta}{0}
\verb{juvenescer}{ê}{}{}{}{v.t.}{Tornar jovem; rejuvenescer.}{ju.ve.nes.cer}{\verboinum{15}}
\verb{juvenil}{}{}{"-is}{}{adj.2g.}{Relativo a juventude.}{ju.ve.nil}{0}
\verb{juvenil}{}{Por ext.}{"-is}{}{}{Jovem.}{ju.ve.nil}{0}
\verb{juvenil}{}{}{"-is}{}{}{Diz"-se de categoria desportiva, equipe, torneio constituído apenas por adolescentes em uma faixa de idade determinada.}{ju.ve.nil}{0}
\verb{juvenilidade}{}{}{}{}{s.f.}{Qualidade de juvenil.}{ju.ve.ni.li.da.de}{0}
\verb{juventude}{}{}{}{}{s.f.}{Período da vida de um ser humano ou de qualquer ser vivo entre o nascimento e o desenvolvimento pleno do organismo.}{ju.ven.tu.de}{0}
\verb{juventude}{}{}{}{}{}{Qualidade do que é jovem ou tem existência recente.}{ju.ven.tu.de}{0}
\verb{juventude}{}{}{}{}{}{O conjunto das pessoas jovens.}{ju.ven.tu.de}{0}
