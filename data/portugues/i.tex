\verb{i}{}{}{}{}{s.m.}{Nona letra e terceira vogal do alfabeto português.}{i}{0}
\verb{I}{}{}{}{}{}{Algarismo romano equivalente a 1.}{I}{0}
\verb{I}{}{Quím.}{}{}{}{Símb. do \textit{iodo}. }{I}{0}
\verb{iaiá}{}{}{}{}{s.f.}{Tratamento dado às meninas e às moças na época da escravidão, e hoje quase abolido.}{ia.iá}{0}
\verb{ialorixá}{ch}{}{}{}{s.f.}{Mulher que dirige as cerimônias da umbanda; mãe"-de"-santo.}{i.a.lo.ri.xá}{0}
\verb{ianque}{}{}{}{}{adj.}{Relativo aos Estados Unidos.}{i.an.que}{0}
\verb{ianque}{}{}{}{}{s.2g.}{Indivíduo natural ou habitante desse país; estadunidense.}{i.an.que}{0}
\verb{ianque}{}{}{}{}{}{Soldado nortista da Guerra de Secessão dos Estados Unidos.}{i.an.que}{0}
\verb{Iansã}{}{Relig.}{}{}{s.f.}{Orixá feminino, mulher de Xangô, relacionada aos ventos e tempestades, que faz sincretismo com Santa Bárbara.  }{I.an.sã}{0}
\verb{iara}{}{Mit.}{}{}{s.f.}{Ser fabuloso que habita o fundo das águas, figurado pelos indígenas na forma de uma mulher de rara beleza, que atrai os homens desprevenidos e os afoga.}{i.a.ra}{0}
\verb{iate}{}{}{}{}{s.m.}{Embarcação a vela ou a motor, usada para recreação ou competições.}{i.a.te}{0}
\verb{iatismo}{}{}{}{}{s.m.}{Técnica de navegar em iate.}{i.a.tis.mo}{0}
\verb{iatismo}{}{Esport.}{}{}{}{Corridas de iate.}{i.a.tis.mo}{0}
\verb{ibérico}{}{}{}{}{adj.}{Relativo à Península Ibérica (Espanha e Portugal).}{i.bé.ri.co}{0}
\verb{ibérico}{}{}{}{}{}{Relativo à Ibéria, antigo nome da Espanha.}{i.bé.ri.co}{0}
\verb{ibérico}{}{}{}{}{s.m.}{Indivíduo natural ou habitante desse país.}{i.bé.ri.co}{0}
\verb{ibérico}{}{}{}{}{}{Partidário da União Ibérica.}{i.bé.ri.co}{0}
\verb{ibero}{é}{}{}{}{adj.}{Relativo à Península Ibérica.}{i.be.ro}{0}
\verb{ibero}{é}{}{}{}{s.m.}{Indivíduo dos iberos, antigos habitantes da Ibéria.}{i.be.ro}{0}
\verb{ibero}{é}{}{}{}{}{A língua falada pelos antigos iberos.}{i.be.ro}{0}
\verb{ibero"-americano}{é}{}{ibero"-americanos ⟨é⟩}{}{adj.}{Relativo aos povos americanos colonizados pelos países da Península Ibérica. }{i.be.ro"-a.me.ri.ca.no}{0}
\verb{ibero"-americano}{é}{}{ibero"-americanos ⟨é⟩}{}{s.m.}{Indivíduo natural ou habitante dos países latinos da América.}{i.be.ro"-a.me.ri.ca.no}{0}
\verb{ibidem}{}{}{}{}{adv.}{No mesmo lugar; aí mesmo.}{\textit{ibidem}}{0}
\verb{íbis}{}{Zool.}{}{}{s.2g.}{Ave de hábitos aquáticos, caracterizadas pelo bico longo, largo e recurvo.}{í.bis}{0}
\verb{ibope}{ó}{Pop.}{}{}{s.m.}{Índice de audiência; sigla de Instituto Brasileiro de Opinião Pública e Estatística.}{i.bo.pe}{0}
\verb{ibope}{ó}{Por ext.}{}{}{}{Prestígio, sucesso.}{i.bo.pe}{0}
\verb{içá}{}{Zool.}{}{}{s.2g.}{Fêmea da saúva, provida de asas.}{i.çá}{0}
\verb{içar}{}{}{}{}{v.t.}{Erguer, levantar, alçar.}{i.çar}{\verboinum{3}}
\verb{iceberg}{}{}{}{}{s.m.}{Bloco de gelo flutuante que se desprendeu de uma geleira polar e que fica à deriva nos mares árticos e antárticos ou, às vezes, encalha na costa.}{\textit{iceberg}}{0}
\verb{ícone}{}{}{}{}{s.m.}{Signo que representa relação de semelhança ou analogia com o referente.}{í.co.ne}{0}
\verb{ícone}{}{Informát.}{}{}{}{Figura apresentada na tela do computador, usada para identificar ou acionar um programa ou um recurso de programa.}{í.co.ne}{0}
\verb{ícone}{}{}{}{}{}{Representação em superfície plana da figura de Cristo, da Virgem Maria ou de um santo, na Igreja russa e na grega.}{í.co.ne}{0}
\verb{iconoclasta}{}{}{}{}{adj.2g.}{Diz"-se daquele que destrói imagens ou ídolos.}{i.co.no.clas.ta}{0}
\verb{iconoclasta}{}{}{}{}{s.2g.}{Indivíduo que destrói símbolos, imagens religiosas e imagens em geral.}{i.co.no.clas.ta}{0}
\verb{iconografia}{}{}{}{}{s.f.}{A arte de representar por meio da imagem.}{i.co.no.gra.fi.a}{0}
\verb{iconografia}{}{}{}{}{}{Conhecimento e descrição de imagens.}{i.co.no.gra.fi.a}{0}
\verb{iconografia}{}{}{}{}{}{Conjunto das ilustrações de uma obra impressa.}{i.co.no.gra.fi.a}{0}
\verb{iconoteca}{é}{}{}{}{s.f.}{Coleção sistematizada de imagens.}{i.co.no.te.ca}{0}
\verb{iconoteca}{é}{}{}{}{}{Local destinado, em museu, biblioteca, à guarda de material iconográfico.}{i.co.no.te.ca}{0}
\verb{icosaedro}{é}{Geom.}{}{}{s.m.}{Poliedro com vinte faces.}{i.co.sa.e.dro}{0}
\verb{icoságono}{}{Geom.}{}{}{s.m.}{Polígono de vinte lados.}{i.co.sá.go.no}{0}
\verb{icterícia}{}{Med.}{}{}{s.f.}{Doença caracterizada pela coloração amarela dos tecidos e das secreções orgânicas, resultante da presença anormal de pigmentos biliares.}{ic.te.rí.cia}{0}
\verb{ictíico}{}{}{}{}{adj.}{Relativo a peixe.}{ic.tí.i.co}{0}
\verb{ictiofagia}{}{}{}{}{s.f.}{Hábito de se alimentar especialmente de peixe.  }{ic.ti.o.fa.gi.a}{0}
\verb{ictiófago}{}{}{}{}{adj.}{Que pratica a ictiofagia, que se alimenta de peixes.}{ic.ti.ó.fa.go}{0}
\verb{ictiófago}{}{}{}{}{s.m.}{Indivíduo que se alimenta de peixes.}{ic.ti.ó.fa.go}{0}
\verb{ictiologia}{}{}{}{}{s.f.}{Ramo da zoologia que estuda os peixes.}{ic.ti.o.lo.gi.a}{0}
\verb{ictiológico}{}{}{}{}{adj.}{Relativo à ictiologia.}{ic.ti.o.ló.gi.co}{0}
\verb{ictiólogo}{}{}{}{}{s.m.}{Especialista em ictiologia.}{ic.ti.ó.lo.go}{0}
\verb{id}{}{}{}{}{s.m.}{Na psicanálise, a divisão da psique referente aos impulsos instintivos e às exigências para a satisfação imediata das necessidades primárias.}{id}{0}
\verb{ida}{}{}{}{}{s.f.}{Ação ou movimento de ir; partida.}{i.da}{0}
\verb{ida}{}{}{}{}{}{Caminhada, passeio.}{i.da}{0}
\verb{ida}{}{}{}{}{}{Bilhete de viagem só de ida.}{i.da}{0}
\verb{idade}{}{}{}{}{s.f.}{Número de anos de alguém ou de algo.}{i.da.de}{0}
\verb{idade}{}{}{}{}{}{Duração ordinária da vida.}{i.da.de}{0}
\verb{idade}{}{}{}{}{}{Idade avançada; velhice.}{i.da.de}{0}
\verb{idade}{}{}{}{}{}{Cada um dos períodos em que se costuma dividir a vida do homem; época, tempo.}{i.da.de}{0}
\verb{idade}{}{}{}{}{}{Qualquer época da civilização que apresenta determinadas características culturais ou sociais; era.}{i.da.de}{0}
\verb{ideal}{}{}{"-ais}{}{adj.2g.}{Relativo à ideia, que só existe no pensamento.}{i.de.al}{0}
\verb{ideal}{}{}{"-ais}{}{}{Que é a síntese de tudo a que aspiramos, de toda perfeição que concebemos e que se pode conceber.}{i.de.al}{0}
\verb{ideal}{}{}{"-ais}{}{s.m.}{O que é objeto da nossa mais alta aspiração, alvo supremo de ambições e afetos.}{i.de.al}{0}
\verb{ideal}{}{}{"-ais}{}{}{Modelo de perfeição ou excelência.}{i.de.al}{0}
\verb{idealismo}{}{}{}{}{s.m.}{Atitude do que aspira a um ideal, frequentemente utópico.}{i.de.a.lis.mo}{0}
\verb{idealismo}{}{}{}{}{}{Doutrina que considera a ideia como princípio ou só do conhecimento, ou do conhecimento e do ser.}{i.de.a.lis.mo}{0}
\verb{idealista}{}{}{}{}{adj.2g.}{Relativo ao idealismo.}{i.de.a.lis.ta}{0}
\verb{idealista}{}{}{}{}{}{Que é sectário do idealismo.}{i.de.a.lis.ta}{0}
\verb{idealista}{}{}{}{}{s.2g.}{Indivíduo partidário do idealismo.}{i.de.a.lis.ta}{0}
\verb{idealista}{}{}{}{}{}{Indivíduo sonhador, devaneador.}{i.de.a.lis.ta}{0}
\verb{idealizar}{}{}{}{}{v.t.}{Tornar ideal.}{i.de.a.li.zar}{0}
\verb{idealizar}{}{}{}{}{}{Criar na imaginação; imaginar, fantasiar.}{i.de.a.li.zar}{0}
\verb{idealizar}{}{}{}{}{}{Projetar, planejar, programar.}{i.de.a.li.zar}{\verboinum{1}}
\verb{idear}{}{}{}{}{v.t.}{Criar na ideia, na imaginação; fantasiar, idealizar.}{i.de.ar}{0}
\verb{idear}{}{}{}{}{}{Projetar, planejar, delinear.}{i.de.ar}{\verboinum{4}}
\verb{ideia}{é}{}{}{}{s.f.}{Imagem de uma coisa na mente. (\textit{Preciso de boas ideias para pintar minhas telas.})}{i.dei.a}{0}
\verb{ideia}{é}{}{}{}{}{O que se pensa de alguma coisa; opinião. (\textit{Não tenho ideia do que ele foi fazer naquela cidade.})}{i.dei.a}{0}
\verb{idem}{}{}{}{}{pron.}{O mesmo; do mesmo modo.}{\textit{idem}}{0}
\verb{idêntico}{}{}{}{}{adj.}{Que é perfeitamente igual.}{i.dên.ti.co}{0}
\verb{idêntico}{}{Por ext.}{}{}{}{Que é muito parecido; semelhante, análogo.}{i.dên.ti.co}{0}
\verb{identidade}{}{}{}{}{s.f.}{Qualidade de idêntico.}{i.den.ti.da.de}{0}
\verb{identidade}{}{}{}{}{}{Conjunto de caracteres próprios e exclusivos de uma pessoa: nome, idade, profissão, sexo, impressões digitais, estado etc.}{i.den.ti.da.de}{0}
\verb{identidade}{}{}{}{}{}{Cédula de identidade.}{i.den.ti.da.de}{0}
\verb{identificação}{}{}{"-ões}{}{s.f.}{Ato ou efeito de identificar.}{i.den.ti.fi.ca.ção}{0}
\verb{identificação}{}{}{"-ões}{}{}{Documento comprovatório de identidade.}{i.den.ti.fi.ca.ção}{0}
\verb{identificação}{}{}{"-ões}{}{}{Processo pelo qual um indivíduo assimila algo da representação que possui do outro indivíduo, e se transforma com isso. }{i.den.ti.fi.ca.ção}{0}
\verb{identificar}{}{}{}{}{v.t.}{Tornar idêntico, igual.}{i.den.ti.fi.car}{0}
\verb{identificar}{}{}{}{}{}{Distinguir os traços característicos de alguém ou de algo; reconhecer.}{i.den.ti.fi.car}{0}
\verb{identificar}{}{}{}{}{}{Permitir a identificação, tornar conhecido.}{i.den.ti.fi.car}{0}
\verb{identificar}{}{}{}{}{v.pron.}{Confundir"-se com outrem; compenetrar"-se com as ideias ou sentimentos de outrem.}{i.den.ti.fi.car}{0}
\verb{identificar}{}{}{}{}{}{Conformar"-se, ajustar"-se.}{i.den.ti.fi.car}{\verboinum{2}}
\verb{ideograma}{}{}{}{}{s.m.}{Símbolo gráfico que corresponde a um conceito, utilizado em algumas escritas como o hieróglifo egípcio, ou os símbolos abstratos das escritas cuneiforme e chinesa.}{i.de.o.gra.ma}{0}
\verb{ideologia}{}{}{}{}{s.f.}{Conjunto das ideias gerais que constituem uma doutrina.}{i.de.o.lo.gi.a}{0}
\verb{ideologia}{}{}{}{}{}{Maneira de pensar que caracteriza um indivíduo ou um grupo de pessoas.}{i.de.o.lo.gi.a}{0}
\verb{ideológico}{}{}{}{}{adj.}{Relativo a ideologia.}{i.de.o.ló.gi.co}{0}
\verb{idílico}{}{}{}{}{adj.}{Relativo a idílio.}{i.dí.li.co}{0}
\verb{idílio}{}{}{}{}{s.m.}{Poema curto que celebra geralmente o amor em um ambiente pastoril.}{i.dí.lio}{0}
\verb{idílio}{}{Fig.}{}{}{}{Sonho, devaneio, utopia.}{i.dí.lio}{0}
\verb{idílio}{}{Fig.}{}{}{}{Entretenimento amoroso; amor poético e suave. }{i.dí.lio}{0}
\verb{idioma}{}{}{}{}{s.m.}{A língua própria de um povo, de uma nação, com o léxico e as formas gramaticais e fonológicas que lhe são peculiares.}{i.di.o.ma}{0}
\verb{idioma}{}{}{}{}{}{Linguagem, expressão.}{i.di.o.ma}{0}
\verb{idiomático}{}{}{}{}{adj.}{Relativo a idioma.}{i.di.o.má.ti.co}{0}
\verb{idiomatismo}{}{}{}{}{s.m.}{Traço ou construção peculiar a uma determinada língua, que não se encontra na maioria dos outros idiomas.}{i.di.o.ma.tis.mo}{0}
\verb{idiossincrasia}{}{}{}{}{s.f.}{Maneira de ver, sentir, reagir, própria, peculiar de cada pessoa.}{i.di.os.sin.cra.si.a}{0}
\verb{idiota}{ó}{}{}{}{adj.}{Em que se nota a falta de inteligência; bobo, imbecil, tolo.}{i.di.o.ta}{0}
\verb{idiota}{ó}{}{}{}{}{Que tem pouca inteligência.}{i.di.o.ta}{0}
\verb{idiota}{ó}{}{}{}{}{Que tem inteligência ou comportamento muito abaixo do comum; retardado.}{i.di.o.ta}{0}
\verb{idiotia}{}{}{}{}{s.f.}{Idiotice.}{i.di.o.ti.a}{0}
\verb{idiotia}{}{Med.}{}{}{}{Atraso intelectual profundo.}{i.di.o.ti.a}{0}
\verb{idiotice}{}{}{}{}{s.f.}{Qualidade ou procedimento de idiota.}{i.di.o.ti.ce}{0}
\verb{idiotismo}{}{}{}{}{s.m.}{Idiotice.}{i.di.o.tis.mo}{0}
\verb{idiotismo}{}{Gram.}{}{}{}{Palavra, locução ou expressão própria de uma língua, que não tem correspondente em outra.}{i.di.o.tis.mo}{0}
\verb{idiotizar}{}{}{}{}{v.t.}{Tornar idiota, imbecil.}{i.di.o.ti.zar}{\verboinum{1}}
\verb{idólatra}{}{}{}{}{adj.}{Relativo a idolatria.}{i.dó.la.tra}{0}
\verb{idólatra}{}{}{}{}{}{Que cultua, que adora ídolos.}{i.dó.la.tra}{0}
\verb{idólatra}{}{}{}{}{s.2g.}{Indivíduo que adora ídolos.}{i.dó.la.tra}{0}
\verb{idólatra}{}{Fig.}{}{}{}{Indivíduo que é admirador exaltado; amante.}{i.dó.la.tra}{0}
\verb{idolatrar}{}{}{}{}{v.t.}{Adorar ídolos, cultuar com idolatria.}{i.do.la.trar}{0}
\verb{idolatrar}{}{Fig.}{}{}{}{Amar excessivamente.}{i.do.la.trar}{\verboinum{1}}
\verb{idolatria}{}{}{}{}{s.f.}{Culto prestado a ídolos.}{i.do.la.tri.a}{0}
\verb{idolatria}{}{Fig.}{}{}{}{Amor excessivo; admiração exagerada.}{i.do.la.tri.a}{0}
\verb{ídolo}{}{}{}{}{s.m.}{Figura, imagem que representa uma divindade e é objeto de culto.}{í.do.lo}{0}
\verb{ídolo}{}{Fig.}{}{}{}{Pessoa a quem se tributa admiração e afeto excessivo.}{í.do.lo}{0}
\verb{idoneidade}{}{}{}{}{s.f.}{Qualidade de idôneo; capacidade, competência.}{i.do.nei.da.de}{0}
\verb{idôneo}{}{}{}{}{adj.}{Que é conveniente, adequado, próprio.}{i.dô.neo}{0}
\verb{idôneo}{}{}{}{}{}{Que tem condições para desempenhar certos cargos ou realizar certas obras.}{i.dô.neo}{0}
\verb{idos}{}{}{}{}{s.m.pl.}{Os tempos, os dias passados, decorridos.}{i.dos}{0}
\verb{idoso}{ô}{}{"-osos ⟨ó⟩}{"-osa ⟨ó⟩}{adj.}{Que tem muitos anos de idade; velho.}{i.do.so}{0}
\verb{idoso}{ô}{}{"-osos ⟨ó⟩}{"-osa ⟨ó⟩}{s.m.}{Indivíduo idoso.}{i.do.so}{0}
\verb{Iemanjá}{}{Relig.}{}{}{s.f.}{Orixá feminino, o próprio mar divinizado; mãe"-d'água, rainha do mar.}{I.e.man.já}{0}
\verb{iene}{}{}{}{}{s.m.}{Unidade monetária e moeda japonesa.}{i.e.ne}{0}
\verb{igaçaba}{}{Bras.}{}{}{s.f.}{Pote de barro, geralmente de boca larga, para guardar líquidos, farinha e outros gêneros.}{i.ga.ça.ba}{0}
\verb{igaçaba}{}{Bras.}{}{}{}{Urna funerária de alguns povos indígenas.}{i.ga.ça.ba}{0}
\verb{igapó}{}{Bras.}{}{}{s.m.}{Trecho de mata que fica cheio de água ou permanece assim por algum tempo após enchentes.}{i.ga.pó}{0}
\verb{igara}{}{Bras.}{}{}{s.f.}{Espécie de canoa escavada em um único tronco de árvore.}{i.ga.ra}{0}
\verb{igara}{}{Por ext.}{}{}{}{Qualquer embarcação.}{i.ga.ra}{0}
\verb{igarapé}{}{Bras.}{}{}{s.m.}{Pequeno curso de água que nasce e atravessa a mata e deságua em um rio.}{i.ga.ra.pé}{0}
\verb{igarapé}{}{Bras.}{}{}{}{Canal natural estreito navegável, formado entre ilhas fluviais ou entre uma dessas e a terra firme.}{i.ga.ra.pé}{0}
\verb{igarité}{}{Bras.}{}{}{s.f.}{Embarcação movida a remo, varejão ou motor, com capacidade para até 2 toneladas de carga.}{i.ga.ri.té}{0}
\verb{iglu}{}{}{}{}{s.m.}{Habitação em forma de cúpula construída com blocos de gelo, usada por certos povos do Polo Norte.}{i.glu}{0}
\verb{ignaro}{}{}{}{}{adj.}{Sem instrução; ignorante, rude.}{ig.na.ro}{0}
\verb{ignavo}{}{}{}{}{adj.}{Indolente, preguiçoso, ocioso.}{ig.na.vo}{0}
\verb{ignavo}{}{}{}{}{}{Covarde, fraco.}{ig.na.vo}{0}
\verb{ígneo}{}{}{}{}{adj.}{Relativo ao fogo.}{íg.neo}{0}
\verb{ígneo}{}{Fig.}{}{}{}{Entusiasmado, ardente.}{íg.neo}{0}
\verb{ígneo}{}{Geol.}{}{}{}{Diz"-se de rocha formada por magma solidificado.}{íg.neo}{0}
\verb{ignição}{}{}{"-ões}{}{s.f.}{Processo pelo qual se inflama um material combustível; nos motores a combustão, é uma centelha elétrica que provoca a explosão da mistura combustível.}{ig.ni.ção}{0}
\verb{ignição}{}{Quím.}{"-ões}{}{}{Estado das substâncias em combustão.}{ig.ni.ção}{0}
\verb{ignóbil}{}{}{"-eis}{}{adj.2g.}{Sem nobreza; vil, desprezível, baixo.}{ig.nó.bil}{0}
\verb{ignomínia}{}{}{}{}{s.f.}{Grande desonra; degradação pública; infâmia.}{ig.no.mí.nia}{0}
\verb{ignominioso}{ô}{}{"-osos ⟨ó⟩}{"-osa ⟨ó⟩}{adj.}{Que provoca ignomínia; desonroso.}{ig.no.mi.ni.o.so}{0}
\verb{ignorado}{}{}{}{}{adj.}{Não sabido; desconhecido.}{ig.no.ra.do}{0}
\verb{ignorado}{}{}{}{}{}{Obscuro, apagado, sem brilho.}{ig.no.ra.do}{0}
\verb{ignorância}{}{}{}{}{s.f.}{Falta de conhecimentos.}{ig.no.rân.cia}{0}
\verb{ignorância}{}{}{}{}{}{Estado de quem desconhece alguma coisa.}{ig.no.rân.cia}{0}
\verb{ignorância}{}{Bras.}{}{}{}{Falta de educação; grosseria.}{ig.no.rân.cia}{0}
\verb{ignorante}{}{}{}{}{}{Que desconhece determinada coisa.}{ig.no.ran.te}{0}
\verb{ignorante}{}{}{}{}{adj.2g.}{Diz"-se de pessoa sem instrução.}{ig.no.ran.te}{0}
\verb{ignorante}{}{Bras.}{}{}{}{Estúpido, grosseiro.}{ig.no.ran.te}{0}
\verb{ignorar}{}{}{}{}{v.t.}{Não saber; desconhecer.}{ig.no.rar}{0}
\verb{ignorar}{}{}{}{}{}{Não saber por não ter conhecimento específico em um assunto.}{ig.no.rar}{0}
\verb{ignorar}{}{}{}{}{}{Não prestar atenção.}{ig.no.rar}{\verboinum{1}}
\verb{ignoto}{ô}{}{}{}{adj.}{Desconhecido, ignorado.}{ig.no.to}{0}
\verb{igreja}{ê}{}{}{}{s.f.}{Construção de forma especial em que os cristãos vão rezar. (\textit{As crianças vão à igreja rezar.})}{i.gre.ja}{0}
\verb{igreja}{ê}{}{}{}{}{Com inicial maiúscula, cada uma das comunidades cristãs. (\textit{Igreja Católica.})}{i.gre.ja}{0}
\verb{igrejinha}{}{}{}{}{s.f.}{Pequena igreja.}{i.gre.ji.nha}{0}
\verb{igrejinha}{}{}{}{}{}{Grupo fechado de pessoas que se beneficiam ou elogiam mutuamente e dificultam o ingresso de outras pessoas; panelinha.}{i.gre.ji.nha}{0}
\verb{igual}{}{}{do adj.: -ais}{}{adj.2g.}{Diz"-se de elementos que não apresentam diferenças qualitativas ou quantitativas.}{i.gual}{0}
\verb{igual}{}{}{do adj.: -ais}{}{}{De mesma natureza, aparência, proporção, valor ou intensidade.}{i.gual}{0}
\verb{igual}{}{}{do adj.: -ais}{}{}{Diz"-se de indivíduos cujos direitos e deveres não diferem.}{i.gual}{0}
\verb{igual}{}{}{do adj.: -ais}{}{conj.}{Tal qual; como.}{i.gual}{0}
\verb{igual}{}{}{do adj.: -ais}{}{adv.}{Igualmente.}{i.gual}{0}
\verb{igualação}{}{}{"-ões}{}{s.f.}{Ato ou efeito de igualar.}{i.gua.la.ção}{0}
\verb{igualar}{}{}{}{}{v.t.}{Tornar igual qualitativa ou quantitativamente.}{i.gua.lar}{0}
\verb{igualar}{}{}{}{}{}{Pôr em pé de igualdade; ombrear, equiparar.}{i.gua.lar}{0}
\verb{igualar}{}{}{}{}{}{Tornar plano; nivelar, alisar.}{i.gua.lar}{\verboinum{1}}
\verb{igualdade}{}{}{}{}{s.f.}{Condição em que não há diferença; condição de igual.}{i.gual.da.de}{0}
\verb{igualdade}{}{Mat.}{}{}{}{Relação entre duas grandezas iguais.}{i.gual.da.de}{0}
\verb{igualdade}{}{Filos.}{}{}{}{Princípio segundo o qual todos os homens têm direitos e deveres iguais.}{i.gual.da.de}{0}
\verb{igualha}{}{}{}{}{s.f.}{Identidade de posição social ou de comportamento.}{i.gua.lha}{0}
\verb{igualitário}{}{}{}{}{adj.}{Relativo ao ou partidário do igualitarismo.}{i.gua.li.tá.rio}{0}
\verb{igualitário}{}{}{}{}{s.m.}{Indivíduo partidário do igualitarismo.}{i.gua.li.tá.rio}{0}
\verb{igualitarismo}{}{}{}{}{s.m.}{Teoria política segundo a qual todos os homens devem ter igualdade de condições perante a lei e a sociedade.}{i.gua.li.ta.ris.mo}{0}
\verb{iguana}{}{Zool.}{}{}{s.f.}{Réptil de grande porte caracterizado por uma crista que vai da nuca até a cauda.}{i.gua.na}{0}
\verb{iguaria}{}{}{}{}{s.f.}{Comida fina, delicada ou apetitosa.}{i.gua.ri.a}{0}
\verb{iguaria}{}{Por ext.}{}{}{}{Qualquer comida.}{i.gua.ri.a}{0}
\verb{ih}{}{}{}{}{interj.}{Expressão que denota admiração, espanto, medo, ironia.}{ih}{0}
\verb{iídiche}{}{}{}{}{s.m.}{Língua germânica falada pelos judeus.}{i.í.di.che}{0}
\verb{ilação}{}{}{"-ões}{}{s.f.}{Conclusão que se tira de certos fatos; dedução.}{i.la.ção}{0}
\verb{ilaquear}{}{}{}{}{v.t.}{Prender, enlaçar, confundir.}{i.la.que.ar}{0}
\verb{ilaquear}{}{}{}{}{}{Fazer cair em logro; enganar.}{i.la.que.ar}{0}
\verb{ilaquear}{}{}{}{}{v.i.}{Cair em tentação.}{i.la.que.ar}{\verboinum{4}}
\verb{ilativo}{}{}{}{}{adj.}{Em que há ilação; dedutivo, conclusivo.}{i.la.ti.vo}{0}
\verb{ilativo}{}{Gram.}{}{}{}{Diz"-se da conjunção que exprime ideia de consequência ou conclusão (logo, portanto).}{i.la.ti.vo}{0}
\verb{ilativo}{}{Gram.}{}{}{}{Diz"-se do caso gramatical que exprime ideia de movimento de fora para dentro, existente nas declinações de algumas línguas.}{i.la.ti.vo}{0}
\verb{ilegal}{}{}{"-ais}{}{adj.2g.}{Contrário à lei; ilícito, ilegítimo.}{i.le.gal}{0}
\verb{ilegalidade}{}{}{}{}{s.f.}{Qualidade de ilegal.}{i.le.ga.li.da.de}{0}
\verb{ilegalidade}{}{}{}{}{}{Situação ou procedimento ilegal.}{i.le.ga.li.da.de}{0}
\verb{ilegitimidade}{}{}{}{}{s.f.}{Qualidade de ilegítimo.}{i.le.gi.ti.mi.da.de}{0}
\verb{ilegitimidade}{}{Jur.}{}{}{}{Falta de condições para que algo seja juridicamente válido.}{i.le.gi.ti.mi.da.de}{0}
\verb{ilegítimo}{}{}{}{}{adj.}{Que não atende às condições legais.}{i.le.gí.ti.mo}{0}
\verb{ilegítimo}{}{}{}{}{}{Que não tem justificativa; desarrazoado.}{i.le.gí.ti.mo}{0}
\verb{ilegítimo}{}{}{}{}{}{Dizia"-se de filho gerado em relação fora do casamento.}{i.le.gí.ti.mo}{0}
\verb{ilegível}{}{}{"-eis}{}{adj.2g.}{De leitura difícil ou impossível.}{i.le.gí.vel}{0}
\verb{íleo}{}{Anat.}{}{}{s.m.}{Último segmento do intestino delgado.}{í.leo}{0}
\verb{íleo}{}{Med.}{}{}{}{Síndrome de obstrução intestinal.}{í.leo}{0}
\verb{ileocecal}{}{}{"-ais}{}{adj.2g.}{Relativo ao íleo e ao ceco; diz"-se da válvula localizada entre essas duas partes dos intestinos.}{i.le.o.ce.cal}{0}
\verb{ileso}{ê/ ou /é}{}{}{}{adj.}{Sem lesão; são e salvo.}{i.le.so}{0}
\verb{iletrado}{}{}{}{}{adj.}{Que não tem conhecimentos literários.}{i.le.tra.do}{0}
\verb{iletrado}{}{}{}{}{}{Analfabeto.}{i.le.tra.do}{0}
\verb{ilha}{}{Geogr.}{}{}{s.f.}{Porção de terra cercada de água por todos os lados.}{i.lha}{0}
\verb{ilha}{}{Fig.}{}{}{}{Aquele ou aquilo que se encontra isolado, incomunicável.}{i.lha}{0}
\verb{ilha}{}{Bras.}{}{}{}{Área reservada que existe no meio de grandes avenidas para proteção dos pedestres ou para organizar e separar as mãos de tráfego.}{i.lha}{0}
\verb{ilhado}{}{}{}{}{adj.}{Isolado.}{i.lha.do}{0}
\verb{ilhal}{}{}{"-ais}{}{s.m.}{Cada uma das depressões laterais abaixo do lombo do cavalo.}{i.lhal}{0}
\verb{ilhal}{}{}{"-ais}{}{}{Cada uma das partes da rês entre a última costela e o lombo.}{i.lhal}{0}
\verb{ilhal}{}{}{"-ais}{}{}{No homem, cada um dos lados do corpo; flanco.}{i.lhal}{0}
\verb{ilhar}{}{}{}{}{v.t.}{Tornar isolado ou incomunicável; isolar.}{i.lhar}{\verboinum{1}}
\verb{ilharga}{}{}{}{}{s.f.}{Cada um dos lados do corpo humano, na altura do baixo ventre; flanco.}{i.lhar.ga}{0}
\verb{ilhéu}{}{}{}{ilhoa ⟨ô⟩}{adj.}{Relativo a ilha.}{i.lhéu}{0}
\verb{ilhéu}{}{}{}{ilhoa ⟨ô⟩}{s.m.}{Indivíduo natural ou habitante de uma ilha.}{i.lhéu}{0}
\verb{ilhéu}{}{}{}{ilhoa ⟨ô⟩}{}{Rochedo no meio do mar.}{i.lhéu}{0}
\verb{ilhó}{}{}{}{}{}{Var. de \textit{ilhós}.}{i.lhó}{0}
\verb{ilhós}{}{}{"-oses ⟨ó⟩}{}{}{Aro geralmente de metal para dar acabamento a esse orifício.}{i.lhós}{0}
\verb{ilhós}{}{}{"-oses ⟨ó⟩}{}{s.2g.}{Orifício por onde se passa uma fita ou cordão.}{i.lhós}{0}
\verb{ilhota}{ó}{}{}{}{s.f.}{Pequena ilha; ilhéu.}{i.lho.ta}{0}
\verb{ilíaco}{}{}{}{}{adj.}{Relativo a Ílion ou Troia, antiga cidade da Ásia Menor.}{i.lí.a.co}{0}
\verb{ilíaco}{}{}{}{}{s.m.}{Indivíduo natural ou habitante dessa cidade.}{i.lí.a.co}{0}
\verb{ilibado}{}{}{}{}{adj.}{Não tocado; puro, incorrupto.}{i.li.ba.do}{0}
\verb{ilibado}{}{}{}{}{}{Dado como livre de culpa ou suspeita; reabilitado, justificado.}{i.li.ba.do}{0}
\verb{ilibar}{}{}{}{}{v.t.}{Purificar, depurar.}{i.li.bar}{0}
\verb{ilibar}{}{}{}{}{}{Restituir a estima; reabilitar, justificar.}{i.li.bar}{\verboinum{1}}
\verb{ilícito}{}{}{}{}{adj.}{Que não está conforme a lei; ilegítimo.}{i.lí.ci.to}{0}
\verb{ilícito}{}{}{}{}{}{Que não é moralmente aceitável.}{i.lí.ci.to}{0}
\verb{ilícito}{}{}{}{}{s.m.}{Ato ilícito; ilicitude.}{i.lí.ci.to}{0}
\verb{ilicitude}{}{}{}{}{s.f.}{Qualidade de ilícito.}{i.li.ci.tu.de}{0}
\verb{ilicitude}{}{}{}{}{}{Ato ilícito.}{i.li.ci.tu.de}{0}
\verb{ilimitado}{}{}{}{}{adj.}{Que não tem limite; infinito.}{i.li.mi.ta.do}{0}
\verb{ilimitado}{}{}{}{}{}{Cujo tamanho ou quantidade não pode ser calculada.}{i.li.mi.ta.do}{0}
\verb{ilimitado}{}{}{}{}{}{Cujo término não é definido.}{i.li.mi.ta.do}{0}
\verb{ílio}{}{Anat.}{}{}{s.m.}{A maior das três partes do osso ilíaco.}{í.lio}{0}
\verb{ilógico}{}{}{}{}{adj.}{Que não tem lógica; incoerente, absurdo.}{i.ló.gi.co}{0}
\verb{ilógico}{}{}{}{}{}{Em que existe contradição.}{i.ló.gi.co}{0}
\verb{ilogismo}{}{}{}{}{s.m.}{Qualidade do que não é lógico; falta de lógica.}{i.lo.gis.mo}{0}
\verb{ilogismo}{}{}{}{}{}{Proposição, conclusão ou fato sem lógica.}{i.lo.gis.mo}{0}
\verb{iludir}{}{}{}{}{v.t.}{Causar ilusão; enganar, lograr.}{i.lu.dir}{0}
\verb{iludir}{}{}{}{}{}{Causar frustração; baldar.}{i.lu.dir}{0}
\verb{iludir}{}{}{}{}{v.pron.}{Cair ou viver em ilusão.}{i.lu.dir}{\verboinum{18}}
\verb{iluminação}{}{}{"-ões}{}{s.f.}{Ato ou efeito de iluminar.}{i.lu.mi.na.ção}{0}
\verb{iluminação}{}{}{"-ões}{}{}{Conjunto de luzes que iluminam determinado local.}{i.lu.mi.na.ção}{0}
\verb{iluminação}{}{Fig.}{"-ões}{}{}{Inspiração.}{i.lu.mi.na.ção}{0}
\verb{iluminação}{}{Relig.}{"-ões}{}{}{No budismo, estágio final da evolução espiritual, no qual há ausência de sofrimento.}{i.lu.mi.na.ção}{0}
\verb{iluminado}{}{}{}{}{adj.}{Que recebe luz.}{i.lu.mi.na.do}{0}
\verb{iluminado}{}{}{}{}{}{Dotado de saber ou inspiração.}{i.lu.mi.na.do}{0}
\verb{iluminado}{}{}{}{}{s.m.}{Indivíduo inspirado ou que atingiu a iluminação.}{i.lu.mi.na.do}{0}
\verb{iluminar}{}{}{}{}{v.t.}{Tornar claro, irradiando luz.}{i.lu.mi.nar}{0}
\verb{iluminar}{}{}{}{}{}{Tornar esclarecido; ilustrar.}{i.lu.mi.nar}{0}
\verb{iluminar}{}{}{}{}{}{Inspirar, orientar.}{i.lu.mi.nar}{0}
\verb{iluminar}{}{}{}{}{v.pron.}{Alegrar"-se.}{i.lu.mi.nar}{\verboinum{1}}
\verb{Iluminismo}{}{Filos.}{}{}{s.m.}{Movimento intelectual do século \textsc{xviii}, caracterizado pela exaltação da razão e pela recusa de qualquer forma de dogmatismo.}{I.lu.mi.nis.mo}{0}
\verb{Iluminismo}{}{Relig.}{}{}{}{Crença em uma intuição mística do ser humano, que pode guiá"-lo para a verdade religiosa.}{I.lu.mi.nis.mo}{0}
\verb{iluminista}{}{}{}{}{adj.2g.}{Relativo ao iluminismo.}{i.lu.mi.nis.ta}{0}
\verb{iluminista}{}{}{}{}{s.2g.}{Indivíduo partidário do iluminsimo.}{i.lu.mi.nis.ta}{0}
\verb{iluminura}{}{}{}{}{s.f.}{Grafismo decorativo feito em livros, principalmente na Idade Média, composto por letras capitais ornamentadas e por desenhos, arabescos, miniaturas.}{i.lu.mi.nu.ra}{0}
\verb{ilusão}{}{}{"-ões}{}{s.f.}{Engano dos sentidos ou da mente; interpretação falsa da realidade.}{i.lu.são}{0}
\verb{ilusão}{}{}{"-ões}{}{}{Sonho, devaneio, fantasia.}{i.lu.são}{0}
\verb{ilusionismo}{}{}{}{}{s.m.}{Técnica de criar fenômenos que parecem reais, por meios não naturais; prestidigitação.}{i.lu.si.o.nis.mo}{0}
\verb{ilusionista}{}{}{}{}{s.2g.}{Indivíduo que cria ilusões por meio de truques; prestidigitador.}{i.lu.si.o.nis.ta}{0}
\verb{ilusório}{}{}{}{}{adj.}{Que causa ilusão; enganoso.}{i.lu.só.rio}{0}
\verb{ilusório}{}{}{}{}{}{Falso, vão, errôneo.}{i.lu.só.rio}{0}
\verb{ilustração}{}{}{"-ões}{}{s.f.}{Ato ou efeito de ilustrar.}{i.lus.tra.ção}{0}
\verb{ilustração}{}{}{"-ões}{}{}{Gravura, fotografia, reprodução em uma obra impressa.}{i.lus.tra.ção}{0}
\verb{ilustração}{}{}{"-ões}{}{}{Conjunto de conhecimentos científicos, artísticos etc; saber, instrução.}{i.lus.tra.ção}{0}
\verb{ilustrado}{}{}{}{}{adj.}{Diz"-se de obra impressa que tem figuras, imagens, fotos etc.}{i.lus.tra.do}{0}
\verb{ilustrado}{}{}{}{}{}{Que se instruiu; esclarecido, culto.}{i.lus.tra.do}{0}
\verb{ilustrador}{ô}{}{}{}{adj.}{Que  ilustra, comentando ou exemplificando.}{i.lus.tra.dor}{0}
\verb{ilustrador}{ô}{}{}{}{s.m.}{Desenhista de ilustrações.}{i.lus.tra.dor}{0}
\verb{ilustrar}{}{}{}{}{v.t.}{Tornar compreensível; esclarecer, elucidar.}{i.lus.trar}{0}
\verb{ilustrar}{}{}{}{}{}{Servir como exemplo, modelo; demonstrar.}{i.lus.trar}{0}
\verb{ilustrar}{}{}{}{}{}{Enfeitar texto com imagens, estampas etc.}{i.lus.trar}{0}
\verb{ilustrar}{}{}{}{}{}{Transmitir conhecimento; ensinar, instruir.}{i.lus.trar}{0}
\verb{ilustrar}{}{}{}{}{}{Tornar ilustre, célebre; glorificar.}{i.lus.trar}{\verboinum{1}}
\verb{ilustrativo}{}{}{}{}{adj.}{Que serve para ilustrar, exemplificar.}{i.lus.tra.ti.vo}{0}
\verb{ilustre}{}{}{}{}{adj.}{Que se distingue por suas qualidades ou méritos; eminente, notável.}{i.lus.tre}{0}
\verb{ilustre}{}{}{}{}{}{Célebre, renomado, famoso.}{i.lus.tre}{0}
\verb{ilustre}{}{}{}{}{}{Dotado de nobreza; fidalgo.}{i.lus.tre}{0}
\verb{ímã}{}{}{}{}{s.m.}{Magneto natural que atrai o ferro e outros metais.}{í.mã}{0}
\verb{ímã}{}{}{}{}{}{Qualquer objeto imantado; ferradura, barra etc.}{í.mã}{0}
\verb{ímã}{}{Fig.}{}{}{}{Coisa que atrai.}{í.mã}{0}
\verb{imaculado}{}{}{}{}{adj.}{Sem mancha ou qualquer impureza; limpo.}{i.ma.cu.la.do}{0}
\verb{imaculado}{}{}{}{}{}{Sem pecado; puro, inocente.}{i.ma.cu.la.do}{0}
\verb{imagem}{}{}{"-ens}{}{s.f.}{Reprodução da forma de um corpo. (\textit{A imagem da árvore se refletia nas águas do lago.})}{i.ma.gem}{0}
\verb{imagem}{}{Relig.}{"-ens}{}{}{Representação de um santo ou santa. (\textit{Naquela gruta há uma imagem de Santa Bárbara.})}{i.ma.gem}{0}
\verb{imaginação}{}{}{"-ões}{}{s.f.}{Faculdade mental que permite elaborar ou evocar, no presente, imagens anteriormente percebidas.}{i.ma.gi.na.ção}{0}
\verb{imaginação}{}{}{"-ões}{}{}{Capacidade de formar imagens e concepções originais, de encontrar soluções novas para problemas.}{i.ma.gi.na.ção}{0}
\verb{imaginação}{}{}{"-ões}{}{}{Faculdade de criar, conceber, inventar.}{i.ma.gi.na.ção}{0}
\verb{imaginação}{}{}{"-ões}{}{}{Crença, superstição, fantasia.}{i.ma.gi.na.ção}{0}
\verb{imaginar}{}{}{}{}{v.t.}{Conceber por meio da imaginação; criar, inventar.}{i.ma.gi.nar}{0}
\verb{imaginar}{}{}{}{}{}{Fantasiar, idealizar, sonhar.}{i.ma.gi.nar}{0}
\verb{imaginar}{}{}{}{}{}{Fazer ideia de algo; visualizar.}{i.ma.gi.nar}{0}
\verb{imaginar}{}{}{}{}{}{Conceber ideia; supor, presumir.}{i.ma.gi.nar}{\verboinum{1}}
\verb{imaginário}{}{Mat.}{}{}{}{Diz"-se de número complexo cuja parte real é zero.}{i.ma.gi.ná.rio}{0}
\verb{imaginário}{}{}{}{}{adj.}{Que só existe na imaginação; ilusório, fantástico.}{i.ma.gi.ná.rio}{0}
\verb{imaginário}{}{}{}{}{s.m.}{Indivíduo que confecciona estátuas ou imagens de santos; santeiro.}{i.ma.gi.ná.rio}{0}
\verb{imaginativa}{}{}{}{}{s.f.}{Capacidade de imaginar; imaginação.}{i.ma.gi.na.ti.va}{0}
\verb{imaginativo}{}{}{}{}{adj.}{Que tem muita imaginação; fértil de ideias; engenhoso.}{i.ma.gi.na.ti.vo}{0}
\verb{imaginável}{}{}{"-eis}{}{adj.2g.}{Que se pode imaginar; concebível.}{i.ma.gi.ná.vel}{0}
\verb{imaginoso}{ô}{}{"-osos ⟨ó⟩}{"-osa ⟨ó⟩}{adj.}{Que tem facilidade de criar, tem imaginação fértil.}{i.ma.gi.no.so}{0}
\verb{imaginoso}{ô}{}{"-osos ⟨ó⟩}{"-osa ⟨ó⟩}{}{Fabuloso, fantástico, inverossímil.}{i.ma.gi.no.so}{0}
\verb{imanar}{}{}{}{}{v.t.}{Imantar.}{i.ma.nar}{\verboinum{1}}
\verb{imane}{}{}{}{}{adj.2g.}{De grande porte; enorme, desmedido.}{i.ma.ne}{0}
\verb{imane}{}{}{}{}{}{Feroz, cruel, desumano.}{i.ma.ne}{0}
\verb{imanência}{}{}{}{}{s.f.}{Qualidade do que é imanente; inerência.}{i.ma.nên.cia}{0}
\verb{imanente}{}{}{}{}{adj.2g.}{Que está contido intrinsecamente em um objeto e não pode ser separado dele.}{i.ma.nen.te}{0}
\verb{imanente}{}{}{}{}{}{Permanente, constante, perdurável.}{i.ma.nen.te}{0}
\verb{imanizar}{}{}{}{}{v.t.}{Imantar.}{i.ma.ni.zar}{\verboinum{1}}
\verb{imantar}{}{}{}{}{v.t.}{Conferir a metal propriedades magnéticas; imanar, imanizar.}{i.man.tar}{\verboinum{1}}
\verb{imarcescível}{}{}{"-eis}{}{adj.2g.}{Que não perde o viço, não murcha.}{i.mar.ces.cí.vel}{0}
\verb{imarcescível}{}{Fig.}{"-eis}{}{}{Que não se altera, nem se corrompe.}{i.mar.ces.cí.vel}{0}
\verb{imaterial}{}{}{"-ais}{}{adj.2g.}{Que não tem consistência material; incorpóreo, impalpável.}{i.ma.te.ri.al}{0}
\verb{imaturidade}{}{}{}{}{s.f.}{Estado ou qualidade de imaturo; sem maturidade.}{i.ma.tu.ri.da.de}{0}
\verb{imaturo}{}{}{}{}{adj.}{Que ainda não chegou à maturidade.}{i.ma.tu.ro}{0}
\verb{imaturo}{}{}{}{}{}{Precoce, prematuro, antecipado.}{i.ma.tu.ro}{0}
\verb{imaturo}{}{}{}{}{}{Inconsciente, próprio de criança.}{i.ma.tu.ro}{0}
\verb{imbatível}{}{}{"-eis}{}{adj.2g.}{Que não se consegue bater ou derrotar; invencível.}{im.ba.tí.vel}{0}
\verb{imbaúba}{}{}{}{}{}{Var. de \textit{umbaúba}.}{im.ba.ú.ba}{0}
\verb{imbé}{}{Bot.}{}{}{s.m.}{Nome comum a diversas plantas trepadeiras, cujas raízes aéreas fornecem fibras para um tipo de barbante ou corda.}{im.bé}{0}
\verb{imbecil}{}{}{"-is}{}{adj.2g.}{Que revela pouca inteligência; estúpido, parvo, tolo.}{im.be.cil}{0}
\verb{imbecil}{}{}{"-is}{}{}{Que apresenta retardo mental, imbecilidade.}{im.be.cil}{0}
\verb{imbecil}{}{}{"-is}{}{}{Que não tem coragem; fraco, covarde.}{im.be.cil}{0}
\verb{imbecilidade}{}{}{}{}{s.f.}{Falta de inteligência; ignorância, tolice, estupidez.}{im.be.ci.li.da.de}{0}
\verb{imbecilidade}{}{}{}{}{}{Retardo mental.}{im.be.ci.li.da.de}{0}
\verb{imbecilidade}{}{}{}{}{}{Covardia, fraqueza.}{im.be.ci.li.da.de}{0}
\verb{imbecilizar}{}{}{}{}{v.t.}{Tornar imbecil.}{im.be.ci.li.zar}{\verboinum{1}}
\verb{imbele}{é}{}{}{}{adj.2g.}{Que não tem espírito belicoso; que não é apto para a guerra.}{im.be.le}{0}
\verb{imbele}{é}{}{}{}{}{Fraco, tímido, covarde.}{im.be.le}{0}
\verb{imberbe}{é}{}{}{}{adj.}{Que não tem barba.}{im.ber.be}{0}
\verb{imberbe}{é}{}{}{}{}{Novo, jovem, iniciante.}{im.ber.be}{0}
\verb{imbricação}{}{}{"-ões}{}{s.f.}{Disposição, arranjo de objetos de modo que uns se sobreponham, em parte, aos outros, como as telhas de um telhado.}{im.bri.ca.ção}{0}
\verb{imbricar}{}{}{}{}{v.t.}{Ligar estreitamente objetos uns aos outros sobrepondo"-os parcialmente como as escamas dos peixes.}{im.bri.car}{\verboinum{2}}
\verb{imbu}{}{}{}{}{}{Var. de \textit{umbu}.}{im.bu}{0}
\verb{imbuia}{}{Bot.}{}{}{s.f.}{Árvore de grande porte, que fornece madeira durável, parda e rica em desenhos.}{im.bui.a}{0}
\verb{imbuir}{}{}{}{}{v.t.}{Colocar num líquido; embeber, impregnar.}{im.bu.ir}{0}
\verb{imbuir}{}{}{}{}{}{Fazer penetrar; entranhar, embutir.}{im.bu.ir}{0}
\verb{imbuir}{}{}{}{}{}{Insinuar, convencer, incutir.}{im.bu.ir}{\verboinum{26}}
\verb{imbuzeiro}{ê}{}{}{}{}{Var. de \textit{umbuzeiro}.}{im.bu.zei.ro}{0}
\verb{imediação}{}{}{"-ões}{}{s.f.}{O fato de ser ou estar próximo, imediato; contiguidade.}{i.me.di.a.ção}{0}
\verb{imediações}{}{}{}{}{s.f.pl.}{Vizinhanças, cercanias, arredores.}{i.me.di.a.ções}{0}
\verb{imediatamente}{}{}{}{}{adv.}{Sem demora, sem intervalo.}{i.me.di.a.ta.men.te}{0}
\verb{imediatismo}{}{}{}{}{s.m.}{Maneira direta de agir, sem intervenções ou rodeios.}{i.me.di.a.tis.mo}{0}
\verb{imediatista}{}{}{}{}{adj.2g.}{Que procede de maneira direta, sem rodeios.}{i.me.di.a.tis.ta}{0}
\verb{imediato}{}{}{}{}{adj.}{Que não tem mediação de espaço; sem intermediário; direto.}{i.me.di.a.to}{0}
\verb{imediato}{}{}{}{}{}{Que segue numa série, sem intervalo; contíguo, próximo.}{i.me.di.a.to}{0}
\verb{imediato}{}{}{}{}{}{Que acontece sem demora; rápido, instantâneo.}{i.me.di.a.to}{0}
\verb{imediato}{}{}{}{}{s.m.}{Funcionário de categoria logo abaixo da do chefe, e que o substitui na sua ausência.}{i.me.di.a.to}{0}
\verb{imemorável}{}{}{"-eis}{}{adj.2g.}{Diz"-se do que não pode haver ou não há memória; imemorial.}{i.me.mo.rá.vel}{0}
\verb{imemorial}{}{}{"-ais}{}{adj.2g.}{Anterior à época de que se tem memória; muito antigo.}{i.me.mo.ri.al}{0}
\verb{imemorial}{}{}{"-ais}{}{}{Imemorável.}{i.me.mo.ri.al}{0}
\verb{imensidade}{}{}{}{}{s.f.}{Qualidade do que é imenso; imensidão.}{i.men.si.da.de}{0}
\verb{imensidade}{}{}{}{}{}{Extensão muito vasta, ilimitada; vastidão.}{i.men.si.da.de}{0}
\verb{imensidade}{}{}{}{}{}{Quantidade grande ou incontável; infinito.}{i.men.si.da.de}{0}
\verb{imensidão}{}{}{"-ões}{}{s.f.}{Imensidade.}{i.men.si.dão}{0}
\verb{imenso}{}{}{}{}{adj.}{De grande extensão; ilimitado, infinito.}{i.men.so}{0}
\verb{imenso}{}{}{}{}{}{Muito grande; enorme, colossal.}{i.men.so}{0}
\verb{imenso}{}{}{}{}{}{Em grande quantidade; muito numeroso; inumerável.}{i.men.so}{0}
\verb{imensurável}{}{}{"-eis}{}{adj.2g.}{Que não pode ser medido; ilimitado, incomensurável.}{i.men.su.rá.vel}{0}
\verb{imerecido}{}{}{}{}{adj.}{Que não é merecido; injusto, indevido, imérito.}{i.me.re.ci.do}{0}
\verb{imergir}{}{}{}{}{v.t.}{Mergulhar em um líquido; afundar, submergir.}{i.mer.gir}{0}
\verb{imergir}{}{}{}{}{}{Entrar, penetrar, introduzir.}{i.mer.gir}g\verboinum{34}\verboirregular{\emph{def.} imerges, imerge, imergimos, imergis, imergem}}
\verb{imérito}{}{}{}{}{adj.}{Imerecido.}{i.mé.ri.to}{0}
\verb{imersão}{}{}{"-ões}{}{s.f.}{Ato ou efeito de imergir, afundar; submersão.}{i.mer.são}{0}
\verb{imerso}{é}{}{}{}{adj.}{Que imergiu; afundado, submerso.}{i.mer.so}{0}
\verb{imerso}{é}{}{}{}{}{Que se absorveu; mergulhado, abismado.}{i.mer.so}{0}
\verb{imigração}{}{}{"-ões}{}{s.f.}{Ato ou efeito de imigrar; estabelecimento de um indivíduo em um país estrangeiro de forma temporária ou definitiva.}{i.mi.gra.ção}{0}
\verb{imigrante}{}{}{}{}{adj.2g.}{Diz"-se do indivíduo que se estabeleceu em país estrangeiro de modo temporário ou definitivo.}{i.mi.gran.te}{0}
\verb{imigrar}{}{}{}{}{v.t.}{Fixar residência em país estrangeiro, temporária ou definitivamente. }{i.mi.grar}{\verboinum{1}}
\verb{imigratório}{}{}{}{}{adj.}{Relativo a imigração ou a imigrante.}{i.mi.gra.tó.rio}{0}
\verb{iminência}{}{}{}{}{s.f.}{Qualidade ou condição do que está iminente; urgência, aproximação.}{i.mi.nên.cia}{0}
\verb{iminente}{}{}{}{}{adj.2g.}{Que está próximo de acontecer; imediato.}{i.mi.nen.te}{0}
\verb{imiscuir"-se}{}{}{}{}{v.pron.}{Envolver"-se em algo que não lhe diz respeito; tomar parte; intrometer"-se.}{i.mis.cu.ir"-se}{0}
\verb{imiscuir}{}{}{}{}{}{Misturar"-se; confundir"-se; mesclar"-se.}{i.mis.cu.ir"-se}{\verboinum{18}}
\verb{imitação}{}{}{"-ões}{}{s.f.}{Ato ou efeito de imitar.}{i.mi.ta.ção}{0}
\verb{imitação}{}{}{"-ões}{}{}{Material trabalhado que reproduz outro mais rico ou mais caro.}{i.mi.ta.ção}{0}
\verb{imitação}{}{}{"-ões}{}{}{Cópia malfeita; falsificação, arremedo.}{i.mi.ta.ção}{0}
\verb{imitador}{ô}{}{}{}{adj.}{Que faz imitações, reproduções.}{i.mi.ta.dor}{0}
\verb{imitar}{}{}{}{}{v.t.}{Reproduzir ou tentar reproduzir exatamente; copiar.}{i.mi.tar}{0}
\verb{imitar}{}{}{}{}{}{Tentar produzir o mesmo efeito; assemelhar; lembrar.}{i.mi.tar}{0}
\verb{imitar}{}{}{}{}{}{Tomar por modelo ou norma; inspirar"-se.}{i.mi.tar}{0}
\verb{imitar}{}{}{}{}{}{Dar falsa aparência; falsificar, plagiar.}{i.mi.tar}{\verboinum{1}}
\verb{imobiliária}{}{}{}{}{s.f.}{Empresa que se dedica a negociar imóveis ou administrar seu aluguel.}{i.mo.bi.li.á.ria}{0}
\verb{imobiliário}{}{}{}{}{adj.}{Relativo a um bem imóvel ou a imóveis.}{i.mo.bi.li.á.rio}{0}
\verb{imobilidade}{}{}{}{}{}{Estado ou condição do que está imóvel, sem movimento; inércia.}{i.mo.bi.li.da.de}{0}
\verb{imobilidade}{}{}{}{}{s.f.}{Ato de permanecer muito tempo em um local; fixidez, estabilidade.}{i.mo.bi.li.da.de}{0}
\verb{imobilidade}{}{}{}{}{}{Estado de quem não se deixa perturbar por emoções; impassibilidade, serenidade.}{i.mo.bi.li.da.de}{0}
\verb{imobilismo}{}{}{}{}{s.m.}{Oposição sistemática a qualquer tipo de progresso; apego às tradições; conservadorismo.}{i.mo.bi.lis.mo}{0}
\verb{imobilização}{}{}{"-ões}{}{s.f.}{Ato ou efeito de imobilizar; tornar imóvel.}{i.mo.bi.li.za.ção}{0}
\verb{imobilização}{}{}{"-ões}{}{}{Parada ou diminuição do ritmo de uma atividade.}{i.mo.bi.li.za.ção}{0}
\verb{imobilização}{}{Med.}{"-ões}{}{}{Medida terapêutica que mantém imóvel um membro ou parte do corpo de um paciente por meio de bandagem ou engessamento.}{i.mo.bi.li.za.ção}{0}
\verb{imobilizar}{}{}{}{}{v.t.}{Tornar imóvel, sem movimento.}{i.mo.bi.li.zar}{0}
\verb{imobilizar}{}{}{}{}{}{Obrigar a permanecer inativo.}{i.mo.bi.li.zar}{0}
\verb{imobilizar}{}{}{}{}{}{Impedir o progresso; reter, estacionar.}{i.mo.bi.li.zar}{0}
\verb{imobilizar}{}{}{}{}{}{Reduzir ou fazer parar o movimento de certo membro ou parte do corpo, para fins terapêuticos.}{i.mo.bi.li.zar}{\verboinum{1}}
\verb{imoderação}{}{}{"-ões}{}{s.f.}{Falta de moderação; exagero, excesso.}{i.mo.de.ra.ção}{0}
\verb{imoderado}{}{}{}{}{adj.}{Que passa dos limites; excessivo, exagerado, descomedido.}{i.mo.de.ra.do}{0}
\verb{imodéstia}{}{}{}{}{s.f.}{Falta de modéstia; vaidade, orgulho.}{i.mo.dés.tia}{0}
\verb{imodéstia}{}{}{}{}{}{Falta de pudor; indecência, despudor.}{i.mo.dés.tia}{0}
\verb{imodesto}{é}{}{}{}{adj.}{Que não tem modéstia; vaidoso, presunçoso.}{i.mo.des.to}{0}
\verb{imodesto}{é}{}{}{}{}{Que não tem pudor; indecente, despudorado.}{i.mo.des.to}{0}
\verb{imolação}{}{}{"-ões}{}{s.f.}{Ato ou efeito de imolar, sacrificar.}{i.mo.la.ção}{0}
\verb{imolação}{}{}{"-ões}{}{}{Grande massacre; matança, carnificina.}{i.mo.la.ção}{0}
\verb{imolado}{}{}{}{}{adj.}{Que se imolou; sacrificado.}{i.mo.la.do}{0}
\verb{imolado}{}{}{}{}{}{Vítima de massacre; chacinado, assassinado.}{i.mo.la.do}{0}
\verb{imolar}{}{}{}{}{v.t.}{Oferecer em sacrifício.}{i.mo.lar}{0}
\verb{imolar}{}{}{}{}{}{Massacrar, assassinar, chacinar.}{i.mo.lar}{\verboinum{1}}
\verb{imoral}{}{}{"-ais}{}{adj.2g.}{Que é contrário aos princípios da moral; indecoroso, vergonhoso.}{i.mo.ral}{0}
\verb{imoral}{}{}{"-ais}{}{}{Contrário ao pudor; indecente, devasso, libertino.}{i.mo.ral}{0}
\verb{imoralidade}{}{}{}{}{s.f.}{Qualidade do que é imoral; indecência. }{i.mo.ra.li.da.de}{0}
\verb{imoralidade}{}{}{}{}{}{Devassidão, libertinagem, depravação.}{i.mo.ra.li.da.de}{0}
\verb{imoralismo}{}{}{}{}{s.m.}{Tendência a desprezar os valores morais normalmente aceitos.}{i.mo.ra.lis.mo}{0}
\verb{imorredoiro}{ô}{}{}{}{}{Var. de \textit{imorredouro}.}{i.mor.re.doi.ro}{0}
\verb{imorredouro}{ô}{}{}{}{adj.}{Que não morre; eterno, imortal.}{i.mor.re.dou.ro}{0}
\verb{imorredouro}{ô}{}{}{}{}{Perdurável, constante, duradouro.}{i.mor.re.dou.ro}{0}
\verb{imortal}{}{}{"-ais}{}{adj.2g.}{Que não está sujeito à morte; eterno, imorredouro.}{i.mor.tal}{0}
\verb{imortal}{}{}{"-ais}{}{}{Que não terá fim; infinito, infindável.}{i.mor.tal}{0}
\verb{imortal}{}{Fig.}{"-ais}{}{}{De que se lembra durante muito tempo; inesquecível.}{i.mor.tal}{0}
\verb{imortal}{}{}{"-ais}{}{s.2g.}{Membro da Academia Brasileira de Letras.}{i.mor.tal}{0}
\verb{imortalidade}{}{}{}{}{}{Qualidade ou estado do que é imortal; que não perece.}{i.mor.ta.li.da.de}{0}
\verb{imortalidade}{}{}{}{}{s.f.}{Vida perpétua na memória dos homens; perenidade.}{i.mor.ta.li.da.de}{0}
\verb{imortalizar}{}{}{}{}{v.t.}{Tornar imortal; eternizar.}{i.mor.ta.li.zar}{0}
\verb{imortalizar}{}{}{}{}{}{Tornar lembrado na memória dos homens para sempre.}{i.mor.ta.li.zar}{\verboinum{1}}
\verb{imóvel}{}{}{"-eis}{}{adj.2g.}{Que não se move; inerte, parado.}{i.mó.vel}{0}
\verb{imóvel}{}{}{"-eis}{}{s.m.}{Bem fixo, que não pode ser transportado, como casa, apartamento, terreno etc.}{i.mó.vel}{0}
\verb{impaciência}{}{}{}{}{s.f.}{Falta de paciência; incapacidade de esperar ou suportar algo ou alguém; irritação.}{im.pa.ci.ên.cia}{0}
\verb{impaciência}{}{}{}{}{}{Pressa em fazer algo; sofreguidão.}{im.pa.ci.ên.cia}{0}
\verb{impacientar}{}{}{}{}{v.t.}{Tornar impaciente; irritar.}{im.pa.ci.en.tar}{\verboinum{1}}
\verb{impaciente}{}{}{}{}{adj.2g.}{Que não tem paciência; inquieto, irritado.}{im.pa.ci.en.te}{0}
\verb{impaciente}{}{}{}{}{}{Apressado, agitado, sôfrego.}{im.pa.ci.en.te}{0}
\verb{impacto}{}{}{}{}{s.m.}{Colisão de dois ou vários corpos.}{im.pac.to}{0}
\verb{impacto}{}{}{}{}{}{Impressão forte deixada por alguma ação ou acontecimento.}{im.pac.to}{0}
\verb{impacto}{}{Fig.}{}{}{}{Abalo moral; choque.}{im.pac.to}{0}
\verb{impagável}{}{}{"-eis}{}{adj.2g.}{Que não se pode ou não se deve pagar.}{im.pa.gá.vel}{0}
\verb{impagável}{}{}{"-eis}{}{}{Muito engraçado; cômico, ridículo.}{im.pa.gá.vel}{0}
\verb{impalpável}{}{}{"-eis}{}{adj.2g.}{Que não se pode apalpar; que não é percebido pelo tato; intangível.}{im.pal.pá.vel}{0}
\verb{impalpável}{}{}{"-eis}{}{}{Imaterial, incorpóreo.}{im.pal.pá.vel}{0}
\verb{impaludar}{}{}{}{}{v.t.}{Infeccionar com agente causador de impaludismo.}{im.pa.lu.dar}{\verboinum{1}}
\verb{impaludismo}{}{Med.}{}{}{s.m.}{Doença aguda ou crônica transmitida pela picada de mosquitos e que se caracteriza por calafrios e febres periódicas; malária.}{im.pa.lu.dis.mo}{0}
\verb{impar}{}{}{}{}{v.i.}{Respirar com dificuldade; ofegar, arfar.}{im.par}{0}
\verb{ímpar}{}{Mat.}{}{}{adj.2g.}{Diz"-se do número não divisível por dois números inteiros iguais. }{ím.par}{0}
\verb{impar}{}{}{}{}{}{Entupir"-se de comida ou bebida; empanturrar"-se.}{im.par}{0}
\verb{impar}{}{}{}{}{}{Mostrar"-se soberbo, arrogante.}{im.par}{\verboinum{1}}
\verb{ímpar}{}{}{}{}{}{Que não tem par; não tem igual; único.}{ím.par}{0}
\verb{imparcial}{}{}{"-ais}{}{adj.2g.}{Que não toma partido ao julgar algo; justo, reto.}{im.par.ci.al}{0}
\verb{imparcial}{}{}{"-ais}{}{}{Que não sacrifica a verdade ou a justiça a considerações particulares.}{im.par.ci.al}{0}
\verb{imparcialidade}{}{}{}{}{s.f.}{Qualidade ou caráter do que é imparcial; equidade, retidão, isenção.}{im.par.ci.a.li.da.de}{0}
%\verb{}{}{}{}{}{}{}{}{0}
\verb{imparissílabo}{}{Gram.}{}{}{adj.}{Diz"-se do vocábulo que tem número de sílabas diferente em suas flexões.}{im.pa.ris.sí.la.bo}{0}
\verb{impasse}{}{}{}{}{s.m.}{Situação que não oferece solução favorável.}{im.pas.se}{0}
\verb{impasse}{}{}{}{}{}{Dificuldade, empecilho, embaraço.}{im.pas.se}{0}
\verb{impassibilidade}{}{}{}{}{s.f.}{Qualidade do que é impassível; tranquilidade, serenidade.}{im.pas.si.bi.li.da.de}{0}
\verb{impassível}{}{}{"-eis}{}{adj.2g.}{Que não experimenta ou não revela exteriormente sensibilidade à dor ou às paixões; imperturbável, sereno.}{im.pas.sí.vel}{0}
\verb{impatriótico}{}{}{}{}{adj.}{Que não tem patriotismo.}{im.pa.tri.ó.ti.co}{0}
\verb{impavidez}{ê}{}{}{}{s.f.}{Qualidade ou caráter de impávido; intrepidez, coragem.}{im.pa.vi.dez}{0}
\verb{impávido}{}{}{}{}{adj.}{Que não tem medo; intrépido, corajoso.}{im.pá.vi.do}{0}
\verb{impeachment}{}{}{}{}{s.m.}{No regime presidencialista, processo que se instaura contra uma autoridade do governo que pratica crime de responsabilidade, a fim de destituí"-la do cargo; impedimento.}{\textit{impeachment}}{0}
\verb{impecável}{}{}{"-eis}{}{adj.2g.}{Que é incapaz de pecar, de errar.}{im.pe.cá.vel}{0}
\verb{impecável}{}{}{"-eis}{}{}{Que não apresenta defeito; correto, perfeito.}{im.pe.cá.vel}{0}
\verb{impedância}{}{Fís.}{}{}{s.f.}{Medida da resistência de um circuito elétrico percorrido por uma corrente alternada.}{im.pe.dân.cia}{0}
\verb{impedimento}{}{}{}{}{s.m.}{Ato ou efeito de impedir; obstrução.}{im.pe.di.men.to}{0}
\verb{impedimento}{}{}{}{}{}{Aquilo que serve para impedir; obstáculo, estorvo.}{im.pe.di.men.to}{0}
\verb{impedimento}{}{Esport.}{}{}{}{No futebol, posição irregular de um jogador que, ao receber a bola, tem apenas o goleiro entre ele e o gol.}{im.pe.di.men.to}{0}
\verb{impedir}{}{}{}{}{v.t.}{Pôr obstáculo; opor"-se; proibir.}{im.pe.dir}{0}
\verb{impedir}{}{}{}{}{}{Dificultar a ação; não permitir, barrar.}{im.pe.dir}{0}
\verb{impedir}{}{}{}{}{}{Interromper; obstruir.}{im.pe.dir}{\verboinum{20}}
\verb{impeditivo}{}{}{}{}{adj.}{Que impede, obstrui.}{im.pe.di.ti.vo}{0}
\verb{impelir}{}{}{}{}{v.t.}{Arremessar com força; lançar, impulsionar.}{im.pe.lir}{0}
\verb{impelir}{}{}{}{}{}{Dar estímulo; incentivar, instigar.}{im.pe.lir}{0}
\verb{impelir}{}{}{}{}{}{Coagir, obrigar, constranger.}{im.pe.lir}{\verboinum{29}}
\verb{impene}{}{}{}{}{adj.2g.}{Diz"-se da ave sem penas ou plumas.}{im.pe.ne}{0}
\verb{impenetrável}{}{}{"-eis}{}{adj.2g.}{Que não se pode penetrar ou atravessar; inacessível.}{im.pe.ne.trá.vel}{0}
\verb{impenetrável}{}{}{"-eis}{}{}{Que não se pode explicar; incompreensível.}{im.pe.ne.trá.vel}{0}
\verb{impenetrável}{}{}{"-eis}{}{}{Que não revela o que sente ou pensa; reservado, fechado.}{im.pe.ne.trá.vel}{0}
\verb{impenitência}{}{}{}{}{s.f.}{Ausência de arrependimento; persistência no erro.}{im.pe.ni.tên.cia}{0}
\verb{impenitência}{}{Fig.}{}{}{}{Obstinação, apego.}{im.pe.ni.tên.cia}{0}
\verb{impenitente}{}{}{}{}{adj.2g.}{Que não se penitencia, não se arrepende.}{im.pe.ni.ten.te}{0}
\verb{impenitente}{}{}{}{}{}{Que persiste no erro ou no vício; inveterado.}{im.pe.ni.ten.te}{0}
\verb{impensado}{}{}{}{}{adj.}{Que não foi pensado ou premeditado; irrefletido, imprevisto.}{im.pen.sa.do}{0}
\verb{impensável}{}{}{"-eis}{}{adj.2g.}{Que não se pode pensar ou supor; inconcebível.}{im.pen.sá.vel}{0}
\verb{imperador}{ô}{}{}{imperatriz}{s.m.}{Chefe ou monarca supremo de um império.}{im.pe.ra.dor}{0}
\verb{imperar}{}{}{}{}{v.t.}{Exercer poder supremo; governar como imperador.}{im.pe.rar}{0}
\verb{imperar}{}{}{}{}{}{Exercer grande influência; dominar, prevalecer.}{im.pe.rar}{\verboinum{1}}
\verb{imperativo}{}{}{}{}{adj.}{Que exprime uma ordem; que impõe; autoritário.}{im.pe.ra.ti.vo}{0}
\verb{imperativo}{}{Gram.}{}{}{}{Diz"-se do modo verbal que exorta o interlocutor a cumprir a ação expressa pelo verbo.}{im.pe.ra.ti.vo}{0}
\verb{imperativo}{}{}{}{}{s.m.}{Imposição, dever, mando.}{im.pe.ra.ti.vo}{0}
\verb{imperatriz}{}{}{}{}{}{Mulher que governa um império; soberana.}{im.pe.ra.triz}{0}
\verb{imperatriz}{}{}{}{}{s.f.}{Esposa do imperador.}{im.pe.ra.triz}{0}
\verb{imperceptível}{}{}{"-eis}{}{adj.2g.}{Que não pode ser percebido ou apreendido pelos sentidos.}{im.per.cep.tí.vel}{0}
\verb{imperceptível}{}{}{"-eis}{}{}{De pouca importância; diminuto, insignificante.}{im.per.cep.tí.vel}{0}
\verb{imperdível}{}{}{"-eis}{}{adj.2g.}{Que não se pode perder; em que se tem a vitória como certa.}{im.per.dí.vel}{0}
\verb{imperdoável}{}{}{"-eis}{}{adj.2g.}{Que não merece perdão; indesculpável, irremissível.}{im.per.do.á.vel}{0}
\verb{imperecedoiro}{ô}{}{}{}{}{Var. de \textit{imperecedouro}.}{im.pe.re.ce.doi.ro}{0}
\verb{imperecedouro}{ô}{}{}{}{adj.}{Imperecível.}{im.pe.re.ce.dou.ro}{0}
\verb{imperecível}{}{}{"-eis}{}{adj.2g.}{Que não perece; que dura muito tempo; eterno, perene, imperecedouro.}{im.pe.re.cí.vel}{0}
\verb{imperfeição}{}{}{"-ões}{}{s.f.}{Qualidade ou estado do que não está perfeito ou está por terminar.}{im.per.fei.ção}{0}
\verb{imperfeição}{}{}{"-ões}{}{}{Falha, irregularidade, defeito.}{im.per.fei.ção}{0}
\verb{imperfeito}{ê}{}{}{}{adj.}{Que não está perfeito; falho, incompleto.}{im.per.fei.to}{0}
\verb{imperfeito}{ê}{}{}{}{}{Mal executado; defeituoso, incorreto.}{im.per.fei.to}{0}
\verb{imperfeito}{ê}{Gram.}{}{}{}{Diz"-se da forma verbal que exprime ação incompleta ou não realizada.}{im.per.fei.to}{0}
\verb{imperial}{}{}{"-ais}{}{adj.2g.}{Relativo a império ou a imperador.}{im.pe.ri.al}{0}
\verb{imperial}{}{}{"-ais}{}{}{Que apresenta pompa; luxuoso, magnificente.}{im.pe.ri.al}{0}
\verb{imperialismo}{}{}{}{}{s.m.}{Política expansionista de um Estado, visando ao domínio territorial ou à submissão econômica, política ou cultural de outros Estados.}{im.pe.ri.a.lis.mo}{0}
\verb{imperialista}{}{}{}{}{adj.2g.}{Relativo ao imperialismo.}{im.pe.ri.a.lis.ta}{0}
\verb{imperialista}{}{}{}{}{s.2g.}{Partidário do imperialismo exercido por um Estado.}{im.pe.ri.a.lis.ta}{0}
\verb{imperícia}{}{}{}{}{s.f.}{Falta de habilidade ou de destreza; inexperiência, incompetência.}{im.pe.rí.cia}{0}
\verb{império}{}{}{}{}{s.m.}{Forma de governo monárquico, cujo soberano recebe o título de imperador.}{im.pé.rio}{0}
\verb{império}{}{}{}{}{}{Conjunto de territórios dependentes de um imperador.}{im.pé.rio}{0}
\verb{império}{}{Por ext.}{}{}{}{Nação de grande porte ou muito influente, não importando a forma de governo.}{im.pé.rio}{0}
\verb{império}{}{}{}{}{}{Autoridade, domínio, poder.}{im.pé.rio}{0}
\verb{imperioso}{ô}{}{"-osos ⟨ó⟩}{"-osa ⟨ó⟩}{adj.}{Que manda com energia; autoritário, dominador.}{im.pe.ri.o.so}{0}
\verb{imperioso}{ô}{}{"-osos ⟨ó⟩}{"-osa ⟨ó⟩}{}{Que revela altivez; arrogante, orgulhoso.}{im.pe.ri.o.so}{0}
\verb{imperioso}{ô}{}{"-osos ⟨ó⟩}{"-osa ⟨ó⟩}{}{Que exerce grande influência; irresistível, impreterível.}{im.pe.ri.o.so}{0}
\verb{imperito}{}{}{}{}{adj.}{Que não tem perícia; inábil, inexperiente.}{im.pe.ri.to}{0}
\verb{impermeabilizar}{}{}{}{}{v.t.}{Tornar impermeável.}{im.per.me.a.bi.li.zar}{\verboinum{1}}
\verb{impermeável}{}{}{"-eis}{}{adj.2g.}{Que não se deixa atravessar por fluidos, especialmente pela água. (\textit{O vidro da janela é impermeável.})}{im.per.me.á.vel}{0}
\verb{impermeável}{}{Fig.}{"-eis}{}{}{Que não se deixa penetrar, atravessar, atingir.}{im.per.me.á.vel}{0}
\verb{imperscrutável}{}{}{"-eis}{}{adj.2g.}{Que não se pode perscrutar; impenetrável.}{im.pers.cru.tá.vel}{0}
\verb{impersistente}{}{}{}{}{adj.2g.}{Que não é persistente; inconstante.}{im.per.sis.ten.te}{0}
\verb{impertérrito}{}{}{}{}{adj.}{Que não tem medo; corajoso, destemido.}{im.per.tér.ri.to}{0}
\verb{impertinência}{}{}{}{}{s.f.}{Qualidade de impertinente; desrespeito, insolência. (\textit{Sua impertinência pode atrapalhá"-lo no futuro.})}{im.per.ti.nên.cia}{0}
\verb{impertinência}{}{}{}{}{}{Ato de uma pessoa impertinente; inconveniência. (\textit{As pessoas não suportam impertinências de crianças mimadas.})}{im.per.ti.nên.cia}{0}
\verb{impertinente}{}{}{}{}{adj.2g.}{Que é desrespeitoso, inconveniente.}{im.per.ti.nen.te}{0}
\verb{impertinente}{}{}{}{}{}{Fora do assunto e do momento; descabido, inoportuno.}{im.per.ti.nen.te}{0}
\verb{imperturbável}{}{}{"-eis}{}{adj.2g.}{Que não é perturbável; a que nada pode abalar ou comover; inabalável, inalterável.}{im.per.tur.bá.vel}{0}
\verb{impérvio}{}{}{}{}{adj.}{Por onde não se pode passar; intransitável, impenetrável.}{im.pér.vio}{0}
\verb{impessoal}{}{}{"-ais}{}{adj.2g.}{Que não possui os atributos de pessoa.}{im.pes.so.al}{0}
\verb{impessoal}{}{}{"-ais}{}{}{Que não se refere ou não pertence a uma pessoa em particular.}{im.pes.so.al}{0}
\verb{impessoal}{}{Gram.}{"-ais}{}{}{Diz"-se do verbo que não admite sujeito.}{im.pes.so.al}{0}
\verb{impessoalidade}{}{}{}{}{s.f.}{Qualidade do que é impessoal.}{im.pes.so.a.li.da.de}{0}
\verb{impetigem}{}{Med.}{"-ens}{}{s.f.}{Impetigo.}{im.pe.ti.gem}{0}
\verb{impetigo}{}{Med.}{}{}{s.m.}{Doença cutânea, inflamatória bacteriana, caracterizada pelo aparecimento de pústulas.}{im.pe.ti.go}{0}
\verb{ímpeto}{}{}{}{}{s.m.}{Movimento rápido e violento. (\textit{Quando a mãe chamou, o menino se levantou num ímpeto.})}{ím.pe.to}{0}
\verb{ímpeto}{}{}{}{}{}{Vontade súbita e violenta; impulso.}{ím.pe.to}{0}
\verb{impetração}{}{}{"-ões}{}{s.f.}{Ato ou efeito de impetrar; rogo, súplica.}{im.pe.tra.ção}{0}
\verb{impetrante}{}{}{}{}{adj.2g.}{Diz"-se de quem impetra.}{im.pe.tran.te}{0}
\verb{impetrar}{}{}{}{}{v.t.}{Interpor um recurso judicial.}{im.pe.trar}{0}
\verb{impetrar}{}{}{}{}{}{Rogar, suplicar.}{im.pe.trar}{\verboinum{1}}
\verb{impetuosidade}{}{}{}{}{s.f.}{Qualidade do que é impetuoso.}{im.pe.tu.o.si.da.de}{0}
\verb{impetuosidade}{}{}{}{}{}{Violência, fúria.}{im.pe.tu.o.si.da.de}{0}
\verb{impetuosidade}{}{}{}{}{}{Entusiasmo, espalhafato.}{im.pe.tu.o.si.da.de}{0}
\verb{impetuoso}{ô}{}{"-osos ⟨ó⟩}{"-osa ⟨ó⟩}{adj.}{Que se move com ímpeto.}{im.pe.tu.o.so}{0}
\verb{impetuoso}{ô}{}{"-osos ⟨ó⟩}{"-osa ⟨ó⟩}{}{Que revela, em seu comportamento, ardor, violência; arrebatado, veemente.}{im.pe.tu.o.so}{0}
\verb{impiedade}{}{}{}{}{s.f.}{Falta de piedade; desumanidade, insensibilidade.}{im.pi.e.da.de}{0}
\verb{impiedoso}{ô}{}{"-osos ⟨ó⟩}{"-osa ⟨ó⟩}{adj.}{Que não tem piedade; sem compaixão; desumano.}{im.pi.e.do.so}{0}
\verb{impigem}{}{}{}{}{}{Var. de \textit{impingem}.}{im.pi.gem}{0}
\verb{impingem}{}{Med.}{"-ens}{}{s.f.}{Nome comum a várias doenças da pele.}{im.pin.gem}{0}
\verb{impingir}{}{}{}{}{v.t.}{Fazer acreditar numa coisa falsa.}{im.pin.gir}{0}
\verb{impingir}{}{}{}{}{}{Fazer aceitar ou receber contra a vontade; impor.}{im.pin.gir}{\verboinum{22}}
%\verb{}{}{}{}{}{}{}{}{0}
\verb{ímpio}{}{}{}{}{adj.}{Diz"-se de indivíduo que não tem fé; incrédulo.}{ím.pio}{0}
\verb{ímpio}{}{}{}{}{}{Diz"-se de indivíduo que não tem piedade; desumano.  }{ím.pio}{0}
\verb{implacável}{}{}{"-eis}{}{adj.2g.}{Que não se pode aplacar ou abrandar.}{im.pla.cá.vel}{0}
\verb{implacável}{}{}{"-eis}{}{}{Que não perdoa; inexorável, insensível.}{im.pla.cá.vel}{0}
\verb{implantação}{}{}{"-ões}{}{s.f.}{Ato ou efeito de implantar.}{im.plan.ta.ção}{0}
\verb{implantação}{}{}{"-ões}{}{}{Introdução de algo novo.}{im.plan.ta.ção}{0}
\verb{implantar}{}{}{}{}{v.t.}{Plantar alguma coisa em outra; inserir; arraigar.}{im.plan.tar}{0}
\verb{implantar}{}{}{}{}{}{Estabelecer, introduzir.}{im.plan.tar}{\verboinum{1}}
\verb{implante}{}{}{}{}{s.m.}{Implantação.}{im.plan.te}{0}
\verb{implante}{}{}{}{}{}{Matéria orgânica ou inorgânica que é inserida num animal ou num ser humano.}{im.plan.te}{0}
\verb{implementação}{}{}{"-ões}{}{s.f.}{Ato ou efeito de implementar; realização.}{im.ple.men.ta.ção}{0}
\verb{implementar}{}{}{}{}{v.t.}{Pôr em prática um plano, um projeto etc.}{im.ple.men.tar}{\verboinum{1}}
\verb{implemento}{}{}{}{}{s.m.}{Parte necessária para a realização de algo; complemento.}{im.ple.men.to}{0}
\verb{implemento}{}{}{}{}{}{Execução, cumprimento.}{im.ple.men.to}{0}
\verb{implicação}{}{}{"-ões}{}{s.f.}{Ato ou efeito de implicar.}{im.pli.ca.ção}{0}
\verb{implicação}{}{}{"-ões}{}{}{Consequência, encadeamento.}{im.pli.ca.ção}{0}
\verb{implicação}{}{Fig.}{"-ões}{}{}{Implicância, incompatibilidade.}{im.pli.ca.ção}{0}
\verb{implicância}{}{}{}{}{s.f.}{Ato ou efeito de implicar.}{im.pli.cân.cia}{0}
\verb{implicância}{}{Pop.}{}{}{}{Má vontade, birra, antipatia gratuita.}{im.pli.cân.cia}{0}
\verb{implicante}{}{}{}{}{adj.2g.}{Diz"-se do que implica, do que envolve contradição ou contrariedade.}{im.pli.can.te}{0}
\verb{implicante}{}{}{}{}{}{Diz"-se de quem demonstra implicância; que é birrento.}{im.pli.can.te}{0}
\verb{implicante}{}{Pop.}{}{}{}{Diz"-se de quem é provocador; briguento.}{im.pli.can.te}{0}
\verb{implicar}{}{}{}{}{v.t.}{Ter como consequência; acarretar.}{im.pli.car}{0}
\verb{implicar}{}{}{}{}{}{Dar a entender; supor.}{im.pli.car}{0}
\verb{implicar}{}{}{}{}{}{Tornar necessário.}{im.pli.car}{0}
\verb{implicar}{}{}{}{}{}{Provocar, hostilizar.}{im.pli.car}{0}
\verb{implicar}{}{}{}{}{}{Envolver alguém ou si mesmo em complicação, embaraço; comprometer"-se.}{im.pli.car}{\verboinum{2}}
\verb{implicativo}{}{}{}{}{s.m.}{Que implica, que faz ter determinada consequência.}{im.pli.ca.ti.vo}{0}
\verb{implícito}{}{}{}{}{adj.}{Que está envolvido, mas não expresso claramente.}{im.plí.ci.to}{0}
\verb{implícito}{}{}{}{}{}{Que não é expresso por palavras; subentendido.}{im.plí.ci.to}{0}
\verb{implodir}{}{}{}{}{v.t.}{Provocar a explosão de uma construção, de modo que suas partes não se espalhem, caindo num mesmo ponto central.}{im.plo.dir}{\verboinum{34}}
\verb{implorar}{}{}{}{}{v.t.}{Pedir com humildade; suplicar. (\textit{O homem implorou ao chefe para continuar trabalhando na empresa.})}{im.plo.rar}{\verboinum{1}}
\verb{implosão}{}{}{"-ões}{}{s.f.}{Ato ou efeito de implodir.}{im.plo.são}{0}
\verb{implume}{}{}{}{}{adj.2g.}{Sem plumas. (\textit{Os passarinhos nascem implumes.})}{im.plu.me}{0}
\verb{impolido}{}{}{}{}{adj.}{Que não recebeu polimento.}{im.po.li.do}{0}
\verb{impolido}{}{Fig.}{}{}{}{Que é grosseiro; descortês.}{im.po.li.do}{0}
\verb{impoluto}{}{}{}{}{adj.}{Que não é poluído; puro; sem manchas.}{im.po.lu.to}{0}
\verb{impoluto}{}{Fig.}{}{}{}{Que é honesto, virtuoso.}{im.po.lu.to}{0}
\verb{imponderado}{}{}{}{}{adj.}{Que não é ponderado; precipitado.}{im.pon.de.ra.do}{0}
\verb{imponderável}{}{}{"-eis}{}{adj.2g.}{Que não se pode pesar.}{im.pon.de.rá.vel}{0}
\verb{imponderável}{}{}{"-eis}{}{}{Que não se pode avaliar.}{im.pon.de.rá.vel}{0}
\verb{imponderável}{}{}{"-eis}{}{s.m.}{Circunstância ou fator cuja influência numa questão é imprevisível.}{im.pon.de.rá.vel}{0}
\verb{imponência}{}{}{}{}{s.f.}{Qualidade do que é imponente.}{im.po.nên.cia}{0}
\verb{imponente}{}{}{}{}{adj.2g.}{Que impõe admiração; majestoso.}{im.po.nen.te}{0}
\verb{imponente}{}{}{}{}{}{Arrogante, soberbo.}{im.po.nen.te}{0}
\verb{impontual}{}{}{"-ais}{}{adj.2g.}{Que não é pontual; que chega fora da hora combinada.}{im.pon.tu.al}{0}
\verb{impontualidade}{}{}{}{}{s.f.}{Qualidade de quem é impontual; falta de pontualidade.}{im.pon.tu.a.li.da.de}{0}
\verb{impopular}{}{}{}{}{adj.2g.}{Que não agrada ao povo.}{im.po.pu.lar}{0}
\verb{impopularidade}{}{}{}{}{s.f.}{Falta de popularidade; desprestígio, descrédito.}{im.po.pu.la.ri.da.de}{0}
\verb{impopularizar}{}{}{}{}{v.t.}{Tornar impopular.}{im.po.pu.la.ri.zar}{\verboinum{1}}
\verb{impor}{}{}{}{}{v.t.}{Tornar algo obrigatório para alguém ou para si mesmo.}{im.por}{0}
\verb{impor}{}{}{}{}{}{Aplicar pena, castigo etc.}{im.por}{\verboinum{60}}
\verb{importação}{}{}{"-ões}{}{s.f.}{Ato ou efeito de importar.}{im.por.ta.ção}{0}
\verb{importação}{}{}{"-ões}{}{}{A coisa importada.}{im.por.ta.ção}{0}
\verb{importado}{}{}{}{}{adj.}{Diz"-se daquilo que se importou, que veio de outro país ou de outra região.}{im.por.ta.do}{0}
\verb{importador}{ô}{}{}{}{adj.}{Que importa, que traz de fora do país.}{im.por.ta.dor}{0}
\verb{importadora}{ô}{}{}{}{s.f.}{Empresa que faz importações.}{im.por.ta.do.ra}{0}
\verb{importância}{}{}{}{}{s.f.}{Grande valor que se atribui a alguém ou a algo.}{im.por.tân.cia}{0}
\verb{importância}{}{}{}{}{}{Quantia em dinheiro.}{im.por.tân.cia}{0}
\verb{importante}{}{}{}{}{adj.2g.}{Que merece atenção por seu valor.}{im.por.tan.te}{0}
\verb{importar}{}{}{}{}{v.t.}{Fazer vir de outro país. (\textit{O Brasil importa produtos industrializados dos Estados Unidos.})}{im.por.tar}{0}
\verb{importar}{}{}{}{}{}{Atingir certo preço.}{im.por.tar}{0}
\verb{importar}{}{}{}{}{}{Ter valor para alguém; interessar.}{im.por.tar}{0}
\verb{importar}{}{}{}{}{v.pron.}{Mostrar interesse por pessoa ou coisa a que se dá valor; preocupar"-se. (\textit{A professora se importa com o aprendizado dos alunos.})}{im.por.tar}{\verboinum{1}}
\verb{importável}{}{}{"-eis}{}{adj.2g.}{Que pode ser importado.}{im.por.tá.vel}{0}
\verb{importe}{ó}{}{}{}{s.m.}{Quantidade total de alguma coisa; importância, soma.}{im.por.te}{0}
\verb{importunação}{}{}{"-ões}{}{s.f.}{Ato ou efeito de importunar.}{im.por.tu.na.ção}{0}
\verb{importunação}{}{}{"-ões}{}{}{Insistência enfadonha.}{im.por.tu.na.ção}{0}
\verb{importunação}{}{}{"-ões}{}{}{Desgosto, aborrecimento.}{im.por.tu.na.ção}{0}
\verb{importunar}{}{}{}{}{v.t.}{Aborrecer com alguma insistência; incomodar.}{im.por.tu.nar}{\verboinum{1}}
\verb{importuno}{}{}{}{}{adj.}{Que importuna, incomoda; impertinente, insuportável.}{im.por.tu.no}{0}
\verb{imposição}{}{}{"-ões}{}{s.f.}{Ato de impor, de obrigar.}{im.po.si.ção}{0}
\verb{imposição}{}{}{"-ões}{}{}{Determinação, ordem.}{im.po.si.ção}{0}
\verb{imposição}{}{}{"-ões}{}{}{A coisa imposta.}{im.po.si.ção}{0}
\verb{impossibilitar}{}{}{}{}{v.t.}{Tornar impossível ou irrealizável.}{im.pos.si.bi.li.tar}{0}
\verb{impossibilitar}{}{}{}{}{}{Fazer perder as forças ou a habilidade; incapacitar.}{im.pos.si.bi.li.tar}{\verboinum{1}}
\verb{impossível}{}{}{"-eis}{}{adj.2g.}{Que não pode acontecer; irrealizável.}{im.pos.sí.vel}{0}
\verb{impossível}{}{}{"-eis}{}{}{Em que não se pode acreditar; incrível, extraordinário.}{im.pos.sí.vel}{0}
\verb{impossível}{}{Fig.}{"-eis}{}{}{Que faz muitas travessuras; peralta; arteiro.}{im.pos.sí.vel}{0}
\verb{impostar}{}{}{}{}{v.t.}{Falar de maneira que a voz saia clara, sem tremor nem vacilação.}{im.pos.tar}{\verboinum{1}}
\verb{imposto}{ô}{}{"-s ⟨ó⟩}{"-a ⟨ó⟩}{adj.}{Que se impôs; colocado, posto.}{im.pos.to}{0}
\verb{imposto}{ô}{}{"-s ⟨ó⟩}{"-a ⟨ó⟩}{}{Que se obrigou a aceitar ou a realizar.}{im.pos.to}{0}
\verb{imposto}{ô}{}{"-s ⟨ó⟩}{"-a ⟨ó⟩}{s.m.}{Pagamento que se deve fazer ao governo pelos serviços prestados; tributo.}{im.pos.to}{0}
\verb{impostor}{ô}{}{}{}{adj.}{Diz"-se de indivíduo que tem ou pratica impostura, que engana os outros fingindo ser o que não é. (\textit{Aquele médico é um impostor.})}{im.pos.tor}{0}
\verb{impostura}{}{}{}{}{s.f.}{Artifício para iludir; embuste.}{im.pos.tu.ra}{0}
\verb{impostura}{}{}{}{}{}{Hipocrisia, fingimento.}{im.pos.tu.ra}{0}
\verb{impostura}{}{}{}{}{}{Vaidade ou presunção extrema; falsa superioridade.}{im.pos.tu.ra}{0}
\verb{impotável}{}{}{"-eis}{}{adj.2g.}{Que não é potável.}{im.po.tá.vel}{0}
\verb{impotência}{}{}{}{}{s.f.}{Falta de poder, meios ou capacidade física para realizar algo.}{im.po.tên.cia}{0}
\verb{impotência}{}{}{}{}{}{Qualidade de impotente.}{im.po.tên.cia}{0}
\verb{impotência}{}{}{}{}{}{Incapacidade masculina de obter ereção.}{im.po.tên.cia}{0}
\verb{impotente}{}{}{}{}{adj.2g.}{Que não tem poder ou capacidade física para realizar algo; débil, fraco.}{im.po.ten.te}{0}
\verb{impotente}{}{}{}{}{}{Que não é capaz de obter ereção.}{im.po.ten.te}{0}
\verb{impotente}{}{}{}{}{s.m.}{Homem que não tem capacidade de obter ereção.}{im.po.ten.te}{0}
\verb{impraticável}{}{}{"-eis}{}{adj.2g.}{Que não pode ser posto em prática; impossível, inexequível.}{im.pra.ti.cá.vel}{0}
\verb{impraticável}{}{}{"-eis}{}{}{Diz"-se de caminho ou estrada sem condições de ser percorrida; intransitável.}{im.pra.ti.cá.vel}{0}
\verb{imprecação}{}{}{"-ões}{}{s.f.}{Ato ou efeito de imprecar.}{im.pre.ca.ção}{0}
\verb{imprecação}{}{}{"-ões}{}{}{Desejo expresso de que algo ruim aconteça; praga, maldição.}{im.pre.ca.ção}{0}
\verb{imprecar}{}{}{}{}{v.t.}{Pedir males ou bens às divindades.}{im.pre.car}{0}
\verb{imprecar}{}{}{}{}{}{Rogar pragas; praguejar.}{im.pre.car}{\verboinum{2}}
\verb{imprecisão}{}{}{}{}{s.f.}{Qualidade de impreciso; falta de precisão.}{im.pre.ci.são}{0}
\verb{impreciso}{}{}{}{}{adj.}{Em que não há precisão ou clareza; vago, inexato.}{im.pre.ci.so}{0}
\verb{impregnação}{}{}{"-ões}{}{s.f.}{Ato ou efeito de impregnar.}{im.preg.na.ção}{0}
\verb{impregnação}{}{Biol.}{"-ões}{}{}{Fecundação do óvulo pelo espermatozoide.}{im.preg.na.ção}{0}
\verb{impregnar}{}{}{}{}{v.t.}{Fazer líquido penetrar em um corpo; embeber, encharcar.}{im.preg.nar}{0}
\verb{impregnar}{}{}{}{}{}{Infiltrar. (\textit{O mau cheiro impregnava o ambiente.})}{im.preg.nar}{0}
\verb{impregnar}{}{Fig.}{}{}{}{Influenciar, imbuir.}{im.preg.nar}{\verboinum{1}}
\verb{imprensa}{}{}{}{}{s.f.}{Máquina que imprime; prelo, prensa.}{im.pren.sa}{0}
\verb{imprensa}{}{}{}{}{}{A técnica da tipografia.}{im.pren.sa}{0}
\verb{imprensa}{}{}{}{}{}{O conjunto de publicações periódicas de determinada localidade.}{im.pren.sa}{0}
\verb{imprensa}{}{}{}{}{}{O conjunto dos meios de difusão de notícias, sejam eles impressos ou eletrônicos.}{im.pren.sa}{0}
\verb{imprensa}{}{Bras.}{}{}{}{O conjunto dos jornalistas na cobertura de determinado evento.}{im.pren.sa}{0}
\verb{imprensar}{}{}{}{}{v.t.}{Apertar na imprensa; imprimir.}{im.pren.sar}{0}
\verb{imprensar}{}{Por ext.}{}{}{}{Apertar como uma imprensa.}{im.pren.sar}{0}
\verb{imprensar}{}{Bras.}{}{}{}{Pressionar alguém exigindo determinada atitude; coagir.}{im.pren.sar}{\verboinum{1}}
\verb{imprescindível}{}{}{"-eis}{}{adj.2g.}{Que não se pode deixar de ter; indispensável.}{im.pres.cin.dí.vel}{0}
\verb{imprescritível}{}{}{"-eis}{}{adj.2g.}{Que não prescreve.}{im.pres.cri.tí.vel}{0}
\verb{impressão}{}{}{"-ões}{}{s.f.}{Reprodução de texto ou imagem por meio de chapas nas quais estão gravadas letras e outros sinais. (\textit{Pedimos um orçamento para a impressão de um livro.})}{im.pres.são}{0}
\verb{impressão}{}{}{"-ões}{}{}{Marca ou sinal deixado pela impressão de um corpo sobre o outro. (\textit{O documento de identidade traz a impressão digital do dono.})}{im.pres.são}{0}
\verb{impressão}{}{}{"-ões}{}{}{Marca deixada na mente por um acontecimento. (\textit{O candidato à vaga de assessor causou boa impressão no diretor.})}{im.pres.são}{0}
\verb{impressão}{}{}{"-ões}{}{}{Opinião não confirmada. (\textit{Tenho a impressão de que ele não chegará a tempo.})}{im.pres.são}{0}
\verb{impressionabilidade}{}{}{}{}{s.f.}{Qualidade de impressionável.}{im.pres.si.o.na.bi.li.da.de}{0}
\verb{impressionante}{}{}{}{}{adj.2g.}{Que impressiona, que causa estranheza; tocante, comovente.}{im.pres.si.o.nan.te}{0}
\verb{impressionar}{}{}{}{}{v.t.}{Produzir impressão moral; abalar, comover.}{im.pres.si.o.nar}{\verboinum{1}}
\verb{impressionável}{}{}{"-eis}{}{adj.2g.}{Diz"-se de material que pode receber impressões.}{im.pres.si.o.ná.vel}{0}
\verb{impressionável}{}{}{"-eis}{}{}{Diz"-se de pessoa passível de receber impressões psicológicas.}{im.pres.si.o.ná.vel}{0}
\verb{impressionismo}{}{Art.}{}{}{s.m.}{Movimento artístico originado entre pintores franceses do final do século \textsc{xix} caracterizado pelo uso de efeitos de luz e movimento e poucos contornos.}{im.pres.si.o.nis.mo}{0}
\verb{impressionismo}{}{Liter.}{}{}{}{Crítica caracterizada pelo subjetivismo.}{im.pres.si.o.nis.mo}{0}
\verb{impressionista}{}{}{}{}{adj.2g.}{Relativo ao Impressionismo.}{im.pres.si.o.nis.ta}{0}
\verb{impressionista}{}{}{}{}{}{Baseado meramente em impressões pessoais.}{im.pres.si.o.nis.ta}{0}
\verb{impressionista}{}{}{}{}{s.2g.}{Artista seguidor do Impressionismo.}{im.pres.si.o.nis.ta}{0}
\verb{impresso}{é}{}{}{}{adj.}{Que se imprimiu.}{im.pres.so}{0}
\verb{impresso}{é}{Fig.}{}{}{}{Que se fixou, gravou.}{im.pres.so}{0}
\verb{impresso}{é}{}{}{}{s.m.}{Qualquer produto impresso.}{im.pres.so}{0}
\verb{impresso}{é}{}{}{}{}{Papel impresso para uso específico em escritórios e repartições.}{im.pres.so}{0}
\verb{impressor}{ô}{}{}{}{adj.}{Que serve para imprimir.}{im.pres.sor}{0}
\verb{impressor}{ô}{}{}{}{s.m.}{Indivíduo que opera máquina gráfica.}{im.pres.sor}{0}
\verb{impressor}{ô}{}{}{}{}{Proprietário de oficina gráfica.}{im.pres.sor}{0}
\verb{impressora}{ô}{}{}{}{s.f.}{Máquina que imprime.}{im.pres.so.ra}{0}
\verb{imprestável}{}{}{"-eis}{}{adj.2g.}{Que não tem serventia ou utilidade.}{im.pres.tá.vel}{0}
\verb{imprestável}{}{}{"-eis}{}{}{Diz"-se de pessoa que não é prestativa.}{im.pres.tá.vel}{0}
\verb{impreterível}{}{}{"-eis}{}{adj.2g.}{Que não pode deixar de ser feito, executado ou cumprido.}{im.pre.te.rí.vel}{0}
\verb{imprevidência}{}{}{}{}{s.f.}{Falta de previdência.}{im.pre.vi.dên.cia}{0}
\verb{imprevidente}{}{}{}{}{adj.2g.}{Que não é previdente; imprudente, descuidado.}{im.pre.vi.den.te}{0}
\verb{imprevisão}{}{}{"-ões}{}{s.f.}{Falta de previsão; negligência.}{im.pre.vi.são}{0}
\verb{imprevisível}{}{}{"-eis}{}{adj.2g.}{Que não pode ser previsto; fortuito, casual.}{im.pre.vi.sí.vel}{0}
\verb{imprevisível}{}{}{"-eis}{}{s.m.}{Aquilo que não se pode prever.}{im.pre.vi.sí.vel}{0}
\verb{imprevisto}{}{}{}{}{adj.}{Que não é previsto; inesperado.}{im.pre.vis.to}{0}
\verb{imprevisto}{}{}{}{}{s.m.}{Aquilo que não se prevê.}{im.pre.vis.to}{0}
\verb{imprimir}{}{}{}{}{v.t.}{Fazer pressão sobre alguma coisa para deixar nela marca ou sinal.}{im.pri.mir}{0}
\verb{imprimir}{}{}{}{}{}{Fazer letras ou desenhos aparecer sobre a superfície de algum material. (\textit{Vamos imprimir um livro de poemas da escola.})}{im.pri.mir}{0}
\verb{imprimir}{}{}{}{}{}{Fazer uma pessoa passar a ter alguma qualidade ou sentimento; infundir. (\textit{A professora imprimiu nos alunos o gosto pela leitura.})}{im.pri.mir}{\verboinum{18}}
\verb{improbidade}{}{}{}{}{s.f.}{Ausência de probidade; mau caráter; desonestidade.}{im.pro.bi.da.de}{0}
\verb{improbidade}{}{}{}{}{}{Ação perversa; maldade.}{im.pro.bi.da.de}{0}
\verb{ímprobo}{}{}{}{}{adj.}{Que não tem probidade; desonesto.}{ím.pro.bo}{0}
\verb{ímprobo}{}{}{}{}{}{Exaustivo, árduo, fatigante, difícil.}{ím.pro.bo}{0}
\verb{improcedência}{}{}{}{}{s.f.}{Qualidade de improcedente.}{im.pro.ce.dên.cia}{0}
\verb{improcedente}{}{}{}{}{adj.2g.}{Que não procede, que não se justifica.}{im.pro.ce.den.te}{0}
\verb{improcedente}{}{}{}{}{}{Incoerente, ilógico.}{im.pro.ce.den.te}{0}
\verb{improdutivo}{}{}{}{}{adj.}{Que não é produtivo.}{im.pro.du.ti.vo}{0}
\verb{improferível}{}{}{"-eis}{}{adj.2g.}{Que não se pode proferir.}{im.pro.fe.rí.vel}{0}
\verb{improfícuo}{}{}{}{}{adj.}{Não profícuo, que não produz o resultado esperado.}{im.pro.fí.cu.o}{0}
\verb{improfícuo}{}{}{}{}{}{Fracassado, vão, baldado.}{im.pro.fí.cu.o}{0}
\verb{impropério}{}{}{}{}{s.m.}{Ofensa, insulto.}{im.pro.pé.rio}{0}
\verb{impropério}{}{}{}{}{}{Repreensão, censura.}{im.pro.pé.rio}{0}
\verb{impropriedade}{}{}{}{}{s.f.}{Qualidade de impróprio.}{im.pro.pri.e.da.de}{0}
\verb{impropriedade}{}{}{}{}{}{Ação, gesto ou dito impróprio; lapso, incoerência.}{im.pro.pri.e.da.de}{0}
\verb{impróprio}{}{}{}{}{adj.}{Que não é próprio, adequado, justo ou exato.}{im.pró.prio}{0}
\verb{impróprio}{}{}{}{}{}{Indecoroso, indecente, inconveniente.}{im.pró.prio}{0}
\verb{improrrogável}{}{}{"-eis}{}{adj.2g.}{Que não se pode prorrogar.}{im.pror.ro.gá.vel}{0}
\verb{improvável}{}{}{"-eis}{}{adj.2g.}{Que tem pouca probabilidade de ocorrer.}{im.pro.vá.vel}{0}
\verb{improvável}{}{}{"-eis}{}{}{Que não se pode provar.}{im.pro.vá.vel}{0}
\verb{improvidência}{}{}{}{}{s.f.}{Qualidade de improvidente.}{im.pro.vi.dên.cia}{0}
\verb{improvidente}{}{}{}{}{adj.2g.}{Que não é providente.}{im.pro.vi.den.te}{0}
\verb{improvidente}{}{}{}{}{}{Negligente, desleixado, desmazelado.}{im.pro.vi.den.te}{0}
\verb{improvidente}{}{Por ext.}{}{}{}{Esbanjador, perdulário.}{im.pro.vi.den.te}{0}
\verb{improvisação}{}{}{"-ões}{}{s.f.}{Ato ou efeito de improvisar.}{im.pro.vi.sa.ção}{0}
\verb{improvisação}{}{}{"-ões}{}{}{Aquilo que é feito de improviso.}{im.pro.vi.sa.ção}{0}
\verb{improvisação}{}{Mús.}{"-ões}{}{}{Prática musical em que o músico cria enquanto executa.}{im.pro.vi.sa.ção}{0}
\verb{improvisar}{}{}{}{}{v.t.}{Arranjar ou preparar às pressas, sem providências antecipadas e com os recursos disponíveis.}{im.pro.vi.sar}{0}
\verb{improvisar}{}{}{}{}{}{Escrever, compor, falar sem preparação prévia.}{im.pro.vi.sar}{0}
\verb{improvisar}{}{}{}{}{}{Forjar, falsificar.}{im.pro.vi.sar}{\verboinum{1}}
\verb{improviso}{}{}{}{}{adj.}{Súbito, improvisado, repentino.}{im.pro.vi.so}{0}
\verb{improviso}{}{}{}{}{s.m.}{Aquilo que é feito sem preparação.}{im.pro.vi.so}{0}
\verb{improviso}{}{Mús.}{}{}{}{Criação de uma música durante a sua própria execução.}{im.pro.vi.so}{0}
\verb{imprudência}{}{}{}{}{s.f.}{Qualidade de imprudente.}{im.pru.dên.cia}{0}
\verb{imprudência}{}{}{}{}{}{Ato ou dito imprudente.}{im.pru.dên.cia}{0}
\verb{imprudente}{}{}{}{}{adj.2g.}{Que não é prudente, cauteloso.}{im.pru.den.te}{0}
\verb{impúbere}{}{}{}{}{adj.2g.}{Que ainda não atingiu a puberdade.}{im.pú.be.re}{0}
\verb{impúbere}{}{Jur.}{}{}{}{Diz"-se de menor entre zero e 16 anos, considerado juridicamente incapaz por não responder civilmente por seus atos.}{im.pú.be.re}{0}
\verb{impudência}{}{}{}{}{s.f.}{Qualidade de impudente.}{im.pu.dên.cia}{0}
\verb{impudência}{}{}{}{}{}{Ato ou dito impudente.}{im.pu.dên.cia}{0}
\verb{impudente}{}{}{}{}{adj.2g.}{Que não tem pudor; despudorado.}{im.pu.den.te}{0}
\verb{impudicícia}{}{}{}{}{s.f.}{Falta de pudicícia.}{im.pu.di.cí.cia}{0}
\verb{impudicícia}{}{}{}{}{}{Ato ou dito impudico.}{im.pu.di.cí.cia}{0}
\verb{impudico}{}{}{}{}{adj.}{Que não tem pudor.}{im.pu.di.co}{0}
\verb{impudor}{ô}{}{}{}{s.m.}{Falta de pudor.}{im.pu.dor}{0}
\verb{impugnação}{}{}{"-ões}{}{s.f.}{Ato ou efeito de impugnar; contestação.}{im.pug.na.ção}{0}
\verb{impugnação}{}{Jur.}{"-ões}{}{}{Petição que se apresenta visando anular os efeitos de uma decisão judicial ou administrativa.}{im.pug.na.ção}{0}
\verb{impugnar}{}{}{}{}{v.t.}{Contestar, refutar, contrariar, argumentar contra algo.}{im.pug.nar}{0}
\verb{impugnar}{}{}{}{}{}{Opor"-se, resistir.}{im.pug.nar}{\verboinum{1}}
\verb{impugnativo}{}{}{}{}{adj.}{Que serve para impugnar.}{im.pug.na.ti.vo}{0}
\verb{impugnativo}{}{}{}{}{}{Próprio de quem impugna.}{im.pug.na.ti.vo}{0}
\verb{impulsão}{}{}{"-ões}{}{s.f.}{Impulso.}{im.pul.são}{0}
\verb{impulsionar}{}{}{}{}{v.t.}{Dar impulso; empurrar, impelir.}{im.pul.si.o.nar}{0}
\verb{impulsionar}{}{Fig.}{}{}{}{Dar incentivo; motivar, estimular.}{im.pul.si.o.nar}{\verboinum{1}}
\verb{impulsividade}{}{}{}{}{s.f.}{Qualidade de impulsivo.}{im.pul.si.vi.da.de}{0}
\verb{impulsivo}{}{}{}{}{adj.}{Que dá impulso.}{im.pul.si.vo}{0}
\verb{impulsivo}{}{}{}{}{}{Diz"-se de quem cede totalmente a impulsos momentâneos e age sem pensar.}{im.pul.si.vo}{0}
\verb{impulsivo}{}{}{}{}{}{Diz"-se de quem se empolga ou enfurece facilmente.}{im.pul.si.vo}{0}
\verb{impulso}{}{}{}{}{s.m.}{Força que se faz para movimentar alguma coisa.}{im.pul.so}{0}
\verb{impulso}{}{Fig.}{}{}{}{Grande vontade de fazer alguma coisa de repente; estímulo, incitamento, instigação.   }{im.pul.so}{0}
\verb{impulsor}{ô}{}{}{}{adj.}{Que impulsiona.}{im.pul.sor}{0}
\verb{impune}{}{}{}{}{adj.2g.}{Que não recebe ou não recebeu a devida punição; não punido.}{im.pu.ne}{0}
\verb{impunidade}{}{}{}{}{s.f.}{Qualidade de impune.}{im.pu.ni.da.de}{0}
\verb{impunidade}{}{}{}{}{}{Ausência de punição.}{im.pu.ni.da.de}{0}
\verb{impureza}{ê}{}{}{}{s.f.}{Falta de pureza.}{im.pu.re.za}{0}
\verb{impureza}{ê}{}{}{}{}{Qualidade de impuro.}{im.pu.re.za}{0}
\verb{impureza}{ê}{}{}{}{}{Material ou substância que compromete a pureza de algo.}{im.pu.re.za}{0}
\verb{impureza}{ê}{Fig.}{}{}{}{Falta de pudor, de castidade; imoralidade.}{im.pu.re.za}{0}
\verb{impuro}{}{}{}{}{adj.}{Que não é puro.}{im.pu.ro}{0}
\verb{impuro}{}{}{}{}{}{Contaminado.}{im.pu.ro}{0}
\verb{impuro}{}{}{}{}{}{Adulterado.}{im.pu.ro}{0}
\verb{impuro}{}{}{}{}{}{Impudico, imoral, devasso.}{im.pu.ro}{0}
\verb{imputação}{}{}{"-ões}{}{}{Ato ou efeito de imputar.}{im.pu.ta.ção}{0}
\verb{imputação}{}{}{"-ões}{}{s.f.}{A acusação feita a alguém.}{im.pu.ta.ção}{0}
\verb{imputação}{}{}{"-ões}{}{}{Aquilo que é imputado.}{im.pu.ta.ção}{0}
\verb{imputação}{}{Fig.}{"-ões}{}{}{Consciência da responsabilidade sobre o ato praticado.}{im.pu.ta.ção}{0}
\verb{imputar}{}{}{}{}{v.t.}{Atribuir responsabilidade por algum ato cometido; incriminar, assacar.}{im.pu.tar}{\verboinum{1}}
%\verb{}{}{}{}{}{}{}{}{0}
\verb{imputrescível}{}{}{"-eis}{}{adj.2g.}{Que não é putrescível, que não apodrece.}{im.pu.tres.cí.vel}{0}
\verb{imundícia}{}{}{}{}{s.f.}{Qualidade de imundo.}{i.mun.dí.cia}{0}
\verb{imundícia}{}{}{}{}{}{Falta de limpeza.}{i.mun.dí.cia}{0}
\verb{imundícia}{}{}{}{}{}{Sujeira.}{i.mun.dí.cia}{0}
\verb{imundícia}{}{Bras.}{}{}{}{Grande quantidade; abundância.}{i.mun.dí.cia}{0}
\verb{imundície}{}{}{}{}{}{Var. de \textit{imundícia}.}{i.mun.dí.cie}{0}
\verb{imundo}{}{}{}{}{adj.}{Muito sujo; porco, emporcalhado.}{i.mun.do}{0}
\verb{imundo}{}{}{}{}{}{Imoral, obsceno, indecente.}{i.mun.do}{0}
\verb{imune}{}{}{}{}{adj.2g.}{Que apresenta imunidade.}{i.mu.ne}{0}
\verb{imune}{}{Fig.}{}{}{}{Que não é atingido.}{i.mu.ne}{0}
\verb{imune}{}{}{}{}{}{Relativo a imunidade.}{i.mu.ne}{0}
\verb{imune}{}{}{}{}{}{Que é livre de impostos, encargos ou deveres.}{i.mu.ne}{0}
\verb{imunidade}{}{}{}{}{s.f.}{Qualidade de imune.}{i.mu.ni.da.de}{0}
\verb{imunidade}{}{Jur.}{}{}{}{Conjunto de isenções de ônus ou encargos concedidos a pessoas em função de cargo ou função exercida.}{i.mu.ni.da.de}{0}
\verb{imunidade}{}{Biol.}{}{}{}{Resistência de um organismo vivo a agentes infecciosos.}{i.mu.ni.da.de}{0}
\verb{imunização}{}{}{ões}{}{s.f.}{Ato ou efeito de imunizar.}{i.mu.ni.za.ção}{0}
\verb{imunizar}{}{}{}{}{v.t.}{Tornar imune, livre, resguardado.}{i.mu.ni.zar}{0}
\verb{imunizar}{}{}{}{}{}{Tornar um organismo imune, resistente ao contágio de certas doenças, ou à ação de alguns venenos.}{i.mu.ni.zar}{\verboinum{1}}
\verb{imunodeficiência}{}{Med.}{}{}{s.f.}{Deficiência do organismo de se defender de agentes estranhos, especialmente aqueles que podem provocar doenças, como vírus, bactérias etc.  }{i.mu.no.de.fi.ci.ên.cia}{0}
\verb{imunoglobulina}{}{Bioquím.}{}{}{s.f.}{Grupo de proteínas produzido pelos plasmócitos dos linfócitos \textsc{b}, responsável pela defesa do organismo às substâncias estranhas; anticorpo.}{i.mu.no.glo.bu.li.na}{0}
\verb{imunologia}{}{Med.}{}{}{s.f.}{Ramo da medicina que estuda os mecanismos de defesa do organismo contra antígenos.}{i.mu.no.lo.gi.a}{0}
\verb{imutabilidade}{}{}{}{}{s.f.}{Qualidade ou condição de imutável.}{i.mu.ta.bi.li.da.de}{0}
\verb{imutável}{}{}{"-eis}{}{adj.2g.}{Diz"-se do que não está sujeito a mudança; imudável, inalterável, permanente.}{i.mu.tá.vel}{0}
\verb{In}{}{Quím.}{}{}{}{Símb. do \textit{índio}.}{In}{0}
\verb{inabalável}{}{}{"-eis}{}{adj.2g.}{Diz"-se do que não pode ser abalado.}{i.na.ba.lá.vel}{0}
\verb{inabalável}{}{Fig.}{"-eis}{}{}{Firmemente apoiado; fixo, arraigado.}{i.na.ba.lá.vel}{0}
\verb{inabalável}{}{Fig.}{"-eis}{}{}{Inquebrantável, inflexível.}{i.na.ba.lá.vel}{0}
\verb{inábil}{}{}{"-eis}{}{adj.2g.}{Diz"-se do que não é hábil; inapto, desajeitado.}{i.ná.bil}{0}
\verb{inábil}{}{Jur.}{"-eis}{}{}{Que não tem capacidade legal ou validade jurídica; incapaz.}{i.ná.bil}{0}
\verb{inabilidade}{}{}{}{}{s.f.}{Qualidade ou condição de inábil.}{i.na.bi.li.da.de}{0}
\verb{inabilitar}{}{}{}{}{v.t.}{Deixar uma pessoa sem condições de fazer algo; incapacitar.}{i.na.bi.li.tar}{\verboinum{1}}
\verb{inabitado}{}{}{}{}{adj.}{Que não é habitado; despovoado, deserto.}{i.na.bi.ta.do}{0}
\verb{inabitável}{}{}{"-eis}{}{adj.2g.}{Que não se pode habitar ou não tem condições de ser habitado.}{i.na.bi.tá.vel}{0}
\verb{inabitual}{}{}{"-ais}{}{adj.2g.}{Que não é habitual; insólito.}{i.na.bi.tu.al}{0}
\verb{inabordável}{}{}{"-eis}{}{adj.2g.}{Que não se pode abordar.}{i.na.bor.dá.vel}{0}
\verb{inacabado}{}{}{}{}{adj.}{Que não está ou não foi acabado; incompleto.}{i.na.ca.ba.do}{0}
\verb{inacabável}{}{}{"-eis}{}{adj.2g.}{Que não se pode acabar; interminável.}{i.na.ca.bá.vel}{0}
\verb{inação}{}{}{"-ões}{}{s.f.}{Falta de ação; inércia, ociosidade.}{i.na.ção}{0}
\verb{inação}{}{Por ext.}{"-ões}{}{}{Hesitação, irresolução, indecisão.}{i.na.ção}{0}
\verb{inaceitável}{}{}{"-eis}{}{adj.2g.}{Que não se pode ou não se deve aceitar; inadmissível.}{i.na.cei.tá.vel}{0}
\verb{inacessibilidade}{}{}{}{}{s.f.}{Qualidade ou condição do que é inacessível.}{i.na.ces.si.bi.li.da.de}{0}
\verb{inacessível}{}{}{"-eis}{}{adj.2g.}{Que não dá ou não tem acesso; a que não se pode chegar; inatingível.}{i.na.ces.sí.vel}{0}
\verb{inacessível}{}{}{"-eis}{}{}{Intratável, insociável.   }{i.na.ces.sí.vel}{0}
\verb{inacessível}{}{}{"-eis}{}{}{Que não se pode compreender,  impenetrável.}{i.na.ces.sí.vel}{0}
\verb{inacessível}{}{}{"-eis}{}{}{Que não está sujeito; imune, isento.}{i.na.ces.sí.vel}{0}
\verb{inacreditável}{}{}{"-eis}{}{adj.2g.}{Em que não se pode acreditar; incrível.}{i.na.cre.di.tá.vel}{0}
\verb{inacusável}{}{}{"-eis}{}{adj.2g.}{Que não se pode acusar.}{i.na.cu.sá.vel}{0}
\verb{inadaptação}{}{}{"-ões}{}{s.f.}{Dificuldade ou incapacidade de adaptação.}{i.na.dap.ta.ção}{0}
\verb{inadaptável}{}{}{"-eis}{}{adj.2g.}{Que não se adapta ou não se pode adaptar.}{i.na.dap.tá.vel}{0}
\verb{inadequação}{}{}{"-ões}{}{s.f.}{Falta ou incapacidade de adequação.}{i.na.de.qua.ção}{0}
\verb{inadequado}{}{}{}{}{adj.}{Que não é adequado; impróprio.}{i.na.de.qua.do}{0}
\verb{inadiável}{}{}{"-eis}{}{adj.2g.}{Que não se pode adiar; impreterível, improrrogável.}{i.na.di.á.vel}{0}
\verb{inadimplemento}{}{Jur.}{}{}{s.m.}{Falta de cumprimento de um contrato ou de parte dele; descumprimento, inadimplência.}{i.na.dim.ple.men.to}{0}
\verb{inadimplência}{}{}{}{}{s.f.}{Falta de cumprimento de um contrato ou de uma das suas condições; descumprimento.}{i.na.dim.plên.cia}{0}
\verb{inadimplente}{}{Jur.}{}{}{adj.2g.}{Diz"-se do devedor que não cumpre suas obrigações contratuais; descumpridor.  }{i.na.dim.plen.te}{0}
\verb{inadmissível}{}{}{"-eis}{}{adj.2g.}{Que não se pode admitir; inaceitável.}{i.nad.mis.sí.vel}{0}
\verb{inadvertência}{}{}{}{}{s.f.}{Falta de advertência; descuido, imprevidência, negligência.}{i.nad.ver.tên.cia}{0}
\verb{inadvertido}{}{}{}{}{adj.}{Que não foi advertido, avisado; desavisado.}{i.nad.ver.ti.do}{0}
\verb{inadvertido}{}{}{}{}{}{Que foi feito sem reflexão; descuidado, distraído.}{i.nad.ver.ti.do}{0}
\verb{inafiançável}{}{}{"-eis}{}{adj.2g.}{Que não é afiançável.}{i.na.fi.an.çá.vel}{0}
\verb{inafiançável}{}{Jur.}{"-eis}{}{}{Que não admite livramento por fiança.}{i.na.fi.an.çá.vel}{0}
\verb{inalação}{}{}{"-ões}{}{s.f.}{Ato ou efeito de inalar; aspirar, inspirar.}{i.na.la.ção}{0}
\verb{inalação}{}{}{"-ões}{}{}{Absorção de medicamento ou droga por via respiratória.}{i.na.la.ção}{0}
\verb{inalador}{ô}{}{}{}{adj.}{Que serve para fazer inalação.}{i.na.la.dor}{0}
\verb{inalante}{}{}{}{}{adj.2g.}{Que é absorvido por inalação.}{i.na.lan.te}{0}
\verb{inalante}{}{Med.}{}{}{s.m.}{Medicamento próprio para inalação.}{i.na.lan.te}{0}
\verb{inalar}{}{}{}{}{v.t.}{Absorver pelas vias respiratórias; aspirar.}{i.na.lar}{\verboinum{1}}
\verb{inalienável}{}{}{"-eis}{}{adj.2g.}{Que não pode ser vendido ou cedido; intransferível. }{i.na.li.e.ná.vel}{0}
\verb{inalterado}{}{}{}{}{adj.}{Que não foi alterado, não sofreu modificação.}{i.nal.te.ra.do}{0}
\verb{inalterado}{}{Por ext.}{}{}{}{Imperturbado, sereno, tranquilo.}{i.nal.te.ra.do}{0}
\verb{inalterável}{}{}{"-eis}{}{adj.2g.}{Que não se pode alterar; imutável, constante.}{i.nal.te.rá.vel}{0}
\verb{inalterável}{}{Por ext.}{"-eis}{}{}{Inabalável, imperturbável, impassível.}{i.nal.te.rá.vel}{0}
\verb{inambu}{}{Zool.}{}{}{s.2g.}{Inhambu.}{i.nam.bu}{0}
\verb{inamissível}{}{}{"-eis}{}{adj.2g.}{Que não se pode perder.}{i.na.mis.sí.vel}{0}
\verb{inamistoso}{ô}{}{"-osos ⟨ó⟩}{"-osa ⟨ó⟩}{adj.}{Que não é próprio de amigo; hostil, inimigo.}{i.na.mis.to.so}{0}
\verb{inamovível}{}{}{"-eis}{}{adj.2g.}{Que não pode ser deslocado ou removido.}{i.na.mo.ví.vel}{0}
\verb{inamovível}{}{}{"-eis}{}{}{Que não pode ser destituído de um posto ou cargo.}{i.na.mo.ví.vel}{0}
\verb{inane}{}{}{}{}{adj.2g.}{Que não contém nada; vazio, oco.}{i.na.ne}{0}
\verb{inanição}{}{}{"-ões}{}{s.f.}{Enfraquecimento agudo por falta de alimentação.}{i.na.ni.ção}{0}
\verb{inanidade}{}{}{}{}{s.f.}{Qualidade ou condição de inane. }{i.na.ni.da.de}{0}
\verb{inanimado}{}{}{}{}{adj.}{Que não tem alma.}{i.na.ni.ma.do}{0}
\verb{inanimado}{}{}{}{}{}{Que não tem ânimo, vivacidade.}{i.na.ni.ma.do}{0}
\verb{inanimado}{}{}{}{}{}{Desfalecido, desmaiado.}{i.na.ni.ma.do}{0}
\verb{inanimado}{}{}{}{}{}{Diz"-se do que não tem ou nunca teve vida.}{i.na.ni.ma.do}{0}
\verb{inanir}{}{Desus.}{}{}{v.t.}{Reduzir ao estado de inanição.}{i.na.nir}{\verboinum{34}}
\verb{inapagável}{}{}{"-eis}{}{adj.2g.}{Que não se pode apagar, extinguir; indelével.}{i.na.pa.gá.vel}{0}
\verb{inapelável}{}{}{"-eis}{}{adj.2g.}{De que não se pode apelar, irrecorrível.}{i.na.pe.lá.vel}{0}
\verb{inapetência}{}{}{}{}{s.f.}{Falta de apetite; anorexia.}{i.na.pe.tên.cia}{0}
\verb{inapetência}{}{Por ext.}{}{}{}{Ausência de vontade, desejo.}{i.na.pe.tên.cia}{0}
\verb{inapetente}{}{}{}{}{adj.2g.}{Que não tem apetite; que não tem desejo.}{i.na.pe.ten.te}{0}
\verb{inaplicável}{}{}{"-eis}{}{adj.2g.}{Que não pode ser aplicado.}{i.na.pli.cá.vel}{0}
\verb{inapreciável}{}{}{"-eis}{}{adj.2g.}{Que não se pode apreciar, avaliar, dada a sua insignificância ou pequenez.}{i.na.pre.ci.á.vel}{0}
\verb{inapreciável}{}{}{"-eis}{}{}{Que não pode ser estimado, em apreço ou dinheiro, dado o seu imenso valor.}{i.na.pre.ci.á.vel}{0}
\verb{inapreensível}{}{}{"-eis}{}{adj.2g.}{Que não pode ser apreendido; incompreensível.}{i.na.pre.en.sí.vel}{0}
\verb{inaproveitável}{}{}{"-eis}{}{adj.2g.}{Que não se pode aproveitar. }{i.na.pro.vei.tá.vel}{0}
\verb{inaptidão}{}{}{"-ões}{}{s.f.}{Qualidade ou estado de inapto; incapacidade, inabilidade.}{i.nap.ti.dão}{0}
\verb{inapto}{}{}{}{}{adj.}{Que não tem aptidão; incapaz, inábil.}{i.nap.to}{0}
\verb{inarmônico}{}{}{}{}{adj.}{Que não é harmonioso; desarmônico.}{i.nar.mô.ni.co}{0}
\verb{inarmonioso}{ô}{}{"-osos ⟨ó⟩}{"-osa ⟨ó⟩}{adj.}{Sem harmonia.}{i.nar.mo.ni.o.so}{0}
\verb{inarrável}{}{}{"-eis}{}{adj.2g.}{Que não pode ser narrado; inenarrável.}{i.nar.rá.vel}{0}
\verb{inarticulado}{}{}{}{}{adj.}{Que não foi articulado.}{i.nar.ti.cu.la.do}{0}
\verb{inarticulado}{}{}{}{}{}{Pronunciado com dificuldade.}{i.nar.ti.cu.la.do}{0}
\verb{inarticulável}{}{}{"-eis}{}{adj.2g.}{Diz"-se de som que não pode ser articulado, pronunciado.}{i.nar.ti.cu.lá.vel}{0}
\verb{inassimilável}{}{}{"-eis}{}{adj.2g.}{Que não pode ser assimilado.}{i.nas.si.mi.lá.vel}{0}
\verb{inatacável}{}{}{"-eis}{}{adj.2g.}{Que não se pode atacar.}{i.na.ta.cá.vel}{0}
\verb{inatacável}{}{Por ext.}{"-eis}{}{}{Que não se pode censurar ou criticar;  irrepreensível, irreprochável.}{i.na.ta.cá.vel}{0}
\verb{inatingível}{}{}{"-eis}{}{adj.2g.}{Que não se pode atingir. }{i.na.tin.gí.vel}{0}
\verb{inatividade}{}{}{}{}{s.f.}{Qualidade ou estado de inativo; inércia.}{i.na.ti.vi.da.de}{0}
\verb{inatividade}{}{Jur.}{}{}{}{Situação do funcionário público que se encontra afastado de suas funções com salário mensal, por invalidez permanente, idade avançada ou tempo de serviço.}{i.na.ti.vi.da.de}{0}
\verb{inativo}{}{}{}{}{adj.}{Que não está ativo; parado, inerte, paralisado. }{i.na.ti.vo}{0}
\verb{inativo}{}{}{}{}{s.m.}{Funcionário, empregado ou militar retirado do serviço efetivo; aposentado; reformado; jubilado.}{i.na.ti.vo}{0}
\verb{inato}{}{}{}{}{adj.}{Que nasce com a pessoa; congênito. }{i.na.to}{0}
\verb{inaudito}{}{}{}{}{adj.}{De que nunca se ouviu falar; incrível, extraordinário.}{i.nau.di.to}{0}
\verb{inaudível}{}{}{"-eis}{}{adj.2g.}{Diz"-se de som cujo volume é muito baixo para ser ouvido, ou cuja frequência não pode ser percebida pelo ouvido humano.}{i.nau.dí.vel}{0}
\verb{inauguração}{}{}{"-ões}{}{s.f.}{Ato ou efeito de inaugurar.}{i.nau.gu.ra.ção}{0}
\verb{inaugural}{}{}{"-ais}{}{adj.2g.}{Relativo a inauguração.}{i.nau.gu.ral}{0}
\verb{inaugural}{}{}{"-ais}{}{}{Que inaugura, dá início; inicial.}{i.nau.gu.ral}{0}
\verb{inaugurar}{}{}{}{}{v.t.}{Apresentar algo ao público pela primeira vez; estreia.}{i.nau.gu.rar}{\verboinum{1}}
\verb{inautêntico}{}{}{}{}{adj.}{Que não é autêntico; falso, espúrio.}{i.nau.tên.ti.co}{0}
\verb{inavegável}{}{}{"-eis}{}{adj.2g.}{Que não se pode navegar.}{i.na.ve.gá.vel}{0}
\verb{inca}{}{}{}{}{adj.}{Relativo aos incas, povo indígena que habitava os Andes antes da colonização espanhola; incaico.}{in.ca}{0}
\verb{inca}{}{}{}{}{s.2g.}{Indivíduo dos incas.}{in.ca}{0}
\verb{incabível}{}{}{"-eis}{}{adj.2g.}{Que não tem cabimento; inaceitável.}{in.ca.bí.vel}{0}
\verb{incaico}{}{}{}{}{adj.}{Relativo aos incas; inca.}{in.cai.co}{0}
\verb{incalculável}{}{}{"-eis}{}{adj.2g.}{Que não se pode calcular.}{in.cal.cu.lá.vel}{0}
\verb{incalculável}{}{Por ext.}{"-eis}{}{}{Muito numeroso; incontável, inumerável.}{in.cal.cu.lá.vel}{0}
\verb{incandescência}{}{}{}{}{s.f.}{Estado ou condição de incandescente.}{in.can.des.cên.cia}{0}
\verb{incandescência}{}{Fís.}{}{}{}{Emissão de radiação luminosa de um corpo aquecido.}{in.can.des.cên.cia}{0}
\verb{incandescente}{}{}{}{}{adj.2g.}{Que está em brasa; candente, ardente.}{in.can.des.cen.te}{0}
\verb{incandescente}{}{Fig.}{}{}{}{Que está muito excitado; arrebatado.}{in.can.des.cen.te}{0}
\verb{incandescer}{ê}{}{}{}{v.t.}{Tornar candente, pôr em brasas.}{in.can.des.cer}{\verboinum{15}}
\verb{incansável}{}{}{"-eis}{}{adj.2g.}{Que não se cansa, não está sujeito ao cansaço; infatigável.}{in.can.sá.vel}{0}
\verb{incansável}{}{}{"-eis}{}{}{Diz"-se de quem é muito ativo; laborioso. (\textit{Ele é um trabalhador incansável.})}{in.can.sá.vel}{0}
\verb{incansável}{}{}{"-eis}{}{}{Constante, assíduo.}{in.can.sá.vel}{0}
\verb{incapacidade}{}{}{}{}{s.f.}{Qualidade ou estado de incapaz; inaptidão.}{in.ca.pa.ci.da.de}{0}
\verb{incapacidade}{}{}{}{}{}{Falta de capacidade intelectual ou física.}{in.ca.pa.ci.da.de}{0}
\verb{incapacidade}{}{}{}{}{}{Falta de qualificação; incompetência.}{in.ca.pa.ci.da.de}{0}
\verb{incapacidade}{}{Jur.}{}{}{}{Estado de um indivíduo privado por lei de exercer certos direitos.}{in.ca.pa.ci.da.de}{0}
\verb{incapacitar}{}{}{}{}{v.t.}{Tornar incapaz; inabilitar.}{in.ca.pa.ci.tar}{\verboinum{1}}
\verb{incapaz}{}{}{}{}{adj.2g.}{Que não é capaz.}{in.ca.paz}{0}
\verb{incapaz}{}{}{}{}{}{Impossibilitado, inabilitado, impedido.}{in.ca.paz}{0}
\verb{incapaz}{}{Jur.}{}{}{}{Que não tem capacidade legal.}{in.ca.paz}{0}
\verb{inçar}{}{}{}{}{v.t.}{Encher, aparecer em grande quantidade.}{in.çar}{0}
\verb{inçar}{}{}{}{}{}{Encher muito uma plantação de insetos.}{in.çar}{\verboinum{3}}
\verb{incaracterístico}{}{}{}{}{adj.}{Que não tem característica, não possui traço marcante; confundível, comum.}{in.ca.rac.te.rís.ti.co}{0}
\verb{incaraterístico}{}{}{}{}{adj.}{Incaracterístico.}{in.ca.ra.te.rís.ti.co}{0}
\verb{incauto}{}{}{}{}{adj.}{Que não tem cautela; imprudente, descuidado.}{in.cau.to}{0}
\verb{incauto}{}{}{}{}{}{Ingênuo, crédulo.}{in.cau.to}{0}
\verb{incender}{ê}{}{}{}{v.t.}{Incendiar.}{in.cen.der}{\verboinum{12}}
\verb{incendiar}{}{}{}{}{v.t.}{Pôr fogo em alguma coisa, fazer arder; queimar.}{in.cen.di.ar}{\verboinum{6}}
\verb{incendiário}{}{}{}{}{adj.}{Que incendeia, ateia fogo, ou provoca incêndio. }{in.cen.di.á.rio}{0}
\verb{incendiário}{}{Fig.}{}{}{}{Que inflama, incita à revolta; sedicioso.}{in.cen.di.á.rio}{0}
\verb{incendiário}{}{}{}{}{s.m.}{Indivíduo que provoca incêndio ou ateia fogo em algo intencionalmente; pirômano.}{in.cen.di.á.rio}{0}
\verb{incêndio}{}{}{}{}{s.m.}{Ato ou efeito de incendiar.}{in.cên.dio}{0}
\verb{incêndio}{}{}{}{}{}{Fogo que se espalha rapidamente, causando prejuízo.}{in.cên.dio}{0}
\verb{incensar}{}{}{}{}{v.t.}{Defumar, perfumar um ambiente queimando incenso.}{in.cen.sar}{0}
\verb{incensar}{}{Fig.}{}{}{}{Lisonjear, adular, bajular.}{in.cen.sar}{\verboinum{1}}
\verb{incenso}{}{}{}{}{s.m.}{Resina que se retira de certas plantas e que se queima para perfumar ou defumar um ambiente.}{in.cen.so}{0}
\verb{incensório}{}{}{}{}{s.m.}{Objeto próprio para queimar incensos; incensário, turíbulo.}{in.cen.só.rio}{0}
\verb{incensurável}{}{}{"-eis}{}{adj.2g.}{Que não pode ser censurado.}{in.cen.su.rá.vel}{0}
\verb{incentivar}{}{}{}{}{v.t.}{Dar incentivo; estimular, animar, encorajar, incitar.}{in.cen.ti.var}{\verboinum{1}}
\verb{incentivo}{}{}{}{}{s.m.}{Coisa que incentiva, encoraja; estímulo.}{in.cen.ti.vo}{0}
\verb{incerteza}{ê}{}{}{}{s.f.}{Falta de certeza; indecisão, hesitação, dúvida.  }{in.cer.te.za}{0}
\verb{incerto}{é}{}{}{}{adj.}{Que não é certo, ou de que não se tem certeza; duvidoso, hipotético.}{in.cer.to}{0}
\verb{incerto}{é}{}{}{}{}{Em que não há firmeza; inseguro, vacilante, indeciso.}{in.cer.to}{0}
\verb{incessante}{}{}{}{}{adj.2g.}{Que não cessa; contínuo, ininterrupto.}{in.ces.san.te}{0}
\verb{incessante}{}{}{}{}{}{Assíduo, constante, repetido.}{in.ces.san.te}{0}
\verb{incesto}{é/ ou /ê}{}{}{}{s.m.}{Relação ou união sexual entre parentes próximos, proibida pela lei e pelos costumes.}{in.ces.to}{0}
\verb{incestuoso}{ô}{}{"-osos ⟨ó⟩}{"-osa ⟨ó⟩}{adj.}{Relativo a incesto.}{in.ces.tu.o.so}{0}
\verb{incestuoso}{ô}{}{"-osos ⟨ó⟩}{"-osa ⟨ó⟩}{}{Que pratica ou praticou incesto.}{in.ces.tu.o.so}{0}
\verb{incestuoso}{ô}{}{"-osos ⟨ó⟩}{"-osa ⟨ó⟩}{}{Nascido de união incestuosa.}{in.ces.tu.o.so}{0}
\verb{inchação}{}{}{"-ões}{}{s.f.}{Ato ou efeito de inchar; inchamento.}{in.cha.ção}{0}
\verb{inchaço}{}{Pop.}{}{}{s.m.}{Inchação.}{in.cha.ço}{0}
\verb{inchado}{}{}{}{}{adj.}{Que tem inchação.}{in.cha.do}{0}
\verb{inchado}{}{Fig.}{}{}{}{Empolado, presunçoso, vaidoso, arrogante.}{in.cha.do}{0}
\verb{inchar}{}{}{}{}{v.t.}{Aumentar o volume; dilatar, engrossar.}{in.char}{0}
\verb{inchar}{}{}{}{}{}{Encher de orgulho, de vaidade.}{in.char}{\verboinum{1}}
\verb{incidência}{}{}{}{}{s.f.}{Qualidade do que é incidente. }{in.ci.dên.cia}{0}
\verb{incidência}{}{}{}{}{}{Ação de incidir.}{in.ci.dên.cia}{0}
\verb{incidental}{}{}{"-ais}{}{adj.2g.}{Relativo a incidente; eventual, episódico.}{in.ci.den.tal}{0}
\verb{incidente}{}{}{}{}{adj.2g.}{Que incide, ocorre.}{in.ci.den.te}{0}
\verb{incidente}{}{}{}{}{s.m.}{Acontecimento inesperado, acidental.}{in.ci.den.te}{0}
\verb{incidir}{}{}{}{}{v.t.}{Cair sobre; recair.}{in.ci.dir}{0}
\verb{incidir}{}{}{}{}{}{Cometer uma falta; incorrer.}{in.ci.dir}{0}
\verb{incidir}{}{}{}{}{}{Acometer, atacar, atingir.}{in.ci.dir}{0}
\verb{incidir}{}{}{}{}{v.i.}{Ocorrer, acontecer, sobrevir.}{in.ci.dir}{\verboinum{18}}
\verb{incineração}{}{}{"-ões}{}{s.f.}{Ato ou efeito de incinerar; cremação.}{in.ci.ne.ra.ção}{0}
\verb{incinerar}{}{}{}{}{v.t.}{Queimar algo até reduzir a cinzas; cremar.}{in.ci.ne.rar}{\verboinum{1}}
\verb{incipiência}{}{}{}{}{s.f.}{Qualidade ou condição de incipiente.}{in.ci.pi.ên.cia}{0}
\verb{incipiente}{}{}{}{}{adj.2g.}{Que está no começo de alguma coisa; principiante, bisonho.}{in.ci.pi.en.te}{0}
\verb{incircunciso}{}{}{}{}{adj.}{Incircuncidado.}{in.cir.cun.ci.so}{0}
\verb{incisão}{}{Med.}{"-ões}{}{s.f.}{Abertura na pele, durante cirurgia, feita com instrumento cortante.}{in.ci.são}{0}
\verb{incisão}{}{}{"-ões}{}{}{Corte, talho.}{in.ci.são}{0}
\verb{incisivo}{}{}{}{}{adj.}{Que é próprio para cortar.}{in.ci.si.vo}{0}
\verb{incisivo}{}{}{}{}{}{Que é direto, sem rodeios; decisivo.}{in.ci.si.vo}{0}
\verb{incisivo}{}{}{}{}{s.m.}{Cada um dos quatro dentes da frente de cada maxilar.}{in.ci.si.vo}{0}
\verb{inciso}{}{}{}{}{adj.}{Golpeado com objeto cortante; cortado.}{in.ci.so}{0}
\verb{inciso}{}{}{}{}{s.m.}{Cada uma das subdivisões de um artigo da lei; parágrafo.}{in.ci.so}{0}
\verb{incitação}{}{}{"-ões}{}{s.f.}{Ato ou efeito de incitar; incitamento.}{in.ci.ta.ção}{0}
\verb{incitador}{ô}{}{}{}{adj.}{Que incita, estimula; animador, instigador.}{in.ci.ta.dor}{0}
\verb{incitamento}{}{}{}{}{s.m.}{Incitação.}{in.ci.ta.men.to}{0}
\verb{incitante}{}{}{}{}{adj.2g.}{Que incita; incitador.}{in.ci.tan.te}{0}
\verb{incitar}{}{}{}{}{v.t.}{Estimular pessoa ou animal a fazer algo; instigar, encorajar, impelir, excitar. (\textit{Os torcedores incitaram os jogadores à vitória.})}{in.ci.tar}{\verboinum{1}}
\verb{incivil}{}{}{"-is}{}{adj.2g.}{Que não é civil, não possui civilidade; grosseiro, descortês, impolido.}{in.ci.vil}{0}
\verb{incivil}{}{}{"-is}{}{}{Que se opõe ao direito civil.}{in.ci.vil}{0}
\verb{incivilidade}{}{}{}{}{s.f.}{Qualidade ou caráter de incivil.}{in.ci.vi.li.da.de}{0}
\verb{incivilidade}{}{}{}{}{}{Ato ou expressão grosseira; descortesia, impolidez.}{in.ci.vi.li.da.de}{0}
\verb{incivilizado}{}{}{}{}{adj.}{Que não é civilizado; inculto, selvagem, grosseiro, rústico.}{in.ci.vi.li.za.do}{0}
\verb{inclassificável}{}{}{"-eis}{}{adj.2g.}{Quem não se pode classificar.}{in.clas.si.fi.cá.vel}{0}
\verb{inclassificável}{}{Por ext.}{"-eis}{}{}{Que não se pode definir, qualificar com precisão.}{in.clas.si.fi.cá.vel}{0}
\verb{inclassificável}{}{Por ext.}{"-eis}{}{}{Digno de censura; reprovável, condenável, inqualificável.}{in.clas.si.fi.cá.vel}{0}
\verb{inclemência}{}{}{}{}{s.f.}{Qualidade ou caráter de inclemente.}{in.cle.mên.cia}{0}
\verb{inclemência}{}{Fig.}{}{}{}{Aspereza, severidade, rigor.}{in.cle.mên.cia}{0}
\verb{inclemente}{}{}{}{}{adj.2g.}{Que não é clemente, não é indulgente.}{in.cle.men.te}{0}
\verb{inclemente}{}{Fig.}{}{}{}{Severo, áspero, rigoroso.}{in.cle.men.te}{0}
\verb{inclinação}{}{}{"-ões}{}{s.f.}{Ação de inclinar.}{in.cli.na.ção}{0}
\verb{inclinação}{}{Fig.}{"-ões}{}{}{Disposição, propensão, tendência.}{in.cli.na.ção}{0}
\verb{inclinado}{}{}{}{}{adj.}{Desviado da linha vertical; pendente.}{in.cli.na.do}{0}
\verb{inclinado}{}{Fig.}{}{}{}{Propenso, disposto, tendente.}{in.cli.na.do}{0}
\verb{inclinar}{}{}{}{}{v.t.}{Desviar da linha vertical ou reta; obliquar.}{in.cli.nar}{0}
\verb{inclinar}{}{}{}{}{}{Curvar; abaixar.}{in.cli.nar}{0}
\verb{inclinar}{}{}{}{}{v.pron.}{Mostrar preferência por alguma coisa;  ter propensão, tender.}{in.cli.nar}{\verboinum{1}}
\verb{ínclito}{}{}{}{}{adj.}{Diz"-se do que tem méritos ou qualidades notáveis; ilustre, famoso, emérito, celebrado, insigne.}{ín.cli.to}{0}
\verb{incluir}{}{}{}{}{v.t.}{Colocar alguma pessoa ou alguma coisa a mais em um conjunto; juntar.}{in.clu.ir}{0}
\verb{incluir}{}{}{}{}{}{Compreender, abarcar, abranger.}{in.clu.ir}{\verboinum{26}}
\verb{inclusão}{}{}{"-ões}{}{s.f.}{Ato ou efeito de incluir.}{in.clu.são}{0}
\verb{inclusive}{}{}{}{}{adv.}{De modo inclusivo; com inclusão, sem deixar de fora.}{in.clu.si.ve}{0}
\verb{inclusive}{}{}{}{}{}{Até; até mesmo.}{in.clu.si.ve}{0}
\verb{inclusivo}{}{}{}{}{adj.}{Que inclui, encerra, abrange, compreende.}{in.clu.si.vo}{0}
\verb{inclusivo}{}{Gram.}{}{}{}{Diz"-se da 1ª pessoa do plural (\textit{nós}), que inclui aquele que fala e aquele que ouve.}{in.clu.si.vo}{0}
\verb{incluso}{}{}{}{}{adj.}{Que foi incluído, compreendido.}{in.clu.so}{0}
\verb{incoativo}{}{}{}{}{adj.}{Que incoa, dá início; inceptivo, inicial.}{in.co.a.ti.vo}{0}
\verb{incoercível}{}{}{"-eis}{}{adj.2g.}{Que não se pode coagir; irreprimível. }{in.co.er.cí.vel}{0}
\verb{incoerência}{}{}{}{}{s.f.}{Falta de coerência; contradição, inconsequência.}{in.co.e.rên.cia}{0}
\verb{incoerente}{}{}{}{}{adj.2g.}{Que não tem, não demonstra coerência; inconsequente, contraditório.}{in.co.e.ren.te}{0}
\verb{incógnita}{}{}{}{}{s.f.}{Aquilo que se desconhece e se procura saber; enigma, segredo.}{in.cóg.ni.ta}{0}
\verb{incógnita}{}{Mat.}{}{}{}{Quantidade a ser descoberta na solução de uma equação, de um problema.Símb.: x}{in.cóg.ni.ta}{0}
\verb{incógnito}{}{}{}{}{adj.}{Que não se conhece; ignorado, ignoto, desconhecido.}{in.cóg.ni.to}{0}
\verb{incógnito}{}{}{}{}{s.m.}{Aquilo que é secreto, enigmático.}{in.cóg.ni.to}{0}
\verb{incógnito}{}{}{}{}{adv.}{Secretamente, às ocultas; sob nome suposto ou falsa identidade.}{in.cóg.ni.to}{0}
\verb{incognoscível}{}{}{"-eis}{}{adj.2g.}{Que não se pode conhecer.}{in.cog.nos.cí.vel}{0}
\verb{íncola}{}{}{}{}{s.2g.}{Indivíduo que habita determinado local; morador, habitante.}{ín.co.la}{0}
\verb{incolor}{ô}{}{}{}{adj.2g.}{Diz"-se do que não tem cor; descolorido.}{in.co.lor}{0}
\verb{incolor}{ô}{Fig.}{}{}{}{Que é sem graça, não tem interesse ou atrativo; insípido.}{in.co.lor}{0}
\verb{incólume}{}{}{}{}{adj.2g.}{Livre de perigo, são e salvo; ileso, intato.}{in.có.lu.me}{0}
\verb{incolumidade}{}{}{}{}{s.f.}{Qualidade ou estado de incólume.}{in.co.lu.mi.da.de}{0}
\verb{incombustível}{}{}{"-eis}{}{adj.2g.}{Diz"-se do que não entra em combustão, que não pode arder nem se queimar.}{in.com.bus.tí.vel}{0}
\verb{incombusto}{}{}{}{}{adj.}{Que não ardeu, não se queimou.}{in.com.bus.to}{0}
\verb{incomensurável}{}{}{"-eis}{}{adj.2g.}{Que não é comensurável, não se pode medir; imensurável.}{in.co.men.su.rá.vel}{0}
\verb{incomensurável}{}{}{"-eis}{}{}{Que não tem medida em comum com outra grandeza.}{in.co.men.su.rá.vel}{0}
\verb{incomensurável}{}{Mat.}{"-eis}{}{}{Diz"-se de relação de grandeza que não pode ser expressa por um número inteiro ou fracionário.}{in.co.men.su.rá.vel}{0}
\verb{incomensurável}{}{Por ext.}{"-eis}{}{}{Que não pode ser medido ou avaliado devido a sua ordem de grandeza ou importância.}{in.co.men.su.rá.vel}{0}
\verb{incomodar}{}{}{}{}{v.t.}{Causar incômodo, dificuldade ou aborrecimento a alguém; perturbar, importunar, indispor, molestar.}{in.co.mo.dar}{0}
\verb{incomodar}{}{}{}{}{v.pron.}{Ficar aborrecido com alguma coisa ou com alguma pessoa; irritar"-se, apoquentar"-se.}{in.co.mo.dar}{\verboinum{1}}
\verb{incomodativo}{}{}{}{}{adj.}{Que causa incômodo.}{in.co.mo.da.ti.vo}{0}
\verb{incomodidade}{}{}{}{}{s.f.}{Qualidade ou estado de incômodo.}{in.co.mo.di.da.de}{0}
\verb{incômodo}{}{}{}{}{adj.}{Que incomoda; desconfortável, desagradável. }{in.cô.mo.do}{0}
\verb{incômodo}{}{}{}{}{s.m.}{Dificuldade que se causa a alguém; embaraço, transtorno, inconveniência.}{in.cô.mo.do}{0}
\verb{incômodo}{}{}{}{}{}{Indisposição, mal"-estar.}{in.cô.mo.do}{0}
\verb{incomparável}{}{}{"-eis}{}{adj.2g.}{Que não admite comparação; não comparável.}{in.com.pa.rá.vel}{0}
\verb{incomparável}{}{}{"-eis}{}{}{Que está acima de qualquer comparação; extraordinário, prodigioso.}{in.com.pa.rá.vel}{0}
\verb{incompatibilidade}{}{}{}{}{s.f.}{Qualidade ou situação de incompatível; falta de compatibilidade.}{in.com.pa.ti.bi.li.da.de}{0}
\verb{incompatibilizar}{}{}{}{}{v.t.}{Tornar incompatível, inconciliável, incombinável; desarmonizar.}{in.com.pa.ti.bi.li.zar}{0}
\verb{incompatibilizar}{}{}{}{}{}{Pôr em discórdia, criar inimizade; indispor.}{in.com.pa.ti.bi.li.zar}{\verboinum{1}}
\verb{incompatível}{}{}{"-eis}{}{adj.2g.}{Que não pode harmonizar"-se; inconciliável, incombinável.}{in.com.pa.tí.vel}{0}
\verb{incompatível}{}{}{"-eis}{}{}{Diz"-se de cargos, funções que não podem ser exercidos ao mesmo tempo pela mesma pessoa.}{in.com.pa.tí.vel}{0}
\verb{incompetência}{}{}{}{}{s.f.}{Falta de competência.}{in.com.pe.tên.cia}{0}
\verb{incompetência}{}{}{}{}{}{Falta de conhecimento, de capacidade, de habilidade; inaptidão.}{in.com.pe.tên.cia}{0}
\verb{incompetente}{}{}{}{}{adj.2g.}{Que revela falta de competência, de habilidade, de aptidão; inábil.}{in.com.pe.ten.te}{0}
\verb{incompetente}{}{}{}{}{}{Que não é idôneo para realizar, resolver ou apreciar algo. }{in.com.pe.ten.te}{0}
\verb{incompetente}{}{}{}{}{s.m.}{Indivíduo que não tem aptidão, capacidade, habilidade; incapaz.}{in.com.pe.ten.te}{0}
\verb{incompleto}{é}{}{}{}{adj.}{Que não é completo; inacabado, inconciso, imperfeito, mutilado.}{in.com.ple.to}{0}
\verb{incomportável}{}{}{"-eis}{}{adj.2g.}{Que não é comportável, não se pode comportar, tolerar, admitir.}{in.com.por.tá.vel}{0}
\verb{incompreendido}{}{}{}{}{adj.}{Que não é compreendido, entendido ou percebido.}{in.com.pre.en.di.do}{0}
\verb{incompreendido}{}{}{}{}{s.m.}{Pessoa ou coisa que não é bem julgada, admirada pelo seu valor.}{in.com.pre.en.di.do}{0}
\verb{incompreendido}{}{}{}{}{}{Que não é compreendido, aceito ou reconhecido.}{in.com.pre.en.di.do}{0}
\verb{incompreensão}{}{}{"-ões}{}{s.f.}{Falta de compreensão.}{in.com.pre.en.são}{0}
\verb{incompreensível}{}{}{"-eis}{}{adj.2g.}{Que não é compreensível; ininteligível.}{in.com.pre.en.sí.vel}{0}
\verb{incompreensível}{}{}{"-eis}{}{}{Que é impossível ou extremamente difícil de se compreender, explicar, perceber, alcançar ou admitir.}{in.com.pre.en.sí.vel}{0}
\verb{incompreensível}{}{}{"-eis}{}{}{Que é misterioso, enigmático.}{in.com.pre.en.sí.vel}{0}
\verb{incompreensível}{}{}{"-eis}{}{s.m.}{Aquilo que não se pode compreender.}{in.com.pre.en.sí.vel}{0}
\verb{incompreensivo}{}{}{}{}{adj.}{Que é incapaz de compreender, entender, perceber, ou que revela falta de compreensão.}{in.com.pre.en.si.vo}{0}
\verb{incompreensivo}{}{}{}{}{}{Que não pode ou não tenta compreender o outro, não é tolerante ou condescendente; inflexível, rígido.}{in.com.pre.en.si.vo}{0}
\verb{incompressível}{}{}{"-eis}{}{adj.2g.}{Que não se pode comprimir.}{in.com.pres.sí.vel}{0}
\verb{incompressível}{}{Fig.}{"-eis}{}{}{Que não se pode reprimir ou coagir; irreprimível, incoercível.}{in.com.pres.sí.vel}{0}
\verb{incomum}{}{}{"-uns}{}{adj.2g.}{Que não é comum; anormal, extraordinário, fora do comum.}{in.co.mum}{0}
\verb{incomunicável}{}{}{"-eis}{}{adj.2g.}{Que não pode ser comunicado, transferido.}{in.co.mu.ni.cá.vel}{0}
\verb{incomunicável}{}{}{"-eis}{}{}{Que não se consegue comunicar a outrem, que não pode ser exprimido; indizível, inexprimível.}{in.co.mu.ni.cá.vel}{0}
\verb{incomunicável}{}{}{"-eis}{}{}{Que não se pode comunicar, manter contato com outrem.}{in.co.mu.ni.cá.vel}{0}
\verb{incomunicável}{}{}{"-eis}{}{}{Que é insociável, intratável.}{in.co.mu.ni.cá.vel}{0}
\verb{incomutável}{}{}{"-eis}{}{adj.2g.}{Que não é comutável, não se pode mudar, trocar, permutar ou substituir.}{in.co.mu.tá.vel}{0}
\verb{inconcebível}{}{}{"-eis}{}{adj.2g.}{Que não se pode conceber, perceber ou explicar.}{in.con.ce.bí.vel}{0}
\verb{inconcebível}{}{Por ext.}{"-eis}{}{}{Que é fora do comum, que surpreende; inacreditável, extraordinário.}{in.con.ce.bí.vel}{0}
\verb{inconciliável}{}{}{"-eis}{}{adj.2g.}{Que não se pode conciliar, acordar, conformar, harmonizar, reunir.}{in.con.ci.li.á.vel}{0}
\verb{inconcludente}{}{}{}{}{adj.2g.}{Que não é concludente, não leva a conclusão, não prova, demonstra ou resolve algo.}{in.con.clu.den.te}{0}
\verb{inconcluso}{}{}{}{}{adj.}{Que não se concluiu, não foi terminado; inacabado, incompleto.}{in.con.clu.so}{0}
\verb{inconcusso}{}{}{}{}{adj.}{Que está solidamente estabelecido, fixado; firme, inabalável.}{in.con.cus.so}{0}
\verb{inconcusso}{}{}{}{}{}{Que se impõe pela sua evidência; indiscutível, incontestável.}{in.con.cus.so}{0}
\verb{inconcusso}{}{}{}{}{}{Que é austero, incorruptível.}{in.con.cus.so}{0}
\verb{incondicional}{}{}{"-ais}{}{adj.2g.}{Que não é sujeito a condições; absoluto, irrestrito.}{in.con.di.ci.o.nal}{0}
\verb{incondicional}{}{}{"-ais}{}{}{Que, em quaisquer circunstâncias e sem discussão, toma partido de alguém, de uma corrente, de uma causa.}{in.con.di.ci.o.nal}{0}
\verb{inconfessado}{}{}{}{}{adj.}{Que não se confessou; que se ocultou ou que se dissimulou.}{in.con.fes.sa.do}{0}
\verb{inconfessável}{}{}{"-eis}{}{adj.2g.}{Que não se pode ou não se deve confessar.}{in.con.fes.sá.vel}{0}
\verb{inconfesso}{é}{}{}{}{adj.}{Que não é confesso, que não confessou.}{in.con.fes.so}{0}
\verb{inconfidência}{}{}{}{}{s.f.}{Falta de fidelidade para com alguém, especialmente para com o Estado ou o soberano.}{in.con.fi.dên.cia}{0}
\verb{inconfidência}{}{}{}{}{}{Abuso de confiança; infidelidade, deslealdade.}{in.con.fi.dên.cia}{0}
\verb{inconfidência}{}{}{}{}{}{Revelação de segredo confiado.}{in.con.fi.dên.cia}{0}
\verb{inconfidente}{}{}{}{}{adj.2g.}{Que está envolvido em inconfidência, deslealdade com o Estado ou governante.}{in.con.fi.den.te}{0}
\verb{inconfidente}{}{}{}{}{}{Que é infiel, desleal, traiçoeiro.}{in.con.fi.den.te}{0}
\verb{inconfidente}{}{}{}{}{}{Que divulga os segredos que lhe confiaram.}{in.con.fi.den.te}{0}
\verb{inconfidente}{}{}{}{}{s.m.}{Cada um dos cidadãos que tomaram parte na Inconfidência Mineira.}{in.con.fi.den.te}{0}
\verb{inconformado}{}{}{}{}{adj.}{Que não se conforma; não resignado.}{in.con.for.ma.do}{0}
\verb{inconformado}{}{}{}{}{s.m.}{Indivíduo que não aceita determinada situação.}{in.con.for.ma.do}{0}
\verb{inconformidade}{}{}{}{}{s.f.}{Ausência de conformidade; divergência, desacordo, rebeldia, resistência.}{in.con.for.mi.da.de}{0}
\verb{inconformismo}{}{}{}{}{s.m.}{Procedimento de inconformado, de quem não aceita condições ou situações desfavoráveis.}{in.con.for.mis.mo}{0}
\verb{inconformismo}{}{}{}{}{}{Tendência ou atitude de não aceitar passivamente o modo de agir ou de pensar das pessoas do grupo em que se vive.}{in.con.for.mis.mo}{0}
\verb{inconfundível}{}{}{"-eis}{}{adj.2g.}{Que não se confunde com outro; distinto, diferente.}{in.con.fun.dí.vel}{0}
\verb{incongelável}{}{}{"-eis}{}{adj.2g.}{Que não se pode congelar.}{in.con.ge.lá.vel}{0}
\verb{incongruência}{}{}{}{}{s.f.}{Qualidade de incongruente.}{in.con.gru.ên.cia}{0}
\verb{incongruência}{}{}{}{}{}{Falta de congruência; incompatibilidade, incoerência.}{in.con.gru.ên.cia}{0}
\verb{incongruente}{}{}{}{}{adj.2g.}{Que é inconveniente, impróprio, incompatível.}{in.con.gru.en.te}{0}
\verb{inconho}{}{Bot.}{}{}{adj.}{Diz"-se de fruto que nasce acoplado a outro.}{in.co.nho}{0}
\verb{inconho}{}{}{}{}{}{Que está muito ligado a outro ser ou coisa.}{in.co.nho}{0}
\verb{inconjugável}{}{}{"-eis}{}{adj.2g.}{Que não é conjugável, não é passível de conjugação.}{in.con.ju.gá.vel}{0}
\verb{inconquistável}{}{}{"-eis}{}{adj.2g.}{Que não se pode conquistar; inexpugnável.}{in.con.quis.tá.vel}{0}
\verb{inconquistável}{}{Fig.}{"-eis}{}{}{Que é indomável, invencível.}{in.con.quis.tá.vel}{0}
\verb{inconsciência}{}{}{}{}{s.f.}{Qualidade ou estado de inconsciente.}{in.cons.ci.ên.cia}{0}
\verb{inconsciência}{}{}{}{}{}{Falta de consciência; irresponsabilidade, irreflexão, leviandade.}{in.cons.ci.ên.cia}{0}
\verb{inconsciente}{}{}{}{}{adj.2g.}{Que não tem consciência.}{in.cons.ci.en.te}{0}
\verb{inconsciente}{}{}{}{}{}{Que perdeu a consciência, que está desacordado.}{in.cons.ci.en.te}{0}
\verb{inconsciente}{}{}{}{}{}{Que é irresponsável, leviano.}{in.cons.ci.en.te}{0}
\verb{inconsciente}{}{}{}{}{s.2g.}{Indivíduo que procede sem consciência do que faz.}{in.cons.ci.en.te}{0}
\verb{inconsciente}{}{}{}{}{s.m.}{Parte da mente, autônoma em relação à consciência, que infuencia o comportamento do ser.}{in.cons.ci.en.te}{0}
\verb{inconsequência}{}{}{}{}{s.f.}{Falta de consequência nas ideias e nas ações; incoerência, incongruência, contradição.}{in.con.se.quên.cia}{0}
\verb{inconsequente}{}{}{}{}{adj.2g.}{Em que há inconsequência; ilógico, contraditório.}{in.con.se.quen.te}{0}
\verb{inconsequente}{}{}{}{}{}{Que revela falta de reflexão, de ponderação, de prudência; irrefletido, imprudente.}{in.con.se.quen.te}{0}
\verb{inconsequente}{}{}{}{}{s.2g.}{Indivíduo que é irresponsável, imprudente, leviano.}{in.con.se.quen.te}{0}
\verb{inconsequente}{}{}{}{}{}{Indivíduo que não é coerente, está em contradição consigo mesmo.}{in.con.se.quen.te}{0}
\verb{inconsiderado}{}{}{}{}{adj.}{Que é dito e feito sem consideração, sem ponderação, sem reflexão.}{in.con.si.de.ra.do}{0}
\verb{inconsiderado}{}{}{}{}{s.m.}{Indivíduo imprudente, leviano.}{in.con.si.de.ra.do}{0}
\verb{inconsistência}{}{}{}{}{s.f.}{Falta de consistência, de estabilidade ou de firmeza.}{in.con.sis.tên.cia}{0}
\verb{inconsistência}{}{}{}{}{}{Falta de lógica, de nexo; incoerência.}{in.con.sis.tên.cia}{0}
\verb{inconsistente}{}{}{}{}{adj.2g.}{A que falta consistência, coesão, estabilidade, firmeza física.}{in.con.sis.ten.te}{0}
\verb{inconsistente}{}{}{}{}{}{Que não tem conteúdo, que tem pouca profundidade.}{in.con.sis.ten.te}{0}
\verb{inconsistente}{}{}{}{}{}{Que é incoerente, infundado.}{in.con.sis.ten.te}{0}
\verb{inconsistente}{}{Fig.}{}{}{}{Que não tem firmeza moral ou personalidade; amorfo, inconstante.}{in.con.sis.ten.te}{0}
\verb{inconsolado}{}{}{}{}{adj.}{Que não tem consolo; desolado.}{in.con.so.la.do}{0}
\verb{inconsolável}{}{}{"-eis}{}{adj.2g.}{Que não se pode consolar.}{in.con.so.lá.vel}{0}
\verb{inconsolável}{}{Por ext.}{"-eis}{}{}{Que está muito triste, desesperado.}{in.con.so.lá.vel}{0}
\verb{inconstância}{}{}{}{}{s.f.}{Falta de constância; volubilidade, instabilidade, mobilidade.}{in.cons.tân.cia}{0}
\verb{inconstância}{}{}{}{}{}{Leviandade, infidelidade.}{in.cons.tân.cia}{0}
\verb{inconstante}{}{}{}{}{adj.2g.}{Que é volúvel, mutável, instável.}{in.cons.tan.te}{0}
\verb{inconstante}{}{}{}{}{}{Que é leviano, infiel.}{in.cons.tan.te}{0}
\verb{inconstante}{}{}{}{}{s.2g.}{Indivíduo volúvel, inconstante, instável.}{in.cons.tan.te}{0}
\verb{inconstitucional}{}{}{"-ais}{}{adj.2g.}{Que não é constitucional ou que se opõe à constituição do Estado.}{in.cons.ti.tu.ci.o.nal}{0}
\verb{inconstitucionalidade}{}{}{}{}{s.f.}{Qualidade de inconstitucional.}{in.cons.ti.tu.ci.o.na.li.da.de}{0}
\verb{inconsútil}{}{}{"-eis}{}{adj.2g.}{Que não tem costura.}{in.con.sú.til}{0}
\verb{inconsútil}{}{}{"-eis}{}{}{Que não apresenta emendas, que é feito de uma só peça; inteiriço.}{in.con.sú.til}{0}
\verb{incontável}{}{}{"-eis}{}{adj.2g.}{Que é impossível de contar; inumerável.}{in.con.tá.vel}{0}
\verb{incontável}{}{}{"-eis}{}{}{Que não se pode relatar; inarrável.}{in.con.tá.vel}{0}
\verb{incontentável}{}{}{"-eis}{}{adj.2g.}{Que não se pode contentar, ou que é difícil de se contentar.}{in.con.ten.tá.vel}{0}
\verb{incontestado}{}{}{}{}{adj.}{Que não se contestou; que não se pôs em dúvida ou em questão.}{in.con.tes.ta.do}{0}
\verb{incontestável}{}{}{"-eis}{}{adj.2g.}{Que não pode sofrer contestação; indiscutível.}{in.con.tes.tá.vel}{0}
\verb{inconteste}{é}{}{}{}{adj.2g.}{Que não se põe em dúvida.}{in.con.tes.te}{0}
\verb{inconteste}{é}{}{}{}{}{Que não é conteste, que não está em harmonia ou de acordo com outras afirmações.}{in.con.tes.te}{0}
\verb{incontido}{}{}{}{}{adj.}{Que não pode ser contido; que não se pode reprimir.}{in.con.ti.do}{0}
\verb{incontinência}{}{}{}{}{s.f.}{Falta de continência, de temperança; excesso.}{in.con.ti.nên.cia}{0}
\verb{incontinência}{}{}{}{}{}{Falta de comedimento nos prazeres sexuais; luxúria, sensualidade.}{in.con.ti.nên.cia}{0}
\verb{incontinência}{}{Med.}{}{}{}{Emissão involuntária de urina ou matéria fecal.}{in.con.ti.nên.cia}{0}
\verb{incontinente}{}{}{}{}{adj.2g.}{Que não tem controle ou moderação.}{in.con.ti.nen.te}{0}
\verb{incontinenti}{}{}{}{}{adv.}{Sem demora; sem intervalo; sem interrupção; imediatamente.}{\textit{incontinenti}}{0}
\verb{incontrastável}{}{}{"-eis}{}{adj.2g.}{Que é irrefutável, irrespondível, irrecusável.}{in.con.tras.tá.vel}{0}
\verb{incontrastável}{}{}{"-eis}{}{}{Irrevogável, decisivo, inabalável.}{in.con.tras.tá.vel}{0}
\verb{incontrolável}{}{}{"-eis}{}{adj.2g.}{Que não se pode controlar; que não se submete a nenhuma forma de controle; incoercível, irrefreável.}{in.con.tro.lá.vel}{0}
\verb{incontroverso}{é}{}{}{}{adj.}{Que é indiscutível, indubitável, incontestável.}{in.con.tro.ver.so}{0}
\verb{inconveniência}{}{}{}{}{s.f.}{Qualidade do que é inconveniente; falta de conveniência.}{in.con.ve.ni.ên.cia}{0}
\verb{inconveniência}{}{}{}{}{}{Incongruência, inadequação.}{in.con.ve.ni.ên.cia}{0}
\verb{inconveniência}{}{}{}{}{}{Indelicadeza, incivilidade, grosseria.}{in.con.ve.ni.ên.cia}{0}
\verb{inconveniente}{}{}{}{}{adj.2g.}{Que não fica bem, considerando o lugar e o momento; impróprio, inadequado, inoportuno.}{in.con.ve.ni.en.te}{0}
\verb{inconveniente}{}{}{}{}{}{Coisa que pode atrapalhar; embaraço, estorvo, transtorno.}{in.con.ve.ni.en.te}{0}
\verb{inconversível}{}{}{"-eis}{}{adj.2g.}{Que não se pode converter, que não se pode trocar.}{in.con.ver.sí.vel}{0}
\verb{inconvertível}{}{}{"-eis}{}{adj.2g.}{Que não se pode converter; inconversível.}{in.con.ver.tí.vel}{0}
\verb{incorporação}{}{}{"-ões}{}{s.f.}{Ato ou efeito de incorporar.}{in.cor.po.ra.ção}{0}
\verb{incorporação}{}{}{"-ões}{}{}{Agrupamento, inclusão.}{in.cor.po.ra.ção}{0}
\verb{incorporação}{}{Bras.}{"-ões}{}{}{Constituição de um condomínio para construção de edifícios comerciais ou de apartamentos.}{in.cor.po.ra.ção}{0}
\verb{incorporação}{}{Bras.}{"-ões}{}{}{Transe mediúnico.}{in.cor.po.ra.ção}{0}
\verb{incorporador}{ô}{}{}{}{adj.}{Que incorpora.}{in.cor.po.ra.dor}{0}
\verb{incorporador}{ô}{}{}{}{s.m.}{Indivíduo que incorpora.}{in.cor.po.ra.dor}{0}
\verb{incorporador}{ô}{Bras.}{}{}{}{Indivíduo que promove a incorporação de prédios de apartamentos, lojas etc., em condomínio.}{in.cor.po.ra.dor}{0}
\verb{incorporador}{ô}{Bras.}{}{}{}{Fundador de uma sociedade anônima.}{in.cor.po.ra.dor}{0}
\verb{incorporadora}{ô}{}{}{}{s.f.}{Firma, empresa que promove ou administra incorporações imobiliárias.}{in.cor.po.ra.do.ra}{0}
\verb{incorporar}{}{}{}{}{v.t.}{Dar forma corpórea, material.}{in.cor.po.rar}{0}
\verb{incorporar}{}{}{}{}{}{Admitir ou receber em corporação.}{in.cor.po.rar}{0}
\verb{incorporar}{}{}{}{}{}{Reunir diversas companhias mercantis em uma só.}{in.cor.po.rar}{0}
\verb{incorporar}{}{}{}{}{}{Juntar num só corpo; dar unidade; reunir.}{in.cor.po.rar}{0}
\verb{incorporar}{}{Bras.}{}{}{}{Realizar contrato para construção em condomínio, começando logo a vender, em prestações, as futuras unidades.}{in.cor.po.rar}{0}
\verb{incorporar}{}{}{}{}{v.pron.}{Começar a fazer parte; ingressar.}{in.cor.po.rar}{\verboinum{1}}
\verb{incorpóreo}{}{}{}{}{adj.}{Que não tem corpo; impalpável, imaterial.}{in.cor.pó.re.o}{0}
\verb{incorreção}{}{}{"-ões}{}{s.f.}{Falta de correção; erro.}{in.cor.re.ção}{0}
\verb{incorreção}{}{}{"-ões}{}{}{Qualidade de incorreto.}{in.cor.re.ção}{0}
\verb{incorreção}{}{}{"-ões}{}{}{Ato, dito ou procedimento incorreto.}{in.cor.re.ção}{0}
\verb{incorrer}{ê}{}{}{}{v.t.}{Ficar incluído, implicado ou comprometido; incidir.}{in.cor.rer}{0}
\verb{incorrer}{ê}{}{}{}{}{Levar a efeito; cometer.}{in.cor.rer}{0}
\verb{incorrer}{ê}{}{}{}{}{Ficar sujeito.}{in.cor.rer}{0}
\verb{incorrer}{ê}{}{}{}{}{Atrair sobre si; causar.}{in.cor.rer}{\verboinum{12}}
\verb{incorreto}{é}{}{}{}{adj.}{Que não foi corrigido; que apresenta erros.}{in.cor.re.to}{0}
\verb{incorreto}{é}{}{}{}{}{Que está em desacordo com as regras.}{in.cor.re.to}{0}
\verb{incorreto}{é}{}{}{}{}{Que denota inconveniência, inadequação, incivilidade.}{in.cor.re.to}{0}
\verb{incorreto}{é}{}{}{}{}{Que é desleal, desonesto, indigno.}{in.cor.re.to}{0}
\verb{incorrigível}{}{}{"-eis}{}{adj.2g.}{Que não se pode corrigir, consertar, restaurar.}{in.cor.ri.gí.vel}{0}
\verb{incorrigível}{}{}{"-eis}{}{}{Que persiste em seus erros.}{in.cor.ri.gí.vel}{0}
\verb{incorrigível}{}{}{"-eis}{}{s.2g.}{Indivíduo que persiste em seus defeitos, vícios etc.}{in.cor.ri.gí.vel}{0}
\verb{incorruptível}{}{}{"-eis}{}{adj.2g.}{Que não se deteriora; inalterável, inatacável.}{in.cor.rup.tí.vel}{0}
\verb{incorruptível}{}{}{"-eis}{}{}{Incapaz de deixar"-se corromper, seduzir, subornar; reto, honesto.}{in.cor.rup.tí.vel}{0}
\verb{incorrupto}{}{}{}{}{adj.}{Que não sofreu processo de corrupção.}{in.cor.rup.to}{0}
\verb{incorrupto}{}{}{}{}{}{Que não se deixou corromper, seduzir, subornar.}{in.cor.rup.to}{0}
\verb{incorrutível}{}{}{}{}{}{Var. de \textit{incorruptível}.}{in.cor.ru.tí.vel}{0}
\verb{incorruto}{}{}{}{}{}{Var. de \textit{incorrupto}.}{in.cor.ru.to}{0}
\verb{incredulidade}{}{}{}{}{s.f.}{Qualidade de incrédulo; descrença.}{in.cre.du.li.da.de}{0}
\verb{incredulidade}{}{}{}{}{}{Falta de fé; ateísmo.}{in.cre.du.li.da.de}{0}
\verb{incrédulo}{}{}{}{}{adj.}{Que não crê; ímpio.}{in.cré.du.lo}{0}
\verb{incrédulo}{}{}{}{}{}{Próprio de quem está em dúvida; que denota ceticismo.}{in.cré.du.lo}{0}
\verb{incrédulo}{}{}{}{}{s.m.}{Indivíduo que não crê, que duvida; ímpio, ateu.}{in.cré.du.lo}{0}
\verb{incrementação}{}{}{"-ões}{}{s.f.}{Ato ou efeito de incrementar, de tornar maior, mais desenvolvido.}{in.cre.men.ta.ção}{0}
\verb{incrementado}{}{}{}{}{adj.}{Desenvolvido, aumentado; tornado maior.}{in.cre.men.ta.do}{0}
\verb{incrementado}{}{}{}{}{}{Mais apurado, mais rico, mais variado.}{in.cre.men.ta.do}{0}
\verb{incrementado}{}{Bras.}{}{}{}{Cheio de elementos que dão mais brilho.}{in.cre.men.ta.do}{0}
\verb{incrementado}{}{Bras.}{}{}{}{Cheio de acessórios, enfeites, recursos.}{in.cre.men.ta.do}{0}
\verb{incrementar}{}{}{}{}{v.t.}{Tornar maior; aumentar, desenvolver.}{in.cre.men.tar}{0}
\verb{incrementar}{}{Bras.}{}{}{}{Enriquecer, enfeitar, equipar.}{in.cre.men.tar}{0}
\verb{incrementar}{}{Mat.}{}{}{}{Adicionar a uma variável; aumentar o valor.}{in.cre.men.tar}{\verboinum{1}}
\verb{incremento}{}{}{}{}{s.m.}{Ato de incrementar.}{in.cre.men.to}{0}
\verb{incremento}{}{}{}{}{}{Aumento, desenvolvimento, melhoria.}{in.cre.men.to}{0}
\verb{incremento}{}{Mat.}{}{}{}{A quantidade adicionada a uma variável.}{in.cre.men.to}{0}
\verb{increpar}{}{}{}{}{v.t.}{Repreender severamente; censurar, acusar.}{in.cre.par}{\verboinum{1}}
\verb{incréu}{}{}{}{}{adj.}{Incrédulo.}{in.créu}{0}
\verb{incriminação}{}{}{"-ões}{}{s.f.}{Ato ou efeito de incriminar.}{in.cri.mi.na.ção}{0}
\verb{incriminar}{}{}{}{}{v.t.}{Declarar criminoso; atribuir um crime a alguém.}{in.cri.mi.nar}{0}
\verb{incriminar}{}{}{}{}{}{Considerar como crime. \textit{abon}}{in.cri.mi.nar}{0}
\verb{incriminar}{}{}{}{}{}{Oferecer involuntariamente evidências de culpa em um crime.}{in.cri.mi.nar}{\verboinum{1}}
\verb{incriticável}{}{}{"-eis}{}{adj.2g.}{Diz"-se de quem, por condição precária ou mérito incomum, não vale a pena criticar ou não pode ser criticado.}{in.cri.ti.cá.vel}{0}
\verb{incrível}{}{}{"-eis}{}{adj.2g.}{Difícil ou impossível de acreditar.}{in.crí.vel}{0}
\verb{incrível}{}{}{"-eis}{}{}{Extraordinário, incompreensível, singular, fantástico, fora do comum.}{in.crí.vel}{0}
\verb{incruento}{}{}{}{}{adj.}{Que não é sanguinário ou cruel; sem crueldade.}{in.cru.en.to}{0}
\verb{incrustação}{}{}{"-ões}{}{s.f.}{Ato ou efeito de incrustar; formação de crosta.}{in.crus.ta.ção}{0}
\verb{incrustação}{}{}{"-ões}{}{}{Aquilo que forma a crosta.}{in.crus.ta.ção}{0}
\verb{incrustação}{}{Art.}{"-ões}{}{}{Nas artes plásticas, qualquer ornamento embutido em uma peça.}{in.crus.ta.ção}{0}
\verb{incrustar}{}{}{}{}{v.t.}{Inserir, embutir.}{in.crus.tar}{0}
\verb{incrustar}{}{}{}{}{}{Ornar com incrustações.}{in.crus.tar}{0}
\verb{incrustar}{}{}{}{}{v.i.}{Tornar"-se crosta.}{in.crus.tar}{\verboinum{1}}
\verb{incubação}{}{}{"-ões}{}{s.f.}{Ato ou efeito de incubar.}{in.cu.ba.ção}{0}
\verb{incubação}{}{Fig.}{"-ões}{}{}{Preparação, elaboração, concepção.}{in.cu.ba.ção}{0}
\verb{incubação}{}{Med.}{"-ões}{}{}{O período de uma doença durante o qual ainda não há sintomas.}{in.cu.ba.ção}{0}
\verb{incubadeira}{ê}{}{}{}{s.f.}{Chocadeira, incubadora.}{in.cu.ba.dei.ra}{0}
\verb{incubadora}{ô}{Med.}{}{}{s.f.}{Câmara com temperatura e umidade controladas para auxiliar na sobrevivência de recém"-nascidos que exigem cuidados particulares.}{in.cu.ba.do.ra}{0}
\verb{incubadora}{ô}{Biol.}{}{}{}{Ambiente em laboratório para cultivar micro"-organismos para serem usados em testes biológicos.}{in.cu.ba.do.ra}{0}
\verb{incubadora}{ô}{Veter.}{}{}{}{Aparelho usado na criação de galináceos; chocadeira.}{in.cu.ba.do.ra}{0}
\verb{incubar}{}{}{}{}{v.t.}{Chocar.}{in.cu.bar}{0}
\verb{incubar}{}{Fig.}{}{}{}{Planejar, arquitetar, elaborar.}{in.cu.bar}{0}
\verb{incubar}{}{Med.}{}{}{}{Trazer moléstia em estado latente.}{in.cu.bar}{\verboinum{1}}
\verb{inculcar}{}{}{}{}{v.t.}{Colocar ideia ou doutrina na mente de alguém.}{in.cul.car}{0}
\verb{inculcar}{}{}{}{}{}{Sugerir, propor, demonstrar. (\textit{Sua voz inculcava que ainda estava preocupada.})}{in.cul.car}{0}
\verb{inculcar}{}{}{}{}{}{Recomendar, indicar, apregoar, aconselhar, insinuar. (\textit{A televisão inculca maus hábitos às pessoas.})}{in.cul.car}{0}
\verb{inculcar}{}{}{}{}{v.pron.}{Insinuar"-se, impor"-se, apresentar"-se.}{in.cul.car}{\verboinum{1}}
\verb{inculpabilidade}{}{}{}{}{s.f.}{Qualidade de inculpável.}{in.cul.pa.bi.li.da.de}{0}
\verb{inculpado}{}{}{}{}{adj.}{Que se inculpou; acusado.}{in.cul.pa.do}{0}
\verb{inculpado}{}{}{}{}{}{Isento de culpa.}{in.cul.pa.do}{0}
\verb{inculpar}{}{}{}{}{v.t.}{Atribuir culpa; acusar, incriminar.}{in.cul.par}{\verboinum{1}}
\verb{inculpável}{}{}{"-eis}{}{adj.2g.}{Que não se pode culpar.}{in.cul.pá.vel}{0}
\verb{incultivável}{}{}{"-eis}{}{adj.2g.}{Que não se pode cultivar; improdutivo.}{in.cul.ti.vá.vel}{0}
\verb{inculto}{}{}{}{}{adj.}{Não cultivado; árido, agreste.}{in.cul.to}{0}
\verb{inculto}{}{}{}{}{}{Sem cultura, sem instrução.}{in.cul.to}{0}
\verb{inculto}{}{Fig.}{}{}{}{Tosco, singelo.}{in.cul.to}{0}
\verb{incultura}{}{}{}{}{s.f.}{Qualidade de inculto.}{in.cul.tu.ra}{0}
\verb{incultura}{}{}{}{}{}{Falta de cultura, de instrução.}{in.cul.tu.ra}{0}
\verb{incumbência}{}{}{}{}{s.f.}{Ato ou efeito de incumbir.}{in.cum.bên.cia}{0}
\verb{incumbência}{}{}{}{}{}{A missão que se incumbe a alguém.}{in.cum.bên.cia}{0}
\verb{incumbir}{}{}{}{}{v.t.}{Dar encargo; encarregar.}{in.cum.bir}{0}
\verb{incumbir}{}{}{}{}{}{Ser da obrigação de algo ou alguém; competir a algo ou alguém.}{in.cum.bir}{\verboinum{18}}
\verb{incunábulo}{}{}{}{}{adj.}{Diz"-se de livro impresso nos primeiros anos da imprensa, até 1500.}{in.cu.ná.bu.lo}{0}
\verb{incunábulo}{}{}{}{}{s.m.}{Começo, origem.}{in.cu.ná.bu.lo}{0}
\verb{incurável}{}{}{"-eis}{}{adj.2g.}{Que não tem cura.}{in.cu.rá.vel}{0}
\verb{incúria}{}{}{}{}{s.f.}{Falta de cuidado; desleixo.}{in.cú.ria}{0}
\verb{incursão}{}{}{"-ões}{}{s.f.}{Invasão.}{in.cur.são}{0}
\verb{incursão}{}{}{"-ões}{}{}{Passeio por determinada região.}{in.cur.são}{0}
\verb{incursionar}{}{}{}{}{v.t.}{Realizar incursão; penetrar, entrar.}{in.cur.si.o.nar}{\verboinum{1}}
\verb{incurso}{}{}{}{}{adj.}{Que incorreu em culpa.}{in.cur.so}{0}
\verb{incurso}{}{}{}{}{s.m.}{Ato de incorrer; incursão.}{in.cur.so}{0}
\verb{incutir}{}{}{}{}{v.t.}{Inspirar, infundir.}{in.cu.tir}{0}
\verb{incutir}{}{}{}{}{}{Insinuar, sugerir, suscitar.}{in.cu.tir}{\verboinum{18}}
\verb{inda}{}{}{}{}{adv.}{Ainda.}{in.da}{0}
\verb{indagação}{}{}{"-ões}{}{s.f.}{Ato ou efeito de indagar.}{in.da.ga.ção}{0}
\verb{indagação}{}{}{"-ões}{}{}{Investigação, pesquisa, busca, devassa.}{in.da.ga.ção}{0}
\verb{indagar}{}{}{}{}{v.t.}{Investigar, buscar.}{in.da.gar}{0}
\verb{indagar}{}{}{}{}{}{Perguntar, inquirir.}{in.da.gar}{\verboinum{5}}
\verb{indaiá}{}{Bot.}{}{}{s.m.}{Palmeira com folhas eretas e crespas e frutos amarelos e comestíveis.}{in.dai.á}{0}
\verb{indébito}{}{}{}{}{adj.}{Que não é devido; imerecido.}{in.dé.bi.to}{0}
\verb{indecência}{}{}{}{}{s.m.}{Falta de decência.}{in.de.cên.cia}{0}
\verb{indecência}{}{}{}{}{}{Ato ou dito indecente.}{in.de.cên.cia}{0}
\verb{indecente}{}{}{}{}{adj.2g.}{Que não é decente; imoral, indecoroso, obsceno.}{in.de.cen.te}{0}
\verb{indecifrável}{}{}{"-eis}{}{adj.2g.}{Que não pode ser decifrado.}{in.de.ci.frá.vel}{0}
\verb{indecisão}{}{}{"-ões}{}{s.f.}{Qualidade ou estado de indeciso; hesitação, perplexidade.}{in.de.ci.são}{0}
\verb{indeciso}{}{}{}{}{adj.}{Não decidido; hesitante.}{in.de.ci.so}{0}
\verb{indeciso}{}{Fig.}{}{}{}{Indeterminado, vago.}{in.de.ci.so}{0}
\verb{indeciso}{}{}{}{}{s.m.}{Indivíduo hesitante.}{in.de.ci.so}{0}
\verb{indeclinável}{}{}{"-eis}{}{adj.2g.}{De que é impossível declinar, desviar"-se; irrecusável.}{in.de.cli.ná.vel}{0}
\verb{indecomponível}{}{}{"-eis}{}{adj.2g.}{Que não se pode decompor.}{in.de.com.po.ní.vel}{0}
\verb{indecoroso}{ô}{}{"-osos ⟨ó⟩}{"-osa ⟨ó⟩}{adj.}{Sem decoro; indecente, vergonhoso.}{in.de.co.ro.so}{0}
\verb{indefectível}{}{}{"-eis}{}{adj.2g.}{Que não falha; infalível.}{in.de.fec.tí.vel}{0}
\verb{indefectível}{}{}{"-eis}{}{}{Indestrutível, imperecível.}{in.de.fec.tí.vel}{0}
\verb{indefensável}{}{}{"-eis}{}{adj.2g.}{Que não pode ser defendido.}{in.de.fen.sá.vel}{0}
\verb{indefenso}{}{}{}{}{adj.}{Indefeso, desarmado, fraco.}{in.de.fen.so}{0}
\verb{indeferido}{}{}{}{}{adj.}{Não deferido; que não teve despacho ou que teve despacho contrário ao requerido.}{in.de.fe.ri.do}{0}
\verb{indeferimento}{}{}{}{}{s.m.}{Ato ou efeito de indeferir.}{in.de.fe.ri.men.to}{0}
\verb{indeferir}{}{}{}{}{v.t.}{Não deferir; dar despacho contrário ao requerido.}{in.de.fe.rir}{\verboinum{18}}
\verb{indefeso}{ê}{}{}{}{adj.}{Sem defesa; desarmado, fraco, frágil.}{in.de.fe.so}{0}
\verb{indefesso}{é}{}{}{}{adj.}{Não cansado; incansável.}{in.de.fes.so}{0}
\verb{indefinição}{}{}{"-ões}{}{s.f.}{Estado de quem não se define.}{in.de.fi.ni.ção}{0}
\verb{indefinido}{}{}{}{}{adj.}{Não definido; indeterminado, incerto, genérico, vago.}{in.de.fi.ni.do}{0}
\verb{indefinível}{}{}{"-eis}{}{adj.2g.}{Que não se pode definir.}{in.de.fi.ní.vel}{0}
\verb{indeiscência}{}{Bot.}{}{}{s.f.}{Qualidade de indeiscente.}{in.de.is.cên.cia}{0}
\verb{indeiscente}{}{Bot.}{}{}{adj.2g.}{Diz"-se de órgão vegetal que não se abre ao atingir a maturidade.}{in.de.is.cen.te}{0}
\verb{indelével}{}{}{"-eis}{}{adj.2g.}{Que não se pode destruir.}{in.de.lé.vel}{0}
\verb{indelével}{}{}{"-eis}{}{}{Que não se pode apagar.}{in.de.lé.vel}{0}
\verb{indelicadeza}{ê}{}{}{}{s.f.}{Falta de delicadeza.}{in.de.li.ca.de.za}{0}
\verb{indelicadeza}{ê}{}{}{}{}{Ato ou dito indelicado.}{in.de.li.ca.de.za}{0}
\verb{indelicado}{}{}{}{}{adj.}{Não delicado; rude, grosseiro, inconveniente.}{in.de.li.ca.do}{0}
\verb{indemonstrável}{}{}{"-eis}{}{adj.2g.}{Que não pode ser demonstrado.}{in.de.mons.trá.vel}{0}
\verb{indene}{}{}{}{}{adj.2g.}{Que não sofreu dano; ileso, íntegro.}{in.de.ne}{0}
\verb{indenidade}{}{}{}{}{s.f.}{Qualidade de indene.}{in.de.ni.da.de}{0}
\verb{indenidade}{}{}{}{}{}{Absolvição, perdão.}{in.de.ni.da.de}{0}
\verb{indenização}{}{}{"-ões}{}{s.f.}{Ato ou efeito de indenizar.}{in.de.ni.za.ção}{0}
\verb{indenizar}{}{}{}{}{v.t.}{Dar indenização; compensar, ressarcir.}{in.de.ni.zar}{\verboinum{1}}
\verb{independência}{}{}{}{}{s.f.}{Qualidade ou condição de quem ou do que é independente.}{in.de.pen.dên.cia}{0}
\verb{independência}{}{}{}{}{}{Qualidade de quem rejeita qualquer dependência.}{in.de.pen.dên.cia}{0}
\verb{independência}{}{}{}{}{}{Obtenção da condição de independente.}{in.de.pen.dên.cia}{0}
\verb{independente}{}{}{}{}{adj.2g.}{Que está livre de qualquer dependência; livre, autônomo.}{in.de.pen.den.te}{0}
\verb{independente}{}{}{}{}{}{Que recorre somente aos próprios meios; autossuficiente.}{in.de.pen.den.te}{0}
\verb{independente}{}{}{}{}{}{Diz"-se de entidade, organização, publicação etc. que não é ligada a qualquer outra entidade, partido, ideologia.}{in.de.pen.den.te}{0}
\verb{independer}{ê}{}{}{}{v.t.}{Não depender, não ter relação, não estar subordinado.}{in.de.pen.der}{\verboinum{12}}
\verb{indescritível}{}{}{"-eis}{}{adj.2g.}{Que não se pode descrever.}{in.des.cri.tí.vel}{0}
\verb{indescritível}{}{}{"-eis}{}{}{Extraordinário, espantoso, surpreendente.}{in.des.cri.tí.vel}{0}
\verb{indesculpável}{}{}{"-eis}{}{adj.2g.}{Que não se pode desculpar; imperdoável, injustificável.}{in.des.cul.pá.vel}{0}
\verb{indesejável}{}{}{"-eis}{}{adj.2g.}{Que não se deve ou não se pode desejar.}{in.de.se.já.vel}{0}
\verb{indesejável}{}{Jur.}{"-eis}{}{s.2g.}{Diz"-se de pessoa estrangeira, cuja entrada ou permanência em um país é considerada inconveniente, sendo, portanto, proibida.}{in.de.se.já.vel}{0}
\verb{indestrutibilidade}{}{}{}{}{s.f.}{Qualidade ou condição do que é indestrutível, inabalável.}{in.des.tru.ti.bi.li.da.de}{0}
\verb{indestrutível}{}{}{"-eis}{}{adj.2g.}{Que não pode ser destruído.}{in.des.tru.tí.vel}{0}
\verb{indestrutível}{}{}{"-eis}{}{}{Que deve perdurar por tempo indeterminado; inabalável, inalterável.}{in.des.tru.tí.vel}{0}
\verb{indeterminação}{}{}{"-ões}{}{s.f.}{Falta de determinação; indefinição, imprecisão.}{in.de.ter.mi.na.ção}{0}
\verb{indeterminação}{}{}{"-ões}{}{}{Ausência de vontade; indecisão, perplexidade.}{in.de.ter.mi.na.ção}{0}
\verb{indeterminado}{}{}{}{}{adj.}{Que não é determinado ou fixado com clareza; impreciso, indefinido.}{in.de.ter.mi.na.do}{0}
\verb{indeterminado}{}{}{}{}{}{De significado incerto, ambíguo, vago.}{in.de.ter.mi.na.do}{0}
\verb{indeterminado}{}{}{}{}{}{Que não se decide; vacilante, hesitante.}{in.de.ter.mi.na.do}{0}
\verb{indeterminado}{}{Gram.}{}{}{}{Diz"-se do sujeito que não é conhecido ou que não se deseja identificar.}{in.de.ter.mi.na.do}{0}
\verb{indeterminar}{}{}{}{}{v.t.}{Tornar indeterminado, vago, indefinido.}{in.de.ter.mi.nar}{\verboinum{1}}
\verb{indeterminável}{}{}{"-eis}{}{adj.2g.}{Que não se pode determinar; indefinível.}{in.de.ter.mi.ná.vel}{0}
\verb{indevassável}{}{}{"-eis}{}{adj.2g.}{Que não se pode devassar, conhecer por completo; impenetrável.}{in.de.vas.sá.vel}{0}
\verb{indevido}{}{}{}{}{adj.}{Que não é devido; impróprio, inconveniente.}{in.de.vi.do}{0}
\verb{indevido}{}{}{}{}{}{Que não é merecido; injusto.}{in.de.vi.do}{0}
\verb{índex}{cs}{}{}{}{s.m.}{Índice.}{ín.dex}{0}
\verb{índex}{cs}{}{}{}{}{Lista de livros cuja leitura era proibida pela Igreja Católica.}{ín.dex}{0}
\verb{indexação}{cs}{}{"-ões}{}{s.f.}{Classificação em forma de índice.}{in.de.xa.ção}{0}
\verb{indexação}{cs}{}{"-ões}{}{}{Correção automática de preços, aluguéis, prêmios etc., em função de índices determinados pelo estado ou por consensos econômicos.}{in.de.xa.ção}{0}
\verb{indexar}{cs}{}{}{}{v.t.}{Classificar ou organizar em forma de índice.}{in.de.xar}{0}
\verb{indexar}{cs}{}{}{}{}{Proceder à indexação de preços em função de um índice determinado.}{in.de.xar}{\verboinum{1}}
\verb{indez}{ê}{}{}{}{s.m.}{Ovo que se deixa no ninho para servir de chamariz às galinhas.}{in.dez}{0}
\verb{indez}{ê}{Fig.}{}{}{adj.2g.}{Diz"-se de pessoa muito suscetível ou delicada.}{in.dez}{0}
\verb{indianismo}{}{Liter.}{}{}{s.m.}{Corrente do romantismo literário voltada para a vida dos índios brasileiros.}{in.di.a.nis.mo}{0}
\verb{indianismo}{}{}{}{}{}{Estudo das línguas e das civilizações do subcontinente indiano.}{in.di.a.nis.mo}{0}
\verb{indianismo}{}{}{}{}{}{Vocábulo hindu introduzido em outra língua.}{in.di.a.nis.mo}{0}
\verb{indiano}{}{}{}{}{adj.}{Relativo à Índia.}{in.di.a.no}{0}
\verb{indiano}{}{}{}{}{s.m.}{Indivíduo natural ou habitante desse país; hindu, índio.}{in.di.a.no}{0}
\verb{indicação}{}{}{"-ões}{}{s.f.}{Ato ou efeito de indicar ou apontar.}{in.di.ca.ção}{0}
\verb{indicação}{}{}{"-ões}{}{}{Recomendação, conselho, sugestão.}{in.di.ca.ção}{0}
\verb{indicador}{ô}{}{}{}{adj.}{Que indica, aponta; indicativo.}{in.di.ca.dor}{0}
\verb{indicador}{ô}{}{}{}{}{Diz"-se do dedo localizado entre o polegar e o médio.}{in.di.ca.dor}{0}
\verb{indicador}{ô}{}{}{}{s.m.}{Elemento de referência que serve para medir ou regular um outro elemento variável.}{in.di.ca.dor}{0}
\verb{indicar}{}{}{}{}{v.t.}{Apontar com o dedo ou por meio de algum sinal; mostrar.}{in.di.car}{0}
\verb{indicar}{}{}{}{}{}{Designar, sugerir, recomendar.}{in.di.car}{0}
\verb{indicar}{}{}{}{}{}{Dar a conhecer; mencionar, registrar.}{in.di.car}{0}
\verb{indicar}{}{}{}{}{}{Esclarecer, informar, instruir.}{in.di.car}{\verboinum{2}}
\verb{indicativo}{}{}{}{}{adj.}{Que indica ou aponta; indicador.}{in.di.ca.ti.vo}{0}
\verb{indicativo}{}{Gram.}{}{}{}{Diz"-se do modo verbal em que se expressa uma ação ou estado considerados certos ou reais.}{in.di.ca.ti.vo}{0}
\verb{índice}{}{}{}{}{s.m.}{Lista de capítulos, seções ou assuntos de um livro; sumário.}{ín.di.ce}{0}
\verb{índice}{}{}{}{}{}{Padrão indicador; paradigma, sinal.}{ín.di.ce}{0}
\verb{indiciado}{}{}{}{}{adj.}{Que é notado ou percebido por indícios, sinais.}{in.di.ci.a.do}{0}
\verb{indiciado}{}{}{}{}{s.m.}{Indivíduo sobre o qual recaem indícios de um delito; acusado, pronunciado.}{in.di.ci.a.do}{0}
\verb{indiciar}{}{}{}{}{v.t.}{Fornecer indícios; mostrar por indícios.}{in.di.ci.ar}{0}
\verb{indiciar}{}{}{}{}{}{Acusar, denunciar.}{in.di.ci.ar}{\verboinum{1}}
\verb{indício}{}{}{}{}{s.m.}{Aquilo que indica a provável existência de algo.}{in.dí.cio}{0}
\verb{indício}{}{}{}{}{}{Vestígio, sinal, traço.}{in.dí.cio}{0}
\verb{índico}{}{}{}{}{adj.}{Relativo ao Oceano Índico.}{ín.di.co}{0}
\verb{índico}{}{}{}{}{adj. e s.m.  }{Indiano.}{ín.di.co}{0}
\verb{indiferença}{}{}{}{}{s.f.}{Falta de interesse; descaso, negligência.}{in.di.fe.ren.ça}{0}
\verb{indiferença}{}{}{}{}{}{Ausência de sensibilidade; frieza, distanciamento.}{in.di.fe.ren.ça}{0}
\verb{indiferença}{}{}{}{}{}{Falta de consideração; altivez, orgulho.}{in.di.fe.ren.ça}{0}
\verb{indiferente}{}{}{}{}{adj.2g.}{Que não revela interesse; banal, trivial.}{in.di.fe.ren.te}{0}
\verb{indiferente}{}{}{}{}{}{Que não se comove; insensível, frio.}{in.di.fe.ren.te}{0}
\verb{indiferente}{}{}{}{}{}{Altivo, orgulhoso.}{in.di.fe.ren.te}{0}
\verb{indiferentismo}{}{}{}{}{s.m.}{Atitude de desinteresse sistemático em relação a questões políticas ou religiosas.}{in.di.fe.ren.tis.mo}{0}
\verb{indígena}{}{}{}{}{adj.2g.}{Diz"-se de indivíduo ou povo originário da região ou país onde vive; autóctone, aborígene, nativo.}{in.dí.ge.na}{0}
\verb{indígena}{}{}{}{}{}{Relativo a esse indivíduo ou povo.}{in.dí.ge.na}{0}
\verb{indigência}{}{}{}{}{s.f.}{Situação de extrema pobreza; penúria, miséria.}{in.di.gên.cia}{0}
\verb{indigência}{}{Fig.}{}{}{}{Necessidade, carência, privação.}{in.di.gên.cia}{0}
\verb{indigenismo}{}{}{}{}{s.m.}{Qualidade ou condição do que é indígena.}{in.di.ge.nis.mo}{0}
\verb{indigenismo}{}{}{}{}{}{Conjunto de estudos ou conhecimento a respeito dos índios brasileiros.}{in.di.ge.nis.mo}{0}
\verb{indigente}{}{}{}{}{adj.2g.}{Que vive em extrema penúria; miserável, pobre, necessitado.}{in.di.gen.te}{0}
\verb{indigestão}{}{Med.}{"-ões}{}{s.f.}{Perturbação nas funções do aparelho digestivo provocando indisposição, náusea, vômito, diarreia etc.}{in.di.ges.tão}{0}
\verb{indigestão}{}{}{"-ões}{}{}{Estado de saturação provocado por excesso; saciedade.}{in.di.ges.tão}{0}
\verb{indigesto}{é}{}{}{}{adj.}{Que é difícil de digerir ou que causa indigestão.}{in.di.ges.to}{0}
\verb{indigesto}{é}{Fig.}{}{}{}{Que é difícil de assimilar; confuso, desconexo.}{in.di.ges.to}{0}
\verb{indigitado}{}{}{}{}{adj.}{Apontado com o dedo; indicado.}{in.di.gi.ta.do}{0}
\verb{indigitado}{}{}{}{}{}{Que se propõe; designado, recomendado.}{in.di.gi.ta.do}{0}
\verb{indigitado}{}{}{}{}{}{Que é apontado como culpado; indiciado.}{in.di.gi.ta.do}{0}
\verb{indigitar}{}{}{}{}{v.t.}{Apontar com o dedo; indicar.}{in.di.gi.tar}{0}
\verb{indigitar}{}{}{}{}{}{Recomendar, propor, designar.}{in.di.gi.tar}{0}
\verb{indigitar}{}{}{}{}{}{Apontar como culpado; indiciar.}{in.di.gi.tar}{\verboinum{1}}
\verb{indignação}{}{}{"-ões}{}{s.f.}{Sentimento de desprezo ou cólera despertado por uma ofensa ou uma ação injusta.}{in.dig.na.ção}{0}
\verb{indignação}{}{}{"-ões}{}{}{Ódio intenso, ira, raiva.}{in.dig.na.ção}{0}
\verb{indignado}{}{}{}{}{adj.}{Tomado pela indignação; revoltado.}{in.dig.na.do}{0}
\verb{indignado}{}{}{}{}{}{Que demonstra indignação; colérico, enfurecido.}{in.dig.na.do}{0}
\verb{indignar}{}{}{}{}{v.t.}{Provocar indignação, revolta.}{in.dig.nar}{0}
\verb{indignar}{}{}{}{}{v.pron.}{Sentir indignação; zangar"-se, enfurecer"-se.}{in.dig.nar}{\verboinum{1}}
\verb{indignidade}{}{}{}{}{s.f.}{Falta de dignidade; indecência, infâmia, baixeza.}{in.dig.ni.da.de}{0}
\verb{indigno}{}{}{}{}{adj.}{Que não é digno; que não merece consideração.}{in.dig.no}{0}
\verb{indigno}{}{}{}{}{}{Desprezível, vil, odioso.}{in.dig.no}{0}
\verb{índigo}{}{}{}{}{s.m.}{Forte tonalidade de azul; anil.}{ín.di.go}{0}
\verb{índio}{}{}{}{}{s.m.}{Indivíduo que pertence a um grupo indígena e é reconhecido como membro por este.}{ín.dio}{0}
\verb{índio}{}{}{}{}{adj. e s.m.  }{Indiano.}{ín.dio}{0}
\verb{índio}{}{Quím.}{}{}{s.m.}{Elemento químico metálico, brilhante, maleável, semelhante ao alumínio e ao gálio, usado em ligas de prata como condutor e em soldas especiais. \elemento{49}{114.818}{In}.}{ín.dio}{0}
\verb{indireta}{é}{}{}{}{s.f.}{Observação disfarçada, ambígua, não explícita.}{in.di.re.ta}{0}
\verb{indireto}{é}{}{}{}{adj.}{Que não é direto, nem reto.}{in.di.re.to}{0}
\verb{indireto}{é}{}{}{}{}{Que se faz com desvios ou rodeios; oblíquo, torto.}{in.di.re.to}{0}
\verb{indireto}{é}{}{}{}{}{Disfarçado, dissimulado, ambíguo.}{in.di.re.to}{0}
\verb{indireto}{é}{Gram.}{}{}{}{Diz"-se do complemento que é regido por preposição.}{in.di.re.to}{0}
\verb{indiscernível}{}{}{"-eis}{}{adj.2g.}{Que não se pode discernir ou perceber com clareza; indistinto.}{in.dis.cer.ní.vel}{0}
\verb{indisciplina}{}{}{}{}{s.f.}{Falta de disciplina; desobediência, rebeldia, insubordinação.}{in.dis.ci.pli.na}{0}
\verb{indisciplinado}{}{}{}{}{adj.}{Que não segue disciplina, regulamento; rebelde, insubordinado.}{in.dis.ci.pli.na.do}{0}
\verb{indisciplinado}{}{}{}{}{}{Que não tem organização; bagunceiro.}{in.dis.ci.pli.na.do}{0}
\verb{indisciplinar}{}{}{}{}{v.t.}{Provocar a indisciplina, a desordem; rebelar, subverter.}{in.dis.ci.pli.nar}{0}
\verb{indisciplinar}{}{}{}{}{}{Diminuir a rigidez; afrouxar, relaxar.}{in.dis.ci.pli.nar}{\verboinum{1}}
\verb{indiscreto}{é}{}{}{}{adj.}{Que não é discreto; inconveniente, imprudente.}{in.dis.cre.to}{0}
\verb{indiscreto}{é}{}{}{}{}{Que revela um segredo ou quebra um sigilo; inconfidente.}{in.dis.cre.to}{0}
\verb{indiscreto}{é}{}{}{}{}{Bisbilhoteiro, intrometido, tagarela.}{in.dis.cre.to}{0}
\verb{indiscrição}{}{}{"-ões}{}{s.f.}{Falta de discrição; impropriedade, inconveniência.}{in.dis.cri.ção}{0}
\verb{indiscrição}{}{}{"-ões}{}{}{Revelação de segredo; quebra de sigilo; inconfidência.}{in.dis.cri.ção}{0}
\verb{indiscrição}{}{}{"-ões}{}{}{Bisbilhotice, mexerico.}{in.dis.cri.ção}{0}
\verb{indiscriminado}{}{}{}{}{adj.}{Que não está discriminado; indistinto, confuso.}{in.dis.cri.mi.na.do}{0}
\verb{indiscutível}{}{}{"-eis}{}{}{Que não admite discussão por ser evidente, por não haver dúvida; incontestável.}{in.dis.cu.tí.vel}{0}
\verb{indiscutível}{}{}{"-eis}{}{adj.2g.}{Que não se pode ou não se deve discutir.}{in.dis.cu.tí.vel}{0}
\verb{indisfarçável}{}{}{"-eis}{}{adj.2g.}{Que não se pode disfarçar ou dissimular.}{in.dis.far.çá.vel}{0}
\verb{indispensável}{}{}{"-eis}{}{adj.2g.}{Que não se pode dispensar; imprescindível, essencial.}{in.dis.pen.sá.vel}{0}
\verb{indisponível}{}{}{"-eis}{}{adj.2g.}{Que não está à disposição.}{in.dis.po.ní.vel}{0}
\verb{indispor}{}{}{}{}{v.t.}{Modificar levemente a saúde.}{in.dis.por}{0}
\verb{indispor}{}{}{}{}{}{Atrair a inimizade de alguém; aborrecer, malquistar.}{in.dis.por}{\verboinum{60}}
\verb{indispor}{}{}{}{}{}{Tornar pouco favorável; descontentar, irritar.}{in.dis.por}{0}
\verb{indisposição}{}{}{"-ões}{}{s.f.}{Ato ou efeito de indispor; falta de disposição ou vontade.}{in.dis.po.si.ção}{0}
\verb{indisposição}{}{}{"-ões}{}{}{Leve enfermidade; mal"-estar.}{in.dis.po.si.ção}{0}
\verb{indisposição}{}{}{"-ões}{}{}{Falta de entendimento; conflito, desavença.}{in.dis.po.si.ção}{0}
\verb{indisposição}{}{}{"-ões}{}{}{Má vontade; aversão, repulsão.}{in.dis.po.si.ção}{0}
\verb{indisposto}{ô}{}{"-s ⟨ó⟩}{"-a ⟨ó⟩}{adj.}{Levemente doente; adoentado, incomodado.}{in.dis.pos.to}{0}
\verb{indisposto}{ô}{}{"-s ⟨ó⟩}{"-a ⟨ó⟩}{}{Irritado, agastado, mal"-humorado.}{in.dis.pos.to}{0}
\verb{indisputável}{}{}{"-eis}{}{adj.2g.}{Que não se pode disputar; indiscutível, incontestável.}{in.dis.pu.tá.vel}{0}
\verb{indissimulável}{}{}{"-eis}{}{adj.2g.}{Que não se pode dissimular; indisfarçável.}{in.dis.si.mu.lá.vel}{0}
\verb{indissociabilidade}{}{}{}{}{s.f.}{Qualidade do que é indissociável, inseparável.}{in.dis.so.ci.a.bi.li.da.de}{0}
\verb{indissociável}{}{}{"-eis}{}{adj.2g.}{Que não se pode dissociar; inseparável.}{in.dis.so.ci.á.vel}{0}
\verb{indissolubilidade}{}{}{}{}{s.f.}{Qualidade do que é indissolúvel; insolubilidade.}{in.dis.so.lu.bi.li.da.de}{0}
\verb{indissolúvel}{}{}{"-eis}{}{adj.2g.}{Que não se pode dissolver; insolúvel.}{in.dis.so.lú.vel}{0}
\verb{indistinguível}{}{}{"-eis}{}{adj.2g.}{Que não se pode distinguir, diferenciar.}{in.dis.tin.guí.vel}{0}
\verb{indistinto}{}{}{}{}{adj.}{Que não se pode distinguir; incerto, indeterminado, vago.}{in.dis.tin.to}{0}
\verb{inditoso}{ô}{}{"-osos ⟨ó⟩}{"-osa ⟨ó⟩}{adj.}{Que foi atingido pela má sorte; desventurado, infeliz, desditoso.}{in.di.to.so}{0}
\verb{individual}{}{}{"-ais}{}{adj.2g.}{Que se refere ao indivíduo.}{in.di.vi.du.al}{0}
\verb{individual}{}{}{"-ais}{}{}{Feito ou executado por um só indivíduo; pessoal.}{in.di.vi.du.al}{0}
\verb{individual}{}{}{"-ais}{}{}{Que diz respeito a uma só pessoa; singular, particular, único.}{in.di.vi.du.al}{0}
\verb{individualidade}{}{}{}{}{s.f.}{Conjunto de características e atributos que distinguem um indivíduo.}{in.di.vi.du.a.li.da.de}{0}
\verb{individualidade}{}{}{}{}{}{Originalidade, personalidade, unicidade.}{in.di.vi.du.a.li.da.de}{0}
\verb{individualismo}{}{}{}{}{s.m.}{Tendência de quem demonstra pouca ou nenhuma solidariedade, vivendo exclusivamente para si; egoísmo.}{in.di.vi.du.a.lis.mo}{0}
\verb{individualismo}{}{}{}{}{}{Doutrina que valoriza a autonomia individual, sem considerar os interesses coletivos.}{in.di.vi.du.a.lis.mo}{0}
\verb{individualista}{}{}{}{}{adj.2g.}{Que revela atitudes egocêntricas; egoísta.}{in.di.vi.du.a.lis.ta}{0}
\verb{individualista}{}{}{}{}{}{Que é partidário do individualismo.}{in.di.vi.du.a.lis.ta}{0}
\verb{individualização}{}{}{"-ões}{}{s.f.}{Ato ou efeito de individualizar.}{in.di.vi.du.a.li.za.ção}{0}
\verb{individualização}{}{}{"-ões}{}{}{Distinção, particularização.}{in.di.vi.du.a.li.za.ção}{0}
\verb{individualizar}{}{}{}{}{v.t.}{Considerar uma pessoa isoladamente; distinguir, individuar.}{in.di.vi.du.a.li.zar}{0}
\verb{individualizar}{}{}{}{}{}{Particularizar, especializar, caracterizar.}{in.di.vi.du.a.li.zar}{\verboinum{1}}
\verb{individuar}{}{}{}{}{v.t.}{Narrar ou expor com detalhes; especificar, discriminar.}{in.di.vi.du.ar}{0}
\verb{individuar}{}{}{}{}{}{Individualizar, distinguir.}{in.di.vi.du.ar}{\verboinum{1}}
\verb{indivíduo}{}{}{}{}{s.m.}{Qualquer ser concreto, animal ou vegetal, em relação à sua espécie.}{in.di.ví.du.o}{0}
\verb{indivíduo}{}{}{}{}{}{O ser humano considerado isoladamente em relação à coletividade.}{in.di.ví.du.o}{0}
\verb{indivíduo}{}{}{}{}{}{Uma pessoa qualquer, indeterminada; anônimo.}{in.di.ví.du.o}{0}
\verb{indivíduo}{}{}{}{}{adj.}{Que não é dividido; indivisível, indiviso.}{in.di.ví.du.o}{0}
\verb{indivisível}{}{}{"-eis}{}{adj.2g.}{Que não se pode dividir ou decompor; indissociável, inseparável.}{in.di.vi.sí.vel}{0}
\verb{indiviso}{}{}{}{}{adj.}{Que não apresenta qualquer divisão; inteiro.}{in.di.vi.so}{0}
\verb{indiviso}{}{}{}{}{}{Que não pode ser dividido; indecomponível.}{in.di.vi.so}{0}
\verb{indiviso}{}{}{}{}{}{Que pertence a várias pessoas e que não pode ser materialmente desmembrado.}{in.di.vi.so}{0}
\verb{indiviso}{}{}{}{}{}{Que detém a posse em comum de um ou mais bens.}{in.di.vi.so}{0}
\verb{indizível}{}{}{"-eis}{}{adj.2g.}{Que não se pode dizer; inefável.}{in.di.zí.vel}{0}
\verb{indizível}{}{}{"-eis}{}{}{Que é extraordinário, raro, incomum.}{in.di.zí.vel}{0}
\verb{indochinês}{}{}{}{}{adj.}{Relativo à Indochina, península do sudoeste da Ásia.}{in.do.chi.nês}{0}
\verb{indochinês}{}{}{}{}{s.m.}{Indivíduo natural ou habitante da Indochina.}{in.do.chi.nês}{0}
\verb{indócil}{}{}{"-eis}{}{adj.2g.}{Que não é dócil; pouco meigo ou submisso; rebelde.}{in.dó.cil}{0}
\verb{indócil}{}{}{"-eis}{}{}{Que é indisciplinável, indomável.}{in.dó.cil}{0}
\verb{indócil}{}{}{"-eis}{}{}{Que está em estado de impaciência; inquieto, irritado.}{in.dó.cil}{0}
\verb{indo"-europeu}{}{}{indo"-europeus}{indo"-europeia}{adj.}{Diz"-se de indivíduo de povos antigos que se expandiram da Ásia Central para Europa, Ásia e Índia.}{in.do"-eu.ro.peu}{0}
\verb{indo"-europeu}{}{}{indo"-europeus}{indo"-europeia}{s.m.}{Tronco linguístico que abrange línguas da Europa e de parte da Índia.}{in.do"-eu.ro.peu}{0}
\verb{indo"-europeu}{}{}{indo"-europeus}{indo"-europeia}{}{Língua que teria dado origem a essas línguas.}{in.do"-eu.ro.peu}{0}
\verb{índole}{}{}{}{}{s.f.}{Conjunto de traços e qualidades inerentes ao indivíduo desde o seu nascimento; caráter, temperamento.}{ín.do.le}{0}
\verb{índole}{}{Fig.}{}{}{}{Conjunto de traços ou características; tipo específico, feitio.}{ín.do.le}{0}
\verb{indolência}{}{}{}{}{s.f.}{Ausência de dor.}{in.do.lên.cia}{0}
\verb{indolência}{}{}{}{}{}{Caráter do que revela indiferença, apatia; distanciamento.}{in.do.lên.cia}{0}
\verb{indolência}{}{}{}{}{}{Ócio, preguiça.}{in.do.lên.cia}{0}
\verb{indolente}{}{}{}{}{adj.2g.}{Que não é doloroso; indolor.}{in.do.len.te}{0}
\verb{indolente}{}{}{}{}{}{Que é insensível, indiferente, apático.}{in.do.len.te}{0}
\verb{indolente}{}{}{}{}{}{Que é vagaroso, preguiçoso.}{in.do.len.te}{0}
\verb{indolente}{}{}{}{}{}{Que é descuidado, negligente.}{in.do.len.te}{0}
\verb{indolor}{ô}{}{}{}{adj.2g.}{Que não provoca dor.}{in.do.lor}{0}
\verb{indolor}{ô}{Fig.}{}{}{}{Que é realizado com facilidade, sem esforço; leve, suave.}{in.do.lor}{0}
\verb{indomável}{}{}{"-eis}{}{adj.2g.}{Que não se consegue domar.}{in.do.má.vel}{0}
\verb{indomável}{}{}{"-eis}{}{}{Que é invencível, irredutível.}{in.do.má.vel}{0}
\verb{indomesticável}{}{}{"-eis}{}{adj.2g.}{Que não se pode domesticar.}{in.do.mes.ti.cá.vel}{0}
\verb{indomesticável}{}{}{"-eis}{}{}{Que é rude, selvagem, bravio.}{in.do.mes.ti.cá.vel}{0}
\verb{indômito}{}{}{}{}{adj.}{Que não é domesticado; indomado.}{in.dô.mi.to}{0}
\verb{indômito}{}{}{}{}{}{Que é invencível, indomável.}{in.dô.mi.to}{0}
\verb{indômito}{}{Fig.}{}{}{}{Que é altivo, soberbo.}{in.dô.mi.to}{0}
\verb{indonésio}{}{}{}{}{adj.}{Relativo à Indonésia.}{in.do.né.sio}{0}
\verb{indonésio}{}{}{}{}{s.m.}{Indivíduo natural ou habitante desse país.}{in.do.né.sio}{0}
\verb{indonésio}{}{}{}{}{}{Um dos idiomas falados nesse país.}{in.do.né.sio}{0}
\verb{indouto}{ô}{}{}{}{adj.}{Que tem pouco saber; que não tem erudição.}{in.dou.to}{0}
\verb{indouto}{ô}{}{}{}{s.m.}{Indivíduo que não é douto, que não tem erudição.}{in.dou.to}{0}
\verb{indubitável}{}{}{"-eis}{}{adj.2g.}{Que não pode ser objeto de dúvida; incontestável, indiscutível.}{in.du.bi.tá.vel}{0}
\verb{indução}{}{}{"-ões}{}{s.f.}{Ato ou efeito de induzir.}{in.du.ção}{0}
\verb{indução}{}{}{"-ões}{}{}{Processo de raciocínio pelo qual se parte de fatos particulares para estabelecer leis gerais.}{in.du.ção}{0}
\verb{indução}{}{}{"-ões}{}{}{Incentivo, estímulo.}{in.du.ção}{0}
\verb{indução}{}{}{"-ões}{}{}{Processo medicamentoso para estimular o trabalho de parto.}{in.du.ção}{0}
\verb{indulgência}{}{}{}{}{s.f.}{Qualidade de indulgente.}{in.dul.gên.cia}{0}
\verb{indulgência}{}{}{}{}{}{Clemência, misericórdia.}{in.dul.gên.cia}{0}
\verb{indulgência}{}{}{}{}{}{Tolerância, benevolência.}{in.dul.gên.cia}{0}
\verb{indulgência}{}{}{}{}{}{Remição das penas; perdão.}{in.dul.gên.cia}{0}
\verb{indulgenciar}{}{}{}{}{v.t.}{Proceder sem rigor.}{in.dul.gen.ci.ar}{0}
\verb{indulgenciar}{}{}{}{}{}{Conceder perdão; desculpar, perdoar.}{in.dul.gen.ci.ar}{\verboinum{6}}
\verb{indulgente}{}{}{}{}{adj.2g.}{Que está pronto a perdoar; tolerante.}{in.dul.gen.te}{0}
\verb{indulgente}{}{}{}{}{}{Que é benigno, complacente.}{in.dul.gen.te}{0}
\verb{indulgente}{}{}{}{}{}{Que denota ou envolve indulgência.}{in.dul.gen.te}{0}
\verb{indultar}{}{}{}{}{v.t.}{Conceder indulto, privilégio; suavizar castigo.}{in.dul.tar}{0}
\verb{indultar}{}{}{}{}{}{Perdoar, relevar, desculpar.}{in.dul.tar}{\verboinum{1}}
\verb{indulto}{}{}{}{}{s.m.}{Clemência com relação a faltas; perdão, desculpa.}{in.dul.to}{0}
\verb{indulto}{}{Jur.}{}{}{}{Perdão, redução ou comutação de pena de um ou mais sentenciados, concedidos espontaneamente por autoridades administrativas.}{in.dul.to}{0}
\verb{indulto}{}{Jur.}{}{}{}{Decreto pelo qual se concede o indulto.}{in.dul.to}{0}
\verb{indumentária}{}{}{}{}{s.f.}{Arte do vestuário.}{in.du.men.tá.ria}{0}
\verb{indumentária}{}{}{}{}{}{História do vestuário; uso do vestuário em relação às épocas ou povos.}{in.du.men.tá.ria}{0}
\verb{indumentária}{}{}{}{}{}{Traje, indumento, vestuário.}{in.du.men.tá.ria}{0}
\verb{indumentário}{}{}{}{}{adj.}{Relativo a indumentária, a vestuário.}{in.du.men.tá.rio}{0}
\verb{indumento}{}{}{}{}{s.m.}{Traje, vestuário.}{in.du.men.to}{0}
\verb{indumento}{}{Bot.}{}{}{}{Qualquer revestimento dos órgãos ou partes vegetais, que pode ser formado de pelos, escamas, glândulas etc. }{in.du.men.to}{0}
\verb{indústria}{}{}{}{}{s.f.}{Destreza ou arte na execução de um trabalho manual; aptidão, perícia.}{in.dús.tria}{0}
\verb{indústria}{}{Fig.}{}{}{}{Invenção, engenho.}{in.dús.tria}{0}
\verb{indústria}{}{Econ.}{}{}{}{A atividade secundária da economia que engloba as atividades de produção ou qualquer de seus ramos, em contraposição à atividade agrícola e à prestação de serviço.}{in.dús.tria}{0}
\verb{indústria}{}{}{}{}{}{O conjunto das empresas industriais; o complexo industrial.}{in.dús.tria}{0}
\verb{indústria}{}{}{}{}{}{Manufatura, fábrica.}{in.dús.tria}{0}
\verb{industrial}{}{}{"-ais}{}{adj.2g.}{Relativo a indústria.}{in.dus.tri.al}{0}
\verb{industrial}{}{}{"-ais}{}{}{Produzido pela indústria.}{in.dus.tri.al}{0}
\verb{industrial}{}{}{"-ais}{}{s.2g.}{Indivíduo proprietário ou administrador de indústria.}{in.dus.tri.al}{0}
\verb{industrialização}{}{}{"-ões}{}{s.f.}{Ato ou efeito de industrializar; aplicação de técnicas industriais.}{in.dus.tri.a.li.za.ção}{0}
\verb{industrialização}{}{}{"-ões}{}{}{Desenvolvimento com base na indústria.}{in.dus.tri.a.li.za.ção}{0}
\verb{industrializar}{}{}{}{}{v.t.}{Promover desenvolvimento industrial.}{in.dus.tri.a.li.zar}{0}
\verb{industrializar}{}{}{}{}{}{Dar caráter industrial; tornar industrial.}{in.dus.tri.a.li.zar}{0}
\verb{industrializar}{}{}{}{}{}{Aproveitar algo como matéria"-prima industrial.}{in.dus.tri.a.li.zar}{\verboinum{1}}
\verb{industriar}{}{}{}{}{v.t.}{Instruir previamente; explicar, ensinar.}{in.dus.tri.ar}{\verboinum{1}}
\verb{industriário}{}{}{}{}{s.m.}{Funcionário de indústria; operário.}{in.dus.tri.á.rio}{0}
\verb{industrioso}{ô}{}{"-osos ⟨ó⟩}{"-osa ⟨ó⟩}{adj.}{Que tem indústria; que tem capacidade de ação; ativo, trabalhador.}{in.dus.tri.o.so}{0}
\verb{industrioso}{ô}{}{"-osos ⟨ó⟩}{"-osa ⟨ó⟩}{}{Que é executado com indústria, com arte.}{in.dus.tri.o.so}{0}
\verb{industrioso}{ô}{}{"-osos ⟨ó⟩}{"-osa ⟨ó⟩}{}{Que é hábil, esperto, astuto.}{in.dus.tri.o.so}{0}
\verb{indutância}{}{Fís. e Quím.}{}{}{s.f.}{Propriedade de indução de força eletromotriz em um circuito, por efeito da variação de uma corrente que passa pelo próprio circuito ou por um circuito próximo.}{in.du.tân.cia}{0}
\verb{indutivo}{}{}{}{}{adj.}{Relativo a indução.}{in.du.ti.vo}{0}
\verb{indutivo}{}{}{}{}{}{Que procede por indução.}{in.du.ti.vo}{0}
\verb{indutivo}{}{}{}{}{}{Que induz, estimula, incentiva.}{in.du.ti.vo}{0}
\verb{indutor}{ô}{}{}{}{adj.}{Que induz, incita.}{in.du.tor}{0}
\verb{indutor}{ô}{}{}{}{s.m.}{Aquilo que induz, que instiga.}{in.du.tor}{0}
\verb{indutor}{ô}{Fís.}{}{}{}{Componente passivo de um circuito elétrico, que tem a função de introduzir neste circuito uma indutância.}{in.du.tor}{0}
\verb{induzimento}{}{}{}{}{s.m.}{Ato ou efeito de induzir; indução, incitação.}{in.du.zi.men.to}{0}
\verb{induzir}{}{}{}{}{v.t.}{Levar alguém a fazer alguma coisa; conduzir, instigar.}{in.du.zir}{0}
\verb{induzir}{}{}{}{}{}{Chegar a uma conclusão geral a partir da observação de fatos particulares; concluir, inferir.}{in.du.zir}{\verboinum{21}}
\verb{inebriante}{}{}{}{}{adj.2g.}{Que embriaga, entontece.}{i.ne.bri.an.te}{0}
\verb{inebriante}{}{Fig.}{}{}{}{Que provoca êxtase.}{i.ne.bri.an.te}{0}
\verb{inebriar}{}{}{}{}{v.t.}{Tornar ébrio; embriagar, embebedar.}{i.ne.bri.ar}{0}
\verb{inebriar}{}{Fig.}{}{}{}{Entusiasmar, extasiar.}{i.ne.bri.ar}{\verboinum{6}}
\verb{ineditismo}{}{}{}{}{s.m.}{Qualidade do que é inédito.}{i.ne.di.tis.mo}{0}
\verb{inédito}{}{}{}{}{adj.}{Que não foi publicado ou impresso.}{i.né.di.to}{0}
\verb{inédito}{}{Fig.}{}{}{}{Que nunca foi visto; original, incomum.}{i.né.di.to}{0}
\verb{inédito}{}{}{}{}{s.m.}{Obra que ainda não foi publicada.}{i.né.di.to}{0}
\verb{inefável}{}{}{"-eis}{}{adj.2g.}{Que não se pode exprimir por meio de palavras; indizível.}{i.ne.fá.vel}{0}
\verb{inefável}{}{Por ext.}{"-eis}{}{}{Encantador, inebriante.}{i.ne.fá.vel}{0}
\verb{ineficácia}{}{}{}{}{s.f.}{Qualidade de ineficaz; falta de eficácia.}{i.ne.fi.cá.cia}{0}
\verb{ineficaz}{}{}{}{}{adj.2g.}{Que não produz o efeito desejado.}{i.ne.fi.caz}{0}
\verb{ineficaz}{}{}{}{}{}{Que é impróprio, inconveniente.}{i.ne.fi.caz}{0}
\verb{ineficiência}{}{}{}{}{s.f.}{Falta de eficiência; inutilidade.}{i.ne.fi.ci.ên.cia}{0}
\verb{ineficiente}{}{}{}{}{adj.2g.}{Que é desprovido de eficiência.}{i.ne.fi.ci.en.te}{0}
\verb{inegável}{}{}{"-eis}{}{adj.2g.}{Que não se pode negar; incontestável, claro, evidente.}{i.ne.gá.vel}{0}
\verb{inegociável}{}{}{"-eis}{}{adj.2g.}{Que não se pode negociar.}{i.ne.go.ci.á.vel}{0}
\verb{inelegibilidade}{}{}{}{}{s.f.}{Qualidade de inelegível.}{i.ne.le.gi.bi.li.da.de}{0}
\verb{inelegível}{}{}{"-eis}{}{adj.2g.}{Que não se pode eleger.}{i.ne.le.gí.vel}{0}
\verb{inelutável}{}{}{"-eis}{}{adj.2g.}{Com que se luta em vão; invencível, irresistível.}{i.ne.lu.tá.vel}{0}
\verb{inelutável}{}{}{"-eis}{}{}{Que é indiscutível, irrefutável.}{i.ne.lu.tá.vel}{0}
\verb{inenarrável}{}{}{"-eis}{}{adj.2g.}{Inarrável.}{i.ne.nar.rá.vel}{0}
\verb{inépcia}{}{}{}{}{s.f.}{Falta de inteligência; imbecilidade.}{i.nép.cia}{0}
\verb{inépcia}{}{}{}{}{}{Falta de aptidão; incapacidade.}{i.nép.cia}{0}
\verb{inépcia}{}{}{}{}{}{Dito ou afirmação absurda; despropósito, disparate.}{i.nép.cia}{0}
\verb{inepto}{é}{}{}{}{adj.}{A quem falta aptidão.}{i.nep.to}{0}
\verb{inepto}{é}{}{}{}{}{A quem falta inteligência; imbecil, idiota.}{i.nep.to}{0}
\verb{inepto}{é}{Jur.}{}{}{}{Que não produz efeitos jurídicos por não atender às exigências legais.}{i.nep.to}{0}
\verb{inepto}{é}{}{}{}{s.m.}{Indivíduo sem nenhuma aptidão.}{i.nep.to}{0}
\verb{inepto}{é}{}{}{}{}{Indivíduo que não tem inteligência; tolo.}{i.nep.to}{0}
\verb{inequação}{}{Mat.}{"-ões}{}{s.f.}{Relação entre os membros de um conjunto, que envolve os sinais de \textit{maior que} ou \textit{menor que}; desigualdade.}{i.ne.qua.ção}{0}
\verb{inequívoco}{}{}{}{}{adj.}{Em que não há equívoco ou ambiguidade; claro, evidente.}{ine.quí.vo.co}{0}
\verb{inércia}{}{}{}{}{s.f.}{Ausência de reação; imobilismo, inatividade.}{i.nér.cia}{0}
\verb{inércia}{}{}{}{}{}{Estado de abatimento; falta de energia; apatia, indolência.}{i.nér.cia}{0}
\verb{inércia}{}{Fís.}{}{}{}{Resistência que a matéria oferece a alguma força que intervenha em seu estado de repouso ou de movimento.}{i.nér.cia}{0}
\verb{inerência}{}{}{}{}{s.f.}{Estado de coisas que são inseparáveis por natureza; indissociabilidade.}{i.ne.rên.cia}{0}
\verb{inerente}{}{}{}{}{adj.2g.}{Que está, por natureza, ligado intimamente a alguma coisa ou pessoa; indissociável, inseparável.}{i.ne.ren.te}{0}
\verb{inerme}{é}{}{}{}{adj.2g.}{Que não tem armas ou meios de defesa; desarmado, indefeso.}{i.ner.me}{0}
\verb{inerme}{é}{Bot.}{}{}{}{Diz"-se de vegetal que não possui espinhos.}{i.ner.me}{0}
\verb{inerte}{é}{}{}{}{adj.2g.}{Sem atividade ou sem movimentos próprios.}{i.ner.te}{0}
\verb{inerte}{é}{}{}{}{}{Carente de energia; apático, ocioso.}{i.ner.te}{0}
\verb{inerte}{é}{}{}{}{}{Que produz inércia, apatia.}{i.ner.te}{0}
\verb{inervar}{}{}{}{}{v.t.}{Prover com nervos certa parte do organismo.}{i.ner.var}{0}
\verb{inervar}{}{}{}{}{}{Atravessar, atingir um órgão do corpo.}{i.ner.var}{\verboinum{1}}
\verb{inescrupuloso}{ô}{}{"-osos ⟨ó⟩}{"-osa ⟨ó⟩}{adj.}{Que não tem escrúpulos; desonesto.}{i.nes.cru.pu.lo.so}{0}
\verb{inescrutável}{}{}{"-eis}{}{adj.2g.}{Que não se pode investigar, compreender; impenetrável, insondável.}{i.nes.cru.tá.vel}{0}
\verb{inescusável}{}{}{"-eis}{}{adj.2g.}{Que não se dispensa; imprescindível, indispensável.}{i.nes.cu.sá.vel}{0}
\verb{inescusável}{}{}{"-eis}{}{}{Em que não há desculpa; imperdoável, indesculpável.}{i.nes.cu.sá.vel}{0}
\verb{inesgotável}{}{}{"-eis}{}{adj. e s.f.}{Que não se pode esgotar; inexaurível.}{i.nes.go.tá.vel}{0}
\verb{inesgotável}{}{}{"-eis}{}{adj.2g.}{De fonte ilimitada; copioso, abundante.}{i.nes.go.tá.vel}{0}
\verb{inesperado}{}{}{}{}{adj.}{Que não se espera ou não se esperava; imprevisto, inopinado.}{i.nes.pe.ra.do}{0}
\verb{inesperado}{}{}{}{}{}{Que muda repentinamente de aspecto ou de enfoque.}{i.nes.pe.ra.do}{0}
\verb{inesquecível}{}{}{"-eis}{}{adj.2g.}{Que não se pode esquecer; inolvidável, memorável.}{i.nes.que.cí.vel}{0}
\verb{inestimável}{}{}{"-eis}{}{adj.2g.}{Que não se pode estimar ou avaliar; incalculável.}{i.nes.ti.má.vel}{0}
\verb{inestimável}{}{}{"-eis}{}{}{De grande valor ou de grande estima; precioso.}{i.nes.ti.má.vel}{0}
\verb{inevitável}{}{}{"-eis}{}{adj.2g.}{Que não se pode evitar ou impedir; fatal.}{i.ne.vi.tá.vel}{0}
\verb{inexatidão}{z}{}{"-ões}{}{s.f.}{Falta de exatidão, de precisão; desajuste, erro.}{i.ne.xa.ti.dão}{0}
\verb{inexatidão}{z}{}{"-ões}{}{}{Aquilo que é inexato, falso.}{i.ne.xa.ti.dão}{0}
\verb{inexato}{z}{}{}{}{adj.}{Que não é exato; que contém erros; falso, errôneo.}{i.ne.xa.to}{0}
\verb{inexaurível}{z}{}{"-eis}{}{adj.2g.}{Que não se pode exaurir; inesgotável.}{i.ne.xau.rí.vel}{0}
\verb{inexcedível}{s}{}{"-eis}{}{adj.2g.}{Que não se pode exceder ou ultrapassar; insuperável.}{i.nex.ce.dí.vel}{0}
\verb{inexecutável}{z}{}{"-eis}{}{adj.2g.}{Que não pode ser executado; inexequível.}{i.ne.xe.cu.tá.vel}{0}
\verb{inexequível}{z}{}{"-eis}{}{adj.2g.}{Que não pode ser executado; inexecutável, irrealizável.}{i.ne.xe.quí.vel}{0}
\verb{inexistência}{z}{}{}{}{s.f.}{Não existência; ausência, falta.}{i.ne.xis.tên.cia}{0}
\verb{inexistente}{z}{}{}{}{adj.2g.}{Que não existe; nulo, irreal.}{i.ne.xis.ten.te}{0}
\verb{inexistir}{z}{}{}{}{v.i.}{Não existir, não haver.}{i.ne.xis.tir}{\verboinum{18}}
\verb{inexorável}{z}{}{"-eis}{}{adj.2g.}{Que não se abala ou se deixa mover diante de rogos ou súplicas; implacável, inflexível.}{i.ne.xo.rá.vel}{0}
\verb{inexorável}{z}{}{"-eis}{}{}{Cujo rigor não pode ser amenizado; severo, reto.}{i.ne.xo.rá.vel}{0}
\verb{inexperiência}{s}{}{}{}{s.f.}{Falta de experiência; imperícia, engano.}{i.nex.pe.ri.ên.cia}{0}
\verb{inexperiente}{s}{}{}{}{adj.2g.}{Que não tem experiência ou prática em algo; imperito.}{i.nex.pe.ri.en.te}{0}
\verb{inexperiente}{s}{}{}{}{}{Sem malícia; ingênuo, inocente.}{i.nex.pe.ri.en.te}{0}
\verb{inexplicável}{s}{}{"-eis}{}{adj.2g.}{Que não se pode explicar; indefinível.}{i.nex.pli.cá.vel}{0}
\verb{inexplicável}{s}{}{"-eis}{}{}{Enigmático, obscuro, incompreensível.}{i.nex.pli.cá.vel}{0}
\verb{inexplorado}{s}{}{}{}{adj.}{Que não foi explorado; desconhecido.}{i.nex.plo.ra.do}{0}
\verb{inexplorável}{s}{}{"-eis}{}{adj.2g.}{Que é impossível ou muito difícil de ser explorado.}{i.nex.plo.rá.vel}{0}
\verb{inexpressável}{s}{}{"-eis}{}{adj.2g.}{Que não pode ser expresso; inexprimível.}{i.nex.pres.sá.vel}{0}
\verb{inexpressivo}{s}{}{}{}{adj.}{Que não indica nada; sem poder de expressão.}{i.nex.pres.si.vo}{0}
\verb{inexpressivo}{s}{}{}{}{}{Que tem pouca importância.}{i.nex.pres.si.vo}{0}
\verb{inexprimível}{s}{}{"-eis}{}{adj.2g.}{Que não se pode exprimir; indescritível, inexpressável.}{i.nex.pri.mí.vel}{0}
\verb{inexprimível}{s}{Fig.}{"-eis}{}{}{Encantador, inefável, delicioso.}{i.nex.pri.mí.vel}{0}
\verb{inexpugnável}{s}{}{"-eis}{}{adj.2g.}{Que não se pode tomar de assalto, nem se apoderar pela força, inconquistável.}{i.nex.pug.ná.vel}{0}
\verb{inexpugnável}{s}{}{"-eis}{}{}{Que resiste a qualquer investida ou ameaça; inabalável.}{i.nex.pug.ná.vel}{0}
\verb{inextensível}{s}{}{"-eis}{}{adj.2g.}{Que não se pode estender ou esticar.}{i.nex.ten.sí.vel}{0}
\verb{inextinguível}{s}{}{"-eis}{}{adj.2g.}{Que não se pode extinguir; indestrutível.}{i.nex.tin.guí.vel}{0}
\verb{inextinto}{s}{}{}{}{adj.}{Que não está extinto; subsistente, duradouro.}{i.nex.tin.to}{0}
\verb{inextirpável}{s}{}{"-eis}{}{adj.2g.}{Que não se pode extirpar ou arrancar; indestrutível.}{i.nex.tir.pá.vel}{0}
\verb{inextricável}{s}{}{"-eis}{}{adj.2g.}{Que não se pode desembaraçar ou dissociar; emaranhado.}{i.nex.tri.cá.vel}{0}
\verb{inextricável}{s}{}{"-eis}{}{}{Que não se pode resolver; indecifrável.}{i.nex.tri.cá.vel}{0}
\verb{inextrincável}{s}{}{}{}{}{Var. de \textit{inextricável}.}{i.nex.trin.cá.vel}{0}
\verb{infalibilidade}{}{}{}{}{s.f.}{Qualidade do que é infalível, do que nunca erra.}{in.fa.li.bi.li.da.de}{0}
\verb{infalível}{}{}{"-eis}{}{adj.2g.}{Que não falha, não comete erros; indefectível.}{in.fa.lí.vel}{0}
\verb{infalível}{}{}{"-eis}{}{}{Que não deixa de acontecer; fatal, impreterível.}{in.fa.lí.vel}{0}
\verb{infalível}{}{}{"-eis}{}{}{Preciso, exato, garantido.}{in.fa.lí.vel}{0}
\verb{infamante}{}{}{}{}{adj.2g.}{Que lança ou envolve infâmia; desonroso, ignomioso.}{in.fa.man.te}{0}
\verb{infamar}{}{}{}{}{v.t.}{Cobrir de vergonha ou desonra; tornar infame.}{in.fa.mar}{0}
\verb{infamar}{}{}{}{}{}{Desacreditar, difamar, caluniar.}{in.fa.mar}{\verboinum{1}}
\verb{infame}{}{}{}{}{adj.2g.}{Que pratica infâmias; baixo, vil, desprezível.}{in.fa.me}{0}
\verb{infame}{}{}{}{}{}{Que está desacreditado, desonrado.}{in.fa.me}{0}
\verb{infâmia}{}{}{}{}{s.f.}{Atitude vergonhosa, vil, torpe.}{in.fâ.mia}{0}
\verb{infâmia}{}{}{}{}{}{Dito contra a reputação ou a honra; injúria, calúnia.}{in.fâ.mia}{0}
\verb{infâmia}{}{}{}{}{}{Perda da fama; descrédito, desonra.}{in.fâ.mia}{0}
\verb{infância}{}{}{}{}{s.f.}{Fase da vida humana entre o nascimento e a puberdade.}{in.fân.cia}{0}
\verb{infância}{}{}{}{}{}{As crianças.}{in.fân.cia}{0}
\verb{infância}{}{Fig.}{}{}{}{Começo, origem, nascimento de algo.}{in.fân.cia}{0}
\verb{infanta}{}{}{}{}{s.f.}{Na Espanha e em Portugal, filha de reis que não é herdeira do trono.}{in.fan.ta}{0}
\verb{infanta}{}{}{}{}{}{Esposa do infante.}{in.fan.ta}{0}
\verb{infantaria}{}{}{}{}{s.f.}{Tropa militar que combate a pé.}{in.fan.ta.ri.a}{0}
\verb{infante}{}{}{}{}{s.m.}{Na Espanha e em Portugal, filho de reis que não é herdeiro do trono.}{in.fan.te}{0}
\verb{infante}{}{}{}{}{s.m.}{Soldado de infantaria.}{in.fan.te}{0}
\verb{infante}{}{}{}{}{adj.2g.}{Relativo à infância; infantil.}{in.fan.te}{0}
\verb{infanticida}{}{}{}{}{adj.}{Que assassinou uma criança, que praticou infanticídio.}{in.fan.ti.ci.da}{0}
\verb{infanticídio}{}{}{}{}{s.m.}{Assassínio de uma criança, geralmente de recém"-nascido.}{in.fan.ti.cí.dio}{0}
\verb{infanticídio}{}{Jur.}{}{}{}{Crime perpetrado pela mãe ao matar o próprio filho, logo após o parto.}{in.fan.ti.cí.dio}{0}
\verb{infantil}{}{}{"-is}{}{adj.2g.}{Relativo à infância ou aos infantes.}{in.fan.til}{0}
\verb{infantil}{}{Fig.}{"-is}{}{}{Simples, ingênuo, inocente.}{in.fan.til}{0}
\verb{infantilidade}{}{}{}{}{s.f.}{Qualidade do que é infantil; puerilidade.}{in.fan.ti.li.da.de}{0}
\verb{infantilidade}{}{Fig.}{}{}{}{Comportamento do que é ingênuo, tolo; criancice.}{in.fan.ti.li.da.de}{0}
\verb{infantilismo}{}{Med.}{}{}{s.m.}{Estado anormal de um indivíduo adulto em que persistem certos caracteres fisiológicos e psicológicos próprios da infância.}{in.fan.ti.lis.mo}{0}
\verb{infantilizar}{}{}{}{}{v.t.}{Tornar infantil; dar aspecto infantil.}{in.fan.ti.li.zar}{\verboinum{1}}
\verb{infantojuvenil}{}{}{infantojuvenis}{}{adj.2g.}{Relativo tanto à infância quanto à juventude.}{in.fan.to.ju.ve.nil}{0}
\verb{infarto}{}{Med.}{}{}{s.m.}{Morte das células da região de um órgão produzida pela obstrução da circulação arterial devido a uma trombose ou embolia.}{in.far.to}{0}
\verb{infatigável}{}{}{"-eis}{}{adj.2g.}{Que não se fatiga; incansável.}{in.fa.ti.gá.vel}{0}
\verb{infatigável}{}{}{"-eis}{}{}{Que revela persistência e desvelo; zeloso, prestimoso.}{in.fa.ti.gá.vel}{0}
\verb{infausto}{}{}{}{}{adj.}{Que não é venturoso; infeliz; não fausto.}{in.faus.to}{0}
\verb{infausto}{}{}{}{}{}{Aziago, agourento.}{in.faus.to}{0}
\verb{infeção}{}{}{}{}{}{Var. de \textit{infecção}.}{in.fe.ção}{0}
\verb{infecção}{}{}{"-ões}{}{s.f.}{Ato ou efeito de infeccionar, infectar.}{in.fec.ção}{0}
\verb{infecção}{}{Med.}{"-ões}{}{}{Enfermidade causada pela presença e desenvolvimento de micro"-organismos patogênicos, como vírus, bactérias e fungos, e que pode ficar localizada em uma região do corpo ou difundir"-se, produzindo uma septicemia.}{in.fec.ção}{0}
\verb{infecção}{}{Fig.}{"-ões}{}{}{Contaminação moral, corrupção, vício.}{in.fec.ção}{0}
\verb{infeccionado}{}{}{}{}{adj.}{Em que há infecção; infectado, contaminado.}{in.fec.ci.o.na.do}{0}
\verb{infeccionado}{}{}{}{}{}{Que se viciou moralmente; pervertido.}{in.fec.ci.o.na.do}{0}
\verb{infeccionar}{}{}{}{}{v.t.}{Tornar infecto por introdução de agentes nocivos; contaminar, viciar, infectar.}{in.fec.ci.o.nar}{0}
\verb{infeccionar}{}{}{}{}{}{Tornar pervertido, vicioso; corromper, perverter.}{in.fec.ci.o.nar}{\verboinum{1}}
\verb{infeccioso}{ô}{}{"-osos ⟨ó⟩}{"-osa ⟨ó⟩}{adj.}{Que produz infecção.}{in.fec.ci.o.so}{0}
\verb{infeccioso}{ô}{}{"-osos ⟨ó⟩}{"-osa ⟨ó⟩}{}{Que resulta de infecção.}{in.fec.ci.o.so}{0}
\verb{infecioso}{ô}{}{}{}{}{Var. de \textit{infeccioso}.}{in.fe.ci.o.so}{0}
\verb{infectar}{}{}{}{}{v.t.}{Contaminar com um agente patogênico; infeccionar.}{in.fec.tar}{0}
\verb{infectar}{}{Fig.}{}{}{}{Corromper moralmente; viciar, perverter.}{in.fec.tar}{\verboinum{1}}
\verb{infecto}{é}{}{}{}{adj.}{Que produz infecção; infeccionado, contaminado.}{in.fec.to}{0}
\verb{infecto}{é}{}{}{}{}{Que exala cheiro ruim; fétido, pestilento.}{in.fec.to}{0}
\verb{infecto}{é}{Fig.}{}{}{}{Repugnante, repulsivo, de péssima qualidade.}{in.fec.to}{0}
\verb{infecto"-contagioso}{é\ldots{}ô}{}{infecto"-contagiosos ⟨é\ldots{}ó⟩}{infecto"-contagiosa ⟨é\ldots{}ó⟩}{adj.}{Que produz infecção e se propaga por contágio.}{in.fec.to"-con.ta.gi.o.so}{0}
\verb{infecundidade}{}{}{}{}{s.f.}{Qualidade do que é infecundo; improdutividade, infertilidade.}{in.fe.cun.di.da.de}{0}
\verb{infecundo}{}{}{}{}{adj.}{Que não produz nada; infértil, improdutivo.}{in.fe.cun.do}{0}
\verb{infelicidade}{}{}{}{}{s.f.}{Ausência de felicidade; desventura, desdita, desgraça.}{in.fe.li.ci.da.de}{0}
\verb{infelicidade}{}{}{}{}{}{Circunstância desfavorável; falta de sorte; revés, infortúnio.}{in.fe.li.ci.da.de}{0}
\verb{infelicitar}{}{}{}{}{v.t.}{Causar a infelicidade, o infortúnio; desgraçar.}{in.fe.li.ci.tar}{\verboinum{1}}
\verb{infeliz}{}{}{}{}{adj.2g.}{Que não é feliz; desventurado, desgraçado, desditoso.}{in.fe.liz}{0}
\verb{infeliz}{}{}{}{}{}{Desfavorecido pela sorte; desafortunado.}{in.fe.liz}{0}
\verb{infeliz}{}{}{}{}{}{Desastroso, funesto.}{in.fe.liz}{0}
\verb{infenso}{}{}{}{}{adj.}{Que se opõe; contrário, adverso, inimigo.}{in.fen.so}{0}
\verb{inferência}{}{}{}{}{s.f.}{Ato ou efeito de inferir; conclusão, dedução.}{in.fe.rên.cia}{0}
\verb{inferência}{}{}{}{}{}{Operação lógica pela qual se tira uma conclusão partindo de um fato, um princípio.}{in.fe.rên.cia}{0}
\verb{inferior}{ô}{}{}{}{adj.2g.}{Que está abaixo ou por baixo.}{in.fe.ri.or}{0}
\verb{inferior}{ô}{}{}{}{}{Que é de valor ou qualidade menor.}{in.fe.ri.or}{0}
\verb{inferior}{ô}{}{}{}{}{Que está abaixo em dignidade, merecimento, importância.}{in.fe.ri.or}{0}
\verb{inferioridade}{}{}{}{}{s.f.}{Qualidade ou estado do que é inferior, do que está abaixo.}{in.fe.ri.o.ri.da.de}{0}
\verb{inferiorizar}{}{}{}{}{v.t.}{Tornar inferior; rebaixar, diminuir.}{in.fe.ri.o.ri.zar}{\verboinum{1}}
\verb{inferir}{}{}{}{}{v.t.}{Tirar uma conclusão a partir de um fato, de um princípio; deduzir.}{in.fe.rir}{\verboinum{29}}
\verb{infernal}{}{}{"-ais}{}{adj.2g.}{Relativo ao inferno.}{in.fer.nal}{0}
\verb{infernal}{}{}{"-ais}{}{}{Diabólico, demoníaco, atroz.}{in.fer.nal}{0}
\verb{infernal}{}{}{"-ais}{}{}{Que não é suportável; intolerável.}{in.fer.nal}{0}
\verb{infernar}{}{}{}{}{v.t.}{Causar aflição; atormentar, infernizar.}{in.fer.nar}{\verboinum{1}}
\verb{inferninho}{}{Bras.}{}{}{s.m.}{Boate pequena, pouco iluminada e com música muito barulhenta.}{in.fer.ni.nho}{0}
\verb{infernizar}{}{}{}{}{v.t.}{Tornar ruim como o inferno; atormentar, infernar.}{in.fer.ni.zar}{0}
\verb{infernizar}{}{}{}{}{}{Tornar incômodo, insuportável; irritar, encolerizar.}{in.fer.ni.zar}{\verboinum{1}}
\verb{inferno}{é}{}{}{}{s.m.}{Na tradição cristã, lugar onde as almas condenadas após a morte sofrem o castigo eterno.}{in.fer.no}{0}
\verb{inferno}{é}{Por ext.}{}{}{}{Extremo sofrimento; tormento, martírio.}{in.fer.no}{0}
\verb{inferno}{é}{Fís.}{}{}{}{Completa desordem; grande confusão; balbúrdia.}{in.fer.no}{0}
\verb{infértil}{}{}{"-eis}{}{adj.2g.}{Que não produz nada; infecundo, estéril.}{in.fér.til}{0}
\verb{infertilizar}{}{}{}{}{v.t.}{Tornar infértil, improdutivo; esterilizar.}{in.fer.ti.li.zar}{\verboinum{1}}
\verb{infestação}{}{}{"-ões}{}{s.f.}{Ato ou efeito de infestar; invasão, devastação.}{in.fes.ta.ção}{0}
\verb{infestação}{}{Med.}{"-ões}{}{}{Em um organismo, presença de parasitas, como piolhos e vermes, que provocam ou não doenças.}{in.fes.ta.ção}{0}
\verb{infestação}{}{}{"-ões}{}{}{Proliferação de pragas em uma plantação.}{in.fes.ta.ção}{0}
\verb{infestado}{}{}{}{}{adj.}{Em que há ou onde se produziu infestação. }{in.fes.ta.do}{0}
\verb{infestar}{}{}{}{}{v.t.}{Invadir com violência; devastar, assolar.}{in.fes.tar}{0}
\verb{infestar}{}{Med.}{}{}{}{Causar infestação em um organismo provocando ou não doenças.}{in.fes.tar}{0}
\verb{infestar}{}{Fig.}{}{}{}{Multiplicar, disseminar, propagar.}{in.fes.tar}{\verboinum{1}}
\verb{infetar}{}{}{}{}{}{Var. de \textit{infectar}.}{in.fe.tar}{0}
\verb{infeto}{é}{}{}{}{}{Var. de \textit{infecto}.}{in.fe.to}{0}
\verb{infidelidade}{}{}{}{}{s.f.}{Falta de fidelidade; deslealdade, traição.}{in.fi.de.li.da.de}{0}
\verb{infidelidade}{}{}{}{}{}{Falta de exatidão, de verdade.}{in.fi.de.li.da.de}{0}
\verb{infidelidade}{}{Relig.}{}{}{}{Estado daqueles que não professam uma religião; paganismo.}{in.fi.de.li.da.de}{0}
\verb{infiel}{é}{}{"-éis}{}{adj.2g.}{Que não é fiel; desleal, traiçoeiro.}{in.fi.el}{0}
\verb{infiel}{é}{}{"-éis}{}{}{Que falta à verdade; inexato, inverídico.}{in.fi.el}{0}
\verb{infiel}{é}{Relig.}{"-éis}{}{}{Diz"-se daquele que não professa uma religião; pagão.}{in.fi.el}{0}
\verb{infiltração}{}{}{"-ões}{}{s.f.}{Ato ou efeito de infiltrar; penetração.}{in.fil.tra.ção}{0}
\verb{infiltração}{}{}{"-ões}{}{}{Passagem lenta de um líquido através dos poros de um corpo sólido.}{in.fil.tra.ção}{0}
\verb{infiltração}{}{Fig.}{"-ões}{}{}{Ato de penetrar sutilmente na mente de alguém; insinuação.}{in.fil.tra.ção}{0}
\verb{infiltrar}{}{}{}{}{v.t.}{Penetrar como por um filtro através dos poros de um corpo sólido.}{in.fil.trar}{0}
\verb{infiltrar}{}{}{}{}{}{Introduzir"-se aos poucos, de forma secreta.}{in.fil.trar}{0}
\verb{infiltrar}{}{}{}{}{}{Insinuar, incutir.}{in.fil.trar}{\verboinum{1}}
\verb{ínfimo}{}{}{}{}{adj.}{Muito pequeno; diminuto.}{ín.fi.mo}{0}
\verb{ínfimo}{}{}{}{}{}{De pouca ou nenhuma importância; insignificante.}{ín.fi.mo}{0}
\verb{infindável}{}{}{"-eis}{}{adj.2g.}{Que não tem fim ou parece não ter fim; interminável, infinito. }{in.fin.dá.vel}{0}
\verb{infindo}{}{}{}{}{adj.}{Que não tem limite; inumerável, inesgotável, infinito.}{in.fin.do}{0}
\verb{infinidade}{}{}{}{}{s.f.}{Qualidade do que é infinito, ilimitado.}{in.fi.ni.da.de}{0}
\verb{infinidade}{}{}{}{}{}{Um número muito grande; uma quantidade enorme.}{in.fi.ni.da.de}{0}
\verb{infinitesimal}{}{}{"-ais}{}{adj.2g.}{Extremamente pequeno; ínfimo, mínimo.}{in.fi.ni.te.si.mal}{0}
\verb{infinitesimal}{}{Mat.}{"-ais}{}{}{Diz"-se dos números ou das operações entre números muito pequenos.}{in.fi.ni.te.si.mal}{0}
\verb{infinitivo}{}{Gram.}{}{}{adj.}{Diz"-se da forma nominal do verbo que exprime o estado ou a ação, sendo neutra em outras categorias gramaticais, como tempo, aspecto, modo.}{in.fi.ni.ti.vo}{0}
\verb{infinito}{}{}{}{}{adj.}{Que não tem fim; ilimitado, infindável, infindo.}{in.fi.ni.to}{0}
\verb{infinito}{}{}{}{}{}{Que não se pode medir; inumerável, incomensurável, incalculável.}{in.fi.ni.to}{0}
\verb{infinito}{}{}{}{}{}{Que transcende o humano, especialmente no que diz respeito a Deus.}{in.fi.ni.to}{0}
\verb{infinito}{}{}{}{}{s.m.}{O que não tem limites; o absoluto.}{in.fi.ni.to}{0}
\verb{infirmar}{}{}{}{}{v.t.}{Tirar a força, a firmeza, a autoridade; enfraquecer.}{in.fir.mar}{\verboinum{1}}
\verb{infixo}{cs}{Gram.}{}{}{s.m.}{Afixo localizado no interior da raiz dos vocábulos primitivos.}{in.fi.xo}{0}
\verb{inflação}{}{}{"-ões}{}{s.f.}{Ato ou efeito de inflar; inchação, intumescimento.}{in.fla.ção}{0}
\verb{inflação}{}{Fig.}{"-ões}{}{}{Ausência de modéstia; presunção, vaidade.}{in.fla.ção}{0}
\verb{inflação}{}{}{"-ões}{}{}{Aumento geral de preços que acarreta a perda do poder aquisitivo do dinheiro.}{in.fla.ção}{0}
\verb{inflação}{}{Por ext.}{"-ões}{}{}{Aumento excessivo de qualquer coisa.}{in.fla.ção}{0}
\verb{inflacionar}{}{}{}{}{v.t.}{Promover o desequilíbrio econômico de um país.}{in.fla.ci.o.nar}{0}
\verb{inflacionar}{}{}{}{}{}{Emitir grande quantidade de papel"-moeda, causando a sua desvalorização.}{in.fla.ci.o.nar}{0}
\verb{inflacionar}{}{}{}{}{}{Tornar a oferta maior que a demanda.}{in.fla.ci.o.nar}{\verboinum{1}}
\verb{inflacionário}{}{}{}{}{adj.}{Relativo a inflação.}{in.fla.ci.o.ná.rio}{0}
\verb{inflacionário}{}{}{}{}{}{Em que há inflação.}{in.fla.ci.o.ná.rio}{0}
\verb{inflado}{}{}{}{}{adj.}{Que aumentou de volume; inchado, intumescido.}{in.fla.do}{0}
\verb{inflado}{}{}{}{}{}{Cheio de soberba; vaidoso, pretensioso.}{in.fla.do}{0}
\verb{inflamação}{}{}{"-ões}{}{s.f.}{Ato ou efeito de inflamar, incendiar.}{in.fla.ma.ção}{0}
\verb{inflamação}{}{Fig.}{"-ões}{}{}{Ardor intenso; excitação, exaltação.}{in.fla.ma.ção}{0}
\verb{inflamação}{}{Med.}{"-ões}{}{}{Resposta local do organismo frente a uma agressão, caracterizada por rubor, edema, dor e, às vezes, febre.}{in.fla.ma.ção}{0}
\verb{inflamar}{}{}{}{}{v.t.}{Atear fogo; incendiar.}{in.fla.mar}{0}
\verb{inflamar}{}{Fig.}{}{}{}{Encher de ardor; estimular, excitar.}{in.fla.mar}{0}
\verb{inflamar}{}{Med.}{}{}{}{Causar inflamação.}{in.fla.mar}{\verboinum{1}}
\verb{inflamatório}{}{}{}{}{adj.}{Relativo a inflamação.}{in.fla.ma.tó.rio}{0}
\verb{inflamatório}{}{}{}{}{}{Que provoca inflamação.}{in.fla.ma.tó.rio}{0}
\verb{inflamável}{}{}{"-eis}{}{adj.2g.}{Que pode inflamar ou que se inflama facilmente.}{in.fla.má.vel}{0}
\verb{inflar}{}{}{}{}{v.t.}{Inchar com ar ou gás; dilatar.}{in.flar}{0}
\verb{inflar}{}{}{}{}{}{Encher de presunção; envaidecer.}{in.flar}{\verboinum{1}}
\verb{inflexão}{cs}{}{"-ões}{}{s.f.}{Ato ou efeito de curvar, dobrar.}{in.fle.xão}{0}
\verb{inflexão}{cs}{}{"-ões}{}{}{Mudança da direção ou da posição; desvio.}{in.fle.xão}{0}
\verb{inflexão}{cs}{}{"-ões}{}{}{Mudança de tom ou de acento na voz; modulação.}{in.fle.xão}{0}
\verb{inflexibilidade}{cs}{}{}{}{s.f.}{Qualidade ou comportamento do que é inflexível; impassibilidade, inexorabilidade.}{in.fle.xi.bi.li.da.de}{0}
\verb{inflexível}{cs}{}{"-eis}{}{adj.2g.}{Que não se pode curvar ou dobrar.}{in.fle.xí.vel}{0}
\verb{inflexível}{cs}{}{"-eis}{}{}{Que não se deixa comover; que não cede; insensível, indiferente.}{in.fle.xí.vel}{0}
\verb{inflexível}{cs}{}{"-eis}{}{}{Implacável, impassível, inexorável.}{in.fle.xí.vel}{0}
\verb{infligir}{}{}{}{}{v.t.}{Aplicar pena ou castigo por um crime ou uma falta.}{in.fli.gir}{\verboinum{22}}
\verb{inflorescência}{}{Bot.}{}{}{s.f.}{Conjunto de flores agrupadas na mesma haste de uma planta.}{in.flo.res.cên.cia}{0}
\verb{influência}{}{}{}{}{s.f.}{Ato ou efeito de influir, de agir sobre o outro.}{in.flu.ên.cia}{0}
\verb{influência}{}{}{}{}{}{Poder que alguém exerce sobre outrem ou sobre fatos ou negócios; autoridade, ascendência.}{in.flu.ên.cia}{0}
\verb{influência}{}{}{}{}{}{Prestígio, crédito, fascinação.}{in.flu.ên.cia}{0}
\verb{influenciar}{}{}{}{}{v.t.}{Exercer uma ascendência psicológica ou intelectual sobre algo ou alguém.}{in.flu.en.ci.ar}{\verboinum{1}}
\verb{influente}{}{}{}{}{adj.2g.}{Que influi; que tem autoridade, prestígio.}{in.flu.en.te}{0}
\verb{influenza}{}{Med.}{}{}{s.f.}{Doença infecciosa viral muito contagiosa, caracterizada por febre, coriza, tosse e dores musculares; gripe.}{in.flu.en.za}{0}
\verb{influir}{}{}{}{}{v.t.}{Fazer fluir para dentro.}{in.flu.ir}{0}
\verb{influir}{}{}{}{}{}{Fazer penetrar; incutir, insinuar.}{in.flu.ir}{0}
\verb{influir}{}{}{}{}{}{Exercer ação sobre outro; ter influência, prestígio.}{in.flu.ir}{\verboinum{26}}
\verb{influxo}{cs}{}{}{}{s.m.}{Ato ou efeito de influir; ação, influência.}{in.flu.xo}{0}
\verb{influxo}{cs}{}{}{}{}{Corrente abundante; afluência, convergência.}{in.flu.xo}{0}
\verb{influxo}{cs}{}{}{}{}{Nível máximo da maré; maré cheia, preamar.}{in.flu.xo}{0}
\verb{informação}{}{}{"-ões}{}{s.f.}{Conjunto de conhecimentos sobre um determinado assunto ou fato; notícia.}{in.for.ma.ção}{0}
\verb{informal}{}{}{"-ais}{}{adj.2g.}{Que não segue formalidades, nem cerimônias.}{in.for.mal}{0}
\verb{informante}{}{}{}{}{adj.2g.}{Que informa, comunica, noticia.}{in.for.man.te}{0}
\verb{informante}{}{}{}{}{s.2g.}{Indivíduo que dá informações à polícia.}{in.for.man.te}{0}
\verb{informar}{}{}{}{}{v.t.}{Dar notícia ou parecer; avisar.}{in.for.mar}{0}
\verb{informar}{}{}{}{}{}{Fazer saber; comunicar, cientificar.}{in.for.mar}{0}
\verb{informar}{}{}{}{}{}{Dar orientação; instruir, ensinar.}{in.for.mar}{\verboinum{1}}
\verb{informática}{}{}{}{}{s.f.}{Ciência que tem por objetivo o tratamento da informação e o processamento de dados por meio de computadores.}{in.for.má.ti.ca}{0}
\verb{informativo}{}{}{}{}{adj.}{Destinado a informar, noticiar, anunciar. }{in.for.ma.ti.vo}{0}
\verb{informatizar}{}{}{}{}{v.t.}{Adaptar um fato, processo ou serviço a um sistema de computadores.}{in.for.ma.ti.zar}{\verboinum{1}}
\verb{informe}{ó}{}{}{}{adj.}{Sem forma determinada; disforme.}{in.for.me}{0}
\verb{informe}{ó}{}{}{}{s.m.}{Ato ou efeito de informar; notícia, informação, aviso.}{in.for.me}{0}
\verb{informe}{ó}{}{}{}{}{Grosseiro, tosco, rudimentar.}{in.for.me}{0}
\verb{infortunado}{}{}{}{}{adj.}{Que não tem fortuna, felicidade; desgraçado, infeliz.}{in.for.tu.na.do}{0}
\verb{infortúnio}{}{}{}{}{s.m.}{Infelicidade, desventura, desgraça.}{in.for.tú.nio}{0}
\verb{infra}{}{}{}{}{adv.}{Citado ou mencionado mais abaixo; posteriormente.}{in.fra}{0}
\verb{infra"-assinado}{}{}{infra"-assinados}{}{adj.}{Assinado abaixo de que se está tratando.}{in.fra"-as.si.na.do}{0}
\verb{infra"-assinado}{}{}{infra"-assinados}{}{s.m.}{Pessoa que assina abaixo.}{in.fra"-as.si.na.do}{0}
\verb{infração}{}{}{"-ões}{}{s.f.}{Ato ou efeito de infringir.}{in.fra.ção}{0}
\verb{infração}{}{}{"-ões}{}{}{Violação de uma lei, ordem, tratado etc.}{in.fra.ção}{0}
\verb{infração}{}{}{"-ões}{}{}{Transgressão das regras de um jogo; falta.}{in.fra.ção}{0}
\verb{infraestrutura}{}{}{infraestruturas}{}{s.f.}{Parte inferior de uma estrutura.}{in.fra.es.tru.tu.ra}{0}
\verb{infraestrutura}{}{}{infraestruturas}{}{}{Base material ou econômica de uma sociedade ou de uma organização.}{in.fra.es.tru.tu.ra}{0}
\verb{infraestrutura}{}{}{infraestruturas}{}{}{Conjunto de relações sociais e econômicas que determinam as ideologias.}{in.fra.es.tru.tu.ra}{0}
\verb{infrator}{ô}{}{}{}{s.m.}{Indivíduo que infringe, que viola, transgride.}{in.fra.tor}{0}
\verb{infravermelho}{ê}{}{}{}{adj.2g.}{Diz"-se de radiação invisível, de espectro muito extenso, e que é emitida por corpos aquecidos.}{in.fra.ver.me.lho}{0}
\verb{infrene}{}{}{}{}{adj.2g.}{Que não tem freio; desenfreado, desordenado, descomedido.}{in.fre.ne}{0}
\verb{infrequente}{}{}{}{}{adj.2g.}{Que não é frequente; raro.}{in.fre.quen.te}{0}
\verb{infringir}{}{}{}{}{v.t.}{Violar, transgredir, desrespeitar.}{in.frin.gir}{\verboinum{22}}
\verb{infrutescência}{}{Bot.}{}{}{s.f.}{Frutificação de uma inflorescência; que resulta num fruto composto íntegro, como o abacaxi, a jaca etc.}{in.fru.tes.cên.cia}{0}
\verb{infrutífero}{}{}{}{}{adj.}{Que não produz fruto; estéril, infecundo.}{in.fru.tí.fe.ro}{0}
\verb{infrutífero}{}{Fig.}{}{}{}{Que não dá resultado; vão, inútil.}{in.fru.tí.fe.ro}{0}
\verb{infundado}{}{}{}{}{adj.}{Que não tem fundamento, base, razão, alicerce.}{in.fun.da.do}{0}
\verb{infundir}{}{}{}{}{v.t.}{Verter, derramar.}{in.fun.dir}{\verboinum{18}}
\verb{infusão}{}{}{"-ões}{}{s.f.}{Ato ou efeito de infundir.}{in.fu.são}{0}
\verb{infusão}{}{}{"-ões}{}{}{Procedimento que consiste em extrair de uma erva ou outra substância um líquido para dela extrair seus princípios ativos.}{in.fu.são}{0}
\verb{infusão}{}{}{"-ões}{}{}{O líquido resultante desse procedimento.}{in.fu.são}{0}
\verb{infusível}{}{}{"-eis}{}{adj.2g.}{Que não se pode fundir ou derreter.}{in.fu.sí.vel}{0}
\verb{infuso}{}{}{}{}{adj.}{Posto em infusão; derramado, vertido.}{in.fu.so}{0}
\verb{infuso}{}{Fig.}{}{}{}{Diz"-se do conhecimento adquirido naturalmente, sem ensinamento.}{in.fu.so}{0}
\verb{ingá}{}{Bot.}{}{}{s.m.}{Árvore pequena ou arbusto, de madeira branca ou avermelhada e fruto comestível.}{in.gá}{0}
\verb{ingá}{}{}{}{}{}{O fruto dessa árvore, que é uma vagem cujas sementes vêm envolvidas numa massa carnosa.}{in.gá}{0}
\verb{ingazeira}{ê}{Bot.}{}{}{s.f.}{Árvore cujo fruto é o ingá.}{in.ga.zei.ra}{0}
\verb{ingazeiro}{ê}{Bot.}{}{}{s.m.}{Ingazeira.}{in.ga.zei.ro}{0}
\verb{ingênito}{}{}{}{}{adj.}{Que é congênito, inato, de nascença.}{in.gê.ni.to}{0}
\verb{ingente}{}{}{}{}{adj.2g.}{Que é muito grande; enorme, desmedido.}{in.gen.te}{0}
\verb{ingente}{}{}{}{}{}{Que é estrondoso, retumbante.}{in.gen.te}{0}
\verb{ingenuidade}{}{}{}{}{s.f.}{Qualidade de ingênuo; simplicidade, singeleza.}{in.ge.nu.i.da.de}{0}
\verb{ingenuidade}{}{}{}{}{}{Ação própria de pessoa ingênua.}{in.ge.nu.i.da.de}{0}
\verb{ingênuo}{}{}{}{}{adj.}{Que é simples, franco, sincero.}{in.gê.nu.o}{0}
\verb{ingênuo}{}{}{}{}{}{Que denota inocência, sinceridade, simplicidade.}{in.gê.nu.o}{0}
\verb{ingênuo}{}{}{}{}{s.m.}{Indivíduo que não tem malícia; simples, franco.}{in.gê.nu.o}{0}
\verb{ingerência}{}{}{}{}{s.f.}{Ato ou efeito de ingerir; intervenção, influência.}{in.ge.rên.cia}{0}
\verb{ingerir}{}{}{}{}{v.t.}{Introduzir no estômago; engolir.}{in.ge.rir}{0}
\verb{ingerir}{}{Fig.}{}{}{}{Fazer penetrar, intrometer.}{in.ge.rir}{0}
\verb{ingerir}{}{}{}{}{v.pron.}{Intervir, intrometer"-se.}{in.ge.rir}{\verboinum{29}}
\verb{ingestão}{}{}{"-ões}{}{s.f.}{Ato ou efeito de ingerir; deglutição.}{in.ges.tão}{0}
\verb{inglês}{}{}{}{}{adj.}{Relativo à Inglaterra.}{in.glês}{0}
\verb{inglês}{}{}{}{}{s.m.}{Indivíduo natural ou habitante desse país.}{in.glês}{0}
\verb{inglês}{}{}{}{}{}{Língua germânica oficial da Inglaterra, Austrália, \textsc{e.u.a}., Nova Zelândia e, simultaneamente com o francês, do Canadá.}{in.glês}{0}
\verb{inglório}{}{}{}{}{}{Que não é reconhecido; ignorado, obscuro.}{in.gló.rio}{0}
\verb{inglório}{}{}{}{}{adj.}{Em que não há glória.}{in.gló.rio}{0}
\verb{ingovernável}{}{}{"-eis}{}{adj.2g.}{Que não se pode governar, dirigir.}{in.go.ver.ná.vel}{0}
\verb{ingovernável}{}{}{"-eis}{}{}{Que não se deixa dobrar; indisciplinável.}{in.go.ver.ná.vel}{0}
\verb{ingovernável}{}{}{"-eis}{}{}{Que não se pode controlar, dominar; irrefreável.}{in.go.ver.ná.vel}{0}
\verb{ingratidão}{}{}{"-ões}{}{s.f.}{Qualidade ou ação de ingrato; falta de gratidão.}{in.gra.ti.dão}{0}
\verb{ingrato}{}{}{}{}{adj.}{Que não é grato, que não reconhece os benefícios que recebeu.}{in.gra.to}{0}
\verb{ingrato}{}{}{}{}{s.m.}{Indivíduo que não é grato.}{in.gra.to}{0}
\verb{ingrato}{}{Fig.}{}{}{}{Que é improdutivo, infecundo.}{in.gra.to}{0}
\verb{ingrato}{}{}{}{}{}{Que não é aprazível; desagradável.}{in.gra.to}{0}
\verb{ingrediente}{}{}{}{}{s.m.}{Elemento que entra numa composição, num preparado ou numa mistura.}{in.gre.di.en.te}{0}
\verb{ingrediente}{}{Por ext.}{}{}{}{Qualquer componente ou constituinte.}{in.gre.di.en.te}{0}
\verb{íngreme}{}{}{}{}{adj.2g.}{Que é difícil de subir, que tem forte declive; escarpado, alcantilado.}{ín.gre.me}{0}
\verb{íngreme}{}{Fig.}{}{}{}{Que é árduo, difícil.}{ín.gre.me}{0}
\verb{ingressar}{}{}{}{}{v.t.}{Entrar.}{in.gres.sar}{0}
\verb{ingressar}{}{}{}{}{}{Passar a fazer parte de algo.}{in.gres.sar}{\verboinum{1}}
\verb{ingresso}{é}{}{}{}{s.m.}{Ato de ingressar ou entrar.}{in.gres.so}{0}
\verb{ingresso}{é}{}{}{}{}{Introdução, admissão.}{in.gres.so}{0}
\verb{ingresso}{é}{Bras.}{}{}{}{Bilhete de entrada em teatro, \textit{show}, cinema etc.}{in.gres.so}{0}
\verb{íngua}{}{Med.}{}{}{s.f.}{Inflamação ou inchação do gânglio linfático inguinal.}{ín.gua}{0}
\verb{íngua}{}{Por ext.}{}{}{}{Intumescimento dos gânglios da região das axilas, do pescoço etc.}{ín.gua}{0}
\verb{inguinal}{}{}{"-ais}{}{adj.2g.}{Relativo à virilha.}{in.gui.nal}{0}
\verb{ingurgitação}{}{}{"-ões}{}{s.f.}{Ato ou efeito de ingurgitar.}{in.gur.gi.ta.ção}{0}
\verb{ingurgitação}{}{Med.}{"-ões}{}{}{Excesso de sangue ou de outro líquido, fisiológico ou patológico, em tecido, órgão ou conduto qualquer.}{in.gur.gi.ta.ção}{0}
\verb{ingurgitamento}{}{}{}{}{s.m.}{Ingurgitação.}{in.gur.gi.ta.men.to}{0}
\verb{ingurgitar}{}{}{}{}{v.t.}{Engolir com muita rapidez; devorar.}{in.gur.gi.tar}{0}
\verb{ingurgitar}{}{}{}{}{}{Tornar repleto; encher até transbordar.}{in.gur.gi.tar}{0}
\verb{ingurgitar}{}{}{}{}{}{Aumentar de volume; inchar.}{in.gur.gi.tar}{0}
\verb{ingurgitar}{}{}{}{}{}{Obstruir.}{in.gur.gi.tar}{\verboinum{1}}
\verb{inhaca}{}{Pop.}{}{}{s.f.}{Mau cheiro.}{i.nha.ca}{0}
\verb{inhambu}{}{Zool.}{}{}{s.m.}{Ave de corpo robusto, pernas grossas e cauda rudimentar ou inexistente.}{i.nham.bu}{0}
\verb{inhame}{}{Bot.}{}{}{s.m.}{Erva originária do sul da Ásia, que fornece rizoma comestível, rico em amido, proteínas, vitaminas e açúcar.}{i.nha.me}{0}
\verb{inibição}{}{}{"-ões}{}{s.f.}{Ato ou efeito de inibir.}{i.ni.bi.ção}{0}
\verb{inibição}{}{}{"-ões}{}{}{Estado ou condição de pessoa inibida.}{i.ni.bi.ção}{0}
\verb{inibido}{}{}{}{}{adj.}{Que sofre de inibição; tímido.}{i.ni.bi.do}{0}
\verb{inibido}{}{}{}{}{s.m.}{Indivíduo que tem inibição; tímido.}{i.ni.bi.do}{0}
\verb{inibir}{}{}{}{}{v.t.}{Impedir, impossibilitar, tolher.}{i.ni.bir}{0}
\verb{inibir}{}{}{}{}{}{Embaraçar, acanhar.}{i.ni.bir}{\verboinum{18}}
\verb{inibitório}{}{}{}{}{adj.}{Que inibe, embaraça, proíbe.}{i.ni.bi.tó.rio}{0}
\verb{iniciação}{}{}{"-ões}{}{s.f.}{Ato ou efeito de iniciar.}{i.ni.ci.a.ção}{0}
\verb{iniciação}{}{}{"-ões}{}{}{Preparação pela qual se inicia alguém nos mistérios de alguma religião ou doutrina e a cerimônia dela decorrente. }{i.ni.ci.a.ção}{0}
\verb{iniciação}{}{}{"-ões}{}{}{Recebimento das primeiras noções relativas a uma ciência, uma arte, uma prática.}{i.ni.ci.a.ção}{0}
\verb{iniciado}{}{}{}{}{adj.}{Que foi principiado, começado.}{i.ni.ci.a.do}{0}
\verb{iniciado}{}{}{}{}{}{Que foi instruído em conhecimento, arte etc.}{i.ni.ci.a.do}{0}
\verb{iniciado}{}{}{}{}{s.m.}{Indivíduo a quem foram revelados os mistérios e práticas de um culto, uma seita, uma ordem. }{i.ni.ci.a.do}{0}
\verb{inicial}{}{}{"-ais}{}{adj.2g.}{Que inicia, que está no começo, no princípio.}{i.ni.ci.al}{0}
\verb{inicial}{}{}{"-ais}{}{}{Que é primeiro, primitivo.}{i.ni.ci.al}{0}
\verb{inicial}{}{}{"-ais}{}{s.f.}{A primeira letra de uma palavra.}{i.ni.ci.al}{0}
\verb{inicial}{}{Jur.}{"-ais}{}{}{A petição inauguratória da ação.}{i.ni.ci.al}{0}
\verb{inicializar}{}{Informát.}{}{}{v.t.}{Dar a partida no computador, preparando para o uso.}{i.ni.ci.a.li.zar}{0}
\verb{inicializar}{}{Informát.}{}{}{}{Carregar ou abrir programa.}{i.ni.ci.a.li.zar}{\verboinum{1}}
\verb{iniciante}{}{}{}{}{adj.2g.}{Que inicia.}{i.ni.ci.an.te}{0}
\verb{iniciante}{}{}{}{}{s.2g.}{Indivíduo que está começando a adquirir a experiência ou a prática de algo; neófito, novato.}{i.ni.ci.an.te}{0}
\verb{iniciar}{}{}{}{}{v.t.}{Dar início, começar, inaugurar.}{i.ni.ci.ar}{0}
\verb{iniciar}{}{}{}{}{}{Informar, instruir nos segredos de uma técnica ou de uma arte.}{i.ni.ci.ar}{0}
\verb{iniciar}{}{}{}{}{}{Introduzir no conhecimento e na participação de mistérios religiosos.}{i.ni.ci.ar}{0}
\verb{iniciar}{}{Informát.}{}{}{}{Executar o procedimento de partida do computador.}{i.ni.ci.ar}{\verboinum{6}}
\verb{iniciativa}{}{}{}{}{s.f.}{Ação de quem propõe ou inicia algo.}{i.ni.ci.a.ti.va}{0}
\verb{iniciativo}{}{}{}{}{adj.}{Que inicia, que está no começo; inicial.}{i.ni.ci.a.ti.vo}{0}
\verb{início}{}{}{}{}{s.m.}{Ato ou efeito de iniciar.}{i.ní.cio}{0}
\verb{início}{}{}{}{}{}{O que vem em primeiro lugar; princípio, começo.}{i.ní.cio}{0}
\verb{início}{}{}{}{}{}{Inauguração, estreia,  fundação.}{i.ní.cio}{0}
\verb{início}{}{}{}{}{}{Parte preliminar; preâmbulo.}{i.ní.cio}{0}
\verb{inidôneo}{}{}{}{}{adj.}{Que não é adequado, que não convém.}{i.ni.dô.neo}{0}
\verb{inidôneo}{}{}{}{}{}{Que não goza de boa fama.}{i.ni.dô.neo}{0}
\verb{inigualável}{}{}{"-eis}{}{adj.2g.}{Que não pode ser igualado, que não tem igual; incomparável.}{i.ni.gua.lá.vel}{0}
\verb{iniludível}{}{}{"-eis}{}{adj.2g.}{Que não se pode iludir.}{i.ni.lu.dí.vel}{0}
\verb{iniludível}{}{}{"-eis}{}{}{Que não admite dúvidas.}{i.ni.lu.dí.vel}{0}
\verb{inimaginável}{}{}{"-eis}{}{adj.2g.}{Que não se pode imaginar, que ultrapassa o poder da imaginação.}{i.ni.ma.gi.ná.vel}{0}
\verb{inimigo}{}{}{}{}{adj.}{Que é hostil, adverso, contrário.}{i.ni.mi.go}{0}
\verb{inimigo}{}{}{}{}{}{Que pertence a grupo, facção ou partido oposto.}{i.ni.mi.go}{0}
\verb{inimigo}{}{}{}{}{s.m.}{Indivíduo que odeia ou detesta alguém ou algo.}{i.ni.mi.go}{0}
\verb{inimigo}{}{}{}{}{}{Grupo, facção ou partido hostil.}{i.ni.mi.go}{0}
\verb{inimigo}{}{}{}{}{}{Coisa prejudicial, nociva.}{i.ni.mi.go}{0}
\verb{inimitável}{}{}{"-eis}{}{adj.2g.}{Que não se pode imitar.}{i.ni.mi.tá.vel}{0}
\verb{inimizade}{}{}{}{}{s.f.}{Falta de amizade; ódio, aversão.}{i.ni.mi.za.de}{0}
\verb{inimizar}{}{}{}{}{v.t.}{Tornar inimigo; indispor.}{i.ni.mi.zar}{\verboinum{1}}
\verb{ininteligível}{}{}{"-eis}{}{adj.2g.}{Que não é inteligível; obscuro, incompreensível, confuso.}{i.nin.te.li.gí.vel}{0}
\verb{ininterrupto}{}{}{}{}{adj.}{Que não é interrompido; constante, contínuo.}{i.nin.ter.rup.to}{0}
\verb{iniquidade}{}{}{}{}{s.f.}{Caráter daquilo ou daquele que é iníquo, que é contrário à equidade.}{i.ni.qui.da.de}{0}
\verb{iniquidade}{}{}{}{}{}{Ação ou coisa contrária à moral, à religião ou à justiça.}{i.ni.qui.da.de}{0}
\verb{iníquo}{}{}{}{}{adj.}{Que é contrário à equidade, ao que é justo.}{i.ní.quo}{0}
\verb{iníquo}{}{}{}{}{}{Que é mau, perverso.}{i.ní.quo}{0}
\verb{injeção}{}{}{"-ões}{}{s.f.}{Ato ou efeito de injetar.}{in.je.ção}{0}
\verb{injeção}{}{Med.}{"-ões}{}{}{Introdução, em tecido, órgão ou formação patológica, de líquido, geralmente medicamentoso, por meio de seringa e agulha.}{in.je.ção}{0}
\verb{injeção}{}{}{"-ões}{}{}{Aquilo que se injeta.}{in.je.ção}{0}
\verb{injetar}{}{}{}{}{v.t.}{Introduzir sob pressão em um corpo.}{in.je.tar}{0}
\verb{injetar}{}{}{}{}{}{Tornar corado ou vermelho pelo afluxo de sangue.}{in.je.tar}{0}
\verb{injetar}{}{}{}{}{}{Aplicar, investir, como reforço.}{in.je.tar}{\verboinum{1}}
\verb{injetável}{}{}{"-eis}{}{adj.2g.}{Diz"-se de medicamento ou produto que deve ser administrado por injeção.}{in.je.tá.vel}{0}
\verb{injetor}{ô}{}{}{}{adj.}{Que injeta.}{in.je.tor}{0}
\verb{injetor}{ô}{}{}{}{s.m.}{Aparelho com que se injeta um fluido em algum órgão mecânico.}{in.je.tor}{0}
\verb{injetor}{ô}{}{}{}{}{Aparelho usado para a aplicação de inseticidas no solo.}{in.je.tor}{0}
\verb{injunção}{}{}{"-ões}{}{s.f.}{Ato ou efeito de injungir, de ordenar expressamente uma coisa; ordem precisa e formal.}{in.jun.ção}{0}
\verb{injunção}{}{}{"-ões}{}{}{Imposição, exigência, pressão.}{in.jun.ção}{0}
\verb{injúria}{}{}{}{}{s.f.}{Ato ou efeito de injuriar.}{in.jú.ria}{0}
\verb{injúria}{}{}{}{}{}{Aquilo que é injusto.}{in.jú.ria}{0}
\verb{injúria}{}{}{}{}{}{Ato ou dito ofensivo a alguém; insulto.}{in.jú.ria}{0}
\verb{injuriar}{}{}{}{}{v.t.}{Ofender com ações ou ditos; insultar.}{in.ju.ri.ar}{0}
\verb{injuriar}{}{}{}{}{}{Tornar infame; desonrar.}{in.ju.ri.ar}{0}
\verb{injuriar}{}{}{}{}{}{Causar estrago, dano ou lesão.}{in.ju.ri.ar}{\verboinum{6}}
\verb{injurioso}{ô}{}{"-osos ⟨ó⟩}{"-osa ⟨ó⟩}{adj.}{Em que há injúria; ofensivo.}{in.ju.ri.o.so}{0}
\verb{injustiça}{}{}{}{}{s.f.}{Falta de justiça, ação contrária à justiça.}{in.jus.ti.ça}{0}
\verb{injustiçado}{}{}{}{}{adj.}{Que não teve justiça.}{in.jus.ti.ça.do}{0}
\verb{injustiçado}{}{}{}{}{s.m.}{Indivíduo a quem não se fez justiça.}{in.jus.ti.ça.do}{0}
\verb{injustificável}{}{}{"-eis}{}{adj.2g.}{Que não pode ser justificado.}{in.jus.ti.fi.cá.vel}{0}
\verb{injusto}{}{}{}{}{adj.}{Escasso de justiça ou contrário à justiça.}{in.jus.to}{0}
\verb{injusto}{}{}{}{}{}{Sem fundamento; injustificado.}{in.jus.to}{0}
\verb{inobservância}{}{}{}{}{s.f.}{Falta de observância ou de cumprimento.}{i.nob.ser.vân.cia}{0}
\verb{inobservável}{}{}{"-eis}{}{adj.2g.}{Que não pode ser observado ou cumprido.}{i.nob.ser.vá.vel}{0}
\verb{inocência}{}{}{}{}{s.f.}{Qualidade de inocente.}{i.no.cên.cia}{0}
\verb{inocência}{}{}{}{}{}{Ingenuidade, simplicidade.}{i.no.cên.cia}{0}
\verb{inocência}{}{}{}{}{}{Pureza, candura.}{i.no.cên.cia}{0}
\verb{inocentar}{}{}{}{}{v.t.}{Considerar inocente.}{i.no.cen.tar}{0}
\verb{inocentar}{}{}{}{}{}{Desculpar, absolver.}{i.no.cen.tar}{\verboinum{1}}
\verb{inocente}{}{}{}{}{adj.2g.}{Que não tem culpa.}{i.no.cen.te}{0}
\verb{inocente}{}{}{}{}{}{Que não tem malícia; ingênuo.}{i.no.cen.te}{0}
\verb{inocente}{}{}{}{}{}{Inofensivo, inócuo.}{i.no.cen.te}{0}
\verb{inoculação}{}{}{"-ões}{}{s.f.}{Ato ou efeito de inocular.}{i.no.cu.la.ção}{0}
\verb{inocular}{}{}{}{}{v.t.}{Inserir, introduzir.}{i.no.cu.lar}{0}
\verb{inocular}{}{}{}{}{}{Introduzir no organismo.}{i.no.cu.lar}{0}
\verb{inocular}{}{Med.}{}{}{}{Introduzir agente etiológico em um organismo.}{i.no.cu.lar}{0}
\verb{inocular}{}{}{}{}{}{Incutir, sugerir, inspirar.}{i.no.cu.lar}{\verboinum{1}}
\verb{inócuo}{}{}{}{}{adj.}{Que não causa dano; inofensivo.}{i.nó.cu.o}{0}
\verb{inócuo}{}{Por ext.}{}{}{}{Que não produz o efeito esperado.}{i.nó.cu.o}{0}
\verb{inodoro}{ó}{}{}{}{adj.}{Sem odor, sem cheiro.}{i.no.do.ro}{0}
\verb{inofensivo}{}{}{}{}{adj.}{Que não prejudica.}{i.no.fen.si.vo}{0}
\verb{inolvidável}{}{}{"-eis}{}{adj.2g.}{Inesquecível.}{i.nol.vi.dá.vel}{0}
\verb{inominado}{}{}{}{}{adj.}{Que não tem ou não se sabe o nome.}{i.no.mi.na.do}{0}
\verb{inominável}{}{}{"-eis}{}{}{Que não pode ser designado por um nome.}{i.no.mi.ná.vel}{0}
\verb{inominável}{}{}{"-eis}{}{adj.2g.}{Abjeto, horroroso, péssimo.}{i.no.mi.ná.vel}{0}
\verb{inoperante}{}{}{}{}{adj.2g.}{Que não opera ou não produz o resultado esperado.}{i.no.pe.ran.te}{0}
\verb{inópia}{}{}{}{}{s.f.}{Grande pobreza; miséria, escassez, penúria.}{i.nó.pia}{0}
\verb{inopinado}{}{}{}{}{adj.}{Súbito, imprevisto, extraordinário.}{i.no.pi.na.do}{0}
\verb{inopino}{}{}{}{}{adj.}{Inesperado, inopinado.}{i.no.pi.no}{0}
\verb{inoportuno}{}{}{}{}{adj.}{Que não é oportuno; que vem fora de hora.}{i.no.por.tu.no}{0}
\verb{inorgânico}{}{}{}{}{adj.}{Que não contém matéria animal ou vegetal.}{i.nor.gâ.ni.co}{0}
\verb{inorgânico}{}{}{}{}{}{Sem órgãos; sem vida; inanimado.}{i.nor.gâ.ni.co}{0}
\verb{inóspito}{}{}{}{}{adj.}{Em que não há condições de viver; inabitável.}{i.nós.pi.to}{0}
\verb{inóspito}{}{}{}{}{}{Não apto para hospedar.}{i.nós.pi.to}{0}
\verb{inovação}{}{}{"-ões}{}{s.f.}{Ato ou efeito de inovar.}{i.no.va.ção}{0}
\verb{inovação}{}{Por ext.}{"-ões}{}{}{Novidade.}{i.no.va.ção}{0}
\verb{inovador}{ô}{}{}{}{adj.}{Que inova.}{i.no.va.dor}{0}
\verb{inovador}{ô}{}{}{}{}{Em que há inovação.}{i.no.va.dor}{0}
\verb{inovar}{}{}{}{}{v.t.}{Introduzir novidade, mudança, melhoria.}{i.no.var}{\verboinum{1}}
\verb{inoxidável}{cs}{}{"-eis}{}{adj.2g.}{Que não oxida.}{i.no.xi.dá.vel}{0}
\verb{input}{}{}{}{}{s.m.}{Entrada.}{\textit{input}}{0}
\verb{input}{}{Informát.}{}{}{}{Dados que são inseridos em determinado programa de computador para serem processados.}{\textit{input}}{0}
\verb{inqualificável}{}{}{"-eis}{}{adj.2g.}{Que é impossível de qualificar.}{in.qua.li.fi.cá.vel}{0}
\verb{inqualificável}{}{}{"-eis}{}{}{Que não merece qualificação por ser demasiado vil, grosseiro, inconveniente.}{in.qua.li.fi.cá.vel}{0}
\verb{inquebrantável}{}{}{"-eis}{}{adj.2g.}{Que não se pode quebrantar; sólido, incansável, inflexível.}{in.que.bran.tá.vel}{0}
\verb{inquebrável}{}{}{"-eis}{}{adj.2g.}{Que não pode ser quebrado.}{in.que.brá.vel}{0}
\verb{inquérito}{}{}{}{}{s.m.}{Ato ou efeito de inquirir.}{in.qué.ri.to}{0}
\verb{inquérito}{}{}{}{}{}{Conjunto de procedimentos e providências com que se busca esclarecer ou investigar algo; sindicância.}{in.qué.ri.to}{0}
\verb{inquestionável}{}{}{"-eis}{}{adj.2g.}{Que não pode ser questionado; indiscutível.}{in.ques.ti.o.ná.vel}{0}
\verb{inquietação}{}{}{"-ões}{}{s.f.}{Falta de sossego; preocupação, nervosismo.}{in.qui.e.ta.ção}{0}
\verb{inquietação}{}{}{"-ões}{}{}{Agitação, excitação.}{in.qui.e.ta.ção}{0}
\verb{inquietação}{}{}{"-ões}{}{}{Preocupação com questões metafísicas.}{in.qui.e.ta.ção}{0}
\verb{inquietador}{ô}{}{}{}{adj.}{Inquietante.}{in.qui.e.ta.dor}{0}
\verb{inquietante}{}{}{}{}{adj.2g.}{Que causa inquietação.}{in.qui.e.tan.te}{0}
\verb{inquietar}{}{}{}{}{v.t.}{Tornar inquieto; perturbar.}{in.qui.e.tar}{0}
\verb{inquietar}{}{}{}{}{}{Amotinar, alvorotar, revoltar.}{in.qui.e.tar}{0}
\verb{inquietar}{}{}{}{}{}{Suscitar preocupação com questões metafísicas.}{in.qui.e.tar}{\verboinum{1}}
\verb{inquieto}{é}{}{}{}{adj.}{Desassossegado, agitado, turbulento.}{in.qui.e.to}{0}
\verb{inquieto}{é}{}{}{}{}{Ansioso, aflito, apreensivo.}{in.qui.e.to}{0}
\verb{inquietude}{}{}{}{}{s.f.}{Qualidade de inquieto.}{in.qui.e.tu.de}{0}
\verb{inquilinato}{}{}{}{}{s.m.}{Condição de inquilino.}{in.qui.li.na.to}{0}
\verb{inquilinato}{}{}{}{}{}{O conjunto dos inquilinos.}{in.qui.li.na.to}{0}
\verb{inquilino}{}{}{}{}{s.m.}{Indivíduo que mora em residência alugada.}{in.qui.li.no}{0}
\verb{inquinar}{}{}{}{}{v.t.}{Manchar, sujar, poluir.}{in.qui.nar}{0}
\verb{inquinar}{}{}{}{}{}{Tirar a pureza; corromper.}{in.qui.nar}{\verboinum{1}}
\verb{inquirição}{}{}{"-ões}{}{s.f.}{Inquérito, averiguação, sindicância.}{in.qui.ri.ção}{0}
\verb{inquirir}{}{}{}{}{v.t.}{Investigar, pesquisar.}{in.qui.rir}{0}
\verb{inquirir}{}{}{}{}{}{Interrogar.}{in.qui.rir}{\verboinum{18}}
\verb{inquisição}{}{}{"-ões}{}{s.f.}{Inquérito, averiguação, sindicância.}{in.qui.si.ção}{0}
\verb{inquisição}{}{}{"-ões}{}{}{Antigo tribunal eclesiástico que investigava e punia cruelmente os crimes contra a fé cristã. (Usa"-se maiúscula nesta acepção.)}{in.qui.si.ção}{0}
\verb{inquisidor}{ô}{}{}{}{adj.}{Que interroga; inquiridor.}{in.qui.si.dor}{0}
\verb{inquisidor}{ô}{}{}{}{}{Juiz do antigo tribunal da Inquisição.}{in.qui.si.dor}{0}
\verb{inquisitivo}{}{}{}{}{adj.}{Relativo a inquisição; interrogativo.}{in.qui.si.ti.vo}{0}
\verb{inquisitorial}{}{}{"-ais}{}{adj.2g.}{Relativo a inquisição.}{in.qui.si.to.ri.al}{0}
\verb{inquisitorial}{}{}{"-ais}{}{}{Muito severo, rigoroso; desumano, vexatório.}{in.qui.si.to.ri.al}{0}
\verb{insaciável}{}{}{"-eis}{}{adj.2g.}{Que não se farta; sôfrego, ávido, insaturável.}{in.sa.ci.á.vel}{0}
\verb{insalivar}{}{}{}{}{v.t.}{Impregnar de saliva.}{in.sa.li.var}{\verboinum{1}}
\verb{insalubre}{}{}{}{}{adj.2g.}{Que não faz bem à saúde.}{in.sa.lu.bre}{0}
\verb{insalubre}{}{}{}{}{}{Que causa doença.}{in.sa.lu.bre}{0}
\verb{insalubridade}{}{}{}{}{s.f.}{Qualidade de insalubre.}{in.sa.lu.bri.da.de}{0}
\verb{insanável}{}{}{"-eis}{}{adj.2g.}{Que não pode ser sanado; sem remédio; incurável.}{in.sa.ná.vel}{0}
\verb{insânia}{}{}{}{}{s.f.}{Ausência de razão; loucura, demência.}{in.sâ.nia}{0}
\verb{insanidade}{}{}{}{}{s.f.}{Qualidade de insano.}{in.sa.ni.da.de}{0}
\verb{insanidade}{}{}{}{}{}{Loucura, demência.}{in.sa.ni.da.de}{0}
\verb{insano}{}{}{}{}{adj.}{Louco, demente.}{in.sa.no}{0}
\verb{insano}{}{Fig.}{}{}{}{Diz"-se de trabalho excessivo, exagerado, muito árduo.}{in.sa.no}{0}
\verb{insatisfação}{}{}{"-ões}{}{s.f.}{Falta de satisfação; desagrado, descontentamento.}{in.sa.tis.fa.ção}{0}
\verb{insatisfatório}{}{}{}{}{adj.}{Que não satisfaz.}{in.sa.tis.fa.tó.rio}{0}
\verb{insatisfatório}{}{}{}{}{}{Abaixo do esperado ou do necessário; insuficiente, fraco.}{in.sa.tis.fa.tó.rio}{0}
\verb{insatisfeito}{ê}{}{}{}{adj.}{Que não está satisfeito ou saciado; descontente.}{in.sa.tis.fei.to}{0}
\verb{insaturado}{}{Quím.}{}{}{adj.}{Diz"-se de compostos orgânicos que apresentam ao menos uma ligação dupla ou tripla.}{in.sa.tu.ra.do}{0}
\verb{insaturado}{}{Quím.}{}{}{}{Diz"-se das soluções cuja concentração fica abaixo da solubilidade.}{in.sa.tu.ra.do}{0}
\verb{insciência}{}{}{}{}{s.f.}{Falta de conhecimento; ignorância.}{ins.ci.ên.cia}{0}
\verb{insciência}{}{}{}{}{}{Falta de habilidade; imperícia, inaptidão.}{ins.ci.ên.cia}{0}
\verb{insciente}{}{}{}{}{adj.2g.}{Não ciente; ignorante.}{ins.ci.en.te}{0}
\verb{insciente}{}{}{}{}{}{Inapto, inábil.}{ins.ci.en.te}{0}
\verb{inscrever}{ê}{}{}{}{v.t.}{Escrever entalhando ou gravando.}{ins.cre.ver}{0}
\verb{inscrever}{ê}{}{}{}{}{Escrever em registro; assentar.}{ins.cre.ver}{0}
\verb{inscrever}{ê}{Geom.}{}{}{}{Traçar uma figura dentro de outra.}{ins.cre.ver}{0}
\verb{inscrever}{ê}{}{}{}{v.pron.}{Matricular"-se, alistar"-se.}{ins.cre.ver}{\verboinum{12}}
\verb{inscrição}{}{}{"-ões}{}{s.f.}{Ato ou efeito de inscrever.}{ins.cri.ção}{0}
\verb{inscrição}{}{}{"-ões}{}{}{Desenho, símbolo ou letra gravados em monumentos, lápides, medalhas, pedras.}{ins.cri.ção}{0}
\verb{inscrição}{}{}{"-ões}{}{}{Assentamento, alistamento.}{ins.cri.ção}{0}
\verb{inscrição}{}{}{"-ões}{}{}{Matrícula.}{ins.cri.ção}{0}
\verb{inscrito}{}{}{}{}{adj.}{Incluído em registro ou lista; registrado, assentado.}{ins.cri.to}{0}
\verb{inscrito}{}{}{}{}{}{Diz"-se de desenho traçado dentro de outro.}{ins.cri.to}{0}
\verb{inscrito}{}{}{}{}{s.m.}{Indivíduo incluído em lista ou matriculado em determinada atividade.}{ins.cri.to}{0}
\verb{insculpir}{}{}{}{}{v.t.}{Abrir cortes ou sulcos, geralmente em madeira, pedra ou metal; entalhar, gravar.}{ins.cul.pir}{\verboinum{18}}
\verb{insegurança}{}{}{}{}{s.f.}{Ausência de segurança.}{in.se.gu.ran.ça}{0}
\verb{insegurança}{}{}{}{}{}{Sensação de estar desprotegido, vulnerável ou de não confiar nas próprias capacidades.}{in.se.gu.ran.ça}{0}
\verb{inseguro}{}{}{}{}{adj.}{Que não oferece segurança.}{in.se.gu.ro}{0}
\verb{inseguro}{}{}{}{}{}{Que tem sensação de insegurança.}{in.se.gu.ro}{0}
\verb{inseminação}{}{Biol.}{"-ões}{}{s.f.}{Fecundação do óvulo.}{in.se.mi.na.ção}{0}
\verb{inseminação}{}{Veter.}{"-ões}{}{}{Introdução do sêmen na cavidade uterina.}{in.se.mi.na.ção}{0}
\verb{inseminar}{}{}{}{}{v.t.}{Fazer inseminação.}{in.se.mi.nar}{\verboinum{1}}
\verb{insensatez}{ê}{}{}{}{s.f.}{Qualidade de insensato.}{in.sen.sa.tez}{0}
\verb{insensatez}{ê}{}{}{}{}{Ato ou dito insensato.}{in.sen.sa.tez}{0}
\verb{insensato}{}{}{}{}{adj.}{Sem razão; louco, demente.}{in.sen.sa.to}{0}
\verb{insensato}{}{}{}{}{}{Contrário ao bom"-senso.}{in.sen.sa.to}{0}
\verb{insensibilidade}{}{}{}{}{s.f.}{Falta de sensibilidade.}{in.sen.si.bi.li.da.de}{0}
\verb{insensibilidade}{}{}{}{}{}{Indiferença, apatia, desinteresse.}{in.sen.si.bi.li.da.de}{0}
\verb{insensibilizar}{}{}{}{}{v.t.}{Tornar insensível.}{in.sen.si.bi.li.zar}{\verboinum{1}}
\verb{insensível}{}{}{"-eis}{}{adj.2g.}{Que não tem sensibilidade; indiferente, apático.}{in.sen.sí.vel}{0}
\verb{insensível}{}{}{"-eis}{}{}{Impiedoso, inexorável, cruel.}{in.sen.sí.vel}{0}
\verb{insensível}{}{}{"-eis}{}{}{Que não é percebido pelos sentidos.}{in.sen.sí.vel}{0}
\verb{inseparável}{}{}{"-eis}{}{adj.2g.}{Que não é separável.}{in.se.pa.rá.vel}{0}
\verb{inseparável}{}{}{"-eis}{}{}{Que está sempre com outro.}{in.se.pa.rá.vel}{0}
\verb{insepulto}{}{}{}{}{adj.}{Que não foi sepultado.}{in.se.pul.to}{0}
\verb{inserção}{}{}{"-ões}{}{s.f.}{Ato ou efeito de inserir; introdução.}{in.ser.ção}{0}
\verb{inserir}{}{}{}{}{v.t.}{Introduzir, colocar, incluir.}{in.se.rir}{0}
\verb{inserir}{}{}{}{}{}{Fixar, implantar.}{in.se.rir}{\verboinum{29}}
\verb{inserto}{é}{}{}{}{adj.}{Introduzido, inserido.}{in.ser.to}{0}
\verb{inservível}{}{}{"-eis}{}{adj.2g.}{Que não serve; sem utilidade.}{in.ser.ví.vel}{0}
\verb{inseticida}{}{}{}{}{adj.2g.}{Que mata insetos.}{in.se.ti.ci.da}{0}
\verb{inseticida}{}{}{}{}{s.m.}{Substância ou produto próprio para matar insetos.}{in.se.ti.ci.da}{0}
\verb{insetívoro}{}{}{}{}{adj.}{Que come insetos.}{in.se.tí.vo.ro}{0}
\verb{inseto}{é}{Zool.}{}{}{s.m.}{Espécime dos insetos, animais invertebrados com três pares de patas, com até dois pares de asas e um par de antenas.}{in.se.to}{0}
\verb{inseto}{é}{Fig.}{}{}{}{Pessoa desprezível, insignificante.}{in.se.to}{0}
\verb{insídia}{}{}{}{}{s.f.}{Ação ou efeito de esperar o inimigo às escondidas para atacá"-lo de surpresa; emboscada, cilada, tocaia.}{in.sí.dia}{0}
\verb{insídia}{}{}{}{}{}{Falta de lealdade; perfídia, traição.}{in.sí.dia}{0}
\verb{insidioso}{ô}{}{"-osos ⟨ó⟩}{"-osa ⟨ó⟩}{adj.}{Que arma insídias; traiçoeiro, enganador, pérfido.}{in.si.di.o.so}{0}
\verb{insigne}{}{}{}{}{adj.2g.}{Que se destaca naquilo que faz; distinto, ilustre, notável, célebre.}{in.sig.ne}{0}
\verb{insígnia}{}{}{}{}{s.f.}{Sinal distintivo de um cargo, de posto, de função etc.; emblema, divisa.}{in.síg.nia}{0}
\verb{insignificância}{}{}{}{}{s.f.}{Qualidade de insignificante.}{in.sig.ni.fi.cân.cia}{0}
\verb{insignificância}{}{}{}{}{}{Coisa de pouco valor, de mínima importância; ninharia.}{in.sig.ni.fi.cân.cia}{0}
\verb{insignificante}{}{}{}{}{adj.2g.}{Que não é significativo, que tem pouco valor ou importância; reles.}{in.sig.ni.fi.can.te}{0}
\verb{insignificante}{}{}{}{}{}{Que é muito pequeno, diminuto, minúsculo.}{in.sig.ni.fi.can.te}{0}
\verb{insinceridade}{}{}{}{}{s.f.}{Falta de sinceridade; hipocrisia, falsidade, fingimento.}{in.sin.ce.ri.da.de}{0}
\verb{insincero}{é}{}{}{}{adj.}{Que não é sincero; fingido, falso, dissimulado.}{in.sin.ce.ro}{0}
\verb{insinuação}{}{}{"-ões}{}{s.f.}{Ato ou efeito de insinuar.}{in.si.nu.a.ção}{0}
\verb{insinuação}{}{}{"-ões}{}{}{Aviso disfarçado ou indireto.}{in.si.nu.a.ção}{0}
\verb{insinuante}{}{}{}{}{adj.2g.}{Que se insinua, que sabe atrair a simpatia ou o interesse dos outros; cativante, sedutor, persuasivo.}{in.si.nu.an.te}{0}
\verb{insinuar}{}{}{}{}{v.t.}{Dar a entender de maneira indireta e sutil; instilar, sugerir.}{in.si.nu.ar}{0}
\verb{insinuar}{}{}{}{}{v.i.}{Fazer insinuações.}{in.si.nu.ar}{0}
\verb{insinuar}{}{}{}{}{v.pron.}{Introduzir"-se habilidosamente na confiança de alguém.}{in.si.nu.ar}{\verboinum{1}}
\verb{insípido}{}{}{}{}{adj.}{Que não tem sabor; insosso.}{in.sí.pi.do}{0}
\verb{insípido}{}{Fig.}{}{}{}{Tedioso, monótono, sem graça.}{in.sí.pi.do}{0}
\verb{insipiência}{}{}{}{}{s.f.}{Qualidade de insipiente; ignorância.}{in.si.pi.ên.cia}{0}
\verb{insipiente}{}{}{}{}{adj.2g.}{Que não é sapiente, que não tem saber; ignorante.}{in.si.pi.en.te}{0}
\verb{insistência}{}{}{}{}{s.f.}{Ato ou efeito de insistir; persistência, obstinação.}{in.sis.tên.cia}{0}
\verb{insistente}{}{}{}{}{adj.2g.}{Que insiste; teimoso, obstinado, perseverante. }{in.sis.ten.te}{0}
\verb{insistente}{}{}{}{}{}{Maçante, importuno, aborrecido.}{in.sis.ten.te}{0}
\verb{insistir}{}{}{}{}{v.t.}{Dizer ou pedir alguma coisa com persistência; teimar, obstinar"-se, instar.}{in.sis.tir}{0}
\verb{insistir}{}{}{}{}{}{Não desistir após muitas tentativas frustradas; perseverar, continuar, persistir.}{in.sis.tir}{\verboinum{18}}
\verb{insociável}{}{}{"-eis}{}{adj.2g.}{Que não é dado ao convívio social; misantropo, retraído, solitário, intratável.}{in.so.ci.á.vel}{0}
\verb{insofismável}{}{}{"-eis}{}{adj.2g.}{Que não se pode sofismar; indiscutível, incontestável.}{in.so.fis.má.vel}{0}
\verb{insofrido}{}{}{}{}{adj.}{Impaciente, sôfrego.}{in.so.fri.do}{0}
\verb{insofrido}{}{}{}{}{}{Que não é sofredor, ou que o é pouco. }{in.so.fri.do}{0}
\verb{insofrível}{}{}{"-eis}{}{adj.2g.}{Que não se pode sofrer; insuportável.}{in.so.frí.vel}{0}
\verb{insolação}{}{Med.}{"-ões}{}{s.f.}{Mal"-estar causado pela exposição prolongada de uma pessoa aos raios solares.}{in.so.la.ção}{0}
\verb{insolência}{}{}{}{}{s.f.}{Qualidade ou caráter de insolente; atrevimento, ousadia, desaforo.}{in.so.lên.cia}{0}
\verb{insolente}{}{}{}{}{adj.2g.}{Que ofende e desrespeita com atos ou palavras; atrevido, desaforado, petulante.}{in.so.len.te}{0}
\verb{insólito}{}{}{}{}{adj.}{Que não acontece com frequência; inusitado, raro, incomum, anormal.}{in.só.li.to}{0}
\verb{insolúvel}{}{}{"-eis}{}{adj.2g.}{Que não se dissolve. }{in.so.lú.vel}{0}
\verb{insolúvel}{}{}{"-eis}{}{}{Que não se pode desfazer; indissolúvel.}{in.so.lú.vel}{0}
\verb{insolúvel}{}{Fig.}{"-eis}{}{}{Que não tem solução, que não se pode resolver.}{in.so.lú.vel}{0}
\verb{insolúvel}{}{Fig.}{"-eis}{}{}{Que não se pode pagar ou cobrar; insolvível, impagável.}{in.so.lú.vel}{0}
\verb{insolvência}{}{}{}{}{s.f.}{Condição de insolvente; inadimplência.}{in.sol.vên.cia}{0}
\verb{insolvente}{}{}{}{}{adj.2g.}{Que não tem como pagar o que deve; inadimplente.}{in.sol.ven.te}{0}
\verb{insondável}{}{}{"-eis}{}{adj.2g.}{Que não se pode sondar; imperscrutável.}{in.son.dá.vel}{0}
\verb{insondável}{}{Fig.}{"-eis}{}{}{Incompreensível, inexplicável.}{in.son.dá.vel}{0}
\verb{insone}{}{}{}{}{adj.2g.}{Que não dorme ou não consegue dormir.}{in.so.ne}{0}
\verb{insone}{}{}{}{}{s.2g.}{Indivíduo que sofre de insônia.}{in.so.ne}{0}
\verb{insônia}{}{}{}{}{s.f.}{Dificuldade para dormir, falta de sono.}{in.sô.nia}{0}
\verb{insonoro}{ó}{}{}{}{adj.}{Que não soa, ou não tem sonoridade.}{in.so.no.ro}{0}
\verb{insopitável}{}{}{"-eis}{}{adj.2g.}{Que não é sopitável; irreprimível, incontrolável.}{in.so.pi.tá.vel}{0}
\verb{insosso}{ô}{}{}{}{adj.}{Diz"-se de alimento que não tem sal, ou não o tem em quantidade suficiente; insulso.}{in.sos.so}{0}
\verb{insosso}{ô}{Por ext.}{}{}{}{Diz"-se de alimento que não tem gosto, que não possui sabor marcante ou não leva tempero nenhum; insípido.}{in.sos.so}{0}
\verb{insosso}{ô}{Fig.}{}{}{}{Que é sem graça; monótono, tedioso, aborrecido, desinteressante, desenxabido.}{in.sos.so}{0}
\verb{inspeção}{}{}{"-ões}{}{s.f.}{Ato ou efeito de inspecionar, examinar alguma coisa com muita atenção; vistoria, fiscalização, exame.}{ins.pe.ção}{0}
\verb{inspecionar}{}{}{}{}{v.t.}{Examinar alguma coisa com muita atenção para saber se está funcionando devidamente; vistoriar, revistar.}{ins.pe.ci.o.nar}{\verboinum{1}}
\verb{inspetor}{ô}{}{}{}{s.m.}{Pessoa encarregada de fazer a inspeção, de examinar.}{ins.pe.tor}{0}
\verb{inspetoria}{}{}{}{}{s.f.}{Cargo ou função de inspetor.}{ins.pe.to.ri.a}{0}
\verb{inspetoria}{}{}{}{}{}{Repartição cujo trabalho é inspecionar.}{ins.pe.to.ri.a}{0}
\verb{inspiração}{}{}{"-ões}{}{s.f.}{Ato ou efeito de inspirar ou de ser inspirado.}{ins.pi.ra.ção}{0}
\verb{inspiração}{}{}{"-ões}{}{}{Movimento da respiração pelo qual o ar entra nos pulmões.}{ins.pi.ra.ção}{0}
\verb{inspiração}{}{Fig.}{"-ões}{}{}{Sentimentos, ideias, pensamentos que parecem sugeridos por uma força sobrenatural.}{ins.pi.ra.ção}{0}
\verb{inspirado}{}{}{}{}{adj.}{Que se inspirou, que foi introduzido nos pulmões.}{ins.pi.ra.do}{0}
\verb{inspirado}{}{Fig.}{}{}{}{Que tem ou revela inspiração.}{ins.pi.ra.do}{0}
\verb{inspirador}{ô}{}{}{}{adj.}{Que inspira.}{ins.pi.ra.dor}{0}
\verb{inspirador}{ô}{Fig.}{}{}{}{Que entusiasma, arrebata.}{ins.pi.ra.dor}{0}
\verb{inspirar}{}{}{}{}{v.t.}{Puxar o ar para dentro dos pulmões.}{ins.pi.rar}{0}
\verb{inspirar}{}{}{}{}{}{Fazer alguém ter ideias.}{ins.pi.rar}{0}
\verb{inspirar}{}{}{}{}{}{Fazer alguém ter determinado pensamento; motivar, sugerir.}{ins.pi.rar}{\verboinum{1}}
\verb{inspiratório}{}{}{}{}{adj.}{Que é próprio para inspirar.}{ins.pi.ra.tó.rio}{0}
\verb{inspiratório}{}{Fig.}{}{}{}{Relativo à inspiração.}{ins.pi.ra.tó.rio}{0}
\verb{instabilidade}{}{}{}{}{s.f.}{Falta de estabilidade, de firmeza.}{ins.ta.bi.li.da.de}{0}
\verb{instabilidade}{}{}{}{}{}{Inconstância, volubilidade, variabilidade.}{ins.ta.bi.li.da.de}{0}
\verb{instalação}{}{}{"-ões}{}{s.f.}{Ato ou efeito de instalar. (\textit{Hoje é que foi feita a instalação da rede elétrica no prédio novo.})}{ins.ta.la.ção}{0}
\verb{instalação}{}{}{"-ões}{}{}{Colocação dos objetos necessários a um trabalho de modo que funcionem corretamente.}{ins.ta.la.ção}{0}
\verb{instalação}{}{}{"-ões}{}{}{Conjunto de peças ou aparelhos dispostos assim.}{ins.ta.la.ção}{0}
\verb{instalar}{}{}{}{}{v.t.}{Dar a alguém um lugar (casa, cômodo) para ficar; alojar, acomodar. (\textit{Instalei nossos hóspedes na edícula.})}{ins.ta.lar}{0}
\verb{instalar}{}{}{}{}{}{Colocar ou preparar algo para funcionar em algum lugar. (\textit{Instalamos a antena parabólica na nossa casa.})}{ins.ta.lar}{\verboinum{1}}
\verb{instância}{}{}{}{}{s.f.}{Qualidade do que é instante.}{ins.tân.cia}{0}
\verb{instância}{}{}{}{}{}{Pedido, solicitação insistente, instante.}{ins.tân.cia}{0}
\verb{instância}{}{Jur.}{}{}{}{Jurisdição, foro.}{ins.tân.cia}{0}
\verb{instantâneo}{}{}{}{}{adj.}{Que acontece num instante, de repente; imediato, rápido, súbito.}{ins.tan.tâ.neo}{0}
\verb{instante}{}{}{}{}{adj.2g.}{Que está prestes a acontecer; iminente.}{ins.tan.te}{0}
\verb{instante}{}{}{}{}{s.m.}{Espaço de tempo curto; momento.}{ins.tan.te}{0}
\verb{instar}{}{}{}{}{v.t.}{Pedir alguma coisa com insistência; suplicar, rogar, insistir.}{ins.tar}{\verboinum{1}}
\verb{instauração}{}{}{"-ões}{}{s.f.}{Ato ou efeito de instaurar; fundação, organização. }{ins.tau.ra.ção}{0}
\verb{instaurar}{}{}{}{}{v.t.}{Fazer com que alguma coisa comece a existir; estabelecer, fundar, iniciar.}{ins.tau.rar}{\verboinum{1}}
\verb{instável}{}{}{"-eis}{}{adj.2g.}{Que não é estável, não tem firmeza, solidez.}{ins.tá.vel}{0}
\verb{instável}{}{Fig.}{"-eis}{}{}{Que muda com facilidade; inconstante, variável.}{ins.tá.vel}{0}
\verb{instigação}{}{}{"-ões}{}{s.f.}{Ato ou efeito de instigar; estímulo, incitação.}{ins.ti.ga.ção}{0}
\verb{instigador}{ô}{}{}{}{adj.}{Que instiga; incitador, estimulador.}{ins.ti.ga.dor}{0}
\verb{instigador}{ô}{}{}{}{s.m.}{Indivíduo que instiga.}{ins.ti.ga.dor}{0}
\verb{instigante}{}{}{}{}{adj.2g.}{Que instiga; instigador, estimulante, incitador.}{ins.ti.gan.te}{0}
\verb{instigar}{}{}{}{}{v.t.}{Incitar; estimular.}{ins.ti.gar}{\verboinum{5}}
\verb{instilar}{}{}{}{}{v.t.}{Introduzir gota a gota.}{ins.ti.lar}{0}
\verb{instilar}{}{Fig.}{}{}{}{Insinuar, insuflar.}{ins.ti.lar}{\verboinum{1}}
\verb{instintivo}{}{}{}{}{adj.}{Relativo ao instinto; inato.}{ins.tin.ti.vo}{0}
\verb{instintivo}{}{}{}{}{}{Que se faz sem ter aprendido, por instinto; automático, natural, maquinal.}{ins.tin.ti.vo}{0}
\verb{instinto}{}{}{}{}{s.m.}{Capacidade que todo animal tem, e que já nasce com ele e o acompanha por toda a vida, de se comportar de determinada maneira e fazer certas coisas sem precisar aprendê"-las.}{ins.tin.to}{0}
\verb{institucional}{}{}{"-ais}{}{adj.2g.}{Relativo a instituição, ou a instituições.}{ins.ti.tu.ci.o.nal}{0}
\verb{institucionalizar}{}{}{}{}{v.t.}{Tornar institucional.}{ins.ti.tu.ci.o.na.li.zar}{\verboinum{1}}
\verb{instituição}{}{}{"-ões ⟨"-ões⟩}{}{s.f.}{Ato de instituir; criação, estabelecimento.}{ins.ti.tu.i.ção}{0}
\verb{instituição}{}{}{"-ões ⟨"-ões⟩}{}{}{Organização permanente, pública ou particular, de interesse social.}{ins.ti.tu.i.ção}{0}
\verb{instituições}{}{}{}{}{s.f.pl.}{Leis, princípios ou regras que orientam uma sociedade política; regime.}{ins.ti.tu.i.ções}{0}
\verb{instituições}{}{}{}{}{}{Conjunto de organizações sociais estabelecidas pela tradição e relacionadas à coisa pública.}{ins.ti.tu.i.ções}{0}
\verb{instituir}{}{}{}{}{v.t.}{Fazer algum tipo de regra começar a valer; criar, estabelecer.}{ins.ti.tu.ir}{0}
\verb{instituir}{}{}{}{}{}{Dar início a alguma atividade; criar, estabelecer, fundar. (\textit{Nossa escola instituiu a hora da leitura para todas as classes.})}{ins.ti.tu.ir}{\verboinum{26}}
\verb{instituto}{}{}{}{}{s.m.}{Organização de interesse social, sem fins lucrativos. (\textit{Visitamos com frequência o Instituto Butantã.})}{ins.ti.tu.to}{0}
\verb{instituto}{}{}{}{}{}{Conjunto de regras para atender a um grupo social.}{ins.ti.tu.to}{0}
\verb{instrução}{}{}{"-ões}{}{s.f.}{Ato de instruir.}{ins.tru.ção}{0}
\verb{instrução}{}{}{"-ões}{}{}{Soma de conhecimentos de uma pessoa. (\textit{Com pouca instrução é difícil conseguir esse emprego.})}{ins.tru.ção}{0}
\verb{instrução}{}{}{"-ões}{}{}{Explicação para se poder fazer ou usar alguma coisa. (\textit{Leio as instruções com atenção sempre que compro um aparelho novo.})}{ins.tru.ção}{0}
\verb{instruído}{}{}{}{}{adj.}{Que recebeu instrução, ou se instruiu; escolarizado, culto.}{ins.tru.í.do}{0}
\verb{instruir}{}{}{}{}{v.t.}{Passar a alguém um conjunto de conhecimentos sobre algum assunto; ensinar.}{ins.tru.ir}{0}
\verb{instruir}{}{}{}{}{}{Falar a uma pessoa sobre o que ela vai fazer; dar instrução, explicar, informar.}{ins.tru.ir}{0}
\verb{instruir}{}{}{}{}{}{Fazer pessoa ou animal aprender a maneira de fazer alguma coisa; treinar, adestrar.}{ins.tru.ir}{\verboinum{26}}
\verb{instrumentador}{ô}{Med.}{}{}{s.m.}{Pessoa que instrumenta numa cirurgia.}{ins.tru.men.ta.dor}{0}
\verb{instrumental}{}{}{"-ais}{}{adj.2g.}{Que serve de instrumento.}{ins.tru.men.tal}{0}
\verb{instrumental}{}{}{"-ais}{}{s.m.}{Conjunto de instrumentos necessários para realizar uma tarefa específica.}{ins.tru.men.tal}{0}
\verb{instrumentar}{}{Mús.}{}{}{v.t.}{Escrever as partes de cada instrumento de uma composição musical, ou escolher quais instrumentos executarão uma peça musical.}{ins.tru.men.tar}{0}
\verb{instrumentar}{}{Med.}{}{}{}{Auxiliar o cirurgião, passando"-lhe os instrumentos de que precisa para operar, durante uma intervenção cirúrgica.}{ins.tru.men.tar}{\verboinum{1}}
\verb{instrumentista}{}{}{}{}{s.2g.}{Pessoa que toca um instrumento musical; músico.}{ins.tru.men.tis.ta}{0}
\verb{instrumento}{}{}{}{}{s.m.}{Objeto feito para uso de um profissional.}{ins.tru.men.to}{0}
\verb{instrumento}{}{}{}{}{}{Pessoa ou coisa que se usa para determinado ato.}{ins.tru.men.to}{0}
\verb{instrutivo}{}{}{}{}{adj.}{Que instrui, que é próprio para instruir, para educar; educativo.}{ins.tru.ti.vo}{0}
\verb{instrutor}{ô}{}{}{}{adj.}{Que instrui, que ensina ou adestra.}{ins.tru.tor}{0}
\verb{instrutor}{ô}{}{}{}{s.m.}{Essa pessoa.}{ins.tru.tor}{0}
\verb{insubmissão}{}{}{"-ões}{}{s.f.}{Falta de submissão; caráter de insubmisso; insubordinação, rebeldia.}{in.sub.mis.são}{0}
\verb{insubmisso}{}{}{}{}{adj.}{Que não se submete; independente, insubordinado.}{in.sub.mis.so}{0}
\verb{insubmisso}{}{Bras.}{}{}{s.m.}{Cidadão que não se apresenta às autoridades quando convocado para o serviço militar.}{in.sub.mis.so}{0}
\verb{insubordinação}{}{}{"-ões}{}{s.f.}{Falta de subordinação; desobediência.}{in.su.bor.di.na.ção}{0}
\verb{insubordinação}{}{}{"-ões}{}{}{Rebelião, revolta, motim.}{in.su.bor.di.na.ção}{0}
\verb{insubordinado}{}{}{}{}{adj.}{Que se insubordinou; rebelde, indisciplinado.}{in.su.bor.di.na.do}{0}
\verb{insubordinado}{}{}{}{}{s.m.}{Pessoa insubordinada.}{in.su.bor.di.na.do}{0}
\verb{insubordinar}{}{}{}{}{v.t.}{Tornar insubordinado; rebelar, sublevar, amotinar. }{in.su.bor.di.nar}{\verboinum{1}}
\verb{insubornável}{}{}{"-eis}{}{adj.2g.}{Que não se deixa subornar; incorruptível, íntegro, probo.}{in.su.bor.ná.vel}{0}
\verb{insubsistente}{zis}{}{}{}{adj.2g.}{Que não pode subsistir. }{in.sub.sis.ten.te}{0}
\verb{insubsistente}{zis}{}{}{}{}{Que não tem fundamento, não tem razão de ser; insustentável.}{in.sub.sis.ten.te}{0}
\verb{insubstituível}{}{}{"-eis}{}{adj.2g.}{Que não pode ser substituído, trocado por outro; inigualável.}{in.subs.ti.tu.í.vel}{0}
\verb{insucesso}{é}{}{}{}{s.m.}{Mau resultado; fracasso, malogro.}{in.su.ces.so}{0}
\verb{insuficiência}{}{}{}{}{s.f.}{Característica ou condição de insuficiente; carência, falta.}{in.su.fi.ci.ên.cia}{0}
\verb{insuficiência}{}{}{}{}{}{Incapacidade, incompetência.}{in.su.fi.ci.ên.cia}{0}
\verb{insuficiente}{}{}{}{}{adj.2g.}{Que não é suficiente, que não basta; pouco.}{in.su.fi.ci.en.te}{0}
\verb{insuficiente}{}{}{}{}{}{Incapaz, incompetente, insatisfatório.}{in.su.fi.ci.en.te}{0}
\verb{insuflar}{}{}{}{}{v.t.}{Encher de ar alguma coisa, soprando.}{in.su.flar}{0}
\verb{insuflar}{}{Fig.}{}{}{}{Insinuar.}{in.su.flar}{\verboinum{1}}
\verb{insulado}{}{}{}{}{adj.}{Que se insulou; ilhado, isolado, separado, apartado.}{in.su.la.do}{0}
\verb{insulano}{}{}{}{}{adj.}{Relativo a ilha.}{in.su.la.no}{0}
\verb{insulano}{}{}{}{}{s.m.}{Natural ou habitante de uma ilha; ilhéu, insular.}{in.su.la.no}{0}
\verb{insular}{}{}{}{}{adj.2g. e s.2g.}{Insulano.}{in.su.lar}{0}
\verb{insulina}{}{Bioquím.}{}{}{s.f.}{Hormônio secretado pelo pâncreas, responsável pelo metabolismo dos açúcares no organismo.}{in.su.li.na}{0}
\verb{insulso}{}{}{}{}{adj.}{Sem sal; insosso.}{in.sul.so}{0}
\verb{insulso}{}{Por ext.}{}{}{}{Sem sabor; insípido.}{in.sul.so}{0}
\verb{insultador}{ô}{}{}{}{adj.}{Que insulta; insultante.}{in.sul.ta.dor}{0}
\verb{insultador}{ô}{}{}{}{s.m.}{Pessoa que tem por hábito insultar os outros.}{in.sul.ta.dor}{0}
\verb{insultante}{}{}{}{}{adj.2g.}{Que insulta, ou que é dado a insultar. }{in.sul.tan.te}{0}
\verb{insultar}{}{}{}{}{v.t.}{Ofender alguém por palavra ou ação; injuriar, ultrajar.}{in.sul.tar}{\verboinum{1}}
\verb{insulto}{}{}{}{}{s.m.}{Ato ou palavra com que se ofende pessoa ou coisa.}{in.sul.to}{0}
\verb{insultuoso}{ô}{}{"-osos ⟨ó⟩}{"-osa ⟨ó⟩}{adj.}{Que insulta, afronta, ofende; insultante.}{in.sul.tu.o.so}{0}
\verb{insumo}{}{Econ.}{}{}{s.m.}{Qualquer coisa que faça parte do processo de produção de mercadorias ou serviços, como máquinas, matéria"-prima, equipamentos, trabalho humano etc.}{in.su.mo}{0}
\verb{insuperável}{}{}{"-eis}{}{adj.2g.}{Que não pode ser superado, ultrapassado; inexcedível, invencível. }{in.su.pe.rá.vel}{0}
\verb{insuportável}{}{}{"-eis}{}{adj.2g.}{Que não é suportável; intolerável.}{in.su.por.tá.vel}{0}
\verb{insurgente}{}{}{}{}{adj.2g.}{Que se insurge ou insurgiu; rebelde, insurrecionado, revoltoso, insubordinado.  }{in.sur.gen.te}{0}
\verb{insurgir}{}{}{}{}{v.pron.}{Passar a lutar contra o poder de pessoa ou ação dela; revoltar"-se.}{in.sur.gir"-se}{\verboinum{22}}
\verb{insurrecto}{é}{}{}{}{}{Var. de \textit{insurreto}.}{in.sur.rec.to}{0}
\verb{insurreição}{}{}{"-ões}{}{s.f.}{Ato ou efeito de se insurgir contra um poder constituído; revolta, rebelião, levante, motim.}{in.sur.rei.ção}{0}
\verb{insurreto}{é}{}{}{}{adj.}{Que se insurgiu; insurgente. }{in.sur.re.to}{0}
\verb{insusceptível}{}{}{}{}{}{Var. de \textit{insuscetível}.}{in.sus.cep.tí.vel}{0}
\verb{insuscetível}{}{}{"-eis}{}{adj.2g.}{Que não é suscetível; incapaz.}{in.sus.ce.tí.vel}{0}
\verb{insuspeição}{}{}{"-ões}{}{s.f.}{Falta de suspeição; qualidade de insuspeito.}{in.sus.pei.ção}{0}
\verb{insuspeito}{ê}{}{}{}{adj.}{De que, ou de quem não se pode levantar suspeitas, desconfiar, duvidar; fidedigno.}{in.sus.pei.to}{0}
\verb{insustável}{}{}{"-eis}{}{adj.2g.}{Que não se pode sustar; não sustável.}{in.sus.tá.vel}{0}
\verb{insustentável}{}{}{"-eis}{}{adj.2g.}{Que não se pode sustentar, manter. }{in.sus.ten.tá.vel}{0}
\verb{insustentável}{}{}{"-eis}{}{}{Que não se pode sustentar, tolerar mais; intolerável, insuportável.}{in.sus.ten.tá.vel}{0}
\verb{insustentável}{}{}{"-eis}{}{}{Que não tem fundamento, que não pode ser defendido; insubsistente, indefensável.}{in.sus.ten.tá.vel}{0}
\verb{intacto}{}{}{}{}{adj.}{Que não foi tocado, mexido ou alterado.}{in.tac.to}{0}
\verb{intacto}{}{}{}{}{}{Que não sofreu nenhum dano; incólume, ileso.}{in.tac.to}{0}
\verb{intangibilidade}{}{}{}{}{s.f.}{Qualidade ou condição de intangível.}{in.tan.gi.bi.li.da.de}{0}
\verb{intangível}{}{}{"-eis}{}{adj.2g.}{Que não se pode tocar; intocável.}{in.tan.gí.vel}{0}
\verb{intanha}{}{}{}{}{s.f.}{Untanha.}{in.ta.nha}{0}
\verb{intato}{}{}{}{}{}{Var. de \textit{intacto}.}{in.ta.to}{0}
\verb{íntegra}{}{}{}{}{s.f.}{A totalidade de alguma coisa; integridade.}{ín.te.gra}{0}
\verb{íntegra}{}{}{}{}{}{Palavra usada na locução adverbial \textit{na íntegra}: integralmente, a que não falta nada; sem faltar nenhuma palavra.}{ín.te.gra}{0}
\verb{integração}{}{}{"-ões}{}{s.f.}{Ato ou efeito de integrar, de incluir alguma coisa num conjunto. }{in.te.gra.ção}{0}
\verb{integral}{}{}{"-ais}{}{adj.2g.}{Que contém todas as partes, a que não falta nada; completo, total, inteiro.}{in.te.gral}{0}
\verb{integral}{}{Por ext.}{"-ais}{}{}{Diz"-se de um alimento preparado com um produto integral, como um cereal que foi apenas descascado, conservando sua película com os nutrientes, vitaminas etc. (\textit{O arroz integral é um alimento saudável.})}{in.te.gral}{0}
\verb{integralidade}{}{}{}{}{s.f.}{Qualidade ou condição do que é integral; totalidade.}{in.te.gra.li.da.de}{0}
\verb{integralismo}{}{}{}{}{s.m.}{Aplicação integral de uma doutrina ou sistema. }{in.te.gra.lis.mo}{0}
\verb{integralismo}{}{Hist.}{}{}{}{Movimento político brasileiro de inspiração fascista, fundado em 1932, por Plínio Salgado, e extinto em 1937.  }{in.te.gra.lis.mo}{0}
\verb{integralista}{}{}{}{}{adj.2g.}{Relativo ao integralismo (movimento político).}{in.te.gra.lis.ta}{0}
\verb{integralista}{}{}{}{}{s.2g.}{Partidário ou simpatizante do integralismo.}{in.te.gra.lis.ta}{0}
\verb{integralizar}{}{}{}{}{v.t.}{Integrar.}{in.te.gra.li.zar}{\verboinum{1}}
\verb{integrante}{}{}{}{}{adj.2g.}{Que integra, que completa.}{in.te.gran.te}{0}
\verb{integrante}{}{}{}{}{s.2g.}{Pessoa que integra, que compõe.}{in.te.gran.te}{0}
\verb{integrar}{}{}{}{}{v.t.}{Tornar inteiro ou integrar; completar, integralizar.}{in.te.grar}{0}
\verb{integrar}{}{}{}{}{v.pron.}{Juntar"-se, fazendo parte de um todo; reunir"-se, incorporar"-se.}{in.te.grar}{\verboinum{1}}
\verb{integridade}{}{}{}{}{s.f.}{Qualidade de íntegro; inteireza.}{in.te.gri.da.de}{0}
\verb{integridade}{}{Fig.}{}{}{}{Integridade moral; retidão, honestidade.}{in.te.gri.da.de}{0}
\verb{íntegro}{}{}{}{}{adj.}{Que está inteiro; completo.}{ín.te.gro}{0}
\verb{íntegro}{}{Fig.}{}{}{}{Honesto, reto, incorruptível, imparcial.}{ín.te.gro}{0}
\verb{inteirar}{}{}{}{}{v.t.}{Fazer alguma coisa ficar completa, sem faltar mais nada; completar, integrar.}{in.tei.rar}{0}
\verb{inteirar}{}{}{}{}{}{Fazer alguém ficar sabendo de alguma coisa; cientificar, informar, notificar.}{in.tei.rar}{\verboinum{1}}
\verb{inteireza}{ê}{}{}{}{s.f.}{Qualidade ou estado do que é inteiro.}{in.tei.re.za}{0}
\verb{inteireza}{ê}{}{}{}{}{Integridade física ou moral.}{in.tei.re.za}{0}
\verb{inteiriçar}{}{}{}{}{v.t.}{Tornar ou ficar inteiriço, rígido; entesar, enrijecer.}{in.tei.ri.çar}{\verboinum{3}}
\verb{inteiriço}{}{}{}{}{adj.}{Que só tem uma peça, ou uma parte.}{in.tei.ri.ço}{0}
\verb{inteiro}{ê}{}{}{}{adj.}{Com todas as suas partes.}{in.tei.ro}{0}
\verb{inteiro}{ê}{}{}{}{}{Em perfeito estado. (\textit{Quando vendi o carro, ele estava inteiro.})}{in.tei.ro}{0}
\verb{inteiro}{ê}{Mat.}{}{}{}{Diz"-se de qualquer dos números.}{in.tei.ro}{0}
\verb{intelecção}{}{}{"-ões}{}{s.f.}{Ato ou processo de entender; compreensão, entendimento.}{in.te.lec.ção}{0}
\verb{intelectivo}{}{}{}{}{adj.}{Relativo ao intelecto, ao entendimento.}{in.te.lec.ti.vo}{0}
\verb{intelecto}{é}{}{}{}{s.m.}{Capacidade de pensar e combinar os pensamentos; inteligência.}{in.te.lec.to}{0}
\verb{intelectual}{}{}{"-ais}{}{adj.2g.}{Relativo ao intelecto.}{in.te.lec.tu.al}{0}
\verb{intelectual}{}{}{"-ais}{}{s.2g.}{Pessoa que se ocupa de coisas ligadas ao intelecto.}{in.te.lec.tu.al}{0}
\verb{intelectualidade}{}{}{}{}{s.f.}{Qualidade de intelectual.}{in.te.lec.tu.a.li.da.de}{0}
\verb{intelectualidade}{}{}{}{}{}{Conjunto de intelectuais.}{in.te.lec.tu.a.li.da.de}{0}
\verb{intelectualismo}{}{Filos.}{}{}{s.m.}{Doutrina que afirma o predomínio da inteligência sobre os sentidos, as emoções e os instintos.}{in.te.lec.tu.a.lis.mo}{0}
\verb{intelectualista}{}{}{}{}{adj.2g.}{Relativo ao intelectualismo.}{in.te.lec.tu.a.lis.ta}{0}
\verb{intelectualista}{}{}{}{}{s.2g.}{Adepto do intelectualismo (doutrina).}{in.te.lec.tu.a.lis.ta}{0}
\verb{intelectualizar}{}{}{}{}{v.t.}{Tornar intelectual.}{in.te.lec.tu.a.li.zar}{\verboinum{1}}
\verb{inteligência}{}{}{}{}{s.f.}{Capacidade de pensar e combinar os pensamentos; intelecto.}{in.te.li.gên.cia}{0}
\verb{inteligência}{}{}{}{}{}{Capacidade de resolver as dificuldades apresentadas pela vida.}{in.te.li.gên.cia}{0}
\verb{inteligente}{}{}{}{}{adj.2g.}{Que é dotado de ou revela inteligência.}{in.te.li.gen.te}{0}
\verb{inteligível}{}{}{"-eis}{}{adj.2g.}{Que se pode entender, compreender; compreensível. }{in.te.li.gí.vel}{0}
\verb{intemerato}{}{}{}{}{adj.}{Que não foi corrompido; íntegro, puro.}{in.te.me.ra.to}{0}
\verb{intemperança}{}{}{}{}{s.f.}{Falta de temperança, de comedimento; imoderação.}{in.tem.pe.ran.ça}{0}
\verb{intemperante}{}{}{}{}{adj.2g.}{Que não tem temperança; dissoluto, descomedido, imoderado, incontinente.  }{in.tem.pe.ran.te}{0}
\verb{intempérie}{}{}{}{}{s.f.}{Mau tempo; tempestade.}{in.tem.pé.rie}{0}
\verb{intempestivo}{}{}{}{}{adj.}{Que ocorre fora do tempo normal; inoportuno.}{in.tem.pes.ti.vo}{0}
\verb{intempestivo}{}{}{}{}{}{Que ocorre sem ser esperado; imprevisto, súbito, repentino.}{in.tem.pes.ti.vo}{0}
\verb{intemporal}{}{}{"-ais}{}{adj.2g.}{Que não é temporal, transitório; eterno, perene, atemporal.}{in.tem.po.ral}{0}
\verb{intemporal}{}{}{"-ais}{}{}{Que não é temporal, profano; espiritual.}{in.tem.po.ral}{0}
\verb{intenção}{}{}{"-ões}{}{s.f.}{Desejo que se pretende realizar; intento, objetivo, plano, propósito.}{in.ten.ção}{0}
\verb{intencionado}{}{}{}{}{adj.}{Feito com intenção, de propósito; intencional, proposital.}{in.ten.ci.o.na.do}{0}
\verb{intencional}{}{}{"-ais}{}{adj.2g.}{Feito por querer, de propósito; proposital.}{in.ten.ci.o.nal}{0}
\verb{intencionar}{}{}{}{}{v.t.}{Ter a intenção de fazer alguma coisa; tencionar.  }{in.ten.ci.o.nar}{\verboinum{1}}
\verb{intendência}{}{}{}{}{s.f.}{Cargo de intendente.  	 }{in.ten.dên.cia}{0}
\verb{intendência}{}{}{}{}{}{Repartição onde o intendente trabalha.}{in.ten.dên.cia}{0}
\verb{intendente}{}{}{}{}{s.2g.}{Pessoa que coordena, administra alguma coisa.}{in.ten.den.te}{0}
\verb{intensão}{}{}{"-ões}{}{s.f.}{Intensidade, força, veemência, energia.}{in.ten.são}{0}
\verb{intensidade}{}{}{}{}{s.f.}{Qualidade do que é intenso.}{in.ten.si.da.de}{0}
\verb{intensificação}{}{}{"-ões}{}{s.f.}{Ato ou efeito de intensificar; aumento.}{in.ten.si.fi.ca.ção}{0}
\verb{intensificar}{}{}{}{}{v.t.}{Tornar intenso ou mais intenso.}{in.ten.si.fi.car}{\verboinum{2}}
\verb{intensivo}{}{}{}{}{adj.}{Que se faz em um tempo menor  que o usual.}{in.ten.si.vo}{0}
\verb{intenso}{}{}{}{}{adj.}{Que se faz sentir com violência. (\textit{Hoje à tarde a chuva foi intensa.})}{in.ten.so}{0}
\verb{intenso}{}{}{}{}{}{Em que se faz muito esforço; exaustivo. (\textit{A recepção do hotel teve um final de semana intenso com a chegada das bandas de })rock\textit{.}}{in.ten.so}{0}
\verb{intentar}{}{}{}{}{v.t.}{Tentar, planejar, projetar.}{in.ten.tar}{0}
\verb{intentar}{}{}{}{}{}{Esforçar"-se; diligenciar.}{in.ten.tar}{0}
\verb{intentar}{}{}{}{}{}{Empreender, praticar.}{in.ten.tar}{0}
\verb{intentar}{}{Jur.}{}{}{}{Propor em juízo.}{in.ten.tar}{\verboinum{1}}
\verb{intento}{}{}{}{}{s.m.}{Aquilo que se pretende fazer; desígnio, intenção, propósito, objetivo.}{in.ten.to}{0}
\verb{intentona}{}{}{}{}{s.f.}{Intento louco, plano insensato.}{in.ten.to.na}{0}
\verb{intentona}{}{}{}{}{}{Ataque imprevisto.}{in.ten.to.na}{0}
\verb{intentona}{}{}{}{}{}{Conspiração para revolta ou motim.}{in.ten.to.na}{0}
\verb{interação}{}{}{"-ões}{}{s.f.}{Ação que se exerce mutuamente entre duas ou mais coisas, ou duas ou mais pessoas; ação recíproca.}{in.te.ra.ção}{0}
\verb{interagir}{}{}{}{}{v.i.}{Agir, influenciar mutuamente; exercer interação.}{in.te.ra.gir}{\verboinum{22}}
\verb{interativo}{}{}{}{}{adj.}{Em que ocorre interação.}{in.te.ra.ti.vo}{0}
\verb{interativo}{}{}{}{}{}{Que permite ao indivíduo interagir com a fonte ou o emissor.}{in.te.ra.ti.vo}{0}
\verb{intercadente}{}{}{}{}{adj.2g.}{Que não é contínuo; intermitente, interrompido.}{in.ter.ca.den.te}{0}
\verb{intercadente}{}{}{}{}{}{Que oscila; irregular, variável.}{in.ter.ca.den.te}{0}
\verb{intercalação}{}{}{"-ões}{}{s.f.}{Ato ou efeito de intercalar.}{in.ter.ca.la.ção}{0}
\verb{intercalação}{}{}{"-ões}{}{}{Colocação de uma coisa no meio de outras.}{in.ter.ca.la.ção}{0}
\verb{intercalação}{}{}{"-ões}{}{}{Adição de um elemento a uma sequência ou conjunto.}{in.ter.ca.la.ção}{0}
\verb{intercalar}{}{}{}{}{adj.2g.}{Que se intercala, que se mete de permeio.}{in.ter.ca.lar}{0}
\verb{intercalar}{}{}{}{}{v.t.}{Pôr de permeio; entremear, interpor.}{in.ter.ca.lar}{\verboinum{1}}
\verb{intercambiar}{}{}{}{}{v.t.}{Fazer intercâmbio; permutar, trocar.}{in.ter.cam.bi.ar}{\verboinum{6}}
\verb{intercâmbio}{}{}{}{}{s.m.}{Troca, permuta.}{in.ter.câm.bio}{0}
\verb{intercâmbio}{}{}{}{}{}{Reciprocidade de relações entre nações.}{in.ter.câm.bio}{0}
\verb{interceder}{ê}{}{}{}{v.t.}{Pedir, suplicar; intervir a favor de alguém ou de algo.}{in.ter.ce.der}{\verboinum{12}}
\verb{intercelular}{}{}{}{}{adj.2g.}{Que se localiza entre as células.}{in.ter.ce.lu.lar}{0}
\verb{interceptação}{}{}{"-ões}{}{s.f.}{Ato ou efeito de interceptar; interrompimento.}{in.ter.cep.ta.ção}{0}
\verb{interceptar}{}{}{}{}{v.t.}{Interromper no seu curso; deter ou impedir na passagem.}{in.ter.cep.tar}{0}
\verb{interceptar}{}{}{}{}{}{Fazer para; cortar, interromper.}{in.ter.cep.tar}{0}
\verb{interceptar}{}{}{}{}{}{Captar ou apreender o que era destinado a outrem.}{in.ter.cep.tar}{\verboinum{1}}
\verb{intercessão}{}{}{"-ões}{}{s.f.}{Ato ou efeito de interceder; intervenção.}{in.ter.ces.são}{0}
\verb{intercessor}{ô}{}{}{}{adj.}{Que intercede; medianeiro.}{in.ter.ces.sor}{0}
\verb{intercessor}{ô}{}{}{}{s.m.}{Indivíduo que intercede.}{in.ter.ces.sor}{0}
\verb{interclube}{}{}{}{}{adj.}{Que se realiza ou se disputa entre dois ou mais clubes.}{in.ter.clu.be}{0}
\verb{intercomunicação}{}{}{"-ões}{}{s.f.}{Ato ou efeito de intercomunicar"-se; comunicação recíproca.}{in.ter.co.mu.ni.ca.ção}{0}
\verb{intercomunicar"-se}{}{}{}{}{v.pron.}{Comunicar"-se mutuamente.}{in.ter.co.mu.ni.car"-se}{\verboinum{2}}
\verb{intercontinental}{}{}{"-ais}{}{adj.2g.}{Situado entre continentes.}{in.ter.con.ti.nen.tal}{0}
\verb{intercontinental}{}{}{"-ais}{}{}{Que se realiza entre dois continentes.}{in.ter.con.ti.nen.tal}{0}
\verb{intercontinental}{}{}{"-ais}{}{}{Que se faz de um continente a outro.}{in.ter.con.ti.nen.tal}{0}
\verb{intercorrência}{}{}{}{}{s.f.}{Qualidade de intercorrente.}{in.ter.cor.rên.cia}{0}
\verb{intercorrência}{}{}{}{}{}{Alternativa, variação.}{in.ter.cor.rên.cia}{0}
\verb{intercorrente}{}{}{}{}{adj.2g.}{Que intercorre, que sobrevém no curso de algo.}{in.ter.cor.ren.te}{0}
\verb{intercorrente}{}{}{}{}{}{Que é irregular, variável.}{in.ter.cor.ren.te}{0}
\verb{intercorrer}{ê}{}{}{}{v.i.}{Correr pelo meio ou no interior.}{in.ter.cor.rer}{0}
\verb{intercorrer}{ê}{}{}{}{}{Decorrer entre uma coisa e outra.}{in.ter.cor.rer}{0}
\verb{intercorrer}{ê}{}{}{}{}{Sobrevir, suceder.}{in.ter.cor.rer}{\verboinum{12}}
\verb{intercostal}{}{}{"-ais}{}{adj.2g.}{Que se localiza entre as costelas.}{in.ter.cos.tal}{0}
\verb{intercurso}{}{}{}{}{s.m.}{Comunicação, trato.}{in.ter.cur.so}{0}
\verb{interdependência}{}{}{}{}{s.f.}{Dependência mútua.}{in.ter.de.pen.dên.cia}{0}
\verb{interdependente}{}{}{}{}{adj.2g.}{Que tem interdependência.}{in.ter.de.pen.den.te}{0}
\verb{interdepender}{ê}{}{}{}{v.i.}{Depender mutuamente.}{in.ter.de.pen.der}{\verboinum{12}}
\verb{interdição}{}{}{"-ões}{}{s.f.}{Ato ou efeito de interdizer; proibição.}{in.ter.di.ção}{0}
\verb{interdigital}{}{}{"-ais}{}{adj.2g.}{Que se localiza entre os dedos.}{in.ter.di.gi.tal}{0}
\verb{interdisciplinar}{}{}{}{}{adj.2g.}{Que é comum a duas ou mais disciplinas ou ramos do conhecimento.}{in.ter.dis.ci.pli.nar}{0}
\verb{interditar}{}{}{}{}{v.t.}{Proibir o acesso a determinado local.}{in.ter.di.tar}{0}
\verb{interditar}{}{}{}{}{}{Impedir a locomoção; apreender.}{in.ter.di.tar}{0}
\verb{interditar}{}{}{}{}{}{Impedir ou proibir a realização de algo; interdizer.}{in.ter.di.tar}{\verboinum{1}}
\verb{interdito}{}{}{}{}{adj.}{Que está sob interdição; interditado.}{in.ter.di.to}{0}
\verb{interdito}{}{}{}{}{s.m.}{Proibição; interdição.}{in.ter.di.to}{0}
\verb{interdizer}{ê}{}{}{}{v.t.}{Privar alguém do direito de exercer suas funções; suspender.}{in.ter.di.zer}{\verboinum{41}}
\verb{interessado}{}{}{}{}{adj.}{Que tem interesse em algo.}{in.te.res.sa.do}{0}
\verb{interessado}{}{}{}{}{}{Que tem por base interesses pessoais.}{in.te.res.sa.do}{0}
\verb{interessado}{}{}{}{}{s.m.}{Indivíduo que coparticipa dos lucros de uma firma.}{in.te.res.sa.do}{0}
\verb{interessante}{}{}{}{}{adj.2g.}{Que oferece interesse; que chama a atenção.}{in.te.res.san.te}{0}
\verb{interessante}{}{}{}{}{}{Que é atraente, simpático.}{in.te.res.san.te}{0}
\verb{interessar}{}{}{}{}{v.t.}{Ter interesse, importância ou utilidade para alguém; importar.}{in.te.res.sar}{0}
\verb{interessar}{}{}{}{}{}{Dizer respeito; concernir.}{in.te.res.sar}{0}
\verb{interessar}{}{}{}{}{}{Provocar o interesse, a curiosidade; cativar.}{in.te.res.sar}{0}
\verb{interessar}{}{}{}{}{}{Ferir, lesar, atingir. }{in.te.res.sar}{0}
\verb{interessar}{}{}{}{}{}{Tornar alguém favorável e solidário.}{in.te.res.sar}{0}
\verb{interessar}{}{}{}{}{v.pron.}{Sentir atração por alguém ou alguma coisa.}{in.te.res.sar}{\verboinum{1}}
\verb{interesse}{ê}{}{}{}{s.m.}{Sentimento de querer saber ainda mais sobre alguma coisa; curiosidade.}{in.te.res.se}{0}
\verb{interesse}{ê}{}{}{}{}{Aquilo que interessa, que convém, que importa.}{in.te.res.se}{0}
\verb{interesseiro}{ê}{}{}{}{adj.}{Diz"-se do indivíduo que só atende a seus interesses; egoísta.}{in.te.res.sei.ro}{0}
\verb{interesseiro}{ê}{}{}{}{}{Que é feito ou inspirado por interesse.}{in.te.res.sei.ro}{0}
\verb{interestadual}{}{}{}{}{adj.2g.}{Que se refere ou se efetua entre dois ou mais estados do mesmo país.}{in.te.res.ta.du.al}{0}
\verb{interestelar}{}{}{}{}{adj.2g.}{Que se situa entre as estrelas.}{in.te.res.te.lar}{0}
\verb{interface}{}{}{}{}{s.f.}{Elemento que proporciona uma ligação física ou lógica entre dois sistemas ou partes de um sistema que não poderiam ser conectados diretamente.}{in.ter.fa.ce}{0}
\verb{interface}{}{Informát.}{}{}{}{Dispositivo que liga ao computador algum elemento de fora dele.}{in.ter.fa.ce}{0}
\verb{interferência}{}{}{}{}{s.f.}{Ato ou efeito de interferir; intervenção, interposição.}{in.ter.fe.rên.cia}{0}
\verb{interferente}{}{}{}{}{adj.2g.}{Que interfere, intervém.}{in.ter.fe.ren.te}{0}
\verb{interferir}{}{}{}{}{v.i.}{Intervir.}{in.ter.fe.rir}{0}
\verb{interferir}{}{}{}{}{}{Produzir interferência.}{in.ter.fe.rir}{\verboinum{29}}
\verb{interfonar}{}{}{}{}{v.t.}{Comunicar"-se por meio de interfone.}{in.ter.fo.nar}{\verboinum{1}}
\verb{interfone}{ô}{}{}{}{s.m.}{Aparelho eletrônico usado para a comunicação entre salas, escritórios, apartamentos etc.}{in.ter.fo.ne}{0}
\verb{intergaláctico}{}{Astron.}{}{}{adj.}{Que se situa entre as galáxias.}{in.ter.ga.lác.ti.co}{0}
\verb{intergovernamental}{}{}{"-ais}{}{adj.2g.}{Que se realiza entre dois ou mais governos ou governadores.}{in.ter.go.ver.na.men.tal}{0}
\verb{ínterim}{}{}{}{}{s.m.}{Qualidade do que é interino; provisório.}{ín.te.rim}{0}
\verb{ínterim}{}{}{}{}{}{Intervalo de tempo entre dois fatos.}{ín.te.rim}{0}
\verb{interinidade}{}{}{}{}{s.f.}{Qualidade ou estado do que é interino; temporário.}{in.te.ri.ni.da.de}{0}
\verb{interinidade}{}{}{}{}{}{Exercício provisório de função ou cargo público.}{in.te.ri.ni.da.de}{0}
\verb{interino}{}{}{}{}{adj.}{Temporário, provisório.}{in.te.ri.no}{0}
\verb{interino}{}{}{}{}{s.m.}{Indivíduo que exerce funções só durante o tempo de impedimento de outrem.}{in.te.ri.no}{0}
\verb{interior}{ô}{}{}{}{adj.2g.}{Que está dentro; interno.}{in.te.ri.or}{0}
\verb{interior}{ô}{}{}{}{}{Que se passa no âmago, no mais íntimo da alma.}{in.te.ri.or}{0}
\verb{interior}{ô}{}{}{}{s.m.}{Aquilo que está dentro; parte interna.}{in.te.ri.or}{0}
\verb{interior}{ô}{}{}{}{}{Seio, coração.}{in.te.ri.or}{0}
\verb{interior}{ô}{}{}{}{}{Índole, caráter.}{in.te.ri.or}{0}
\verb{interior}{ô}{}{}{}{}{Toda a região de um estado, com exclusão de sua capital.}{in.te.ri.or}{0}
\verb{interiorano}{}{}{}{}{adj.}{Relativo ao interior do país.}{in.te.ri.o.ra.no}{0}
\verb{interiorano}{}{}{}{}{}{Diz"-se daquele que é natural ou habitante do interior.}{in.te.ri.o.ra.no}{0}
\verb{interiorano}{}{}{}{}{s.m.}{Indivíduo que nasceu ou vive no interior.}{in.te.ri.o.ra.no}{0}
\verb{interiorizar}{}{}{}{}{v.t.}{Trazer para dentro de si; incorporar aquilo que é exterior ao seu mundo interior.}{in.te.ri.o.ri.zar}{\verboinum{1}}
\verb{interjeição}{}{}{"-ões}{}{s.f.}{Palavra com que se exprime um sentimento como dor, alegria, espanto etc.}{in.ter.jei.ção}{0}
\verb{interjetivo}{}{}{}{}{adj.}{Que tem valor de uma interjeição.}{in.ter.je.ti.vo}{0}
\verb{interjetivo}{}{}{}{}{}{Que é expresso por interjeição.}{in.ter.je.ti.vo}{0}
\verb{interligar}{}{}{}{}{v.t.}{Ligar duas ou mais coisas entre si.}{in.ter.li.gar}{\verboinum{5}}
\verb{interlocução}{}{}{"-ões}{}{s.f.}{Conversação entre duas ou mais pessoas.}{in.ter.lo.cu.ção}{0}
\verb{interlocução}{}{}{"-ões}{}{}{Interrupção do discurso pela fala de novos interlocutores.}{in.ter.lo.cu.ção}{0}
\verb{interlocutor}{ô}{}{}{}{s.m.}{Cada uma das pessoas que participam de uma conversa, de um diálogo.}{in.ter.lo.cu.tor}{0}
\verb{interlocutor}{ô}{}{}{}{}{Indivíduo que fala em nome do outro.}{in.ter.lo.cu.tor}{0}
\verb{interlúdio}{}{}{}{}{s.m.}{Composição instrumental com a função de separar partes musicais, litúrgicas ou cênicas.}{in.ter.lú.dio}{0}
\verb{interlúdio}{}{}{}{}{}{Intervalo, pausa.}{in.ter.lú.dio}{0}
\verb{intermediação}{}{}{"-ões}{}{s.f.}{Ato ou efeito de intermediar; intervenção,   intercessão.}{in.ter.me.di.a.ção}{0}
\verb{intermediar}{}{}{}{}{v.t.}{Existir de permeio; situar"-se entre.}{in.ter.me.di.ar}{0}
\verb{intermediar}{}{}{}{}{}{Pôr de permeio; entremear, intercalar.}{in.ter.me.di.ar}{0}
\verb{intermediar}{}{}{}{}{}{Interceder, intervir.}{in.ter.me.di.ar}{\verboinum{6}}
\verb{intermediário}{}{}{}{}{adj.}{Que está de permeio; interposto.}{in.ter.me.di.á.rio}{0}
\verb{intermediário}{}{}{}{}{s.m.}{Pessoa que medeia entre duas ou mais pessoas ou coisas.}{in.ter.me.di.á.rio}{0}
\verb{intermediário}{}{}{}{}{}{Agente de negócios; corretor.}{in.ter.me.di.á.rio}{0}
\verb{intermediário}{}{}{}{}{}{Negociante que exerce suas atividades colocando"-se entre o produtor e o consumidor.}{in.ter.me.di.á.rio}{0}
\verb{intermédio}{}{}{}{}{adj.}{Que está de permeio; interposto.}{in.ter.mé.dio}{0}
\verb{intermédio}{}{}{}{}{s.m.}{Indivíduo que medeia ou intervém.}{in.ter.mé.dio}{0}
\verb{intermédio}{}{}{}{}{}{Pequena cena dramática que se apresenta no intervalo dos atos de uma peça de teatro.}{in.ter.mé.dio}{0}
\verb{intermédio}{}{}{}{}{}{Na música instrumental, interlúdio.}{in.ter.mé.dio}{0}
\verb{interminável}{}{}{"-eis}{}{adj.2g.}{Que não se pode terminar.}{in.ter.mi.ná.vel}{0}
\verb{interminável}{}{}{"-eis}{}{}{Que é demorado, prolongado.}{in.ter.mi.ná.vel}{0}
\verb{interministerial}{}{}{}{}{adj.2g.}{Que se realiza entre ministérios ou ministros.}{in.ter.mi.nis.te.ri.al}{0}
\verb{intermitência}{}{}{}{}{s.f.}{Qualidade do que é intermitente.}{in.ter.mi.tên.cia}{0}
\verb{intermitência}{}{}{}{}{}{Interrupção temporária; intervalo.}{in.ter.mi.tên.cia}{0}
\verb{intermitente}{}{}{}{}{adj.2g.}{Que apresenta interrupções ou suspensões; descontínuo.}{in.ter.mi.ten.te}{0}
\verb{intermunicipal}{}{}{}{}{adj.2g.}{Que se efetua entre dois ou mais municípios.}{in.ter.mu.ni.ci.pal}{0}
\verb{internação}{}{}{"-ões}{}{s.f.}{Ato ou efeito de internar.}{in.ter.na.ção}{0}
\verb{internacional}{}{}{"-ais}{}{adj.2g.}{Relativo às relações entre nações.}{in.ter.na.ci.o.nal}{0}
\verb{internacional}{}{}{"-ais}{}{}{Que se realiza entre nações.}{in.ter.na.ci.o.nal}{0}
\verb{internacionalismo}{}{}{}{}{s.m.}{Sistema de política internacional.}{in.ter.na.ci.o.na.lis.mo}{0}
\verb{internacionalismo}{}{}{}{}{}{Princípio segundo o qual deve existir uma aliança internacional dos trabalhadores, independentemente das fronteiras nacionais.}{in.ter.na.ci.o.na.lis.mo}{0}
\verb{internacionalizar}{}{}{}{}{v.t.}{Tornar internacional.}{in.ter.na.ci.o.na.li.zar}{\verboinum{1}}
\verb{internacionalizar}{}{}{}{}{}{Difundir em várias nações.}{in.ter.na.ci.o.na.li.zar}{0}
\verb{internado}{}{}{}{}{adj.}{Diz"-se do indivíduo que foi colocado em asilo, escola ou casa de saúde.}{in.ter.na.do}{0}
\verb{internado}{}{}{}{}{}{Que foi confinado em alguma localidade do país, de onde não pode sair. }{in.ter.na.do}{0}
\verb{internar}{}{}{}{}{v.t.}{Pôr ou colocar dentro; introduzir.}{in.ter.nar}{0}
\verb{internar}{}{}{}{}{}{Pôr em colégio, asilo, hospital etc.}{in.ter.nar}{0}
\verb{internar}{}{}{}{}{}{Obrigar ou forçar a residir no interior de um país, com a proibição de sair dali.}{in.ter.nar}{0}
\verb{internar}{}{}{}{}{v.pron.}{Meter"-se, introduzir"-se.}{in.ter.nar}{\verboinum{1}}
\verb{internato}{}{}{}{}{s.m.}{Escola ou instituição de assistência onde os alunos residem, fazem as refeições e recebem educação e instrução.}{in.ter.na.to}{0}
\verb{internato}{}{}{}{}{}{O conjunto dos alunos internos.}{in.ter.na.to}{0}
\verb{internauta}{}{Informát.}{}{}{s.2g.}{Usuário da \textit{internet}.}{in.ter.nau.ta}{0}
\verb{internet}{}{Informát.}{}{}{s.f.}{Rede de alcance mundial que une computadores os quais se comunicam por meio de um protocolo de comunicação comum, permitindo o fluxo de dados, mensagens e informações diversas.}{\textit{internet}}{0}
\verb{interno}{é}{}{}{}{adj.}{Que está dentro.}{in.ter.no}{0}
\verb{interno}{é}{}{}{}{}{Relativo ao lado interior de algo.}{in.ter.no}{0}
\verb{interno}{é}{}{}{}{}{Diz"-se de aluno que mora no colégio em que estuda.}{in.ter.no}{0}
\verb{interno}{é}{}{}{}{}{Diz"-se do uso de certos medicamentos, que podem ser ingeridos.}{in.ter.no}{0}
\verb{interno}{é}{}{}{}{s.m.}{Indivíduo recluso em uma casa de recuperação.}{in.ter.no}{0}
\verb{interno}{é}{}{}{}{}{Estudante de medicina que já presta serviços em um hospital.}{in.ter.no}{0}
\verb{interoceânico}{}{}{}{}{adj.}{Que está entre ou liga dois oceanos.}{in.te.ro.ce.â.ni.co}{0}
\verb{interpartidário}{}{}{}{}{adj.}{Que envolve dois ou mais partidos.}{in.ter.par.ti.dá.rio}{0}
\verb{interpelação}{}{}{"-ões}{}{s.f.}{Ato ou efeito de interpelar.}{in.ter.pe.la.ção}{0}
\verb{interpelante}{}{}{}{}{adj.2g.}{Que interpela.}{in.ter.pe.lan.te}{0}
\verb{interpelar}{}{}{}{}{v.t.}{Dirigir a palavra para perguntar algo.}{in.ter.pe.lar}{0}
\verb{interpelar}{}{}{}{}{}{Dirigir a palavra para pedir explicações, esclarecimentos ou fazer cobranças.}{in.ter.pe.lar}{\verboinum{1}}
\verb{interpenetração}{}{}{"-ões}{}{s.f.}{Ato ou efeito de interpenetrar"-se.}{in.ter.pe.ne.tra.ção}{0}
\verb{interpenetrar"-se}{}{}{}{}{v.pron.}{Penetrar"-se reciprocamente.}{in.ter.pe.ne.trar"-se}{\verboinum{1}}
\verb{interplanetário}{}{}{}{}{adj.}{Que existe entre planetas.}{in.ter.pla.ne.tá.rio}{0}
\verb{interplanetário}{}{}{}{}{}{Que se dá entre planetas.}{in.ter.pla.ne.tá.rio}{0}
\verb{interpolação}{}{}{"-ões}{}{s.f.}{Ato ou efeito de interpolar.}{in.ter.po.la.ção}{0}
\verb{interpolação}{}{}{"-ões}{}{}{Trecho interpolado em uma obra.}{in.ter.po.la.ção}{0}
\verb{interpolar}{}{}{}{}{v.t.}{Inserir palavras ou frases em um texto para completá"-lo, alterá"-lo ou esclarecê"-lo.}{in.ter.po.lar}{0}
\verb{interpolar}{}{}{}{}{adj.2g.}{Situado entre dois polos.}{in.ter.po.lar}{0}
\verb{interpolar}{}{}{}{}{}{Pôr de permeio; alternar.}{in.ter.po.lar}{\verboinum{1}}
\verb{interpor}{}{}{}{}{v.t.}{Pôr de permeio.}{in.ter.por}{0}
\verb{interpor}{}{}{}{}{}{Opor, contrapor.}{in.ter.por}{0}
\verb{interpor}{}{}{}{}{}{Fazer intervir.}{in.ter.por}{0}
\verb{interpor}{}{Jur.}{}{}{}{Colocar em juízo.}{in.ter.por}{\verboinum{60}}
\verb{interposição}{}{Jur.}{"-ões}{}{s.f.}{Ato de interpor.}{in.ter.po.si.ção}{0}
\verb{interposição}{}{Fig.}{"-ões}{}{}{Intervenção, interrupção.}{in.ter.po.si.ção}{0}
\verb{interposição}{}{}{"-ões}{}{}{Posição entre duas coisas.}{in.ter.po.si.ção}{0}
\verb{interposto}{ô}{}{"-s ⟨ó⟩}{"-a ⟨ó⟩}{adj.}{Que se interpôs.}{in.ter.pos.to}{0}
\verb{interposto}{ô}{}{"-s ⟨ó⟩}{"-a ⟨ó⟩}{s.m.}{Depósito de mercadorias; entreposto.}{in.ter.pos.to}{0}
\verb{interpretação}{}{}{"-ões}{}{s.f.}{Ato ou efeito de interpretar.}{in.ter.pre.ta.ção}{0}
\verb{interpretação}{}{}{"-ões}{}{}{Comentário, julgamento, explicação.}{in.ter.pre.ta.ção}{0}
\verb{interpretação}{}{}{"-ões}{}{}{Técnica de representação em teatro, cinema, televisão.}{in.ter.pre.ta.ção}{0}
\verb{interpretar}{}{}{}{}{v.t.}{Determinar o significado; dar sentido.}{in.ter.pre.tar}{0}
\verb{interpretar}{}{}{}{}{}{Avaliar a intenção de gesto ou palavra; considerar, julgar. (\textit{Não me interprete mal.})}{in.ter.pre.tar}{0}
\verb{interpretar}{}{}{}{}{}{Representar papel no teatro.}{in.ter.pre.tar}{0}
\verb{interpretar}{}{}{}{}{}{Cantar ou executar obra musical.}{in.ter.pre.tar}{\verboinum{1}}
\verb{interpretativo}{}{}{}{}{adj.}{Relativo a interpretação.}{in.ter.pre.ta.ti.vo}{0}
\verb{interpretativo}{}{}{}{}{}{Que fornece uma interpretação; explicativo.}{in.ter.pre.ta.ti.vo}{0}
\verb{interpretativo}{}{}{}{}{}{Que contém elementos da natureza da interpretação.}{in.ter.pre.ta.ti.vo}{0}
\verb{intérprete}{}{}{}{}{s.2g.}{Indivíduo que interpreta.}{in.tér.pre.te}{0}
\verb{intérprete}{}{}{}{}{}{Indivíduo que serve de intermediário entre pessoas de língua diferentes fazendo sua tradução, auxiliando na comunicação entre eles.}{in.tér.pre.te}{0}
\verb{intérprete}{}{}{}{}{}{Comentarista, exegeta.}{in.tér.pre.te}{0}
\verb{inter"-racial}{}{}{"-ais}{}{adj.2g.}{Que se dá entre raças.}{in.ter"-ra.ci.al}{0}
\verb{interregno}{é}{}{}{}{s.m.}{Tempo entre dois reinados.}{in.ter.reg.no}{0}
\verb{interregno}{é}{}{}{}{}{Intervalo, interrupção.}{in.ter.reg.no}{0}
\verb{inter"-relação}{}{}{"-ões}{}{s.f.}{Relação mútua.}{in.ter"-re.la.ção}{0}
\verb{inter"-resistente}{}{}{}{}{adj.2g.}{Diz"-se de alavanca que tem a resistência entre a potência e o apoio.}{in.ter"-re.sis.ten.te}{0}
\verb{interrogação}{}{}{"-ões}{}{s.f.}{Ato de interrogar.}{in.ter.ro.ga.ção}{0}
\verb{interrogação}{}{}{"-ões}{}{}{Pergunta.}{in.ter.ro.ga.ção}{0}
\verb{interrogação}{}{}{"-ões}{}{}{Dúvida, incerteza.}{in.ter.ro.ga.ção}{0}
\verb{interrogação}{}{Gram.}{"-ões}{}{}{O sinal gráfico que marca uma frase interrogativa; ponto de interrogação.}{in.ter.ro.ga.ção}{0}
\verb{interrogar}{}{}{}{}{v.t.}{Fazer perguntas; indagar.}{in.ter.ro.gar}{0}
\verb{interrogar}{}{}{}{}{}{Consultar.}{in.ter.ro.gar}{0}
\verb{interrogar}{}{}{}{}{}{Propor questões.}{in.ter.ro.gar}{0}
\verb{interrogar}{}{Jur.}{}{}{}{Fazer interrogatório.}{in.ter.ro.gar}{\verboinum{5}}
\verb{interrogativo}{}{}{}{}{adj.}{Que denota interrogação.}{in.ter.ro.ga.ti.vo}{0}
\verb{interrogativo}{}{Gram.}{}{}{}{Que serve para ou contém interrogação.}{in.ter.ro.ga.ti.vo}{0}
\verb{interrogatório}{}{}{}{}{s.m.}{Ato ou efeito de interrogar.}{in.ter.ro.ga.tó.rio}{0}
\verb{interrogatório}{}{Jur.}{}{}{}{Arguição feita com o réu sobre questões relacionadas à acusação.}{in.ter.ro.ga.tó.rio}{0}
\verb{interrogatório}{}{}{}{}{adj.}{Que denota interrogação; interrogativo.}{in.ter.ro.ga.tó.rio}{0}
\verb{interromper}{ê}{}{}{}{v.t.}{Fazer parar; suspender a continuidade.}{in.ter.rom.per}{0}
\verb{interromper}{ê}{}{}{}{}{Obstar, estorvar.}{in.ter.rom.per}{0}
\verb{interromper}{ê}{}{}{}{}{Deixar de fazer.}{in.ter.rom.per}{0}
\verb{interromper}{ê}{}{}{}{}{Cortar.}{in.ter.rom.per}{\verboinum{12}}
\verb{interrupção}{}{}{"-ões}{}{s.f.}{Ato ou efeito de interromper.}{in.ter.rup.ção}{0}
\verb{interrupção}{}{}{"-ões}{}{}{Ato de interromper a fala de alguém, ou aquilo que provoca isso.}{in.ter.rup.ção}{0}
\verb{interrupção}{}{}{"-ões}{}{}{Suspensão de um enunciado.}{in.ter.rup.ção}{0}
\verb{interrupto}{}{}{}{}{adj.}{Interrompido.}{in.ter.rup.to}{0}
\verb{interruptor}{ô}{}{}{}{adj.}{Que interrompe.}{in.ter.rup.tor}{0}
\verb{interruptor}{ô}{}{}{}{s.m.}{Dispositivo que interrompe e reestabelece a passagem de energia elétrica ou sinal em um circuito.}{in.ter.rup.tor}{0}
\verb{interseção}{}{}{}{}{}{Var. de \textit{intersecção}.}{in.ter.se.ção}{0}
\verb{intersecção}{}{}{"-ões}{}{s.f.}{Ato de cortar; corte.}{in.ter.sec.ção}{0}
\verb{intersecção}{}{Mat.}{"-ões}{}{}{O subconjunto dos elementos que pertencem simultaneamente a dois ou mais conjuntos.}{in.ter.sec.ção}{0}
\verb{intersindical}{}{}{"-ais}{}{adj.2g.}{Comum a dois ou mais sindicatos.}{in.ter.sin.di.cal}{0}
\verb{intersindical}{}{}{"-ais}{}{}{Que acontece entre sindicatos.}{in.ter.sin.di.cal}{0}
\verb{intersticial}{}{}{"-ais}{}{adj.2g.}{Relativo a interstício.}{in.ters.ti.ci.al}{0}
\verb{interstício}{}{}{}{}{s.m.}{Pequeno intervalo entre as partes de um todo.}{in.ters.tí.cio}{0}
\verb{interstício}{}{Anat.}{}{}{}{Intervalo entre dois órgãos contíguos.}{in.ters.tí.cio}{0}
\verb{interstício}{}{}{}{}{}{Fenda, fresta.}{in.ters.tí.cio}{0}
\verb{intertropical}{}{Geogr.}{"-ais}{}{adj.2g.}{Situado entre os trópicos.}{in.ter.tro.pi.cal}{0}
\verb{interurbano}{}{}{}{}{adj.}{Que se dá entre cidades.}{in.te.rur.ba.no}{0}
\verb{interurbano}{}{}{}{}{}{Que envolve duas ou mais cidades.}{in.te.rur.ba.no}{0}
\verb{interurbano}{}{}{}{}{s.m.}{Comunicação telefônica entre cidades.}{in.te.rur.ba.no}{0}
\verb{intervalar}{}{}{}{}{adj.2g.}{Situado em um intervalo.}{in.ter.va.lar}{0}
\verb{intervalar}{}{}{}{}{}{Relativo a intervalo.}{in.ter.va.lar}{0}
\verb{intervalar}{}{}{}{}{v.t.}{Dispor com intervalos; alternar, entremear. }{in.ter.va.lar}{\verboinum{1}}
\verb{intervalo}{}{}{}{}{s.m.}{Espaço de tempo entre dois eventos ou épocas.}{in.ter.va.lo}{0}
\verb{intervalo}{}{}{}{}{}{Distância entre dois pontos.}{in.ter.va.lo}{0}
\verb{intervenção}{}{}{"-ões}{}{s.f.}{Ato ou efeito de intervir.}{in.ter.ven.ção}{0}
\verb{intervenção}{}{Med.}{"-ões}{}{}{Cirurgia.}{in.ter.ven.ção}{0}
\verb{intervenção}{}{}{"-ões}{}{}{Administração de uma empresa por um delegado em caso de irregularidades.}{in.ter.ven.ção}{0}
\verb{intervenção}{}{}{"-ões}{}{}{Ação direta do governo federal em um estado.}{in.ter.ven.ção}{0}
\verb{intervencionismo}{}{}{}{}{s.m.}{Doutrina que defende a interferência do Estado na economia do país, ou de um país nos assuntos de outros países.}{in.ter.ven.ci.o.nis.mo}{0}
\verb{intervencionismo}{}{}{}{}{}{A prática dessa doutrina.}{in.ter.ven.ci.o.nis.mo}{0}
\verb{interveniente}{}{}{}{}{adj.2g.}{Que intervém; interventor.}{in.ter.ve.ni.en.te}{0}
\verb{interventor}{ô}{}{}{}{adj.}{Que intervém; interveniente.}{in.ter.ven.tor}{0}
\verb{interventor}{ô}{}{}{}{s.m.}{Delegado escolhido pelo Presidente da República para assumir o governo de um estado que se encontra sob intervenção.}{in.ter.ven.tor}{0}
\verb{intervir}{}{}{}{}{v.t.}{Tomar parte voluntariamente; intrometer"-se.}{in.ter.vir}{0}
\verb{intervir}{}{}{}{}{}{Interpor a sua própria autoridade, iniciativa, competência.}{in.ter.vir}{0}
\verb{intervir}{}{}{}{}{}{Pôr"-se como árbitro; mediar, assistir.}{in.ter.vir}{0}
\verb{intervir}{}{}{}{}{v.i.}{Ocorrer incidentalmente; sobrevir.}{in.ter.vir}{\verboinum{56}}
\verb{intervocálico}{}{Gram.}{}{}{adj.}{Que se encontra posicionado entre vogais.}{in.ter.vo.cá.li.co}{0}
\verb{intestinal}{}{}{"-ais}{}{adj.2g.}{Relativo ao intestino.}{in.tes.ti.nal}{0}
\verb{intestino}{}{Anat.}{}{}{s.m.}{Porção final do tubo digestivo que se estende do estômago ao ânus.}{in.tes.ti.no}{0}
\verb{intestino}{}{}{}{}{adj.}{Íntimo, interno.}{in.tes.ti.no}{0}
\verb{intestino}{}{}{}{}{}{Nacional, civil.}{in.tes.ti.no}{0}
\verb{intimação}{}{}{"-ões}{}{s.f.}{Ato ou efeito de intimar.}{in.ti.ma.ção}{0}
\verb{intimação}{}{Jur.}{"-ões}{}{}{Ciência dada a alguém acerca de ato judicial; notificação.}{in.ti.ma.ção}{0}
\verb{intimar}{}{Jur.}{}{}{v.t.}{Fazer notificação judicial.}{in.ti.mar}{0}
\verb{intimar}{}{}{}{}{}{Determinar com autoridade; ordenar, impor.}{in.ti.mar}{0}
\verb{intimar}{}{}{}{}{v.i.}{Falar com arrogância.}{in.ti.mar}{\verboinum{1}}
\verb{intimativa}{}{}{}{}{s.f.}{Afirmação enérgica.}{in.ti.ma.ti.va}{0}
\verb{intimativa}{}{}{}{}{}{Arrogância, energia.}{in.ti.ma.ti.va}{0}
\verb{intimativo}{}{}{}{}{adj.}{Próprio de intimação; ordenativo, severo, enérgico.}{in.ti.ma.ti.vo}{0}
\verb{intimidação}{}{}{"-ões}{}{s.f.}{Ato ou efeito de intimidar.}{in.ti.mi.da.ção}{0}
\verb{intimidade}{}{}{}{}{}{Relação próxima; familiaridade.}{in.ti.mi.da.de}{0}
\verb{intimidade}{}{}{}{}{s.f.}{Qualidade de íntimo.}{in.ti.mi.da.de}{0}
\verb{intimidade}{}{}{}{}{}{Vida particular, doméstica; privacidade, aconchego.}{in.ti.mi.da.de}{0}
\verb{intimidar}{}{}{}{}{v.t.}{Tornar receoso; causar medo; assustar, apavorar.}{in.ti.mi.dar}{0}
\verb{intimidar}{}{}{}{}{v.pron.}{Acanhar"-se; tornar"-se tímido.}{in.ti.mi.dar}{\verboinum{1}}
\verb{íntimo}{}{}{}{}{adj.}{Que está muito dentro.}{ín.ti.mo}{0}
\verb{íntimo}{}{}{}{}{}{Que se passa no interior da mente ou do espírito.}{ín.ti.mo}{0}
\verb{íntimo}{}{}{}{}{}{Que tem ligação próxima, com afeição e confiança.}{ín.ti.mo}{0}
\verb{íntimo}{}{}{}{}{s.m.}{O fundo da alma.}{ín.ti.mo}{0}
\verb{intimorato}{}{}{}{}{adj.}{Sem temor; intrépido, destemido.}{in.ti.mo.ra.to}{0}
\verb{intitulação}{}{}{"-ões}{}{s.f.}{Ato ou efeito de intitular.}{in.ti.tu.la.ção}{0}
\verb{intitular}{}{}{}{}{v.t.}{Designar um título.}{in.ti.tu.lar}{0}
\verb{intitular}{}{}{}{}{v.pron.}{Chamar"-se, denominar"-se.}{in.ti.tu.lar}{\verboinum{1}}
\verb{intocável}{}{}{"-eis}{}{adj.2g.}{Que não pode ser tocado; intangível.}{in.to.cá.vel}{0}
\verb{intocável}{}{}{"-eis}{}{}{Que não pode ser atacado ou incriminado.}{in.to.cá.vel}{0}
\verb{intocável}{}{}{"-eis}{}{}{Que não pode ser objeto de crítica, geralmente por seu prestígio.}{in.to.cá.vel}{0}
\verb{intolerância}{}{}{}{}{s.f.}{Falta de tolerância; intransigência.}{in.to.le.rân.cia}{0}
\verb{intolerante}{}{}{}{}{adj.2g.}{Que não é tolerante; inflexível, rígido.}{in.to.le.ran.te}{0}
\verb{intolerável}{}{}{"-eis}{}{adj.2g.}{Que não pode ser tolerado; insuportável.}{in.to.le.rá.vel}{0}
\verb{intoxicação}{cs}{}{"-ões}{}{s.f.}{Ato ou efeito de intoxicar; envenenamento.}{in.to.xi.ca.ção}{0}
\verb{intoxicar}{cs}{}{}{}{v.t.}{Envenenar pela ação de substância tóxica.}{in.to.xi.car}{\verboinum{2}}
\verb{intraduzível}{}{}{"-eis}{}{adj.2g.}{Que não se pode traduzir ou verter.}{in.tra.du.zí.vel}{0}
\verb{intraduzível}{}{}{"-eis}{}{}{Inexprimível, indizível.}{in.tra.du.zí.vel}{0}
\verb{intragável}{}{}{"-eis}{}{adj.2g.}{Que não se pode tragar.}{in.tra.gá.vel}{0}
\verb{intragável}{}{}{"-eis}{}{}{Insuportável, incômodo.}{in.tra.gá.vel}{0}
\verb{intramuscular}{}{Anat.}{}{}{adj.2g.}{Relativo ao interior dos músculos.}{in.tra.mus.cu.lar}{0}
\verb{intramuscular}{}{Med.}{}{}{}{Diz"-se de injeção que se aplica no interior de um músculo.}{in.tra.mus.cu.lar}{0}
\verb{intranet}{}{Informát.}{}{}{s.f.}{Rede local de computadores que utiliza os mesmos protocolos de comunicação empregados na internet, mas sem usar a sua infraestrutura física.}{\textit{intranet}}{0}
\verb{intranquilidade}{}{}{}{}{s.f.}{Falta de tranquilidade; desassossego.}{in.tran.qui.li.da.de}{0}
\verb{intranquilizar}{}{}{}{}{v.t.}{Tornar intranquilo; inquietar, desassossegar.}{in.tran.qui.li.zar}{\verboinum{1}}
\verb{intranquilo}{}{}{}{}{adj.}{Que não está tranquilo; preocupado, aflito, inquieto.}{in.tran.qui.lo}{0}
\verb{intransferível}{}{}{"-eis}{}{adj.2g.}{Que não pode ser transferido; inalienável, intransmissível.}{in.trans.fe.rí.vel}{0}
\verb{intransigência}{}{}{}{}{s.f.}{Falta de transigência, de condescendência; intolerância.}{in.tran.si.gên.cia}{0}
\verb{intransigência}{}{}{}{}{}{Rigidez, austeridade, inflexibilidade.}{in.tran.si.gên.cia}{0}
\verb{intransigente}{}{}{}{}{adj.2g.}{Que não transige, não cede; intolerante.}{in.tran.si.gen.te}{0}
\verb{intransigente}{}{}{}{}{}{Austero, rígido, inflexível.}{in.tran.si.gen.te}{0}
\verb{intransitável}{}{}{"-eis}{}{adj.2g.}{Por onde não se consegue transitar; proibido ao trânsito.}{in.tran.si.tá.vel}{0}
\verb{intransitável}{}{Fig.}{"-eis}{}{}{Que não se pode alcançar ou atingir; inacessível, impraticável.}{in.tran.si.tá.vel}{0}
\verb{intransitivo}{}{}{}{}{adj.}{Que não se pode transmitir; intransmissível.}{in.tran.si.ti.vo}{0}
\verb{intransitivo}{}{Gram.}{}{}{}{Diz"-se do verbo que não aceita complemento verbal, devido a sua significação ou a alguma de suas acepções. }{in.tran.si.ti.vo}{0}
\verb{intransmissível}{}{}{"-eis}{}{adj.2g.}{Que não pode ser passado a outrem; intransitivo.}{in.trans.mis.sí.vel}{0}
\verb{intransponível}{}{}{"-eis}{}{adj.2g.}{Que não se pode transpor, ultrapassar; insuperável.}{in.trans.po.ní.vel}{0}
\verb{intraocular}{}{Anat.}{intraoculares}{}{adj.2g.}{Que está situado dentro do olho.}{in.tra.o.cu.lar}{0}
\verb{intraocular}{}{}{intraoculares}{}{}{Que se aplica dentro do olho.}{in.tra.o.cu.lar}{0}
\verb{intrapulmonar}{}{Anat.}{}{}{adj.2g.}{Que se localiza ou ocorre dentro dos pulmões.}{in.tra.pul.mo.nar}{0}
\verb{intratável}{}{}{"-eis}{}{adj.2g.}{Que não se pode tratar. }{in.tra.tá.vel}{0}
\verb{intratável}{}{}{"-eis}{}{}{Com quem não se pode tratar; insociável, grosseiro, rude.}{in.tra.tá.vel}{0}
\verb{intra"-uterino}{}{}{}{}{adj.}{Relativo ao interior do útero.}{in.tra"-u.te.ri.no}{0}
\verb{intra"-uterino}{}{}{}{}{}{Que se localiza ou ocorre dentro do útero.}{in.tra"-u.te.ri.no}{0}
\verb{intravenoso}{ô}{Anat.}{"-osos ⟨ó⟩}{"-osa ⟨ó⟩}{adj.}{Relativo ao interior da veia.}{in.tra.ve.no.so}{0}
\verb{intravenoso}{ô}{}{"-osos ⟨ó⟩}{"-osa ⟨ó⟩}{}{Que se aplica ou se introduz na veia.}{in.tra.ve.no.so}{0}
\verb{intrepidez}{ê}{}{}{}{s.f.}{Qualidade de intrépido; ousadia, destemor, coragem.}{in.tre.pi.dez}{0}
\verb{intrépido}{}{}{}{}{adj.}{Diz"-se do indivíduo que não receia; destemido, ousado, corajoso.}{in.tré.pi.do}{0}
\verb{intricado}{}{}{}{}{}{Var. de \textit{intrincado}.}{in.tri.ca.do}{0}
\verb{intricar}{}{}{}{}{}{Var. de \textit{intrincar}.}{in.tri.car}{0}
\verb{intriga}{}{}{}{}{s.f.}{Maquinação para se obter vantagem ou prejudicar alguém; bisbilhotice, mexerico.}{in.tri.ga}{0}
\verb{intriga}{}{}{}{}{}{Cilada, insídia, perfídia.}{in.tri.ga}{0}
\verb{intrigado}{}{}{}{}{adj.}{Em que há intriga; insidioso.}{in.tri.ga.do}{0}
\verb{intrigado}{}{}{}{}{}{Que revela curiosidade; desconfiado.}{in.tri.ga.do}{0}
\verb{intrigante}{}{}{}{}{adj.2g.}{Diz"-se daquele que faz intriga; mexeriqueiro.}{in.tri.gan.te}{0}
\verb{intrigante}{}{}{}{}{}{Que desperta a curiosidade; surpreendente.}{in.tri.gan.te}{0}
\verb{intrigar}{}{}{}{}{v.t.}{Despertar a curiosidade; surpreender.}{in.tri.gar}{0}
\verb{intrigar}{}{}{}{}{}{Envolver em mexericos; criar inimizades.}{in.tri.gar}{\verboinum{5}}
\verb{intrincado}{}{}{}{}{adj.}{Difícil de entender ou resolver; confuso, obscuro, complicado.}{in.trin.ca.do}{0}
\verb{intrincado}{}{}{}{}{}{Entrelaçado, emaranhado, embaraçado.}{in.trin.ca.do}{0}
\verb{intrincar}{}{}{}{}{}{Tornar emaranhado; embaraçar, atrapalhar.}{in.trin.car}{\verboinum{2}}
\verb{intrincar}{}{}{}{}{v.t.}{Tornar obscuro; complicar, confundir.}{in.trin.car}{0}
\verb{intrínseco}{}{}{}{}{adj.}{Que faz parte da essência; próprio, inerente.}{in.trín.se.co}{0}
\verb{intrínseco}{}{}{}{}{}{Que existe por si mesmo; que é real.}{in.trín.se.co}{0}
\verb{introdução}{}{}{"-ões}{}{s.f.}{Ato ou efeito de introduzir; admissão.}{in.tro.du.ção}{0}
\verb{introdução}{}{}{"-ões}{}{}{Parte inicial de um livro, em que se expõem os argumentos e objetivos da obra.}{in.tro.du.ção}{0}
\verb{introdução}{}{Mús.}{"-ões}{}{}{Pequeno trecho que inicia uma peça musical.}{in.tro.du.ção}{0}
\verb{introdutivo}{}{}{}{}{adj.}{Introdutório.}{in.tro.du.ti.vo}{0}
\verb{introdutor}{ô}{}{}{}{adj.}{Que introduz, insere.}{in.tro.du.tor}{0}
\verb{introdutório}{}{}{}{}{adj.}{Relativo a introdução ou a começo; introdutivo.}{in.tro.du.tó.rio}{0}
\verb{introduzir}{}{}{}{}{v.t.}{Levar para dentro; fazer entrar.}{in.tro.du.zir}{0}
\verb{introduzir}{}{}{}{}{}{Fazer alguma novidade ser adotada por um grupo.}{in.tro.du.zir}{\verboinum{21}}
\verb{intróito}{}{}{}{}{s.m.}{Parte inicial; entrada, princípio.}{in.trói.to}{0}
\verb{intróito}{}{Relig.}{}{}{s.m.}{Na liturgia da missa católica, oração que inicia o ofício.}{in.trói.to}{0}
\verb{intrometer}{ê}{}{}{}{v.t.}{Introduzir um corpo no interior de outro.}{in.tro.me.ter}{0}
\verb{intrometer}{ê}{}{}{}{v.pron.}{Meter"-se com as coisas dos outros; dar opinião sobre algo que não lhe diz respeito; imiscuir"-se.}{in.tro.me.ter}{\verboinum{12}}
\verb{intrometido}{}{}{}{}{adj.}{Introduzido; inserido dentro de algo.}{in.tro.me.ti.do}{0}
\verb{intrometido}{}{}{}{}{}{Diz"-se daquele que se mete no que não lhe diz respeito; atrevido.}{in.tro.me.ti.do}{0}
\verb{intrometimento}{}{}{}{}{s.m.}{Intromissão.}{in.tro.me.ti.men.to}{0}
\verb{intromissão}{}{}{"-ões}{}{s.f.}{Ato ou efeito de intrometer; intrometimento.}{in.tro.mis.são}{0}
\verb{intromissão}{}{}{"-ões}{}{}{Introdução de um corpo no interior de outro.}{in.tro.mis.são}{0}
\verb{introspeção}{}{}{}{}{}{Var. de \textit{instrospecção}.}{in.tros.pe.ção}{0}
\verb{introspecção}{}{}{"-ões}{}{s.f.}{Observação ou exame que a pessoa faz sobre o que ocorre no seu íntimo, seus pensamentos e sentimentos.}{in.tros.pec.ção}{0}
\verb{introspectivo}{}{}{}{}{adj.}{Em que se examinam os próprios sentimentos e pensamentos.}{in.tros.pec.ti.vo}{0}
\verb{introversão}{}{}{"-ões }{}{s.f.}{Ação de se voltar para dentro de si, de ensimesmar"-se.}{in.tro.ver.são}{0}
\verb{introverter}{ê}{}{}{}{v.pron.}{Voltar"-se para dentro de si; recolher"-se; ensimesmar"-se.}{in.tro.ver.ter"-se}{\verboinum{12}}
\verb{introvertido}{}{}{}{}{adj.}{Voltado para dentro de si mesmo; absorto, ensimesmado.}{in.tro.ver.ti.do}{0}
\verb{intrujão}{}{}{"-ões}{}{s.m.}{Pessoa que intruja, ludibria; trapaceiro.}{in.tru.jão}{0}
\verb{intrujão}{}{Pop.}{"-ões}{}{}{Pessoa que recebe objetos furtados.}{in.tru.jão}{0}
\verb{intrujar}{}{}{}{}{v.t.}{Intrometer"-se com outras pessoas com o intuito de enganá"-las ou explorá"-las em proveito próprio; ludibriar.}{in.tru.jar}{\verboinum{1}}
\verb{intrujice}{}{}{}{}{s.f.}{Ato de intrujar; logro, trapaça.}{in.tru.ji.ce}{0}
\verb{intrusão}{}{}{"-ões}{}{s.f.}{Ato de se introduzir em uma sociedade ou de ocupar um cargo sem ter direito a tanto.}{in.tru.são}{0}
\verb{intrusão}{}{Geol.}{"-ões}{}{}{Penetração de material magmático no interior da crosta.}{in.tru.são}{0}
\verb{intruso}{}{}{}{}{adj.}{Diz"-se daquele que se introduz em algo que não lhe diz respeito; metido, intrometido.}{in.tru.so}{0}
\verb{intruso}{}{}{}{}{}{Diz"-se daquele que entra em festas ou reuniões sem ser convidado; penetra.}{in.tru.so}{0}
\verb{intuição}{}{}{"-ões}{}{s.f.}{Ato de discernir, conhecer a verdade de forma imediata e clara sem o auxílio do raciocínio.}{in.tu.i.ção}{0}
\verb{intuição}{}{}{"-ões}{}{}{Pressentimento, presságio.}{in.tu.i.ção}{0}
\verb{intuir}{}{}{}{}{v.t.}{Deduzir ou concluir algo por meio da intuição, sem o uso do raciocínio; pressentir.}{in.tu.ir}{\verboinum{26}}
\verb{intuitivo}{}{}{}{}{adj.}{Relativo a intuição.}{in.tu.i.ti.vo}{0}
\verb{intuitivo}{}{}{}{}{}{Diz"-se daquele que age por intuição.}{in.tu.i.ti.vo}{0}
\verb{intuito}{}{}{}{}{s.m.}{Objetivo que se tem em vista; propósito, intenção.}{in.tui.to}{0}
\verb{intumescência}{}{}{}{}{s.f.}{Estado de intumescer, de aumentar de volume; dilatação, inchaço, intumescimento.}{in.tu.mes.cên.cia}{0}
\verb{intumescente}{}{}{}{}{adj.2g.}{Que está em processo de intumescer; inchado, dilatado.}{in.tu.mes.cen.te}{0}
\verb{intumescer}{ê}{}{}{}{v.t.}{Aumentar de volume; inchar, dilatar.}{in.tu.mes.cer}{0}
\verb{intumescer}{ê}{}{}{}{}{Tornar orgulhoso; envaidecer.}{in.tu.mes.cer}{\verboinum{15}}
\verb{intumescimento}{}{}{}{}{s.m.}{Intumescência.}{in.tu.mes.ci.men.to}{0}
\verb{inturgescência}{}{}{}{}{s.f.}{Inchação, intumescência, turgidez.}{in.tur.ges.cên.cia}{0}
\verb{inturgescer}{ê}{}{}{}{v.t.}{Aumentar de volume; dilatar, inchar.}{in.tur.ges.cer}{\verboinum{15}}
\verb{inúbia}{}{Mús.}{}{}{s.f.}{Trombeta de guerra usada pelos índios, feita de dois pedaços de maçaranduba.}{i.nú.bia}{0}
\verb{inúbil}{}{}{"-eis}{}{adj.2g.}{Que ainda não é núbil; que ainda não está na idade de se casar.}{i.nú.bil}{0}
\verb{inumação}{}{}{"-ões}{}{s.f.}{Ato ou efeito de inumar; enterro, sepultamento.}{i.nu.ma.ção}{0}
\verb{inumano}{}{}{}{}{adj.}{Que não apresenta sentimentos de respeito, consideração, generosidade, próprios do ser humano.}{i.nu.ma.no}{0}
\verb{inumano}{}{}{}{}{}{Que não pertence ou parece não pertencer à condição humana.}{i.nu.ma.no}{0}
\verb{inumar}{}{}{}{}{v.t.}{Sepultar um corpo; enterrar.}{i.nu.mar}{\verboinum{1}}
\verb{inumerável}{}{}{"-eis}{}{adj.2g.}{Que não se pode contar ou numerar; incontável.}{i.nu.me.rá.vel}{0}
\verb{inumerável}{}{}{"-eis}{}{}{Muito numeroso; abundante, copioso, inúmero.}{i.nu.me.rá.vel}{0}
\verb{inúmero}{}{}{}{}{adj.}{Inumerável.}{i.nú.me.ro}{0}
\verb{inundação}{}{}{"-ões}{}{s.f.}{Ato ou efeito de inundar; alagamento, enchente.}{i.nun.da.ção}{0}
\verb{inundação}{}{Fig.}{"-ões}{}{}{Grande quantidade de objetos ou pessoas.}{i.nun.da.ção}{0}
\verb{inundar}{}{}{}{}{v.t.}{Cobrir com água, fazendo transbordar; submergir, alagar.}{i.nun.dar}{0}
\verb{inundar}{}{}{}{}{}{Banhar, molhar, umedecer.}{i.nun.dar}{0}
\verb{inundar}{}{Fig.}{}{}{}{Invadir em tumulto ou com agressão.}{i.nun.dar}{\verboinum{1}}
\verb{inusitado}{}{}{}{}{adj.}{Que não é usual ou que causa estranhamento; incomum, esquisito, insólito.}{i.nu.si.ta.do}{0}
\verb{inusual}{}{}{"-ais}{}{adj.2g.}{Que é pouco usual, incomum.  }{i.nu.su.al}{0}
\verb{inútil}{}{}{"-eis}{}{adj.2g.}{Que não tem utilidade; que não serve para nada; imprestável.}{i.nú.til}{0}
\verb{inútil}{}{}{"-eis}{}{}{Que não vale a pena; que é em vão; infrutífero.}{i.nú.til}{0}
\verb{inutilidade}{}{}{}{}{s.f.}{Falta de utilidade, de serventia.}{i.nu.ti.li.da.de}{0}
\verb{inutilidade}{}{}{}{}{}{Coisa ou pessoa sem valor, sem préstimo.}{i.nu.ti.li.da.de}{0}
\verb{inutilizar}{}{}{}{}{v.t.}{Tornar inútil ou imprestável.}{i.nu.ti.li.zar}{0}
\verb{inutilizar}{}{}{}{}{}{Tornar inválido; anular, frustrar.}{i.nu.ti.li.zar}{\verboinum{1}}
\verb{invadir}{}{}{}{}{v.t.}{Entrar violentamente em um lugar e ocupá"-lo pela força; tomar, conquistar.}{in.va.dir}{0}
\verb{invadir}{}{}{}{}{}{Difundir, alastrar, espalhar.}{in.va.dir}{0}
\verb{invadir}{}{}{}{}{}{Avassalar, usurpar, devastar.}{in.va.dir}{\verboinum{18}}
\verb{invalidação}{}{}{"-ões}{}{s.f.}{Ato ou efeito de invalidar; anulação.}{in.va.li.da.ção}{0}
\verb{invalidar}{}{}{}{}{v.t.}{Tornar ou declarar inválido; anular.}{in.va.li.dar}{0}
\verb{invalidar}{}{}{}{}{}{Tirar a credibilidade ou a importância.}{in.va.li.dar}{0}
\verb{invalidar}{}{}{}{}{}{Tornar incapaz ou inapto para exercer uma função.}{in.va.li.dar}{\verboinum{1}}
\verb{invalidez}{ê}{}{}{}{s.f.}{Estado de incapacidade mental ou física causado por uma enfermidade ou por velhice.}{in.va.li.dez}{0}
\verb{inválido}{}{}{}{}{adj.}{Que não tem validade; nulo.}{in.vá.li.do}{0}
\verb{inválido}{}{}{}{}{}{Que não tem valor algum; insubsistente, vão.}{in.vá.li.do}{0}
\verb{inválido}{}{}{}{}{}{Desprovido de vigor físico ou mental; incapacitado para exercer uma função.}{in.vá.li.do}{0}
\verb{invariabilidade}{}{}{}{}{s.f.}{Qualidade ou estado daquilo que não se altera; imutabilidade, constância.}{in.va.ri.a.bi.li.da.de}{0}
\verb{invariante}{}{}{}{}{adj.2g.}{Que não varia; constante, estável.}{in.va.ri.an.te}{0}
\verb{invariável}{}{}{"-eis}{}{adj.2g.}{Que não varia; imutável, inalterável.}{in.va.ri.á.vel}{0}
\verb{invariável}{}{Gram.}{"-eis}{}{}{Diz"-se da palavra que não sofre flexão de gênero, número, pessoa ou caso.}{in.va.ri.á.vel}{0}
\verb{invasão}{}{}{"-ões}{}{s.f.}{Ato de penetrar ou ocupar algum lugar usando"-se a força.}{in.va.são}{0}
\verb{invasão}{}{}{"-ões}{}{}{Difusão, propagação maciça e rápida.}{in.va.são}{0}
\verb{invasão}{}{Med.}{"-ões}{}{}{Irrupção de uma epidemia em uma região.}{in.va.são}{0}
\verb{invasor}{ô}{}{}{}{adj.}{Que invade, que penetra na propriedade alheia usando"-se de força e violência.}{in.va.sor}{0}
\verb{invectiva}{}{}{}{}{s.f.}{Expressão violenta e injuriosa contra algo ou alguém; insulto.}{in.vec.ti.va}{0}
\verb{invectivar}{}{}{}{}{v.t.}{Atacar alguém com expressões violentas, injuriosas.}{in.vec.ti.var}{\verboinum{1}}
\verb{inveja}{é}{}{}{}{s.f.}{Sentimento de desgosto ou ódio em face da felicidade ou da prosperidade de outrem.}{in.ve.ja}{0}
\verb{inveja}{é}{}{}{}{}{Desejo intenso de possuir o que é de outrem.}{in.ve.ja}{0}
\verb{invejar}{}{}{}{}{v.t.}{Sentir desgosto ou ódio diante da felicidade ou do bem"-estar alheio.}{in.ve.jar}{0}
\verb{invejar}{}{}{}{}{}{Cobiçar, desejar ardentemente o que é de outrem.}{in.ve.jar}{\verboinum{1}}
\verb{invejável}{}{}{"-eis}{}{adj.2g.}{Que é digno de inveja; cobiçável, desejável.}{in.ve.já.vel}{0}
\verb{invejável}{}{}{"-eis}{}{}{Que tem muito valor; precioso.}{in.ve.já.vel}{0}
\verb{invejoso}{ô}{}{"-osos ⟨ó⟩}{"-osa ⟨ó⟩}{adj.}{Que sente inveja, despeito pela felicidade de outrem.}{in.ve.jo.so}{0}
\verb{invenção}{}{}{"-ões}{}{}{Coisa criada ou descoberta; invento.}{in.ven.ção}{0}
\verb{invenção}{}{}{"-ões}{}{s.f.}{Ato ou efeito de inventar, criar algo novo.}{in.ven.ção}{0}
\verb{invenção}{}{}{"-ões}{}{}{Aquilo que não pertence à realidade; ficção, fantasia.}{in.ven.ção}{0}
\verb{invencibilidade}{}{}{}{}{s.f.}{Qualidade ou estado do que é invencível, insuperável.}{in.ven.ci.bi.li.da.de}{0}
\verb{invencionice}{}{}{}{}{s.f.}{Mentira enganosa; intriga, embuste.}{in.ven.ci.o.ni.ce}{0}
\verb{invencível}{}{}{"-eis}{}{}{Que não se pode realizar; impraticável.}{in.ven.cí.vel}{0}
\verb{invencível}{}{}{"-eis}{}{adj.2g.}{Que não se pode vencer ou superar; inconquistável.}{in.ven.cí.vel}{0}
\verb{invencível}{}{}{"-eis}{}{}{Que não se pode contestar; indiscutível, irrefutável.}{in.ven.cí.vel}{0}
\verb{invendável}{}{}{"-eis}{}{adj.2g.}{Que não se pode vender ou que não se vende com facilidade; invendível.}{in.ven.dá.vel}{0}
\verb{invendível}{}{}{"-eis}{}{adj.2g.}{Invendável.}{in.ven.dí.vel}{0}
\verb{inventar}{}{}{}{}{v.t.}{Descobrir, criar algo novo.}{in.ven.tar}{0}
\verb{inventar}{}{}{}{}{}{Imaginar uma coisa que se dá por real; fantasiar, idealizar.}{in.ven.tar}{0}
\verb{inventar}{}{}{}{}{}{Contar falsamente; fingir, tramar.}{in.ven.tar}{0}
\verb{inventar}{}{}{}{}{}{Conceber explicações ou falsos argumentos; arquitetar, urdir.}{in.ven.tar}{\verboinum{1}}
\verb{inventariante}{}{Jur.}{}{}{s.2g.}{Indivíduo que inventaria ou que dá os bens a inventário.}{in.ven.ta.ri.an.te}{0}
\verb{inventariante}{}{Jur.}{}{}{}{Indivíduo nomeado juridicamente para arrolar, administrar e partilhar uma herança.}{in.ven.ta.ri.an.te}{0}
\verb{inventariar}{}{}{}{}{v.t.}{Fazer o inventário de bens.}{in.ven.ta.ri.ar}{0}
\verb{inventariar}{}{}{}{}{}{Enumerar, catalogar, relacionar.}{in.ven.ta.ri.ar}{0}
\verb{inventariar}{}{}{}{}{}{Descrever minuciosamente; detalhar.}{in.ven.ta.ri.ar}{\verboinum{1}}
\verb{inventário}{}{}{}{}{s.m.}{Rol, registro ou relação do patrimônio de pessoa falecida para que se possa efetuar a partilha dos bens.}{in.ven.tá.rio}{0}
\verb{inventário}{}{}{}{}{}{O documento ou papel em que estão descritos esses bens.}{in.ven.tá.rio}{0}
\verb{inventário}{}{}{}{}{}{Avaliação do patrimônio de uma sociedade comercial, para conhecer lucros e perdas.}{in.ven.tá.rio}{0}
\verb{inventiva}{}{}{}{}{s.f.}{Imaginação produtiva; engenhosidade.}{in.ven.ti.va}{0}
\verb{inventividade}{}{}{}{}{s.f.}{Faculdade de inventar; imaginação criativa.}{in.ven.ti.vi.da.de}{0}
\verb{inventivo}{}{}{}{}{adj.}{Que tem o dom, o talento de inventar, de criar; inventor.}{in.ven.ti.vo}{0}
\verb{inventivo}{}{}{}{}{}{Que é produto da imaginação, da criatividade.}{in.ven.ti.vo}{0}
\verb{invento}{}{}{}{}{s.m.}{Mecanismo criado; equipamento novo; invenção.}{in.ven.to}{0}
\verb{inventor}{ô}{}{}{}{s.m.}{Indivíduo que cria, inventa algo novo, original; inventivo. }{in.ven.tor}{0}
\verb{inventor}{ô}{}{}{}{}{Indivíduo que mente, que forja falsidades.}{in.ven.tor}{0}
\verb{inverdade}{}{}{}{}{s.f.}{Falta de verdade; inexatidão, falsidade, mentira.}{in.ver.da.de}{0}
\verb{inverídico}{}{}{}{}{adj.}{Que não é verídico; inexato, inautêntico.}{in.ve.rí.di.co}{0}
\verb{inverificável}{}{}{"-eis}{}{adj.2g.}{Que não pode ser verificado ou averiguado.}{in.ve.ri.fi.cá.vel}{0}
\verb{invernada}{}{}{}{}{s.f.}{Inverno rigoroso; invernia.}{in.ver.na.da}{0}
\verb{invernada}{}{}{}{}{}{Temporada em que há chuvas prolongadas no Norte e Nordeste durante o inverno.}{in.ver.na.da}{0}
\verb{invernada}{}{}{}{}{}{Local, nas fazendas, onde se recolhe o gado para invernar. }{in.ver.na.da}{0}
\verb{invernal}{}{}{"-ais}{}{adj.2g.}{Relativo ao inverno; hibernal.}{in.ver.nal}{0}
\verb{invernar}{}{}{}{}{v.i.}{Fazer tempo frio e chuvoso.}{in.ver.nar}{0}
\verb{invernar}{}{}{}{}{}{Passar o inverno em local abrigado; hibernar.}{in.ver.nar}{0}
\verb{invernar}{}{}{}{}{v.t.}{Colocar o gado em descanso ou a engordar em invernada.}{in.ver.nar}{\verboinum{1}}
\verb{invernia}{}{}{}{}{s.f.}{Inverno rigoroso; invernada.}{in.ver.ni.a}{0}
\verb{inverno}{é}{Fig.}{}{}{}{Idade avançada; velhice.}{in.ver.no}{0}
\verb{inverno}{é}{}{}{}{s.m.}{A mais fria estação do ano, situada entre o outono e a primavera; no Hemisfério Sul vai de 21 de junho a 22 de setembro e no Hemisfério Norte, de 22 de dezembro a 20 de março. }{in.ver.no}{0}
\verb{inverno}{é}{Por ext.}{}{}{}{Tempo chuvoso e frio.}{in.ver.no}{0}
\verb{inverno}{é}{Bras.}{}{}{}{Estação das chuvas, no Norte e no Nordeste, que corresponde à época entre o verão e o outono.}{in.ver.no}{0}
\verb{invernoso}{ô}{}{"-osos ⟨ó⟩}{"-osa ⟨ó⟩}{adj.}{Relativo ao inverno; invernal, hibernal.}{in.ver.no.so}{0}
\verb{inverossímil}{}{}{"-eis}{}{adj.2g.}{Que não é ou não parece verdade; pouco provável.}{in.ve.ros.sí.mil}{0}
\verb{inverossimilhança}{}{}{}{}{s.f.}{Ausência de verossimilhança; improbabilidade, incoerência.}{in.ve.ros.si.mi.lhan.ça}{0}
\verb{inversão}{}{}{"-ões}{}{s.f.}{Ato ou efeito de inverter; alteração da ordem.}{in.ver.são}{0}
\verb{inversão}{}{}{"-ões}{}{}{Mudança de marcha, de direção, de circulação.}{in.ver.são}{0}
\verb{inverso}{é}{}{}{}{adj.}{Que está simetricamente oposto ao sentido ou direção natural; invertido.}{in.ver.so}{0}
\verb{inverso}{é}{}{}{}{}{Voltado de cima para baixo ou de trás para diante; contrário, avesso.}{in.ver.so}{0}
\verb{inversor}{ô}{}{}{}{adj.}{Que inverte, altera a ordem, opondo os elementos.}{in.ver.sor}{0}
\verb{inversor}{ô}{Fís.}{}{}{s.m.}{Dispositivo que converte a corrente contínua em alternada.}{in.ver.sor}{0}
\verb{invertebrado}{}{Biol.}{}{}{adj.}{Diz"-se do animal desprovido de coluna vertebral, como os insetos, os vermes, os crustáceos etc.}{in.ver.te.bra.do}{0}
\verb{inverter}{ê}{}{}{}{v.t.}{Dispor em sentido contrário ao sentido original; colocar às avessas.}{in.ver.ter}{0}
\verb{inverter}{ê}{}{}{}{}{Alterar a ordem dos termos de um conjunto.}{in.ver.ter}{0}
\verb{inverter}{ê}{}{}{}{}{Trocar, mudar totalmente.}{in.ver.ter}{\verboinum{12}}
\verb{invés}{}{}{}{}{s.m.}{O lado oposto ao normal; o avesso.}{in.vés}{0}
\verb{investida}{}{}{}{}{s.f.}{Ato ou efeito de investir; arremetida, ímpeto, ataque.}{in.ves.ti.da}{0}
\verb{investida}{}{}{}{}{}{Experiência breve; ensaio, tentativa.}{in.ves.ti.da}{0}
\verb{investidor}{ô}{}{}{}{s.m.}{Pessoa física ou jurídica que realiza aplicações com fins de investimento.}{in.ves.ti.dor}{0}
\verb{investidura}{}{}{}{}{s.f.}{Ato de investir uma pessoa na posse de algum cargo.}{in.ves.ti.du.ra}{0}
\verb{investigação}{}{}{"-ões}{}{s.f.}{Ato ou efeito de investigar, perscrutar; averiguação.}{investigação}{0}
\verb{investigação}{}{}{"-ões}{}{}{Pesquisa atenta e detalhada; indagação, apuração.}{investigação}{0}
\verb{investigador}{ô}{}{}{}{adj.}{Que investiga, averigua.}{in.ves.ti.ga.dor}{0}
\verb{investigador}{ô}{}{}{}{s.m.}{Agente da polícia civil encarregado das investigações.}{in.ves.ti.ga.dor}{0}
\verb{investigar}{}{}{}{}{v.t.}{Seguir os vestígios, as pistas.}{in.ves.ti.gar}{0}
\verb{investigar}{}{}{}{}{}{Pesquisar, indagar, inquirir.}{in.ves.ti.gar}{0}
\verb{investigar}{}{}{}{}{}{Examinar detalhada e metodicamente.}{in.ves.ti.gar}{\verboinum{5}}
\verb{investimento}{}{}{}{}{s.m.}{Ato ou efeito de investir; aplicação de recursos, tempo, esforço, para se obter algo.}{in.ves.ti.men.to}{0}
\verb{investimento}{}{}{}{}{}{Aplicação de capital em empreendimentos que renderão juros ou lucros.}{in.ves.ti.men.to}{0}
\verb{investir}{}{}{}{}{v.t.}{Atirar"-se com ímpeto; atacar, assaltar.}{in.ves.tir}{0}
\verb{investir}{}{}{}{}{}{Empregar recursos, tempo, esforço com o fim de se obter algo.}{in.ves.tir}{0}
\verb{investir}{}{}{}{}{}{Aplicar capital em empreendimentos para se obter juros ou lucros.}{in.ves.tir}{\verboinum{29}}
\verb{inveterado}{}{}{}{}{adj.}{Bastante antigo; de velha data.}{in.ve.te.ra.do}{0}
\verb{inveterado}{}{}{}{}{}{Firmemente estabelecido; arraigado, enraizado.}{in.ve.te.ra.do}{0}
\verb{inveterar}{}{}{}{}{v.t.}{Tornar velho, antigo.}{in.ve.te.rar}{0}
\verb{inveterar}{}{}{}{}{}{Estabelecer por muito tempo; arraigar, enraizar.}{in.ve.te.rar}{\verboinum{1}}
\verb{inviabilidade}{}{}{}{}{s.f.}{Qualidade ou estado de inviável; inacessibilidade, inexequibilidade.}{in.vi.a.bi.li.da.de}{0}
\verb{inviabilizar}{}{}{}{}{v.t.}{Tornar inviável, impossível de ser realizado.}{in.vi.a.bi.li.zar}{\verboinum{1}}
\verb{inviável}{}{}{"-eis}{}{adj.2g.}{Que não possui condições de se praticar; irrealizável, inexequível.}{in.vi.á.vel}{0}
\verb{inviável}{}{}{"-eis}{}{}{Por onde não se pode passar; intransitável, inacessível.}{in.vi.á.vel}{0}
\verb{invicto}{}{}{}{}{adj.}{Que nunca foi vencido ou derrotado.}{in.vic.to}{0}
\verb{invicto}{}{}{}{}{}{Que não pode ser vencido; invencível.}{in.vic.to}{0}
\verb{ínvio}{}{}{}{}{adj.}{Em que não há via, nem caminho; intransitável.}{ín.vio}{0}
\verb{inviolabilidade}{}{}{}{}{s.f.}{Qualidade ou caráter do que é inviolável, que permanece intacto.}{in.vi.o.la.bi.li.da.de}{0}
\verb{inviolado}{}{}{}{}{adj.}{Que não foi violado; protegido.}{in.vi.o.la.do}{0}
\verb{inviolável}{}{}{"-eis}{}{adj.2g.}{Que não se pode ou não se deve violar.}{in.vi.o.lá.vel}{0}
\verb{invisibilidade}{}{}{}{}{s.f.}{Qualidade ou atributo do que não é visível.}{in.vi.si.bi.li.da.de}{0}
\verb{invisível}{}{}{"-eis}{}{adj.2g.}{Que, por sua natureza, não pode ser visto.}{in.vi.sí.vel}{0}
\verb{invisível}{}{}{"-eis}{}{}{Que se esconde; que age de modo oculto, fora da visão.}{in.vi.sí.vel}{0}
\verb{in vitro}{}{}{}{}{loc. adv.}{Fora do organismo vivo, em tubo de ensaio.}{\textit{in vitro}}{0}
\verb{in vivo}{}{}{}{}{loc. adv.}{No organismo vivo.}{\textit{in vivo}}{0}
\verb{invocação}{}{}{"-ões}{}{s.f.}{Ato ou efeito de invocar; chamamento, rogo.}{in.vo.ca.ção}{0}
\verb{invocação}{}{}{"-ões}{}{}{Pedido de proteção divina ou sobrenatural a uma instituição religiosa ou a um templo.}{in.vo.ca.ção}{0}
\verb{invocação}{}{}{"-ões}{}{}{Consagração, dedicação, proteção.}{in.vo.ca.ção}{0}
\verb{invocação}{}{Liter.}{"-ões}{}{}{Súplica do poeta a uma divindade para pedir inspiração.}{in.vo.ca.ção}{0}
\verb{invocado}{}{}{}{}{adj.}{Que foi chamado, evocado.}{in.vo.ca.do}{0}
\verb{invocado}{}{}{}{}{}{Que desconfia de algo ou alguém; cismado.}{in.vo.ca.do}{0}
\verb{invocar}{}{}{}{}{v.t.}{Chamar em auxílio; evocar.}{in.vo.car}{0}
\verb{invocar}{}{}{}{}{}{Pedir, suplicar, recorrer.}{in.vo.car}{0}
\verb{invocar}{}{}{}{}{}{Deixar cismado, desconfiado.}{in.vo.car}{\verboinum{2}}
\verb{invocativo}{}{}{}{}{s.f.}{Que contém invocação.}{in.vo.ca.ti.vo}{0}
\verb{invocatório}{}{}{}{}{adj.}{Relativo a invocação.}{in.vo.ca.tó.rio}{0}
\verb{involução}{}{}{"-ões}{}{s.f.}{Movimento ou processo de regressão.}{in.vo.lu.ção}{0}
\verb{involução}{}{Anat.}{"-ões}{}{}{Condição de um órgão que se encontra voltado para dentro.}{in.vo.lu.ção}{0}
\verb{invólucro}{}{}{}{}{s.m.}{Envoltório que recobre algo; revestimento.}{in.vó.lu.cro}{0}
\verb{invólucro}{}{Bot.}{}{}{}{Conjunto de pequenas folhas situado logo abaixo de uma flor ou inflorescência.}{in.vó.lu.cro}{0}
\verb{involuir}{}{}{}{}{v.i.}{Sofrer involução; regredir.}{in.vo.lu.ir}{\verboinum{26}}
\verb{involuntário}{}{}{}{}{adj.}{Que ocorre independentemente da vontade; automático, inconsciente.}{in.vo.lun.tá.rio}{0}
\verb{involuntário}{}{Anat.}{}{}{}{Diz"-se dos fenômenos orgânicos que não estão submetidos à vontade, como a digestão.}{in.vo.lun.tá.rio}{0}
\verb{involuto}{}{Bot.}{}{}{adj.}{Diz"-se da disposição das folhas jovens quando estão enroladas sobre a face interna. }{in.vo.lu.to}{0}
\verb{invulgar}{}{}{}{}{adj.2g.}{Que não é vulgar; incomum, raro.}{in.vul.gar}{0}
\verb{invulnerável}{}{}{"-eis}{}{adj.2g.}{Que não pode ser ferido fisicamente.}{in.vul.ne.rá.vel}{0}
\verb{invulnerável}{}{}{"-eis}{}{}{Que não se deixa atingir; inatacável, inatingível.}{in.vul.ne.rá.vel}{0}
\verb{inzona}{}{}{}{}{s.f.}{Ato de induzir alguém conscientemente a um erro; embuste, intriga.}{in.zo.na}{0}
\verb{inzonar}{}{}{}{}{v.t.}{Armar intrigas, embustes; mexericar.}{in.zo.nar}{\verboinum{1}}
\verb{inzoneiro}{ê}{}{}{}{adj.}{Que arma intrigas; fuxiqueiro, mexeriqueiro.}{in.zo.nei.ro}{0}
\verb{iodar}{}{}{}{}{v.t.}{Cobrir ou misturar com iodo.}{i.o.dar}{\verboinum{1}}
\verb{iodeto}{ê}{Quím.}{}{}{s.m.}{Sal ou ânion derivado do ácido iodídrico.}{i.o.de.to}{0}
\verb{iodeto}{ê}{}{}{}{}{Qualquer sal que contenha esse ânion.}{i.o.de.to}{0}
\verb{iódico}{}{Quím.}{}{}{adj.}{Diz"-se do ácido resultante da oxidação do iodo, usado como desinfetante.}{i.ó.di.co}{0}
\verb{iodo}{ô}{Quím.}{}{}{s.m.}{Elemento químico do grupo dos halogênios, sólido acinzentado, com brilho metálico, muito usado na medicina. \elemento{53}{126.90447}{I}.}{i.o.do}{0}
\verb{iodofórmio}{}{Farm.}{}{}{s.m.}{Composto obtido pela ação do iodo sobre a acetona em meio básico, usado como antisséptico tópico.}{i.o.do.fór.mio}{0}
\verb{ioga}{ó/ ou /ô}{}{}{}{s.f.}{Sistema místico"-filosófico da Índia que procura o domínio absoluto do espírito sobre a matéria por meio de exercícios corporais, respiratórios, mentais etc.}{i.o.ga}{0}
\verb{iogue}{ô}{}{}{}{s.m.}{Pessoa que pratica ioga.}{i.o.gue}{0}
\verb{iogue}{ô}{}{}{}{}{Adepto da filosofia da ioga.}{i.o.gue}{0}
\verb{iogurte}{}{}{}{}{s.m.}{Leite coalhado pelo fermento láctico, de consistência cremosa.}{i.o.gur.te}{0}
\verb{ioiô}{}{}{}{}{s.m.}{Brinquedo que consiste em dois discos unidos pelo centro, no qual se prende e enrola um cordel que os faz subir e descer com um movimento de rotação.}{io.iô}{0}
\verb{iole}{ó}{Esport.}{}{}{s.m.}{Embarcação esportiva leve e delgada, de boca muito estreita, movida a remo.}{i.o.le}{0}
\verb{íon}{}{Fís. e Quím.}{}{}{s.m.}{Átomo ou agrupamento atômico eletricamente carregado.}{í.on}{0}
\verb{iônio}{}{Fís. e Quím.}{}{}{s.m.}{Íon.}{i.ô.nio}{0}
\verb{ionização}{}{Fís. e Quím.}{"-ões}{}{s.f.}{Processo por meio do qual uma molécula ou um átomo neutro se torna portador de uma carga elétrica positiva ou negativa.}{i.o.ni.za.ção}{0}
\verb{ionizar}{}{Fís. e Quím.}{}{}{v.t.}{Perder ou ganhar elétrons para formar íons; ser decomposto em íons.}{i.o.ni.zar}{\verboinum{1}}
\verb{ionosfera}{é}{Geol.}{}{}{s.f.}{Camada da atmosfera terrestre, acima da estratosfera, caracterizada pela presença de elétrons e íons. }{i.o.nos.fe.ra}{0}
\verb{ionte}{}{Fís. e Quím.}{}{}{s.m.}{Íon.}{i.on.te}{0}
\verb{ioruba}{}{}{}{}{adj.2g.}{Relativo aos iorubas.}{i.o.ru.ba}{0}
\verb{iorubá}{}{}{}{}{}{Var. de \textit{ioruba}.}{i.o.ru.bá}{0}
\verb{ioruba}{}{}{}{}{s.2g.}{Indivíduo pertencente aos iorubas, povo negro da África ocidental, habitante da Nigéria, Togo e Benin.}{i.o.ru.ba}{0}
\verb{iota}{ó}{}{}{}{s.m.}{Nona letra do alfabeto grego, correspondente ao \textit{I} do latim e das línguas neolatinas.}{i.o.ta}{0}
\verb{ipé}{}{}{}{}{}{Var. de \textit{ipê}.}{i.pé}{0}
\verb{ipê}{}{Bot.}{}{}{s.m.}{Árvore ornamental de flores amarelas ou violáceas, considerada símbolo do Brasil.}{i.pê}{0}
\verb{ipecacuanha}{}{Bras.}{}{}{s.f.}{Erva de raízes grossas e nodulosas com propriedades medicinais, que vive no solo de florestas pluviais.}{i.pe.ca.cu.a.nha}{0}
\verb{ipsílon}{}{}{}{}{s.m.}{Ípsilon.}{ip.sí.lon}{0}
\verb{ípsilon}{}{}{}{}{s.m.}{Vigésima letra do alfabeto grego.}{íp.si.lon}{0}
\verb{ípsilon}{}{}{}{}{}{Nome da letra \textit{y}.}{íp.si.lon}{0}
\verb{ipsilone}{}{}{}{}{s.m.}{Ípsilon.}{ip.si.lo.ne}{0}
\verb{ipueira}{ê}{Bras.}{}{}{s.f.}{Pântano formado em lugares baixos pelo transbordamento dos rios.}{i.pu.ei.ra}{0}
\verb{iquebana}{}{}{}{}{s.m.}{Arte e técnica de composição floral de origem japonesa.}{i.que.ba.na}{0}
\verb{iquebana}{}{}{}{}{}{Arranjo floral realizado com essa técnica.}{i.que.ba.na}{0}
\verb{Ir}{}{Quím.}{}{}{}{Símb. do \textit{irídio}.}{Ir}{0}
\verb{ir}{}{}{}{}{v.i.}{Sair de um lugar para chegar a outro.}{ir}{0}
\verb{ir}{}{}{}{}{}{Seguir por um caminho; andar, caminhar.}{ir}{0}
\verb{ir}{}{}{}{}{}{Ter a intenção de fazer alguma coisa.}{ir}{0}
\verb{ir}{}{}{}{}{}{Estar em determinada situação.}{ir}{0}
\verb{ir}{}{}{}{}{v.pron.}{Desaparecer de onde está; sumir.}{ir}{\verboinum{55}}
\verb{ira}{}{}{}{}{s.f.}{Grande ódio; raiva, cólera, indignação.}{i.ra}{0}
\verb{irá}{}{Zool.}{}{}{s.m.}{Espécie de abelha que faz seu ninho no chão.}{i.rá}{0}
\verb{iracúndia}{}{}{}{}{s.f.}{Fúria, ira, raiva.}{i.ra.cún.dia}{0}
\verb{iracundo}{}{}{}{}{adj.}{Irado, colérico, furibundo.}{i.ra.cun.do}{0}
\verb{irado}{}{}{}{}{adj.}{Que está com raiva; colérico, enraivecido.}{i.ra.do}{0}
\verb{iraniano}{}{}{}{}{adj.}{Relativo ao Irã.}{i.ra.ni.a.no}{0}
\verb{iraniano}{}{}{}{}{s.m.}{Indivíduo natural ou habitante desse país.}{i.ra.ni.a.no}{0}
\verb{iraquiano}{}{}{}{}{adj.}{Relativo ao Iraque.}{i.ra.qui.a.no}{0}
\verb{iraquiano}{}{}{}{}{s.m.}{Indivíduo natural ou habitante desse país.}{i.ra.qui.a.no}{0}
\verb{irar}{}{}{}{}{v.t.}{Causar ira; enfurecer, encolerizar.}{i.rar}{\verboinum{1}}
\verb{irara}{}{Zool.}{}{}{s.f.}{Animal carnívoro de corpo esguio, pelo escuro, curto e áspero e pernas curtas; papa"-mel, jaguapé.}{i.ra.ra}{0}
\verb{irascibilidade}{}{}{}{}{s.f.}{Qualidade de irascível; iracúndia.}{i.ras.ci.bi.li.da.de}{0}
\verb{irascível}{}{}{"-eis}{}{adj.2g.}{Que se ira facilmente; irritável, iracundo.}{i.ras.cí.vel}{0}
\verb{irerê}{}{Zool.}{}{}{s.2g.}{Ave aquática de cabeça branca, corpo listrado e asas negras, que vive em rios e lagoas.}{i.re.rê}{0}
\verb{iriar}{}{}{}{}{v.t.}{Revestir com as cores do arco"-íris; matizar.}{i.ri.ar}{\verboinum{1}}
\verb{iridescente}{}{}{}{}{adj.2g.}{Que tem ou reflete as cores do arco"-íris.}{i.ri.des.cen.te}{0}
\verb{irídio}{}{Quím.}{}{}{s.m.}{Elemento químico metálico, prateado, com um leve tom amarelado, muito parecido com a platina, utilizado em ligas especiais. \elemento{77}{192.217}{Ir}.}{i.rí.dio}{0}
\verb{íris}{}{}{}{}{s.2g.}{O espectro solar.}{í.ris}{0}
\verb{íris}{}{Anat.}{}{}{}{Membrana circular retrátil com um orifício no centro, que regula a entrada de luz no olho.}{í.ris}{0}
\verb{íris}{}{}{}{}{}{Certa pedra preciosa.}{í.ris}{0}
\verb{irisar}{}{}{}{}{v.t.}{Revestir com as cores do arco"-íris.}{i.ri.sar}{\verboinum{1}}
\verb{irlandês}{}{}{}{}{adj.}{Relativo à Irlanda.}{ir.lan.dês}{0}
\verb{irlandês}{}{}{}{}{s.m.}{Indivíduo natural ou habitante desse país.}{ir.lan.dês}{0}
\verb{irlandês}{}{}{}{}{}{Língua falada na Irlanda.}{ir.lan.dês}{0}
\verb{irmã}{}{}{}{}{s.f.}{Forma feminina de \textit{irmão}.}{ir.mã}{0}
\verb{irmã}{}{}{}{}{}{Membro de organização religiosa feminina, especialmente das congregações católicas; freira.}{ir.mã}{0}
\verb{irmanar}{}{}{}{}{v.t.}{Tornar irmão; igualar, unir.}{ir.ma.nar}{\verboinum{1}}
\verb{irmandade}{}{}{}{}{}{Relação de parentesco existente entre irmãos.}{ir.man.da.de}{0}
\verb{irmandade}{}{}{}{}{s.f.}{Associação de caráter religioso; liga, confraria.}{ir.man.da.de}{0}
\verb{irmandade}{}{Por ext.}{}{}{}{Agremiação de pessoas com um mesmo objetivo, geralmente de caráter social; confederação.}{ir.man.da.de}{0}
\verb{irmão}{}{}{"-ãos}{irmã}{s.m.}{Filho de mesma mãe ou mesmo pai.}{ir.mão}{0}
\verb{irmão}{}{}{"-ãos}{irmã}{}{Membro de irmandade ou confraria.}{ir.mão}{0}
\verb{irmão}{}{}{"-ãos}{irmã}{}{Membro de organização eclesiástica que não recebeu ordens sacras.}{ir.mão}{0}
\verb{irmão}{}{}{"-ãos}{irmã}{adj.}{Semelhante, igual, idêntico.}{ir.mão}{0}
\verb{ironia}{}{Gram.}{}{}{s.f.}{Figura de linguagem em que o que se deve entender é o oposto do que se diz. }{i.ro.ni.a}{0}
\verb{ironia}{}{}{}{}{}{Evento marcado por certa incongruência ou contraste entre o esperado e o obtido.}{i.ro.ni.a}{0}
\verb{ironia}{}{}{}{}{}{Zombaria, sarcasmo.}{i.ro.ni.a}{0}
\verb{irônico}{}{}{}{}{adj.}{Que contém ironia.}{i.rô.ni.co}{0}
\verb{irônico}{}{}{}{}{}{Zombeteiro, sarcástico.}{i.rô.ni.co}{0}
\verb{ironizar}{}{}{}{}{v.t.}{Manifestar"-se com ironia.}{i.ro.ni.zar}{0}
\verb{ironizar}{}{}{}{}{v.i.}{Fazer ironia; zombar, troçar.}{i.ro.ni.zar}{\verboinum{1}}
\verb{iroso}{ô}{}{"-osos ⟨ó⟩}{"-osa ⟨ó⟩}{adj.}{Cheio de ira; raivoso.}{i.ro.so}{0}
\verb{iroso}{ô}{Fig.}{"-osos ⟨ó⟩}{"-osa ⟨ó⟩}{}{Tempestuoso, tormentoso, revoltoso.}{i.ro.so}{0}
\verb{irra}{}{}{}{}{interj.}{Expressão que denota raiva, desaprovação, desprezo.}{ir.ra}{0}
\verb{irracional}{}{}{"-ais}{}{adj.2g.}{Que não tem uso da razão; que não provém do raciocínio; ilógico.}{ir.ra.ci.o.nal}{0}
\verb{irracional}{}{}{"-ais}{}{}{Contrário à razão; insensato, desarrazoado.}{ir.ra.ci.o.nal}{0}
\verb{irracional}{}{}{"-ais}{}{}{Destituído de raciocínio. (\textit{O cachorro é um animal irracional.})}{ir.ra.ci.o.nal}{0}
\verb{irracional}{}{Mat.}{"-ais}{}{}{Diz"-se de número com infinitas casas decimais e, portanto, impossível de ser expresso como a razão entre dois números inteiros.}{ir.ra.ci.o.nal}{0}
\verb{irradiação}{}{}{"-ões}{}{s.f.}{Ato ou efeito de irradiar.}{ir.ra.di.a.ção}{0}
\verb{irradiação}{}{Fís.}{"-ões}{}{}{Exposição de um material a um feixe de partículas.}{ir.ra.di.a.ção}{0}
\verb{irradiação}{}{}{"-ões}{}{}{Transmissão radiofônica.}{ir.ra.di.a.ção}{0}
\verb{irradiante}{}{}{}{}{adj.2g.}{Que irradia; irradiador.}{ir.ra.di.an.te}{0}
\verb{irradiante}{}{Fig.}{}{}{}{Que projeta em todas as direções.}{ir.ra.di.an.te}{0}
\verb{irradiar}{}{}{}{}{v.t.}{Emitir em todas as direções.}{ir.ra.di.ar}{0}
\verb{irradiar}{}{}{}{}{}{Propagar, espalhar.}{ir.ra.di.ar}{0}
\verb{irradiar}{}{}{}{}{}{Transmitir pelo rádio.}{ir.ra.di.ar}{0}
\verb{irradiar}{}{}{}{}{v.i.}{Emitir luz.}{ir.ra.di.ar}{\verboinum{1}}
\verb{irreal}{}{}{"-ais}{}{adj.2g.}{Que não é ou não parece real.}{ir.re.al}{0}
\verb{irreal}{}{}{"-ais}{}{}{Que é produto da fantasia, da imaginação; imaginário, fantasioso.}{ir.re.al}{0}
\verb{irrealizável}{}{}{"-eis}{}{adj.2g.}{Que não pode ser realizado.}{ir.re.a.li.zá.vel}{0}
\verb{irrebatível}{}{}{"-eis}{}{adj.2g.}{Que não pode ser rebatido; irrefutável.}{ir.re.ba.tí.vel}{0}
\verb{irreconciliável}{}{}{"-eis}{}{adj.2g.}{Que não se pode reconciliar.}{ir.re.con.ci.li.á.vel}{0}
\verb{irreconhecível}{}{}{"-eis}{}{adj.2g.}{Impossível de ser reconhecido por haver mudado muito.}{ir.re.co.nhe.cí.vel}{0}
\verb{irrecorrível}{}{Jur.}{"-eis}{}{adj.2g.}{Diz"-se de decisão à qual não cabe mais recurso; inapelável.}{ir.re.cor.rí.vel}{0}
\verb{irrecuperável}{}{}{"-eis}{}{adj.2g.}{Que não pode ser recuperado.}{ir.re.cu.pe.rá.vel}{0}
\verb{irrecusável}{}{}{"-eis}{}{adj.2g.}{Que não se pode recusar.}{ir.re.cu.sá.vel}{0}
\verb{irredutível}{}{}{"-eis}{}{adj.2g.}{Que não se pode reduzir.}{ir.re.du.tí.vel}{0}
\verb{irredutível}{}{}{"-eis}{}{}{Que não se pode decompor; indecomponível.}{ir.re.du.tí.vel}{0}
\verb{irredutível}{}{Fig.}{"-eis}{}{}{Que não se deixa convencer por opiniões diferentes das suas; inflexível.}{ir.re.du.tí.vel}{0}
\verb{irreduzível}{}{}{"-eis}{}{adj.2g.}{Que não se pode reduzir; irredutível.}{ir.re.du.zí.vel}{0}
\verb{irrefletido}{}{}{}{}{adj.}{Diz"-se de pessoa que não reflete.}{ir.re.fle.ti.do}{0}
\verb{irrefletido}{}{}{}{}{}{Diz"-se de ato ou dito não precedido de reflexão; imponderado.}{ir.re.fle.ti.do}{0}
\verb{irreflexão}{cs}{}{"-ões}{}{s.f.}{Ausência de reflexão; imprudência.}{ir.re.fle.xão}{0}
\verb{irrefreável}{}{}{"-eis}{}{adj.2g.}{Que não se pode refrear, reprimir; irreprimível.}{ir.re.fre.á.vel}{0}
\verb{irrefutável}{}{}{"-eis}{}{adj.2g.}{Que não se pode refutar; evidente, incontestável.}{ir.re.fu.tá.vel}{0}
\verb{irregenerável}{}{}{"-eis}{}{adj.2g.}{Que não se pode regenerar; incorrigível.}{ir.re.ge.ne.rá.vel}{0}
\verb{irregular}{}{}{}{}{adj.2g.}{Contra as regras.}{ir.re.gu.lar}{0}
\verb{irregular}{}{}{}{}{}{Em que não há regularidade.}{ir.re.gu.lar}{0}
\verb{irregular}{}{}{}{}{}{Anormal, diferente, excepcional.}{ir.re.gu.lar}{0}
\verb{irregular}{}{}{}{}{}{Que não apresenta uniformidade; desigual, assimétrico.}{ir.re.gu.lar}{0}
\verb{irregular}{}{}{}{}{}{Inconstante, volúvel.}{ir.re.gu.lar}{0}
\verb{irregularidade}{}{}{}{}{s.f.}{Qualidade de irregular.}{ir.re.gu.la.ri.da.de}{0}
\verb{irregularidade}{}{}{}{}{}{Ação ou procedimento irregular.}{ir.re.gu.la.ri.da.de}{0}
\verb{irrelevância}{}{}{}{}{s.f.}{Ausência de relevância ou importância.}{ir.re.le.vân.cia}{0}
\verb{irrelevante}{}{}{}{}{adj.2g.}{Que não é relevante, que tem pouca ou nenhuma importância.}{ir.re.le.van.te}{0}
\verb{irreligiosidade}{}{}{}{}{s.f.}{Ausência de religiosidade.}{ir.re.li.gi.o.si.da.de}{0}
\verb{irreligiosidade}{}{}{}{}{}{Ato ou dito irreligioso.}{ir.re.li.gi.o.si.da.de}{0}
\verb{irreligioso}{ô}{}{"-osos ⟨ó⟩}{"-osa ⟨ó⟩}{}{Não religioso; ateu.}{ir.re.li.gi.o.so}{0}
\verb{irreligioso}{ô}{}{"-osos ⟨ó⟩}{"-osa ⟨ó⟩}{adj.}{Contrário à religião; herege.}{ir.re.li.gi.o.so}{0}
\verb{irremediável}{}{}{"-eis}{}{adj.2g.}{Que não pode ser remediado.}{ir.re.me.di.á.vel}{0}
\verb{irremediável}{}{}{"-eis}{}{}{Que não pode deixar de acontecer; fatal, inevitável.}{ir.re.me.di.á.vel}{0}
\verb{irremissível}{}{}{"-eis}{}{}{Que não pode ou não merece ser remitido; imperdoável.}{ir.re.mis.sí.vel}{0}
\verb{irremissível}{}{}{"-eis}{}{adj.2g.}{Que não se pode evitar; fatal, infalível.}{ir.re.mis.sí.vel}{0}
\verb{irremovível}{}{}{"-eis}{}{adj.2g.}{Que não se pode remover ou mudar de lugar.}{ir.re.mo.ví.vel}{0}
\verb{irremovível}{}{Fig.}{"-eis}{}{}{Que não se pode evitar.}{ir.re.mo.ví.vel}{0}
\verb{irreparável}{}{}{"-eis}{}{adj.2g.}{Que não se pode reparar; irrecuperável, irremediável.}{ir.re.pa.rá.vel}{0}
\verb{irreplicável}{}{}{"-eis}{}{adj.2g.}{Que não se pode replicar; irrespondível, irrefutável.}{ir.re.pli.cá.vel}{0}
\verb{irrepreensível}{}{}{"-eis}{}{adj.2g.}{Que não merece ou não dá margem a repreensão; perfeito.}{ir.re.pre.en.sí.vel}{0}
\verb{irreprimível}{}{}{"-eis}{}{adj.2g.}{Que não pode ser reprimido; irrefreável.}{ir.re.pri.mí.vel}{0}
\verb{irreprochável}{}{}{"-eis}{}{adj.2g.}{Que não merece censura; irrepreensível, impecável.}{ir.re.pro.chá.vel}{0}
\verb{irrequieto}{é}{}{}{}{adj.}{Que não sossega; agitado, turbulento.}{ir.re.qui.e.to}{0}
\verb{irresgatável}{}{}{"-eis}{}{adj.2g.}{Que não se pode resgatar.}{ir.res.ga.tá.vel}{0}
\verb{irresistível}{}{}{"-eis}{}{adj.2g.}{A que ou a quem não se pode resistir; sedutor.}{ir.re.sis.tí.vel}{0}
\verb{irresistível}{}{}{"-eis}{}{}{Que não se pode dominar; invencível.}{ir.re.sis.tí.vel}{0}
\verb{irresoluto}{}{}{}{}{adj.}{Que não foi resolvido.}{ir.re.so.lu.to}{0}
\verb{irresoluto}{}{}{}{}{}{Que tem dificuldade em tomar uma decisão; indeciso.}{ir.re.so.lu.to}{0}
\verb{irresolúvel}{}{}{"-eis}{}{adj.2g.}{Que não tem solução; insolúvel.}{ir.re.so.lú.vel}{0}
\verb{irrespirável}{}{}{"-eis}{}{adj.2g.}{Que não se pode respirar.}{ir.res.pi.rá.vel}{0}
\verb{irrespirável}{}{}{"-eis}{}{}{Diz"-se de ambiente no qual não se pode respirar direito.}{ir.res.pi.rá.vel}{0}
\verb{irrespondível}{}{}{"-eis}{}{adj.2g.}{A que não se pode responder; irrefutável.}{ir.res.pon.dí.vel}{0}
\verb{irresponsabilidade}{}{}{}{}{adj.}{Qualidade de irresponsável; falta de responsabilidade.}{ir.res.pon.sa.bi.li.da.de}{0}
\verb{irresponsabilidade}{}{}{}{}{}{Ato ou dito que denota falta de responsabilidade.}{ir.res.pon.sa.bi.li.da.de}{0}
\verb{irresponsabilidade}{}{Jur.}{}{}{}{Qualidade do indivíduo que não pode ser responsabilizado por atos ilícitos.}{ir.res.pon.sa.bi.li.da.de}{0}
\verb{irresponsável}{}{}{"-eis}{}{adj.2g.}{Que revela falta de responsabilidade.}{ir.res.pon.sá.vel}{0}
\verb{irresponsável}{}{Jur.}{"-eis}{}{}{Que não pode ser responsabilizado.}{ir.res.pon.sá.vel}{0}
\verb{irresponsável}{}{}{"-eis}{}{s.m.}{Indivíduo que age irresponsavelmente.}{ir.res.pon.sá.vel}{0}
\verb{irrestringível}{}{}{"-eis}{}{adj.2g.}{Que não se pode restringir.}{ir.res.trin.gí.vel}{0}
\verb{irrestrito}{}{}{}{}{adj.}{Que não tem restrição; ilimitado, amplo.}{ir.res.tri.to}{0}
\verb{irretorquível}{}{}{"-eis}{}{adj.2g.}{A que não se pode retorquir; irrespondível.}{ir.re.tor.quí.vel}{0}
\verb{irretratável}{}{}{"-eis}{}{adj.2g.}{Não suscetível de retratação; irrevogável.}{ir.re.tra.tá.vel}{0}
\verb{irretratável}{}{}{"-eis}{}{adj.2g.}{Que não se pode retratar, reproduzir, fotografar.}{ir.re.tra.tá.vel}{0}
\verb{irreverência}{}{}{}{}{s.f.}{Tratamento inusual dado às coisas ou pessoas consideradas sérias.}{ir.re.ve.rên.cia}{0}
\verb{irreverência}{}{}{}{}{}{Ato ou dito irreverente.}{ir.re.ve.rên.cia}{0}
\verb{irreverente}{}{}{}{}{adj.2g.}{Que trata com irreverência.  }{ir.re.ve.ren.te}{0}
\verb{irreversível}{}{}{"-eis}{}{adj.2g.}{Que não é reversível; que se dá em uma única direção.}{ir.re.ver.sí.vel}{0}
\verb{irrevogável}{}{}{"-eis}{}{adj.2g.}{Que não se pode revogar.}{ir.re.vo.gá.vel}{0}
\verb{irrigação}{}{}{"-ões}{}{s.f.}{Ato ou efeito de irrigar; rega.}{ir.ri.ga.ção}{0}
\verb{irrigador}{ô}{}{}{}{s.m.}{Utensílio para irrigar jardins; regador.}{ir.ri.ga.dor}{0}
\verb{irrigar}{}{}{}{}{v.t.}{Molhar com água ou outro líquido; aguar.}{ir.ri.gar}{0}
\verb{irrigar}{}{}{}{}{}{Regar, molhar por meio de processos não naturais.}{ir.ri.gar}{0}
\verb{irrigar}{}{Med.}{}{}{}{Conduzir líquido para determinada área, especialmente sangue e linfa.}{ir.ri.gar}{\verboinum{5}}
\verb{irrisão}{}{}{"-ões}{}{s.f.}{Zombaria, escárnio.}{ir.ri.são}{0}
\verb{irrisão}{}{Por ext.}{"-ões}{}{}{Aquele ou aquilo que é alvo de risos e zombarias.}{ir.ri.são}{0}
\verb{irrisão}{}{Por ext.}{"-ões}{}{}{Acontecimento irrisório.}{ir.ri.são}{0}
\verb{irrisório}{}{}{}{}{adj.}{Em que há irrisão.}{ir.ri.só.rio}{0}
\verb{irrisório}{}{}{}{}{}{Que é dito ou feito com intenção de provocar irrisão; risível, cômico.}{ir.ri.só.rio}{0}
\verb{irrisório}{}{}{}{}{}{Que é demasiado insignificante para ser levado em consideração.}{ir.ri.só.rio}{0}
\verb{irritabilidade}{}{}{}{}{s.f.}{Qualidade ou estado de irritável.}{ir.ri.ta.bi.li.da.de}{0}
\verb{irritabilidade}{}{}{}{}{}{Capacidade que tem uma estrutura de reagir a certos agentes químicos ou físicos.}{ir.ri.ta.bi.li.da.de}{0}
\verb{irritação}{}{}{"-ões}{}{s.f.}{Ato ou efeito de irritar.}{ir.ri.ta.ção}{0}
\verb{irritação}{}{}{"-ões}{}{}{Estado de nervosismo ou de cólera contida.}{ir.ri.ta.ção}{0}
\verb{irritação}{}{}{"-ões}{}{}{Exacerbação, exasperação.}{ir.ri.ta.ção}{0}
\verb{irritação}{}{Med.}{"-ões}{}{}{Lesão de natureza inflamatória, localizada em pele ou mucosa, resultante de estímulos químicos ou físicos.}{ir.ri.ta.ção}{0}
\verb{irritadiço}{}{}{}{}{adj.}{Que se irrita ou se exalta com facilidade.}{ir.ri.ta.di.ço}{0}
\verb{irritante}{}{}{}{}{adj.2g.}{Que leva a um estado de irritação, por vezes próximo ao da cólera.}{ir.ri.tan.te}{0}
\verb{irritante}{}{}{}{}{s.m.}{Substância que irrita, excita.}{ir.ri.tan.te}{0}
\verb{irritar}{}{}{}{}{v.t.}{Encolerizar, exasperar, exaltar.}{ir.ri.tar}{0}
\verb{irritar}{}{}{}{}{}{Excitar, provocar.}{ir.ri.tar}{0}
\verb{irritar}{}{}{}{}{}{Impacientar, importunar.}{ir.ri.tar}{0}
\verb{irritar}{}{}{}{}{}{Causar dor, inflamação a um órgão.}{ir.ri.tar}{\verboinum{1}}
\verb{irritável}{}{}{"-eis}{}{adj.2g.}{Que se irrita facilmente; irritadiço.}{ir.ri.tá.vel}{0}
\verb{irritável}{}{Med.}{"-eis}{}{}{Que apresenta capacidade de reagir a certos agentes químicos ou físicos.}{ir.ri.tá.vel}{0}
\verb{írrito}{}{}{}{}{adj.}{Que é nulo, sem efeito.}{ír.ri.to}{0}
\verb{irromper}{ê}{}{}{}{v.i.}{Entrar, surgir com ímpeto, com violência.}{ir.rom.per}{0}
\verb{irromper}{ê}{}{}{}{}{Mostrar"-se ou fazer"-se ouvir, de repente.}{ir.rom.per}{0}
\verb{irromper}{ê}{}{}{}{}{Intervir, sobrevir.}{ir.rom.per}{\verboinum{12}}
\verb{irrupção}{}{}{"-ões}{}{s.f.}{Ato ou efeito de irromper.}{ir.rup.ção}{0}
\verb{irrupção}{}{}{"-ões}{}{}{Invasão, aparecimento súbito e impetuoso.}{ir.rup.ção}{0}
\verb{irrupção}{}{Por ext.}{"-ões}{}{}{Intervenção, superveniência.}{ir.rup.ção}{0}
\verb{irruptivo}{}{}{}{}{adj.}{Que causa irrupção.}{ir.rup.ti.vo}{0}
\verb{isca}{}{}{}{}{}{Engodo que se põe no anzol para pescar.}{is.ca}{0}
\verb{isca}{}{}{}{}{s.f.}{Aquilo que atrai e seduz uma pessoa.}{is.ca}{0}
\verb{isca}{}{}{}{}{}{Combustível que recebe as faíscas do fuzil para comunicar fogo.}{is.ca}{0}
\verb{isca}{}{}{}{}{}{A mecha do isqueiro.}{is.ca}{0}
\verb{iscar}{}{}{}{}{v.t.}{Pôr isca; cevar.}{is.car}{0}
\verb{iscar}{}{Fig.}{}{}{}{Atrair alguém seduzindo"-o com algo que é de seu interesse; engodar.}{is.car}{0}
\verb{iscar}{}{}{}{}{}{Contaminar, contagiar.}{is.car}{0}
\verb{iscar}{}{}{}{}{}{Untar, besuntar.}{is.car}{\verboinum{2}}
\verb{isenção}{}{}{"-ões}{}{s.f.}{Ato ou efeito de isentar ou eximir.}{i.sen.ção}{0}
\verb{isenção}{}{}{"-ões}{}{}{Independência de caráter; desinteresse, abnegação.}{i.sen.ção}{0}
\verb{isenção}{}{}{"-ões}{}{}{Imparcialidade, neutralidade.}{i.sen.ção}{0}
\verb{isentar}{}{}{}{}{v.t.}{Tornar isento; livrar, dispensar, eximir.}{i.sen.tar}{\verboinum{1}}
\verb{isento}{}{}{}{}{adj.}{Que se encontra desobrigado, dispensado, eximido.}{i.sen.to}{0}
\verb{isento}{}{}{}{}{}{Desembaraçado, livre, limpo.}{i.sen.to}{0}
\verb{isento}{}{}{}{}{}{Imparcial, neutro.}{i.sen.to}{0}
\verb{isento}{}{}{}{}{}{Que é desprovido de algo.}{i.sen.to}{0}
\verb{islã}{}{}{}{}{s.m.}{O mundo muçulmano; o conjunto dos povos de civilização islâmica, que professam o islamismo.}{is.lã}{0}
\verb{islame}{}{}{}{}{s.m.}{Islã.}{is.la.me}{0}
\verb{islâmico}{}{}{}{}{adj.}{Relativo ao Islã.}{is.lâ.mi.co}{0}
\verb{islamismo}{}{Relig.}{}{}{s.m.}{Religião caracterizada pelo monoteísmo e a síntese entre fé religiosa e organização sociopolítica, fundada pelo profeta árabe Maomé, cujo livro sagrado é o Corão, que se tornou o fundamento escrito da fé muçulmana.}{is.la.mis.mo}{0}
\verb{islamita}{}{}{}{}{s.2g.}{Seguidor do islamismo.}{is.la.mi.ta}{0}
\verb{islandês}{}{}{}{}{adj.}{Relativo à Islândia.}{is.lan.dês}{0}
\verb{islandês}{}{}{}{}{s.m.}{Indivíduo natural ou habitante desse país.}{is.lan.dês}{0}
\verb{islandês}{}{}{}{}{}{Língua falada na Islândia.}{is.lan.dês}{0}
\verb{islão}{}{}{}{}{s.m.}{Islã.}{is.lão}{0}
\verb{ismaelita}{}{}{}{}{s.2g.}{Indivíduo descendente de Ismael, filho do patriarca Abraão e sua escrava Agar, que viviam, segundo a Bíblia, numa confederação de tribos no deserto da Arábia.}{is.ma.e.li.ta}{0}
\verb{ismaelita}{}{}{}{}{}{Relativo aos ismaelitas.}{is.ma.e.li.ta}{0}
\verb{isócrono}{}{}{}{}{adj.}{Diz"-se de movimento simultâneo.}{i.só.cro.no}{0}
\verb{isolacionismo}{}{}{}{}{s.m.}{Política de um país que se isola do cenário internacional, mediante recusa a formar alianças e assumir compromissos econômicos externos.}{i.so.la.ci.o.nis.mo}{0}
\verb{isolador}{ô}{}{}{}{adj.}{Que isola.}{i.so.la.dor}{0}
\verb{isolador}{ô}{Fís.}{}{}{s.m.}{Componente de um circuito elétrico ou eletrônico que tem a função de isolá"-lo eletricamente do exterior.}{i.so.la.dor}{0}
\verb{isolamento}{}{}{}{}{s.m.}{Ato ou efeito de isolar.}{i.so.la.men.to}{0}
\verb{isolamento}{}{}{}{}{}{Estado de pessoa isolada.}{i.so.la.men.to}{0}
\verb{isolamento}{}{Fís.}{}{}{}{Separação feita entre um corpo eletrizado e os corpos que o rodeiam.}{i.so.la.men.to}{0}
\verb{isolamento}{}{Med.}{}{}{}{Ação de manter um doente contagioso confinado para não contagiar outras pessoas.}{i.so.la.men.to}{0}
\verb{isolante}{}{}{}{}{adj.2g.}{Isolador.}{i.so.lan.te}{0}
\verb{isolante}{}{Fís.}{}{}{s.m.}{Substância que conduz muito pouca ou nenhuma corrente elétrica.}{i.so.lan.te}{0}
\verb{isolar}{}{}{}{}{v.t.}{Tornar solitário; separar ou estremar de qualquer comunicação.}{i.so.lar}{0}
\verb{isolar}{}{Fís.}{}{}{}{Aplicar isolador ou isolante.}{i.so.lar}{0}
\verb{isolar}{}{}{}{}{v.i.}{Afastar mau agouro.}{i.so.lar}{\verboinum{1}}
\verb{isonomia}{}{}{}{}{s.f.}{Estado daqueles que são governados pelas mesmas leis.}{i.so.no.mi.a}{0}
\verb{isonomia}{}{Jur.}{}{}{}{Igualdade de todos perante a lei, assegurada como princípio constitucional.}{i.so.no.mi.a}{0}
\verb{isopor}{ô}{Por ext.}{}{}{}{Artefato feito com esse material.}{i.so.por}{0}
\verb{isopor}{ô}{}{}{}{s.m.}{Espuma de poliestireno, utilizada como isolante térmico.}{i.so.por}{0}
\verb{isóscele}{}{}{}{}{}{Var. de \textit{isósceles}.}{i.sós.ce.le}{0}
\verb{isósceles}{}{Geom.}{}{}{adj.2g.}{Diz"-se de triângulo ou trapézio que tem dois lados iguais.}{i.sós.ce.les}{0}
\verb{isotérmico}{}{}{}{}{adj.}{Que tem a mesma temperatura.}{i.so.tér.mi.co}{0}
\verb{isótopo}{}{Quím.}{}{}{adj.}{Diz"-se de cada um de dois ou mais átomos  de um mesmo elemento, cujo núcleo atômico possui o mesmo número de prótons, mas número de nêutrons diferentes.}{i.só.to.po}{0}
\verb{isqueiro}{ê}{}{}{}{s.m.}{Pequeno aparelho que provoca faísca pelo atrito de uma roda dentada e uma pedra, que produz chama, próprio para acender cigarros, charutos e cachimbos.}{is.quei.ro}{0}
\verb{isquemia}{}{Med.}{}{}{s.f.}{Diminuição ou suspensão da irrigação sanguínea, numa parte do organismo, ocasionada por obstrução arterial ou pela diminuição do diâmetro dos vasos sanguíneos.}{is.que.mi.a}{0}
\verb{ísquio}{}{Anat.}{}{}{s.m.}{Porção inferior e posterior do osso ilíaco.}{ís.quio}{0}
\verb{ísquion}{}{}{}{}{}{Var. de \textit{ísquio}.}{ís.quion}{0}
\verb{israelense}{}{}{}{}{adj.2g.}{Relativo ao Estado de Israel.}{is.ra.e.len.se}{0}
\verb{israelense}{}{}{}{}{s.2g.}{Indivíduo natural ou habitante desse Estado.}{is.ra.e.len.se}{0}
\verb{israelita}{}{}{}{}{adj.2g.}{Relativo ao povo descendente do patriarca bíblico Jacó, também chamado Israel.}{is.ra.e.li.ta}{0}
\verb{israelita}{}{}{}{}{s.2g.}{Indivíduo pertencente a esse povo.}{is.ra.e.li.ta}{0}
\verb{issei}{}{}{}{}{adj.2g.}{Diz"-se do japonês que emigra para o continente americano.}{is.sei}{0}
\verb{isso}{}{}{}{}{pron.}{Essa coisa.}{is.so}{0}
\verb{isso}{}{}{}{}{}{Essa pessoa, em sentido depreciativo.}{is.so}{0}
\verb{isso}{}{}{}{}{interj.}{Expressão que denota aprovação.}{is.so}{0}
\verb{istmo}{}{Geogr.}{}{}{s.m.}{Faixa estreita de terra que liga dois continentes, ou uma península a um continente.}{ist.mo}{0}
\verb{isto}{}{}{}{}{pron.}{Esta coisa.}{is.to}{0}
\verb{isto}{}{}{}{}{}{Esta pessoa, em sentido depreciativo.}{is.to}{0}
\verb{isto}{}{}{}{}{interj.}{Expressão que denota apoio, concordância.}{is.to}{0}
\verb{italianismo}{}{}{}{}{s.m.}{Imitação da língua ou dos costumes italianos.}{i.ta.li.a.nis.mo}{0}
\verb{italianismo}{}{}{}{}{}{Expressão ou construção própria do italiano emprestada a uma outra língua.}{i.ta.li.a.nis.mo}{0}
\verb{italiano}{}{}{}{}{adj.}{Relativo à Itália.}{i.ta.li.a.no}{0}
\verb{italiano}{}{}{}{}{s.m.}{Indivíduo natural ou habitante desse país.}{i.ta.li.a.no}{0}
\verb{italiano}{}{}{}{}{}{Língua oficial da Itália, falada também na Suíça italiana e em São Marinho na Europa Meridional.}{i.ta.li.a.no}{0}
\verb{itálico}{}{}{}{}{adj.}{Relativo à Itália antiga.}{i.tá.li.co}{0}
\verb{itálico}{}{}{}{}{}{Diz"-se do caractere de impressão levemente inclinado para a direita.}{i.tá.li.co}{0}
\verb{ítalo}{}{}{}{}{adj.}{Relativo à Itália antiga ou atual.}{í.ta.lo}{0}
\verb{ítalo}{}{}{}{}{}{Italiano, romano, latino.}{í.ta.lo}{0}
\verb{ítalo}{}{}{}{}{s.m.}{Indivíduo habitante ou natural da Itália.}{í.ta.lo}{0}
\verb{itaoca}{ó}{}{}{}{s.f.}{Furna, lapa, caverna.}{i.ta.o.ca}{0}
\verb{itapeba}{é}{}{}{}{s.f.}{Recife de pedra que corre paralelamente à margem do rio.}{i.ta.pe.ba}{0}
\verb{itapeva}{é}{}{}{}{}{Var. de \textit{itapeba}.}{i.ta.pe.va}{0}
\verb{itararé}{}{}{}{}{s.m.}{Curso subterrâneo das águas de um rio através de rochas calcárias.}{i.ta.ra.ré}{0}
\verb{ité}{}{Bras.}{}{}{adj.}{Que não tem gosto; insípido.}{i.té}{0}
\verb{item}{}{}{}{}{s.m.}{Cada um dos artigos ou argumentos de uma exposição escrita, de um requerimento, de um regulamento, de um contrato etc.}{i.tem}{0}
\verb{iteração}{}{}{"-ões}{}{s.f.}{Ato ou efeito de iterar; repetição.}{i.te.ra.ção}{0}
\verb{iterar}{}{}{}{}{v.t.}{Tornar a fazer ou a dizer; repetir, reiterar.}{i.te.rar}{\verboinum{1}}
\verb{iterativo}{}{}{}{}{adj.}{Que é repetido, reiterado; feito mais de uma vez; frequente.}{i.te.ra.ti.vo}{0}
\verb{itérbio}{}{Quím.}{}{}{s.m.}{Elemento químico metálico, sólido, brilhante, prateado, estável ao ar, da família dos lantanídeos; usado em \textit{lasers} e aparelhos de raios \textsc{x}. \elemento{70}{173.04}{Yb}.}{i.tér.bio}{0}
\verb{iterícia}{}{}{}{}{}{Var. de \textit{icterícia}.}{i.te.rí.cia}{0}
\verb{itinerante}{}{}{}{}{adj.2g.}{Que viaja, que percorre itinerários.}{i.ti.ne.ran.te}{0}
\verb{itinerante}{}{}{}{}{}{Que se desloca de lugar em lugar no exercício de uma função.}{i.ti.ne.ran.te}{0}
\verb{itinerante}{}{}{}{}{s.2g.}{Indivíduo itinerante.}{i.ti.ne.ran.te}{0}
\verb{itinerário}{}{}{}{}{adj.}{Relativo a caminhos.}{i.ti.ne.rá.rio}{0}
\verb{itinerário}{}{}{}{}{s.m.}{Descrição de viagem; roteiro.}{i.ti.ne.rá.rio}{0}
\verb{itinerário}{}{}{}{}{}{Caminho que se vai percorrer, ou se percorreu.}{i.ti.ne.rá.rio}{0}
\verb{itinerário}{}{}{}{}{}{Caminho, trajeto, percurso.}{i.ti.ne.rá.rio}{0}
\verb{itororó}{}{}{}{}{s.m.}{Pequena cachoeira.}{i.to.ro.ró}{0}
\verb{ítrio}{}{Quím.}{}{}{s.m.}{Elemento químico metálico, prateado, usado na fabricação de ímãs permanentes, reatores nucleares e semicondutores. \elemento{39}{88.90585}{Y}.}{í.trio}{0}
\verb{iugoslavo}{}{}{}{}{adj.}{Relativo à Iugoslávia.}{i.u.gos.la.vo}{0}
\verb{iugoslavo}{}{}{}{}{s.m.}{Indivíduo natural ou habitante desse país.}{i.u.gos.la.vo}{0}
\verb{ixe}{ch}{}{}{}{interj.}{Expressão que denota ironia, desdém, desprezo.}{i.xe}{0}
