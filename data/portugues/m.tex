\verb{m}{}{}{}{}{s.m.}{Décima terceira letra do alfabeto português.}{m}{0}
\verb{m}{}{}{}{}{}{Abrev. de \textit{metro}.}{m}{0}
\verb{M}{}{}{}{}{}{Algarismo romano equivalente a \textit{1000}.}{M}{0}
\verb{ma}{}{}{}{}{}{Contração dos pronomes \textit{me} e \textit{a}.}{ma}{0}
\verb{má}{}{}{}{}{adj.}{Feminino de \textit{mau}.}{má}{0}
\verb{maca}{}{}{}{}{s.f.}{Cama especial para transportar doente ou ferido.}{ma.ca}{0}
\verb{maça}{}{}{}{}{s.f.}{Pau pesado, mais grosso numa ponta, que era usado como arma; clava.}{ma.ça}{0}
\verb{maça}{}{}{}{}{}{Arma de ferro ou madeira, com uma extremidade em forma de bola cheia de pontas afiadas.}{ma.ça}{0}
\verb{maçã}{}{}{}{}{s.f.}{Fruto comestível da macieira, de polpa branca e casca lisa e fina, avermelhada ou esverdeada quando madura.}{ma.çã}{0}
\verb{macabro}{}{}{}{}{adj.}{Que lembra a morte; fúnebre, trágico.}{ma.ca.bro}{0}
\verb{macaca}{}{Zool.}{}{}{s.f.}{Fêmea do macaco.}{ma.ca.ca}{0}
\verb{macaca}{}{}{}{}{}{Brincadeira em que a criança pula com um pé ou com dois, conforme aparecer à frente um ou dois quadrados riscados no chão, e pega uma pedra jogada num desses quadrados, sem pisar nele; amarelinha.}{ma.ca.ca}{0}
\verb{macacada}{}{}{}{}{s.f.}{Grupo de macacos.}{ma.ca.ca.da}{0}
\verb{macacada}{}{Pop.}{}{}{}{O pessoal, bando.}{ma.ca.ca.da}{0}
\verb{macaca"-de"-auditório}{}{}{macacas"-de"-auditório}{}{s.f.}{Moça que participa de programas de auditório com muito entusiasmo.}{ma.ca.ca"-de"-au.di.tó.rio}{0}
\verb{macacão}{}{}{"-ões}{}{s.m.}{Roupa, em geral de tecido grosso, com a calça e a camisa formando uma única peça.}{ma.ca.cão}{0}
\verb{macacaria}{}{}{}{}{s.f.}{Bando de macacos; macacada.}{ma.ca.ca.ri.a}{0}
\verb{macaco}{}{Zool.}{}{}{s.m.}{Animal mamífero fisicamente parecido com o homem, com cérebro desenvolvido e inteligência superior à dos outros animais.}{ma.ca.co}{0}
\verb{macaco}{}{}{}{}{}{Aparelho usado para levantar grandes pesos.}{ma.ca.co}{0}
\verb{macacoa}{ô}{}{}{}{s.f.}{Doença sem gravidade.}{ma.ca.co.a}{0}
\verb{maçada}{}{}{}{}{s.f.}{Pancada dada com a maça ou o maço.}{ma.ça.da}{0}
\verb{maçada}{}{}{}{}{}{Situação que aborrece muito; transtorno.}{ma.ça.da}{0}
\verb{macadame}{}{}{}{}{s.m.}{Calçamento de estrada com pedra britada.}{ma.ca.da.me}{0}
\verb{macadamizar}{}{}{}{}{v.t.}{Calçar estrada, rua, com macadame.}{ma.ca.da.mi.zar}{\verboinum{1}}
\verb{maçador}{ô}{}{}{}{adj.}{Diz"-se de indivíduo irritante; maçante.}{ma.ça.dor}{0}
\verb{macambira}{}{Bot.}{}{}{s.f.}{Planta de folhas rígidas e espinhosas, muito dispersa nas regiões secas nordestinas, onde o povo prepara com as folhas dela uma espécie de pão.}{ma.cam.bi.ra}{0}
\verb{macambúzio}{}{}{}{}{adj.}{Diz"-se de quem se mostra triste, silencioso.}{ma.cam.bú.zio}{0}
\verb{maçaneta}{ê}{}{}{}{s.f.}{Puxador em porta ou janela.}{ma.ça.ne.ta}{0}
\verb{maçante}{}{}{}{}{adj.2g.}{Que maça, entedia, aborrece.}{ma.çan.te}{0}
\verb{macapaense}{}{}{}{}{adj.2g.}{Relativo a Macapá, capital do Amapá.}{ma.ca.pa.en.se}{0}
\verb{macapaense}{}{}{}{}{s.2g.}{Indivíduo natural ou habitante dessa cidade.}{ma.ca.pa.en.se}{0}
\verb{maçapão}{}{Cul.}{"-ães}{}{s.m.}{Bolo de farinha de trigo com pasta de amêndoa e açúcar.}{ma.ça.pão}{0}
\verb{macaqueação}{}{}{"-ões}{}{s.f.}{Macaquice.}{ma.ca.que.a.ção}{0}
\verb{macaquear}{}{}{}{}{v.t.}{Reproduzir por imitação; arremedar, imitar.}{ma.ca.que.ar}{\verboinum{4}}
\verb{macaquice}{}{}{}{}{s.f.}{Ato ou efeito de macaquear; macaqueação.}{ma.ca.qui.ce}{0}
\verb{macaquice}{}{}{}{}{}{Postura ou gesticulação que provoca riso, semelhante à dos macacos.}{ma.ca.qui.ce}{0}
\verb{macaquice}{}{}{}{}{}{Comportamento hipócrita e interesseiro; carinho interesseiro; adulação.}{ma.ca.qui.ce}{0}
\verb{maçar}{}{}{}{}{v.t.}{Bater com maço ou maça.}{ma.çar}{0}
\verb{maçar}{}{}{}{}{}{Bater ou golpear com pau ou outro instrumento.}{ma.çar}{0}
\verb{maçar}{}{}{}{}{}{Enfadar, repetindo assuntos, conversas etc.; aborrecer, importunar, amolar.}{ma.çar}{\verboinum{3}}
\verb{maçaranduba}{}{Bot.}{}{}{s.f.}{Árvore de madeira vermelha e dura, usada em obras externas.}{ma.ça.ran.du.ba}{0}
\verb{macaréu}{}{}{}{}{s.m.}{Elevação brusca das águas do mar no sentido oposto ao do fluxo das águas de um rio, a qual sobe rio acima devastando as margens.}{ma.ca.réu}{0}
\verb{maçarico}{}{}{}{}{s.m.}{Aparelho de solda que lança chama.}{ma.ça.ri.co}{0}
\verb{maçarico}{}{Zool.}{}{}{}{Ave ribeirinha.}{ma.ça.ri.co}{0}
\verb{maçaroca}{ó}{}{}{}{s.f.}{Porção de fio torcido e enrolado no fuso.}{ma.ça.ro.ca}{0}
\verb{maçaroca}{ó}{}{}{}{}{Emaranhado.}{ma.ça.ro.ca}{0}
\verb{macarrão}{}{}{"-ões}{}{s.m.}{Massa de farinha de trigo, com ou sem ovos, que se corta em fios, em canudinhos ou em outros formatos.}{ma.car.rão}{0}
\verb{macarronada}{}{Cul.}{}{}{s.f.}{Iguaria de macarrão cozido a que se acrescentam queijo, manteiga e molho de tomate ou qualquer outro molho.}{ma.car.ro.na.da}{0}
\verb{macarrônico}{}{}{}{}{adj.}{Gênero de poesia ou prosa no qual se mesclam ironicamente à lingua original palavras de outra língua.}{ma.car.rô.ni.co}{0}
\verb{macarrônico}{}{}{}{}{}{Diz"-se de idioma mal escrito ou falado.}{ma.car.rô.ni.co}{0}
\verb{macaxeira}{ch}{}{}{}{s.f.}{Mandioca.}{ma.ca.xei.ra}{0}
\verb{macaxera}{chê}{}{}{}{}{Var. de \textit{macaxeira}.}{ma.ca.xe.ra}{0}
\verb{macedônio}{}{}{}{}{adj.}{Relativo à Macedônia (Europa).}{ma.ce.dô.nio}{0}
\verb{macedônio}{}{}{}{}{s.m.}{Indivíduo natural ou habitante desse país.}{ma.ce.dô.nio}{0}
\verb{macega}{é}{Bot.}{}{}{s.f.}{Erva daninha que cresce nas searas.}{ma.ce.ga}{0}
\verb{macega}{é}{Bras.}{}{}{}{Capim muito alto que dificulta o trânsito.}{ma.ce.ga}{0}
\verb{macegal}{}{}{"-ais}{}{s.m.}{Extensão considerável de terreno coberto de macega.}{ma.ce.gal}{0}
\verb{maceió}{}{}{}{}{s.m.}{Lagoeiro que se forma no litoral em virtude das marés e das águas pluviais.}{ma.cei.ó}{0}
\verb{maceioense}{}{}{}{}{adj.2g.}{Relativo a Maceió, capital de Alagoas.}{ma.cei.o.en.se}{0}
\verb{maceioense}{}{}{}{}{s.2g.}{Indivíduo natural ou habitante dessa cidade.}{ma.cei.o.en.se}{0}
\verb{macela}{é}{Bot.}{}{}{s.f.}{Erva aromática de florzinha amarela.}{ma.ce.la}{0}
\verb{maceração}{}{}{"-ões}{}{s.f.}{Ato ou efeito de macerar.}{ma.ce.ra.ção}{0}
\verb{maceração}{}{}{"-ões}{}{}{Operação que consiste em pôr uma substância sólida num líquido para que este se impregne dos princípios solúveis daquela substância.}{ma.ce.ra.ção}{0}
\verb{maceração}{}{Fig.}{"-ões}{}{}{Mortificação do corpo por meio de jejuns, penitências etc.}{ma.ce.ra.ção}{0}
\verb{macerado}{}{}{}{}{adj.}{Que sofreu maceração.}{ma.ce.ra.do}{0}
\verb{macerado}{}{}{}{}{}{Mortificado.}{ma.ce.ra.do}{0}
\verb{macerado}{}{}{}{}{}{Aflito, desgostoso.}{ma.ce.ra.do}{0}
\verb{macerar}{}{}{}{}{v.t.}{Submeter uma substância sólida à maceração}{ma.ce.rar}{0}
\verb{macerar}{}{Fig.}{}{}{}{Mortificar, torturar.}{ma.ce.rar}{\verboinum{1}}
\verb{macérrimo}{}{}{}{}{adj.}{Superlativo absoluto sintético de magro; extremamente magro; magérrimo.}{ma.cér.ri.mo}{0}
\verb{maceta}{ê}{}{}{}{s.f.}{Cilindro para moer tintas.}{ma.ce.ta}{0}
\verb{maceta}{ê}{}{}{}{}{Pequeno martelo.}{ma.ce.ta}{0}
\verb{macetar}{}{}{}{}{v.t.}{Golpear com maceta.}{ma.ce.tar}{\verboinum{1}}
\verb{macete}{ê}{}{}{}{s.m.}{Pequeno maço.}{ma.ce.te}{0}
\verb{macete}{ê}{}{}{}{}{Maço de escultores.}{ma.ce.te}{0}
\verb{macete}{ê}{}{}{}{}{Pequeno maço de madeira com que os carpinteiros e marceneiros batem sobre o cabo do formão. }{ma.ce.te}{0}
\verb{macete}{ê}{Pop.}{}{}{}{Recurso astucioso para facilitar a execução de alguma tarefa; truque.}{ma.ce.te}{0}
\verb{machadada}{}{}{}{}{s.f.}{Golpe de machado.}{ma.cha.da.da}{0}
\verb{machadiano}{}{}{}{}{adj.}{Relativo ao escritor Machado de Assis.}{ma.cha.di.a.no}{0}
\verb{machadiano}{}{}{}{}{}{Diz"-se de estudioso de sua obra.}{ma.cha.di.a.no}{0}
\verb{machadinha}{}{}{}{}{s.f.}{Tipo de machado pequeno e largo.}{ma.cha.di.nha}{0}
\verb{machadinho}{}{}{}{}{s.m.}{Pequeno machado; machadinha.}{ma.cha.di.nho}{0}
\verb{machado}{}{}{}{}{s.m.}{Instrumento cortante, com cabo, usado para cortar árvores ou rachar lenha.}{ma.cha.do}{0}
\verb{machão}{}{}{"-ões}{}{adj.}{Diz"-se de homem que se mostra excessivamente orgulhoso de sua condição masculina; que alardeia ridiculamente sua masculinidade.}{ma.chão}{0}
\verb{machismo}{}{}{}{}{s.m.}{Atitude de machão.}{ma.chis.mo}{0}
\verb{machista}{}{}{}{}{adj.2g.}{Diz"-se de pessoa partidária do machismo, que acha as mulheres inferiores aos homens.}{ma.chis.ta}{0}
\verb{macho}{}{}{}{}{s.m.}{Animal do sexo masculino.}{ma.cho}{0}
\verb{macho}{}{}{}{}{}{Homem.}{ma.cho}{0}
\verb{macho}{}{}{}{}{}{Tipo de dobra de costura.}{ma.cho}{0}
\verb{machucado}{}{}{}{}{s.m.}{Resultado de uma batida do corpo contra alguma coisa; ferimento, contusão.}{ma.chu.ca.do}{0}
\verb{machucado}{}{}{}{}{adj.}{Que sofreu machucadura.}{ma.chu.ca.do}{0}
\verb{machucado}{}{}{}{}{}{Que está magoado, triste.}{ma.chu.ca.do}{0}
\verb{machucadura}{}{}{}{}{s.f.}{Ferida, machucado.}{ma.chu.ca.du.ra}{0}
\verb{machucar}{}{}{}{}{v.t.}{Produzir um machucado no corpo; ferir.}{ma.chu.car}{0}
\verb{machucar}{}{}{}{}{}{Esmagar alguma coisa apertando; amassar, macerar.}{ma.chu.car}{0}
\verb{machucar}{}{}{}{}{}{Magoar.}{ma.chu.car}{\verboinum{2}}
\verb{maciço}{}{}{}{}{adj.}{Feito em peça única de material compacto e sem vãos no interior.}{ma.ci.ço}{0}
\verb{maciço}{}{}{}{}{}{Sólido, espesso, consistente, denso.}{ma.ci.ço}{0}
\verb{maciço}{}{}{}{}{}{Relativo a um conjunto grande de pessoas.}{ma.ci.ço}{0}
\verb{macieira}{ê}{Bot.}{}{}{s.f.}{Árvore que dá a maçã.}{ma.ci.ei.ra}{0}
\verb{maciez}{ê}{}{}{}{s.f.}{Qualidade de macio.}{ma.ci.ez}{0}
\verb{macieza}{ê}{}{}{}{s.f.}{Maciez.}{ma.ci.e.za}{0}
\verb{macilento}{}{}{}{}{adj.}{Com pouca carne; magro, descarnado.}{ma.ci.len.to}{0}
\verb{macilento}{}{Fig.}{}{}{}{Sem brilho; mortiço.}{ma.ci.len.to}{0}
\verb{macio}{}{}{}{}{adj.}{Que cede ao ser apertado; mole, tenro.}{ma.ci.o}{0}
\verb{macio}{}{}{}{}{}{Não áspero; liso, aveludado.}{ma.ci.o}{0}
\verb{macio}{}{Fig.}{}{}{}{Ameno, delicado, meigo.}{ma.ci.o}{0}
\verb{maciota}{ó}{}{}{}{s.f.}{Maciez, delicadeza, suavidade.}{ma.ci.o.ta}{0}
\verb{maço}{}{}{}{}{s.m.}{Conjunto de objetos reunidos amarrados ou colocados em um mesmo invólucro.}{ma.ço}{0}
\verb{maço}{}{}{}{}{}{Ferramenta feita de uma peça maciça de madeira presa a um cabo, usada de maneira semelhante ao martelo para bater em ferramentas de entalhe.}{ma.ço}{0}
\verb{maçom}{}{}{"-ons}{}{adj.}{Diz"-se de indivíduo ligado à maçonaria.}{ma.çom}{0}
\verb{maçonaria}{}{}{}{}{s.f.}{Sociedade parcialmente secreta com princípios de fraternidade e que tem como objetivo principal praticar a filantropia.}{ma.ço.na.ri.a}{0}
\verb{maconha}{}{}{}{}{s.f.}{Fumo preparado com as folhas e ramos do cânhamo, com propriedades relaxantes, analgésicas e alucinógenas, e cujo consumo é proibido por lei no Brasil.}{ma.co.nha}{0}
\verb{maconheiro}{ê}{}{}{}{adj.}{Consumidor de maconha.}{ma.co.nhei.ro}{0}
\verb{maçônico}{}{}{}{}{adj.}{Relativo à maçonaria.}{ma.çô.ni.co}{0}
\verb{má"-criação}{}{}{má"-criações \textit{ou} más"-criações}{}{s.f.}{Ato de uma pessoa mal"-educada; grosseria, malcriação.}{má"-cri.a.ção}{0}
\verb{macro}{}{Gram.}{}{}{s.m.}{Sinal gráfico em forma de traço horizontal que se coloca sobre vogais para indicar duração longa.}{ma.cro}{0}
\verb{macro}{}{Informát.}{}{}{s.f.}{Sequência de comandos executados automaticamente por um aplicativo.}{ma.cro}{0}
\verb{macróbio}{}{}{}{}{adj.}{Diz"-se de indivíduo de idade muito avançada.}{ma.cró.bio}{0}
\verb{macrobiótica}{}{}{}{}{s.f.}{Estudo e conjunto de práticas que buscam aprimorar a saúde integral do ser humano e proporcionar vida longa.}{ma.cro.bi.ó.ti.ca}{0}
\verb{macrobiótico}{}{}{}{}{adj.}{Relativo à macrobiótica.}{ma.cro.bi.ó.ti.co}{0}
\verb{macrocefalia}{}{Med.}{}{}{s.f.}{Qualidade de quem tem o crânio muito desenvolvido.}{ma.cro.ce.fa.li.a}{0}
\verb{macrocefálico}{}{}{}{}{adj.}{Relativo à macrocefalia.}{ma.cro.ce.fá.li.co}{0}
\verb{macrocéfalo}{}{}{}{}{adj.}{Que apresenta macrocefalia.}{ma.cro.cé.fa.lo}{0}
\verb{macrocosmo}{ó}{}{}{}{s.m.}{O Universo, considerado como ser orgânico, à imagem do ser humano.}{ma.cro.cos.mo}{0}
\verb{macroeconomia}{}{}{}{}{s.f.}{Ramo da economia que estuda os fenômenos econômicos de uma nação ou de um determinado setor da economia.}{ma.cro.e.co.no.mi.a}{0}
\verb{macro"-jê}{}{}{}{}{s.m.}{Tronco linguístico que compreende nove famílias linguísticas que se estendem desde o Pará até o Rio Grande do Sul.   }{ma.cro"-jê}{0}
\verb{macroscópico}{}{}{}{}{adj.}{Que pode ser observado a olho nu.}{ma.cros.có.pi.co}{0}
\verb{macroscópico}{}{}{}{}{}{Sem detalhes; superficial.}{ma.cros.có.pi.co}{0}
\verb{macuco}{}{Bras.}{}{}{s.m.}{Ave de grande porte, cauda pequena e dorso pardo, atualmente ameaçada de extinção.}{ma.cu.co}{0}
\verb{maçudo}{}{}{}{}{adj.}{Semelhante a uma maça.}{ma.çu.do}{0}
\verb{maçudo}{}{}{}{}{}{Monótono, maçante, entediante.}{ma.çu.do}{0}
\verb{mácula}{}{}{}{}{s.f.}{Marca de sujeira; mancha.}{má.cu.la}{0}
\verb{mácula}{}{Fig.}{}{}{}{Mancha moral; infâmia.}{má.cu.la}{0}
\verb{maculado}{}{}{}{}{adj.}{Manchado.}{ma.cu.la.do}{0}
\verb{maculado}{}{Fig.}{}{}{}{Infamado, desonrado.}{ma.cu.la.do}{0}
\verb{macular}{}{}{}{}{v.t.}{Sujar, manchar.}{ma.cu.lar}{0}
\verb{macular}{}{}{}{}{}{Comprometer a honra; infamar.}{ma.cu.lar}{\verboinum{1}}
\verb{macumba}{}{}{}{}{s.f.}{Culto religioso afro"-brasileiro com influências do candomblé, catolicismo e espiritismo.}{ma.cum.ba}{0}
\verb{macumba}{}{}{}{}{}{Oferenda feita a Exu, geralmente em encruzilhadas; despacho.}{ma.cum.ba}{0}
\verb{macumba}{}{Mús.}{}{}{}{Antigo instrumento de percussão de origem africana, semelhante ao reco"-reco.}{ma.cum.ba}{0}
\verb{macumbeiro}{ê}{}{}{}{s.m.}{Praticante ou frequentador de macumba.}{ma.cum.bei.ro}{0}
\verb{macumbeiro}{ê}{Mús.}{}{}{}{Tocador de macumba.}{ma.cum.bei.ro}{0}
%\verb{}{}{}{}{}{}{}{}{0}
%\verb{}{}{}{}{}{}{}{}{0}
%\verb{}{}{}{}{}{}{}{}{0}
\verb{madagascarense}{}{}{}{}{adj.2g. e s.2g.}{Malgaxe.}{ma.da.gas.ca.ren.se}{0}
\verb{madama}{}{Pop.}{}{}{s.f.}{Madame.}{ma.da.ma}{0}
\verb{madame}{}{}{}{}{s.f.}{Mulher adulta; senhora, dama.}{ma.da.me}{0}
\verb{madame}{}{Pop.}{}{}{}{Mulher adulta de hábitos petulantes, consumistas e exibicionistas.}{ma.da.me}{0}
\verb{madame}{}{Pop.}{}{}{}{Meretriz, prostituta.}{ma.da.me}{0}
\verb{madeira}{ê}{}{}{}{s.f.}{Parte dura que forma a raiz, o tronco e os galhos de uma árvore. (\textit{Há ainda muita extração e venda ilegal de madeira no Brasil.})}{ma.dei.ra}{0}
\verb{madeirame}{}{}{}{}{s.m.}{Madeiramento.}{ma.dei.ra.me}{0}
\verb{madeiramento}{}{}{}{}{s.m.}{Conjunto de madeiras, especialmente os usados em uma construção.}{ma.dei.ra.men.to}{0}
\verb{madeiramento}{}{}{}{}{}{A estrutura de madeira de uma edificação ou de parte dela.}{ma.dei.ra.men.to}{0}
\verb{madeireira}{ê}{}{}{}{s.f.}{Estabelecimento que produz ou comercializa chapas e artefatos de madeira.}{ma.dei.rei.ra}{0}
\verb{madeireiro}{ê}{}{}{}{s.m.}{Indivíduo que negocia ou trabalha na produção de madeira.}{ma.dei.rei.ro}{0}
\verb{madeireiro}{ê}{}{}{}{adj.}{Relativo a indústria e comércio de madeira.}{ma.dei.rei.ro}{0}
\verb{madeiro}{ê}{}{}{}{s.m.}{Peça de madeira robusta; lenho.}{ma.dei.ro}{0}
\verb{madeiro}{ê}{}{}{}{}{Tronco de madeira que sustenta as vigas do teto de uma casa.}{ma.dei.ro}{0}
\verb{madeixa}{ch}{}{}{}{s.f.}{Pequeno novelo de linha para costura.}{ma.dei.xa}{0}
\verb{madeixa}{ch}{}{}{}{}{Feixe de cabelos; mecha, cacho.}{ma.dei.xa}{0}
\verb{madona}{}{Relig.}{}{}{s.f.}{A mãe de Jesus Cristo [usa"-se com maiúscula inicial nesta acepção].}{ma.do.na}{0}
\verb{madona}{}{}{}{}{}{Pintura ou escultura que representa a mãe de Jesus Cristo.}{ma.do.na}{0}
\verb{madorna}{ó}{}{}{}{s.f.}{Modorra.}{ma.dor.na}{0}
\verb{madraço}{}{}{}{}{adj.}{Que se dedica pouco a suas atividades; preguiçoso.}{ma.dra.ço}{0}
\verb{madrasta}{}{}{}{}{s.f.}{A nova mulher do pai.}{ma.dras.ta}{0}
\verb{madre}{}{}{}{}{s.f.}{Mãe.}{ma.dre}{0}
\verb{madre}{}{}{}{}{}{Membro feminino de um convento; freira, irmã.}{ma.dre}{0}
\verb{madrepérola}{}{}{}{}{s.f.}{Substância calcária encontrada na concha de alguns moluscos, utilizada na confecção de pequenos objetos.}{ma.dre.pé.ro.la}{0}
\verb{madressilva}{}{Bot.}{}{}{s.f.}{Arbusto com flores aromáticas e bagas vermelhas, cultivado como ornamental.}{ma.dres.sil.va}{0}
\verb{madrigal}{}{Liter.}{"-ais}{}{s.m.}{Composição poética sucinta, com temática amorosa e galante.}{ma.dri.gal}{0}
\verb{madrigal}{}{}{"-ais}{}{}{Poesia pastoril.}{ma.dri.gal}{0}
\verb{madrigal}{}{}{"-ais}{}{}{Galanteio.}{ma.dri.gal}{0}
\verb{madrileno}{}{}{}{}{adj.}{Relativo a Madri, capital da Espanha.}{ma.dri.le.no}{0}
\verb{madrileno}{}{}{}{}{s.m.}{Indivíduo natural ou habitante dessa cidade.}{ma.dri.le.no}{0}
\verb{madrilense}{}{}{}{}{adj.2g. e s.2g.}{Madrileno.}{ma.dri.len.se}{0}
\verb{madrinha}{}{}{}{}{s.f.}{Mulher que, na cerimônia de batismo ou crisma, assume o compromisso de exercer as atribuições dos pais na ausência deles.}{ma.dri.nha}{0}
\verb{madrinha}{}{}{}{}{}{Mulher que, no ritual de casamento civil ou religioso, é escolhida por um dos noivos como testemunha.}{ma.dri.nha}{0}
\verb{madrinha}{}{}{}{}{}{Mulher que representa o espírito de uma entidade, grupo ou corporação.}{ma.dri.nha}{0}
\verb{madrugada}{}{}{}{}{s.f.}{Período entre meia"-noite e o amanhecer.}{ma.dru.ga.da}{0}
\verb{madrugada}{}{Fig.}{}{}{}{Os momentos iniciais de algo.}{ma.dru.ga.da}{0}
\verb{madrugada}{}{}{}{}{}{Ato de madrugar.}{ma.dru.ga.da}{0}
\verb{madrugador}{ô}{}{}{}{adj.}{Que acorda cedo, madruga.}{ma.dru.ga.dor}{0}
\verb{madrugar}{}{}{}{}{v.i.}{Levantar"-se da cama muito cedo ou antes da hora habitual.}{ma.dru.gar}{0}
\verb{madrugar}{}{}{}{}{}{Iniciar"-se o dia; amanhecer.}{ma.dru.gar}{0}
\verb{madrugar}{}{}{}{}{}{Fazer algo ou surgir antes do tempo próprio.}{ma.dru.gar}{\verboinum{5}}
\verb{maduração}{}{}{"-ões}{}{s.f.}{Ato ou efeito de amadurecer; maturação, amadurecimento.}{ma.du.ra.ção}{0}
\verb{madurar}{}{}{}{}{v.t.}{Tornar maduro; amadurecer.}{ma.du.rar}{\verboinum{1}}
\verb{madurecer}{ê}{}{}{}{v.t.}{Amadurecer.}{ma.du.re.cer}{\verboinum{15}}
\verb{madureza}{ê}{}{}{}{s.f.}{Qualidade de maduro.}{ma.du.re.za}{0}
\verb{madureza}{ê}{Fig.}{}{}{}{Prudência, maturidade, sensatez.}{ma.du.re.za}{0}
\verb{maduro}{}{}{}{}{adj.}{Diz"-se de produto vegetal próprio para ser utilizado como alimento.}{ma.du.ro}{0}
\verb{maduro}{}{Fig.}{}{}{}{Diz"-se de qualquer coisa que atingiu seu desenvolvimento completo.}{ma.du.ro}{0}
\verb{maduro}{}{Fig.}{}{}{}{Ponderado, refletido, prudente.}{ma.du.ro}{0}
\verb{mãe}{}{}{}{}{s.f.}{Mulher que deu à luz um ou mais filhos.}{mãe}{0}
\verb{mãe}{}{}{}{}{}{Fêmea de animal que teve cria.}{mãe}{0}
\verb{mãe}{}{Fig.}{}{}{}{Origem, causa.}{mãe}{0}
\verb{mãe"-benta}{}{Cul.}{mães"-bentas}{}{s.f.}{Espécie de bolo pequeno, feito com farinha de arroz, manteiga, açúcar e ovos. }{mãe"-ben.ta}{0}
\verb{mãe"-d'água}{}{}{mães"-d'água}{}{s.f.}{Reservatório de água com uma nascente.}{mãe"-d'á.gua}{0}
\verb{mãe"-d'água}{}{}{mães"-d'água}{}{}{Ser mitológico, espécie de sereia de rios e lagos, também conhecida como iara.}{mãe"-d'á.gua}{0}
\verb{mãe"-de"-santo}{}{}{mães"-de"-santo}{}{s.f.}{Mulher que dirige as cerimônias da umbanda.}{mãe"-de"-san.to}{0}
\verb{maestria}{}{}{}{}{s.f.}{Qualidade de mestre; sabedoria, perícia.}{ma.es.tri.a}{0}
\verb{maestrina}{}{}{}{}{s.f.}{Forma feminina de \textit{maestro}.}{ma.es.tri.na}{0}
\verb{maestro}{é}{}{}{}{s.m.}{Indivíduo que rege orquestra, banda ou coro.}{ma.es.tro}{0}
\verb{maestro}{é}{}{}{}{}{Compositor musical.}{ma.es.tro}{0}
\verb{má"-fé}{}{}{más"-fés}{}{s.f.}{Má intenção; intenção dolosa. (\textit{Os rapazes agiram de má"-fé quando convidaram o vizinho para ajudá"-los.})}{má"-fé}{0}
\verb{máfia}{}{}{}{}{s.f.}{Sociedade secreta existente na Itália no século \textsc{xix} que tinha como objetivo garantir a segurança pública.}{má.fia}{0}
\verb{máfia}{}{}{}{}{}{Grupo criminoso organizado.}{má.fia}{0}
\verb{máfia}{}{}{}{}{}{Qualquer grupo de pessoas que age de modo bastante unido exclusivamente em nome de seus próprios interesses.}{má.fia}{0}
\verb{mafioso}{ô}{}{"-osos ⟨ó⟩}{"-osa ⟨ó⟩}{adj.}{Relativo a máfia.}{ma.fi.o.so}{0}
\verb{mafioso}{ô}{}{"-osos ⟨ó⟩}{"-osa ⟨ó⟩}{s.m.}{Membro da máfia.}{ma.fi.o.so}{0}
\verb{má"-formação}{}{}{más"-formações}{}{s.f.}{Formação anormal ou defeituosa, de origem congênita ou hereditária; deformidade.}{má"-for.ma.ção}{0}
\verb{maga}{}{}{}{}{s.f.}{Mulher que domina técnicas de magia; feiticeira, mágica.}{ma.ga}{0}
\verb{maganão}{}{}{"-ões}{}{adj.}{Muito magano.}{ma.ga.não}{0}
\verb{magano}{}{}{}{}{adj.}{Travesso, atrevido, engraçado.}{ma.ga.no}{0}
\verb{magarefe}{é}{}{}{}{s.m.}{Indivíduo que abate as reses nos matadouros; carniceiro.}{ma.ga.re.fe}{0}
\verb{magarefe}{é}{Pop.}{}{}{}{Médico ou cirurgião inábil.}{ma.ga.re.fe}{0}
\verb{magazine}{}{}{}{}{s.m.}{Estabelecimento comercial que vende grande variedade de mercadorias; loja.}{ma.ga.zi.ne}{0}
\verb{magazine}{}{}{}{}{}{Publicação periódica em formato de revista e que trata de assuntos variados.}{ma.ga.zi.ne}{0}
\verb{magérrimo}{}{}{}{}{adj.}{Forma superlativa absoluta sintética de \textit{magro}; muito magro, magríssimo.}{ma.gér.ri.mo}{0}
\verb{magia}{}{}{}{}{s.f.}{Ciência oculta que busca produzir fenômenos extraordinários por meio de atos, palavras e interferências de seres sobrenaturais.}{ma.gi.a}{0}
\verb{magia}{}{Fig.}{}{}{}{Magnetismo, encanto, fascínio.}{ma.gi.a}{0}
\verb{magiar}{}{}{}{}{adj.2g.}{Relativo à Hungria; húngaro.}{ma.gi.ar}{0}
\verb{magiar}{}{}{}{}{s.2g.}{Indivíduo natural ou habitante desse país.  }{ma.gi.ar}{0}
\verb{magiar}{}{}{}{}{}{O idioma falado na Hungria. }{ma.gi.ar}{0}
\verb{mágica}{}{}{}{}{s.f.}{Técnica de ilusionismo por meio de qual se transfigura, faz aparecer ou desaparecer coisas, animais ou pessoas; prestidigitação.}{má.gi.ca}{0}
\verb{mágica}{}{}{}{}{}{Ato ou efeito produzido por essa técnica.}{má.gi.ca}{0}
\verb{mágica}{}{}{}{}{}{Maga.}{má.gi.ca}{0}
\verb{mágico}{}{}{}{}{adj.}{Relativo a magia.}{má.gi.co}{0}
\verb{mágico}{}{}{}{}{}{Sobrenatural, fantástico, extraordinário.}{má.gi.co}{0}
\verb{mágico}{}{}{}{}{}{Que fascina; encantador, sedutor.}{má.gi.co}{0}
\verb{mágico}{}{}{}{}{s.m.}{Indivíduo que realiza mágicas; prestidigitador, ilusionista.}{má.gi.co}{0}
\verb{magistério}{}{}{}{}{s.m.}{Cargo de professor.}{ma.gis.té.rio}{0}
\verb{magistério}{}{}{}{}{}{A carreira de professor; ensino.}{ma.gis.té.rio}{0}
\verb{magistério}{}{}{}{}{}{A classe dos professores; professorado.}{ma.gis.té.rio}{0}
\verb{magistrado}{}{}{}{}{s.m.}{Indivíduo investido de autoridade que participa da administração política de um Estado.}{ma.gis.tra.do}{0}
\verb{magistrado}{}{}{}{}{}{Os altos cargos do Poder Judiciário.}{ma.gis.tra.do}{0}
\verb{magistral}{}{}{"-ais}{}{adj.2g.}{Relativo a mestre.}{ma.gis.tral}{0}
\verb{magistral}{}{}{"-ais}{}{}{Digno de um mestre; perfeito, exemplar.}{ma.gis.tral}{0}
\verb{magistratura}{}{}{}{}{s.f.}{Dignidade, função ou tempo de magistrado.}{ma.gis.tra.tu.ra}{0}
\verb{magistratura}{}{}{}{}{}{A classe dos magistrados.}{ma.gis.tra.tu.ra}{0}
\verb{magma}{}{}{}{}{s.m.}{Material mineral fundido, de consistência pastosa, encontrado nos vulcões e em grandes profundidades, e que se cristaliza ao esfriar, formando rochas ígneas.}{mag.ma}{0}
\verb{magma}{}{}{}{}{}{Qualquer substância pastosa e espessa que fica ao espremer algum material.}{mag.ma}{0}
\verb{magmático}{}{}{}{}{adj.}{Relativo a magma.}{mag.má.ti.co}{0}
\verb{magmático}{}{Geol.}{}{}{}{Diz"-se de rocha formada pela solidificação do magma.}{mag.má.ti.co}{0}
\verb{magnanimidade}{}{}{}{}{s.f.}{Qualidade de magnânimo.}{mag.na.ni.mi.da.de}{0}
\verb{magnanimidade}{}{}{}{}{}{Ato ou dito de quem é magnânimo.}{mag.na.ni.mi.da.de}{0}
\verb{magnânimo}{}{}{}{}{adj.}{De alma grandiosa; generoso, nobre.}{mag.nâ.ni.mo}{0}
\verb{magnata}{}{}{}{}{s.m.}{Indivíduo poderoso, influente ou extremamente rico.}{mag.na.ta}{0}
\verb{magnata}{}{}{}{}{}{Nome que era dado aos membros da alta nobreza na Polônia e Hungria.}{mag.na.ta}{0}
\verb{magnésia}{}{Quím.}{}{}{s.f.}{Óxido de magnésio, substância branca e cristalina usada como laxante e antiácido.}{mag.né.sia}{0}
\verb{magnésio}{}{Quím.}{}{}{s.m.}{Elemento químico metálico, branco"-prateado, maleável, dúctil, do grupo dos alcalino"-terrosos, utilizado em processos metalúrgicos e químicos, em fotografia e em pirotecnia, pela intensa luz branca que produz. \elemento{12}{4.305}{Mg}.}{mag.né.sio}{0}
\verb{magnete}{é}{}{}{}{s.m.}{Minério com propriedade de atrair metais; ímã.}{mag.ne.te}{0}
\verb{magnético}{}{}{}{}{adj.}{Relativo a magneto ou ímã.}{mag.né.ti.co}{0}
\verb{magnético}{}{}{}{}{}{Que tem a propriedade de atrair metais.}{mag.né.ti.co}{0}
\verb{magnético}{}{Fig.}{}{}{}{Sedutor, atraente.}{mag.né.ti.co}{0}
\verb{magnetismo}{}{Fís.}{}{}{s.m.}{Conjunto de propriedades características dos circuitos elétricos e das substâncias magnéticas.}{mag.ne.tis.mo}{0}
\verb{magnetismo}{}{Fig.}{}{}{}{Atração exercida por uma pessoa sobre outra; fascínio, sedução.}{mag.ne.tis.mo}{0}
\verb{magnetita}{}{Geol.}{}{}{s.f.}{Mineral preto e altamente magnético.}{mag.ne.ti.ta}{0}
\verb{magnetização}{}{}{"-ões}{}{s.f.}{Ato ou efeito de magnetizar.}{mag.ne.ti.za.ção}{0}
\verb{magnetizador}{ô}{}{}{}{adj.}{Que magnetiza.}{mag.ne.ti.za.dor}{0}
\verb{magnetizar}{}{}{}{}{v.t.}{Dar a um corpo propriedades magnéticas.}{mag.ne.ti.zar}{0}
\verb{magnetizar}{}{Fig.}{}{}{}{Exercer atração sobre alguém; seduzir, fascinar.}{mag.ne.ti.zar}{\verboinum{1}}
\verb{magneto}{ê}{Fís.}{}{}{s.m.}{Material que tem a propriedade de atrair metais; ímã.}{mag.ne.to}{0}
\verb{magnetômetro}{}{Fís.}{}{}{s.m.}{Instrumento para medir campos magnéticos.}{mag.ne.tô.me.tro}{0}
\verb{magnificação}{}{}{"-ões}{}{s.f.}{Ato ou efeito de magnificar.}{mag.ni.fi.ca.ção}{0}
\verb{magnificar}{}{}{}{}{v.t.}{Exaltar, glorificar.}{mag.ni.fi.car}{0}
\verb{magnificar}{}{}{}{}{}{Ampliar as dimensões; aumentar.}{mag.ni.fi.car}{\verboinum{2}}
\verb{magnificência}{}{}{}{}{s.f.}{Qualidade de magnificente; grandiosidade, esplendor.}{mag.ni.fi.cên.cia}{0}
\verb{magnificência}{}{}{}{}{}{Luxo, ostentação.}{mag.ni.fi.cên.cia}{0}
\verb{magnificente}{}{}{}{}{adj.2g.}{Grandioso, luxuoso, suntuoso.}{mag.ni.fi.cen.te}{0}
\verb{magnífico}{}{}{}{}{adj.}{Muito bom ou belo; excelente, formidável.}{mag.ní.fi.co}{0}
\verb{magnífico}{}{}{}{}{}{Magnificente.}{mag.ní.fi.co}{0}
\verb{magnitude}{}{}{}{}{s.f.}{Qualidade de magno; importância, grandeza.}{mag.ni.tu.de}{0}
\verb{magnitude}{}{Astron.}{}{}{}{Medida do brilho de um astro.}{mag.ni.tu.de}{0}
\verb{magno}{}{}{}{}{adj.}{Importante, grande, relevante.}{mag.no}{0}
\verb{magnólia}{}{Bot.}{}{}{s.f.}{Certa flor, grande e perfumada, existente nas cores branca, amarela, rosada ou purpúrea.}{mag.nó.lia}{0}
\verb{magnoliópsida}{}{Bot.}{}{}{s.f.}{Espécime das dicotiledôneas, classe das angiospermas caracterizada por plantas cuja semente possui dois cotilédones; dicotiledônea.}{mag.no.li.óp.si.da}{0}
\verb{mago}{}{}{}{}{s.m.}{Homem que pratica magia.}{ma.go}{0}
\verb{mágoa}{}{}{}{}{s.f.}{Desgosto, amargura, pesar.}{má.go.a}{0}
\verb{magoar}{}{}{}{}{v.t.}{Causar mágoa; desgostar, ofender.}{ma.go.ar}{\verboinum{7}}
\verb{magote}{ó}{}{}{}{s.m.}{Grande quantidade de pessoas; multidão, amontoado.}{ma.go.te}{0}
\verb{magrelo}{é}{Bras.}{}{}{adj.}{Excessivamente magro.}{ma.gre.lo}{0}
\verb{magreza}{ê}{}{}{}{s.f.}{Qualidade de magro.}{ma.gre.za}{0}
\verb{magricela}{é}{}{}{}{adj.}{Excessivamente magro.}{ma.gri.ce.la}{0}
\verb{magro}{}{}{}{}{adj.}{Em que há pouca ou nenhuma gordura.}{ma.gro}{0}
\verb{magro}{}{}{}{}{}{Pouco espesso; ralo.}{ma.gro}{0}
\verb{magro}{}{}{}{}{}{Comprido e estreito; fino.}{ma.gro}{0}
\verb{maguari}{}{Zool.}{}{}{s.m.}{Ave de plumagem branca e cauda preta, encontrada em várias regiões da América do Sul.}{ma.gua.ri}{0}
\verb{maia}{}{}{}{}{adj.2g.}{Relativo aos maias, povo da América Central, bastante desenvolvido até ser dizimado pela colonização europeia, e existente até os dias atuais com população reduzida.}{mai.a}{0}
\verb{maia}{}{}{}{}{s.m.}{Indivíduo desse povo.}{mai.a}{0}
\verb{maia}{}{}{}{}{}{A língua falada por esse povo.}{mai.a}{0}
\verb{maiêutica}{}{Filos.}{}{}{s.f.}{Método socrático de conduzir o interlocutor às descobertas por meio de sucessivas indagações.}{mai.êu.ti.ca}{0}
\verb{maiêutica}{}{Desus.}{}{}{}{Obstetrícia.}{mai.êu.ti.ca}{0}
\verb{maio}{}{}{}{}{s.m.}{O quinto mês do ano civil.}{mai.o}{0}
\verb{maiô}{}{}{}{}{s.m.}{Traje de banho feminino, em uma única peça, que cobre do busto ao alto das coxas.}{mai.ô}{0}
\verb{maiô}{}{}{}{}{}{Qualquer traje de banho.}{mai.ô}{0}
\verb{maionese}{é}{Cul.}{}{}{s.f.}{Molho de consistência pastosa, feito de óleo vegetal, sal e gema de ovo batidos juntos.}{mai.o.ne.se}{0}
\verb{maior}{ó}{}{}{}{adj.2g.}{Que supera o outro em número, tamanho, extensão, intensidade, duração, excelência.}{mai.or}{0}
\verb{maior}{ó}{}{}{}{}{Que atingiu a maioridade civil.}{mai.or}{0}
\verb{maior}{ó}{}{}{}{}{Que tem mais que determinada idade.}{mai.or}{0}
\verb{maior}{ó}{}{}{}{s.m.}{Indivíduo maior de idade.}{mai.or}{0}
\verb{maioral}{}{}{"-ais}{}{s.m.}{O chefe.}{mai.o.ral}{0}
\verb{maioral}{}{}{"-ais}{}{adj.2g.}{Que se distingue dos outros por ser superior a eles.}{mai.o.ral}{0}
\verb{maioria}{}{}{}{}{s.f.}{A maior parte de um conjunto ou de um todo.}{mai.o.ri.a}{0}
\verb{maioria}{}{}{}{}{}{O número de votos que dá a vitória em uma eleição.}{mai.o.ri.a}{0}
\verb{maioria}{}{}{}{}{}{O grupo de pessoas que conta com o maior número de votos favoráveis.}{mai.o.ri.a}{0}
\verb{maioria}{}{Desus.}{}{}{}{Qualidade de maior; superioridade.}{mai.o.ri.a}{0}
\verb{maioridade}{}{}{}{}{s.f.}{Idade (no Brasil, 21 anos) em que uma pessoa passa a ser reconhecida como plenamente capaz no mundo jurídico e civil.}{mai.o.ri.da.de}{0}
\verb{mais}{}{}{}{}{adv.}{Em maior grau ou quantidade.}{mais}{0}
\verb{mais}{}{}{}{}{s.m.}{Aquilo que sobra; restante.}{mais}{0}
\verb{maisena}{}{}{}{}{s.f.}{Farinha feita de amido de milho.}{mai.se.na}{0}
\verb{mais"-que"-perfeito}{ê}{Gram.}{}{}{s.m.}{Tempo do verbo que exprime uma ação passada antes de outra também passada. (\textit{A professora nos pediu que escrevêssemos uma narrativa empregando os tempos perfeito e mais"-que"-perfeito.})}{mais"-que"-per.fei.to}{0}
\verb{mais"-valia}{}{}{mais"-valias}{}{s.f.}{Na economia marxista, diferença entre o valor do produto criado pelo trabalho do operário e o valor obtido com sua venda e de que se apodera o capitalista em proveito próprio.}{mais"-va.li.a}{0}
\verb{maitaca}{}{Zool.}{}{}{s.f.}{Maritaca.}{mai.ta.ca}{0}
\verb{maître}{}{}{}{}{s.m.}{O chefe dos garçons.}{\textit{maître}}{0}
\verb{maiúscula}{}{}{}{}{adj.}{Diz"-se de letra de tamanho maior e formato diferenciado, usada geralmente em início de período e de nomes próprios.}{mai.ús.cu.la}{0}
\verb{maiúsculo}{}{}{}{}{adj.}{De tamanho maior e formato próprio, usado geralmente em início de período e nomes próprios; capitular.}{mai.ús.cu.lo}{0}
\verb{maiúsculo}{}{Fig.}{}{}{}{De grande importância; excelente.}{mai.ús.cu.lo}{0}
\verb{majestade}{}{}{}{}{s.f.}{Superioridade, excelência, grandeza.}{ma.jes.ta.de}{0}
\verb{majestade}{}{}{}{}{}{Aspecto grandioso; imponência.}{ma.jes.ta.de}{0}
\verb{majestade}{}{}{}{}{}{Título que se dá ao soberano hereditário de um Estado e à sua mulher.}{ma.jes.ta.de}{0}
\verb{majestático}{}{}{}{}{adj.}{Relativo a majestade.}{ma.jes.tá.ti.co}{0}
\verb{majestático}{}{}{}{}{}{Majestoso, respeitável.}{ma.jes.tá.ti.co}{0}
\verb{majestático}{}{}{}{}{}{Diz"-se de plural em cujo uso se emprega a forma \textit{nós} em lugar de \textit{eu} para denotar modéstia.}{ma.jes.tá.ti.co}{0}
\verb{majestoso}{ô}{}{"-osos ⟨ó⟩}{"-osa ⟨ó⟩}{adj.}{Que tem majestade; nobre.}{ma.jes.to.so}{0}
\verb{majestoso}{ô}{}{"-osos ⟨ó⟩}{"-osa ⟨ó⟩}{}{Belo, sublime, imponente.}{ma.jes.to.so}{0}
\verb{major}{ó}{}{}{}{s.m.}{Na hierarquia militar do Exército, oficial militar que ocupa o posto imediatamente acima do capitão e imediatamente abaixo do tenente"-coronel.}{ma.jor}{0}
\verb{majoração}{}{Bras.}{"-ões}{}{s.f.}{Ato ou efeito de majorar.}{ma.jo.ra.ção}{0}
\verb{majorar}{}{Bras.}{}{}{v.t.}{Tornar maior; aumentar.}{ma.jo.rar}{\verboinum{1}}
\verb{major"-aviador}{ó\ldots{}ô}{}{majores"-aviadores ⟨ó\ldots{}ô⟩}{}{s.m.}{Posto da hierarquia militar imediatamente inferior ao do tenente"-coronel"-aviador e imediatamente superior ao do capitão"-aviador.}{ma.jor"-a.vi.a.dor}{0}
\verb{major"-aviador}{ó\ldots{}ô}{}{majores"-aviadores ⟨ó\ldots{}ô⟩}{}{}{Militar que ocupa esse posto.}{ma.jor"-a.vi.a.dor}{0}
\verb{major"-brigadeiro}{ó}{}{majores"-brigadeiros ⟨ó⟩}{}{s.m.}{Posto da hierarquia da Aeronáutica imediatamente inferior ao de tenente"-brigadeiro e imediatamente superior ao de brigadeiro"-do"-ar. }{ma.jor"-bri.ga.dei.ro}{0}
\verb{major"-brigadeiro}{ó}{}{majores"-brigadeiros ⟨ó⟩}{}{}{Militar que ocupa esse posto.}{ma.jor"-bri.ga.dei.ro}{0}
\verb{majoritário}{}{Bras.}{}{}{adj.}{Relativo a maioria.}{ma.jo.ri.tá.rio}{0}
%\verb{}{}{}{}{}{}{}{}{0}
%\verb{}{}{}{}{}{}{}{}{0}
%\verb{}{}{}{}{}{}{}{}{0}
%\verb{}{}{}{}{}{}{}{}{0}
\verb{mal}{}{}{}{}{adv.}{De maneira ruim, inadequada ou errada.}{mal}{0}
\verb{mal}{}{}{}{}{}{Com esforço ou dificuldade.}{mal}{0}
\verb{mal}{}{}{}{}{s.m.}{Coisa nociva ou prejudicial. [Pl.: \textit{males}].}{mal}{0}
\verb{mal}{}{}{}{}{conj.}{Assim que, logo que.}{mal}{0}
\verb{mala}{}{}{}{}{s.f.}{Objeto provido de alça, geralmente feito de couro, madeira ou lona, para transportar objetos.}{ma.la}{0}
\verb{malabar}{}{}{}{}{adj.2g.}{Diz"-se de jogo que envolve movimentos hábeis e extravagantes.}{ma.la.bar}{0}
\verb{malabarismo}{}{}{}{}{s.m.}{Execução de jogos malabares.}{ma.la.ba.ris.mo}{0}
\verb{malabarismo}{}{Fig.}{}{}{}{Habilidade de lidar com situações desfavoráveis ou instáveis.}{ma.la.ba.ris.mo}{0}
\verb{malabarista}{}{}{}{}{s.2g.}{Indivíduo que executa jogos malabares.}{ma.la.ba.ris.ta}{0}
\verb{mal"-acabado}{}{}{mal"-acabados}{}{adj.}{Que não foi correta ou adequadamente executado; malfeito.}{mal"-a.ca.ba.do}{0}
\verb{malacacheta}{ê}{Bras.}{}{}{s.f.}{Nome comum a um grande número de minerais que se fendem em lâminas delgadas; mica.}{ma.la.ca.che.ta}{0}
\verb{mal"-acostumado}{}{}{mal"-acostumados}{}{adj.}{Diz"-se de indivíduo cheio de caprichos por ter recebido regalias e facilidades demais.}{mal"-a.cos.tu.ma.do}{0}
%\verb{}{}{}{}{}{}{}{}{0}
%\verb{}{}{}{}{}{}{}{}{0}
\verb{mal"-afortunado}{}{}{mal"-afortunados}{}{adj.}{Infeliz, infortunado, mal"-aventurado.}{mal"-a.for.tu.na.do}{0}
\verb{mal"-agradecido}{}{Bras.}{mal"-agradecidos}{}{adj.}{Que não reconhece os benefícios recebidos; ingrato.}{mal"-a.gra.de.ci.do}{0}
\verb{malaguenha}{}{Mús.}{}{}{s.f.}{Canto popular originário da região de Málaga (Espanha), similar ao fandango, composto de melodia improvisada e de sequência típica de acordes descendentes.}{ma.la.gue.nha}{0}
\verb{malaguenha}{}{}{}{}{}{Dança espanhola, similar ao fandango, que acompanha esse canto.}{ma.la.gue.nha}{0}
\verb{malaguenho}{}{}{}{}{adj.}{Relativo a Málaga (Espanha).}{ma.la.gue.nho}{0}
\verb{malaguenho}{}{}{}{}{}{Indivíduo natural ou habitante dessa cidade.}{ma.la.gue.nho}{0}
\verb{malagueta}{ê}{}{}{}{s.f.}{Tipo de pimenta muito ardida.}{ma.la.gue.ta}{0}
\verb{malaio}{}{}{}{}{adj.}{Relativo à Malásia (Ásia).}{ma.lai.o}{0}
\verb{malaio}{}{}{}{}{s.m.}{Indivíduo natural ou habitante desse país.}{ma.lai.o}{0}
\verb{malaio}{}{}{}{}{}{O idioma da Malásia.}{ma.lai.o}{0}
\verb{mal"-ajambrado}{}{}{mal"-ajambrados}{}{adj.}{Malvestido; desarrumado, deselegante.}{mal"-a.jam.bra.do}{0}
\verb{mal"-amanhado}{}{}{mal"-amanhados}{}{adj.}{Que está mal vestido.}{mal"-a.ma.nha.do}{0}
\verb{malandragem}{}{}{"-ens}{}{s.f.}{Bando de malandros.}{ma.lan.dra.gem}{0}
\verb{malandragem}{}{}{"-ens}{}{}{Qualidade, estilo de vida ou ação própria de malandro.}{ma.lan.dra.gem}{0}
\verb{malandrar}{}{}{}{}{v.i.}{Viver como malandro.}{ma.lan.drar}{\verboinum{1}}
\verb{malandrice}{}{}{}{}{s.f.}{Dito ou gesto de malandro.}{ma.lan.dri.ce}{0}
\verb{malandro}{}{}{}{}{adj.}{Diz"-se de quem é astuto, esperto.}{ma.lan.dro}{0}
\verb{malandro}{}{}{}{}{}{Vagabundo, vadio.}{ma.lan.dro}{0}
%\verb{}{}{}{}{}{}{}{}{0}
\verb{malar}{}{}{}{}{adj.2g.}{Relativo às maçãs do rosto.}{ma.lar}{0}
\verb{malar}{}{}{}{}{s.m.}{Cada um dos dois ossos sob as bochechas.}{ma.lar}{0}
\verb{malária}{}{Med.}{}{}{s.f.}{Doença aguda ou crônica transmitida pela picada de mosquitos e que se caracteriza por calafrios e febres periódicas; impaludismo.  }{ma.lá.ria}{0}
%\verb{}{}{}{}{}{}{}{}{0}
%\verb{}{}{}{}{}{}{}{}{0}
%\verb{}{}{}{}{}{}{}{}{0}
\verb{mal"-assombrado}{}{Bras.}{mal"-assombrados}{}{adj.}{Habitado ou frequentado por assombrações, por fantasmas.}{mal"-as.som.bra.do}{0}
\verb{mal"-aventurado}{}{}{mal"-aventurados}{}{adj.}{Diz"-se de indivíduo infeliz, infortunado, desgraçado.}{mal"-a.ven.tu.ra.do}{0}
%\verb{}{}{}{}{}{}{}{}{0}
\verb{malbaratar}{}{}{}{}{v.t.}{Vender abaixo do custo.}{mal.ba.ra.tar}{0}
\verb{malbaratar}{}{}{}{}{}{Desperdiçar.}{mal.ba.ra.tar}{\verboinum{1}}
\verb{malcheiroso}{ô}{}{"-osos ⟨ó⟩}{"-osa ⟨ó⟩}{adj.}{Que cheira mal; fedorento.}{mal.chei.ro.so}{0}
\verb{malcriação}{}{}{malcriações}{}{s.f.}{Qualidade, ação ou dito de malcriado; má"-criação. }{mal.cri.a.ção}{0}
\verb{malcriado}{}{}{}{}{adj.}{Diz"-se de indivíduo descortês, mal- educado.}{mal.cri.a.do}{0}
\verb{maldade}{}{}{}{}{s.f.}{Qualidade ou ato de quem é mau.}{mal.da.de}{0}
\verb{maldade}{}{}{}{}{}{Ação prejudicial e mal"-intencionada.}{mal.da.de}{0}
\verb{maldade}{}{}{}{}{}{Malícia.}{mal.da.de}{0}
\verb{maldar}{}{}{}{}{v.t.}{Formar mau juízo, suspeitar maldade de alguém ou algo.}{mal.dar}{0}
\verb{maldar}{}{}{}{}{}{Julgar com dados poucos seguros; suspeitar, maliciar.}{mal.dar}{\verboinum{1}}
\verb{maldição}{}{}{"-ões}{}{s.f.}{Desejo mau lançado contra alguém.}{mal.di.ção}{0}
\verb{maldição}{}{}{"-ões}{}{}{Castigo divino.}{mal.di.ção}{0}
\verb{maldito}{}{}{}{}{adj.}{Diz"-se de quem sofreu maldição.}{mal.di.to}{0}
\verb{maldito}{}{}{}{}{}{Malvado, perverso.}{mal.di.to}{0}
\verb{maldizente}{}{}{}{}{adj.2g.}{Diz"-se de quem fala mal dos outros.}{mal.di.zen.te}{0}
\verb{maldizer}{ê}{}{}{}{v.t.}{Dirigir imprecações contra; amaldiçoar, praguejar.}{mal.di.zer}{0}
\verb{maldizer}{ê}{}{}{}{}{Dizer mal de alguém; difamar.}{mal.di.zer}{\verboinum{41}}
\verb{maldoso}{ô}{}{"-osos ⟨ó⟩}{"-osa ⟨ó⟩}{adj.}{Que tem maldade.}{mal.do.so}{0}
\verb{maldoso}{ô}{}{"-osos ⟨ó⟩}{"-osa ⟨ó⟩}{}{Malicioso.}{mal.do.so}{0}
\verb{malear}{}{}{}{}{v.t.}{Distender metal a golpes de martelo; transformar em ou reduzir a lâmina.}{ma.le.ar}{0}
\verb{malear}{}{}{}{}{}{Tornar flexível.}{ma.le.ar}{\verboinum{4}}
\verb{maleável}{}{}{"-eis}{}{adj.2g.}{Que se pode malear ou malhar.}{ma.le.á.vel}{0}
\verb{maleável}{}{Fig.}{"-eis}{}{}{Que tem elasticidade; flexível, dobrável.}{ma.le.á.vel}{0}
\verb{maleável}{}{Fig.}{"-eis}{}{}{Dócil, flexível, brando.}{ma.le.á.vel}{0}
\verb{maledicência}{}{}{}{}{s.f.}{Qualidade de quem é maledicente ou maldizente.}{ma.le.di.cên.cia}{0}
\verb{maledicência}{}{}{}{}{}{Ação ou hábito de dizer mal dos outros; difamação, maldizer.}{ma.le.di.cên.cia}{0}
\verb{maledicência}{}{}{}{}{}{Comentário maldoso; difamação, injúria.}{ma.le.di.cên.cia}{0}
\verb{maledicente}{}{}{}{}{adj.2g.}{Diz"-se de quem diz mal dos outros.}{ma.le.di.cen.te}{0}
\verb{mal"-educado}{}{}{mal"-educados}{}{adj.}{Malcriado.}{mal"-e.du.ca.do}{0}
\verb{malefício}{}{}{}{}{s.m.}{O que tem efeito nocivo, prejudicial; dano, prejuízo.}{ma.le.fí.cio}{0}
\verb{malefício}{}{}{}{}{}{Feitiço contra alguém.}{ma.le.fí.cio}{0}
\verb{maléfico}{}{}{}{}{adj.}{Que provoca dano, prejuízo; nocivo.}{ma.lé.fi.co}{0}
\verb{maléfico}{}{}{}{}{}{Que faz mal; que tende para o mal; malvado.}{ma.lé.fi.co}{0}
\verb{maleiro}{ê}{}{}{}{s.m.}{Local para guardar malas.}{ma.lei.ro}{0}
\verb{maleiro}{ê}{}{}{}{}{Fabricante ou vendedor de malas.}{ma.lei.ro}{0}
\verb{maleita}{ê}{Med.}{}{}{s.f.}{Malária.}{ma.lei.ta}{0}
\verb{mal"-e"-mal}{}{}{}{}{adv.}{De modo sofrível; escassamente.}{mal"-e"-mal}{0}
\verb{mal"-encarado}{}{}{mal"-encarados}{}{adj.}{Diz"-se de quem é carrancudo.}{mal"-en.ca.ra.do}{0}
\verb{mal"-encarado}{}{}{mal"-encarados}{}{}{Diz"-se de quem tem aparência suspeita.}{mal"-en.ca.ra.do}{0}
\verb{mal"-entendido}{}{}{mal"-entendidos}{}{s.m.}{Divergência de interpretação quanto ao sentido de palavra, ação, ordem; equívoco, engano.}{mal"-en.ten.di.do}{0}
\verb{mal"-entendido}{}{}{mal"-entendidos}{}{adj.}{Mal"-interpretado ou avaliado.}{mal"-en.ten.di.do}{0}
\verb{maléolo}{}{Anat.}{}{}{s.m.}{Cada uma das proeminências ósseas arredondadas que ficam de ambos os lados da articulação do tornozelo.}{ma.lé.o.lo}{0}
\verb{mal"-estar}{}{}{mal"-estares}{}{s.m.}{Incômodo físico que não chega a ser doença.}{mal"-es.tar}{0}
\verb{mal"-estar}{}{}{mal"-estares}{}{}{Desassossego.}{mal"-es.tar}{0}
\verb{mal"-estar}{}{}{mal"-estares}{}{}{Embaraço, constrangimento.}{mal"-es.tar}{0}
\verb{maleta}{ê}{}{}{}{s.f.}{Pequena mala.}{ma.le.ta}{0}
\verb{malevolência}{}{}{}{}{s.f.}{Qualidade ou ação de malevolente.}{ma.le.vo.lên.cia}{0}
\verb{malevolência}{}{}{}{}{}{Disposição para julgar desfavoravelmente; má vontade; malquerença; hostilidade.}{ma.le.vo.lên.cia}{0}
\verb{malevolente}{}{}{}{}{adj.2g.}{Que tem má índole.}{ma.le.vo.len.te}{0}
\verb{malevolente}{}{}{}{}{}{Que deseja o mal aos outros.}{ma.le.vo.len.te}{0}
\verb{malévolo}{}{}{}{}{adj.}{Que é muito mau; malvado.}{ma.lé.vo.lo}{0}
\verb{malfadado}{}{}{}{}{adj.}{Diz"-se de quem tem má sorte; infeliz.}{mal.fa.da.do}{0}
\verb{malfadar}{}{}{}{}{v.t.}{Prever desgraças para.}{mal.fa.dar}{\verboinum{1}}
\verb{malfazejo}{ê}{}{}{}{adj.}{Que faz mal.}{mal.fa.ze.jo}{0}
\verb{malfeito}{ê}{}{}{}{adj.}{Realizado sem cuidado ou competência.}{mal.fei.to}{0}
\verb{malfeito}{ê}{}{}{}{}{Defeituoso.}{mal.fei.to}{0}
\verb{malfeito}{ê}{}{}{}{s.m.}{Feitiço.}{mal.fei.to}{0}
\verb{malfeitor}{ô}{}{}{}{adj.}{Diz"-se daquele que comete crimes ou ações condenáveis.}{mal.fei.tor}{0}
\verb{malfeitor}{ô}{}{}{}{}{Malfazejo.}{mal.fei.tor}{0}
\verb{malfeitoria}{}{}{}{}{s.f.}{O que é prejudicial, nocivo; dano, malefício.}{mal.fei.to.ri.a}{0}
\verb{malfeitoria}{}{}{}{}{}{Ação que infringe as leis de uma sociedade, que merece punição; infração, delito.}{mal.fei.to.ri.a}{0}
\verb{malferir}{}{}{}{}{v.t.}{Ferir gravemente.}{mal.fe.rir}{\verboinum{29}}
\verb{malformação}{}{}{"-ões}{}{s.f.}{Formação anormal de parte do corpo.}{mal.for.ma.ção}{0}
\verb{malgastar}{}{}{}{}{v.t.}{Gastar descontroladamente; esbanjar, desperdiçar.}{mal.gas.tar}{\verboinum{1}}
\verb{malgaxe}{ch}{}{}{}{adj.2g.}{Relativo à República Malgaxe (antiga ilha de Madagascar).}{mal.ga.xe}{0}
\verb{malgaxe}{ch}{}{}{}{s.2g.}{Indivíduo natural ou habitante dessa república; madagascarense.}{mal.ga.xe}{0}
\verb{malgaxe}{ch}{}{}{}{s.m.}{O idioma malgaxe.}{mal.ga.xe}{0}
\verb{malgrado}{}{}{}{}{prep.}{Apesar de; não obstante. (\textit{Malgrado os 40 graus de calor que fazia, as pessoas aproveitaram a festa.})}{mal.gra.do}{0}
\verb{malha}{}{}{}{}{s.f.}{Cada um dos nós ou trançados de um tipo de tecido flexível e elástico.}{ma.lha}{0}
\verb{malha}{}{}{}{}{s.f.}{Ato ou efeito de malhar; malhação, surra.}{ma.lha}{0}
\verb{malha}{}{}{}{}{}{Roupa feita desse tecido.}{ma.lha}{0}
\verb{malha}{}{}{}{}{}{Sinal de coloração diferente na pele ou no pelo de animais; mancha.}{ma.lha}{0}
\verb{malha}{}{Esport.}{}{}{}{Disco ou chapa de metal que se arremessa em um jogo de mesmo nome.}{ma.lha}{0}
%\verb{}{}{}{}{}{}{}{}{0}
\verb{malhação}{}{}{"-ões}{}{s.f.}{Ato ou efeito de malhar; surra, espancamento.}{ma.lha.ção}{0}
\verb{malhação}{}{Esport.}{"-ões}{}{}{Conjunto de exercícios físicos praticados com muita intensidade com fins de modelamento do corpo ou fortalecimento dos músculos.}{ma.lha.ção}{0}
\verb{malhada}{}{}{}{}{s.f.}{Pancada desferida com malho.}{ma.lha.da}{0}
\verb{malhada}{}{}{}{}{}{Curral de gado.}{ma.lha.da}{0}
\verb{malhado}{}{}{}{}{adj.}{Que tem malhas, manchas; manchado.}{ma.lha.do}{0}
\verb{malhado}{}{}{}{}{adj.}{Batido com malho; surrado, espancado.}{ma.lha.do}{0}
\verb{malhado}{}{}{}{}{}{Diz"-se do corpo que foi moldado por ginástica intensa.}{ma.lha.do}{0}
\verb{malhar}{}{}{}{}{v.t.}{Bater com malho; surrar, espancar.}{ma.lhar}{0}
\verb{malhar}{}{}{}{}{}{Praticar exercícios físicos com intensidade.}{ma.lhar}{\verboinum{1}}
\verb{malharia}{}{}{}{}{s.f.}{Local onde se fabricam ou se vendem roupas de malha ou de jérsei.}{ma.lha.ri.a}{0}
\verb{malhete}{ê}{}{}{}{s.m.}{Entalhe nas extremidades de duas tábuas para que se adaptem perfeitamente.}{ma.lhe.te}{0}
\verb{malhete}{ê}{}{}{}{}{Pequeno malho ou martelo.}{ma.lhe.te}{0}
\verb{malho}{}{}{}{}{s.m.}{Martelo usado para bater o ferro na bigorna.}{ma.lho}{0}
\verb{mal"-humorado}{}{}{mal"-humorados}{}{adj.}{Que demonstra mau humor; irritado, ranzinza. \textit{Às segundas"-feiras ele sempre acorda mal"-humorado.}}{mal"-hu.mo.ra.do}{0}
\verb{malícia}{}{}{}{}{s.f.}{Inclinação para encontrar maldade em tudo; má índole; má"-fé.}{ma.lí.cia}{0}
\verb{malícia}{}{}{}{}{}{Habilidade para enganar; astúcia, esperteza, manha.}{ma.lí.cia}{0}
\verb{maliciar}{}{}{}{}{v.t.}{Desconfiar de outrem; fazer mau juízo; suspeitar.}{ma.li.ci.ar}{\verboinum{1}}
\verb{malicioso}{ô}{}{"-osos ⟨ó⟩}{"-osa ⟨ó⟩}{adj.}{Que tem malícia; astuto, mordaz, picante. (\textit{Seu olhar malicioso não combinava com as palavras inocentes que dizia.})}{ma.li.ci.o.so}{0}
\verb{maligno}{}{}{}{}{adj.}{Que é mau por natureza; mal"-intencionado, malévolo.}{ma.lig.no}{0}
\verb{maligno}{}{Med.}{}{}{}{Diz"-se de doença que pode levar à morte; fatal.}{ma.lig.no}{0}
\verb{malinês}{}{}{}{}{adj.}{Relativo à República do Mali (oeste da África).}{ma.li.nês}{0}
\verb{malinês}{}{}{}{}{s.m.}{Indivíduo natural ou habitante dessa república.}{ma.li.nês}{0}
\verb{má"-língua}{}{}{más"-línguas}{}{s.f.}{Vício de dizer mal de pessoas ou coisas.}{má"-lín.gua}{0}
\verb{mal"-intencionado}{}{}{mal"-intencionados}{}{adj.}{Que age com intenção de fazer o mal; de má índole; maligno.}{mal"-in.ten.ci.o.na.do}{0}
\verb{maljeitoso}{ô}{}{"-osos ⟨ó⟩}{"-osa ⟨ó⟩}{adj.}{Que não tem habilidade; desajeitado, desastrado.}{mal.jei.to.so}{0}
\verb{malmequer}{é}{Bot.}{malmequeres ⟨é⟩}{}{s.m.}{Planta ornamental, de flores amarelas e frutos cobertos de pelos; bem"-me"-quer.}{mal.me.quer}{0}
\verb{malnascido}{}{}{}{}{adj.}{Que nasceu com má sorte; desventurado, desgraçado.}{mal.nas.ci.do}{0}
\verb{malnascido}{}{}{}{}{}{De baixa condição social.}{mal.nas.ci.do}{0}
\verb{maloca}{ó}{}{}{}{s.f.}{Habitação indígena em que moram várias famílias.}{ma.lo.ca}{0}
\verb{maloca}{ó}{}{}{}{}{Casebre muito pobre; choupana.}{ma.lo.ca}{0}
\verb{malogrado}{}{}{}{}{adj.}{Que teve mau êxito; frustrado, fracassado.}{ma.lo.gra.do}{0}
\verb{malograr}{}{}{}{}{v.t.}{Fazer frustrar; levar ao fracasso.}{ma.lo.grar}{\verboinum{1}}
\verb{malogro}{ô}{}{}{}{s.m.}{Falta de sucesso; mau êxito; fracasso.}{ma.lo.gro}{0}
\verb{maloqueiro}{ê}{Bras.}{}{}{s.m.}{Indivíduo que faz arruaças pela rua; desordeiro.}{ma.lo.quei.ro}{0}
\verb{malote}{ó}{}{}{}{s.m.}{Pequena mala de viagem; maleta.}{ma.lo.te}{0}
\verb{malote}{ó}{}{}{}{}{Serviço particular de transporte e entrega rápida de correspondência ou encomendas.}{ma.lo.te}{0}
\verb{malparado}{}{}{}{}{adj.}{Que se encontra em situação ruim, desfavorável; arriscado.}{mal.pa.ra.do}{0}
\verb{malpassado}{}{Cul.}{}{}{adj.}{Diz"-se de alimento, principalmente carne, pouco assado ou frito.}{mal.pas.sa.do}{0}
\verb{malquerença}{}{}{}{}{s.f.}{Estado de malquerente; inimizade, hostilidade.}{mal.que.ren.ça}{0}
\verb{malquerente}{}{}{}{}{adj.2g.}{Que quer mal a outrem; inimigo, hostil.}{mal.que.ren.te}{0}
\verb{malquerer}{ê}{}{}{}{v.t.}{Desejar mal a outrem; não gostar de; detestar.}{mal.que.rer}{\verboinum{49}}
\verb{malquistar}{}{}{}{}{v.t.}{Tornar malquisto; inimizar, indispor.}{mal.quis.tar}{\verboinum{1}}
\verb{malquisto}{}{}{}{}{adj.}{Que não é querido; detestado, odiado.}{mal.quis.to}{0}
\verb{malsão}{}{}{"-ãos}{malsã}{adj.}{Que é nocivo à saúde; daninho, maléfico.}{mal.são}{0}
\verb{malsão}{}{}{"-ãos}{malsã}{}{De saúde precária; doentio.}{mal.são}{0}
\verb{malsinar}{}{}{}{}{v.t.}{Fazer acusações; denunciar, delatar.}{mal.si.nar}{\verboinum{1}}
\verb{malsoante}{}{}{}{}{adj.2g.}{Que soa mal; desafinado, dissonante.}{mal.so.an.te}{0}
\verb{malsofrido}{}{}{}{}{adj.}{Que não consegue suportar o sofrimento; sem resignação; impaciente.}{mal.so.fri.do}{0}
\verb{malsucedido}{}{}{}{}{adj.}{Que não teve êxito; fracassado, frustrado, malogrado.}{mal.su.ce.di.do}{0}
\verb{malta}{}{}{}{}{s.f.}{Grupo de desordeiros; corja, bando.}{mal.ta}{0}
\verb{maltado}{}{}{}{}{adj.}{Que contém malte ou a que foi adicionado malte.}{mal.ta.do}{0}
\verb{maltar}{}{}{}{}{v.t.}{Converter a cevada em malte.}{mal.tar}{0}
\verb{maltar}{}{}{}{}{}{Adicionar malte a.}{mal.tar}{\verboinum{1}}
\verb{malte}{}{}{}{}{s.m.}{Cevada posta para germinar e secar, usada na fermentação e destilação de bebidas.}{mal.te}{0}
\verb{maltês}{}{}{}{}{adj.}{Relativo a Malta, ilha do Mediterrâneo Central.}{mal.tês}{0}
\verb{maltês}{}{}{}{}{}{Diz"-se de gato cinzento.}{mal.tês}{0}
\verb{maltês}{}{}{}{}{s.m.}{Indivíduo natural ou habitante dessa ilha.}{mal.tês}{0}
\verb{maltês}{}{}{}{}{}{O idioma de Malta.}{mal.tês}{0}
\verb{malthusianismo}{}{}{}{}{s.m.}{Teoria do economista inglês Malthus, que viveu no final do século \textsc{xviii}, segundo a qual o crescimento demográfico desordenado levaria à escassez de alimentos no mundo.}{mal.thu.si.a.nis.mo}{0}
\verb{malthusiano}{}{}{}{}{adj.}{Relativo a Malthus ou ao malthusianismo.}{mal.thu.si.a.no}{0}
\verb{malthusiano}{}{}{}{}{s.m.}{Indivíduo partidário do malthusianismo.}{mal.thu.si.a.no}{0}
\verb{maltose}{ó}{Bioquím.}{}{}{s.f.}{Açúcar cristalino extraído do amido.}{mal.to.se}{0}
\verb{maltrapilho}{}{}{}{}{adj.}{Que anda mal vestido; esfarrapado.}{mal.tra.pi.lho}{0}
\verb{maltratar}{}{}{}{}{v.t.}{Tratar mal, com grosseria; insultar, ofender.}{mal.tra.tar}{\verboinum{1}}
\verb{malucar}{}{}{}{}{v.i.}{Praticar ou dizer maluquices; delirar, desvairar.}{ma.lu.car}{\verboinum{2}}
\verb{maluco}{}{}{}{}{adj.}{Que não tem juízo; doido, louco.}{ma.lu.co}{0}
\verb{maluco}{}{}{}{}{}{Que tem paixão por algo; excêntrico, extravagante. (\textit{Meu primo é maluco por futebol.})}{ma.lu.co}{0}
\verb{maluqueira}{ê}{}{}{}{s.f.}{Maluquice.}{ma.lu.quei.ra}{0}
\verb{maluquice}{}{}{}{}{s.f.}{Ato ou dito de maluco; doidice, asneira, absurdo, maluqueira.}{ma.lu.qui.ce}{0}
\verb{malva}{}{Bot.}{}{}{s.f.}{Planta medicinal de propriedades anti"-inflamatórias e calmantes.}{mal.va}{0}
\verb{malvadez}{ê}{}{ ⟨ê⟩}{}{s.f.}{Malvadeza.}{mal.va.dez}{0}
\verb{malvadeza}{ê}{}{}{}{s.f.}{Qualidade ou ação de malvado; maldade, malvadez, crueldade. }{mal.va.de.za}{0}
\verb{malvado}{}{}{}{}{adj.}{Que pratica maldades, atos de crueldade; mau, perverso, cruel.}{mal.va.do}{0}
\verb{malvado}{}{}{}{}{s.m.}{Pessoa má, malvada.}{mal.va.do}{0}
\verb{malversação}{}{}{"-ões}{}{s.f.}{Ato de malversar.}{mal.ver.sa.ção}{0}
\verb{malversação}{}{}{"-ões}{}{}{Má administração, má gerência.}{mal.ver.sa.ção}{0}
\verb{malversação}{}{}{"-ões}{}{}{Desvio de dinheiro, especialmente público; dilapidação, dissipação.}{mal.ver.sa.ção}{0}
\verb{malversar}{}{}{}{}{v.t.}{Usar de modo impróprio ou desviar dinheiro, especialmente público.}{mal.ver.sar}{0}
\verb{malversar}{}{}{}{}{}{Administrar ou gerenciar mal uma empresa.}{mal.ver.sar}{\verboinum{1}}
\verb{malvestido}{}{}{}{}{adj.}{Que não }{mal.ves.ti.do}{0}
\verb{malvisto}{}{}{}{}{adj.}{Que não é bem conceituado, aceito, ou que tem má fama; malquisto.}{mal.vis.to}{0}
\verb{mama}{}{}{}{}{s.f.}{Órgão glandular característico dos mamíferos e que, nas fêmeas, secreta leite; seio, peito, teta, úbere.}{ma.ma}{0}
\verb{mama}{}{}{}{}{}{Mamadura.}{ma.ma}{0}
\verb{mamada}{}{}{}{}{s.f.}{Ato ou efeito de mamar; mamadura.}{ma.ma.da}{0}
\verb{mamada}{}{}{}{}{}{Tempo que dura cada amamentação.}{ma.ma.da}{0}
\verb{mamadeira}{ê}{}{}{}{s.f.}{Garrafinha de vidro ou de plástico, com bico de borracha, usada para amamentar artificialmente as crianças.}{ma.ma.dei.ra}{0}
\verb{mamãe}{}{}{"-ães}{}{s.f.}{Tratamento carinhoso que um filho dá a sua mãe; mãe.}{ma.mãe}{0}
\verb{mamangaba}{}{Zool.}{}{}{s.f.}{Nome comum a certas abelhas sociais, de coloração preta e amarela, largamente distribuídas pelo Brasil, que fazem ninho no solo ou em barrancos e cujo mel é ralo e de má qualidade.}{ma.man.ga.ba}{0}
\verb{mamangava}{}{Zool.}{}{}{s.f.}{Mamangaba.}{ma.man.ga.va}{0}
\verb{mamão}{}{}{"-ões}{}{s.m.}{O fruto comestível do mamoeiro, de forma ovalada, semelhante a uma mama, de polpa alaranjada, densa e suculenta.}{ma.mão}{0}
\verb{mamão}{}{}{"-ões}{}{}{Mamoeiro.}{ma.mão}{0}
\verb{mamão}{}{}{"-ões}{}{adj.}{Que não desmamou ainda.}{ma.mão}{0}
\verb{mamão}{}{}{"-ões}{}{}{Que mama muito.}{ma.mão}{0}
\verb{mamar}{}{}{}{}{v.t.}{Sugar o leite (de mama, teta, mamadeira etc.).}{ma.mar}{0}
\verb{mamar}{}{}{}{}{}{Sugar uma coisa qualquer.}{ma.mar}{0}
\verb{mamar}{}{}{}{}{v.i.}{Sugar o leite.}{ma.mar}{0}
\verb{mamar}{}{}{}{}{}{Embriagar"-se, embebedar"-se.}{ma.mar}{\verboinum{1}}
\verb{mamário}{}{}{}{}{adj.}{Relativo às mamas.}{ma.má.rio}{0}
\verb{mamata}{}{Pop.}{}{}{s.f.}{Negociata, roubalheira.}{ma.ma.ta}{0}
\verb{mamata}{}{}{}{}{}{Trabalho fácil e lucrativo, rendoso.}{ma.ma.ta}{0}
\verb{mambembe}{}{}{}{}{adj.2g.}{Que tem pouco valor; medíocre, ordinário, reles, inferior. }{mam.bem.be}{0}
\verb{mambembe}{}{}{}{}{s.m.}{Grupo teatral ambulante e amador, com poucos recursos e de má qualidade.}{mam.bem.be}{0}
\verb{mambo}{}{}{}{}{s.m.}{Música e dança afro"-cubana, inspiradas na rumba e no \textit{swing}. }{mam.bo}{0}
\verb{mameluco}{}{}{}{}{s.m.}{Mestiço de branco e índio.}{ma.me.lu.co}{0}
\verb{mamífero}{}{}{}{}{adj.}{Que tem mamas.}{ma.mí.fe.ro}{0}
\verb{mamífero}{}{}{}{}{}{Relativo ou pertencente aos mamíferos.}{ma.mí.fe.ro}{0}
\verb{mamífero}{}{Zool.}{}{}{s.m.}{Espécime dos mamíferos, classe de animais vertebrados de sangue quente, temperatura constante, com respiração pulmonar, vivíparos, de fecundação interna, geralmente cobertos por pelos e cuja principal característica é a de possuírem mamas que secretam leite para alimentar os filhotes.}{ma.mí.fe.ro}{0}
\verb{mamilo}{}{}{}{}{s.m.}{O bico da mama, do peito; maminha.}{ma.mi.lo}{0}
\verb{maminha}{}{}{}{}{s.f.}{Mama pequena.}{ma.mi.nha}{0}
\verb{maminha}{}{}{}{}{}{Bico do peito; mamilo.}{ma.mi.nha}{0}
\verb{maminha}{}{Cul.}{}{}{}{A parte mais macia da alcatra.}{ma.mi.nha}{0}
\verb{mamoeiro}{ê}{Bot.}{}{}{s.m.}{Árvore cujo fruto, comestível, é o mamão.}{ma.mo.ei.ro}{0}
\verb{mamografia}{}{Med.}{}{}{s.f.}{Radiografia de mama; mastografia.}{ma.mo.gra.fi.a}{0}
\verb{mamona}{}{}{}{}{s.f.}{O fruto da mamoneira, do qual se extrai o óleo de rícino.}{ma.mo.na}{0}
\verb{mamona}{}{}{}{}{}{A semente da mamoneira.}{ma.mo.na}{0}
\verb{mamona}{}{}{}{}{}{Mamoneira.}{ma.mo.na}{0}
\verb{mamoneira}{ê}{Bot.}{}{}{s.f.}{Arbusto cujo fruto é a mamona; carrapateira, rícino.}{ma.mo.nei.ra}{0}
\verb{mamoneira}{ê}{}{}{}{}{Mamona.}{ma.mo.nei.ra}{0}
\verb{mamoplastia}{}{Med.}{}{}{s.f.}{Mastoplastia.}{ma.mo.plas.ti.a}{0}
\verb{mamute}{}{Paleo.}{}{}{s.m.}{Nome comum aos elefantes pré"-históricos, conhecidos apenas por meio de fósseis, dotados de presas longas e curvas e com o corpo coberto de pelos.}{ma.mu.te}{0}
\verb{maná}{}{}{}{}{s.m.}{Segundo a Bíblia, alimento que Deus enviou do céu aos israelitas quando atravessavam o deserto.}{ma.ná}{0}
\verb{maná}{}{Fig.}{}{}{}{Alimento delicioso.}{ma.ná}{0}
\verb{mana}{}{Pop.}{}{}{s.f.}{Irmã.}{ma.na}{0}
\verb{manacá}{}{Bot.}{}{}{s.m.}{Nome comum a várias árvores e arbustos com flores perfumadas e de cores variadas, muito cultivadas como ornamentais em praças e jardins.}{ma.na.cá}{0}
\verb{manada}{}{}{}{}{s.f.}{Rebanho de gado.}{ma.na.da}{0}
\verb{managuense}{}{}{}{}{adj.2g.}{Relativo a Manágua, capital da Nicarágua.}{ma.na.guen.se}{0}
\verb{managuense}{}{}{}{}{s.2g.}{Natural ou habitante dessa cidade.}{ma.na.guen.se}{0}
\verb{manancial}{}{}{"-ais}{}{adj.2g.}{Que mana, que jorra sem parar.}{ma.nan.ci.al}{0}
\verb{manancial}{}{}{"-ais}{}{s.m.}{Nascente de água; fonte, olho"-d'água.}{ma.nan.ci.al}{0}
\verb{manancial}{}{Fig.}{"-ais}{}{}{Fonte abundante de alguma coisa.}{ma.nan.ci.al}{0}
\verb{manar}{}{}{}{}{v.t.}{Verter, jorrar sem parar e em abundância; minar, brotar.}{ma.nar}{\verboinum{1}}
\verb{manauense}{}{}{}{}{adj.2g.}{Relativo a Manaus, capital do Amazonas.}{ma.nau.en.se}{0}
\verb{manauense}{}{}{}{}{s.2g.}{Indivíduo natural ou habitante dessa cidade.}{ma.nau.en.se}{0}
\verb{mancada}{}{Pop.}{}{}{s.f.}{Atitude ou fala errônea ou inoportuna; falha, erro, lapso, fora, gafe, vacilo, rata.}{man.ca.da}{0}
\verb{mancal}{}{}{"-ais}{}{s.m.}{Dobradiça de porta.}{man.cal}{0}
\verb{mancar}{}{}{}{}{v.i.}{Coxear, claudicar, manquejar.}{man.car}{0}
\verb{mancar}{}{Pop.}{}{}{v.t.}{Deixar de cumprir um compromisso.}{man.car}{0}
\verb{mancar}{}{Pop.}{}{}{v.pron.}{Dar"-se conta de que está sendo inconveniente, inoportuno, ou de que está cometendo um erro.}{man.car}{\verboinum{2}}
\verb{mancebia}{}{}{}{}{s.f.}{Estado de quem vive com alguém sem estar religiosa ou juridicamente casado.}{man.ce.bi.a}{0}
\verb{mancebia}{}{Desus.}{}{}{}{Mocidade, juventude.}{man.ce.bi.a}{0}
\verb{mancebo}{ê}{Bras.}{}{}{s.m.}{Cabide para roupa, formado por uma haste vertical com diversos braços.}{man.ce.bo}{0}
\verb{mancebo}{ê}{}{}{}{}{Rapaz.}{man.ce.bo}{0}
\verb{mancebo}{ê}{Desus.}{}{}{}{Indivíduo que vive em mancebia.}{man.ce.bo}{0}
\verb{mancha}{}{}{}{}{s.f.}{Marca de sujeira deixada por alguma coisa; nódoa.}{man.cha}{0}
\verb{mancha}{}{}{}{}{}{Pinta de outra cor na pele de pessoas ou pelo de animais.}{man.cha}{0}
\verb{mancha}{}{}{}{}{}{Erro que prejudica a boa reputação de uma pessoa.}{man.cha}{0}
\verb{manchar}{}{}{}{}{v.t.}{Pôr mancha em; enodoar, sujar.}{man.char}{0}
\verb{manchar}{}{Fig.}{}{}{}{Prejudicar a boa reputação de alguém; desonrar.}{man.char}{\verboinum{1}}
\verb{mancheia}{ê}{}{}{}{s.f.}{Quantidade de qualquer coisa que caiba numa mão; punhado, mão"-cheia.}{man.chei.a}{0}
\verb{manchete}{é}{}{}{}{s.f.}{Título de notícia em jornal ou revista, geralmente em letras garrafais.}{man.che.te}{0}
\verb{manco}{}{}{}{}{adj.}{Que perdeu uma das mãos ou um dos pés; mutilado.}{man.co}{0}
\verb{manco}{}{}{}{}{}{Que coxeia, que claudica; coxo.}{man.co}{0}
\verb{manco}{}{}{}{}{s.m.}{Pessoa que manca.}{man.co}{0}
\verb{mancomunação}{}{}{"-ões}{}{s.f.}{Ato ou efeito de mancomunar; conluio, combinação.}{man.co.mu.na.ção}{0}
\verb{mancomunado}{}{}{}{}{adj.}{Que se mancomunou; combinado.}{man.co.mu.na.do}{0}
\verb{mancomunar}{}{}{}{}{v.t.}{Fazer um acordo desonesto; combinar, ajustar.}{man.co.mu.nar}{\verboinum{1}}
\verb{mandacaru}{}{Bot.}{}{}{s.m.}{Cacto de grande porte, de tronco muito ramificado, com flores grandes e brancas, característico da caatinga nordestina, e que serve de alimento ao gado durante a seca.}{man.da.ca.ru}{0}
\verb{manda"-chuva}{}{Pop.}{manda"-chuvas}{}{s.m.}{Pessoa muito importante ou influente; figurão, chefão, maioral.  }{man.da"-chu.va}{0}
\verb{manda"-chuva}{}{}{manda"-chuvas}{}{}{Líder, chefe.}{man.da"-chu.va}{0}
\verb{manda"-chuva}{}{}{manda"-chuvas}{}{}{Líder político nas cidades do interior; coronel.}{man.da"-chu.va}{0}
\verb{mandado}{}{}{}{}{adj.}{Que se mandou; enviado, remetido, expedido.}{man.da.do}{0}
\verb{mandado}{}{}{}{}{s.m.}{Ato ou efeito de mandar; mandamento.}{man.da.do}{0}
\verb{mandado}{}{}{}{}{}{Ordem, determinação.}{man.da.do}{0}
\verb{mandado}{}{}{}{}{}{Recado, incumbência.}{man.da.do}{0}
\verb{mandado}{}{}{}{}{}{Ordem por escrito de autoridade judicial ou administrativa.}{man.da.do}{0}
\verb{mandamento}{}{}{}{}{}{Cada um dos preceitos estabelecidos por uma religião.  }{man.da.men.to}{0}
\verb{mandamento}{}{}{}{}{s.m.}{Ato ou efeito de mandar; ordem, mandado, mando.}{man.da.men.to}{0}
\verb{mandante}{}{}{}{}{adj.2g.}{Que manda; mandador.}{man.dan.te}{0}
\verb{mandante}{}{}{}{}{s.2g.}{Pessoa que manda.}{man.dan.te}{0}
\verb{mandante}{}{}{}{}{}{Pessoa que estimula outra a fazer certos atos; instigador.}{man.dan.te}{0}
\verb{mandante}{}{Jur.}{}{}{}{Pessoa que outorga um mandato.}{man.dan.te}{0}
\verb{mandão}{}{}{"-ões}{mandona}{s.m.}{Pessoa que manda de forma autoritária e arrogante, ou que gosta de mandar.}{man.dão}{0}
\verb{mandar}{}{}{}{}{v.t.}{Ordenar, determinar.}{man.dar}{0}
\verb{mandar}{}{}{}{}{}{Comandar, dirigir.}{man.dar}{0}
\verb{mandar}{}{}{}{}{}{Remeter, enviar, endereçar.}{man.dar}{\verboinum{1}}
\verb{mandarim}{}{}{}{mandarina}{adj.2g.}{Pertencente ou relativo ao mandarim (dialeto).}{man.da.rim}{0}
\verb{mandarim}{}{}{}{mandarina}{s.m.}{Na China antiga, alto funcionário público.}{man.da.rim}{0}
\verb{mandarim}{}{}{}{mandarina}{}{Pessoa muito influente ou importante; manda"-chuva, chefão, figurão.}{man.da.rim}{0}
\verb{mandarim}{}{}{}{mandarina}{}{Dialeto oficial da China.}{man.da.rim}{0}
\verb{mandatário}{}{}{}{}{s.m.}{Pessoa que recebe ou executa mandatos.}{man.da.tá.rio}{0}
\verb{mandatário}{}{}{}{}{}{Representante, procurador.}{man.da.tá.rio}{0}
\verb{mandato}{}{}{}{}{s.m.}{Autorização que alguém dá a outra pessoa para, em seu nome, praticar determinados atos ou administrar interesses; procuração.}{man.da.to}{0}
\verb{mandato}{}{}{}{}{}{Poderes políticos que os eleitores conferem a um cidadão que elegem e que os representa.}{man.da.to}{0}
\verb{mandi}{}{Zool.}{}{}{s.m.}{Nome comum a certos tipos de peixes de água doce.}{man.di}{0}
\verb{mandíbula}{}{Anat.}{}{}{s.f.}{Osso da face no qual se encaixam os dentes inferiores da arcada dentária; maxila inferior.}{man.dí.bu.la}{0}
\verb{mandibular}{}{}{}{}{adj.2g.}{Relativo a mandíbula.}{man.di.bu.lar}{0}
\verb{mandinga}{}{}{}{}{s.f.}{Obra de feitiçaria; feitiço, bruxaria.}{man.din.ga}{0}
\verb{mandingar}{}{}{}{}{v.t.}{Praticar mandingas; enfeitiçar.}{man.din.gar}{\verboinum{5}}
\verb{mandioca}{ó}{}{}{}{s.f.}{Arbusto nativo da América do Sul, largamente cultivado pelas grossas raízes ricas em amido, utilizadas na alimentação.}{man.di.o.ca}{0}
\verb{mandioca}{ó}{}{}{}{}{A raiz desse arbusto.}{man.di.o.ca}{0}
\verb{mandiocal}{}{}{"-ais}{}{s.m.}{Extensa plantação de pés de mandioca em determinada área.}{man.di.o.cal}{0}
\verb{mandioquinha}{}{}{}{}{s.f.}{Tipo de erva com grandes raízes amarelas, usada na alimentação.}{man.di.o.qui.nha}{0}
\verb{mando}{}{}{}{}{s.m.}{Ato ou efeito de mandar; comando, ordem, autoridade.}{man.do}{0}
\verb{mandonismo}{}{}{}{}{s.m.}{Tendência ou hábito de mandar ou abusar do poder em qualquer circunstância.}{man.do.nis.mo}{0}
\verb{mandrião}{}{}{"-ões}{mandriona}{adj.}{Que não gosta de trabalhar ou estudar; preguiçoso, indolente.}{man.dri.ão}{0}
\verb{mandriar}{}{}{}{}{v.i.}{Viver como mandrião, sem trabalhar nem estudar por preguiça.}{man.dri.ar}{\verboinum{1}}
\verb{mandril}{}{}{"-is}{}{s.m.}{Ponta ou haste que se introduz em um instrumento cujo motor a faz girar.}{man.dril}{0}
\verb{mandril}{}{}{"-is}{}{}{Peça usada para alargar e alisar os furos grandes, em certos trabalhos mecânicos.}{man.dril}{0}
\verb{mandril}{}{Zool.}{"-is}{}{s.m.}{Espécime de macaco de cauda curta e pelo esverdeado encontrado nas florestas úmidas africanas.}{man.dril}{0}
\verb{manducar}{}{}{}{}{v.i.}{Ingerir alimento; comer, mastigar.}{man.du.car}{\verboinum{2}}
\verb{mané}{}{Pop.}{}{}{adj.2g.}{Diz"-se do indivíduo pouco inteligente; bobo, tolo.}{ma.né}{0}
\verb{manear}{}{}{}{}{v.t.}{Manejar.}{ma.ne.ar}{\verboinum{4}}
\verb{maneio}{ê}{}{}{}{s.m.}{Ato ou efeito de manear; manejo.}{ma.nei.o}{0}
\verb{maneira}{ê}{}{}{}{s.f.}{Forma ou modo de ser, de agir; jeito, feitio.}{ma.nei.ra}{0}
\verb{maneira}{ê}{}{}{}{}{Característica inerente de algo; qualidade, tipo.}{ma.nei.ra}{0}
\verb{maneira}{ê}{}{}{}{}{Modo de compor, pintar ou escrever peculiar a um artista.}{ma.nei.ra}{0}
\verb{maneirar}{}{}{}{}{v.t.}{Dar um jeito; resolver com habilidade uma situação difícil.}{ma.nei.rar}{\verboinum{1}}
\verb{maneirismo}{}{Art.}{}{}{s.m.}{Movimento artístico"-literário europeu que precedeu o Barroco e se caracterizou pelo uso excessivo de certos procedimentos como o alongamento das figuras humanas e a visão extremamente pessimista do mundo.}{ma.nei.ris.mo}{0}
\verb{maneirismo}{}{}{}{}{}{Maneira afetada de falar e gesticular.}{ma.nei.ris.mo}{0}
\verb{maneirista}{}{}{}{}{adj.2g.}{Diz"-se do artista que adota o maneirismo como estilo.}{ma.nei.ris.ta}{0}
\verb{maneirista}{}{}{}{}{}{Que mostra afetação no modo de falar ou gesticular.}{ma.nei.ris.ta}{0}
\verb{maneiro}{ê}{}{}{}{adj.}{Que se maneja ou usa sem dificuldade; prático, jeitoso.}{ma.nei.ro}{0}
\verb{maneiro}{ê}{}{}{}{}{Que se move com facilidade; ligeiro.}{ma.nei.ro}{0}
\verb{maneiroso}{ô}{}{"-osos ⟨ó⟩}{"-osa ⟨ó⟩}{adj.}{Que apresenta boas maneiras; educado, polido.}{ma.nei.ro.so}{0}
\verb{manejar}{}{}{}{}{v.t.}{Usar as mãos para executar alguma atividade; manusear, manear.}{ma.ne.jar}{\verboinum{1}}
\verb{manejo}{ê}{}{}{}{s.m.}{Ato ou efeito de manejar; maneio, manuseio.}{ma.ne.jo}{0}
%Jorge: \verb{manemolência}{}{}{}{}{}{}{}
%Jorge: \verb{manemolente}{}{}{}{}{}{}{}
\verb{manequim}{}{}{"-ins}{}{s.m.}{Figura, representando homem ou mulher, usada por costureiros, artistas etc.}{ma.ne.quim}{0}
\verb{manequim}{}{}{"-ins}{}{s.2g.}{Pessoa que desfila modelos de costureiros ou posa para revistas de moda; modelo.}{ma.ne.quim}{0}
\verb{maneta}{ê}{}{}{}{adj.2g.}{Diz"-se do indivíduo a quem falta um dos braços ou uma das mãos.}{ma.ne.ta}{0}
\verb{manga}{}{}{}{}{s.f.}{Parte do vestuário que recobre o braço.}{man.ga}{0}
\verb{manga}{}{}{}{}{s.f.}{Fruto comestível da mangueira, de casca amarela a avermelhada, polpa fibrosa e caroço achatado.}{man.ga}{0}
\verb{mangaba}{}{}{}{}{s.f.}{Fruto comestível da mangabeira.}{man.ga.ba}{0}
\verb{mangabeira}{ê}{Bot.}{}{}{s.f.}{Árvore de madeira vermelha, da qual se extrai látex e cujo fruto comestível é usado no fabrico de vinho.}{man.ga.bei.ra}{0}
\verb{mangação}{}{}{"-ões}{}{s.f.}{Ato ou efeito de mangar; troça, zombaria.}{man.ga.ção}{0}
\verb{manga"-d'água}{}{}{mangas"-d'água}{}{s.f.}{Chuva repentina e de pouca duração; aguaceiro.}{man.ga"-d'á.gua}{0}
\verb{manga"-larga}{}{}{mangas"-largas}{}{adj.2g.}{Diz"-se de raça de cavalos marchadores, obtida do cruzamento de um puro"-sangue com égua.}{man.ga"-lar.ga}{0}
\verb{manganês}{}{Quím.}{}{}{s.m.}{Elemento químico, branco"-acinzentado, semelhante ao ferro, mas não magnético; usado na fabricação de aço. \elemento{25}{54.93805}{Mn}.}{man.ga.nês}{0}
\verb{mangar}{}{}{}{}{v.t.}{Expor ao ridículo; caçoar, zombar.}{man.gar}{\verboinum{5}}
\verb{mangual}{}{}{"-ais}{}{s.m.}{Instrumento rústico usado para debulhar cereais.}{man.gual}{0}
\verb{mangue}{}{}{}{}{s.m.}{Terreno pantanoso próximo de rio ou mar, coberto de lama escura e plantas baixas; brejo, charco, manguezal.}{man.gue}{0}
\verb{mangueira}{ê}{}{}{}{s.f.}{Tubo de borracha, lona, plástico ou outro material que conduz líquido ou gás.}{man.guei.ra}{0}
\verb{mangueira}{ê}{Bot.}{}{}{s.f.}{Árvore frondosa que produz frutos muito apreciados por sua polpa doce e suculenta.}{man.guei.ra}{0}
\verb{manguezal}{}{}{"-ais}{}{s.m.}{Mangue.}{man.gue.zal}{0}
\verb{manha}{}{}{}{}{s.f.}{Habilidade em realizar algo; destreza, astúcia.}{ma.nha}{0}
\verb{manha}{}{}{}{}{}{Choro ou lamento sem motivo; birra.}{ma.nha}{0}
\verb{manhã}{}{}{}{}{s.f.}{Período entre o nascer do Sol e o meio"-dia; o amanhecer.}{ma.nhã}{0}
\verb{manhã}{}{Fig.}{}{}{}{Princípio, início.}{ma.nhã}{0}
\verb{manhoso}{ô}{}{"-osos ⟨ó⟩}{"-osa ⟨ó⟩}{adj.}{Que tem manha; habilidoso, astuto.}{ma.nho.so}{0}
\verb{manhoso}{ô}{}{"-osos ⟨ó⟩}{"-osa ⟨ó⟩}{}{Birrento, chorão.}{ma.nho.so}{0}
\verb{mania}{}{Med.}{}{}{s.f.}{Estado de desordem mental caracterizado por superexcitação, euforia etc.}{ma.ni.a}{0}
\verb{mania}{}{}{}{}{}{Extravagância, capricho, excentricidade.}{ma.ni.a}{0}
\verb{maníaco}{}{}{}{}{adj.}{Relativo a mania.}{ma.ní.a.co}{0}
\verb{maníaco}{}{}{}{}{s.m.}{Indivíduo que possui manias.}{ma.ní.a.co}{0}
\verb{maniatar}{}{}{}{}{v.t.}{Manietar.}{ma.ni.a.tar}{\verboinum{1}}
\verb{maniçoba}{ó}{Bot.}{}{}{s.f.}{Árvore de que se extrai látex para fabricação de borracha.}{ma.ni.ço.ba}{0}
\verb{maniçoba}{ó}{Cul.}{}{}{}{Guisado preparado com as folhas da mandioca acrescentando"-se carne de porco temperada com alho, sal, louro e pimenta.}{ma.ni.ço.ba}{0}
\verb{manicômio}{}{}{}{}{s.m.}{Hospital para internação e tratamento de doentes mentais; hospício.}{ma.ni.cô.mio}{0}
\verb{manícula}{}{Zool.}{}{}{s.f.}{Cada um dos membros anteriores dos mamíferos.}{ma.ní.cu.la}{0}
\verb{manicura}{}{}{}{}{s.f.}{Mulher especializada no tratamento e embelezamento de mãos e unhas; manicure.}{ma.ni.cu.ra}{0}
\verb{manicure}{}{}{}{}{s.f.}{Manicura.}{ma.ni.cu.re}{0}
\verb{manicuro}{}{}{}{}{s.m.}{Indivíduo que se dedica ao tratamento e ao embelezamento de mãos e unhas.}{ma.ni.cu.ro}{0}
\verb{manidestro}{é}{}{}{}{adj.}{Diz"-se de indivíduo que tem mais facilidade em usar a mão direita; destro.}{ma.ni.des.tro}{0}
\verb{manietar}{}{}{}{}{v.t.}{Amarrar as mãos de outrem; imobilizar, prender.}{ma.ni.e.tar}{\verboinum{1}}
\verb{manifestação}{}{}{"-ões}{}{s.f.}{Ato ou efeito de manifestar.}{ma.ni.fes.ta.ção}{0}
\verb{manifestação}{}{}{"-ões}{}{}{Expressão coletiva e pública de pontos de vista e opiniões a respeito de algo.}{ma.ni.fes.ta.ção}{0}
\verb{manifestante}{}{}{}{}{adj.2g.}{Que manifesta; que participa de manifestações.}{ma.ni.fes.tan.te}{0}
\verb{manifestar}{}{}{}{}{v.t.}{Tornar manifesto, público; revelar, exprimir.}{ma.ni.fes.tar}{0}
\verb{manifestar}{}{}{}{}{}{Apresentar, divulgar, declarar.}{ma.ni.fes.tar}{0}
\verb{manifestar}{}{}{}{}{v.pron.}{Emitir opinião sobre algo; declarar"-se.}{ma.ni.fes.tar}{\verboinum{1}}
\verb{manifesto}{é}{}{}{}{adj.}{Patente, evidente, claro.}{ma.ni.fes.to}{0}
\verb{manifesto}{é}{}{}{}{s.m.}{Declaração pública de programas, objetivos, razões etc.}{ma.ni.fes.to}{0}
\verb{manifesto}{é}{}{}{}{}{Relação de mercadorias sujeitas a pagamentos de direitos alfandegários.}{ma.ni.fes.to}{0}
\verb{manilha}{}{}{}{}{s.f.}{Tubo de barro cozido ou de concreto, usado em canalizações.}{ma.ni.lha}{0}
\verb{manilha}{}{}{}{}{}{Certo tipo de papel feito de fibras muito resistentes.}{ma.ni.lha}{0}
\verb{maninho}{}{}{}{}{adj.}{Diz"-se do terreno que não serve para o cultivo; estéril, infecundo.}{ma.ni.nho}{0}
\verb{maninho}{}{}{}{}{s.m.}{Diminutivo de \textit{mano}; irmãozinho.}{ma.ni.nho}{0}
\verb{manipanso}{}{}{}{}{s.m.}{Ídolo africano; fetiche.}{ma.ni.pan.so}{0}
\verb{manipanso}{}{Fig.}{}{}{}{Indivíduo muito baixo e gordo.}{ma.ni.pan.so}{0}
\verb{manipulação}{}{}{"-ões}{}{s.f.}{Manejo, utilização.}{ma.ni.pu.la.ção}{0}
\verb{manipulação}{}{}{"-ões}{}{}{Elaboração de produtos químicos ou farmacêuticos.}{ma.ni.pu.la.ção}{0}
\verb{manipulação}{}{}{"-ões}{}{}{Manobra com o fim de enganar.}{ma.ni.pu.la.ção}{0}
\verb{manipulador}{ô}{}{}{}{adj.}{Que manipula.}{ma.ni.pu.la.dor}{0}
\verb{manipulador}{ô}{}{}{}{s.m.}{Instrumento sobre o qual o telegrafista faz pressões para transmitir os sinais.}{ma.ni.pu.la.dor}{0}
\verb{manipular}{}{}{}{}{v.t.}{Preparar, acionar ou controlar com as mãos.}{ma.ni.pu.lar}{0}
\verb{manipular}{}{}{}{}{}{Misturar manualmente.}{ma.ni.pu.lar}{0}
\verb{manipular}{}{}{}{}{}{Adulterar.}{ma.ni.pu.lar}{0}
\verb{manipular}{}{}{}{}{}{Influenciar indivíduo, coletividade.}{ma.ni.pu.lar}{\verboinum{1}}
\verb{manirroto}{ô}{}{}{}{adj.}{Que gasta à larga; dissipador, perdulário.}{ma.nir.ro.to}{0}
\verb{manitó}{}{}{}{}{s.m.}{Gênio tutelar ou demônio entre os índios americanos.               }{ma.ni.tó}{0}
\verb{manivela}{é}{}{}{}{s.f.}{Barra que, com rotação manual, aciona máquina.}{ma.ni.ve.la}{0}
\verb{manjado}{}{Pop.}{}{}{adj.}{Muito conhecido.}{man.ja.do}{0}
\verb{manjar}{}{}{}{}{v.t.}{Levar algum alimento à boca e engolir; comer.}{man.jar}{0}
\verb{manjar}{}{}{}{}{}{Ficar sabendo de alguma coisa que o outro quer esconder; compreender, perceber.}{man.jar}{0}
\verb{manjar}{}{Pop.}{}{}{}{Ficar olhando alguma coisa com atenção; observar.}{man.jar}{0}
\verb{manjar}{}{Pop.}{}{}{}{Ter conhecimento de algum assunto.}{man.jar}{\verboinum{1}}
\verb{manjar}{}{Cul.}{}{}{}{Comida delicada e gostosa; iguaria.}{man.jar}{0}
\verb{manjar"-branco}{}{Cul.}{manjares"-brancos}{}{s.m.}{Pudim feito com leite e maisena, servido geralmente com calda de ameixa"-preta. }{man.jar"-bran.co}{0}
\verb{manjedoura}{ô}{}{}{}{s.f.}{Local de um estábulo onde os animais comem.}{man.je.dou.ra}{0}
\verb{manjericão}{}{}{"-ões}{}{s.m.}{Erva de cheiro muito agradável, com folhas pequenas e flores em cachos, usada como remédio ou tempero.}{man.je.ri.cão}{0}
\verb{manjerona}{}{}{}{}{s.f.}{Erva de cheiro agradável, com galhos avermelhados e folhas com pelos curtos, usada como remédio ou tempero.}{man.je.ro.na}{0}
\verb{manjuba}{}{Zool.}{}{}{s.f.}{Tipo de peixe muito pequeno encontrado no mar, de grande valor econômico para as indústrias pesqueiras. }{man.ju.ba}{0}
\verb{mano}{}{Pop.}{}{}{s.m.}{Irmão.}{ma.no}{0}
\verb{manobra}{ó}{}{}{}{s.f.}{Cada um dos movimentos executados quando se usa uma máquina.}{ma.no.bra}{0}
\verb{manobra}{ó}{}{}{}{}{Conjunto de movimentos para se alcançar uma finalidade.}{ma.no.bra}{0}
\verb{manobra}{ó}{}{}{}{}{Maneira esperta de conseguir alguma coisa; ardil, artimanha, jogada.}{ma.no.bra}{0}
\verb{manobrar}{}{}{}{}{v.t.}{Fazer veículo ou máquina se mover para determinado fim.}{ma.no.brar}{0}
\verb{manobrar}{}{}{}{}{}{Usar uma maneira esperta para levar uma pessoa a fazer alguma coisa que se quer dela.}{ma.no.brar}{0}
\verb{manobrar}{}{}{}{}{v.i.}{Realizar exercícios militares.}{ma.no.brar}{\verboinum{1}}
\verb{manobreiro}{ê}{}{}{}{s.m.}{Indivíduo que faz manobras.}{ma.no.brei.ro}{0}
\verb{manobreiro}{ê}{}{}{}{}{Indivíduo que manobra automóveis em estacionamento ou garagens; manobrista.}{ma.no.brei.ro}{0}
\verb{manobrista}{}{}{}{}{s.2g.}{Manobreiro.}{ma.no.bris.ta}{0}
\verb{manômetro}{}{}{}{}{s.m.}{Instrumento que mede a pressão em fluidos.}{ma.nô.me.tro}{0}
\verb{manopla}{ó}{Pop.}{}{}{s.f.}{Mão grande, grossa.}{ma.no.pla}{0}
\verb{manopla}{ó}{}{}{}{}{Luva de ferro em armadura.}{ma.no.pla}{0}
\verb{manqueira}{ê}{}{}{}{s.f.}{Defeito no andar.}{man.quei.ra}{0}
\verb{manqueira}{ê}{}{}{}{}{Doença que deixa bois e cavalos mancos.}{man.quei.ra}{0}
\verb{manquejante}{}{}{}{}{adj.2g.}{Que manqueja; claudicante.}{man.que.jan.te}{0}
\verb{manquejar}{}{}{}{}{v.i.}{Andar pouco ou devagar, apresentando dificuldade.}{man.que.jar}{\verboinum{1}}
\verb{manquitola}{ó}{Pop.}{}{}{s.2g.}{Manco.}{man.qui.to.la}{0}
\verb{manquitolar}{}{}{}{}{v.i.}{Mancar.}{man.qui.to.lar}{\verboinum{1}}
\verb{mansão}{}{}{"-ões}{}{s.f.}{Casa muito grande e luxuosa.}{man.são}{0}
\verb{mansarda}{}{}{}{}{s.f.}{Quarto abaixo do telhado.}{man.sar.da}{0}
\verb{mansidão}{}{}{"-ões}{}{s.f.}{Qualidade ou condição do que é manso; brandura, calma.}{man.si.dão}{0}
\verb{mansidão}{}{}{"-ões}{}{}{Tranquilidade, sossego.}{man.si.dão}{0}
\verb{manso}{}{}{}{}{adj.}{Que não ataca.}{man.so}{0}
\verb{manso}{}{}{}{}{}{Que não se agita; tranquilo.}{man.so}{0}
\verb{mansuetude}{}{}{}{}{s.f.}{Serenidade, mansidão.}{man.su.e.tu.de}{0}
\verb{manta}{}{}{}{}{s.f.}{Cobertor ou certo agasalho geralmente de lã.}{man.ta}{0}
\verb{manta}{}{}{}{}{}{Faixa de carne da costela ou do peito da rês.}{man.ta}{0}
\verb{manteiga}{ê}{}{}{}{s.f.}{Substância gordurosa que se tira da nata do leite e serve como alimento.}{man.tei.ga}{0}
\verb{manteiga}{ê}{}{}{}{}{Substância gordurosa de algumas plantas.}{man.tei.ga}{0}
\verb{manteigueira}{ê}{}{}{}{s.f.}{Vasilha para servir manteiga.}{man.tei.guei.ra}{0}
\verb{mantelete}{ê}{}{}{}{s.m.}{Tipo de capa curta.}{man.te.le.te}{0}
\verb{mantenedor}{ô}{}{}{}{adj.}{Que mantém, sustenta; mantedor.}{man.te.ne.dor}{0}
\verb{mantenedor}{ô}{}{}{}{}{Que defende, protege.}{man.te.ne.dor}{0}
\verb{manter}{ê}{}{}{}{v.t.}{Fornecer o necessário para uma pessoa ou organização continuar existindo; sustentar.}{man.ter}{0}
\verb{manter}{ê}{}{}{}{}{Conservar pessoa ou coisa em algum estado ou lugar.}{man.ter}{0}
\verb{manter}{ê}{}{}{}{}{Ficar firme em alguma coisa que disse; cumprir.}{man.ter}{\verboinum{39}}
\verb{mantilha}{}{}{}{}{s.f.}{Véu que cai sobre os ombros.}{man.ti.lha}{0}
\verb{mantimento}{}{}{}{}{s.m.}{Sustento, comida.}{man.ti.men.to}{0}
\verb{manto}{}{}{}{}{s.m.}{Peça de vestuário, sem mangas, larga e comprida, usada sobre outra roupa.}{man.to}{0}
\verb{mantô}{}{}{}{}{s.m.}{Casaco comprido semelhante ao manto, em geral de lã, usado pelas mulheres sobre outra roupa.}{man.tô}{0}
\verb{mantuano}{}{}{}{}{adj.}{Relativo a Mântua, cidade localizada às margens do rio Pó, na Itália.}{man.tu.a.no}{0}
\verb{mantuano}{}{}{}{}{s.m.}{Indivíduo natural ou habitante dessa cidade.}{man.tu.a.no}{0}
\verb{manual}{}{}{"-ais}{}{adj.2g.}{Das mãos.}{ma.nu.al}{0}
\verb{manual}{}{}{"-ais}{}{}{Feito com as mãos.}{ma.nu.al}{0}
\verb{manual}{}{}{"-ais}{}{s.m.}{Livro que contém os conhecimentos básicos de alguma coisa.}{ma.nu.al}{0}
\verb{manufatura}{}{}{}{}{s.f.}{Trabalho feito à mão.}{ma.nu.fa.tu.ra}{0}
\verb{manufatura}{}{}{}{}{}{Fábrica mecanizada.}{ma.nu.fa.tu.ra}{0}
\verb{manufatura}{}{}{}{}{}{Produto fabril.}{ma.nu.fa.tu.ra}{0}
\verb{manufaturar}{}{}{}{}{v.t.}{Fabricar manualmente.}{ma.nu.fa.tu.rar}{0}
\verb{manufaturar}{}{}{}{}{}{Produzir em manufatura.}{ma.nu.fa.tu.rar}{\verboinum{1}}
\verb{manufatureiro}{ê}{}{}{}{adj.}{Referente a manufatura.}{ma.nu.fa.tu.rei.ro}{0}
\verb{manuscrever}{ê}{}{}{}{v.t.}{Escrever à mão.}{ma.nus.cre.ver}{\verboinum{12}}
\verb{manuscrito}{}{}{}{}{adj.}{Escrito à mão.}{ma.nus.cri.to}{0}
\verb{manuscrito}{}{}{}{}{s.m.}{Obra escrita à mão.}{ma.nus.cri.to}{0}
\verb{manuscrito}{}{}{}{}{}{Original de uma obra.}{ma.nus.cri.to}{0}
\verb{manusear}{}{}{}{}{v.t.}{Servir"-se das mãos para usar alguma coisa.}{ma.nu.se.ar}{0}
\verb{manusear}{}{}{}{}{}{Servir"-se das mãos para examinar alguma obra escrita.}{ma.nu.se.ar}{\verboinum{4}}
\verb{manuseio}{ê}{}{}{}{s.m.}{Utilização de algo servindo"-se das mãos; manejo.}{ma.nu.sei.o}{0}
\verb{manutenção}{}{}{"-ões}{}{s.f.}{Cuidado para fazer algo durar.}{ma.nu.ten.ção}{0}
\verb{manzorra}{ô}{}{}{}{s.f.}{Mão grande e feia; manopla.}{man.zor.ra}{0}
\verb{mão}{}{}{mãos}{}{s.f.}{Parte do corpo do homem ou do macaco em que o braço termina.}{mão}{0}
\verb{mão}{}{}{mãos}{}{}{Pata dianteira de animal de quatro pernas.}{mão}{0}
\verb{mão}{}{}{mãos}{}{}{Direção que o veículo deve tomar numa rua.}{mão}{0}
\verb{mão}{}{}{mãos}{}{}{Camada de alguma coisa que se passa numa superfície.}{mão}{0}
\verb{mão"-aberta}{é}{}{mãos"-abertas ⟨é⟩}{}{s.m.}{Indivíduo que se desfaz do que tem para ajudar os outros; generoso.}{mão"-a.ber.ta}{0}
\verb{mão"-aberta}{é}{}{mãos"-abertas ⟨é⟩}{}{}{Indivíduo que gasta muito; perdulário, gastador.}{mão"-a.ber.ta}{0}
\verb{mão"-boba}{ô}{}{mãos"-bobas ⟨ô⟩}{}{s.f.}{Gesto de quem se finge distraído e com a mão procura tocar o corpo de uma pessoa de forma libidinosa, ou com a intenção de furtar.}{mão"-bo.ba}{0}
\verb{mão"-boba}{ô}{}{mãos"-bobas ⟨ô⟩}{}{s.m.}{Indivíduo que pratica esse gesto.}{mão"-bo.ba}{0}
\verb{mão"-cheia}{ê}{}{mãos"-cheias}{}{s.f.}{Punhado, o que se pode conter dentro da mão fechada.}{mão"-chei.a}{0}
\verb{mão"-cheia}{ê}{}{mãos"-cheias}{}{}{Usado na expressão \textit{de mão"-cheia}: ótimo, excelente. (\textit{A moça é uma cozinheira de mão"-cheia.})}{mão"-chei.a}{0}
\verb{mão"-de"-obra}{ó}{}{mãos"-de"-obra ⟨ó⟩}{}{s.f.}{Trabalho manual de um operário.}{mão"-de"-o.bra}{0}
\verb{mão"-de"-obra}{ó}{}{mãos"-de"-obra ⟨ó⟩}{}{}{Conjunto de trabalhadores necessários para a execução de um trabalho.}{mão"-de"-o.bra}{0}
\verb{mão"-de"-obra}{ó}{}{mãos"-de"-obra ⟨ó⟩}{}{}{Custo desse trabalho.}{mão"-de"-o.bra}{0}
\verb{mão"-de"-obra}{ó}{}{mãos"-de"-obra ⟨ó⟩}{}{}{Trabalho difícil de fazer.}{mão"-de"-o.bra}{0}
\verb{maometano}{}{}{}{}{adj.}{Que se refere à religião fundada por Maomé.}{ma.o.me.ta.no}{0}
\verb{maometano}{}{}{}{}{s.m.}{Seguidor dessa religião; muçulmano.}{ma.o.me.ta.no}{0}
\verb{maometismo}{}{}{}{}{s.m.}{Religião fundada pelo profeta Maomé; islamismo.}{ma.o.me.tis.mo}{0}
\verb{mãos"-rotas}{ô}{}{}{}{s.2g.2n}{Indivíduo que gasta excessivamente; esbanjador, gastador.}{mãos"-ro.tas}{0}
\verb{mãozada}{}{Pop.}{}{}{s.f.}{Quantidade de coisas que podem caber numa das mãos.}{mão.za.da}{0}
\verb{mãozada}{}{}{}{}{}{Aperto de mão dado com força.}{mão.za.da}{0}
\verb{mãozada}{}{}{}{}{}{Golpe desferido com as mãos.}{mão.za.da}{0}
\verb{mãozudo}{}{}{}{}{adj.}{Que tem mãos grandes.}{mão.zu.do}{0}
\verb{mapa}{}{}{}{}{s.m.}{Representação gráfica de um território ou região, com pontos de referência relevantes e em uma escala determinada. (\textit{Chegamos até o local graças ao mapa que tínhamos na bagagem.})}{ma.pa}{0}
\verb{mapa}{}{Por ext.}{}{}{}{Descrição ou representação de qualquer situação, com dados fartos e relevantes; quadro, lista. (\textit{Foi feito um mapa da situação atual dos meninos de rua.})}{ma.pa}{0}
\verb{mapa"-múndi}{}{}{mapas"-múndi}{}{s.m.}{Mapa que representa a totalidade da superfície terrestre.}{ma.pa"-mún.di}{0}
\verb{mapeamento}{}{}{}{}{s.m.}{Ato ou efeito de mapear.}{ma.pe.a.men.to}{0}
\verb{mapear}{}{}{}{}{v.t.}{Selecionar pontos de referência de um território ou dados relevantes de determinado assunto e organizá"-los, produzindo um mapa.}{ma.pe.ar}{\verboinum{4}}
\verb{mapoteca}{é}{}{}{}{s.f.}{Conjunto organizado de mapas.}{ma.po.te.ca}{0}
\verb{maqueta}{é}{}{}{}{s.f.}{Maquete.}{ma.que.ta}{0}
\verb{maquete}{é}{}{}{}{s.f.}{Representação tridimensional em miniatura de obra arquitetônica ou de engenharia.}{ma.que.te}{0}
\verb{maquete}{é}{}{}{}{}{Esboço em miniatura de escultura.}{ma.que.te}{0}
\verb{maquiagem}{}{}{"-ens}{}{s.f.}{Ato ou efeito de maquiar.}{ma.qui.a.gem}{0}
\verb{maquiagem}{}{}{"-ens}{}{}{O conjunto dos produtos usados para maquiar.}{ma.qui.a.gem}{0}
\verb{maquiar}{}{}{}{}{v.t.}{Aplicar no rosto produtos que alteram sua aparência, para torná"-lo mais belo, disfarçar imperfeições ou mudar a fisionomia.}{ma.qui.ar}{0}
\verb{maquiar}{}{Fig.}{}{}{}{Disfarçar, mascarar, ocultar.}{ma.qui.ar}{\verboinum{6}}
\verb{maquiavélico}{}{}{}{}{adj.}{Relativo ao maquiavelismo.}{ma.qui.a.vé.li.co}{0}
\verb{maquiavélico}{}{Fig.}{}{}{}{Em que há astúcia, manha.}{ma.qui.a.vé.li.co}{0}
\verb{maquiavelismo}{}{}{}{}{s.m.}{Sistema político de Maquiavel (1469--1527), estadista florentino, regido pela crença de que os fins justificam os meios.}{ma.qui.a.ve.lis.mo}{0}
\verb{maquiavelismo}{}{Fig.}{}{}{}{Procedimento astucioso, manhoso.}{ma.qui.a.ve.lis.mo}{0}
\verb{maquilagem}{}{}{"-ens}{}{s.f.}{Maquiagem.}{ma.qui.la.gem}{0}
\verb{maquilar}{}{}{}{}{v.t.}{Maquiar.}{ma.qui.lar}{\verboinum{1}}
\verb{máquina}{}{}{}{}{s.f.}{Aparelho que utiliza energia para produzir movimento e realizar uma tarefa.}{má.qui.na}{0}
\verb{máquina}{}{}{}{}{}{O núcleo de peças que efetivamente produz o movimento em qualquer aparelho.}{má.qui.na}{0}
\verb{máquina}{}{}{}{}{}{Veículo que utiliza energia para produzir movimento e transmiti"-lo a outros.}{má.qui.na}{0}
\verb{máquina}{}{Fig.}{}{}{}{Qualquer sistema organizado, especialmente os que apresentam complexidade e oferecem dificuldade de se lidar.}{má.qui.na}{0}
\verb{maquinação}{}{}{"-ões}{}{s.f.}{Ato ou efeito de maquinar; trama, manobra.}{ma.qui.na.ção}{0}
\verb{maquinador}{ô}{}{}{}{adj.}{Que maquina; tramador, conspirador.}{ma.qui.na.dor}{0}
\verb{maquinal}{}{}{"-ais}{}{adj.2g.}{Relativo a máquina.}{ma.qui.nal}{0}
\verb{maquinal}{}{Fig.}{"-ais}{}{}{Realizado de maneira automática; mecânico, inconsciente.}{ma.qui.nal}{0}
\verb{maquinar}{}{}{}{}{v.t.}{Projetar em segredo e com artifícios; tramar.}{ma.qui.nar}{\verboinum{1}}
\verb{maquinaria}{}{}{}{}{s.f.}{Maquinário.}{ma.qui.na.ri.a}{0}
\verb{maquinário}{}{}{}{}{s.m.}{O conjunto das máquinas de uma instituição ou de um veículo.}{ma.qui.ná.rio}{0}
\verb{maquinismo}{}{}{}{}{s.m.}{O conjunto de peças e mecanismos de um aparelho.}{ma.qui.nis.mo}{0}
\verb{maquinismo}{}{}{}{}{}{Maquinário.}{ma.qui.nis.mo}{0}
\verb{maquinista}{}{}{}{}{s.2g.}{Condutor ou operador de máquinas de locomotiva ou navio.}{ma.qui.nis.ta}{0}
\verb{maquinista}{}{}{}{}{}{Inventor ou construtor de máquinas.}{ma.qui.nis.ta}{0}
\verb{maquinista}{}{}{}{}{}{Nas artes cênicas, o profissional que opera os mecanismos do palco.}{ma.qui.nis.ta}{0}
\verb{mar}{}{}{}{}{s.m.}{A extensão de águas salgadas do globo terrestre; oceano. (\textit{O mar ocupa dois terços da superfície terrestre.})}{mar}{0}
\verb{mar}{}{}{}{}{}{Cada porção em que se divide, arbitrariamente, o oceano. (\textit{A cidade é banhada pelo Mar do Caribe.})}{mar}{0}
\verb{mar}{}{}{}{}{}{Grande lago de água salgada situado no interior de um continente. (\textit{Mar Morto, Mar Cáspio.})}{mar}{0}
\verb{mar}{}{Fig.}{}{}{}{Grande quantidade ou extensão; imensidão, oceano. (\textit{Há um mar entre nossos sonhos e a realidade.})}{mar}{0}
\verb{marabá}{}{Bras.}{}{}{s.2g.}{Mestiço de um francês com uma índia.}{ma.ra.bá}{0}
\verb{marabá}{}{Bras.}{}{}{}{Mestiço entre índio e não índio; mameluco.}{ma.ra.bá}{0}
\verb{maracá}{}{Bras.}{}{}{s.m.}{Tipo de chocalho de origem indígena usado em música popular, danças e cerimônias religiosas de alguns povos.}{ma.ra.cá}{0}
\verb{maracanã}{}{Zool.}{}{}{s.f.}{Designação comum de algumas espécies de pássaros verdes encontrados por todo o Brasil.}{ma.ra.ca.nã}{0}
\verb{maracatu}{}{Bras.}{}{}{s.m.}{Dança popular em que um bloco fantasiado segue uma boneca enfeitada ao som de tambores.}{ma.ra.ca.tu}{0}
\verb{maracatu}{}{Mús.}{}{}{}{O padrão rítmico característico dessa dança.}{ma.ra.ca.tu}{0}
\verb{maracatu}{}{Mús.}{}{}{}{Música composta sobre esse padrão rítmico.}{ma.ra.ca.tu}{0}
\verb{maracujá}{}{}{}{}{s.m.}{Fruto comestível do maracujazeiro, com propriedades calmantes.}{ma.ra.cu.já}{0}
\verb{maracujazeiro}{ê}{Bras.}{}{}{s.m.}{Planta trepadeira que dá o maracujá.}{ma.ra.cu.ja.zei.ro}{0}
\verb{maracutaia}{}{Pop.}{}{}{s.m.}{Negócio ilícito, especialmente em administração pública; fraude, corrupção.}{ma.ra.cu.tai.a}{0}
\verb{marafona}{}{}{}{}{s.f.}{Boneca de trapos, geralmente sem rosto.}{ma.ra.fo.na}{0}
\verb{marafona}{}{Pejor.}{}{}{}{Prostituta.}{ma.ra.fo.na}{0}
\verb{marajá}{}{}{}{}{s.m.}{Título dado aos príncipes na Índia.}{ma.ra.já}{0}
\verb{marajá}{}{Bras.}{}{}{s.2g.}{Funcionário público com salário incrivelmente alto, muitas regalias e que geralmente trabalha pouco.}{ma.ra.já}{0}
\verb{marajoara}{}{}{}{}{adj.2g.}{Relativo à ilha de Marajó, no estado do Pará.}{ma.ra.jo.a.ra}{0}
\verb{marajoara}{}{}{}{}{s.2g.}{Indivíduo natural ou habitante dessa ilha.}{ma.ra.jo.a.ra}{0}
\verb{maranha}{}{}{}{}{s.f.}{Porção de fios enredados.}{ma.ra.nha}{0}
\verb{maranha}{}{}{}{}{}{Crespidão, desalinho dos cabelos.}{ma.ra.nha}{0}
\verb{maranha}{}{}{}{}{}{Coisa ou negócio intrincado; confusão, intriga.}{ma.ra.nha}{0}
\verb{maranha}{}{}{}{}{}{Pacto, conluio.}{ma.ra.nha}{0}
\verb{maranhense}{}{}{}{}{adj.2g.}{Relativo a Maranhão.}{ma.ra.nhen.se}{0}
\verb{maranhense}{}{}{}{}{s.2g.}{Indivíduo natural ou habitante desse estado.}{ma.ra.nhen.se}{0}
\verb{marani}{}{}{}{}{s.f.}{Mulher de marajá.}{ma.ra.ni}{0}
\verb{marasca}{}{}{}{}{s.f.}{Variedade de cereja amarga, empregada na fabricação do marasquino.}{ma.ras.ca}{0}
\verb{marasmo}{}{}{}{}{s.m.}{Grande fraqueza; debilidade.}{ma.ras.mo}{0}
\verb{marasmo}{}{}{}{}{}{Ausência de atividade; estagnação, paralisação, monotonia.}{ma.ras.mo}{0}
\verb{marasmo}{}{}{}{}{}{Tristeza profunda, melancolia, apatia.}{ma.ras.mo}{0}
\verb{marasquino}{}{}{}{}{s.m.}{Licor feito de marascas.}{ma.ras.qui.no}{0}
\verb{maratona}{}{Esport.}{}{}{s.f.}{Corrida de 42 km de extensão feita a pé, em referência ao soldado que atravessou a planície de Maratona em direção a Atenas.}{ma.ra.to.na}{0}
\verb{maratona}{}{}{}{}{}{Corrida de longo percurso a pé.}{ma.ra.to.na}{0}
\verb{maratona}{}{Fig.}{}{}{}{Atividade muito extensa ou desgastante.}{ma.ra.to.na}{0}
\verb{maravalhas}{}{}{}{}{s.f.pl.}{Aparas de madeira; lascas, gravetos.}{ma.ra.va.lhas}{0}
\verb{maravalhas}{}{}{}{}{}{Pedacinhos, fragmentos, coisas insignificantes.}{ma.ra.va.lhas}{0}
\verb{maravilha}{}{}{}{}{s.f.}{Fato extraordinário que causa grande admiração.}{ma.ra.vi.lha}{0}
\verb{maravilha}{}{}{}{}{}{Coisa ou pessoa que se destaca causando admiração.}{ma.ra.vi.lha}{0}
\verb{maravilha}{}{}{}{}{}{Fato incompreensível; milagre.}{ma.ra.vi.lha}{0}
\verb{maravilha}{}{Bras.}{}{}{}{Planta com flores de cores variadas e raiz com propriedades purgativas.}{ma.ra.vi.lha}{0}
\verb{maravilhar}{}{}{}{}{v.t.}{Provocar admiração; fascinar.}{ma.ra.vi.lhar}{\verboinum{1}}
\verb{maravilhoso}{ô}{}{"-osos ⟨ó⟩}{"-osa ⟨ó⟩}{}{Belo, magnífico, primoroso.}{ma.ra.vi.lho.so}{0}
\verb{maravilhoso}{ô}{}{"-osos ⟨ó⟩}{"-osa ⟨ó⟩}{adj.}{Que causa admiração; espantoso, fascinante.}{ma.ra.vi.lho.so}{0}
\verb{marca}{}{}{}{}{s.f.}{Sinal deixado por alguma coisa; vestígio.}{mar.ca}{0}
\verb{marca}{}{}{}{}{}{Sinal colocado em um produto industrial para indicar o fabricante; etiqueta.}{mar.ca}{0}
\verb{marcação}{}{}{"-ões}{}{s.f.}{Ato ou efeito de marcar.}{mar.ca.ção}{0}
\verb{marca"-d'água}{}{}{marcas"-d'água}{}{s.f.}{Desenho ou escrito feito em folha de papel e visível somente contra a luz, por transparência; filigrana.}{mar.ca"-d'á.gua}{0}
\verb{marcado}{}{}{}{}{adj.}{Que recebeu uma marca; assinalado.}{mar.ca.do}{0}
\verb{marcado}{}{}{}{}{}{Combinado, reservado.}{mar.ca.do}{0}
\verb{marcado}{}{}{}{}{}{Condenado, estigmatizado.}{mar.ca.do}{0}
\verb{marcador}{ô}{}{}{}{adj.}{Que marca.}{mar.ca.dor}{0}
\verb{marcador}{ô}{}{}{}{s.m.}{Quadro em que se marca a pontuação de um jogo ou competição.}{mar.ca.dor}{0}
\verb{marcante}{}{}{}{}{adj.2g.}{Que deixa marca, que se torna evidente ou é lembrado devido a alguma qualidade particular.}{mar.can.te}{0}
\verb{marcar}{}{}{}{}{v.t.}{Pôr sinal ou marca em; assinalar.}{mar.car}{0}
\verb{marcar}{}{}{}{}{}{Determinar, fixar, combinar, designar.}{mar.car}{0}
\verb{marcar}{}{}{}{}{}{Deixar sequela, vestígio; macular.}{mar.car}{0}
\verb{marcar}{}{Esport.}{}{}{}{Acompanhar os movimentos do jogador adversário para dificultar sua atuação.}{mar.car}{\verboinum{2}}
\verb{marcela}{é}{Bot.}{}{}{s.f.}{Macela.}{mar.ce.la}{0}
\verb{marcenaria}{}{}{}{}{s.f.}{Conjunto de técnicas de trabalho com madeira, para se produzir móveis e objetos de decoração.}{mar.ce.na.ri.a}{0}
\verb{marcenaria}{}{}{}{}{}{Ofício de marceneiro.}{mar.ce.na.ri.a}{0}
\verb{marcenaria}{}{}{}{}{}{Estabelecimento de marceneiro.}{mar.ce.na.ri.a}{0}
\verb{marceneiro}{ê}{}{}{}{s.m.}{Profissional que fabrica peças de madeira e que domina a técnica da marcenaria.}{mar.ce.nei.ro}{0}
\verb{marcha}{}{}{}{}{s.f.}{Ato de marchar.}{mar.cha}{0}
\verb{marcha}{}{}{}{}{}{Maneira ou velocidade com a qual algo se desenvolve; progresso, andamento.}{mar.cha}{0}
\verb{marcha}{}{}{}{}{}{Cada uma das combinações de engrenagens que dão diferentes velocidades a um veículo.}{mar.cha}{0}
\verb{marcha}{}{Mús.}{}{}{}{Padrão rítmico binário.}{mar.cha}{0}
\verb{marcha}{}{}{}{}{}{Música composta de acordo com esse padrão rítmico, eventualmente para acompanhar a marcha de uma tropa ou grupo de pessoas.}{mar.cha}{0}
\verb{marchand}{}{Bras.}{}{}{s.2g.}{Indivíduo que negocia com obras de arte.}{\textit{marchand}}{0}
\verb{marchante}{}{}{}{}{s.2g.}{Indivíduo que compra gado e vende para açougues.}{mar.chan.te}{0}
\verb{marchar}{}{}{}{}{v.t.}{Caminhar, andar.}{mar.char}{0}
\verb{marchar}{}{}{}{}{}{Andar em ritmo de marcha, com passo cadenciado.}{mar.char}{0}
\verb{marchar}{}{}{}{}{}{Avançar, evoluir, progredir.}{mar.char}{\verboinum{1}}
\verb{marchetar}{}{}{}{}{v.t.}{Aplicar ou embutir peças de madeira ou outros materiais, formando desenhos.}{mar.che.tar}{0}
\verb{marchetar}{}{Fig.}{}{}{}{Adornar, realçar.}{mar.che.tar}{\verboinum{1}}
\verb{marchetaria}{}{}{}{}{s.f.}{Técnica de aplicar ou embutir peças de madeira, metal, marfim ou outros materiais para formar desenhos em uma superfície.}{mar.che.ta.ri.a}{0}
\verb{marchetaria}{}{}{}{}{}{A obra executada com essa técnica.}{mar.che.ta.ri.a}{0}
\verb{marchete}{ê}{}{}{}{s.m.}{Cada uma das peças que se aplicam ou embutem na madeira em uma obra de marchetaria.}{mar.che.te}{0}
\verb{marchinha}{}{}{}{}{s.f.}{Tipo de música carnavalesca.}{mar.chi.nha}{0}
\verb{marcial}{}{}{"-ais}{}{adj.2g.}{Que se refere a militares ou a guerra.}{mar.ci.al}{0}
\verb{marciano}{}{}{}{}{adj.}{Relativo ao planeta Marte.}{mar.ci.a.no}{0}
\verb{marciano}{}{}{}{}{s.m.}{Suposto habitante desse planeta.}{mar.ci.a.no}{0}
\verb{março}{}{}{}{}{s.m.}{O terceiro mês do ano civil.}{mar.ço}{0}
\verb{marco}{}{}{}{}{s.m.}{Sinal que indica algum limite ou alguma distância.}{mar.co}{0}
\verb{marco}{}{}{}{}{}{Coluna que assinala um local ou um acontecimento.}{mar.co}{0}
\verb{marco}{}{}{}{}{}{Data importante que define uma época.}{mar.co}{0}
\verb{marco}{}{}{}{}{}{Moeda da Alemanha.}{mar.co}{0}
\verb{maré}{}{}{}{}{s.f.}{Movimento das águas do mar, que sobem ou descem regular e periodicamente, produzido pela ação da Lua ou do Sol.}{ma.ré}{0}
\verb{maré}{}{}{}{}{}{Vaivém dos acontecimentos.}{ma.ré}{0}
\verb{maré}{}{}{}{}{}{Brincadeira em que a criança pula com um ou dois pés, conforme aparecer à frente um ou dois quadrados riscados no chão, e pega uma pedra jogada num desses quadrados, sem pisar nele; amarelinha.}{ma.ré}{0}
\verb{mareação}{}{}{"-ões}{}{s.f.}{Ato ou efeito de marear.}{ma.re.a.ção}{0}
\verb{mareação}{}{}{"-ões}{}{}{Sensação desagradável de quem está para vomitar; enjoo.}{ma.re.a.ção}{0}
\verb{marear}{}{}{}{}{v.t.}{Governar navio.}{ma.re.ar}{0}
\verb{marear}{}{}{}{}{}{Fazer uma pessoa enjoar.}{ma.re.ar}{\verboinum{4}}
\verb{marechal}{}{}{"-ais}{}{s.m.}{Posto mais alto do Exército.}{ma.re.chal}{0}
\verb{marechal}{}{}{"-ais}{}{}{Militar que ocupa esse posto.}{ma.re.chal}{0}
\verb{marechalado}{}{}{}{}{s.m.}{Cargo de marechal.}{ma.re.cha.la.do}{0}
\verb{marechalato}{}{}{}{}{}{Var. de \textit{marechalado}.}{ma.re.cha.la.to}{0}
\verb{marechal"-do"-ar}{}{}{marechais"-do"-ar}{}{s.m.}{Posto mais alto da hierarquia da Força Aérea.}{ma.re.chal"-do"-ar}{0}
\verb{marechal"-do"-ar}{}{}{marechais"-do"-ar}{}{}{Militar que ocupa esse posto.}{ma.re.chal"-do"-ar}{0}
\verb{marejar}{}{}{}{}{v.i.}{Ficar gotejando.}{ma.re.jar}{\verboinum{1}}
\verb{maremoto}{ó}{}{}{}{s.m.}{Grande tremor de terra que acontece embaixo do mar e causa ondas enormes.}{ma.re.mo.to}{0}
\verb{maresia}{}{}{}{}{s.f.}{Cheiro que vem do mar.}{ma.re.si.a}{0}
\verb{mareta}{ê}{}{}{}{s.f.}{Pequena onda.}{ma.re.ta}{0}
\verb{mareta}{ê}{}{}{}{}{Onda de rio.}{ma.re.ta}{0}
\verb{marfim}{}{}{"-ins}{}{s.m.}{Matéria branca e resistente das presas do elefante.}{mar.fim}{0}
\verb{marfim}{}{}{"-ins}{}{}{A parte branca dos dentes.}{mar.fim}{0}
%\verb{}{}{}{}{}{}{}{}{0}
\verb{marfíneo}{}{}{}{}{adj.}{Que se assemelha ao marfim; ebúrneo.}{mar.fí.neo}{0}
\verb{marfíneo}{}{}{}{}{}{Feito de marfim.}{mar.fí.neo}{0}
\verb{marga}{}{}{}{}{s.f.}{Tipo de solo cimentado resultante da mistura de um solo argiloso com carbonato de cálcio.}{mar.ga}{0}
\verb{margarida}{}{Bot.}{}{}{s.f.}{Planta de jardim que possui poucos galhos.}{mar.ga.ri.da}{0}
\verb{margarida}{}{}{}{}{}{A flor dessa planta, que tem um miolo amarelo, cercado de muitas pétalas brancas.}{mar.ga.ri.da}{0}
\verb{margarida}{}{}{}{}{}{Peça arredondada que se coloca em certas máquinas de escrever para trocar o tipo de letra.}{mar.ga.ri.da}{0}
\verb{margarina}{}{}{}{}{s.f.}{Produto feito com gordura animal ou vegetal, parecido com a manteiga.}{mar.ga.ri.na}{0}
\verb{margear}{}{}{}{}{v.t.}{Estar ao longo ou ao lado de alguma coisa; beirar.}{mar.ge.ar}{0}
\verb{margear}{}{}{}{}{}{Fazer a margem em folha de papel.}{mar.ge.ar}{\verboinum{4}}
\verb{margem}{}{}{"-ens}{}{s.f.}{Faixa de terra que determina os lados de alguma coisa; beira, orla.}{mar.gem}{0}
\verb{margem}{}{}{"-ens}{}{}{A parte branca de uma página que cerca a parte escrita.}{mar.gem}{0}
\verb{marginal}{}{}{"-ais}{}{adj.2g.}{Feito sobre a margem.}{mar.gi.nal}{0}
\verb{marginal}{}{}{"-ais}{}{}{Que acompanha a margem.}{mar.gi.nal}{0}
\verb{marginal}{}{}{"-ais}{}{s.2g.}{Indivíduo que vive fora da lei.}{mar.gi.nal}{0}
\verb{marginalidade}{}{}{}{}{s.f.}{Condição de marginal.}{mar.gi.na.li.da.de}{0}
\verb{marginalidade}{}{}{}{}{}{Posição marginal em relação a uma forma social.}{mar.gi.na.li.da.de}{0}
\verb{marginalizar}{}{}{}{}{v.t.}{Considerar que pessoa ou coisa não faz parte da sociedade ou cultura.}{mar.gi.na.li.zar}{\verboinum{1}}
\verb{marginar}{}{}{}{}{v.t.}{Margear.}{mar.gi.nar}{\verboinum{1}}
\verb{maria"-chiquinha}{}{}{marias"-chiquinhas}{}{s.f.}{Penteado que separa os cabelos em duas partes, prendendo"-os nas laterais da cabeça.}{ma.ri.a"-chi.qui.nha}{0}
\verb{maria"-fumaça}{}{}{marias"-fumaças \textit{ou}  marias"-fumaça}{}{s.f.}{Trem puxado por uma máquina movida a vapor e que solta a fumaça da lenha queimada.}{ma.ri.a"-fu.ma.ça}{0}
\verb{maria"-mole}{ó}{}{marias"-moles ⟨ó⟩}{}{s.f.}{Doce feito de gelatina, coco e clara de ovo.}{ma.ri.a"-mo.le}{0}
\verb{mariano}{}{}{}{}{adj.}{Relativo à Virgem Maria ou ao seu culto.}{ma.ri.a.no}{0}
\verb{mariano}{}{}{}{}{}{Diz"-se de membro da Ordem dos Marianos.}{ma.ri.a.no}{0}
\verb{maria"-sem"-vergonha}{}{Bot.}{marias"-sem"-vergonha}{}{s.f.}{Planta que cresce espontaneamente nas matas úmidas e cuja flor tem cinco pétalas.}{ma.ri.a"-sem"-ver.go.nha}{0}
\verb{maria"-vai"-com"-as"-outras}{}{}{}{}{s.2g.2n}{Indivíduo sem personalidade, que faz o que vê os outros fazendo, que se deixa levar pelos outros.}{ma.ri.a"-vai"-com"-as"-ou.tras}{0}
\verb{maricas}{}{Pop.}{}{}{adj.}{Diz"-se de homem afeminado.}{ma.ri.cas}{0}
\verb{maricas}{}{}{}{}{}{Covarde.}{ma.ri.cas}{0}
\verb{marido}{}{}{}{}{s.m.}{Homem em relação à mulher com quem está casado.}{ma.ri.do}{0}
\verb{marimba}{}{}{}{}{s.f.}{Instrumento musical de percussão, formado por lâminas de madeira ou metal, uma para cada nota, e tocado com duas varetas.}{ma.rim.ba}{0}
\verb{marimbondo}{}{Zool.}{}{}{s.m.}{Grande vespa venenosa.}{ma.rim.bon.do}{0}
\verb{marina}{}{}{}{}{s.f.}{Cais equipado para guarda e manutenção de iates, lanchas etc.}{ma.ri.na}{0}
\verb{marinada}{}{Cul.}{}{}{s.f.}{Salmoura ou molho condimentado para conservar, temperar ou amaciar carnes.}{ma.ri.na.da}{0}
\verb{marinha}{}{}{}{}{s.f.}{Força armada marítima nacional.}{ma.ri.nha}{0}
\verb{marinha}{}{}{}{}{}{Seu conjunto de navios.}{ma.ri.nha}{0}
\verb{marinha}{}{}{}{}{}{Pintura de paisagem marítima.}{ma.ri.nha}{0}
\verb{marinhagem}{}{}{"-ens}{}{s.f.}{O conjunto dos marinheiros, do pessoal encarregado da mareação em navio de guerra ou mercante.}{ma.ri.nha.gem}{0}
\verb{marinhagem}{}{}{"-ens}{}{}{A técnica e a faina de navegar.}{ma.ri.nha.gem}{0}
\verb{marinharia}{}{}{}{}{s.f.}{A técnica ou profissão de marinheiro.}{ma.ri.nha.ri.a}{0}
\verb{marinharia}{}{}{}{}{}{Estudo e conhecimento do aparelho e da manobra de um navio.}{ma.ri.nha.ri.a}{0}
\verb{marinheiro}{ê}{}{}{}{s.m.}{Soldado da marinha.}{ma.ri.nhei.ro}{0}
\verb{marinheiro}{ê}{}{}{}{}{Indivíduo que trabalha em navio; homem do mar, marítimo, marujo.}{ma.ri.nhei.ro}{0}
\verb{marinho}{}{}{}{}{adj.}{Do mar ou próprio dele.}{ma.ri.nho}{0}
\verb{mariola}{ó}{Cul.}{}{}{s.f.}{Docinho retangular de banana, goiaba etc.}{ma.ri.o.la}{0}
\verb{marionete}{é}{}{}{}{s.f.}{Boneco manipulado especialmente por fios.}{ma.ri.o.ne.te}{0}
\verb{mariposa}{ô}{Zool.}{}{}{s.f.}{Nome comum aos insetos noturnos que são atraídos pela luz.}{ma.ri.po.sa}{0}
\verb{mariscar}{}{}{}{}{v.i.}{Apanhar mariscos.}{ma.ris.car}{0}
\verb{mariscar}{}{}{}{}{v.t.}{Pescar.}{ma.ris.car}{0}
\verb{mariscar}{}{}{}{}{}{Catar pedras preciosas em meio ao cascalho.}{ma.ris.car}{\verboinum{2}}
\verb{marisco}{}{Zool.}{}{}{s.m.}{Nome comum a diversos moluscos marinhos, dotados de concha, que servem de alimento para o homem.}{ma.ris.co}{0}
\verb{marisma}{}{}{}{}{s.f.}{Terreno alagado à beira de mar ou rio.}{ma.ris.ma}{0}
\verb{marista}{}{}{}{}{adj.2g.}{Relativo ou pertencente a uma das congregações religiosas devotadas à Virgem Maria, dedicadas ao ensino.}{ma.ris.ta}{0}
\verb{marista}{}{}{}{}{s.2g.}{Membro de uma congregação marista.}{ma.ris.ta}{0}
\verb{maritaca}{}{Zool.}{}{}{s.f.}{Nome comum a diversas aves semelhantes ao papagaio, notáveis pelo grande barulho que fazem; jandaia.}{ma.ri.ta.ca}{0}
\verb{maritaca}{}{Pop.}{}{}{}{Pessoa que fala muito; tagarela.}{ma.ri.ta.ca}{0}
\verb{marital}{}{}{"-ais}{}{adj.2g.}{Que se refere a casamento; conjugal. }{ma.ri.tal}{0}
\verb{mariticida}{}{}{}{}{s.f.}{Mulher que mata o marido.}{ma.ri.ti.ci.da}{0}
\verb{mariticídio}{}{}{}{}{s.m.}{Crime no qual a esposa assassina o marido.}{ma.ri.ti.cí.dio}{0}
\verb{marítimo}{}{}{}{}{adj.}{Relativo ao mar; marinho.}{ma.rí.ti.mo}{0}
\verb{marítimo}{}{}{}{}{}{Referente à marinha; naval.}{ma.rí.ti.mo}{0}
\verb{marítimo}{}{}{}{}{s.m.}{Indivíduo que trabalha em navio; marinheiro, marujo.}{ma.rí.ti.mo}{0}
\verb{marketing}{}{Econ.}{}{}{s.m.}{Conjunto de conhecimentos e estratégias sobre como vender melhor.}{\textit{marketing}}{0}
\verb{marmanjo}{}{}{}{}{s.m.}{Rapaz crescido.}{mar.man.jo}{0}
\verb{marmanjo}{}{}{}{}{}{Homem desonesto; velhaco, malandro.}{mar.man.jo}{0}
\verb{marmelada}{}{}{}{}{s.f.}{Doce de marmelo.}{mar.me.la.da}{0}
\verb{marmelada}{}{Bras.}{}{}{}{Combinação prévia, secreta e desonesta entre duas partes para determinar resultado ou placar; trapaça, conluio.}{mar.me.la.da}{0}
\verb{marmeleiro}{ê}{Bot.}{}{}{s.m.}{Planta cujo fruto comestível é o marmelo.}{mar.me.lei.ro}{0}
\verb{marmelo}{é}{}{}{}{s.m.}{Fruto do marmeleiro, levemente ácido, muito usado no fabrico de doces.}{mar.me.lo}{0}
\verb{marmita}{}{}{}{}{s.f.}{Vasilha com tampa para transportar comida.}{mar.mi.ta}{0}
\verb{marmita}{}{}{}{}{}{Conjunto de vasilhas encaixadas umas sobre as outras para transportar comida.}{mar.mi.ta}{0}
\verb{marmiteiro}{ê}{}{}{}{s.m.}{Indivíduo que entrega marmitas em domicílio.}{mar.mi.tei.ro}{0}
\verb{marmiteiro}{ê}{Bras.}{}{}{}{Indivíduo que leva sua refeição numa marmita para o local de trabalho.}{mar.mi.tei.ro}{0}
\verb{marmoraria}{}{}{}{}{s.f.}{Estabelecimento que vende mármore ou que faz trabalhos ou peças com esse material.}{mar.mo.ra.ri.a}{0}
\verb{mármore}{}{}{}{}{s.m.}{Pedra de várias cores, usada para estátuas e construções.}{már.mo.re}{0}
\verb{marmóreo}{}{}{}{}{adj.}{Que se assemelha ao mármore.}{mar.mó.re.o}{0}
\verb{marmóreo}{}{}{}{}{}{Que tem a cor do mármore, ou que é feito de mármore.}{mar.mó.re.o}{0}
\verb{marmorista}{}{}{}{}{s.2g.}{Pessoa que trabalha com mármore.}{mar.mo.ris.ta}{0}
\verb{marmota}{ó}{Zool.}{}{}{s.f.}{Mamífero roedor que passa o inverno hibernando em tocas.}{mar.mo.ta}{0}
\verb{marnel}{é}{}{"-éis}{}{s.m.}{Terreno alagado; brejo.}{mar.nel}{0}
\verb{marola}{ó}{}{}{}{s.f.}{Ondulação na superfície do mar.}{ma.ro.la}{0}
\verb{marola}{ó}{}{}{}{}{Onda pequena.}{ma.ro.la}{0}
\verb{maroma}{}{}{}{}{s.f.}{Corda grossa; cabo. }{ma.ro.ma}{0}
\verb{maroma}{}{}{}{}{}{Corda sobre a qual se equilibram funâmbulos, arlequins e outras personagens cômicas. }{ma.ro.ma}{0}
\verb{maromba}{}{}{}{}{s.f.}{Vara com que os funâmbulos mantêm o equilíbrio sobre a maroma.}{ma.rom.ba}{0}
\verb{maromba}{}{Fig.}{}{}{}{Situação difícil de sustentar.}{ma.rom.ba}{0}
\verb{maromba}{}{Pop.}{}{}{}{Musculação.}{ma.rom.ba}{0}
\verb{marombar}{}{}{}{}{v.i.}{Equilibrar"-se na maromba (corda bamba).}{ma.rom.bar}{0}
\verb{marombar}{}{Fig.}{}{}{}{Faltar ao trabalho; fugir a compromisso.}{ma.rom.bar}{0}
\verb{marombar}{}{Pop.}{}{}{}{Fazer musculação; malhar.}{ma.rom.bar}{\verboinum{1}}
\verb{maronita}{}{}{}{}{adj.2g.}{Diz"-se do rito católico sírio"-libanês.}{ma.ro.ni.ta}{0}
\verb{maronita}{}{}{}{}{s.2g.}{Fiel que pertence a esse rito.}{ma.ro.ni.ta}{0}
\verb{marosca}{ó}{}{}{}{s.f.}{Trapaça, ardil, logro.}{ma.ros.ca}{0}
\verb{maroteira}{ê}{}{}{}{s.f.}{Ação própria de maroto; malandragem, esperteza.}{ma.ro.tei.ra}{0}
\verb{maroto}{ô}{}{}{}{adj.}{Diz"-se de pessoa esperta, ladina.}{ma.ro.to}{0}
\verb{maroto}{ô}{}{}{}{s.m.}{Indivíduo malandro, esperto.}{ma.ro.to}{0}
\verb{marquês}{}{}{}{marquesa ⟨ê⟩}{s.m.}{Pessoa com título de nobreza entre o de conde e o de duque.}{mar.quês}{0}
\verb{marquesa}{ê}{}{}{}{s.f.}{Mulher com título de nobreza entre o de condessa e o de duquesa.}{mar.que.sa}{0}
\verb{marquesa}{ê}{}{}{}{}{Esposa de um marquês.}{mar.que.sa}{0}
\verb{marquesado}{}{}{}{}{s.m.}{Terras que correspondiam ao domínio do marquês ou marquesa.}{mar.que.sa.do}{0}
\verb{marquesado}{}{}{}{}{}{Título de marquês ou marquesa.}{mar.que.sa.do}{0}
\verb{marqueteiro}{ê}{Bras.}{}{}{s.m.}{Pessoa que trabalha com \textit{marketing}. }{mar.que.tei.ro}{0}
\verb{marqueteiro}{ê}{Por ext.}{}{}{}{Pessoa que se promove de forma oportunista e sistemática.}{mar.que.tei.ro}{0}
\verb{marquise}{}{}{}{}{s.f.}{Laje saliente à frente de uma porta, que serve de abrigo.}{mar.qui.se}{0}
\verb{marra}{}{}{}{}{s.f.}{Usada na expressão \textit{na marra}: à força, com luta.}{mar.ra}{0}
\verb{marrada}{}{}{}{}{s.f.}{Golpe dado com os chifres ou com a cabeça.}{mar.ra.da}{0}
\verb{marrão}{}{}{"-ões}{marrã}{s.m.}{Porco desmamado.}{mar.rão}{0}
\verb{marrar}{}{}{}{}{v.i.}{Dar cabeçadas ou chifradas, como fazem os touros, bodes etc.}{mar.rar}{\verboinum{1}}
\verb{marreca}{é}{Zool.}{}{}{s.f.}{A fêmea do marreco.  }{mar.re.ca}{0}
\verb{marreco}{é}{Zool.}{}{}{s.m.}{Ave menor que o pato, de pescoço curto, bico largo, e com uma pele fina ligando os dedos.}{mar.re.co}{0}
\verb{marreta}{ê}{}{}{}{s.f.}{Martelo grande para quebrar pedras.}{mar.re.ta}{0}
\verb{marretada}{}{}{}{}{s.f.}{Golpe de marreta.}{mar.re.ta.da}{0}
\verb{marretar}{}{}{}{}{v.t.}{Bater, golpear com marreta.}{mar.re.tar}{0}
\verb{marretar}{}{}{}{}{}{Dar uma surra.}{mar.re.tar}{0}
\verb{marretar}{}{Fig.}{}{}{}{Falar mal de alguém; malhar, difamar.}{mar.re.tar}{\verboinum{1}}
\verb{marreteiro}{ê}{Bras.}{}{}{s.m.}{Pessoa que trabalha com marreta.}{mar.re.tei.ro}{0}
\verb{marreteiro}{ê}{Pop.}{}{}{}{Pessoa que faz mal seu trabalho.}{mar.re.tei.ro}{0}
\verb{marreteiro}{ê}{}{}{}{}{Vendedor ambulante.}{mar.re.tei.ro}{0}
\verb{marrom}{}{}{"-ons}{}{adj.2g.}{Que tem a cor da castanha.}{mar.rom}{0}
\verb{marrom}{}{}{"-ons}{}{s.m.}{Essa cor.}{mar.rom}{0}
\verb{marroquim}{}{}{"-ins}{}{s.m.}{Pele de cabra ou bode, já preparada para artefatos (bolsas, sapatos etc.) e encadernações.}{mar.ro.quim}{0}
\verb{marroquino}{}{}{}{}{adj.}{Relativo a Marrocos.}{mar.ro.qui.no}{0}
\verb{marroquino}{}{}{}{}{s.m.}{Indivíduo natural ou habitante desse país.}{mar.ro.qui.no}{0}
\verb{marruá}{}{Bras.}{}{}{s.m.}{Novilho não domesticado.}{mar.ru.á}{0}
\verb{marrueiro}{ê}{Bras.}{}{}{s.m.}{Pessoa que doma marruás.}{mar.ru.ei.ro}{0}
\verb{marselhês}{}{}{}{}{adj.}{Relativo ou pertencente a Marselha, cidade da França.}{mar.se.lhês}{0}
\verb{marselhês}{}{}{}{}{s.m.}{Natural ou habitante dessa cidade.}{mar.se.lhês}{0}
\verb{marselhesa}{ê}{}{}{}{s.f.}{O hino nacional da França.}{mar.se.lhe.sa}{0}
\verb{marshmallow}{}{}{}{}{s.m.}{Doce feito com xarope de milho, clara de ovo, gelatina e açúcar.  }{\textit{marshmallow}}{0}
\verb{marsupial}{}{Zool.}{"-ais}{}{s.m.}{Espécime dos marsupiais, ordem de mamíferos caracterizados por terem as fêmeas uma bolsa, formada por uma dobra na pele, na qual amamentam e carregam os filhotes; são representados pelos cangurus, coalas e gambás, entre outros.}{mar.su.pi.al}{0}
\verb{marsupial}{}{}{"-ais}{}{adj.2g.}{Relativo a marsúpio.}{mar.su.pi.al}{0}
\verb{marsúpio}{}{Anat.}{}{}{s.m.}{Nas fêmeas dos marsupiais, bolsa no abdômen formada por uma dobra da pele, que serve para recobrir as tetas e abrigar os filhotes.}{mar.sú.pio}{0}
\verb{marta}{}{Zool.}{}{}{s.f.}{Mamífero carnívoro cuja pele é muito apreciada pela beleza.}{mar.ta}{0}
\verb{marta}{}{Por ext.}{}{}{}{A pele desse animal.}{mar.ta}{0}
\verb{Marte}{}{Astron.}{}{}{s.m.}{Quarto planeta em ordem de afastamento do Sol.}{Mar.te}{0}
\verb{Marte}{}{Mit.}{}{}{}{Deus da guerra, entre os antigos romanos.}{Mar.te}{0}
\verb{martelada}{}{}{}{}{s.f.}{Golpe, pancada de martelo.}{mar.te.la.da}{0}
\verb{martelar}{}{}{}{}{v.t.}{Bater com o martelo. }{mar.te.lar}{0}
\verb{martelar}{}{Fig.}{}{}{}{Aborrecer, chatear alguém dizendo sempre a mesma coisa ou insistindo muito sobre um assunto.}{mar.te.lar}{\verboinum{1}}
\verb{martelete}{ê}{}{}{}{s.m.}{Pequeno martelo; martelinho.}{mar.te.le.te}{0}
\verb{martelo}{é}{}{}{}{s.m.}{Ferramenta usada para cravar ou arrancar pregos, geralmente com cabo de pau e cabeça de aço.}{mar.te.lo}{0}
\verb{martelo}{é}{Anat.}{}{}{}{O maior dos três ossinhos que compõem o ouvido médio.}{mar.te.lo}{0}
\verb{martim}{}{Zool.}{"-ins}{}{s.m.}{Martim"-pescador.}{mar.tim}{0}
\verb{martim"-pescador}{ô}{Zool.}{martins"-pescadores ⟨ô⟩}{}{s.m.}{Ave que vive à beira de rios e lagoas, de bico longo, coloração azul ou verde, e que se alimenta de peixes.}{mar.tim"-pes.ca.dor}{0}
\verb{martinete}{ê}{}{}{}{s.m.}{Martelo grande e pesado, movido por água ou vapor, usado para bater o aço ou o ferro a frio.}{mar.ti.ne.te}{0}
\verb{martinete}{ê}{}{}{}{}{O martelo que percute as cordas do piano ou do cravo.}{mar.ti.ne.te}{0}
\verb{martíni}{}{}{}{}{s.m.}{Coquetel feito com vermute seco e gim, servido gelado.}{mar.tí.ni}{0}
\verb{martíni}{}{}{}{}{}{Espécie de vermute.}{mar.tí.ni}{0}
\verb{mártir}{}{}{}{}{s.2g.}{Pessoa que foi torturada, condenada ou morta por suas crenças ou ideias.}{már.tir}{0}
\verb{mártir}{}{}{}{}{}{Pessoa que se sacrifica por um ideal, trabalho, experiência etc.}{már.tir}{0}
\verb{mártir}{}{}{}{}{}{Pessoa que sofre muito.}{már.tir}{0}
\verb{martírio}{}{}{}{}{s.m.}{Sofrimento de mártir.}{mar.tí.rio}{0}
\verb{martírio}{}{}{}{}{}{Grande sofrimento; suplício, tormento.}{mar.tí.rio}{0}
\verb{martirizar}{}{}{}{}{v.t.}{Fazer alguém sofrer martírio.}{mar.ti.ri.zar}{0}
\verb{martirizar}{}{}{}{}{}{Fazer alguém sofrer muito; atormentar, torturar.}{mar.ti.ri.zar}{\verboinum{1}}
%\verb{}{}{}{}{}{}{}{}{0}
%\verb{}{}{}{}{}{}{}{}{0}
\verb{maruí}{}{Zool.}{}{}{s.m.}{Maruim. }{ma.ru.í}{0}
\verb{maruim}{}{Zool.}{}{}{s.m.}{Mosquito pequenino que vive em lugares pantanosos, de picada dolorosa, e que transmite a filariose ao homem e aos animais domésticos; maruí.}{ma.ru.im}{0}
\verb{maruja}{}{}{}{}{s.f.}{Conjunto de marinheiros; marinhagem, tripulação.}{ma.ru.ja}{0}
\verb{marujada}{}{}{}{}{s.f.}{Ajuntamento de marujos.}{ma.ru.ja.da}{0}
\verb{marujo}{}{}{}{}{s.m.}{Pessoa que trabalha no mar, a bordo de uma embarcação; marinheiro.}{ma.ru.jo}{0}
\verb{marulhar}{}{}{}{}{v.i.}{Produzir o som do marulho.}{ma.ru.lhar}{\verboinum{1}}
\verb{marulho}{}{}{}{}{s.m.}{Agitação das ondas do mar, que produz um determinado barulho.}{ma.ru.lho}{0}
\verb{marxismo}{cs}{Filos.}{}{}{s.m.}{Conjunto das teorias e doutrinas do filósofo alemão Karl Marx (1818--1883).}{mar.xis.mo}{0}
\verb{marxista}{cs}{}{}{}{adj.2g.}{Relativo ao marxismo.}{mar.xis.ta}{0}
\verb{marxista}{cs}{}{}{}{s.2g.}{Partidário do marxismo.}{mar.xis.ta}{0}
\verb{marzipã}{}{Cul.}{}{}{s.m.}{Doce feito de massa de amêndoas e açúcar.}{mar.zi.pã}{0}
\verb{mas}{}{}{}{}{conj.}{Palavra que indica oposição ou restrição; porém, contudo, todavia, entretanto.}{mas}{0}
\verb{mas}{}{}{}{}{}{Palavra que indica o motivo de um fato dito antes. (\textit{Eu o trouxe até aqui, mas ele é que me pediu.})}{mas}{0}
\verb{mas}{}{}{}{}{}{Palavra que reforça aquilo que se diz.}{mas}{0}
\verb{mascar}{}{}{}{}{v.t.}{Mastigar sem engolir.}{mas.car}{\verboinum{2}}
\verb{máscara}{}{}{}{}{s.f.}{Objeto de papelão, pano etc., que representa um rosto, usado para cobrir a face, para disfarçar a pessoa que o usa.}{más.ca.ra}{0}
\verb{máscara}{}{}{}{}{}{Qualquer coisa usada para cobrir, esconder ou disfarçar o rosto.}{más.ca.ra}{0}
\verb{mascarada}{}{}{}{}{s.f.}{Festa com pessoas mascaradas; baile de máscaras.}{mas.ca.ra.da}{0}
\verb{mascarado}{}{}{}{}{adj.}{Que está com máscara.}{mas.ca.ra.do}{0}
\verb{mascarado}{}{Fig.}{}{}{}{Falso, fingido.}{mas.ca.ra.do}{0}
\verb{mascarado}{}{}{}{}{s.m.}{Pessoa com máscara.}{mas.ca.ra.do}{0}
\verb{mascarado}{}{}{}{}{}{Pessoa falsa, fingida, dissimulada.}{mas.ca.ra.do}{0}
\verb{mascarar}{}{}{}{}{v.t.}{Pôr máscara em alguém.}{mas.ca.rar}{0}
\verb{mascarar}{}{}{}{}{}{Disfarçar, dissimular.}{mas.ca.rar}{\verboinum{1}}
\verb{mascate}{}{}{}{}{s.m.}{Vendedor ambulante, que vai de porta em porta oferecendo suas mercadorias. }{mas.ca.te}{0}
\verb{mascatear}{}{}{}{}{v.t.}{Vender mercadorias de porta em porta.}{mas.ca.te.ar}{\verboinum{4}}
\verb{mascavado}{}{}{}{}{adj.}{Diz"-se do açúcar que não foi refinado; mascavo.}{mas.ca.va.do}{0}
\verb{mascavo}{}{}{}{}{adj.}{Diz"-se do açúcar que não foi refinado, de cor escura.}{mas.ca.vo}{0}
\verb{mascote}{ó}{}{}{}{s.f.}{Animal, pessoa ou coisa que dá sorte ou traz felicidade.}{mas.co.te}{0}
\verb{mascote}{ó}{}{}{}{}{Animal ou objeto de estimação de uma pessoa ou grupo.}{mas.co.te}{0}
\verb{masculinidade}{}{}{}{}{s.f.}{Qualidade de másculo ou masculino; virilidade.}{mas.cu.li.ni.da.de}{0}
\verb{masculinizar}{}{}{}{}{v.t.}{Tornar masculino.}{mas.cu.li.ni.zar}{0}
\verb{masculinizar}{}{}{}{}{}{Dar aparência de sexo masculino.}{mas.cu.li.ni.zar}{\verboinum{1}}
\verb{masculino}{}{}{}{}{adj.}{Que é próprio do homem ou do macho; viril, varonil, másculo.}{mas.cu.li.no}{0}
\verb{masculino}{}{}{}{}{}{Que pertence ao sexo dos animais machos.}{mas.cu.li.no}{0}
\verb{masculino}{}{}{}{}{}{Diz"-se das palavras ou nomes que designam seres masculinos.}{mas.cu.li.no}{0}
\verb{masculino}{}{Gram.}{}{}{s.m.}{O gênero masculino.}{mas.cu.li.no}{0}
\verb{másculo}{}{}{}{}{adj.}{Referente ao homem ou a animal macho.}{más.cu.lo}{0}
\verb{másculo}{}{Por ext.}{}{}{}{Viril, enérgico, vigoroso, macho.}{más.cu.lo}{0}
\verb{masmorra}{ô}{}{}{}{s.f.}{Prisão subterrânea, fria e escura, onde antigamente se prendiam os condenados; calabouço.}{mas.mor.ra}{0}
\verb{masoquismo}{}{}{}{}{s.m.}{Perversão sexual na qual a pessoa só tem prazer ao ser maltratada.}{ma.so.quis.mo}{0}
\verb{masoquista}{}{}{}{}{adj.2g.}{Relativo ao masoquismo.}{ma.so.quis.ta}{0}
\verb{masoquista}{}{}{}{}{s.2g.}{Pessoa que sente prazer com a própria dor.}{ma.so.quis.ta}{0}
\verb{massa}{}{}{}{}{s.f.}{Qualquer mistura de farinha e água ou outro líquido que forma uma pasta.}{mas.sa}{0}
\verb{massa}{}{}{}{}{}{Grande quantidade de matéria.}{mas.sa}{0}
\verb{massa}{}{}{}{}{}{Grande quantidade de pessoas.}{mas.sa}{0}
\verb{massa}{}{Fís.}{}{}{}{Quantidade de matéria que um corpo contém.}{mas.sa}{0}
\verb{massacrar}{}{}{}{}{v.t.}{Matar uma pessoa ou um grupo de pessoas com crueldade; chacinar.}{mas.sa.crar}{0}
\verb{massacrar}{}{Fig.}{}{}{}{Causar sofrimento a alguém; atormentar, oprimir, torturar.}{mas.sa.crar}{\verboinum{1}}
\verb{massacre}{}{}{}{}{s.m.}{Ato ou efeito de massacrar; matança cruel; carnificina, chacina.}{mas.sa.cre}{0}
\verb{massagear}{}{}{}{}{v.t.}{Fazer massagem.}{mas.sa.ge.ar}{\verboinum{4}}
\verb{massagem}{}{}{"-ens}{}{s.f.}{Pressão que se faz sobre parte do corpo de uma pessoa, batendo ou friccionando com as mãos, para melhorar a circulação do sangue ou com algum outro propósito terapêutico.}{mas.sa.gem}{0}
\verb{massagista}{}{}{}{}{s.2g.}{Pessoa cuja profissão é massagear.}{mas.sa.gis.ta}{0}
\verb{massapé}{}{Bras.}{}{}{s.m.}{Massapê.}{mas.sa.pé}{0}
\verb{massapê}{}{Bras.}{}{}{s.m.}{Tipo de terra argilosa, de cor escura, muito fértil, excelente para a cultura de cana"-de"-açúcar; massapé.}{mas.sa.pê}{0}
\verb{masseira}{ê}{}{}{}{s.f.}{Tabuleiro onde se amassa a farinha para fazer pão.}{mas.sei.ra}{0}
\verb{masseter}{é}{Anat.}{}{}{s.m.}{Músculo da face responsável pela mastigação e pelo abrir e fechar da boca.}{mas.se.ter}{0}
\verb{massificação}{}{}{"-ões}{}{s.f.}{Ato ou efeito de massificar.}{mas.si.fi.ca.ção}{0}
\verb{massificar}{}{}{}{}{v.t.}{Fazer com que algo seja muito conhecido ou consumido pela população.}{mas.si.fi.car}{0}
\verb{massificar}{}{}{}{}{v.pron.}{Deixar"-se influenciar pelos meios de comunicação de massa, como a televisão, o rádio, os jornais etc.}{mas.si.fi.car}{\verboinum{2}}
\verb{massudo}{}{}{}{}{adj.}{Cheio de massa; encorpado, grosso, volumoso.}{mas.su.do}{0}
\verb{mastectomia}{}{Med.}{}{}{s.f.}{Cirurgia feita para a remoção de mama.}{mas.tec.to.mi.a}{0}
\verb{mastigação}{}{}{"-ões}{}{s.f.}{Ato ou efeito de mastigar.}{mas.ti.ga.ção}{0}
%\verb{}{}{}{}{}{}{}{}{0}
%\verb{}{}{}{}{}{}{}{}{0}
\verb{mastigador}{ô}{}{}{}{adj.}{Que mastiga ou tem o hábito de mastigar.}{mas.ti.ga.dor}{0}
\verb{mastigar}{}{}{}{}{v.t.}{Triturar um alimento com os dentes, fazendo com que ele se transforme numa pasta.}{mas.ti.gar}{\verboinum{5}}
\verb{mastigóforo}{}{Zool.}{}{}{s.m.}{Espécime dos mastigóforos, classe de seres muito pequenos, formados de uma célula apenas, que possuem flagelos para se movimentarem. }{mas.ti.gó.fo.ro}{0}
\verb{mastim}{}{}{"-ins}{}{s.m.}{Raça de cachorro que serve para guarda de gado.}{mas.tim}{0}
\verb{mastite}{}{Med.}{}{}{s.f.}{Inflamação de mama.}{mas.ti.te}{0}
\verb{mastodonte}{}{Paleo.}{}{}{s.m.}{Animal que viveu e se extinguiu há muito tempo, conhecido apenas por fósseis, parecido com um elefante, mas com dois pares de presas e mais corpulento.}{mas.to.don.te}{0}
\verb{mastoplastia}{}{Med.}{}{}{s.f.}{Cirurgia estética nos seios, para aumentá"-los (geralmente) ou reduzi"-los; mamoplastia.}{mas.to.plas.ti.a}{0}
\verb{mastreação}{}{}{"-ões}{}{s.f.}{Numa embarcação, o conjunto de mastros e seus acessórios.}{mas.tre.a.ção}{0}
\verb{mastrear}{}{}{}{}{v.t.}{Pôr mastros em uma embarcação.}{mas.tre.ar}{\verboinum{4}}
\verb{mastro}{}{}{}{}{s.m.}{Peça de madeira ou de ferro, comprida e circular, que sustenta as velas das embarcações.}{mas.tro}{0}
\verb{mastro}{}{}{}{}{}{Haste na qual se içam as bandeiras.}{mas.tro}{0}
\verb{mastruço}{}{Bot.}{}{}{s.m.}{Erva da família das crucíferas, com muitas propriedades medicinais; mastruz.}{mas.tru.ço}{0}
\verb{mastruz}{}{Bot.}{}{}{s.m.}{Mastruço.}{mas.truz}{0}
\verb{masturbação}{}{}{"-ões}{}{s.f.}{Ato ou efeito de masturbar.}{mas.tur.ba.ção}{0}
\verb{masturbador}{ô}{}{}{}{adj.}{Que (se) masturba; onanista. }{mas.tur.ba.dor}{0}
\verb{masturbar}{}{}{}{}{v.t.}{Provocar orgasmo em alguém pela estimulação dos órgãos genitais.}{mas.tur.bar}{\verboinum{1}}
\verb{mata}{}{}{}{}{s.f.}{Área muito extensa, coberta de árvores silvestres; floresta, bosque, mato, selva.}{ma.ta}{0}
\verb{mata"-borrão}{}{}{mata"-borrões}{}{s.m.}{Papel poroso, usado para absorver o excesso de tinta.}{ma.ta"-bor.rão}{0}
\verb{mata"-burro}{}{Bras.}{mata"-burros}{}{s.m.}{Buraco ou fosso com ponte de traves  espaçadas, cavado à frente das porteiras de fazendas, chácaras etc. para impedir que o gado saia.}{ma.ta"-bur.ro}{0}
\verb{matacão}{}{}{"-ães}{}{s.m.}{Pedra solta, grande e arredondada.}{ma.ta.cão}{0}
\verb{matacão}{}{}{"-ães}{}{}{Grande fatia ou pedaço de alguma coisa.}{ma.ta.cão}{0}
\verb{matado}{}{Bras.}{}{}{adj.}{Feito de qualquer maneira, às pressas; malfeito, mal"-acabado, marretado.}{ma.ta.do}{0}
\verb{matado}{}{Bras.}{}{}{}{Diz"-se do fruto que foi colhido antes do tempo e amadurecido de forma artificial.}{ma.ta.do}{0}
\verb{matador}{ô}{}{}{}{adj.}{Que mata, que causa ou provoca morte; mortal, mortífero.}{ma.ta.dor}{0}
\verb{matador}{ô}{}{}{}{s.m.}{Pessoa que mata; assassino.}{ma.ta.dor}{0}
\verb{matadouro}{ô}{}{}{}{s.m.}{Lugar onde são abatidos os animais destinados ao consumo; abatedouro. (\textit{O matadouro da cidade ficava à margem do rio.})}{ma.ta.dou.ro}{0}
\verb{matagal}{}{}{"-ais}{}{s.m.}{Mata densa, cerrada, difícil de penetrar e de atravessar; brenha.}{ma.ta.gal}{0}
\verb{matagal}{}{}{"-ais}{}{}{Terreno coberto de mato espesso; mato.}{ma.ta.gal}{0}
\verb{matalotagem}{}{}{"-ens}{}{s.f.}{Matula.}{ma.ta.lo.ta.gem}{0}
\verb{matalote}{ó}{}{}{}{s.m.}{Homem do mar; marinheiro, marujo.}{ma.ta.lo.te}{0}
\verb{matalote}{ó}{Desus.}{}{}{}{Companheiro de serviço ou de viagem.}{ma.ta.lo.te}{0}
\verb{matança}{}{}{}{}{s.f.}{Ato ou efeito de matar grande quantidade de pessoas ou animais; massacre, carnificina.}{ma.tan.ça}{0}
\verb{mata"-piolho}{ô}{Pop.}{mata"-piolhos ⟨ô⟩}{}{s.m.}{O dedo mais grosso e curto da mão; polegar, cata"-piolho.}{ma.ta"-pi.o.lho}{0}
%\verb{}{}{}{}{}{}{}{}{0}
\verb{matar}{}{}{}{}{v.t.}{Tirar a vida.}{ma.tar}{0}
\verb{matar}{}{}{}{}{}{Fazer murchar, secar.}{ma.tar}{0}
\verb{matar}{}{}{}{}{}{Executar um trabalho às pressas e com pouco cuidado.}{ma.tar}{0}
\verb{matar}{}{Pop.}{}{}{}{Deixar de comparecer à aula.}{ma.tar}{0}
\verb{matar}{}{Fig.}{}{}{}{Resolver uma incógnita; decifrar.}{ma.tar}{\verboinum{1}}
\verb{mata"-rato}{}{}{mata"-ratos}{}{s.m.}{Veneno para matar ratos.}{ma.ta"-ra.to}{0}
\verb{mataria}{}{Bras.}{}{}{s.f.}{Grande extensão de terreno coberto de mato.}{ma.ta.ri.a}{0}
\verb{mate}{}{}{}{}{s.f.}{Forma reduzida de \textit{erva"-mate}.}{ma.te}{0}
\verb{mate}{}{}{}{}{adj.2g.}{Que não tem brilho; fosco, embaciado.}{ma.te}{0}
\verb{mateiro}{ê}{}{}{}{s.m.}{Pessoa que corta lenha nas matas.}{ma.tei.ro}{0}
\verb{mateiro}{ê}{}{}{}{}{Pessoa que vigia, que guarda matas, florestas ou bosques.}{ma.tei.ro}{0}
\verb{mateiro}{ê}{}{}{}{s.m.}{Pessoa que se dedica à cultura de erva"-mate ou à sua venda.}{ma.tei.ro}{0}
\verb{matemática}{}{}{}{}{s.f.}{Ciência que estuda, por meio do raciocínio dedutivo, medidas e propriedades das grandezas que podem ser expressas por números, valores ou letras.}{ma.te.má.ti.ca}{0}
\verb{matemático}{}{}{}{}{adj.}{Relativo a matemática.}{ma.te.má.ti.co}{0}
\verb{matemático}{}{}{}{}{}{Que é preciso como a matemática.}{ma.te.má.ti.co}{0}
\verb{matemático}{}{}{}{}{s.m.}{Pessoa versada em matemática.}{ma.te.má.ti.co}{0}
\verb{matéria}{}{}{}{}{s.f.}{Qualquer substância, sólida, líquida ou gasosa, de que são formados os corpos ou que ocupa lugar no espaço. }{ma.té.ria}{0}
\verb{matéria}{}{}{}{}{}{Assunto, tema, teor.}{ma.té.ria}{0}
\verb{matéria}{}{}{}{}{}{Texto jornalístico; notícia, reportagem.}{ma.té.ria}{0}
\verb{matéria}{}{}{}{}{}{Conteúdo específico que é dado em sala de aula; disciplina.}{ma.té.ria}{0}
\verb{material}{}{}{"-ais}{}{adj.2g.}{Relativo a matéria.}{ma.te.ri.al}{0}
\verb{material}{}{}{"-ais}{}{}{Que não é espiritual; concreto, palpável, tangível, sensível.}{ma.te.ri.al}{0}
\verb{material}{}{}{"-ais}{}{s.m.}{Utensílios, equipamentos, petrechos.}{ma.te.ri.al}{0}
\verb{material}{}{}{"-ais}{}{}{Conjunto de objetos que constituem uma obra, uma construção etc.}{ma.te.ri.al}{0}
\verb{materialidade}{}{}{}{}{s.f.}{Qualidade do que é material.}{ma.te.ri.a.li.da.de}{0}
\verb{materialismo}{}{Filos.}{}{}{s.m.}{Doutrina segundo a qual o fundamento último de todas as coisas é material, sem que haja algo que o transcenda.}{ma.te.ri.a.lis.mo}{0}
\verb{materialismo}{}{Por ext.}{}{}{}{Modo de vida voltado exclusivamente para os valores e prazeres materiais.}{ma.te.ri.a.lis.mo}{0}
\verb{materialista}{}{}{}{}{adj.2g.}{Relativo ao materialismo.}{ma.te.ri.a.lis.ta}{0}
\verb{materialista}{}{}{}{}{}{Que é partidário ou simpatizante do materialismo.}{ma.te.ri.a.lis.ta}{0}
\verb{materialista}{}{}{}{}{s.2g.}{Essa pessoa.}{ma.te.ri.a.lis.ta}{0}
\verb{materialização}{}{}{"-ões}{}{s.f.}{Ato ou efeito de materializar. }{ma.te.ri.a.li.za.ção}{0}
\verb{materializar}{}{}{}{}{v.t.}{Tornar material; transformar em realidade.}{ma.te.ri.a.li.zar}{\verboinum{1}}
\verb{matéria"-prima}{}{}{matérias"-primas}{}{s.f.}{A substância principal para fabricar alguma coisa.}{ma.té.ri.a"-pri.ma}{0}
\verb{matéria"-prima}{}{Fig.}{matérias"-primas}{}{}{Base, fundamento.}{ma.té.ri.a"-pri.ma}{0}
\verb{maternal}{}{}{"-ais}{}{adj.2g.}{Materno.}{ma.ter.nal}{0}
\verb{maternal}{}{}{"-ais}{}{}{Diz"-se de escola que ensina crianças com menos de quatro anos de idade.}{ma.ter.nal}{0}
\verb{maternidade}{}{}{}{}{s.f.}{Estado, condição ou qualidade de mãe.}{ma.ter.ni.da.de}{0}
\verb{maternidade}{}{}{}{}{}{Lugar onde são assistidas as mulheres grávidas e onde são feitos os partos.}{ma.ter.ni.da.de}{0}
\verb{materno}{é}{}{}{}{adj.}{Relativo a mãe; maternal.}{ma.ter.no}{0}
\verb{materno}{é}{}{}{}{}{Que se assemelha à relação da mãe com o filho.}{ma.ter.no}{0}
\verb{materno}{é}{}{}{}{}{Que tem parentesco por parte de mãe.}{ma.ter.no}{0}
\verb{matilha}{}{}{}{}{s.f.}{Grupo de cães de caça.}{ma.ti.lha}{0}
\verb{matilha}{}{Fig.}{}{}{}{Bando de malfeitores; corja, súcia, malta.}{ma.ti.lha}{0}
\verb{matina}{}{Pop.}{}{}{s.f.}{Madrugada, alvorada, matinada. }{ma.ti.na}{0}
\verb{matinada}{}{}{}{}{s.f.}{Madrugada, alvorada, matina.}{ma.ti.na.da}{0}
\verb{matinada}{}{}{}{}{}{O canto das matinas.}{ma.ti.na.da}{0}
\verb{matinal}{}{}{"-ais}{}{adj.2g.}{Da manhã; matutino.}{ma.ti.nal}{0}
\verb{matinas}{}{}{}{}{s.f.pl.}{Na liturgia católica, primeira parte do ofício divino, rezada de madrugada.}{ma.ti.nas}{0}
\verb{matinê}{}{}{}{}{s.f.}{Festa, espetáculo, reunião ou sessão de cinema realizada à tarde; vesperal.}{ma.ti.nê}{0}
%\verb{}{}{}{}{}{}{}{}{0}
%\verb{}{}{}{}{}{}{}{}{0}
%\verb{}{}{}{}{}{}{}{}{0}
%\verb{}{}{}{}{}{}{}{}{0}
\verb{matiz}{}{}{}{}{s.m.}{Variedade de uma cor; nuança, tom, tonalidade.}{ma.tiz}{0}
\verb{matizar}{}{}{}{}{v.t.}{Dar variedade a uma cor; nuançar.}{ma.ti.zar}{\verboinum{1}}
\verb{mato}{}{}{}{}{s.m.}{Porção de terreno coberto por vegetação silvestre; mata, brenha.}{ma.to}{0}
\verb{mato}{}{}{}{}{}{O conjunto de plantas rasteiras, daninhas, que prejudicam jardins, lavouras etc.}{ma.to}{0}
\verb{mato}{}{Bras.}{}{}{}{O campo; roça.}{ma.to}{0}
\verb{mato"-grossense}{}{}{mato"-grossenses}{}{adj.2g.}{Relativo ao Mato Grosso.}{ma.to"-gros.sen.se}{0}
\verb{mato"-grossense}{}{}{mato"-grossenses}{}{s.2g.}{Indivíduo natural ou habitante desse estado.}{ma.to"-gros.sen.se}{0}
\verb{mato"-grossense"-do"-sul}{}{}{mato"-grossenses"-do"-sul}{}{adj.}{Relativo ao estado do Mato Grosso do Sul; sul"-mato"-grossense.}{ma.to"-gros.sen.se"-do"-sul}{0}
\verb{mato"-grossense"-do"-sul}{}{}{mato"-grossenses"-do"-sul}{}{s.2g.}{Indivíduo natural ou habitante desse estado.}{ma.to"-gros.sen.se"-do"-sul}{0}
\verb{matraca}{}{}{}{}{s.f.}{Instrumento de percussão, formado por tabuinhas que produzem um ruído de estalos ao serem agitadas; malho. }{ma.tra.ca}{0}
\verb{matraca}{}{Fig.}{}{}{}{Pessoa que fala muito; tagarela.}{ma.tra.ca}{0}
\verb{matraquear}{}{}{}{}{v.i.}{Tocar matraca.}{ma.tra.que.ar}{0}
\verb{matraquear}{}{Fig.}{}{}{}{Falar muito, sem parar; tagarelar.}{ma.tra.que.ar}{\verboinum{4}}
\verb{matreiro}{ê}{}{}{}{adj.}{Que sabe lidar com ou contornar qualquer situação; esperto, astuto, sabido, manhoso, experimentado.}{ma.trei.ro}{0}
\verb{matriarca}{}{}{}{}{s.f.}{Mulher que é chefe de família, ou que exerce autoridade sobre um grupo de pessoas.}{ma.tri.ar.ca}{0}
\verb{matriarcado}{}{}{}{}{s.m.}{Tipo de organização social e política em que a autoridade é exercida pela mulher, pela matriarca.}{ma.tri.ar.ca.do}{0}
\verb{matriarcal}{}{}{"-ais}{}{adj.2g.}{Que se refere a matriarca ou ao matriarcado.}{ma.tri.ar.cal}{0}
\verb{matricida}{}{}{}{}{s.2g.}{Pessoa que matou a própria mãe.}{ma.tri.ci.da}{0}
\verb{matricídio}{}{}{}{}{s.m.}{Crime de quem matou a própria mãe.}{ma.tri.cí.dio}{0}
\verb{matrícula}{}{}{}{}{s.f.}{Ato ou efeito de matricular.}{ma.trí.cu.la}{0}
\verb{matrícula}{}{}{}{}{}{Registro, inscrição de uma pessoa sujeita a certas obrigações ou para facilitar sua identificação. }{ma.trí.cu.la}{0}
\verb{matricular}{}{}{}{}{v.t.}{Fazer a matrícula de alguém.}{ma.tri.cu.lar}{\verboinum{1}}
\verb{matrilinear}{}{}{}{}{adj.2g.}{Que se refere aos parentes pela linha feminina.}{ma.tri.li.ne.ar}{0}
\verb{matrimônio}{}{}{}{}{s.m.}{Casamento.}{ma.tri.mô.nio}{0}
\verb{mátrio}{}{}{}{}{adj.}{Que se refere a mãe.}{má.trio}{0}
\verb{matriz}{}{}{}{}{adj.2g.}{Que é fonte, origem de uma coisa; básico.}{ma.triz}{0}
\verb{matriz}{}{}{}{}{s.f.}{Parte do corpo da fêmea na qual o filhote é gerado; útero.}{ma.triz}{0}
\verb{matriz}{}{}{}{}{}{A principal e mais importante igreja de um local.}{ma.triz}{0}
\verb{matriz}{}{}{}{}{}{O principal estabelecimento de uma empresa ou instituição, do qual dependem as outras lojas, as filiais.}{ma.triz}{0}
\verb{matroca}{ó}{}{}{}{s.f.}{Palavra usada na expressão \textit{à matroca}: ao acaso, à toa, de qualquer jeito.}{ma.tro.ca}{0}
\verb{matrona}{ô}{}{}{}{s.f.}{Mulher respeitável pela idade.  }{ma.tro.na}{0}
\verb{matrona}{ô}{Pejor.}{}{}{}{Mulher de meia"-idade corpulenta.}{ma.tro.na}{0}
%\verb{}{}{}{}{}{}{}{}{0}
%\verb{}{}{}{}{}{}{}{}{0}
\verb{matula}{}{Bras.}{}{}{s.f.}{Saco em que se carrega comida para uma viagem; farnel, matalotagem.}{ma.tu.la}{0}
\verb{matulagem}{}{}{"-ens}{}{s.f.}{Vida de vadio, de vagabundo; vadiagem.}{ma.tu.la.gem}{0}
\verb{maturação}{}{}{"-ões}{}{s.f.}{Ato ou efeito de maturar; amadurecimento.}{ma.tu.ra.ção}{0}
\verb{maturar}{}{}{}{}{v.t.}{Tornar maduro; amadurecer.}{ma.tu.rar}{\verboinum{1}}
\verb{maturi}{}{Bot.}{}{}{s.m.}{A castanha verde e mole do caju, antes do desenvolvimento do pedúnculo carnoso.  }{ma.tu.ri}{0}
\verb{maturidade}{}{}{}{}{s.f.}{Estado de pleno desenvolvimento físico e mental de uma pessoa.}{ma.tu.ri.da.de}{0}
\verb{maturidade}{}{}{}{}{}{A época desse desenvolvimento; idade madura.}{ma.tu.ri.da.de}{0}
\verb{maturidade}{}{}{}{}{}{A experiência, a ponderação e a prudência próprias da vida adulta ou da idade madura.}{ma.tu.ri.da.de}{0}
\verb{matusalém}{}{Pop.}{"-ens}{}{s.m.}{Pessoa muito velha; ancião.}{ma.tu.sa.lém}{0}
\verb{matutar}{}{}{}{}{v.i.}{Pensar longamente; refletir.}{ma.tu.tar}{0}
\verb{matutar}{}{}{}{}{v.t.}{Planejar, maquinar, tramar.}{ma.tu.tar}{\verboinum{1}}
\verb{matutice}{}{}{}{}{s.f.}{Qualidade de matuto.}{ma.tu.ti.ce}{0}
\verb{matutino}{}{}{}{}{adj.}{Relativo à manhã; matinal.}{ma.tu.ti.no}{0}
\verb{matutino}{}{}{}{}{}{Diz"-se de indivíduo que acorda cedo.}{ma.tu.ti.no}{0}
\verb{matuto}{}{}{}{}{adj.}{Diz"-se de indivíduo que vive no mato.}{ma.tu.to}{0}
\verb{matuto}{}{}{}{}{}{Diz"-se de quem costuma matutar.}{ma.tu.to}{0}
\verb{matuto}{}{}{}{}{}{Sabido, astuto.}{ma.tu.to}{0}
\verb{matuto}{}{}{}{}{s.m.}{Caipira.}{ma.tu.to}{0}
\verb{matuto}{}{Pejor.}{}{}{}{Indivíduo sem instrução, ignorante.}{ma.tu.to}{0}
\verb{mau}{}{}{}{má}{adj.}{Que causa dano, desgraça ou moléstia; prejudicial, nocivo.}{mau}{0}
\verb{mau}{}{}{}{má}{}{De pouca ou nenhuma qualidade; inferior, ruim.}{mau}{0}
\verb{mau}{}{}{}{má}{}{Sem habilidade, talento, educação ou delicadeza; incapaz, inábil, rude, grosseiro.}{mau}{0}
\verb{mau"-caráter}{}{}{maus"-caracteres}{}{s.m.}{Indivíduo sem escrúpulos, não confiável, traiçoeiro.}{mau"-ca.rá.ter}{0}
%\verb{}{}{}{}{}{}{}{}{0}
\verb{mau"-olhado}{}{}{maus"-olhados}{}{s.m.}{Olhar ao qual é atribuída a capacidade de causar danos ou infortúnios.}{mau"-o.lha.do}{0}
\verb{mau"-olhado}{}{}{maus"-olhados}{}{}{O efeito desse olhar.}{mau"-o.lha.do}{0}
\verb{mauritano}{}{}{}{}{adj.}{Relativo à Mauritânia (África Ocidental).}{mau.ri.ta.no}{0}
\verb{mauritano}{}{}{}{}{s.m.}{Indivíduo natural ou habitante desse país.}{mau.ri.ta.no}{0}
\verb{mausoléu}{}{}{}{}{s.m.}{Sepultura com abrigo de alvenaria ou monumento imponente, e que geralmente abriga os despojos de várias pessoas de uma mesma família ou relacionadas por um evento histórico.}{mau.so.léu}{0}
%\verb{}{}{}{}{}{}{}{}{0}
\verb{mavioso}{ô}{}{"-osos ⟨ó⟩}{"-osa ⟨ó⟩}{adj.}{Afetuoso, doce, brando.}{ma.vi.o.so}{0}
\verb{mavioso}{ô}{}{"-osos ⟨ó⟩}{"-osa ⟨ó⟩}{}{Compassivo, piedoso.}{ma.vi.o.so}{0}
%\verb{}{}{}{}{}{}{}{}{0}
%\verb{}{}{}{}{}{}{}{}{0}
%\verb{}{}{}{}{}{}{}{}{0}
%\verb{}{}{}{}{}{}{}{}{0}
\verb{maxidesvalorização}{cs}{Econ.}{"-ões}{}{s.f.}{Elevada desvalorização da moeda de um país decretada pelas autoridades responsáveis pela economia.}{ma.xi.des.va.lo.ri.za.ção}{0}
\verb{maxidesvalorizar}{cs}{Econ.}{}{}{v.t.}{Desvalorizar demais a moeda de um país de uma só vez.}{ma.xi.des.va.lo.ri.zar}{\verboinum{1}}
\verb{maxila}{cs}{Anat.}{}{}{s.f.}{Cada um dos dois ossos em que estão implantados os dentes superiores. }{ma.xi.la}{0}
\verb{maxilar}{cs}{}{}{}{adj.2g.}{Relativo a maxila.}{ma.xi.lar}{0}
\verb{maxilar}{cs}{Anat.}{}{}{s.m.}{Osso que forma a maxila.}{ma.xi.lar}{0}
\verb{máxima}{ss}{}{}{}{s.f.}{Frase que contém um conceito ou norma de conduta que pode ou deve ser seguida; axioma, aforismo.}{má.xi.ma}{0}
\verb{máxime}{cs}{}{}{}{adv.}{Principalmente, especialmente, mormente.}{má.xi.me}{0}
\verb{maximizar}{cs}{}{}{}{v.t.}{Elevar ao máximo possível; aumentar, superestimar.}{ma.xi.mi.zar}{\verboinum{1}}
\verb{máximo}{ss}{}{}{}{adj.}{Maior que todos; supremo, excelso.}{má.xi.mo}{0}
\verb{máximo}{ss}{}{}{}{s.m.}{O ponto mais alto; o mais alto grau.}{má.xi.mo}{0}
\verb{maxixe}{ch\ldots{}ch}{}{}{}{s.m.}{Dança brasileira de movimentos rápidos, resultante da mistura da polca com o tango.}{ma.xi.xe}{0}
\verb{maxixe}{ch\ldots{}ch}{}{}{}{s.m.}{Fruto do maxixeiro, cultivado por sua  baga carnosa, usada em saladas.}{ma.xi.xe}{0}
\verb{maxixeiro}{ch\ldots{}ch}{Bot.}{}{}{s.m.}{Planta trepadeira que produz o maxixe.}{ma.xi.xei.ro}{0}
\verb{mazela}{é}{}{}{}{}{Tudo o que aflige.}{ma.ze.la}{0}
\verb{mazela}{é}{}{}{}{s.f.}{Ferida que apresenta lesão externa; chaga.}{ma.ze.la}{0}
\verb{mazurca}{}{Mús.}{}{}{s.f.}{Dança ou música polonesa a três tempos, misto de valsa e polca.}{ma.zur.ca}{0}
\verb{MB}{}{Informát.}{}{}{}{Símb. de \textit{megabyte}.}{MB}{0}
\verb{Mb}{}{Informát.}{}{}{}{Símb. de \textit{megabit}.}{Mb}{0}
\verb{Md}{}{Quím.}{}{}{}{Símb. do \textit{mendelévio}.}{Md}{0}
\verb{mê}{}{}{}{}{s.m.}{Nome da letra \textit{m}; eme.}{mê}{0}
\verb{me}{}{}{}{}{pron.}{Para mim; para minha pessoa.}{me}{0}
\verb{me}{}{}{}{}{}{A mim; a minha pessoa.}{me}{0}
\verb{me}{}{}{}{}{}{De mim.}{me}{0}
\verb{meação}{}{}{"-ões}{}{s.f.}{Divisão em duas partes iguais.}{me.a.ção}{0}
\verb{meada}{}{}{}{}{s.f.}{Quantidade de fio de linha ou de lã enovelada.}{me.a.da}{0}
\verb{meada}{}{Fig.}{}{}{}{Intriga, mexerico, confusão.}{me.a.da}{0}
\verb{meado}{}{}{}{}{adj.}{Que está no meio ou foi dividido ao meio.}{me.a.do}{0}
\verb{mealheiro}{ê}{}{}{}{s.m.}{Pequeno cofre ou caixa onde se guardam moedas.}{me.a.lhei.ro}{0}
\verb{meandro}{}{}{}{}{s.m.}{Cada uma das curvas ou voltas de um caminho ou rio; sinuosidade.}{me.an.dro}{0}
\verb{meão}{}{}{"-ãos}{}{adj.}{Que está no meio, em posição intermediária.}{me.ão}{0}
\verb{meão}{}{}{"-ãos}{}{}{Médio, mediano, normal.}{me.ão}{0}
\verb{mear}{}{}{}{}{v.t.}{Dividir ao meio, em duas partes iguais.}{me.ar}{\verboinum{4}}
\verb{meato}{}{}{}{}{s.m.}{Abertura externa de um canal.}{me.a.to}{0}
\verb{mecânica}{}{Fís.}{}{}{s.f.}{Ciência que estuda as leis do movimento, suas ações e as forças que o criam.}{me.câ.ni.ca}{0}
\verb{mecânica}{}{}{}{}{}{Organização e funcionamento das peças de uma máquina.}{me.câ.ni.ca}{0}
\verb{mecânico}{}{}{}{}{adj.}{Relativo a mecânica.}{me.câ.ni.co}{0}
\verb{mecânico}{}{}{}{}{}{Feito com o emprego de uma máquina ou de um mecanismo.}{me.câ.ni.co}{0}
\verb{mecânico}{}{}{}{}{}{Que se faz sem pensar; maquinal, automático.}{me.câ.ni.co}{0}
\verb{mecânico}{}{}{}{}{s.m.}{Indivíduo especializado no conserto de máquinas ou motores em geral.}{me.câ.ni.co}{0}
\verb{mecanismo}{}{}{}{}{s.m.}{Conjunto de peças que constituem uma máquina; maquinismo.}{me.ca.nis.mo}{0}
\verb{mecanização}{}{}{"-ões}{}{s.f.}{Ato ou efeito de mecanizar.}{me.ca.ni.za.ção}{0}
\verb{mecanizar}{}{}{}{}{v.t.}{Tornar mecânico, semelhante a uma máquina.}{me.ca.ni.zar}{0}
\verb{mecanizar}{}{}{}{}{}{Efetuar por meio de máquina; automatizar.}{me.ca.ni.zar}{\verboinum{1}}
\verb{mecanografia}{}{}{}{}{s.f.}{Fabricação e venda de máquinas datilográficas e taquigráficas.}{me.ca.no.gra.fi.a}{0}
\verb{mecenas}{}{}{}{}{s.2g.2n}{Indivíduo rico que protege e patrocina artistas e intelectuais.}{me.ce.nas}{0}
\verb{mecha}{é}{}{}{}{s.f.}{Porção de corda desfiada embebida em substância inflamável usada para levar fogo e peças de artilharia; pavio.}{me.cha}{0}
\verb{mecha}{é}{}{}{}{}{Porção de fios ou filamentos.}{me.cha}{0}
\verb{mecha}{é}{}{}{}{}{Porção de cabelo de cor diferente do restante; madeixa.}{me.cha}{0}
\verb{medalha}{}{}{}{}{s.f.}{Chapa metálica, em geral de forma circular, com figura de um lado e inscrição de outro.}{me.da.lha}{0}
\verb{medalhão}{}{}{"-ões}{}{s.m.}{Peça ornamental com tampa, na qual se guarda objeto de recordação.}{me.da.lhão}{0}
\verb{medalhão}{}{Fig.}{"-ões}{}{}{Indivíduo importante; figurão.}{me.da.lhão}{0}
\verb{medalhão}{}{Cul.}{"-ões}{}{}{Fatia grossa e redonda de carne bovina ou de peixe, para fritar ou grelhar.}{me.da.lhão}{0}
\verb{médão}{}{}{"-ãos}{}{s.m.}{Monte de areia junto ao mar; duna.}{mé.dão}{0}
\verb{média}{}{Mat.}{}{}{s.f.}{Soma de quantidades diferentes dividida pelo número delas.}{mé.dia}{0}
\verb{média}{}{}{}{}{}{Nota mínima exigida para aprovação em um exame.}{mé.dia}{0}
\verb{média}{}{}{}{}{}{Xícara ou copo de café com leite.}{mé.dia}{0}
\verb{mediação}{}{}{"-ões}{}{s.f.}{Ato ou efeito de mediar; intervenção.}{me.di.a.ção}{0}
\verb{mediador}{ô}{}{}{}{adj.}{Que serve de intermediário; que intervém; medianeiro.}{me.di.a.dor}{0}
\verb{medial}{}{}{"-ais}{}{adj.2g.}{Que medeia; que está no meio.}{me.di.al}{0}
\verb{mediana}{}{Geom.}{}{}{s.f.}{Segmento de reta que vai de um vértice do triângulo ao meio do lado oposto.}{me.di.a.na}{0}
\verb{medianeiro}{ê}{}{}{}{adj.}{Mediador.}{me.di.a.nei.ro}{0}
\verb{mediania}{}{}{}{}{s.f.}{Qualidade de mediano; normalidade, mediocridade.}{me.di.a.ni.a}{0}
\verb{mediano}{}{}{}{}{adj.}{Que fica no meio; médio, meão.}{me.di.a.no}{0}
\verb{mediano}{}{}{}{}{}{Nem muito bom nem muito mal; sofrível, medíocre. }{me.di.a.no}{0}
\verb{mediante}{}{}{}{}{prep.}{Por meio de; por intermédio de.}{me.di.an.te}{0}
\verb{mediar}{}{}{}{}{v.t.}{Repartir ao meio; mear.}{me.di.ar}{0}
\verb{mediar}{}{}{}{}{}{Intervir como mediador.}{me.di.ar}{\verboinum{6}}
\verb{mediato}{}{}{}{}{adj.}{Que se refere a uma coisa por meio de uma terceira; que precisa de intermediário; indireto.}{me.di.a.to}{0}
\verb{mediatriz}{}{Geom.}{}{}{s.f.}{Reta perpendicular ao meio de um segmento de reta.}{me.di.a.triz}{0}
\verb{medicação}{}{}{"-ões}{}{s.f.}{Tratamento feito com remédios.}{me.di.ca.ção}{0}
\verb{medicamentar}{}{}{}{}{v.t.}{Ministrar medicamento a; medicar.}{me.di.ca.men.tar}{\verboinum{1}}
\verb{medicamento}{}{Farm.}{}{}{s.m.}{Qualquer substância usada no tratamento de uma afecção ou doença, ministrada interna ou externamente ao paciente; medicação, remédio, fármaco.}{me.di.ca.men.to}{0}
\verb{medicamentoso}{ô}{}{"-osos ⟨ó⟩}{"-osa ⟨ó⟩}{adj.}{Que tem propriedades terapêuticas.}{me.di.ca.men.to.so}{0}
\verb{medição}{}{}{"-ões}{}{s.f.}{Ato ou efeito de medir; medida.}{me.di.ção}{0}
\verb{medicar}{}{}{}{}{v.t.}{Tratar com medicamento; medicamentar.}{me.di.car}{\verboinum{2}}
\verb{medicina}{}{}{}{}{s.f.}{Ciência que tem por objetivo prevenir, atenuar ou curar doenças.}{me.di.ci.na}{0}
\verb{medicina}{}{}{}{}{}{A profissão de médico.}{me.di.ci.na}{0}
\verb{medicinal}{}{}{"-ais}{}{adj.2g.}{Relativo a medicina.}{me.di.ci.nal}{0}
\verb{medicinal}{}{}{"-ais}{}{}{Que tem propriedades terapêuticas; medicamentoso.}{me.di.ci.nal}{0}
\verb{médico}{}{}{}{}{adj.}{Relativo a medicina.}{mé.di.co}{0}
\verb{médico}{}{}{}{}{s.m.}{Profissional diplomado que pratica a medicina.}{mé.di.co}{0}
\verb{médico"-hospitalar}{}{}{médico"-hospitalares}{}{adj.2g.}{Relativo ao atendimento tanto médico quanto hospitalar.}{mé.di.co"-hos.pi.ta.lar}{0}
\verb{médico"-legal}{}{}{médico"-legais}{}{adj.2g.}{Relativo à medicina legal.}{mé.di.co"-le.gal}{0}
\verb{médico"-legista}{}{Med.}{médicos"-legistas}{médica"-legista}{s.m.}{Médico especializado em medicina legal; legista.}{mé.di.co"-le.gis.ta}{0}
\verb{medida}{}{}{}{}{s.f.}{Grandeza conhecida que serve para avaliar outras; padrão, parâmetro.}{me.di.da}{0}
\verb{medida}{}{}{}{}{}{Dimensão de um corpo; tamanho.}{me.di.da}{0}
\verb{medida}{}{}{}{}{}{Limite, termo.}{me.di.da}{0}
\verb{medida}{}{}{}{}{}{Iniciativa, providência.}{me.di.da}{0}
\verb{medida}{}{}{}{}{}{Número de sílabas de um verso.}{me.di.da}{0}
\verb{medida}{}{}{}{}{}{Usada na locução \textit{à medida que}: ao mesmo tempo; à proporção que; conforme.}{me.di.da}{0}
\verb{medidor}{ô}{}{}{}{adj.}{Diz"-se do instrumento que serve para medir; contador.}{me.di.dor}{0}
\verb{medieval}{}{}{"-ais}{}{adj.2g.}{Relativo ou pertencente à Idade Média.}{me.di.e.val}{0}
\verb{médio}{}{}{}{}{adj.}{Que está no meio de duas coisas; mediano.}{mé.dio}{0}
\verb{médio}{}{}{}{}{s.m.}{O terceiro dedo da mão, entre o anular e o indicador.}{mé.dio}{0}
\verb{medíocre}{}{}{}{}{adj.2g.}{Que não é nem muito bom nem muito mau; mediano, vulgar, ordinário.}{me.dí.o.cre}{0}
\verb{mediocridade}{}{}{}{}{s.f.}{Qualidade de medíocre; vulgaridade.}{me.di.o.cri.da.de}{0}
\verb{medir}{}{}{}{}{v.t.}{Determinar quantas vezes uma grandeza cabe em alguma coisa.}{me.dir}{0}
\verb{medir}{}{}{}{}{}{Avaliar, calcular, ponderar. (\textit{O advogado mediu bem as palavras antes de se pronunciar ao júri.})}{me.dir}{0}
\verb{medir}{}{}{}{}{v.pron.}{Combater, rivalizar, lutar.}{me.dir}{\verboinum{20}}
\verb{meditabundo}{}{}{}{}{adj.}{Que medita, reflete; meditativo, pensativo.}{me.di.ta.bun.do}{0}
\verb{meditação}{}{}{"-ões}{}{s.f.}{Ato ou efeito de meditar; reflexão, concentração.}{me.di.ta.ção}{0}
\verb{meditar}{}{}{}{}{v.t.}{Pensar longamente sobre; refletir, ponderar.}{me.di.tar}{0}
\verb{meditar}{}{}{}{}{v.i.}{Praticar meditações.}{me.di.tar}{\verboinum{1}}
\verb{meditativo}{}{}{}{}{adj.}{Que se entrega à meditação; meditabundo.}{me.di.ta.ti.vo}{0}
\verb{mediterrâneo}{}{}{}{}{adj.}{Diz"-se do mar situado no meio de terras.}{me.di.ter.râ.neo}{0}
\verb{mediterrâneo}{}{}{}{}{}{Relativo ao ou situado no mar Mediterrâneo.}{me.di.ter.râ.neo}{0}
\verb{mediterrâneo}{}{Geogr.}{}{}{s.m.}{Mar interior que se situa entre a Europa e a África.}{me.di.ter.râ.neo}{0}
\verb{médium}{}{Relig.}{}{}{s.2g.}{Segundo o Espiritismo, indivíduo que serve de intermediário entre o mundo dos encarnados e o dos desencarnados.}{mé.dium}{0}
\verb{mediunidade}{}{Relig.}{}{}{s.f.}{Segundo o Espiritismo, faculdade que todo ser encarnado possui de se relacionar, em diferentes níveis, com o mundo dos desencarnados.}{me.di.u.ni.da.de}{0}
\verb{medo}{ê}{}{}{}{s.m.}{Sensação desagradável de inquietação diante de um perigo real ou imaginário; temor, apreensão, receio.}{me.do}{0}
\verb{medonho}{}{}{}{}{adj.}{Que causa medo; horrível, horrendo.}{me.do.nho}{0}
\verb{medrar}{}{}{}{}{v.t.}{Criar raízes; fazer crescer; desenvolver, prosperar.}{me.drar}{\verboinum{1}}
\verb{medroso}{ô}{}{"-osos ⟨ó⟩}{"-osa ⟨ó⟩}{adj.}{Que tem medo; receoso, temeroso.}{me.dro.so}{0}
\verb{medula}{}{Anat.}{}{}{s.f.}{Substância mole e gordurosa contida dentros dos ossos; tutano.}{me.du.la}{0}
\verb{medula}{}{Anat.}{}{}{}{Feixe de fibras nervosas localizado dentro da coluna vertebral; medula espinhal.}{me.du.la}{0}
\verb{medular}{}{}{}{}{adj.2g.}{Relativo à ou da natureza da medula.}{me.du.lar}{0}
\verb{medusa}{}{Zool.}{}{}{s.f.}{Animal celenterado, marinho, de corpo gelatinoso, em forma de sino e com tentáculos, geralmente venenosos; água"-viva.}{me.du.sa}{0}
\verb{meeiro}{ê}{}{}{}{s.m.}{Colono que planta em terreno de outrem e com este divide o resultado da colheita.}{me.ei.ro}{0}
\verb{meeting}{}{}{}{}{s.m.}{Reunião pública para discutir questões de ordem social ou política; comício.}{\textit{meeting}}{0}
\verb{mefistofélico}{}{}{}{}{adj.}{Que envolve perfídia, traição; desleal.}{me.fis.to.fé.li.co}{0}
\verb{mega}{é}{Informát.}{}{}{s.m.}{Forma reduzida de \textit{megabyte}.}{me.ga}{0}
\verb{megabit}{}{Informát.}{}{}{s.m.}{Unidade de medida de informação, equivalente a 1.024 \textit{bits}. Símb.: Mb.}{\textit{megabit}}{0}
\verb{megabyte}{}{Informát.}{}{}{s.m.}{Unidade de medida de informação, equivalente a 1.024 \textit{bytes}. Símb.: MB.}{\textit{megabyte}}{0}
\verb{megafone}{ô}{}{}{}{s.m.}{Instrumento em forma de funil utilizado para ampliar o som da voz; alto"-falante.}{me.ga.fo.ne}{0}
\verb{mega"-hertz}{é\ldots{}é}{Fís.}{}{}{s.m.}{Unidade de medida de frequência que equivale a um milhão de hertz.}{me.ga"-hertz}{0}
\verb{megalítico}{}{Arqueol.}{}{}{adj.}{Diz"-se de monumento pré"-histórico feito de grandes blocos de pedras, como os dolmens e os menires.}{me.ga.lí.ti.co}{0}
\verb{megalocéfalo}{}{}{}{}{adj.}{De cabeça anormalmente grande; macrocéfalo.}{me.ga.lo.cé.fa.lo}{0}
\verb{megalomania}{}{}{}{}{s.f.}{Mania de grandeza ou de poder.}{me.ga.lo.ma.ni.a}{0}
\verb{megalomaníaco}{}{}{}{}{adj.}{Que sofre de megalomania.}{me.ga.lo.ma.ní.a.co}{0}
\verb{megalópole}{}{}{}{}{s.f.}{Região densamente povoada, formada por uma metrópole e várias cidades ao seu redor.}{me.ga.ló.po.le}{0}
\verb{megaton}{}{Fís.}{}{}{s.m.}{Unidade de medida equivalente à energia liberada pela explosão de um milhão de toneladas de \textsc{tnt}.}{me.ga.ton}{0}
\verb{megera}{é}{}{}{}{s.f.}{Mulher má, cruel.}{me.ge.ra}{0}
%\verb{}{}{}{}{}{}{}{}{0}
%\verb{}{}{}{}{}{}{}{}{0}
\verb{meia}{ê}{}{}{}{s.f.}{Peça de malha, algodão, lã ou outro tecido, usada para cobrir o pé e parte da perna.}{mei.a}{0}
\verb{meia}{ê}{}{}{}{s.f.}{Redução de meia dúzia, quantidade equivalente ao número \textit{6}.}{mei.a}{0}
\verb{meia"-água}{ê}{}{meias"-águas}{}{s.f.}{Telhado de um só plano.}{mei.a"-á.gua}{0}
\verb{meia"-armador}{ê\ldots{}ô}{Esport.}{meias"-armadores}{}{s.m.}{No futebol, jogador que atua no meio de campo, armando as jogadas.}{mei.a"-ar.ma.dor}{0}
\verb{meia"-calça}{ê}{}{meias"-calças}{}{s.f.}{Meia feminina que veste pés e pernas e chega até a cintura.}{mei.a"-cal.ça}{0}
\verb{meia"-direita}{ê\ldots{}ê}{Esport.}{meias"-direitas}{}{s.m.}{No futebol, jogador que atua na linha direita do campo, entre o centro e a ponta direita.}{mei.a"-di.rei.ta}{0}
\verb{meia"-direita}{ê\ldots{}ê}{}{meias"-direitas}{}{}{Essa posição ou função.}{mei.a"-di.rei.ta}{0}
\verb{meia"-entrada}{}{}{meias"-entradas}{}{s.f.}{Bilhete para eventos vendido pela metade do preço, geralmente para menores de idade, estudantes e aposentados.}{mei.a"-en.tra.da}{0}
\verb{meia"-esquerda}{}{Esport.}{meias"-esquerdas}{}{s.m.}{No futebol, jogador que atua na linha esquerda do campo, entre o centro e a ponta esquerda.}{mei.a"-es.quer.da}{0}
\verb{meia"-esquerda}{}{}{meias"-esquerdas}{}{}{Essa posição ou função.}{mei.a"-es.quer.da}{0}
\verb{meia"-estação}{}{}{meias"-estações}{}{s.f.}{Época do ano em que não faz nem muito frio nem muito calor.}{mei.a"-es.ta.ção}{0}
\verb{meia"-idade}{}{}{meias"-idades}{}{s.f.}{Idade que compreende entre os 40 e os 60 anos aproximadamente.}{mei.a"-i.da.de}{0}
\verb{meia"-lua}{}{}{meias"-luas}{}{s.f.}{Forma semicircular da lua quando ela está na fase crescente ou minguante.}{mei.a"-lu.a}{0}
\verb{meia"-luz}{}{}{meias"-luzes}{}{s.f.}{Luz suave própria do entardecer ou do amanhecer, ou de ambiente fracamente iluminado; penumbra.}{mei.a"-luz}{0}
\verb{meia"-noite}{}{}{meias"-noites}{}{s.f.}{Momento que divide a noite em duas partes iguais ou separa um dia do outro; zero hora; vinte e quatro horas.}{mei.a"-noi.te}{0}
\verb{meia"-sola}{ó}{}{meia"-solas ⟨ó⟩}{}{s.f.}{Conserto que o sapateiro faz num sapato usado, substituindo a parte da sola gasta.}{mei.a"-so.la}{0}
\verb{meia"-tigela}{é}{}{}{}{s.f.}{Usado na locução \textit{de meia"-tigela}: de pouco valor, medíocre, insignificante, vulgar.}{mei.a"-ti.ge.la}{0}
\verb{meia"-tinta}{}{}{meias"-tintas}{}{s.f.}{Graduação suave de cores, entre o claro e o escuro.}{mei.a"-tin.ta}{0}
\verb{meigo}{ê}{}{}{}{adj.}{Que é delicado e carinhoso com outras pessoas; terno, afetuoso. (\textit{Seu sorriso meigo encantava a todos.})}{mei.go}{0}
\verb{meiguice}{}{}{}{}{s.f.}{Qualidade de meigo; ternura, suavidade, brandura.}{mei.gui.ce}{0}
\verb{meio}{ê}{}{}{}{num.}{Metade de um; fracionário de dois.}{mei.o}{0}
\verb{meio}{ê}{}{}{}{adj.}{Que indica a metade de algo. (\textit{O garçom trouxe meia porção de batata frita.})}{mei.o}{0}
\verb{meio}{ê}{}{}{}{adv.}{Um tanto; em parte. (\textit{Minha irmã andou meio adoentada ultimamente.})}{mei.o}{0}
\verb{meio}{ê}{}{}{}{s.m.}{O centro de um espaço; núcleo, âmago.}{mei.o}{0}
\verb{meio}{ê}{}{}{}{}{Momento em que se separa um espaço de tempo pela metade. (\textit{O filme estava no meio quando acabou a luz.})}{mei.o}{0}
\verb{meio}{ê}{}{}{}{}{Expediente, condição, maneira.}{mei.o}{0}
\verb{meio"-dia}{ê}{}{meios"-dias}{}{s.m.}{Momento que divide o dia em duas partes iguais, a manhã e a tarde; as doze horas.}{mei.o"-di.a}{0}
\verb{meio"-fio}{ê}{}{meios"-fios}{}{s.m.}{Fileira de pedras colocadas na calçada que a separam da pista por onde transitam os veículos.}{mei.o"-fi.o}{0}
\verb{meio"-irmão}{ê}{}{meios"-irmãos}{}{s.m.}{Irmão só por parte do pai ou da mãe.}{mei.o"-ir.mão}{0}
\verb{meio"-soprano}{ê}{Mús.}{meios"-sopranos}{}{s.m.}{Voz feminina situada entre o contralto e o soprano.}{mei.o"-so.pra.no}{0}
\verb{meio"-tempo}{ê}{}{meios"-tempos}{}{s.m.}{Espaço de tempo que divide dois acontecimentos; intervalo.}{mei.o"-tem.po}{0}
\verb{meio"-termo}{ê\ldots{}ê}{}{meios"-termos}{}{s.m.}{Posição aceita por todos; moderação, comedimento.}{mei.o"-ter.mo}{0}
\verb{meio"-tom}{ê}{Mús.}{meios"-tons}{}{s.m.}{Intervalo que vale a metade de um tom; semitom.}{mei.o"-tom}{0}
\verb{meirinho}{}{}{}{}{s.m.}{Nome que antigamente se dava ao atual oficial de justiça.}{mei.ri.nho}{0}
\verb{meitinério}{}{Quím.}{}{}{s.m.}{Elemento químico obtido artificialmente por fissão nuclear. \elemento{109}{(266)}{Mt}.}{mei.ti.né.rio}{0}
\verb{mel}{é}{}{méis \textit{ou} meles ⟨é⟩}{}{s.m.}{Substância líquida e doce produzida pelas abelhas com o néctar das flores e usada como alimento.}{mel}{0}
\verb{melaço}{}{}{}{}{s.m.}{Substância viscosa resultante do processo de refinação do açúcar.}{me.la.ço}{0}
\verb{melado}{}{}{}{}{adj.}{Muito doce; próximo ao mel.}{me.la.do}{0}
\verb{melado}{}{}{}{}{s.m.}{Calda grossa da cana"-de"-açúcar, usada para fazer rapadura.}{me.la.do}{0}
\verb{melado}{}{}{}{}{adj.}{Pegajoso, grudento.}{me.la.do}{0}
\verb{melancia}{}{Bot.}{}{}{s.f.}{Planta rasteira, que produz frutos grandes e de casca esverdeada, com polpa vermelha e suculenta.}{me.lan.ci.a}{0}
\verb{melancia}{}{}{}{}{}{O fruto dessa planta.}{me.lan.ci.a}{0}
\verb{melancolia}{}{}{}{}{s.f.}{Estado de tristeza profunda e persistente; depressão.}{me.lan.co.li.a}{0}
\verb{melancólico}{}{}{}{}{adj.}{Que sofre de melancolia; triste, sombrio.}{me.lan.có.li.co}{0}
\verb{melanésio}{}{}{}{}{adj.}{Relativo a Melanésia, conjunto de ilhas localizadas na Oceania.}{me.la.né.sio}{0}
\verb{melanésio}{}{}{}{}{s.m.}{Indivíduo natural ou habitante da Melanésia.}{me.la.né.sio}{0}
\verb{melanésio}{}{}{}{}{}{Conjunto de línguas faladas na Oceania.}{me.la.né.sio}{0}
\verb{melanina}{}{Bioquím.}{}{}{s.f.}{Nome comum a diversas proteínas encontradas em vegetais e animais, responsáveis pela pigmentação preta ou marrom, e que nos homens determina a cor dos pelos, dos cabelos, da pele e da íris.}{me.la.ni.na}{0}
\verb{melão}{}{}{"-ões}{}{s.m.}{Fruto comestível do meloeiro, de casca amarela ou verde, polpa branca e muitas sementes.}{me.lão}{0}
\verb{melar}{}{}{}{}{v.t.}{Cobrir ou adoçar com mel.}{me.lar}{0}
\verb{melar}{}{}{}{}{}{Sujar com mel ou outra substância pegajosa.}{me.lar}{0}
\verb{melar}{}{Pop.}{}{}{v.i.}{Frustrar, malograr.}{me.lar}{\verboinum{1}}
\verb{meleca}{é}{}{}{}{s.f.}{Coisa suja ou nojenta.}{me.le.ca}{0}
\verb{meleca}{é}{}{}{}{}{Secreção do nariz quando ressequida.}{me.le.ca}{0}
\verb{melena}{}{}{}{}{s.f.}{Cabelo comprido, solto e despenteado; guedelha.}{me.le.na}{0}
\verb{melhor}{ó}{}{}{}{adj.2g.}{Comparativo de superioridade e superlativo de \textit{bom}; superior.}{me.lhor}{0}
\verb{melhor}{ó}{}{}{}{s.m.}{O que é superior a tudo o mais.}{me.lhor}{0}
\verb{melhor}{ó}{}{}{}{adv.}{Comparativo de superioridade de \textit{bem}; mais bem.}{me.lhor}{0}
\verb{melhora}{ó}{}{}{}{s.f.}{Ato ou efeito de melhorar; melhoria.}{me.lho.ra}{0}
\verb{melhora}{ó}{}{}{}{}{Mudança para um estado ou condição melhor.}{me.lho.ra}{0}
\verb{melhoramento}{}{}{}{}{s.m.}{Ato ou efeito de melhorar; melhoria, benfeitoria.}{me.lho.ra.men.to}{0}
\verb{melhorar}{}{}{}{}{v.t.}{Tornar melhor; aperfeiçoar.}{me.lho.rar}{0}
\verb{melhorar}{}{}{}{}{v.i.}{Adquirir melhor estado de saúde; recuperar"-se, convalescer.}{me.lho.rar}{\verboinum{1}}
\verb{melhoria}{}{}{}{}{s.f.}{Mudança para melhor; melhora.}{me.lho.ri.a}{0}
\verb{melhoria}{}{}{}{}{}{Benfeitoria, melhoramento.}{me.lho.ri.a}{0}
\verb{meliante}{}{}{}{}{s.2g.}{Indivíduo que abusa da confiança de outrem para trapacear; malandro, trapaceiro.}{me.li.an.te}{0}
\verb{melífero}{}{}{}{}{adj.}{Que produz mel.}{me.lí.fe.ro}{0}
\verb{melificar}{}{}{}{}{v.t.}{Converter em mel.}{me.li.fi.car}{0}
\verb{melificar}{}{}{}{}{v.i.}{Fabricar mel.}{me.li.fi.car}{\verboinum{2}}
\verb{melífluo}{}{}{}{}{adj.}{Que flui como o mel.}{me.lí.flu.o}{0}
\verb{melífluo}{}{Fig.}{}{}{}{Suave, doce, brando.}{me.lí.flu.o}{0}
\verb{melindrar}{}{}{}{}{v.t.}{Deixar (alguém) ofendido; magoar.}{me.lin.drar}{\verboinum{1}}
\verb{melindre}{}{}{}{}{s.m.}{Facilidade de se magoar, de se ofender; suscetibilidade.}{me.lin.dre}{0}
\verb{melindroso}{ô}{}{"-osos ⟨ó⟩}{"-osa ⟨ó⟩}{adj.}{Que se ofende com facilidade; suscetível.}{me.lin.dro.so}{0}
\verb{melindroso}{ô}{}{"-osos ⟨ó⟩}{"-osa ⟨ó⟩}{}{Que requer muito cuidado; arriscado.}{me.lin.dro.so}{0}
\verb{melissa}{}{Bot.}{}{}{s.f.}{Erva aromática cultivada devido a seus efeitos calmantes; erva"-cidreira.}{me.lis.sa}{0}
\verb{melissa}{}{Zool.}{}{}{}{Abelha.}{me.lis.sa}{0}
\verb{melissografia}{}{Biol.}{}{}{s.f.}{Estudo ou tratado acerca das abelhas.}{me.lis.so.gra.fi.a}{0}
\verb{melodia}{}{}{}{}{s.f.}{Combinação de sons agradáveis ao ouvido.}{me.lo.di.a}{0}
\verb{melodia}{}{}{}{}{}{Musicalidade, sonoridade, suavidade.}{me.lo.di.a}{0}
\verb{melódico}{}{}{}{}{adj.}{Relativo a melodia; melodioso.}{me.ló.di.co}{0}
\verb{melodioso}{ô}{}{"-osos ⟨ó⟩}{"-osa ⟨ó⟩}{adj.}{Cheio de melodia; harmonioso.}{me.lo.di.o.so}{0}
\verb{melodrama}{}{}{}{}{s.f.}{Peça dramática de caráter popular caracterizada pelo extremo sentimentalismo e romantismo.}{me.lo.dra.ma}{0}
\verb{melodramático}{}{}{}{}{adj.}{Relativo a melodrama.}{me.lo.dra.má.ti.co}{0}
\verb{melodramático}{}{}{}{}{}{Exagerado, sentimentalista.}{me.lo.dra.má.ti.co}{0}
\verb{meloeiro}{ê}{Bot.}{}{}{s.m.}{Planta rasteira cultivada devido a seus frutos de polpa branca e adocicada.}{me.lo.ei.ro}{0}
\verb{melomania}{}{}{}{}{s.f.}{Gosto exagerado pela música.}{me.lo.ma.ni.a}{0}
\verb{melopeia}{é}{Mús.}{}{}{s.f.}{Peça musical que acompanha uma recitação.}{me.lo.pei.a}{0}
\verb{meloso}{ô}{}{"-osos ⟨ó⟩}{"-osa ⟨ó⟩}{adj.}{Semelhante ao mel.}{me.lo.so}{0}
\verb{meloso}{ô}{Fig.}{"-osos ⟨ó⟩}{"-osa ⟨ó⟩}{}{Excessivamente sentimental; piegas.}{me.lo.so}{0}
\verb{melro}{}{Zool.}{}{}{s.m.}{Pássaro de plumagem preta, bico amarelo e canto muito melodioso.}{mel.ro}{0}
\verb{membrana}{}{Biol.}{}{}{s.f.}{Película que reveste certos órgãos animais ou vegetais.}{mem.bra.na}{0}
\verb{membranoso}{ô}{}{"-osos ⟨ó⟩}{"-osa ⟨ó⟩}{adj.}{Que possui membrana.}{mem.bra.no.so}{0}
\verb{membranoso}{ô}{}{"-osos ⟨ó⟩}{"-osa ⟨ó⟩}{}{Que tem consistência ou natureza de membrana.}{mem.bra.no.so}{0}
\verb{membro}{}{}{}{}{s.m.}{Cada uma das partes alongadas do corpo do homem e de certos animais constituídos por segmentos articulados.}{mem.bro}{0}
\verb{membro}{}{}{}{}{}{Que faz parte de uma entidade, corporação ou família.}{mem.bro}{0}
\verb{memento}{}{}{}{}{s.m.}{Caderneta ou agenda usada para anotar o que não se pode esquecer.}{me.men.to}{0}
\verb{memento}{}{Relig.}{}{}{}{Prece litúrgica feita em lembrança dos vivos e dos mortos.}{me.men.to}{0}
\verb{memorando}{}{}{}{}{s.m.}{Nota ou comunicação por escrito usada em empresas ou instituições.}{me.mo.ran.do}{0}
\verb{memorar}{}{}{}{}{v.t.}{Trazer à memória; recordar, lembrar.}{me.mo.rar}{0}
\verb{memorar}{}{}{}{}{}{Celebrar, comemorar.}{me.mo.rar}{\verboinum{1}}
\verb{memorável}{}{}{"-eis}{}{adj.2g.}{Digno de ser lembrado.}{me.mo.rá.vel}{0}
\verb{memorável}{}{}{"-eis}{}{}{Célebre, notável, afamado.}{me.mo.rá.vel}{0}
\verb{memória}{}{}{}{}{s.f.}{Capacidade do ser humano em reter as ideias, as experiências, as impressões.}{me.mó.ria}{0}
\verb{memória}{}{}{}{}{}{Ato ou efeito de lembrar; lembrança, recordação, reminiscência.}{me.mó.ria}{0}
\verb{memória}{}{Informát.}{}{}{}{Local em que o computador armazena as informações.}{me.mó.ria}{0}
\verb{memorial}{}{}{"-ais}{}{s.m.}{Relato por escrito que conta os fatos e feitos importantes da vida de alguém.}{me.mo.ri.al}{0}
\verb{memorial}{}{}{"-ais}{}{}{Obra construída para preservar a memória de alguém; monumento.}{me.mo.ri.al}{0}
\verb{memorialista}{}{}{}{}{s.2g.}{Indivíduo que escreve memórias.}{me.mo.ri.a.lis.ta}{0}
\verb{memórias}{}{}{}{}{s.f.pl.}{Relato dos fatos que se passaram com uma pessoa.}{me.mó.ri.as}{0}
\verb{memorizar}{}{}{}{}{v.t.}{Guardar na memória; decorar.}{me.mo.ri.zar}{\verboinum{1}}
\verb{menarca}{}{}{}{}{s.f.}{A primeira menstruação.}{me.nar.ca}{0}
\verb{menção}{}{}{"-ões}{}{s.f.}{Ato ou efeito de mencionar.}{men.ção}{0}
\verb{menção}{}{}{"-ões}{}{}{Referência, alusão.}{men.ção}{0}
\verb{menchevique}{}{}{}{}{adj.2g.}{Relativo ao partido russo moderado que se opunha ao bolchevista, de tendências radicais.}{men.che.vi.que}{0}
\verb{menchevique}{}{}{}{}{s.2g.}{Membro desse partido.}{men.che.vi.que}{0}
\verb{mencionar}{}{}{}{}{v.t.}{Fazer menção de; referir, aludir.}{men.ci.o.nar}{\verboinum{1}}
\verb{mendaz}{}{}{}{}{adj.2g.}{Que mente ou revela falsidade; mentiroso, hipócrita.}{men.daz}{0}
\verb{mendelévio}{}{Quím.}{}{}{s.m.}{Elemento químico radioativo, do grupo dos actinídeos, obtido artificialmente. \elemento{101}{(258)}{Md}.}{men.de.lé.vio}{0}
\verb{mendicância}{}{}{}{}{s.f.}{Estado ou condição de quem mendiga; mendicidade, miséria.}{men.di.cân.cia}{0}
\verb{mendicante}{}{}{}{}{adj.2g.}{Que pede esmola, que mendiga.}{men.di.can.te}{0}
\verb{mendicante}{}{}{}{}{s.2g.}{Pessoa que mendiga; mendigo, pedinte.}{men.di.can.te}{0}
\verb{mendicidade}{}{}{}{}{s.f.}{Mendicância.}{men.di.ci.da.de}{0}
\verb{mendigar}{}{}{}{}{v.t.}{Pedir esmola para viver; esmolar.}{men.di.gar}{0}
\verb{mendigar}{}{}{}{}{}{Pedir com humildade; implorar.}{men.di.gar}{\verboinum{5}}
\verb{mendigo}{}{}{}{}{s.m.}{Indivíduo que pede esmola para viver; pedinte.}{men.di.go}{0}
\verb{menear}{}{}{}{}{v.t.}{Mover de um lado para outro; mexer, saracotear.}{me.ne.ar}{\verboinum{4}}
\verb{meneio}{ê}{}{}{}{s.m.}{Ato ou efeito de menear; gesto, aceno.}{me.nei.o}{0}
\verb{menestrel}{é}{Mús.}{"-éis}{}{s.m.}{Poeta e músico que atuava nos castelos durante a Idade Média; trovador. }{me.nes.trel}{0}
\verb{menina}{}{}{}{}{s.f.}{Criança do sexo feminino; garota.}{me.ni.na}{0}
\verb{menina}{}{}{}{}{}{Forma carinhosa e familiar de tratar as mulheres.}{me.ni.na}{0}
\verb{meninada}{}{}{}{}{s.f.}{Grupo de meninos ou meninas; criançada.}{me.ni.na.da}{0}
\verb{menina"-moça}{}{}{meninas"-moças}{}{s.f.}{Menina que já entrou na puberdade.}{me.ni.na"-mo.ça}{0}
\verb{meninge}{}{Anat.}{}{}{s.f.}{Cada uma das três membranas que envolvem o cérebro e a medula espinhal.}{me.nin.ge}{0}
\verb{meningite}{}{Med.}{}{}{s.f.}{Inflamação das meninges.}{me.nin.gi.te}{0}
\verb{meninice}{}{}{}{}{s.f.}{Idade de menino; infância.}{me.ni.ni.ce}{0}
\verb{meninice}{}{}{}{}{}{Ato ou dito próprio de menino; criancice.}{me.ni.ni.ce}{0}
\verb{menino}{}{}{}{}{s.m.}{Criança do sexo masculino; garoto.}{me.ni.no}{0}
\verb{menino}{}{}{}{}{}{Forma carinhosa e familiar de tratar um parente ou amigo, mesmo adulto.}{me.ni.no}{0}
\verb{meninota}{ó}{}{}{}{s.f.}{Menina crescida; mocinha.}{me.ni.no.ta}{0}
\verb{menir}{}{Arqueol.}{}{}{s.m.}{Monumento pré"-histórico constituído de um único bloco de pedra fincado verticalmente no solo.}{me.nir}{0}
\verb{menisco}{}{Anat.}{}{}{s.m.}{Cartilagem fibrosa e muscular, em forma de meia"-lua, situada entre os ossos ou em certas articulações.}{me.nis.co}{0}
%\verb{}{}{}{}{}{}{}{}{0}
%\verb{}{}{}{}{}{}{}{}{0}
\verb{menopausa}{}{}{}{}{s.f.}{Época da vida da mulher em que cessa definitivamente a menstruação; climatério.}{me.no.pau.sa}{0}
\verb{menor}{ó}{}{}{}{adj.2g.}{Comparativo de superioridade de \textit{pequeno}; mais pequeno.}{me.nor}{0}
\verb{menor}{ó}{}{}{}{}{Diz"-se do indivíduo que ainda não atingiu a maioridade.}{me.nor}{0}
\verb{menor}{ó}{}{}{}{}{Ínfimo, mínimo.}{me.nor}{0}
\verb{menoridade}{}{Jur.}{}{}{s.f.}{Período da vida de um indivíduo que não atingiu a idade legal para exercer diretamente os atos da vida civil.}{me.no.ri.da.de}{0}
\verb{menorragia}{}{Med.}{}{}{s.f.}{Aumento excessivo do fluxo menstrual.}{me.nor.ra.gi.a}{0}
\verb{menorreia}{é}{Biol.}{}{}{s.f.}{Fluxo menstrual; menstruação.}{me.nor.rei.a}{0}
\verb{menos}{}{}{}{}{adv.}{Em quantidade ou intensidade menor.}{me.nos}{0}
\verb{menos}{}{}{}{}{prep.}{À exceção de; exceto, salvo. (\textit{Todos poderão entrar, menos os que se atrasarem.})}{me.nos}{0}
\verb{menos}{}{}{}{}{pron.}{Expressa menor número ou quantidade.}{me.nos}{0}
\verb{menos}{}{}{}{}{s.m.}{Aquilo que tem menor importância; o mínimo.}{me.nos}{0}
\verb{menoscabar}{}{}{}{}{v.t.}{Fazer pouco caso de; desprezar, menosprezar.}{me.nos.ca.bar}{\verboinum{1}}
\verb{menoscabo}{}{}{}{}{s.m.}{Ato ou efeito de menoscabar; menosprezo, depreciação.}{me.nos.ca.bo}{0}
\verb{menosprezar}{}{}{}{}{v.t.}{Fazer pouco caso de; desdenhar, desprezar.}{me.nos.pre.zar}{\verboinum{1}}
\verb{menosprezo}{ê}{}{}{}{s.m.}{Ato ou efeito de menosprezar; desprezo, desdém.}{me.nos.pre.zo}{0}
\verb{mensageiro}{ê}{}{}{}{s.m.}{Indivíduo que porta mensagens ou encomendas.}{men.sa.gei.ro}{0}
\verb{mensagem}{}{}{}{}{s.f.}{Comunicação verbal; recado.}{men.sa.gem}{0}
\verb{mensagem}{}{}{}{}{}{Comunicação oficial entre autoridades; despacho.}{men.sa.gem}{0}
\verb{mensagem}{}{}{}{}{}{O que há de essencial e original em uma obra artística ou literária.}{men.sa.gem}{0}
\verb{mensal}{}{}{}{}{adj.2g.}{Relativo a mês.}{men.sal}{0}
\verb{mensal}{}{}{}{}{}{Que dura um mês.}{men.sal}{0}
\verb{mensal}{}{}{}{}{}{Que ocorre ou que se faz de mês em mês.}{men.sal}{0}
\verb{mensalidade}{}{}{}{}{s.f.}{Quantia que se paga a cada mês por algum serviço; mesada.}{men.sa.li.da.de}{0}
\verb{mensalista}{}{}{}{}{s.2g.}{Funcionário que recebe salário mensal.}{men.sa.lis.ta}{0}
\verb{mensário}{}{}{}{}{s.m.}{Periódico publicado mensalmente.}{men.sá.rio}{0}
\verb{menstruação}{}{Biol.}{"-ões}{}{s.f.}{Fluxo periódico de sangue que provém do útero da mulher a partir da puberdade; mênstruo.}{mens.tru.a.ção}{0}
\verb{menstruar}{}{}{}{}{v.i.}{Ter o fluxo menstrual.}{mens.tru.ar}{\verboinum{1}}
\verb{mênstruo}{}{}{}{}{s.m.}{Menstruação.}{mêns.tru.o}{0}
\verb{mensuração}{}{}{"-ões}{}{s.f.}{Ato ou efeito de mensurar; medição.}{men.su.ra.ção}{0}
\verb{mensurar}{}{}{}{}{v.t.}{Determinar as dimensões de; medir.}{men.su.rar}{\verboinum{1}}
\verb{mensurável}{}{}{"-eis}{}{adj.2g.}{Que pode ser medido.}{men.su.rá.vel}{0}
\verb{menta}{}{Bot.}{}{}{s.f.}{Nome dado a diversas espécies de hortelã; hortelã"-pimenta.}{men.ta}{0}
\verb{mental}{}{}{"-ais}{}{adj.2g.}{Relativo a mente; intelectual, cognitivo.}{men.tal}{0}
\verb{mental}{}{Fig.}{"-ais}{}{}{Que parece atuar mais com as faculdades da mente do que com as do corpo; espiritual}{men.tal}{0}
\verb{mentalidade}{}{}{}{}{s.f.}{Modo individual de pensar e julgar; opinião, convicção.}{men.ta.li.da.de}{0}
\verb{mentalidade}{}{}{}{}{}{Conjunto das faculdades intelectuais do indivíduo; mente, inteligência.}{men.ta.li.da.de}{0}
\verb{mentalizar}{}{}{}{}{v.t.}{Conceber na mente; imaginar, inventar.}{men.ta.li.zar}{\verboinum{1}}
\verb{mente}{}{}{}{}{s.f.}{Parte inteligente e sensível do ser humano; intelecto, espírito.}{men.te}{0}
\verb{mente}{}{}{}{}{}{Intenção, propósito, intuito.}{men.te}{0}
\verb{mentecapto}{}{}{}{}{adj.}{Que perdeu o uso da razão; alienado, louco, imbecil.}{men.te.cap.to}{0}
\verb{mentir}{}{}{}{}{v.t.}{Faltar com a verdade; enganar, iludir.}{men.tir}{\verboinum{29}}
\verb{mentira}{}{}{}{}{s.f.}{Afirmação contrária à verdade; falsidade, fraude.}{men.ti.ra}{0}
\verb{mentiroso}{ô}{}{"-osos ⟨ó⟩}{"-osa ⟨ó⟩}{adj.}{Que mente, engana; falso, ilusório.}{men.ti.ro.so}{0}
\verb{mento}{}{Anat.}{}{}{s.m.}{Parte inferior e média da face, abaixo do lábio inferior; queixo.}{men.to}{0}
\verb{mentol}{ó}{Quím.}{"-óis}{}{s.m.}{Substância alcoólica extraída da essência da menta e utilizada em licores, perfumaria, inalantes nasais e pastilhas para tosse.}{men.tol}{0}
\verb{mentolado}{}{}{}{}{adj.}{Que contém mentol. (\textit{Minha mãe costumava passar talco mentolado nas assaduras de meu irmão.})}{men.to.la.do}{0}
\verb{mentor}{ô}{}{}{}{s.m.}{Indivíduo que serve de guia intelectual ou espiritual.}{men.tor}{0}
\verb{menu}{}{}{}{}{s.m.}{Relação das refeições disponíveis para consumo em um restaurante; cardápio.}{me.nu}{0}
\verb{menu}{}{Informát.}{}{}{}{Lista das opções disponíveis para o usuário exibidas no monitor de um computador ou de uma televisão.}{me.nu}{0}
\verb{mequetrefe}{é}{}{}{}{s.m.}{Indivíduo sem importância, insignificante; joão"-ninguém.}{me.que.tre.fe}{0}
\verb{mercadejar}{}{}{}{}{v.i.}{Fazer negócios como mercador; comprar e vender; negociar.}{mer.ca.de.jar}{\verboinum{1}}
\verb{mercadinho}{}{Bras.}{}{}{s.m.}{Pequeno estabelecimento onde se vendem frutas, verduras, cereais; quitanda.}{mer.ca.di.nho}{0}
\verb{mercado}{}{}{}{}{s.m.}{Estabelecimento onde se vendem gêneros alimentícios, produtos de higiene e limpeza etc. (\textit{Minha prima foi ao mercado comprar frutas.})}{mer.ca.do}{0}
\verb{mercado}{}{}{}{}{}{Transação de produtos ou valores segundo a lei da oferta e da procura. (\textit{O preço de mercado para esse tipo de imóvel costuma ser alto.})}{mer.ca.do}{0}
\verb{mercado}{}{}{}{}{}{Centro de negócios e transações financeiras. (\textit{São Paulo tem o maior mercado financeiro da América Latina.})}{mer.ca.do}{0}
\verb{mercadologia}{}{}{}{}{s.f.}{Estratégia empresarial para estimular vendas; \textit{marketing}.}{mer.ca.do.lo.gi.a}{0}
\verb{mercador}{ô}{}{}{}{s.m.}{Indivíduo que compra e vende mercadorias; negociante, comerciante.}{mer.ca.dor}{0}
\verb{mercadoria}{}{}{}{}{s.f.}{Qualquer produto passível de ser comprado ou vendido; mercancia.}{mer.ca.do.ri.a}{0}
\verb{mercancia}{}{}{}{}{s.f.}{Mercadoria.}{mer.can.ci.a}{0}
\verb{mercancia}{}{}{}{}{}{Ato ou efeito de mercanciar.}{mer.can.ci.a}{0}
\verb{mercanciar}{}{}{}{}{v.i.}{Mercadejar.}{mer.can.ci.ar}{\verboinum{1}}
\verb{mercante}{}{}{}{}{adj.2g.}{Relativo ao comércio, ao movimento comercial; mercantil.}{mer.can.te}{0}
\verb{mercantil}{}{}{"-is}{}{adj.2g.}{Relativo a comércio, a mercador ou a mercadoria; comercial.}{mer.can.til}{0}
\verb{mercantilismo}{}{Hist.}{}{}{s.m.}{Política econômica que imperou na Europa nos séculos \textsc{xvi} e \textsc{xvii}, baseada no acúmulo de metais preciosos, no estímulo à manufatura e no colonialismo.}{mer.can.ti.lis.mo}{0}
\verb{mercantilismo}{}{}{}{}{}{Propensão a relacionar qualquer coisa ao interesse comercial, ao lucro ou às vantagens financeiras.}{mer.can.ti.lis.mo}{0}
\verb{mercar}{}{}{}{}{v.t.}{Comprar para revender; negociar, comerciar.}{mer.car}{\verboinum{2}}
\verb{mercê}{}{}{}{}{s.f.}{Benefício que se concede ou se faz a outrem. (\textit{Viemos aqui para pedir uma mercê a Vossa Majestade.})}{mer.cê}{0}
\verb{mercê}{}{}{}{}{}{(\textit{à mercê}) À vontade de; ao sabor de. (\textit{Não podemos ficar à  mercê dos acontecimentos.})}{mer.cê}{0}
\verb{mercearia}{}{}{}{}{s.f.}{Estabelecimento em que se vendem gêneros alimentícios e especiarias; armazém, venda.}{mer.ce.a.ria}{0}
\verb{merceeiro}{ê}{}{}{}{s.m.}{Proprietário de mercearia.}{mer.ce.ei.ro}{0}
\verb{mercenário}{}{}{}{}{adj.}{Que trabalha apenas por interesse financeiro ou vantagem comercial; interesseiro, venal.}{mer.ce.ná.rio}{0}
\verb{mercenarismo}{}{}{}{}{s.m.}{Caráter ou atitude de quem é mercenário.}{mer.ce.na.ris.mo}{0}
\verb{merceologia}{}{}{}{}{s.f.}{Parte da ciência comercial que trata da compra e venda de mercadorias.}{mer.ce.o.lo.gi.a}{0}
\verb{merchandising}{}{}{}{}{s.m.}{Designação corrente da propaganda não declarada feita por intermédio da menção ou aparição de um produto, serviço ou marca durante um programa de televisão ou de rádio, filme, espetáculo teatral, entre outros.  }{\textit{merchandising}}{0}
\verb{mercurial}{}{}{}{}{adj.2g.}{Que se refere ao mercúrio.}{mer.cu.ri.al}{0}
\verb{mercurial}{}{}{}{}{}{Que contém mercúrio.}{mer.cu.ri.al}{0}
\verb{mercurial}{}{}{}{}{s.m.}{Medicamento em que há mercúrio.}{mer.cu.ri.al}{0}
\verb{mercúrico}{}{Quím.}{}{}{adj.}{Diz"-se dos compostos que contêm mercúrio divalente.}{mer.cú.ri.co}{0}
\verb{mercúrio}{}{Quím.}{}{}{s.m.}{Elemento químico metálico pesado, branco prateado, tóxico, único em estado líquido na temperatura ambiente, bom condutor de eletricidade, utilizado em vários instrumentos científicos (termômetros, barômetros etc.) e na indústria farmacêutica, de tintas e de explosivos. \elemento{80}{200.59}{Hg}.}{mer.cú.rio}{0}
\verb{mercúrio}{}{}{}{}{}{Medicamento em que entra essa substância.}{mer.cú.rio}{0}
\verb{Mercúrio}{}{Astron.}{}{}{s.m.}{Planeta do Sistema Solar mais próximo do Sol, e o oitavo maior, com relação ao tamanho.}{Mer.cú.rio}{0}
\verb{Mercúrio}{}{Mit.}{}{}{}{Deus da indústria e do comércio na religião romana, correspondente ao Hermes grego; considerado também o protetor dos viajantes e aventureiros.  }{Mer.cú.rio}{0}
\verb{mercuriocromo}{}{}{}{}{}{Var. de \textit{mercurocromo}.}{mer.cu.ri.o.cro.mo}{0}
\verb{mercurocromo}{}{Farm.}{}{}{s.m.}{Solução líquida vermelha com fluorescência amarelo"-esverdeada,  preparada com mercúrio, de aplicação tópica, usado como antisséptico e germicida; mercuriocromo.}{mer.cu.ro.cro.mo}{0}
\verb{merda}{é}{Chul.}{}{}{s.f.}{Matéria fecal; excrementos, fezes.}{mer.da}{0}
\verb{merda}{é}{}{}{}{}{Coisa sem valor, desprezível; porcaria.}{mer.da}{0}
\verb{merda}{é}{}{}{}{interj.}{Expressão que denota desprezo, indignação.}{mer.da}{0}
\verb{merecedor}{ô}{}{}{}{adj.}{Que merece, tem direito a; digno.}{me.re.ce.dor}{0}
\verb{merecer}{ê}{}{}{}{v.t.}{Fazer jus a; ter direito a; ser digno de.}{me.re.cer}{\verboinum{15}}
\verb{merecido}{}{}{}{}{adj.}{Que é devido, digno, justo.}{me.re.ci.do}{0}
\verb{merecimento}{}{}{}{}{s.m.}{Qualidade que torna alguém digno de estima, de recompensa, de apreço; mérito.}{me.re.ci.men.to}{0}
\verb{merencório}{}{}{}{}{adj.}{Que é dado à tristeza; melancólico, triste.}{me.ren.có.rio}{0}
\verb{merenda}{}{}{}{}{s.f.}{Lanche que as crianças levam à escola para comer durante o intervalo.}{me.ren.da}{0}
\verb{merendar}{}{}{}{}{v.i.}{Comer a merenda; lanchar.}{me.ren.dar}{\verboinum{1}}
\verb{merendeira}{ê}{}{}{}{s.f.}{Maleta ou bolsa para levar a merenda; lancheira.}{me.ren.dei.ra}{0}
\verb{merendeira}{ê}{}{}{}{}{Mulher que prepara ou distribui merendas na escola.}{me.ren.dei.ra}{0}
\verb{merengue}{}{}{}{}{s.m.}{Doce feito com claras de ovos e açúcar; suspiro.}{me.ren.gue}{0}
\verb{meretrício}{}{}{}{}{adj.}{Relativo a meretriz.}{me.re.trí.cio}{0}
\verb{meretrício}{}{}{}{}{s.m.}{Atividade de meretriz; prostituição.}{me.re.trí.cio}{0}
\verb{meretriz}{}{}{}{}{s.f.}{Mulher que pratica o ato sexual em troca de dinheiro; prostituta.}{me.re.triz}{0}
\verb{mergulhador}{ô}{}{}{}{adj.}{Que mergulha.}{mer.gu.lha.dor}{0}
\verb{mergulhador}{ô}{}{}{}{s.m.}{Indivíduo que pratica o mergulho por esporte ou profissão.}{mer.gu.lha.dor}{0}
\verb{mergulhão}{}{Zool.}{"-ões}{}{s.m.}{Nome comum às aves pelicaniformes, aquáticas, de distribuição mundial; corvo"-marinho, biguá.   }{mer.gu.lhão}{0}
\verb{mergulhar}{}{}{}{}{v.t.}{Fazer imergir parcial ou totalmente na água ou em outro líquido.}{mer.gu.lhar}{0}
\verb{mergulhar}{}{}{}{}{v.i.}{Afundar na água a ponto de ficar coberto por ela.}{mer.gu.lhar}{\verboinum{1}}
\verb{mergulho}{}{}{}{}{s.m.}{Ato ou efeito de mergulhar, de lançar à água.}{mer.gu.lho}{0}
\verb{mergulho}{}{}{}{}{}{Prática que consiste em submergir em águas rasas a fim de estudar a vida marinha.}{mer.gu.lho}{0}
\verb{meridiano}{}{Geogr.}{}{}{s.m.}{Círculo imaginário que liga um polo a outro passando pela linha do equador.}{me.ri.di.a.no}{0}
\verb{meridiano}{}{}{}{}{adj.}{Relativo ao meio"-dia.}{me.ri.di.a.no}{0}
\verb{meridional}{}{}{}{}{adj.2g.}{Que diz respeito a ou é próprio da região sul; austral.}{me.ri.di.o.nal}{0}
\verb{merino}{}{}{}{}{adj.}{Diz"-se de uma raça de carneiros que apresenta uma lã muito fina.}{me.ri.no}{0}
\verb{merinó}{}{}{}{}{}{Var. de \textit{merino}.}{me.ri.nó}{0}
\verb{meritíssimo}{}{}{}{}{adj.}{De grande mérito; digníssimo.}{me.ri.tís.si.mo}{0}
\verb{meritíssimo}{}{}{}{}{}{Tratamento dado aos juízes de direito.}{me.ri.tís.si.mo}{0}
\verb{mérito}{}{}{}{}{s.m.}{Aquilo que torna alguém digno de reconhecimento, de recompensa; merecimento, valor.}{mé.ri.to}{0}
\verb{meritório}{}{}{}{}{adj.}{Que é digno de louvor, de recompensa; merecedor.}{me.ri.tó.rio}{0}
\verb{merluza}{}{Zool.}{}{}{s.f.}{Peixe marinho de corpo alongado e boca ampla, parecido com a pescada.}{mer.lu.za}{0}
\verb{mero}{é}{}{}{}{adj.}{Sem mistura; puro, legítimo.}{me.ro}{0}
\verb{mero}{é}{Zool.}{}{}{s.m.}{Peixe marinho de corpo robusto e coloração castanha, parecido com a garoupa.}{me.ro}{0}
\verb{mero}{é}{}{}{}{}{Ordinário, comum, vulgar.}{me.ro}{0}
\verb{mês}{}{}{}{}{s.m.}{Cada uma das doze partes em que se divide o ano solar.}{mês}{0}
\verb{mês}{}{}{}{}{}{Espaço de trinta dias.}{mês}{0}
\verb{mesa}{ê}{}{}{}{s.f.}{Móvel, em geral de madeira, formado por uma superfície plana e lisa, fixada em suportes verticais, próprio para se sentar ao redor e realizar atividades várias, tais como comer, escrever e jogar.}{me.sa}{0}
\verb{mesa}{ê}{Fig.}{}{}{}{Alimentação, comida.}{me.sa}{0}
\verb{mesa}{ê}{}{}{}{}{Em uma assembleia, conjunto de presidente e secretários. (\textit{A reunião não começou enquanto não se compôs a mesa.})}{me.sa}{0}
\verb{mesada}{}{}{}{}{s.f.}{Quantia que se paga ou se recebe por mês.}{me.sa.da}{0}
\verb{mesa"-de"-cabeceira}{ê}{}{mesas"-de"-cabeceira}{}{s.f.}{Pequeno móvel que se põe ao lado da cabeceira da cama em que se colocam abajur, livros ou outros objetos que podem ser necessários durante a noite; criado"-mudo.}{me.sa"-de"-ca.be.cei.ra}{0}
\verb{mesa"-redonda}{}{}{mesas"-redondas}{}{s.f.}{Reunião de pessoas que debatem, em pé de igualdade, ideias sobre algum assunto.}{me.sa"-re.don.da}{0}
\verb{mesário}{}{}{}{}{s.m.}{Membro de mesa eleitoral que fiscaliza e dirige os trabalhos relativos a votação.}{me.sá.rio}{0}
\verb{mescla}{é}{}{}{}{s.f.}{Mistura de coisas ou pessoas diversas.}{mes.cla}{0}
\verb{mescla}{é}{}{}{}{}{Tecido composto de fios de diversas cores.}{mes.cla}{0}
\verb{mesclar}{}{}{}{}{v.t.}{Misturar coisas diferentes; amalgamar.}{mes.clar}{\verboinum{1}}
\verb{mesencéfalo}{}{Anat.}{}{}{s.m.}{Parte média do encéfalo.}{me.sen.cé.fa.lo}{0}
\verb{mesentério}{}{Anat.}{}{}{s.m.}{Membrana serosa que liga o intestino à parede posterior do abdômen. }{me.sen.té.rio}{0}
\verb{meseta}{ê}{}{}{}{s.f.}{Pequeno planalto.}{me.se.ta}{0}
\verb{mesmice}{}{}{}{}{s.f.}{Qualidade ou estado do que permanece inalterado; igualdade, identidade, uniformidade.}{mes.mi.ce}{0}
\verb{mesmice}{}{}{}{}{}{Falta de variedade; monotonia, marasmo. }{mes.mi.ce}{0}
\verb{mesmo}{ê}{}{}{}{adj.}{Que não é outro.}{mes.mo}{0}
\verb{mesmo}{ê}{}{}{}{}{Que não se distingue de outro; idêntico, igual. (\textit{O aluno apresenta sempre as mesmas desculpas por ter faltado à aula.})}{mes.mo}{0}
\verb{mesmo}{ê}{}{}{}{}{Em pessoa; próprio. (\textit{Eu mesma farei o serviço de limpeza pesada.})}{mes.mo}{0}
\verb{mesmo}{ê}{}{}{}{adv.}{De fato; realmente. (\textit{É mesmo uma situação difícil perder um amigo.})}{mes.mo}{0}
\verb{mesmo}{ê}{}{}{}{}{Apesar de alguma coisa; ainda. (\textit{Mesmo não querendo, ela teve de entrar na água.})}{mes.mo}{0}
\verb{mesmo}{ê}{}{}{}{s.m.}{Essa pessoa; ele. (\textit{As duas moças são as mesmas nos dois filmes.})}{mes.mo}{0}
\verb{mesocarpo}{}{}{}{}{s.m.}{Série inferior dos ossos do carpo.}{me.so.car.po}{0}
\verb{mesocarpo}{}{}{}{}{}{Parte média do pericarpo dos frutos; polpa.}{me.so.car.po}{0}
\verb{mesóclise}{}{Gram.}{}{}{s.f.}{Intercalação de pronome átono dentro do verbo.}{me.só.cli.se}{0}
\verb{mesofalange}{}{Anat.}{}{}{s.f.}{Falanginha.}{me.so.fa.lan.ge}{0}
\verb{mesologia}{}{}{}{}{s.f.}{Ecologia.}{me.so.lo.gi.a}{0}
\verb{méson}{}{Fís.}{}{}{s.m.}{Nome genérico das partículas subatômicas que se encontram no núcleo de um átomo, entre os prótons e elétrons.}{mé.son}{0}
\verb{mesozoico}{ó}{}{}{}{adj.}{Diz"-se de era geológica situada entre a paleozoica e a cenozoica, caracterizada pela preponderância de répteis, pelo aparecimento das aves e de algumas espécies de mamíferos.}{me.so.zoi.co}{0}
\verb{mesquinharia}{}{}{}{}{s.f.}{Mesquinhez.}{mes.qui.nha.ri.a}{0}
\verb{mesquinhez}{ê}{}{}{}{s.f.}{Qualidade de mesquinho; insignificância, pequenez, miudeza.}{mes.qui.nhez}{0}
\verb{mesquinhez}{ê}{}{}{}{}{Estreiteza, acanhamento.}{mes.qui.nhez}{0}
\verb{mesquinhez}{ê}{}{}{}{}{Usura, avareza.}{mes.qui.nhez}{0}
\verb{mesquinhez}{ê}{}{}{}{}{Desdita, infelicidade.}{mes.qui.nhez}{0}
\verb{mesquinho}{}{}{}{}{adj.}{Que é apegado em excesso aos bens materiais.}{mes.qui.nho}{0}
\verb{mesquinho}{}{}{}{}{}{Medíocre.}{mes.qui.nho}{0}
\verb{mesquinho}{}{}{}{}{}{Que não tem generosidade. }{mes.qui.nho}{0}
\verb{mesquita}{}{}{}{}{s.f.}{Templo sagrado dos mulçumanos.}{mes.qui.ta}{0}
\verb{messe}{é}{}{}{}{s.f.}{Cereal em bom estado de se ceifar.}{mes.se}{0}
\verb{messe}{é}{}{}{}{}{Colheita, ceifa.}{mes.se}{0}
\verb{messe}{é}{Fig.}{}{}{}{Aquisição, conquista.}{mes.se}{0}
\verb{messe}{é}{Relig.}{}{}{}{Conversão de almas.}{mes.se}{0}
\verb{messianismo}{}{}{}{}{s.m.}{Crença na vinda do Messias.}{mes.si.a.nis.mo}{0}
\verb{messias}{}{}{}{}{s.m.}{Para os judeus, aquele que vem para salvá"-los.}{mes.si.as}{0}
\verb{messias}{}{}{}{}{}{Para os cristãos, o próprio Cristo.}{mes.si.as}{0}
\verb{messias}{}{}{}{}{}{Alguém esperado como libertador, salvador.}{mes.si.as}{0}
\verb{mestiçagem}{}{}{"-ens}{}{s.f.}{Cruzamento entre pessoas ou animais de etnias ou raças diferentes.}{mes.ti.ça.gem}{0}
\verb{mestiçamento}{}{}{}{}{s.m.}{Miscigenação.}{mes.ti.ça.men.to}{0}
\verb{mestiçar}{}{}{}{}{v.t.}{Cruzar os indivíduos de uma etnia ou raça com os de outra, gerando mestiços; miscigenar.  }{mes.ti.çar}{\verboinum{3}}
\verb{mestiço}{}{}{}{}{adj.}{Nascido de pais de raças ou etnias diferentes.}{mes.ti.ço}{0}
\verb{mestra}{é}{}{}{}{s.f.}{Mulher que se dedica ao ensino; professora.}{mes.tra}{0}
\verb{mestra}{é}{}{}{}{}{Fato do qual se pode extrair ensinamento útil.}{mes.tra}{0}
\verb{mestrado}{}{}{}{}{s.m.}{Grau universitário obtido após o bacharelado.}{mes.tra.do}{0}
\verb{mestrado}{}{}{}{}{}{Curso de pós"-graduação que possibilita a obtenção desse grau.}{mes.tra.do}{0}
\verb{mestre}{é}{}{}{}{s.m.}{Indivíduo que ensina; professor.}{mes.tre}{0}
\verb{mestre}{é}{}{}{}{}{Indivíduo que sabe bem uma arte ou uma ciência; sábio.}{mes.tre}{0}
\verb{mestre}{é}{}{}{}{}{Indivíduo formado em curso superior que defendeu uma dissertação depois de estudos em curso de pós"-graduação.}{mes.tre}{0}
\verb{mestre"-cuca}{é}{Pop.}{mestres"-cucas ⟨é⟩}{}{s.m.}{Pessoa que inventa ou cozinha pratos de qualidade, com grande habilidade; cozinheiro.}{mes.tre"-cu.ca}{0}
\verb{mestre"-de"-armas}{é}{}{mestres"-de"-armas ⟨é⟩}{}{s.2g.}{Pessoa que ensina esgrima. }{mes.tre"-de"-ar.mas}{0}
\verb{mestre"-de"-cerimônias}{é}{}{mestres"-de"-cerimônias ⟨é⟩}{}{s.m.}{Oficial encarregado do cerimonial nas recepções de uma corte ou em outros atos solenes; mestre"-sala. }{mes.tre"-de"-ce.ri.mô.ni.as}{0}
\verb{mestre"-de"-obras}{é\ldots{}ó}{}{mestres"-de"-obras ⟨é\ldots{}ó⟩}{}{s.m.}{Trabalhador que, orientado pelo engenheiro e pelo arquiteto, dirige a execução de uma obra civil.  }{mes.tre"-de"-o.bras}{0}
\verb{mestre"-escola}{é\ldots{}ó}{Desus.}{mestres"-escolas ⟨é\ldots{}ó⟩}{}{s.m.}{Professor de instrução primária.}{mes.tre"-es.co.la}{0}
\verb{mestre"-sala}{é}{}{mestres"-salas ⟨é⟩}{}{s.m.}{Oficial encarregado do cerimonial nas recepções de uma corte, ou em outros atos solenes; mestre"-de"-cerimônia. }{mes.tre"-sa.la}{0}
\verb{mestre"-sala}{é}{}{mestres"-salas ⟨é⟩}{}{}{Indivíduo que acompanha a porta"-bandeira no desfile de uma escola de samba.}{mes.tre"-sa.la}{0}
\verb{mestria}{}{}{}{}{s.f.}{Grande saber ou habilidade.}{mes.tri.a}{0}
\verb{mesura}{}{}{}{}{s.f.}{Cumprimento cerimonioso; reverência.}{me.su.ra}{0}
\verb{mesurar}{}{}{}{}{v.t.}{Fazer mesuras a; cumprimentar, cortejar.}{me.su.rar}{0}
\verb{mesurar}{}{}{}{}{v.pron.}{Portar"-se moderadamente; comedir"-se, moderar"-se.}{me.su.rar}{\verboinum{1}}
\verb{mesureiro}{ê}{}{}{}{adj.}{Que é dado a fazer mesuras; atencioso, reverenciador, cerimonioso.}{me.su.rei.ro}{0}
\verb{mesureiro}{ê}{Pejor.}{}{}{}{Diz"-se daquele que, pelo excesso de polidez, mostra"-se servil, bajulador.}{me.su.rei.ro}{0}
\verb{meta}{é}{}{}{}{s.f.}{Sinal que marca o fim de uma corrida.}{me.ta}{0}
\verb{meta}{é}{}{}{}{}{Coisa que se quer alcançar com alguma atividade; alvo, objetivo.}{me.ta}{0}
\verb{meta}{é}{}{}{}{}{Conjunto de traves do campo de vários esportes; arco, baliza.}{me.ta}{0}
\verb{metabólico}{}{}{}{}{adj.}{Relativo a metabolismo.}{me.ta.bó.li.co}{0}
\verb{metabolismo}{}{Biol.}{}{}{s.m.}{Conjunto de fenômenos químicos do organismo.}{me.ta.bo.lis.mo}{0}
\verb{metacarpo}{}{Anat.}{}{}{s.m.}{A parte da mão entre o carpo e os dedos.}{me.ta.car.po}{0}
\verb{metade}{}{}{}{}{s.f.}{Cada uma das duas partes iguais em que se divide alguma coisa.}{me.ta.de}{0}
\verb{metafalange}{}{}{}{}{s.f.}{A terceira falange dos dedos, a das unhas; falangeta.}{me.ta.fa.lan.ge}{0}
\verb{metafísica}{}{Filos.}{}{}{s.f.}{Parte da filosofia que estuda o que está para além do mundo físico ou dos fenômenos.}{me.ta.fí.si.ca}{0}
\verb{metafísico}{}{}{}{}{adj.}{Relativo à metafísica.}{me.ta.fí.si.co}{0}
\verb{metafísico}{}{}{}{}{}{Transcendente.}{me.ta.fí.si.co}{0}
\verb{metáfora}{}{Gram.}{}{}{s.f.}{Figura de linguagem que consiste na transferência de uma palavra para um âmbito semântico que não é o do objeto que ela designa, e que se fundamenta numa relação de semelhança subentendida entre o sentido próprio e o figurado. }{me.tá.fo.ra}{0}
\verb{metafórico}{}{}{}{}{adj.}{Relativo a metáfora.}{me.ta.fó.ri.co}{0}
\verb{metafórico}{}{}{}{}{}{Que contém metáfora.}{me.ta.fó.ri.co}{0}
\verb{metal}{}{}{"-ais}{}{s.m.}{Elemento quase sempre duro e brilhante que, em geral, é bom condutor de calor e eletricidade.}{me.tal}{0}
\verb{metálico}{}{}{}{}{adj.}{Relativo a metal.}{me.tá.li.co}{0}
\verb{metalizar}{}{}{}{}{v.t.}{Tornar um metal puro.}{me.ta.li.zar}{0}
\verb{metalizar}{}{}{}{}{}{Dar o aspecto de metal a.}{me.ta.li.zar}{0}
\verb{metalizar}{}{}{}{}{}{Transformar em metal.}{me.ta.li.zar}{\verboinum{1}}
\verb{metalografia}{}{}{}{}{s.f.}{Estudo dos metais.}{me.ta.lo.gra.fi.a}{0}
\verb{metaloide}{}{Quím.}{}{}{adj.2g.}{Diz"-se de elemento químico que não tem as propriedades do metal, mas a ele se assemelha.}{me.ta.loi.de}{0}
\verb{metalurgia}{}{}{}{}{s.f.}{Técnica de extrair metais dos minerais.}{me.ta.lur.gi.a}{0}
\verb{metalurgia}{}{}{}{}{}{Arte de purificar e trabalhar os metais.}{me.ta.lur.gi.a}{0}
\verb{metalúrgica}{}{}{}{}{s.f.}{Oficina especializada em metalurgia.}{me.ta.lúr.gi.ca}{0}
\verb{metalúrgico}{}{}{}{}{adj.}{Relativo a metalurgia.}{me.ta.lúr.gi.co}{0}
\verb{metalúrgico}{}{}{}{}{}{Diz"-se de indivíduo que trabalha em metalurgia.}{me.ta.lúr.gi.co}{0}
\verb{metamorfose}{ó}{}{}{}{s.f.}{Mudança da forma ou da estrutura do corpo, que acontece na vida de certos animais.}{me.ta.mor.fo.se}{0}
\verb{metamorfosear}{}{}{}{}{v.t.}{Mudar a forma; transformar.}{me.ta.mor.fo.se.ar}{\verboinum{4}}
\verb{metano}{}{Quím.}{}{}{s.m.}{Hidrocarboneto usado em petroquímica e como combustível.}{me.ta.no}{0}
\verb{metanol}{ó}{Quím.}{"-óis}{}{s.m.}{Álcool usado como solvente e combustível.}{me.ta.nol}{0}
\verb{metaplasmo}{}{Gram.}{}{}{s.m.}{Nome comum a todas as alterações que acrescentam, suprimem, permutam ou transpõem fonemas nos vocábulos.}{me.ta.plas.mo}{0}
\verb{metapsíquica}{}{}{}{}{s.f.}{Estudo dos fenômenos psíquicos.}{me.tap.sí.qui.ca}{0}
\verb{metástase}{}{}{}{}{s.f.}{Alastramento no organismo de um fenômeno patológico já existente.}{me.tás.ta.se}{0}
\verb{metatarso}{}{Anat.}{}{}{s.m.}{Osso longo do pé.}{me.ta.tar.so}{0}
\verb{metátese}{}{Gram.}{}{}{s.f.}{Processo de mudança linguística caracterizado pela troca entre si dos lugares de dois fonemas de um mesmo vocábulo.}{me.tá.te.se}{0}
\verb{metazoário}{}{Biol.}{}{}{adj.}{Relativo aos metazoários, animais de corpo constituído por muitas células e tecidos com funções especializadas.}{me.ta.zo.á.rio}{0}
\verb{metazoário}{}{}{}{}{s.m.}{Espécime dos metazoários.}{me.ta.zo.á.rio}{0}
\verb{metediço}{}{}{}{}{adj.}{Intrometido.}{me.te.di.ço}{0}
\verb{metempsicose}{}{Relig.}{}{}{s.f.}{Teoria que acredita na migração das almas entre os corpos.}{me.temp.si.co.se}{0}
\verb{meteórico}{}{}{}{}{adj.}{Relativo a meteoro.}{me.te.ó.ri.co}{0}
\verb{meteórico}{}{Fig.}{}{}{}{De grande visibilidade, mas pouca duração. (\textit{Aquela modelo teve uma carreira meteórica.})}{me.te.ó.ri.co}{0}
\verb{meteorismo}{}{Med.}{}{}{s.m.}{Excesso de gases no estômago e nos intestinos.}{me.te.o.ris.mo}{0}
\verb{meteorito}{}{Astron.}{}{}{s.m.}{Corpo rochoso ou metálico que atravessa a atmosfera terrestre e cai na superfície do planeta.}{me.te.o.ri.to}{0}
\verb{meteoro}{ó}{}{}{}{s.m.}{Qualquer fenômeno que ocorre na atmosfera terrestre, como chuva, raio, trovão, arco"-íris.}{me.te.o.ro}{0}
\verb{meteoro}{ó}{}{}{}{}{Rastro luminoso causado na atmosfera pela passagem de um meteorito.}{me.te.o.ro}{0}
\verb{meteoro}{ó}{Fig.}{}{}{}{Qualquer pessoa ou fenômeno de brilho forte e curta duração.}{me.te.o.ro}{0}
\verb{meteorologia}{}{}{}{}{s.f.}{Ciência que estuda os fenômenos atmosféricos e cujas técnicas permitem fazer prognósticos a respeito das variações climáticas.}{me.te.o.ro.lo.gi.a}{0}
\verb{meteorológico}{}{}{}{}{adj.}{Relativo à meteorologia.}{me.te.o.ro.ló.gi.co}{0}
\verb{meteorologista}{}{}{}{}{s.2g.}{Especialista em meteorologia.}{me.te.o.ro.lo.gis.ta}{0}
\verb{meter}{ê}{}{}{}{v.t.}{Fazer algo ou alguém entrar em um lugar; colocar, enfiar, introduzir, pôr.}{me.ter}{0}
\verb{meter}{ê}{}{}{}{v.pron.}{Aventurar"-se a fazer algo.}{me.ter}{0}
\verb{meter}{ê}{}{}{}{}{Dar palpites sem permissão; intrometer"-se.}{me.ter}{\verboinum{12}}
\verb{meticulosidade}{}{}{}{}{s.f.}{Qualidade de meticuloso.}{me.ti.cu.lo.si.da.de}{0}
\verb{meticuloso}{ô}{}{"-osos ⟨ó⟩}{"-osa ⟨ó⟩}{adj.}{Cuidadoso, minucioso, cauteloso.}{me.ti.cu.lo.so}{0}
\verb{meticuloso}{ô}{}{"-osos ⟨ó⟩}{"-osa ⟨ó⟩}{}{Tímido, medroso.}{me.ti.cu.lo.so}{0}
\verb{metido}{}{}{}{}{adj.}{Que convive intimamente com algo; familiarizado.}{me.ti.do}{0}
\verb{metido}{}{}{}{}{}{Intrometido.}{me.ti.do}{0}
\verb{metílico}{}{Quím.}{}{}{adj.}{Relativo ao radical metila.}{me.tí.li.co}{0}
\verb{metódico}{}{}{}{}{adj.}{Em que há método, técnica apurada.}{me.tó.di.co}{0}
\verb{metodismo}{}{Relig.}{}{}{s.m.}{Movimento religioso protestante e evangélico, fundado no século \textsc{xviii} na Inglaterra por John Wesley.}{me.to.dis.mo}{0}
\verb{metodismo}{}{}{}{}{}{Qualidade de metódico.}{me.to.dis.mo}{0}
\verb{metodismo}{}{}{}{}{}{Procedimento metódico.}{me.to.dis.mo}{0}
\verb{metodista}{}{}{}{}{adj.2g.}{Relativo ao metodismo.}{me.to.dis.ta}{0}
\verb{metodista}{}{}{}{}{s.2g.}{Indivíduo adepto do metodismo.}{me.to.dis.ta}{0}
\verb{metodizar}{}{}{}{}{v.t.}{Tornar metódico; ordenar.}{me.to.di.zar}{\verboinum{1}}
\verb{método}{}{}{}{}{s.m.}{Procedimento para se atingir um objetivo.}{mé.to.do}{0}
\verb{método}{}{}{}{}{}{Conjunto de regras que regulam ou orientam a prática de algo.}{mé.to.do}{0}
\verb{método}{}{}{}{}{}{Maneira particular de se comportar.}{mé.to.do}{0}
\verb{metodologia}{}{Filos.}{}{}{s.f.}{Estudo dos métodos das ciências.}{me.to.do.lo.gi.a}{0}
\verb{metodologia}{}{}{}{}{}{Conjunto dos critérios utilizados na realização de uma pesquisa.}{me.to.do.lo.gi.a}{0}
\verb{metodológico}{}{}{}{}{adj.}{Relativo a metodologia.}{me.to.do.ló.gi.co}{0}
\verb{metonímia}{}{}{}{}{s.f.}{Figura retórica que permite o uso de uma palavra no lugar de outra, de forma que seus referentes estejam relacionados por contiguidade e não por similaridade. (\textit{Gosto de ler Machado de Assis} por \textit{Gosto de ler a obra de Machado de Assis.})}{me.to.ní.mia}{0}
\verb{metragem}{}{}{"-ens}{}{s.f.}{Medição em metros.}{me.tra.gem}{0}
\verb{metragem}{}{}{"-ens}{}{}{O comprimento de um filme.}{me.tra.gem}{0}
\verb{metralha}{}{}{}{}{s.f.}{Fragmentos de metal com que se carregam projéteis ocos.}{me.tra.lha}{0}
\verb{metralha}{}{}{}{}{}{Tiro ou conjunto de tiros de metralha.}{me.tra.lha}{0}
\verb{metralha}{}{Fig.}{}{}{}{Grande porção de algo.}{me.tra.lha}{0}
\verb{metralha}{}{}{}{}{}{Fragmentos de tijolo.}{me.tra.lha}{0}
\verb{metralhadora}{ô}{}{}{}{s.f.}{Arma de fogo de repetição, que dispara grande quantidade de projéteis em pouco tempo.}{me.tra.lha.do.ra}{0}
\verb{metralhar}{}{}{}{}{v.t.}{Dar tiros de metralha ou metralhadora.}{me.tra.lhar}{\verboinum{1}}
\verb{métrica}{}{}{}{}{s.f.}{Conjunto de regras que orientam a construção rítmica de versos.}{mé.tri.ca}{0}
\verb{métrica}{}{}{}{}{}{As características métricas de um poema, de uma música ou das obras de um escritor.}{mé.tri.ca}{0}
\verb{métrico}{}{}{}{}{adj.}{Relativo ao metro.}{mé.tri.co}{0}
\verb{métrico}{}{}{}{}{}{Relativo a metrificação.}{mé.tri.co}{0}
\verb{metrificação}{}{}{"-ões}{}{s.f.}{Ato, efeito ou técnica de metrificar.}{me.tri.fi.ca.ção}{0}
\verb{metrificar}{}{}{}{}{v.t.}{Compor versos tendo em vista o seu tamanho e ritmo.}{me.tri.fi.car}{\verboinum{2}}
\verb{metrite}{}{Med.}{}{}{s.f.}{Inflamação do útero.}{me.tri.te}{0}
\verb{metro}{é}{}{}{}{s.m.}{Unidade fundamental de medida de comprimento do Sistema Internacional de Medidas.}{me.tro}{0}
\verb{metro}{é}{}{}{}{}{Qualquer objeto que tenha marcações da medida de um metro, utilizado para fazer medições.}{me.tro}{0}
\verb{metro}{é}{}{}{}{}{Forma rítmica de uma obra poética.}{me.tro}{0}
\verb{metrô}{}{}{}{}{s.m.}{Sistema de transporte urbano composto por trens de tração elétrica de grande velocidade e frequência, e que circulam em vias geralmente subterrâneas.}{me.trô}{0}
\verb{metrologia}{}{}{}{}{s.f.}{Estudo dos sistemas de pesos e medidas utilizados pelos diferentes povos e nas diversas épocas.}{me.tro.lo.gi.a}{0}
\verb{metrônomo}{}{Mús.}{}{}{s.m.}{Instrumento que emite estalos em intervalos regulares, usado para regular os andamentos musicais.}{me.trô.no.mo}{0}
\verb{metrópole}{}{}{}{}{s.f.}{Cidade de maior tamanho ou importância em determinada região.}{me.tró.po.le}{0}
\verb{metrópole}{}{}{}{}{}{A nação dominante em relação às suas colônias.}{me.tró.po.le}{0}
\verb{metropolitano}{}{}{}{}{adj.}{Relativo a metrópole.}{me.tro.po.li.ta.no}{0}
\verb{metropolitano}{}{}{}{}{}{Diz"-se da região que compreende uma metrópole e as cidades dela dependentes.}{me.tro.po.li.ta.no}{0}
\verb{metroviário}{}{}{}{}{adj.}{Relativo a metrô.}{me.tro.vi.á.rio}{0}
\verb{metroviário}{}{}{}{}{s.m.}{Funcionário da empresa que opera o metrô.}{me.tro.vi.á.rio}{0}
\verb{meu}{}{Gram.}{}{}{pron.}{Pronome possessivo que se refere à pessoa que fala.}{meu}{0}
\verb{meu}{}{Gram.}{}{}{s.m.}{Aquilo que pertence à pessoa que fala.}{meu}{0}
\verb{mexediço}{ch}{}{}{}{adj.}{Que se mexe muito ou que tende a se mexer; inquieto, movediço.}{me.xe.di.ço}{0}
\verb{mexer}{ch}{}{}{}{v.t.}{Dar movimento a; mover, movimentar.}{me.xer}{0}
\verb{mexer}{ch}{}{}{}{}{Agitar chacoalhando ou com auxílio de um utensílio. (\textit{Mexer a sopa.})}{me.xer}{0}
\verb{mexer}{ch}{}{}{}{}{Tirar alguma coisa do lugar ou da posição em que se encontrava.}{me.xer}{0}
\verb{mexer}{ch}{}{}{}{}{Abordar ou dirigir gracejos ou gozações. (\textit{O garoto mexeu com a velhinha que passava na rua.})}{me.xer}{0}
\verb{mexer}{ch}{}{}{}{v.pron.}{Mover"-se, agitar"-se.}{me.xer}{\verboinum{12}}
\verb{mexerica}{ch}{Bras.}{}{}{s.f.}{Tangerina.}{me.xe.ri.ca}{0}
\verb{mexericar}{ch}{}{}{}{v.t.}{Tratar de coisas secretamente para produzir intriga.}{me.xe.ri.car}{\verboinum{2}}
\verb{mexerico}{ch}{}{}{}{s.m.}{Ato ou efeito de mexericar; intriga, fofoca.}{me.xe.ri.co}{0}
\verb{mexeriqueira}{ch}{Bot.}{}{}{s.f.}{Árvore que dá a tangerina; tangerineira.}{me.xe.ri.quei.ra}{0}
\verb{mexeriqueiro}{ch}{}{}{}{adj.}{Que costuma mexericar, fazer mexericos.}{me.xe.ri.quei.ro}{0}
\verb{mexicano}{ch}{}{}{}{adj.}{Relativo ao México.}{me.xi.ca.no}{0}
\verb{mexicano}{ch}{}{}{}{s.m.}{Indivíduo natural ou habitante desse país.}{me.xi.ca.no}{0}
\verb{mexida}{ch}{}{}{}{s.f.}{Desordem, confusão, discórdia.}{me.xi.da}{0}
\verb{mexido}{ch}{}{}{}{adj.}{Que se mexeu ou costuma mexer"-se.}{me.xi.do}{0}
\verb{mexido}{ch}{Cul.}{}{}{s.m.}{Prato preparado com farofa, ovos, sobras de comida, linguiça e torresmo.}{me.xi.do}{0}
\verb{mexilhão}{ch}{Zool.}{"-ões}{}{s.m.}{Designação comum aos moluscos bivalves.}{me.xi.lhão}{0}
\verb{mezanino}{}{}{}{}{s.m.}{Andar intermediário entre dois pavimentos, cujo acesso se dá por meio de uma escada.}{me.za.ni.no}{0}
\verb{mezinha}{}{}{}{}{s.f.}{Líquido medicamentoso utilizado para lavagem intestinal.}{me.zi.nha}{0}
\verb{mezinha}{}{Pop.}{}{}{}{Medicamento caseiro.}{me.zi.nha}{0}
\verb{mg}{}{}{}{}{}{Abrev. de \textit{miligrama}.}{mg}{0}
\verb{Mg}{}{Quím.}{}{}{}{Símb. do \textit{magnésio}.}{Mg}{0}
\verb{MG}{}{}{}{}{}{Sigla do estado de Minas Gerais.}{MG}{0}
\verb{MHz}{}{}{}{}{}{Símb. de \textit{mega"-hertz}.}{MHz}{0}
\verb{mi}{}{}{}{}{s.m.}{Décima segunda letra do alfabeto grego.}{mi}{0}
\verb{mi}{}{}{}{}{s.m.}{A terceira nota musical na escala de \textit{dó}.}{mi}{0}
\verb{miada}{}{}{}{}{s.f.}{O som de muitos gatos miando ao mesmo tempo.}{mi.a.da}{0}
\verb{miada}{}{}{}{}{}{Miado.}{mi.a.da}{0}
\verb{miado}{}{}{}{}{s.m.}{A voz do gato.}{mi.a.do}{0}
\verb{miar}{}{}{}{}{v.i.}{Dar miados.}{mi.ar}{0}
\verb{miar}{}{Bras.}{}{}{}{Choramingar.}{mi.ar}{\verboinum{1}}
\verb{miasma}{}{}{}{}{s.m.}{O cheiro e os gases que emanam de animais ou vegetais em decomposição.}{mi.as.ma}{0}
\verb{miasmático}{}{}{}{}{adj.}{Que produz miasma.}{mi.as.má.ti.co}{0}
\verb{miau}{}{Pop.}{}{}{s.m.}{A voz do gato.}{mi.au}{0}
\verb{mica}{}{}{}{}{s.f.}{Nome comum a um grande número de minerais que se fendem em lâminas delgadas; malacacheta. }{mi.ca}{0}
\verb{micado}{}{}{}{}{s.m.}{Título do imperador no Japão.}{mi.ca.do}{0}
\verb{micagem}{}{}{"-ens}{}{s.f.}{Gesto ou trejeito próprio de mico.}{mi.ca.gem}{0}
\verb{miçanga}{}{}{}{}{s.f.}{Conta de vidro, geralmente de tamanho bem pequeno.}{mi.çan.ga}{0}
\verb{miçanga}{}{}{}{}{}{Coisa de pouco valor; bugiganga.}{mi.çan.ga}{0}
\verb{micção}{}{Med.}{"-ões}{}{s.f.}{Ato de urinar.}{mic.ção}{0}
\verb{michê}{}{Bras.}{}{}{s.m.}{Ato de prostituir"-se.}{mi.chê}{0}
\verb{michê}{}{}{}{}{}{Quantia paga ao prostituto ou prostituta pelo serviço.}{mi.chê}{0}
\verb{michê}{}{Por ext.}{}{}{}{Indivíduo que se prostitui.}{mi.chê}{0}
\verb{mico}{}{Zool.}{}{}{s.m.}{Designação comum dada a alguns macacos de porte médio; sagui, macaco"-prego.}{mi.co}{0}
\verb{mico}{}{Pop.}{}{}{}{Situação embaraçosa ou desagradável.}{mi.co}{0}
\verb{mico"-leão}{}{Zool.}{micos"-leões \textit{ou} micos"-leão}{}{s.m.}{Nome comum dado aos pequenos macacos ou saguis da América tropical, que se alimentam de frutos e estão ameaçados de extinção.}{mi.co"-le.ão}{0}
\verb{mico"-leão"-dourado}{}{Zool.}{micos"-leões"-dourados \textit{ou}  micos"-leão"-dourados}{}{s.m.}{Variedade de mico"-leão.}{mi.co"-le.ão"-dou.ra.do}{0}
\verb{micologia}{}{Biol.}{}{}{s.f.}{Ramo da biologia que estuda os fungos.}{mi.co.lo.gi.a}{0}
\verb{micose}{ó}{Med.}{}{}{s.m.}{Doença de pele causada por fungo.}{mi.co.se}{0}
\verb{micreiro}{ê}{Bras.}{}{}{adj.}{Relativo a microcomputador.}{mi.crei.ro}{0}
\verb{micreiro}{ê}{}{}{}{s.m.}{Usuário aficionado de computadores.}{mi.crei.ro}{0}
\verb{micro}{}{}{}{}{s.m.}{Redução de \textit{microcomputador}.}{mi.cro}{0}
\verb{micro}{}{}{}{}{}{Denominação do Sistema Internacional de Medidas equivalente à milionésima parte de qualquer unidade.}{mi.cro}{0}
\verb{microbiano}{}{}{}{}{adj.}{Relativo a micróbio.}{mi.cro.bi.a.no}{0}
\verb{microbiano}{}{}{}{}{}{Causado por micróbio.}{mi.cro.bi.a.no}{0}
\verb{micróbio}{}{Biol.}{}{}{s.m.}{Designação comum de qualquer ser vivo de tamanho microscópico, como fungos, bactérias e protozoários.}{mi.cró.bio}{0}
\verb{microbiologia}{}{Biol.}{}{}{s.f.}{Ramo das ciências biomédicas que estuda os micro"-organismos, em especial os causadores de doenças.}{mi.cro.bi.o.lo.gi.a}{0}
\verb{microbiologista}{}{}{}{}{s.2g.}{Indivíduo especialista em microbiologia.}{mi.cro.bi.o.lo.gis.ta}{0}
\verb{microcefalia}{}{Med.}{}{}{s.f.}{Anormalidade caracterizada pela pequenez anormal da cabeça e, geralmente, pela deficiência mental.}{mi.cro.ce.fa.li.a}{0}
\verb{microcirurgia}{}{Med.}{}{}{s.f.}{Intervenção cirúrgica realizada com auxílio de microscópio binocular.}{mi.cro.ci.rur.gi.a}{0}
\verb{microcomputador}{ô}{Informát.}{}{}{s.m.}{Computador em que o processamento dos dados e instruções é realizado por um microprocessador.}{mi.cro.com.pu.ta.dor}{0}
\verb{microcosmo}{ó}{}{}{}{s.m.}{Qualquer coisa que possa ser considerada, por analogia, um universo em miniatura.}{mi.cro.cos.mo}{0}
\verb{microeconomia}{}{Econ.}{}{}{s.f.}{Ramo da economia que estuda o funcionamento e o comportamento dos agentes econômicos individuais (empresas, consumidores).}{mi.cro.e.co.no.mi.a}{0}
\verb{microempresa}{ê}{Econ.}{}{}{s.f.}{Empresa cuja receita anual é inferior a um valor estabelecido pelo governo, e por isso obtém isenção de alguns impostos.}{mi.cro.em.pre.sa}{0}
\verb{microempresário}{}{}{}{}{s.m.}{Proprietário de microempresa.}{mi.cro.em.pre.sá.rio}{0}
\verb{microfilmar}{}{}{}{}{v.t.}{Fotografar (um documento) para transformá"-lo em microfilme.}{mi.cro.fil.mar}{\verboinum{1}}
\verb{microfilme}{}{}{}{}{s.m.}{Reprodução bastante reduzida de um documento.}{mi.cro.fil.me}{0}
\verb{microfilme}{}{}{}{}{}{A película fotossensível utilizada nesse processo.}{mi.cro.fil.me}{0}
\verb{microflora}{ó}{Biol.}{}{}{s.f.}{Flora composta de micro"-organismos.}{mi.cro.flo.ra}{0}
\verb{microfone}{}{}{}{}{s.m.}{Aparelho que converte ondas sonoras em sinais elétricos.}{mi.cro.fo.ne}{0}
\verb{microfonia}{}{}{}{}{s.f.}{Ruído em um sistema de som causado pela realimentação, ou seja, o som amplificado retornando ao microfone e sendo amplificado novamente.}{mi.cro.fo.ni.a}{0}
\verb{microfotografia}{}{}{}{}{s.f.}{Técnica e processo de obtenção de reproduções fotográficas de tamanho bastante reduzido, como microfilmes.}{mi.cro.fo.to.gra.fi.a}{0}
\verb{micrômetro}{}{}{}{}{s.m.}{Unidade de medida de comprimento equivalente à milionésima parte do metro.}{mi.crô.me.tro}{0}
\verb{mícron}{}{Desus.}{}{}{s.m.}{Micrômetro.}{mí.cron}{0}
\verb{micronésio}{}{}{}{}{adj.}{Relativo à Micronésia, um dos três grandes agrupamentos das ilhas do Pacífico equatorial.}{mi.cro.né.sio}{0}
\verb{micronésio}{}{}{}{}{s.m.}{Indivíduo natural ou habitante desse local.}{mi.cro.né.sio}{0}
\verb{micronésio}{}{}{}{}{}{Um dos subgrupos do idioma malaio"-polinésio falado nesse local.}{mi.cro.né.sio}{0}
\verb{micro"-onda}{}{Fís.}{}{}{s.f.}{Radiação eletromagnética de frequência muito alta, geralmente superior a 300 MHz.}{mi.cro"-on.da}{0}
\verb{micro"-ondas}{}{}{}{}{s.m.}{Redução de \textit{forno de micro"-ondas}, aparelho que utiliza radiação eletromagnética de alta frequência para produzir calor.}{mi.cro"-on.das}{0}
\verb{micro"-ônibus}{}{}{}{}{s.m.}{Veículo de transporte coletivo de tamanho e capacidade menor que um ônibus.}{mi.cro"-ô.ni.bus}{0}
\verb{micro"-organismo}{}{Biol.}{}{}{s.m.}{Designação comum a organismos microscópicos, como bactérias, vírus, fungos e protozoários; microrganismo.}{mi.cro"-or.ga.nis.mo}{0}
\verb{microprocessador}{ô}{Informát.}{}{}{s.m.}{Circuito integrado dotado de unidade lógica capaz de executar um número determinado de instruções e utilizado como unidade central de processamento em microcomputadores.}{mi.cro.pro.ces.sa.dor}{0}
\verb{microrganismo}{}{Biol.}{}{}{s.m.}{Micro"-organismo.}{mi.cror.ga.nis.mo}{0}
\verb{microscópico}{}{}{}{}{adj.}{Relativo a microscópio ou a microscopia.}{mi.cros.có.pi.co}{0}
\verb{microscópico}{}{}{}{}{}{Visível somente com auxílio de um microscópico.}{mi.cros.có.pi.co}{0}
\verb{microscópico}{}{Por ext.}{}{}{}{Muito pequeno; minúsculo.}{mi.cros.có.pi.co}{0}
\verb{microscópio}{}{Fís.}{}{}{s.m.}{Sistema óptico que produz uma imagem ampliada dos objetos e é usado para observar coisas e organismos de dimensões bastante pequenas.}{mi.cros.có.pio}{0}
\verb{microssulco}{}{}{}{}{s.m.}{Ranhura em forma de \textsc{v}, muito fina e compacta, na qual se registra a trilha sonora em um \textit{long"-play}.  }{mi.cros.sul.co}{0}
\verb{micrótomo}{}{}{}{}{s.m.}{Instrumento para fazer cortes microscópicos nos tecidos; histótomo.}{mi.cró.to.mo}{0}
\verb{microzoário}{}{Zool.}{}{}{s.m.}{Animal de dimensões muito reduzidas, visível somente através de microscópio.}{mi.cro.zo.á.rio}{0}
\verb{mictório}{}{}{}{}{adj.}{Que provoca micção.}{mic.tó.rio}{0}
\verb{mictório}{}{}{}{}{s.m.}{Local próprio para urinar; banheiro.}{mic.tó.rio}{0}
\verb{mictório}{}{}{}{}{}{Peça sanitária que se presta somente para urinar.}{mic.tó.rio}{0}
\verb{micuim}{}{Zool.}{"-ins}{}{s.m.}{Ácaro parasita da pele dos vertebrados que provoca intensa coceira.}{mi.cu.im}{0}
\verb{mídia}{}{}{}{}{s.m.}{Qualquer sistema capaz de transmitir mensagens, especialmente em larga escala, como rádio, televisão, jornal, computador, telefone, videocassete.}{mí.dia}{0}
\verb{mídia}{}{}{}{}{}{O conjunto dos meios de comunicação, especialmente aqueles utilizados para veiculação de notícias.}{mí.dia}{0}
\verb{mídia}{}{}{}{}{}{Suporte físico utilizado para registro de informações, como fita magnética, disco digital, jornal, película fotográfica.}{mí.dia}{0}
\verb{migalha}{}{}{}{}{s.m.}{Fragmento de alimentos à base de farinha (pão, bolo, bolacha).}{mi.ga.lha}{0}
\verb{migalha}{}{Fig.}{}{}{}{Quantidade insignificante de qualquer coisa.}{mi.ga.lha}{0}
\verb{migração}{}{}{"-ões}{}{s.f.}{Deslocamento de populações (pessoas ou animais) entre regiões diferentes, geralmente em busca de melhores condições de vida.}{mi.gra.ção}{0}
\verb{migrante}{}{}{}{}{adj.2g.}{Que migra.}{mi.gran.te}{0}
\verb{migrar}{}{}{}{}{v.t.}{Deslocar"-se para outra região, em caráter temporário ou permanente.}{mi.grar}{\verboinum{1}}
\verb{migratório}{}{}{}{}{adj.}{Relativo a migração.}{mi.gra.tó.rio}{0}
\verb{miiologia}{}{}{}{}{s.f.}{Ramo da biologia que estuda as moscas.}{mi.i.o.lo.gi.a}{0}
\verb{mijada}{}{}{}{}{s.f.}{Ato ou efeito de mijar.}{mi.ja.da}{0}
\verb{mijada}{}{Pop.}{}{}{}{Repreensão severa; bronca.}{mi.ja.da}{0}
\verb{mijar}{}{}{}{}{v.t.}{Urinar.}{mi.jar}{0}
\verb{mijar}{}{Pop.}{}{}{}{Repreender severamente.}{mi.jar}{0}
\verb{mijar}{}{Pop.}{}{}{v.pron.}{Mostrar"-se com medo.}{mi.jar}{\verboinum{1}}
\verb{mijo}{}{Pop.}{}{}{s.m.}{Urina.}{mi.jo}{0}
\verb{mil}{}{}{}{}{num.}{Nome dado à quantidade expressa pelo número 1.000.}{mil}{0}
\verb{milagre}{}{}{}{}{s.m.}{Evento fora do comum e que não pode ser explicado com base no conhecimento científico disponível.}{mi.la.gre}{0}
\verb{milagre}{}{}{}{}{}{Acontecimento surpreendente, admirável, formidável.}{mi.la.gre}{0}
\verb{milagre}{}{Relig.}{}{}{}{Qualquer evidência de intervenção divina na vida dos homens.}{mi.la.gre}{0}
\verb{milagreiro}{ê}{}{}{}{adj.}{Que pratica milagres ou é tido como tal.}{mi.la.grei.ro}{0}
\verb{milagroso}{ô}{}{"-osos ⟨ó⟩}{"-osa ⟨ó⟩}{adj.}{Em que há milagre; extraordinário.}{mi.la.gro.so}{0}
\verb{milagroso}{ô}{}{"-osos ⟨ó⟩}{"-osa ⟨ó⟩}{}{Milagreiro.}{mi.la.gro.so}{0}
\verb{milanês}{}{}{}{}{adj.}{Relativo a Milão, cidade italiana.}{mi.la.nês}{0}
\verb{milanês}{}{}{}{}{s.m.}{Indivíduo natural ou habitante dessa cidade.}{mi.la.nês}{0}
\verb{milenar}{}{}{}{}{adj.2g.}{Que tem um milênio ou mais.}{mi.le.nar}{0}
\verb{milenar}{}{}{}{}{}{Muito antigo.}{mi.le.nar}{0}
\verb{milenário}{}{}{}{}{adj.}{Relativo a mil ou a milhar.}{mi.le.ná.rio}{0}
\verb{milenário}{}{}{}{}{}{Milenar.}{mi.le.ná.rio}{0}
\verb{milenário}{}{}{}{}{s.m.}{Período de mil anos.}{mi.le.ná.rio}{0}
\verb{milênio}{}{}{}{}{s.m.}{Período de mil anos.}{mi.lê.nio}{0}
\verb{milésimo}{}{}{}{}{num.}{Ordinal ou fracionário correspondente ao número 1000.}{mi.lé.si.mo}{0}
\verb{mil"-folhas}{ô}{Cul.}{}{}{s.m.}{Doce feito com diversas camadas finas de massa recheadas de creme.}{mil"-fo.lhas}{0}
\verb{milha}{}{}{}{}{s.f.}{Unidade de medida de distância equivalente a 1.609 metros.}{mi.lha}{0}
\verb{milhafre}{}{}{}{}{s.m.}{Designação comum a diversas espécies de gavião.}{mi.lha.fre}{0}
\verb{milhafre}{}{Pop.}{}{}{}{Ladrão, gatuno.}{mi.lha.fre}{0}
\verb{milhagem}{}{}{"-ens}{}{s.f.}{Distância, percorrida ou a percorrer, calculada em milhas.}{mi.lha.gem}{0}
\verb{milhão}{}{}{"-ões}{}{s.m.}{Quantidade equivalente a mil milhares, ou mil vezes mil.}{mi.lhão}{0}
\verb{milhar}{}{}{}{}{s.m.}{Conjunto de mil unidades.}{mi.lhar}{0}
\verb{milhar}{}{Bras.}{}{}{}{Qualquer número de quatro algarismos nas loterias.}{mi.lhar}{0}
\verb{milharal}{}{}{"-ais}{}{s.m.}{Aglomerado de pés de milho.}{mi.lha.ral}{0}
\verb{milheiro}{ê}{}{}{}{s.m.}{Milho.}{mi.lhei.ro}{0}
\verb{milheiro}{ê}{}{}{}{}{Certo tipo de uva escura.}{mi.lhei.ro}{0}
\verb{milho}{}{Bot.}{}{}{s.m.}{Planta que dá espigas com grãos nutritivos utilizados na alimentação humana, na criação de animais e na indústria alimentícia. }{mi.lho}{0}
\verb{milho}{}{}{}{}{}{A espiga com grãos dessa planta.}{mi.lho}{0}
\verb{miliário}{}{}{}{}{adj.}{Relativo a milha.}{mi.li.á.rio}{0}
\verb{miliário}{}{}{}{}{}{Diz"-se dos marcos que indicam distâncias nas estradas.}{mi.li.á.rio}{0}
\verb{miliário}{}{Fig.}{}{}{}{Que assinala data ou evento memorável.}{mi.li.á.rio}{0}
\verb{milícia}{}{}{}{}{s.f.}{O conjunto das técnicas de guerra.}{mi.lí.cia}{0}
\verb{milícia}{}{}{}{}{}{Conjunto das tropas de um país; exército.}{mi.lí.cia}{0}
\verb{milícia}{}{}{}{}{}{Conjunto de cidadãos armados e que não fazem parte do exército e, especialmente, aqueles ligados a organizações religiosas ou políticas.}{mi.lí.cia}{0}
\verb{miliciano}{}{}{}{}{adj.}{Relativo a milícia.}{mi.li.ci.a.no}{0}
\verb{miliciano}{}{}{}{}{}{Que faz parte de uma milícia.}{mi.li.ci.a.no}{0}
\verb{milico}{}{Bras.}{}{}{s.m.}{Membro das Forças Armadas; militar.}{mi.li.co}{0}
\verb{miligrama}{}{}{}{}{s.m.}{Unidade de medida de massa equivalente a um milésimo de um grama. Símb.: mg.}{mi.li.gra.ma}{0}
\verb{mililitro}{}{}{}{}{s.m.}{Unidade de medida de capacidade equivalente à milésima parte de um litro. Símb.: ml.}{mi.li.li.tro}{0}
\verb{milímetro}{}{}{}{}{s.m.}{Unidade de medida de comprimento equivalente à milésima parte de um metro. Símb.: mm.}{mi.lí.me.tro}{0}
\verb{milionário}{}{}{}{}{adj.}{Diz"-se de homem que possui milhões em patrimônio ou que possui uma grande fortuna.}{mi.li.o.ná.rio}{0}
\verb{milionário}{}{}{}{}{}{Diz"-se de quantia, orçamento, dotação que envolve milhões em valores.}{mi.li.o.ná.rio}{0}
\verb{milionésimo}{}{}{}{}{num.}{Ordinal ou fracionário correspondente a um milhão.}{mi.li.o.né.si.mo}{0}
\verb{militância}{}{}{}{}{s.f.}{Atividade, atitude ou atuação de militante.}{mi.li.tân.cia}{0}
\verb{militante}{}{}{}{}{adj.2g.}{Que milita, que atua politicamente em defesa de uma causa.}{mi.li.tan.te}{0}
\verb{militar}{}{}{}{}{adj.2g.}{Relativo às Forças Armadas, ou a guerra.}{mi.li.tar}{0}
\verb{militar}{}{}{}{}{v.i.}{Atuar politicamente em defesa de uma causa.}{mi.li.tar}{\verboinum{1}}
\verb{militar}{}{}{}{}{s.2g.}{Indivíduo membro das Forças Armadas.}{mi.li.tar}{0}
\verb{militarismo}{}{}{}{}{s.m.}{Tendência ideológica de considerar as intervenções militares e a guerra como solução para os problemas sociais e políticos.}{mi.li.ta.ris.mo}{0}
\verb{militarismo}{}{}{}{}{}{Sistema político muito sujeito às influências dos militares.}{mi.li.ta.ris.mo}{0}
\verb{militarizar}{}{}{}{}{v.t.}{Dar caráter ou organização militar a.}{mi.li.ta.ri.zar}{0}
\verb{milk"-shake}{}{Cul.}{}{}{s.m.}{Bebida à base de leite misturado com sorvete e aromatizante. }{\textit{milk"-shake}}{0}
\verb{milonga}{}{Mús.}{}{}{s.f.}{Tipo de música lamentosa, acompanhada de violão, encontrada no sul do Brasil.}{mi.lon.ga}{0}
\verb{milonga}{}{Bras.}{}{}{}{Habilidade de ludibriar; astúcia. (Usa"-se no plural nesta acepção.)}{mi.lon.ga}{0}
\verb{milonga}{}{Bras.}{}{}{}{Boato, intriga, mexerico. (Usa"-se no plural nesta acepção.)}{mi.lon.ga}{0}
\verb{mil"-réis}{}{Econ.}{}{}{s.m.}{Antiga unidade monetária brasileira.}{mil"-réis}{0}
\verb{mim}{}{Gram.}{}{}{pron.}{Forma oblíqua tônica do pronome pessoal de primeira pessoa singular, sempre regida de preposição.}{mim}{0}
\verb{mimar}{}{}{}{}{v.t.}{Dar carinho a.}{mi.mar}{0}
\verb{mimar}{}{}{}{}{}{Dar (a alguém) atenção ou cuidados em excesso.}{mi.mar}{\verboinum{1}}
\verb{mimeografar}{}{}{}{}{v.t.}{Imprimir usando mimeógrafo.}{mi.me.o.gra.far}{\verboinum{1}}
\verb{mimeógrafo}{}{}{}{}{s.m.}{Máquina de impressão que utiliza uma matriz de estêncil em volta de um cilindro giratório.}{mi.me.ó.gra.fo}{0}
\verb{mimetismo}{}{Biol.}{}{}{s.m.}{Propriedade de um organismo de adquirir características visuais que fazem com que ele seja confundido com um indivíduo de outra espécie ou com o meio em que vive.}{mi.me.tis.mo}{0}
\verb{mímica}{}{}{}{}{s.f.}{Meio de expressão que utiliza apenas gestos e expressões faciais e corporais.}{mí.mi.ca}{0}
\verb{mímico}{}{}{}{}{adj.}{Relativo a mímica.}{mí.mi.co}{0}
\verb{mímico}{}{}{}{}{s.m.}{Indivíduo que se expressa por mímica.}{mí.mi.co}{0}
\verb{mimo}{}{}{}{}{s.m.}{Carinho, agrado, atenção.}{mi.mo}{0}
\verb{mimo}{}{Bot.}{}{}{}{Certa planta com flores vistosas cultivada como ornamental; brinco"-de"-princesa.}{mi.mo}{0}
\verb{mimosa}{ó}{Bot.}{}{}{s.f.}{Designação comum a diversas plantas, cultivadas como ornamentais e algumas com propriedades medicinais.}{mi.mo.sa}{0}
\verb{mimosear}{}{}{}{}{v.t.}{Presentear, obsequiar.}{mi.mo.se.ar}{0}
\verb{mimosear}{}{}{}{}{}{Tratar com delicadezas; mimar, agradar.}{mi.mo.se.ar}{\verboinum{4}}
\verb{mimoso}{ô}{}{"-osos ⟨ó⟩}{"-osa ⟨ó⟩}{adj.}{Acostumado com muitos carinhos e cuidados; mimado.}{mi.mo.so}{0}
\verb{mimoso}{ô}{}{"-osos ⟨ó⟩}{"-osa ⟨ó⟩}{}{Delicado, suave, meigo.}{mi.mo.so}{0}
\verb{mimoso}{ô}{}{"-osos ⟨ó⟩}{"-osa ⟨ó⟩}{}{Diz"-se de fubá de milho moído muito fino.}{mi.mo.so}{0}
\verb{mina}{}{}{}{}{s.f.}{Depósito natural de minérios explorado pelo homem.}{mi.na}{0}
\verb{mina}{}{}{}{}{}{Artefato explosivo que fica enterrado no solo e explode ao ser pisado.}{mi.na}{0}
\verb{mina}{}{}{}{}{}{Fonte de água.}{mi.na}{0}
\verb{mina}{}{Pop.}{}{}{}{Mulher adolescente; menina, garota.}{mi.na}{0}
\verb{minar}{}{}{}{}{v.t.}{Abrir buracos em algum lugar.}{mi.nar}{0}
\verb{minar}{}{}{}{}{}{Colocar bombas em algum lugar.}{mi.nar}{0}
\verb{minar}{}{}{}{}{}{Corroer pouco a pouco; solapar, consumir.}{mi.nar}{0}
\verb{minar}{}{}{}{}{v.i.}{Sair de uma fonte sem parar; brotar, verter, fluir.}{mi.nar}{\verboinum{1}}
\verb{minarete}{ê}{}{}{}{s.m.}{Torre existente nas mesquitas e de onde se conclamam os fiéis para as orações.}{mi.na.re.te}{0}
\verb{mindinho}{}{}{}{}{adj.}{Diz"-se do quinto dedo da mão, a partir do dedo polegar.}{min.di.nho}{0}
\verb{mindinho}{}{}{}{}{s.m.}{Esse dedo; dedo mínimo.}{min.di.nho}{0}
\verb{mineiro}{ê}{}{}{}{adj.}{Referente a mina.}{mi.nei.ro}{0}
\verb{mineiro}{ê}{}{}{}{}{Relativo a Minas Gerais.}{mi.nei.ro}{0}
\verb{mineiro}{ê}{}{}{}{s.m.}{Operário ou proprietário de mina.}{mi.nei.ro}{0}
\verb{mineiro}{ê}{}{}{}{}{Indivíduo natural ou habitante do estado de Minas Gerais.}{mi.nei.ro}{0}
\verb{mineração}{}{}{"-ões}{}{s.f.}{Atividade de exploração de minas.}{mi.ne.ra.ção}{0}
\verb{mineral}{}{}{"-ais}{}{adj.2g.}{Diz"-se dos corpos inorgânicos encontrados na superfície ou no interior da Terra.}{mi.ne.ral}{0}
\verb{mineral}{}{}{"-ais}{}{}{Que não é orgânico.}{mi.ne.ral}{0}
\verb{mineral}{}{}{"-ais}{}{}{Diz"-se de água coletada e engarrafada diretamente das fontes de água potável, sem passar pelo sistema de encanamentos de uma cidade.}{mi.ne.ral}{0}
\verb{mineralizar}{}{}{}{}{v.t.}{Converter em mineral ou em minério.}{mi.ne.ra.li.zar}{\verboinum{1}}
\verb{mineralogia}{}{Geol.}{}{}{s.f.}{Ramo da geologia que estuda os minerais.}{mi.ne.ra.lo.gi.a}{0}
\verb{mineralogista}{}{}{}{}{s.2g.}{Especialista em mineralogia.}{mi.ne.ra.lo.gis.ta}{0}
\verb{minerar}{}{}{}{}{v.t.}{Explorar mina.}{mi.ne.rar}{\verboinum{1}}
\verb{minério}{}{}{}{}{s.m.}{Mineral composto de várias substâncias do qual se podem extrair materiais de vasta utilidade.}{mi.né.rio}{0}
\verb{mingau}{}{}{}{}{s.m.}{Alimento de consistência pastosa, em geral feito de farinha misturada com leite e açúcar.}{min.gau}{0}
\verb{míngua}{}{}{}{}{s.f.}{Falta de alguma coisa necessária; carência, escassez.}{mín.gua}{0}
\verb{minguado}{}{}{}{}{adj.}{Que carece do necessário.}{min.gua.do}{0}
\verb{minguado}{}{}{}{}{}{Escasso, limitado.}{min.gua.do}{0}
\verb{minguante}{}{}{}{}{adj.2g.}{Que míngua; decrescente, declinante.}{min.guan.te}{0}
\verb{minguante}{}{}{}{}{s.m.}{Aparência que a lua toma quando a parte iluminada diminui; quarto minguante.}{min.guan.te}{0}
\verb{minguar}{}{}{}{}{v.t.}{Tornar escasso ou mais escasso.}{min.guar}{0}
\verb{minguar}{}{}{}{}{}{Diminuir.}{min.guar}{\verboinum{9}\verboirregular[minguo]{mínguo}}
\verb{minguinho}{}{}{}{}{adj.}{Diz"-se do menor dos dedos.}{min.gui.nho}{0}
\verb{minguinho}{}{}{}{}{s.m.}{Esse dedo; dedo mínimo; mindinho.}{min.gui.nho}{0}
\verb{minha}{}{}{}{}{pron.}{Feminino de \textit{meu}.}{mi.nha}{0}
\verb{minhoca}{ó}{Zool.}{}{}{s.f.}{Nome comum aos anelídeos, sobretudo  terrestres, que cavam galerias no solo e apresentam coloração acinzentada ou rósea, e são muito usados como isca em pescaria.}{mi.nho.ca}{0}
\verb{minhocão}{}{}{"-ões}{}{s.m.}{Grande minhoca.}{mi.nho.cão}{0}
\verb{minhocão}{}{}{"-ões}{}{}{Ceroula, calça de malha usada por baixo da calça comprida, durante o inverno.}{mi.nho.cão}{0}
\verb{minhocão}{}{Bras.}{"-ões}{}{}{Tipo de viaduto urbano sinuoso que liga zonas residenciais.}{mi.nho.cão}{0}
\verb{minhoto}{ô}{}{}{}{adj.}{Relativo ao Minho (Portugal).}{mi.nho.to}{0}
\verb{minhoto}{ô}{}{}{}{s.m.}{Indivíduo natural ou habitante do Minho.}{mi.nho.to}{0}
\verb{míni}{}{}{}{}{s.f.}{Forma reduzida de \textit{minissaia}.}{mí.ni}{0}
\verb{miniatura}{}{}{}{}{s.f.}{Qualquer coisa de tamanho reduzido.}{mi.ni.a.tu.ra}{0}
\verb{miniaturista}{}{}{}{}{adj.2g.}{Diz"-se de quem faz miniaturas.}{mi.ni.a.tu.ris.ta}{0}
\verb{minidesvalorização}{}{Econ.}{"-ões}{}{s.f.}{Desvalorização pequena da moeda, geralmente parcelada.}{mi.ni.des.va.lo.ri.za.ção}{0}
\verb{minifundiário}{}{}{}{}{adj.}{Relativo a minifúndio.}{mi.ni.fun.di.á.rio}{0}
\verb{minifundiário}{}{}{}{}{s.m.}{Proprietário de minifúndio.}{mi.ni.fun.di.á.rio}{0}
\verb{minifúndio}{}{}{}{}{s.m.}{Pequena propriedade rural.}{mi.ni.fún.dio}{0}
\verb{mínima}{}{}{}{}{s.f.}{Nota musical que vale metade da semibreve.}{mí.ni.ma}{0}
\verb{mínima}{}{}{}{}{}{Usado na expressão \textit{não ligar a mínima}, que significa \textit{não dar a menor importância}.}{mí.ni.ma}{0}
\verb{minimalismo}{}{}{}{}{s.m.}{Técnica ou estilo artístico caracterizado por extrema enxutez, concisão e simplicidade.}{mi.ni.ma.lis.mo}{0}
\verb{minimizar}{}{}{}{}{v.t.}{Tornar mínimo.}{mi.ni.mi.zar}{0}
\verb{minimizar}{}{}{}{}{}{Subestimar.}{mi.ni.mi.zar}{\verboinum{1}}
\verb{mínimo}{}{}{}{}{adj.}{Que é o menor.}{mí.ni.mo}{0}
\verb{mínimo}{}{}{}{}{s.m.}{A menor porção de qualquer coisa.}{mí.ni.mo}{0}
\verb{mínimo}{}{}{}{}{}{Diz"-se de dedo que sucede o anelar.}{mí.ni.mo}{0}
\verb{minissaia}{}{}{}{}{s.f.}{Saia muito curta.}{mi.nis.sai.a}{0}
\verb{minissérie}{}{Bras.}{}{}{s.f.}{Telenovela em poucos capítulos.}{mi.nis.sé.rie}{0}
\verb{ministerial}{}{}{}{}{adj.2g.}{Relativo a ministro ou a ministério.}{mi.nis.te.ri.al}{0}
\verb{ministerial}{}{}{}{}{}{Que defende ou apoia a política de um governo ou ministro.}{mi.nis.te.ri.al}{0}
\verb{ministério}{}{}{}{}{s.m.}{Cargo, função.}{mi.nis.té.rio}{0}
\verb{ministério}{}{}{}{}{}{Função de ministro.}{mi.nis.té.rio}{0}
\verb{ministério}{}{}{}{}{}{Conjunto de ministros.}{mi.nis.té.rio}{0}
\verb{ministério}{}{}{}{}{}{Prédio onde trabalham os ministros.}{mi.nis.té.rio}{0}
\verb{ministra}{}{}{}{}{s.f.}{Mulher que exerce funções de ministro.}{mi.nis.tra}{0}
\verb{ministra}{}{}{}{}{}{Mulher de ministro.}{mi.nis.tra}{0}
\verb{ministrar}{}{}{}{}{v.t.}{Aplicar, dar.}{mi.nis.trar}{0}
\verb{ministrar}{}{}{}{}{}{Conferir, administrar.}{mi.nis.trar}{\verboinum{1}}
\verb{ministro}{}{}{}{}{s.m.}{Chefe de ministério.}{mi.nis.tro}{0}
\verb{ministro}{}{}{}{}{}{Sacerdote.}{mi.nis.tro}{0}
\verb{ministro}{}{}{}{}{}{Pastor protestante.}{mi.nis.tro}{0}
\verb{minoração}{}{}{"-ões}{}{s.f.}{Ato ou efeito de minorar; diminuição, redução.}{mi.no.ra.ção}{0}
\verb{minorar}{}{}{}{}{v.t.}{Tornar menor; diminuir.}{mi.no.rar}{\verboinum{1}}
\verb{minorativo}{}{}{}{}{adj.}{Que minora, que diminui.}{mi.no.ra.ti.vo}{0}
\verb{minorativo}{}{Med.}{}{}{s.m.}{Laxante.}{mi.no.ra.ti.vo}{0}
\verb{minoria}{}{}{}{}{s.f.}{Condição do que é numericamente inferior a outro.}{mi.no.ri.a}{0}
\verb{minoridade}{}{}{}{}{}{Var. de \textit{menoridade}.}{mi.no.ri.da.de}{0}
\verb{minoritário}{}{Bras.}{}{}{adj.}{Relativo à minoria.}{mi.no.ri.tá.rio}{0}
\verb{minoritário}{}{}{}{}{}{Diz"-se do partido político que detém a minoria dos votos numa assembleia legislativa.}{mi.no.ri.tá.rio}{0}
\verb{minuano}{}{Bras.}{}{}{s.m.}{Vento frio e seco que sopra do sudoeste, durante o inverno.}{mi.nu.a.no}{0}
\verb{minúcia}{}{}{}{}{s.f.}{Coisa muito miúda.}{mi.nú.cia}{0}
\verb{minúcia}{}{}{}{}{}{Detalhe, pormenor, particularidade.}{mi.nú.cia}{0}
\verb{minuciar}{}{}{}{}{v.t.}{Relatar com detalhes; pormenorizar.}{mi.nu.ci.ar}{\verboinum{6}}
\verb{minucioso}{ô}{}{"-osos ⟨ó⟩}{"-osa ⟨ó⟩}{adj.}{Que se ocupa de minúcias.}{mi.nu.ci.o.so}{0}
\verb{minucioso}{ô}{}{"-osos ⟨ó⟩}{"-osa ⟨ó⟩}{}{Descrito sem esquecer os mínimos detalhes.}{mi.nu.ci.o.so}{0}
\verb{minucioso}{ô}{}{"-osos ⟨ó⟩}{"-osa ⟨ó⟩}{}{Feito com grande cuidado e atenção; meticuloso.}{mi.nu.ci.o.so}{0}
\verb{minudência}{}{}{}{}{s.f.}{Minúcia.}{mi.nu.dên.cia}{0}
\verb{minudência}{}{Por ext.}{}{}{}{Rigor, cuidado, atenção no que se faz.}{mi.nu.dên.cia}{0}
\verb{minudenciar}{}{}{}{}{v.t.}{Narrar com pormenores; detalhar, minuciar.}{mi.nu.den.ci.ar}{\verboinum{6}}
\verb{minuendo}{}{Mat.}{}{}{s.m.}{Diminuendo.}{mi.nu.en.do}{0}
\verb{minuete}{ê}{}{}{}{}{Var. de \textit{minueto}.}{mi.nu.e.te}{0}
\verb{minueto}{ê}{}{}{}{s.m.}{Dança francesa surgida no século \textsc{xvii}.}{mi.nu.e.to}{0}
\verb{minueto}{ê}{}{}{}{}{Música composta para essa dança.}{mi.nu.e.to}{0}
\verb{minúscula}{}{}{}{}{adj.}{Diz"-se da letra de formato próprio e menor que a maiúscula.}{mi.nús.cu.la}{0}
\verb{minúsculo}{}{}{}{}{adj.}{Muito pequeno; diminuto, mínimo.}{mi.nús.cu.lo}{0}
\verb{minúsculo}{}{}{}{}{}{Sem importância ou valor; insignificante.}{mi.nús.cu.lo}{0}
\verb{minuta}{}{}{}{}{s.f.}{Rascunho, primeira redação de qualquer escrito.}{mi.nu.ta}{0}
\verb{minuta}{}{Bras.}{}{}{}{Prato preparado no momento, nos restaurantes.}{mi.nu.ta}{0}
\verb{minutar}{}{}{}{}{v.t.}{Fazer ou ditar a minuta de um documento.}{mi.nu.tar}{\verboinum{1}}
\verb{minuto}{}{}{}{}{s.m.}{Unidade de tempo equivalente a 60 segundos.}{mi.nu.to}{0}
\verb{mio}{}{}{}{}{s.m.}{Miado.}{mi.o}{0}
\verb{miocárdio}{}{Anat.}{}{}{s.m.}{Músculo do coração que funciona de forma involuntária e autônoma.}{mi.o.cár.dio}{0}
\verb{miocardite}{}{Med.}{}{}{s.f.}{Inflamação do miocárdio.}{mi.o.car.di.te}{0}
\verb{mioceno}{}{}{}{}{s.m.}{Época do período terciário, entre o oligoceno e o plioceno, marcada pelo grande desenvolvimento dos primatas.}{mi.o.ce.no}{0}
\verb{miolo}{ô}{}{}{}{s.m.}{A parte de dentro do pão.}{mi.o.lo}{0}
\verb{miolo}{ô}{}{}{}{}{A polpa de alguns frutos.}{mi.o.lo}{0}
\verb{miolo}{ô}{}{}{}{}{O cérebro. (Usa"-se geralmente no plural nesta acepção.)}{mi.o.lo}{0}
\verb{miolo}{ô}{}{}{}{}{O interior ou o essencial de algo.}{mi.o.lo}{0}
\verb{miologia}{}{}{}{}{s.f.}{Estudo dos músculos.}{mi.o.lo.gi.a}{0}
\verb{mioma}{}{}{}{}{s.m.}{Tumor formado especialmente por tecidos musculares.}{mi.o.ma}{0}
\verb{míope}{}{}{}{}{adj.2g.}{Diz"-se da pessoa que tem miopia.}{mí.o.pe}{0}
\verb{míope}{}{Fig.}{}{}{}{Pouco perspicaz.}{mí.o.pe}{0}
\verb{miopia}{}{Med.}{}{}{s.f.}{Deficiência ocular que dificulta a visão de objetos distantes do observador.}{mi.o.pi.a}{0}
\verb{miosótis}{}{Bot.}{}{}{s.2g.2n}{Planta de flor miúda azul"-clara.}{mi.o.só.tis}{0}
\verb{mira}{}{}{}{}{s.f.}{Apêndice metálico das armas de fogo pelo qual se dirige a vista nas pontarias.}{mi.ra}{0}
\verb{mira}{}{}{}{}{}{Ato ou efeito de mirar.}{mi.ra}{0}
\verb{mira}{}{}{}{}{}{Alvo, fim, intuito, meta, objetivo. }{mi.ra}{0}
\verb{mirabolante}{}{}{}{}{adj.2g.}{Grandioso demais para se realizar.}{mi.ra.bo.lan.te}{0}
\verb{mirabolante}{}{}{}{}{}{Extraordinário.}{mi.ra.bo.lan.te}{0}
\verb{miraculoso}{ô}{}{"-osos ⟨ó⟩}{"-osa ⟨ó⟩}{adj.}{Milagroso, maravilhoso.}{mi.ra.cu.lo.so}{0}
\verb{miraculoso}{ô}{}{"-osos ⟨ó⟩}{"-osa ⟨ó⟩}{}{Que faz milagres.}{mi.ra.cu.lo.so}{0}
\verb{miragem}{}{}{"-ens}{}{s.f.}{Ilusão de óptica em que objetos distantes produzem uma imagem invertida, como se refletissem na água.}{mi.ra.gem}{0}
\verb{miragem}{}{Fig.}{"-ens}{}{}{Engano dos sentidos; ilusão.}{mi.ra.gem}{0}
\verb{miramar}{}{}{}{}{s.m.}{Mirante voltado para o mar.}{mi.ra.mar}{0}
%\verb{}{}{}{}{}{}{}{}{0}
%\verb{}{}{}{}{}{}{}{}{0}
\verb{mirante}{}{}{}{}{s.m.}{Edificação com vista panorâmica.}{mi.ran.te}{0}
\verb{mirar}{}{}{}{}{v.t.}{Fixar a vista em.}{mi.rar}{0}
\verb{mirar}{}{}{}{}{}{Fazer pontaria.}{mi.rar}{0}
\verb{mirar}{}{}{}{}{v.pron.}{Olhar ou contemplar a própria imagem refletida.}{mi.rar}{0}
\verb{mirar}{}{}{}{}{}{Refletir"-se, reproduzir"-se.}{mi.rar}{\verboinum{1}}
\verb{miríada}{}{}{}{}{}{Var. de \textit{miríade}.}{mi.rí.a.da}{0}
\verb{miríade}{}{}{}{}{s.f.}{Número de dez mil.}{mi.rí.a.de}{0}
\verb{miríade}{}{Fig.}{}{}{}{Grande quantidade indeterminada.}{mi.rí.a.de}{0}
\verb{miriagrama}{}{}{}{}{s.m.}{Unidade de massa de dez mil gramas.}{mi.ri.a.gra.ma}{0}
\verb{mirialitro}{}{}{}{}{s.m.}{Medida de capacidade de dez mil litros.}{mi.ri.a.li.tro}{0}
\verb{miriâmetro}{}{}{}{}{s.m.}{Medida de comprimento de dez mil metros.}{mi.ri.â.me.tro}{0}
\verb{miriápode}{}{}{}{}{adj.2g.}{Que tem muitos pés.}{mi.ri.á.po.de}{0}
\verb{miriápode}{}{Zool.}{}{}{s.m.}{Gênero de artrópodes, dividido em duas clases, em que se incluem diversos invertebrados alongados de muitas pernas; piolho"-de"-cobra, lacraia, embuá.}{mi.ri.á.po.de}{0}
\verb{miriare}{}{}{}{}{s.m.}{Superfície de um quilômetro quadrado.}{mi.ri.a.re}{0}
\verb{mirificar}{}{}{}{}{v.t.}{Tornar admirável.}{mi.ri.fi.car}{0}
\verb{mirificar}{}{}{}{}{}{Causar espanto.}{mi.ri.fi.car}{\verboinum{2}}
\verb{mirim}{}{}{"-ins}{}{adj.2g.}{Pequeno.}{mi.rim}{0}
\verb{mirim}{}{}{"-ins}{}{}{De pouca idade.}{mi.rim}{0}
%\verb{}{}{}{}{}{}{}{}{0}
%\verb{}{}{}{}{}{}{}{}{0}
\verb{mirra}{}{Bot.}{}{}{s.f.}{Árvore de resina perfumada.}{mir.ra}{0}
\verb{mirra}{}{}{}{}{}{Essa resina usada como incenso, e no preparo de remédio e perfume.}{mir.ra}{0}
\verb{mirrado}{}{}{}{}{adj.}{Seco, magro, definhado.}{mir.ra.do}{0}
\verb{mirrar}{}{}{}{}{v.t.}{Tornar cada vez mais fraco e seco; definhar.}{mir.rar}{\verboinum{1}}
\verb{misantropia}{}{}{}{}{s.f.}{Aversão à sociedade.}{mi.san.tro.pi.a}{0}
\verb{misantropia}{}{}{}{}{}{Melancolia.}{mi.san.tro.pi.a}{0}
\verb{misantropo}{ô}{}{}{}{adj.}{Que odeia a humanidade ou sente aversão às pessoas.}{mi.san.tro.po}{0}
\verb{miscelânea}{}{}{}{}{s.f.}{Coletânea de estudos.}{mis.ce.lâ.nea}{0}
\verb{miscelânea}{}{}{}{}{}{Mistura de várias coisas.}{mis.ce.lâ.nea}{0}
\verb{miscigenação}{}{}{"-ões}{}{s.f.}{Cruzamento de raças ou etnias; mestiçagem.}{mis.ci.ge.na.ção}{0}
\verb{miscível}{}{}{"-eis}{}{adj.2g.}{Que se pode misturar.}{mis.cí.vel}{0}
\verb{miserando}{}{}{}{}{adj.}{Lastimoso, deplorável; digno de dó.}{mi.se.ran.do}{0}
\verb{miserável}{}{}{"-eis}{}{adj.2g.}{Que é digno de piedade.}{mi.se.rá.vel}{0}
\verb{miserável}{}{}{"-eis}{}{}{Que vive em extrema pobreza.}{mi.se.rá.vel}{0}
\verb{miserável}{}{}{"-eis}{}{}{Que merece castigo por sua maldade; infame, vil.}{mi.se.rá.vel}{0}
\verb{miséria}{}{}{}{}{s.f.}{Estado de grande pobreza.}{mi.sé.ria}{0}
\verb{miséria}{}{}{}{}{}{Quantidade muito pequena de alguma coisa; insignificância.}{mi.sé.ria}{0}
\verb{misericórdia}{}{}{}{}{s.f.}{Vontade de ajudar quem passa por grande dificuldade; compaixão.}{mi.se.ri.cór.dia}{0}
\verb{misericórdia}{}{}{}{}{}{Vontade de perdoar; indulgência.}{mi.se.ri.cór.dia}{0}
\verb{misericordioso}{ô}{}{"-osos ⟨ó⟩}{"-osa ⟨ó⟩}{adj.}{Que tem misericórdia; compassivo, bondoso, piedoso.}{mi.se.ri.cor.di.o.so}{0}
\verb{mísero}{}{}{}{}{adj.}{Que causa piedade; miserável.}{mí.se.ro}{0}
\verb{misógamo}{}{}{}{}{adj.}{Que tem aversão ao casamento.}{mi.só.ga.mo}{0}
\verb{misoneísmo}{}{}{}{}{s.m.}{Aversão a tudo que é novo.}{mi.so.ne.ís.mo}{0}
%\verb{}{}{}{}{}{}{}{}{0}
%\verb{}{}{}{}{}{}{}{}{0}
\verb{missa}{}{}{}{}{s.f.}{Ato religioso em que o padre oferece a Deus o corpo e o sangue de Cristo sob a forma de pão e vinho, respectivamente.}{mis.sa}{0}
\verb{missal}{}{}{"-ais}{}{s.m.}{Livro que contém os textos e os cânticos da missa.}{mis.sal}{0}
\verb{missão}{}{}{"-ões}{}{s.f.}{Tarefa dada a uma pessoa por uma autoridade.}{mis.são}{0}
\verb{missão}{}{}{"-ões}{}{}{Trabalho de catequese para difundir uma religião.}{mis.são}{0}
\verb{missão}{}{}{"-ões}{}{}{Trabalho que o profissional aceita como um dever.}{mis.são}{0}
\verb{misse}{}{}{}{}{s.f.}{Moça selecionada em concurso de beleza.}{mis.se}{0}
\verb{míssil}{}{}{"-eis}{}{s.m.}{Projétil de longo alcance.}{mís.sil}{0}
\verb{míssil}{}{}{"-eis}{}{adj.2g.}{Que pode ser arremessado.}{mís.sil}{0}
\verb{missionário}{}{}{}{}{s.m.}{Participante de uma missão religiosa.}{mis.sio.ná.rio}{0}
\verb{missiva}{}{}{}{}{s.f.}{Comunicação escrita que se manda a alguém; carta, bilhete.}{mis.si.va}{0}
\verb{missivista}{}{}{}{}{s.2g.}{Indivíduo que leva missivas ou cartas.}{mis.si.vis.ta}{0}
\verb{missivista}{}{}{}{}{}{Indivíduo que escreve cartas a alguém.}{mis.si.vis.ta}{0}
\verb{mister}{é}{}{}{}{s.m.}{Coisa que se deve fazer; obrigação.}{mis.ter}{0}
\verb{mistério}{}{}{}{}{s.m.}{Coisa que não tem explicação; enigma.}{mis.té.rio}{0}
\verb{mistério}{}{}{}{}{}{Ponto de doutrina religiosa que se deve aceitar sem discutir; dogma.}{mis.té.rio}{0}
\verb{misterioso}{ô}{}{"-osos ⟨ó⟩}{"-osa ⟨ó⟩}{adj.}{Em que há, ou que envolve mistério; oculto, secreto.}{mis.te.ri.o.so}{0}
\verb{misterioso}{ô}{}{"-osos ⟨ó⟩}{"-osa ⟨ó⟩}{}{Inexplicável, enigmático.}{mis.te.ri.o.so}{0}
\verb{misterioso}{ô}{}{"-osos ⟨ó⟩}{"-osa ⟨ó⟩}{}{Estranho, imponderável.}{mis.te.ri.o.so}{0}
\verb{misterioso}{ô}{}{"-osos ⟨ó⟩}{"-osa ⟨ó⟩}{}{Suspeito, dissimulado.}{mis.te.ri.o.so}{0}
\verb{mística}{}{}{}{}{s.f.}{Estudo do que é divino, espiritual.}{mís.ti.ca}{0}
\verb{mística}{}{}{}{}{}{Vida religiosa ou contemplativa.}{mís.ti.ca}{0}
\verb{misticismo}{}{}{}{}{s.m.}{Doutrina religiosa dos místicos.}{mis.ti.cis.mo}{0}
\verb{misticismo}{}{}{}{}{}{Contemplação espiritual.}{mis.ti.cis.mo}{0}
\verb{misticismo}{}{}{}{}{}{Tendência para acreditar no sobrenatural.}{mis.ti.cis.mo}{0}
\verb{místico}{}{}{}{}{adj.}{Referente aos mistérios, às cerimônias religiosas secretas. }{mís.ti.co}{0}
\verb{místico}{}{}{}{}{}{Em que existe uma grande ligação com Deus.}{mís.ti.co}{0}
\verb{mistificação}{}{}{"-ões}{}{s.f.}{Ato ou efeito de mistificar; engano, farsa, ludibrio.}{mis.ti.fi.ca.ção}{0}
\verb{mistificar}{}{}{}{}{v.t.}{Fazer alguém crer em uma mentira ou em algo falso, abusando de sua credulidade, enganar, ludibriar, iludir.}{mis.ti.fi.car}{\verboinum{2}}
\verb{mistifório}{}{Pejor.}{}{}{s.m.}{Mistura indistinta de coisas ou pessoas; mixórdia, confusão.}{mis.ti.fó.rio}{0}
\verb{misto}{}{}{}{}{adj.}{Misturado.}{mis.to}{0}
\verb{misto}{}{Mat.}{}{}{}{Diz"-se de número que tem uma parte inteira e outra fracionária.}{mis.to}{0}
\verb{misto}{}{}{}{}{}{Que compreende pessoas de ambos os sexos.}{mis.to}{0}
\verb{misto"-quente}{}{}{mistos"-quentes}{}{s.m.}{Sanduíche quente de presunto e queijo.}{mis.to"-quen.te}{0}
\verb{mistral}{}{}{"-ais}{}{s.m.}{Vento forte, frio e seco.}{mis.tral}{0}
\verb{mistura}{}{}{}{}{s.f.}{Reunião de coisas diversas.}{mis.tu.ra}{0}
\verb{mistura}{}{}{}{}{}{Cruzamento de raças; miscigenação.}{mis.tu.ra}{0}
\verb{misturada}{}{}{}{}{s.f.}{Coisa muito mal organizada; confusão, mixórdia.}{mis.tu.ra.da}{0}
\verb{misturada}{}{}{}{}{}{Mistura de uma bebida alcoólica com outra qualquer.}{mis.tu.ra.da}{0}
\verb{misturar}{}{}{}{}{v.t.}{Reunir coisas diversas.}{mis.tu.rar}{0}
\verb{misturar}{}{}{}{}{}{Confundir.}{mis.tu.rar}{\verboinum{1}}
\verb{mitene}{}{}{}{}{s.f.}{Luva feminina que deixa livre os dedos.}{mi.te.ne}{0}
\verb{mítico}{}{}{}{}{adj.}{Relativo a ou próprio de mito.}{mí.ti.co}{0}
\verb{mitificar}{}{}{}{}{v.t.}{Tranformar alguma coisa ou alguém em mito. (\textit{Alguns filmes mitificam os atores.})}{mi.ti.fi.car}{\verboinum{1}}
\verb{mitigar}{}{}{}{}{v.t.}{Tornar mais brando; acalmar, suavizar, aliviar.}{mi.ti.gar}{\verboinum{5}}
\verb{mito}{}{}{}{}{s.m.}{Narrativa alegórica sobre feitos de seres com poderes sobre"-humanos; fábula, lenda.}{mi.to}{0}
\verb{mito}{}{}{}{}{}{Pessoa famosa que é tida como herói.}{mi.to}{0}
\verb{mito}{}{}{}{}{}{História fictícia, irreal.}{mi.to}{0}
\verb{mitologia}{}{}{}{}{s.f.}{Ciência ou tratado acerca dos mitos.}{mi.to.lo.gi.a}{0}
\verb{mitologia}{}{}{}{}{}{Conjunto das lendas e mitos de um povo.}{mi.to.lo.gi.a}{0}
\verb{mitológico}{}{}{}{}{adj.}{Relativo a mitologia.}{mi.to.ló.gi.co}{0}
\verb{mitólogo}{}{}{}{}{s.m.}{Pessoa que estuda ou escreve sobre mitos.}{mi.tó.lo.go}{0}
\verb{mitomania}{}{Med.}{}{}{s.f.}{Tendência mórbida para mentir ou exagerar os fatos.}{mi.to.ma.ni.a}{0}
\verb{mitomania}{}{}{}{}{}{Hábito de mentir, fantasiar ou exagerar os fatos.}{mi.to.ma.ni.a}{0}
\verb{mitômano}{}{}{}{}{adj.}{Que sofre de mitomania.}{mi.tô.ma.no}{0}
\verb{mitômano}{}{}{}{}{s.m.}{Essa pessoa.}{mi.tô.ma.no}{0}
\verb{mitra}{}{}{}{}{s.f.}{Chapéu alto e pontudo usado pelos bispos, arcebispos e cardeais.}{mi.tra}{0}
\verb{mitrado}{}{}{}{}{adj.}{Que tem mitra ou detém o direito de usá"-la.}{mi.tra.do}{0}
\verb{mitral}{}{Anat.}{"-ais}{}{s.f.}{Válvula do coração, situada na comunicação do ventrículo com a aurícula esquerda.}{mi.tral}{0}
\verb{mitridatismo}{}{Med.}{}{}{s.m.}{Imunização de um paciente contra os efeitos de um veneno, que se adquire aplicando doses gradualmente crescentes do mesmo veneno.}{mi.tri.da.tis.mo}{0}
%\verb{}{}{}{}{}{}{}{}{0}
\verb{miuçalha}{}{}{}{}{s.f.}{Conjunto de coisas miúdas e de pouca serventia.}{mi.u.ça.lha}{0}
\verb{miuçalha}{}{}{}{}{}{Grupo de crianças pequenas; pirralhada.}{mi.u.ça.lha}{0}
\verb{miudeza}{ê}{}{}{}{s.f.}{Qualidade de miúdo; pequenez.}{mi.u.de.za}{0}
\verb{miudeza}{ê}{}{}{}{}{Minúcia, pormenor, particularidade.}{mi.u.de.za}{0}
\verb{miudeza}{ê}{Por ext.}{}{}{}{Atenção, cuidado, rigor, escrúpulo ao examinar ou fazer alguma coisa.}{mi.u.de.za}{0}
\verb{miúdo}{}{}{}{}{adj.}{Que é muito pequeno; diminuto.}{mi.ú.do}{0}
\verb{miúdo}{}{}{}{}{}{Dinheiro em pequenas moedas.}{mi.ú.do}{0}
\verb{miúdo}{}{}{}{}{}{Víscera de animais como rim, fígado, moela etc.}{mi.ú.do}{0}
\verb{miúdos}{}{}{}{}{s.m.pl.}{Vísceras de animal (fígado, rim, moela etc.), usadas no preparo de alguns pratos; miudezas.}{mi.ú.dos}{0}
\verb{mixagem}{cs}{}{"-ens}{}{s.f.}{Ato ou efeito de sobrepor vários canais de som que foram gravados separadamente.}{mi.xa.gem}{0}
\verb{mixagem}{cs}{}{"-ens}{}{}{No cinema, a justaposição de diálogos, música, sonoplastia etc. e as imagens correspondentes.}{mi.xa.gem}{0}
\verb{mixar}{ch}{}{}{}{v.i.}{Acabar, chegar ao fim, terminar, findar.}{mi.xar}{0}
\verb{mixar}{ch}{}{}{}{}{Não dar certo; falhar, gorar, frustrar, malograr. }{mi.xar}{0}
\verb{mixar}{ch}{}{}{}{}{Perder a força ou ânimo; exaurir"-se, enfraquecer, diminuir.}{mi.xar}{\verboinum{1}}
\verb{mixaria}{ch}{Pop.}{}{}{s.f.}{Coisa sem valor, insignificante.}{mi.xa.ri.a}{0}
\verb{mixaria}{ch}{}{}{}{}{Pequena quantia de dinheiro; ninharia; bagatela.}{mi.xa.ri.a}{0}
\verb{mixe}{ch}{Bras.}{}{}{adj.2g.}{Que é insignificante, pequeno ou escasso; mixo.}{mi.xe}{0}
\verb{mixe}{ch}{}{}{}{}{Que é ruim, de má qualidade; imprestável, inútil.}{mi.xe}{0}
\verb{mixo}{ch}{}{}{}{adj.}{Mixe.}{mi.xo}{0}
\verb{mixórdia}{ch}{}{}{}{s.f.}{Mistura confusa e desordenada de coisas diversas; maçarocada, confusão, miscelânea, bagunça.}{mi.xór.dia}{0}
\verb{mixuruca}{ch}{}{}{}{adj.2g.}{Mixe.}{mi.xu.ru.ca}{0}
\verb{ml}{}{}{}{}{}{Abrev. de \textit{mililitro}.}{ml}{0}
\verb{Mn}{}{Quím.}{}{}{}{Símb. do \textit{manganês}.}{Mn}{0}
\verb{mnemônica}{}{}{}{}{s.f.}{Arte ou técnica de desenvolver a memória por meio de processos artificiais.}{mne.mô.ni.ca}{0}
\verb{mnemônico}{}{}{}{}{adj.}{Relativo a ou próprio da memória.}{mne.mô.ni.co}{0}
\verb{mnemônico}{}{}{}{}{}{Que se usa para desenvolver a memória ou tornar a memorização mais fácil.}{mne.mô.ni.co}{0}
\verb{mo}{}{}{}{ma}{}{Contração dos pronomes pessoais \textit{me} e \textit{o}.}{mo}{0}
\verb{Mo}{}{Quím.}{}{}{}{Símb. do \textit{molibdênio}.}{Mo}{0}
\verb{mó}{}{}{}{}{s.f.}{Pedra circular usada para triturar grãos nos moinhos ou extrair azeite das azeitonas.}{mó}{0}
\verb{mó}{}{}{}{}{}{Pedra circular usada para afiar instrumentos de corte.}{mó}{0}
\verb{moagem}{}{}{"-ens}{}{s.f.}{Ato de moer.}{mo.a.gem}{0}
\verb{móbil}{}{}{móbeis \textit{ou} móbiles}{}{s.m.}{Aquilo que move alguém a realizar determinada ação; causa, motivo.}{mó.bil}{0}
\verb{móbile}{}{Art.}{}{}{s.m.}{Escultura feita com formas de material leve suspensas no ar por fios presos em hastes equilibradas, que se move com os movimentos do ar.}{mó.bi.le}{0}
\verb{mobilhar}{}{}{}{}{}{Var. de \textit{mobiliar}.}{mo.bi.lhar}{0}
\verb{mobília}{}{}{}{}{s.f.}{O conjunto dos móveis de um ambiente, de caráter funcional ou decorativo.}{mo.bí.lia}{0}
\verb{mobiliar}{}{Bras.}{}{}{v.t.}{Escolher e dispor a mobília em um ambiente.}{mo.bi.li.ar}{\verboinum{6}}
\verb{mobiliário}{}{}{}{}{adj.}{Relativo a bens móveis.}{mo.bi.li.á.rio}{0}
\verb{mobiliário}{}{}{}{}{}{Mobília.}{mo.bi.li.á.rio}{0}
\verb{mobilidade}{}{}{}{}{s.f.}{Qualidade do que é móvel.}{mo.bi.li.da.de}{0}
\verb{mobilidade}{}{}{}{}{}{Facilidade para mover"-se, variar ou mudar de estado.}{mo.bi.li.da.de}{0}
\verb{mobilização}{}{}{"-ões}{}{s.f.}{Ato ou efeito de mobilizar.}{mo.bi.li.za.ção}{0}
\verb{mobilizar}{}{}{}{}{v.t.}{Movimentar, mover.}{mo.bi.li.zar}{0}
\verb{mobilizar}{}{}{}{}{}{Arregimentar e pôr em ação pessoas para lutarem juntos por uma determinada causa.}{mo.bi.li.zar}{0}
\verb{mobilizar}{}{}{}{}{}{Colocar (tropas militares) em atividade para a realização de ações militares.}{mo.bi.li.zar}{0}
\verb{mobilizar}{}{Desus.}{}{}{}{Colocar (valores) em circulação.}{mo.bi.li.zar}{\verboinum{1}}
\verb{moca}{ó}{Bras.}{}{}{s.f.}{Zombaria, caçoada, tolice.}{mo.ca}{0}
\verb{moca}{ó}{}{}{}{s.m.}{Variedade de café de alta qualidade, originária da Arábia.}{mo.ca}{0}
\verb{moça}{ô}{}{}{}{s.f.}{Mulher jovem.}{mo.ça}{0}
\verb{moçada}{}{Pop.}{}{}{s.f.}{Conjunto de pessoas jovens; mocidade.}{mo.ça.da}{0}
\verb{mocambeiro}{ê}{Bras.}{}{}{adj.}{Dizia"-se de escravo que morava ou se foragia em mocambo.}{mo.cam.bei.ro}{0}
\verb{moçambicano}{}{}{}{}{adj.}{Relativo a Moçambique.}{mo.çam.bi.ca.no}{0}
\verb{moçambicano}{}{}{}{}{s.m.}{Indivíduo natural ou habitante desse país.}{mo.çam.bi.ca.no}{0}
\verb{mocambo}{}{Bras.}{}{}{s.m.}{Refúgio de escravos foragidos; quilombo.}{mo.cam.bo}{0}
\verb{mocambo}{}{Por ext.}{}{}{}{Habitação de condições precárias.}{mo.cam.bo}{0}
\verb{moção}{}{}{"-ões}{}{s.f.}{Ato ou efeito de mover; deslocamento, movimento.}{mo.ção}{0}
\verb{moção}{}{Fig.}{"-ões}{}{}{Comoção, abalo.}{mo.ção}{0}
\verb{moçárabe}{}{}{}{}{adj.2g.}{Diz"-se de indivíduos de origem hispânica e religião cristã que habitavam as terras do sul da Península Ibérica ocupadas pelos muçulmanos.}{mo.çá.ra.be}{0}
\verb{moçárabe}{}{Gram.}{}{}{}{Diz"-se dos dialetos românicos que eram falados por essas pessoas.}{mo.çá.ra.be}{0}
\verb{mocassim}{}{}{"-ins}{}{s.m.}{Tipo de sapato baixo e macio, geralmente sem cadarço.}{mo.cas.sim}{0}
\verb{mocetão}{}{}{"-ões}{}{s.m.}{Rapaz alto, robusto e vistoso.}{mo.ce.tão}{0}
\verb{mochila}{}{}{}{}{s.f.}{Saco de lona ou tecido resistente que se prende às costas com correias, utilizado por estudantes, viajantes e militares para transportar pertences.}{mo.chi.la}{0}
\verb{mocho}{ô}{}{}{}{adj.}{Diz"-se de animal cujos chifres foram aparados.}{mo.cho}{0}
\verb{mocho}{ô}{}{}{}{}{Diz"-se de animal que sofreu mutilação de algum membro, orelhas, garras.}{mo.cho}{0}
\verb{mocidade}{}{}{}{}{s.f.}{Período da vida humana que compreende a adolescência e a primeira fase da vida adulta; juventude.}{mo.ci.da.de}{0}
\verb{mocidade}{}{}{}{}{}{Conjunto das pessoas jovens; juventude.}{mo.ci.da.de}{0}
\verb{mocidade}{}{}{}{}{}{A energia que caracteriza as pessoas jovens.}{mo.ci.da.de}{0}
\verb{mocinho}{}{}{}{}{s.m.}{Homem jovem; moçoilo.}{mo.ci.nho}{0}
\verb{mocinho}{}{}{}{}{}{O herói de filmes e histórias de aventura.}{mo.ci.nho}{0}
\verb{moço}{ô}{}{}{}{adj.}{Jovem.}{mo.ço}{0}
\verb{moço}{ô}{Fig.}{}{}{}{Sem experiência.}{mo.ço}{0}
\verb{moço}{ô}{}{}{}{s.m.}{Indivíduo jovem.}{mo.ço}{0}
\verb{mocotó}{}{Bras.}{}{}{s.m.}{Pata de bovino, sem o casco, utilizada como alimento.}{mo.co.tó}{0}
\verb{mocotó}{}{Pop.}{}{}{}{Tornozelo, calcanhar.}{mo.co.tó}{0}
\verb{moda}{ó}{}{}{}{s.f.}{Conjunto de práticas sociais tidas como dignas de admiração e, portanto, imitadas em determinado local e época.}{mo.da}{0}
\verb{moda}{ó}{}{}{}{}{Modinha.}{mo.da}{0}
\verb{moda}{ó}{}{}{}{}{Peça de vestuário. (Usa"-se geralmente no plural nesta acepção.)}{mo.da}{0}
\verb{modal}{}{}{"-ais}{}{adj.2g.}{Relativo a modo ou a modalidade.}{mo.dal}{0}
\verb{modal}{}{Gram.}{"-ais}{}{}{Diz"-se das formas gramaticais que alteram a categoria de modo de uma sentença.}{mo.dal}{0}
\verb{modalidade}{}{}{}{}{s.f.}{Maneira peculiar de cada indivíduo.}{mo.da.li.da.de}{0}
\verb{modalidade}{}{Gram.}{}{}{}{Modo.}{mo.da.li.da.de}{0}
\verb{modalidade}{}{Esport.}{}{}{}{Cada uma das espécies particulares de esporte, com regras e práticas específicas.}{mo.da.li.da.de}{0}
\verb{modelador}{ô}{}{}{}{s.m.}{Que modela.}{mo.de.la.dor}{0}
\verb{modelagem}{}{}{"-ens}{}{s.f.}{Ato ou efeito de modelar.}{mo.de.la.gem}{0}
\verb{modelar}{}{}{}{}{v.t.}{Fazer o molde de.}{mo.de.lar}{0}
\verb{modelar}{}{}{}{}{}{Criar o relevo ou os contornos de.}{mo.de.lar}{0}
\verb{modelar}{}{}{}{}{}{Destacar os contornos ou as formas, especialmente do corpo humano; contornar.}{mo.de.lar}{0}
\verb{modelar}{}{Fig.}{}{}{}{Traçar as linhas de; delinear, planejar, arquitetar.}{mo.de.lar}{\verboinum{1}}
\verb{modelo}{ê}{}{}{}{s.m.}{Representação em escala reduzida de objeto ou obra arquitetônica; maquete.}{mo.de.lo}{0}
\verb{modelo}{ê}{}{}{}{}{Representação abstrata de um fenômeno, capaz de captar as características formais ou funcionais do objeto de estudo.}{mo.de.lo}{0}
\verb{modelo}{ê}{}{}{}{}{Tipo exemplar; exemplo a ser seguido.}{mo.de.lo}{0}
\verb{modelo}{ê}{}{}{}{s.2g.}{Indivíduo que posa para exercício prático de escultura, pintura ou fotografia.}{mo.de.lo}{0}
\verb{modelo}{ê}{}{}{}{}{Indivíduo que serve para provar ou exibir peças de vestuário; manequim.}{mo.de.lo}{0}
\verb{modem}{}{Informát.}{"-ems}{}{s.m.}{Dispositivo que serve para enviar e receber dados digitais por intermédio de linha telefônica.}{mo.dem}{0}
\verb{moderação}{}{}{"-ões}{}{s.f.}{Ato ou efeito de moderar.}{mo.de.ra.ção}{0}
\verb{moderação}{}{}{"-ões}{}{}{Qualidade de saber evitar excessos; comedimento, prudência.}{mo.de.ra.ção}{0}
\verb{moderado}{}{}{}{}{adj.}{Não excessivo; razoável, mediano.}{mo.de.ra.do}{0}
\verb{moderado}{}{}{}{}{}{Que age com moderação; prudente, comedido.}{mo.de.ra.do}{0}
\verb{moderador}{ô}{}{}{}{adj.}{Que modera.}{mo.de.ra.dor}{0}
\verb{moderador}{ô}{}{}{}{s.m.}{Indivíduo que comanda discussão ou debate em grupo, controlando a ordem e os tempos das falas.}{mo.de.ra.dor}{0}
\verb{moderar}{}{}{}{}{v.t.}{Guardar as proporções ou a intensidade adequada.}{mo.de.rar}{0}
\verb{moderar}{}{}{}{}{}{Tornar menos intenso; diminuir, controlar.}{mo.de.rar}{0}
\verb{moderar}{}{}{}{}{}{Evitar excessos; comedir.}{mo.de.rar}{\verboinum{1}}
\verb{modernice}{}{}{}{}{s.f.}{Qualidade de moderno.}{mo.der.ni.ce}{0}
\verb{modernice}{}{}{}{}{}{Preferência incondicional por tudo que é moderno.}{mo.der.ni.ce}{0}
\verb{modernidade}{}{}{}{}{s.f.}{Qualidade do que é moderno; modernismo.}{mo.der.ni.da.de}{0}
\verb{modernidade}{}{}{}{}{}{Época considerada moderna segundo critérios determinados.}{mo.der.ni.da.de}{0}
\verb{modernismo}{}{}{}{}{s.m.}{Tendência pelo que é moderno.}{mo.der.nis.mo}{0}
\verb{modernismo}{}{Art.}{}{}{}{Designação de certos movimentos artísticos surgidos no fim do século \textsc{xix} e início do século \textsc{xx}.}{mo.der.nis.mo}{0}
\verb{modernista}{}{}{}{}{adj.2g.}{Relativo ao modernismo.}{mo.der.nis.ta}{0}
\verb{modernista}{}{Art.}{}{}{s.2g.}{Pessoa adepta de ou ligada aos movimentos artísticos chamados de modernismo.}{mo.der.nis.ta}{0}
\verb{modernizar}{}{}{}{}{v.t.}{Tornar moderno.}{mo.der.ni.zar}{\verboinum{1}}
\verb{moderno}{é}{}{}{}{adj.}{Relativo à época atual.}{mo.der.no}{0}
\verb{moderno}{é}{}{}{}{}{Que tem características técnicas ou valores contemporâneos.}{mo.der.no}{0}
\verb{moderno}{é}{Art.}{}{}{}{Relativo ao período classificado como Modernismo.}{mo.der.no}{0}
\verb{modéstia}{}{}{}{}{s.f.}{Qualidade de modesto; ausência de vaidade; simplicidade.}{mo.dés.tia}{0}
\verb{modesto}{é}{}{}{}{adj.}{Cujas características ou aspirações não revelam pretensão ou vaidade.}{mo.des.to}{0}
\verb{modesto}{é}{}{}{}{}{Não excessivo; sóbrio, moderado.}{mo.des.to}{0}
\verb{modesto}{é}{}{}{}{}{Que revela pobreza.}{mo.des.to}{0}
\verb{módico}{}{}{}{}{adj.}{Reduzido, escasso, limitado, modesto.}{mó.di.co}{0}
\verb{modificação}{}{}{"-ões}{}{s.f.}{Ato ou efeito de modificar; alteração.}{mo.di.fi.ca.ção}{0}
\verb{modificar}{}{}{}{}{v.t.}{Alterar, transformar, mudar.}{mo.di.fi.car}{\verboinum{2}}
\verb{modilhão}{}{}{"-ões}{}{s.m.}{Tipo de ornamento arquitetônico em forma de \textsc{s} invertido.}{mo.di.lhão}{0}
\verb{modinha}{}{Mús.}{}{}{s.f.}{Gênero de canção popular tradicional, acompanhada por violão ou viola e geralmente com tema amoroso.}{mo.di.nha}{0}
\verb{modismo}{}{}{}{}{s.m.}{Coisa ou costume de caráter efêmero, já que é praticado meramente por estar na moda.}{mo.dis.mo}{0}
\verb{modista}{}{}{}{}{s.2g.}{Profissional que dirige ou trabalha em ateliê de costura de roupa feminina.}{mo.dis.ta}{0}
\verb{modista}{}{}{}{}{s.2g.}{Cantor de modinhas.}{mo.dis.ta}{0}
\verb{modo}{ó}{}{}{}{s.m.}{Maneira ou forma particular; jeito.}{mo.do}{0}
\verb{modo}{ó}{}{}{}{}{Maneira, meio.}{mo.do}{0}
\verb{modo}{ó}{}{}{}{}{Prática, uso, sistema.}{mo.do}{0}
\verb{modo}{ó}{Gram.}{}{}{}{Categoria verbal que exprime a atitude do falante em relação ao que está sendo dito, como desejo, possibilidade, dever, incerteza.}{mo.do}{0}
\verb{modo}{ó}{}{}{}{}{Atitude, maneira, educação. [usa"-se geralmente no plural nesta acepção]}{mo.do}{0}
\verb{modorra}{ô}{}{}{}{s.f.}{Prostração ou sonolência excessiva que acomete certos doentes.}{mo.dor.ra}{0}
\verb{modorra}{ô}{}{}{}{}{Moleza, preguiça, sonolência.}{mo.dor.ra}{0}
\verb{modorra}{ô}{Fig.}{}{}{}{Indiferença, apatia.}{mo.dor.ra}{0}
\verb{modulação}{}{}{"-ões}{}{s.f.}{Ato ou efeito de modular.}{mo.du.la.ção}{0}
\verb{modulação}{}{}{"-ões}{}{}{Variações controladas de altura e intensidade na produção de um som.}{mo.du.la.ção}{0}
\verb{modular}{}{}{}{}{adj.2g.}{Relativo a módulo.}{mo.du.lar}{0}
\verb{modular}{}{}{}{}{v.t.}{Estabelecer e controlar a altura e a frequência na geração de sinais sonoros.}{mo.du.lar}{0}
\verb{modular}{}{}{}{}{}{Cantar ou dizer com sonoridade agradável e harmoniosa.}{mo.du.lar}{0}
\verb{modular}{}{Fís.}{}{}{}{Usar uma onda para modificar parâmetros de outra onda.}{mo.du.lar}{0}
\verb{modular}{}{Informát.}{}{}{}{Transformar informação digital em uma frequência sonora para que possa trafegar por meio de linha telefônica.}{mo.du.lar}{\verboinum{1}}
\verb{módulo}{}{}{}{}{s.m.}{Modulação da frequência de um som.}{mó.du.lo}{0}
\verb{módulo}{}{}{}{}{}{Quantidade específica que é adotada como padrão de medida.}{mó.du.lo}{0}
\verb{módulo}{}{}{}{}{}{Unidade de qualquer coisa planejada para ajustar"-se a outras semelhantes.}{mó.du.lo}{0}
\verb{módulo}{}{Mat.}{}{}{}{Valor absoluto de um número, desconsiderado o sinal positivo ou negativo.}{mó.du.lo}{0}
\verb{moeda}{é}{}{}{}{s.f.}{Placa metálica, geralmente de forma circular, cunhada por uma autoridade governamental, usada como medida de valor e meio de troca.}{mo.e.da}{0}
\verb{moeda}{é}{Fig.}{}{}{}{Aquilo que tem valor de troca; dinheiro.}{mo.e.da}{0}
\verb{moedeiro}{ê}{}{}{}{s.m.}{Pequena bolsa ou recipiente para guardar moedas.}{mo.e.dei.ro}{0}
\verb{moedeiro}{ê}{}{}{}{}{Fabricante de moedas.}{mo.e.dei.ro}{0}
\verb{moedor}{ô}{}{}{}{adj.}{Que mói.}{mo.e.dor}{0}
\verb{moedor}{ô}{Fig.}{}{}{}{Cansativo, maçante.}{mo.e.dor}{0}
\verb{moedor}{ô}{}{}{}{s.m.}{Aparelho para moer ou triturar.}{mo.e.dor}{0}
\verb{moedura}{}{}{}{}{s.f.}{Ato de moer; moagem.}{mo.e.du.ra}{0}
\verb{moedura}{}{}{}{}{}{Porção que se mói a cada vez.}{mo.e.du.ra}{0}
\verb{moela}{é}{Zool.}{}{}{s.f.}{Parte do estômago de aves, insetos e moluscos que tritura os alimentos.}{mo.e.la}{0}
\verb{moenda}{}{}{}{}{s.f.}{Conjunto de peças que servem para moer; moinho.}{mo.en.da}{0}
\verb{moenda}{}{}{}{}{}{O local onde estão instaladas essas peças.}{mo.en.da}{0}
\verb{moer}{ê}{}{}{}{v.t.}{Esmagar ou triturar, reduzindo a pó.}{mo.er}{0}
\verb{moer}{ê}{}{}{}{}{Esmagar para extrair o suco.}{mo.er}{0}
\verb{moer}{ê}{}{}{}{}{Ruminar.}{mo.er}{\verboinum{17}}
\verb{mofa}{ó}{}{}{}{s.f.}{Ato de mofar; zombaria.}{mo.fa}{0}
\verb{mofa}{ó}{}{}{}{}{Indivíduo ou coisa que é objeto de zombaria.}{mo.fa}{0}
\verb{mofar}{}{}{}{}{v.t.}{Fazer caçoada; zombar.}{mo.far}{\verboinum{1}}
\verb{mofar}{}{}{}{}{v.t.}{Encher de mofo.}{mo.far}{0}
\verb{mofar}{}{Pop.}{}{}{v.i.}{Permanecer por muito tempo em um mesmo local ou posição, na espera de que algo aconteça.}{mo.far}{\verboinum{1}}
\verb{mofino}{}{}{}{}{adj.}{Azarado, infeliz.}{mo.fi.no}{0}
\verb{mofino}{}{}{}{}{}{Sovina, avarento.}{mo.fi.no}{0}
\verb{mofino}{}{Bras.}{}{}{}{Covarde, fraco.}{mo.fi.no}{0}
\verb{mofino}{}{Bras.}{}{}{}{Que vive doente; doentio.}{mo.fi.no}{0}
\verb{mofo}{ô}{Biol.}{}{}{s.m.}{Designação genérica dos fungos que aparecem nos alimentos ou em locais úmidos e mal ventilados; bolor.}{mo.fo}{0}
\verb{mogno}{ó}{Bot.}{}{}{s.m.}{Árvore de flores brancas que tem madeira nobre, resistente e de tom avermelhado.}{mog.no}{0}
\verb{moído}{}{}{}{}{adj.}{Esmagado, triturado.}{mo.í.do}{0}
\verb{moído}{}{Fig.}{}{}{}{Extremamente cansado; exausto.}{mo.í.do}{0}
\verb{moído}{}{Fig.}{}{}{}{Magoado, triste, aborrecido.}{mo.í.do}{0}
\verb{moinho}{}{}{}{}{s.m.}{Aparelho que serve para moer cereais, movido pelo vento, pela água ou por um motor.}{mo.i.nho}{0}
\verb{moirão}{}{}{}{}{}{Var. de \textit{mourão}.}{moi.rão}{0}
\verb{moirisco}{}{}{}{}{}{Var. de \textit{mourisco}.}{moi.ris.co}{0}
\verb{moiro}{ô}{}{}{}{}{Var. de \textit{mouro}.}{moi.ro}{0}
\verb{moita}{ô}{}{}{}{s.f.}{Porção espessa de plantas arbustivas ou arvorecentes.}{moi.ta}{0}
\verb{mojica}{}{Bras.}{}{}{s.f.}{Processo para engrossar o caldo, por meio de cozimento lento e prolongado.}{mo.ji.ca}{0}
\verb{mojica}{}{Cul.}{}{}{}{O caldo engrossado por esse processo.}{mo.ji.ca}{0}
\verb{mojica}{}{Cul.}{}{}{}{Peixe cozido e misturado com farinha, que engrossa o caldo.}{mo.ji.ca}{0}
\verb{mola}{ó}{}{}{}{s.f.}{Peça dotada de elasticidade, que tende a voltar a sua forma original quando comprimida ou esticada.}{mo.la}{0}
\verb{mola}{ó}{Fig.}{}{}{}{Aquilo que fornece o estímulo inicial para que algo ocorra.}{mo.la}{0}
\verb{molambento}{}{}{}{}{adj.}{Diz"-se de indivíduo esfarrapado ou sujo.}{mo.lam.ben.to}{0}
\verb{molambo}{}{}{}{}{s.m.}{Roupa ou pedaço de pano velho e esfarrapado.}{mo.lam.bo}{0}
\verb{molambo}{}{Fig.}{}{}{}{Indivíduo sem firmeza ou determinação.}{mo.lam.bo}{0}
\verb{molar}{}{}{}{}{adj.2g.}{Que tem a casca mole e fácil de partir. }{mo.lar}{0}
\verb{molar}{}{}{}{}{adj.2g.}{Que mói, próprio para moer.  }{mo.lar}{0}
\verb{molar}{}{}{}{}{s.m.}{Cada um dos dentes de coroa com superfície larga, que ficam situados depois dos caninos e são usados para triturar alimentos.}{mo.lar}{0}
\verb{moldador}{ô}{}{}{}{adj.}{Que molda ou faz moldes.}{mol.da.dor}{0}
\verb{moldagem}{}{}{"-ens}{}{s.f.}{Ato ou efeito de moldar; modelagem.}{mol.da.gem}{0}
\verb{moldar}{}{}{}{}{v.t.}{Fazer os moldes de.}{mol.dar}{0}
\verb{moldar}{}{}{}{}{}{Acomodar ao molde; amoldar.}{mol.dar}{0}
\verb{moldar}{}{}{}{}{}{Dar forma a; modelar.}{mol.dar}{0}
\verb{moldar}{}{}{}{}{v.pron.}{Acomodar"-se, conformar"-se, adaptar"-se.}{mol.dar}{\verboinum{1}}
\verb{moldávio}{}{}{}{}{adj.}{Relativo à República da Moldávia (Europa).}{mol.dá.vio}{0}
\verb{moldávio}{}{}{}{}{s.m.}{Indivíduo natural ou habitante dessa república.}{mol.dá.vio}{0}
\verb{molde}{ó}{}{}{}{s.m.}{Peça oca na qual se modelam obras de metal.}{mol.de}{0}
\verb{molde}{ó}{}{}{}{}{Peça de qualquer material que fornece os contornos para cortar tecido, plástico, madeira.}{mol.de}{0}
\verb{molde}{ó}{Fig.}{}{}{}{Aquilo que serve de modelo de comportamento.}{mol.de}{0}
\verb{moldura}{}{}{}{}{s.f.}{Peça lisa ou trabalhada, geralmente de madeira ou metal, usada para guarnecer e enfeitar o contorno de pinturas, retratos, espelhos.}{mol.du.ra}{0}
\verb{moldurar}{}{}{}{}{v.t.}{Colocar em moldura; emoldurar.}{mol.du.rar}{\verboinum{1}}
\verb{moldureiro}{ê}{}{}{}{s.m.}{Indivíduo que fabrica ou lida com molduras.}{mol.du.rei.ro}{0}
\verb{mole}{}{}{}{}{s.f.}{Grande massa sem forma.}{mo.le}{0}
\verb{mole}{}{}{}{}{adj.2g.}{Que cede à pressão; macio, tenro, flácido.}{mo.le}{0}
\verb{mole}{}{Fig.}{}{}{}{Sem energia; fraco, preguiçoso, lento.}{mo.le}{0}
\verb{mole}{}{Fig.}{}{}{}{Que não oferece dificuldade; fácil.}{mo.le}{0}
\verb{molecada}{}{Bras.}{}{}{s.f.}{Grupo de moleques.}{mo.le.ca.da}{0}
\verb{molecagem}{}{}{"-ens}{}{s.f.}{Ato ou dito de moleque.}{mo.le.ca.gem}{0}
\verb{molécula}{}{Quím.}{}{}{s.f.}{Representação da estrutura e composição de uma substância segundo a teoria atômica.}{mo.lé.cu.la}{0}
\verb{molecular}{}{}{}{}{adj.2g.}{Relativo a molécula.}{mo.le.cu.lar}{0}
\verb{moleira}{ê}{}{}{}{s.f.}{Mulher proprietária ou trabalhadora de moinho.}{mo.lei.ra}{0}
\verb{moleira}{ê}{Anat.}{}{}{s.f.}{Espaço membranoso no alto do crânio que não se encontra ainda totalmente ossificado em crianças muito novas; fontanela.}{mo.lei.ra}{0}
\verb{moleirão}{}{}{"-ões}{"-ona}{adj.}{Diz"-se de indivíduo muito molenga; molengão.}{mo.lei.rão}{0}
\verb{moleiro}{ê}{}{}{}{s.m.}{Indivíduo proprietário ou trabalhador de moinho.}{mo.lei.ro}{0}
\verb{molejo}{ê}{Bras.}{}{}{s.m.}{O resultado da ação das molas do sistema de suspensão de um veículo, tornando"-o confortável e macio.}{mo.le.jo}{0}
\verb{molejo}{ê}{Pop.}{}{}{}{Capacidade de requebrar o corpo, especialmente em danças.}{mo.le.jo}{0}
\verb{molenga}{}{}{}{}{adj.2g.}{Sem energia; preguiçoso, mole.}{mo.len.ga}{0}
\verb{molenga}{}{}{}{}{}{Covarde, medroso, frouxo.}{mo.len.ga}{0}
\verb{molengo}{}{Bras.}{}{}{adj.}{Molenga.}{mo.len.go}{0}
\verb{moleque}{é}{Bras.}{}{}{s.m.}{Garoto travesso e de pouca idade.}{mo.le.que}{0}
\verb{moleque}{é}{}{}{}{}{Menino de rua.}{mo.le.que}{0}
\verb{molestador}{ô}{}{}{}{adj.}{Que molesta.}{mo.les.ta.dor}{0}
\verb{molestar}{}{}{}{}{v.t.}{Importunar, incomodar, enfadar.}{mo.les.tar}{0}
\verb{molestar}{}{}{}{}{}{Tentar aproximação sexual de maneira inconveniente.}{mo.les.tar}{0}
\verb{molestar}{}{}{}{}{}{Causar moléstia; atacar, afetar.}{mo.les.tar}{0}
\verb{molestar}{}{}{}{}{}{Causar desgosto; magoar, ofender.}{mo.les.tar}{\verboinum{1}}
\verb{moléstia}{}{}{}{}{s.f.}{Sofrimento físico; doença, mal.}{mo.lés.tia}{0}
\verb{molesto}{é}{}{}{}{adj.}{Que aborrece; incômodo, enfadonho.}{mo.les.to}{0}
\verb{molesto}{é}{}{}{}{}{Diz"-se de trabalho penoso, árduo; trabalhoso.}{mo.les.to}{0}
\verb{moletom}{}{}{"-ons}{}{s.m.}{Tecido macio e quente, semelhante a uma flanela grossa.}{mo.le.tom}{0}
\verb{moleza}{ê}{}{}{}{s.f.}{Qualidade de mole.}{mo.le.za}{0}
\verb{moleza}{ê}{}{}{}{}{Falta de energia; preguiça, indolência.}{mo.le.za}{0}
\verb{moleza}{ê}{}{}{}{}{Situação que não requer esforço; ócio, facilidade.}{mo.le.za}{0}
\verb{molhadela}{é}{}{}{}{s.f.}{Ato de molhar rápida ou ligeiramente.}{mo.lha.de.la}{0}
\verb{molhadela}{é}{}{}{}{}{Banho rápido.}{mo.lha.de.la}{0}
\verb{molhado}{}{}{}{}{adj.}{Úmido ou embebido de qualquer líquido.}{mo.lha.do}{0}
\verb{molhados}{}{}{}{}{s.m.pl.}{Gêneros alimentícios líquidos.}{mo.lha.dos}{0}
\verb{molhar}{}{}{}{}{v.t.}{Umedecer ou embeber em líquido.}{mo.lhar}{\verboinum{1}}
\verb{molhe}{ó}{}{}{}{s.m.}{Tipo de paredão existente nos portos e que avança pelo mar para atenuar a sua força, servindo de abrigo a embarcações.}{mo.lhe}{0}
\verb{molheira}{ê}{}{}{}{s.f.}{Recipiente para servir molhos à mesa.}{mo.lhei.ra}{0}
\verb{molho}{ô}{Cul.}{"-s ⟨ô⟩}{}{s.m.}{Caldo mais ou menos grosso que se prepara com um ingrediente principal e temperos. (\textit{Há vários tipos de molho: à bolonhesa, inglês, branco etc.})}{mo.lho}{0}
\verb{molho}{ó}{}{"-s ⟨ó⟩}{}{s.m.}{Conjunto de coisas da mesma espécie, reunidas e amarradas ao comprido; maço.}{mo.lho}{0}
\verb{molho}{ó}{}{"-s ⟨ó⟩}{}{}{Conjunto de objetos reunidos e presos em alguma coisa. (\textit{Na pressa, deixei meu molho de chaves sobre a mesa.})}{mo.lho}{0}
\verb{molibdênio}{}{Quím.}{}{}{s.m.}{Elemento químico metálico, branco"-prateado, muito duro, denso, usado em aços e ligas com o ferro ou o níquel, na fabricação de mísseis, aviões, foguetes, eletrodos resistentes ao calor e à corrosão,  etc. \elemento{42}{95.94}{Mo}.}{mo.lib.dê.nio}{0}
\verb{molícia}{}{}{}{}{s.f.}{Molície.}{mo.lí.cia}{0}
\verb{molície}{}{}{}{}{s.f.}{Indolência, moleza, preguiça.}{mo.lí.cie}{0}
\verb{molície}{}{}{}{}{}{Sensualidade, languidez.}{mo.lí.cie}{0}
\verb{molinete}{ê}{}{}{}{s.m.}{Tipo de manivela que permite lançar uma linha a muitos metros de distância, usada em varas de pescar.}{mo.li.ne.te}{0}
\verb{moloide}{}{}{}{}{adj.2g.}{Molenga.}{mo.loi.de}{0}
\verb{molosso}{ô}{}{}{}{s.m.}{Grande cão de fila para caçar e guardar o gado; mastim.}{mo.los.so}{0}
\verb{molosso}{ô}{Fig.}{}{}{}{Indivíduo valentão; desordeiro.}{mo.los.so}{0}
\verb{molusco}{}{Zool.}{}{}{s.m.}{Filo de animais invertebrados, geralmente dotados de concha.}{mo.lus.co}{0}
\verb{momentâneo}{}{}{}{}{adj.}{Que dura um momento e depois desaparece; passageiro.}{mo.men.tâ.neo}{0}
\verb{momento}{}{}{}{}{s.m.}{Espaço pequeno de tempo.}{mo.men.to}{0}
\verb{momento}{}{}{}{}{}{Tempo em que alguma coisa acontece; ocasião.}{mo.men.to}{0}
\verb{momentoso}{ô}{}{"-osos ⟨ó⟩}{"-osa ⟨ó⟩}{adj.}{Grave, de grande importância para o momento, para a ocasião.}{mo.men.to.so}{0}
\verb{momice}{}{}{}{}{s.f.}{Trejeito, careta.}{mo.mi.ce}{0}
\verb{momo}{ô}{}{}{}{s.m.}{Rei do carnaval.}{mo.mo}{0}
\verb{monacal}{}{}{"-ais}{}{adj.2g.}{Relativo a monge ou convento.}{mo.na.cal}{0}
\verb{monarca}{}{}{}{}{s.2g.}{Chefe de uma monarquia; soberano.}{mo.nar.ca}{0}
\verb{monarquia}{}{}{}{}{s.f.}{Forma de governo em que o poder é exercido por um monarca, e passa de pai para filho. }{mo.nar.qui.a}{0}
\verb{monarquia}{}{}{}{}{}{Estado que tem essa forma de governo.}{mo.nar.qui.a}{0}
\verb{monárquico}{}{}{}{}{adj.}{Relativo a monarca ou a monarquia.}{mo.nár.qui.co}{0}
\verb{monarquismo}{}{}{}{}{s.m.}{Forma de governo na qual o poder máximo é exercido por um rei.}{mo.nar.quis.mo}{0}
\verb{monarquista}{}{}{}{}{s.2g.}{Indivíduo partidário da monarquia ou do sistema monárquico.}{mo.nar.quis.ta}{0}
\verb{monastério}{}{}{}{}{s.m.}{Convento de monges.}{mo.nas.té.rio}{0}
\verb{monástico}{}{}{}{}{adj.}{Relativo a monge ou convento.}{mo.nás.ti.co}{0}
\verb{monazita}{}{Quím.}{}{}{s.f.}{Mineral amarelado, composto de fosfato e vários metais raros, combinado com outros metais e que se encontra disseminado nas rochas eruptivas ou nas areias amarelas, ditas monazíticas.  }{mo.na.zi.ta}{0}
\verb{monazítico}{}{}{}{}{adj.}{Relativo à monazita; que contém monazita.}{mo.na.zí.ti.co}{0}
\verb{monção}{}{}{"-ões}{}{s.f.}{Vento que sopra do continente para o mar no inverno e do mar para o continente no verão.}{mon.ção}{0}
\verb{monção}{}{}{"-ões}{}{}{Boa oportunidade.}{mon.ção}{0}
\verb{monção}{}{}{"-ões}{}{}{Nome dado às expedições fluviais das capitanias de São Paulo e do Mato Grosso, nos séculos \textsc{xviii} e \textsc{xix}.}{mon.ção}{0}
\verb{mondar}{}{}{}{}{v.t.}{Arrancar o mato que nasce no meio de uma plantação; capinar, limpar.}{mon.dar}{\verboinum{1}}
\verb{monegasco}{}{}{}{}{adj.}{Relativo ao Principado de Mônaco.}{mo.ne.gas.co}{0}
\verb{monegasco}{}{}{}{}{s.m.}{Indivíduo natural ou habitante desse país.}{mo.ne.gas.co}{0}
\verb{monetário}{}{}{}{}{adj.}{Relativo a moeda.}{mo.ne.tá.rio}{0}
\verb{monge}{}{}{monja ⟨ô⟩}{}{s.m.}{Frade de mosteiro.}{mon.ge}{0}
\verb{mongol}{ó}{}{"-óis}{}{adj.2g.}{Relativo à Mongólia (Ásia).}{mon.gol}{0}
\verb{mongol}{ó}{}{"-óis}{}{s.2g.}{Indivíduo natural ou habitante desse país.}{mon.gol}{0}
\verb{mongol}{ó}{}{"-óis}{}{s.m.}{Língua falada pelos mongóis.}{mon.gol}{0}
\verb{mongólico}{}{}{}{}{adj.}{Relativo aos mongóis.}{mon.gó.li.co}{0}
\verb{mongolismo}{}{}{}{}{s.m.}{Distúrbio cromossômico que causa retardo mental e traços físicos característicos.}{mon.go.lis.mo}{0}
\verb{mongoloide}{}{}{}{}{adj.2g.}{Que sofre de mongolismo.}{mon.go.loi.de}{0}
\verb{monitor}{ô}{}{}{}{s.m.}{Pessoa que dá conselhos, lições.}{mo.ni.tor}{0}
\verb{monitor}{ô}{}{}{}{}{Aluno que auxilia o professor no ensino de uma matéria, em geral na aplicação de exercícios, no esclarecimento de dúvidas etc., fora das aulas regulares.}{mo.ni.tor}{0}
\verb{monitor}{ô}{}{}{}{}{Dispositivo de aparelho eletrônico que exibe informações sobre seu funcionamento; visor, \textit{display}, mostrador. }{mo.ni.tor}{0}
\verb{monitoração}{}{}{"-ões}{}{s.f.}{Acompanhamento, controle, observação, fiscalização.}{mo.ni.to.ra.ção}{0}
\verb{monitorar}{}{}{}{}{v.t.}{Acompanhar uma atividade com o objetivo de corrigir possíveis erros.}{mo.ni.to.rar}{\verboinum{1}}
\verb{monitoria}{}{}{}{}{s.f.}{Cargo ou função de monitor.}{mo.ni.to.ri.a}{0}
\verb{monitória}{}{}{}{}{adj.}{Diz"-se de carta judicial em que alguém é convidado a ir dizer o que souber sobre um fato.}{mo.ni.tó.ria}{0}
\verb{monitório}{}{}{}{}{adj.}{Que adverte, repreende.}{mo.ni.tó.rio}{0}
\verb{monitorizar}{}{}{}{}{v.t.}{Monitorar.}{mo.ni.to.ri.zar}{\verboinum{1}}
\verb{monja}{}{}{}{}{s.f.}{Religiosa que vive em convento.}{mon.ja}{0}
\verb{monjolo}{ô}{}{}{}{s.m.}{Engenho primitivo movido a água para pilar o milho e descascar o café.}{mon.jo.lo}{0}
\verb{monjolo}{ô}{}{}{}{}{Bezerro novo.}{mon.jo.lo}{0}
\verb{mono}{}{}{}{}{s.m.}{Macaco.}{mo.no}{0}
\verb{monobloco}{ó}{}{}{}{s.m.}{O que é fabricado em um só bloco.}{mo.no.blo.co}{0}
\verb{monociclo}{}{}{}{}{s.m.}{Veículo de uma roda, usado geralmente por malabaristas.}{mo.no.ci.clo}{0}
\verb{monocórdio}{}{}{}{}{adj.}{De um só tom; monótono.}{mo.no.cór.dio}{0}
\verb{monocórdio}{}{}{}{}{s.m.}{Instrumento musical de uma só corda.}{mo.no.cór.dio}{0}
\verb{monocotiledôneo}{}{Bot.}{}{}{adj.}{Que tem um só cotilédone.}{mo.no.co.ti.le.dô.neo}{0}
\verb{monocotiledôneo}{}{}{}{}{}{Relativo às monocotiledôneas.}{mo.no.co.ti.le.dô.neo}{0}
\verb{monocromático}{}{}{}{}{adj.}{Que tem uma só cor.}{mo.no.cro.má.ti.co}{0}
\verb{monóculo}{}{}{}{}{s.m.}{Lente para um só olho.}{mo.nó.cu.lo}{0}
\verb{monocultura}{}{}{}{}{s.f.}{Cultura de um só produto agrícola.}{mo.no.cul.tu.ra}{0}
\verb{monofobia}{}{}{}{}{s.f.}{Horror à solidão.}{mo.no.fo.bi.a}{0}
\verb{monogamia}{}{}{}{}{s.f.}{Sistema conjugal no qual não se pode ter mais de um cônjuge simultaneamente.}{mo.no.ga.mi.a}{0}
\verb{monogâmico}{}{}{}{}{adj.}{Relativo à monogamia.}{mo.no.gâ.mi.co}{0}
\verb{monógamo}{}{}{}{}{adj.}{Que tem uma só esposa.}{mo.nó.ga.mo}{0}
\verb{monógamo}{}{}{}{}{}{Diz"-se do animal que se acasala com uma só fêmea.}{mo.nó.ga.mo}{0}
\verb{monografia}{}{}{}{}{s.f.}{Estudo minucioso sobre algum tema restrito.}{mo.no.gra.fi.a}{0}
\verb{monograma}{}{}{}{}{s.m.}{Sigla formada por uma ou várias letras, conjuntas ou entralaçadas, significando um símbolo ou iniciais de um nome; cifra.}{mo.no.gra.ma}{0}
\verb{monólito}{}{}{}{}{s.m.}{Pedra de grandes dimensões.}{mo.nó.li.to}{0}
\verb{monólito}{}{}{}{}{}{Monumento feito de um único bloco de pedra.}{mo.nó.li.to}{0}
\verb{monologar}{}{}{}{}{v.i.}{Falar sozinho.}{mo.no.lo.gar}{\verboinum{5}}
\verb{monólogo}{}{}{}{}{s.m.}{Conversa de uma pessoa consigo mesma.}{mo.nó.lo.go}{0}
\verb{monólogo}{}{}{}{}{}{Cena de teatro em que o ator representa sozinho.}{mo.nó.lo.go}{0}
\verb{monomania}{}{}{}{}{s.f.}{Ideia fixa.}{mo.no.ma.ni.a}{0}
\verb{monomotor}{ô}{}{}{}{adj.}{Diz"-se de veículo de um só motor.}{mo.no.mo.tor}{0}
\verb{monoplano}{}{}{}{}{s.m.}{Aeroplano com apenas uma asa de cada lado.}{mo.no.pla.no}{0}
\verb{monopólio}{}{}{}{}{s.m.}{Privilégio exclusivo de vender, fabricar, explorar algum produto ou serviço.}{mo.no.pó.lio}{0}
\verb{monopolização}{}{}{"-ões}{}{s.f.}{Ato ou efeito de monopolizar; exclusividade.}{mo.no.po.li.za.ção}{0}
\verb{monopolizar}{}{}{}{}{v.t.}{Ter o monopólio de determinado produto.}{mo.no.po.li.zar}{0}
\verb{monopolizar}{}{}{}{}{}{Tomar alguma coisa apenas para si; centralizar, concentrar.}{mo.no.po.li.zar}{\verboinum{1}}
\verb{monossilábico}{}{}{}{}{adj.}{Que tem uma só sílaba.}{mo.nos.si.lá.bi.co}{0}
\verb{monossílabo}{}{Gram.}{}{}{adj.}{Que tem uma só sílaba. }{mo.nos.sí.la.bo}{0}
\verb{monoteísmo}{}{}{}{}{s.m.}{Crença em um só Deus.}{mo.no.te.ís.mo}{0}
\verb{monoteísta}{}{}{}{}{adj.2g.}{Que adora um só Deus.}{mo.no.te.ís.ta}{0}
\verb{monotipo}{}{}{}{}{s.m.}{Máquina de composição mecânica cujo teclado converte o texto em perfurações numa bobina de papel que, por sua vez, comanda o mecanismo de fundição dos caracteres tipográficos.}{mo.no.ti.po}{0}
\verb{monotonia}{}{}{}{}{s.f.}{Falta de variedade de sons.}{mo.no.to.ni.a}{0}
\verb{monotonia}{}{}{}{}{}{Falta de acontecimentos novos.}{mo.no.to.ni.a}{0}
\verb{monótono}{}{}{}{}{adj.}{Que não muda de tom; invariável, uniforme.}{mo.nó.to.no}{0}
\verb{monótono}{}{}{}{}{}{Que aborrece por se repetir; desinteressante, enfadonho.}{mo.nó.to.no}{0}
\verb{monovalente}{}{}{}{}{adj.2g.}{Que só tem uma valência; univalente.}{mo.no.va.len.te}{0}
\verb{monóxido}{cs}{Quím.}{}{}{s.m.}{Óxido que contém um átomo de oxigênio.}{mo.nó.xi.do}{0}
\verb{monsenhor}{ô}{}{}{}{s.m.}{Título honorífico concedido pelo Papa a alguns sacerdotes.}{mon.se.nhor}{0}
\verb{monstrengo}{}{}{}{}{s.m.}{Indivíduo deformado ou muito feio; mostrengo.}{mons.tren.go}{0}
\verb{monstrengo}{}{}{}{}{}{Algo desproporcional, disforme.}{mons.tren.go}{0}
\verb{monstro}{}{}{}{}{s.m.}{Ser fantástico de mau aspecto.}{mons.tro}{0}
\verb{monstro}{}{}{}{}{}{Indivíduo perverso.}{mons.tro}{0}
\verb{monstro}{}{}{}{}{}{Aberração.}{mons.tro}{0}
\verb{monstruosidade}{}{}{}{}{s.f.}{Qualidade de monstruoso.}{mons.tru.o.si.da.de}{0}
\verb{monstruosidade}{}{}{}{}{}{Coisa extraordinária ou abominável.}{mons.tru.o.si.da.de}{0}
\verb{monstruoso}{ô}{}{"-osos ⟨ó⟩}{"-osa ⟨ó⟩}{adj.}{Que tem um aspecto que mete muito medo; horroroso, medonho.}{mons.tru.o.so}{0}
\verb{monstruoso}{ô}{}{"-osos ⟨ó⟩}{"-osa ⟨ó⟩}{}{De uma maldade muito grande.}{mons.tru.o.so}{0}
\verb{monstruoso}{ô}{}{"-osos ⟨ó⟩}{"-osa ⟨ó⟩}{}{De tamanho gigante; colossal, enorme.}{mons.tru.o.so}{0}
\verb{monta}{}{}{}{}{s.f.}{Quantia, montante; o valor total de uma conta.}{mon.ta}{0}
\verb{monta}{}{}{}{}{}{O preço, o custo de alguma coisa.}{mon.ta}{0}
\verb{montada}{}{}{}{}{s.f.}{Parte elevada e curva do freio por baixo do qual a cavalgadura passa a língua.}{mon.ta.da}{0}
\verb{montada}{}{}{}{}{}{Ato de montar.}{mon.ta.da}{0}
\verb{montada}{}{}{}{}{}{Cavalgadura sobre a qual uma pessoa está montada.}{mon.ta.da}{0}
\verb{montado}{}{}{}{}{adj.}{Posto sobre um cavalo ou sobre outro animal.}{mon.ta.do}{0}
\verb{montado}{}{}{}{}{}{Colocado ou posto ao jeito de cavaleiro.}{mon.ta.do}{0}
\verb{montador}{ô}{}{}{}{adj.}{Que faz montagens.}{mon.ta.dor}{0}
\verb{montador}{ô}{Informát.}{}{}{}{Categoria de programa que lê um programa"-fonte e transcreve instruções e símbolos para seus correspondentes em linguagem de máquina.}{mon.ta.dor}{0}
\verb{montadora}{ô}{}{}{}{s.f.}{Fábrica de produtos resultantes de uma linha de montagem.}{mon.ta.do.ra}{0}
\verb{montagem}{}{}{"-ens}{}{s.f.}{Ato ou efeito de montar.}{mon.ta.gem}{0}
\verb{montagem}{}{}{"-ens}{}{}{Preparação das peças de um maquinismo, para que funcione.}{mon.ta.gem}{0}
\verb{montagem}{}{}{"-ens}{}{}{Encenação de uma peça teatral.}{mon.ta.gem}{0}
\verb{montagem}{}{}{"-ens}{}{}{Seleção e organização de materiais de um filme, de um programa de rádio ou televisão, de uma gravação em disco ou fita etc.}{mon.ta.gem}{0}
\verb{montanha}{}{}{}{}{s.f.}{Alta elevação natural da superfície da Terra.}{mon.ta.nha}{0}
\verb{montanha}{}{}{}{}{}{Grande quantidade.}{mon.ta.nha}{0}
\verb{montanha"-russa}{}{}{montanhas"-russas}{}{s.f.}{Rede de trilhos, com descidas e curvas bruscas, percorrida por vagões em alta velocidade.}{mon.ta.nha"-rus.sa}{0}
\verb{montanhês}{}{}{}{"-esa ⟨ê⟩}{adj.2g.}{Que vive nas montanhas.}{mon.ta.nhês}{0}
\verb{montanhismo}{}{}{}{}{s.m.}{Esporte que se pratica subindo montanhas; alpinismo.}{mon.ta.nhis.mo}{0}
\verb{montanhista}{}{}{}{}{s.2g.}{Indivíduo que pratica o montanhismo; alpinista.}{mon.ta.nhis.ta}{0}
\verb{montanhoso}{ô}{}{"-osos ⟨ó⟩}{"-osa ⟨ó⟩}{adj.}{Cheio de montanhas.}{mon.ta.nho.so}{0}
\verb{montante}{}{}{}{}{s.m.}{Quantidade total; importância, soma.}{mon.tan.te}{0}
\verb{montão}{}{}{"-ões}{}{s.m.}{Grande quantidade de coisas; amontoado.}{mon.tão}{0}
\verb{montar}{}{}{}{}{v.t.}{Reunir, uma por uma, as peças de alguma coisa para que ela fique pronta para ser usada; armar. }{mon.tar}{0}
\verb{montar}{}{}{}{}{}{Preparar um espetáculo para o apresentar ao público }{mon.tar}{0}
\verb{montar}{}{}{}{}{}{Chegar a uma certa soma ou quantia.}{mon.tar}{0}
\verb{montar}{}{}{}{}{}{Colocar (alguém) em cima de um animal.}{mon.tar}{\verboinum{1}}
\verb{montaria}{}{}{}{}{s.f.}{Animal em que se monta; cavalgadura.}{mon.ta.ri.a}{0}
\verb{montaria}{}{}{}{}{}{Pequeno barco feito de um tronco; canoa.}{mon.ta.ri.a}{0}
\verb{monte}{}{}{}{}{s.m.}{Elevação de terreno; morro.}{mon.te}{0}
\verb{monte}{}{}{}{}{}{Grande quantidade de coisas; acúmulo, amontoado.}{mon.te}{0}
\verb{montepio}{}{}{}{}{s.m.}{Organização em que os membros pagam uma cota mensal para garantir uma pensão em caso de morte.}{mon.te.pi.o}{0}
\verb{montês}{}{}{}{}{adj.2g.}{Que vive nas montanhas.}{mon.tês}{0}
\verb{montesinho}{}{}{}{}{adj. e s.m.  }{Montês.}{mon.te.si.nho}{0}
\verb{montesino}{}{}{}{}{adj. e s.m.  }{Montês.}{mon.te.si.no}{0}
\verb{montevideano}{}{}{}{}{adj.}{Relativo a Montevidéu, capital do Uruguai.}{mon.te.vi.de.a.no}{0}
\verb{montevideano}{}{}{}{}{s.m.}{Indivíduo natural ou habitante de Montevidéu.}{mon.te.vi.de.a.no}{0}
\verb{montículo}{}{Geogr.}{}{}{s.m.}{Pequeno monte.}{mon.tí.cu.lo}{0}
\verb{montículo}{}{}{}{}{}{Pequena quantidade de qualquer coisa amontoada.}{mon.tí.cu.lo}{0}
\verb{montoeira}{ê}{}{}{}{s.f.}{Grande amontoado de coisas.}{mon.to.ei.ra}{0}
\verb{montra}{}{}{}{}{s.f.}{Vitrina de casa comercial.}{mon.tra}{0}
\verb{montra}{}{Mús.}{}{}{}{Fachada de órgão, na qual se acham os tubos mais vistosos desse instrumento.}{mon.tra}{0}
\verb{monturo}{}{}{}{}{s.m.}{Lugar onde se deposita lixo.}{mon.tu.ro}{0}
\verb{monumental}{}{}{"-ais}{}{adj.2g.}{Relativo a monumento.}{mo.nu.men.tal}{0}
\verb{monumental}{}{}{"-ais}{}{}{Grandioso, magnífico, esplêndido, soberbo.}{mo.nu.men.tal}{0}
\verb{monumento}{}{}{}{}{s.m.}{Construção feita para lembrar fato ou pessoa importante. }{mo.nu.men.to}{0}
\verb{monumento}{}{}{}{}{}{Edifício majestoso, imponente.}{mo.nu.men.to}{0}
\verb{monumento}{}{}{}{}{}{Obra de grande valor.}{mo.nu.men.to}{0}
\verb{moquear}{}{}{}{}{v.t.}{Secar (carne, peixe etc.) no moquém.}{mo.que.ar}{0}
\verb{moquear}{}{}{}{}{}{Assar a carne no moquém.}{mo.que.ar}{\verboinum{4}}
\verb{moqueca}{é}{}{}{}{s.f.}{Guisado de peixe, frutos do mar, com leite de coco, azeite"-de"-dendê e diversos temperos, preparado e servido numa panela de barro.}{mo.que.ca}{0}
\verb{moquém}{}{Bras.}{}{}{s.m.}{Grelha de varas, de origem indígena, usada para secar ou assar ligeiramente peixe ou carne.}{mo.quém}{0}
\verb{mor}{ó}{}{}{}{adj.2g.}{Forma reduzida de \textit{maior}.}{mor}{0}
\verb{mora}{ó}{}{}{}{s.f.}{Demora, adiamento, prorrogação.}{mo.ra}{0}
\verb{mora}{ó}{}{}{}{}{Quantia que se paga a mais em uma dívida pelo atraso no seu pagamento. (\textit{Juros de mora.})}{mo.ra}{0}
\verb{morada}{}{}{}{}{s.f.}{Lugar em que se mora; habitação, casa.}{mo.ra.da}{0}
\verb{morada}{}{}{}{}{}{Lugar em que uma coisa existe habitualmente.}{mo.ra.da}{0}
\verb{moradia}{}{}{}{}{s.f.}{Morada.}{mo.ra.di.a}{0}
\verb{moradia}{}{}{}{}{}{Tempo em que se morou em um lugar.}{mo.ra.di.a}{0}
\verb{morador}{ô}{}{}{}{s.m.}{Pessoa que mora; habitante.}{mo.ra.dor}{0}
\verb{moral}{}{}{"-ais}{}{s.m.}{Estado de espírito; ânimo, disposição.}{mo.ral}{0}
\verb{moral}{}{}{"-ais}{}{s.f.}{Parte da Filosofia que estuda os costumes, deveres e o comportamento dos homens em sociedade; ética.}{mo.ral}{0}
\verb{moral}{}{}{"-ais}{}{}{Conjunto de preceitos estabelecidos pela sociedade ou por um grupo social.}{mo.ral}{0}
\verb{moral}{}{}{"-ais}{}{}{Ensinamento que se tira na conclusão de uma história.}{mo.ral}{0}
\verb{moral}{}{}{"-ais}{}{adj.2g.}{Relativo à moral; ético.}{mo.ral}{0}
\verb{moralidade}{}{}{}{}{s.f.}{Qualidade do que é moral, do que segue os princípios da moral.}{mo.ra.li.da.de}{0}
\verb{moralismo}{}{}{}{}{s.m.}{Sistema ou comportamento filosófico que se ocupa exclusivamente da moral.}{mo.ra.lis.mo}{0}
\verb{moralismo}{}{}{}{}{}{Adesão total a uma determinada moral.}{mo.ra.lis.mo}{0}
\verb{moralista}{}{}{}{}{adj.2g.}{Diz"-se do indivíduo que prega os preceitos morais.}{mo.ra.lis.ta}{0}
\verb{moralização}{}{}{"-ões}{}{s.f.}{Ato ou efeito de moralizar.}{mo.ra.li.za.ção}{0}
\verb{moralizar}{}{}{}{}{v.t.}{Adequar um comportamento às regras da moral.}{mo.ra.li.zar}{\verboinum{1}}
\verb{moranga}{}{}{}{}{s.f.}{Variedade de abóbora.}{mo.ran.ga}{0}
\verb{morango}{}{}{}{}{s.m.}{Fruto do morangueiro.}{mo.ran.go}{0}
\verb{morangueiro}{ê}{}{}{}{s.m.}{Planta de origem europeia, cujo fruto, comestível, é o morango.}{mo.ran.guei.ro}{0}
\verb{morar}{}{}{}{}{v.i.}{Ocupar permanentemente uma residência; ter domicílio; residir, habitar.}{mo.rar}{\verboinum{1}}
\verb{moratória}{}{Jur.}{}{}{s.f.}{Adiamento do prazo de pagamento de uma dívida concedida ao devedor por um tribunal ou uma autoridade competente.}{mo.ra.tó.ria}{0}
\verb{moratório}{}{}{}{}{adj.}{Que envolve demora; que retarda.}{mo.ra.tó.rio}{0}
\verb{morbidez}{ê}{}{}{}{s.f.}{Qualidade ou estado de mórbido; morbidade.}{mor.bi.dez}{0}
\verb{mórbido}{}{}{}{}{adj.}{Que está doente; enfermo.}{mór.bi.do}{0}
\verb{mórbido}{}{}{}{}{}{Relativo a doença, enfermidade.}{mór.bi.do}{0}
\verb{morbo}{ô}{Med.}{}{}{s.m.}{Estado de quem apresenta alguma patologia; doença, enfermidade, moléstia.}{mor.bo}{0}
\verb{morcego}{ê}{}{}{}{s.m.}{Nome comum aos mamíferos voadores, de hábitos noturnos, com várias espécies que se alimentam de insetos ou frutos e algumas que se alimentam de sangue.}{mor.ce.go}{0}
\verb{morcego}{ê}{Pop.}{}{}{}{Pessoa que só sai à noite.}{mor.ce.go}{0}
\verb{morcela}{é}{}{}{}{s.f.}{Tipo de chouriço, feito com sangue e miúdos de porco; morcilha. }{mor.ce.la}{0}
\verb{mordaça}{}{}{}{}{s.f.}{Pano ou outro objeto usado para tapar a boca de uma pessoa, impedindo"-a de falar ou gritar.}{mor.da.ça}{0}
\verb{mordaça}{}{}{}{}{}{Cabresto que se põe no focinho dos animais para evitar que eles mordam ou comam; açaimo.}{mor.da.ça}{0}
\verb{mordacidade}{}{}{}{}{s.f.}{Qualidade de mordaz.}{mor.da.ci.da.de}{0}
\verb{mordacidade}{}{}{}{}{}{Crítica dura, severa.}{mor.da.ci.da.de}{0}
\verb{mordaz}{}{}{}{}{adj.2g.}{Que morde.}{mor.daz}{0}
\verb{mordaz}{}{}{}{}{}{Que corrói; cáustico, corrosivo.}{mor.daz}{0}
\verb{mordaz}{}{}{}{}{}{Que ironiza ou critica duramente, com muito rigor; satírico, maledicente, mordente.}{mor.daz}{0}
\verb{mordedor}{ô}{}{}{}{adj.}{Que morde.}{mor.de.dor}{0}
\verb{mordedor}{ô}{}{}{}{s.m.}{Pequeno brinquedo anatômico de borracha que é dado para os bebês morderem na época do nascimento dos dentes.}{mor.de.dor}{0}
\verb{mordedor}{ô}{Pop.}{}{}{}{Pessoa que tem o hábito de pedir dinheiro emprestado aos amigos e conhecidos.}{mor.de.dor}{0}
\verb{mordedura}{}{}{}{}{s.f.}{Ato ou efeito de morder; dentada.}{mor.de.du.ra}{0}
\verb{mordedura}{}{}{}{}{}{Marca de dentada.}{mor.de.du.ra}{0}
\verb{mordente}{}{}{}{}{adj.2g.}{Que morde.}{mor.den.te}{0}
\verb{mordente}{}{}{}{}{}{Mordaz.}{mor.den.te}{0}
\verb{mordente}{}{}{}{}{s.m.}{Preparado usado para fixar cores.}{mor.den.te}{0}
\verb{morder}{ê}{}{}{}{v.t.}{Apertar ou ferir com os dentes; dar dentadas.}{mor.der}{\verboinum{12}}
\verb{mordida}{}{}{}{}{s.f.}{Ato ou efeito de morder; mordedura, dentada.}{mor.di.da}{0}
\verb{mordiscar}{}{}{}{}{v.t.}{Morder de leve ou repetidamente.}{mor.dis.car}{\verboinum{2}}
\verb{mordomia}{}{}{}{}{s.f.}{Função exercida por mordomo. (\textit{A mordomia do palácio era exercida por funcionários especialmente contratados.})}{mor.do.mi.a}{0}
\verb{mordomia}{}{Bras.}{}{}{}{Vantagens ou facilidades próprias de determinados cargos. (\textit{Em regime de contenção de despesas, a primeira coisa a ser eliminada são as mordomias.})}{mor.do.mi.a}{0}
\verb{mordomia}{}{Pop.}{}{}{}{Bem"-estar, regalia ou conforto que se desfruta sem ter de pagar ou sem despender esforço. (\textit{Uma parcela dos políticos que ocupam altos cargos tem mordomias que ninguém conseguiria pagar.})}{mor.do.mi.a}{0}
\verb{mordomo}{}{}{}{}{s.m.}{Chefe dos empregados, encarregado de administrar e cuidar de uma casa.}{mor.do.mo}{0}
\verb{moreia}{é}{Zool.}{}{}{s.f.}{Nome comum aos peixes de corpo alongado e quase cilíndrico, semelhante a uma enguia, dotado de muitos dentes.}{mo.rei.a}{0}
\verb{moreno}{}{}{}{}{adj.}{Que tem a cor (da pele e dos cabelos) um pouco escura; trigueiro.}{mo.re.no}{0}
\verb{moreno}{}{}{}{}{s.m.}{Pessoa morena.}{mo.re.no}{0}
\verb{morfeia}{é}{Med.}{}{}{s.f.}{Lepra, hanseníase.}{mor.fei.a}{0}
\verb{morfema}{ê}{}{}{}{s.m.}{A menor unidade linguística com significado (prefixo, radical, conjunção etc.).}{mor.fe.ma}{0}
\verb{morfético}{}{}{}{}{adj.}{Relativo a morfeia (hanseníase); leproso.}{mor.fé.ti.co}{0}
\verb{morfético}{}{}{}{}{s.m.}{Pessoa que sofre desse mal; leproso.}{mor.fé.ti.co}{0}
\verb{morfina}{}{}{}{}{s.f.}{Alcaloide extraído do ópio, usado como medicamento para diminuir ou cessar a dor.}{mor.fi.na}{0}
\verb{morfologia}{}{Biol.}{}{}{}{Estudo da forma ou estrutura dos seres vivos.}{mor.fo.lo.gi.a}{0}
\verb{morfologia}{}{Gram.}{}{}{s.f.}{Estudo dos processos de formação das palavras e da sua constituição e classificação.}{mor.fo.lo.gi.a}{0}
\verb{morfológico}{}{}{}{}{adj.}{Relativo a morfologia.}{mor.fo.ló.gi.co}{0}
\verb{morgadio}{}{}{}{}{adj.}{Relativo a morgado.}{mor.ga.di.o}{0}
\verb{morgadio}{}{}{}{}{s.m.}{Bens ou qualidade de morgado.}{mor.ga.di.o}{0}
\verb{morgado}{}{}{}{}{s.m.}{Propriedades ou bens vinculados que passam para o filho primogênito em caso de morte do possuidor.}{mor.ga.do}{0}
\verb{morgado}{}{}{}{}{}{Filho primogênito, ou filho único.}{mor.ga.do}{0}
\verb{morgue}{}{}{}{}{s.f.}{Necrotério.}{mor.gue}{0}
\verb{moribundo}{}{}{}{}{adj.}{Que está morrendo ou agonizando.}{mo.ri.bun.do}{0}
\verb{moribundo}{}{}{}{}{s.m.}{Pessoa que está prestes a morrer.}{mo.ri.bun.do}{0}
\verb{morigerado}{}{}{}{}{adj.}{Que demonstra bons costumes; que leva uma vida exemplar ou irrepreensível.}{mo.ri.ge.ra.do}{0}
\verb{morigerar}{}{}{}{}{v.t.}{Moderar os costumes; ensinar bons costumes.}{mo.ri.ge.rar}{\verboinum{1}}
\verb{morim}{}{}{"-ins}{}{s.m.}{Pano branco e fino, feito de algodão.}{mo.rim}{0}
\verb{moringa}{}{}{}{}{s.f.}{Vaso de barro bojudo e de gargalo estreito, que se usa para acondicionar e refrescar a água; bilha.}{mo.rin.ga}{0}
\verb{moringue}{}{}{}{}{s.m.}{Moringa.}{mo.rin.gue}{0}
\verb{mormaceira}{ê}{}{}{}{s.f.}{Mormaço forte.}{mor.ma.cei.ra}{0}
\verb{mormacento}{}{}{}{}{adj.}{Diz"-se do tempo abafado, quente e úmido.}{mor.ma.cen.to}{0}
\verb{mormaço}{}{}{}{}{s.m.}{Tempo abafado, quente e úmido.}{mor.ma.ço}{0}
\verb{mormente}{}{}{}{}{adv.}{Principalmente, maiormente, sobretudo.}{mor.men.te}{0}
\verb{mormo}{ô}{Med.}{}{}{s.m.}{Doença de equídeos, causada por um bacilo, contagiosa, e que é transmissível ao homem.}{mor.mo}{0}
\verb{mórmon}{}{}{}{}{adj.2g.}{Que segue o mormonismo.}{mór.mon}{0}
\verb{mórmon}{}{}{}{}{s.2g.}{Pessoa adepta ou seguidora do mormonismo.}{mór.mon}{0}
\verb{mormonismo}{}{}{}{}{s.m.}{Doutrina religiosa norte"-americana, protestante, fundada em 1827 por Joseph Smith (1805--1844) e posteriormente denominada Igreja de Jesus Cristo dos Santos dos Últimos Dias.}{mor.mo.nis.mo}{0}
\verb{morno}{ô}{}{mornos ⟨ó⟩}{morna ⟨ó⟩}{adj.}{Pouco quente; tépido, cálido.}{mor.no}{0}
\verb{morno}{ô}{Fig.}{mornos ⟨ó⟩}{morna ⟨ó⟩}{}{Sem graça; insípido, monótono.}{mor.no}{0}
\verb{morosidade}{}{}{}{}{s.f.}{Qualidade de moroso; lentidão, demora, vagareza.}{mo.ro.si.da.de}{0}
\verb{moroso}{ô}{}{"-osos ⟨ó⟩}{"-osa ⟨ó⟩}{adj.}{Que anda ou faz as coisas com lentidão; lento, vagaroso.}{mo.ro.so}{0}
\verb{moroso}{ô}{}{"-osos ⟨ó⟩}{"-osa ⟨ó⟩}{}{Difícil ou custoso de fazer.}{mo.ro.so}{0}
\verb{morra}{ô}{}{}{}{interj.}{Exprime o desejo de que alguma coisa acabe, de que alguém seja morto ou afastado de um cargo.}{mor.ra}{0}
\verb{morrão}{}{}{"-ões}{}{s.m.}{Extremidade queimada de mecha ou pavio.}{mor.rão}{0}
\verb{morrão}{}{}{"-ões}{}{}{Grão apodrecido na espiga.}{mor.rão}{0}
\verb{morrer}{ê}{}{}{}{v.i.}{Perder a vida; falecer.}{mor.rer}{0}
\verb{morrer}{ê}{}{}{}{}{Deixar de existir; terminar, acabar; extinguir.}{mor.rer}{0}
\verb{morrer}{ê}{Fig.}{}{}{}{Sentir alguma coisa com grande intensidade; desejar. (\textit{Morro de vontade de conhecer a Europa!})}{mor.rer}{0}
\verb{morrer}{ê}{}{}{}{}{Parar de funcionar. (\textit{O problema do meu carro é que morria quando eu freava nas esquinas.})}{mor.rer}{\verboinum{12}}
\verb{morrinha}{}{}{}{}{s.f.}{Sarna epidêmica do gado.}{mor.ri.nha}{0}
\verb{morrinha}{}{Bras.}{}{}{}{Mau cheiro de pessoa ou animal; fedor, bodum, catinga.}{mor.ri.nha}{0}
\verb{morrinha}{}{Fig.}{}{}{}{Tristeza, melancolia, desânimo.}{mor.ri.nha}{0}
\verb{morrinhento}{}{}{}{}{adj.}{Que tem morrinha.}{mor.ri.nhen.to}{0}
\verb{morro}{ô}{}{}{}{s.m.}{Pequena elevação de terreno, menor que um monte; colina, outeiro.}{mor.ro}{0}
\verb{morro}{ô}{}{}{}{}{Favela.}{mor.ro}{0}
\verb{morsa}{ó}{}{}{}{s.f.}{Nome comum aos mamíferos marinhos encontrados no oceano Ártico, que chegam a medir 3 m de comprimento e a pesar mais de uma tonelada, e que são dotados de duas grandes presas superiores.}{mor.sa}{0}
\verb{mortadela}{é}{}{}{}{s.f.}{Tipo de salame feito de carne de porco ou de boi, temperado, que se come frio.}{mor.ta.de.la}{0}
\verb{mortal}{}{}{"-ais}{}{adj.2g.}{Que está sujeito à morte, que morrerá um dia.}{mor.tal}{0}
\verb{mortal}{}{}{"-ais}{}{}{Que causa a morte, que mata; letal, fatal.}{mor.tal}{0}
\verb{mortal}{}{}{"-ais}{}{}{Que tem vida curta; efêmero, transitório, passageiro. }{mor.tal}{0}
\verb{mortal}{}{}{"-ais}{}{s.m.}{O ser humano; a espécie humana.}{mor.tal}{0}
\verb{mortalha}{}{}{}{}{s.f.}{Pano com que se envolve o cadáver que será enterrado.}{mor.ta.lha}{0}
\verb{mortalha}{}{}{}{}{}{Tira de papel, de palha etc. que serve para embrulhar o fumo para fazer o cigarro.}{mor.ta.lha}{0}
\verb{mortalidade}{}{}{}{}{s.f.}{Qualidade ou condição do que é mortal.}{mor.ta.li.da.de}{0}
\verb{mortalidade}{}{}{}{}{}{Quantidade de seres que morrem em determinado lugar num certo espaço de tempo; obituário.}{mor.ta.li.da.de}{0}
\verb{mortandade}{}{}{}{}{s.f.}{Grande matança de homens ou animais; carnificina, extermínio, morticínio, chacina. }{mor.tan.da.de}{0}
\verb{morte}{ó}{}{}{}{s.f.}{Ato de morrer; fim da vida.}{mor.te}{0}
\verb{morte}{ó}{}{}{}{}{Fim, desaparecimento, destruição.}{mor.te}{0}
\verb{morte}{ó}{}{}{}{}{Grande sofrimento; grande pesar.}{mor.te}{0}
\verb{morteiro}{ê}{}{}{}{s.m.}{Canhão curto e de boca larga.}{mor.tei.ro}{0}
\verb{morticínio}{}{}{}{}{s.m.}{Mortandade.}{mor.ti.cí.nio}{0}
\verb{mortiço}{}{}{}{}{adj.}{Que está próximo de se apagar ou de se extinguir; tíbio, amortecido.}{mor.ti.ço}{0}
\verb{mortiço}{}{}{}{}{}{Sem brilho; embaciado, apagado.}{mor.ti.ço}{0}
\verb{mortífero}{}{}{}{}{adj.}{Que causa a morte; mortal.}{mor.tí.fe.ro}{0}
\verb{mortificação}{}{}{"-ões}{}{s.f.}{Ato ou efeito de mortificar.   }{mor.ti.fi.ca.ção}{0}
\verb{mortificação}{}{}{"-ões}{}{}{Sofrimento, tormento, desgosto.}{mor.ti.fi.ca.ção}{0}
\verb{mortificação}{}{}{"-ões}{}{}{Flagelação, tortura.}{mor.ti.fi.ca.ção}{0}
\verb{mortificar}{}{}{}{}{v.t.}{Atormentar, afligir; causar desgosto.}{mor.ti.fi.car}{0}
\verb{mortificar}{}{}{}{}{}{Torturar o corpo com penitências; penitenciar.}{mor.ti.fi.car}{\verboinum{2}}
\verb{morto}{ô}{}{"-s ⟨ó⟩}{"-a ⟨ó⟩}{adj.}{Que morreu, perdeu a vida; defunto, falecido.}{mor.to}{0}
\verb{morto}{ô}{}{"-s ⟨ó⟩}{"-a ⟨ó⟩}{}{Que alguém matou.}{mor.to}{0}
\verb{morto}{ô}{Fig.}{"-s ⟨ó⟩}{"-a ⟨ó⟩}{}{Que está muito desejoso de algo; ávido.}{mor.to}{0}
\verb{morto}{ô}{}{"-s ⟨ó⟩}{"-a ⟨ó⟩}{s.m.}{Pessoa que morreu; defunto, falecido.}{mor.to}{0}
\verb{mortuário}{}{}{}{}{adj.}{Que se refere a morte ou ao morto; fúnebre, funerário, funéreo.}{mor.tu.á.rio}{0}
\verb{morubixaba}{ch}{Bras.}{}{}{s.m.}{Chefe, cacique nas nações indígenas do Brasil.}{mo.ru.bi.xa.ba}{0}
\verb{mosaico}{}{}{}{}{s.m.}{Conjunto de pedrinhas ou ladrilhos coloridos, reunidos para formar um desenho.}{mo.sai.co}{0}
\verb{mosaico}{}{}{}{}{adj.}{Relativo a ou próprio de Moisés, profeta do Velho Testamento.}{mo.sai.co}{0}
\verb{mosca}{ô}{Zool.}{}{}{s.f.}{Nome comum a vários insetos de duas asas.}{mos.ca}{0}
\verb{moscadeira}{ê}{Bot.}{}{}{s.f.}{Árvore que produz a noz"-moscada; noz"-moscada.}{mos.ca.dei.ra}{0}
\verb{moscado}{}{}{}{}{adj.}{Que tem cheiro forte; aromático, odorífero, almiscarado.}{mos.ca.do}{0}
\verb{mosca"-morta}{ô\ldots{}ó}{Pop.}{moscas"-mortas ⟨ô\ldots{}ó⟩}{}{s.2g.}{Pessoa sem ânimo, sem vida.}{mos.ca"-mor.ta}{0}
\verb{mosca"-morta}{ô\ldots{}ó}{Pop.}{moscas"-mortas ⟨ô\ldots{}ó⟩}{}{}{Pessoa dissimulada, que finge inocência.}{mos.ca"-mor.ta}{0}
\verb{moscar}{}{}{}{}{v.i.}{Fugir das moscas.}{mos.car}{0}
\verb{moscar}{}{Fig.}{}{}{}{Não perceber, não se dar conta de certas coisas; comer mosca.}{mos.car}{0}
\verb{moscar}{}{}{}{}{v.pron.}{Desaparecer, sumir.}{mos.car}{\verboinum{2}}
\verb{moscardo}{}{}{}{}{s.m.}{Mosca grande; mutuca.}{mos.car.do}{0}
\verb{moscatel}{é}{}{"-éis}{}{adj.2g.}{Diz"-se de uma variedade de uva, muito doce.}{mos.ca.tel}{0}
\verb{moscatel}{é}{}{"-éis}{}{s.m.}{Vinho feito dessa uva.}{mos.ca.tel}{0}
\verb{moscovita}{}{}{}{}{adj.2g.}{De Moscou (Rússia).}{mos.co.vi.ta}{0}
\verb{moscovita}{}{}{}{}{s.2g.}{Pessoa natural ou habitante de Moscou.}{mos.co.vi.ta}{0}
\verb{mosqueado}{}{}{}{}{adj.}{Salpicado de pintas; pintalgado, sarapintado.}{mos.que.a.do}{0}
\verb{mosquear}{}{}{}{}{v.t.}{Salpicar de pintas ou de manchas.}{mos.que.ar}{0}
\verb{mosquear}{}{}{}{}{v.i.}{Afugentar as moscas, abanando ou por qualquer outro meio.}{mos.que.ar}{\verboinum{4}}
\verb{mosquetão}{}{Bras.}{"-ões}{}{s.m.}{Fuzil de cano curto, usado na artilharia e na cavalaria.}{mos.que.tão}{0}
\verb{mosquetão}{}{}{"-ões}{}{s.m.}{O elo de metal que une o relógio à corrente. }{mos.que.tão}{0}
\verb{mosquetão}{}{}{"-ões}{}{}{Elo de metal, alumínio etc. que une as cordas ou outros mosquetões, empregado no montanhismo.}{mos.que.tão}{0}
\verb{mosquetaria}{}{}{}{}{s.f.}{Grande quantidade de mosquetes ou de mosqueteiros.}{mos.que.ta.ri.a}{0}
\verb{mosquetaria}{}{}{}{}{}{Série de tiros de qualquer arma de fogo.}{mos.que.ta.ri.a}{0}
\verb{mosquete}{ê}{}{}{}{s.m.}{Arma de fogo antiga, muito pesada, que se disparava apoiada numa forquilha.    }{mos.que.te}{0}
\verb{mosqueteiro}{ê}{Desus.}{}{}{s.m.}{Soldado armado de mosquete.}{mos.que.tei.ro}{0}
\verb{mosquiteiro}{ê}{}{}{}{s.m.}{Tela ou rede muito fina usada em torno da cama para impedir a entrada de mosquitos e outros insetos.}{mos.qui.tei.ro}{0}
\verb{mosquito}{}{Zool.}{}{}{s.m.}{Nome comum aos insetos pequenos, de duas asas, especialmente os que se alimentam de sangue, e que podem transmitir doenças ao homem; mosca, pernilongo, carapanã, muriçoca.}{mos.qui.to}{0}
\verb{mossa}{ó}{}{}{}{s.f.}{Marca de pancada ou pressão no corpo.}{mos.sa}{0}
\verb{mostarda}{}{}{}{}{s.f.}{Verdura de folhas largas e enrugadas, com sabor picante; mostardeira.}{mos.tar.da}{0}
\verb{mostarda}{}{}{}{}{}{Semente dessa verdura.}{mos.tar.da}{0}
\verb{mostarda}{}{}{}{}{}{Pasta de sabor picante que se prepara com essa semente. (\textit{Gosto muito de cachorro"-quente com mostarda.})}{mos.tar.da}{0}
\verb{mostardeira}{ê}{}{}{}{s.f.}{Mostarda.}{mos.tar.dei.ra}{0}
\verb{mosteiro}{ê}{}{}{}{s.m.}{Lugar onde vivem os monges ou as monjas.}{mos.tei.ro}{0}
\verb{mosto}{ô}{}{}{}{s.m.}{Sumo de uvas, antes de concluída a fermentação.}{mos.to}{0}
\verb{mostra}{ó}{}{}{}{s.f.}{Ato ou efeito de mostrar.}{mos.tra}{0}
\verb{mostra}{ó}{}{}{}{}{Manifestação, demonstração, sinal. (\textit{Isso que ele fez é uma mostra da sua boa educação.})}{mos.tra}{0}
\verb{mostra}{ó}{}{}{}{}{Exposição; apresentação. (\textit{A artista fará uma mostra do seu trabalho em aquarela.})}{mos.tra}{0}
\verb{mostrador}{ô}{}{}{}{adj.}{Que mostra.}{mos.tra.dor}{0}
\verb{mostrador}{ô}{}{}{}{s.m.}{Armário ou balcão em que são expostas, nas lojas, as mercadorias para venda; mostruário.}{mos.tra.dor}{0}
\verb{mostrador}{ô}{}{}{}{}{Dispositivo de aparelho eletrônico que exibe informações sobre seu funcionamento; visor, monitor, \textit{display}.}{mos.tra.dor}{0}
\verb{mostrador}{ô}{}{}{}{}{A parte do relógio em que estão os ponteiros e os números.}{mos.tra.dor}{0}
\verb{mostrar}{}{}{}{}{v.t.}{Fazer ver; exibir.}{mos.trar}{0}
\verb{mostrar}{}{}{}{}{}{Apontar, indicar.}{mos.trar}{0}
\verb{mostrar}{}{}{}{}{}{Demonstrar, provar, manifestar.}{mos.trar}{\verboinum{1}}
\verb{mostrengo}{}{}{}{}{s.m.}{Pessoa que apresenta disformidade ou é muito feia; monstrengo, monstro.}{mos.tren.go}{0}
\verb{mostruário}{}{}{}{}{s.m.}{Armário, balcão ou pasta em que são expostas as mercadorias para venda; mostrador.}{mos.tru.á.rio}{0}
\verb{mote}{ó}{}{}{}{s.m.}{Frase, ditado curto que faz rir; gracejo.}{mo.te}{0}
\verb{mote}{ó}{}{}{}{}{Assunto; tema.}{mo.te}{0}
\verb{motejar}{}{}{}{}{v.t.}{Fazer motes; satirizar, escarnecer, zombar.}{mo.te.jar}{0}
\verb{motejar}{}{}{}{}{}{Criar ou sugerir mote(s).}{mo.te.jar}{\verboinum{1}}
\verb{motejo}{ê}{}{}{}{s.m.}{Dito ou comentário zombeteiro ou satírico; motete, zombaria, caçoada, gracejo, troça.}{mo.te.jo}{0}
\verb{motel}{é}{}{"-éis}{}{s.m.}{Hotel de beira de estrada, destinado a motoristas e viajantes em trânsito.}{mo.tel}{0}
\verb{motel}{é}{}{"-éis}{}{}{Hotel de curta permanência, geralmente para encontro de casais.}{mo.tel}{0}
\verb{motete}{ê}{}{}{}{s.m.}{Dito engraçado ou zombeteiro; motejo.}{mo.te.te}{0}
\verb{motete}{ê}{}{}{}{}{Composição poética para ser cantada com música.}{mo.te.te}{0}
\verb{motilidade}{}{}{}{}{s.f.}{Faculdade de mover; mobilidade.}{mo.ti.li.da.de}{0}
\verb{motim}{}{}{"-ins}{}{s.m.}{Revolta contra uma autoridade, visando perturbar a ordem; rebelião, barulho, desordem. (\textit{Houve vários motins no Instituto de Menores.})}{mo.tim}{0}
\verb{motivação}{}{}{"-ões}{}{s.f.}{Ato ou efeito de motivar.}{mo.ti.va.ção}{0}
\verb{motivação}{}{}{"-ões}{}{}{Interesse espontâneo; estímulo, impulso.}{mo.ti.va.ção}{0}
\verb{motivar}{}{}{}{}{v.t.}{Dar motivo; provocar, causar.}{mo.ti.var}{0}
\verb{motivar}{}{}{}{}{}{Apresentar motivo; justificar, fundamentar.}{mo.ti.var}{0}
\verb{motivar}{}{}{}{}{}{Incentivar, estimular, impulsionar.}{mo.ti.var}{\verboinum{1}}
\verb{motivo}{}{}{}{}{s.m.}{Fato que leva alguém a fazer alguma atividade; causa, razão.}{mo.ti.vo}{0}
\verb{motivo}{}{}{}{}{}{Fim, intuito, intuição.}{mo.ti.vo}{0}
\verb{moto}{ó}{}{}{}{s.f.}{Forma reduzida de \textit{motocicleta}.}{mo.to}{0}
\verb{moto}{ó}{}{}{}{s.m.}{Ato ou efeito de mover; movimento, circulação.}{mo.to}{0}
\verb{motoca}{ó}{Pop.}{}{}{s.f.}{Motocicleta.}{mo.to.ca}{0}
\verb{motocicleta}{é}{}{}{}{s.f.}{Veículo de duas rodas, semelhante a bicicleta, mas acionado por motor a explosão.}{mo.to.ci.cle.ta}{0}
\verb{motociclismo}{}{Esport.}{}{}{s.m.}{Competição de motociclistas.}{mo.to.ci.clis.mo}{0}
\verb{motociclista}{}{}{}{}{adj.2g.}{Que dirige motocicleta.}{mo.to.ci.clis.ta}{0}
\verb{motociclista}{}{Esport.}{}{}{s.2g.}{Indivíduo que pratica motociclismo.}{mo.to.ci.clis.ta}{0}
\verb{motociclo}{}{}{}{}{s.m.}{Bicicleta munida de pequeno motor.}{mo.to.ci.clo}{0}
\verb{moto"-contínuo}{}{}{motos"-contínuos}{}{s.m.}{Máquina ideal que, teoricamente, funcionaria indefinidamente sem gastar energia ou transformando em trabalho toda a energia recebida.}{mo.to"-con.tí.nu.o}{0}
\verb{motocross}{}{Esport.}{}{}{s.m.}{Competição de motocicletas em terreno muito acidentado.}{\textit{motocross}}{0}
\verb{motoneta}{ê}{}{}{}{s.f.}{Veículo motorizado de duas rodas, semelhante a motocicleta, mas com assento em vez de selim; lambreta.}{mo.to.ne.ta}{0}
\verb{motoqueiro}{ê}{}{}{}{s.m.}{Motociclista.}{mo.to.quei.ro}{0}
\verb{motor}{ô}{}{}{do adj.: motriz ou motora ⟨ô⟩}{adj.}{Relativo a movimento.}{mo.tor}{0}
\verb{motor}{ô}{}{}{do adj.: motriz ou motora ⟨ô⟩}{}{Que faz mover ou gera movimento.}{mo.tor}{0}
\verb{motor}{ô}{}{}{do adj.: motriz ou motora ⟨ô⟩}{s.m.}{Mecanismo que movimenta um veículo ou máquina. }{mo.tor}{0}
\verb{motorista}{}{}{}{}{s.2g.}{Indivíduo que dirige um veículo motorizado; chofer.}{mo.to.ris.ta}{0}
\verb{motorizado}{}{}{}{}{adj.}{Provido de ou acionado por motor.}{mo.to.ri.za.do}{0}
\verb{motorizado}{}{}{}{}{}{Que dispõe de automóvel.}{mo.to.ri.za.do}{0}
\verb{motorizar}{}{}{}{}{v.t.}{Prover de motor.}{mo.to.ri.zar}{0}
\verb{motorizar}{}{}{}{}{}{Adquirir veículo motorizado.}{mo.to.ri.zar}{\verboinum{1}}
\verb{motorneiro}{ê}{}{}{}{s.m.}{Indivíduo encarregado de regular o motor do bonde.}{mo.tor.nei.ro}{0}
\verb{motosserra}{é}{}{}{}{s.f.}{Serra portátil acionada por motor.}{mo.tos.ser.ra}{0}
\verb{motriz}{}{}{}{}{adj.}{Diz"-se da força que produz movimento.}{mo.triz}{0}
\verb{mouco}{ô}{}{}{}{adj.}{Que não ouve nada ou ouve pouco; surdo.}{mou.co}{0}
\verb{mourão}{}{}{"-ões}{}{s.m.}{Estaca grossa que se finca no chão para sustentar alguma coisa; esteio. }{mou.rão}{0}
\verb{mourejar}{}{}{}{}{v.i.}{Trabalhar sem descanso.}{mou.re.jar}{\verboinum{1}}
\verb{mourisco}{}{}{}{}{adj.}{Relativo a mouro.}{mou.ris.co}{0}
\verb{mourisco}{}{}{}{}{s.m.}{Mouro que permaneceu na Península Ibérica após a reconquista. }{mou.ris.co}{0}
\verb{mouro}{ô}{}{}{}{adj.}{Relativo aos mouros; mourisco.}{mou.ro}{0}
\verb{mouro}{ô}{Hist.}{}{}{s.m.}{Indivíduo do povo árabe muçulmano originário da África do Norte, antiga Mauritânia, que conquistou a Península Ibérica.}{mou.ro}{0}
\verb{mouro}{ô}{Fig.}{}{}{}{Indivíduo que trabalha muito. }{mou.ro}{0}
\verb{mouse}{}{Informát.}{}{}{s.m.}{Acessório do computador que permite movimentar o cursor na tela.}{\textit{mouse}}{0}
\verb{movediço}{}{}{}{}{adj.}{Que se move com facilidade; instável.}{mo.ve.di.ço}{0}
\verb{móvel}{}{}{"-eis}{}{adj.2g.}{Que se pode mover; movediço.}{mó.vel}{0}
\verb{móvel}{}{}{"-eis}{}{s.m.}{Peça de mobília.}{mó.vel}{0}
\verb{movelaria}{}{}{}{}{s.f.}{Estabelecimento onde se fazem ou se vendem móveis.}{mo.ve.la.ri.a}{0}
\verb{moveleiro}{ê}{}{}{}{s.m.}{Fabricante ou vendedor de móveis.}{mo.ve.lei.ro}{0}
\verb{mover}{ê}{}{}{}{v.t.}{Dar movimento a; movimentar.}{mo.ver}{0}
\verb{mover}{ê}{}{}{}{}{Deslocar de um lugar a outro.}{mo.ver}{0}
\verb{mover}{ê}{}{}{}{}{Induzir, instigar, incitar.}{mo.ver}{\verboinum{12}}
\verb{movimentação}{}{}{"-ões}{}{s.f.}{Ato ou efeito de movimentar"-se; movimento.}{mo.vi.men.ta.ção}{0}
\verb{movimentado}{}{}{}{}{adj.}{Que possui movimento.}{mo.vi.men.ta.do}{0}
\verb{movimentado}{}{}{}{}{}{Agitado, animado, movido.}{mo.vi.men.ta.do}{0}
\verb{movimentar}{}{}{}{}{v.t.}{Pôr em movimento; mover.}{mo.vi.men.tar}{0}
\verb{movimentar}{}{}{}{}{}{Dar movimento; animar.}{mo.vi.men.tar}{\verboinum{1}}
\verb{movimento}{}{}{}{}{s.m.}{Ato ou efeito de movimentar; movimentação.}{mo.vi.men.to}{0}
\verb{movimento}{}{}{}{}{}{Mudança de um corpo do local ou da posição em que se encontra para outro.}{mo.vi.men.to}{0}
\verb{movimento}{}{}{}{}{}{Agitação, animação, alvoroço.}{mo.vi.men.to}{0}
\verb{movimento}{}{}{}{}{}{Atividade organizada para se conseguir um objetivo; campanha. (\textit{O ano de 1984 foi marcado pelo movimento das Diretas"-Já.})}{mo.vi.men.to}{0}
\verb{moviola}{ó}{}{}{}{s.f.}{Equipamento usado na edição de filmes.}{mo.vi.o.la}{0}
\verb{movível}{}{}{"-eis}{}{adj.2g.}{Que se pode mover; móvel.}{mo.ví.vel}{0}
\verb{moxa}{ôch}{}{}{}{s.f.}{Pequeno cone composto de plantas usado na medicina chinesa para cauterizar feridas ou como anestésico local.}{mo.xa}{0}
\verb{moxinifada}{ch}{}{}{}{s.f.}{Mistura de coisas; confusão, miscelânea.}{mo.xi.ni.fa.da}{0}
\verb{mozarela}{é}{}{}{}{s.f.}{Forma aportuguesada de \textit{mozzarella}.}{mo.za.re.la}{0}
\verb{mozzarella}{}{}{}{}{s.f.}{Tipo de queijo de origem italiana, usado como complemento em vários pratos quentes e frios; mozarela, muçarela.}{\textit{mozzarella}}{0}
\verb{MS}{}{}{}{}{}{Sigla do estado do Mato Grosso do Sul.}{MS}{0}
\verb{MT}{}{}{}{}{}{Sigla do estado do Mato Grosso.}{MT}{0}
\verb{mu}{}{Zool.}{}{mula}{s.m.}{Filho de jumento e égua ou de cavalo e jumenta; mulo, burro.}{mu}{0}
\verb{muamba}{}{}{}{}{s.f.}{Comércio ilícito de mercadorias; contrabando.}{mu.am.ba}{0}
\verb{muamba}{}{}{}{}{}{Mercadoria contrabandeada.}{mu.am.ba}{0}
\verb{muamba}{}{}{}{}{}{Ato de má"-fe; fraude.}{mu.am.ba}{0}
\verb{muambeiro}{ê}{}{}{}{adj.}{Diz"-se do indivíduo que faz muambas.}{mu.am.bei.ro}{0}
\verb{muar}{}{}{}{}{adj.2g.}{Relativo a ou da raça das mulas.}{mu.ar}{0}
\verb{muar}{}{}{}{}{s.2g.}{Besta, mula ou mulo.}{mu.ar}{0}
\verb{mucama}{}{}{}{}{s.f.}{Escrava negra escolhida para fazer os serviços caseiros, como acompanhar a senhora em passeios ou servir de ama de leite.}{mu.ca.ma}{0}
\verb{muçarela}{é}{}{}{}{s.f.}{Forma aportuguesada de \textit{mozzarella}.}{mu.ça.re.la}{0}
\verb{mucilagem}{}{Bot.}{"-ens}{}{s.f.}{Substância gelatinosa excretada por certas plantas que reage com a água, aumentando de volume.}{mu.ci.la.gem}{0}
\verb{mucilaginoso}{ô}{}{"-osos ⟨ó⟩}{"-osa ⟨ó⟩}{adj.}{Que contém ou apresenta mucilagem.}{mu.ci.la.gi.no.so}{0}
\verb{muco}{}{}{}{}{s.m.}{Substância viscosa excretada pelas glândulas mucosas; mucosidade.}{mu.co}{0}
\verb{mucosa}{ó}{Anat.}{}{}{s.f.}{Membrana que recobre algumas cavidades do corpo e segrega muco.}{mu.co.sa}{0}
\verb{mucosidade}{}{}{}{}{s.f.}{Muco.}{mu.co.si.da.de}{0}
\verb{mucoso}{ô}{}{"-osos ⟨ó⟩}{"-osa ⟨ó⟩}{adj.}{Que produz ou tem a natureza do muco.}{mu.co.so}{0}
\verb{muçulmanismo}{}{Relig.}{}{}{s.m.}{Islamismo.}{mu.çul.ma.nis.mo}{0}
\verb{muçulmano}{}{}{}{}{adj.}{Relativo ao muçulmanismo ou islamismo; maometano, islâmico.}{mu.çul.ma.no}{0}
\verb{muçulmano}{}{}{}{}{s.m.}{Indivíduo seguidor do Islamismo; maometano, islamita.}{mu.çul.ma.no}{0}
\verb{muçum}{}{Zool.}{"-uns}{}{s.m.}{Peixe de água doce, de corpo esguio e comprido, e desprovido de escamas, semelhante a enguia.}{mu.çum}{0}
\verb{muçurana}{}{Zool.}{}{}{s.f.}{Cobra não venenosa que se alimenta de pequenos animais, inclusive de outras cobras.}{mu.çu.ra.na}{0}
\verb{muçurana}{}{}{}{}{}{Corda utilizada pelos índios para amarrar os prisioneiros.}{mu.çu.ra.na}{0}
\verb{muda}{}{}{}{}{s.f.}{Ato ou efeito de mudar; mudança.}{mu.da}{0}
\verb{muda}{}{}{}{}{}{Troca de pele, penas ou pelos em certos animais.}{mu.da}{0}
\verb{muda}{}{}{}{}{}{Troca de roupa.}{mu.da}{0}
\verb{muda}{}{}{}{}{}{Planta que se muda de canteiro.}{mu.da}{0}
\verb{mudança}{}{}{}{}{s.f.}{Ato ou efeito de mudar.}{mu.dan.ça}{0}
\verb{mudança}{}{}{}{}{}{Conjunto dos móveis e utensílios de uma casa que estão sendo transferidos de residência.}{mu.dan.ça}{0}
\verb{mudar}{}{}{}{}{v.t.}{Modificar, alterar.}{mu.dar}{0}
\verb{mudar}{}{}{}{}{}{Colocar em outro lugar; deslocar.}{mu.dar}{0}
\verb{mudar}{}{}{}{}{}{Desviar.}{mu.dar}{0}
\verb{mudar}{}{}{}{}{}{Trocar, permutar, substituir.}{mu.dar}{0}
\verb{mudar}{}{}{}{}{v.i.}{Tornar"-se diferente; alterar"-se.}{mu.dar}{0}
\verb{mudar}{}{}{}{}{v.pron.}{Transferir"-se.}{mu.dar}{\verboinum{1}}
\verb{mudável}{}{}{"-eis}{}{adj.2g.}{Que pode ser mudado; sujeito a mudança.}{mu.dá.vel}{0}
\verb{mudável}{}{Fig.}{"-eis}{}{}{Volúvel.}{mu.dá.vel}{0}
\verb{mudez}{ê}{}{}{}{s.f.}{Qualidade de mudo.}{mu.dez}{0}
\verb{mudo}{}{}{}{}{adj.}{Incapaz de falar por defeito fisiológico ou psicológico.}{mu.do}{0}
\verb{mudo}{}{}{}{}{}{Que não fala ou não fala muito; calado, silencioso.}{mu.do}{0}
\verb{mudo}{}{Gram.}{}{}{}{Diz"-se de letra que não representa som na fala.}{mu.do}{0}
\verb{mugido}{}{}{}{}{s.m.}{A voz da vaca e dos bovinos em geral.}{mu.gi.do}{0}
\verb{mugir}{}{}{}{}{v.i.}{Dar mugidos.}{mu.gir}{0}
\verb{mugir}{}{}{}{}{}{Dar gritos semelhantes a mugidos; berrar.}{mu.gir}{0}
\verb{mugir}{}{}{}{}{}{Fazer barulho ou estrondo forte.}{mu.gir}{\verboinum{22}}
\verb{mui}{}{}{}{}{adv.}{Forma reduzida de \textit{muito}, empregada antes de adjetivos ou advérbios terminados em \textit{"-mente}.}{mui}{0}
\verb{muiraquitã}{}{Bras.}{}{}{s.m.}{Figura talhada em pedra ou madeira, representando pessoas ou animais, à qual atribuem"-se poderes sobrenaturais.}{mui.ra.qui.tã}{0}
\verb{muito}{}{}{}{}{pron.}{Em grande quantidade ou intensidade. (\textit{Muito conhecimento.})}{mui.to}{0}
\verb{muito}{}{}{}{}{adv.}{Com intensidade; em alto grau. (\textit{Ele come muito.})}{mui.to}{0}
\verb{mula}{}{}{}{}{s.f.}{Fêmea do burro.}{mu.la}{0}
\verb{mula"-sem"-cabeça}{ê}{Bras.}{mulas"-sem"-cabeça}{}{s.f.}{Criatura fantástica que em vida foi amante de padre e agora, transformada em mula, sai correndo pela noite.}{mu.la"-sem"-ca.be.ça}{0}
\verb{mulatinho}{}{}{}{}{s.m.}{Certa variedade de feijão.}{mu.la.ti.nho}{0}
\verb{mulato}{}{}{}{}{adj.}{Mestiço de etnias branca e negra; pardo.}{mu.la.to}{0}
\verb{muleta}{ê}{}{}{}{s.f.}{Bastão que serve de apoio a pessoas com problemas de locomoção.}{mu.le.ta}{0}
\verb{muleta}{ê}{Fig.}{}{}{}{Qualquer coisa que serve de amparo, sustentação.}{mu.le.ta}{0}
\verb{mulher}{é}{}{}{}{s.f.}{Ser humano do sexo feminino.}{mu.lher}{0}
\verb{mulher}{é}{}{}{}{}{Ser humano do sexo feminino após a puberdade.}{mu.lher}{0}
\verb{mulher}{é}{}{}{}{}{Esposa, senhora.}{mu.lher}{0}
\verb{mulheraça}{}{Pop.}{}{}{s.f.}{Mulher alta e forte, ou fisicamente atraente.}{mu.lhe.ra.ça}{0}
\verb{mulherão}{}{Pop.}{"-ões}{}{s.m.}{Mulheraça.}{mu.lhe.rão}{0}
\verb{mulherão}{}{}{"-ões}{}{}{Mulher admirável.}{mu.lhe.rão}{0}
\verb{mulherengo}{}{}{}{}{adj.}{Homem galanteador.}{mu.lhe.ren.go}{0}
\verb{mulheril}{}{}{"-is}{}{adj.2g.}{Relativo a mulher ou ao sexo feminino.}{mu.lhe.ril}{0}
\verb{mulherio}{}{}{}{}{s.m.}{Grupo de mulheres.}{mu.lhe.ri.o}{0}
%\verb{}{}{}{}{}{}{}{}{0}
\verb{mulherona}{}{Bras.}{}{}{s.f.}{Mulheraça.}{mu.lhe.ro.na}{0}
\verb{mulo}{}{}{}{}{s.m.}{Burro.}{mu.lo}{0}
\verb{multa}{}{}{}{}{s.f.}{Pena pecuniária.}{mul.ta}{0}
\verb{multar}{}{}{}{}{v.t.}{Aplicar multa a.}{mul.tar}{\verboinum{1}}
\verb{multicelular}{}{}{}{}{adj.2g.}{Formado por mais de uma célula; pluricelular.}{mul.ti.ce.lu.lar}{0}
\verb{multicolor}{ô}{}{}{}{adj.2g.}{Que tem muitas cores; multicolorido.}{mul.ti.co.lor}{0}
\verb{multicor}{ô}{}{}{}{adj.2g.}{Multicolor.}{mul.ti.cor}{0}
\verb{multidão}{}{}{"-ões}{}{s.f.}{Grande quantidade de pessoas.}{mul.ti.dão}{0}
\verb{multidão}{}{}{"-ões}{}{}{Abundância, profusão.}{mul.ti.dão}{0}
\verb{multifário}{}{}{}{}{adj.}{Que apresenta vários aspectos; variado.}{mul.ti.fá.rio}{0}
\verb{multiforme}{ó}{}{}{}{adj.2g.}{Que tem ou apresenta muitas formas; polimorfo.}{mul.ti.for.me}{0}
\verb{multilateral}{}{}{"-ais}{}{adj.2g.}{Que efetivamente envolve várias pessoas, organizações ou nações.}{mul.ti.la.te.ral}{0}
\verb{multilíngue}{}{}{}{}{adj.2g.}{Que tem muitas línguas.}{mul.ti.lí.ngue}{0}
\verb{multilíngue}{}{}{}{}{}{Que fala muitas línguas; poliglota.}{mul.ti.lí.ngue}{0}
\verb{multimídia}{}{}{}{}{adj.2g.}{Que envolve meios de comunicação de diferentes naturezas.}{mul.ti.mí.dia}{0}
\verb{multimídia}{}{Informát.}{}{}{s.f.}{Equipamento necessário para a apresentação de texto, som e imagem em um computador.}{mul.ti.mí.dia}{0}
\verb{multimilionário}{}{}{}{}{adj.}{Que tem ou envolve muitos milhões em valores.}{mul.ti.mi.li.o.ná.rio}{0}
\verb{multinacional}{}{}{"-ais}{}{adj.2g.}{Relativo a muitos países.}{mul.ti.na.ci.o.nal}{0}
\verb{multinacional}{}{}{"-ais}{}{}{Que envolve muitos países.}{mul.ti.na.ci.o.nal}{0}
\verb{multinacional}{}{}{"-ais}{}{s.f.}{Organização que atua em vários países.}{mul.ti.na.ci.o.nal}{0}
\verb{multiplicação}{}{}{"-ões}{}{s.f.}{Ato ou efeito de multiplicar.}{mul.ti.pli.ca.ção}{0}
\verb{multiplicação}{}{Mat.}{"-ões}{}{}{Operação básica em que se calcula o valor da soma de \textit{n} parcelas de um número \textit{x}.}{mul.ti.pli.ca.ção}{0}
\verb{multiplicação}{}{Biol.}{"-ões}{}{}{Reprodução assexual.}{mul.ti.pli.ca.ção}{0}
\verb{multiplicador}{ô}{}{}{}{adj.}{Que multiplica.}{mul.ti.pli.ca.dor}{0}
\verb{multiplicador}{ô}{Mat.}{}{}{s.m.}{Em uma operação de multiplicação, o fator que indica o número de vezes que o outro fator é somado.}{mul.ti.pli.ca.dor}{0}
\verb{multiplicando}{}{Mat.}{}{}{s.m.}{Em uma operação de multiplicação, o fator que é somado repetidas vezes.}{mul.ti.pli.can.do}{0}
\verb{multiplicar}{}{}{}{}{v.t.}{Aumentar significativamente a quantidade de.}{mul.ti.pli.car}{0}
\verb{multiplicar}{}{}{}{}{}{Produzir muito do mesmo; proliferar.}{mul.ti.pli.car}{0}
\verb{multiplicar}{}{Mat.}{}{}{}{Realizar a operação de multiplicação.}{mul.ti.pli.car}{0}
\verb{multiplicar}{}{}{}{}{v.pron.}{Aumentar de quantidade ocupando um grande espaço; proliferar"-se.}{mul.ti.pli.car}{\verboinum{2}}
\verb{multiplicativo}{}{}{}{}{adj.}{Relativo a multiplicação.}{mul.ti.pli.ca.ti.vo}{0}
\verb{multiplicativo}{}{}{}{}{}{Que se repete muitas vezes.}{mul.ti.pli.ca.ti.vo}{0}
\verb{multíplice}{}{}{}{}{adj.2g.}{Que se manifesta de diversas maneiras; variado.}{mul.tí.pli.ce}{0}
\verb{multiplicidade}{}{}{}{}{s.f.}{Grande quantidade; abundância.}{mul.ti.pli.ci.da.de}{0}
\verb{múltiplo}{}{}{}{}{adj.}{Que se refere a uma grande quantidade de coisas.}{múl.ti.plo}{0}
\verb{multissecular}{}{}{}{}{adj.2g.}{Que existe há muitos séculos.}{mul.tis.se.cu.lar}{0}
\verb{multissecular}{}{}{}{}{}{Muito antigo.}{mul.tis.se.cu.lar}{0}
\verb{múmia}{}{}{}{}{s.f.}{Cadáver embalsamado.}{mú.mia}{0}
\verb{múmia}{}{Fig.}{}{}{}{Pessoa muito magra ou fraca.}{mú.mia}{0}
\verb{mumificação}{}{}{"-ões}{}{s.f.}{Ato ou efeito de mumificar.}{mu.mi.fi.ca.ção}{0}
\verb{mumificar}{}{}{}{}{v.t.}{Transformar em múmia; embalsamar.}{mu.mi.fi.car}{\verboinum{2}}
\verb{mundana}{}{}{}{}{s.f.}{Prostituta.}{mun.da.na}{0}
\verb{mundanismo}{}{}{}{}{s.m.}{Qualidade de mundano.}{mun.da.nis.mo}{0}
\verb{mundano}{}{}{}{}{adj.}{Relativo ao mundo, em seu aspecto material.}{mun.da.no}{0}
\verb{mundano}{}{}{}{}{}{Dado a prazeres materiais.}{mun.da.no}{0}
\verb{mundão}{}{}{"-ões}{}{s.m.}{Grande extensão de terras.}{mun.dão}{0}
\verb{mundão}{}{}{"-ões}{}{}{Grande quantidade de qualquer coisa.}{mun.dão}{0}
\verb{mundaréu}{}{Bras.}{"-éis}{}{s.m.}{Mundão.}{mun.da.réu}{0}
\verb{mundial}{}{}{"-ais}{}{adj.2g.}{Relativo ao mundo.}{mun.di.al}{0}
\verb{mundial}{}{}{"-ais}{}{}{Que envolve o mundo inteiro.}{mun.di.al}{0}
\verb{mundícia}{}{Bras.}{}{}{s.f.}{Mundície.}{mun.dí.cia}{0}
\verb{mundície}{}{Bras.}{}{}{s.f.}{Limpeza, esmero, asseio.}{mun.dí.cie}{0}
\verb{mundo}{}{}{}{}{s.m.}{Conjunto das coisas que existem.}{mun.do}{0}
\verb{mundo}{}{}{}{}{}{O globo terrestre; Terra.}{mun.do}{0}
\verb{mundo}{}{}{}{}{}{O gênero humano.}{mun.do}{0}
\verb{mundo}{}{}{}{}{}{Grande quantidade de qualquer coisa.}{mun.do}{0}
\verb{mundo}{}{}{}{}{}{Os prazeres materiais.}{mun.do}{0}
%\verb{}{}{}{}{}{}{}{}{0}
%\verb{}{}{}{}{}{}{}{}{0}
\verb{mungir}{}{}{}{}{v.t.}{Ordenhar.}{mun.gir}{\verboinum{22}}
\verb{mungunzá}{}{Cul.}{}{}{s.m.}{Doce feito geralmente com milho branco, açúcar, leite e canela.}{mun.gun.zá}{0}
\verb{munguzá}{}{}{}{}{}{Var. de \textit{mungunzá}.}{mun.gu.zá}{0}
\verb{munheca}{é}{}{}{}{s.f.}{A parte do corpo que faz a junção da mão com o antebraço; pulso.}{mu.nhe.ca}{0}
\verb{munheca}{é}{Pop.}{}{}{}{Indivíduo pouco ou nada generoso.}{mu.nhe.ca}{0}
\verb{munição}{}{}{"-ões}{}{s.f.}{Material com que se carrega uma arma de fogo.}{mu.ni.ção}{0}
\verb{municiar}{}{}{}{}{v.t.}{Prover de munição.}{mu.ni.ci.ar}{\verboinum{6}}
\verb{municionar}{}{}{}{}{v.t.}{Abastecer de munição; municiar. }{mu.ni.ci.o.nar}{\verboinum{1}}
\verb{municipal}{}{}{"-ais}{}{adj.2g.}{Relativo ao município.}{mu.ni.ci.pal}{0}
\verb{municipal}{}{}{"-ais}{}{s.m.}{Qualquer teatro pertencente à municipalidade.}{mu.ni.ci.pal}{0}
\verb{municipalidade}{}{}{}{}{s.f.}{Câmara municipal; prefeitura.}{mu.ni.ci.pa.li.da.de}{0}
\verb{municipalidade}{}{}{}{}{}{O município.}{mu.ni.ci.pa.li.da.de}{0}
\verb{municipalismo}{}{}{}{}{s.m.}{Sistema administrativo para atender aos municípios.}{mu.ni.ci.pa.lis.mo}{0}
\verb{munícipe}{}{}{}{}{s.2g.}{Habitante de um município.}{mu.ní.ci.pe}{0}
\verb{município}{}{}{}{}{s.m.}{Unidade territorial e administrativa governada pelo prefeito e pela câmara municipal.}{mu.ni.cí.pio}{0}
\verb{município}{}{}{}{}{}{O conjunto de moradores desse território.}{mu.ni.cí.pio}{0}
\verb{munificência}{}{}{}{}{s.f.}{Qualidade de munificente; generosidade, liberalidade.}{mu.ni.fi.cên.cia}{0}
\verb{munificente}{}{}{}{}{adj.2g.}{Que tem magnificência, esplendor, opulência; grandioso, suntuoso.}{mu.ni.fi.cen.te}{0}
\verb{munificente}{}{}{}{}{}{Que revela generosidade, bondade.}{mu.ni.fi.cen.te}{0}
\verb{munir}{}{}{}{}{v.t.}{Fazer com que um lugar possa se defender, armando ou construindo fortalezas; fortificar.}{mu.nir}{0}
\verb{munir}{}{}{}{}{}{Fornecer alguma coisa a alguém para uma finalidade; abastecer.}{mu.nir}{\verboinum{18}\verboirregular{\emph{def.} munimos, munis}}
\verb{múnus}{}{}{}{}{s.m.pl.}{Funções que um indivíduo tem de exercer; encargo, emprego.}{mú.nus}{0}
\verb{muque}{}{Pop.}{}{}{s.m.}{Força dos músculos; musculatura.}{mu.que}{0}
\verb{muquirana}{}{}{}{}{adj.2g.}{Que faz de tudo para não gastar dinheiro; pão"-duro.}{mu.qui.ra.na}{0}
\verb{mural}{}{}{"-ais}{}{s.m.}{Quadro de avisos.}{mu.ral}{0}
\verb{mural}{}{}{"-ais}{}{adj.2g.}{Que é exposto em paredes ou muros.}{mu.ral}{0}
\verb{muralha}{}{}{}{}{s.f.}{Muro de proteção de fortalezas.}{mu.ra.lha}{0}
\verb{muralha}{}{}{}{}{}{Paredão.}{mu.ra.lha}{0}
\verb{murar}{}{}{}{}{v.t.}{Erguer muro.}{mu.rar}{0}
\verb{murar}{}{}{}{}{}{Proteger, cercar.}{mu.rar}{\verboinum{1}}
\verb{murça}{}{}{}{}{s.f.}{Vestimenta usada pelos cônegos por cima da sobrepeliz.}{mur.ça}{0}
\verb{murça}{}{}{}{}{}{Lima serreada de finos dentes.}{mur.ça}{0}
\verb{murchar}{}{}{}{}{v.t.}{Tornar murcho; fazer perder a vida, o viço, a força.}{mur.char}{\verboinum{1}}
\verb{murcho}{}{}{}{}{adj.}{Que está sem viço ou força.}{mur.cho}{0}
\verb{murcho}{}{}{}{}{}{Esvaziado.}{mur.cho}{0}
\verb{mureta}{ê}{}{}{}{s.f.}{Muro baixo.}{mu.re.ta}{0}
\verb{muriático}{}{}{}{}{adj.}{Diz"-se de ácido formado de hidrogênio e cloro.}{mu.ri.á.ti.co}{0}
\verb{murici}{}{Bot.}{}{}{s.m.}{Frutinha amarela e redonda de cheiro acentuado.}{mu.ri.ci}{0}
\verb{muriçoca}{ó}{Zool.}{}{}{s.f.}{Inseto pequeno, de corpo fino e longo, duas asas e seis pernas compridas, que nasce na água; mosquito.}{mu.ri.ço.ca}{0}
\verb{murídeo}{}{}{}{}{adj.}{Que se refere ao rato.}{mu.rí.deo}{0}
\verb{murídeo}{}{}{}{}{}{Espécime dos murídeos, grande família de roedores, que inclui os ratos.}{mu.rí.deo}{0}
\verb{murino}{}{}{}{}{adj.}{Relativo a rato.}{mu.ri.no}{0}
\verb{murmuração}{}{}{"-ões}{}{s.f.}{Ato ou efeito de murmurar; murmúrio.}{mur.mu.ra.ção}{0}
\verb{murmuração}{}{}{"-ões}{}{}{Rumor infundado; boato, falatório.}{mur.mu.ra.ção}{0}
\verb{murmuração}{}{}{"-ões}{}{}{Falatório depreciativo; maledicência.}{mur.mu.ra.ção}{0}
%\verb{}{}{}{}{}{}{}{}{0}
%\verb{}{}{}{}{}{}{}{}{0}
\verb{murmurar}{}{}{}{}{v.t.}{Falar alguma coisa baixinho.}{mur.mu.rar}{0}
\verb{murmurar}{}{}{}{}{}{Falar mal de uma pessoa ou do comportamento de alguém ou de alguma coisa em voz baixa.}{mur.mu.rar}{\verboinum{1}}
\verb{murmurejar}{}{}{}{}{v.i.}{Produzir murmúrio; rumorejar.}{mur.mu.re.jar}{\verboinum{1}}
\verb{murmurinho}{}{}{}{}{s.m.}{Sussurro de vozes simultâneas.}{mur.mu.ri.nho}{0}
\verb{murmurinho}{}{}{}{}{}{Ruído brando das águas, das folhas etc.; murmúrio.}{mur.mu.ri.nho}{0}
\verb{murmúrio}{}{}{}{}{s.m.}{Ato de murmurar; sussurrar.}{mur.mú.rio}{0}
\verb{murmúrio}{}{}{}{}{}{Som de palavras que mal se ouvem.}{mur.mú.rio}{0}
\verb{murmúrio}{}{}{}{}{}{Barulho que mal se ouve; rumor.}{mur.mú.rio}{0}
\verb{muro}{}{}{}{}{s.m.}{Cerca de tijolos ou pedras.}{mu.ro}{0}
\verb{murro}{}{}{}{}{s.m.}{Golpe forte dado com a mão fechada; soco.}{mur.ro}{0}
\verb{murta}{}{Bot.}{}{}{s.f.}{Arbusto de folhagem sempre verde, com pequenas flores brancas aromáticas, cultivado em cercas vivas.}{mur.ta}{0}
\verb{musa}{}{}{}{}{s.f.}{Cada uma das nove divindades da mitologia grega que protegiam as artes e as ciências.}{mu.sa}{0}
\verb{musa}{}{}{}{}{}{Tudo que inspira um artista, em especial uma mulher; inspiradora.}{mu.sa}{0}
\verb{musculação}{}{}{"-ões}{}{s.f.}{Conjunto de exrcícios para desenvolver os músculos.}{mus.cu.la.ção}{0}
\verb{musculação}{}{}{"-ões}{}{}{Conjunto das ações dos músculos.}{mus.cu.la.ção}{0}
\verb{muscular}{}{}{}{}{adj.2g.}{Relativo a músculo.}{mus.cu.lar}{0}
\verb{musculatura}{}{}{}{}{s.f.}{Conjunto dos músculos do corpo.}{mus.cu.la.tu.ra}{0}
\verb{musculatura}{}{}{}{}{}{Força dos músculos; muque.}{mus.cu.la.tu.ra}{0}
\verb{músculo}{}{Anat.}{}{}{s.m.}{Parte do corpo com capacidade para encolher ou esticar, que permite os movimentos.}{mús.cu.lo}{0}
\verb{musculoso}{ô}{}{"-osos ⟨ó⟩}{"-osa ⟨ó⟩}{adj.}{Que tem músculos bem desenvolvidos; forte.}{mus.cu.lo.so}{0}
\verb{museologia}{}{}{}{}{s.f.}{Ciência que trata da organização e manutenção dos museus.}{mu.se.o.lo.gi.a}{0}
\verb{museu}{}{}{}{}{s.m.}{Lugar onde se estudam e se expõem objetos de arte ou antigos e peças de valor científico ou histórico.}{mu.seu}{0}
\verb{musgo}{}{Bot.}{}{}{s.m.}{Vegetal muito pequeno, de caule e folhas verdes, que forma uma espécie de tapete que cobre o lugar onde cresce.}{mus.go}{0}
\verb{musgoso}{ô}{}{"-osos ⟨ó⟩}{"-osa ⟨ó⟩}{adj.}{Coberto de musgo.}{mus.go.so}{0}
\verb{musgoso}{ô}{}{"-osos ⟨ó⟩}{"-osa ⟨ó⟩}{}{Semelhante ao musgo.}{mus.go.so}{0}
\verb{música}{}{}{}{}{s.f.}{Arte de combinar os sons de forma rítmica, harmônica, melodiosa etc.}{mú.si.ca}{0}
\verb{música}{}{}{}{}{}{O produto dessa combinação.}{mú.si.ca}{0}
\verb{musical}{}{}{"-ais}{}{adj.2g.}{Relativo à música.}{mu.si.cal}{0}
\verb{musical}{}{}{"-ais}{}{}{Agradável ao ouvido; harmonioso, melodioso.}{mu.si.cal}{0}
\verb{musical}{}{}{"-ais}{}{s.m.}{Espetáculo teatral ou cinematográfico cantado ou dançado.}{mu.si.cal}{0}
\verb{musicalidade}{}{}{}{}{s.f.}{Qualidade do que é musical; harmonia, afinação, sonoridade.}{mu.si.ca.li.da.de}{0}
\verb{musicar}{}{}{}{}{v.t.}{Colocar música em um texto.}{mu.si.car}{\verboinum{2}}
\verb{musicista}{}{}{}{}{s.2g.}{Especialista em música.}{mu.si.cis.ta}{0}
\verb{músico}{}{}{}{}{s.m.}{Indivíduo que faz ou toca músicas.}{mú.si.co}{0}
\verb{musicomania}{}{}{}{}{s.f.}{Paixão pela música.}{mu.si.co.ma.ni.a}{0}
\verb{mussarela}{é}{}{}{}{}{Var. de \textit{muçarela} ou de \textit{mozarela}, formas aportuguesadas de \textit{mozzarella}. Tem uso mais corrente do que as formas preconizadas pelas gramáticas e vocabulários ortográficos.}{mus.sa.re.la}{0}
\verb{musse}{}{Cul.}{}{}{s.m.}{Iguaria doce ou salgada de consistência cremosa.}{mus.se}{0}
\verb{musse}{}{}{}{}{}{Fixador ou modelador de cabelos.}{mus.se}{0}
\verb{musselina}{}{}{}{}{s.f.}{Tecido leve e transparente, muito usado para roupa feminina.}{mus.se.li.na}{0}
\verb{mustelídeo}{}{}{}{}{adj.}{Relativo aos mustelídeos.}{mus.te.lí.deo}{0}
\verb{mustelídeo}{}{Zool.}{}{}{}{Espécime dos mustelídeos, família dos mamíferos da ordem dos carnívoros, de pequeno porte, corpo longo e esguio e patas curtas, como as doninhas, os furões, as lontras e a ariranha.}{mus.te.lí.deo}{0}
\verb{mutação}{}{}{"-ões}{}{s.f.}{Processo de mudar; mudança, transformação.}{mu.ta.ção}{0}
\verb{mutante}{}{}{}{}{adj.2g.}{Diz"-se de organismo, célula ou gene que sofreu mutação.}{mu.tan.te}{0}
\verb{mutante}{}{}{}{}{s.m.}{Em ficção científica, ser extraordinário resultante de mutação da espécie humana.}{mu.tan.te}{0}
\verb{mutatório}{}{}{}{}{adj.}{Que muda, que serve para fazer mudança.}{mu.ta.tó.rio}{0}
\verb{mutável}{}{}{"-eis}{}{adj.2g.}{Que pode mudar; sujeito a mudança; alterável, mudável.}{mu.tá.vel}{0}
\verb{mutável}{}{}{"-eis}{}{}{Diz"-se de gene passível de sofrer mutação.}{mu.tá.vel}{0}
\verb{mutilação}{}{}{"-ões}{}{s.f.}{Ato ou efeito de mutilar; corte.}{mu.ti.la.ção}{0}
\verb{mutilado}{}{}{}{}{adj.}{Diz"-se de indivíduo a que falta um membro, ou parte dele, ou qualquer parte do corpo.}{mu.ti.la.do}{0}
\verb{mutilar}{}{}{}{}{v.t.}{Cortar alguma parte do corpo.}{mu.ti.lar}{0}
\verb{mutilar}{}{}{}{}{}{Destruir parcialmente.}{mu.ti.lar}{\verboinum{1}}
\verb{mutirão}{}{}{"-ões}{}{s.m.}{Trabalho que muitas pessoas fazem de graça em benefício de uma delas ou de todas.}{mu.ti.rão}{0}
\verb{mutismo}{}{}{}{}{s.m.}{Estado de mudo; mudez.}{mu.tis.mo}{0}
\verb{mutreta}{ê}{}{}{}{s.f.}{Negócio desonesto; maracutaia, negociata.}{mu.tre.ta}{0}
\verb{mutualidade}{}{}{}{}{s.f.}{Qualidade ou estado do que é mútuo; reciprocidade, permutação, troca.}{mu.tu.a.li.da.de}{0}
\verb{mutuar}{}{}{}{}{v.t.}{Dar ou tomar por empréstimo.}{mu.tu.ar}{0}
\verb{mutuar}{}{}{}{}{}{Trocar reciprocamente.}{mu.tu.ar}{\verboinum{1}}
\verb{mutuário}{}{}{}{}{s.m.}{Indivíduo que recebeu algo por empréstimo.}{mu.tu.á.rio}{0}
\verb{mutuca}{}{Zool.}{}{}{s.f.}{Espécie de mosca grande, de picada muito dolorosa; butuca.}{mu.tu.ca}{0}
\verb{mutum}{}{Zool.}{"-uns}{}{s.m.}{Ave galiforme com penas da crista curtas nas extremidades.}{mu.tum}{0}
\verb{mútuo}{}{}{}{}{adj.}{Que se faz reciprocamente, entre duas ou mais pessoas; recíproco.}{mú.tu.o}{0}
\verb{muxiba}{ch}{}{}{}{s.f.}{Carne magra, que se atira aos cães.}{mu.xi.ba}{0}
\verb{muxiba}{ch}{}{}{}{}{Peles enrugadas e magras da carne; pelanca.}{mu.xi.ba}{0}
\verb{muxinga}{ch}{}{}{}{s.f.}{Chicote.}{mu.xin.ga}{0}
\verb{muxirão}{ch}{}{"-ões}{}{s.m.}{Mutirão.}{mu.xi.rão}{0}
\verb{muxoxo}{chô\ldots{}ch}{}{}{}{s.m.}{Resmungo que demonstra enfado, aborrecimento, desdém.}{mu.xo.xo}{0}
