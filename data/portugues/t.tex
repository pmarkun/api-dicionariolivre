\verb{t}{}{}{}{}{s.m.}{Vigésima letra do alfabeto português.}{t}{0}
\verb{Ta}{}{Quím.}{}{}{}{Símb. do \textit{tântalo}.}{Ta}{0}
\verb{tá}{}{Pop.}{}{}{interj.}{Está bem; está combinado; sim; aceito, concordo. }{tá}{0}
\verb{tá}{}{}{}{}{interj.}{Basta; já chega. }{tá}{0}
\verb{taba}{}{Bras.}{}{}{s.f.}{Aldeia de índios; conjunto de habitações de índios. }{ta.ba}{0}
\verb{tabacal}{}{}{"-ais}{}{adj.2g.}{Relativo a tabaco.}{ta.ba.cal}{0}
\verb{tabacal}{}{}{"-ais}{}{s.m.}{Plantação de tabaco. }{ta.ba.cal}{0}
\verb{tabacaria}{}{}{}{}{s.f.}{Loja onde se vendem cigarros, charutos, cachimbo, tabaco e outros artigos para fumantes. }{ta.ba.ca.ri.a}{0}
\verb{tabaco}{}{Bot.}{}{}{s.m.}{Planta solanácea, cujas folhas podem ser aspiradas, fumadas ou mascadas. }{ta.ba.co}{0}
\verb{tabagismo}{}{}{}{}{s.m.}{Abuso de tabaco.}{ta.ba.gis.mo}{0}
\verb{tabagismo}{}{}{}{}{}{Intoxicação causada pelo abuso de tabaco. }{ta.ba.gis.mo}{0}
\verb{tabaqueira}{ê}{}{}{}{s.f.}{Caixa ou bolsa para guardar fumo ou rapé. }{ta.ba.quei.ra}{0}
\verb{tabaréu}{}{Desus.}{}{tabaroa}{s.m.}{Indivíduo que mora no campo, na zona rural; caipira, capiau. }{ta.ba.réu}{0}
\verb{tabatinga}{}{Bras.}{}{}{s.f.}{Qualquer tipo de argila mole com certo teor de matéria orgânica e de coloração variada. }{ta.ba.tin.ga}{0}
\verb{tabe}{}{Med.}{}{}{s.f.}{Tabes. }{ta.be}{0}
\verb{tabefe}{é}{Pop.}{}{}{s.m.}{Bofetada, tapa, soco, sopapo. }{ta.be.fe}{0}
\verb{tabela}{é}{}{}{}{s.f.}{Quadro sistemático para consulta de dados.   }{ta.be.la}{0}
\verb{tabela}{é}{}{}{}{}{Lista, rol.}{ta.be.la}{0}
\verb{tabela}{é}{}{}{}{}{Relação oficial de preços de mercadorias.}{ta.be.la}{0}
\verb{tabelamento}{}{}{}{}{s.m.}{Ato ou efeito de tabelar.}{ta.be.la.men.to}{0}
\verb{tabelamento}{}{}{}{}{}{O controle oficial de preços de certos produtos.}{ta.be.la.men.to}{0}
\verb{tabelar}{}{}{}{}{v.t.}{Fazer uma tabela (de horários, preços, dados etc.).}{ta.be.lar}{0}
\verb{tabelar}{}{}{}{}{}{Submeter (produto, serviço etc.) a uma tabela de preços. }{ta.be.lar}{\verboinum{1}}
\verb{tabelar}{}{}{}{}{adj.2g.}{Relativo a tabela. }{ta.be.lar}{0}
\verb{tabelião}{}{}{"-ães}{tabelioa \textit{ou} tabeliã}{s.m.}{Escrivão público.}{ta.be.li.ão}{0}
\verb{tabelioa}{}{}{}{}{s.f.}{Mulher que exerce as funções de tabelião.}{ta.be.li.o.a}{0}
\verb{tabelioa}{}{}{}{}{adj.}{Diz"-se de certas palavras ou expressões que constituem forma usual.}{ta.be.li.o.a}{0}
\verb{tabelioa}{}{}{}{}{}{Diz"-se das fórmulas usadas nos instrumentos lavrados por tabeliães.}{ta.be.li.o.a}{0}
\verb{tabelionato}{}{}{}{}{s.m.}{Ofício ou escritório de tabelião.}{ta.be.li.o.na.to}{0}
\verb{taberna}{é}{}{}{}{s.f.}{Restaurante barato; botequim.}{ta.ber.na}{0}
\verb{tabernáculo}{}{Relig.}{}{}{s.m.}{Tenda portátil usada pelos hebreus como santuário durante a peregrinação pelo deserto.}{ta.ber.ná.cu.lo}{0}
\verb{tabernáculo}{}{}{}{}{}{Divisão do templo de Jerusalém onde se guardava a Arca da Aliança. }{ta.ber.ná.cu.lo}{0}
\verb{taberneiro}{ê}{}{}{}{s.m.}{Dono ou empregado de taberna.}{ta.ber.nei.ro}{0}
\verb{tabes}{}{Med.}{}{}{s.f.}{Infecção da medula que acarreta perturbações graves no corpo todo ou em parte dele; tabe.}{ta.bes}{0}
\verb{tábido}{}{}{}{}{adj.}{Em que há podridão, decomposição; podre.}{tá.bi.do}{0}
\verb{tabique}{}{}{}{}{s.m.}{Tapume usado para dividir ou fechar as áreas ou quartos de uma casa; divisória.}{ta.bi.que}{0}
\verb{tablado}{}{}{}{}{s.m.}{Piso mais alto, onde artistas ou lutadores se apresentam; palanque.}{ta.bla.do}{0}
\verb{tablete}{é}{Bras.}{}{}{s.m.}{Produto alimentar ou farmacêutico na forma de um pequeno retângulo ou quadrado.}{ta.ble.te}{0}
\verb{tabloide}{ó}{}{}{}{s.m.}{Publicação com formato de meio jornal.  }{ta.bloi.de}{0}
\verb{taboca}{ó}{}{}{}{s.f.}{Tipo de bambu; taquara.}{ta.bo.ca}{0}
\verb{tabu}{}{}{}{}{s.m.}{Tudo o que, convencionalmente, deve ser evitado ou proibido por razões religiosas ou morais.}{ta.bu}{0}
\verb{tabu}{}{}{}{}{}{Proibição imposta por costume social ou como medida de proteção.}{ta.bu}{0}
\verb{tabu}{}{}{}{}{adj.2g.}{Que é proibido, interdito.}{ta.bu}{0}
\verb{tábua}{}{}{}{}{s.f.}{Peça de madeira serrada, plana, pouco espessa e relativamente larga.}{tá.bu.a}{0}
\verb{tabuada}{}{}{}{}{s.f.}{Tabela com as operações aritméticas elementares e seus resultados.}{ta.bu.a.da}{0}
\verb{tabuada}{}{}{}{}{}{O livro que contém essa tabela.}{ta.bu.a.da}{0}
\verb{tabuado}{}{}{}{}{s.m.}{Conjunto de tábuas dispostas uma do lado da outra, que serve de cerca, revestimento etc.}{ta.bu.a.do}{0}
\verb{tábula}{}{}{}{}{s.f.}{Pequena peça redonda, geralmente de osso ou de marfim, usada em vários jogos.}{tá.bu.la}{0}
\verb{tabulador}{ô}{}{}{}{s.m.}{Mecanismo das máquinas de escrever que permite alinhar o carro em várias posições   usando apenas uma tecla.}{ta.bu.la.dor}{0}
\verb{tabular}{}{}{}{}{v.t.}{Organizar dados em tabela.}{ta.bu.lar}{0}
\verb{tabular}{}{}{}{}{}{Ajustar a máquina de escrever, marcando os pontos em que o carro deve parar.}{ta.bu.lar}{\verboinum{1}}
\verb{tabular}{}{}{}{}{adj.2g.}{Relativo a tábua, tabela, mapa, ou ao seu uso.}{ta.bu.lar}{0}
\verb{tabule}{}{Cul.}{}{}{s.m.}{Especialidade da cozinha libanesa, salada feita com trigo cru moído grosso, coentro, hortelã, cebolas e tomates picados, e temperada com azeite de oliva e limão.}{ta.bu.le}{0}
\verb{tabuleiro}{ê}{}{}{}{s.f.}{Mesa de tábuas na qual os feirantes expõem seus produtos.}{ta.bu.lei.ro}{0}
\verb{tabuleiro}{ê}{}{}{}{}{Superfície de madeira ou de outro material, usada para se disputarem diversos tipos de jogos.}{ta.bu.lei.ro}{0}
\verb{tabuleiro}{ê}{}{}{}{}{Assadeira.}{ta.bu.lei.ro}{0}
\verb{tabuleta}{ê}{}{}{}{s.f.}{Tábua pequena que contém algum tipo de informação e é colocada em lugar de destaque.}{ta.bu.le.ta}{0}
\verb{taça}{}{}{}{}{s.f.}{Copo largo e raso provido de pé, usado para beber vinho, champanhe etc.}{ta.ça}{0}
\verb{taça}{}{}{}{}{}{O conteúdo desse copo.}{ta.ça}{0}
\verb{taça}{}{}{}{}{}{Troféu esportivo com a forma desse copo.}{ta.ça}{0}
\verb{tacacá}{}{Cul.}{}{}{s.m.}{Iguaria típica do Pará, que consiste num caldo espesso, feito com goma de mandioca, camarões, tucupi, jambu e pimenta.}{ta.ca.cá}{0}
\verb{tacada}{}{}{}{}{s.f.}{Golpe de taco.}{ta.ca.da}{0}
\verb{tacada}{}{}{}{}{}{Acontecimento súbito, inesperado; golpe de sorte.}{ta.ca.da}{0}
\verb{tacanho}{}{}{}{}{adj.}{De pouca altura; baixo.}{ta.ca.nho}{0}
\verb{tacanho}{}{}{}{}{}{Que tem pouca inteligência; estúpido.}{ta.ca.nho}{0}
\verb{tacão}{}{}{"-ões}{}{s.m.}{O salto de um calçado.}{ta.cão}{0}
\verb{tacape}{}{}{}{}{s.m.}{Arma indígena, semelhante a uma clava, usada em ataques e sacrifícios; borduna.}{ta.ca.pe}{0}
\verb{tacar}{}{}{}{}{v.t.}{Bater com o taco em uma bola a fim de empurrá"-la.}{ta.car}{0}
\verb{tacar}{}{}{}{}{}{Fazer alguma coisa sair com violência em direção a outra; arremessar, atirar, jogar. }{ta.car}{\verboinum{2}}
\verb{tacha}{}{}{}{}{s.f.}{Prego de cabeça redonda, chata e larga.}{ta.cha}{0}
\verb{tachada}{}{}{}{}{s.f.}{O que um tacho pode conter.}{ta.cha.da}{0}
\verb{tachada}{}{}{}{}{}{Tacho cheio.}{ta.cha.da}{0}
\verb{tachar}{}{}{}{}{v.t.}{Acusar alguém de alguma coisa.}{ta.char}{\verboinum{1}}
\verb{tachear}{}{Bras.}{}{}{v.t.}{Pregar tachas.}{ta.che.ar}{0}
\verb{tachear}{}{}{}{}{}{Enfeitar com tachas.}{ta.che.ar}{\verboinum{4}}
\verb{tacho}{}{}{}{}{s.m.}{Tipo de panela grande, larga e pouco funda, geralmente com alças.}{ta.cho}{0}
\verb{tácito}{}{}{}{}{adj.}{Que não é preciso dizer por estar implícito ou subentendido.}{tá.ci.to}{0}
\verb{tácito}{}{}{}{}{}{Silencioso.}{tá.ci.to}{0}
\verb{taciturno}{}{}{}{}{adj.}{Que é de poucas palavras.}{ta.ci.tur.no}{0}
\verb{taciturno}{}{}{}{}{}{Que é ou está tomado pela tristeza.}{ta.ci.tur.no}{0}
\verb{taciturno}{}{}{}{}{}{Carrancudo.}{ta.ci.tur.no}{0}
\verb{taco}{}{}{}{}{s.m.}{Bastão de madeira longo e roliço.}{ta.co}{0}
\verb{taco}{}{}{}{}{}{Pequena peça de madeira usada para revestir pisos.}{ta.co}{0}
\verb{tacômetro}{}{}{}{}{s.m.}{Instrumento que mede as rotações por minuto de um motor ou de um eixo, usado para determinar a velocidade de um veículo.}{ta.cô.me.tro}{0}
\verb{táctil}{}{Lus.}{"-eis}{}{adj.2g.}{Tátil. }{tác.til}{0}
\verb{tacto}{}{Lus.}{}{}{s.m.}{Tato.}{tac.to}{0}
\verb{tafetá}{}{}{}{}{s.m.}{Tecido de seda, muito lustroso, de trama fina e sem avesso.}{ta.fe.tá}{0}
\verb{tagarela}{é}{}{}{}{adj.2g.}{Que fala muito.}{ta.ga.re.la}{0}
\verb{tagarela}{é}{}{}{}{}{Fofoqueiro.}{ta.ga.re.la}{0}
\verb{tagarelar}{}{}{}{}{v.t.}{Falar muito.}{ta.ga.re.lar}{0}
\verb{tagarelar}{}{}{}{}{}{Fofocar, bisbilhotar.}{ta.ga.re.lar}{\verboinum{1}}
\verb{tagarelice}{}{}{}{}{s.f.}{Costume, hábito de tagarelar.}{ta.ga.re.li.ce}{0}
\verb{tagarelice}{}{}{}{}{}{Modos de tagarela.}{ta.ga.re.li.ce}{0}
\verb{tagarelice}{}{}{}{}{}{Dito indiscreto ou inconveniente; indiscrição, intrometimento.}{ta.ga.re.li.ce}{0}
\verb{tágide}{}{Poét.}{}{}{s.f.}{A ninfa do rio Tejo (Portugal).}{tá.gi.de}{0}
\verb{taifa}{}{Bras.}{}{}{s.f.}{A criadagem de uma embarcação.}{tai.fa}{0}
\verb{taifeiro}{ê}{}{}{}{s.m.}{Serviçal de um navio mercante.}{tai.fei.ro}{0}
\verb{tailandês}{}{}{}{}{adj.}{Relativo à Tailândia (Sudeste Asiático).}{tai.lan.dês}{0}
\verb{tailandês}{}{}{}{}{s.m.}{Indivíduo natural ou habitante desse país.}{tai.lan.dês}{0}
\verb{tailandês}{}{}{}{}{}{A língua oficial da Tailândia.}{tai.lan.dês}{0}
\verb{tailleur}{}{}{}{}{s.m.}{Conjunto feminino de casaco e saia.}{\textit{tailleur}}{0}
\verb{tainha}{}{Zool.}{}{}{s.f.}{Nome comum a várias espécies de peixes encontrados no Atlântico, com nadadeiras dorsal e anal sem escamas, corpo com listras escuras, e cuja carne é apreciada.}{ta.i.nha}{0}
\verb{taioba}{ó}{Bot.}{}{}{s.f.}{Erva de folhas largas, nativa da América tropical, cujas folhas e tubérculos são comestíveis.}{tai.o.ba}{0}
\verb{taipa}{}{}{}{}{s.f.}{Parede construída com uma trama de ripas ou varas fincadas no chão, que se cobre com barro; pau"-a"-pique.}{tai.pa}{0}
\verb{taipar}{}{}{}{}{v.t.}{Cercar, delimitar ou separar com taipa.}{tai.par}{0}
\verb{taipar}{}{}{}{}{}{Construir muro, casa etc. de taipa.}{tai.par}{\verboinum{1}}
\verb{taitiano}{}{}{}{}{adj.}{Relativo ao Taiti (Polinésia Francesa).}{ta.i.ti.a.no}{0}
\verb{taitiano}{}{}{}{}{s.m.}{Natural ou habitante do Taiti.}{ta.i.ti.a.no}{0}
\verb{taitiano}{}{}{}{}{}{A língua falada nessa ilha.}{ta.i.ti.a.no}{0}
\verb{taiuanês}{}{}{}{}{adj.}{Relativo a Taiwan, antiga Formosa ou República Nacional da China (Ásia).}{tai.u.a.nês}{0}
\verb{taiuanês}{}{}{}{}{s.m.}{Indivíduo natural ou habitante desse país.}{tai.u.a.nês}{0}
\verb{tal}{}{}{tais}{}{pron.}{Este, aquele.}{tal}{0}
\verb{tal}{}{}{tais}{}{s.2g.}{Indivíduo que tem ou pensa ter alguma boa qualidade.}{tal}{0}
\verb{tala}{}{}{}{}{s.f.}{Dispositivo usado para imobilizar parte do corpo fraturada ou luxada.}{ta.la}{0}
\verb{talabarte}{}{}{}{}{s.m.}{Correia a tiracolo, à qual se prende a espada ou outra arma; cinturão; boldrié.}{ta.la.bar.te}{0}
\verb{talagada}{}{Bras.}{}{}{s.f.}{Gole de bebida alcoólica que se toma de uma só vez; trago.}{ta.la.ga.da}{0}
\verb{talagarça}{}{}{}{}{s.f.}{Tecido de fios ralos, sobre o qual se borda.}{ta.la.gar.ça}{0}
\verb{tálamo}{}{}{}{}{s.m.}{Leito de casal.}{tá.la.mo}{0}
\verb{tálamo}{}{Anat.}{}{}{}{Parte do cérebro que participa da recepção, coordenação e integração dos impulsos nervosos sensoriais.}{tá.la.mo}{0}
\verb{talante}{}{Desus.}{}{}{s.m.}{Vontade, desejo, arbítrio, alvedrio.}{ta.lan.te}{0}
\verb{talante}{}{}{}{}{}{Empenho, esforço, disposição, interesse.}{ta.lan.te}{0}
\verb{talão}{}{}{"-ões}{}{s.m.}{Bloco de folhas destacáveis.}{ta.lão}{0}
\verb{talar}{}{}{}{}{adj.2g.}{Relativo a talão}{ta.lar}{0}
\verb{talar}{}{}{}{}{}{Que desce até o calcanhar.}{ta.lar}{0}
\verb{talar}{}{}{}{}{v.t.}{Abrir valas em, a fim de escoar os campos; sulcar.}{ta.lar}{0}
\verb{talar}{}{}{}{}{}{Destruir, assolar, devastar.}{ta.lar}{\verboinum{1}}
\verb{talássico}{}{}{}{}{adj.}{Relativo ao mar.}{ta.lás.si.co}{0}
\verb{talassofobia}{}{}{}{}{s.f.}{Medo mórbido de mar.}{ta.las.so.fo.bi.a}{0}
\verb{talassófobo}{}{}{}{}{s.m.}{Pessoa que sofre de talassofobia.}{ta.las.só.fo.bo}{0}
\verb{talassômetro}{}{}{}{}{s.m.}{Sonda marítima.}{ta.las.sô.me.tro}{0}
\verb{talco}{}{}{}{}{s.m.}{Mineral macio que parece uma graxa ao toque.}{tal.co}{0}
\verb{talco}{}{}{}{}{}{Pó desse mineral para uso medicinal ou higiênico.}{tal.co}{0}
\verb{talento}{}{}{}{}{s.m.}{Facilidade natural para fazer alguma coisa.}{ta.len.to}{0}
\verb{talento}{}{}{}{}{}{Antiga moeda grega e romana.}{ta.len.to}{0}
\verb{talentoso}{ô}{}{"-osos ⟨ó⟩}{"-osa ⟨ó⟩}{adj.}{Que tem talento, inteligência, habilidade.}{ta.len.to.so}{0}
\verb{talha}{}{}{}{}{s.f.}{Vaso bojudo para armazenar líquidos ou cereais.}{ta.lha}{0}
\verb{talha}{}{}{}{}{}{Obra de arte que se faz entalhando a madeira.}{ta.lha}{0}
\verb{talhada}{}{}{}{}{s.f.}{Porção que se corta de certas coisas; fatia, naco, pedaço, lasca. }{ta.lha.da}{0}
\verb{talhadeira}{ê}{Bras.}{}{}{s.f.}{Ferramenta usada para talhar.}{ta.lha.dei.ra}{0}
\verb{talhado}{}{}{}{}{adj.}{Que foi dividido, cortado.}{ta.lha.do}{0}
\verb{talhado}{}{}{}{}{}{Esculpido, moldado.}{ta.lha.do}{0}
\verb{talhado}{}{}{}{}{}{Diz"-se de leite coalhado.}{ta.lha.do}{0}
\verb{talhado}{}{}{}{}{}{Abismo, despenhadeiro.}{ta.lha.do}{0}
\verb{talhador}{ô}{}{}{}{adj.}{Diz"-se de máquina que se usa para talhar, cortar.}{ta.lha.dor}{0}
\verb{talha"-mar}{}{}{talha"-mares}{}{s.m.}{Muralha ou barreira natural destinada a oferecer resistência ao embate das ondas ou à força das correntes marítimas; quebra"-mar.}{ta.lha"-mar}{0}
\verb{talha"-mar}{}{}{talha"-mares}{}{}{Parte da proa do navio que corta as águas.}{ta.lha"-mar}{0}
\verb{talhamento}{}{}{}{}{s.m.}{Ato ou efeito de talhar.}{ta.lha.men.to}{0}
\verb{talhar}{}{}{}{}{v.t.}{Dar ou fazer talho em; cortar, golpear.}{ta.lhar}{0}
\verb{talhar}{}{}{}{}{}{Modelar, moldar.}{ta.lhar}{0}
\verb{talhar}{}{}{}{}{}{Esculpir.}{ta.lhar}{0}
\verb{talhar}{}{}{}{}{}{Coalhar.}{ta.lhar}{\verboinum{1}}
\verb{talharim}{}{}{}{}{s.m.}{Tipo de massa feita com ovos e farinha de trigo, cortada em tiras mais ou menos delgadas.}{ta.lha.rim}{0}
\verb{talharim}{}{Cul.}{}{}{}{\textit{Por ext.} Prato preparado com essa massa cozida, servida com vários tipos de molhos.}{ta.lha.rim}{0}
\verb{talhe}{}{}{}{}{s.m.}{Feição do corpo ou de qualquer objeto.}{ta.lhe}{0}
\verb{talhe}{}{}{}{}{}{Tronco do corpo humano.}{ta.lhe}{0}
\verb{talher}{é}{}{}{}{s.m.}{O conjunto de garfo, faca e colher.}{ta.lher}{0}
\verb{talho}{}{}{}{}{s.m.}{Corte, rasgo.}{ta.lho}{0}
\verb{talho}{}{}{}{}{}{Aspecto, forma.}{ta.lho}{0}
\verb{talião}{}{}{}{}{s.m.}{Castigo igual ao crime cometido.}{ta.li.ão}{0}
\verb{talim}{}{}{"-ins}{}{s.m.}{Correia a tiracolo, à qual se prende a espada ou outra arma; talabarte, cinturão; boldrié.}{ta.lim}{0}
\verb{tálio}{}{Quím.}{}{}{s.m.}{Elemento químico metálico, branco azulado, tóxico, pouco utilizado. \elemento{81}{204.3833}{Ti}.}{tá.lio}{0}
\verb{talisca}{}{}{}{}{s.f.}{Fenda em rocha; greta.}{ta.lis.ca}{0}
\verb{talisca}{}{}{}{}{}{Lasca, pedaço, estilha.}{ta.lis.ca}{0}
\verb{talismã}{}{}{}{}{s.m.}{Objeto usado para atrair sorte.}{ta.lis.mã}{0}
\verb{Talmude}{}{Relig.}{}{}{s.m.}{Livro que contém a doutrina e a jurisprudência da religião judaica.}{Tal.mu.de}{0}
\verb{talmúdico}{}{}{}{}{adj.}{Relativo ao Talmude.  }{tal.mú.di.co}{0}
\verb{talo}{}{}{}{}{s.m.}{Fibra grossa que se estende pelo meio das folhas das plantas.}{ta.lo}{0}
\verb{talonário}{}{}{}{}{s.m.}{Conjunto de talões em bloco; talão.}{ta.lo.ná.rio}{0}
\verb{taloso}{ô}{}{"-osos ⟨ó⟩}{"-osa ⟨ó⟩}{adj.}{Relativo a talo, ou que tem talo. }{ta.lo.so}{0}
\verb{talude}{}{}{}{}{s.m.}{Superfície inclinada à beira de uma escavação, aterro ou qualquer outra obra.}{ta.lu.de}{0}
\verb{talude}{}{}{}{}{}{Terreno inclinado; rampa.}{ta.lu.de}{0}
\verb{taludo}{}{Bot.}{}{}{adj.}{Que tem o talo rijo, resistente.}{ta.lu.do}{0}
\verb{taludo}{}{Fig.}{}{}{}{Que tem o corpo bem desenvolvido (criança, adolescente); parrudo, corpulento, desenvolvido. }{ta.lu.do}{0}
\verb{talvegue}{é}{Geogr.}{}{}{s.m.}{A parte mais profunda do leito de um rio. }{tal.ve.gue}{0}
\verb{talvez}{ê}{}{}{}{adv.}{Possivelmente, porventura.}{tal.vez}{0}
\verb{tamancada}{}{}{}{}{s.f.}{Pancada, golpe de tamanco.}{ta.man.ca.da}{0}
\verb{tamanco}{}{}{}{}{s.m.}{Calçado cuja base é de madeira inteiriça.}{ta.man.co}{0}
\verb{tamanduá}{}{Zool.}{}{}{s.m.}{Nome comum aos mamíferos xenartros, desdentados, de focinho longo e cilíndrico, que se alimentam principalmente de cupins.}{ta.man.du.á}{0}
\verb{tamanduá"-bandeira}{ê}{Zool.}{tamanduás"-bandeiras \textit{ou} tamanduás"-bandeira}{}{s.m.}{Mamífero desdentado, encontrado nas Américas Central e do Sul, dotado de um focinho longo e cilíndrico e de uma língua comprida e aderente para capturar cupins e formigas, dos quais se alimenta.}{ta.man.du.á"-ban.dei.ra}{0}
\verb{tamanho}{}{}{}{}{s.m.}{Medida de alguma coisa.}{ta.ma.nho}{0}
\verb{tamanho}{}{}{}{}{adj.}{Tão grande, tão extenso.}{ta.ma.nho}{0}
\verb{tamanqueiro}{ê}{}{}{}{s.m.}{Pessoa que fabrica ou vende tamanco.}{ta.man.quei.ro}{0}
\verb{tâmara}{}{}{}{}{s.f.}{Fruto da tamareira.}{tâ.ma.ra}{0}
\verb{tamareira}{ê}{Bot.}{}{}{s.f.}{Palmeira nativa da África, muito cultivada pelo seu fruto comestível, a tâmara.}{ta.ma.rei.ra}{0}
\verb{tamarindeiro}{ê}{}{}{}{}{Tamarindo.}{ta.ma.rin.dei.ro}{0}
\verb{tamarindeiro}{ê}{Bot.}{}{}{s.m.}{Árvore, nativa da África, cujo fruto, comestível, é o tamarindo.}{ta.ma.rin.dei.ro}{0}
\verb{tamarindo}{}{}{}{}{s.m.}{O fruto do tamarindeiro, de polpa marrom escuro, ácida e adstringente, muito apreciado em sucos, doces e sorvetes.}{ta.ma.rin.do}{0}
\verb{tamarindo}{}{}{}{}{}{Tamarindeiro.}{ta.ma.rin.do}{0}
\verb{tambaqui}{}{Zool.}{}{}{s.m.}{Peixe encontrado no rio Amazonas e afluentes, com dorso cinza escuro e ventre claro.}{tam.ba.qui}{0}
\verb{também}{}{}{}{}{adv.}{Indica comparação e expressa condição de equivalência; da mesma forma.}{tam.bém}{0}
\verb{também}{}{}{}{}{}{Indica inclusão; além disso.}{tam.bém}{0}
\verb{também}{}{}{}{}{}{Indica o contrário; por outro lado.}{tam.bém}{0}
\verb{também}{}{}{}{}{}{Indica ênfase; realmente, na verdade.}{tam.bém}{0}
\verb{tambor}{ô}{}{}{}{s.m.}{Recipiente de metal, em forma de cilindro, usado para guardar uma grande quantidade de líquido.}{tam.bor}{0}
\verb{tambor}{ô}{Mús.}{}{}{}{Qualquer dos instrumentos de percussão, com uma ou duas membranas esticadas, as quais, percutidas, produzem sons indeterminados.}{tam.bor}{0}
\verb{tambor}{ô}{}{}{}{}{Peça de revólver, na qual se acomodam as balas.}{tam.bor}{0}
\verb{tamborete}{ê}{}{}{}{s.m.}{Assento quadrado ou redondo, sem encosto e braços, com quatro pés; banco.}{tam.bo.re.te}{0}
\verb{tamboril}{}{}{"-is}{}{s.m.}{Tamborim.}{tam.bo.ril}{0}
\verb{tamborilar}{}{}{}{}{v.t.}{Percutir levemente com os dedos sobre uma superfície.}{tam.bo.ri.lar}{0}
\verb{tamborilar}{}{}{}{}{}{Imitar o som do tambor.}{tam.bo.ri.lar}{\verboinum{1}}
\verb{tamborim}{}{}{"-ins}{}{s.m.}{Pequeno tambor.}{tam.bo.rim}{0}
\verb{tamis}{}{}{}{}{s.m.}{Peneira de seda usada em farmácia ou laboratório.}{ta.mis}{0}
\verb{tamis}{}{}{}{}{}{Tecido inglês de lã.}{ta.mis}{0}
\verb{tamis}{}{}{}{}{}{Peneira, filtro.}{ta.mis}{0}
\verb{tamisar}{}{}{}{}{v.t.}{Passar pelo tamis; peneirar.}{ta.mi.sar}{\verboinum{1}}
\verb{tamoio}{ô}{}{}{}{adj.}{Diz"-se do indivíduo dos tamoios, povo indígena que habitava o Vale do Paraíba, o litoral de \textsc{sp} e o \textsc{rj}.}{ta.moi.o}{0}
\verb{tampa}{}{}{}{}{s.f.}{Peça móvel com que se cobre ou se fecha um recipiente.}{tam.pa}{0}
\verb{tampão}{}{}{}{}{s.m.}{Tampa grande para fechar buracos.}{tam.pão}{0}
\verb{tampão}{}{}{}{}{}{Porção de algodão para estancar hemorragia ou impedir saída de um medicamento.}{tam.pão}{0}
\verb{tampar}{}{}{}{}{v.t.}{Cobrir ou fechar com tampa ou tampo.}{tam.par}{\verboinum{1}}
\verb{tampinha}{}{}{}{}{s.f.}{Tampa pequena.}{tam.pi.nha}{0}
\verb{tampinha}{}{}{}{}{}{Jogo infantil em que se usam tampas metálicas de garrafas de cerveja ou refrigerantes.}{tam.pi.nha}{0}
\verb{tampinha}{}{Pop.}{}{}{s.m.}{Indivíduo de estatura muito baixa.}{tam.pi.nha}{0}
\verb{tampo}{}{}{}{}{s.m.}{Peça de madeira ou plástico que guarnece a boca do vaso sanitário.}{tam.po}{0}
\verb{tamponar}{}{}{}{}{v.t.}{Obstruir com tampão; tapar.}{tam.po.nar}{\verboinum{1}}
\verb{tampouco}{ô}{}{}{}{adv.}{Usado para reforçar uma negação; também não, muito menos.}{tam.pou.co}{0}
\verb{tanajura}{}{Zool.}{}{}{s.f.}{Fêmea das saúvas que perde as asas quando começa a formar um novo formigueiro.}{ta.na.ju.ra}{0}
\verb{tanajura}{}{Pop.}{}{}{}{Mulher de cintura fina e quadris largos.}{ta.na.ju.ra}{0}
\verb{tanatologia}{}{}{}{}{s.f.}{Teoria ou estudo científico sobre a morte, suas causas e fenômenos a ela relacionados.}{ta.na.to.lo.gi.a}{0}
\verb{tandem}{}{}{"-ens}{}{s.m.}{Bicicleta de dois assentos, um atrás do outro.}{tan.dem}{0}
\verb{tanga}{}{}{}{}{s.f.}{Espécie de avental usado por certos povos naturais para cobrir o corpo desde o ventre até as coxas.}{tan.ga}{0}
\verb{tanga}{}{}{}{}{}{Roupa feminina íntima ou de banho, muito pequena, usada entre a cintura e as coxas.}{tan.ga}{0}
\verb{tangará}{}{Zool.}{}{}{s.m.}{Pássaro encontrado em toda a América do Sul, que dança em grupo para atrair a fêmea.}{tan.ga.rá}{0}
\verb{tangedor}{ô}{}{}{}{adj.}{Que tange ou toca instrumentos.}{tan.ge.dor}{0}
\verb{tangedor}{ô}{}{}{}{}{Que tange ou toca animais.}{tan.ge.dor}{0}
\verb{tangência}{}{Geom.}{}{}{s.f.}{Contato entre curvas ou superfícies de modo que suas vizinhanças fiquem em lados opostos relativamente a uma reta ou a um plano.}{tan.gên.cia}{0}
\verb{tangencial}{}{}{"-ais}{}{adj.2g.}{Relativo a tangência ou a tangente.}{tan.gen.ci.al}{0}
\verb{tangencial}{}{}{"-ais}{}{}{Que é feito de maneira superficial, sem aprofundamento.}{tan.gen.ci.al}{0}
\verb{tangenciar}{}{}{}{}{v.t.}{Traçar uma tangente ou seguir uma tangente de.}{tan.gen.ci.ar}{0}
\verb{tangenciar}{}{}{}{}{}{Passar ou estar muito próximo de; tocar.}{tan.gen.ci.ar}{\verboinum{6}}
\verb{tangente}{}{}{}{}{adj.2g.}{Que tange ou tangencia.}{tan.gen.te}{0}
\verb{tangente}{}{}{}{}{}{Que passa muito perto de.}{tan.gen.te}{0}
\verb{tangente}{}{Geom.}{}{}{}{Reta que toca outra reta ou uma superfície em um único ponto.}{tan.gen.te}{0}
\verb{tanger}{ê}{}{}{}{v.t.}{Fazer um sino ou instrumento de cordas soar; tocar.}{tan.ger}{0}
\verb{tanger}{ê}{}{}{}{}{Tocar o gado.}{tan.ger}{0}
\verb{tanger}{ê}{}{}{}{}{Dizer respeito a pessoa ou coisa; concernir.}{tan.ger}{\verboinum{16}}
\verb{tangerina}{}{}{}{}{s.f.}{Fruta cítrica da família da laranja, cuja casca se tira facilmente com a mão; mexerica.}{tan.ge.ri.na}{0}
\verb{tangerineira}{ê}{Bot.}{}{}{s.f.}{Árvore de flores brancas, nativa do Sudeste da Ásia e cultivada em várias regiões tropicais, de frutos refrescantes e casca rica em óleo essencial.}{tan.ge.ri.nei.ra}{0}
\verb{tangível}{}{}{"-eis}{}{adj.2g.}{Que pode ser tangido, tocado ou apalpado; palpável, sensível. }{tan.gí.vel}{0}
\verb{tango}{}{}{}{}{s.m.}{Canto e dança de origem argentina, criada sob influência da habanera, da milonga e de certas melodias populares europeias.}{tan.go}{0}
\verb{tangolomango}{}{}{}{}{s.m.}{Doença atribuída a feitiçaria.}{tan.go.lo.man.go}{0}
\verb{tanino}{}{}{}{}{s.m.}{Substância vegetal que dá coloração azul em certos sais de ferro, usada no curtimento de couro e como mordente. }{ta.ni.no}{0}
\verb{taninoso}{ô}{}{"-osos ⟨ó⟩}{"-osa ⟨ó⟩}{adj.}{Que contém tanino.}{ta.ni.no.so}{0}
\verb{tanoaria}{}{}{}{}{s.f.}{Local onde trabalha o tanoeiro.}{ta.no.a.ri.a}{0}
\verb{tanoaria}{}{}{}{}{}{Ofício, obra de tanoeiro.}{ta.no.a.ri.a}{0}
\verb{tanoeiro}{ê}{}{}{}{s.m.}{Indivíduo que fabrica ou conserta pipas, barris ou outras vasilhas semelhantes.}{ta.no.ei.ro}{0}
\verb{tanque}{}{}{}{}{s.m.}{Reservatório para conter água.}{tan.que}{0}
\verb{tanque}{}{}{}{}{}{Pequeno lago, construído para criar peixes.}{tan.que}{0}
\verb{tanque}{}{}{}{}{}{Cuba para lavagem de roupa.}{tan.que}{0}
\verb{tantã}{}{}{}{}{s.m.}{Tipo de tambor.}{tan.tã}{0}
\verb{tantã}{}{Pop.}{}{}{adj.2g.}{Que não está no uso da razão; desequilibrado, maluco.}{tan.tã}{0}
\verb{tantálico}{}{Quím.}{}{}{adj.}{Relativo a tântalo.}{tan.tá.li.co}{0}
\verb{tantálico}{}{Quím.}{}{}{}{Diz"-se de ácido derivado do tântalo.}{tan.tá.li.co}{0}
\verb{tantalizar}{}{}{}{}{v.t.}{Ser supliciante, terrível para alguém.}{tan.ta.li.zar}{0}
\verb{tantalizar}{}{}{}{}{}{Atrair fortemente a atenção ou admiração de.}{tan.ta.li.zar}{\verboinum{1}}
\verb{tântalo}{}{Quím.}{}{}{s.m.}{Elemento químico metálico, cinzento, pesado, dúctil, usado em ligas, em equipamentos sujeitos à corrosão, em instrumentos cirúrgicos, na indústria aeronáutica  etc. \elemento{73}{180.9479}{Ta}.}{tân.ta.lo}{0}
\verb{tanto}{}{}{}{}{pron.}{Tão grande, tamanho.}{tan.to}{0}
\verb{tanto}{}{}{}{}{s.m.}{Porção ou quantia indeterminada.}{tan.to}{0}
\verb{tanto}{}{}{}{}{adv.}{Em tal quantidade.}{tan.to}{0}
\verb{tanzaniano}{}{}{}{}{adj.}{Relativo à Tanzânia (África Oriental).}{tan.za.ni.a.no}{0}
\verb{tanzaniano}{}{}{}{}{s.m.}{Indivíduo natural ou habitante desse país.}{tan.za.ni.a.no}{0}
\verb{tão}{}{}{}{}{adv.}{Em tal grau, maneira ou quantidade.}{tão}{0}
\verb{taoísmo}{}{}{}{}{s.m.}{Doutrina filosófico"-religiosa chinesa, desenvolvida sobretudo por Lao"-Tsé, no século \textsc{vi} a.C, que trata dos espíritos, da natureza, dos ancestrais e de outras crenças.}{tao.ís.mo}{0}
\verb{tão"-só}{}{}{}{}{adv.}{Não mais que; apenas, unicamente.}{tão"-só}{0}
\verb{tão"-somente}{}{}{}{}{adv.}{Apenas; não mais que.}{tão"-so.men.te}{0}
\verb{tapa}{}{}{}{}{s.m.}{Pancada dada com a palma da mão; palmada.}{ta.pa}{0}
\verb{tapa"-buraco}{}{}{}{}{s.2g.2n}{Indivíduo que substitui outro numa emergência.}{ta.pa"-bu.ra.co}{0}
\verb{tapada}{}{}{}{}{s.f.}{Terreno murado; cerca.}{ta.pa.da}{0}
\verb{tapada}{}{}{}{}{}{Área rodeada de muros, com bosques, destinada a criação e preservação da caça; parque.}{ta.pa.da}{0}
\verb{tapado}{}{}{}{}{adj.}{Que se tapou; fechado.}{ta.pa.do}{0}
\verb{tapado}{}{Pop.}{}{}{}{Que é estúpido, burro, ignorante.}{ta.pa.do}{0}
\verb{tapadura}{}{}{}{}{s.f.}{Ato ou efeito de tapar; tapamento.}{ta.pa.du.ra}{0}
\verb{tapadura}{}{}{}{}{}{Tampa.}{ta.pa.du.ra}{0}
\verb{tapadura}{}{}{}{}{}{Cerca ou vala guarnecida de sebe; tapume.}{ta.pa.du.ra}{0}
\verb{tapagem}{}{}{"-ens}{}{s.f.}{Barreira usada para defesa militar.}{ta.pa.gem}{0}
\verb{tapagem}{}{}{"-ens}{}{}{Cerca guarnecida de sebe.}{ta.pa.gem}{0}
\verb{tapagem}{}{}{"-ens}{}{}{Barragem feita com cipós, na margem dos rios, para represar água ou evitar que os peixes saiam.}{ta.pa.gem}{0}
\verb{tapamento}{}{}{}{}{s.m.}{Ato ou efeito de tapar; tapadura, tapagem.}{ta.pa.men.to}{0}
\verb{tapamento}{}{}{}{}{}{Cerca ou vala guarnecida de sebe; tapume.}{ta.pa.men.to}{0}
\verb{tapa"-olho}{ô}{}{tapa"-olhos ⟨ó⟩}{}{s.m.}{Venda usada para proteger o olho. (\textit{Camões usava um tapa"-olho.})}{ta.pa"-o.lho}{0}
\verb{tapa"-olhos}{ó}{}{}{}{s.m.}{Tapa"-olho.}{ta.pa"-o.lhos}{0}
\verb{tapar}{}{}{}{}{v.t.}{Pôr tampa ou tampo; cobrir, tampar.}{ta.par}{0}
\verb{tapar}{}{}{}{}{}{Fechar a abertura de; vedar, cerrar.}{ta.par}{0}
\verb{tapar}{}{}{}{}{}{Encobrir, ocultar, esconder.}{ta.par}{\verboinum{1}}
\verb{tapeação}{}{}{"-ões}{}{s.f.}{Ato ou efeito de tapear; logro, enganação.}{ta.pe.a.ção}{0}
\verb{tapeador}{ô}{}{}{}{adj.}{Que tapeia, ilude; enganador.}{ta.pe.a.dor}{0}
\verb{tapear}{}{}{}{}{v.t.}{Enganar por meio de artimanhas; lograr, ludibriar.}{ta.pe.ar}{\verboinum{4}}
\verb{tapeçar}{}{}{}{}{v.t.}{Cobrir com tapetes, tapeçarias; atapetar, carpetar.}{ta.pe.çar}{\verboinum{3}}
\verb{tapeçaria}{}{}{}{}{s.f.}{Tecido grosso, geralmente bordado ou trabalhado, com que se forram assoalhos e paredes; alcatifa.}{ta.pe.ça.ri.a}{0}
\verb{tapeçaria}{}{}{}{}{}{Fábrica ou loja de tapetes.}{ta.pe.ça.ri.a}{0}
\verb{tapeceiro}{ê}{}{}{}{s.m.}{Indivíduo que fabrica ou vende tapeçarias.}{ta.pe.cei.ro}{0}
\verb{tapera}{é}{}{}{}{s.f.}{Habitação em ruínas, abandonada.}{ta.pe.ra}{0}
\verb{tapera}{é}{}{}{}{}{Lugar deserto, abandonado.}{ta.pe.ra}{0}
\verb{taperebá}{}{}{}{}{s.m.}{Fruto amarelo, muito aromático, suculento e azedo; cajá.}{ta.pe.re.bá}{0}
\verb{tapetar}{}{}{}{}{v.t.}{Forrar com tapetes; atapetar, tapeçar.}{ta.pe.tar}{\verboinum{1}}
\verb{tapete}{ê}{}{}{}{s.m.}{Peça de tecido grosso ou de pelo usada para cobrir assoalhos, móveis, paredes etc.}{ta.pe.te}{0}
\verb{tapioca}{ó}{}{}{}{s.f.}{Farinha extraída das raízes da mandioca ou aipim, com a qual se preparam pratos salgados ou doces; goma, polvilho.}{ta.pi.o.ca}{0}
\verb{tapioca}{ó}{Cul.}{}{}{}{Beiju feito com essa farinha e que pode ser recheado com coco ralado ou coberto com manteiga.}{ta.pi.o.ca}{0}
\verb{tapir}{}{Zool.}{}{}{s.m.}{Mamífero de grande porte, membros curtos e focinho longo e flexível, que vive próximo a lagos e rios e se alimenta de plantas; anta. }{ta.pir}{0}
\verb{tapona}{}{}{}{}{s.f.}{Tapa forte aplicado com a mão; tabefe.}{ta.po.na}{0}
\verb{tapuio}{}{Hist.}{}{}{s.m.}{Nome que os índios da nação tupi davam a todos os que não pertenciam a seu grupo e que eram considerados inimigos.}{ta.pui.o}{0}
\verb{tapuio}{}{}{}{}{}{Índio bravio e guerreiro.}{ta.pui.o}{0}
\verb{tapuio}{}{}{}{}{}{Mestiço de branco e índia.}{ta.pui.o}{0}
\verb{tapume}{}{}{}{}{s.m.}{Cerca feita com tábuas para vedar um terreno ou uma construção.}{ta.pu.me}{0}
\verb{taquara}{}{}{}{}{s.f.}{Espécie de bambu de caule muito fino, do qual se fazem cestas e peneiras; taboca.}{ta.qua.ra}{0}
\verb{taquaral}{}{}{"-ais}{}{s.m.}{Terreno extenso com muitas taquaras; bambuzal.}{ta.qua.ral}{0}
\verb{taquear}{}{}{}{}{v.t.}{Revestir assoalho de tacos.}{ta.que.ar}{\verboinum{4}}
\verb{taquicardia}{}{Med.}{}{}{s.f.}{Aceleração dos batimentos do coração.}{ta.qui.car.di.a}{0}
\verb{taquigrafar}{}{}{}{}{v.t.}{Escrever de forma abreviada, com caracteres especiais; estenografar.}{ta.qui.gra.far}{\verboinum{1}}
\verb{taquigrafia}{}{}{}{}{s.f.}{Sistema de escrita rápida que permite acompanhar a fala de um orador; estenografia.}{ta.qui.gra.fi.a}{0}
\verb{taquígrafo}{}{}{}{}{s.m.}{Indivíduo que sabe taquigrafia ou que a pratica profissionalmente; estenógrafo.}{ta.quí.gra.fo}{0}
\verb{tara}{}{}{}{}{s.f.}{Peso equivalente a embalagem de um produto.}{ta.ra}{0}
\verb{tara}{}{}{}{}{}{Peso da carroceria de um caminhão sem a carga.}{ta.ra}{0}
\verb{tara}{}{}{}{}{}{Desvio moral; perversão, depravação.}{ta.ra}{0}
\verb{tarado}{}{}{}{}{adj.}{Marcado com o peso da embalagem ou do veículo.}{ta.ra.do}{0}
\verb{tarado}{}{}{}{}{}{Diz"-se do indivíduo moralmente devasso ou que comete crimes sexuais; depravado, pervertido.}{ta.ra.do}{0}
\verb{tarado}{}{Pop.}{}{}{}{Apaixonado, gamado.}{ta.ra.do}{0}
\verb{taramela}{é}{}{}{}{s.f.}{Peça de madeira ou metal, que gira presa a prego fixado em porta ou janela servindo para fechá"-las; tramela.}{ta.ra.me.la}{0}
\verb{taramelar}{}{}{}{}{v.t.}{Fechar com taramela.}{ta.ra.me.lar}{0}
\verb{taramelar}{}{}{}{}{v.i.}{Falar muito; tagarelar.}{ta.ra.me.lar}{\verboinum{1}}
\verb{tarantela}{é}{Mús.}{}{}{s.f.}{Dança e música populares originárias de Nápoles, Itália, caracterizadas por movimento vivo e rápido.}{ta.ran.te.la}{0}
\verb{tarântula}{}{Zool.}{}{}{s.f.}{Aranha grande, de coloração marrom ou preta, cuja picada é venenosa.}{ta.rân.tu.la}{0}
\verb{tarar}{}{}{}{}{v.t.}{Pesar o produto para descontar a tara.}{ta.rar}{0}
\verb{tarar}{}{Pop.}{}{}{v.i.}{Apaixonar"-se, enamorar"-se, gamar.}{ta.rar}{\verboinum{1}}
\verb{tardança}{}{}{}{}{s.f.}{Ato ou efeito de tardar; demora, delonga.}{tar.dan.ça}{0}
\verb{tardar}{}{}{}{}{v.t.}{Demorar, retardar, adiar.}{tar.dar}{\verboinum{1}}
\verb{tarde}{}{}{}{}{s.f.}{Espaço de tempo entre o meio"-dia e o anoitecer.}{tar.de}{0}
\verb{tarde}{}{}{}{}{adv.}{Depois do tempo ou da hora combinada.}{tar.de}{0}
\verb{tardígrado}{}{}{}{}{adj.}{Que se move ou flui lentamente; lentígrado.}{tar.dí.gra.do}{0}
\verb{tardígrado}{}{Zool.}{}{}{s.m.}{Espécime dos tardígrados, filo de invertebrados microscópicos, com quatro pares de patas dotadas de garras, principalmente terrestres, com algumas espécies de água doce e salgada, capazes de resistir, inativos, a condições críticas, como o congelamento e a desidratação, por tempo indeterminado, retornando efetivamente à vida em condições favoráveis.}{tar.dí.gra.do}{0}
\verb{tardinha}{}{}{}{}{s.f.}{Fim da tarde.}{tar.di.nha}{0}
\verb{tardio}{}{}{}{}{adj.}{Que ocorre após o tempo apropriado.}{tar.di.o}{0}
\verb{tardio}{}{}{}{}{}{Que demora; lento, moroso.}{tar.di.o}{0}
\verb{tardo}{}{}{}{}{adj.}{Que se desloca ou age vagarosamente; lento, moroso.}{tar.do}{0}
\verb{tareco}{é}{}{}{}{s.m.}{Objeto velho que não tem mais utilidade; traste, cacareco.}{ta.re.co}{0}
\verb{tarefa}{é}{}{}{}{s.f.}{Trabalho ou atividade que deve ser feita dentro de um determinado prazo; incumbência, encargo. }{ta.re.fa}{0}
\verb{tarefeiro}{ê}{}{}{}{s.m.}{Indivíduo que se encarrega do cumprimento de uma tarefa; empreiteiro.}{ta.re.fei.ro}{0}
\verb{tarefeiro}{ê}{}{}{}{}{Indivíduo que recebe por tarefa executada.}{ta.re.fei.ro}{0}
\verb{tarifa}{}{}{}{}{s.f.}{Preço fixado para o transporte de passageiros a determinada distância. (\textit{A tarifa do ônibus subiu novamente nesse ano.})}{ta.ri.fa}{0}
\verb{tarifa}{}{}{}{}{}{Tabela que indica o preço dos serviços públicos.}{ta.ri.fa}{0}
\verb{tarifar}{}{}{}{}{v.t.}{Aplicar tarifa a; taxar.}{ta.ri.far}{\verboinum{1}}
\verb{tarifário}{}{}{}{}{adj.}{Relativo a tarifa.}{ta.ri.fá.rio}{0}
\verb{tarimba}{}{}{}{}{s.f.}{Tempo de prática em uma profissão ou atividade; experiência, jeito.}{ta.rim.ba}{0}
\verb{tarimba}{}{}{}{}{}{Estrado de madeira sobre o qual dormem os soldados no quartel.}{ta.rim.ba}{0}
\verb{tarimbado}{}{}{}{}{adj.}{Que tem muita prática; experiente.}{ta.rim.ba.do}{0}
\verb{tarimbar}{}{}{}{}{v.i.}{Servir nas Forças Armadas.}{ta.rim.bar}{\verboinum{1}}
\verb{tarja}{}{}{}{}{s.f.}{Faixa que contorna um objeto; orla, ornato. (\textit{A tarja preta na embalagem de certos medicamentos indica que eles podem provocar dependência.})}{tar.ja}{0}
\verb{tarja}{}{}{}{}{}{Faixa preta em sinal de luto que se usa no canto de papéis de carta.}{tar.ja}{0}
\verb{tarjar}{}{}{}{}{v.t.}{Cercar de tarja; marcar.}{tar.jar}{\verboinum{1}}
\verb{tarjeta}{ê}{}{}{}{s.f.}{Tarja pequena ou estreita.}{tar.je.ta}{0}
\verb{tarlatana}{}{}{}{}{s.f.}{Tecido de algodão de fios grossos, usado para forrar roupas.}{tar.la.ta.na}{0}
\verb{tarô}{}{}{}{}{s.m.}{Conjunto de 78 cartas, maiores que as do baralho, com desenhos simbólicos, utilizadas na cartomancia.}{ta.rô}{0}
\verb{tarol}{ó}{Mús.}{"-óis}{}{s.m.}{Pequeno tambor de som claro e forte, que se percute com duas baquetas.}{ta.rol}{0}
\verb{tarrafa}{}{}{}{}{s.f.}{Rede de pesca de malha fina e com chumbo nas pontas, que se lança com as mãos.}{tar.ra.fa}{0}
\verb{tarrafar}{}{}{}{}{v.i.}{Pescar com tarrafa.}{tar.ra.far}{\verboinum{1}}
\verb{tarraxa}{ch}{}{}{}{s.f.}{Conjunto de voltas de um parafuso que se usa para receber a porca; rosca.}{tar.ra.xa}{0}
\verb{tarraxa}{ch}{}{}{}{}{Utensílio de serralheiro usado para fazer as roscas dos parafusos.}{tar.ra.xa}{0}
\verb{tarraxar}{ch}{}{}{}{v.t.}{Apertar e rosquear a porca no parafuso; atarraxar.}{tar.ra.xar}{\verboinum{1}}
\verb{tarro}{}{}{}{}{s.m.}{Vasilha em que se recolhe o leite da ordenha.}{tar.ro}{0}
\verb{tarso}{}{Anat.}{}{}{s.m.}{Parte posterior do esqueleto do pé.}{tar.so}{0}
\verb{tartamudear}{}{}{}{}{v.i.}{Falar com dificuldade, pronunciando as palavras e repetindo as sílabas; gaguejar.}{tar.ta.mu.de.ar}{\verboinum{4}}
\verb{tartamudez}{ê}{}{}{}{s.f.}{Embaraço no falar; gagueira.}{tar.ta.mu.dez}{0}
\verb{tartamudo}{}{}{}{}{s.m.}{Indivíduo que gagueja; gago.}{tar.ta.mu.do}{0}
\verb{tartárico}{}{Quím.}{}{}{adj.}{Diz"-se de um ácido que se encontra na uva e em outros frutos e que é usado na fabricação de xaropes e refrescos.}{tar.tá.ri.co}{0}
\verb{tártaro}{}{}{}{}{adj.}{Relativo à Tartária, antiga região da Sibéria.}{tár.ta.ro}{0}
\verb{tártaro}{}{}{}{}{s.m.}{Substância espessa que o vinho ou o suco de uva deixa nas garrafas e barris; sarro.}{tár.ta.ro}{0}
\verb{tártaro}{}{}{}{}{s.m.}{Indivíduo natural ou habitante da antiga Tartária.}{tár.ta.ro}{0}
\verb{tártaro}{}{Mit.}{}{}{}{Lugar profundo e subterrâneo; inferno.}{tár.ta.ro}{0}
\verb{tártaro}{}{}{}{}{}{Substância dura, composta por cálcio e magnésio, que se forma na borda dos dentes.}{tár.ta.ro}{0}
\verb{tartaruga}{}{Zool.}{}{}{s.f.}{Nome comum dado aos répteis quelônios que têm o corpo coberto por uma carapaça e vivem no mar e nos grandes rios.}{tar.ta.ru.ga}{0}
\verb{tartufo}{}{}{}{}{s.m.}{Indivíduo hipócrita, mentiroso.}{tar.tu.fo}{0}
\verb{tartufo}{}{}{}{}{}{Beato falso, enganador.}{tar.tu.fo}{0}
\verb{tartufo}{}{Cul.}{}{}{}{Doce feito de sorvete, chocolate meio amargo e \textit{marshmallow}.}{tar.tu.fo}{0}
\verb{tarugo}{}{}{}{}{s.m.}{Pino com que se prendem duas peças de madeira.}{ta.ru.go}{0}
\verb{tasca}{}{}{}{}{s.f.}{Restaurante de baixa categoria; baiuca, boteco.}{tas.ca}{0}
\verb{tascar}{}{}{}{}{v.t.}{Tirar pedaço mordendo; mastigar.}{tas.car}{0}
\verb{tascar}{}{}{}{}{}{Despedaçar balões ou pipas caindo; rasgar.}{tas.car}{\verboinum{2}}
\verb{tasco}{}{}{}{}{s.m.}{A parte mais grossa, a casca do linho, partida em pequenos pedaços.}{tas.co}{0}
\verb{tassalho}{}{}{}{}{s.m.}{Fatia ou pedaço grande.}{tas.sa.lho}{0}
\verb{tatá}{}{Onomat.}{}{}{s.m.}{Papá, papai.}{ta.tá}{0}
\verb{tatarana}{}{Zool.}{}{}{s.f.}{Taturana.}{ta.ta.ra.na}{0}
\verb{tataraneto}{é}{}{}{}{s.m.}{Neto de bisneta ou de bisneto.}{ta.ta.ra.ne.to}{0}
\verb{tataravó}{}{}{}{}{s.f.}{Avó da bisavó ou do bisavô.}{ta.ta.ra.vó}{0}
\verb{tataravô}{}{}{}{}{s.m.}{Avô de bisavó ou de bisavô.}{ta.ta.ra.vô}{0}
\verb{tateante}{}{}{}{}{adj.2g.}{Que tateia, apalpa.}{ta.te.an.te}{0}
\verb{tatear}{}{}{}{}{v.i.}{Tocar nas coisas para se orientar, usando o sentido do tato; apalpar.}{ta.te.ar}{\verboinum{4}}
\verb{tatibitate}{}{}{}{}{adj.2g.}{Diz"-se daquele que, ao falar, troca certas consoantes, gagueja.}{ta.ti.bi.ta.te}{0}
\verb{tática}{}{}{}{}{s.f.}{Maneira hábil que se usa para se sair bem em alguma atividade; sagacidade, esperteza.}{tá.ti.ca}{0}
\verb{tática}{}{}{}{}{}{Arte de dispor e movimentar as tropas em combate.}{tá.ti.ca}{0}
\verb{tático}{}{}{}{}{adj.}{Relativo a tática.}{tá.ti.co}{0}
\verb{tático}{}{}{}{}{}{Em que há tática.}{tá.ti.co}{0}
\verb{tátil}{}{}{"-eis}{}{adj.2g.}{Relativo ao tato.}{tá.til}{0}
\verb{tátil}{}{}{"-eis}{}{}{Que é percebido pelo tato.}{tá.til}{0}
\verb{tato}{}{}{}{}{s.m.}{Sentido pelo qual se percebem e se conhecem as sensações de extensão, consistência e temperatura pelo toque da pele.}{ta.to}{0}
\verb{tato}{}{}{}{}{}{Jeito, habilidade, capacidade.}{ta.to}{0}
\verb{tatu}{}{Zool.}{}{}{s.m.}{Nome comum a diversos mamíferos xenartros sem dentes, cujo corpo é coberto por uma carapaça e que vivem em tocas e buracos que cavam no solo.}{ta.tu}{0}
\verb{tatuador}{ô}{}{}{}{s.m.}{Indivíduo que faz tatuagens.}{ta.tu.a.dor}{0}
\verb{tatuagem}{}{}{"-ens}{}{s.f.}{Ato ou efeito de tatuar.}{ta.tu.a.gem}{0}
\verb{tatuagem}{}{}{"-ens}{}{}{Desenho ou pintura que se tatuou.}{ta.tu.a.gem}{0}
\verb{tatuar}{}{}{}{}{v.t.}{Pintar ou gravar desenhos no corpo pela introdução de substâncias corantes na pele.}{ta.tu.ar}{\verboinum{1}}
\verb{tatu"-bola}{ó}{Zool.}{tatus"-bola \textit{ou} tatus"-bolas ⟨ó⟩}{}{s.m.}{Nome dado a certos tatus que, para se defenderem, conseguem enrolar o próprio corpo dentro da carapaça, formando uma bola.}{ta.tu"-bo.la}{0}
%\verb{}{}{}{}{}{}{}{}{0}
\verb{tatuí}{}{Zool.}{}{}{s.m.}{Crustáceo, semelhante ao camarão, encontrado na zona de arrebentação das praias, onde vive enterrado na areia.}{ta.tu.í}{0}
\verb{tatupeba}{é}{Zool.}{}{}{s.m.}{Tatu de coloração marrom e cuja carapaça apresenta pelos esparsos.}{ta.tu.pe.ba}{0}
\verb{taturana}{}{Zool.}{}{}{s.f.}{Lagarta capaz de produzir queimadura na pele de quem as toca; tatarana.}{ta.tu.ra.na}{0}
\verb{tatuzinho}{}{Zool.}{}{}{s.m.}{Espécie de crustáceo que, ao ser tocado, enrola o próprio corpo, semelhante ao tatu"-bola.}{ta.tu.zi.nho}{0}
\verb{tau}{}{}{}{}{s.m.}{Décima nona letra do alfabeto grego, correspondente ao \textit{T} do latim e das línguas neolatinas.}{tau}{0}
\verb{tau}{}{}{}{}{}{Figura outrora usada nas vestes dos cônegos de Santo Antão.}{tau}{0}
\verb{tau}{}{Onomat.}{}{}{}{Expressão que tenta representar o som de tiro, detonação ou pancada.}{tau}{0}
\verb{taumaturgia}{}{}{}{}{s.f.}{Talento para atrair e impressionar pessoas com milagres ou feitos prodigiosos.}{tau.ma.tur.gi.a}{0}
\verb{taumaturgo}{}{}{}{}{adj.}{Que opera milagres ou feitos prodigiosos; milagreiro.}{tau.ma.tur.go}{0}
\verb{tauriano}{}{Astrol.}{}{}{adj.}{Diz"-se do indivíduo nascido sob o signo de touro; taurino.}{tau.ri.a.no}{0}
\verb{tauriforme}{ó}{}{}{}{adj.2g.}{Em forma de touro.}{tau.ri.for.me}{0}
\verb{taurino}{}{}{}{}{adj.}{Relativo a ou próprio de touro.}{tau.ri.no}{0}
\verb{taurino}{}{Astrol.}{}{}{s.m.}{Indivíduo que nasceu sob o signo de touro.   }{tau.ri.no}{0}
\verb{taurino}{}{Astrol.}{}{}{adj.}{Relativo ou pertencente a esse signo. }{tau.ri.no}{0}
\verb{tauromaquia}{}{}{}{}{s.f.}{Técnica de tourear; tourada.}{tau.ro.ma.qui.a}{0}
\verb{tautologia}{}{}{}{}{s.f.}{Repetição das mesmas ideias com palavras diferentes; redundância.}{tau.to.lo.gi.a}{0}
\verb{tautológico}{}{}{}{}{adj.}{Relativo a tautologia; redundante.}{tau.to.ló.gi.co}{0}
\verb{tauxia}{ch}{}{}{}{s.f.}{Incrustação de ouro, prata ou cobre em obra de aço ou ferro.}{tau.xi.a}{0}
\verb{tauxiar}{ch}{}{}{}{v.t.}{Lavrar ou enfeitar com tauxia; embutir.}{tau.xi.ar}{\verboinum{1}}
\verb{taverna}{é}{}{}{}{s.f.}{Taberna.}{ta.ver.na}{0}
\verb{taverneiro}{ê}{}{}{}{s.m.}{Taberneiro.}{ta.ver.nei.ro}{0}
\verb{tavolagem}{}{}{"-ens}{}{s.f.}{Vício de jogar; jogo.}{ta.vo.la.gem}{0}
\verb{tavolagem}{}{}{"-ens}{}{}{Casa de jogo.}{ta.vo.la.gem}{0}
\verb{taxa}{ch}{}{}{}{s.f.}{Preço que se paga por um serviço público; tributo, imposto.}{ta.xa}{0}
\verb{taxa}{ch}{}{}{}{}{Preço cobrado ao usuário pela prestação de algum serviço. (\textit{Os pais tiveram que pagar a taxa de matrícula dos filhos.})}{ta.xa}{0}
\verb{taxa}{ch}{}{}{}{}{Relação de uma quantidade com outra, expressa por porcentagem.}{ta.xa}{0}
\verb{taxação}{ch}{}{"-ões}{}{s.f.}{Ato ou efeito de taxar; tributação.}{ta.xa.ção}{0}
\verb{taxar}{ch}{}{}{}{v.t.}{Estabelecer taxa ou imposto sobre algum serviço; tributar.}{ta.xar}{\verboinum{1}}
\verb{taxativo}{ch}{}{}{}{adj.}{Que não admite contestação; categórico, terminante, imperativo.}{ta.xa.ti.vo}{0}
\verb{táxi}{cs}{}{}{}{s.m.}{Carro de aluguel para transporte de passageiros conduzido por um motorista e provido de um taxímetro.}{tá.xi}{0}
\verb{taxiar}{cs}{}{}{}{v.t.}{Rodar (o avião) na pista.}{ta.xi.ar}{\verboinum{1}}
\verb{taxidermia}{cs}{}{}{}{s.f.}{Técnica de empalhar animais mortos a fim de conservar suas características externas.}{ta.xi.der.mi.a}{0}
\verb{taxidermista}{cs}{}{}{}{s.2g.}{Indivíduo que se dedica à taxidermia.}{ta.xi.der.mis.ta}{0}
\verb{taxímetro}{cs}{}{}{}{s.m.}{Aparelho que, em carros de aluguel, mede e registra o preço que se deve pagar pelo percurso efetuado.  }{ta.xí.me.tro}{0}
\verb{taxinomia}{cs}{}{}{}{s.f.}{Taxionomia.}{ta.xi.no.mi.a}{0}
\verb{taxionomia}{cs}{}{}{}{s.f.}{Ciência ou técnica de classificação.}{ta.xi.o.no.mi.a}{0}
\verb{taxionomia}{cs}{Biol.}{}{}{}{Parte da botânica e da zoologia que trata da descrição, identificação e classificação das plantas e dos animais.}{ta.xi.o.no.mi.a}{0}
\verb{taxista}{cs}{}{}{}{s.2g.}{Motorista de táxi. (\textit{Meu pai e meus irmãos são taxistas.})}{ta.xis.ta}{0}
\verb{taxonomia}{cs}{}{}{}{s.f.}{Taxionomia.}{ta.xo.no.mi.a}{0}
\verb{Tb}{}{Quím.}{}{}{}{Símb. do \textit{térbio}. }{Tb}{0}
\verb{TB}{}{Informát.}{}{}{}{Símb. de \textit{terabyte}.}{TB}{0}
\verb{Tc}{}{Quím.}{}{}{}{Símb. do \textit{tecnécio}.}{Tc}{0}
\verb{tchã}{}{Pop.}{}{}{s.m.}{Toque especial; apuro, requinte. }{tchã}{0}
\verb{tchã}{}{}{}{}{}{Charme, encanto pessoal.}{tchã}{0}
\verb{tchau}{}{}{}{}{interj.}{Expressão usada como despedida.}{tchau}{0}
\verb{tchau}{}{}{}{}{s.m.}{Aceno, adeus; até logo.}{tchau}{0}
\verb{tcheco}{é}{}{}{}{adj.}{Relativo à Boêmia e à Morávia, ou à atual República Tcheca.}{tche.co}{0}
\verb{tcheco}{é}{}{}{}{s.m.}{Indivíduo natural ou habitante desse país.}{tche.co}{0}
\verb{tcheco}{é}{}{}{}{}{A língua falada pelos tchecos.}{tche.co}{0}
\verb{tchecoslovaco}{}{}{}{}{adj.}{Relativo à antiga República Socialista Tchecoslovaca (Europa Central), desmembrada em República Tcheca e República Eslovaca.}{tche.cos.lo.va.co}{0}
\verb{tchecoslovaco}{}{}{}{}{s.m.}{Indivíduo natural ou habitante dessa república.}{tche.cos.lo.va.co}{0}
\verb{Te}{}{Quím.}{}{}{}{Símb. do \textit{telúrio}. }{Te}{0}
\verb{tê}{}{}{}{}{s.m.}{Nome da letra \textit{t}.  }{tê}{0}
\verb{te}{}{}{}{}{pron.}{Tua pessoa; a ti.}{te}{0}
\verb{te}{}{}{}{}{}{Para tua pessoa; para ti.}{te}{0}
\verb{te}{}{}{}{}{}{Em ti.}{te}{0}
\verb{te}{}{}{}{}{}{De ti.}{te}{0}
\verb{tear}{}{}{}{}{s.m.}{Máquina para transformar fios em pano.}{te.ar}{0}
\verb{teatral}{}{}{"-ais}{}{adj.2g.}{Relativo a teatro.}{te.a.tral}{0}
\verb{teatral}{}{}{"-ais}{}{}{Pouco natural; forçado.}{te.a.tral}{0}
\verb{teatralidade}{}{}{}{}{s.f.}{Qualidade do que é teatral.}{te.a.tra.li.da.de}{0}
\verb{teatralidade}{}{}{}{}{}{Tom teatral.}{te.a.tra.li.da.de}{0}
\verb{teatralizar}{}{}{}{}{v.t.}{Tornar representável no teatro.}{te.a.tra.li.zar}{0}
\verb{teatralizar}{}{Fig.}{}{}{}{Tornar comovente, dramático.}{te.a.tra.li.zar}{\verboinum{1}}
\verb{teatro}{}{}{}{}{s.m.}{A arte de representar.}{te.a.tro}{0}
\verb{teatro}{}{}{}{}{}{Local destinado a apresentação de obras dramáticas, óperas etc.}{te.a.tro}{0}
\verb{teatro}{}{}{}{}{}{Conjunto das obras escritas para serem representadas.}{te.a.tro}{0}
\verb{teatrólogo}{}{}{}{}{s.m.}{Indivíduo que escreve peças teatrais.}{te.a.tró.lo.go}{0}
\verb{tebaida}{}{}{}{}{s.f.}{Profunda solidão, retiro, ermo.}{te.bai.da}{0}
\verb{teca}{é}{Bot.}{}{}{s.f.}{Árvore de grande porte, nativa da Ásia, cuja madeira é usada na construção naval.}{te.ca}{0}
\verb{tecedor}{ô}{}{}{}{adj.}{Que tece pano.}{te.ce.dor}{0}
\verb{tecedor}{ô}{Pop.}{}{}{}{Intrigante, mexeriqueiro.}{te.ce.dor}{0}
\verb{tecedura}{}{}{}{}{s.f.}{Ato de tecer; tecelagem.}{te.ce.du.ra}{0}
\verb{tecedura}{}{}{}{}{}{Conjunto dos fios de um tecido.}{te.ce.du.ra}{0}
\verb{tecelagem}{}{}{"-ens}{}{s.f.}{Ato ou efeito de tecer; tecedura.}{te.ce.la.gem}{0}
\verb{tecelagem}{}{}{"-ens}{}{}{Ofício do tecelão.}{te.ce.la.gem}{0}
\verb{tecelagem}{}{}{"-ens}{}{}{Indústria de tecido.}{te.ce.la.gem}{0}
\verb{tecelão}{}{}{"-ões}{tecelã \textit{ou} teceloa ⟨ô⟩}{s.m.}{Indivíduo que tece o pano no tear.}{te.ce.lão}{0}
\verb{tecer}{ê}{}{}{}{v.t.}{Confeccionar tecidos com fios.}{te.cer}{0}
\verb{tecer}{ê}{}{}{}{}{Fazer alguma tipo de avaliação ou crítica.}{te.cer}{\verboinum{15}}
\verb{tecido}{}{}{}{}{s.m.}{Produto artesanal ou industrial resultante da tecedura; fazenda, pano.}{te.ci.do}{0}
\verb{tecido}{}{Biol.}{}{}{}{Conjunto de células do organismo para desenvolver determinada função.}{te.ci.do}{0}
\verb{tecla}{é}{}{}{}{s.f.}{Pequena peça que, ao ser pressionada, faz um mecanismo funcionar.}{te.cla}{0}
\verb{tecladista}{}{}{}{}{s.2g.}{Músico que toca teclado.}{te.cla.dis.ta}{0}
\verb{teclado}{}{}{}{}{s.m.}{Conjunto de teclas de um instrumento musical, computador etc.}{te.cla.do}{0}
\verb{teclar}{}{}{}{}{v.t.}{Pressionar as teclas.}{te.clar}{\verboinum{1}}
\verb{tecnécio}{}{Quím.}{}{}{s.m.}{Elemento químico metálico, cinzento, denso, usado em radiologia. \elemento{43}{(98)}{Tc}.}{tec.né.cio}{0}
\verb{técnica}{}{}{}{}{s.f.}{Conjunto dos processos necessários à execução de uma arte ou profissão.}{téc.ni.ca}{0}
\verb{técnica}{}{}{}{}{}{Prática, especialização.}{téc.ni.ca}{0}
\verb{tecnicalidade}{}{}{}{}{s.f.}{Qualidade de técnico; tecnicidade.}{tec.ni.ca.li.da.de}{0}
\verb{tecnicidade}{}{}{}{}{s.f.}{Qualidade de técnico.}{tec.ni.ci.da.de}{0}
\verb{tecnicismo}{}{}{}{}{s.m.}{Linguagem própria de uma arte ou ciência, com muitos termos técnicos; tecnicidade.}{tec.ni.cis.mo}{0}
\verb{técnico}{}{}{}{}{adj.}{Que é próprio de arte, técnica ou ciência.}{téc.ni.co}{0}
\verb{técnico}{}{}{}{}{s.m.}{Indivíduo que conhece uma técnica; especialista, perito.}{téc.ni.co}{0}
\verb{técnico}{}{}{}{}{}{Profissional que treina, dirige ou orienta treinos ou jogos; treinador.}{téc.ni.co}{0}
\verb{tecnicolor}{ô}{}{}{}{s.m.}{Processo utilizado para a produção de filmes cinematográficos coloridos.}{tec.ni.co.lor}{0}
\verb{tecnicolor}{ô}{}{}{}{adj.2g.}{Diz"-se de qualquer filme colorido.}{tec.ni.co.lor}{0}
\verb{tecnocracia}{}{}{}{}{s.f.}{Sistema político e social em que predominam os técnicos.}{tec.no.cra.ci.a}{0}
\verb{tecnocrata}{}{}{}{}{s.2g.}{Indivíduo que é adepto ou partidário da tecnocracia.}{tec.no.cra.ta}{0}
\verb{tecnocrático}{}{}{}{}{adj.}{Relativo a tecnocracia.}{tec.no.crá.ti.co}{0}
\verb{tecnologia}{}{}{}{}{s.f.}{Conjunto de conhecimentos, processos, métodos que se aplicam a um determinado ramo de atividade. }{tec.no.lo.gi.a}{0}
\verb{tecnológico}{}{}{}{}{adj.}{Relativo a tecnologia.}{tec.no.ló.gi.co}{0}
\verb{tecnólogo}{}{}{}{}{s.m.}{Indivíduo versado em tecnologia.}{tec.nó.lo.go}{0}
\verb{teco"-teco}{é\ldots{}é}{}{teco"-tecos ⟨é\ldots{}é⟩}{}{s.m.}{Pequeno avião, próprio para treinamento de pilotos ou para trajetos curtos.}{te.co"-te.co}{0}
\verb{tectônica}{}{Geol.}{}{}{s.f.}{Estudo da formação das montanhas e das deformações da crosta terrestre.}{tec.tô.ni.ca}{0}
\verb{tectônica}{}{}{}{}{}{Arte de construir edifícios.}{tec.tô.ni.ca}{0}
\verb{tédio}{}{}{}{}{s.m.}{Aborrecimento, desgosto, enfado, fastio, com ou sem motivo explícito.}{té.dio}{0}
\verb{tedioso}{ô}{}{"-osos ⟨ó⟩}{"-osa ⟨ó⟩}{adj.}{Que causa ou revela tédio.}{te.di.o.so}{0}
\verb{tedioso}{ô}{}{"-osos ⟨ó⟩}{"-osa ⟨ó⟩}{}{Aborrecido, enfadonho, fastidioso.}{te.di.o.so}{0}
\verb{tegumentar}{}{}{}{}{adj.2g.}{Relativo a tegumento, ou da natureza dele.}{te.gu.men.tar}{0}
\verb{tegumento}{}{}{}{}{s.m.}{Conjunto formado pela pele e seus anexos (pelos, cabelos, unhas e glândulas).}{te.gu.men.to}{0}
\verb{teia}{ê}{}{}{}{s.f.}{Emaranhado de fios.}{tei.a}{0}
\verb{teia}{ê}{}{}{}{}{Rede tecida pela aranha.}{tei.a}{0}
\verb{teia}{ê}{}{}{}{}{Organização, estrutura.}{tei.a}{0}
\verb{teima}{}{}{}{}{s.f.}{Repetição proposital de uma atitude, comportamento; insistência.}{tei.ma}{0}
\verb{teima}{}{}{}{}{}{Perseverança.}{tei.ma}{0}
\verb{teimar}{}{}{}{}{v.t.}{Fazer algo repetidamente.}{tei.mar}{0}
\verb{teimar}{}{}{}{}{}{Não desistir; perseverar.}{tei.mar}{\verboinum{1}}
\verb{teimosia}{}{}{}{}{s.f.}{Insistência obstinada.}{tei.mo.si.a}{0}
\verb{teimoso}{ô}{}{"-osos ⟨ó⟩}{"-osa ⟨ó⟩}{adj.}{Que teima; insistente, obstinado, persistente.}{tei.mo.so}{0}
\verb{teína}{}{}{}{}{s.f.}{Princípio ativo do chá.}{te.í.na}{0}
\verb{teísmo}{}{}{}{}{s.m.}{Doutrina da existência de um único Deus.}{te.ís.mo}{0}
\verb{teísmo}{}{Med.}{}{}{}{Intoxicação por ingestão excessiva de chá.}{te.ís.mo}{0}
\verb{teiú}{}{Zool.}{}{}{s.m.}{Réptil, semelhante a um lagarto, de corpo coberto de escamas azuis, cauda comprida e pernas curtas, que anda com a barriga quase colada no chão e pode ter dois metros de comprimento; lagarto.}{tei.ú}{0}
\verb{tejadilho}{}{}{}{}{s.m.}{Teto de veículos.}{te.ja.di.lho}{0}
\verb{tela}{é}{}{}{}{s.f.}{Tecido preparado e esticado sobre o qual se pintam quadros.}{te.la}{0}
\verb{tela}{é}{}{}{}{}{Quadro, pintura.}{te.la}{0}
\verb{tela}{é}{}{}{}{}{Painel sobre o qual se projetam filmes, \textit{slides} etc.}{te.la}{0}
\verb{tela}{é}{}{}{}{}{Tecido de arame próprio para cercados.}{te.la}{0}
\verb{tela}{é}{}{}{}{}{Tubo de imgem de \textsc{tv}, computador etc.}{te.la}{0}
\verb{telão}{}{}{"-ões}{}{s.m.}{Sistema de projeção de imagens em tela grande, como no cinema.}{te.lão}{0}
\verb{telecomandar}{}{}{}{}{v.t.}{Emitir sinais por linha de comunicação para comandar a distância.}{te.le.co.man.dar}{\verboinum{1}}
\verb{telecomunicação}{}{}{"-ões}{}{s.f.}{Sistema de comunicação a distância por fios, ondas eletromagnéticas etc.}{te.le.co.mu.ni.ca.ção}{0}
\verb{teleconferência}{}{}{}{}{s.f.}{Conferência em que os interlocutores, distanciados, comunicam"-se por telefonia, televisão ou computador.}{te.le.con.fe.rên.cia}{0}
\verb{telecurso}{}{}{}{}{s.m.}{Curso de ensino transmitido por televisão educativa.}{te.le.cur.so}{0}
\verb{teledifusão}{}{}{"-ões}{}{s.f.}{Divulgação pela televisão.}{te.le.di.fu.são}{0}
\verb{teleducação}{}{}{"-ões}{}{s.f.}{Ensino a distância.}{te.le.du.ca.ção}{0}
\verb{teleférico}{}{}{}{}{adj.}{Que transporta a distância.}{te.le.fé.ri.co}{0}
\verb{teleférico}{}{}{}{}{s.m.}{Cabine ou assento suspenso por cabos móveis para transportar pessoas.}{te.le.fé.ri.co}{0}
\verb{telefonada}{}{Pop.}{}{}{s.f.}{Ato ou efeito de telefonar; telefonema.}{te.le.fo.na.da}{0}
\verb{telefonar}{}{}{}{}{v.t.}{Comunicar"-se por telefone.}{te.le.fo.nar}{\verboinum{1}}
\verb{telefone}{}{}{}{}{s.m.}{Aparelho para comunicação a distância, que converte sons em sinais eletromagnéticos.}{te.le.fo.ne}{0}
\verb{telefone}{}{}{}{}{}{Conjunto específico de números para chamar um local ou uma pessoa por via telefônica.}{te.le.fo.ne}{0}
\verb{telefonema}{}{}{}{}{s.m.}{Ato ou efeito de telefonar.}{te.le.fo.ne.ma}{0}
\verb{telefonia}{}{}{}{}{s.f.}{Sistema de transmissão de sons a distância, por meio de cabos ou ondas de radiofrequência.}{te.le.fo.ni.a}{0}
\verb{telefônico}{}{}{}{}{adj.}{Relativo a telefone ou a telefonia.}{te.le.fô.ni.co}{0}
\verb{telefonista}{}{}{}{}{s.2g.}{Pessoa encarregada de atender e distribuir as ligações em uma empresa ou repartição.}{te.le.fo.nis.ta}{0}
\verb{telefonista}{}{}{}{}{}{Funcionário de empresa telefônica que atende os usuários e opera os sistemas de telefonia.}{te.le.fo.nis.ta}{0}
\verb{telefoto}{ó}{}{}{}{s.f.}{Sistema de transmissão de imagens fotográficas a distância.}{te.le.fo.to}{0}
\verb{telefotografia}{}{}{}{}{s.f.}{Técnica de se tomar fotografias a distância.}{te.le.fo.to.gra.fi.a}{0}
\verb{telefotografia}{}{}{}{}{}{Telefoto.}{te.le.fo.to.gra.fi.a}{0}
\verb{telegrafar}{}{}{}{}{v.t.}{Comunicar"-se por telégrafo.}{te.le.gra.far}{\verboinum{1}}
\verb{telegrafia}{}{}{}{}{s.f.}{Sistema de comunicação a distância que utiliza um código de sinais (Morse) para codificar as palavras.}{te.le.gra.fi.a}{0}
\verb{telegráfico}{}{}{}{}{adj.}{Relativo a telégrafo ou a telegrafia.}{te.le.grá.fi.co}{0}
\verb{telegrafista}{}{}{}{}{s.2g.}{Profissional conhecedor do código Morse e que opera um telégrafo.}{te.le.gra.fis.ta}{0}
\verb{telégrafo}{}{}{}{}{s.m.}{Aparelho que transmite e recebe sinais de código Morse.}{te.lé.gra.fo}{0}
\verb{telegrama}{}{}{}{}{s.m.}{Mensagem recebida por meio de telégrafo.}{te.le.gra.ma}{0}
\verb{teleguiado}{}{}{}{}{adj.}{Diz"-se de aparelho comandado ou guiado a distância por meio de ondas eletromagnéticas.}{te.le.gui.a.do}{0}
\verb{teleguiar}{}{}{}{}{v.t.}{Guiar ou comandar a distância.}{te.le.gui.ar}{\verboinum{1}}
\verb{telejornal}{}{}{"-ais}{}{s.m.}{Noticiário televisivo.}{te.le.jor.nal}{0}
\verb{telejornalismo}{}{}{}{}{s.m.}{Atividade de produção de matérias e programas jornalísticos para serem transmitidos pela televisão.}{te.le.jor.na.lis.mo}{0}
\verb{telemarketing}{}{}{}{}{s.m.}{Atendimento ou prospecção de clientes por meio de telefone.}{\textit{telemarketing}}{0}
\verb{telemetria}{}{}{}{}{s.f.}{Técnica de medição de distâncias.}{te.le.me.tri.a}{0}
\verb{telêmetro}{}{}{}{}{s.m.}{Aparelho utilizado para medir distâncias.}{te.lê.me.tro}{0}
\verb{telenovela}{é}{}{}{}{s.f.}{Obra de dramaturgia apresentada pela televisão, geralmente em capítulos diários.}{te.le.no.ve.la}{0}
\verb{teleobjetiva}{}{}{}{}{s.f.}{Objetiva com lentes de grande distância focal, própria para fotografar objetos a grande distância.}{te.le.ob.je.ti.va}{0}
\verb{teleologia}{}{Filos.}{}{}{s.f.}{Doutrina que concebe a existência de metas ou finalidades guiando os eventos da natureza e da história.}{te.le.o.lo.gi.a}{0}
\verb{teleósteo}{}{Zool.}{}{}{adj.}{Relativo aos teleósteos, grupo que engloba os peixes ósseos.}{te.le.ós.teo}{0}
\verb{teleósteo}{}{}{}{}{s.m.}{Espécime dos telósteos.}{te.le.ós.teo}{0}
\verb{telepata}{}{}{}{}{adj.2g.}{Que pratica telepatia.}{te.le.pa.ta}{0}
\verb{telepatia}{}{}{}{}{s.f.}{Comunicação entre duas mentes sem intermédio da língua.}{te.le.pa.ti.a}{0}
\verb{telepático}{}{}{}{}{adj.}{Relativo a telepatia.}{te.le.pá.ti.co}{0}
\verb{teleprompter}{}{}{}{}{s.m.}{Dispositivo que se adapta a uma câmera para expor o texto aos locutores ou atores.}{te.le.promp.ter}{0}
\verb{telescopia}{}{}{}{}{s.f.}{Utilização de telescópio para fins específicos.}{te.les.co.pi.a}{0}
\verb{telescópico}{}{}{}{}{adj.}{Relativo a telescópico.}{te.les.có.pi.co}{0}
\verb{telescópico}{}{}{}{}{}{Feito com auxílio de um telescópio.}{te.les.có.pi.co}{0}
\verb{telescópico}{}{}{}{}{}{Diz"-se de objeto que, formado por um conjunto de cilindros adaptados uns dentro de outros, pode variar de comprimento.}{te.les.có.pi.co}{0}
\verb{telescópio}{}{}{}{}{s.m.}{Instrumento óptico constituído por lentes que produzem uma imagem ampliada de objetos muito distantes.}{te.les.có.pio}{0}
\verb{telespectador}{ô}{}{}{}{s.m.}{Indivíduo que assiste a algo pela televisão.}{te.les.pec.ta.dor}{0}
\verb{teletipo}{}{}{}{}{s.m.}{Aparelho de transmissão de textos a longa distância, sendo o transmissor um teclado e o receptor um dispositivo de impressão.}{te.le.ti.po}{0}
\verb{televisão}{}{}{"-ões}{}{}{Aparelho receptor de sons e imagens por esse sistema; televisor.}{te.le.vi.são}{0}
\verb{televisão}{}{}{"-ões}{}{s.f.}{Sistema de transmissão de som e imagem por onda eletromagnética ou cabo.}{te.le.vi.são}{0}
\verb{televisar}{}{}{}{}{v.t.}{Televisionar.}{te.le.vi.sar}{\verboinum{1}}
\verb{televisionar}{}{}{}{}{v.t.}{Transmitir programa, evento pelo sistema de televisão.}{te.le.vi.si.o.nar}{\verboinum{1}}
\verb{televisivo}{}{}{}{}{adj.}{Relativo a televisão.}{te.le.vi.si.vo}{0}
\verb{televisor}{ô}{}{}{}{s.m.}{Aparelho receptor de som e imagem transmitidos pelo sistema de televisão.}{te.le.vi.sor}{0}
\verb{televisora}{ô}{}{}{}{s.f.}{Estação transmissora de som e imagem pelo sistema de televisão.}{te.le.vi.so.ra}{0}
\verb{televisual}{}{}{"-ais}{}{adj.2g.}{Relativo a televisão.}{te.le.vi.su.al}{0}
\verb{telex}{écs}{}{}{}{s.m.}{Sistema de transmissão e recepção de mensagens telegráficas em que os usuários comunicam"-se diretamente.}{te.lex}{0}
\verb{telex}{écs}{}{}{}{}{Mensagem enviada ou recebida por esse sistema.}{te.lex}{0}
\verb{telha}{ê}{}{}{}{s.f.}{Peça, geralmente de barro cozido ou cimento"-amianto, usada na cobertura de casas.}{te.lha}{0}
\verb{telhado}{}{}{}{}{s.m.}{Cobertura de uma casa, constituída de uma estrutura e uma camada de telhas.}{te.lha.do}{0}
\verb{telhar}{}{}{}{}{v.t.}{Cobrir com telhas.}{te.lhar}{\verboinum{1}}
\verb{telha"-vã}{}{}{}{}{s.f.}{Telhado sem forro.}{te.lha"-vã}{0}
\verb{telheiro}{ê}{}{}{}{s.m.}{Fabricante ou vendedor de telha.}{te.lhei.ro}{0}
\verb{telheiro}{ê}{}{}{}{}{Cobertura de telhas ou de outro material, para proteção ou abrigo; alpendre.}{te.lhei.ro}{0}
\verb{telúrico}{}{}{}{}{adj.}{Relativo à Terra ou ao solo.}{te.lú.ri.co}{0}
\verb{telúrico}{}{Quím.}{}{}{}{Relativo ao telúrio.}{te.lú.ri.co}{0}
\verb{telúrio}{}{Quím.}{}{}{s.m.}{Elemento químico metálico, branco"-prateado, usado em ligas com o aço e o chumbo, em cerâmica e em dispositivos termoelétricos. \elemento{52}{127.6}{Te}.}{te.lú.rio}{0}
\verb{tema}{}{}{}{}{s.m.}{Assunto pelo qual se desenvolve um texto.}{te.ma}{0}
\verb{tema}{}{Mús.}{}{}{}{Frase melódica sobre a qual se desenvolve a composição.}{te.ma}{0}
\verb{tema}{}{Gram.}{}{}{}{Conjunto formado pelo radical e uma vogal temática ao qual se acrescentam os sufixos flexionais.}{te.ma}{0}
\verb{temário}{}{}{}{}{s.m.}{Conjunto dos temas a serem discutidos em um congresso; encontro.}{te.má.rio}{0}
\verb{temática}{}{}{}{}{s.f.}{Conjunto dos temas literários ou artísticos que caracterizam um movimento, escola ou artista.}{te.má.ti.ca}{0}
\verb{temático}{}{}{}{}{adj.}{Relativo a tema.}{te.má.ti.co}{0}
\verb{temente}{}{}{}{}{adj.2g.}{Que teme; receoso, obediente.}{te.men.te}{0}
\verb{temer}{ê}{}{}{}{v.t.}{Ter medo de; sentir temor; recear.}{te.mer}{\verboinum{12}}
\verb{temerário}{}{}{}{}{adj.}{Que é arriscado, audacioso, perigoso.}{te.me.rá.rio}{0}
\verb{temeridade}{}{}{}{}{s.f.}{Qualidade, ato ou dito de temerário.}{te.me.ri.da.de}{0}
\verb{temeridade}{}{}{}{}{}{Ousadia excessiva; imprudência.}{te.me.ri.da.de}{0}
\verb{temeroso}{ô}{}{"-osos ⟨ó⟩}{"-osa ⟨ó⟩}{adj.}{Que teme; receoso, medroso.}{te.me.ro.so}{0}
\verb{temeroso}{ô}{}{"-osos ⟨ó⟩}{"-osa ⟨ó⟩}{}{Que causa temor; assustador.}{te.me.ro.so}{0}
\verb{temido}{}{}{}{}{adj.}{Que provoca medo; temeroso.}{te.mi.do}{0}
\verb{temido}{}{}{}{}{}{Que nada teme; valente.}{te.mi.do}{0}
\verb{temível}{}{}{"-eis}{}{adj.2g.}{Que se deve temer.}{te.mí.vel}{0}
\verb{temor}{ô}{}{}{}{s.m.}{Medo, receio, susto.}{te.mor}{0}
\verb{temor}{ô}{}{}{}{}{Diligência, empenho, pontualidade.}{te.mor}{0}
\verb{têmpera}{}{}{}{}{s.f.}{Ato ou efeito de temperar; temperamento.  }{têm.pe.ra}{0}
\verb{têmpera}{}{}{}{}{}{Consistência que se dá aos metais, principalmente ao aço, introduzindo"-os muito quentes em água fria. }{têm.pe.ra}{0}
\verb{têmpera}{}{}{}{}{}{Mistura de pigmento e cola usada em pinturas artísticas.}{têm.pe.ra}{0}
\verb{temperado}{}{}{}{}{adj.}{Diz"-se de metal que possui têmpera.}{tem.pe.ra.do}{0}
\verb{temperado}{}{}{}{}{}{Que tem tempero.}{tem.pe.ra.do}{0}
\verb{temperado}{}{}{}{}{}{Diz"-se de clima cuja temperatura não é nem baixa nem elevada.}{tem.pe.ra.do}{0}
\verb{temperamental}{}{}{"-ais}{}{adj.2g.}{Relativo a temperamento.}{tem.pe.ra.men.tal}{0}
\verb{temperamental}{}{}{"-ais}{}{}{Diz"-se de quem age obedecendo apenas aos impulsos de seu temperamento.}{tem.pe.ra.men.tal}{0}
\verb{temperamento}{}{}{}{}{s.m.}{Conjunto das qualidades que determinam o comportamento de uma pessoa; índole, caráter.}{tem.pe.ra.men.to}{0}
\verb{temperamento}{}{}{}{}{}{Estado fisiológico; constituição física particular.}{tem.pe.ra.men.to}{0}
\verb{temperança}{}{}{}{}{s.f.}{Qualidade de quem é moderado; moderação, parcimônia.}{tem.pe.ran.ça}{0}
\verb{temperar}{}{}{}{}{v.t.}{Pôr tempero em.}{tem.pe.rar}{0}
\verb{temperar}{}{}{}{}{}{Amenizar, suavizar.}{tem.pe.rar}{0}
\verb{temperar}{}{}{}{}{}{Dar temperatura amena a .}{tem.pe.rar}{0}
\verb{temperar}{}{}{}{}{}{Dar a um metal consistência ou rijeza.}{tem.pe.rar}{\verboinum{1}}
\verb{temperatura}{}{}{}{}{s.f.}{Grau ou quantidade de calor existente num corpo ou num lugar.}{tem.pe.ra.tu.ra}{0}
\verb{temperatura}{}{}{}{}{}{Clima.}{tem.pe.ra.tu.ra}{0}
\verb{tempero}{ê}{}{}{}{s.m.}{Substância usada para condimentar o alimento.}{tem.pe.ro}{0}
\verb{tempestade}{}{}{}{}{s.f.}{Vento ou chuva violenta, com raio, trovão etc.}{tem.pes.ta.de}{0}
\verb{tempestivo}{}{}{}{}{adj.}{Que ocorre no momento certo; oportuno.}{tem.pes.ti.vo}{0}
\verb{tempestuoso}{ô}{}{"-osos ⟨ó⟩}{"-osa ⟨ó⟩}{adj.}{Que é sujeito a tempestades.}{tem.pes.tu.o.so}{0}
\verb{tempestuoso}{ô}{}{"-osos ⟨ó⟩}{"-osa ⟨ó⟩}{}{Que causa ou traz tempestade.}{tem.pes.tu.o.so}{0}
\verb{tempestuoso}{ô}{}{"-osos ⟨ó⟩}{"-osa ⟨ó⟩}{}{Violento, revolto.}{tem.pes.tu.o.so}{0}
\verb{templo}{}{}{}{}{s.m.}{Edifício destinado ao culto religioso.}{tem.plo}{0}
\verb{tempo}{}{}{}{}{}{Época.}{tem.po}{0}
\verb{tempo}{}{}{}{}{s.m.}{Medida de duração dos fenômenos.}{tem.po}{0}
\verb{tempo}{}{}{}{}{}{Estado da atmosfera.}{tem.po}{0}
\verb{tempo}{}{Gram.}{}{}{}{Variação do verbo que indica o momento em que se dá o fato expresso pelo verbo.}{tem.po}{0}
\verb{tempo"-quente}{}{}{tempos"-quentes}{}{s.m.}{Confusão, desordem.}{tem.po"-quen.te}{0}
\verb{tempo"-quente}{}{}{tempos"-quentes}{}{}{Discussão acalorada; briga.}{tem.po"-quen.te}{0}
\verb{têmpora}{}{Anat.}{}{}{s.f.}{Cada uma das duas porções laterais da cabeça.}{têm.po.ra}{0}
\verb{temporada}{}{}{}{}{s.f.}{Certo espaço de tempo.}{tem.po.ra.da}{0}
\verb{temporada}{}{}{}{}{}{Estação do ano.}{tem.po.ra.da}{0}
\verb{temporada}{}{}{}{}{}{Época propícia para a realização de certas atividades.}{tem.po.ra.da}{0}
\verb{temporal}{}{}{"-ais}{}{adj.2g.}{Relativo a tempo; temporário.}{tem.po.ral}{0}
\verb{temporal}{}{}{"-ais}{}{}{Relativo ao mundo, às coisas materiais; mundano.}{tem.po.ral}{0}
\verb{temporal}{}{Anat.}{"-ais}{}{}{Relativo às têmporas.}{tem.po.ral}{0}
\verb{temporal}{}{Anat.}{"-ais}{}{s.m.}{Cada um dos ossos situados na parte lateral inferior do crânio.  }{tem.po.ral}{0}
\verb{temporal}{}{}{"-ais}{}{}{Chuva forte com vento; tempestade. }{tem.po.ral}{0}
\verb{temporão}{}{}{"-ões}{}{adj.}{Que nasce, vem, aparece ou acontece fora  do tempo próprio; extemporâneo.}{tem.po.rão}{0}
\verb{temporão}{}{}{"-ões}{}{s.m.}{Filho que nasce muito depois do irmão antecessor ou quando os pais já estão casados há muito tempo.}{tem.po.rão}{0}
\verb{temporário}{}{}{}{}{adj.}{Que não é definitivo.}{tem.po.rá.rio}{0}
\verb{temporizar}{}{}{}{}{v.t.}{Levar tempo para se fazer alguma coisa; demorar, retardar.}{tem.po.ri.zar}{0}
\verb{temporizar}{}{}{}{}{}{Concordar com o que lhe pedem; condescender, contemporizar.}{tem.po.ri.zar}{\verboinum{1}}
\verb{tenacidade}{}{}{}{}{s.f.}{Capacidade de manter"-se firme numa ideia; obstinação.}{te.na.ci.da.de}{0}
\verb{tênar}{}{Anat.}{}{}{s.m.}{Saliência carnosa que, na palma da mão, corresponde à base do polegar.}{tê.nar}{0}
\verb{tenaz}{}{}{}{}{adj.2g.}{Que nunca desiste; incansável, perseverante, persistente.}{te.naz}{0}
\verb{tenaz}{}{}{}{}{}{Que se prende fortemente; pegajoso, viscoso.}{te.naz}{0}
\verb{tenaz}{}{}{}{}{s.f.}{Instrumento próprio para pegar coisas muito quentes, formado por dois cabos que se prolongam adiante de um eixo.}{te.naz}{0}
\verb{tença}{}{}{}{}{s.f.}{Pensão com que se remuneram serviços.}{ten.ça}{0}
\verb{tenção}{}{}{"-ões}{}{s.f.}{O que se pretende fazer; intenção, propósito.}{ten.ção}{0}
\verb{tenção}{}{}{"-ões}{}{}{Objeto de especial adoração; devoção.}{ten.ção}{0}
\verb{tencionar}{}{}{}{}{v.t.}{Ter o plano de fazer alguma coisa; intencionar, planejar, pretender.}{ten.ci.o.nar}{\verboinum{1}}
\verb{tenda}{}{}{}{}{s.f.}{Barraca de acampamento.}{ten.da}{0}
\verb{tenda}{}{}{}{}{}{Oficina de artesão.}{ten.da}{0}
\verb{tendal}{}{}{"-ais}{}{s.m.}{Lugar em que se tosquiam ovelhas.}{ten.dal}{0}
\verb{tendal}{}{}{"-ais}{}{}{Varal no qual se estende o charque ou o peixe, para secar.}{ten.dal}{0}
\verb{tendal}{}{}{"-ais}{}{}{Armação na qual se expõe a roupa lavada para enxugar; varal.}{ten.dal}{0}
\verb{tendão}{}{Anat.}{"-ões}{}{s.m.}{Tecido fibroso que une o músculo ao osso.}{ten.dão}{0}
\verb{tendência}{}{}{}{}{s.f.}{Ação que determina o movimento de um corpo.}{ten.dên.cia}{0}
\verb{tendência}{}{}{}{}{}{Inclinação, vocação.}{ten.dên.cia}{0}
\verb{tendencioso}{ô}{}{"-osos ⟨ó⟩}{"-osa ⟨ó⟩}{adj.}{Em que há alguma intenção oculta.}{ten.den.ci.o.so}{0}
\verb{tendente}{}{}{}{}{adj.2g.}{Que se inclina ou caminha para determinado alvo ou fim.}{ten.den.te}{0}
\verb{tendente}{}{}{}{}{}{Que tem vocação; propenso.}{ten.den.te}{0}
\verb{tender}{ê}{}{}{}{v.t.}{Ter alguma tendência; inclinar"-se.}{ten.der}{0}
\verb{tender}{ê}{}{}{}{}{Dirigir"-se para determinado resultado.}{ten.der}{\verboinum{12}}
\verb{tênder}{}{}{}{}{adj.2g.}{Diz"-se de alimento defumado industrialmente.}{tên.der}{0}
\verb{tendinha}{}{}{}{}{s.f.}{Pequena mercearia de favela; birosca.}{ten.di.nha}{0}
\verb{tendinite}{}{Med.}{}{}{s.f.}{Inflamação do tendão.}{ten.di.ni.te}{0}
\verb{tenebroso}{ô}{}{"-osos ⟨ó⟩}{"-osa ⟨ó⟩}{adj.}{Que causa medo; horrível, medonho.}{te.ne.bro.so}{0}
\verb{tenebroso}{ô}{}{"-osos ⟨ó⟩}{"-osa ⟨ó⟩}{}{Sombrio, escuro.}{te.ne.bro.so}{0}
\verb{tenência}{}{}{}{}{s.f.}{Cargo de tenente.}{te.nên.cia}{0}
\verb{tenência}{}{}{}{}{}{Cautela, prudência.}{te.nên.cia}{0}
\verb{tenente}{}{}{}{}{s.m.}{Forma abreviada de \textit{primeiro"-tenente} ou \textit{segundo"-tenente}. }{te.nen.te}{0}
\verb{tenente"-brigadeiro}{ê}{}{tenentes"-brigadeiros}{}{s.m.}{Posto da hierarquia da Aeronáutica imediatamente superior ao de major"-brigadeiro e imediatamente inferior ao de marechal"-do"-ar.   }{te.nen.te"-bri.ga.dei.ro}{0}
\verb{tenente"-brigadeiro}{ê}{}{tenentes"-brigadeiros}{}{}{Militar que ocupa esse posto.}{te.nen.te"-bri.ga.dei.ro}{0}
\verb{tenente"-coronel}{é}{}{tenentes"-coronéis}{}{s.m.}{Posto da hierarquia do Exército imediatamente superior ao de major e imediatamente inferior ao de coronel.   }{te.nen.te"-co.ro.nel}{0}
\verb{tenente"-coronel}{é}{}{tenentes"-coronéis}{}{}{Militar que ocupa esse posto.}{te.nen.te"-co.ro.nel}{0}
\verb{tenente"-coronel}{é}{}{tenentes"-coronéis}{}{}{Forma abreviada de \textit{tenente"-coronel"-aviador}.}{te.nen.te"-co.ro.nel}{0}
\verb{tenente"-coronel"-aviador}{é\ldots{}ô}{}{tenentes"-coronéis"-aviadores ⟨ô⟩}{}{s.m.}{Posto na hierarquia da Aeronáutica imediatamente superior ao de major"-aviador e imediatamente inferior ao de coronel"-aviador; tenente"-coronel.}{te.nen.te"-co.ro.nel"-a.vi.a.dor}{0}
\verb{tenente"-coronel"-aviador}{é\ldots{}ô}{}{tenentes"-coronéis"-aviadores ⟨ô⟩}{}{}{Militar que ocupa esse posto.}{te.nen.te"-co.ro.nel"-a.vi.a.dor}{0}
\verb{tenesmo}{}{}{}{}{s.m.}{Contração involuntária e dolorosa dos músculos do intestino.}{te.nes.mo}{0}
\verb{tênia}{}{}{}{}{s.f.}{Verme muito comprido e achatado que se aloja no intestino do homem e de alguns animais; solitária.}{tê.nia}{0}
\verb{teníase}{}{Med.}{}{}{s.f.}{Infecção provocada pela tênia.}{te.ní.a.se}{0}
\verb{tenífugo}{}{}{}{}{adj.}{Diz"-se de agente ou medicamento que expulsa do organismo vermes do gênero \textit{Taenia}.}{te.ní.fu.go}{0}
\verb{tênis}{}{}{}{}{s.m.}{Sapato de material leve e sola flexível, em geral para uso esportivo.}{tê.nis}{0}
\verb{tênis}{}{Esport.}{}{}{}{Esporte de quadra que se joga com raquete e bola.}{tê.nis}{0}
\verb{tênis"-de"-mesa}{}{Esport.}{}{}{s.m.}{Tênis praticado sobre mesa própria com bola de celuloide.}{tê.nis"-de"-me.sa}{0}
\verb{tenista}{}{Esport.}{}{}{s.2g.}{Indivíduo que joga tênis.}{te.nis.ta}{0}
\verb{tenor}{ô}{}{}{}{s.m.}{Tipo de voz aguda masculina.}{te.nor}{0}
\verb{tenor}{ô}{}{}{}{}{Cantor com essa voz.}{te.nor}{0}
\verb{tenor}{ô}{}{}{}{}{Diz"-se de instrumento de registro comparável ao da voz dos tenores.}{te.nor}{0}
\verb{tenorino}{}{}{}{}{s.m.}{Tenor que canta em falsete.}{te.no.ri.no}{0}
\verb{tenro}{}{}{}{}{adj.}{Que é mole, macio.}{ten.ro}{0}
\verb{tensão}{}{}{"-ões}{}{s.f.}{Estado do que está esticado.}{ten.são}{0}
\verb{tensão}{}{}{"-ões}{}{}{Condição do que ameaça romper"-se, desfazer"-se.}{ten.são}{0}
\verb{tensão}{}{}{"-ões}{}{}{Estado de sobrecarga física ou mental.}{ten.são}{0}
\verb{tensão}{}{}{"-ões}{}{}{Pressão sanguínea.}{ten.são}{0}
\verb{tensão}{}{Fís.}{"-ões}{}{}{Diferença de potencial entre dois pontos de um circuito elétrico.}{ten.são}{0}
\verb{tensivo}{}{}{}{}{adj.}{Que provoca tensão.}{ten.si.vo}{0}
\verb{tenso}{}{}{}{}{adj.}{Esticado com força; esticado, estirado.}{ten.so}{0}
\verb{tenso}{}{}{}{}{}{Difícil, preocupante.}{ten.so}{0}
\verb{tensor}{ô}{}{}{}{adj.}{Que estende.}{ten.sor}{0}
\verb{tensor}{ô}{Anat.}{}{}{}{Diz"-se de músculo que estende qualquer órgão ou membro.}{ten.sor}{0}
\verb{tentação}{}{}{"-ões}{}{s.f.}{Disposição para prática de atos censuráveis.}{ten.ta.ção}{0}
\verb{tentação}{}{}{"-ões}{}{}{Desejo intenso.}{ten.ta.ção}{0}
\verb{tentação}{}{Pop.}{"-ões}{}{}{O diabo.}{ten.ta.ção}{0}
\verb{tentacular}{}{}{}{}{adj.2g.}{Relativo a tentáculo.}{ten.ta.cu.lar}{0}
\verb{tentacular}{}{}{}{}{}{Que tem tentáculos.}{ten.ta.cu.lar}{0}
\verb{tentáculo}{}{}{}{}{s.m.}{Apêndice móvel, em certos animais, que serve para o tato ou como garra.}{ten.tá.cu.lo}{0}
\verb{tentador}{ô}{}{}{}{adj.}{Que tenta.}{ten.ta.dor}{0}
\verb{tentador}{ô}{}{}{}{}{Que inspira desejo ou apetite; sedutor.}{ten.ta.dor}{0}
\verb{tentador}{ô}{Fig.}{}{}{s.m.}{O diabo.}{ten.ta.dor}{0}
\verb{tentame}{}{}{}{}{s.m.}{Ato de tentar; tentativa, ensaio.}{ten.ta.me}{0}
\verb{tentâmen}{}{}{}{}{}{Var. de \textit{tentame}.}{ten.tâ.men}{0}
\verb{tentar}{}{}{}{}{v.t.}{Esforçar"-se para.}{ten.tar}{0}
\verb{tentar}{}{}{}{}{}{Provocar desejo em.}{ten.tar}{0}
\verb{tentar}{}{}{}{}{}{Buscar, procurar.}{ten.tar}{0}
\verb{tentar}{}{}{}{}{}{Arriscar, empreender.}{ten.tar}{\verboinum{1}}
\verb{tentativa}{}{}{}{}{s.f.}{Ação que tem por fim pôr em execução um projeto ou uma ideia.}{ten.ta.ti.va}{0}
\verb{tentativa}{}{}{}{}{}{Teste experimental; ensaio, prova.}{ten.ta.ti.va}{0}
\verb{tentear}{}{}{}{}{v.t.}{Examinar, investigar, observar. }{ten.te.ar}{0}
\verb{tentear}{}{}{}{}{}{Remediar, paliar.}{ten.te.ar}{0}
\verb{tentear}{}{}{}{}{}{Pôr em prática, experimentar.}{ten.te.ar}{0}
\verb{tentear}{}{}{}{}{}{Apalpar, tatear.}{ten.te.ar}{\verboinum{4}}
\verb{tento}{}{}{}{}{s.m.}{Juízo, precaução, cuidado.}{ten.to}{0}
\verb{tento}{}{}{}{}{s.m.}{Qualquer peça com que se marcam os pontos no jogo.}{ten.to}{0}
\verb{tento}{}{Bot.}{}{}{s.m.}{Planta trepadeira de sementes vermelhas com uma parte preta.}{ten.to}{0}
\verb{tento}{}{}{}{}{}{Ponto marcado no jogo.}{ten.to}{0}
\verb{tento}{}{Por ext.}{}{}{}{No futebol, ponto que se faz quando a bola passa pelo gol; gol.}{ten.to}{0}
\verb{tênue}{}{}{}{}{adj.2g.}{Que tem pouca consistência ou espessura; fino, frágil.}{tê.nu.e}{0}
\verb{tênue}{}{}{}{}{}{Que é de pouca importância ou de pouco valor.}{tê.nu.e}{0}
\verb{tênue}{}{}{}{}{}{Muito pequeno. }{tê.nu.e}{0}
\verb{tênue}{}{}{}{}{}{Sutil.}{tê.nu.e}{0}
\verb{teocracia}{}{}{}{}{s.f.}{Forma de governo em que a autoridade, emanada dos deuses ou de Deus, é exercida pela casta sacerdotal ou por um soberano considerado representante de Deus na terra.}{te.o.cra.ci.a}{0}
\verb{teocrata}{}{}{}{}{s.2g.}{Pessoa que exerce o poder na teocracia, ou que é partidário dela.}{te.o.cra.ta}{0}
\verb{teocrático}{}{}{}{}{adj.}{Relativo a teocracia.}{te.o.crá.ti.co}{0}
\verb{teodolito}{}{}{}{}{s.m.}{Instrumento de precisão que mede ângulos horizontais e ângulos verticais, muito usado em trabalhos geodésicos e topográficos.  }{te.o.do.li.to}{0}
\verb{teofania}{}{Relig.}{}{}{s.f.}{Manifestação de Deus.}{te.o.fa.ni.a}{0}
\verb{teogonia}{}{}{}{}{s.f.}{Nas religiões politeístas, conjunto de narrativas que explicam o nascimento das divindades e apresentam a sua genealogia.}{te.o.go.ni.a}{0}
\verb{teologal}{}{}{"-ais}{}{adj.2g.}{Relativo a teologia.}{te.o.lo.gal}{0}
\verb{teologia}{}{}{}{}{s.f.}{Estudo de Deus, das religiões e das coisas divinas.}{te.o.lo.gi.a}{0}
\verb{teológico}{}{}{}{}{adj.}{Relativo a teologia.  }{te.o.ló.gi.co}{0}
\verb{teólogo}{}{}{}{}{s.m.}{Pessoa que é especialista em teologia, ou que estuda essa doutrina.}{te.ó.lo.go}{0}
\verb{teomania}{}{Med.}{}{}{s.f.}{Tendência mórbida para a religiosidade.}{te.o.ma.ni.a}{0}
\verb{teomania}{}{}{}{}{}{Loucura em que o paciente acredita ser Deus ou estar por Ele inspirado.}{te.o.ma.ni.a}{0}
\verb{teor}{ô}{}{}{}{s.m.}{Texto de uma mensagem ou escrito; conteúdo.}{te.or}{0}
\verb{teor}{ô}{}{}{}{}{Proporção, em um todo, de determinada substância.}{te.or}{0}
\verb{teorema}{}{}{}{}{s.m.}{Proposição que precisa ser demonstrada para ser admitida ou se tornar evidente.}{te.o.re.ma}{0}
\verb{teorético}{}{}{}{}{adj.}{Relativo a teoria; teórico.}{te.o.ré.ti.co}{0}
\verb{teoria}{}{}{}{}{s.f.}{Conjunto de princípios mais ou menos organizados que se aplicam a uma área específica.}{te.o.ri.a}{0}
\verb{teoria}{}{}{}{}{}{Conhecimento baseado na razão, sem passar à ação.}{te.o.ri.a}{0}
\verb{teoria}{}{}{}{}{}{Opinião, hipótese, especulação.}{te.o.ri.a}{0}
\verb{teórico}{}{}{}{}{adj.}{Relativo a teoria; teorético.  }{te.ó.ri.co}{0}
\verb{teórico}{}{}{}{}{s.m.}{Pessoa dada à formulação de teorias ou ao pensamento especulativo.}{te.ó.ri.co}{0}
\verb{teorizar}{}{}{}{}{v.t.}{Criar, estabelecer ou fundar teoria sobre.}{te.o.ri.zar}{0}
\verb{teorizar}{}{}{}{}{}{Tratar de um assunto teoricamente, sem passar à prática.}{te.o.ri.zar}{\verboinum{1}}
\verb{teosofia}{}{Filos.}{}{}{s.f.}{Doutrina mística que prega a elevação do homem à divindade.}{te.o.so.fi.a}{0}
\verb{tepidez}{ê}{}{}{}{s.f.}{Qualidade ou estado de tépido.}{te.pi.dez}{0}
\verb{tepidez}{ê}{Fig.}{}{}{}{Que tem pouca força ou intensidade; fraqueza, frouxidão, tibieza, debilidade.}{te.pi.dez}{0}
\verb{tépido}{}{}{}{}{adj.}{Não muito quente; morno, temperado, tíbio.}{té.pi.do}{0}
\verb{tequila}{}{}{}{}{s.f.}{Tipo de aguardente mexicana produzida pela destilação do agave.}{te.qui.la}{0}
\verb{ter}{ê}{}{}{}{v.t.}{Estar na posse de algo; possuir, usufruir.}{ter}{0}
\verb{ter}{ê}{}{}{}{}{Sofrer, padecer, sentir. (\textit{Ele quase teve um colapso quando soube do resultado das eleições.})}{ter}{0}
\verb{ter}{ê}{}{}{}{}{Estar com certa idade; contar anos.}{ter}{0}
\verb{ter}{ê}{}{}{}{}{Ser formado de; constar de.}{ter}{0}
\verb{ter}{ê}{}{}{}{}{Dar à luz; parir.}{ter}{0}
\verb{ter}{ê}{}{}{}{}{Necessitar, precisar.}{ter}{\verboinum{39}}
\verb{terabyte}{}{Informát.}{}{}{s.m.}{Unidade de medida de informação, equivalente a 1.024 \textit{gigabytes}. Símb.: \textsc{tb}.}{\textit{terabyte}}{0}
\verb{terapeuta}{}{Med.}{}{}{s.2g.}{Indivíduo que dá tratamento ou cuidado médico a alguém.}{te.ra.peu.ta}{0}
\verb{terapêutica}{}{}{}{}{s.f.}{Parte da medicina que estuda os meios de curar ou tratar as doenças.}{te.ra.pêu.ti.ca}{0}
\verb{terapêutica}{}{}{}{}{}{Tratamento de doenças; terapia.}{te.ra.pêu.ti.ca}{0}
\verb{terapêutico}{}{}{}{}{adj.}{Relativo a terapêutica.}{te.ra.pêu.ti.co}{0}
\verb{terapêutico}{}{}{}{}{}{Que tem propriedades medicinais; curativo, medicinal.}{te.ra.pêu.ti.co}{0}
\verb{terapia}{}{}{}{}{s.f.}{Terapêutica.}{te.ra.pi.a}{0}
\verb{teratologia}{}{Med.}{}{}{s.f.}{Ramo da medicina que estuda as anomalias e as malformações no desenvolvimento dos fetos.}{te.ra.to.lo.gi.a}{0}
\verb{térbio}{}{Quím.}{}{}{s.m.}{Elemento químico metálico, prateado, mole, dúctil, da família dos lantanídeos (terras"-raras); usado em \textit{lasers} de estado sólido, aparelhos de televisão, etc. \elemento{65}{158.92534}{Tb}.}{tér.bio}{0}
\verb{terça}{ê}{}{}{}{s.f.}{Cada uma das três partes iguais em que pode ser dividido um todo.}{ter.ça}{0}
\verb{terça}{ê}{}{}{}{}{Forma reduzida de \textit{terça"-feira}.}{ter.ça}{0}
\verb{terçã}{}{Med.}{}{}{s.f.}{Febre terçã, decorrente da malária, que se manifesta de três em três dias aproximadamente.}{ter.çã}{0}
\verb{terçado}{}{}{}{}{adj.}{Que se terçou.}{ter.ça.do}{0}
\verb{terçado}{}{}{}{}{}{Espada de folha larga e curta; sabre.}{ter.ça.do}{0}
\verb{terça"-feira}{ê}{}{terças"-feiras ⟨ê⟩}{}{s.f.}{O terceiro dia da semana.}{ter.ça"-fei.ra}{0}
\verb{terçar}{}{}{}{}{v.t.}{Dividir em três partes.}{ter.çar}{0}
\verb{terçar}{}{}{}{}{}{Misturar três coisas.}{ter.çar}{0}
\verb{terçar}{}{}{}{}{}{Combater, lutar.}{ter.çar}{\verboinum{3}}
\verb{terceiranista}{}{}{}{}{s.2g.}{Estudante do terceiro ano de um curso ou de uma faculdade.  }{ter.cei.ra.nis.ta}{0}
\verb{terceirização}{}{}{"-ões}{}{s.f.}{Ato ou efeito de terceirizar.}{ter.cei.ri.za.ção}{0}
\verb{terceirizar}{}{}{}{}{v.t.}{Contratar terceiros, por parte de uma empresa pública ou privada, para a execução de atividades geralmente não essenciais.}{ter.cei.ri.zar}{\verboinum{1}}
\verb{terceiro}{ê}{}{}{}{num.}{Numa sequência, o que ocupa a posição de número três.}{ter.cei.ro}{0}
\verb{terceiro}{ê}{}{}{}{}{Medianeiro, intercessor.}{ter.cei.ro}{0}
\verb{terceiro}{ê}{Jur.}{}{}{}{Pessoa que, sem ser autor nem réu, participa da demanda legitimamente.}{ter.cei.ro}{0}
\verb{terceiro"-sargento}{}{}{terceiros"-sargentos.}{}{s.m.}{Patente do Exército, Marinha e Aeronáutica  hierarquicamente inferior ao segundo"-sargento e superior ao cabo.}{ter.cei.ro"-sar.gen.to}{0}
\verb{terceiro"-sargento}{}{}{terceiros"-sargentos.}{}{}{Militar que ocupa esse posto.}{ter.cei.ro"-sar.gen.to}{0}
\verb{terceto}{ê}{Gram.}{}{}{s.m.}{Estrofe de três versos.}{ter.ce.to}{0}
\verb{terciário}{}{}{}{}{adj.}{Que está ou vem em terceiro lugar.}{ter.ci.á.rio}{0}
\verb{terço}{ê}{}{}{}{num.}{Cada uma das partes de um todo dividido em três partes iguais.}{ter.ço}{0}
\verb{terço}{ê}{Relig.}{}{}{}{A terça parte do rosário, composta de cinco mistérios.}{ter.ço}{0}
\verb{terçol}{ó}{Med.}{}{}{s.m.}{Pequeno abscesso na borda da pálpebra; hordéolo.  }{ter.çol}{0}
\verb{terebintina}{}{Quím.}{}{}{s.f.}{Nome comum às resinas extraídas do terebinto e de certas coníferas, muito usadas na fabricação de vernizes e tintas.}{te.re.bin.ti.na}{0}
\verb{tererê}{}{}{}{}{s.m.}{Refresco de mate que se toma com bombilha, preparada com água fria. }{te.re.rê}{0}
\verb{teres}{ê}{}{}{}{s.m.pl.}{Tudo aquilo que alguém possui; bens, posses, haveres.}{te.res}{0}
\verb{teresinense}{}{}{}{}{adj.2g.}{Relativo a Teresina, capital do Piauí.}{te.re.si.nen.se}{0}
\verb{teresinense}{}{}{}{}{s.2g.}{Indivíduo natural ou habitante dessa cidade.}{te.re.si.nen.se}{0}
\verb{tergal}{}{}{}{}{s.m.}{Espécie de tecido muito usado em roupas.}{ter.gal}{0}
\verb{tergiversação}{}{}{"-ões}{}{s.f.}{Ato ou efeito de tergiversar; rodeio, evasiva, subterfúgio.}{ter.gi.ver.sa.ção}{0}
\verb{tergiversar}{}{}{}{}{v.i.}{Usar de evasivas, rodeios ou subterfúgios. }{ter.gi.ver.sar}{\verboinum{1}}
\verb{termal}{}{}{"-ais}{}{adj.2g.}{Relativo a termas ou a calor.}{ter.mal}{0}
\verb{termas}{é}{}{}{}{s.f.pl.}{Estabelecimento em que se faz uso terapêutico de águas termais. }{ter.mas}{0}
\verb{termeletricidade}{}{Fís.}{}{}{s.f.}{Eletricidade obtida pela ação do calor. }{ter.me.le.tri.ci.da.de}{0}
\verb{termelétrico}{}{Fís.}{}{}{adj.}{Que diz respeito à eletricidade  e ao calor.}{ter.me.lé.tri.co}{0}
\verb{térmico}{}{}{}{}{adj.}{Relativo a termas ou a calor; termal.}{tér.mi.co}{0}
\verb{térmico}{}{}{}{}{}{Que conserva o calor. (\textit{Minha mãe coou o café e o colocou na garrafa térmica.})}{tér.mi.co}{0}
\verb{terminação}{}{}{"-ões}{}{s.f.}{Ato ou efeito de terminar; conclusão.}{ter.mi.na.ção}{0}
\verb{terminação}{}{}{"-ões}{}{}{Parte final de algo; extremidade.}{ter.mi.na.ção}{0}
\verb{terminação}{}{Gram.}{"-ões}{}{}{Parte final de uma palavra; desinência.}{ter.mi.na.ção}{0}
\verb{terminal}{}{}{"-ais}{}{adj.2g.}{Que se refere à ou está situado na extremidade; final.}{ter.mi.nal}{0}
\verb{terminal}{}{}{"-ais}{}{s.m.}{Estação final de uma linha férrea ou rodoviária.}{ter.mi.nal}{0}
\verb{terminal}{}{}{"-ais}{}{}{Dispositivo de um sistema que recebe ou transmite informações.}{ter.mi.nal}{0}
\verb{terminante}{}{}{}{}{adj.2g.}{Que termina; conclusivo.}{ter.mi.nan.te}{0}
\verb{terminante}{}{}{}{}{}{Categórico, decisivo.}{ter.mi.nan.te}{0}
\verb{terminar}{}{}{}{}{v.t.}{Pôr termo; findar, concluir, finalizar.}{ter.mi.nar}{0}
\verb{terminar}{}{}{}{}{v.i.}{Chegar ao fim; acabar.}{ter.mi.nar}{\verboinum{1}}
\verb{terminativo}{}{}{}{}{adj.}{Que faz terminar; categórico, irrevogável.}{ter.mi.na.ti.vo}{0}
\verb{término}{}{}{}{}{s.m.}{Conclusão de algo; limite, fim.}{tér.mi.no}{0}
\verb{terminologia}{}{}{}{}{s.f.}{Conjunto dos termos de uma ciência ou arte; nomenclatura.}{ter.mi.no.lo.gi.a}{0}
\verb{térmita}{}{Zool.}{}{}{s.f.}{Pequeno inseto que vive em comunidades populosas e se alimenta de madeira; cupim.}{tér.mi.ta}{0}
\verb{térmite}{}{}{}{}{}{Var. de \textit{térmita}.}{tér.mi.te}{0}
\verb{termo}{ê}{}{}{}{s.m.}{Conclusão no espaço e no tempo; fim, término.}{ter.mo}{0}
\verb{termo}{ê}{Gram.}{}{}{}{Cada um dos elementos de uma oração.}{ter.mo}{0}
\verb{termodinâmica}{}{Fís.}{}{}{s.f.}{Estudo das leis e relações existentes entre os fenômenos caloríficos e os mecânicos.}{ter.mo.di.nâ.mi.ca}{0}
\verb{termodinâmico}{}{}{}{}{adj.}{Relativo à termodinâmica.}{ter.mo.di.nâ.mi.co}{0}
\verb{termoeletricidade}{}{}{}{}{}{Var. de \textit{termeletricidade}.}{ter.mo.e.le.tri.ci.da.de}{0}
\verb{termoelétrico}{}{}{}{}{}{Var. de \textit{termelétrico}.}{ter.mo.e.lé.tri.co}{0}
\verb{termogênese}{}{}{}{}{s.f.}{Produção de calor nos seres vivos.}{ter.mo.gê.ne.se}{0}
\verb{termométrico}{}{}{}{}{adj.}{Relativo a termômetro.}{ter.mo.mé.tri.co}{0}
\verb{termômetro}{}{}{}{}{s.m.}{Instrumento com que se mede a temperatura dos corpos, baseado na dilatação de líquidos ou gases, por ação do calor.}{ter.mô.me.tro}{0}
\verb{termonuclear}{}{Fís.}{}{}{adj.2g.}{Relativo à energia térmica obtida por meio de processos nucleares. }{ter.mo.nu.cle.ar}{0}
\verb{termostato}{}{}{}{}{s.m.}{Dispositivo automático que mantém a temperatura constante em um ambiente.}{ter.mos.ta.to}{0}
\verb{ternário}{}{}{}{}{adj.}{Que contém três partes.}{ter.ná.rio}{0}
\verb{terneiro}{ê}{}{}{}{s.m.}{Cria da vaca, que ainda mama, até atingir um ano de idade; bezerro.}{ter.nei.ro}{0}
\verb{terno}{é}{}{}{}{adj.}{Que revela compaixão, carinho; meigo, delicado.}{ter.no}{0}
\verb{terno}{é}{}{}{}{s.m.}{Conjunto de três; trio.}{ter.no}{0}
\verb{terno}{é}{}{}{}{}{Conjunto de calças, paletó e colete.}{ter.no}{0}
\verb{ternura}{}{}{}{}{s.f.}{Qualidade do que é terno; meiguice, carinho.}{ter.nu.ra}{0}
\verb{terra}{é}{}{}{}{s.f.}{O terceiro planeta do sistema solar. (Nesta acepção, com inicial maiúscula.)}{ter.ra}{0}
\verb{terra}{é}{}{}{}{}{Solo sobre o qual se anda; chão.}{ter.ra}{0}
\verb{terra}{é}{}{}{}{}{Local onde se nasce; país, pátria.}{ter.ra}{0}
\verb{terra}{é}{}{}{}{}{Propriedade rural; campo.}{ter.ra}{0}
\verb{terra"-a"-terra}{é\ldots{}é}{}{}{}{adj.}{Sem grandeza de ideias; prosaico, trivial.}{ter.ra"-a"-ter.ra}{0}
\verb{terraço}{}{}{}{}{s.m.}{Pavimento descoberto, geralmente sobre o último andar de uma construção.}{ter.ra.ço}{0}
\verb{terracota}{ó}{}{}{}{s.f.}{Argila cozida em forno, própria para modelar.}{ter.ra.co.ta}{0}
\verb{terral}{}{}{"-ais}{}{adj.2g.}{Relativo a terra.}{ter.ral}{0}
\verb{terral}{}{}{"-ais}{}{}{Diz"-se do vento que sopra da terra para o mar.}{ter.ral}{0}
\verb{terra"-nova}{é\ldots{}ó}{Zool.}{terras"-novas \textit{ou} terra"-novas ⟨é\ldots{}ó⟩}{}{s.m.}{Cão de pelos sedosos e longos e pés espalmados, originário da ilha de Terra Nova no Canadá.}{ter.ra"-no.va}{0}
\verb{terraplanar}{}{}{}{}{}{Var. de \textit{terraplenar}.}{ter.ra.pla.nar}{0}
\verb{terraplenagem}{}{}{}{}{s.f.}{Ato ou efeito de terraplenar; processo de remover ou pôr terra em um terreno para que fique plano.}{ter.ra.ple.na.gem}{0}
\verb{terraplenar}{}{}{}{}{v.t.}{Encher de terra e pedras o terreno que possui desníveis, preenchendo as depressões.}{ter.ra.ple.nar}{\verboinum{1}}
\verb{terrapleno}{}{}{}{}{s.m.}{Terreno que sofreu terraplenagem, que foi aterrado.}{ter.ra.ple.no}{0}
\verb{terráqueo}{}{}{}{}{adj.}{Que é originário do planeta Terra; terrestre.}{ter.rá.queo}{0}
\verb{terras"-raras}{é}{Quím.}{}{}{s.f.pl.}{Denominação dada aos elementos químicos que pertencem à série dos lantanídeos, como o érbio, o itérbio, o lantânio, o túlio etc. }{ter.ras"-ra.ras}{0}
\verb{terreal}{}{}{"-ais}{}{adj.2g.}{Relativo à Terra; terrestre.}{ter.real}{0}
\verb{terreiro}{ê}{}{}{}{s.m.}{Terreno plano e largo, próximo das casas.}{ter.rei.ro}{0}
\verb{terreiro}{ê}{}{}{}{}{Local onde se realizam cultos afro"-brasileiros.}{ter.rei.ro}{0}
\verb{terremoto}{ó}{}{}{}{s.m.}{Movimento de trepidação na superfície terrestre, originário do interior da crosta; abalo sísmico; sismo.}{ter.re.mo.to}{0}
\verb{terreno}{}{}{}{}{adj.}{Relativo às coisas da Terra; mundano, temporal.}{ter.re.no}{0}
\verb{terreno}{}{}{}{}{}{Espaço de terra próprio para cultivo ou para construções; gleba, lote.}{ter.re.no}{0}
\verb{térreo}{}{}{}{}{adj.}{Que fica ao rés"-do"-chão.}{tér.re.o}{0}
\verb{térreo}{}{}{}{}{s.m.}{O pavimento que fica sobre a superfície do chão.}{tér.re.o}{0}
\verb{terrestre}{é}{}{}{}{adj.2g.}{Relativo à ou próprio da Terra; terráqueo.}{ter.res.tre}{0}
\verb{terrestre}{é}{}{}{}{}{Relativo a terra. (\textit{O pacote da agência da viagens só cobria a locomoção terrestre, não a aérea.})}{ter.res.tre}{0}
\verb{terrícola}{}{}{}{}{adj.}{Diz"-se do ser humano, do animal ou do vegetal habitantes da Terra.}{ter.rí.co.la}{0}
\verb{terrificante}{}{}{}{}{adj.2g.}{Que terrifica; apavorante, assustador.}{ter.ri.fi.can.te}{0}
\verb{terrificar}{}{}{}{}{v.t.}{Causar terror; apavorar, assustar.}{ter.ri.fi.car}{\verboinum{2}}
\verb{terrífico}{}{}{}{}{adj.}{Terrificante.}{ter.rí.fi.co}{0}
\verb{terrina}{}{}{}{}{s.f.}{Recipiente em que se serve sopa ou caldos à mesa.}{ter.ri.na}{0}
\verb{territorial}{}{}{"-ais}{}{adj.2g.}{Relativo ou pertencente a território.}{ter.ri.to.ri.al}{0}
\verb{território}{}{}{}{}{s.m.}{Área de um país, estado, cidade etc.}{ter.ri.tó.rio}{0}
\verb{território}{}{}{}{}{}{Área em que um grupo de animais ocupa e defende contra a invasão de outro grupo ou outra espécie.}{ter.ri.tó.rio}{0}
\verb{terrível}{}{}{"-eis}{}{adj.2g.}{Que causa terror; assustador, horrível.}{ter.rí.vel}{0}
\verb{terrível}{}{}{"-eis}{}{}{Que produz resultados desastrosos.}{ter.rí.vel}{0}
\verb{terrível}{}{}{"-eis}{}{}{Muito forte; violento, feroz.}{ter.rí.vel}{0}
\verb{terror}{ô}{}{}{}{s.m.}{Estado de pavor; grande medo; horror.}{ter.ror}{0}
\verb{terrorismo}{}{}{}{}{s.m.}{Conjunto de atos violentos como atentados e destruições cometidos por grupos cujos objetivos são a desestabilização da sociedade e a tomada do poder político vigente.}{ter.ro.ris.mo}{0}
\verb{terrorista}{}{}{}{}{adj.2g.}{Diz"-se do indivíduo que pratica o terrorismo.}{ter.ro.ris.ta}{0}
\verb{terroso}{ô}{}{"-osos ⟨ó⟩}{"-osa ⟨ó⟩}{adj.}{Que tem aspecto, cor ou é da natureza da terra.}{ter.ro.so}{0}
\verb{terso}{ê}{}{}{}{adj.}{Que se apresenta puro, limpo, sem manchas.}{ter.so}{0}
\verb{terso}{ê}{}{}{}{}{Correto, vernáculo, esmerado.}{ter.so}{0}
\verb{tertúlia}{}{}{}{}{s.f.}{Reunião de amigos com fins artísticos ou literários.}{ter.tú.lia}{0}
\verb{tesão}{}{}{}{}{s.2g.}{Estado de rigidez do pênis; endurecimento.}{te.são}{0}
\verb{tesão}{}{Chul.}{}{}{}{Desejo sexual.}{te.são}{0}
\verb{tesão}{}{Pop.}{}{}{}{Coisa estimulante, excitante.}{te.são}{0}
\verb{tesar}{}{}{}{}{v.t.}{Tornar teso; retesar, endireitar.}{te.sar}{\verboinum{1}}
\verb{tese}{é}{}{}{}{s.f.}{Proposição, afirmação posta em discussão.}{te.se}{0}
\verb{tese}{é}{}{}{}{}{Monografia defendida em universidade, ao final do curso de doutoramento.}{te.se}{0}
\verb{teso}{ê}{}{}{}{adj.}{Que se estirou; esticado, tenso, estendido.}{te.so}{0}
\verb{teso}{ê}{}{}{}{}{Rígido, duro.}{te.so}{0}
\verb{tesoura}{ô}{}{}{}{s.f.}{Instrumento formado por duas lâminas móveis que se cruzam, acionadas com os dedos ou com as mãos, próprio para cortar papéis, tecidos, vegetais.}{te.sou.ra}{0}
\verb{tesourada}{}{}{}{}{s.f.}{Ato ou efeito de tesourar.}{te.sou.ra.da}{0}
\verb{tesourada}{}{}{}{}{}{Golpe dado com uma tesoura.}{te.sou.ra.da}{0}
\verb{tesourar}{}{}{}{}{v.t.}{Cortar com tesoura.}{te.sou.rar}{0}
\verb{tesourar}{}{Pop.}{}{}{}{Fazer intervenções diretas contra uma pessoa, impedindo"-a de manifestar"-se.}{te.sou.rar}{\verboinum{1}}
\verb{tesouraria}{}{}{}{}{s.f.}{Cargo de tesoureiro.}{te.sou.ra.ri.a}{0}
\verb{tesouraria}{}{}{}{}{}{Seção ou gabinete de um tesoureiro.}{te.sou.ra.ri.a}{0}
\verb{tesoureiro}{ê}{}{}{}{s.m.}{Indivíduo responsável pelo tesouro de uma instituição.}{te.sou.rei.ro}{0}
\verb{tesoureiro}{ê}{}{}{}{}{Indivíduo encarregado das operações monetárias de uma instituição.}{te.sou.rei.ro}{0}
\verb{tesouro}{ô}{}{}{}{s.m.}{Conjunto de riquezas, de bens de valor de uma pessoa, instituição ou nação.}{te.sou.ro}{0}
\verb{tesouro}{ô}{}{}{}{}{O local onde essas riquezas ficam guardadas.}{te.sou.ro}{0}
\verb{tessitura}{}{Mús.}{}{}{s.f.}{Conjunto dos sons da escala musical que melhor convêm a uma voz ou instrumento musical.}{tes.si.tu.ra}{0}
\verb{tessitura}{}{Fig.}{}{}{}{Organização, textura, contextura.}{tes.si.tu.ra}{0}
\verb{testa}{é}{Anat.}{}{}{s.f.}{Parte da face situada entre as sobrancelhas e o couro cabeludo; fronte.}{tes.ta}{0}
\verb{testada}{}{}{}{}{s.f.}{Pancada com a testa.}{tes.ta.da}{0}
\verb{testada}{}{Fig.}{}{}{}{Erro, bobagem, asneira.}{tes.ta.da}{0}
\verb{testada}{}{}{}{}{}{Parte da via pública que fica à frente de um edifício; testeira.}{tes.ta.da}{0}
\verb{testa"-de"-ferro}{é\ldots{}é}{}{testas"-de"-ferro ⟨é\ldots{}é⟩}{}{s.m.}{Indivíduo que se apresenta como responsável pelos negócios ou serviços de alguém que não quer aparecer.}{tes.ta"-de"-fer.ro}{0}
\verb{testador}{ô}{Bras.}{}{}{adj.}{Que testa ou que se submete a teste.}{tes.ta.dor}{0}
\verb{testador}{}{}{}{}{adj.}{Que faz testamento.}{tes.ta.dor}{0}
\verb{testamental}{}{}{"-ais}{}{adj.2g.}{Relativo a testamento.}{tes.ta.men.tal}{0}
\verb{testamentário}{}{}{}{}{adj.}{Relativo a testamento; testamental.}{tes.ta.men.tá.rio}{0}
\verb{testamentário}{}{}{}{}{s.m.}{Herdeiro por testamento.}{tes.ta.men.tá.rio}{0}
\verb{testamenteiro}{ê}{}{}{}{s.m.}{Indivíduo que cumpre ou faz cumprir um testamento.}{tes.ta.men.tei.ro}{0}
\verb{testamenteiro}{ê}{}{}{}{}{Indivíduo expressamente encarregado pelo testador de cumprir sua última vontade.}{tes.ta.men.tei.ro}{0}
\verb{testamento}{}{Jur.}{}{}{s.m.}{Ato pelo qual alguém dispõe de seu patrimônio, nomeia tutores e faz reconhecimentos e declarações, o qual vem a público e tem efeito após sua morte.}{tes.ta.men.to}{0}
\verb{testamento}{}{}{}{}{}{Documento que contém essas disposições.}{tes.ta.men.to}{0}
\verb{testar}{}{}{}{}{v.t.}{Submeter a teste; pôr à prova.}{tes.tar}{0}
\verb{testar}{}{}{}{}{}{Experimentar, provar, ensaiar.}{tes.tar}{\verboinum{1}}
\verb{testar}{}{}{}{}{v.t.}{Dispor em testamento; deixar, legar.}{tes.tar}{0}
\verb{testar}{}{}{}{}{}{Dar testemunho de.}{tes.tar}{0}
\verb{testar}{}{}{}{}{v.i.}{Fazer o próprio testamento.}{tes.tar}{\verboinum{1}}
\verb{teste}{é}{}{}{}{s.m.}{Ato ou efeito de testar.}{tes.te}{0}
\verb{teste}{é}{}{}{}{}{Exame, prova, avaliação, experimentação.}{tes.te}{0}
\verb{testemunha}{}{Jur.}{}{}{s.f.}{Indivíduo que, tendo estado presente a um episódio, presta testemunho do que viu e ouviu.}{tes.te.mu.nha}{0}
\verb{testemunha}{}{Jur.}{}{}{}{Indivíduo que assiste a certas cerimônias para torná"-los juridicamente válidos; padrinho.}{tes.te.mu.nha}{0}
\verb{testemunhal}{}{}{"-ais}{}{adj.2g.}{Relativo a testemunha.}{tes.te.mu.nhal}{0}
\verb{testemunhar}{}{}{}{}{v.t.}{Presenciar, ver, ouvir, assistir.}{tes.te.mu.nhar}{0}
\verb{testemunhar}{}{}{}{}{}{Declarar ter visto ou ouvido.}{tes.te.mu.nhar}{0}
\verb{testemunhar}{}{}{}{}{}{Dar testemunho de.}{tes.te.mu.nhar}{\verboinum{1}}
\verb{testemunho}{}{}{}{}{s.m.}{Depoimento prestado por alguém que declara ter testemunhado um episódio.}{tes.te.mu.nho}{0}
\verb{testicular}{}{}{}{}{adj.2g.}{Relativo a testículo.}{tes.ti.cu.lar}{0}
\verb{testículo}{}{Anat.}{}{}{s.m.}{Cada uma das duas glândulas sexuais masculinas, de forma ovoide, que se localizam na bolsa escrotal.}{tes.tí.cu.lo}{0}
\verb{testificação}{}{}{"-ões}{}{s.f.}{Ato ou efeito de testificar.}{tes.ti.fi.ca.ção}{0}
\verb{testificar}{}{}{}{}{v.t.}{Testemunhar, atestar.}{tes.ti.fi.car}{\verboinum{2}}
\verb{testo}{ê}{}{}{}{s.m.}{Tampa para vasilha, geralmente feita de barro ou metal.}{tes.to}{0}
\verb{testosterona}{}{Quím.}{}{}{s.f.}{Hormônio masculino, produzido pelos testículos.}{tes.tos.te.ro.na}{0}
\verb{testudo}{}{}{}{}{adj.}{Diz"-se de indivíduo que tem testa ou cabeça grande.}{tes.tu.do}{0}
\verb{testudo}{}{Fig.}{}{}{}{Teimoso, cabeçudo.}{tes.tu.do}{0}
\verb{tesudo}{}{Chul.}{}{}{adj.}{Diz"-se de indivíduo muito atraente, que provoca desejo sexual.}{te.su.do}{0}
\verb{teta}{é}{}{}{}{s.m.}{Oitava letra do alfabeto grego.}{te.ta}{0}
\verb{teta}{ê}{Anat.}{}{}{s.f.}{Cada uma das partes do corpo de uma fêmea que produz leite; glândula mamária; mama; úbere.}{te.ta}{0}
\verb{teta}{ê}{Anat.}{}{}{}{Cada um dos peitos da mulher; seio.}{te.ta}{0}
\verb{teta}{ê}{Fig.}{}{}{}{Fonte; manancial.}{te.ta}{0}
\verb{tetânico}{}{}{}{}{adj.}{Relativo a tétano.}{te.tâ.ni.co}{0}
\verb{tétano}{}{Med.}{}{}{s.m.}{Doença infecciosa caracterizada por espasmos musculares, causada por um determinado bacilo.}{té.ta.no}{0}
\verb{teteia}{é}{}{}{}{s.f.}{Enfeite, berloque.}{te.tei.a}{0}
\verb{teteia}{é}{Pop.}{}{}{}{Coisa ou pessoa muito graciosa ou delicada.}{te.tei.a}{0}
\verb{teto}{é}{}{}{}{s.m.}{Parte interna da cobertura de um aposento, casa ou edificação.}{te.to}{0}
\verb{teto}{é}{Por ext.}{}{}{}{Moradia, casa.}{te.to}{0}
\verb{teto}{é}{}{}{}{}{Altitude máxima em que certas manobras aeronáuticas podem ser realizadas com segurança.}{te.to}{0}
\verb{tetônica}{}{}{}{}{}{Var. de \textit{tectônica}.}{te.tô.ni.ca}{0}
\verb{tetracampeão}{}{}{"-ões}{}{adj.}{Diz"-se do indivíduo, equipe ou clube campeão quatro vezes.}{te.tra.cam.pe.ão}{0}
\verb{tetracampeonato}{}{}{}{}{s.m.}{Campeonato conquistado pela quarta vez.}{te.tra.cam.pe.o.na.to}{0}
\verb{tetraédrico}{}{}{}{}{adj.}{Relativo a ou próprio de um tetraedro.}{te.tra.é.dri.co}{0}
\verb{tetraedro}{é}{Geom.}{}{}{s.m.}{Poliedro de quatro faces.}{te.tra.e.dro}{0}
\verb{tetragonal}{}{}{"-ais}{}{adj.2g.}{Que tem forma de tetrágono.}{te.tra.go.nal}{0}
\verb{tetragonal}{}{}{"-ais}{}{}{Que tem quatro ângulos ou quatro lados.}{te.tra.go.nal}{0}
\verb{tetrágono}{}{Geom.}{}{}{s.m.}{Polígono de quatro lados.}{te.trá.go.no}{0}
\verb{tetraneto}{é}{}{}{}{s.m.}{Filho de trineto ou trineta.}{te.tra.ne.to}{0}
\verb{tetraplegia}{}{Med.}{}{}{s.f.}{Paralisia que afeta os quatro membros.}{te.tra.ple.gi.a}{0}
\verb{tetraplégico}{}{}{}{}{adj.}{Relativo a tetraplegia.}{te.tra.plé.gi.co}{0}
\verb{tetraplégico}{}{}{}{}{}{Que sofre de tetraplegia.}{te.tra.plé.gi.co}{0}
\verb{tetrápode}{}{}{}{}{adj.2g.}{Que tem quatro pés.}{te.trá.po.de}{0}
\verb{tetrápode}{}{Zool.}{}{}{s.m.}{Animal de quatro patas; quadrúpede.}{te.trá.po.de}{0}
\verb{tetrassílabo}{}{}{}{}{adj.}{Diz"-se de palavra ou verso de quatro sílabas.}{te.tras.sí.la.bo}{0}
\verb{tetravó}{}{}{}{}{s.f.}{Forma feminina de \textit{tetravô}.}{te.tra.vó}{0}
\verb{tetravô}{}{}{}{}{s.m.}{Pai de trisavô ou trisavó.}{te.tra.vô}{0}
\verb{tétrico}{}{}{}{}{adj.}{Fúnebre, muito triste.}{té.tri.co}{0}
\verb{tétrico}{}{}{}{}{}{Horrível, medonho.}{té.tri.co}{0}
\verb{teu}{}{}{}{tua}{pron.}{Possessivo que indica a segunda pessoa. (\textit{Menino, teu trabalho está excelente!})}{teu}{0}
\verb{teu}{}{Pop.}{}{tua}{}{Possessivo que indica a terceira pessoa. (\textit{Teu vestido é lindo, onde você o comprou?})}{teu}{0}
\verb{teutão}{}{Hist.}{}{}{adj.}{Relativo aos teutões, povo de provável origem germânica.}{teu.tão}{0}
\verb{teutão}{}{}{}{}{s.m.}{Indivíduo pertencente a esse povo.}{teu.tão}{0}
\verb{teutão}{}{}{}{}{}{Língua germânica falada por esse povo.}{teu.tão}{0}
\verb{teuto}{}{}{}{}{adj. e s.m.  }{Teutão.}{teu.to}{0}
\verb{teutônico}{}{}{}{}{adj.}{Relativo ao povo teutão.}{teu.tô.ni.co}{0}
\verb{teutônico}{}{}{}{}{}{Relativo à Alemanha ou ao povo alemão.}{teu.tô.ni.co}{0}
\verb{tevê}{}{}{}{}{s.f.}{Redução de \textit{televisão}.}{te.vê}{0}
\verb{têxtil}{s}{}{"-eis ⟨s⟩}{}{adj.2g.}{Relativo a tecido.}{têx.til}{0}
\verb{têxtil}{s}{}{"-eis ⟨s⟩}{}{}{Diz"-se de planta ou fibra que se pode ser usada para tecer.}{têx.til}{0}
\verb{texto}{ês}{}{}{}{s.m.}{Conjunto de palavras encadeadas, portador de uma ideia, de um raciocínio, de uma narrativa.}{tex.to}{0}
\verb{textual}{s}{}{"-ais}{}{adj.2g.}{Relativo a texto.}{tex.tu.al}{0}
\verb{textura}{s}{}{}{}{s.f.}{Ato ou efeito de tecer.}{tex.tu.ra}{0}
\verb{textura}{s}{}{}{}{}{Trama, contextura, composição.}{tex.tu.ra}{0}
\verb{texugo}{ch}{Zool.}{}{}{s.m.}{Mamífero carnívoro de cabeça alongada, corpo atarracado e cauda curta, e que cava tocas com rapidez.}{te.xu.go}{0}
\verb{tez}{ê}{}{}{}{s.f.}{Pele que recobre o rosto; cútis.}{tez}{0}
\verb{Th}{}{Quím.}{}{}{}{Símb. do \textit{tório}.}{Th}{0}
\verb{ti}{}{Gram.}{}{}{pron.}{Pronome pessoal da segunda pessoa do singular, que se usa depois de preposição, exceto quando se trata da preposiçao \textit{com}.}{ti}{0}
\verb{Ti}{}{Quím.}{}{}{}{Símb. do \textit{titânio}.}{Ti}{0}
\verb{tia}{}{}{}{}{s.f.}{Irmã do pai ou da mãe em relação aos filhos deles.}{ti.a}{0}
\verb{tia}{}{}{}{}{}{Esposa do tio em relação aos sobrinhos deste.}{ti.a}{0}
\verb{tia}{}{Pop.}{}{}{}{Solteirona.}{ti.a}{0}
\verb{tia"-avó}{}{}{tias"-avós}{}{s.f.}{Irmã do avô ou da avó, em relação aos netos destes.}{ti.a"-a.vó}{0}
\verb{tiamina}{}{Quím.}{}{}{s.f.}{Vitamina B1, encontrada em cereais, ovos e fermentos, e que é essencial ao crescimento.}{ti.a.mi.na}{0}
\verb{tiara}{}{}{}{}{s.f.}{Arco ou diadema com que as mulheres prendem a parte frontal do cabelo.}{ti.a.ra}{0}
\verb{tibetano}{}{}{}{}{adj.}{Relativo ao Tibete, na Ásia.}{ti.be.ta.no}{0}
\verb{tibetano}{}{}{}{}{s.m.}{Indivíduo natural ou habitante dessa região.}{ti.be.ta.no}{0}
\verb{tíbia}{}{Anat.}{}{}{s.f.}{O mais grosso dos dois ossos da perna, abaixo do joelho.}{tí.bia}{0}
\verb{tíbio}{}{}{}{}{adj.}{Pouco aquecido; morno.}{tí.bio}{0}
\verb{tíbio}{}{}{}{}{}{Sem entusiasmo; indolente.}{tí.bio}{0}
\verb{tição}{}{}{"-ões}{}{s.m.}{Pedaço de lenha acesa ou ainda queimando.}{ti.ção}{0}
\verb{ticar}{}{}{}{}{v.t.}{Assinalar palavras ou expressões de um texto escrito com um tique ou um furo.}{ti.car}{\verboinum{2}}
\verb{tico}{}{}{}{}{s.m.}{Pedacinho de alguma coisa; bocado, taco.}{ti.co}{0}
\verb{tico"-tico}{}{Zool.}{tico"-ticos}{}{s.m.}{Passarinho de cabeça listrada de cinza e negro, com pequeno topete e garganta branca.}{ti.co"-ti.co}{0}
\verb{tido}{}{}{}{}{adj.}{Que foi considerado; julgado.}{ti.do}{0}
\verb{tié}{}{Zool.}{}{}{s.m.}{Nome comum a vários pássaros cujo macho possui um topete vermelho na cabeça.}{ti.é}{0}
\verb{tiê}{}{}{}{}{}{Var. de \textit{tié}.}{ti.ê}{0}
\verb{tiete}{é}{}{}{}{s.2g.}{Admirador fanático de alguma personalidade famosa.}{ti.e.te}{0}
\verb{tífico}{}{}{}{}{adj.}{Relativo ao ou da natureza do tifo.}{tí.fi.co}{0}
\verb{tifo}{}{Med.}{}{}{s.m.}{Doença infecto"-contagiosa caracterizada por febre contínua e diarreia, transmitida por pulgas, piolhos e carrapatos.}{ti.fo}{0}
\verb{tifoide}{}{}{}{}{adj.2g.}{Que se assemelha ao tifo.}{ti.foi.de}{0}
\verb{tifoso}{ô}{}{"-osos ⟨ó⟩}{"-osa ⟨ó⟩}{adj.}{Diz"-se do indivíduo acometido de tifo.}{ti.fo.so}{0}
\verb{tigela}{é}{}{}{}{s.f.}{Vasilha de fundo estreito e boca larga, sem asas, própria para se tomar caldos e sopas.}{ti.ge.la}{0}
\verb{tigre}{}{Zool.}{}{tigresa ⟨ê⟩}{s.m.}{Mamífero felídeo, extremamente feroz, de pele amarelada com faixas pretas. }{ti.gre}{0}
\verb{tiguera}{é}{}{}{}{s.f.}{Roça de milho após a colheita.}{ti.gue.ra}{0}
\verb{tijoleiro}{ê}{}{}{}{s.m.}{Indivíduo que fabrica tijolos.}{ti.jo.lei.ro}{0}
\verb{tijolo}{ô}{}{}{}{s.m.}{Bloco de barro cozido usado na construção de muros e paredes.}{ti.jo.lo}{0}
\verb{tijuco}{}{}{}{}{s.m.}{Terreno repleto de lama; lodaçal, atoleiro.}{ti.ju.co}{0}
\verb{tijupá}{}{}{}{}{s.m.}{Pequena palhoça de índios, menor que a oca.}{ti.ju.pá}{0}
\verb{til}{}{Gram.}{}{}{s.m.}{Sinal [~] que se coloca sobre as vogais \textit{a} e \textit{o} para indicar a pronúncia nasal.}{til}{0}
\verb{tílburi}{}{}{}{}{s.m.}{Antigo carro de duas rodas e dois assentos, sem capota, puxado por um só animal.}{tíl.bu.ri}{0}
\verb{tília}{}{Bot.}{}{}{s.f.}{Árvore ornamental de flores aromáticas, usadas em infusão calmante.}{tí.lia}{0}
\verb{tilintar}{}{}{}{}{v.i.}{Soar como metal, campainha ou sino.}{ti.lin.tar}{\verboinum{1}}
\verb{timão}{}{}{"-ões}{}{s.m.}{Em embarcações, barra do leme.}{ti.mão}{0}
\verb{timão}{}{}{"-ões}{}{}{Peça do arado a que se atrelam os animais que o puxam.}{ti.mão}{0}
\verb{timbale}{}{}{}{}{s.m.}{Tambor metálico em forma de meio globo, usado em cavalaria.}{tim.ba.le}{0}
%\verb{}{}{}{}{}{s.m.}{Grupo da família linguística Jê.}{}{0}
%\verb{}{}{}{}{}{s.2g.}{Indivíduo pertencente a qualquer dos grupos timbira.}{}{0}
%\verb{}{}{}{}{}{adj.2g.}{Relativo aos timbira.}{}{0}
\verb{timbó}{}{Bot.}{}{}{s.m.}{Espécie de planta tóxica usada em pesca para paralisar os peixes.}{tim.bó}{0}
\verb{timbrado}{}{}{}{}{adj.}{Que foi marcado com timbre; carimbado.}{tim.bra.do}{0}
\verb{timbrar}{}{}{}{}{v.t.}{Marcar com timbre.}{tim.brar}{\verboinum{1}}
\verb{timbre}{}{}{}{}{s.m.}{Marca que se coloca sobre um impresso para indicar categoria, qualidade, origem etc.}{tim.bre}{0}
\verb{timbre}{}{}{}{}{}{Qualidade distintiva do som da fala ou de um instrumento.}{tim.bre}{0}
\verb{time}{}{}{}{}{s.m.}{Grupo de jogadores que participam de esportes em conjunto; equipe.}{ti.me}{0}
\verb{time}{}{}{}{}{}{Conjunto de pessoas que exercem a mesma atividade; turma.}{ti.me}{0}
\verb{timer}{}{}{}{}{s.m.}{Dispositivo visual ou sonoro que indica o final de intervalo de tempo.}{\textit{timer}}{0}
\verb{timidez}{ê}{}{}{}{s.f.}{Qualidade de tímido; acanhamento.}{ti.mi.dez}{0}
\verb{tímido}{}{}{}{}{adj.}{Que não se sente à vontade em frente a outras pessoas; acanhado, receoso.}{tí.mi.do}{0}
\verb{timing}{}{}{}{}{s.m.}{Senso de oportunidade quanto à escolha do momento adequado para agir ou da duração de um processo etc. }{\textit{timing}}{0}
\verb{timoneiro}{ê}{}{}{}{s.m.}{Indivíduo que governa o timão de um barco.}{ti.mo.nei.ro}{0}
\verb{timorato}{}{}{}{}{adj.}{Que tem medo de errar; receoso, medroso.}{ti.mo.ra.to}{0}
\verb{timpanal}{}{}{"-ais}{}{adj.2g.}{Relativo ou pertencente ao tímpano.}{tim.pa.nal}{0}
\verb{timpânico}{}{}{}{}{adj.}{Timpanal.}{tim.pâ.ni.co}{0}
\verb{tímpano}{}{Anat.}{}{}{s.m.}{Membrana fina e tensa localizada no fundo do ouvido e que vibra com os sons.}{tím.pa.no}{0}
\verb{tímpano}{}{Mús.}{}{}{}{Instrumento de percussão tocado com uma vara; timbale.}{tím.pa.no}{0}
\verb{tina}{}{}{}{}{s.f.}{Vasilha grande de madeira, semelhante a um barril cortado ao meio, usada para lavar roupas ou carregar água.}{ti.na}{0}
\verb{tíner}{}{}{}{}{s.m.}{Solvente usado para desmanchar tinta.}{tí.ner}{0}
\verb{tingir}{}{}{}{}{v.t.}{Dar nova cor, imergindo em tinta.}{tin.gir}{\verboinum{22}}
\verb{tingui}{}{}{}{}{s.m.}{Timbó.}{tin.gui}{0}
\verb{tinha}{}{Med.}{}{}{s.f.}{Doença contagiosa causada por um fungo e que atinge o couro cabeludo fazendo cair o cabelo.}{ti.nha}{0}
\verb{tinhorão}{}{Bot.}{"-ões}{}{s.m.}{Planta herbácea venenosa cultivada por suas belas folhas com manchas variadas.}{ti.nho.rão}{0}
\verb{tinhoso}{ô}{}{"-osos ⟨ó⟩}{"-osa ⟨ó⟩}{adj.}{Diz"-se do indivíduo que sofre de tinha.}{ti.nho.so}{0}
\verb{tinhoso}{ô}{}{"-osos ⟨ó⟩}{"-osa ⟨ó⟩}{}{Teimoso, obstinado, endiabrado.}{ti.nho.so}{0}
\verb{tinido}{}{}{}{}{s.m.}{Som vibrante e agudo de vidro, cristal ou metal.}{ti.ni.do}{0}
\verb{tinir}{}{}{}{}{v.i.}{Soar de modo agudo e vibrante como metal ou cristal.}{ti.nir}{\verboinum{34}}
\verb{tino}{}{}{}{}{s.m.}{Habilidade em avaliar seres, fenômenos, fatos; sensatez, juízo.}{ti.no}{0}
\verb{tino}{}{}{}{}{}{Sagacidade, perspicácia.}{ti.no}{0}
\verb{tinta}{}{}{}{}{s.f.}{Substância líquida própria para escrever, tingir, pintar.}{tin.ta}{0}
\verb{tinteiro}{ê}{}{}{}{s.m.}{Pequeno recipiente para guardar tinta de escrever e carregar de tinta a pena ou caneta.}{tin.tei.ro}{0}
\verb{tintim}{}{}{}{}{interj.}{Expressão usada nos brindes.}{tin.tim}{0}
\verb{tintinábulo}{}{}{}{}{s.m.}{Pequeno sino ou campainha.}{tin.ti.ná.bu.lo}{0}
\verb{tinto}{}{}{}{}{adj.}{Que se tingiu; tingido.}{tin.to}{0}
\verb{tinto}{}{}{}{}{}{Diz"-se de vinho de cor vermelho escuro.}{tin.to}{0}
\verb{tinto}{}{Fig.}{}{}{}{Sujo, manchado.}{tin.to}{0}
\verb{tintura}{}{}{}{}{s.f.}{Substância própria para tingir.}{tin.tu.ra}{0}
\verb{tintura}{}{}{}{}{}{Ato ou efeito de tingir.}{tin.tu.ra}{0}
\verb{tinturaria}{}{}{}{}{s.f.}{Estabelecimento de tintureiro, no qual se fazem tingimentos.}{tin.tu.ra.ri.a}{0}
\verb{tintureiro}{ê}{}{}{}{adj.}{Diz"-se de planta que produz substância corante.}{tin.tu.rei.ro}{0}
\verb{tintureiro}{ê}{}{}{}{s.m.}{Indivíduo que tinge tecidos.}{tin.tu.rei.ro}{0}
\verb{tio}{}{}{}{}{s.m.}{Irmão do pai ou da mãe.}{ti.o}{0}
\verb{tio}{}{}{}{}{}{Marido da tia.}{ti.o}{0}
\verb{tio"-avô}{}{}{}{}{s.m.}{Irmão do avô ou avó.}{ti.o"-a.vô}{0}
\verb{típico}{}{}{}{}{adj.}{Que representa um determinado tipo; característico.}{tí.pi.co}{0}
\verb{tipiti}{}{Bras.}{}{}{s.m.}{Recipiente de palha trançada próprio para espremer mandioca ralada.}{ti.pi.ti}{0}
\verb{tiple}{}{Mús.}{}{}{s.2g.}{A voz mais aguda de mulher ou de menino.}{ti.ple}{0}
\verb{tipo}{}{}{}{}{s.m.}{Classe de coisas ou pessoas agrupadas por algumas características em comum; espécie, categoria.}{ti.po}{0}
\verb{tipo}{}{}{}{}{}{Coisa que serve para produzir outra igual; modelo, padrão.}{ti.po}{0}
\verb{tipo}{}{}{}{}{}{Qualquer indivíduo; sujeito.}{ti.po}{0}
\verb{tipo}{}{}{}{}{}{Bloco de metal com um caractere em relevo, usado para impressão.}{ti.po}{0}
\verb{tipo}{}{Por ext.}{}{}{}{O caractere impresso; letra.}{ti.po}{0}
\verb{tipografia}{}{}{}{}{s.f.}{Técnica de impressão que utiliza tipos.}{ti.po.gra.fi.a}{0}
\verb{tipográfico}{}{}{}{}{adj.}{Relativo a tipografia.}{ti.po.grá.fi.co}{0}
\verb{tipógrafo}{}{}{}{}{s.m.}{Indivíduo que trabalha com tipografia.}{ti.pó.gra.fo}{0}
\verb{tipoia}{ó}{}{}{}{s.f.}{Rede utilizada principalmente por indígenas para carregar bebês presos ao corpo.}{ti.poi.a}{0}
\verb{tipoia}{ó}{}{}{}{}{Tira de pano presa ao pescoço para segurar braço ferido ou imobilizado.}{ti.poi.a}{0}
\verb{tipologia}{}{}{}{}{s.f.}{Estudo que visa agrupar objetos em classes de acordo com suas características.}{ti.po.lo.gi.a}{0}
\verb{tipologia}{}{}{}{}{}{Conjunto dos caracteres tipográficos de um trabalho gráfico.}{ti.po.lo.gi.a}{0}
\verb{tique}{}{}{}{}{s.m.}{Hábito incômodo, repetitivo e, geralmente, fora do controle; cacoete.}{ti.que}{0}
\verb{tique}{}{}{}{}{}{Ato de ticar.}{ti.que}{0}
\verb{tique"-taque}{}{Onomat.}{tique"-taques}{}{s.m.}{Sequência repetitiva e cadenciada de estalidos, como o som produzido pelo relógio.}{ti.que"-ta.que}{0}
\verb{tíquete}{}{}{}{}{s.m.}{Bilhete que comprova o direito a um serviço, como viagem em transporte público, refeição, espetáculo.}{tí.que.te}{0}
\verb{tira}{}{}{}{}{s.f.}{Pedaço de qualquer material de forma longa e estreita.}{ti.ra}{0}
\verb{tira}{}{}{}{}{}{Sequência curta de história em quadrinhos.}{ti.ra}{0}
\verb{tira}{}{Bras.}{}{}{s.m.}{Agente de polícia.}{ti.ra}{0}
\verb{tiracolo}{ó}{}{}{}{s.m.}{Correia presa ao pescoço e colocada de forma diagonal no corpo.}{ti.ra.co.lo}{0}
\verb{tirada}{}{}{}{}{s.f.}{Ato ou efeito de tirar.}{ti.ra.da}{0}
\verb{tirada}{}{}{}{}{}{Longa extensão de caminho ou espaço de tempo.}{ti.ra.da}{0}
\verb{tirada}{}{}{}{}{}{Frase longa.}{ti.ra.da}{0}
\verb{tirada}{}{Pop.}{}{}{}{Ímpeto de espirituosidade.}{ti.ra.da}{0}
\verb{tiragem}{}{}{"-ens}{}{s.f.}{Ato ou efeito de tirar.}{ti.ra.gem}{0}
\verb{tiragem}{}{}{"-ens}{}{}{Número de exemplares de um trabalho gráfico impressos em uma edição, impressão ou reimpressão.}{ti.ra.gem}{0}
\verb{tira"-gosto}{ô}{}{tira"-gostos ⟨ô⟩}{}{s.m.}{Pequena porção de alimento para acompanhar bebidas.}{ti.ra"-gos.to}{0}
\verb{tira"-linhas}{}{}{}{}{s.m.}{Instrumento próprio para traçar linhas de grossura uniforme.}{ti.ra"-li.nhas}{0}
\verb{tira"-manchas}{}{}{}{}{s.m.}{Produto químico próprio para remover manchas.}{ti.ra"-man.chas}{0}
\verb{tirania}{}{}{}{}{s.m.}{Governo ou poder de tirano.}{ti.ra.ni.a}{0}
\verb{tirânico}{}{}{}{}{adj.}{Relativo a tirano; opressivo, violento.}{ti.râ.ni.co}{0}
\verb{tiranizar}{}{}{}{}{v.t.}{Governar ou comandar com tirania, severidade.}{ti.ra.ni.zar}{\verboinum{1}}
\verb{tirano}{}{}{}{}{s.m.}{Governante injusto, cruel e autoritário.}{ti.ra.no}{0}
\verb{tiranossauro}{}{Zool.}{}{}{s.m.}{Dinossauro de pescoço curto, membros anteriores reduzidos e dentes grandes.}{ti.ra.nos.sau.ro}{0}
\verb{tirante}{}{}{}{}{s.m.}{Correias ou cordas que prendem a cavalgadura ao veículo puxado.}{ti.ran.te}{0}
\verb{tirante}{}{}{}{}{}{Peça que serve para puxar.}{ti.ran.te}{0}
\verb{tirante}{}{}{}{}{adj.2g.}{Que tende a; que se aproxima de.}{ti.ran.te}{0}
\verb{tirante}{}{}{}{}{prep.}{Com exceção de; exceto.}{ti.ran.te}{0}
\verb{tira"-prosa}{ó}{Bras.}{tira"-prosas ⟨ó⟩}{}{adj.}{Diz"-se de indivíduo valentão.}{ti.ra"-pro.sa}{0}
\verb{tirar}{}{}{}{}{v.t.}{Arrancar, puxar, sacar, extrair.}{ti.rar}{0}
\verb{tirar}{}{}{}{}{}{Eliminar, retirar, suprimir.}{ti.rar}{0}
\verb{tirar}{}{}{}{}{}{Fazer sair; retirar, expulsar.}{ti.rar}{0}
\verb{tirar}{}{}{}{}{}{Fazer fotografia, radiografia.}{ti.rar}{0}
\verb{tirar}{}{}{}{}{}{Subtrair, diminuir, deduzir.}{ti.rar}{0}
\verb{tirar}{}{}{}{}{}{Roubar, subtrair, desviar.}{ti.rar}{\verboinum{1}}
\verb{tira"-teima}{}{}{}{}{s.m.}{Argumento decisivo; prova, evidência.}{ti.ra"-tei.ma}{0}
\verb{tira"-teimas}{}{}{}{}{s.m.}{Tira"-teima.}{ti.ra"-tei.mas}{0}
\verb{tireoide}{}{Anat.}{}{}{adj.}{Diz"-se de glândula localizada na parte anterior e inferior do pescoço.}{ti.re.oi.de}{0}
\verb{tireoideo}{}{Anat.}{}{}{adj.}{Relativo à glândula tireoide.}{ti.re.oi.de.o}{0}
\verb{tiririca}{}{Bot.}{}{}{s.f.}{Erva daninha que se prolifera rapidamente pelas áreas cultivadas.}{ti.ri.ri.ca}{0}
\verb{tiririca}{}{Bras.}{}{}{adj.}{Furioso, irritado.}{ti.ri.ri.ca}{0}
\verb{tiritante}{}{}{}{}{adj.2g.}{Que tirita; trêmulo.}{ti.ri.tan.te}{0}
\verb{tiritar}{}{}{}{}{v.i.}{Tremer ou bater os dentes, geralmente por causa do frio.}{ti.ri.tar}{\verboinum{1}}
\verb{tiro}{}{}{}{}{s.m.}{Ato ou efeito de atirar, especialmente com arma de fogo.}{ti.ro}{0}
\verb{tiro}{}{}{}{}{}{O projétil disparado.}{ti.ro}{0}
\verb{tiro}{}{}{}{}{}{Ato ou serviço de puxar carros, realizado por animal.}{ti.ro}{0}
\verb{tirocínio}{}{}{}{}{s.m.}{Prática, aprendizado, exercício.}{ti.ro.cí.nio}{0}
\verb{tiro"-de"-guerra}{é}{Bras.}{tiros"-de"-guerra ⟨é⟩}{}{s.m.}{Escola de instrução militar destinada aos cidadãos que não se incorporam ao Exército.}{ti.ro"-de"-guer.ra}{0}
\verb{tiroide}{}{}{}{}{}{Var. de \textit{tireoide}.}{ti.roi.de}{0}
\verb{tirolesa}{ê}{}{}{}{s.f.}{Forma de cantar, originária da região suíça do Tirol, que alterna rapidamente a voz normal com o falsete.}{ti.ro.le.sa}{0}
\verb{tirolesa}{ê}{}{}{}{}{Dança dessa região em compasso ternário e andamento moderado.}{ti.ro.le.sa}{0}
\verb{tirolesa}{ê}{Esport.}{}{}{}{Espécie de técnica vertical em que o praticante, preso a um cinto de segurança e a um sistema de mosquetões, desliza suspenso por um cabo de aço, de comprimento variável, entre dois pontos.}{ti.ro.le.sa}{0}
\verb{tiroteio}{ê}{}{}{}{s.m.}{Sequência ou troca de tiros.}{ti.ro.tei.o}{0}
\verb{tisana}{}{}{}{}{s.f.}{Bebida medicamentosa que o doente bebe regularmente.}{ti.sa.na}{0}
\verb{tísica}{}{Med.}{}{}{s.f.}{Tuberculose pulmonar.}{tí.si.ca}{0}
\verb{tísico}{}{}{}{}{adj.}{Que sofre de tísica; tuberculoso.}{tí.si.co}{0}
\verb{tisnar}{}{}{}{}{v.t.}{Tornar escuro como carvão.}{tis.nar}{0}
\verb{tisnar}{}{}{}{}{}{Tostar, manchar, macular.}{tis.nar}{\verboinum{1}}
\verb{tisne}{}{}{}{}{s.m.}{Cor que fica a pele quando exposta a calor ou fumaça.}{tis.ne}{0}
\verb{tisne}{}{}{}{}{}{Fuligem.}{tis.ne}{0}
\verb{titã}{}{Mit.}{}{}{s.m.}{Cada um dos gigantes que quiseram escalar o céu e destronar Júpiter.}{ti.tã}{0}
\verb{titã}{}{Fig.}{}{}{}{Pessoa grande e forte, física ou intelectualmente.}{ti.tã}{0}
\verb{titânico}{}{}{}{}{adj.}{Relativo aos titãs.}{ti.tâ.ni.co}{0}
\verb{titânico}{}{}{}{}{}{Que revela grande força.}{ti.tâ.ni.co}{0}
\verb{titânico}{}{Quím.}{}{}{adj.}{Relativo ao titânio.}{ti.tâ.ni.co}{0}
\verb{titânio}{}{Quím.}{}{}{s.m.}{Elemento químico metálico, branco, brilhante, dúctil, usado em ligas leves e resistentes, em revestimentos protetores e anticorrosivos, em válvulas eletrônicas etc. \elemento{22}{47.867}{Ti}.}{ti.tâ.nio}{0}
\verb{títere}{}{}{}{}{s.m.}{Fantoche, boneco.}{tí.te.re}{0}
\verb{títere}{}{}{}{}{adj.}{Diz"-se de indivíduo facilmente manipulável ou que representa os interesses alheios.}{tí.te.re}{0}
\verb{titia}{}{Pop.}{}{}{s.f.}{Tia.}{ti.ti.a}{0}
\verb{titica}{}{Pop.}{}{}{s.f.}{Excremento, caca.}{ti.ti.ca}{0}
\verb{titica}{}{}{}{}{}{Coisa ou pessoa insignificante ou ruim.}{ti.ti.ca}{0}
\verb{titilar}{}{}{}{}{v.t.}{Fazer cócegas em.}{ti.ti.lar}{0}
\verb{titilar}{}{}{}{}{}{Lisonjear, afagar.}{ti.ti.lar}{0}
\verb{titilar}{}{}{}{}{v.i.}{Palpitar, estremecer.}{ti.ti.lar}{\verboinum{1}}
\verb{titio}{}{Pop.}{}{}{s.m.}{Tio.}{ti.ti.o}{0}
\verb{tititi}{}{Pop.}{}{}{s.m.}{Mexerico, boataria, intriga.}{ti.ti.ti}{0}
\verb{tititi}{}{Pop.}{}{}{}{Confusão, barulho.}{ti.ti.ti}{0}
\verb{titubeação}{}{}{"-ões}{}{s.f.}{Ato ou efeito de titubear; hesitação.}{ti.tu.be.a.ção}{0}
\verb{titubeante}{}{}{}{}{adj.2g.}{Que titubeia; hesitante.}{ti.tu.be.an.te}{0}
\verb{titubear}{}{}{}{}{v.i.}{Hesitar, vacilar.}{ti.tu.be.ar}{0}
\verb{titubear}{}{}{}{}{}{Ter dificuldade em manter"-se em pé; cambalear.}{ti.tu.be.ar}{\verboinum{4}}
\verb{titulação}{}{}{"-ões}{}{s.f.}{Conjunto de títulos acadêmicos ou profissionais de um indivíduo.}{ti.tu.la.ção}{0}
\verb{titular}{}{}{}{}{adj.2g.}{Que tem título honorífico.}{ti.tu.lar}{0}
\verb{titular}{}{}{}{}{}{Ocupante efetivo de cargo ou função.}{ti.tu.lar}{0}
\verb{titular}{}{}{}{}{}{Que tem só o título, sem posse real; honorário.}{ti.tu.lar}{0}
\verb{titular}{}{}{}{}{}{Nobre, fidalgo.}{ti.tu.lar}{0}
\verb{titular}{}{}{}{}{v.t.}{Dar título a; intitular.}{ti.tu.lar}{0}
\verb{titular}{}{}{}{}{}{Registrar em livro oficial, tornando autêntico.}{ti.tu.lar}{\verboinum{1}}
\verb{título}{}{}{}{}{s.m.}{Nome de obra literária ou artística, artigo e matéria jornalística.}{tí.tu.lo}{0}
\verb{título}{}{}{}{}{}{Denominação honorífica.}{tí.tu.lo}{0}
\verb{título}{}{}{}{}{}{Qualificação, designação, nome.}{tí.tu.lo}{0}
\verb{título}{}{}{}{}{}{Documento oficial que torna autêntico um direito.}{tí.tu.lo}{0}
\verb{título}{}{}{}{}{}{Qualquer papel negociável que representa um valor mobiliário.}{tí.tu.lo}{0}
\verb{tiziu}{}{Zool.}{}{}{s.m.}{Pássaro preto ou pardo encontrado na América do Sul e Central.}{ti.ziu}{0}
\verb{Tl}{}{Quím.}{}{}{}{Símb. do \textit{tálio}.}{Tl}{0}
\verb{Tm}{}{Quím.}{}{}{}{Símb. do \textit{túlio}. }{Tm}{0}
\verb{TO}{}{}{}{}{}{Sigla do estado do Tocantins.}{TO}{0}
\verb{to}{}{}{}{}{pron.}{Contração do pronome \textit{te} com o pronome \textit{o}.}{to}{0}
\verb{toa}{}{}{}{}{s.m.}{Corda para rebocar embarcações.}{to.a}{0}
\verb{toa}{}{}{}{}{}{Usado na locução \textit{à toa}: a esmo, sem motivo, em vão, ao acaso.}{to.a}{0}
\verb{toada}{}{}{}{}{s.f.}{Ato ou efeito de toar.}{to.a.da}{0}
\verb{toada}{}{}{}{}{}{Entoação.}{to.a.da}{0}
\verb{toada}{}{}{}{}{}{Cantiga, tom.}{to.a.da}{0}
\verb{toada}{}{}{}{}{}{Rumor, boato.}{to.a.da}{0}
\verb{toalete}{é}{}{}{}{s.m.}{Sanitário, banheiro.}{to.a.le.te}{0}
\verb{toalete}{é}{}{}{}{}{Aposento para vestir"-se.}{to.a.le.te}{0}
\verb{toalete}{é}{}{}{}{s.f.}{Ato de se lavar, vestir e aprontar.}{to.a.le.te}{0}
\verb{toalha}{}{}{}{}{s.f.}{Peça de pano felpudo e absorvente própria para enxugar o corpo ou as mãos.}{to.a.lha}{0}
\verb{toalha}{}{}{}{}{}{Peça, geralmente de tecido e adornada, para forrar mesas e móveis.}{to.a.lha}{0}
\verb{toalheiro}{ê}{}{}{}{s.m.}{Móvel ou gancho para se pendurarem toalhas.}{to.a.lhei.ro}{0}
\verb{toalheiro}{ê}{}{}{}{}{Fabricante ou vendedor de toalhas.}{to.a.lhei.ro}{0}
\verb{toante}{}{}{}{}{adj.2g.}{Que soa bem.}{to.an.te}{0}
\verb{toar}{}{}{}{}{v.t.}{Fazer som forte; estrondear.}{to.ar}{0}
\verb{toar}{}{}{}{}{}{Ressoar.}{to.ar}{0}
\verb{toar}{}{}{}{}{}{Condizer, convir, agradar.}{to.ar}{0}
\verb{toar}{}{}{}{}{}{Ter ares de; parecer.}{to.ar}{\verboinum{7}}
\verb{tobaguiano}{}{}{}{}{adj.}{Relativo à República de Trinidad e Tobago, no sudeste da América Central, mar do Caribe.}{to.ba.gui.a.no}{0}
\verb{tobaguiano}{}{}{}{}{s.m.}{Indivíduo natural ou habitante desse país.}{to.ba.gui.a.no}{0}
\verb{tobogã}{}{}{}{}{s.m.}{Rampa ondulada para descer escorregando.}{to.bo.gã}{0}
\verb{toca}{ó}{}{}{}{s.f.}{Buraco na terra ou pedra que serve de abrigo a alguns animais.}{to.ca}{0}
\verb{toca}{ó}{}{}{}{}{Caverna, gruta, lapa.}{to.ca}{0}
\verb{toca"-CDs}{ó}{}{}{}{s.m.}{Aparelho para reproduzir o som gravado em discos digitais.}{to.ca"-CDs}{0}
\verb{toca"-discos}{ó}{}{}{}{s.m.}{Aparelho para reproduzir o som gravado em discos.}{to.ca"-dis.cos}{0}
\verb{tocado}{}{}{}{}{adj.}{Que se tocou; encostado.}{to.ca.do}{0}
\verb{tocado}{}{}{}{}{}{Que foi lesado; atingido, afetado.}{to.ca.do}{0}
\verb{tocador}{ô}{}{}{}{adj.}{Que toca.}{to.ca.dor}{0}
\verb{toca"-fitas}{ó}{}{}{}{s.m.}{Aparelho para reproduzir o som gravado em fitas magnéticas.}{to.ca"-fi.tas}{0}
\verb{tocaia}{}{}{}{}{s.f.}{Ato de esconder"-se para caçar ou atacar.}{to.cai.a}{0}
\verb{tocaia}{}{}{}{}{}{Emboscada, cilada.}{to.cai.a}{0}
\verb{tocaiar}{}{}{}{}{v.i.}{Ficar de ou armar tocaia.}{to.cai.ar}{\verboinum{1}}
\verb{tocante}{}{}{}{}{adj.2g.}{Que diz respeito; concernente, referente.}{to.can.te}{0}
\verb{tocante}{}{}{}{}{}{Emocionante, comovente.}{to.can.te}{0}
\verb{tocante}{}{}{}{}{}{Usado na expressão \textit{no tocante a}: com relação a, quanto a, a respeito de.}{to.can.te}{0}
\verb{tocantinense}{}{}{}{}{adj.2g.}{Relativo ao Tocantins.}{to.can.ti.nen.se}{0}
\verb{tocantinense}{}{}{}{}{s.2g.}{Indivíduo natural ou habitante desse estado.}{to.can.ti.nen.se}{0}
\verb{tocar}{}{}{}{}{v.t.}{Fazer parte do corpo ter contato com.}{to.car}{0}
\verb{tocar}{}{}{}{}{}{Ter contato com; encostar.}{to.car}{0}
\verb{tocar}{}{}{}{}{}{Fazer andar; impelir, incitar.}{to.car}{0}
\verb{tocar}{}{}{}{}{}{Fazer ir embora; enxotar, expulsar.}{to.car}{0}
\verb{tocar}{}{}{}{}{}{Impressionar, emocionar, comover, sensibilizar.}{to.car}{0}
\verb{tocar}{}{}{}{}{}{Operar instrumento musical, produzindo música.}{to.car}{0}
\verb{tocar}{}{}{}{}{}{Mencionar, referir, aludir, tratar.}{to.car}{0}
\verb{tocar}{}{}{}{}{}{Dizer respeito a ou ser da obrigação de; caber, competir, referir"-se.}{to.car}{\verboinum{2}}
\verb{tocata}{}{}{}{}{s.f.}{Ato ou efeito de tocar instrumentos musicais.}{to.ca.ta}{0}
\verb{tocata}{}{Mús.}{}{}{}{Composição instrumental livre.}{to.ca.ta}{0}
\verb{tocha}{ó}{}{}{}{s.f.}{Artefato para iluminação, feito de material inflamável com haste para segurar; archote, facho.}{to.cha}{0}
\verb{tocha}{ó}{}{}{}{}{Vela grande e grossa.}{to.cha}{0}
\verb{tocheiro}{ê}{}{}{}{s.m.}{Utensílio para segurar tocha.}{to.chei.ro}{0}
\verb{toco}{ô}{}{}{}{s.m.}{Parte de planta cortada que permaneceu presa ao solo.}{to.co}{0}
\verb{toco}{ô}{}{}{}{}{Pedaço curto de madeira; pau.}{to.co}{0}
\verb{tocologia}{}{Med.}{}{}{s.f.}{Obstetrícia.}{to.co.lo.gi.a}{0}
\verb{tocólogo}{}{Med.}{}{}{s.m.}{Obstetra, parteiro.}{to.có.lo.go}{0}
\verb{todavia}{}{}{}{}{conj.}{Mas, porém, no entanto.}{to.da.vi.a}{0}
\verb{todo}{ô}{}{}{}{pron.}{Cada, qualquer.}{to.do}{0}
\verb{todo}{ô}{}{}{}{adv.}{Completamente, totalmente.}{to.do}{0}
\verb{todo}{ô}{}{}{}{adj.}{Inteiro, completo.}{to.do}{0}
\verb{todo}{ô}{}{}{}{s.m.}{Conjunto, totalidade.}{to.do}{0}
\verb{todo"-poderoso}{ô\ldots{}ô}{}{todo"-poderosos ⟨ô\ldots{}ó⟩}{todo"-poderosa ⟨ô\ldots{}ó⟩}{adj.}{Que tudo pode, que tem muito poder; onipotente.}{to.do"-po.de.ro.so}{0}
\verb{todo"-poderoso}{ô\ldots{}ô}{Relig.}{todo"-poderosos ⟨ô\ldots{}ó⟩}{todo"-poderosa ⟨ô\ldots{}ó⟩}{s.m.}{Deus. (Usa"-se com inicial maiúscula nesta acepção.)}{to.do"-po.de.ro.so}{0}
\verb{toesa}{ê}{}{}{}{s.f.}{Antiga medida francesa de comprimento.}{to.e.sa}{0}
\verb{toesa}{ê}{Pop.}{}{}{}{Pé muito grande.}{to.e.sa}{0}
\verb{toga}{ó}{}{}{}{s.f.}{Na Roma Antiga, manto comprido que se usava diagonalmente ao corpo.}{to.ga}{0}
\verb{toga}{ó}{}{}{}{}{Vestuário de magistrado; beca.}{to.ga}{0}
\verb{toga}{ó}{Fig.}{}{}{}{Magistratura.}{to.ga}{0}
\verb{togado}{}{}{}{}{adj.}{Que usa toga.}{to.ga.do}{0}
\verb{togolês}{}{}{}{}{adj.}{Relativo ao Togo (África Ocidental).}{to.go.lês}{0}
\verb{togolês}{}{}{}{}{s.m.}{Indivíduo natural ou habitante desse país.}{to.go.lês}{0}
\verb{toiça}{}{}{}{}{}{Var. de \textit{touça}.}{toi.ça}{0}
\verb{toiceira}{ê}{}{}{}{}{Var. de \textit{touceira}.}{toi.cei.ra}{0}
\verb{toicinho}{}{}{}{}{s.m.}{Gordura que fica por baixo da pele do porco, usada em culinária.}{toi.ci.nho}{0}
\verb{toirada}{}{}{}{}{}{Var. de \textit{tourada}.}{toi.ra.da}{0}
\verb{toirear}{}{}{}{}{}{Var. de \textit{tourear}.}{toi.re.ar}{0}
\verb{toireiro}{ê}{}{}{}{}{Var. de \textit{toureiro}.}{toi.rei.ro}{0}
\verb{toiro}{ô}{}{}{}{}{Var. de \textit{touro}.}{toi.ro}{0}
\verb{toitiço}{}{}{}{}{}{Var. de \textit{toutiço}.}{toi.ti.ço}{0}
\verb{tolda}{ô}{}{}{}{s.f.}{Toldo.}{tol.da}{0}
\verb{tolda}{ô}{}{}{}{}{Em algumas embarcações, cobertura de palha ou madeira para passageiros ou carga.}{tol.da}{0}
\verb{toldar}{}{}{}{}{v.t.}{Cobrir, encobrir.}{tol.dar}{0}
\verb{toldar}{}{}{}{}{}{Anuviar, nublar.}{tol.dar}{0}
\verb{toldar}{}{}{}{}{}{Turvar.}{tol.dar}{0}
\verb{toldar}{}{Fig.}{}{}{}{Obscurecer.}{tol.dar}{0}
\verb{toldar}{}{Fig.}{}{}{}{Entristecer.}{tol.dar}{\verboinum{1}}
\verb{toldo}{ô}{}{}{}{s.m.}{Peça, geralmente de brim ou lona, própria para fazer cobertura para abrigar pessoas ou objetos.}{tol.do}{0}
\verb{toleirão}{}{}{"-ões}{}{adj.}{Muito tolo; pateta.}{to.lei.rão}{0}
\verb{tolerância}{}{}{}{}{s.f.}{Qualidade de tolerante.}{to.le.rân.cia}{0}
\verb{tolerante}{}{}{}{}{adj.2g.}{Que tolera.}{to.le.ran.te}{0}
\verb{tolerar}{}{}{}{}{v.t.}{Suportar pacientemente, ainda que contra a vontade.}{to.le.rar}{0}
\verb{tolerar}{}{}{}{}{}{Consentir, permitir, deixar.}{to.le.rar}{\verboinum{1}}
\verb{tolerável}{}{}{"-eis}{}{adj.2g.}{Que se pode tolerar.}{to.le.rá.vel}{0}
\verb{tolete}{ê}{}{}{}{s.f.}{Cavilha de ferro ou madeira, colocada na borda de canoa ou outra embarcação, para servir de apoio ao remo.}{to.le.te}{0}
\verb{tolher}{ê}{}{}{}{v.t.}{Causar embaraço ou impedimento; estorvar.}{to.lher}{0}
\verb{tolher}{ê}{}{}{}{}{Privar de.}{to.lher}{0}
\verb{tolher}{ê}{}{}{}{}{Tornar paralisado.}{to.lher}{0}
\verb{tolher}{ê}{}{}{}{}{Reprimir.}{to.lher}{\verboinum{12}}
\verb{tolhimento}{}{}{}{}{s.m.}{Ato ou efeito de tolher.}{to.lhi.men.to}{0}
\verb{tolice}{}{}{}{}{s.f.}{Qualidade, ato ou dito de tolo; asneira, disparate, paspalhice, patetice.}{to.li.ce}{0}
\verb{tolo}{ô}{}{}{}{adj.}{Que tem pouca inteligência; bobo, ignorante.}{to.lo}{0}
\verb{tolo}{ô}{}{}{}{}{Sem razão de ser; sem importância. (\textit{O casal se separou por um motivo tolo.})}{to.lo}{0}
\verb{tolueno}{}{Quím.}{}{}{s.m.}{Hidrocarboneto aromático, obtido na destilação do petróleo e do carvão, empregado como solvente.}{to.lu.e.no}{0}
\verb{tom}{}{Mús.}{tons}{}{s.m.}{Altura de um som na escala musical.}{tom}{0}
\verb{tom}{}{}{tons}{}{}{Grau de intensidade de uma tinta ou cor.}{tom}{0}
\verb{tom}{}{}{tons}{}{}{Modo de expressar"-se.}{tom}{0}
\verb{tomada}{}{}{}{}{s.f.}{Dispositivo terminal de uma instalação elétrica, em que se ligam aparelhos.}{to.ma.da}{0}
\verb{tomada}{}{}{}{}{}{Registro contínuo de cena de filme ou vídeo.}{to.ma.da}{0}
\verb{tomar}{}{}{}{}{v.t.}{Tirar alguma coisa de uma pessoa contra a vontade dela; arrebatar.}{to.mar}{0}
\verb{tomar}{}{}{}{}{}{Passar a segurar alguma coisa; empunhar.}{to.mar}{0}
\verb{tomar}{}{}{}{}{}{Passar a ocupar algo.}{to.mar}{0}
\verb{tomar}{}{}{}{}{}{Beber alguma coisa.}{to.mar}{0}
\verb{tomar}{}{}{}{}{}{Seguir em determinada direção; pegar.}{to.mar}{\verboinum{1}}
\verb{tomara}{}{}{}{}{interj.}{Expressão que denota desejos, votos.}{to.ma.ra}{0}
\verb{tomara"-que"-caia}{}{}{}{}{s.m.}{Peça de vestuário feminino que cobre o busto, sem nada que a prenda aos ombros ou ao pescoço. }{to.ma.ra"-que"-cai.a}{0}
\verb{tomate}{}{}{}{}{s.m.}{Fruto arredondado e vermelho quando maduro, de pele fina e muitas sementes, próprio para salada ou molho.}{to.ma.te}{0}
\verb{tomateira}{ê}{Bot.}{}{}{s.f.}{Tomateiro.}{to.ma.tei.ra}{0}
\verb{tomateiro}{ê}{Bot.}{}{}{s.m.}{Planta hortense cujo fruto, comestível, é o tomate.}{to.ma.tei.ro}{0}
\verb{tombadilho}{}{}{}{}{s.m.}{A parte mais alta de um navio, situada entre a popa e o mastro da mezena.}{tom.ba.di.lho}{0}
\verb{tombamento}{}{}{}{}{s.m.}{Ato ou efeito de tombar.}{tom.ba.men.to}{0}
\verb{tombar}{}{}{}{}{v.t.}{Fazer cair.}{tom.bar}{0}
\verb{tombar}{}{}{}{}{}{Pôr sob proteção estatal uma construção de valor histórico ou artístico.}{tom.bar}{0}
\verb{tombar}{}{}{}{}{v.i.}{Inclinar"-se.}{tom.bar}{\verboinum{1}}
\verb{tombo}{}{}{}{}{s.m.}{Ato de cair no chão; queda.}{tom.bo}{0}
\verb{tomento}{}{Bot.}{}{}{s.m.}{Conjunto de pelos minúsculos, invisíveis a olho nu, que recobrem certos órgãos ou partes das plantas.}{to.men.to}{0}
\verb{tomilho}{}{Bot.}{}{}{s.m.}{Erva cujas folhas, aromáticas, são muito usadas como tempero.}{to.mi.lho}{0}
\verb{tomo}{}{}{}{}{s.m.}{Divisão de uma obra, nem sempre igual ao volume impresso.}{to.mo}{0}
\verb{tomografia}{}{Med.}{}{}{s.f.}{Radiografia que demonstra, detalhadamente, imagens de vários planos de um órgão ou de uma região do corpo.}{to.mo.gra.fi.a}{0}
\verb{tomógrafo}{}{}{}{}{s.m.}{Aparelho empregado na realização de tomografias.}{to.mó.gra.fo}{0}
\verb{tona}{}{}{}{}{s.f.}{Casca, película.}{to.na}{0}
\verb{tona}{}{Fig.}{}{}{}{Superfície.}{to.na}{0}
\verb{tona}{}{}{}{}{}{Palavra usada na expressão \textit{à tona}: na superfície da água.}{to.na}{0}
\verb{tonal}{}{Mús.}{"-ais}{}{adj.2g.}{Relativo ao tom, ou à tonalidade.}{to.nal}{0}
\verb{tonalidade}{}{}{}{}{s.f.}{Variedade de uma cor; matiz, tom.}{to.na.li.da.de}{0}
\verb{tonalidade}{}{Mús.}{}{}{}{Sistema que rege as escalas musicais e seus encadeamentos.}{to.na.li.da.de}{0}
\verb{tonalizar}{}{}{}{}{v.t.}{Dar tom ou tonalidade a.}{to.na.li.zar}{\verboinum{1}}
\verb{tonante}{}{}{}{}{adj.2g.}{Que troveja, atroa. }{to.nan.te}{0}
\verb{tonante}{}{}{}{}{}{Vibrante, forte.}{to.nan.te}{0}
\verb{tonel}{é}{}{"-éis}{}{s.m.}{Vasilha grande, feita de tábuas curvadas e unidas por cintas de madeira ou metal, própria para guardar ou transportar líquidos. }{to.nel}{0}
\verb{tonelada}{}{}{}{}{s.f.}{Unidade de massa igual a 1000 kg.}{to.ne.la.da}{0}
\verb{tonelagem}{}{}{}{}{s.f.}{Capacidade de um veículo de carga.}{to.ne.la.gem}{0}
\verb{tonelagem}{}{}{}{}{}{A medida dessa capacidade.}{to.ne.la.gem}{0}
\verb{tôner}{}{}{}{}{s.m.}{Pó preto, fino e resinoso usado nos processos de reprodução eletrostática para imprimir imagens, sendo o equivalente da tinta nas impressoras a laser; tonalizador.}{tô.ner}{0}
\verb{tônica}{}{Gram.}{}{}{s.f.}{A sílaba mais forte na pronúncia de uma palavra.}{tô.ni.ca}{0}
\verb{tonicidade}{}{}{}{}{s.f.}{Qualidade ou estado do que é tônico. }{to.ni.ci.da.de}{0}
\verb{tonicidade}{}{Gram.}{}{}{}{Propriedade de uma vogal ou de uma sílaba de ser pronunciada com maior intensidade; acentuação.}{to.ni.ci.da.de}{0}
\verb{tônico}{}{}{}{}{adj.}{Diz"-se de remédio revigorante.}{tô.ni.co}{0}
\verb{tônico}{}{}{}{}{}{Relativo a tom.}{tô.ni.co}{0}
\verb{tônico}{}{Gram.}{}{}{}{Que se pronuncia com acento.}{tô.ni.co}{0}
\verb{tonificador}{ô}{}{}{}{adj.}{Tonificante.}{to.ni.fi.ca.dor}{0}
\verb{tonificante}{}{}{}{}{adj.2g.}{Que tonifica; tonificador.}{to.ni.fi.can.te}{0}
\verb{tonificar}{}{}{}{}{v.t.}{Dar mais forças ao corpo evitando doenças; fortalecer, fortificar.}{to.ni.fi.car}{\verboinum{2}}
\verb{toninha}{}{Zool.}{}{}{s.f.}{Nome comum aos cetáceos marinhos de pequeno porte, semelhantes aos golfinhos, mas com o focinho curto; boto.}{to.ni.nha}{0}
\verb{tonitruante}{}{}{}{}{adj.2g.}{Que troveja; atroador.}{to.ni.tru.an.te}{0}
\verb{tonsila}{}{Anat.}{}{}{s.f.}{Amígdala.}{ton.si.la}{0}
\verb{tonsura}{}{}{}{}{s.f.}{Ato ou efeito de tonsurar.}{ton.su.ra}{0}
\verb{tonsura}{}{Relig.}{}{}{}{Corte circular nos cabelos de religiosos.}{ton.su.ra}{0}
\verb{tonsurado}{}{}{}{}{adj.}{Que recebeu a tonsura.}{ton.su.ra.do}{0}
\verb{tonsurado}{}{}{}{}{}{Que teve o cabelo ou os pelos cortados; tosquiado.}{ton.su.ra.do}{0}
\verb{tonsurado}{}{}{}{}{s.m.}{Clérigo.}{ton.su.ra.do}{0}
\verb{tonsurar}{}{}{}{}{v.t.}{Cortar o cabelo de; tosquiar.}{ton.su.rar}{0}
\verb{tonsurar}{}{Relig.}{}{}{}{Praticar a cerimônia religiosa da primeira tonsura, ao ingressar na hierarquia.}{ton.su.rar}{\verboinum{1}}
\verb{tontear}{}{}{}{}{v.i.}{Agir como um tonto.}{ton.te.ar}{0}
\verb{tontear}{}{}{}{}{}{Deixar tombar a cabeça.}{ton.te.ar}{0}
\verb{tontear}{}{}{}{}{}{Fazer ficar tonto ou ter tonturas.}{ton.te.ar}{0}
\verb{tontear}{}{}{}{}{v.i.}{Causar perturbação; alvoroçar}{ton.te.ar}{\verboinum{4}}
\verb{tonteira}{ê}{}{}{}{s.f.}{Ato ou modos de tolo; tolice.}{ton.tei.ra}{0}
\verb{tonteira}{ê}{}{}{}{}{Vertigem.}{ton.tei.ra}{0}
\verb{tontice}{}{}{}{}{s.f.}{Ato, dito ou modos de tonto; asneira, tolice.}{ton.ti.ce}{0}
\verb{tonto}{}{}{}{}{adj.}{Que sente que tudo está rodando em volta de si; zonzo.}{ton.to}{0}
\verb{tonto}{}{}{}{}{}{Que é bobo, tolo.}{ton.to}{0}
\verb{tontura}{}{}{}{}{s.f.}{Estado de tonto. }{ton.tu.ra}{0}
\verb{tontura}{}{}{}{}{}{Sensação de rotação dos objetos e do meio ambiente ao redor do indivíduo; tonteira, vertigem.}{ton.tu.ra}{0}
\verb{tônus}{}{}{}{}{s.m.}{Capacidade de contração de um músculo.}{tô.nus}{0}
\verb{top}{ó}{}{}{}{s.m.}{Espécie de blusa curta, sem mangas, que deixa à mostra os ombros e a barriga. }{top}{0}
\verb{topada}{}{}{}{}{s.f.}{Batida dada com a ponta do pé em alguma coisa.}{to.pa.da}{0}
\verb{topar}{}{}{}{}{v.t.}{Encontrar"-se com pessoa ou coisa sem esperar; defrontar"-se, deparar"-se.}{to.par}{0}
\verb{topar}{}{}{}{}{}{Aceitar uma proposta.}{to.par}{0}
\verb{topar}{}{}{}{}{}{Dar uma topada em alguma coisa.}{to.par}{\verboinum{1}}
\verb{topázio}{}{}{}{}{s.m.}{Pedra preciosa de cor amarela.}{to.pá.zio}{0}
\verb{tope}{ó}{}{}{}{s.m.}{Parte mais alta em que alguma coisa termina; topo.}{to.pe}{0}
\verb{tope}{ó}{}{}{}{}{Laço de fita em chapéu, flâmula etc.}{to.pe}{0}
\verb{topetudo}{}{}{}{}{adj.}{Que traz ou usa topete.}{to.pe.tu.do}{0}
\verb{topetudo}{}{Fig.}{}{}{}{Atrevido, ousado, audacioso.}{to.pe.tu.do}{0}
\verb{tópico}{}{}{}{}{adj.}{Relativo àquilo de que se trata.}{tó.pi.co}{0}
\verb{tópico}{}{}{}{}{}{Diz"-se de medicamento de aplicação externa.}{tó.pi.co}{0}
\verb{tópico}{}{}{}{}{s.m.}{Assunto, tema.}{tó.pi.co}{0}
\verb{tópico}{}{}{}{}{}{Pequeno comentário de jornal.}{tó.pi.co}{0}
\verb{topless}{}{}{}{}{s.m.}{Prática que consiste em usar apenas a parte inferior do biquíni, deixando os seios à mostra.}{\textit{topless}}{0}
\verb{topo}{ô}{}{}{}{s.m.}{Parte mais alta; cume.}{to.po}{0}
\verb{topografia}{}{}{}{}{s.f.}{Descrição detalhada de uma localidade, de um terreno etc.; topologia.}{to.po.gra.fi.a}{0}
\verb{topógrafo}{}{}{}{}{s.m.}{Pessoa especialista em topografia.}{to.pó.gra.fo}{0}
\verb{topológico}{}{}{}{}{adj.}{Relativo a topologia. }{to.po.ló.gi.co}{0}
\verb{toponímia}{}{}{}{}{s.f.}{Estudo dos nomes próprios de lugares.}{to.po.ní.mia}{0}
\verb{toponímico}{}{}{}{}{adj.}{Relativo a toponímia ou a topônimo.}{to.po.ní.mi.co}{0}
\verb{topônimo}{}{}{}{}{s.m.}{Nome próprio de lugar.}{to.pô.ni.mo}{0}
\verb{toque}{ó}{}{}{}{s.m.}{Ato de tocar em alguma coisa; contato.}{to.que}{0}
\verb{toque}{ó}{}{}{}{}{Conjunto de sons produzidos por um instrumento.}{to.que}{0}
\verb{torá}{}{}{}{}{s.f.}{A lei mosaica ou o livro que encerra essa lei.}{to.rá}{0}
\verb{tora}{ó}{}{}{}{s.f.}{Tronco grosso de madeira.}{to.ra}{0}
\verb{torácico}{}{}{}{}{adj.}{Relativo ao tórax.}{to.rá.ci.co}{0}
\verb{tórax}{cs}{Anat.}{}{}{s.m.}{Parte do tronco que guarda pulmões e coração.}{tó.rax}{0}
\verb{torção}{}{}{"-ões}{}{s.f.}{Rotação brusca e doída de articulação, órgão etc.}{tor.ção}{0}
\verb{torcedor}{ô}{}{}{}{adj.}{Que torce por um time.}{tor.ce.dor}{0}
\verb{torcedura}{}{}{}{}{s.f.}{Ato ou efeito de torcer; torção.}{tor.ce.du.ra}{0}
\verb{torcer}{ê}{}{}{}{v.t.}{Pegar as duas pontas de alguma coisa e virar cada uma delas em uma direção.}{tor.cer}{0}
\verb{torcer}{ê}{}{}{}{}{Deslocar articulação; luxar.}{tor.cer}{0}
\verb{torcer}{ê}{}{}{}{}{Contar algo de maneira diferente da que aconteceu; alterar, distorcer.}{tor.cer}{0}
\verb{torcer}{ê}{}{}{}{}{Desejar muito o sucesso ou o fracasso de.}{tor.cer}{0}
\verb{torcer}{ê}{}{}{}{}{Ser simpatizante de time, equipe etc.}{tor.cer}{\verboinum{15}}
\verb{torcicolo}{ó}{}{}{}{s.m.}{Forte contração de músculo do pescoço.}{tor.ci.co.lo}{0}
\verb{torcida}{}{}{}{}{s.f.}{Grupo de torcedores numa competição.}{tor.ci.da}{0}
\verb{torcida}{}{}{}{}{}{Conjunto de fios torcidos para se acenderem; mecha, pavio.}{tor.ci.da}{0}
\verb{torcido}{}{}{}{}{adj.}{Que se torceu.}{tor.ci.do}{0}
\verb{torcido}{}{}{}{}{}{Torto, sinuoso.}{tor.ci.do}{0}
\verb{tordilho}{}{}{}{}{s.m.}{Cavalo de pelos negros com manchas brancas.}{tor.di.lho}{0}
\verb{tordo}{ô}{Zool.}{}{}{s.m.}{Tipo de pássaro branco, com manchas pretas.}{tor.do}{0}
\verb{tório}{}{Quím.}{}{}{s.m.}{Elemento químico metálico, radioativo, dúctil, maleável, do grupo dos actinídeos, usado  em camisas de lampião, em eletrodos de tungstênio ou de níquel em lâmpadas a gás etc. \elemento{90}{232.0381}{Th}.}{tó.rio}{0}
\verb{tormenta}{}{}{}{}{s.f.}{Temporal violento; tempestade.}{tor.men.ta}{0}
\verb{tormento}{}{}{}{}{s.m.}{Sofrimento muito grande; aflição, angústia.}{tor.men.to}{0}
\verb{tormento}{}{}{}{}{}{Punição corporal; suplício, tortura.}{tor.men.to}{0}
\verb{tormentório}{}{}{}{}{adj.}{Relativo a tormenta, ou em que há tormenta; tormentoso.}{tor.men.tó.rio}{0}
\verb{tornado}{}{}{}{}{s.m.}{Tempestade de vento em movimento circular, que destrói e joga para os ares tudo o que encontra à sua passagem.}{tor.na.do}{0}
\verb{tornar}{}{}{}{}{v.t.}{Percorrer de novo o caminho para o ponto do qual saiu; regressar, retornar.}{tor.nar}{0}
\verb{tornar}{}{}{}{}{}{Fazer passar a ser; transformar.}{tor.nar}{\verboinum{1}}
\verb{torneamento}{}{}{}{}{s.m.}{Ato ou efeito de tornear.}{tor.ne.a.men.to}{0}
\verb{tornear}{}{}{}{}{v.t.}{Modelar no torno.}{tor.ne.ar}{0}
\verb{tornear}{}{}{}{}{}{Arredondar.}{tor.ne.ar}{\verboinum{4}}
\verb{tornearia}{}{}{}{}{s.f.}{A arte, o ofício ou a oficina de torneiro. }{tor.ne.a.ri.a}{0}
\verb{torneio}{ê}{}{}{}{s.m.}{Competição esportiva; campeonato.}{tor.nei.o}{0}
\verb{torneira}{ê}{}{}{}{s.f.}{Peça que se abre ou se fecha para controlar a saída da água de um cano ou de um reservatário.}{tor.nei.ra}{0}
\verb{torneiro}{ê}{}{}{}{s.m.}{Indivíduo que trabalha com o torno.}{tor.nei.ro}{0}
\verb{torniquete}{ê}{}{}{}{s.m.}{Faixa que se aperta em volta de um membro ferido do corpo para impedir a saída do sangue.}{tor.ni.que.te}{0}
\verb{torniquete}{ê}{}{}{}{}{Instrumento destinado a apertar.}{tor.ni.que.te}{0}
\verb{torno}{ô}{}{}{}{s.m.}{Aparelho em que se prende ou se faz girar uma peça de madeira ou metal para lhe dar a forma desejada.}{tor.no}{0}
\verb{tornozeleira}{ê}{}{}{}{s.f.}{Peça de tecido elástico usada para proteger os tornozelos .}{tor.no.ze.lei.ra}{0}
\verb{tornozelo}{ê}{}{}{}{s.m.}{Região entre a perna e o pé.}{tor.no.ze.lo}{0}
\verb{toró}{}{}{}{}{s.m.}{Chuva muito forte e súbita.}{to.ró}{0}
\verb{torpe}{ô}{}{}{}{adj.2g.}{Em que há maldade e desonestidade; infame, vil.}{tor.pe}{0}
\verb{torpe}{ô}{}{}{}{}{Repugnante, asqueroso, sórdido.}{tor.pe}{0}
\verb{torpe}{ô}{}{}{}{}{Obsceno, indecente.}{tor.pe}{0}
\verb{torpedear}{}{}{}{}{v.t.}{Lançar torpedo contra.}{tor.pe.de.ar}{\verboinum{4}}
\verb{torpedeiro}{ê}{}{}{}{s.m.}{Navio de guerra que lança torpedos.}{tor.pe.dei.ro}{0}
\verb{torpedo}{ê}{}{}{}{s.m.}{Míssil cilíndrico, que uma embarcação, um submarino ou um avião atira contra um navio para afundá"-lo.}{tor.pe.do}{0}
\verb{torpeza}{ê}{}{}{}{s.f.}{Qualidade de torpe.}{tor.pe.za}{0}
\verb{torpeza}{ê}{}{}{}{}{Comportamento indigno, desonesto, vil.}{tor.pe.za}{0}
\verb{torpor}{ô}{}{}{}{s.m.}{Ausência de reação a estímulos externos.}{tor.por}{0}
\verb{torração}{}{}{"-ões}{}{s.f.}{Ato ou efeito de torrar; torra. }{tor.ra.ção}{0}
\verb{torrada}{}{}{}{}{s.f.}{Fatia de pão torrado.}{tor.ra.da}{0}
\verb{torradeira}{ê}{}{}{}{s.f.}{Utensílio próprio para fazer torradas.}{tor.ra.dei.ra}{0}
\verb{torrão}{}{}{"-ões}{}{s.m.}{Porção sólida de terra ou de qualquer granulado.}{tor.rão}{0}
\verb{torrão}{}{}{"-ões}{}{}{Terra natal.}{tor.rão}{0}
\verb{torrar}{}{}{}{}{v.t.}{Fazer com que o calor deixe alguma coisa muito seca.}{tor.rar}{0}
\verb{torrar}{}{}{}{}{}{Tostar.}{tor.rar}{0}
\verb{torrar}{}{}{}{}{}{Vender alguma coisa por preço mais baixo; liquidar.}{tor.rar}{0}
\verb{torrar}{}{Pop.}{}{}{}{Importunar, encher.}{tor.rar}{\verboinum{1}}
\verb{torre}{ô}{}{}{}{s.m.}{Construção alta e estreita, geralmente fortificada, usada antigamente para defesa em caso de guerra.}{tor.re}{0}
\verb{torre}{ô}{}{}{}{}{Construção alta, anexa a uma igreja, onde fica o sino; campanário.}{tor.re}{0}
\verb{torre}{ô}{}{}{}{}{Estrutura metálica que sustenta fios da rede elétrica ou antenas transmissoras de rádio e televisão.}{tor.re}{0}
\verb{torre}{ô}{}{}{}{}{Peça do jogo de xadrez.}{tor.re}{0}
\verb{torreão}{}{}{"-ões}{}{s.m.}{Torre larga, com ameias, em castelo ou fortaleza; bastilhão. }{tor.re.ão}{0}
\verb{torrefação}{}{}{"-ões}{}{s.f.}{Ato ou efeito de torrefazer.}{tor.re.fa.ção}{0}
\verb{torrefação}{}{}{"-ões}{}{}{Estabelecimento onde se torra café.}{tor.re.fa.ção}{0}
\verb{torrefazer}{ê}{}{}{}{v.t.}{Tornar tórrido; tostar, torrar.}{tor.re.fa.zer}{\verboinum{12}}
\verb{torrencial}{}{}{"-ais}{}{adj.2g.}{Relativo a torrente; abundante, copioso.}{tor.ren.ci.al}{0}
\verb{torrente}{}{}{}{}{s.f.}{Corrente de água, rápida e volumosa, formada por chuvas muito fortes.}{tor.ren.te}{0}
\verb{torrente}{}{}{}{}{}{Multidão impetuosa.}{tor.ren.te}{0}
\verb{torresmo}{ê}{}{}{}{s.m.}{Pequenos pedaços crocantes de toicinho frito.}{tor.res.mo}{0}
\verb{tórrido}{}{}{}{}{adj.}{Muito quente; ardente.}{tór.ri.do}{0}
\verb{torrificar}{}{}{}{}{v.t.}{Tornar tórrido; torrar.}{tor.ri.fi.car}{\verboinum{2}}
\verb{torrinha}{}{}{}{}{s.f.}{Diminutivo de \textit{torre}.}{tor.ri.nha}{0}
\verb{torrinha}{}{}{}{}{}{Nos teatros, galeria mais alta; poleiro, galinheiro.}{tor.ri.nha}{0}
\verb{torso}{ô}{Anat.}{}{}{s.m.}{Parte do corpo entre o pescoço e a parte superior das coxas; tronco.}{tor.so}{0}
\verb{torta}{ó}{Cul.}{}{}{s.f.}{Espécie de pastelão com recheio doce ou salgado.}{tor.ta}{0}
\verb{torto}{ô}{}{"-os ⟨ó⟩}{"-a ⟨ó⟩}{adj.}{Que se torceu; torcido.}{tor.to}{0}
\verb{torto}{ô}{}{"-os ⟨ó⟩}{"-a ⟨ó⟩}{}{Que tem curva; tortuoso, sinuoso.}{tor.to}{0}
\verb{torto}{ô}{}{"-os ⟨ó⟩}{"-a ⟨ó⟩}{}{Mal colocado; inclinado, oblíquo.}{tor.to}{0}
\verb{tortuoso}{ô}{}{"-osos ⟨ó⟩}{"-osa ⟨ó⟩}{adj.}{Que apresenta muitas curvas; sinuoso.}{tor.tu.o.so}{0}
\verb{tortura}{}{}{}{}{s.f.}{Violência física ou moral; suplício, tormento, sofrimento.}{tor.tu.ra}{0}
\verb{torturador}{ô}{}{}{}{adj.}{Diz"-se daquele que pratica tortura; carrasco, verdugo.}{tor.tu.ra.dor}{0}
\verb{torturante}{}{}{}{}{adj.2g.}{Que tortura; aflitivo, angustiante.}{tor.tu.ran.te}{0}
\verb{torturar}{}{}{}{}{v.t.}{Submeter à tortura; atormentar, supliciar.}{tor.tu.rar}{\verboinum{1}}
\verb{torvar}{}{}{}{}{v.t.}{Causar dificuldade; impedir, estorvar.}{torvar}{0}
\verb{torvar}{}{}{}{}{v.i.}{Tornar torvo, carrancudo.}{torvar}{\verboinum{1}}
\verb{torvelinho}{}{}{}{}{s.m.}{Movimento de águas ou vento em espiral; redemoinho.}{tor.ve.li.nho}{0}
\verb{torvo}{ô}{}{}{}{adj.}{Que provoca terror; pavoroso, horrível.}{tor.vo}{0}
\verb{torvo}{ô}{}{}{}{}{De aspecto carrancudo, carregado.}{tor.vo}{0}
\verb{tosa}{ó}{}{}{}{s.f.}{Ato ou efeito de tosar; tosquia.}{to.sa}{0}
\verb{tosão}{}{}{"-ões}{}{s.m.}{Lã de carneiro.}{to.são}{0}
\verb{tosar}{}{}{}{}{v.t.}{Cortar a lã ou o pelo; tosquiar.}{to.sar}{0}
\verb{tosar}{}{}{}{}{}{Cortar rente; rapar.}{to.sar}{\verboinum{1}}
\verb{tosco}{ô}{}{}{}{adj.}{Tal como se encontra na natureza; não lapidado, não polido.}{tos.co}{0}
\verb{tosco}{ô}{}{}{}{}{Feito sem cuidado; grosseiro, rústico.}{tos.co}{0}
\verb{tosco}{ô}{}{}{}{}{Inculto, rude.}{tos.co}{0}
\verb{tosquia}{}{}{}{}{s.f.}{Ato ou efeito de tosquiar; tosa.}{tos.qui.a}{0}
\verb{tosquiar}{}{}{}{}{v.t.}{Cortar os pelos de animal; tosar.}{tosquiar}{\verboinum{1}}
\verb{tosse}{ó}{Med.}{}{}{s.f.}{Saída súbita e ruidosa do ar dos pulmões, causada por irritação ou inflamação das vias respiratórias.}{tos.se}{0}
\verb{tossir}{}{}{}{}{v.i.}{Ter ataque de tosse.}{tos.sir}{\verboinum{18}}
\verb{tostado}{}{}{}{}{adj.}{Que se tostou; queimado de leve; crestado.}{tos.ta.do}{0}
\verb{tostão}{}{}{"-ões}{}{s.m.}{Antiga moeda de cem réis.}{tos.tão}{0}
\verb{tostão}{}{}{"-ões}{}{}{Soma de dinheiro que vale pouco.}{tos.tão}{0}
\verb{tostar}{}{}{}{}{v.t.}{Queimar de leve; crestar. (\textit{Minha mãe deixou tostar o pão do jeito que eu gosto.})}{tos.tar}{\verboinum{1}}
\verb{total}{}{}{"-ais}{}{adj.2g.}{Que forma ou abrange um todo; completo, inteiro.}{to.tal}{0}
\verb{total}{}{}{"-ais}{}{s.m.}{Resultado de uma adição; soma, totalidade.}{to.tal}{0}
\verb{totalidade}{}{}{}{}{s.f.}{Conjunto dos componentes de um todo; soma.}{to.ta.li.da.de}{0}
\verb{totalitário}{}{}{}{}{adj.}{Diz"-se do regime político ou da doutrina que não admite nenhum tipo de oposição e centraliza todos os poderes políticos e administrativos.}{to.ta.li.tá.rio}{0}
\verb{totalitarismo}{}{}{}{}{s.m.}{Sistema de governo totalitário.}{to.ta.li.ta.ris.mo}{0}
\verb{totalitarista}{}{}{}{}{adj.2g.}{Relativo ao totalitarismo.}{to.ta.li.ta.ris.ta}{0}
\verb{totalitarista}{}{}{}{}{s.2g.}{Seguidor ou adepto do totalitarismo.}{to.ta.li.ta.ris.ta}{0}
\verb{totalização}{}{}{"-ões}{}{s.f.}{Ato ou efeito de totalizar; soma.}{to.ta.li.za.ção}{0}
\verb{totalizador}{ô}{}{}{}{adj.}{Que totaliza ou dá a soma de um conjunto de operações.}{to.ta.li.za.dor}{0}
\verb{totalizar}{}{}{}{}{v.t.}{Calcular o total; somar.}{to.ta.li.zar}{\verboinum{1}}
\verb{totem}{}{}{}{}{s.m.}{Animal ou planta que serve como símbolo sagrado de uma comunidade sendo considerado seu ancestral ou protetor.}{to.tem}{0}
\verb{tóteme}{}{}{}{}{s.m.}{Totem.}{tó.te.me}{0}
\verb{totêmico}{}{}{}{}{adj.}{Relativo a totem.}{to.tê.mi.co}{0}
\verb{totemismo}{}{}{}{}{s.m.}{Sistema de crenças religiosas e sociais baseadas na afinidade mística entre uma comunidade e um totem.}{to.te.mis.mo}{0}
\verb{touca}{ô}{}{}{}{s.f.}{Peça de roupa usada para cobrir a cabeça como adorno ou proteção.}{tou.ca}{0}
\verb{touça}{}{}{}{}{s.f.}{Agrupamento de plantas; moita.}{tou.ça}{0}
\verb{toucado}{}{}{}{}{s.m.}{Conjunto dos enfeites que a mulher usa na cabeça.}{tou.ca.do}{0}
\verb{toucador}{ô}{}{}{}{s.m.}{Cômoda com espelho usada para se toucar ou pentear; penteadeira.}{tou.ca.dor}{0}
\verb{toucar}{}{}{}{}{v.t.}{Cobrir com touca.}{tou.car}{0}
\verb{toucar}{}{}{}{}{}{Arrumar ou enfeitar o cabelo; pentear.}{tou.car}{\verboinum{2}}
\verb{touceira}{ê}{}{}{}{s.f.}{Grande touça ou moita.}{tou.cei.ra}{0}
\verb{toucinho}{}{}{}{}{}{Var. de \textit{toicinho}.}{tou.ci.nho}{0}
\verb{toupeira}{ê}{Zool.}{}{}{s.f.}{Mamífero que se alimenta de insetos, tem os olhos atrofiados, corpo alongado, focinho longo e patas  adaptadas para cavar ou nadar.}{tou.pei.ra}{0}
\verb{toupeira}{ê}{Fig.}{}{}{}{Indivíduo ignorante, estúpido.}{tou.pei.ra}{0}
\verb{tour}{}{}{}{}{s.m.}{Viagem ou passeio promovido para mero entretenimento ou conhecimento descompromissado. }{\textit{tour}}{0}
\verb{tour}{}{}{}{}{}{Viagem, com itinerário e paradas predeterminadas, de um artista ou um grupo artístico ao exterior ou pelo interior de um país; excursão artística; turnê.}{\textit{tour}}{0}
\verb{tourada}{}{}{}{}{s.f.}{Tipo de espetáculo em que um homem enfrenta um touro; corrida de touros.}{tou.ra.da}{0}
\verb{tourear}{}{}{}{}{v.i.}{Lutar contra um touro em praça ou circo.}{tou.re.ar}{\verboinum{4}}
\verb{toureiro}{ê}{}{}{}{s.m.}{Profissional que toureia.}{tou.rei.ro}{0}
\verb{touro}{ô}{Astron.}{}{}{s.m.}{Segunda constelação zodiacal.}{tou.ro}{0}
\verb{touro}{ô}{Astrol.}{}{}{}{O signo do zodíaco referente a essa constelação.}{tou.ro}{0}
\verb{touro}{ô}{}{}{}{s.m.}{Boi não castrado, que pode reproduzir.}{tou.ro}{0}
\verb{toutiço}{}{}{}{}{s.m.}{Parte posterior da cabeça; nuca.}{tou.ti.ço}{0}
\verb{toxicidade}{cs}{}{}{}{s.f.}{Qualidade de tóxico.}{to.xi.ci.da.de}{0}
\verb{tóxico}{cs}{}{}{}{adj.}{Que envenena ou faz muito mal para o organismo.}{tó.xi.co}{0}
\verb{toxicologia}{cs}{Med.}{}{}{s.f.}{Ramo da medicina que estuda as substâncias tóxicas, seus efeitos sobre o organismo e o tratamento das intoxicações.}{to.xi.co.lo.gi.a}{0}
\verb{toxicólogo}{cs}{}{}{}{s.m.}{Especialista em toxicologia.}{to.xi.có.lo.go}{0}
\verb{toxicomania}{cs}{Med.}{}{}{s.f.}{Compulsão pelo consumo de substâncias com efeito psíquico, como álcool, cocaína, nicotina, heroína.}{to.xi.co.ma.ni.a}{0}
\verb{toxicômano}{cs}{}{}{}{adj.}{Que apresenta compulsão pelo consumo de substâncias com efeito psíquico.}{to.xi.cô.ma.no}{0}
\verb{toxina}{cs}{Quím.}{}{}{s.f.}{Substância tóxica de origem orgânica.}{to.xi.na}{0}
\verb{TPM}{}{Med.}{}{}{s.f.}{Sigla de \textit{tensão pré"-menstrual}, quadro caracterizado por ansiedade, irritabilidade, dor nos seios, dor de cabeça etc., que pode ocorrer nos dias que precedem a menstruação.}{TPM}{0}
\verb{trabalhadeira}{ê}{}{}{}{adj.}{Diz"-se da mulher que trabalha bastante. }{tra.ba.lha.dei.ra}{0}
\verb{trabalhadeira}{ê}{}{}{}{s.f.}{Essa mulher.}{tra.ba.lha.dei.ra}{0}
\verb{trabalhador}{ô}{}{}{}{adj.}{Que trabalha.}{tra.ba.lha.dor}{0}
\verb{trabalhador}{ô}{}{}{}{}{Dado ao trabalho; esforçado, diligente.}{tra.ba.lha.dor}{0}
\verb{trabalhador}{ô}{}{}{}{s.m.}{Indivíduo que não tem propriedade de meios de produção e que vende apenas sua força de trabalho; operário, empregado.}{tra.ba.lha.dor}{0}
\verb{trabalhão}{}{}{"-ões}{}{s.m.}{Trabalho extenso, árduo ou cansativo; trabalheira.}{tra.ba.lhão}{0}
\verb{trabalhar}{}{}{}{}{v.i.}{Desempenhar atividade, profissão.}{tra.ba.lhar}{0}
\verb{trabalhar}{}{}{}{}{}{Exercer regularmente alguma atividade em caráter profissional; atuar.}{tra.ba.lhar}{0}
\verb{trabalhar}{}{}{}{}{}{Estar em funcionamento (diz"-se de máquina).}{tra.ba.lhar}{0}
\verb{trabalhar}{}{}{}{}{}{Esforçar"-se, empenhar"-se.}{tra.ba.lhar}{0}
\verb{trabalhar}{}{}{}{}{v.t.}{Lavrar, manipular.}{tra.ba.lhar}{\verboinum{1}}
\verb{trabalheira}{ê}{}{}{}{s.f.}{Trabalho extenso, árduo ou cansativo.}{tra.ba.lhei.ra}{0}
\verb{trabalhismo}{}{}{}{}{s.m.}{Doutrina que defende a prioridade aos interesses dos trabalhadores como meio de obter melhorias sociais e econômicas.}{tra.ba.lhis.mo}{0}
\verb{trabalho}{}{}{}{}{s.m.}{Aplicação de esforços físicos ou mentais na realização de algo.}{tra.ba.lho}{0}
\verb{trabalho}{}{}{}{}{}{Esforço, empenho, esmero.}{tra.ba.lho}{0}
\verb{trabalho}{}{}{}{}{}{Ocupação profissional; emprego.}{tra.ba.lho}{0}
\verb{trabalho}{}{}{}{}{}{O resultado dos esforços empregados; obra, realização.}{tra.ba.lho}{0}
\verb{trabalho}{}{Bras.}{}{}{}{Obra de feitiçaria; despacho, feitiço.}{tra.ba.lho}{0}
\verb{trabalhoso}{ô}{}{"-osos ⟨ó⟩}{"-osa ⟨ó⟩}{adj.}{Diz"-se de tarefa extensa, cansativa ou que exige muito esforço, especialmente mental.}{tra.ba.lho.so}{0}
\verb{trabuco}{}{}{}{}{s.m.}{Máquina de guerra que servia para arremessar pedras.}{tra.bu.co}{0}
\verb{trabuco}{}{}{}{}{}{Arma de fogo de cano curto e boca larga, semelhante ao bacamarte.}{tra.bu.co}{0}
\verb{traça}{}{}{}{}{s.f.}{Inseto pequeno e de corpo chato, que rói tecidos e papéis.}{tra.ça}{0}
\verb{traçado}{}{}{}{}{adj.}{Que se traçou; delineado, projetado.}{tra.ça.do}{0}
\verb{traçado}{}{}{}{}{s.m.}{Ato de traçar; traço.}{tra.ça.do}{0}
\verb{traçado}{}{}{}{}{}{Desenho que representa algo que vai ser feito; esboço, plano, projeto.}{tra.ça.do}{0}
\verb{traçador}{ô}{}{}{}{adj.}{Que traça ou serve para traçar.}{tra.ça.dor}{0}
\verb{traçador}{ô}{}{}{}{s.m.}{Tipo de serra de lâmina curva, operada por duas pessoas, para cortar toras e grandes espessuras de madeira.}{tra.ça.dor}{0}
\verb{tracajá}{}{Bras.}{}{}{s.f.}{Tartaruga de grande porte, que vive em água doce, encontrada na Amazônia, cujos ovos são apreciados como alimento.}{tra.ca.já}{0}
\verb{tração}{}{}{"-ões}{}{s.f.}{Ação de uma força para deslocar um corpo.}{tra.ção}{0}
\verb{traçar}{}{}{}{}{v.t.}{Fazer traços em; riscar.}{tra.çar}{0}
\verb{traçar}{}{}{}{}{}{Desenhar por meio de traços.}{tra.çar}{0}
\verb{traçar}{}{}{}{}{}{Planejar, projetar.}{tra.çar}{0}
\verb{traçar}{}{}{}{}{}{Delimitar, marcar.}{tra.çar}{0}
\verb{traçar}{}{}{}{}{}{Colocar em posição transversal; cruzar.}{tra.çar}{\verboinum{3}}
\verb{tracejar}{}{}{}{}{v.t.}{Desenhar (algo) com pequenos traços seguidos.}{tra.ce.jar}{0}
\verb{tracejar}{}{}{}{}{v.i.}{Fazer traços ou linhas.}{tra.ce.jar}{\verboinum{1}}
\verb{tracionar}{}{}{}{}{v.t.}{Fazer tração; puxar.}{tra.ci.o.nar}{\verboinum{1}}
\verb{traço}{}{}{}{}{s.m.}{Ato ou efeito de traçar.}{tra.ço}{0}
\verb{traço}{}{}{}{}{}{Risco feito com instrumento de escrita, desenho ou pintura.}{tra.ço}{0}
\verb{traço}{}{}{}{}{}{Esboço, delineamento.}{tra.ço}{0}
\verb{traço}{}{}{}{}{}{Aspecto, caráter, qualidade.}{tra.ço}{0}
\verb{traço}{}{}{}{}{}{Vestígio, sinal, rastro.}{tra.ço}{0}
\verb{traço"-de"-união}{}{Gram.}{traços"-de"-união}{}{s.m.}{Sinal gráfico usado para ligar os elementos de uma palavra composta ou pronomes átonos a verbos; hífen.}{tra.ço"-de"-u.ni.ão}{0}
\verb{tracoma}{}{Med.}{}{}{s.m.}{Forma grave de doença oftálmica, de causa infecciosa, com comprometimento da córnea e da conjuntiva.}{tra.co.ma}{0}
\verb{tradição}{}{}{"-ões}{}{s.f.}{Informação cultural, religiosa e técnica transmitida através de gerações, especialmente por língua oral.}{tra.di.ção}{0}
\verb{tradição}{}{}{"-ões}{}{}{Qualquer informação ou prática repetida habitualmente e sem questionamento; costume, praxe.}{tra.di.ção}{0}
\verb{tradicional}{}{}{"-ais}{}{adj.2g.}{Relativo a tradição.}{tra.di.ci.o.nal}{0}
\verb{tradicional}{}{}{"-ais}{}{}{Costumeiro, usual.}{tra.di.ci.o.nal}{0}
\verb{tradicionalismo}{}{}{}{}{s.m.}{Apego às tradições.}{tra.di.ci.o.na.lis.mo}{0}
\verb{tradicionalista}{}{}{}{}{adj.2g.}{Diz"-se de indivíduo apegado às tradições.}{tra.di.ci.o.na.lis.ta}{0}
\verb{trado}{}{}{}{}{s.m.}{Verruma grande para abrir furos largos em grandes peças de madeira.}{tra.do}{0}
\verb{tradução}{}{}{"-ões}{}{s.f.}{Ato ou efeito de traduzir.}{tra.du.ção}{0}
\verb{tradução}{}{}{"-ões}{}{}{A obra traduzida.}{tra.du.ção}{0}
\verb{tradutor}{ô}{}{}{}{adj.}{Que traduz.}{tra.du.tor}{0}
\verb{traduzir}{}{}{}{}{v.t.}{Transpor para outra língua.}{tra.du.zir}{0}
\verb{traduzir}{}{}{}{}{}{Explicar, interpretar.}{tra.du.zir}{0}
\verb{traduzir}{}{}{}{}{}{Simbolizar, representar.}{tra.du.zir}{0}
\verb{trafegar}{}{}{}{}{v.i.}{Deslocar"-se por uma via; transitar.}{tra.fe.gar}{\verboinum{5}}
\verb{tráfego}{}{}{}{}{s.m.}{Deslocamento de pessoas, objetos ou informações.}{trá.fe.go}{0}
\verb{tráfego}{}{}{}{}{}{Trânsito de veículos em vias públicas.}{trá.fe.go}{0}
\verb{tráfego}{}{}{}{}{}{Departamento que cuida do fluxo de bens e informações dentro de uma instituição.}{trá.fe.go}{0}
\verb{traficância}{}{}{}{}{s.f.}{Ato ou efeito de traficar.}{tra.fi.cân.cia}{0}
\verb{traficante}{}{}{}{}{s.2g.}{Indivíduo que faz negócios ilícitos ou fraudulentos.}{tra.fi.can.te}{0}
\verb{traficar}{}{Desus.}{}{}{v.t.}{Negociar, comercializar.}{tra.fi.car}{0}
\verb{traficar}{}{}{}{}{}{Praticar negócio fraudulento, ilegal.}{tra.fi.car}{\verboinum{2}}
\verb{tráfico}{}{Desus.}{}{}{s.m.}{Atividade mercantil; comércio.}{trá.fi.co}{0}
\verb{tráfico}{}{}{}{}{}{Negócio ilícito.}{trá.fi.co}{0}
\verb{tragada}{}{}{}{}{s.f.}{Ato de tragar.}{tra.ga.da}{0}
\verb{tragadouro}{ô}{}{}{}{s.m.}{Aquilo que traga, que absorve; sorvedouro.}{tra.ga.dou.ro}{0}
\verb{tragar}{}{}{}{}{v.t.}{Engolir vorazmente sem mastigar; devorar.}{tra.gar}{0}
\verb{tragar}{}{}{}{}{v.i.}{Inspirar engolindo a fumaça de cigarro, charuto.}{tra.gar}{\verboinum{5}}
\verb{tragédia}{}{}{}{}{s.f.}{Texto dramático com personagens heroicos, intenção de provocar piedade e final desastroso.}{tra.gé.dia}{0}
\verb{tragédia}{}{Fig.}{}{}{}{Desgraça, catástrofe.}{tra.gé.dia}{0}
\verb{trágico}{}{}{}{}{adj.}{Relativo a tragédia.}{trá.gi.co}{0}
\verb{trágico}{}{}{}{}{}{Calamitoso, nefasto, sinistro, funesto.}{trá.gi.co}{0}
\verb{tragicomédia}{}{}{}{}{s.f.}{Texto dramatúrgico que mistura características de tragédia e de comédia, com final feliz.}{tra.gi.co.mé.dia}{0}
\verb{tragicômico}{}{}{}{}{adj.}{Que é trágico e cômico ao mesmo tempo.}{tra.gi.cô.mi.co}{0}
\verb{tragicômico}{}{}{}{}{}{Relativo a tragicomédia.}{tra.gi.cô.mi.co}{0}
\verb{trago}{}{}{}{}{s.m.}{Ato de tragar.}{tra.go}{0}
\verb{trago}{}{}{}{}{}{Aquilo que se traga; gole, sorvo, porção.}{tra.go}{0}
\verb{traição}{}{}{"-ões}{}{s.f.}{Ato ou efeito de trair.}{tra.i.ção}{0}
\verb{traiçoeiro}{ê}{}{}{}{adj.}{Em que há traição.}{trai.ço.ei.ro}{0}
\verb{traiçoeiro}{ê}{}{}{}{}{Diz"-se de indivíduo que usa de traição; desleal, pérfido.}{trai.ço.ei.ro}{0}
\verb{traidor}{ô}{}{}{}{adj.}{Que usou ou usa de traição.}{tra.i.dor}{0}
\verb{trailer}{}{}{}{}{s.m.}{Veículo sem tração própria com instalações de uma casa, que se leva geralmente rebocado por um automóvel em viagens de acampamento.}{\textit{trailer}}{0}
\verb{trailer}{}{}{}{}{}{Exibição de trechos de um filme para despertar o interesse do público.}{\textit{trailer}}{0}
\verb{traineira}{ê}{}{}{}{s.f.}{Barco de pesca motorizado que leva rede de arrastar.}{trai.nei.ra}{0}
\verb{trair}{}{}{}{}{v.t.}{Enganar, tornando"-se indigno da confiança de outrem; iludir, atraiçoar.}{tra.ir}{0}
\verb{trair}{}{}{}{}{}{Entregar pessoa ou informação ao inimigo; delatar.}{tra.ir}{0}
\verb{trair}{}{}{}{}{}{Ser infiel a cônjuge ou namorado(a).}{tra.ir}{\verboinum{19}}
\verb{traíra}{}{Bras.}{}{}{s.f.}{Peixe de água doce, cor escura e manchas pelo corpo, com dentes fortes e afiados.}{tra.í.ra}{0}
\verb{trajar}{}{}{}{}{v.t.}{Usar como vestuário; vestir.}{tra.jar}{\verboinum{1}}
\verb{traje}{}{}{}{}{s.m.}{Vestuário específico; vestes.}{tra.je}{0}
\verb{trajeto}{}{}{}{}{s.m.}{Percurso, caminho, intinerário.}{tra.je.to}{0}
\verb{trajetória}{}{Fís.}{}{}{s.f.}{Percurso descrito por um corpo em deslocamento.}{tra.je.tó.ria}{0}
\verb{trajetória}{}{}{}{}{}{Trajeto.}{tra.je.tó.ria}{0}
\verb{trajo}{}{}{}{}{s.m.}{Traje.}{tra.jo}{0}
\verb{tralha}{}{}{}{}{s.f.}{Pequena rede de pesca.}{tra.lha}{0}
\verb{tralha}{}{}{}{}{}{Bugiganga, cacareco.}{tra.lha}{0}
\verb{trama}{}{}{}{}{}{Sucessão de fatos em um história; enredo.}{tra.ma}{0}
\verb{trama}{}{}{}{}{s.f.}{Conjunto de fios paralelos à largura de um tecido.}{tra.ma}{0}
\verb{trama}{}{}{}{}{}{Intriga, conluio contra alguém ou algo; complô.}{tra.ma}{0}
\verb{tramar}{}{}{}{}{v.t.}{Passar a trama entre os fios da urdidura; tecer.}{tra.mar}{0}
\verb{tramar}{}{}{}{}{}{Fazer intriga.}{tra.mar}{0}
\verb{tramar}{}{}{}{}{}{Planejar bem.}{tra.mar}{\verboinum{1}}
\verb{trambicagem}{}{}{"-ens}{}{s.f.}{Negócio ilícito; safadeza, fraude.}{tram.bi.ca.gem}{0}
\verb{trambique}{}{Pop.}{}{}{s.m.}{Negócio fraudulento; trapaça.}{tram.bi.que}{0}
\verb{trambiqueiro}{ê}{}{}{}{adj.}{Diz"-se de indivíduo dado a trambiques; trapaceiro, vigarista.}{tram.bi.quei.ro}{0}
\verb{trambolhão}{}{}{"-ões}{}{s.m.}{Queda em que se rola e faz muito barulho.}{tram.bo.lhão}{0}
\verb{trambolho}{ô}{}{}{}{s.m.}{Peso atado aos pés de animais.}{tram.bo.lho}{0}
\verb{trambolho}{ô}{}{}{}{}{Obstáculo, empecilho.}{tram.bo.lho}{0}
\verb{tramela}{é}{}{}{}{s.f.}{Taramela.}{tra.me.la}{0}
\verb{tramitar}{}{}{}{}{v.i.}{Seguir os trâmites, o curso regular para a consecução de algo.}{tra.mi.tar}{\verboinum{1}}
\verb{trâmite}{}{}{}{}{s.m.}{Aquilo que conduz a algum ponto; caminho, via.}{trâ.mi.te}{0}
\verb{trâmites}{}{}{}{}{s.m.pl.}{Meios apropriados à consecução de um fim.}{trâ.mi.tes}{0}
\verb{trâmites}{}{}{}{}{}{Curso de um processo, segundo as regras; via.}{trâ.mi.tes}{0}
\verb{tramoia}{ó}{}{}{}{s.f.}{Ação planejada em segredo para prejudicar alguém; intriga, trapaça.}{tra.moi.a}{0}
\verb{tramontana}{}{}{}{}{s.f.}{Estrela polar.}{tra.mon.ta.na}{0}
\verb{tramontana}{}{}{}{}{}{Vento norte.}{tra.mon.ta.na}{0}
\verb{tramontana}{}{Pop.}{}{}{}{Rumo, norte.}{tra.mon.ta.na}{0}
\verb{tramontar}{}{}{}{}{v.i.}{Pôr"-se por trás dos montes, da serra.}{tra.mon.tar}{\verboinum{1}}
\verb{trampa}{}{}{}{}{s.f.}{Ação planejada em segredo para prejudicar alguém; tramoia.}{tram.pa}{0}
\verb{trampo}{}{Pop.}{}{}{s.m.}{Trabalho, serviço.}{tram.po}{0}
\verb{trampolim}{}{}{"-ins}{}{s.m.}{Prancha para impulsionar salto, mergulho.}{tram.po.lim}{0}
\verb{trampolinar}{}{}{}{}{v.i.}{Fazer velhacarias; trapacear.}{tram.po.li.nar}{\verboinum{1}}
\verb{trampolineiro}{ê}{}{}{}{adj.}{Diz"-se de indivíduo que tem o hábito de fazer trapaças, velhacarias.}{tram.po.li.nei.ro}{0}
\verb{tranca}{}{}{}{}{s.f.}{Barra de segurança para porta, janela etc.}{tran.ca}{0}
\verb{trança}{}{}{}{}{s.f.}{Entrelace de três porções de fios.}{tran.ça}{0}
\verb{trancado}{}{}{}{}{adj.}{Fechado ou seguro com tranca.}{tran.ca.do}{0}
\verb{trancado}{}{}{}{}{}{Completamente fechado.}{tran.ca.do}{0}
\verb{trançado}{}{}{}{}{adj.}{Que se trançou; diposto em trança; entrelaçado.}{tran.ça.do}{0}
\verb{trançado}{}{}{}{}{s.m.}{Obra trançada; trança.}{tran.ça.do}{0}
\verb{trancafiar}{}{}{}{}{v.t.}{Prender na cadeia; aprisionar.}{tran.ca.fi.ar}{0}
\verb{trancafiar}{}{}{}{}{}{Encarcerar em determinado ambiente para afastar do convívio social; isolar.}{tran.ca.fi.ar}{\verboinum{1}}
\verb{trancar}{}{}{}{}{v.t.}{Fechar com chave ou tranca.}{tran.car}{0}
\verb{trancar}{}{}{}{}{}{Cancelar por certo tempo.}{tran.car}{0}
\verb{trancar}{}{}{}{}{}{Enclausurar.}{tran.car}{0}
\verb{trancar}{}{}{}{}{v.pron.}{Fechar"-se, retrair"-se.}{tran.car}{\verboinum{2}}
\verb{trançar}{}{}{}{}{v.t.}{Pegar três porções de fios de alguma coisa e fazer com que cada uma delas passe sempre pelo meio das outras duas.}{tran.çar}{\verboinum{3}}
\verb{trancelim}{}{}{"-ins}{}{s.m.}{Galão ou trança fina, de seda, ouro ou prata, para guarnições e obras de costura.}{tran.ce.lim}{0}
\verb{trancelim}{}{}{"-ins}{}{}{Cordão delgado, de ouro.}{tran.ce.lim}{0}
\verb{tranco}{}{}{}{}{s.m.}{Salto que dá o cavalo.}{tran.co}{0}
\verb{tranco}{}{}{}{}{}{Solavanco.}{tran.co}{0}
\verb{tranqueira}{ê}{}{}{}{s.f.}{O que impede passagem.}{tran.quei.ra}{0}
\verb{tranqueira}{ê}{}{}{}{}{Conjunto de coisas sem utilidade; cacarecos. }{tran.quei.ra}{0}
\verb{tranqueta}{ê}{}{}{}{s.f.}{Pequena tranca.  }{tran.que.ta}{0}
\verb{tranqueta}{ê}{}{}{}{}{Peça de ferro que se coloca verticalmente por detrás das portas ou das janelas para fechá"-las.}{tran.que.ta}{0}
\verb{tranquilidade}{}{}{}{}{s.f.}{Sentimento de calma; serenidade.}{tran.qui.li.da.de}{0}
\verb{tranquilizador}{ô}{}{}{}{adj.}{Que tranquiliza, que acalma.}{tran.qui.li.za.dor}{0}
\verb{tranquilizante}{}{}{}{}{adj.2g.}{Que tranquiliza; tranquilizador.}{tran.qui.li.zan.te}{0}
\verb{tranquilizante}{}{}{}{}{}{Sedativo.}{tran.qui.li.zan.te}{0}
\verb{tranquilizar}{}{}{}{}{v.t.}{Tornar tranquilo; acalmar.}{tran.qui.li.zar}{\verboinum{1}}
\verb{tranquilo}{}{}{}{}{adj.}{Que é calmo, sossegado.}{tran.qui.lo}{0}
\verb{tranquilo}{}{}{}{}{}{Que não tem agitação, inquietação.}{tran.qui.lo}{0}
\verb{transa}{}{Pop.}{}{}{s.f.}{Ajuste para a consecução de determinado fim; acordo, combinação.}{tran.sa}{0}
\verb{transa}{}{Pop.}{}{}{}{Relação sexual.}{tran.sa}{0}
\verb{transação}{}{}{"-ões}{}{s.f.}{Operação comercial.}{tran.sa.ção}{0}
\verb{transação}{}{}{"-ões}{}{}{Combinação.}{tran.sa.ção}{0}
\verb{transação}{}{Pop.}{"-ões}{}{}{Transa.}{tran.sa.ção}{0}
\verb{transacionar}{}{}{}{}{v.i.}{Fazer negócio; comerciar.}{tran.sa.ci.o.nar}{\verboinum{1}}
\verb{transacto}{}{}{}{}{}{Var. de \textit{transato}.}{tran.sac.to}{0}
\verb{transamazônico}{}{}{}{}{adj.}{Que atravessa a Amazônia.}{tran.sa.ma.zô.ni.co}{0}
\verb{transar}{}{}{}{}{v.t.}{Fazer algum acordo ou negócio com alguém; negociar.}{tran.sar}{0}
\verb{transar}{}{Pop.}{}{}{}{Ter relação sexual.}{tran.sar}{\verboinum{1}}
\verb{transatlântico}{}{}{}{}{adj.}{Que fica além do Atlântico.}{tran.sa.tlân.ti.co}{0}
\verb{transatlântico}{}{}{}{}{}{Diz"-se de navio que atravessa o Atlântico.}{tran.sa.tlân.ti.co}{0}
\verb{transato}{}{}{}{}{adj.}{Que já passou; anterior.}{tran.sa.to}{0}
\verb{transbordamento}{}{}{}{}{s.m.}{Extrapolação dos limites das bordas.}{trans.bor.da.men.to}{0}
\verb{transbordamento}{}{}{}{}{}{Manifestação intensa de emoção.}{trans.bor.da.men.to}{0}
\verb{transbordamento}{}{}{}{}{}{Excesso, exagero.}{trans.bor.da.men.to}{0}
\verb{transbordamento}{}{}{}{}{}{Espalhamento, propagação.}{trans.bor.da.men.to}{0}
\verb{transbordante}{}{}{}{}{adj.2g.}{Que transborda.}{trans.bor.dan.te}{0}
\verb{transbordar}{}{}{}{}{v.t.}{Fazer sair das bordas.}{trans.bor.dar}{0}
\verb{transbordar}{}{}{}{}{}{Derramar, expandir.}{trans.bor.dar}{\verboinum{1}}
\verb{transbordo}{ô}{}{}{}{s.m.}{Ato ou efeito de transbordar.}{trans.bor.do}{0}
\verb{transbordo}{ô}{}{}{}{}{Ato ou efeito de passar mercadorias ou passageiros de uma linha para outra do mesmo serviço de transporte; baldeação.}{trans.bor.do}{0}
\verb{transcendência}{}{}{}{}{s.f.}{Qualidade do que é transcendente; excelência, superioridade.}{trans.cen.dên.cia}{0}
\verb{transcendental}{}{}{"-ais}{}{adj.2g.}{Que transcende; muito elevado; superior.   }{trans.cen.den.tal}{0}
\verb{transcendental}{}{Filos.}{"-ais}{}{}{Na filosofia kantiana, diz"-se de todo conhecimento que precede a experiência, isto é, que é constituído \textit{a priori}. }{trans.cen.den.tal}{0}
\verb{transcendentalismo}{}{Filos.}{}{}{s.m.}{Doutrina baseada na intuição e caracterizada por certo misticismo.}{trans.cen.den.ta.lis.mo}{0}
\verb{transcendente}{}{}{}{}{adj.2g.}{Que é sublime, superior.}{trans.cen.den.te}{0}
\verb{transcendente}{}{}{}{}{}{Que está além do conhecimento e da experiência.}{trans.cen.den.te}{0}
\verb{transcender}{ê}{}{}{}{v.t.}{Ir além ou acima de; ultrapassar.}{trans.cen.der}{\verboinum{12}}
\verb{transcontinental}{}{}{}{}{adj.2g.}{Que atravessa um continente.}{trans.con.ti.nen.tal}{0}
\verb{transcorrer}{ê}{}{}{}{v.i.}{Desenrolar"-se como um todo; decorrer.}{trans.cor.rer}{0}
\verb{transcorrer}{ê}{}{}{}{v.t.}{Passar além de; transpor.}{trans.cor.rer}{\verboinum{12}}
\verb{transcrever}{ê}{}{}{}{v.t.}{Escrever novamente um determinado conteúdo em outro lugar; copiar.}{trans.cre.ver}{0}
\verb{transcrever}{ê}{}{}{}{}{Passar para o papel ou equivalente algo que está sendo ouvido.}{trans.cre.ver}{\verboinum{12}}
\verb{transcrição}{}{}{"-ões}{}{s.f.}{Cópia textual.}{trans.cri.ção}{0}
\verb{transcrição}{}{}{"-ões}{}{}{Escrita fonética de uma língua.}{trans.cri.ção}{0}
\verb{transcrito}{}{}{}{}{adj.}{Que se transcreveu.}{trans.cri.to}{0}
\verb{transcrito}{}{}{}{}{s.m.}{Cópia, traslado.}{trans.cri.to}{0}
\verb{transcurso}{}{}{}{}{s.m.}{Ato ou efeito de transcorrer; decurso, decorrência.}{trans.cur.so}{0}
\verb{transe}{}{}{}{}{s.m.}{Momento crítico, de grande aflição, dificuldade ou perigo.}{tran.se}{0}
\verb{transe}{}{}{}{}{}{Estado de êxtase.}{tran.se}{0}
\verb{transeunte}{}{}{}{}{adj.2g.}{Que passa ou vai passando, que não é permanente.}{tran.se.un.te}{0}
\verb{transeunte}{}{}{}{}{s.2g.}{Pessoa que passa ou vai passando; andante, pedestre.}{tran.se.un.te}{0}
\verb{transexual}{cs}{}{}{}{s.2g.}{Indivíduo que deseja que o seu gênero biológico corresponda à sua identidade de gênero mudando assim o seu corpo através de trajes, hormônios e até de cirurgias. }{tran.se.xu.al}{0}
%\verb{}{}{}{}{}{}{}{}{0}
\verb{transferência}{}{}{}{}{s.f.}{Ato ou efeito de transferir.}{trans.fe.rên.cia}{0}
\verb{transferidor}{ô}{}{}{}{adj.}{Que transfere.}{trans.fe.ri.dor}{0}
\verb{transferidor}{ô}{}{}{}{s.m.}{Instrumento circular ou semicircular, empregado na medição de ângulos.}{trans.fe.ri.dor}{0}
\verb{transferir}{}{}{}{}{v.t.}{Fazer passar de um lugar a outro; deslocar.}{trans.fe.rir}{0}
\verb{transferir}{}{}{}{}{}{Adiar, retardar, procrastinar, delongar.}{trans.fe.rir}{0}
\verb{transferir}{}{}{}{}{}{Enviar um funcionário, empregado etc. para um outro lugar, para um outro posto de trabalho.}{trans.fe.rir}{0}
\verb{transferir}{}{}{}{}{v.pron.}{Mudar"-se de um lugar para outro.}{trans.fe.rir}{\verboinum{29}}
\verb{transfiguração}{}{}{"-ões}{}{s.f.}{Ato ou efeito de transfigurar.}{trans.fi.gu.ra.ção}{0}
\verb{transfigurar}{}{}{}{}{v.t.}{Mudar a forma, feição ou caráter; transformar, modificar.}{trans.fi.gu.rar}{\verboinum{1}}
%\verb{}{}{}{}{}{}{}{}{0}
%\verb{}{}{}{}{}{}{}{}{0}
\verb{transfixar}{cs}{}{}{}{v.t.}{Atravessar de lado a lado. (\textit{Ele morreu após falência múltipla dos órgãos, segundo o hospital, em decorrência do projétil que} transfixou \textit{seu crânio. })}{trans.fi.xar}{\verboinum{1}}
\verb{transformador}{ô}{}{}{}{adj.}{Que transforma; transformante. }{trans.for.ma.dor}{0}
\verb{transformador}{ô}{}{}{}{s.m.}{Aparelho usado para transformar a tensão ou a intensidade da corrente elétrica.}{trans.for.ma.dor}{0}
\verb{transformar}{}{}{}{}{v.t.}{Fazer passar de uma forma para outra; converter. (\textit{A medida econômica do governo transformou a vida de todas as pessoas.})}{trans.for.mar}{0}
\verb{transformar}{}{}{}{}{v.pron.}{Mudar a própria forma; converter"-se. (\textit{Ele se transformou num excelente jogador.})}{trans.for.mar}{\verboinum{1}}
\verb{transformismo}{}{}{}{}{s.m.}{Doutrina segundo a qual os seres vivos derivam uns dos outros por filiação; evolucionismo.}{trans.for.mis.mo}{0}
\verb{transformismo}{}{}{}{}{}{A atividade do transformista.}{trans.for.mis.mo}{0}
\verb{transformista}{}{}{}{}{adj.2g.}{Relativo ao transformismo.}{trans.for.mis.ta}{0}
\verb{transformista}{}{}{}{}{s.2g.}{Pessoa adepta da doutrina do transformismo.}{trans.for.mis.ta}{0}
\verb{transformista}{}{}{}{}{}{Artista que se disfarça, trocando de roupa e se maquiando.}{trans.for.mis.ta}{0}
\verb{transformista}{}{Por ext.}{}{}{}{Travesti.}{trans.for.mis.ta}{0}
\verb{trânsfuga}{}{}{}{}{s.2g.}{Em tempo de guerra, pessoa que deserta para o exército inimigo; desertor, traidor.}{trâns.fu.ga}{0}
\verb{trânsfuga}{}{}{}{}{}{Indivíduo que deixa um partido político por outro, ou que muda de religião.}{trâns.fu.ga}{0}
\verb{transfusão}{}{}{}{}{s.f.}{Ato ou efeito de transfundir.}{trans.fu.são}{0}
\verb{transfusão}{}{Med.}{}{}{}{Introdução na corrente sanguínea de um paciente, por via direta, de plasma,  sangue, soro etc.}{trans.fu.são}{0}
\verb{transgênico}{}{Bioquím.}{}{}{adj.}{Que tem em seu genoma um ou mais genes transferidos artificialmente de outra espécie.}{trans.gê.ni.co}{0}
\verb{transgênico}{}{}{}{}{s.m.}{Esse organismo.}{trans.gê.ni.co}{0}
\verb{transgredir}{}{}{}{}{v.t.}{Desobedecer a leis, normas, regras etc; infringir, violar.}{trans.gre.dir}{\verboinum{30}}
\verb{transgressão}{}{}{"-ões}{}{s.f.}{Ato ou efeito de transgredir; violação, infração.}{trans.gres.são}{0}
\verb{transgressor}{ô}{}{}{}{adj.}{Que transgride; infrator.}{trans.gres.sor}{0}
\verb{transiberiano}{}{}{}{}{adj.}{Que se encontra além da Sibéria.}{tran.si.be.ri.a.no}{0}
\verb{transiberiano}{}{}{}{}{}{Que atravessa a Sibéria.}{tran.si.be.ri.a.no}{0}
\verb{transição}{}{}{"-ões}{}{s.f.}{Passagem de um ponto ou de um estado para outro; mudança.}{tran.si.ção}{0}
\verb{transigência}{}{}{}{}{s.f.}{Ato ou efeito de transigir; tolerância, contemporização.}{tran.si.gên.cia}{0}
\verb{transigente}{}{}{}{}{adj.2g.}{Que transige, cede; tolerante.}{tran.si.gen.te}{0}
\verb{transigir}{}{}{}{}{v.t.}{Chegar a um acordo; ceder, tolerar, contemporizar.}{tran.si.gir}{\verboinum{22}}
\verb{transístor}{z}{}{}{}{s.m.}{Dispositivo com semicondutores que controlam o fluxo de eletricidade em um equipamento eletrônico.}{tran.sís.tor}{0}
\verb{transistor}{z\ldots{}ô}{}{}{}{s.m.}{Transístor.}{tran.sis.tor}{0}
\verb{transistorizar}{}{}{}{}{v.t.}{Construir um circuito eletrônico com transistores em vez de válvulas.}{tran.sis.to.ri.zar}{\verboinum{1}}
\verb{transitar}{}{}{}{}{v.i.}{Passar ou andar de um lugar a outro; percorrer.}{tran.si.tar}{\verboinum{1}}
\verb{transitável}{}{}{"-eis}{}{adj.2g.}{Por onde se pode passar, transitar.}{tran.si.tá.vel}{0}
\verb{transitivar}{}{Gram.}{}{}{v.t.}{Tornar um verbo transitivo.}{tran.si.ti.var}{\verboinum{1}}
\verb{transitivo}{}{}{}{}{adj.}{Que dura pouco; passageiro, transitório.}{tran.si.ti.vo}{0}
\verb{transitivo}{}{Gram.}{}{}{}{Diz"-se do verbo que requer um complemento.}{tran.si.ti.vo}{0}
\verb{trânsito}{}{}{}{}{s.m.}{Ato ou efeito de transitar.}{trân.si.to}{0}
\verb{trânsito}{}{}{}{}{}{Movimento de veículos e pedestres; tráfego.}{trân.si.to}{0}
\verb{transitório}{}{}{}{}{adj.}{Que dura pouco; passageiro, efêmero, transitivo.}{tran.si.tó.rio}{0}
\verb{translação}{}{}{"-ões}{}{s.f.}{Ato ou efeito de transladar; mudança, transferência, transladação.}{trans.la.ção}{0}
\verb{translação}{}{Astron.}{"-ões}{}{}{Movimento dos astros em suas órbitas.}{trans.la.ção}{0}
\verb{transladação}{}{}{"-ões}{}{s.f.}{Ato ou efeito de transladar; mudança, translação.}{trans.la.da.ção}{0}
\verb{transladar}{}{}{}{}{v.t.}{Mudar de um lugar a outro; transferir.}{trans.la.dar}{0}
\verb{transladar}{}{}{}{}{}{Traduzir, verter.}{trans.la.dar}{\verboinum{1}}
\verb{translado}{}{}{}{}{s.m.}{Passagem de um lugar a outro; transporte.}{trans.la.do}{0}
\verb{transliterar}{}{}{}{}{v.t.}{Passar de um sistema de escrita a outro, letra por letra.}{trans.li.te.rar}{\verboinum{1}}
\verb{translúcido}{}{}{}{}{adj.}{Que deixa passar a luz, sem contudo deixar perceber nitidamente os objetos através de sua espessura; diáfano.}{trans.lú.ci.do}{0}
\verb{transluzir}{}{}{}{}{v.t.}{Brilhar através de um corpo; transparecer.}{trans.lu.zir}{\verboinum{21}}
\verb{transmigrar}{}{}{}{}{v.i.}{Passar de uma região a outra.}{trans.mi.grar}{0}
\verb{transmigrar}{}{}{}{}{}{Passar a alma de um corpo para outro, segundo certas crenças religiosas.}{trans.mi.grar}{\verboinum{1}}
\verb{transmissão}{}{}{"-ões}{}{s.f.}{Ato ou efeito de transmitir; transferência, comunicação.}{trans.mis.são}{0}
\verb{transmissível}{}{}{"-eis}{}{adj.2g.}{Que se pode transmitir.}{trans.mis.sí.vel}{0}
\verb{transmissivo}{}{}{}{}{adj.}{Que transmite ou é próprio para transmitir; transmissor.}{trans.mis.si.vo}{0}
\verb{transmissor}{ô}{}{}{}{adj.}{Que transmite, envia, propaga.}{trans.mis.sor}{0}
\verb{transmitir}{}{}{}{}{v.t.}{Fazer informação ou dado chegar a outrem; enviar, expedir, comunicar.}{trans.mi.tir}{0}
\verb{transmitir}{}{}{}{}{}{Propagar, contagiar, difundir.}{trans.mi.tir}{\verboinum{18}}
\verb{transmudar}{}{}{}{}{v.t.}{Transmutar.}{trans.mu.dar}{\verboinum{1}}
\verb{transmutação}{}{}{"-ões}{}{s.f.}{Ato ou efeito de transmutar; transformação, conversão.}{trans.mu.ta.ção}{0}
\verb{transmutar}{}{}{}{}{v.t.}{Passar de um estado ou condição a outro; transformar, converter.}{trans.mu.tar}{\verboinum{1}}
\verb{transoceânico}{}{}{}{}{adj.}{Que atravessa ou está além do oceano.}{tran.so.ce.â.ni.co}{0}
\verb{transparecer}{ê}{}{}{}{v.t.}{Aparecer total ou parcialmente através de; transluzir.}{trans.pa.re.cer}{0}
\verb{transparecer}{ê}{Fig.}{}{}{}{Fazer"-se conhecido; revelar"-se, manifestar"-se.}{trans.pa.re.cer}{\verboinum{15}}
\verb{transparência}{}{}{}{}{s.f.}{Propriedade de certos materiais de serem atravessados pelos raios luminosos.}{trans.pa.rên.cia}{0}
\verb{transparência}{}{Fig.}{}{}{}{Pureza.}{trans.pa.rên.cia}{0}
\verb{transparência}{}{}{}{}{}{Folha transparente em que se escreve ou desenha o que vai ser projetado numa tela.}{trans.pa.rên.cia}{0}
\verb{transparente}{}{}{}{}{adj.2g.}{Que se deixa atravessar pela luz.}{trans.pa.ren.te}{0}
\verb{transparente}{}{Fig.}{}{}{}{Evidente, indubitável.}{trans.pa.ren.te}{0}
\verb{transpassar}{}{}{}{}{v.t.}{Furar de parte a parte; perfurar, penetrar.}{trans.pas.sar}{0}
\verb{transpassar}{}{}{}{}{}{Ir de um extremo a outro de; atravessar, transpor.}{trans.pas.sar}{0}
\verb{transpassar}{}{}{}{}{}{Fechar saia, cinto etc., sobrepondo duas partes.}{trans.pas.sar}{\verboinum{1}}
\verb{transpiração}{}{}{"-ões}{}{s.f.}{Ato ou efeito de transpirar.}{trans.pi.ra.ção}{0}
\verb{transpiração}{}{}{"-ões}{}{}{Eliminação do suor pelas glândulas sudoríparas da pele.}{trans.pi.ra.ção}{0}
\verb{transpiração}{}{}{"-ões}{}{}{Fluido segregado pelas glândulas sudoríparas; suor.}{trans.pi.ra.ção}{0}
\verb{transpirar}{}{}{}{}{v.t.}{Secretar suor pelos poros do corpo; suar.}{trans.pi.rar}{0}
\verb{transpirar}{}{Fig.}{}{}{}{Manifestar por atitude ou gesto; exprimir.}{trans.pi.rar}{\verboinum{1}}
\verb{transplantação}{}{}{"-ões}{}{s.f.}{Ato ou efeito de transplantar; transplante.}{trans.plan.ta.ção}{0}
\verb{transplantação}{}{}{"-ões}{}{}{Ato ou efeito de trasnplantar em outro lugar.}{trans.plan.ta.ção}{0}
\verb{transplantar}{}{}{}{}{v.t.}{Remover planta de um lugar a outro.}{trans.plan.tar}{0}
\verb{transplantar}{}{}{}{}{}{Implantar tecido, ógão etc. em outra parte ou outro corpo.}{trans.plan.tar}{0}
\verb{transplantar}{}{}{}{}{}{Fazer passar de um país a outro.}{trans.plan.tar}{\verboinum{1}}
\verb{transplante}{}{}{}{}{s.m.}{Ato ou efeito de transplantar.}{trans.plan.te}{0}
\verb{transplante}{}{}{}{}{}{Transferência de tecido, órgão etc. para outra parte ou outro corpo.}{trans.plan.te}{0}
\verb{transpor}{ô}{}{}{}{v.t.}{Pôr algo em outro lugar.}{trans.por}{0}
\verb{transpor}{ô}{}{}{}{}{Passar para outro tom.}{trans.por}{0}
\verb{transpor}{ô}{}{}{}{}{Ultrapassar.}{trans.por}{0}
\verb{transpor}{ô}{}{}{}{v.pron.}{Ocultar"-se.}{trans.por}{\verboinum{60}}
\verb{transportadora}{ô}{}{}{}{s.f.}{Empresa que faz transporte de cargas.}{trans.por.ta.do.ra}{0}
\verb{transportar}{}{}{}{}{v.t.}{Levar ou conduzir seres animados ou coisas a determinado lugar; carregar.}{trans.por.tar}{0}
\verb{transportar}{}{Mús.}{}{}{}{Transpor melodia.}{trans.por.tar}{\verboinum{1}}
\verb{transporte}{ó}{}{}{}{s.m.}{Ato ou efeito de transportar.}{trans.por.te}{0}
\verb{transporte}{ó}{}{}{}{}{Veículo que transporta.}{trans.por.te}{0}
\verb{transposição}{}{}{"-ões}{}{s.f.}{Ato ou efeito de transpor.}{trans.po.si.ção}{0}
\verb{transposto}{ô}{}{"-s ⟨ó⟩}{"-a ⟨ó⟩}{adj.}{Que sofreu transposição.}{trans.pos.to}{0}
\verb{transtornar}{}{}{}{}{v.t.}{Deixar em desordem; desorganizar.}{trans.tor.nar}{0}
\verb{transtornar}{}{}{}{}{}{Provocar uma emoção forte; perturbar.}{trans.tor.nar}{0}
\verb{transtorno}{ô}{}{}{}{s.m.}{Situação incômoda.}{trans.tor.no}{0}
\verb{transtorno}{ô}{}{}{}{}{Desordem, desorganização.}{trans.tor.no}{0}
\verb{transtorno}{ô}{}{}{}{}{Contrariedade, decepção.}{trans.tor.no}{0}
\verb{transubstanciação}{}{}{"-ões}{}{s.f.}{Transformação de uma substância em outra.}{tran.subs.tan.ci.a.ção}{0}
\verb{transubstanciação}{}{Relig.}{"-ões}{}{}{No catolicismo, conversão de pão e vinho no corpo e sangue de Cristo.}{tran.subs.tan.ci.a.ção}{0}
\verb{transubstanciar}{}{}{}{}{v.t.}{Converter uma substância em outra.}{tran.subs.tan.ci.ar}{0}
\verb{transubstanciar}{}{Relig.}{}{}{}{Realizar a transubstanciação do pão e do vinho.}{tran.subs.tan.ci.ar}{\verboinum{6}}
\verb{transudar}{}{}{}{}{v.t.}{Passar através dos poros; transpirar.}{tran.su.dar}{0}
\verb{transudar}{}{}{}{}{}{Verter, transvasar.}{tran.su.dar}{\verboinum{1}}
\verb{transurânico}{}{}{}{}{adj.}{Que se localiza além do planeta Urano.}{tran.su.râ.ni.co}{0}
\verb{transurânico}{}{Quím.}{}{}{adj.}{Diz"-se de qualquer elemento cujo número atômico é superior a 92, número atômico do urânio.}{tran.su.râ.ni.co}{0}
\verb{transurânico}{}{Quím.}{}{}{}{Esse elemento químico.}{tran.su.râ.ni.co}{0}
\verb{transvasar}{}{}{}{}{v.t.}{Transferir um líquido de recipiente.}{trans.va.sar}{\verboinum{1}}
\verb{transvazar}{}{}{}{}{v.t.}{Fazer transbordar um líquido para fora do recipiente; derramar, entornar.}{trans.va.zar}{\verboinum{1}}
\verb{transverberar}{}{}{}{}{v.t.}{Deixar passar luz.}{trans.ver.be.rar}{0}
\verb{transverberar}{}{}{}{}{}{Transparecer.}{trans.ver.be.rar}{\verboinum{1}}
\verb{transversal}{}{}{"-ais}{}{adj.2g.}{Diz"-se de linha ou qualquer coisa que atravessa algo em ângulo reto ou oblíquo.}{trans.ver.sal}{0}
\verb{transverso}{é}{}{}{}{adj.}{Situado no sentido oblíquo; atravessado.}{trans.ver.so}{0}
\verb{transviar}{}{}{}{}{v.t.}{Desviar do dever; corromper, seduzir.}{trans.vi.ar}{\verboinum{6}}
\verb{trapaça}{}{}{}{}{s.f.}{Artimanha para enganar, prejudicar.}{tra.pa.ça}{0}
\verb{trapaça}{}{}{}{}{}{Trambique.}{tra.pa.ça}{0}
\verb{trapacear}{}{}{}{}{v.t.}{Tratar negócio, contrato de má"-fé.}{tra.pa.ce.ar}{0}
\verb{trapacear}{}{}{}{}{}{Fazer trapaça.}{tra.pa.ce.ar}{\verboinum{4}}
\verb{trapaceiro}{ê}{}{}{}{adj.}{Diz"-se de quem faz trapaça.}{tra.pa.cei.ro}{0}
\verb{trapalhada}{}{}{}{}{s.f.}{Monte de trapos.}{tra.pa.lha.da}{0}
\verb{trapalhada}{}{}{}{}{s.f.}{Grande desordem; confusão.}{tra.pa.lha.da}{0}
\verb{trapalhão}{}{}{"-ões}{"-ona}{adj.}{Que atrapalha muito, que causa confusão.}{tra.pa.lhão}{0}
\verb{trapalhão}{}{}{"-ões}{"-ona}{}{Que faz trapaças.}{tra.pa.lhão}{0}
\verb{trapeiro}{ê}{}{}{}{adj.}{Diz"-se de catador de lixo.}{tra.pei.ro}{0}
\verb{trapézio}{}{}{}{}{s.m.}{Quadrilátero de dois lados paralelos.}{tra.pé.zio}{0}
\verb{trapézio}{}{}{}{}{}{Balanço para acrobacia usado especialmente em circo.}{tra.pé.zio}{0}
\verb{trapézio}{}{}{}{}{}{Músculo posterior do pescoço.}{tra.pé.zio}{0}
\verb{trapézio}{}{}{}{}{}{Um dos ossos do corpo.}{tra.pé.zio}{0}
\verb{trapezista}{}{}{}{}{s.2g.}{Ginasta de trapézio.}{tra.pe.zis.ta}{0}
\verb{trapezoide}{ó}{}{}{}{adj.2g.}{Em forma de trapézio.}{tra.pe.zoi.de}{0}
\verb{trapiche}{}{}{}{}{s.m.}{Depósito perto de cais.}{tra.pi.che}{0}
\verb{trapiche}{}{}{}{}{}{Pequeno engenho de açúcar.}{tra.pi.che}{0}
\verb{trapicheiro}{ê}{}{}{}{adj.}{Que é proprietário ou administrador de trapiche.}{tra.pi.chei.ro}{0}
\verb{trapicheiro}{ê}{}{}{}{}{Que trabalha em trapiche.}{tra.pi.chei.ro}{0}
\verb{trapista}{}{}{}{}{adj.2g.}{Diz"-se de religioso de certa ordem beneditina.}{tra.pis.ta}{0}
\verb{trapo}{}{}{}{}{s.m.}{Pedaço de pano velho.}{tra.po}{0}
\verb{trapo}{}{}{}{}{}{Roupa muito gasta, surrada.}{tra.po}{0}
\verb{traque}{}{}{}{}{s.m.}{Explosivo recreativo constituído de um tubo pequeno, carregado de pólvora.}{tra.que}{0}
\verb{traque}{}{Pop.}{}{}{}{Flatulência.}{tra.que}{0}
\verb{traqueia}{é}{Anat.}{}{}{s.f.}{Canal que comunica a laringe com os brônquios.}{tra.quei.a}{0}
\verb{traquejado}{}{}{}{}{adj.}{Que é perito em qualquer atividade; experiente.}{tra.que.ja.do}{0}
\verb{traquejar}{}{}{}{}{v.t.}{Tornar apto; exercitar.}{tra.que.jar}{\verboinum{1}}
\verb{traquejo}{ê}{}{}{}{s.m.}{Muita prática ou experiência em qualquer atividade.}{tra.que.jo}{0}
\verb{traqueostomia}{}{Med.}{}{}{s.f.}{Abertura cirúrgica da traqueia para colocação de uma cânula.}{tra.que.os.to.mi.a}{0}
\verb{traqueotomia}{}{}{}{}{}{Var. de \textit{traqueostomia}.}{tra.que.o.to.mi.a}{0}
\verb{traquina}{}{}{}{}{}{Var. de \textit{traquinas}.}{tra.qui.na}{0}
\verb{traquinagem}{}{}{"-ens}{}{s.f.}{Ato de traquinas; travessura.}{tra.qui.na.gem}{0}
\verb{traquinar}{}{}{}{}{v.i.}{Fazer travessuras, traquinagens.}{tra.qui.nar}{\verboinum{1}}
\verb{traquinas}{}{}{}{}{adj.2g.}{Que é travesso.}{tra.qui.nas}{0}
\verb{traquinice}{}{}{}{}{s.f.}{Traquinagem.}{tra.qui.ni.ce}{0}
\verb{traquitana}{}{}{}{}{s.f.}{Carruagem para duas pessoas.}{tra.qui.ta.na}{0}
\verb{traquitana}{}{Pop.}{}{}{}{Carro velho.}{tra.qui.ta.na}{0}
\verb{trás}{}{}{}{}{prep.}{}{trás}{0}
\verb{trás}{}{}{}{}{interj.}{Expressão que denota pancada ruidosa.}{trás}{0}
\verb{trás}{}{Desus.}{}{}{adv.}{Após, atrás, em seguida a.}{trás}{0}
\verb{trasanteontem}{}{}{}{}{adv.}{No dia anterior a anteontem.}{tra.san.te.on.tem}{0}
\verb{trasantontem}{}{}{}{}{}{Var. de \textit{trasanteontem}.}{tra.san.ton.tem}{0}
\verb{trasbordar}{}{}{}{}{}{Var. de \textit{transbordar}.}{tras.bor.dar}{0}
\verb{traseira}{ê}{}{}{}{s.f.}{Parte de trás.}{tra.sei.ra}{0}
\verb{traseiro}{ê}{}{}{}{adj.}{Que fica atrás.}{tra.sei.ro}{0}
\verb{traseiro}{ê}{Pop.}{}{}{s.m.}{O conjunto das nádegas.}{tra.sei.ro}{0}
\verb{trasfegar}{}{}{}{}{v.t.}{Transferir de um recipiente para outro.}{tras.fe.gar}{\verboinum{5}}
\verb{trasladação}{}{}{"-ões}{}{s.f.}{Ato ou efeito de trasladar.}{tras.la.da.ção}{0}
\verb{trasladar}{}{}{}{}{v.t.}{Mudar de lugar; transferir.}{tras.la.dar}{0}
\verb{trasladar}{}{}{}{}{}{Mudar de data; adiar.}{tras.la.dar}{0}
\verb{trasladar}{}{}{}{}{}{Traduzir, verter.}{tras.la.dar}{0}
\verb{trasladar}{}{}{}{}{}{Transcrever, copiar.}{tras.la.dar}{0}
\verb{trasladar}{}{}{}{}{v.pron.}{Mudar"-se.}{tras.la.dar}{\verboinum{1}}
\verb{traslado}{}{}{}{}{s.m.}{Ato ou efeito de trasladar.}{tras.la.do}{0}
\verb{trasorelho}{ê}{Med.}{}{}{s.m.}{Parotidite, caxumba.}{tra.so.re.lho}{0}
\verb{traspassar}{}{}{}{}{v.t.}{Transpassar.}{tras.pas.sar}{\verboinum{1}}
\verb{traspasse}{}{}{}{}{s.m.}{Ato ou efeito de traspassar.}{tras.pas.se}{0}
\verb{traspasse}{}{}{}{}{}{Sublocação.}{tras.pas.se}{0}
\verb{traspasse}{}{}{}{}{}{Morte, falecimento.}{tras.pas.se}{0}
\verb{traste}{}{}{}{}{s.m.}{Objeto sem utilidade ou valor.}{tras.te}{0}
\verb{traste}{}{}{}{}{}{Indivíduo sem caráter.}{tras.te}{0}
\verb{traste}{}{}{}{}{}{Indivíduo inútil, desocupado.}{tras.te}{0}
\verb{trastejar}{}{}{}{}{v.i.}{Ganhar a vida negociando objetos pouco valiosos.}{tras.te.jar}{\verboinum{1}}
\verb{trasvasar}{}{}{}{}{v.t.}{Transvasar.}{tras.va.sar}{\verboinum{1}}
\verb{tratadista}{}{}{}{}{adj.2g.}{Autor de tratado científico.}{tra.ta.dis.ta}{0}
\verb{tratado}{}{}{}{}{adj.}{Que se tratou.}{tra.ta.do}{0}
\verb{tratado}{}{}{}{}{s.m.}{Estudo científico aprofundado sobre determinado assunto.}{tra.ta.do}{0}
\verb{tratado}{}{}{}{}{}{Acordo, convênio, pacto.}{tra.ta.do}{0}
\verb{tratador}{ô}{}{}{}{adj.}{Que cuida de animais, especialmente cavalos.}{tra.ta.dor}{0}
\verb{tratamento}{}{}{}{}{s.m.}{Ato ou efeito de tratar; trato.}{tra.ta.men.to}{0}
\verb{tratamento}{}{}{}{}{}{Processo de recuperação da saúde.}{tra.ta.men.to}{0}
\verb{tratamento}{}{Gram.}{}{}{}{Palavra ou expressão que se usa para se dirigir ou referir a alguém.}{tra.ta.men.to}{0}
\verb{tratante}{}{}{}{}{adj.2g.}{Trapaceiro, velhaco, ardiloso.}{tra.tan.te}{0}
\verb{tratar}{}{}{}{}{v.t.}{Fazer os serviços necessários; cuidar.}{tra.tar}{0}
\verb{tratar}{}{}{}{}{}{Ter como assunto; versar, discutir, debater.}{tra.tar}{0}
\verb{tratar}{}{}{}{}{}{Dar a alguém determinado tratamento ou denominação.}{tra.tar}{0}
\verb{tratar}{}{}{}{}{}{Ter determinado comportamento em relação a.}{tra.tar}{0}
\verb{tratar}{}{}{}{}{}{Entrar em acordo com; ajustar, pactuar, acordar.}{tra.tar}{0}
\verb{tratar}{}{}{}{}{}{Ter conversa profissional com; negociar, comerciar.}{tra.tar}{\verboinum{1}}
\verb{tratativa}{}{}{}{}{s.f.}{Ajuste, pacto, tratado.}{tra.ta.ti.va}{0}
\verb{tratável}{}{}{"-eis}{}{adj.2g.}{Diz"-se de pessoa com quem é fácil relacionar"-se socialmente; afável, sociável.}{tra.tá.vel}{0}
\verb{trato}{}{}{}{}{s.m.}{Ato ou efeito de tratar; tratamento.}{tra.to}{0}
\verb{trato}{}{}{}{}{}{Conversação, negociação.}{tra.to}{0}
\verb{trato}{}{}{}{}{}{Acordo, pacto, contrato, ajuste.}{tra.to}{0}
\verb{trato}{}{}{}{}{}{Alimentação habitual; passadio.}{tra.to}{0}
\verb{trato}{}{}{}{}{}{Maneira de se comportar em relação a alguém; procedimento.}{tra.to}{0}
\verb{trato}{}{}{}{}{}{Pedaço de terra; terreno, região.}{tra.to}{0}
\verb{trator}{ô}{}{}{}{s.m.}{Máquina de tração, própria para rebocar veículos pesados e máquinas agrícolas de grande porte.}{tra.tor}{0}
\verb{tratorista}{}{}{}{}{s.2g.}{Indivíduo que opera tratores.}{tra.to.ris.ta}{0}
\verb{trauma}{}{}{}{}{s.m.}{Traumatismo.}{trau.ma}{0}
\verb{traumático}{}{}{}{}{adj.}{Relativo a trauma.}{trau.má.ti.co}{0}
\verb{traumatismo}{}{Med.}{}{}{s.m.}{Lesão em um tecido provocada por agentes físicos ou químicos.}{trau.ma.tis.mo}{0}
\verb{traumatismo}{}{Psicol.}{}{}{}{Abalo mental forte, que causa perturbações psíquicas ou somáticas.}{trau.ma.tis.mo}{0}
\verb{traumatismo}{}{Fig.}{}{}{}{Sofrimento moral.}{trau.ma.tis.mo}{0}
\verb{traumatizante}{}{}{}{}{adj.2g.}{Que traumatiza.}{trau.ma.ti.zan.te}{0}
\verb{traumatizar}{}{}{}{}{v.t.}{Causar trauma; abalar, ferir.}{trau.ma.ti.zar}{\verboinum{1}}
\verb{traumatologia}{}{Med.}{}{}{s.f.}{Ramo da medicina que se ocupa dos traumatismos.}{trau.ma.to.lo.gi.a}{0}
\verb{trautear}{}{}{}{}{v.t.}{Cantar discretamente, em voz baixa; cantarolar.}{trau.te.ar}{\verboinum{4}}
\verb{trava}{}{}{}{}{s.f.}{Ato ou efeito de travar.}{tra.va}{0}
\verb{trava}{}{}{}{}{}{Dispositivo que serve para travar ou manter fixo.}{tra.va}{0}
\verb{trava}{}{}{}{}{}{Ligação, conexão, nexo.}{tra.va}{0}
\verb{travanca}{}{}{}{}{s.f.}{Obstáculo, empecilho, estorvo.}{tra.van.ca}{0}
\verb{travão}{}{}{"-ões}{}{s.m.}{Dispositivo para travar ou impedir o movimento de máquinas, rodas, animais.}{tra.vão}{0}
\verb{travar}{}{}{}{}{v.t.}{Fazer parar ou manter fixo com trava.}{tra.var}{0}
\verb{travar}{}{}{}{}{}{Restringir os movimentos de.}{tra.var}{0}
\verb{travar}{}{}{}{}{}{Estabelecer, começar conversa, amizade, luta.}{tra.var}{0}
\verb{travar}{}{}{}{}{}{Obstruir, atravancar.}{tra.var}{0}
\verb{travar}{}{}{}{}{}{Causar adstringência, amargor.}{tra.var}{0}
\verb{travar}{}{Pop.}{}{}{v.i.}{Ficar entorpecido por álcool ou droga.}{tra.var}{0}
\verb{travar}{}{Informát.}{}{}{}{Parar de responder aos comandos do usuário.}{tra.var}{\verboinum{1}}
\verb{trave}{}{}{}{}{s.f.}{Viga de madeira ou metal para sustentar armação de edifício, teto, rede.}{tra.ve}{0}
\verb{travejamento}{}{}{}{}{s.m.}{Conjunto de traves.}{tra.ve.ja.men.to}{0}
\verb{travejar}{}{}{}{}{v.t.}{Pôr traves em.}{tra.ve.jar}{\verboinum{1}}
\verb{través}{}{}{}{}{s.m.}{Posição oblíqua, atravessada.}{tra.vés}{0}
\verb{travessa}{é}{}{}{}{s.f.}{Tábua atravessada em relação a outras.}{tra.ves.sa}{0}
\verb{travessa}{é}{}{}{}{}{Trave, viga.}{tra.ves.sa}{0}
\verb{travessa}{é}{}{}{}{}{Rua estreita e de tráfego local.}{tra.ves.sa}{0}
\verb{travessa}{é}{}{}{}{}{Recipiente culinário, geralmente longo, para preparar e servir refeições.}{tra.ves.sa}{0}
\verb{travessão}{}{}{}{}{s.m.}{Trave horizontal.}{tra.ves.são}{0}
\verb{travessão}{}{Gram.}{}{}{}{Sinal gráfico em forma de traço, maior que o hífen, usado para separar partes da frase e iniciar parágrafos que contêm falas de personagens nas narrativas.}{tra.ves.são}{0}
\verb{travesseiro}{ê}{}{}{}{s.m.}{Almofada para apoiar a cabeça quando deitado.}{tra.ves.sei.ro}{0}
\verb{travessia}{}{}{}{}{s.f.}{Ato ou efeito de atravessar rio, mar, região geográfica.}{tra.ves.si.a}{0}
\verb{travesso}{ê}{}{}{}{adj.}{Atravessado; colocado de través.}{tra.ves.so}{0}
\verb{travesso}{ê}{}{}{}{}{Diz"-se de pessoa turbulenta, traquinas, buliçosa.}{tra.ves.so}{0}
\verb{travessura}{}{}{}{}{s.f.}{Ato de indivíduo travesso; traquinice.}{tra.ves.su.ra}{0}
\verb{travesti}{}{}{}{}{s.2g.}{Pessoa que se veste com roupas do sexo oposto por prazer ou diversão, sem ser necessariamente homossexual. }{tra.ves.ti}{0}
\verb{travestir}{}{}{}{}{v.t.}{Pôr fantasia em; fantasiar.}{tra.ves.tir}{0}
\verb{travestir}{}{}{}{}{v.pron.}{Dissimular"-se, disfarçar"-se.}{tra.ves.tir}{\verboinum{29}}
\verb{travo}{}{}{}{}{s.m.}{Sabor amargo e adstringente de alimentos.}{tra.vo}{0}
\verb{travor}{ô}{}{}{}{s.m.}{Travo.}{tra.vor}{0}
\verb{trazer}{ê}{}{}{}{v.t.}{Transportar na direção do falante.}{tra.zer}{0}
\verb{trazer}{ê}{}{}{}{}{Ocasionar, causar, acarretar, resultar em.}{tra.zer}{0}
\verb{trazer}{ê}{}{}{}{}{Ter consigo; portar.}{tra.zer}{0}
\verb{trazer}{ê}{}{}{}{}{Vestir, usar.}{tra.zer}{0}
\verb{trazer}{ê}{}{}{}{}{Herdar.}{tra.zer}{0}
\verb{trazer}{ê}{}{}{}{}{Atrair, chamar.}{tra.zer}{\verboinum{43}}
\verb{trecentésimo}{}{}{}{}{num.}{Número ordinal ou fracionário correspondente a 300; tricentésimo.}{tre.cen.té.si.mo}{0}
\verb{trecho}{ê}{}{}{}{s.m.}{Segmento de espaço ou de tempo.}{tre.cho}{0}
\verb{trecho}{ê}{}{}{}{}{Parte de obra literária ou musical; excerto.}{tre.cho}{0}
\verb{treco}{é}{Pop.}{}{}{s.m.}{Qualquer objeto ou sensação que não se quer ou não se sabe como nomear.}{tre.co}{0}
\verb{trêfego}{}{}{}{}{adj.}{Turbulento, traquinas, travesso.}{trê.fe.go}{0}
\verb{trégua}{}{}{}{}{s.f.}{Suspensão temporária das operações bélicas durante uma guerra, geralmente resultante de comum acordo entre as partes envolvidas.}{tré.gua}{0}
\verb{trêiler}{}{}{}{}{s.m.}{Forma aportuguesada de \textit{trailer}.}{trêi.ler}{0}
\verb{treinador}{ô}{}{}{}{s.m.}{Profissional que treina ou adestra pessoas ou animais.}{trei.na.dor}{0}
\verb{treinamento}{}{}{}{}{s.m.}{Ato ou efeito de treinar.}{trei.na.men.to}{0}
\verb{treinar}{}{}{}{}{v.t.}{Desenvolver as habilidades ou capacidades para determinada tarefa; adestrar, exercitar.}{trei.nar}{0}
\verb{treinar}{}{}{}{}{v.i.}{Exercitar"-se na prática de uma atividade.}{trei.nar}{\verboinum{1}}
\verb{treino}{é}{}{}{}{s.m.}{Ato de treinar; destreza em qualquer ramo de atividade; preparo.}{trei.no}{0}
\verb{trejeito}{ê}{}{}{}{s.m.}{Gesto, especialmente jocoso ou esquisito.}{tre.jei.to}{0}
\verb{trejurar}{}{}{}{}{v.t.}{Afirmar jurando enfaticamente.}{tre.ju.rar}{0}
\verb{trejurar}{}{}{}{}{v.i.}{Jurar insistentemente.}{tre.ju.rar}{\verboinum{1}}
\verb{trela}{é}{}{}{}{s.f.}{Corda ou tira com que se prendem os cães.}{tre.la}{0}
\verb{trela}{é}{Fig.}{}{}{}{Liberdade, intimidade, confiança.}{tre.la}{0}
\verb{treliça}{}{}{}{}{s.f.}{Peça feita de pequenas ripas de madeira entrecruzadas usada como adorno em portas de armário, biombos etc.}{tre.li.ça}{0}
\verb{trem}{}{}{trens}{}{s.m.}{Conjunto de locomotiva e vagões engatados, que conduzem passageiros ou carga, em uma estrada de ferro.}{trem}{0}
\verb{trem}{}{}{trens}{}{}{Conjunto de objetos levados em viagem; bagagem.}{trem}{0}
\verb{trem}{}{}{trens}{}{}{Conjunto de móveis e utensílios domésticos de uma casa.}{trem}{0}
\verb{trem}{}{}{trens}{}{}{Qualquer objeto; traste, troço, treco.}{trem}{0}
\verb{trema}{}{Gram.}{}{}{s.m.}{Sinal gráfico ["] usado para indicar que o \textit{u} é pronunciado nas sequências \textit{que, qui, gue, gui}.}{tre.ma}{0}
\verb{trem"-bala}{}{}{}{}{s.m.}{Trem de transporte de passageiros de alta velocidade.}{trem"-ba.la}{0}
\verb{tremedal}{}{}{"-ais}{}{s.m.}{Área de pântano; lamaçal, lodaçal, brejo.}{tre.me.dal}{0}
\verb{tremedeira}{ê}{}{}{}{s.f.}{Ato de tremer; tremura, tremor.}{tre.me.dei.ra}{0}
\verb{tremelicar}{}{}{}{}{v.i.}{Tremer, agitar"-se repetidamente; estremecer.}{tre.me.li.car}{\verboinum{2}}
\verb{tremelique}{}{}{}{}{s.m.}{Ato ou efeito de tremelicar; tremedeira.}{tre.me.li.que}{0}
\verb{tremeluzente}{}{}{}{}{adj.2g.}{Que tremeluz; cintilante.}{tre.me.lu.zen.te}{0}
\verb{tremeluzir}{}{}{}{}{v.i.}{Brilhar tremulamente; cintilar.}{tre.me.lu.zir}{\verboinum{21}}
\verb{tremendo}{}{}{}{}{adj.}{Que faz tremer de medo; assombroso, horripilante.}{tre.men.do}{0}
\verb{tremendo}{}{}{}{}{}{Incomum, formidável, extraordinário.}{tre.men.do}{0}
\verb{tremer}{ê}{}{}{}{v.t.}{Provocar abalo; agitar, sacudir.}{tre.mer}{\verboinum{12}}
\verb{tremido}{}{}{}{}{adj.}{Que apresenta tremor; incerto, trêmulo.}{tre.mi.do}{0}
\verb{treminhão}{}{}{"-ões}{}{s.m.}{Caminhão de 30 metros de comprimento, usado no transporte de cana"-de"-açúcar.}{tre.mi.nhão}{0}
\verb{tremoceiro}{ê}{}{}{}{s.m.}{Ambulante que vende tremoços.}{tre.mo.cei.ro}{0}
\verb{tremoço}{ô}{}{}{}{s.m.}{Grão amarelo, comestível após de cozido. }{tre.mo.ço}{0}
\verb{tremor}{ô}{}{}{}{s.m.}{Ato ou efeito de tremer; abalo, tremedeira.}{tre.mor}{0}
\verb{trempe}{}{}{}{}{s.f.}{Arco de ferro, assentado sobre três pés, sobre o qual se põem panelas que vão ao fogo.}{trem.pe}{0}
\verb{trempe}{}{}{}{}{}{Reunião de três pessoas; trinca.}{trem.pe}{0}
\verb{tremular}{}{}{}{}{v.t.}{Fazer mover no ar de um lado para outro continuamente.}{tre.mu.lar}{\verboinum{1}}
\verb{trêmulo}{}{}{}{}{adj.}{Que treme; vacilante, hesitante.}{trê.mu.lo}{0}
\verb{tremura}{}{}{}{}{s.f.}{Tremor.}{tre.mu.ra}{0}
\verb{trena}{}{}{}{}{s.f.}{Fita métrica de aço, enrolada em pequena caixa, usada na medição de terrenos.}{tre.na}{0}
\verb{treno}{}{}{}{}{s.m.}{Canto fúnebre; elegia, lamentação.}{tre.no}{0}
\verb{trenó}{}{}{}{}{s.m.}{Veículo sem rodas, provido de trilhos deslizantes, usado para locomoção sobre a neve ou o gelo.}{tre.nó}{0}
\verb{trepada}{}{Chul.}{}{}{}{Ato sexual; cópula.}{tre.pa.da}{0}
\verb{trepada}{}{}{}{}{s.f.}{Caminho íngreme; ladeira, subida.}{tre.pa.da}{0}
\verb{trepadeira}{ê}{Bot.}{}{}{s.f.}{Erva ou arbusto que cresce apoiando"-se sobre outra planta ou sobre uma parede.}{tre.pa.dei.ra}{0}
\verb{trepador}{ô}{Zool.}{}{}{s.m.}{Grupo de aves cujos pés possuem dois dedos voltados para frente e dois dedos voltados para trás, o que lhes permite trepar com facilidade, como o papagaio, o periquito, a arara.}{tre.pa.dor}{0}
\verb{trepanação}{}{}{"-ões}{}{s.f.}{Ato ou efeito de trepanar.}{tre.pa.na.ção}{0}
\verb{trepanar}{}{}{}{}{v.t.}{Perfurar osso com trépano.}{tre.pa.nar}{\verboinum{1}}
\verb{trépano}{}{Med.}{}{}{s.m.}{Instrumento cirúrgico em forma de broca, usado para perfurar ossos, principalmente o crânio.}{tré.pa.no}{0}
\verb{trepar}{}{}{}{}{v.t.}{Subir em lugar alto, usando as mãos e os pés.}{tre.par}{\verboinum{1}}
\verb{trepidação}{}{}{"-ões}{}{s.f.}{Ato ou efeito de trepidar; tremor.}{tre.pi.da.ção}{0}
\verb{trepidante}{}{}{}{}{adj.2g.}{Que trepida; vibrante.}{tre.pi.dan.te}{0}
\verb{trepidar}{}{}{}{}{v.i.}{Apresentar tremor; estremecer, oscilar, abalar.}{tre.pi.dar}{\verboinum{1}}
\verb{trépido}{}{}{}{}{adj.}{Que treme de susto; trêmulo, assustadiço.}{tré.pi.do}{0}
\verb{tréplica}{}{}{}{}{s.f.}{Resposta a uma réplica.}{tré.pli.ca}{0}
\verb{treplicar}{}{}{}{}{v.i.}{Responder a uma réplica; contestar, refutar.}{tre.pli.car}{\verboinum{2}}
\verb{três}{}{}{}{}{num.}{Nome dado à quantidade expressa pelo número 3.  }{três}{0}
\verb{tresandar}{}{}{}{}{v.t.}{Fazer andar; desandar.}{tre.san.dar}{0}
\verb{tresandar}{}{}{}{}{v.i.}{Cheirar mal.}{tre.san.dar}{\verboinum{1}}
\verb{trescalar}{}{}{}{}{v.t.}{Exalar cheiro forte.}{tres.ca.lar}{\verboinum{1}}
\verb{tresdobrar}{}{}{}{}{v.t.}{Tornar três vezes maior; triplicar.}{tres.do.brar}{\verboinum{1}}
\verb{tresler}{ê}{}{}{}{v.i.}{Ler às avessas.}{tres.ler}{0}
\verb{tresler}{ê}{}{}{}{}{Perder o juízo por ler em excesso. }{tres.ler}{\verboinum{13}}
\verb{tresloucado}{}{}{}{}{adj.}{Totalmente louco; alucinado, desvairado.}{tres.lou.ca.do}{0}
\verb{tresloucar}{}{}{}{}{v.t.}{Tornar completamente louco; desvairar.}{tres.lou.car}{\verboinum{2}}
\verb{tresmalhado}{}{}{}{}{adj.}{Diz"-se do animal que se afastou do bando; perdido, desgarrado.}{tres.ma.lha.do}{0}
\verb{tresmalhar}{}{}{}{}{v.i.}{Afastar"-se do rebanho; desgarrar.}{tres.ma.lhar}{\verboinum{1}}
\verb{tresnoitar}{}{}{}{}{v.i.}{Passar a noite sem dormir.}{tres.noi.tar}{\verboinum{1}}
\verb{trespassar}{}{}{}{}{v.t.}{Passar além; ultrapassar, traspassar.}{tres.pas.sar}{\verboinum{1}}
\verb{trespasse}{}{}{}{}{s.m.}{Ato ou efeito de trespassar; falecimento.}{tres.pas.se}{0}
\verb{tresvariar}{}{}{}{}{v.i.}{Ficar fora de si; delirar, desvariar.}{tres.va.ri.ar}{\verboinum{1}}
\verb{tresvario}{}{}{}{}{s.m.}{Ato de tresvariar; desvario, delírio.}{tres.va.ri.o}{0}
\verb{treta}{ê}{}{}{}{s.f.}{Destreza na luta; ardil, estratagema.}{tre.ta}{0}
\verb{treta}{ê}{}{}{}{}{Palavreado para enganar; tapeação.}{tre.ta}{0}
\verb{treva}{é}{}{}{}{s.f.}{Trevas.}{tre.va}{0}
\verb{trevas}{é}{}{}{}{s.f.pl.}{Total falta de luz; escuridão, noite.}{tre.vas}{0}
\verb{trevo}{ê}{Bot.}{}{}{s.m.}{Nome comum a diversas plantas cujas folhas são dotadas de três partes, e, excepcionalmente, de quatro.}{tre.vo}{0}
\verb{trevo}{ê}{}{}{}{}{Entroncamento de vias, com passagens rebaixadas ou elevadas para evitar cruzamento em pontos de tráfego intenso.}{tre.vo}{0}
\verb{treze}{ê}{}{}{}{num.}{Nome dado à quantidade expressa pelo número 13.  }{tre.ze}{0}
\verb{trezena}{}{}{}{}{s.f.}{Conjunto de treze unidades.}{tre.ze.na}{0}
\verb{trezena}{}{}{}{}{}{Espaço de treze dias.}{tre.ze.na}{0}
\verb{trezentos}{}{}{}{}{num.}{Nome dado à quantidade expressa pelo número 300.  }{tre.zen.tos}{0}
\verb{tríada}{}{}{}{}{s.f.}{Tríade.}{trí.a.da}{0}
\verb{tríade}{}{}{}{}{s.f.}{Conjunto de três pessoas ou três objetos; trindade, trio.}{trí.a.de}{0}
\verb{triagem}{}{}{}{}{s.f.}{Ato ou efeito de triar; seleção, escolha.  }{tri.a.gem}{0}
\verb{triangulação}{}{}{"-ões}{}{s.f.}{Ato ou efeito de triangular.}{tri.an.gu.la.ção}{0}
\verb{triangular}{}{}{}{}{adj.2g.}{Que tem três ângulos.}{tri.an.gu.lar}{0}
\verb{triangular}{}{}{}{}{v.t.}{Dividir em triângulos.}{tri.an.gu.lar}{\verboinum{1}}
\verb{triangular}{}{}{}{}{}{Cuja base é um triângulo.}{tri.an.gu.lar}{0}
\verb{triangular}{}{}{}{}{}{Que tem forma de triângulo.}{tri.an.gu.lar}{0}
\verb{triângulo}{}{}{}{}{}{Qualquer objeto que tenha a forma dessa figura. (\textit{A sala da casa é um triângulo, com um lado voltado para o mar.})}{tri.ân.gu.lo}{0}
\verb{triângulo}{}{}{}{}{}{Instrumento musical de percussão, feito de uma peça de metal torcida na forma de um triângulo. (\textit{A música ficou boa, mas o triângulo não se conseguia ouvir.})}{tri.ân.gu.lo}{0}
\verb{triângulo}{}{Gram.}{}{}{s.m.}{Figura geométrica que tem três lados. (\textit{Os triângulos são as figuras geométricas com o menor número de lados.})}{tri.ân.gu.lo}{0}
\verb{triângulo}{}{}{}{}{}{Um dos acessórios de segurança dos veículos, que consiste em uma peça plástica brilhante com a forma de um triângulo. (\textit{Mesmo com o triângulo colocado na pista, o acidente acabou ocorrendo})}{tri.ân.gu.lo}{0}
\verb{triatlo}{}{Esport.}{}{}{s.m.}{Competição composta por três provas esportivas diferentes.}{tri.a.tlo}{0}
\verb{tribal}{}{}{}{}{adj.2g.}{Relativo a tribo.}{tri.bal}{0}
\verb{tribo}{}{}{}{}{s.f.}{Grupo étnico de língua e costumes comuns, vivendo em comunidade.}{tri.bo}{0}
\verb{tribulação}{}{}{"-ões}{}{s.f.}{Sofrimento, aflição, tormento; atribulação.}{tri.bu.la.ção}{0}
\verb{tribuna}{}{}{}{}{s.f.}{Lugar mais alto de onde falam os oradores.}{tri.bu.na}{0}
\verb{tribuna}{}{}{}{}{}{Palanque ou varanda onde ficam as autoridades em comemorações públicas.}{tri.bu.na}{0}
\verb{tribunal}{}{Jur.}{"-ais}{}{s.m.}{Corpo de magistrados que decidem julgamentos em conjunto.}{tri.bu.nal}{0}
\verb{tribunal}{}{Jur.}{"-ais}{}{}{Lugar em que se realizam audiências judiciais.}{tri.bu.nal}{0}
\verb{tribuno}{}{}{}{}{s.m.}{Na Roma antiga, magistrado que defendia o povo perante o Senado. }{tri.bu.no}{0}
\verb{tribuno}{}{}{}{}{}{Orador eloquente, que defende os interesses e os direitos do povo.}{tri.bu.no}{0}
\verb{tributação}{}{}{"-ões}{}{s.f.}{Ato ou efeito de tributar.}{tri.bu.ta.ção}{0}
\verb{tributar}{}{}{}{}{v.t.}{Impor ou cobrar tributo; taxar.}{tri.bu.tar}{0}
\verb{tributar}{}{}{}{}{}{Prestar ou render homenagem a alguém.}{tri.bu.tar}{\verboinum{1}}
\verb{tributário}{}{}{}{}{adj.}{Que paga tributo; contribuinte.}{tri.bu.tá.rio}{0}
\verb{tributário}{}{}{}{}{s.m.}{Pessoa sujeita ao pagamento de tributos.}{tri.bu.tá.rio}{0}
\verb{tributo}{}{}{}{}{s.m.}{Imposto, taxa.}{tri.bu.to}{0}
\verb{tributo}{}{Fig.}{}{}{}{Homenagem, preito.}{tri.bu.to}{0}
\verb{trica}{}{}{}{}{s.f.}{Intriga, ardil, enredo, trama.}{tri.ca}{0}
\verb{tricampeão}{}{}{}{}{adj.}{Que venceu três vezes um campeonato, uma disputa etc.}{tri.cam.pe.ão}{0}
\verb{tricampeão}{}{}{}{}{s.m.}{Essa pessoa.}{tri.cam.pe.ão}{0}
\verb{tricampeonato}{}{}{}{}{s.m.}{Campeonato alcançado pela terceira vez.}{tri.cam.pe.o.na.to}{0}
\verb{tricentenário}{}{}{}{}{adj.}{Que tem trezentos anos de idade.}{tri.cen.te.ná.rio}{0}
\verb{tricentenário}{}{}{}{}{s.m.}{A comemoração dessa data.}{tri.cen.te.ná.rio}{0}
\verb{tricentésimo}{}{}{}{}{num.}{Número ordinal ou fracionário correspondente a 300; trecentésimo.}{tri.cen.té.si.mo}{0}
\verb{triciclo}{}{}{}{}{s.m.}{Velocípede com três rodas, próprio para crianças.}{tri.ci.clo}{0}
\verb{tricô}{}{}{}{}{s.m.}{Tecido feito com um par de agulhas apropriadas, ou feito à máquina, com várias agulhas.}{tri.cô}{0}
\verb{tricolor}{ô}{}{}{}{adj.2g.}{Que tem três cores.}{tri.co.lor}{0}
\verb{tricotar}{}{}{}{}{v.t.}{Fazer tricô.}{tri.co.tar}{0}
\verb{tricotar}{}{Pop.}{}{}{v.i.}{Falar da vida alheia; mexericar.}{tri.co.tar}{\verboinum{1}}
\verb{tridente}{}{}{}{}{adj.2g.}{Que tem três dentes.}{tri.den.te}{0}
\verb{tridente}{}{Mit.}{}{}{s.m.}{Cetro mitológico do deus Netuno.}{tri.den.te}{0}
\verb{tridente}{}{}{}{}{}{Garfo com três dentes.}{tri.den.te}{0}
\verb{tridentino}{}{}{}{}{adj.}{Relativo a Trento, na Itália.}{tri.den.ti.no}{0}
\verb{tridentino}{}{}{}{}{s.m.}{Indivíduo natural ou habitante de Trento.}{tri.den.ti.no}{0}
\verb{tridimensional}{}{}{"-ais}{}{adj.2g.}{Relativo às três dimensões (comprimento, largura e altura).}{tri.di.men.si.o.nal}{0}
\verb{triedro}{é}{Geom.}{}{}{adj.}{Que é formado por três planos ou tem três faces.}{tri.e.dro}{0}
\verb{trienal}{}{}{"-ais}{}{adj.2g.}{Que dura três anos. }{tri.e.nal}{0}
\verb{trienal}{}{}{"-ais}{}{}{Que acontece ou é realizado a cada três anos. }{tri.e.nal}{0}
\verb{triênio}{}{}{}{}{}{Período de três anos seguidos.}{tri.ê.nio}{0}
\verb{trifásico}{}{}{}{}{adj.}{Que apresenta três fases ou variações.}{tri.fá.si.co}{0}
\verb{trifásico}{}{}{}{}{}{Diz"-se de um sistema de corrente alternada, no qual as correntes circulam em três circuitos independentes.}{tri.fá.si.co}{0}
\verb{trifólio}{}{Bot.}{}{}{adj.}{Que tem três folhas ou cujas folhas se apresentam em grupos de três; trifoliado.}{tri.fó.lio}{0}
\verb{trifólio}{}{}{}{}{s.m.}{Trevo.}{tri.fó.lio}{0}
\verb{trigal}{}{}{"-ais}{}{s.m.}{Campo de trigo; seara.}{tri.gal}{0}
\verb{trigêmeo}{}{}{}{}{adj.}{Diz"-se de cada um dos três indivíduos nascidos de um só parto.}{tri.gê.meo}{0}
\verb{trigésimo}{}{}{}{}{num.}{Numa sequência, o que ocupa a posição de número 30.}{tri.gé.si.mo}{0}
\verb{trigo}{}{}{}{}{s.m.}{Nome comum a várias plantas herbáceas da família das gramíneas, de cujos frutos, ricos em amido, se faz o pão.}{tri.go}{0}
\verb{trigo}{}{}{}{}{}{O grão dessa planta.}{tri.go}{0}
\verb{trigonometria}{}{Mat.}{}{}{s.f.}{Ramo da matemática que estuda as funções trigonométricas e os métodos de resolução de triângulos.}{tri.go.no.me.tri.a}{0}
\verb{trigueiro}{ê}{}{}{}{adj.}{Que tem a cor do trigo maduro; moreno.}{tri.guei.ro}{0}
\verb{trigueiro}{ê}{}{}{}{s.m.}{Pessoa de tez morena.}{tri.guei.ro}{0}
\verb{trilar}{}{}{}{}{v.i.}{Emitir uma ave sua voz ou canto; gorjear, trinar.}{tri.lar}{0}
\verb{trilar}{}{}{}{}{v.t.}{Cantar imitando um trilo de ave.}{tri.lar}{0}
\verb{trilateral}{}{}{"-ais}{}{adj.2g.}{Trilátero.}{tri.la.te.ral}{0}
\verb{trilátero}{}{}{}{}{adj.}{Que tem três lados; trilateral.}{tri.lá.te.ro}{0}
\verb{trilha}{}{}{}{}{s.f.}{Ato ou efeito de trilhar.}{tri.lha}{0}
\verb{trilha}{}{}{}{}{}{Vestígio, pista, rasto, sinal. }{tri.lha}{0}
\verb{trilha}{}{}{}{}{}{Caminho estreito e tortuoso, entre a vegetação; trilho, vereda.}{tri.lha}{0}
\verb{trilhão}{}{}{"-ões}{}{num.}{Mil bilhões.}{tri.lhão}{0}
\verb{trilhar}{}{}{}{}{v.t.}{Esmagar, moer, triturar.}{tri.lhar}{0}
\verb{trilhar}{}{}{}{}{}{Deixar rastos ou pegadas.}{tri.lhar}{0}
\verb{trilhar}{}{}{}{}{}{Seguir uma direção ou caminho; percorrer; palmilhar, palmear.}{tri.lhar}{0}
\verb{trilhar}{}{}{}{}{}{Abrir trilha, passagem.}{tri.lhar}{\verboinum{1}}
\verb{trilho}{}{}{}{}{s.m.}{Caminho estreito; senda, passagem, trilha, vereda.}{tri.lho}{0}
\verb{trilho}{}{}{}{}{}{Cada uma das duas barras de ferro paralelas sobre as quais deslizam as rodas do trem.}{tri.lho}{0}
\verb{trilogia}{}{}{}{}{s.f.}{Qualquer obra ou poema dividido em três partes.}{tri.lo.gi.a}{0}
\verb{trimensal}{}{}{"-ais}{}{adj.2g.}{Que acontece ou se realiza três vezes por mês.}{tri.men.sal}{0}
\verb{trimestral}{}{}{"-ais}{}{adj.2g.}{Que dura três meses.}{tri.mes.tral}{0}
\verb{trimestre}{é}{}{}{}{s.m.}{Período de três meses.}{tri.mes.tre}{0}
\verb{trinado}{}{}{}{}{s.m.}{Conjunto de sons agradáveis do canto do passarinho; gorgeio.}{tri.na.do}{0}
\verb{trinar}{}{}{}{}{v.t.}{Cantar com trinados; gorgear.}{tri.nar}{\verboinum{1}}
\verb{trinca}{}{}{}{}{s.f.}{Grupo de três coisas ou seres similares.}{trin.ca}{0}
\verb{trinca}{}{}{}{}{}{Resultado de trincar; rachadura.}{trin.ca}{0}
\verb{trincadura}{}{}{}{}{s.f.}{Rachadura, fenda.}{trin.ca.du.ra}{0}
\verb{trincar}{}{}{}{}{v.t.}{Quebrar ou partir alguma coisa sem os dentes.}{trin.car}{0}
\verb{trincar}{}{}{}{}{}{Quebrar alguma coisa sem a desmanchar; trincar.}{trin.car}{\verboinum{2}}
\verb{trincha}{}{}{}{}{s.f.}{Ferramenta usada para arrancar prego.}{trin.cha}{0}
\verb{trinchante}{}{}{}{}{adj.2g.}{Diz"-se de faca que serve para trinchar.}{trin.chan.te}{0}
\verb{trinchar}{}{}{}{}{v.t.}{Retalhar carnes com esmero.}{trin.char}{\verboinum{1}}
\verb{trincheira}{ê}{}{}{}{s.f.}{Fosso aberto na terra para proteger os soldados dos ataques inimigos.}{trin.chei.ra}{0}
\verb{trinco}{}{}{}{}{s.m.}{Tranca pequena.}{trin.co}{0}
\verb{trindade}{}{}{}{}{s.f.}{Conjunto de três pessoas ou coisas; tríade.}{trin.da.de}{0}
\verb{trindade}{}{Relig.}{}{}{}{Dogma católico que proclama a união de três pessoas distintas, Pai, Filho e Espírito Santo formando um só Deus; o mistério da Santíssima Trindade.}{trin.da.de}{0}
\verb{trineto}{ê}{}{}{}{s.m.}{Filho do bisneto ou da bisneta.}{tri.ne.to}{0}
\verb{trinitino}{}{}{}{}{adj.}{Relativo a Trindad e Tobago (América Central).}{tri.ni.ti.no}{0}
\verb{trinitino}{}{}{}{}{s.m.}{Indivíduo natural ou habitante desse país.}{tri.ni.ti.no}{0}
\verb{trino}{}{}{}{}{adj.}{Formado de três elementos.}{tri.no}{0}
\verb{trinômio}{}{Mat.}{}{}{s.m.}{Polinômio de três termos.}{tri.nô.mio}{0}
\verb{trinômio}{}{}{}{}{adj.}{Que tem três partes.}{tri.nô.mio}{0}
\verb{trinque}{}{}{}{}{s.m.}{Cabide em que mascates ou vendedores de roupas de tecido barato expõem a sua mercadoria.}{trin.que}{0}
\verb{trinque}{}{}{}{}{}{Apuro na apresentação pessoal; elegância, esmero.}{trin.que}{0}
\verb{trinta}{}{}{}{}{num.}{Nome dado à quantidade expressa pelo número 30.}{trin.ta}{0}
\verb{trinta"-e"-um}{}{}{}{}{s.m.}{Espécie de jogo de cartas em que vence o parceiro que fizer 31 ou que mais pontos tiver.}{trin.ta"-e"-um}{0}
\verb{trintão}{}{Pop.}{}{}{adj.}{Que está na casa dos trinta anos.}{trin.tão}{0}
\verb{trio}{}{}{}{}{s.m.}{Conjunto de três; trinca.}{tri.o}{0}
\verb{trio}{}{}{}{}{}{Conjunto de três vozes ou três instrumentos.}{tri.o}{0}
\verb{tripa}{}{}{}{}{s.f.}{Intestino de animal.}{tri.pa}{0}
\verb{tripa}{}{}{}{}{}{Pedaço de intestino de porco usado na fabricação de linguiça.}{tri.pa}{0}
\verb{tripanossomíase}{}{Med.}{}{}{s.f.}{Doença causada por tripanossomo.}{tri.pa.nos.so.mí.a.se}{0}
\verb{tripanossomo}{}{}{}{}{s.m.}{Protozoário que vive como parasito no sangue de diversos vertebrados, causando doenças.}{tri.pa.nos.so.mo}{0}
\verb{tripartição}{}{}{"-ões}{}{s.f.}{Divisão em três partes.}{tri.par.ti.ção}{0}
\verb{tripartir}{}{}{}{}{v.t.}{Dividir em três partes.}{tri.par.tir}{\verboinum{18}}
\verb{tripé}{}{}{}{}{s.m.}{Suporte portátil de três apoios.}{tri.pé}{0}
\verb{tripeça}{}{}{}{}{s.f.}{Banco de três pés.}{tri.pe.ça}{0}
\verb{tríplex}{cs}{}{}{}{num.}{Multiplicado por três.}{trí.plex}{0}
\verb{tríplex}{cs}{}{}{}{adj.2g.}{Diz"-se de casa ou apartamento de três andares.}{trí.plex}{0}
\verb{tríplex}{cs}{}{}{}{}{Que é triplo.}{trí.plex}{0}
\verb{triplicação}{}{}{"-ões}{}{s.f.}{Multiplicação por três.}{tri.pli.ca.ção}{0}
\verb{triplicar}{}{}{}{}{v.t.}{Tornar três vezes maior; multiplicar"-se por três.}{tri.pli.car}{\verboinum{2}}
\verb{triplicata}{}{}{}{}{s.f.}{Terceira cópia.}{tri.pli.ca.ta}{0}
\verb{triplicata}{}{}{}{}{}{Substituto de duplicata extraviada.}{tri.pli.ca.ta}{0}
\verb{tríplice}{}{}{}{}{num.}{Multiplicado por três.}{trí.pli.ce}{0}
\verb{tríplice}{}{}{}{}{adj.2g.}{Composto por três elementos ou desenvolvido em três etapas.}{trí.pli.ce}{0}
\verb{triplo}{}{}{}{}{num.}{Que contém três vezes a mesma quantidade.}{tri.plo}{0}
\verb{triplo}{}{}{}{}{adj.}{Que apresenta três características, três funções, três componentes etc.}{tri.plo}{0}
\verb{tríptico}{}{}{}{}{s.m.}{Pintura ou escultura em painel tripartido.}{tríp.ti.co}{0}
\verb{tripudiante}{}{}{}{}{adj.2g.}{Que tripudia, escarnece.}{tri.pu.di.an.te}{0}
\verb{tripudiar}{}{}{}{}{v.i.}{Saltar ou dançar batendo com os pés.}{tri.pu.di.ar}{0}
\verb{tripudiar}{}{}{}{}{}{Divertir"-se com desenvotura, com animação; exultar.}{tri.pu.di.ar}{0}
\verb{tripudiar}{}{}{}{}{v.t.}{Humilhar um adversário.}{tri.pu.di.ar}{\verboinum{1}}
\verb{tripúdio}{}{}{}{}{s.m.}{Ato ou efeito de tripudiar.}{tri.pú.dio}{0}
\verb{tripúdio}{}{}{}{}{}{Dança de sapateado.}{tri.pú.dio}{0}
\verb{tripúdio}{}{}{}{}{}{Licenciosidade., libertinagem.}{tri.pú.dio}{0}
\verb{tripulação}{}{}{"-ões}{}{s.f.}{Equipe que trabalha a bordo.}{tri.pu.la.ção}{0}
\verb{tripulante}{}{}{}{}{s.2g.}{Membro de tripulação.}{tri.pu.lan.te}{0}
\verb{tripular}{}{}{}{}{v.t.}{Pilotar embarcação, aeronave.}{tri.pu.lar}{0}
\verb{tripular}{}{}{}{}{}{Prover de tripulação.}{tri.pu.lar}{\verboinum{1}}
\verb{trirreme}{}{}{}{}{s.m.}{Embarcação grega da Antiguidade impelida por remos armados em três pavimentos e eventualmente por uma vela redonda.}{trir.re.me}{0}
\verb{trisanual}{}{}{"-ais}{}{adj.2g.}{Que ocorre de três em três anos.}{tri.sa.nu.al}{0}
\verb{trisavó}{}{}{}{}{s.m.}{Pai do bisavô ou da bisavó.}{tri.sa.vó}{0}
\verb{trisavô}{}{}{}{}{s.m.}{Pai de bisavô ou de bisavó.}{tri.sa.vô}{0}
\verb{trissilábico}{}{Gram.}{}{}{adj.}{Que tem três sílabas.}{tris.si.lá.bi.co}{0}
\verb{trissílabo}{}{Gram.}{}{}{adj.}{Diz"-se de palavra de três sílabas.}{tris.sí.la.bo}{0}
\verb{triste}{}{}{}{}{adj.2g.}{Que se encontra sem energia ou ânimo.}{tris.te}{0}
\verb{triste}{}{}{}{}{}{Infeliz, desgraçado.}{tris.te}{0}
\verb{tristeza}{ê}{}{}{}{s.f.}{Falta de alegria, de ânimo; melancolia.}{tris.te.za}{0}
\verb{tristonho}{}{}{}{}{adj.}{Que denota ou causa triteza; melancólico.}{tris.to.nho}{0}
\verb{triticultor}{ô}{}{}{}{adj.}{Diz"-se de indivíduo que se dedica à triticultura.}{tri.ti.cul.tor}{0}
\verb{triticultura}{}{}{}{}{s.f.}{Cultura do trigo.}{tri.ti.cul.tu.ra}{0}
\verb{tritongo}{}{Gram.}{}{}{s.m.}{Emissão de tres fonemas vocálicos numa sílaba.}{tri.ton.go}{0}
\verb{triturador}{ô}{}{}{}{adj.}{Que tritura.}{tri.tu.ra.dor}{0}
\verb{triturador}{ô}{}{}{}{s.m.}{Qualquer aparelho que se emprega para triturar uma substância.}{tri.tu.ra.dor}{0}
\verb{triturar}{}{}{}{}{v.t.}{Reduzir a pó ou a pequenos pedaços; roer.}{tri.tu.rar}{\verboinum{1}}
\verb{triunfador}{ô}{}{}{}{adj.}{Que triunfa; vencedor.}{tri.un.fa.dor}{0}
\verb{triunfal}{}{}{"-ais}{}{adj.2g.}{Que vence, triunfa.}{tri.un.fal}{0}
\verb{triunfante}{}{}{}{}{adj.2g.}{Triunfal.}{tri.un.fan.te}{0}
\verb{triunfar}{}{}{}{}{v.t.}{Obter vitória ou levar vantagem sobre.}{tri.un.far}{\verboinum{1}}
\verb{triunfo}{}{}{}{}{s.m.}{Ato ou efeito de triunfar; grande êxito; vitória, sucesso.}{tri.un.fo}{0}
\verb{triunviral}{}{}{"-ais}{}{adj.2g.}{Relativo a triúnviro ou a triunvirato.}{tri.un.vi.ral}{0}
\verb{triunvirato}{}{}{}{}{s.m.}{Magistratura dos triúnviros.}{tri.un.vi.ra.to}{0}
\verb{triunvirato}{}{Por ext.}{}{}{}{Conjunto de três cidadãos associados para exercer uma função ou autoridade.}{tri.un.vi.ra.to}{0}
\verb{triúnviro}{}{}{}{}{s.m.}{Na Roma Antiga, magistrado que recebia, junto com mais dois semelhantes, autoridade e responsabilidade pela administração pública.}{tri.ún.vi.ro}{0}
\verb{triúnviro}{}{Por ext.}{}{}{}{Membro de qualquer triunvirato.}{tri.ún.vi.ro}{0}
\verb{trivalência}{}{}{}{}{s.f.}{Qualidade de trivalente.}{tri.va.lên.cia}{0}
\verb{trivalente}{}{Gram.}{}{}{adj.2g.}{Diz"-se de elemento que requer três argumentos.}{tri.va.len.te}{0}
\verb{trivalente}{}{Quím.}{}{}{}{Diz"-se de elemento que possui valência três.}{tri.va.len.te}{0}
\verb{trivial}{}{}{"-ais}{}{adj.2g.}{Comum, banal, corriqueiro.}{tri.vi.al}{0}
\verb{trivial}{}{}{"-ais}{}{s.m.}{Refeições simples e quotidianas.}{tri.vi.al}{0}
\verb{triz}{}{}{}{}{s.m.}{Usado na expressão \textit{por um triz}: por pouco, por um fio.}{triz}{0}
\verb{troada}{}{}{}{}{s.f.}{Ato ou efeito de troar; estrondo.}{tro.a.da}{0}
\verb{troante}{}{}{}{}{adj.2g.}{Que faz muito barulho.}{tro.an.te}{0}
\verb{troar}{}{}{}{}{v.i.}{Produzir estrondo.}{tro.ar}{0}
\verb{troar}{}{}{}{}{}{Trovejar.}{tro.ar}{\verboinum{7}}
\verb{troca}{}{}{}{}{s.f.}{Ato ou efeito de trocar.}{tro.ca}{0}
\verb{troça}{ó}{}{}{}{s.f.}{Zombaria, pilhéria, mofa.}{tro.ça}{0}
\verb{troça}{ó}{}{}{}{}{Brincadeira, farra.}{tro.ça}{0}
\verb{trocadilho}{}{}{}{}{s.m.}{Jogo de palavras com forma igual ou parecida que produz efeito engraçado ou significado ambíguo.}{tro.ca.di.lho}{0}
\verb{trocado}{}{}{}{}{adj.}{Que se trocou.}{tro.ca.do}{0}
\verb{trocado}{}{}{}{}{s.m.}{Dinheiro miúdo, em moedas ou cédulas de valor baixo.}{tro.ca.do}{0}
\verb{trocador}{ô}{}{}{}{adj.}{Que troca, que cambia.}{tro.ca.dor}{0}
\verb{trocador}{ô}{}{}{}{s.m.}{Indivíduo que cobra as passagens nos ônibus; cobrador.}{tro.ca.dor}{0}
\verb{trocar}{}{}{}{}{v.t.}{Substituir.}{tro.car}{0}
\verb{trocar}{}{}{}{}{}{Alterar, mudar.}{tro.car}{0}
\verb{trocar}{}{}{}{}{}{Permutar, barganhar.}{tro.car}{0}
\verb{trocar}{}{}{}{}{}{Mudar por engano; confundir.}{tro.car}{0}
\verb{trocar}{}{}{}{}{v.pron.}{Mudar de roupa.}{tro.car}{\verboinum{2}}
\verb{troçar}{}{}{}{}{v.t.}{Fazer troça; zombar.}{tro.çar}{\verboinum{3}}
\verb{trocista}{}{}{}{}{adj.2g.}{Que é dado a fazer troça; caçoador.}{tro.cis.ta}{0}
\verb{troco}{ô}{}{}{}{s.m.}{Ato ou efeito de trocar.}{tro.co}{0}
\verb{troco}{ô}{}{}{}{}{Dinheiro que o pagante recebe quando efetua o pagamento com cédulas que totalizam uma quantia maior que o valor da compra.}{tro.co}{0}
\verb{troco}{ô}{}{}{}{}{Dinheiro miúdo; trocado.}{tro.co}{0}
\verb{troco}{ô}{Fig.}{}{}{}{Revide, resposta, vingança.}{tro.co}{0}
\verb{troço}{ó}{}{}{}{s.m.}{Coisa, objeto, treco.}{tro.ço}{0}
\verb{troféu}{}{}{}{}{s.m.}{Objeto que representa uma vitória.}{tro.féu}{0}
\verb{troglodita}{}{}{}{}{adj.2g.}{Diz"-se de habitante das cavernas.}{tro.glo.di.ta}{0}
\verb{troglodita}{}{}{}{}{}{Diz"-se de indivíduo que se impõe pelo porte físico e pela possibilidade de usar a violência em detrimento de uma resolução racional para um impasse.}{tro.glo.di.ta}{0}
\verb{troglodítico}{}{}{}{}{adj.}{Relativo a troglodita.}{tro.glo.dí.ti.co}{0}
\verb{troiano}{}{}{}{}{adj.}{Relativo a Troia, antiga cidade localizada na Ásia Menor.}{troi.a.no}{0}
\verb{troiano}{}{}{}{}{s.m.}{Indivíduo natural ou habitante dessa cidade.}{troi.a.no}{0}
\verb{trole}{}{}{}{}{s.m.}{Haste metálica que estabelece contato elétrico entre uma locomotiva ou bonde e o cabo da via.}{tro.le}{0}
\verb{trole}{}{}{}{}{}{Pequeno vagão de tração manual.}{tro.le}{0}
\verb{trole}{}{}{}{}{}{Tipo de carruagem rústica.}{tro.le}{0}
\verb{trólebus}{}{}{}{}{s.m.}{Ônibus movido a energia elétrica.}{tró.le.bus}{0}
\verb{trolha}{}{}{}{}{s.f.}{Colher de pedreiro para espalhar argamassa.}{tro.lha}{0}
\verb{trolha}{}{}{}{}{s.m.}{Aprendiz ou ajudante de pedreiro.}{tro.lha}{0}
\verb{trolha}{}{Pejor.}{}{}{}{Pedreiro inábil.}{tro.lha}{0}
\verb{trololó}{}{}{}{}{s.m.}{Música simples e ligeira.}{tro.lo.ló}{0}
\verb{trololó}{}{Bras.}{}{}{}{As nádegas.}{tro.lo.ló}{0}
\verb{trom}{}{Onomat.}{}{}{s.m.}{Som de trovão ou canhão; estrondo.}{trom}{0}
\verb{tromba}{}{}{}{}{s.f.}{Orgão facial de alguns animais, p. ex. o elefante, usado para respirar, levar alimentos à boca.}{trom.ba}{0}
\verb{tromba}{}{Pop.}{}{}{}{Expressão fechada, amarrada.}{trom.ba}{0}
\verb{trombada}{}{}{}{}{s.f.}{Pancada com a tromba ou focinho.}{trom.ba.da}{0}
\verb{trombada}{}{}{}{}{}{Colisão, choque, batida.}{trom.ba.da}{0}
\verb{trombadinha}{}{Pop.}{}{}{s.2g.}{Indivíduo menor de idade que pratica pequenos roubos e furtos nas ruas.}{trom.ba.di.nha}{0}
\verb{tromba"-d’água}{}{}{}{}{s.f.}{Massa de vapor de água que se forma na atmosfera, produzindo núvens espessas em forma de cone invertido.}{trom.ba"-d’á.gua}{0}
\verb{tromba"-d’água}{}{Por ext.}{}{}{}{Chuva muito intensa.}{trom.ba"-d’á.gua}{0}
\verb{trombar}{}{}{}{}{v.t.}{Dar trombada; colidir, chocar"-se.}{trom.bar}{\verboinum{1}}
\verb{trombeta}{ê}{Mús.}{}{}{s.m.}{Instrumento de sopro, semelhante à corneta.}{trom.be.ta}{0}
\verb{trombetear}{}{}{}{}{v.t.}{Divulgar com grande alarde.}{trom.be.te.ar}{\verboinum{4}}
\verb{trombeteiro}{ê}{}{}{}{s.m.}{Indivíduo que toca trombeta.}{trom.be.tei.ro}{0}
\verb{trombicar}{}{}{}{}{v.i.}{Ter relações sexuais.}{trom.bi.car}{0}
\verb{trombicar}{}{Fig.}{}{}{v.t.}{Enganar, ludibriar, burlar.}{trom.bi.car}{0}
\verb{trombicar}{}{}{}{}{v.pron.}{Dar"-se mal; trumbicar"-se.}{trom.bi.car}{0}
\verb{trombo}{}{Med.}{}{}{s.m.}{Coágulo que se forma em vaso sanguíneo e ali permanece.}{trom.bo}{0}
\verb{trombone}{}{Mús.}{}{}{s.m.}{Instrumento de sopro constituído por um longo tubo dobrado de metal.}{trom.bo.ne}{0}
\verb{trombose}{}{Med.}{}{}{s.f.}{Coagulação do sangue em um vaso sanguíneo.}{trom.bo.se}{0}
\verb{trombudo}{}{}{}{}{adj.}{Que tem tromba.}{trom.bu.do}{0}
\verb{trombudo}{}{Fig.}{}{}{}{Que apresenta expressão facial carregada; carrancudo.}{trom.bu.do}{0}
\verb{trompa}{}{Mús.}{}{}{s.f.}{Instrumento de sopro constituído de um tubo de metal espiralado e grande pavilhão.}{trom.pa}{0}
\verb{trompaço}{}{Bras.}{}{}{s.m.}{Pancada de tromba.}{trom.pa.ço}{0}
\verb{trompaço}{}{Por ext.}{}{}{}{Qualquer pancada; choque, batida, bofetada.}{trom.pa.ço}{0}
\verb{trompete}{é}{Mús.}{}{}{s.m.}{Instrumento de sopro de metal e com pistões.}{trom.pe.te}{0}
\verb{trompista}{}{Mús.}{}{}{adj.2g.}{Que toca trompa.}{trom.pis.ta}{0}
\verb{troncho}{}{}{}{}{adj.}{Privado de um membro ou ramo; mutilado.}{tron.cho}{0}
\verb{troncho}{}{}{}{}{s.m.}{Membro cortado.}{tron.cho}{0}
\verb{troncho}{}{}{}{}{}{Talo de couve.}{tron.cho}{0}
\verb{tronco}{}{}{}{}{s.m.}{A parte da árvore entre a raiz e os ramos.}{tron.co}{0}
\verb{tronco}{}{Anat.}{}{}{}{A parte do corpo humano excluídos os membros e a cabeça.}{tron.co}{0}
\verb{tronco}{}{Geom.}{}{}{}{Parte truncada de um sólido.}{tron.co}{0}
\verb{tronco}{}{}{}{}{}{Origem de uma família ou linhagem.}{tron.co}{0}
\verb{troncudo}{}{Bras.}{}{}{adj.}{Que tem o tronco bem desenvolvido; forte, corpulento.}{tron.cu.do}{0}
\verb{trono}{}{}{}{}{s.m.}{Tipo de poltrona grande, elevada e ornamentada, em que ficam os monarcas especialmente em ocasiões solenes.}{tro.no}{0}
\verb{trono}{}{Fig.}{}{}{}{O poder soberano.}{tro.no}{0}
\verb{tropa}{}{}{}{}{s.f.}{Conjunto de soldados ou de militares.}{tro.pa}{0}
\verb{tropa}{}{}{}{}{}{Porção de pessoas.}{tro.pa}{0}
\verb{tropa}{}{}{}{}{}{Porção de animais de carga em caravana.}{tro.pa}{0}
\verb{tropeção}{}{}{"-ões}{}{s.m.}{Ato ou efeito de tropeçar.}{tro.pe.ção}{0}
\verb{tropeção}{}{Bras.}{"-ões}{}{}{Erro, equívoco, especialmente relacionado com questões conceituais ou gramaticais. (\textit{O apresentador deu alguns tropeções durante o programa.})}{tro.pe.ção}{0}
\verb{tropeçar}{}{}{}{}{v.i.}{Bater ou prender, por descuido, a ponta do pé em algum obstáculo.}{tro.pe.çar}{0}
\verb{tropeçar}{}{Fig.}{}{}{}{Cometer erro.}{tro.pe.çar}{\verboinum{3}}
\verb{tropeço}{}{}{}{}{s.m.}{Obstáculo em que se tropeça.}{tro.pe.ço}{0}
\verb{tropeço}{}{}{}{}{}{Ato ou efeito de tropeçar.}{tro.pe.ço}{0}
\verb{trôpego}{}{}{}{}{adj.}{Que tem dificuldade em andar ou mover os membros.}{trô.pe.go}{0}
\verb{tropeiro}{ê}{}{}{}{s.m.}{Condutor de tropa de animais.}{tro.pei.ro}{0}
\verb{tropel}{é}{}{"-éis}{}{s.m.}{Ruído de pessoas ou animais se movendo.}{tro.pel}{0}
\verb{tropel}{é}{}{"-éis}{}{}{Tumulto, confusão.}{tro.pel}{0}
\verb{tropel}{é}{}{"-éis}{}{}{Barulho dos pés ao andar.}{tro.pel}{0}
\verb{tropelia}{}{}{}{}{s.f.}{Ruído, balbúrdia.}{tro.pe.li.a}{0}
\verb{tropelia}{}{}{}{}{}{Travessura, traquinice.}{tro.pe.li.a}{0}
\verb{tropical}{}{Geogr.}{}{}{adj.2g.}{Relativo aos trópicos ou à região geográfica compreendida entre eles.}{tro.pi.cal}{0}
\verb{tropical}{}{}{}{}{}{Muito quente.}{tro.pi.cal}{0}
\verb{tropicão}{}{}{"-ões}{}{s.m.}{Tropeção.}{tro.pi.cão}{0}
\verb{tropicar}{}{}{}{}{v.i.}{Tropeçar várias vezes.}{tro.pi.car}{\verboinum{2}}
\verb{trópico}{}{Geogr.}{}{}{s.m.}{Cada um dos círculos paralelos à linha do equador que separam as regiões mais frias das mais quentes do globo. }{tró.pi.co}{0}
\verb{trópico}{}{Geogr.}{}{}{}{Denominação dada às regiões compreendidas entre esses dois círculos.}{tró.pi.co}{0}
\verb{tropilha}{}{}{}{}{s.f.}{Conjunto de cavalos que têm o mesmo pelame.}{tro.pi.lha}{0}
\verb{tropismo}{}{Biol.}{}{}{s.m.}{Desenvolvimento de um vegetal em certa direção provocado por um estímulo exterior como calor, luz etc. }{tro.pis.mo}{0}
\verb{tropo}{ô}{Gram.}{}{}{s.m.}{Emprego de uma palavra em sentido figurado; figura de linguagem.}{tro.po}{0}
\verb{troposfera}{é}{Geogr.}{}{}{s.f.}{Camada de ar mais próxima da superfície terrestre, entre 10 e 12 km de altitude.}{tro.pos.fe.ra}{0}
\verb{trotador}{ô}{}{}{}{adj.}{Diz"-se de cavalo que trota.}{tro.ta.dor}{0}
\verb{trotar}{}{}{}{}{v.i.}{Andar a cavalo a trote.}{tro.tar}{\verboinum{1}}
\verb{trote}{ó}{}{}{}{s.m.}{Maneira ligeira de andar dos cavalos, entre o passo e o galope.}{tro.te}{0}
\verb{trote}{ó}{}{}{}{}{Brincadeira ou zombaria que os veteranos aplicam nos calouros nas universidades.}{tro.te}{0}
\verb{trote}{ó}{}{}{}{}{Brincadeira de mau"-gosto feita pelo telefone por pessoa não identificada.}{tro.te}{0}
\verb{trotear}{}{}{}{}{v.i.}{Trotar.}{tro.te.ar}{\verboinum{4}}
\verb{trouxa}{ch}{}{}{}{s.f.}{Carga ou fardo de roupa.}{trou.xa}{0}
\verb{trouxa}{ch}{}{}{}{adj.2g.}{Diz"-se do indivíduo que se deixa enganar facilmente; tolo, bobo, otário.}{trou.xa}{0}
\verb{trova}{ó}{Liter.}{}{}{s.f.}{Composição poética de quatro versos de caráter popular, que pode ser acompanhada de música; cantiga.}{tro.va}{0}
\verb{trovador}{ô}{}{}{}{s.m.}{Na Idade Média, poeta que cantava composições líricas nos castelos.}{tro.va.dor}{0}
\verb{trovador}{ô}{}{}{}{}{Poeta que compõe e canta trovas.}{tro.va.dor}{0}
\verb{trovadoresco}{ê}{}{}{}{adj.}{Relativo aos trovadores medievais, à sua poesia e à sua época.}{tro.va.do.res.co}{0}
\verb{trovão}{}{}{"-ões}{}{s.m.}{Estrondo produzido por descarga elétrica na atmosfera.}{tro.vão}{0}
\verb{trovar}{}{}{}{}{v.i.}{Fazer ou cantar trovas.}{tro.var}{\verboinum{1}}
\verb{trovejante}{}{}{}{}{adj.2g.}{Que troveja; forte, estrondoso.}{tro.ve.jan.te}{0}
\verb{trovejar}{}{}{}{}{v.i.}{Estrondar o trovão; ribombar, retumbar.}{tro.ve.jar}{0}
\verb{trovejar}{}{}{}{}{}{Soar fortemente como o trovão; vibrar, ressoar.}{tro.ve.jar}{\verboinum{1}}
\verb{trovoada}{}{}{}{}{s.f.}{Sucessão de trovões seguidos de uma tempestade.}{tro.vo.a.da}{0}
\verb{trovoar}{}{}{}{}{v.i.}{Trovejar.}{tro.vo.ar}{\verboinum{7}}
\verb{truanice}{}{}{}{}{s.f.}{Ato ou dito de truão; palhaçada.}{tru.a.ni.ce}{0}
\verb{truão}{}{}{"-ões}{}{s.m.}{Palhaço, saltimbanco.}{tru.ão}{0}
\verb{trucagem}{}{}{"-ens}{}{s.f.}{Em cinema, recurso usado para criar efeitos inesperados, divertidos ou dramáticos. }{tru.ca.gem}{0}
\verb{trucar}{}{}{}{}{v.i.}{No jogo de truco, propor a primeira parada, desafiando o adversário.}{tru.car}{\verboinum{2}}
\verb{trucidar}{}{}{}{}{v.t.}{Matar com crueldade.}{tru.ci.dar}{\verboinum{1}}
\verb{truco}{}{}{}{}{s.m.}{Jogo de cartas em que participam dois adversários, ou quatro parceiros, em duplas, ou seis parceiros, em trincas, e em que correm apostas.}{tru.co}{0}
\verb{truculência}{}{}{}{}{s.f.}{Qualidade de truculento; crueldade, atrocidade.}{tru.cu.lên.cia}{0}
\verb{truculento}{}{}{}{}{adj.}{Que age com crueldade; atroz, feroz.}{tru.cu.len.to}{0}
\verb{trufa}{}{Bot.}{}{}{s.f.}{Cogumelo subterrâneo, de cor escura, comestível, encontrado sob certas árvores, como o carvalho.}{tru.fa}{0}
\verb{trufa}{}{Cul.}{}{}{}{Tipo de bombom com recheio cremoso.}{tru.fa}{0}
\verb{trufeira}{ê}{}{}{}{s.f.}{Terreno onde se encontram trufas.}{tru.fei.ra}{0}
\verb{trufeiro}{ê}{}{}{}{adj.}{Relativo a trufa.}{tru.fei.ro}{0}
\verb{trufeiro}{ê}{}{}{}{}{Diz"-se daquele que apanha ou vende trufas.}{tru.fei.ro}{0}
\verb{truísmo}{}{}{}{}{s.m.}{Verdade banal; lugar"-comum.}{tru.ís.mo}{0}
\verb{trumbicar"-se}{}{Pop.}{}{}{v.pron.}{Sair"-se mal; estrepar"-se.}{trum.bi.car"-se}{\verboinum{2}}
\verb{truncar}{}{}{}{}{v.t.}{Separar do tronco; cortar, mutilar.}{trun.car}{0}
\verb{truncar}{}{}{}{}{}{Retirar parte de um texto.}{trun.car}{\verboinum{2}}
\verb{trunfo}{}{}{}{}{s.m.}{Vantagem ou recurso que possibilita a vitória em uma luta, um negócio etc.}{trun.fo}{0}
\verb{trunfo}{}{}{}{}{}{Jogo de cartas em que participam dois, quatro ou seis jogadores.}{trun.fo}{0}
\verb{trupe}{}{}{}{}{s.f.}{Grupo de artistas que atuam juntos em peças de teatro.}{tru.pe}{0}
\verb{truque}{}{}{}{}{s.m.}{Ato que visa iludir, enganar; mágica.}{tru.que}{0}
\verb{truque}{}{}{}{}{}{Ardil, estratagema, artifício.}{tru.que}{0}
\verb{truste}{}{}{}{}{s.m.}{Associação comercial ou financeira que visa suprimir a concorrência e controlar o mercado; monopólio.}{trus.te}{0}
\verb{truta}{}{Zool.}{}{}{s.f.}{Peixe de água doce, cuja carne é muito apreciada.}{tru.ta}{0}
\verb{truta}{}{Fig.}{}{}{}{Negócio desonesto; trapaça, maracutaia.}{tru.ta}{0}
\verb{truz}{}{}{}{}{interj.}{Palavra que reproduz o som da queda de um corpo ou o estrondo de um tiro.}{truz}{0}
\verb{tsé"-tsé}{}{Zool.}{}{}{s.f.}{Espécie de mosca africana, portadora do tripanossomo causador da doença do sono.}{tsé"-tsé}{0}
\verb{tu}{}{Gram.}{}{}{pron.}{Pronome pessoal da segunda pessoa do singular.}{tu}{0}
\verb{tuba}{}{Mús.}{}{}{s.f.}{Instrumento de sopro, de grande tamanho, e timbre grave.}{tu.ba}{0}
\verb{tubagem}{}{}{"-ens}{}{s.f.}{Sistema por que estão dispostos ou funcionam certos tubos; tubulação.}{tu.ba.gem}{0}
\verb{tubarão}{}{Zool.}{"-ões}{}{s.m.}{Nome comum a vários peixes marinhos, de grande porte, corpo alongado e apetite voraz.}{tu.ba.rão}{0}
\verb{tubarão}{}{Fig.}{"-ões}{}{}{Empresário sem escrúpulos, que só pensa em ganhar dinheiro.}{tu.ba.rão}{0}
\verb{tuberculado}{}{}{}{}{adj.}{Que tem tubérculos.}{tu.ber.cu.la.do}{0}
\verb{tubérculo}{}{Bot.}{}{}{s.m.}{Protuberância, rica em fécula, que se desenvolve nas raízes de certas plantas, como a batata, a mandioca etc.}{tu.bér.cu.lo}{0}
\verb{tubérculo}{}{Med.}{}{}{}{Pequena massa arredondada localizada na pele ou no osso, causada pelo bacilo de Koch, sendo a principal lesão característica da tuberculose.}{tu.bér.cu.lo}{0}
\verb{tuberculose}{ó}{Med.}{}{}{s.f.}{Doença infecto"-contagiosa, causada pelo bacilo de Koch, que atinge principalmente os pulmões, provocando febre, tosse e escarro de sangue.}{tu.ber.cu.lo.se}{0}
\verb{tuberculoso}{ô}{}{"-osos ⟨ó⟩}{"-osa ⟨ó⟩}{adj.}{Relativo a tuberculose.}{tu.ber.cu.lo.so}{0}
\verb{tuberculoso}{ô}{}{"-osos ⟨ó⟩}{"-osa ⟨ó⟩}{}{Diz"-se do indivíduo que sofre de tuberculose.}{tu.ber.cu.lo.so}{0}
\verb{tuberosidade}{}{}{}{}{s.f.}{Saliência em forma de tubérculo.}{tu.be.ro.si.da.de}{0}
\verb{tuberoso}{ô}{}{"-osos ⟨ó⟩}{"-osa ⟨ó⟩}{adj.}{Que tem forma de tubérculo.}{tu.be.ro.so}{0}
\verb{tubiforme}{ó}{}{}{}{adj.2g.}{Que tem forma ou aspecto de tubo.}{tu.bi.for.me}{0}
\verb{tubinho}{}{}{}{}{s.m.}{Modelo de vestido reto, sem recorte na cintura.}{tu.bi.nho}{0}
\verb{tubo}{}{}{}{}{s.m.}{Peça ou canal cilíndrico, reto ou curvo, pelo qual passam líquidos, gases etc.}{tu.bo}{0}
\verb{tubo}{}{}{}{}{}{Canal do organismo do homem e dos animais; conduto.}{tu.bo}{0}
\verb{tubulação}{}{}{"-ões}{}{s.f.}{Conjunto de tubos em uma instalação.}{tu.bu.la.ção}{0}
\verb{tubulado}{}{}{}{}{adj.}{Tubular.}{tubulado}{0}
\verb{tubular}{}{}{}{}{adj.2g.}{Que tem forma de tubo; tubiforme, tubulado.}{tubular}{0}
\verb{tucano}{}{Zool.}{}{}{s.m.}{Pássaro multicolor, de bico grande e alongado.}{tu.ca.no}{0}
\verb{tucum}{}{Bot.}{}{}{s.m.}{Palmeira de folhas fibrosas e de cujos frutos, comestíveis, extrai"-se um óleo muito usado na fabricação de medicamentos e cosméticos.}{tu.cum}{0}
\verb{tucumã}{}{}{}{}{s.m.}{Certa palmeira com espinhos e frutos comestíveis, que fornece palmito, óleo e fibras para fazer corda.}{tucumã}{0}
\verb{tucunaré}{}{Bras.}{}{}{s.m.}{Peixe de corpo prateado, encontrado nos rios da Amazônia.}{tucunaré}{0}
\verb{tucupi}{}{Bras.}{}{}{s.m.}{Molho de mandioca com pimenta.}{tucupi}{0}
\verb{tudo}{}{}{}{}{pron.}{A totalidade das coisas que existem.}{tudo}{0}
\verb{tudo"-nada}{}{}{}{}{s.m.}{Porção insignificante.}{tudo"-nada}{0}
\verb{tufão}{}{}{}{}{s.m.}{Vento muito forte; vendaval, furacão.}{tufão}{0}
\verb{tufar}{}{}{}{}{v.t.}{Inchar, entufar.}{tufar}{0}
\verb{tufar}{}{}{}{}{v.i.}{Tomar forma de tufo.}{tufar}{\verboinum{1}}
\verb{tufo}{}{}{}{}{s.m.}{Porção densa de plantas, penas, fibras, pelos.}{tufo}{0}
\verb{tugir}{}{}{}{}{v.i.}{Falar baixo; murmurar.}{tugir}{0}
\verb{tugúrio}{}{}{}{}{s.m.}{Habitação pequena e simples; cabana, casebre.}{tugúrio}{0}
\verb{tugúrio}{}{}{}{}{}{Refúgio, abrigo.}{tugúrio}{0}
\verb{tuim}{}{Bras.}{}{}{s.m.}{Pequena ave de cor verde; periquito.}{tuim}{0}
\verb{tuiuiú}{}{Bras.}{}{}{s.m.}{Ave de grande porte encontrada em rios, pantanais, lagos; jaburu.}{tuiuiú}{0}
\verb{tule}{}{}{}{}{s.m.}{Tecido fino e transparente, de seda ou algodão; filó.}{tule}{0}
\verb{tulha}{}{}{}{}{s.f.}{Local onde se depositam cereais; celeiro.}{tulha}{0}
\verb{tulha}{}{Por ext.}{}{}{}{Grande quantidade de cereais.}{tulha}{0}
\verb{túlio}{}{Quím.}{}{}{s.m.}{Elemento químico metálico, prateado, dúctil, maleável,  da família dos lantanídeos (terras"-raras); usado em tubos de raios \textsc{x}. \elemento{69}{168.9342}{Tm}.}{tú.lio}{0}
\verb{tulipa}{}{Bot.}{}{}{s.f.}{Planta cultivada como ornamental por suas flores, encontradas em diversas cores.}{tulipa}{0}
\verb{tumba}{}{}{}{}{s.f.}{Túmulo, sepultura.}{tumba}{0}
\verb{tumefação}{}{}{"-ões}{}{s.f.}{Ato ou efeito de tumefazer"-se; inchação, tumor.}{tu.me.fa.ção}{0}
\verb{tumefacto}{}{}{}{}{adj.}{Tumefato.}{tumefacto}{0}
\verb{tumefato}{}{}{}{}{adj.}{Que se tumefez; inchado.}{tumefato}{0}
\verb{tumefazer}{ê}{}{}{}{v.t.}{Inchar, intumescer.}{tumefazer}{0}
\verb{túmido}{}{}{}{}{adj.}{Inchado, dilatado, saliente.}{túmido}{0}
\verb{tumor}{}{Med.}{}{}{s.m.}{Aumento de volume em um tecido, causado pela proliferação celular.}{tumor}{0}
\verb{tumular}{}{}{}{}{v.t.}{Pôr em túmulo; sepultar.}{tu.mu.lar}{\verboinum{1}}
\verb{tumular}{}{}{}{}{adj.2g.}{Relativo a túmulo.}{tu.mu.lar}{0}
\verb{túmulo}{}{}{}{}{s.m.}{Cova no chão, geralmente coberta por uma pedra, em que se enterra um cadáver; sepultura.}{tú.mu.lo}{0}
\verb{tumulto}{}{}{}{}{s.m.}{Agitação, confusão, desordem.}{tu.mul.to}{0}
\verb{tumulto}{}{}{}{}{}{Motim, levante, rebelião.}{tu.mul.to}{0}
\verb{tumultuar}{}{}{}{}{v.t.}{Tirar da ordem; desordenar.}{tu.mul.tu.ar}{0}
\verb{tumultuar}{}{}{}{}{}{Incitar à revolta; rebelar.}{tu.mul.tu.ar}{0}
\verb{tumultuário}{}{}{}{}{adj.}{Relativo a tumulto; desordenado, agitado, amotinado.}{tu.mul.tu.á.rio}{0}
\verb{tumultuoso}{}{}{}{}{adj.}{Em que há tumulto.}{tu.mul.tu.o.so}{0}
\verb{tunda}{}{}{}{}{s.f.}{Surra, sova.}{tun.da}{0}
\verb{tundra}{}{Geogr.}{}{}{s.f.}{Paisagem geográfica das altas latitudes no hemisfério norte caracterizada por vegetação baixa, musgos e líquens.}{tun.dra}{0}
\verb{túnel}{}{}{"-eis}{}{s.m.}{Passagem subterrânea.}{tú.nel}{0}
\verb{tungar}{}{}{}{}{v.t.}{Agredir, bater, surrar.}{tun.gar}{0}
\verb{tungar}{}{}{}{}{}{Enganar, iludir, ludibriar.}{tun.gar}{0}
\verb{tungar}{}{}{}{}{}{Teimar.}{tun.gar}{\verboinum{5}}
\verb{tungstênio}{}{Quím.}{}{}{s.m.}{Elemento químico metálico, cinzento, duro, denso, usado na fabricação de filamentos  para lâmpadas e válvulas eletrônicas, de ligas duras e resistentes à corrosão, de tintas, abrasivos, etc. \elemento{74}{183.84}{w}.}{tungs.tê.nio}{0}
\verb{túnica}{}{}{}{}{s.f.}{Peça de vestuário longa e ajustada ao corpo.}{túnica}{0}
\verb{túnica}{}{}{}{}{}{Veste sacerdotal.}{túnica}{0}
\verb{tunisiano}{}{}{}{}{adj.}{Relativo à Tunísia (África do Norte).}{tunisiano}{0}
\verb{tunisiano}{}{}{}{}{s.m.}{Indivíduo natural ou habitante desse país.}{tunisiano}{0}
\verb{tupã}{}{}{}{}{s.m.}{Nome que os povos tupis davam ao trovão, cultuado como divindade suprema, e que os padres jesuítas da época da colonização empregaram para designar Deus.}{tu.pã}{0}
\verb{tupi}{}{}{}{}{s.m.}{Tronco linguístico que reúne, no Brasil, dez famílias linguísticas, estendendo"-se também a outros países da América do Sul, como Colômbia, Paraguai e Bolívia. }{tu.pi}{0}
\verb{tupi}{}{}{}{}{adj.2g.}{Relativo às línguas desse tronco ou a seus falantes.}{tu.pi}{0}
\verb{tupi"-guarani}{}{}{}{}{s.m.}{Família linguística filiada ao tronco tupi, que reúne um grande número de línguas de povos indígenas brasileiros e de outros países sul"-americanos.}{tupi"-guarani}{0}
\verb{tupi"-guarani}{}{}{}{}{adj.2g.}{Relativo a uma dessas línguas ou a seus falantes.}{tupi"-guarani}{0}
\verb{turba}{}{}{}{}{s.f.}{Grande porção de pessoas aglomeradas; multidão, povo.}{turba}{0}
\verb{turba}{}{}{}{}{}{Os indivíduos que vivem em piores condições econômicas e culturais em uma sociedade; ralé, gentalha.}{turba}{0}
\verb{turbamulta}{}{}{}{}{s.f.}{Turba.}{turbamulta}{0}
\verb{turbante}{}{}{}{}{s.m.}{Cobertura de tecido que se usa sobre a cabeça, muito comum entre alguns povos orientais.}{turbante}{0}
\verb{turbante}{}{}{}{}{}{Ornamento de cabeça feminino feito com tecido fino e ricamente adornado.}{turbante}{0}
\verb{turbar}{}{}{}{}{v.t.}{Causar perturbação; revolver, agitar, transtornar.}{turbar}{\verboinum{1}}
\verb{túrbido}{}{}{}{}{adj.}{Que causa perturbação.}{túrbido}{0}
\verb{túrbido}{}{}{}{}{}{Turvo, escuro, sombrio, obscuro.}{túrbido}{0}
\verb{turbilhão}{}{}{}{}{s.m.}{Redemoinho de vento.}{turbilhão}{0}
\verb{turbilhão}{}{Por ext.}{}{}{}{Redemoinho de líquidos, partículas, objetos.}{turbilhão}{0}
\verb{turbilhão}{}{Fig.}{}{}{}{Aquilo que arrasta violentamente.}{turbilhão}{0}
\verb{turbilhonar}{}{}{}{}{adj.2g.}{Relativo a turbilhão.}{tur.bi.lho.nar}{0}
\verb{turbilhonar}{}{}{}{}{v.t.}{Formar turbilhão.}{tur.bi.lho.nar}{\verboinum{1}}
\verb{turbina}{}{}{}{}{s.f.}{Dispositivo que transforma a energia cinética de um líquido ou gás em energia mecânica giratória.}{turbina}{0}
\verb{turbinagem}{}{}{}{}{s.f.}{Ato de submeter um líquido ou gás à ação de uma turbina.}{turbinagem}{0}
\verb{turbo}{}{}{}{}{adj.}{Diz"-se de motor a combustão em que o combustível é submetido a um turbocompressor.}{turbo}{0}
\verb{turboélice}{}{}{}{}{adj.}{Diz"-se de motor de aeronave no qual uma turbina produz a energia que aciona a hélice.}{turboélice}{0}
\verb{turbopropulsor}{}{}{}{}{adj.}{Turboélice.}{turbopropulsor}{0}
\verb{turbulência}{}{}{}{}{s.f.}{Perturbação da ordem; agitação.}{turbulência}{0}
\verb{turbulência}{}{}{}{}{}{Grande agitação do ar causada por diferenças de pressão e temperatura nas diversas camadas de ar.}{turbulência}{0}
\verb{turbulento}{}{}{}{}{adj.}{Em que há turbulência.}{tur.bu.len.to}{0}
\verb{turco}{}{}{}{}{adj.}{Relativo à Turquia (Ásia e Europa).}{tur.co}{0}
\verb{turco}{}{}{}{}{s.m.}{Indivíduo natural ou habitante desse país.}{tur.co}{0}
\verb{turco}{}{}{}{}{}{A língua falada pelos turcos.}{tur.co}{0}
\verb{turfa}{}{Biol.}{}{}{s.f.}{Massa de vegetais e musgos em decomposição, encontrada em áreas muito úmidas, utilizada como fertilizante e como combustível.}{tur.fa}{0}
\verb{turfe}{}{Esport.}{}{}{s.m.}{Corrida de cavalos; hipismo.}{tur.fe}{0}
\verb{turfista}{}{}{}{}{s.2g.}{Indivíduo frequentador ou apostador de corrida de cavalos.}{tur.fis.ta}{0}
\verb{turgidez}{ê}{}{}{}{s.f.}{Qualidade de túrgido.}{tur.gi.dez}{0}
\verb{túrgido}{}{}{}{}{adj.}{Dilatado, inchado, saliente, intumescido.}{túr.gi.do}{0}
\verb{turíbulo}{}{}{}{}{s.m.}{Vaso em que se queima o incenso nas igrejas.}{tu.rí.bu.lo}{0}
\verb{turiferário}{}{}{}{}{adj.}{Diz"-se daquele que carrega o turíbulo.}{tu.ri.fe.rá.rio}{0}
\verb{turismo}{}{}{}{}{s.m.}{Prática de viajar ou excursionar com fins recreativos, culturais etc.}{tu.ris.mo}{0}
\verb{turista}{}{}{}{}{s.2g.}{Indivíduo que viaja a turismo.}{tu.ris.ta}{0}
\verb{turístico}{}{}{}{}{adj.}{Relativo a turismo ou a turista.}{tu.rís.ti.co}{0}
\verb{turma}{}{}{}{}{s.f.}{Conjunto de pessoas; grupo.}{tur.ma}{0}
\verb{turma}{}{}{}{}{}{Grupo de alunos de uma sala; classe.}{tur.ma}{0}
\verb{turma}{}{}{}{}{}{Cada um dos grupos de pessoas que se revezam em determinadas atividades; turno.}{tur.ma}{0}
\verb{turmalina}{}{Geol.}{}{}{s.f.}{Diz"-se de pedra semipreciosa, de coloração variada, como rosa, preto e verde.}{tur.ma.li.na}{0}
\verb{turnê}{}{}{}{}{s.f.}{Viagem de um artista ou de um grupo de artistas com apresentações em locais predeterminados.}{tur.nê}{0}
\verb{turno}{}{}{}{}{s.m.}{Cada um dos períodos ou das etapas de uma atividade.}{tur.no}{0}
\verb{turquesa}{ê}{Geol.}{}{}{s.f.}{Pedra semipreciosa, de cor azul, sem transparência.}{tur.que.sa}{0}
\verb{turquesa}{ê}{}{}{}{adj.2g.}{Da cor dessa pedra.}{tur.que.sa}{0}
\verb{turra}{}{}{}{}{s.f.}{Persistência excessiva; teimosia, obstinação.}{tur.ra}{0}
\verb{turrão}{}{}{"-ões}{turrona}{adj.}{Diz"-se daquele que é teimoso, obstinado.}{tur.rão}{0}
\verb{turrar}{}{}{}{}{v.i.}{Teimar, obstinar"-se.}{tur.rar}{\verboinum{1}}
\verb{turturino}{}{Zool.}{}{}{adj.}{Relativo a rola.}{tur.tu.ri.no}{0}
\verb{turuna}{}{}{}{}{adj.2g.}{Diz"-se daquele que é valentão, destemido.}{tu.ru.na}{0}
\verb{turvação}{}{}{"-ões}{}{s.f.}{Ato ou efeito de turvar; perturbação.}{tur.va.ção}{0}
\verb{turvar}{}{}{}{}{v.t.}{Tornar turvo, embaciado, opaco.}{tur.var}{\verboinum{1}}
\verb{turvo}{}{}{}{}{adj.}{Que perdeu a transparência; opaco, embaciado.}{tur.vo}{0}
\verb{turvo}{}{}{}{}{}{Agitado, alterado, violento.}{tur.vo}{0}
\verb{tussor}{ô}{}{}{}{s.m.}{Tecido de seda, muito leve.}{tus.sor}{0}
\verb{tuta"-e"-meia}{ê}{}{tuta"-e"-meias}{}{s.f.}{Coisa sem importância; bagatela, ninharia.}{tu.ta"-e"-mei.a}{0}
\verb{tutameia}{é}{}{}{}{s.f.}{Tuta"-e"-meia.}{tu.ta.mei.a}{0}
\verb{tutano}{}{Anat.}{}{}{s.m.}{Substância mole e gordurosa, que preenche o interior dos ossos; medula.}{tu.ta.no}{0}
\verb{tutear}{}{}{}{}{v.t.}{Tratar por \textit{tu}.}{tu.te.ar}{\verboinum{4}}
\verb{tutela}{é}{}{}{}{s.f.}{Encargo de cuidar de um menor ou de uma pessoa impedida ou incapaz de exercer seus direitos civis, através de uma decisão judiciária.}{tu.te.la}{0}
\verb{tutelar}{}{}{}{}{adj.2g.}{Relativo a tutela.}{tu.te.lar}{0}
\verb{tutelar}{}{}{}{}{v.t.}{Exercer a tutela sobre; amparar, proteger.}{tu.te.lar}{\verboinum{1}}
\verb{tutor}{ô}{}{}{}{s.m.}{Indivíduo encarregado de exercer uma tutela.}{tu.tor}{0}
\verb{tutor}{ô}{}{}{}{}{Protetor, defensor.}{tu.tor}{0}
\verb{tutoria}{}{}{}{}{s.f.}{Encargo ou autoridade de tutor.}{tu.to.ri.a}{0}
\verb{tutu}{}{Cul.}{}{}{s.m.}{Prato feito com feijão cozido engrossado com farinha de mandioca ou de milho.}{tu.tu}{0}
\verb{tutu}{}{Pop.}{}{}{}{Dinheiro.}{tu.tu}{0}
\verb{tuxaua}{ch}{}{}{}{s.m.}{Entre alguns povos indígenas, chefe temporal da tribo; morubixaba, cacique.}{tu.xau.a}{0}
\verb{TV}{}{}{}{}{s.f.}{Abrev. de \textit{televisão}.}{TV}{0}
\verb{tweed}{}{}{}{}{s.m.}{Tecido de lã, de origem escocesa, usado na confecção de roupas esportivas.}{\textit{tweed}}{0}
%\verb{}{}{}{}{}{}{}{}{0}
%\verb{}{}{}{}{}{}{}{}{0}
%\verb{}{}{}{}{}{}{}{}{0}
%\verb{}{}{}{}{}{}{}{}{0}
%\verb{}{}{}{}{}{}{}{}{0}
