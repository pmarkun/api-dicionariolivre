\verb{a}{}{}{}{}{s.m.}{Primeira letra e primeira vogal do alfabeto português.}{a}{0}
\verb{a}{}{}{}{}{prep.}{Indica direção ou destino; para.}{a}{0}
\verb{a}{}{}{}{}{art.}{Feminino do artigo \textit{o}. }{a}{0}
\verb{a}{}{}{}{}{pron.}{Feminino do pronome \textit{o}.}{a}{0}
\verb{à}{}{}{}{}{}{Contração da preposição \textit{a} com o artigo \textit{a}.}{à}{0}
\verb{A}{}{Mús.}{}{}{s.m.}{A nota ou o acorde referente ao \textit{lá}, ou à sexta nota da escala de \textit{dó}.}{A}{0}
\verb{A}{}{Fís.}{}{}{}{Símb. de \textit{ampère}.}{A}{0}
\verb{A}{}{}{}{}{}{Na avaliação escolar, é a nota mais alta. (\textit{Ele só tira \textsc{a} em matemática.})}{A}{0}
\verb{A}{}{Mat.}{}{}{}{No sistema hexadecimal, representa o décimo primeiro algarismo, equivalente ao número decimal 10.}{A}{0}
\verb{aba}{}{}{}{}{s.f.}{Prolongamento de alguns objetos para além de sua parte principal.}{a.ba}{0}
\verb{aba}{}{}{}{}{}{Parte do chapéu que sombreia o rosto.}{a.ba}{0}
\verb{aba}{}{}{}{}{}{Parte da carne da costela do boi.}{a.ba}{0}
\verb{abacate}{}{}{}{}{s.m.}{Fruto do abacateiro, de casca verde, com grande caroço e polpa oleaginosa de um verde amarelado.}{a.ba.ca.te}{0}
\verb{abacateiral}{}{}{"-ais}{}{s.m.}{Coletivo de abacateiro.}{a.ba.ca.tei.ral}{0}
\verb{abacateiro}{ê}{Bot.}{}{}{s.m.}{Árvore frutífera cujo fruto é o abacate.}{a.ba.ca.tei.ro}{0}
\verb{abacaxi}{ch}{}{}{}{s.m.}{Fruta comestível de casca grossa e espinhenta, carnosa, em geral com muito caldo; ananás.}{a.ba.ca.xi}{0}
\verb{abacaxi}{ch}{}{}{}{}{O pé dessa fruta.}{a.ba.ca.xi}{0}
\verb{abacaxi}{ch}{Fig.}{}{}{}{Problema difícil de resolver.}{a.ba.ca.xi}{0}
\verb{abacaxizeiro}{ch\ldots{}ê}{Bot.}{}{}{s.m.}{Planta cuja fruta é o abacaxi.}{a.ba.ca.xi.zei.ro}{0}
\verb{abacial}{}{}{"-ais}{}{adj.2g.}{Relativo ao abade, à abadessa ou à abadia.}{a.ba.ci.al}{0}
\verb{ábaco}{}{}{}{}{s.m.}{Instrumento composto de fios de arame, nos quais correm pequenas contas, utilizado para efetuar operações aritméticas.}{á.ba.co}{0}
\verb{abade}{}{}{}{abadessa ⟨ê⟩}{s.m.}{Indivíduo que dirige a abadia.}{a.ba.de}{0}
\verb{abadessa}{ê}{}{}{}{s.f.}{Feminino de abade.}{a.ba.des.sa}{0}
\verb{abadia}{}{}{}{}{s.f.}{Conjunto dos imóveis eclesiásticos, bem como de seus produtos financeiros, dirigidos por um abade.}{a.ba.di.a}{0}
\verb{abafadiço}{}{}{}{}{adj.}{Que se abafa com facilidade.}{a.ba.fa.di.ço}{0}
\verb{abafado}{}{}{}{}{adj.}{Mal ventilado, sufocante.}{a.ba.fa.do}{0}
\verb{abafado}{}{}{}{}{}{Que foi coberto ou tampado.}{a.ba.fa.do}{0}
\verb{abafado}{}{Fig.}{}{}{}{Que foi encoberto, que não foi divulgado.}{a.ba.fa.do}{0}
\verb{abafador}{ô}{}{}{}{adj.}{Que abafa ou cobre algo para impedir a circulação de ar.}{a.ba.fa.dor}{0}
\verb{abafamento}{}{}{}{}{s.m.}{O ato ou o resultado de abafar.}{a.ba.fa.men.to}{0}
\verb{abafante}{}{}{}{}{adj.2g.}{Que abafa.}{a.ba.fan.te}{0}
\verb{abafante}{}{Fig.}{}{}{}{Que se destaca dos demais.}{a.ba.fan.te}{0}
\verb{abafar}{}{}{}{}{v.t.}{Impedir a circulação do ar.}{a.ba.far}{0}
\verb{abafar}{}{Fig.}{}{}{}{Encobrir algo para impedir sua divulgação.}{a.ba.far}{0}
\verb{abafar}{}{Fig.}{}{}{v.i.}{Destacar"-se de maneira a encobrir os demais.}{a.ba.far}{0}
\verb{abafar}{}{Pop.}{}{}{}{Furtar.}{a.ba.far}{0}
\verb{abafar}{}{Fig.}{}{}{v.pron.}{Cobrir"-se de roupas ou cobertas, agasalhar"-se.}{a.ba.far}{\verboinum{1}}
\verb{abaixa"-língua}{ch}{}{abaixa"-línguas ⟨ch⟩}{}{s.m.}{Instrumento em forma de espátula usado, durante exame ou cirurgia, para manter a língua abaixada.  }{a.bai.xa"-lín.gua}{0}
\verb{abaixa"-luz}{ch}{}{abaixa"-luzes ⟨ch⟩}{}{s.m.}{Parte do abajur que protege a lâmpada; cúpula, quebra"-luz.}{a.bai.xa"-luz}{0}
\verb{abaixamento}{ch}{}{}{}{s.m.}{Ato ou efeito de abaixar.}{a.bai.xa.men.to}{0}
\verb{abaixamento}{ch}{}{}{}{}{Diminuição; humilhação.}{a.bai.xa.men.to}{0}
\verb{abaixamento}{ch}{}{}{}{}{Desvalorização.}{a.bai.xa.men.to}{0}
\verb{abaixar}{ch}{}{}{}{v.t.}{Tornar menor em altura ou intensidade.}{a.bai.xar}{0}
\verb{abaixar}{ch}{}{}{}{}{Dirigir algo para o chão.}{a.bai.xar}{0}
\verb{abaixar}{ch}{}{}{}{v.pron.}{Humilhar"-se, rebaixar"-se.}{a.bai.xar}{\verboinum{1}}
\verb{abaixo}{ch}{}{}{}{adv.}{Em lugar, posição ou situação inferior.}{a.bai.xo}{0}
\verb{abaixo}{ch}{}{}{}{}{Na direção da parte mais baixa.}{a.bai.xo}{0}
\verb{abaixo}{ch}{}{}{}{interj.}{Protesto contra uma pessoa ou situação.}{a.bai.xo}{0}
\verb{abaixo"-assinado}{ch}{}{abaixo"-assinados ⟨ch⟩}{}{s.m.}{Documento de reivindicação, de protesto ou de apoio assinado por muitas pessoas.}{a.bai.xo"-as.si.na.do}{0}
\verb{abajur}{}{}{}{}{s.m.}{Objeto que serve para iluminar, dotado de uma lâmpada protegida por uma cúpula.}{a.ba.jur}{0}
\verb{abajur}{}{Desus.}{}{}{}{A cúpula desse objeto.}{a.ba.jur}{0}
\verb{abalada}{}{}{}{}{s.f.}{Usada na locução \textit{de abalada}: rapidamente, de forma precipitada.}{a.ba.la.da}{0}
\verb{abalado}{}{}{}{}{adj.}{Que sofreu abalo, estremecimento.}{a.ba.la.do}{0}
\verb{abalado}{}{}{}{}{}{Chocado, assustado; comovido.}{a.ba.la.do}{0}
\verb{abalançar"-se}{}{}{}{}{v.pron.}{Atrever"-se, arriscar"-se.}{a.ba.lan.çar"-se}{\verboinum{3}}
\verb{abalar}{}{}{}{}{v.t.}{Diminuir a firmeza, enfraquecer.}{a.ba.lar}{0}
\verb{abalar}{}{}{}{}{}{Chocar, assustar; comover.}{a.ba.lar}{0}
\verb{abalar}{}{}{}{}{}{Pôr"-se a fazer.}{a.ba.lar}{0}
\verb{abalar}{}{}{}{}{v.i.}{Sair, partir rapidamente.}{a.ba.lar}{\verboinum{1}}
\verb{abalizado}{}{}{}{}{adj.}{Que tem competência; que merece respeito e confiança.}{a.ba.li.za.do}{0}
\verb{abalizado}{}{}{}{}{}{Marcado com baliza, assinalado.}{a.ba.li.za.do}{0}
\verb{abalizamento}{}{Desus.}{}{}{s.m.}{Ato ou efeito de sinalizar com balizas.}{a.ba.li.za.men.to}{0}
\verb{abalizar}{}{}{}{}{v.t.}{Dar garantia de qualidade ou veracidade; apoiar.}{a.ba.li.zar}{0}
\verb{abalizar}{}{Desus.}{}{}{}{Colocar sinais, marcar, delimitar.}{a.ba.li.zar}{\verboinum{1}}
\verb{abalo}{}{}{}{}{s.m.}{Tremor, estremecimento.}{a.ba.lo}{0}
\verb{abalo}{}{}{}{}{}{Choque, perturbação; comoção.}{a.ba.lo}{0}
\verb{abalroamento}{}{}{}{}{s.m.}{Ato ou efeito de abalroar, bater.}{a.bal.ro.a.men.to}{0}
\verb{abalroamento}{}{}{}{}{}{Batida, colisão.}{a.bal.ro.a.men.to}{0}
\verb{abalroar}{}{}{}{}{v.t.}{Colidir, chocar.}{a.bal.ro.ar}{0}
\verb{abalroar}{}{}{}{}{}{Ir de encontro a, chocar"-se com ou contra.}{a.bal.ro.ar}{\verboinum{7}}
\verb{abanador}{ô}{}{}{}{adj.}{Que abana.}{a.ba.na.dor}{0}
\verb{abanador}{ô}{}{}{}{s.m.}{Objeto usado para fazer vento; leque, abano.}{a.ba.na.dor}{0}
\verb{abanar}{}{}{}{}{v.t.}{Balançar.}{a.ba.nar}{0}
\verb{abanar}{}{}{}{}{}{Mover um objeto de modo a fazer vento.}{a.ba.nar}{\verboinum{1}}
\verb{abancado}{}{}{}{}{adj.}{Sentado em banco ou qualquer outro assento.}{a.ban.ca.do}{0}
\verb{abancado}{}{}{}{}{adj.}{Que tem banca ou escritório montado.}{a.ban.ca.do}{0}
\verb{abancar}{}{}{}{}{v.i.}{Sentar.}{a.ban.car}{0}
\verb{abancar}{}{Pop.}{}{}{}{Fugir.}{a.ban.car}{0}
\verb{abancar}{}{}{}{}{v.pron.}{Estabelecer"-se com banca ou escritório.}{a.ban.car}{\verboinum{2}}
\verb{abandalhar}{}{}{}{}{v.t.}{Tornar bandalho, reles; humilhar; ridicularizar.}{a.ban.da.lhar}{\verboinum{1}}
\verb{abandidar}{}{}{}{}{v.t.}{Tornar bandido, criminoso.}{a.ban.di.dar}{\verboinum{1}}
\verb{abandonado}{}{}{}{}{adj.}{Largado, deixado de lado.}{a.ban.do.na.do}{0}
\verb{abandonado}{}{}{}{}{}{Desamparado.}{a.ban.do.na.do}{0}
\verb{abandonar}{}{}{}{}{v.t.}{Largar. }{a.ban.do.nar}{0}
\verb{abandonar}{}{}{}{}{}{Deixar desamparado.}{a.ban.do.nar}{0}
\verb{abandonar}{}{}{}{}{v.pron.}{Deixar"-se dominar por uma situação ou por um sentimento.}{a.ban.do.nar}{\verboinum{1}}
\verb{abandono}{}{}{}{}{s.m.}{Ato ou efeito de abandonar.}{a.ban.do.no}{0}
\verb{abandono}{}{}{}{}{}{Desistência.}{a.ban.do.no}{0}
\verb{abandono}{}{}{}{}{}{Desamparo.}{a.ban.do.no}{0}
\verb{abano}{}{}{}{}{s.m.}{Ato ou efeito de abanar.}{a.ba.no}{0}
\verb{abano}{}{}{}{}{}{Objeto plano usado para agitar o ar, provocando vento; leque, abanador. }{a.ba.no}{0}
\verb{abantesma}{ê}{}{}{}{s.2g.}{Fantasma, espectro.}{a.ban.tes.ma}{0}
\verb{abantesma}{ê}{Fig.}{}{}{}{Indivíduo ou coisa que causa espanto.}{a.ban.tes.ma}{0}
\verb{abará}{}{Cul.}{}{}{s.m.}{Bolo de feijão, cozido em banho"-maria, enrolado em folhas verdes de bananeira.}{a.ba.rá}{0}
\verb{abarbado}{}{}{}{}{adj.}{Sobrecarregado de trabalhos ou  problemas difíceis.}{a.bar.ba.do}{0}
\verb{abarbado}{}{}{}{}{}{Atrapalhado.}{a.bar.ba.do}{0}
\verb{abarcar}{}{}{}{}{v.t.}{Cingir, envolver com as mãos, os braços ou as pernas.}{a.bar.car}{0}
\verb{abarcar}{}{}{}{}{}{Alcançar, abranger.}{a.bar.car}{\verboinum{2}}
\verb{abarrotado}{}{}{}{}{adj.}{Cheio demais; lotado.}{a.bar.ro.ta.do}{0}
\verb{abarrotado}{}{}{}{}{}{Empanturrado.}{a.bar.ro.ta.do}{0}
\verb{abarrotamento}{}{}{}{}{s.m.}{Ato ou efeito de abarrotar.}{a.bar.ro.ta.men.to}{0}
\verb{abarrotamento}{}{}{}{}{}{Lotação excessiva, superlotação.}{a.bar.ro.ta.men.to}{0}
\verb{abarrotar}{}{}{}{}{v.t.}{Encher demais.}{a.bar.ro.tar}{0}
\verb{abarrotar}{}{}{}{}{v.pron.}{Empanturrar"-se.}{a.bar.ro.tar}{\verboinum{1}}
\verb{abastado}{}{}{}{}{adj.}{Que tem  muitos bens materiais ou financeiros; farto, rico.}{a.bas.ta.do}{0}
\verb{abastado}{}{}{}{}{s.m.}{Indivíduo bem provido desses bens.}{a.bas.ta.do}{0}
\verb{abastança}{}{}{}{}{s.f.}{Fartura, riqueza, abundância.}{a.bas.tan.ça}{0}
\verb{abastar}{}{Desus.}{}{}{v.t.}{Prover do suficiente; abastecer.}{a.bas.tar}{\verboinum{1}}
\verb{abastecedor}{ô}{}{}{}{adj.}{Que abastece; provedor.}{a.bas.te.ce.dor}{0}
\verb{abastecedor}{ô}{}{}{}{s.m.}{Indivíduo que abastece ou provê.}{a.bas.te.ce.dor}{0}
\verb{abastecer}{ê}{}{}{}{v.t.}{Prover do suficiente; colocar ou conseguir o necessário.}{a.bas.te.cer}{\verboinum{15}}
\verb{abastecido}{}{}{}{}{adj.}{Provido, abastado, cheio; que tem o necessário ou suficiente.}{a.bas.te.ci.do}{0}
\verb{abastecimento}{}{}{}{}{s.m.}{Ato ou efeito de abastecer; provimento.}{a.bas.te.ci.men.to}{0}
\verb{abatatado}{}{}{}{}{adj.}{Que tem forma de batata.}{a.ba.ta.ta.do}{0}
\verb{abatatado}{}{Pop.}{}{}{}{Diz"-se do nariz grosso ou largo.}{a.ba.ta.ta.do}{0}
\verb{abate}{}{}{}{}{s.m.}{Ato de abater um animal para consumo.}{a.ba.te}{0}
\verb{abate}{}{}{}{}{}{Queda, abatimento, diminuição.}{a.ba.te}{0}
\verb{abate}{}{}{}{}{}{Ato de derrubar um inimigo.}{a.ba.te}{0}
\verb{abatedouro}{ô}{}{}{}{s.m.}{Local onde se abatem os animais, matadouro.}{a.ba.te.dou.ro}{0}
\verb{abater}{ê}{}{}{}{v.t.}{Derrubar, fazer cair.}{a.ba.ter}{0}
\verb{abater}{ê}{}{}{}{}{Matar animais para consumo.}{a.ba.ter}{0}
\verb{abater}{ê}{}{}{}{}{Derrubar ou matar inimigos.}{a.ba.ter}{0}
\verb{abater}{ê}{}{}{}{}{Diminuir no valor.}{a.ba.ter}{0}
\verb{abater}{ê}{}{}{}{}{Debilitar, desanimar.}{a.ba.ter}{\verboinum{13}}
\verb{abatido}{}{}{}{}{adj.}{Que se abateu.}{a.ba.ti.do}{0}
\verb{abatido}{}{}{}{}{}{Caído, derrubado.}{a.ba.ti.do}{0}
\verb{abatido}{}{}{}{}{}{Debilitado, desanimado.}{a.ba.ti.do}{0}
\verb{abatimento}{}{}{}{}{s.m.}{Ato de abater.}{a.ba.ti.men.to}{0}
\verb{abatimento}{}{}{}{}{}{Derrubada.}{a.ba.ti.men.to}{0}
\verb{abatimento}{}{}{}{}{}{Diminuição de preço, desconto.}{a.ba.ti.men.to}{0}
\verb{abatimento}{}{}{}{}{}{Desânimo; cansaço.}{a.ba.ti.men.to}{0}
\verb{abatumar}{}{}{}{}{v.i.}{Tornar"-se (pão ou bolo) espesso ou pesado.}{a.ba.tu.mar}{\verboinum{1}}
\verb{abaulado}{}{}{}{}{adj.}{Que tem forma de baú; convexo, arredondado.}{a.ba.u.la.do}{0}
\verb{abaular}{}{}{}{}{v.t.}{Dar forma de baú a; tornar convexo, arredondar.}{a.ba.u.lar}{\verboinum{8}}
\verb{abc}{}{}{}{}{s.m.}{Alfabeto; abecê.}{abc}{0}
\verb{abcesso}{é}{}{}{}{}{Var. de \textit{abscesso}.}{ab.ces.so}{0}
\verb{abdicação}{}{}{"-ões}{}{s.f.}{Ato ou efeito de abdicar; renúncia, desistência.   }{ab.di.ca.ção}{0}
\verb{abdicador}{ô}{}{}{abdicatriz \textit{ou} abdicadora ⟨ô⟩}{adj.}{Que abdica, que renuncia.}{ab.di.ca.dor}{0}
\verb{abdicar}{}{}{}{}{v.t.}{Renunciar por vontade própria a cargo ou dignidade; desistir de, abandonar.}{ab.di.car}{0}
\verb{abdicar}{}{}{}{}{v.i.}{Renunciar a cargo ou dignidade que ocupava; demitir"-se.}{ab.di.car}{\verboinum{2}}
\verb{abdome}{ô/ ou /ó}{}{}{}{}{Var. de \textit{abdômen}.}{ab.do.me}{0}
\verb{abdômen}{}{Anat.}{abdomens \textit{ou} abdômenes}{}{s.m.}{Parte do corpo localizada entre o tórax e a bacia; ventre, barriga.}{ab.dô.men}{0}
\verb{abdominal}{}{}{"-ais}{}{adj.2g.}{Relativo ao abdômen.}{ab.do.mi.nal}{0}
\verb{abdominal}{}{}{"-ais}{}{s.m.}{Exercício para fortalecimento dos músculos dessa região do corpo.}{ab.do.mi.nal}{0}
\verb{abdominoscopia}{}{Med.}{}{}{s.f.}{Exame do abdômen com instrumentos; laparoscopia.}{ab.do.mi.nos.co.pi.a}{0}
\verb{abdução}{}{}{"-ões}{}{s.f.}{Ato ou efeito de abduzir, afastar.}{ab.du.ção}{0}
\verb{abdução}{}{Jur.}{"-ões}{}{}{Rapto com violência, fraude ou sedução.}{ab.du.ção}{0}
\verb{abdução}{}{Pop.}{"-ões}{}{}{Rapto de pessoas ou animais por seres extraterrestres.}{ab.du.ção}{0}
\verb{abdutor}{ô}{}{}{}{adj.}{Que provoca a abdução, o afastamento.}{ab.du.tor}{0}
\verb{abdutor}{ô}{Anat.}{}{}{s.m.}{Músculo que afasta um membro do plano médio do corpo.}{ab.du.tor}{0}
\verb{abduzir}{}{}{}{}{v.t.}{Afastar.}{ab.du.zir}{0}
\verb{abduzir}{}{}{}{}{}{Raptar, arrebatar.}{ab.du.zir}{\verboinum{21}}
\verb{abeberar}{}{}{}{}{v.t.}{Dar de beber a, dessedentar.}{a.be.be.rar}{0}
\verb{abeberar}{}{}{}{}{}{Embeber, encharcar.}{a.be.be.rar}{\verboinum{1}}
\verb{abecar}{}{}{}{}{v.t.}{Segurar com violência, pela gola ou pelo colarinho, com o propósito de agredir.}{a.be.car}{\verboinum{2}}
\verb{abecê}{}{}{}{}{s.m.}{Alfabeto.}{a.be.cê}{0}
\verb{abecê}{}{Fig.}{}{}{}{Ideias básicas de uma ciência ou arte. }{a.be.cê}{0}
\verb{abecedar}{}{}{}{}{v.t.}{Colocar em ordem alfabética.}{a.be.ce.dar}{\verboinum{1}}
\verb{abecedário}{}{}{}{}{s.m.}{Conjunto de todas as letras do alfabeto; alfabeto, abecê.}{a.be.ce.dá.rio}{0}
\verb{abeirar}{}{}{}{}{v.t.}{Chegar à beira de; chegar perto de. }{a.bei.rar}{0}
\verb{abeirar}{}{}{}{}{v.pron.}{Achegar"-se, aproximar"-se.}{a.bei.rar}{\verboinum{1}}
\verb{abelha}{ê}{Zool.}{}{}{s.f.}{Inseto de quatro asas membranosas produtor de cera e mel.}{a.be.lha}{0}
\verb{abelha"-mestra}{ê\ldots{}é}{Zool.}{abelhas"-mestras ⟨ê\ldots{}é⟩}{}{s.f.}{Única abelha da colmeia que gera outras abelhas; rainha.}{a.be.lha"-mes.tra}{0}
\verb{abelheira}{ê}{}{}{}{s.f.}{Colmeia de abelhas.}{a.be.lhei.ra}{0}
\verb{abelheira}{ê}{Desus.}{}{}{}{Orifícios que surgem nas pedras.}{a.be.lhei.ra}{0}
\verb{abelheiro}{ê}{}{}{}{s.m.}{Indivíduo que trata de abelhas; apicultor.}{a.be.lhei.ro}{0}
\verb{abelheiro}{ê}{}{}{}{}{Colmeia, ninho de abelhas.}{a.be.lhei.ro}{0}
\verb{abelheiro}{ê}{Fig.}{}{}{}{Buracos que surgem nas pedras, assemelhando"-as a colmeias.}{a.be.lhei.ro}{0}
\verb{abelhudo}{}{}{}{}{adj.}{Que se intromete nas coisas alheias; bisbilhoteiro, curioso.}{a.be.lhu.do}{0}
\verb{abelhudo}{}{}{}{}{s.m.}{Indivíduo intrometido, bisbilhoteiro, metediço.}{a.be.lhu.do}{0}
\verb{abençoado}{}{}{}{}{adj.}{Que recebeu bênção; bendito.}{a.ben.ço.a.do}{0}
\verb{abençoado}{}{}{}{}{}{Diz"-se de algo que, por suas qualidades positivas intensas, parece compartilhar de natureza divina.}{a.ben.ço.a.do}{0}
\verb{abençoar}{}{}{}{}{v.t.}{Dar a bênção a, benzer.}{a.ben.ço.ar}{0}
\verb{abençoar}{}{}{}{}{}{Favorecer, proteger.}{a.ben.ço.ar}{\verboinum{7}}
\verb{aberração}{}{}{"-ões}{}{s.f.}{Anomalia, deformidade.}{a.ber.ra.ção}{0}
\verb{aberração}{}{Fís.}{"-ões}{}{}{Defeito em uma superfície que gera uma imagem não correspondente ao objeto.}{a.ber.ra.ção}{0}
\verb{aberrante}{}{}{}{}{adj.2g.}{Que se desvia ou se afasta da norma; anômalo.}{a.ber.ran.te}{0}
\verb{aberrar}{}{}{}{}{v.i.}{Desviar"-se, afastar"-se da norma.}{a.ber.rar}{0}
\verb{aberrar}{}{}{}{}{}{Tornar"-se anormal.}{a.ber.rar}{\verboinum{1}}
\verb{aberta}{é}{}{}{}{s.f.}{Abertura, brecha, passagem. (\textit{Saiu pela mesma aberta do canavial por onde havia entrado.})}{a.ber.ta}{0}
\verb{aberto}{é}{}{}{}{adj.}{Que está descoberto, exposto; que não está fechado.}{a.ber.to}{0}
\verb{aberto}{é}{}{}{}{}{Vasto, espaçoso.}{a.ber.to}{0}
\verb{aberto}{é}{Fig.}{}{}{}{Franco, sincero.}{a.ber.to}{0}
\verb{aberto}{é}{Fig.}{}{}{}{Liberal.}{a.ber.to}{0}
\verb{aberto}{é}{}{}{}{}{Em funcionamento. (\textit{As lojas estarão abertas no feriado.})}{a.ber.to}{0}
\verb{abertura}{}{}{}{}{s.f.}{Ato ou efeito de abrir.}{a.ber.tu.ra}{0}
\verb{abertura}{}{}{}{}{}{Fenda, passagem.}{a.ber.tu.ra}{0}
\verb{abertura}{}{Fig.}{}{}{}{Sinceridade, franqueza.}{a.ber.tu.ra}{0}
\verb{abertura}{}{}{}{}{}{Inauguração, começo.}{a.ber.tu.ra}{0}
\verb{abertura}{}{Mús.}{}{}{}{Introdução que precede uma ópera, um oratório.}{a.ber.tu.ra}{0}
\verb{abespinhar}{}{}{}{}{v.t.}{Irritar, exasperar, enfurecer.}{a.bes.pi.nhar}{0}
\verb{abespinhar}{}{}{}{}{}{Amuar, aborrecer.}{a.bes.pi.nhar}{\verboinum{1}}
\verb{abestalhado}{}{}{}{}{adj.}{Que apresenta comportamento inadequado a uma situação ou a um lugar.}{a.bes.ta.lha.do}{0}
\verb{abestalhado}{}{}{}{}{}{Perplexo, pasmado, espantado, embasbacado.}{a.bes.ta.lha.do}{0}
\verb{abestalhar"-se}{}{}{}{}{v.pron.}{Bestificar"-se, embrutecer"-se.}{a.bes.ta.lhar"-se}{0}
\verb{abestalhar"-se}{}{}{}{}{}{Ficar perplexo, pasmo.}{a.bes.ta.lhar"-se}{\verboinum{2}}
\verb{abeto}{ê}{Bot.}{}{}{s.m.}{Árvore conífera de grande porte, originária de regiões de clima temperado, cultivada no Brasil como ornamental.}{a.be.to}{0}
\verb{abetumado}{}{}{}{}{adj.}{Calafetado com betume.}{a.be.tu.ma.do}{0}
\verb{abetumado}{}{Fig.}{}{}{}{Triste, melancólico, abatumado.}{a.be.tu.ma.do}{0}
\verb{abetumar}{}{}{}{}{v.t.}{Untar, cobrir ou tapar com betume; calafetar.}{a.be.tu.mar}{\verboinum{1}}
\verb{abicar}{}{}{}{}{v.i.}{Aportar, chegar}{a.bi.car}{0}
\verb{abicar}{}{Desus.}{}{}{v.t.}{Fazer bico em algum objeto.}{a.bi.car}{\verboinum{2}}
\verb{abieiro}{ê}{Bot.}{}{}{s.m.}{Árvore frutífera, cujo fruto comestível é o abio.}{a.bi.ei.ro}{0}
\verb{abio}{}{}{}{}{s.m.}{Abiu.}{a.bi.o}{0}
\verb{abiogênese}{}{Biol.}{}{}{s.f.}{Geração espontânea de vida a partir de matéria inanimada.}{a.bi.o.gê.ne.se}{0}
\verb{abiose}{ó}{}{}{}{s.f.}{Suspensão aparente da vida; vida latente.}{a.bi.o.se}{0}
\verb{abiose}{ó}{}{}{}{}{Estado do que é inapto para viver.}{a.bi.o.se}{0}
\verb{abiótico}{}{}{}{}{adj.}{Relativo a abiose.}{a.bi.ó.ti.co}{0}
\verb{abiótico}{}{}{}{}{}{Contrário à vida, que não a favorece.}{a.bi.ó.ti.co}{0}
\verb{abiótico}{}{}{}{}{}{Em que não há vida.}{a.bi.ó.ti.co}{0}
\verb{abiscoitado}{}{}{}{}{adj.}{Que tem forma de biscoito.}{a.bis.coi.ta.do}{0}
\verb{abiscoitado}{}{Pop.}{}{}{}{Conseguido, ganho.}{a.bis.coi.ta.do}{0}
\verb{abiscoitar}{}{}{}{}{v.t.}{Cozer como biscoito.}{a.bis.coi.tar}{0}
\verb{abiscoitar}{}{Pop.}{}{}{}{Conseguir, ganhar, lucrar.}{a.bis.coi.tar}{\verboinum{1}}
\verb{abismado}{}{}{}{}{adj.}{Assombrado, atônito, perplexo.}{a.bis.ma.do}{0}
\verb{abismal}{}{}{"-ais}{}{adj.2g.}{Relativo a abismo.}{a.bis.mal}{0}
\verb{abismal}{}{}{"-ais}{}{}{Abissal.}{a.bis.mal}{0}
\verb{abismar}{}{}{}{}{v.t.}{Lançar no abismo.}{a.bis.mar}{0}
\verb{abismar}{}{}{}{}{}{Causar assombro, espanto.}{a.bis.mar}{0}
\verb{abismar}{}{}{}{}{v.pron.}{Concentrar"-se profundamente.}{a.bis.mar}{0}
\verb{abismar}{}{}{}{}{}{(\textit{No meio de tanta gente, eu me abismei e corri.})}{a.bis.mar}{0}
\verb{abismar}{}{}{}{}{}{(\textit{A crise financeira não é razão para abismar"-se.})}{a.bis.mar}{\verboinum{1}}
\verb{abismo}{}{}{}{}{s.m.}{Abertura natural e profunda de um terreno.}{a.bis.mo}{0}
\verb{abismo}{}{}{}{}{}{Precipício.}{a.bis.mo}{0}
\verb{abismo}{}{Fig.}{}{}{}{Desastre, ruína.}{a.bis.mo}{0}
\verb{abissal}{}{}{"-ais}{}{adj.2g.}{Relativo a abismo.}{a.bis.sal}{0}
\verb{abissal}{}{}{"-ais}{}{}{Diz"-se da região profunda dos oceanos.}{a.bis.sal}{0}
\verb{abissal}{}{Fig.}{"-ais}{}{}{Que causa assombro, espanto.}{a.bis.sal}{0}
\verb{abissínio}{}{}{}{}{adj.}{Relativo à Abissínia, atual Etiópia.}{a.bis.sí.nio}{0}
\verb{abissínio}{}{}{}{}{s.m.}{Natural ou habitante da Abissínia. (\textit{Era um engenheiro abissínio, nascido, criado e estudado na Abissínia.})}{a.bis.sí.nio}{0}
\verb{abiu}{}{}{}{}{s.m.}{Fruto do abieiro, comestível, amarelo e doce; abio.}{a.biu}{0}
\verb{abjeção}{}{}{"-ões}{}{s.f.}{Grau extremo de aviltamento, de vilania, de baixeza.}{ab.je.ção}{0}
\verb{abjeto}{é}{}{}{}{adj.}{Vil, desprezível, infame. (\textit{Ele teve uma atitude abjeta, que nem se deve comentar.})}{ab.je.to}{0}
\verb{abjurar}{}{}{}{}{v.t.}{Renegar, renunciar a crença ou doutrina.}{ab.ju.rar}{0}
\verb{abjurar}{}{}{}{}{}{Retratar"-se, desdizer"-se.}{ab.ju.rar}{\verboinum{1}}
\verb{ablação}{}{}{"-ões}{}{s.f.}{Ato de cortar, tirar à força.}{a.bla.ção}{0}
\verb{ablação}{}{Med.}{"-ões}{}{}{Extração cirúrgica de parte de órgão ou parte do corpo.}{a.bla.ção}{0}
\verb{ablativo}{}{}{}{}{adj.}{Que corta, tira.}{a.bla.ti.vo}{0}
\verb{ablativo}{}{Gram.}{}{}{s.m.}{Um dos casos sintáticos morfologicamente marcados de algumas línguas, como o latim.}{a.bla.ti.vo}{0}
\verb{ablução}{}{}{"-ões}{}{s.f.}{Ato de lavar o corpo ou parte dele.}{a.blu.ção}{0}
\verb{ablução}{}{}{"-ões}{}{}{Ritual religioso de purificação por meio da água.}{a.blu.ção}{0}
\verb{abluir}{}{}{}{}{v.t.}{Lavar; limpar. (\textit{Pilatos abluiu as mãos.})}{a.blu.ir}{\verboinum{26}}
\verb{ablutomania}{}{Med.}{}{}{s.f.}{Impulso doentio para lavar"-se.}{a.blu.to.ma.ni.a}{0}
\verb{abnegação}{}{}{"-ões}{}{s.f.}{Ato ou efeito de abnegar.}{ab.ne.ga.ção}{0}
\verb{abnegação}{}{}{"-ões}{}{}{Renúncia, desprendimento, dedicação.}{ab.ne.ga.ção}{0}
\verb{abnegado}{}{}{}{}{adj.}{Que revela abnegação.}{ab.ne.ga.do}{0}
\verb{abnegado}{}{}{}{}{}{Devotado, dedicado.}{ab.ne.ga.do}{0}
\verb{abnegar}{}{}{}{}{v.t.}{Renunciar, abster"-se.}{ab.ne.gar}{0}
\verb{abnegar}{}{}{}{}{v.pron.}{Sacrificar"-se em benefício do próximo ou em nome de uma causa ou religião.}{ab.ne.gar}{\verboinum{5}}
\verb{abóbada}{}{}{}{}{s.f.}{Construção em forma de arco.}{a.bó.ba.da}{0}
\verb{abóbada}{}{}{}{}{}{Cobertura abaulada; cúpula.}{a.bó.ba.da}{0}
\verb{abobadado}{}{}{}{}{adj.}{Que tem forma de abóbada.}{a.bo.ba.da.do}{0}
\verb{abobadar}{}{}{}{}{v.t.}{Construir abóbada.}{a.bo.ba.dar}{0}
\verb{abobadar}{}{}{}{}{}{Cobrir com abóbada.}{a.bo.ba.dar}{\verboinum{1}}
\verb{abobalhado}{}{}{}{}{adj.}{Bobo, tolo, palerma.}{a.bo.ba.lha.do}{0}
\verb{abobar}{}{}{}{}{v.t.}{Tornar bobo, apalermar, causar espanto. (\textit{Ele abobava qualquer discussão, com suas opiniões.})}{a.bo.bar}{0}
\verb{abobar}{}{}{}{}{}{Espantar"-se. (\textit{Ele se abobou com a beleza da moça.})}{a.bo.bar}{\verboinum{1}}
\verb{abóbora}{}{}{}{}{s.f.}{Fruto da aboboreira, de forma arredondada ou alongada, casca grossa, sementes pequenas e achatadas e polpa alaranjada, usado no preparo de pratos doces e salgados; jerimum.}{a.bó.bo.ra}{0}
\verb{abóbora}{}{}{}{}{}{A planta que dá a abóbora, aboboreira.}{a.bó.bo.ra}{0}
\verb{abóbora}{}{}{}{}{s.m.}{A cor da abóbora.}{a.bó.bo.ra}{0}
\verb{aboboral}{}{}{"-ais}{}{s.m.}{Coletivo de aboboreira, plantação de abóboras.}{a.bo.bo.ral}{0}
\verb{aboboreira}{ê}{Bot.}{}{}{s.f.}{Planta cujo fruto é a abóbora; jerimunzeiro.}{a.bo.bo.rei.ra}{0}
\verb{abobrinha}{}{}{}{}{s.f.}{Tipo de abóbora pequena e de forma alongada, colhida antes de amadurecer e usada no preparo de pratos diversos.}{a.bo.bri.nha}{0}
\verb{abobrinha}{}{Pop.}{}{}{}{Bobagem, asneira.}{a.bo.bri.nha}{0}
\verb{abocanhar}{}{}{}{}{v.t.}{Apanhar com a boca, morder, rasgar com os dentes.}{a.bo.ca.nhar}{0}
\verb{abocanhar}{}{Pop.}{}{}{}{Apossar"-se de algo ilegalmente ou com oportunismo.}{a.bo.ca.nhar}{\verboinum{1}}
\verb{aboiar}{}{}{}{}{v.i.}{Boiar, manter"-se à superfície da água.}{a.boi.ar}{\verboinum{1}}
\verb{aboiar}{}{}{}{}{v.i.}{Conduzir o gado cantando.}{a.boi.ar}{\verboinum{1}}
\verb{aboio}{ô}{}{}{}{s.m.}{Canto que os vaqueiros entoam para conduzir o gado.}{a.boi.o}{0}
\verb{aboletado}{}{}{}{}{adj.}{Que se aboletou; alojado, acomodado.}{a.bo.le.ta.do}{0}
\verb{aboletar}{}{}{}{}{v.t.}{Acomodar alguém em algum lugar, alojar.}{a.bo.le.tar}{\verboinum{1}}
\verb{abolição}{}{}{"-ões}{}{s.f.}{Ato ou efeito de abolir, tirar de uso ou extinguir.}{a.bo.li.ção}{0}
\verb{abolição}{}{Hist.}{"-ões}{}{}{Abolição da escravatura.}{a.bo.li.ção}{0}
\verb{abolicionismo}{}{Hist.}{}{}{s.m.}{Conjunto de ideias que pregava a extinção da escravidão.}{a.bo.li.ci.o.nis.mo}{0}
\verb{abolicionista}{}{Hist.}{}{}{adj.2g.}{Relativo ao abolicionismo.}{a.bo.li.ci.o.nis.ta}{0}
\verb{abolicionista}{}{}{}{}{s.2g.}{Indivíduo que prega ou segue o abolicionismo.}{a.bo.li.ci.o.nis.ta}{0}
\verb{abolido}{}{}{}{}{adj.}{Que está fora de uso, extinto, anulado.}{a.bo.li.do}{0}
\verb{abolir}{}{}{}{}{v.t.}{Deixar de usar, fazer cessar, extinguir.}{a.bo.lir}{\verboinum{34}\verboirregular{\emph{def.} abolimos, abolis}}
\verb{abominação}{}{}{"-ões}{}{s.f.}{Ato ou efeito de abominar; aversão, repulsa.}{a.bo.mi.na.ção}{0}
\verb{abominar}{}{}{}{}{v.t.}{Ter aversão a, sentir horror de.}{a.bo.mi.nar}{\verboinum{1}}
\verb{abominável}{}{}{"-eis}{}{adj.2g.}{Que é odioso, detestável.}{a.bo.mi.ná.vel}{0}
\verb{abonação}{}{}{"-ões}{}{s.f.}{Ato ou efeito de abonar; endosso, aprovação.}{a.bo.na.ção}{0}
\verb{abonado}{}{}{}{}{adj.}{Que recebeu abono.}{a.bo.na.do}{0}
\verb{abonado}{}{}{}{}{}{Que tem muito dinheiro.}{a.bo.na.do}{0}
\verb{abonador}{ô}{}{}{}{adj.}{Que abona, confirma, endossa.}{a.bo.na.dor}{0}
\verb{abonar}{}{}{}{}{v.t.}{Apresentar algo como bom; endossar, aprovar, confirmar.}{a.bo.nar}{0}
\verb{abonar}{}{}{}{}{}{Justificar ou perdoar faltas (ao trabalho ou a outros deveres).}{a.bo.nar}{\verboinum{1}}
\verb{abono}{}{}{}{}{s.m.}{Ato de abonar; abonação.}{a.bo.no}{0}
\verb{abono}{}{}{}{}{}{Remuneração adicional ao ordenado.}{a.bo.no}{0}
\verb{abono}{}{}{}{}{}{Ato de relevar faltas ao trabalho, não as descontando no salário.}{a.bo.no}{0}
\verb{abordagem}{}{}{"-ens}{}{s.f.}{Ato ou efeito de abordar.}{a.bor.da.gem}{0}
\verb{abordagem}{}{}{"-ens}{}{}{Aproximação de uma embarcação com finalidade de assaltá"-la.}{a.bor.da.gem}{0}
\verb{abordagem}{}{}{"-ens}{}{}{Aproximação.}{a.bor.da.gem}{0}
\verb{abordagem}{}{}{"-ens}{}{}{Ponto de vista sobre um assunto; maneira de enfocar ou interpretar algo.}{a.bor.da.gem}{0}
\verb{abordar}{}{}{}{}{v.t.}{Aproximar"-se de algo ou alguém com o propósito de fazer contato; encostar em, atingir.}{a.bor.dar}{0}
\verb{abordar}{}{}{}{}{}{Discutir sobre um assunto ou tema.}{a.bor.dar}{\verboinum{1}}
\verb{aborrecer}{ê}{}{}{}{v.t.}{Deixar alguém aborrecido, irritado.}{a.bor.re.cer}{0}
\verb{aborrecer}{ê}{}{}{}{v.pron.}{Ficar insatisfeito, descontente, enfastiado.}{a.bor.re.cer}{\verboinum{15}}
\verb{aborrecido}{}{}{}{}{adj.}{Que causa aborrecimento, tédio, enfado.}{a.bor.re.ci.do}{0}
\verb{aborrecido}{}{}{}{}{}{Chateado, entediado, contrariado.}{a.bor.re.ci.do}{0}
\verb{aborrecimento}{}{}{}{}{s.m.}{Ato ou efeito de aborrecer"-se.}{a.bor.re.ci.men.to}{0}
\verb{aborrecimento}{}{}{}{}{}{Chateação, contrariedade, tédio.}{a.bor.re.ci.men.to}{0}
\verb{abortado}{}{}{}{}{adj.}{Que resultou de aborto, não vingou ou não chegou a nascer.}{a.bor.ta.do}{0}
\verb{abortado}{}{}{}{}{}{Frustrado, interrompido.}{a.bor.ta.do}{0}
\verb{abortar}{}{}{}{}{v.i.}{Não se desenvolver, não ter êxito, fracassar.}{a.bor.tar}{0}
\verb{abortar}{}{Med.}{}{}{}{Interromper a gestação, expulsando o feto ou embrião do útero. }{a.bor.tar}{0}
\verb{abortar}{}{}{}{}{v.t.}{Interromper, frustrar, malograr.}{a.bor.tar}{\verboinum{1}}
\verb{aborteiro}{ê}{}{}{}{adj.}{Que pratica o aborto, sendo médico ou não.}{a.bor.tei.ro}{0}
\verb{aborteiro}{ê}{}{}{}{}{No feminino, diz"-se daquela que, não sendo médica, pratica o aborto com ou sem o consentimento da gestante.}{a.bor.tei.ro}{0}
\verb{abortício}{}{}{}{}{adj.}{Nascido por aborto.}{a.bor.tí.cio}{0}
\verb{abortivo}{}{}{}{}{adj.}{Relativo a aborto.}{a.bor.ti.vo}{0}
\verb{abortivo}{}{}{}{}{s.m.}{Substância que faz abortar.}{a.bor.ti.vo}{0}
\verb{aborto}{ô}{}{}{}{s.m.}{Ato ou efeito de abortar.}{a.bor.to}{0}
\verb{aborto}{ô}{Med.}{}{}{}{Interrupção da gestação.}{a.bor.to}{0}
\verb{aborto}{ô}{}{}{}{}{Fracasso, insucesso.}{a.bor.to}{0}
\verb{aborto}{ô}{Fig.}{}{}{}{Anomalia, aberração.}{a.bor.to}{0}
\verb{abotoadeira}{ê}{}{}{}{s.f.}{Objeto que serve para abotoar, abotoadura.}{a.bo.to.a.dei.ra}{0}
\verb{abotoadeira}{ê}{}{}{}{}{A casa em que se prende o botão.}{a.bo.to.a.dei.ra}{0}
\verb{abotoadura}{}{}{}{}{s.f.}{Ato ou efeito de abotoar.}{a.bo.to.a.du.ra}{0}
\verb{abotoadura}{}{}{}{}{}{Botão ou conjunto de botões removíveis usados em peças de vestuário.}{a.bo.to.a.du.ra}{0}
\verb{abotoar}{}{}{}{}{v.t.}{Prender com botões.}{a.bo.to.ar}{0}
\verb{abotoar}{}{}{}{}{v.i.}{Formar"-se como botão de flor.}{a.bo.to.ar}{0}
\verb{abotoar}{}{Fig.}{}{}{}{Morrer.}{a.bo.to.ar}{0}
\verb{abotoar}{}{Fig.}{}{}{v.t.}{Segurar pela gola, com violência.}{a.bo.to.ar}{\verboinum{7}}
\verb{abra}{}{}{}{}{s.f.}{Baía, enseada, ancoradouro.}{a.bra}{0}
\verb{abracadabra}{}{}{}{}{s.m.}{Palavra a que se atribuem poderes mágicos.}{a.bra.ca.da.bra}{0}
\verb{abraçadeira}{ê}{}{}{}{s.f.}{Peça usada para envolver e prender objetos; braçadeira.}{a.bra.ça.dei.ra}{0}
\verb{abraçar}{}{}{}{}{v.t.}{Envolver com os braços.}{a.bra.çar}{0}
\verb{abraçar}{}{}{}{}{}{Envolver, circundar.}{a.bra.çar}{0}
\verb{abraçar}{}{Por ext.}{}{}{}{Adotar, seguir uma ideia, uma religião.}{a.bra.çar}{\verboinum{3}}
\verb{abraço}{}{}{}{}{s.m.}{Ato ou efeito de abraçar; enlace com os braços.}{a.bra.ço}{0}
\verb{abraço}{}{Fig.}{}{}{}{Expressão de afeto.}{a.bra.ço}{0}
\verb{abrandado}{}{}{}{}{adj.}{Que se tornou brando, calmo, ameno. }{a.bran.da.do}{0}
\verb{abrandamento}{}{}{}{}{s.m.}{Ato ou efeito de abrandar.}{a.bran.da.men.to}{0}
\verb{abrandamento}{}{Gram.}{}{}{}{Mudança de uma consoante surda para sonora.}{a.bran.da.men.to}{0}
\verb{abrandar}{}{}{}{}{v.t.}{Tornar algo ou alguém brando; amolecer. }{a.bran.dar}{0}
\verb{abrandar}{}{}{}{}{}{Acalmar}{a.bran.dar}{\verboinum{1}}
\verb{abrangente}{}{}{}{}{adj.2g.}{Que abrange, que tem grande alcance.}{a.bran.gen.te}{0}
\verb{abranger}{ê}{}{}{}{v.t.}{Conter em si, compreender, abraçar, abarcar.}{a.bran.ger}{0}
\verb{abranger}{ê}{}{}{}{}{Estender"-se por uma área delimitada, atingir.}{a.bran.ger}{\verboinum{16}}
\verb{abrangido}{}{}{}{}{adj.}{Que está contido, abarcado, compreendido.}{a.bran.gi.do}{0}
\verb{abrasado}{}{}{}{}{adj.}{Que está queimado, em brasa.}{a.bra.sa.do}{0}
\verb{abrasado}{}{}{}{}{}{Quente, ardente.}{a.bra.sa.do}{0}
\verb{abrasado}{}{}{}{}{}{Que tem cor de brasa, vermelho.}{a.bra.sa.do}{0}
\verb{abrasado}{}{Fig.}{}{}{}{Animado, excitado.}{a.bra.sa.do}{0}
\verb{abrasador}{ô}{}{}{}{adj.}{Que abrasa, queima.}{a.bra.sa.dor}{0}
\verb{abrasador}{ô}{}{}{}{}{Muito quente, de calor intenso.}{a.bra.sa.dor}{0}
\verb{abrasador}{ô}{Fig.}{}{}{}{Aflitivo.}{a.bra.sa.dor}{0}
\verb{abrasador}{ô}{}{}{}{s.m.}{Algo ou alguém que abrasa.}{a.bra.sa.dor}{0}
\verb{abrasão}{}{}{"-ões}{}{s.f.}{Raspagem, esfolamento.}{a.bra.são}{0}
\verb{abrasão}{}{}{"-ões}{}{}{Destruição causada pelo atrito.}{a.bra.são}{0}
\verb{abrasão}{}{}{"-ões}{}{}{Erosão provocada por agentes como o vento e as ondas do mar.}{a.bra.são}{0}
\verb{abrasão}{}{Med.}{"-ões}{}{}{Desgaste orgânico provocado por ação mecânica, na pele ou mucosa.}{a.bra.são}{0}
\verb{abrasar}{}{}{}{}{v.t.}{Tornar em brasa, incandescer, esquentar muito.}{a.bra.sar}{0}
\verb{abrasar}{}{}{}{}{v.i.}{Produzir calor excessivo.}{a.bra.sar}{0}
\verb{abrasar}{}{Fig.}{}{}{v.pron.}{Entusiasmar"-se, exaltar"-se.}{a.bra.sar}{0}
\verb{abrasar}{}{Fig.}{}{}{}{Ficar da cor da brasa, avermelhar"-se.}{a.bra.sar}{\verboinum{1}}
\verb{abrasileirado}{}{}{}{}{adj.}{Que tem características próprias de brasileiro.}{a.bra.si.lei.ra.do}{0}
\verb{abrasileirar}{}{}{}{}{v.t.}{Dar características próprias de brasileiro.}{a.bra.si.lei.rar}{\verboinum{1}}
\verb{abrasivo}{}{}{}{}{adj.}{Que produz abrasão.}{a.bra.si.vo}{0}
\verb{abrasivo}{}{}{}{}{s.m.}{Designação de substância que produz abrasão.}{a.bra.si.vo}{0}
\verb{abre"-alas}{}{}{}{}{s.m.}{Grupo que abre o desfile de carnaval.}{a.bre"-a.las}{0}
\verb{abrejeirado}{}{}{}{}{adj.}{Que tem aspecto ou modos brejeiros, maliciosos.}{a.bre.jei.ra.do}{0}
\verb{abreugrafia}{}{Med.}{}{}{s.f.}{Radiografia dos pulmões, em tamanho reduzido, para o diagnóstico de tuberculose.}{a.breu.gra.fi.a}{0}
\verb{abreviado}{}{}{}{}{adj.}{Tornado breve; resumido; encurtado.}{a.bre.vi.a.do}{0}
\verb{abreviar}{}{}{}{}{v.t.}{Tornar breve, encurtar.}{a.bre.vi.ar}{0}
\verb{abreviar}{}{}{}{}{}{Diminuir, resumir, reduzir.}{a.bre.vi.ar}{\verboinum{1}}
\verb{abreviatura}{}{}{}{}{s.f.}{Redução de uma palavra ou locução a algumas de suas letras ou sílabas.}{a.bre.vi.a.tu.ra}{0}
\verb{abrideira}{ê}{Bras.}{}{}{s.f.}{Máquina usada nas indústrias de fiação.}{a.bri.dei.ra}{0}
\verb{abrideira}{ê}{Pop.}{}{}{}{Bebida alcoólica que se toma como aperitivo.}{a.bri.dei.ra}{0}
\verb{abridor}{ô}{}{}{}{s.m.}{Instrumento que se usa para abrir.}{a.bri.dor}{0}
\verb{abridor}{ô}{}{}{}{}{Instrumento usado para abrir latas ou garrafas.}{a.bri.dor}{0}
\verb{abrigado}{}{}{}{}{adj.}{Que se abrigou; protegido; acolhido.}{a.bri.ga.do}{0}
\verb{abrigado}{}{}{}{}{s.m.}{Lugar que abriga; abrigo.}{a.bri.ga.do}{0}
\verb{abrigar}{}{}{}{}{v.t.}{Dar abrigo a; proteger; acolher.}{a.bri.gar}{\verboinum{5}}
\verb{abrigo}{}{}{}{}{s.m.}{Lugar que abriga, que oferece proteção.}{a.bri.go}{0}
\verb{abrigo}{}{Fig.}{}{}{}{Proteção, amparo, acolhimento.}{a.bri.go}{0}
\verb{abril}{}{}{}{}{s.m.}{O quarto mês do ano.}{a.bril}{0}
\verb{abril}{}{Fig.}{}{}{}{Mocidade, juventude, viço.}{a.bril}{0}
\verb{abrilhantar}{}{}{}{}{v.t.}{Dar brilho a; tornar brilhante, reluzente.}{a.bri.lhan.tar}{0}
\verb{abrilhantar}{}{Fig.}{}{}{}{Dar destaque a; realçar}{a.bri.lhan.tar}{\verboinum{1}}
\verb{abrimento}{}{}{}{}{s.m.}{Ato ou efeito de abrir; abertura.}{a.bri.men.to}{0}
\verb{abrir}{}{}{}{}{v.t.}{Permitir o acesso ao interior de um objeto ou de um local, mediante a retirada de um obstáculo.}{a.brir}{0}
\verb{abrir}{}{}{}{}{}{Separar uma parte da outra; afastar.}{a.brir}{0}
\verb{abrir}{}{}{}{}{}{Fazer uma incisão; cortar.}{a.brir}{0}
\verb{abrir}{}{}{}{}{}{Desabotoar; desabrochar.}{a.brir}{0}
\verb{abrir}{}{}{}{}{}{Dar início a uma cerimônia; inaugurar.}{a.brir}{0}
\verb{abrir}{}{}{}{}{}{Estender; estirar.}{a.brir}{0}
\verb{abrir}{}{}{}{}{v.pron.}{Fazer confidências; desabafar.}{a.brir}{\verboinum{18}}
\verb{ab"-rogar}{}{}{}{}{v.t.}{Fazer cessar a existência ou obrigatoriedade; tirar de uso; anular; revogar.}{ab"-ro.gar}{\verboinum{5}}
\verb{abrolho}{ô}{}{}{}{s.m.}{Designação de diversas plantas rasteiras e espinhosas.}{a.bro.lho}{0}
\verb{abrolho}{ô}{}{}{}{}{O espinho ou a ponta dessas plantas.}{a.bro.lho}{0}
\verb{abrolho}{ô}{}{}{}{}{Nome de qualquer estrepe; espinho.}{a.bro.lho}{0}
\verb{abrolho}{ô}{}{}{}{}{Rochedo marinho que oferece risco para embarcações.}{a.bro.lho}{0}
\verb{abrolhos}{ó}{}{}{}{s.m.pl.}{Nome dado aos rochedos que atingem a superfície marinha.}{a.bro.lhos}{0}
\verb{abrolhos}{ó}{Desus.}{}{}{}{Amarguras; dificuldades; mágoas.}{a.bro.lhos}{0}
\verb{abrutado}{}{}{}{}{adj.}{Que é abrutalhado; rude; grosseiro.}{a.bru.ta.do}{0}
\verb{abrutalhado}{}{}{}{}{adj.}{Que tem modos ou semelhanças de bruto; grosseiro; rude.}{a.bru.ta.lha.do}{0}
\verb{abrutalhar}{}{}{}{}{v.i.}{Tornar bruto, grosseiro, rude. }{a.bru.ta.lhar}{\verboinum{1}}
\verb{abscesso}{é}{Med.}{}{}{s.m.}{Acúmulo de pus numa cavidade do corpo causado por inflamação ou por germes piogênicos; tumor.}{abs.ces.so}{0}
\verb{abscissa}{}{Geom.}{}{}{s.f.}{Coordenada \textit{x} do sistema cartesiano que define a posição de um ponto sobre um plano ou espaço, ou  estabelece a posição de um ponto sobre uma linha. }{abs.cis.sa}{0}
\verb{absenteísmo}{}{}{}{}{s.m.}{Falta de comparecimento premeditada.}{ab.sen.te.ís.mo}{0}
\verb{absenteísmo}{}{}{}{}{}{Sistema de exploração da terra, em que o proprietário confia a exploração desta a intermediários.}{ab.sen.te.ís.mo}{0}
\verb{absenteísta}{}{}{}{}{adj.2g.}{Relativo a absenteísmo.}{ab.sen.te.ís.ta}{0}
\verb{absenteísta}{}{}{}{}{s.2g.}{Indivíduo que pratica o absenteísmo.}{ab.sen.te.ís.ta}{0}
\verb{abside}{}{}{}{}{s.f.}{Local onde se encontra o altar"-mor nas igrejas.}{ab.si.de}{0}
\verb{absinto}{}{Bot.}{}{}{s.m.}{Nome de uma erva amarga, aromatizada e de essência tóxica.}{ab.sin.to}{0}
\verb{absinto}{}{}{}{}{}{Bebida alcoólica preparada com as folhas dessa erva.}{ab.sin.to}{0}
\verb{absinto}{}{Fig.}{}{}{}{Mágoa; amargura.}{ab.sin.to}{0}
\verb{absolutamente}{}{}{}{}{adv.}{De modo absoluto; completamente; inteiramente.}{ab.so.lu.ta.men.te}{0}
\verb{absolutamente}{}{}{}{}{}{De modo algum; de jeito nenhum.}{ab.so.lu.ta.men.te}{0}
\verb{absolutamente}{}{}{}{}{}{Certamente que sim; sem dúvida que sim.}{ab.so.lu.ta.men.te}{0}
\verb{absolutismo}{}{}{}{}{s.m.}{Sistema político de governo, no qual o governante tem poderes absolutos, ilimitados.}{ab.so.lu.tis.mo}{0}
\verb{absolutismo}{}{}{}{}{}{Qualquer forma de tirania ou despotismo.}{ab.so.lu.tis.mo}{0}
\verb{absolutista}{}{}{}{}{adj.2g.}{Relativo ao absolutismo.}{ab.so.lu.tis.ta}{0}
\verb{absolutista}{}{}{}{}{s.2g.}{Indivíduo que segue o absolutismo.}{ab.so.lu.tis.ta}{0}
\verb{absoluto}{}{}{}{}{adj.}{Que é sem limites ou condições; irrestrito; total; pleno.}{ab.so.lu.to}{0}
\verb{absoluto}{}{}{}{}{}{Que é superior a todos os outros; único; supremo.}{ab.so.lu.to}{0}
\verb{absolver}{ê}{}{}{}{v.t.}{Inocentar, declarar sem pecado ou sem culpa.}{ab.sol.ver}{0}
\verb{absolver}{ê}{}{}{}{}{Perdoar.}{ab.sol.ver}{\verboinum{12}\verboirregular{absolvo, absolves}}
\verb{absolvição}{}{Jur.}{"-ões}{}{s.f.}{Ato ou efeito de absolver, inocentar.}{ab.sol.vi.ção}{0}
\verb{absolvição}{}{Relig.}{"-ões}{}{}{Perdão dos pecados.}{ab.sol.vi.ção}{0}
\verb{absolvido}{}{}{}{}{adj.}{Que recebeu absolvição.}{ab.sol.vi.do}{0}
\verb{absolvido}{}{}{}{}{}{Inocentado.}{ab.sol.vi.do}{0}
\verb{absolvido}{}{}{}{}{}{Perdoado.}{ab.sol.vi.do}{0}
\verb{absorção}{}{}{"-ões}{}{s.f.}{Ato ou efeito de absorver.}{ab.sor.ção}{0}
\verb{absorção}{}{}{"-ões}{}{}{Assimilação.}{ab.sor.ção}{0}
\verb{absorção}{}{Biol.}{"-ões}{}{}{Penetração de substâncias pelo organismo.}{ab.sor.ção}{0}
\verb{absorto}{ô}{}{}{}{adj.}{Que está alheio; pensativo, distante, concentrado.}{ab.sor.to}{0}
\verb{absorto}{ô}{}{}{}{}{Extasiado, arrebatado, contemplativo.}{ab.sor.to}{0}
\verb{absorver}{ê}{}{}{}{v.t.}{Embeber, sorver em si.}{ab.sor.ver}{0}
\verb{absorver}{ê}{}{}{}{}{Aspirar, engolir.}{ab.sor.ver}{0}
\verb{absorver}{ê}{}{}{}{}{Preocupar, requerer a atenção.}{ab.sor.ver}{0}
\verb{absorver}{ê}{}{}{}{v.pron.}{Concentrar"-se, aplicar"-se.}{ab.sor.ver}{\verboinum{12}\verboirregular{absorvo, absorves}}
\verb{absorvido}{}{}{}{}{adj.}{Que se absorveu.}{ab.sor.vi.do}{0}
\verb{absorvido}{}{}{}{}{}{Aspirado, sorvido.}{ab.sor.vi.do}{0}
\verb{absorvido}{}{}{}{}{}{Absorto, concentrado.}{ab.sor.vi.do}{0}
\verb{abstêmio}{}{}{}{}{adj.}{Que se abstém de bebidas alcoólicas; sóbrio, moderado.}{abs.tê.mio}{0}
\verb{abstenção}{}{}{"-ões}{}{s.f.}{Ato ou efeito de abster"-se.}{abs.ten.ção}{0}
\verb{abstenção}{}{}{"-ões}{}{}{Recusa voluntária de participar de um ato, como em uma votação.}{abs.ten.ção}{0}
\verb{abstencionismo}{}{}{}{}{s.m.}{Prática de se abster do exercício do voto;  abstenção.}{abs.ten.ci.o.nis.mo}{0}
\verb{abster"-se}{}{}{}{}{v.pron.}{Abdicar ou privar"-se.}{abs.ter"-se}{0}
\verb{absterse}{}{}{}{}{}{Conter"-se, deter"-se.}{abs.ter"-se}{\verboinum{39}}
\verb{abstinência}{}{}{}{}{s.f.}{Ato ou efeito de se abster.}{abs.ti.nên.cia}{0}
\verb{abstinência}{}{}{}{}{}{Privação forçada ou voluntária de algo; jejum.}{abs.ti.nên.cia}{0}
\verb{abstinente}{}{}{}{}{adj.2g.}{Que pratica abstinência.}{abs.ti.nen.te}{0}
\verb{abstinente}{}{}{}{}{}{Moderado, sóbrio.}{abs.ti.nen.te}{0}
\verb{abstração}{}{}{"-ões}{}{s.f.}{Ato ou efeito de abstrair.}{abs.tra.ção}{0}
\verb{abstração}{}{}{"-ões}{}{}{Separação mental de elementos; conceituação.}{abs.tra.ção}{0}
\verb{abstração}{}{}{"-ões}{}{}{Alheamento, devaneio.}{abs.tra.ção}{0}
\verb{abstracionismo}{}{Art.}{}{}{s.m.}{Corrente estética surgida no início do século \textsc{xx}, cuja característica determinante é o abandono da representação figurativa das imagens.}{abs.tra.ci.o.nis.mo}{0}
\verb{abstracionista}{}{}{}{}{adj.2g.}{Relativo a abstracionismo.}{abs.tra.ci.o.nis.ta}{0}
\verb{abstracionista}{}{}{}{}{s.2g.}{Indivíduo que segue o abstracionismo.}{abs.tra.ci.o.nis.ta}{0}
\verb{abstraído}{}{}{}{}{adj.}{Que se abstraiu.}{abs.tra.í.do}{0}
\verb{abstraído}{}{}{}{}{}{Absorto, distraído, alheio.}{abs.tra.í.do}{0}
\verb{abstrair}{}{}{}{}{v.t.}{Considerar separadamente.}{abs.tra.ir}{0}
\verb{abstrair}{}{}{}{}{}{Separar, apartar, alhear.}{abs.tra.ir}{0}
\verb{abstrair}{}{}{}{}{v.pron.}{Concentrar"-se, absorver"-se.}{abs.tra.ir}{\verboinum{19}}
\verb{abstrato}{}{}{}{}{adj.}{Que resulta de uma abstração.}{abs.tra.to}{0}
\verb{abstrato}{}{}{}{}{}{Abstraído, desatento.}{abs.tra.to}{0}
\verb{abstrato}{}{Gram.}{}{}{}{Que designa qualidade, ação ou estado separado do objeto a que se refere.}{abs.tra.to}{0}
\verb{abstrato}{}{Art.}{}{}{}{Relativo a abstracionismo; arte abstrata.}{abs.tra.to}{0}
\verb{abstruso}{}{}{}{}{adj.}{Que não se compreende; confuso, oculto.}{abs.tru.so}{0}
\verb{absurdez}{ê}{}{}{}{s.f.}{Aquilo que é absurdo.}{ab.sur.dez}{0}
\verb{absurdo}{}{}{}{}{adj.}{Que contraria o bom"-senso, a razão.}{ab.sur.do}{0}
\verb{absurdo}{}{}{}{}{}{Tolo, incoerente.}{ab.sur.do}{0}
\verb{absurdo}{}{}{}{}{s.m.}{Fatos, ideias ou opiniões que fogem às leis da razão ou do bom"-senso; disparate.}{ab.sur.do}{0}
\verb{abulia}{}{Med.}{}{}{s.f.}{Falta ou perda de vontade, por motivo de doença; apatia.}{a.bu.li.a}{0}
\verb{abúlico}{}{}{}{}{adj.}{Que tem abulia; apático.}{a.bú.li.co}{0}
\verb{abundância}{}{}{}{}{s.f.}{Fartura, profusão, grande quantidade.}{a.bun.dân.cia}{0}
\verb{abundante}{}{}{}{}{adj.2g.}{Que existe em abundância; farto.}{a.bun.dan.te}{0}
\verb{abundante}{}{Gram.}{}{}{}{Verbo que possui mais de uma forma.}{a.bun.dan.te}{0}
\verb{abundar}{}{}{}{}{v.i.}{Existir em abundância, em grande quantidade.}{a.bun.dar}{\verboinum{1}}
\verb{aburguesado}{}{}{}{}{adj.}{Que tem hábitos ou costumes próprios de burguês.}{a.bur.gue.sa.do}{0}
\verb{aburguesar}{}{}{}{}{v.t.}{Dar ou adquirir hábitos ou costumes de burguês.}{a.bur.gue.sar}{\verboinum{1}}
\verb{abusado}{}{}{}{}{adj.}{Que abusa, que excede.}{a.bu.sa.do}{0}
\verb{abusado}{}{}{}{}{}{Atrevido, intrometido.}{a.bu.sa.do}{0}
\verb{abusado}{}{}{}{}{}{Aborrecido, enfadonho.}{a.bu.sa.do}{0}
\verb{abusão}{}{}{"-ões}{}{s.f.}{Abuso.}{a.bu.são}{0}
\verb{abusão}{}{}{"-ões}{}{}{Engano, erro.}{a.bu.são}{0}
\verb{abusão}{}{}{"-ões}{}{}{Superstição, crendice.}{a.bu.são}{0}
\verb{abusar}{}{}{}{}{v.t.}{Usar com exagero.}{a.bu.sar}{0}
\verb{abusar}{}{}{}{}{}{Usar mal. }{a.bu.sar}{0}
\verb{abusar}{}{}{}{}{}{Aproveitar"-se, valer"-se.}{a.bu.sar}{0}
\verb{abusar}{}{}{}{}{}{Violentar, estuprar.}{a.bu.sar}{0}
\verb{abusar}{}{}{}{}{v.i.}{Cometer excessos.}{a.bu.sar}{\verboinum{1}}
\verb{abusivo}{}{}{}{}{adj.}{Em que há abuso, exagero.}{a.bu.si.vo}{0}
\verb{abusivo}{}{}{}{}{}{Impróprio.}{a.bu.si.vo}{0}
\verb{abuso}{}{}{}{}{s.m.}{Uso exagerado, impróprio.}{a.bu.so}{0}
\verb{abuso}{}{}{}{}{}{Ultraje ao pudor.}{a.bu.so}{0}
\verb{abuso}{}{}{}{}{}{Excesso de atribuições.}{a.bu.so}{0}
\verb{abuso}{}{}{}{}{}{Desaforo, aborrecimento.}{a.bu.so}{0}
\verb{abutre}{}{Zool.}{}{}{s.m.}{Ave de rapina originária da Europa, Ásia e África, que se alimenta com carniça.}{a.bu.tre}{0}
\verb{abutre}{}{Fig.}{}{}{}{Indivíduo sovina, usurário.}{a.bu.tre}{0}
\verb{Ac}{}{Quím.}{}{}{}{Símb. do \textit{actínio}.}{Ac}{0}
\verb{aC}{}{}{}{}{}{Abrev. de \textit{antes de Cristo}.}{a.C.}{0}
\verb{AC}{}{}{}{}{}{Sigla do estado do Acre.}{AC}{0}
\verb{aca}{}{}{}{}{s.f.}{Fedor, mau cheiro, fetidez.}{a.ca}{0}
\verb{aça}{}{Desus.}{}{}{adj.}{Que tem albinismo; albino.}{a.ça}{0}
\verb{acabado}{}{}{}{}{adj.}{Que foi concluído, terminado.}{a.ca.ba.do}{0}
\verb{acabado}{}{}{}{}{}{Perfeito, notável.}{a.ca.ba.do}{0}
\verb{acabado}{}{}{}{}{}{Gasto, deteriorado.}{a.ca.ba.do}{0}
\verb{acabado}{}{}{}{}{}{Envelhecido, abatido.}{a.ca.ba.do}{0}
\verb{acabamento}{}{}{}{}{s.m.}{Ato ou efeito de acabar.}{a.ca.ba.men.to}{0}
\verb{acabamento}{}{}{}{}{}{Conclusão, remate.}{a.ca.ba.men.to}{0}
\verb{acabamento}{}{}{}{}{}{Aperfeiçoamento de uma obra.}{a.ca.ba.men.to}{0}
\verb{acabanado}{}{}{}{}{adj.}{Que tem forma de cabana.}{a.ca.ba.na.do}{0}
\verb{acabanado}{}{}{}{}{}{Pendente, caído.}{a.ca.ba.na.do}{0}
\verb{acabanado}{}{}{}{}{s.m.}{Animal que possui orelhas ou chifres caídos.}{a.ca.ba.na.do}{0}
\verb{acabar}{}{}{}{}{v.t.}{Levar a cabo, terminar, completar.}{a.ca.bar}{0}
\verb{acabar}{}{}{}{}{}{Dar cabo, destruir, matar.}{a.ca.bar}{0}
\verb{acabar}{}{}{}{}{}{Aperfeiçoar, dar acabamento.}{a.ca.bar}{0}
\verb{acabar}{}{}{}{}{v.i.}{Findar, chegar ao termo, cessar.}{a.ca.bar}{0}
\verb{acabar}{}{}{}{}{v.pron.}{Consumir"-se, esgotar"-se.}{a.ca.bar}{\verboinum{1}}
\verb{acaboclado}{}{}{}{}{adj.}{Semelhante a caboclo.}{a.ca.bo.cla.do}{0}
\verb{acaboclado}{}{}{}{}{}{Que tem feições ou modos rústicos, acaipirados.}{a.ca.bo.cla.do}{0}
\verb{acabrunhado}{}{}{}{}{adj.}{Que se acabrunhou.}{a.ca.bru.nha.do}{0}
\verb{acabrunhado}{}{}{}{}{}{Abatido, oprimido.}{a.ca.bru.nha.do}{0}
\verb{acabrunhado}{}{}{}{}{}{Melancólico, entristecido.}{a.ca.bru.nha.do}{0}
\verb{acabrunhar}{}{}{}{}{v.t.}{Abater, oprimir.}{a.ca.bru.nhar}{0}
\verb{acabrunhar}{}{}{}{}{}{Afligir, entristecer.}{a.ca.bru.nhar}{0}
\verb{acabrunhar}{}{}{}{}{v.pron.}{Adoentar"-se.}{a.ca.bru.nhar}{\verboinum{1}}
\verb{acaçá}{}{Cul.}{}{}{s.m.}{Bolinho feito de farinha de arroz ou de milho branco, envolto, ainda quente, em folhas de bananeira.}{a.ca.çá}{0}
\verb{acaçapado}{}{}{}{}{adj.}{Que fica agachado como um caçapo, filhote de coelho.}{a.ca.ça.pa.do}{0}
\verb{acaçapado}{}{Por ext.}{}{}{}{Indivíduo de pouca estatura, baixo.}{a.ca.ça.pa.do}{0}
\verb{acaçapar}{}{}{}{}{v.t.}{Tornar semelhante a caçapo, filhote de coelho.}{a.ca.ça.par}{0}
\verb{acaçapar}{}{}{}{}{}{Achatar, aplanar.}{a.ca.ça.par}{0}
\verb{acaçapar}{}{}{}{}{v.pron.}{Abaixar"-se, agachar"-se, encolher"-se.}{a.ca.ça.par}{\verboinum{1}}
\verb{acácia}{}{Bot.}{}{}{s.f.}{Nome dado a várias espécies de plantas ornamentais da família das leguminosas, com flores de cores variadas.}{a.cá.cia}{0}
\verb{academia}{}{}{}{}{s.f.}{Sociedade onde se reúnem cientistas, artistas ou literatos.}{a.ca.de.mia}{0}
\verb{academia}{}{}{}{}{}{Local onde se ministram ou se praticam artes ou esportes.}{a.ca.de.mia}{0}
\verb{acadêmia}{}{}{}{}{s.f.}{Figura de gesso destinada ao ensino de formas humanas.}{a.ca.dê.mia}{0}
\verb{academicismo}{}{}{}{}{s.m.}{Modo ou comportamento de quem integra uma academia.}{a.ca.de.mi.cis.mo}{0}
\verb{academicismo}{}{}{}{}{}{Atitude ou mentalidade conservadora.}{a.ca.de.mi.cis.mo}{0}
\verb{acadêmico}{}{}{}{}{adj.}{Relativo a academia.}{a.ca.dê.mi.co}{0}
\verb{acadêmico}{}{}{}{}{s.m.}{Membro ou estudante de uma academia.}{a.ca.dê.mi.co}{0}
\verb{acafajestado}{}{}{}{}{adj.}{Que tem modos ou aparência de cafajeste; canalha.}{a.ca.fa.jes.ta.do}{0}
\verb{acafajestar}{}{}{}{}{v.t.}{Tornar cafajeste, canalha.}{a.ca.fa.jes.tar}{\verboinum{1}}
\verb{açafate}{}{}{}{}{s.m.}{Pequeno cesto de vime, sem asas, para guardar flores ou pequenos objetos.}{a.ça.fa.te}{0}
\verb{açafrão}{}{Bot.}{"-ões}{}{s.m.}{Planta de origem europeia.}{a.ça.frão}{0}
\verb{açafrão}{}{}{"-ões}{}{}{Pó amarelo extraído dessa planta, utilizado como tempero, corante ou medicamento.}{a.ça.frão}{0}
\verb{açaí}{}{}{}{}{s.m.}{Fruto roxo comestível do açaizeiro, consumido em forma de suco, sorvete ou pirão; juçara.}{a.ça.í}{0}
\verb{açaimo}{}{}{}{}{s.m.}{Peça de couro ou metal colocada no focinho de animais para não morderem ou não comerem; mordaça. }{a.çai.mo}{0}
\verb{acaipirado}{}{}{}{}{adj.}{Que tem modos próprios de caipira.}{a.cai.pi.ra.do}{0}
\verb{acaipirado}{}{}{}{}{}{Acanhado, tímido.}{a.cai.pi.ra.do}{0}
\verb{acaipirar"-se}{}{}{}{}{v.pron.}{Tornar"-se caipira.}{a.cai.pi.rar"-se}{0}
\verb{acaipirar"-se}{}{}{}{}{}{Adquirir modos ou costumes de caipira.}{a.cai.pi.rar"-se}{\verboinum{1}}
\verb{açaizal}{}{}{"-ais}{}{s.m.}{Plantação ou aglomeração de açaizeiros.}{a.ça.i.zal}{0}
\verb{açaizeiro}{ê}{Bot.}{}{}{s.m.}{Palmeira cujo fruto comestível é o açaí.}{a.ça.i.zei.ro}{0}
\verb{acaju}{}{Bot.}{}{}{s.m.}{Nome dado a várias espécies de madeiras semelhantes ao mogno por sua cor avermelhada.}{a.ca.ju}{0}
\verb{acaju}{}{}{}{}{adj.2g.}{Que tem essa cor.}{a.ca.ju}{0}
\verb{acalantar}{}{}{}{}{}{Var. de \textit{acalentar}.}{a.ca.lan.tar}{0}
\verb{acalanto}{}{}{}{}{s.m.}{Canção de ninar; acalento.}{a.ca.lan.to}{0}
\verb{acalcanhar}{}{}{}{}{v.t.}{Pisar com o calcanhar.}{a.cal.ca.nhar}{0}
\verb{acalcanhar}{}{}{}{}{}{Andar de forma a entortar o tacão do calçado (próximo do calcanhar), gastando"-o.}{a.cal.ca.nhar}{0}
\verb{acalcanhar}{}{}{}{}{}{Gastar, envelhecer.}{a.cal.ca.nhar}{\verboinum{1}}
\verb{acalentado}{}{}{}{}{adj.}{Que se acalentou.}{a.ca.len.ta.do}{0}
\verb{acalentado}{}{}{}{}{}{Embalado, aconchegado.}{a.ca.len.ta.do}{0}
\verb{acalentado}{}{}{}{}{}{Sossegado, tranquilizado.}{a.ca.len.ta.do}{0}
\verb{acalentar}{}{}{}{}{v.t.}{Aquecer nos braços; embalar.}{a.ca.len.tar}{0}
\verb{acalentar}{}{}{}{}{}{Adormecer criança ao som de cantiga de ninar.}{a.ca.len.tar}{0}
\verb{acalentar}{}{}{}{}{}{Tranquilizar, confortar; acalantar.}{a.ca.len.tar}{\verboinum{1}}
\verb{acalento}{}{}{}{}{s.m.}{Ato de acalentar, adormecer criança.}{a.ca.len.to}{0}
\verb{acalento}{}{}{}{}{}{Carícia, afeto; acalanto.}{a.ca.len.to}{0}
\verb{acalmar}{}{}{}{}{v.t.}{Tornar calmo, sereno; tranquilizar; pacificar.}{a.cal.mar}{\verboinum{1}}
\verb{acalorado}{}{}{}{}{adj.}{Que se acalorou.}{a.ca.lo.ra.do}{0}
\verb{acalorado}{}{}{}{}{}{Animado, excitado, entusiasmado.}{a.ca.lo.ra.do}{0}
\verb{acalorado}{}{}{}{}{}{Aquecido.}{a.ca.lo.ra.do}{0}
\verb{acalorar}{}{}{}{}{v.t.}{Dar ou receber calor; aquecer.}{a.ca.lo.rar}{0}
\verb{acalorar}{}{}{}{}{}{Animar, excitar.}{a.ca.lo.rar}{\verboinum{1}}
\verb{acamado}{}{}{}{}{adj.}{Deitado ou estendido na cama.}{a.ca.ma.do}{0}
\verb{acamado}{}{}{}{}{}{Disposto em camadas.}{a.ca.ma.do}{0}
\verb{acamado}{}{}{}{}{}{Doente de cama.}{a.ca.ma.do}{0}
\verb{acamar}{}{}{}{}{v.t.}{Deitar na cama.}{a.ca.mar}{0}
\verb{acamar}{}{}{}{}{}{Dispor em camadas.}{a.ca.mar}{0}
\verb{acamar}{}{}{}{}{v.pron.}{Adoecer, cair de cama.}{a.ca.mar}{\verboinum{1}}
\verb{acamaradar"-se}{}{}{}{}{v.pron.}{Tornar"-se camarada, amigo, parceiro.}{a.ca.ma.ra.dar"-se}{\verboinum{1}}
\verb{açambarcador}{ô}{}{}{-deira}{adj.}{Que açambarca; monopolizador.}{a.çam.bar.ca.dor}{0}
\verb{açambarcador}{ô}{}{}{}{s.m.}{Indivíduo que açambarca, monopoliza.}{a.çam.bar.ca.dor}{0}
\verb{açambarcamento}{}{}{}{}{s.m.}{Ato ou efeito de açambarcar, apoderar"-se.}{a.çam.bar.ca.men.to}{0}
\verb{açambarcamento}{}{}{}{}{}{Monopólio.}{a.çam.bar.ca.men.to}{0}
\verb{açambarcar}{}{}{}{}{v.t.}{Apoderar"-se de algo em detrimento de outros, monopolizar.}{a.çam.bar.car}{0}
\verb{açambarcar}{}{}{}{}{}{Apossar"-se; apropriar"-se.}{a.çam.bar.car}{\verboinum{2}}
\verb{acampado}{}{}{}{}{adj.}{Que acampou.}{a.cam.pa.do}{0}
\verb{acampado}{}{}{}{}{}{Instalado em acampamento.}{a.cam.pa.do}{0}
\verb{acampamento}{}{}{}{}{s.m.}{Ato ou efeito de acampar.}{a.cam.pa.men.to}{0}
\verb{acampamento}{}{}{}{}{}{Local onde se acampa; alojamento.}{a.cam.pa.men.to}{0}
\verb{acampar}{}{}{}{}{v.t.}{Estabelecer em campo.}{a.cam.par}{0}
\verb{acampar}{}{}{}{}{v.i.}{Instalar"-se provisoriamente em campo ou acampamento.}{a.cam.par}{\verboinum{1}}
\verb{acamurçado}{}{}{}{}{adj.}{Que foi recoberto com camurça.}{a.ca.mur.ça.do}{0}
\verb{acamurçado}{}{}{}{}{}{Semelhante à camurça.}{a.ca.mur.ça.do}{0}
\verb{acanaladura}{}{}{}{}{s.f.}{Escavação ou concavidade no terreno em forma de canal ou rego.}{a.ca.na.la.du.ra}{0}
\verb{acanalar}{}{}{}{}{v.t.}{Escavar em forma de canal ou rego; abrir sulcos, acanaladuras.}{a.ca.na.lar}{0}
\verb{acanalar}{}{}{}{}{}{Sulcar.}{a.ca.na.lar}{\verboinum{1}}
\verb{acanalhado}{}{}{}{}{adj.}{Que se acanalhou.}{a.ca.na.lha.do}{0}
\verb{acanalhado}{}{}{}{}{}{Que tem aparência ou comportamento de canalha.}{a.ca.na.lha.do}{0}
\verb{acanalhado}{}{}{}{}{}{Ridicularizado, escarnecido.}{a.ca.na.lha.do}{0}
\verb{acanalhar}{}{}{}{}{v.t.}{Tornar canalha, desprezível.}{a.ca.na.lhar}{0}
\verb{acanalhar}{}{}{}{}{}{Ridicularizar, escarnecer.}{a.ca.na.lhar}{\verboinum{1}}
\verb{acangatara}{}{}{}{}{s.m.}{Tipo de cocar de penas; canitar.}{a.can.ga.ta.ra}{0}
\verb{acanguçu}{}{Zool.}{}{}{s.m.}{Onça"-pintada.}{a.can.gu.çu}{0}
\verb{acanhado}{}{}{}{}{adj.}{Que se acanhou; que sofre ou sofreu limitação nas suas ações.}{a.ca.nha.do}{0}
\verb{acanhado}{}{}{}{}{}{Envergonhado, retraído.}{a.ca.nha.do}{0}
\verb{acanhamento}{}{}{}{}{s.m.}{Ato ou efeito de acanhar; limitação das ações.}{a.ca.nha.men.to}{0}
\verb{acanhamento}{}{}{}{}{}{Timidez, retraimento.}{a.ca.nha.men.to}{0}
\verb{acanhar}{}{}{}{}{v.t.}{Restringir as ações.}{a.ca.nhar}{0}
\verb{acanhar}{}{}{}{}{}{Envergonhar; tornar tímido, retraído.}{a.ca.nhar}{\verboinum{1}}
\verb{acanto}{}{Bot.}{}{}{s.m.}{Planta espinhosa de folhas recortadas, cultivada para ornamento.}{a.can.to}{0}
\verb{acanto}{}{Por ext.}{}{}{}{Ornato arquitetônico inspirado nas folhas dessa planta.}{a.can.to}{0}
\verb{acantonamento}{}{}{}{}{s.m.}{Ato ou efeito de acantonar.}{a.can.to.na.men.to}{0}
\verb{acantonamento}{}{}{}{}{}{Local onde se alojam provisoriamente tropas militares.}{a.can.to.na.men.to}{0}
\verb{acantonar}{}{}{}{}{v.t.}{Distribuir ou dispor tropas por várias localidades ou cantões.}{a.can.to.nar}{0}
\verb{acantonar}{}{}{}{}{}{Instalar, alojar.}{a.can.to.nar}{\verboinum{1}}
\verb{ação}{}{}{"-ões}{}{s.f.}{Ato ou efeito de agir ou de atuar. }{a.ção}{0}
\verb{ação}{}{}{"-ões}{}{}{Movimento, funcionamento, atividade.}{a.ção}{0}
\verb{ação}{}{}{"-ões}{}{}{Modo de proceder, comportamento.}{a.ção}{0}
\verb{ação}{}{}{"-ões}{}{}{Capacidade ou disposição para agir.}{a.ção}{0}
\verb{ação}{}{}{"-ões}{}{}{Efeito de um comportamento sobre outro. (\textit{A ação eficaz de um remédio. A ação do pai sobre o filho.})}{a.ção}{0}
\verb{ação}{}{}{"-ões}{}{}{Enredo de peça teatral, de filme ou de romance em que fatos se sucedem muito rapidamente.}{a.ção}{0}
\verb{ação}{}{}{"-ões}{}{}{Cada uma das partes, ideais ou não, que compõe o capital de uma empresa, e que pode ser negociada.}{a.ção}{0}
\verb{ação}{}{}{"-ões}{}{}{Pedido formal de intervenção ao órgão competente do Poder Público para fazer valer um direito que se julga ter.}{a.ção}{0}
\verb{ação}{}{}{"-ões}{}{}{Na retórica ou no teatro, é o movimento do corpo ou dos braços que acompanha um discurso ou uma fala.}{a.ção}{0}
\verb{acará}{}{Zool.}{}{}{s.m.}{Nome comum a vários peixes de água doce, geralmente criados como ornamentais.}{a.ca.rá}{0}
\verb{acarajé}{}{Cul.}{}{}{s.m.}{Bolinho feito de feijão, camarão seco, frito em azeite"-de"-dendê e servido com molho de pimenta"-malagueta.}{a.ca.ra.jé}{0}
\verb{acareação}{}{}{"-ões}{}{s.f.}{Ato ou efeito de acarear.}{a.ca.re.a.ção}{0}
\verb{acareação}{}{Jur.}{"-ões}{}{}{Confrontação de depoimentos de réus e testemunhas.}{a.ca.re.a.ção}{0}
\verb{acarear}{}{}{}{}{v.t.}{Colocar cara a cara.}{a.ca.re.ar}{0}
\verb{acarear}{}{}{}{}{}{Comparar, cotejar.}{a.ca.re.ar}{0}
\verb{acarear}{}{Jur.}{}{}{}{Confrontar depoimentos de réus e testemunhas.}{a.ca.re.ar}{\verboinum{4}}
\verb{acari}{}{Zool.}{}{}{s.m.}{Peixe de água doce; cascudo.}{a.ca.ri}{0}
\verb{acaríase}{}{Med.}{}{}{s.f.}{Infestação de ácaros.}{a.ca.rí.a.se}{0}
\verb{acaríase}{}{Med.}{}{}{}{Afecção cutânea causada por ácaros; sarna.}{a.ca.rí.a.se}{0}
\verb{acariciador}{ô}{}{}{}{adj.}{Que acaricia, que afaga; acariciante.}{a.ca.ri.ci.a.dor}{0}
\verb{acariciante}{}{}{}{}{adj.2g.}{Acariciador.}{a.ca.ri.ci.an.te}{0}
\verb{acariciar}{}{}{}{}{v.t.}{Fazer carícias; afagar.}{a.ca.ri.ci.ar}{0}
\verb{acariciar}{}{}{}{}{}{Acarinhar.}{a.ca.ri.ci.ar}{\verboinum{1}}
\verb{acaricida}{}{}{}{}{adj.2g.}{Que mata os ácaros.}{a.ca.ri.ci.da}{0}
\verb{acarídeo}{}{Zool.}{}{}{s.m.}{Espécime de aracnídeos parasitas que têm a cabeça, o tórax e o abdômen formando uma única parte do corpo. }{a.ca.rí.deo}{0}
\verb{acarídeo}{}{}{}{}{adj.}{Relativo aos acarídeos.}{a.ca.rí.deo}{0}
\verb{acarinhar}{}{}{}{}{v.t.}{Tratar com carinho; mimar.}{a.ca.ri.nhar}{0}
\verb{acarinhar}{}{}{}{}{}{Afagar, acariciar.}{a.ca.ri.nhar}{\verboinum{1}}
\verb{acarino}{}{Zool.}{}{}{s.m.}{Espécime dos acarinos, ordem de pequenos ou minúsculos aracnídeos que vivem livres ou como parasitas.}{a.ca.ri.no}{0}
\verb{acarino}{}{}{}{}{adj.}{Relativo aos acarinos.}{a.ca.ri.no}{0}
\verb{ácaro}{}{Zool.}{}{}{s.m.}{Nome comum dado a vários acarídeos, parasitas do homem e de outros animais, causadores de alergias ou afecções cutâneas, como a sarna.}{á.ca.ro}{0}
\verb{acarpetado}{}{}{}{}{adj.}{Que foi revestido de carpete; carpetado.}{a.car.pe.ta.do}{0}
\verb{acarpetar}{}{}{}{}{v.t.}{Revestir de carpete; carpetar.}{a.car.pe.tar}{\verboinum{1}}
\verb{acarretar}{}{}{}{}{v.t.}{Transportar em carreta.}{a.car.re.tar}{0}
\verb{acarretar}{}{}{}{}{}{Carregar, trazer.}{a.car.re.tar}{0}
\verb{acarretar}{}{}{}{}{}{Causar, ocasionar, provocar.}{a.car.re.tar}{\verboinum{1}}
\verb{acasalado}{}{}{}{}{adj.}{Que se acasalou.}{a.ca.sa.la.do}{0}
\verb{acasalado}{}{}{}{}{}{Unido como casal; emparelhado.}{a.ca.sa.la.do}{0}
\verb{acasalamento}{}{}{}{}{s.m.}{Ato ou efeito de acasalar.}{a.ca.sa.la.men.to}{0}
\verb{acasalamento}{}{}{}{}{}{União de macho e fêmea para procriação; cruzamento.}{a.ca.sa.la.men.to}{0}
\verb{acasalar}{}{}{}{}{v.t.}{Unir em casal, em par.}{a.ca.sa.lar}{0}
\verb{acasalar}{}{}{}{}{}{Unir macho e fêmea para procriação; cruzar.}{a.ca.sa.lar}{\verboinum{1}}
\verb{acaso}{}{}{}{}{s.m.}{Acontecimento fortuito, cujas causas são ignoradas.}{a.ca.so}{0}
\verb{acaso}{}{}{}{}{}{Casualidade, eventualidade.}{a.ca.so}{0}
\verb{acaso}{}{}{}{}{}{Destino, sorte.}{a.ca.so}{0}
\verb{acaso}{}{}{}{}{adv.}{Talvez, porventura.}{a.ca.so}{0}
\verb{acastanhado}{}{}{}{}{adj.}{De cor próxima à da castanha.}{a.cas.ta.nha.do}{0}
\verb{acatado}{}{}{}{}{adj.}{Que se acatou.}{a.ca.ta.do}{0}
\verb{acatado}{}{}{}{}{}{Respeitado, considerado.}{a.ca.ta.do}{0}
\verb{acatamento}{}{}{}{}{s.m.}{Ato ou efeito de acatar.}{a.ca.ta.men.to}{0}
\verb{acatamento}{}{}{}{}{}{Respeito, consideração, obediência.}{a.ca.ta.men.to}{0}
\verb{acatar}{}{}{}{}{v.t.}{Seguir, adotar.}{a.ca.tar}{0}
\verb{acatar}{}{}{}{}{}{Respeitar, considerar.}{a.ca.tar}{\verboinum{1}}
\verb{acatólico}{}{}{}{}{adj.}{Que não é católico.}{a.ca.tó.li.co}{0}
\verb{acatólico}{}{}{}{}{s.m.}{Indivíduo que não é católico.}{a.ca.tó.li.co}{0}
\verb{acauã}{}{Zool.}{}{}{s.2g.}{Ave de rapina, espécie de gavião, que ataca os ofídios; tem plumagem amarelo claro, com uma mancha negra em torno dos olhos, prolongando"-se até a nuca, e o alto da cabeça é branco. }{a.cau.ã}{0}
\verb{acaule}{}{}{}{}{adj.2g.}{Que é sem caule, ou não tem caule visível.}{a.cau.le}{0}
\verb{acautelado}{}{}{}{}{adj.}{Que se acautela; precavido; prevenido; prudente. }{a.cau.te.la.do}{0}
\verb{acautelado}{}{}{}{}{}{Que é dotado de astúcia; esperto; manhoso.}{a.cau.te.la.do}{0}
\verb{acautelado}{}{}{}{}{}{O que é guardado com cuidado; resguardado.}{a.cau.te.la.do}{0}
\verb{acautelamento}{}{}{}{}{s.m.}{Ato ou efeito de acautelar; cautela; cuidado; prevenção.}{a.cau.te.la.men.to}{0}
\verb{acautelar}{}{}{}{}{v.t.}{Pôr alguém de prevenção; precaver.}{a.cau.te.lar}{0}
\verb{acautelar}{}{}{}{}{}{Guardar com cautela; resguardar.}{a.cau.te.lar}{0}
\verb{acautelar}{}{}{}{}{v.pron.}{Tomar cuidado; prevenir"-se.}{a.cau.te.lar}{\verboinum{1}}
\verb{acavalado}{}{}{}{}{adj.}{O que está sobreposto; amontoado; empilhado.}{a.ca.va.la.do}{0}
\verb{acavalado}{}{Fig.}{}{}{}{Indivíduo abrutalhado de maneiras; grosseiro.}{a.ca.va.la.do}{0}
\verb{acavalar}{}{}{}{}{v.t.}{Colocar de maneira sobreposta; amontoar; empilhar.}{a.ca.va.lar}{\verboinum{1}}
\verb{accessível}{}{}{}{}{}{Var. de \textit{acessível}.}{ac.ces.sí.vel}{0}
\verb{accessório}{}{}{}{}{}{Var. de \textit{acessório}.}{ac.ces.só.rio}{0}
\verb{acebolado}{}{}{}{}{adj.}{Que tem formato ou gosto de cebola.}{a.ce.bo.la.do}{0}
\verb{acebolado}{}{}{}{}{}{Temperado com cebola.}{a.ce.bo.la.do}{0}
\verb{aceder}{ê}{}{}{}{v.i.}{Estar de acordo; concordar.}{a.ce.der}{0}
\verb{aceder}{ê}{}{}{}{}{Fazer aumentar; acrescentar; juntar.}{a.ce.der}{0}
\verb{aceder}{ê}{Lus.}{}{}{}{Ter acesso.}{a.ce.der}{\verboinum{12}\verboirregular{acedo, acedes}}
\verb{acefalia}{}{Med.}{}{}{s.f.}{Ausência congênita da cabeça ou de parte dela (no embrião ou no feto).}{a.ce.fa.li.a}{0}
\verb{acéfalo}{}{}{}{}{adj.}{Que não tem responsável, sem orientador.}{a.cé.fa.lo}{0}
\verb{acéfalo}{}{Med.}{}{}{}{Que apresenta acefalia.}{a.cé.fa.lo}{0}
\verb{aceirar}{}{}{}{}{v.t.}{Converter em aço; revestir de aço; acerar.}{a.cei.rar}{0}
\verb{aceirar}{}{}{}{}{}{Cortar a vegetação que fica em volta da mata para evitar incêndios.}{a.cei.rar}{0}
\verb{aceirar}{}{}{}{}{}{Andar à volta de; rodear.}{a.cei.rar}{\verboinum{1}}
\verb{aceiro}{ê}{}{}{}{s.m.}{Parte da terra desbastada em volta de propriedades rurais para evitar a propagação de queimadas. }{a.cei.ro}{0}
\verb{aceiro}{ê}{}{}{}{}{Nome dado às pessoas que trabalham com aço.}{a.cei.ro}{0}
\verb{aceiro}{ê}{}{}{}{}{O próprio aço.}{a.cei.ro}{0}
\verb{aceiro}{ê}{}{}{}{adj.}{Relativo ao aço; o que tem as propriedades dele.}{a.cei.ro}{0}
\verb{aceitação}{}{}{"-ões}{}{s.f.}{Ato ou efeito de aceitar; aprovação; receptividade; concordância; acolhimento.}{a.cei.ta.ção}{0}
\verb{aceitante}{}{}{}{}{adj.2g.}{Que aceita.}{a.cei.tan.te}{0}
\verb{aceitante}{}{}{}{}{s.2g.}{Indivíduo que aceita.}{a.cei.tan.te}{0}
\verb{aceitante}{}{}{}{}{}{Indivíduo que assina o aceite de letra de câmbio ou duplicata de fatura.}{a.cei.tan.te}{0}
\verb{aceitar}{}{}{}{}{v.t.}{Concordar em receber algo oferecido ou dado.}{a.cei.tar}{0}
\verb{aceitar}{}{}{}{}{}{Estar de acordo com; anuir; aprovar.}{a.cei.tar}{0}
\verb{aceitar}{}{}{}{}{}{Conformar"-se; submeter"-se.}{a.cei.tar}{0}
\verb{aceitar}{}{}{}{}{}{Assumir; tomar para si ou sobre si.}{a.cei.tar}{0}
\verb{aceitar}{}{}{}{}{}{Ter ou dar como bom ou certo.}{a.cei.tar}{0}
\verb{aceitar}{}{}{}{}{}{Aderir; seguir.}{a.cei.tar}{0}
\verb{aceitar}{}{}{}{}{}{Reconhecer; admitir; receber.}{a.cei.tar}{\verboinum{1}}
\verb{aceitável}{}{}{"-eis}{}{adj.2g.}{Que se pode aceitar; digno de aceitação.}{a.cei.tá.vel}{0}
\verb{aceite}{}{}{}{}{adj.2g.}{Que é aceito.}{a.cei.te}{0}
\verb{aceite}{}{}{}{}{s.m.}{Ato ou efeito de aceitar.}{a.cei.te}{0}
\verb{aceite}{}{}{}{}{}{Assinatura do aceitante no título de crédito.}{a.cei.te}{0}
\verb{aceite}{}{}{}{}{}{O título de crédito assinado.}{a.cei.te}{0}
\verb{aceito}{ê}{}{}{}{adj.}{Que se aceitou; permitido.}{a.cei.to}{0}
\verb{aceito}{ê}{}{}{}{}{Que se aprova; que é acreditado; respeitado; estimado; acolhido.}{a.cei.to}{0}
\verb{aceleração}{}{}{"-ões}{}{s.f.}{Ato ou efeito de acelerar; rapidez; pressa; antecipação.}{a.ce.le.ra.ção}{0}
\verb{aceleração}{}{Fís.}{"-ões}{}{}{Variação da velocidade de um corpo num intervalo de tempo.}{a.ce.le.ra.ção}{0}
\verb{acelerado}{}{}{}{}{adj.}{Que é tornado rápido ou mais rápido; apressado; ligeiro; veloz. }{a.ce.le.ra.do}{0}
\verb{acelerado}{}{}{}{}{s.m.}{Passo de andamento mais rápido que o passo ordinário da tropa que marcha a pé.}{a.ce.le.ra.do}{0}
\verb{acelerador}{ô}{}{}{}{adj.}{Que acelera ou serve para acelerar.}{a.ce.le.ra.dor}{0}
\verb{acelerador}{ô}{}{}{}{s.m.}{Aquilo que serve para acelerar.}{a.ce.le.ra.dor}{0}
\verb{acelerador}{ô}{}{}{}{}{Dispositivo de um veículo destinado a controlar sua velocidade.}{a.ce.le.ra.dor}{0}
\verb{aceleramento}{}{}{}{}{s.m.}{Ato ou efeito de acelerar; aceleração.}{a.ce.le.ra.men.to}{0}
\verb{acelerar}{}{}{}{}{v.t.}{Tornar veloz ou mais veloz; apressar.}{a.ce.le.rar}{0}
\verb{acelerar}{}{}{}{}{}{Instigar; estimular; incitar.}{a.ce.le.rar}{\verboinum{1}}
\verb{acelga}{é}{Bot.}{}{}{s.f.}{Erva de origem europeia, cujas folhas grandes e crespas são verdes ou amareladas com nervuras; é consumida como verdura em saladas e ensopados.  }{a.cel.ga}{0}
\verb{acém}{}{}{}{}{s.m.}{Carne do lombo do boi entre o cachaço e a pá.}{a.cém}{0}
\verb{acenar}{}{}{}{}{v.t.}{Fazer movimentos com as mãos ou os braços, ou com objetos, para mostrar, negar, confirmar, atrair, despedir, prevenir, chamar etc.}{a.ce.nar}{0}
\verb{acenar}{}{}{}{}{}{Agitar; balançar.}{a.ce.nar}{0}
\verb{acenar}{}{}{}{}{}{Dar a perceber; indicar.}{a.ce.nar}{0}
\verb{acenar}{}{}{}{}{}{Ameaçar.}{a.ce.nar}{0}
\verb{acenar}{}{}{}{}{}{Referir"-se; aludir.}{a.ce.nar}{0}
\verb{acenar}{}{}{}{}{}{Procurar seduzir, atrair, aliciar, induzir.}{a.ce.nar}{\verboinum{1}}
\verb{acendalha}{}{Lus.}{}{}{s.f.}{Tudo aquilo que serve para queimar, como gravetos, aparas de madeira, cavacos etc.}{a.cen.da.lha}{0}
\verb{acendedor}{ô}{}{}{}{adj.}{Que acende.}{a.cen.de.dor}{0}
\verb{acendedor}{ô}{}{}{}{s.m.}{Instrumento que acende, produzindo calor ou faísca suficiente para produzir ignição. }{a.cen.de.dor}{0}
\verb{acendedor}{ô}{}{}{}{}{Indivíduo que manipula instrumentos ou objetos que provocam ignição.}{a.cen.de.dor}{0}
\verb{acender}{ê}{}{}{}{v.t.}{Pôr fogo; fazer arder.}{a.cen.der}{0}
\verb{acender}{ê}{}{}{}{}{Provocar; produzir; irromper.}{a.cen.der}{0}
\verb{acender}{ê}{}{}{}{}{Pôr em funcionamento; ligar.}{a.cen.der}{0}
\verb{acender}{ê}{}{}{}{v.i.}{Pegar fogo.}{a.cen.der}{\verboinum{12}}
\verb{acendimento}{}{}{}{}{s.m.}{Ato ou efeito de acender; pôr ou pegar fogo.}{a.cen.di.men.to}{0}
\verb{acendimento}{}{Fig.}{}{}{}{Fervor, entusiasmo, excitação nas ações ou nos sentimentos.}{a.cen.di.men.to}{0}
\verb{acendrado}{}{}{}{}{adj.}{Que se acendrou, que é livre de impurezas; limpo; purificado; acrisolado.}{a.cen.dra.do}{0}
\verb{acendrar}{}{}{}{}{v.t.}{Limpar com cinza.}{a.cen.drar}{0}
\verb{acendrar}{}{}{}{}{}{Purificar; apurar; acrisolar.}{a.cen.drar}{\verboinum{1}}
\verb{aceno}{}{}{}{}{s.m.}{Ato ou efeito de acenar; movimento que se faz com os braços ou as mãos, ou com objetos para exprimir o que se deseja; sinal; gesto.}{a.ce.no}{0}
\verb{acento}{}{}{}{}{s.m.}{Destaque ou realce que uma sílaba ou uma palavra tem em relação às outras numa enunciação, que pode ser pela maior intensidade, duração ou altura com que é pronunciada.}{a.cen.to}{0}
\verb{acento}{}{Gram.}{}{}{}{}{a.cen.to}{0}
\verb{acentuação}{}{}{"-ões}{}{s.f.}{Ato ou efeito, modo ou método de acentuar na escrita ou na fala.}{a.cen.tu.a.ção}{0}
\verb{acentuação}{}{Gram.}{"-ões}{}{}{Regras ortográficas que tratam do emprego do acento.}{a.cen.tu.a.ção}{0}
\verb{acentuado}{}{Gram.}{}{}{adj.}{Que se acentuou, que recebeu acento gráfico.}{a.cen.tu.a.do}{0}
\verb{acentuado}{}{}{}{}{}{Que é saliente; ressaltante; destacado.}{a.cen.tu.a.do}{0}
\verb{acentuar}{}{}{}{}{v.t.}{Pronunciar, enunciar sílabas ou palavras com os devidos acentos.}{a.cen.tu.ar}{0}
\verb{acentuar}{}{Gram.}{}{}{}{Colocar o acento gráfico.}{a.cen.tu.ar}{\verboinum{1}}
\verb{acepção}{}{}{"-ões}{}{s.f.}{Cada um dos vários sentidos que apresenta uma palavra ou uma frase num determinado contexto.}{a.cep.ção}{0}
\verb{acepilhar}{}{}{}{}{v.t.}{Alisar com cepilho; aplainar.}{a.ce.pi.lhar}{0}
\verb{acepilhar}{}{}{}{}{}{Polir com limas finas; brunir.}{a.ce.pi.lhar}{0}
\verb{acepilhar}{}{Fig.}{}{}{}{Aperfeiçoar; aprimorar.}{a.ce.pi.lhar}{\verboinum{1}}
\verb{acepipe}{}{}{}{}{s.m.}{Prato servido para abrir o apetite; aperitivo, petisco. }{a.ce.pi.pe}{0}
\verb{aceração}{}{}{"-ões}{}{s.f.}{Ato ou efeito de acerar, revestir com aço.}{a.ce.ra.ção}{0}
\verb{acerado}{}{}{}{}{adj.}{Que é temperado com aço.}{a.ce.ra.do}{0}
\verb{acerado}{}{}{}{}{}{Que é pontiagudo; cortante.}{a.ce.ra.do}{0}
\verb{acerado}{}{Fig.}{}{}{}{Que fere profundamente; maledicente; mordaz.}{a.ce.ra.do}{0}
\verb{aceragem}{}{}{"-ens}{}{s.f.}{Ato ou efeito de acerar; aceração.}{a.ce.ra.gem}{0}
\verb{acerar}{}{}{}{}{v.t.}{Temperar, revestir com aço.}{a.ce.rar}{0}
\verb{acerar}{}{}{}{}{}{Tornar pontiagudo, cortante.}{a.ce.rar}{0}
\verb{acerar}{}{Bras.}{}{}{}{Aguçar; estimular; exacerbar.}{a.ce.rar}{\verboinum{1}}
\verb{acerbidade}{}{}{}{}{s.f.}{Qualidade do que é acerbo; amargo, azedo, áspero.}{a.cer.bi.da.de}{0}
\verb{acerbo}{é}{}{}{}{adj.}{Que tem sabor áspero, amargo.}{a.cer.bo}{0}
\verb{acerbo}{é}{}{}{}{}{Que tem gosto azedo, ácido.}{a.cer.bo}{0}
\verb{acerbo}{é}{Fig.}{}{}{}{Que é cruel; severo.}{a.cer.bo}{0}
\verb{acerca}{ê}{}{}{}{adv.}{A pequena distância; perto; próximo.}{a.cer.ca}{0}
\verb{acerca}{ê}{}{}{}{}{Quase; cerca.}{a.cer.ca}{0}
\verb{acercamento}{}{}{}{}{s.m.}{Ato ou efeito de acercar"-se; aproximação; avizinhação.}{a.cer.ca.men.to}{0}
\verb{acercar"-se}{}{}{}{}{v.pron.}{Pôr"-se cerca de; avizinhar"-se; aproximar"-se.}{a.cer.car"-se}{0}
\verb{acercar"-se}{}{}{}{}{}{Ficar ao redor de; acompanhar; cercar.}{a.cer.car"-se}{\verboinum{2}}
\verb{acerola}{ó}{Bot.}{}{}{s.f.}{Arbusto nativo da América tropical, cujo fruto comestível de pequeno tamanho tem a forma redonda, a cor vermelha e é rico em vitamina \textsc{c}.}{a.ce.ro.la}{0}
\verb{acerola}{ó}{}{}{}{}{O fruto da acerola.}{a.ce.ro.la}{0}
\verb{acérrimo}{}{}{}{}{adj.}{Superlativo absoluto sintético de \textit{ácido}; muito amargo, muito azedo.}{a.cér.ri.mo}{0}
\verb{acérrimo}{}{}{}{}{}{Que é muito forte; tenaz; obstinado.}{a.cér.ri.mo}{0}
\verb{acertado}{}{}{}{}{adj.}{Que se acertou; feito com acerto; apropriado; adequado; ajustado.}{a.cer.ta.do}{0}
\verb{acertador}{ô}{}{}{}{adj.}{Que acerta.}{a.cer.ta.dor}{0}
\verb{acertador}{ô}{}{}{}{s.m.}{Pessoa que tem por função acertar a marcha dos animais de sela.}{a.cer.ta.dor}{0}
\verb{acertar}{}{}{}{}{v.t.}{Colocar de maneira certa; adequar; endireitar.}{a.cer.tar}{0}
\verb{acertar}{}{}{}{}{}{Dar ou bater; atingir; alcançar.}{a.cer.tar}{0}
\verb{acertar}{}{}{}{}{}{Combinar; ajustar; acordar. }{a.cer.tar}{0}
\verb{acertar}{}{}{}{}{}{Achar ao certo; atinar; encontrar; deparar.}{a.cer.tar}{\verboinum{1}}
\verb{acerto}{ê}{}{}{}{s.m.}{Ato ou efeito de acertar, ajustar, acordar.}{a.cer.to}{0}
\verb{acerto}{ê}{}{}{}{}{Sensatez; prudência.}{a.cer.to}{0}
\verb{acerto}{ê}{}{}{}{}{Acaso; sorte.}{a.cer.to}{0}
\verb{acerto}{ê}{}{}{}{}{Correção no modo de escrever, falar, agir.}{a.cer.to}{0}
\verb{acervo}{ê}{}{}{}{s.m.}{Conjunto de bens que integram um patrimônio.}{a.cer.vo}{0}
\verb{acervo}{ê}{}{}{}{}{Grande quantidade; monte; cúmulo.}{a.cer.vo}{0}
\verb{aceso}{ê}{}{}{}{adj.}{Que se acendeu, a que se pôs fogo.}{a.ce.so}{0}
\verb{aceso}{ê}{}{}{}{}{Brilhante, vivo.}{a.ce.so}{0}
\verb{aceso}{ê}{}{}{}{}{Ligado.}{a.ce.so}{0}
\verb{acessão}{}{}{"-ões}{}{s.f.}{Participação oficial em um acordo, anuência.}{a.ces.são}{0}
\verb{acessão}{}{}{"-ões}{}{}{Promoção a um cargo. }{a.ces.são}{0}
\verb{acessão}{}{Jur.}{"-ões}{}{}{Acréscimo que se incorpora ao bem principal.}{a.ces.são}{0}
\verb{acessar}{}{}{}{}{v.t.}{Alcançar para manipulação, processamento ou passagem.}{a.ces.sar}{\verboinum{1}}
\verb{acessibilidade}{}{}{}{}{s.f.}{Qualidade do que é acessível.}{a.ces.si.bi.li.da.de}{0}
\verb{acessível}{}{}{"-eis}{}{adj.2g.}{A que se tem ou se pode ter acesso; fácil de atingir. }{a.ces.sí.vel}{0}
\verb{acessível}{}{Ant.}{}{}{}{inacessível; insociável; intratável}{a.ces.sí.vel}{0}
\verb{acessível}{}{}{"-eis}{}{}{Que permite aproximação; afável, comunicativo, sociável, accessível.}{a.ces.sí.vel}{0}
\verb{acesso}{é}{}{}{}{s.m.}{Alcance para contato, manipulação ou passagem.}{a.ces.so}{0}
\verb{acesso}{é}{}{}{}{}{Entrada, passagem de um lugar a outro.}{a.ces.so}{0}
\verb{acesso}{é}{}{}{}{}{Ataque intenso e repentino.}{a.ces.so}{0}
\verb{acessório}{}{}{}{}{adj.}{Que se acrescenta a uma coisa, sem fazer parte integrante dela.}{a.ces.só.rio}{0}
\verb{acessório}{}{}{}{}{}{Suplementar; adicional; dispensável.}{a.ces.só.rio}{0}
\verb{acessório}{}{}{}{}{s.m.}{Aquilo que se junta ao objeto principal, ou dele é dependente.}{a.ces.só.rio}{0}
\verb{acessório}{}{}{}{}{}{Complemento, detalhe, accessório.}{a.ces.só.rio}{0}
\verb{acetaldeído}{}{Quím.}{}{}{s.m.}{Substância química resultante da oxidação de alcoóis primários, também chamado de aldeído acético.}{a.ce.tal.de.í.do}{0}
\verb{acetato}{}{Quím.}{}{}{s.m.}{Nome comum aos sais ou aos ânions derivados do ácido acético.}{a.ce.ta.to}{0}
\verb{acetato}{}{}{}{}{}{Disco de alumínio, revestido com substância especial, em que se fazem  gravações sonoras experimentais ou provisórias.  }{a.ce.ta.to}{0}
\verb{acético}{}{Quím.}{}{}{adj.}{Que se refere ao ácido acético encontrado no vinagre, ou que é próprio dele.  }{a.cé.ti.co}{0}
\verb{acético}{}{}{}{}{}{Relativo ou pertencente ao vinagre; ácido; azedo.}{a.cé.ti.co}{0}
\verb{acetileno}{}{Quím.}{}{}{s.m.}{Gás incolor, explosivo, de cheiro desagradável, muito usado na indústria.}{a.ce.ti.le.no}{0}
\verb{acetilsalicílico}{}{Quím.}{}{}{adj.}{Diz"-se de ácido usado em remédio para combater febre e dor.}{a.ce.til.sa.li.cí.li.co}{0}
\verb{acetilsalicílico}{}{}{}{}{s.m.}{Esse medicamento; aspirina.}{a.ce.til.sa.li.cí.li.co}{0}
\verb{acetinado}{}{}{}{}{adj.}{Que tem a aparência, a maciez e o brilho do cetim.}{a.ce.ti.na.do}{0}
\verb{acetinar}{}{}{}{}{v.t.}{Tornar macio e lustroso como o cetim.}{a.ce.ti.nar}{0}
\verb{acetinar}{}{}{}{}{}{Amaciar.}{a.ce.ti.nar}{\verboinum{1}}
\verb{acetona}{}{}{}{}{s.f.}{Líquido incolor, inflamável e volátil, com cheiro de éter, obtido sinteticamente do ácido acético, usado como solvente de ceras, vernizes, esmaltes etc.    }{a.ce.to.na}{0}
\verb{acetonemia}{}{Veter.}{}{}{s.f.}{Patologia, principalmente dos bovinos lactantes, caracterizada pela presença de acetona no sangue.}{a.ce.to.ne.mi.a}{0}
\verb{acetoso}{ô}{}{"-osos ⟨ó⟩}{"-osa ⟨ó⟩}{adj.}{Que tem sabor de vinagre.}{a.ce.to.so}{0}
\verb{acha}{}{}{}{}{s.f.}{Pedaço de madeira para o fogo.}{a.cha}{0}
\verb{acha}{}{}{}{}{s.f.}{Antigo machado, cujo corte era arredondado.}{a.cha}{0}
\verb{achacadiço}{}{}{}{}{adj.}{Que sofre de achaques com muita frequência.}{a.cha.ca.di.ço}{0}
\verb{achacar}{}{}{}{}{v.t.}{Maltratar física ou moralmente.}{a.cha.car}{0}
\verb{achacar}{}{}{}{}{v.pron.}{Sofrer o ataque de algum mal"-estar.}{a.cha.car}{\verboinum{2}}
\verb{achado}{}{}{}{}{adj.}{Que foi encontrado, descoberto.}{a.cha.do}{0}
\verb{achado}{}{}{}{}{s.m.}{Ato ou efeito de achar; achamento, descoberta.}{a.cha.do}{0}
\verb{achamalotado}{}{}{}{}{adj.}{Semelhante ao chamalote, que produz padrões ondulantes.}{a.cha.ma.lo.ta.do}{0}
\verb{achaque}{}{}{}{}{s.m.}{Indisposição recorrente sem causa conhecida que acomete alguns indivíduos.}{a.cha.que}{0}
\verb{achar}{}{}{}{}{v.t.}{Encontrar, descobrir.}{a.char}{0}
\verb{achar}{}{}{}{}{}{Ter sensação de, imaginar.}{a.char}{0}
\verb{achar}{}{}{}{}{v.pron.}{Localizar"-se.}{a.char}{\verboinum{1}}
\verb{achatado}{}{}{}{}{adj.}{Que sofreu achatamento; aplainado.}{a.cha.ta.do}{0}
\verb{achatado}{}{}{}{}{}{Que parece ter sofrido achatamento; plano.}{a.cha.ta.do}{0}
\verb{achatamento}{}{}{}{}{s.m.}{Ato ou efeito de achatar.}{a.cha.ta.men.to}{0}
\verb{achatar}{}{}{}{}{v.t.}{Tornar plano, amassando.}{a.cha.tar}{\verboinum{1}}
\verb{achavascado}{}{}{}{}{adj.}{Que tem aspecto grosseiro, rústico.}{a.cha.vas.ca.do}{0}
\verb{achega}{ê}{}{}{}{s.f.}{Complemento para uma peça principal.}{a.che.ga}{0}
\verb{achega}{ê}{}{}{}{}{Ajuda, contribuição.}{a.che.ga}{0}
\verb{achegar}{}{}{}{}{v.t.}{Colocar próximo, aproximar.}{a.che.gar}{0}
\verb{achegar}{}{}{}{}{v.pron.}{Aproximar"-se.}{a.che.gar}{0}
\verb{achegar}{}{}{}{}{}{Acomodar"-se.}{a.che.gar}{\verboinum{5}}
\verb{achego}{ê}{}{}{}{s.m.}{Amparo, ajuda.}{a.che.go}{0}
\verb{achego}{ê}{}{}{}{}{Contribuição, achega.}{a.che.go}{0}
\verb{achincalhação}{}{}{"-ões}{}{s.f.}{Ato ou efeito de achincalhar; humilhação, ridicularização.}{a.chin.ca.lha.ção}{0}
\verb{achincalhar}{}{}{}{}{v.t.}{Fazer zombaria, escarnecer, ridicularizar.}{a.chin.ca.lhar}{\verboinum{1}}
\verb{achincalhe}{}{}{}{}{s.m.}{Zombaria, ridicularização, achincalhação.}{a.chin.ca.lhe}{0}
\verb{achinesado}{}{}{}{}{adj.}{Que se assemelha às pessoas ou às coisas que são próprias da China.}{a.chi.ne.sa.do}{0}
\verb{achocolatado}{}{}{}{}{adj.}{Com sabor de chocolate ou que contém  chocolate}{a.cho.co.la.ta.do}{0}
\verb{achocolatado}{}{}{}{}{s.m.}{O próprio alimento que contém chocolate.}{a.cho.co.la.ta.do}{0}
\verb{aciaria}{}{}{}{}{s.f.}{A produção do aço.}{a.ci.a.ri.a}{0}
\verb{aciaria}{}{}{}{}{s.f.}{Empresa, ou parte dela, destinada à produção do aço.}{a.ci.a.ri.a}{0}
\verb{acicatar}{}{}{}{}{v.t.}{Estimular com acicate ou espora o animal em que se monta.}{a.ci.ca.tar}{0}
\verb{acicatar}{}{Por ext.}{}{}{}{Estimular.}{a.ci.ca.tar}{\verboinum{1}}
\verb{acicate}{}{}{}{}{s.m.}{Espora com uma só ponta de ferro usada para estimular o animal em que se monta.}{a.ci.ca.te}{0}
\verb{acicate}{}{}{}{}{}{Estímulo.}{a.ci.ca.te}{0}
\verb{acíclico}{}{}{}{}{adj.}{Que não tem comportamento repetitivo regular.}{a.cí.cli.co}{0}
\verb{acicular}{}{}{}{}{adj.2g.}{Que tem forma de agulha.}{a.ci.cu.lar}{0}
\verb{acidentado}{}{}{}{}{adj.}{Que tem muitos acidentes ou imprevistos.}{a.ci.den.ta.do}{0}
\verb{acidentado}{}{}{}{}{}{Que foi vítima de um acidente.}{a.ci.den.ta.do}{0}
\verb{acidental}{}{}{"-ais}{}{adj.2g.}{Que ocorreu inesperadamente, como um acidente.}{a.ci.den.tal}{0}
\verb{acidentar}{}{}{}{}{v.t.}{Produzir acidente.}{a.ci.den.tar}{0}
\verb{acidentar}{}{}{}{}{}{Tornar irregular, imprevisível.}{a.ci.den.tar}{0}
\verb{acidentar}{}{}{}{}{v.pron.}{Ser vítima de um acidente.}{a.ci.den.tar}{\verboinum{1}}
\verb{acidente}{}{}{}{}{s.m.}{Acontecimento inesperado, que foge à expectativa.}{a.ci.den.te}{0}
\verb{acidente}{}{}{}{}{}{Mudança repentina de um padrão.}{a.ci.den.te}{0}
\verb{acidente}{}{}{}{}{}{Desastre que coloca em risco a vida de pessoas.}{a.ci.den.te}{0}
\verb{acidez}{ê}{}{}{}{s.f.}{Qualidade do que é ou está ácido, azedo; azedume.}{a.ci.dez}{0}
\verb{acidificar}{}{}{}{}{v.t.}{Tornar ácido, acidular.}{a.ci.di.fi.car}{\verboinum{2}}
\verb{acidimetria}{}{Quím.}{}{}{s.f.}{Medição da acidez de uma solução.}{a.ci.di.me.tri.a}{0}
\verb{acidímetro}{}{Quím.}{}{}{s.m.}{Instrumento empregado para a medição da acidez de uma solução.}{a.ci.dí.me.tro}{0}
\verb{ácido}{}{Quím.}{}{}{s.m.}{Nome genérico das substâncias que doam prótons.}{á.ci.do}{0}
\verb{ácido}{}{}{}{}{}{Droga alucinógena conhecida como \textsc{lsd}.}{á.ci.do}{0}
\verb{ácido}{}{}{}{}{adj.}{Que dá sensação picante ao olfato ou ao paladar; azedo, acre.}{á.ci.do}{0}
\verb{ácido}{}{Fig.}{}{}{}{Desagradável.}{á.ci.do}{0}
\verb{ácido}{}{Ant.}{}{}{}{antiácido; alcalino; doce}{á.ci.do}{0}
\verb{acidose}{ó}{Med.}{}{}{s.f.}{Aumento dos ácidos nos líquidos do organismo.}{a.ci.do.se}{0}
\verb{acidulado}{}{}{}{}{adj.}{Tornado levemente ácido.}{a.ci.du.la.do}{0}
\verb{acidulante}{}{Quím.}{}{}{adj.2g.}{Diz"-se da substância que tem a propriedade de tornar ácida alguma coisa; acidificante.}{a.ci.du.lan.te}{0}
\verb{acidulante}{}{}{}{}{s.m.}{Essa substância.}{a.ci.du.lan.te}{0}
\verb{acidular}{}{}{}{}{v.t.}{Tornar ácido, acidificar.}{a.ci.du.lar}{\verboinum{1}}
\verb{acídulo}{}{}{}{}{adj.}{Que é levemente ácido.}{a.cí.du.lo}{0}
\verb{aciganado}{}{}{}{}{adj.}{Que se assemelha aos ciganos ou às coisas que são próprias dos ciganos.}{a.ci.ga.na.do}{0}
\verb{acima}{}{}{}{}{adv.}{Em lugar mais alto; sobre; na parte superior.}{a.ci.ma}{0}
\verb{acima}{}{}{}{}{}{Para cima.}{a.ci.ma}{0}
\verb{acinte}{}{}{}{}{s.m.}{Ação premeditada com intenção de provocar ou ofender.}{a.cin.te}{0}
\verb{acintoso}{ô}{}{"-osos ⟨ó⟩}{"-osa ⟨ó⟩}{adj.}{Que é feito com acinte; para ofensa ou provocação.}{a.cin.to.so}{0}
\verb{acinzentado}{}{}{}{}{adj.}{Que se acinzentou.}{a.cin.zen.ta.do}{0}
\verb{acinzentado}{}{}{}{}{}{Da cor próxima ao cinzento.}{a.cin.zen.ta.do}{0}
\verb{acinzentar}{}{}{}{}{v.t.}{Tornar cinzento.}{a.cin.zen.tar}{0}
\verb{acinzentar}{}{}{}{}{}{Escurecer.}{a.cin.zen.tar}{\verboinum{1}}
\verb{acionado}{}{}{}{}{adj.}{Que se acionou.}{a.ci.o.na.do}{0}
\verb{acionado}{}{}{}{}{}{Posto em ação, em movimento.}{a.ci.o.na.do}{0}
\verb{acionado}{}{Jur.}{}{}{}{Processado por ação judicial.}{a.ci.o.na.do}{0}
\verb{acionado}{}{Econ.}{}{}{}{Incorporado por ações.}{a.ci.o.na.do}{0}
\verb{acionado}{}{Desus.}{}{}{s.m.}{Gesticulação de quem fala.}{a.ci.o.na.do}{0}
\verb{acionador}{ô}{}{}{}{adj.}{Que aciona.}{a.ci.o.na.dor}{0}
\verb{acionador}{ô}{Jur.}{}{}{s.m.}{Indivíduo que demanda uma ação judicial.}{a.ci.o.na.dor}{0}
\verb{acionar}{}{}{}{}{v.t.}{Pôr em ação, em funcionamento.}{a.ci.o.nar}{0}
\verb{acionar}{}{Desus.}{}{}{}{Fazer acionados, gesticular.}{a.ci.o.nar}{0}
\verb{acionar}{}{Jur.}{}{}{}{Propor ação judicial contra; processar.}{a.ci.o.nar}{0}
\verb{acionar}{}{Econ.}{}{}{}{Incorporar ou fundir (sociedades, companhias) por ações.}{a.ci.o.nar}{\verboinum{1}}
\verb{acionista}{}{}{}{}{adj.2g.}{Que possui ações de uma companhia ou sociedade.}{a.ci.o.nis.ta}{0}
\verb{acionista}{}{}{}{}{s.2g.}{Indivíduo que possui ações de uma companhia ou sociedade.}{a.ci.o.nis.ta}{0}
\verb{acirrado}{}{}{}{}{adj.}{Que se acirrou.}{a.cir.ra.do}{0}
\verb{acirrado}{}{}{}{}{}{Irritado, exacerbado.}{a.cir.ra.do}{0}
\verb{acirrado}{}{}{}{}{}{Provocado, incitado.}{a.cir.ra.do}{0}
\verb{acirrado}{}{}{}{}{}{Estimulado, excitado.}{a.cir.ra.do}{0}
\verb{acirrado}{}{}{}{}{}{Obstinado, teimoso.}{a.cir.ra.do}{0}
\verb{acirramento}{}{}{}{}{s.m.}{Ato ou efeito de acirrar.}{a.cir.ra.men.to}{0}
\verb{acirramento}{}{}{}{}{}{Irritação.}{a.cir.ra.men.to}{0}
\verb{acirramento}{}{}{}{}{}{Estimulação.}{a.cir.ra.men.to}{0}
\verb{acirrar}{}{}{}{}{v.t.}{Irritar, exacerbar.}{a.cir.rar}{0}
\verb{acirrar}{}{}{}{}{}{Provocar, incitar.}{a.cir.rar}{0}
\verb{acirrar}{}{}{}{}{}{Estimular, excitar.}{a.cir.rar}{0}
\verb{acirrar}{}{Ant.}{}{}{}{brandar; acalmar}{a.cir.rar}{\verboinum{1}}
\verb{aclamação}{}{}{"-ões}{}{s.f.}{Ato ou efeito de aclamar, de aplaudir; ovação.}{a.cla.ma.ção}{0}
\verb{aclamação}{}{}{"-ões}{}{}{Reconhecimento solene da ascensão a algum cargo ou alguma função importante.}{a.cla.ma.ção}{0}
\verb{aclamar}{}{}{}{}{v.t.}{Aplaudir com entusiasmo.}{a.cla.mar}{0}
\verb{aclamar}{}{}{}{}{}{Proclamar, declarar.}{a.cla.mar}{0}
\verb{aclamar}{}{}{}{}{}{Eleger por aclamação.}{a.cla.mar}{0}
\verb{aclamar}{}{}{}{}{v.pron.}{Atribuir a si mesmo cargo ou função.}{a.cla.mar}{\verboinum{1}}
\verb{aclaração}{}{}{"-ões}{}{s.f.}{Ato ou efeito de aclarar.}{a.cla.ra.ção}{0}
\verb{aclaração}{}{}{"-ões}{}{}{Esclarecimento, explicação.}{a.cla.ra.ção}{0}
\verb{aclaração}{}{Jur.}{"-ões}{}{}{Aditamento que se faz a um texto legal com o propósito de esclarecer cláusulas ou artigos.}{a.cla.ra.ção}{0}
\verb{aclarar}{}{}{}{}{v.t.}{Tornar claro ou mais claro; iluminar.}{a.cla.rar}{0}
\verb{aclarar}{}{}{}{}{}{Esclarecer, explicar.}{a.cla.rar}{0}
\verb{aclarar}{}{}{}{}{}{Purificar, limpar.}{a.cla.rar}{0}
\verb{aclarar}{}{}{}{}{v.i.}{Amanhecer, alvorecer.}{a.cla.rar}{\verboinum{1}}
\verb{aclimação}{}{}{"-ões}{}{s.f.}{Aclimatação.}{a.cli.ma.ção}{0}
\verb{aclimar}{}{}{}{}{v.t.}{Aclimatar.}{a.cli.mar}{\verboinum{1}}
\verb{aclimatação}{}{}{"-ões}{}{s.f.}{Ato ou efeito de aclimatar; aclimação.}{a.cli.ma.ta.ção}{0}
\verb{aclimatação}{}{Biol.}{"-ões}{}{}{Adaptação de um ser vivo a um clima ou ambiente diverso do habitual.}{a.cli.ma.ta.ção}{0}
\verb{aclimatação}{}{Por ext.}{"-ões}{}{}{Acomodação.}{a.cli.ma.ta.ção}{0}
\verb{aclimatar}{}{}{}{}{v.t.}{Adaptar a um novo clima ou ambiente; aclimar.}{a.cli.ma.tar}{0}
\verb{aclimatar}{}{}{}{}{}{Harmonizar, ajustar.}{a.cli.ma.tar}{0}
\verb{aclimatar}{}{Por ext.}{}{}{}{Habituar, acostumar.}{a.cli.ma.tar}{\verboinum{1}}
\verb{aclive}{}{}{}{}{s.m.}{Inclinação do terreno (de baixo para cima); ladeira.}{a.cli.ve}{0}
\verb{aclive}{}{Ant.}{}{}{}{declive}{a.cli.ve}{0}
\verb{acne}{}{Med.}{}{}{s.f.}{Inflamação das glândulas sebáceas da pele, causada por acúmulo de secreção, caracterizada geralmente por espinhas.}{ac.ne}{0}
\verb{aço}{}{}{}{}{s.m.}{Liga de ferro com pequena porcentagem de carbono.}{a.ço}{0}
\verb{aço}{}{Por ext.}{}{}{}{Qualquer tipo de arma branca.}{a.ço}{0}
\verb{aço}{}{Fig.}{}{}{}{O que é duro, rígido e resistente como o aço.}{a.ço}{0}
\verb{aço}{}{Fig.}{}{}{}{Dureza, rigidez, resistência.}{a.ço}{0}
\verb{aço}{}{Pop.}{}{}{}{Cachaça.}{a.ço}{0}
\verb{acobardar}{}{}{}{}{}{Var. de \textit{acovardar}.}{a.co.bar.dar}{0}
\verb{acobertamento}{}{}{}{}{s.m.}{Ato ou efeito de acobertar.}{a.co.ber.ta.men.to}{0}
\verb{acobertamento}{}{}{}{}{}{Dissimulação, encobrimento.}{a.co.ber.ta.men.to}{0}
\verb{acobertar}{}{}{}{}{v.t.}{Colocar coberta ou manta; cobrir.}{a.co.ber.tar}{0}
\verb{acobertar}{}{}{}{}{}{Proteger, abrigar.}{a.co.ber.tar}{0}
\verb{acobertar}{}{}{}{}{}{Encobrir, esconder, dissimular.}{a.co.ber.tar}{0}
\verb{acobertar}{}{Ant.}{}{}{}{}{revelar; descobrir}{\verboinum{1}}
\verb{acobreado}{}{}{}{}{adj.}{Que se acobreou.}{a.co.bre.a.do}{0}
\verb{acobreado}{}{}{}{}{}{Que tem cor ou aspecto de cobre.}{a.co.bre.a.do}{0}
\verb{acobrear}{}{}{}{}{v.t.}{Dar cor ou aspecto de cobre.}{a.co.bre.ar}{0}
\verb{acobrear}{}{}{}{}{}{Revestir com cobre.}{a.co.bre.ar}{\verboinum{4}}
\verb{acochar}{}{}{}{}{v.t.}{Dispor em camadas, apertando ou comprimindo.}{a.co.char}{0}
\verb{acochar}{}{}{}{}{}{Apertar, comprimir, arrochar.}{a.co.char}{0}
\verb{acochar}{}{}{}{}{}{Achegar, aconchegar.}{a.co.char}{0}
\verb{acochar}{}{}{}{}{}{Aborrecer, importunar.}{a.co.char}{\verboinum{1}}
\verb{acocorado}{}{}{}{}{adj.}{Que se acocorou.}{a.co.co.ra.do}{0}
\verb{acocorado}{}{}{}{}{}{Posto de cócoras; agachado.}{a.co.co.ra.do}{0}
\verb{acocorar}{}{}{}{}{v.t.}{Abaixar na posição de cócoras.}{a.co.co.rar}{0}
\verb{acocorar}{}{}{}{}{}{Rebaixar moralmente, humilhar.}{a.co.co.rar}{\verboinum{1}}
\verb{açodado}{}{}{}{}{adj.}{Que se açodou.}{a.ço.da.do}{0}
\verb{açodado}{}{}{}{}{}{Precipitado, apressado, acelerado.}{a.ço.da.do}{0}
\verb{açodado}{}{}{}{}{}{Instigado, estimulado.}{a.ço.da.do}{0}
\verb{açodamento}{}{}{}{}{s.m.}{Ato ou efeito de açodar.}{a.ço.da.men.to}{0}
\verb{açodamento}{}{}{}{}{}{Pressa, precipitação.}{a.ço.da.men.to}{0}
\verb{açodar}{}{}{}{}{v.t.}{Precipitar, acelerar, apressar.}{a.ço.dar}{0}
\verb{açodar}{}{}{}{}{}{Instigar, estimular.}{a.ço.dar}{\verboinum{1}}
\verb{acogular}{}{}{}{}{v.t.}{Encher demasiadamente até formar cogulo; a ponto de transbordar.}{a.co.gu.lar}{\verboinum{1}}
\verb{acoimar}{}{}{}{}{v.t.}{Obrigar a pagar coima; multar.}{a.coi.mar}{0}
\verb{acoimar}{}{}{}{}{}{Castigar, punir.}{a.coi.mar}{0}
\verb{acoimar}{}{}{}{}{}{Repreender, censurar.}{a.coi.mar}{\verboinum{1}}
\verb{açoita"-cavalo}{}{Bot.}{açoita"-cavalos}{}{s.m.}{Árvore cujos galhos flexíveis servem de açoite, chicote.}{a.çoi.ta"-ca.va.lo}{0}
\verb{açoitamento}{}{}{}{}{s.m.}{Ato ou efeito de açoitar, fustigar.}{a.çoi.ta.men.to}{0}
\verb{acoitar}{}{}{}{}{v.t.}{Dar abrigo ou asilo a; acolher.}{a.coi.tar}{0}
\verb{acoitar}{}{}{}{}{}{Esconder, ocultar.}{a.coi.tar}{\verboinum{1}}
\verb{açoitar}{}{}{}{}{v.t.}{Golpear com açoite ou outro instrumento (chicote, vara).}{a.çoi.tar}{0}
\verb{açoitar}{}{}{}{}{}{Chocar, ir de encontro a.}{a.çoi.tar}{0}
\verb{açoitar}{}{}{}{}{}{Afligir, ferir.}{a.çoi.tar}{0}
\verb{açoitar}{}{}{}{}{}{Devastar, assolar.}{a.çoi.tar}{\verboinum{1}}
\verb{açoite}{}{}{}{}{s.m.}{Instrumento de tiras de couro; chicote, látego.}{a.çoi.te}{0}
\verb{açoite}{}{}{}{}{}{Golpe aplicado com esse instrumento ou qualquer outro (vara, cipó etc.) utilizado para castigar.}{a.çoi.te}{0}
\verb{acolá}{}{}{}{}{adv.}{Em lugar distante daquele da pessoa que fala ou com quem se fala; naquele lugar; além.}{a.co.lá}{0}
\verb{acolá}{}{}{}{}{}{Para aquele lugar; mais além.}{a.co.lá}{0}
\verb{acolchetar}{}{}{}{}{v.t.}{Colocar colchetes ou prender com colchetes.}{a.col.che.tar}{\verboinum{1}}
\verb{acolchoado}{}{}{}{}{adj.}{Tecido à maneira de colcha.}{a.col.cho.a.do}{0}
\verb{acolchoado}{}{}{}{}{}{Forrado ou estofado com tecido macio ou flexível, como o colchão.}{a.col.cho.a.do}{0}
\verb{acolchoado}{}{}{}{}{}{Edredom.}{a.col.cho.a.do}{0}
\verb{acolchoar}{}{}{}{}{v.t.}{Tecer à maneira de colcha.}{a.col.cho.ar}{0}
\verb{acolchoar}{}{}{}{}{}{Forrar ou estofar com material macio ou flexível, como o colchão.}{a.col.cho.ar}{\verboinum{7}}
\verb{acolhedor}{ô}{}{}{}{adj.}{Que acolhe bem; hospitaleiro, aconchegante.}{a.co.lhe.dor}{0}
\verb{acolher}{ê}{}{}{}{v.t.}{Abrigar, agasalhar, hospedar.}{a.co.lher}{0}
\verb{acolher}{ê}{}{}{}{}{Receber, admitir, aceitar.}{a.co.lher}{0}
\verb{acolher}{ê}{}{}{}{}{Levar em consideração; dar crédito a.}{a.co.lher}{\verboinum{12}}
\verb{acolhida}{}{}{}{}{s.f.}{Ato ou efeito de acolher; acolhimento.}{a.co.lhi.da}{0}
\verb{acolhida}{}{}{}{}{}{Hospitalidade, recepção, abrigo.}{a.co.lhi.da}{0}
\verb{acolhida}{}{}{}{}{}{Consideração, atenção.}{a.co.lhi.da}{0}
\verb{acolhido}{}{}{}{}{adj.}{Que se acolheu.}{a.co.lhi.do}{0}
\verb{acolhido}{}{}{}{}{}{Hospedado, abrigado.}{a.co.lhi.do}{0}
\verb{acolhido}{}{}{}{}{}{Recebido, admitido.}{a.co.lhi.do}{0}
\verb{acolhimento}{}{}{}{}{s.m.}{Acolhida.}{a.co.lhi.men.to}{0}
\verb{acolitar}{}{}{}{}{v.t.}{Servir ou ajudar como acólito.}{a.co.li.tar}{0}
\verb{acolitar}{}{}{}{}{}{Acompanhar, auxiliar.}{a.co.li.tar}{0}
\verb{acolitar}{}{}{}{}{v.i.}{Servir de acólito.}{a.co.li.tar}{\verboinum{1}}
\verb{acólito}{}{}{}{}{s.m.}{Indivíduo que acompanha, auxiliar; assistente, ajudante.}{a.có.li.to}{0}
\verb{acólito}{}{Relig.}{}{}{}{Indivíduo que recebeu o grau superior das ordens menores na Igreja Católica.}{a.có.li.to}{0}
\verb{acólito}{}{Relig.}{}{}{}{Indivíduo que acompanha e auxilia o celebrante na condução dos atos litúrgicos.}{a.có.li.to}{0}
\verb{acometer}{ê}{}{}{}{v.t.}{Investir contra; atacar.}{a.co.me.ter}{0}
\verb{acometer}{ê}{}{}{}{}{Insultar, hostilizar, provocar.}{a.co.me.ter}{0}
\verb{acometer}{ê}{}{}{}{}{Dominar, subornar, seduzir.}{a.co.me.ter}{0}
\verb{acometer}{ê}{}{}{}{v.i.}{Iniciar ataque ou assalto.}{a.co.me.ter}{\verboinum{12}}
\verb{acometida}{}{}{}{}{s.f.}{Acometimento.}{a.co.me.ti.da}{0}
\verb{acometimento}{}{}{}{}{s.m.}{Ato ou efeito de acometer; acometida.}{a.co.me.ti.men.to}{0}
\verb{acometimento}{}{}{}{}{}{Investida, ataque, assalto.}{a.co.me.ti.men.to}{0}
\verb{acometimento}{}{}{}{}{}{Insulto, hostilização.}{a.co.me.ti.men.to}{0}
\verb{acomodação}{}{}{"-ões}{}{s.f.}{Ato ou efeito de acomodar.}{a.co.mo.da.ção}{0}
\verb{acomodação}{}{}{"-ões}{}{}{Adaptação, adequação.}{a.co.mo.da.ção}{0}
\verb{acomodação}{}{}{"-ões}{}{}{Divisão ou compartimento de um local; cômodo.}{a.co.mo.da.ção}{0}
\verb{acomodação}{}{}{"-ões}{}{}{Falta de ambição; conformismo.}{a.co.mo.da.ção}{0}
\verb{acomodado}{}{}{}{}{adj.}{Que se acomodou.}{a.co.mo.da.do}{0}
\verb{acomodado}{}{}{}{}{}{Adequado, adaptado.}{a.co.mo.da.do}{0}
\verb{acomodado}{}{}{}{}{}{Sossegado, tranquilizado.}{a.co.mo.da.do}{0}
\verb{acomodado}{}{}{}{}{}{Conformado, resignado.}{a.co.mo.da.do}{0}
\verb{acomodar}{}{}{}{}{v.t.}{Dispor em lugar cômodo, conveniente.}{a.co.mo.dar}{0}
\verb{acomodar}{}{}{}{}{}{Tornar cômodo, confortável.}{a.co.mo.dar}{0}
\verb{acomodar}{}{}{}{}{}{Dar ocupação a, empregar.}{a.co.mo.dar}{0}
\verb{acomodar}{}{}{}{}{v.pron.}{Retirar"-se para o quarto ou para os aposentos; recolher"-se.}{a.co.mo.dar}{0}
\verb{acomodar}{}{}{}{}{}{Adaptar"-se, conformar"-se a uma situação.}{a.co.mo.dar}{\verboinum{1}}
\verb{acomodatício}{}{}{}{}{adj.}{Que se acomoda facilmente; adaptável, maleável.}{a.co.mo.da.tí.cio}{0}
\verb{acomodatício}{}{}{}{}{}{Transigente, tolerante.}{a.co.mo.da.tí.cio}{0}
\verb{acompadrar"-se}{}{}{}{}{v.pron.}{Tornar"-se amigo, camarada.}{a.com.pa.drar"-se}{\verboinum{1}}
\verb{acompanhador}{ô}{}{}{}{adj.}{Que acompanha; acompanhante.}{a.com.pa.nha.dor}{0}
\verb{acompanhamento}{}{}{}{}{s.m.}{Ato ou efeito de acompanhar.}{a.com.pa.nha.men.to}{0}
\verb{acompanhamento}{}{}{}{}{}{Comitiva, cortejo, séquito.}{a.com.pa.nha.men.to}{0}
\verb{acompanhamento}{}{}{}{}{}{Assistência dada por profissional a quem esteja sob seus cuidados.}{a.com.pa.nha.men.to}{0}
\verb{acompanhamento}{}{}{}{}{}{O que acompanha ou segue.}{a.com.pa.nha.men.to}{0}
\verb{acompanhamento}{}{Mús.}{}{}{}{Parte secundária da música que segue a melodia principal.}{a.com.pa.nha.men.to}{0}
\verb{acompanhante}{}{}{}{}{adj.2g.}{Que acompanha.}{a.com.pa.nhan.te}{0}
\verb{acompanhante}{}{}{}{}{s.2g.}{Indivíduo que acompanha outro, exercendo função secundária.}{a.com.pa.nhan.te}{0}
\verb{acompanhante}{}{}{}{}{}{Pessoa que presta assistência a doentes, inválidos ou idosos.}{a.com.pa.nhan.te}{0}
\verb{acompanhar}{}{}{}{}{v.t.}{Ir em companhia de.}{a.com.pa.nhar}{0}
\verb{acompanhar}{}{}{}{}{}{Seguir a mesma direção de.}{a.com.pa.nhar}{0}
\verb{acompanhar}{}{}{}{}{}{Observar a marcha de; assistir.}{a.com.pa.nhar}{0}
\verb{acompanhar}{}{}{}{}{}{Participar das mesmas ideias ou sentimentos.}{a.com.pa.nhar}{0}
\verb{acompanhar}{}{}{}{}{}{Unir, aliar.}{a.com.pa.nhar}{0}
\verb{acompanhar}{}{}{}{}{}{Estar associado a.}{a.com.pa.nhar}{0}
\verb{acompanhar}{}{Mús.}{}{}{}{Executar o acompanhamento.}{a.com.pa.nhar}{\verboinum{1}}
\verb{aconchegante}{}{}{}{}{adj.2g.}{Que aconchega, agasalha, protege; confortável.}{a.con.che.gan.te}{0}
\verb{aconchegar}{}{}{}{}{v.t.}{Aproximar, procurando conforto, abrigo, alento; chegar perto, achegar, conchegar.}{a.con.che.gar}{\verboinum{5}}
\verb{aconchego}{ê}{}{}{}{s.m.}{Ato ou efeito de aconchegar, chegar junto de si.}{a.con.che.go}{0}
\verb{aconchego}{ê}{}{}{}{}{Conforto, alento, abrigo.}{a.con.che.go}{0}
\verb{aconchego}{ê}{}{}{}{}{Pessoa que protege; amparo. }{a.con.che.go}{0}
\verb{acondicionado}{}{}{}{}{adj.}{Que está de acordo com; adequado.}{a.con.di.ci.o.na.do}{0}
\verb{acondicionado}{}{}{}{}{}{Colocado em ordem; empacotado, guardado, embalado.}{a.con.di.ci.o.na.do}{0}
\verb{acondicionamento}{}{}{}{}{s.m.}{Ato ou efeito de acondicionar, armazenar; empacotamento.}{a.con.di.ci.o.na.men.to}{0}
\verb{acondicionar}{}{}{}{}{v.t.}{Guardar em lugar apropriado; armazenar; colocar em ordem.}{a.con.di.ci.o.nar}{\verboinum{1}}
\verb{acônito}{}{Bot.}{}{}{s.m.}{Planta venenosa de uso medicinal.}{a.cô.ni.to}{0}
\verb{acônito}{}{}{}{}{}{Medicamento preparado com essa planta e usado como sedativo cardíaco, respiratório e analgésico.}{a.cô.ni.to}{0}
\verb{aconselhamento}{}{}{}{}{s.m.}{Ato ou efeito de aconselhar.}{a.con.se.lha.men.to}{0}
\verb{aconselhamento}{}{}{}{}{}{Etapa de orientação pedagógica.}{a.con.se.lha.men.to}{0}
\verb{aconselhamento}{}{}{}{}{}{Assistência psicológica.}{a.con.se.lha.men.to}{0}
\verb{aconselhar}{}{}{}{}{v.t.}{Dar conselho a.}{a.con.se.lhar}{0}
\verb{aconselhar}{}{}{}{}{}{Indicar, recomendar, sugerir.}{a.con.se.lhar}{0}
\verb{aconselhar}{}{}{}{}{}{Procurar convencer, persuadir.}{a.con.se.lhar}{\verboinum{1}}
\verb{aconselhável}{}{}{"-eis}{}{adj.2g.}{Que se pode ou se deve aconselhar; recomendável. }{a.con.se.lhá.vel}{0}
\verb{acontecer}{ê}{}{}{}{v.i.}{Tornar"-se realidade; suceder inesperadamente; ocorrer.}{a.con.te.cer}{\verboinum{15}}
\verb{acontecido}{}{}{}{}{adj.}{Que aconteceu; sucedido, ocorrido.}{a.con.te.ci.do}{0}
\verb{acontecido}{}{}{}{}{s.m.}{Acontecimento, ocorrência.}{a.con.te.ci.do}{0}
\verb{acontecimento}{}{}{}{}{s.m.}{Que acontece ou aconteceu; ocorrência.}{a.con.te.ci.men.to}{0}
\verb{acontecimento}{}{}{}{}{}{Fato ou pessoa que causa sensação; sucesso.}{a.con.te.ci.men.to}{0}
\verb{acontecimento}{}{}{}{}{}{Eventualidade, acaso.}{a.con.te.ci.men.to}{0}
\verb{acoplagem}{}{}{"-ens}{}{s.f.}{Acoplamento.}{a.co.pla.gem}{0}
\verb{acoplamento}{}{}{}{}{s.m.}{Ato ou efeito de acoplar; acoplagem.}{a.co.pla.men.to}{0}
\verb{acoplamento}{}{Fís.}{}{}{}{Ligação ou conexão entre dois sistemas, transferindo"-se energia de um para outro.}{a.co.pla.men.to}{0}
\verb{acoplamento}{}{Astron.}{}{}{}{Junção ou união de dois componentes de uma nave ou de uma estação espacial.}{a.co.pla.men.to}{0}
\verb{acoplar}{}{}{}{}{v.t.}{Realizar acoplamento; unir dois a dois; formar par.}{a.co.plar}{\verboinum{1}}
\verb{açor}{ô}{Zool.}{}{}{s.m.}{Ave de rapina, de hábitos diurnos, semelhante ao falcão.}{a.çor}{0}
\verb{acorçoar}{}{}{}{}{}{Var. de \textit{acoroçoar}.}{a.cor.ço.ar}{0}
\verb{açorda}{ô}{Cul.}{}{}{s.f.}{Sopa portuguesa, feita de miolo de pão, temperada com azeite, alho e coentro.}{a.çor.da}{0}
\verb{acordado}{}{}{}{}{adj.}{Que acordou ou foi acordado; desperto.}{a.cor.da.do}{0}
\verb{acordado}{}{}{}{}{}{Animado, excitado.}{a.cor.da.do}{0}
\verb{acordado}{}{}{}{}{adj.}{Que foi resolvido em acordo; combinado.}{a.cor.da.do}{0}
\verb{acórdão}{}{Jur.}{"-ãos}{}{s.m.}{Sentença final sobre um recurso, proferida em tribunais coletivos; aresto.}{a.cór.dão}{0}
\verb{acordar}{}{}{}{}{v.t.}{Tirar do sono; despertar.}{a.cor.dar}{0}
\verb{acordar}{}{}{}{}{v.i.}{Sair do estado de sono ou de sonolência.}{a.cor.dar}{\verboinum{1}}
\verb{acordar}{}{}{}{}{v.t.}{Chegar a um acordo; concordar.}{a.cor.dar}{0}
\verb{acordar}{}{Mús.}{}{}{}{Afinar, harmonizar.}{a.cor.dar}{\verboinum{1}}
\verb{acorde}{ó}{}{}{}{adj.}{Que está de acordo; em harmonia; concorde.}{a.cor.de}{0}
\verb{acorde}{ó}{Mús.}{}{}{s.m.}{Conjunto de três ou mais sons diferentes combinados harmonicamente.}{a.cor.de}{0}
\verb{acordeão}{}{Mús.}{"-ões}{}{s.m.}{Instrumento de sopro, dotado de fole, teclado e botões, semelhante à sanfona e à harmônica.}{a.cor.de.ão}{0}
\verb{acordeonista}{}{}{}{}{s.2g.}{Pessoa que toca ou fabrica acordeões.}{a.cor.de.o.nis.ta}{0}
\verb{acordo}{ô}{}{}{}{s.m.}{Pacto, combinação, ajuste.}{a.cor.do}{0}
\verb{acordo}{ô}{}{}{}{}{Concordância, harmonia.}{a.cor.do}{0}
\verb{acordo}{ô}{}{}{}{s.m.}{Domínio perfeito dos sentidos; conhecimento inteiro.}{a.cor.do}{0}
\verb{acordo}{ó}{Mús.}{}{}{s.m.}{Instrumento italiano de 15 cordas, popular nos séculos \textsc{xvii} e \textsc{xviii}.}{a.cor.do}{0}
\verb{açoriano}{}{}{}{}{adj.}{Relativo aos Açores.}{a.ço.ri.a.no}{0}
\verb{açoriano}{}{}{}{}{s.m.}{Indivíduo natural ou habitante dos Açores.}{a.ço.ri.a.no}{0}
\verb{acoroçoar}{}{}{}{}{v.t.}{Inspirar coragem; animar, estimular, acorçoar.}{a.co.ro.ço.ar}{\verboinum{7}}
\verb{acorrentar}{}{}{}{}{v.t.}{Prender com corrente; encadear.}{a.cor.ren.tar}{0}
\verb{acorrentar}{}{Fig.}{}{}{}{Subjugar, escravizar.}{a.cor.ren.tar}{\verboinum{1}}
\verb{acorrer}{ê}{}{}{}{v.i.}{Acudir, socorrer.}{a.cor.rer}{0}
\verb{acorrer}{ê}{}{}{}{}{Dirigir"-se diretamente a algum lugar.}{a.cor.rer}{0}
\verb{acorrer}{ê}{}{}{}{}{Voltar, especialmente, à memória. (\textit{Aquelas palavras acudiram à minha memória.})}{a.cor.rer}{\verboinum{12}}
\verb{acossado}{}{}{}{}{adj.}{Que se acossou.}{a.cos.sa.do}{0}
\verb{acossado}{}{}{}{}{}{Acuado, perseguido.}{a.cos.sa.do}{0}
\verb{acossado}{}{}{}{}{}{Ferido, agredido.}{a.cos.sa.do}{0}
\verb{acossar}{}{}{}{}{v.t.}{Seguir ao encalço de; acuar, perseguir.}{a.cos.sar}{0}
\verb{acossar}{}{}{}{}{}{Agredir, ferir, machucar.}{a.cos.sar}{\verboinum{1}}
\verb{acostamento}{}{}{}{}{s.m.}{Ato ou efeito de acostar.}{a.cos.ta.men.to}{0}
\verb{acostamento}{}{}{}{}{}{Área contígua à pista das rodovias, destinada a paradas de emergência de veículos.}{a.cos.ta.men.to}{0}
\verb{acostar}{}{}{}{}{v.t.}{Encostar junto a.}{a.cos.tar}{0}
\verb{acostar}{}{}{}{}{v.pron.}{Aproximar"-se da costa; costear.}{a.cos.tar}{0}
\verb{acostar}{}{}{}{}{}{Buscar auxílio.}{a.cos.tar}{0}
\verb{acostar}{}{}{}{}{}{Deitar"-se, recostar"-se.}{a.cos.tar}{\verboinum{1}}
\verb{acostumar}{}{}{}{}{v.t.}{Tomar o costume de; habituar, adaptar.}{a.cos.tu.mar}{\verboinum{1}}
\verb{acotiledôneo}{}{Bot.}{}{}{adj.}{Diz"-se da planta que não possui cotilédone ou folha primordial.}{a.co.ti.le.dô.neo}{0}
\verb{acotovelar}{}{}{}{}{v.t.}{Tocar com o cotovelo, para chamar a atenção de alguém.}{a.co.to.ve.lar}{0}
\verb{acotovelar}{}{}{}{}{}{Dar cotoveladas; empurrar.}{a.co.to.ve.lar}{0}
\verb{acotovelar}{}{}{}{}{v.pron.}{Amontoar"-se, disputando espaço com outras pessoas por meio de cotoveladas.}{a.co.to.ve.lar}{\verboinum{1}}
\verb{açougue}{}{}{}{}{s.m.}{Local onde se vendem carnes.}{a.çou.gue}{0}
\verb{açougue}{}{Fig.}{}{}{}{Matança, carnificina.}{a.çou.gue}{0}
\verb{açougueiro}{ê}{}{}{}{s.m.}{Proprietário ou funcionário de açougue.}{a.çou.guei.ro}{0}
\verb{açougueiro}{ê}{Fig.}{}{}{}{Indivíduo carniceiro, sanguinário.}{a.çou.guei.ro}{0}
\verb{açougueiro}{ê}{Fig.}{}{}{}{Mau cirurgião ou dentista.}{a.çou.guei.ro}{0}
\verb{acovardar}{}{}{}{}{v.t.}{Tornar covarde, amedrontar, intimidar.}{a.co.var.dar}{0}
\verb{acovardar}{}{}{}{}{}{Fazer perder a coragem, desanimar; acobardar.}{a.co.var.dar}{\verboinum{1}}
\verb{acracia}{}{}{}{}{s.f.}{Ausência ou negação de governo; falta de autoridade; anarquia.}{a.cra.ci.a}{0}
\verb{acracia}{}{Med.}{}{}{}{Fraqueza, debilidade.}{a.cra.ci.a}{0}
\verb{acrania}{}{Med.}{}{}{s.f.}{Ausência total ou parcial do crânio.}{a.cra.ni.a}{0}
\verb{acre}{}{}{}{}{adj.}{Que tem sabor azedo ou ácido.}{a.cre}{0}
\verb{acre}{}{}{}{}{}{De cheiro forte e seco.}{a.cre}{0}
\verb{acre}{}{Fig.}{}{}{}{Rude, áspero, mordaz.}{a.cre}{0}
\verb{acre}{}{}{}{}{s.m.}{Unidade de medida agrária inglesa, equivalente a 4.047 m\textsuperscript{2}.}{a.cre}{0}
\verb{acreditar}{}{}{}{}{v.t.}{Admitir como verdadeiro ou existente; crer.}{a.cre.di.tar}{0}
\verb{acreditar}{}{}{}{}{}{Dar crédito a; afiançar.}{a.cre.di.tar}{0}
\verb{acreditar}{}{}{}{}{}{Conferir autoridade a alguém para representar uma nação num país estrangeiro; credenciar.}{a.cre.di.tar}{\verboinum{1}}
\verb{acrescentar}{}{}{}{}{v.t.}{Juntar uma coisa a outra; adicionar.}{a.cres.cen.tar}{0}
\verb{acrescentar}{}{}{}{}{}{Fazer crescer alguém em bens, vantagens ou graças.}{a.cres.cen.tar}{0}
\verb{acrescentar}{}{Ant.}{}{}{}{diminuir; desuniri; retirar}{a.cres.cen.tar}{\verboinum{1}}
\verb{acrescer}{ê}{}{}{}{v.t.}{Tornar maior; aumentar.}{a.cres.cer}{0}
\verb{acrescer}{ê}{}{}{}{}{Juntar, acrescentar.}{a.cres.cer}{0}
\verb{acrescer}{ê}{}{}{}{}{Incorporar, incluir.}{a.cres.cer}{0}
\verb{acrescer}{ê}{Ant.}{}{}{}{decrescer; diminuir}{a.cres.cer}{\verboinum{15}}
\verb{acréscimo}{}{}{}{}{s.m.}{Aquilo que se acrescenta.}{a.crés.ci.mo}{0}
\verb{acréscimo}{}{}{}{}{}{Aumento, elevação, adicionamento.}{a.crés.ci.mo}{0}
\verb{acriançado}{}{}{}{}{adj.}{Que tem aparência ou modos infantis, próprios de criança.}{a.cri.an.ça.do}{0}
\verb{acriano}{}{}{}{}{adj.}{Relativo ao estado do Acre.}{a.cri.a.no}{0}
\verb{acriano}{}{}{}{}{s.m.}{Indivíduo natural ou habitante desse estado.}{a.cri.a.no}{0}
\verb{acrídio}{}{Zool.}{}{}{s.m.}{Gênero de insetos a que pertence o gafanhoto.}{a.crí.dio}{0}
\verb{acrílico}{}{}{}{}{s.m.}{Denominação genérica de várias resinas sintéticas usadas na fabricação de plásticos.}{a.crí.li.co}{0}
\verb{acrílico}{}{}{}{}{adj.}{Feito ou derivado dessa resina.}{a.crí.li.co}{0}
\verb{acrimônia}{}{}{}{}{s.f.}{Estado ou qualidade do que é acre, ácido.}{a.cri.mô.nia}{0}
\verb{acrimônia}{}{}{}{}{}{Comportamento áspero, mordaz.}{a.cri.mô.nia}{0}
\verb{acrimonioso}{ô}{}{"-osos ⟨ó⟩}{"-osa ⟨ó⟩}{adj.}{Que tem acrimônia.}{a.cri.mo.ni.o.so}{0}
\verb{acrimonioso}{ô}{}{"-osos ⟨ó⟩}{"-osa ⟨ó⟩}{}{Rude, áspero, mordaz.}{a.cri.mo.ni.o.so}{0}
\verb{acrisolado}{}{}{}{}{adj.}{Que foi purificado no crisol.}{a.cri.so.la.do}{0}
\verb{acrisolado}{}{}{}{}{}{Depurado, aperfeiçoado, apurado.}{a.cri.so.la.do}{0}
\verb{acrisolar}{}{}{}{}{v.t.}{Purificar metais no crisol.}{a.cri.so.lar}{0}
\verb{acrisolar}{}{}{}{}{}{Depurar, aperfeiçoar, apurar.}{a.cri.so.lar}{\verboinum{1}}
\verb{acrobacia}{}{}{}{}{s.f.}{Exercícios executados por acrobata; acrobatismo.}{a.cro.ba.ci.a}{0}
\verb{acrobacia}{}{}{}{}{}{Movimento que demonstra audácia e destreza.}{a.cro.ba.ci.a}{0}
\verb{acrobacia}{}{Fig.}{}{}{}{Ação que revela habilidade ou astúcia.}{a.cro.ba.ci.a}{0}
\verb{acrobata}{}{}{}{}{s.2g.}{Indivíduo que executa movimentos audaciosos e habilidosos; ginasta, malabarista.}{a.cro.ba.ta}{0}
\verb{acrobata}{}{}{}{}{}{Indivíduo hábil, audacioso.}{a.cro.ba.ta}{0}
\verb{acrobático}{}{}{}{}{adj.}{Relativo a acrobata ou à acrobacia.}{a.cro.bá.ti.co}{0}
\verb{acrobatismo}{}{}{}{}{s.m.}{Acrobacia.}{a.cro.ba.tis.mo}{0}
\verb{acrobatismo}{}{Fig.}{}{}{}{Instabilidade de opiniões; volubilidade.}{a.cro.ba.tis.mo}{0}
\verb{acrocianose}{ó}{Med.}{}{}{s.f.}{Distúrbio circulatório em que as mãos e os pés se apresentam frios, azulados e úmidos.}{a.cro.ci.a.no.se}{0}
\verb{acrofobia}{}{Med.}{}{}{s.f.}{Medo doentio de altura.}{a.cro.fo.bi.a}{0}
\verb{acromático}{}{}{}{}{adj.}{Que não apresenta ou não distingue cores.}{a.cro.má.ti.co}{0}
\verb{acromático}{}{Biol.}{}{}{}{Que não possui cromatina, que é difícil de corar.}{a.cro.má.ti.co}{0}
\verb{acromegalia}{}{Med.}{}{}{s.f.}{Síndrome clínica, resultante de prolongada e excessiva secreção do hormônio de crescimento pela hipófise, caracterizada pelo maior crescimento dos tecidos moles, incluindo nariz, boca e orelhas, e pelo aumento do tamanho dos ossos das mãos e dos pés.}{a.cro.me.ga.li.a}{0}
\verb{acrônimo}{}{}{}{}{s.m.}{Palavra formada pelas letras iniciais das partes de uma locução ou pela maioria dessas partes.}{a.crô.ni.mo}{0}
\verb{acrópole}{}{}{}{}{s.f.}{Local mais elevado das antigas cidades gregas, onde se construía a cidadela, com templos e palácios.}{a.cró.po.le}{0}
\verb{acróstico}{}{}{}{}{s.m.}{Composição poética em que as primeiras letras de cada verso formam verticalmente uma palavra que lhe serve de tema ou conceito.}{a.crós.ti.co}{0}
\verb{actínia}{}{Zool.}{}{}{s.f.}{Animal marinho de espécimes vivamente coloridas com tentáculos móveis e urticantes, que vive fixado às rochas litorâneas; anêmona"-do"-mar.}{ac.tí.nia}{0}
\verb{actinídeos}{}{Quím.}{}{}{s.m.}{Série de quinze elementos químicos metálicos pesados, radioativos, de número atômico crescente. }{ac.ti.ní.de.os}{0}
\verb{actínio}{}{Quím.}{}{}{s.m.}{Elemento químico metálico, radioativo, de cor branco prateado, do grupo dos actinídeos. \elemento{89}{(227)}{Ac}.}{ac.tí.nio}{0}
\verb{acuação}{}{}{"-ões}{}{s.f.}{Ato ou efeito de acuar.}{a.cu.a.ção}{0}
\verb{acuação}{}{}{"-ões}{}{}{Perseguição da caça, obrigando"-a a se refugiar na toca.}{a.cu.a.ção}{0}
\verb{acuado}{}{}{}{}{adj.}{Que está agachado, acocorado.}{a.cu.a.do}{0}
\verb{acuado}{}{}{}{}{}{Diz"-se do animal que está empacado, emperrado.}{a.cu.a.do}{0}
\verb{acuado}{}{}{}{}{}{Que está parado em atitude defensiva.}{a.cu.a.do}{0}
\verb{acuamento}{}{}{}{}{s.m.}{Ato ou efeito de acuar.}{a.cu.a.men.to}{0}
\verb{acuamento}{}{}{}{}{}{Retirada humilhante; embaraço.}{a.cu.a.men.to}{0}
\verb{acuar}{}{}{}{}{v.t.}{Posicionar"-se com as pernas dobradas e o corpo encolhido próximo ao chão, preparando"-se para o salto.}{a.cu.ar}{0}
\verb{acuar}{}{}{}{}{}{Parar diante de perigo ou ameaça; agir defensivamente; recuar.}{a.cu.ar}{\verboinum{1}}
\verb{açúcar}{}{}{}{}{s.m.}{Produto alimentar de sabor doce, fabricado industrialmente, geralmente extraído da cana"-de"-açúcar e da beterraba.}{a.çú.car}{0}
\verb{açúcar}{}{Fig.}{}{}{}{Doçura; suavidade.}{a.çú.car}{0}
\verb{açucarado}{}{}{}{}{adj.}{Que se açucarou; temperado com açúcar; adoçado.}{a.çu.ca.ra.do}{0}
\verb{açucarar}{}{}{}{}{v.t.}{Adoçar com açúcar; tornar doce.}{a.çu.ca.rar}{0}
\verb{açucarar}{}{}{}{}{}{Adquirir a consistência do açúcar.}{a.çu.ca.rar}{\verboinum{1}}
\verb{açúcar"-cande}{}{}{açúcares"-candes \textit{ou} açúcares"-cande}{}{s.m.}{Açúcar obtido pela cristalização da sacarose; cristalizado.}{a.çú.car"-can.de}{0}
\verb{açucareiro}{ê}{}{}{}{adj.}{Que é relativo ao açúcar.}{a.çu.ca.rei.ro}{0}
\verb{açucareiro}{ê}{}{}{}{}{Que fabrica ou negocia o açúcar.}{a.çu.ca.rei.ro}{0}
\verb{açucareiro}{ê}{}{}{}{s.m.}{Recipiente em que se serve o açúcar à mesa.}{a.çu.ca.rei.ro}{0}
\verb{açucareiro}{ê}{}{}{}{}{Indivíduo que trabalha com a fabricação e o comércio do açúcar.}{a.çu.ca.rei.ro}{0}
\verb{açucena}{ê}{Bot.}{}{}{s.f.}{Planta ornamental cuja flor perfumada é da família do lírio. }{a.çu.ce.na}{0}
\verb{açucena}{ê}{}{}{}{}{A flor dessa planta.}{a.çu.ce.na}{0}
\verb{açudagem}{}{}{"-ens}{}{s.f.}{Ato ou efeito de açudar, de deter o curso de águas.}{a.çu.da.gem}{0}
\verb{açudar}{}{}{}{}{v.t.}{Represar água no açude.}{a.çu.dar}{\verboinum{1}}
\verb{açude}{}{}{}{}{s.m.}{Construção destinada a represar águas para serem usadas na geração de força, na agricultura ou no abastecimento.}{a.çu.de}{0}
\verb{acudir}{}{}{}{}{v.t.}{Estar disposto a auxiliar; pôr"-se à disposição; socorrer.}{a.cu.dir}{0}
\verb{acudir}{}{}{}{}{}{Atender; arcar.}{a.cu.dir}{0}
\verb{acudir}{}{}{}{}{}{Responder prontamente.}{a.cu.dir}{\verboinum{18}}
\verb{acuidade}{}{}{}{}{s.f.}{Qualidade do que é agudo; agudeza.}{a.cui.da.de}{0}
\verb{acuidade}{}{Fig.}{}{}{}{Grande capacidade de percepção.}{a.cui.da.de}{0}
\verb{aculálio}{}{Med.}{}{}{s.m.}{Aparelho com o qual se ensinam os surdos"-mudos a articularem os sons da fala.}{a.cu.lá.lio}{0}
\verb{açular}{}{}{}{}{v.t.}{Incitar o cão a morder.}{a.çu.lar}{0}
\verb{açular}{}{}{}{}{}{Estimular; provocar; excitar.}{a.çu.lar}{\verboinum{1}}
\verb{acúleo}{}{}{}{}{s.m.}{Ponta aguçada; espinho.}{a.cú.leo}{0}
\verb{acúleo}{}{Bot.}{}{}{}{Formação epidérmica dura e pontiaguda de alguns vegetais, de origem foliar, que pode ser tirada com facilidade, sem lesar o vegetal, o que a difere dos espinhos.}{a.cú.leo}{0}
\verb{acúleo}{}{Zool.}{}{}{}{Ferrão dos insetos.}{a.cú.leo}{0}
\verb{aculturação}{}{}{"-ões}{}{s.f.}{Ato, processo ou efeito de aculturar;  modificação cultural aparente de indivíduo ou grupo sofrida sob a influência de outras culturas com que entra em contato.}{a.cul.tu.ra.ção}{0}
\verb{aculturado}{}{}{}{}{adj.}{Que sofreu o processo de aculturação; adaptado a outra cultura.}{a.cul.tu.ra.do}{0}
\verb{aculturar}{}{}{}{}{v.t.}{Promover a mudança de hábitos culturais de um povo no sentido de torná"-lo semelhante àquele que o domina ou influencia.}{a.cul.tu.rar}{0}
\verb{aculturar}{}{}{}{}{v.pron.}{Submeter"-se a esse processo; adaptar"-se a outra cultura.}{a.cul.tu.rar}{\verboinum{1}}
\verb{acume}{}{}{}{}{s.m.}{Ponta aguda ou penetrante; gume.}{a.cu.me}{0}
\verb{acume}{}{}{}{}{}{Parte mais alta; cume.}{a.cu.me}{0}
\verb{acume}{}{}{}{}{}{Agudeza; astúcia.}{a.cu.me}{0}
\verb{acuminado}{}{}{}{}{adj.}{Que se acuminou, afinando em ponta comprida e aguda; aguçado; pontiagudo.}{a.cu.mi.na.do}{0}
\verb{acumpliciar}{}{}{}{}{v.t.}{Tornar cúmplice ou conivente; cumpliciar.}{a.cum.pli.ci.ar}{\verboinum{6}}
\verb{acumulação}{}{}{"-ões}{}{s.f.}{Ato ou efeito de acumular; acúmulo; amontoamento.}{a.cu.mu.la.ção}{0}
\verb{acumulação}{}{}{"-ões}{}{}{Armazenamento em grande quantidade.}{a.cu.mu.la.ção}{0}
\verb{acumulação}{}{}{"-ões}{}{}{Aumento; ampliação; acréscimo.}{a.cu.mu.la.ção}{0}
\verb{acumulada}{}{}{}{}{s.f.}{Aposta nas corridas de cavalos de vários páreos, na qual o dinheiro aplicado se acumula à medida que os cavalos apostados vencem.}{a.cu.mu.la.da}{0}
\verb{acumulador}{ô}{}{}{}{adj.}{Que acumula.}{a.cu.mu.la.dor}{0}
\verb{acumulador}{ô}{Bras.}{}{}{s.m.}{Aparelho que armazena energia,  restituindo"-a sob a forma de eletricidade.}{a.cu.mu.la.dor}{0}
\verb{acumular}{}{}{}{}{v.t.}{Pôr ou dispor em cúmulo; amontoar; ajuntar.}{a.cu.mu.lar}{0}
\verb{acumular}{}{}{}{}{}{Juntar; reunir.}{a.cu.mu.lar}{0}
\verb{acumular}{}{}{}{}{}{Exercer simultaneamente.}{a.cu.mu.lar}{\verboinum{1}}
\verb{acúmulo}{}{}{}{}{s.m.}{Amontoamento de coisas; acumulação; sobrecarga.}{a.cú.mu.lo}{0}
\verb{acupuntura}{}{}{}{}{s.f.}{Prática terapêutica da medicina chinesa, que consiste em introduzir agulhas em partes precisas do corpo, a fim de tratar diferentes doenças ou aliviar dores.}{a.cu.pun.tu.ra}{0}
\verb{acupuntura}{}{}{}{}{}{Picada feita com agulha, nessa prática.}{a.cu.pun.tu.ra}{0}
\verb{acurado}{}{}{}{}{adj.}{Que se acurou; feito ou tratado com cuidado, capricho; aprimorado.}{a.cu.ra.do}{0}
\verb{acurar}{}{}{}{}{v.t.}{Cuidar ou tratar de alguém ou de algo com dedicação.}{a.cu.rar}{0}
\verb{acurar}{}{}{}{}{}{Tornar aprimorado; aperfeiçoar.}{a.cu.rar}{\verboinum{1}}
\verb{acusação}{}{}{"-ões}{}{s.f.}{Ato ou efeito de acusar.}{a.cu.sa.ção}{0}
\verb{acusação}{}{}{"-ões}{}{}{Incriminação.}{a.cu.sa.ção}{0}
\verb{acusação}{}{}{"-ões}{}{}{Delação; denúncia.}{a.cu.sa.ção}{0}
\verb{acusação}{}{}{"-ões}{}{}{Confissão de culpa.}{a.cu.sa.ção}{0}
\verb{acusação}{}{Ant.}{"-ões}{}{}{defesa}{a.cu.sa.ção}{0}
\verb{acusado}{}{}{}{}{adj.}{Que sofreu acusação; incriminado.}{a.cu.sa.do}{0}
\verb{acusado}{}{}{}{}{}{Denunciado; delatado.}{a.cu.sa.do}{0}
\verb{acusado}{}{}{}{}{}{Censurado; repreendido.}{a.cu.sa.do}{0}
\verb{acusado}{}{}{}{}{s.m.}{Indivíduo que é alvo de acusação.}{a.cu.sa.do}{0}
\verb{acusador}{ô}{}{}{}{adj.}{Que acusa, denuncia.}{a.cu.sa.dor}{0}
\verb{acusador}{ô}{}{}{}{s.m.}{Indivíduo que acusa; denunciante.}{a.cu.sa.dor}{0}
\verb{acusador}{ô}{Jur.}{}{}{}{Indivíduo a quem compete a acusação.}{a.cu.sa.dor}{0}
\verb{acusar}{}{}{}{}{v.t.}{Atribuir falta, delito ou crime; incriminar.}{a.cu.sar}{0}
\verb{acusar}{}{}{}{}{}{Denunciar; delatar.}{a.cu.sar}{0}
\verb{acusar}{}{}{}{}{}{Confessar espontaneamente.}{a.cu.sar}{0}
\verb{acusar}{}{Ant.}{}{}{}{defender}{a.cu.sar}{\verboinum{1}}
\verb{acusativo}{}{}{}{}{adj.}{Relativo à acusação; acusatório.}{a.cu.sa.ti.vo}{0}
\verb{acusativo}{}{Gram.}{}{}{s.m.}{Um dos casos sintáticos morfologicamente marcados de algumas línguas, como o latim.}{a.cu.sa.ti.vo}{0}
\verb{acústica}{}{Fís.}{}{}{s.f.}{Estudo das leis e dos fenômenos do som.}{a.cús.ti.ca}{0}
\verb{acústica}{}{}{}{}{}{Propagação do som num local.}{a.cús.ti.ca}{0}
\verb{acústico}{}{}{}{}{adj.}{Relativo ao som ou à acústica.}{a.cús.ti.co}{0}
\verb{acústico}{}{Mús.}{}{}{}{Produção de sons que não envolve meios eletrônicos.}{a.cús.ti.co}{0}
\verb{acutângulo}{}{Geom.}{}{}{adj.}{Que tem todos os ângulos agudos.}{a.cu.tân.gu.lo}{0}
\verb{acutilar}{}{}{}{}{v.t.}{Bater, machucar com cutelo ou arma branca.}{a.cu.ti.lar}{0}
\verb{acutilar}{}{Fig.}{}{}{}{Maltratar; agredir intencionalmente.}{a.cu.ti.lar}{\verboinum{1}}
\verb{adaga}{}{}{}{}{s.f.}{Espada de lâmina larga e curta, usada por bárbaros durante a Idade Média.}{a.da.ga}{0}
\verb{adagiário}{}{}{}{}{adj.}{Relativo a adágio.}{a.da.gi.á.rio}{0}
\verb{adagiário}{}{}{}{}{s.m.}{Coleção ou compêndio de adágios ou provérbios populares.}{a.da.gi.á.rio}{0}
\verb{adágio}{}{}{}{}{s.m.}{Sentença moral de origem popular; ditado, máxima.}{a.dá.gio}{0}
\verb{adágio}{}{Mús.}{}{}{s.m.}{Andamento lento de um trecho musical.}{a.dá.gio}{0}
\verb{adamantino}{}{}{}{}{adj.}{Que tem características do diamante; diamantino.}{a.da.man.ti.no}{0}
\verb{adamantino}{}{Fig.}{}{}{}{Pessoa íntegra; incorruptível.}{a.da.man.ti.no}{0}
\verb{adamascado}{}{}{}{}{adj.}{Que é semelhante ao damasco na cor ou no sabor.}{a.da.mas.ca.do}{0}
\verb{adamascado}{}{}{}{}{}{Tecido que tem a aparência e a cor do damasco.}{a.da.mas.ca.do}{0}
\verb{adamascado}{}{}{}{}{s.m.}{Tecido adamascado.}{a.da.mas.ca.do}{0}
\verb{adâmico}{}{}{}{}{adj.}{Relativo a Adão, o primeiro homem segundo a Bíblia. }{a.dâ.mi.co}{0}
\verb{adâmico}{}{Fig.}{}{}{}{Primitivo; antigo.}{a.dâ.mi.co}{0}
\verb{adaptabilidade}{}{}{}{}{s.f.}{Qualidade do que é adaptável.}{a.dap.ta.bi.li.da.de}{0}
\verb{adaptabilidade}{}{}{}{}{}{Capacidade de alguém ou de algo se adaptar.}{a.dap.ta.bi.li.da.de}{0}
\verb{adaptação}{}{}{"-ões}{}{s.f.}{Ato ou efeito de adaptar; adequação, ajuste.}{a.dap.ta.ção}{0}
\verb{adaptador}{ô}{}{}{}{adj.}{Que adapta, possibilitando o encaixe ou a associação de peças ou sistemas com formatos diferentes.}{a.dap.ta.dor}{0}
\verb{adaptador}{ô}{}{}{}{s.m.}{Dispositivo que torna possível essa associação ou esse encaixe.}{a.dap.ta.dor}{0}
\verb{adaptador}{ô}{}{}{}{}{Indivíduo que promove uma adaptação.}{a.dap.ta.dor}{0}
\verb{adaptar}{}{}{}{}{v.t.}{Fazer os ajustes necessários nos fatos de modo que possam encaixar"-se uns nos outros; configurar, conformar.}{a.dap.tar}{0}
\verb{adaptar}{}{Por ext.}{}{}{v.pron.}{Familiarizar"-se, adequar"-se ou acomodar"-se com determinada situação.}{a.dap.tar}{\verboinum{1}}
\verb{adaptável}{}{}{"-eis}{}{adj.2g.}{Que se pode adaptar, ajustar.}{a.dap.tá.vel}{0}
\verb{adega}{é}{}{}{}{s.f.}{Lugar, geralmente subterrâneo, de temperatura baixa, onde se guardam vinho, azeite e outras bebidas.  }{a.de.ga}{0}
\verb{adega}{é}{}{}{}{}{Conjunto das bebidas guardadas nesse lugar.}{a.de.ga}{0}
\verb{adejar}{}{}{}{}{v.i.}{Dar pequenos e repetidos voos sem direção certa; esvoaçar, voejar, volitar.}{a.de.jar}{0}
\verb{adejar}{}{Por ext.}{}{}{}{Tocar levemente; aflorar; perpassar.}{a.de.jar}{\verboinum{1}}
\verb{adejo}{ê}{}{}{}{s.m.}{Ato ou efeito de adejar; mover as asas; esvoaçar.}{a.de.jo}{0}
\verb{adelgaçar}{}{}{}{}{v.t.}{Tornar delgado, fino.}{a.del.ga.çar}{0}
\verb{adelgaçar}{}{}{}{}{}{Tornar menos denso; rarefazer.}{a.del.ga.çar}{0}
\verb{adelgaçar}{}{Por ext.}{}{}{}{Emagrecer.}{a.del.ga.çar}{\verboinum{3}}
\verb{ademais}{}{}{}{}{adv.}{Além disso, além do mais.}{a.de.mais}{0}
\verb{ademanes}{}{}{}{}{s.m.pl.}{Gestos ensaiados e estilosos, feitos geralmente com as mãos, para expressar sentimentos ou ideias. Usa"-se também no singular \textit{ademane}.}{a.de.ma.nes}{0}
\verb{adenda}{}{}{}{}{s.f.}{Adendo.}{a.den.da}{0}
\verb{adendo}{}{}{}{}{s.m.}{Aquilo que se acrescenta a um texto, para corrigi"-lo, ratificá"-lo, complementá"-lo.}{a.den.do}{0}
\verb{adendo}{}{}{}{}{}{Apêndice; suplemento.}{a.den.do}{0}
\verb{adenite}{}{Med.}{}{}{s.f.}{Inflamação de uma glândula ou dos gânglios linfáticos.}{a.de.ni.te}{0}
\verb{adenoide}{}{}{}{}{adj.2g.}{Que é semelhante a uma glândula.}{a.de.noi.de}{0}
\verb{adenoide}{}{Med.}{}{}{s.f.}{Hipertrofia do tecido esponjoso entre as fossas nasais e a garganta.}{a.de.noi.de}{0}
\verb{adenoma}{}{Med.}{}{}{s.m.}{Tumor benigno do tecido glandular.  }{a.de.no.ma}{0}
\verb{adenovírus}{}{Biol.}{}{}{s.m.}{Vírus tropical, que causa conjuntivite e problemas nas vias respiratórias.}{a.de.no.ví.rus}{0}
\verb{adensar}{}{}{}{}{v.t.}{Tornar mais denso, mais espesso.}{a.den.sar}{0}
\verb{adensar}{}{}{}{}{}{Juntar, condensar.}{a.den.sar}{\verboinum{1}}
\verb{adentrar}{}{}{}{}{v.i.}{Entrar, penetrar.}{a.den.trar}{0}
\verb{adentrar}{}{}{}{}{}{Empurrar para dentro; fazer entrar (à força).}{a.den.trar}{\verboinum{1}}
\verb{adentro}{}{}{}{}{adv.}{No interior, dentro, interiormente.}{a.den.tro}{0}
\verb{adentro}{}{}{}{}{}{Para a parte interior; para dentro.}{a.den.tro}{0}
\verb{adepto}{é}{}{}{}{s.m.}{Indivíduo que se inicia em dogmas ou princípios de uma ciência, religião, e se torna partidário dela.}{a.dep.to}{0}
\verb{adepto}{é}{}{}{}{adj.}{Que é partidário, simpatizante, admirador. }{a.dep.to}{0}
\verb{adequação}{}{}{"-ões}{}{s.f.}{Ato ou efeito de adequar.}{a.de.qua.ção}{0}
\verb{adequação}{}{}{"-ões}{}{}{Adaptação, ajustamento, acomodação.}{a.de.qua.ção}{0}
\verb{adequado}{}{}{}{}{adj.}{Que é apropriado; que está em conformidade com. }{a.de.qua.do}{0}
\verb{adequado}{}{}{}{}{}{Conveniente.}{a.de.qua.do}{0}
\verb{adequar}{}{}{}{}{v.t.}{Tornar apropriado, oportuno.}{a.de.quar}{0}
\verb{adequar}{}{}{}{}{}{Adaptar, ambientar.}{a.de.quar}{\verboinum{11}\verboirregular{adequo, adequas}}
\verb{adereçar}{}{}{}{}{v.t.}{Enfeitar com adereços, adornos.}{a.de.re.çar}{0}
\verb{adereçar}{}{}{}{}{}{Enviar; endereçar.}{a.de.re.çar}{\verboinum{3}}
\verb{adereço}{ê}{}{}{}{s.m.}{Objeto usado para enfeitar, ornar. }{a.de.re.ço}{0}
\verb{aderência}{}{}{}{}{s.f.}{Efeito de aderir, colar, fixar.}{a.de.rên.cia}{0}
\verb{aderência}{}{}{}{}{}{Qualidade do que adere. }{a.de.rên.cia}{0}
\verb{aderente}{}{}{}{}{adj.2g.}{Que adere, cola, gruda.}{a.de.ren.te}{0}
\verb{aderente}{}{}{}{}{s.2g.}{Indivíduo que se torna seguidor de uma ideia, causa, religião; partidário. }{a.de.ren.te}{0}
\verb{aderir}{}{}{}{}{v.i.}{Colar, grudar, ser aderente.}{a.de.rir}{0}
\verb{aderir}{}{}{}{}{v.t.}{Tornar"-se adepto, simpatizante, partidário.}{a.de.rir}{\verboinum{29}}
\verb{adernar}{}{}{}{}{v.i.}{Inclinar a embarcação sobre um dos lados.}{a.der.nar}{0}
\verb{adernar}{}{}{}{}{}{Virar, inclinar.}{a.der.nar}{\verboinum{1}}
\verb{adesão}{}{}{"-ões}{}{s.f.}{Ato ou efeito de aderir, colar; junção.}{a.de.são}{0}
\verb{adesão}{}{}{"-ões}{}{}{Filiação a uma associação ou partido.}{a.de.são}{0}
\verb{adesismo}{}{}{}{}{s.m.}{Tendência ou hábito oportunista de filiar"-se a partidos ou associações.}{a.de.sis.mo}{0}
\verb{adesista}{}{}{}{}{adj.2g.}{Que pratica o adesismo.}{a.de.sis.ta}{0}
\verb{adesivo}{}{}{}{}{adj.}{Que pode se fixar, colar em algo.}{a.de.si.vo}{0}
\verb{adesivo}{}{Por ext.}{}{}{s.m.}{Ilustração ou mensagem em papel ou outro material flexível, com um dos lados recoberto por um produto que adere a determinadas superfícies.}{a.de.si.vo}{0}
\verb{adestramento}{}{}{}{}{s.m.}{Ato ou efeito de adestrar; treinamento.}{a.des.tra.men.to}{0}
\verb{adestrar}{}{}{}{}{v.t.}{Tornar destro; habilitar; ensinar; treinar.}{a.des.trar}{\verboinum{1}}
\verb{adeus}{}{}{}{}{interj.}{Expressão que denota despedida ou saudade.}{a.deus}{0}
\verb{adeus}{}{}{}{}{s.m.}{Palavra, gesto, sinal de despedida ou de saudação.}{a.deus}{0}
\verb{adiamento}{}{}{}{}{s.m.}{Ato ou efeito de adiar, retardar; delonga.}{a.di.a.men.to}{0}
\verb{adiantado}{}{}{}{}{adj.}{Que se adiantou, avançou.}{a.di.an.ta.do}{0}
\verb{adiantado}{}{}{}{}{}{Desenvolvido; avançado.}{a.di.an.ta.do}{0}
\verb{adiantado}{}{}{}{}{}{Precipitado.}{a.di.an.ta.do}{0}
\verb{adiantado}{}{}{}{}{adv.}{Antecipadamente; adiantadamente.}{a.di.an.ta.do}{0}
\verb{adiantamento}{}{}{}{}{s.m.}{Ato ou efeito de adiantar.}{a.di.an.ta.men.to}{0}
\verb{adiantamento}{}{}{}{}{}{Estado do que se encontra próximo do fim ou da conclusão.}{a.di.an.ta.men.to}{0}
\verb{adiantamento}{}{}{}{}{}{Prioridade; primazia.}{a.di.an.ta.men.to}{0}
\verb{adiantamento}{}{}{}{}{}{Desenvolvimento; progresso; avanço.}{a.di.an.ta.men.to}{0}
\verb{adiantamento}{}{}{}{}{}{Quantia paga antecipadamente.}{a.di.an.ta.men.to}{0}
\verb{adiantar}{}{}{}{}{v.t.}{Ir para a frente.}{a.di.an.tar}{0}
\verb{adiantar}{}{}{}{}{}{Desenvolver; progredir.}{a.di.an.tar}{0}
\verb{adiantar}{}{}{}{}{}{Acelerar; apressar.}{a.di.an.tar}{0}
\verb{adiantar}{}{}{}{}{}{Antecipar; precipitar.}{a.di.an.tar}{0}
\verb{adiantar}{}{}{}{}{v.i.}{Valer a pena, compensar.}{a.di.an.tar}{0}
\verb{adiantar}{}{Ant.}{}{}{}{atrasar; retardar}{a.di.an.tar}{\verboinum{1}}
\verb{adiante}{}{}{}{}{adv.}{Na frente; em frente; diante.}{a.di.an.te}{0}
\verb{adiante}{}{}{}{}{}{Para a frente; avante.}{a.di.an.te}{0}
\verb{adiante}{}{}{}{}{}{Em primeiro lugar; primeiramente.}{a.di.an.te}{0}
\verb{adiante}{}{}{}{}{}{Mais à frente; além; após.}{a.di.an.te}{0}
\verb{adiante}{}{}{}{}{interj.}{Expressão que denota uma intimação para continuar algo interrompido.}{a.di.an.te}{0}
\verb{adiante}{}{Ant.}{}{}{}{atrás}{a.di.an.te}{0}
\verb{adiar}{}{}{}{}{v.t.}{Deixar para outro dia; postergar; protelar; prorrogar; procrastinar.}{a.di.ar}{0}
\verb{adiar}{}{}{}{}{}{Reprovar em exame.}{a.di.ar}{\verboinum{1}}
\verb{adiável}{}{}{"-eis}{}{adj.2g.}{Que se pode ou se deve adiar; prorrogável.}{a.di.á.vel}{0}
\verb{adição}{}{}{"-ões}{}{s.f.}{Ato ou efeito de adicionar.}{a.di.ção}{0}
\verb{adição}{}{}{"-ões}{}{}{Acréscimo; soma.}{a.di.ção}{0}
\verb{adição}{}{}{"-ões}{}{}{A primeira das quatro operações fundamentais da aritmética.}{a.di.ção}{0}
\verb{adicional}{}{}{"-ais}{}{adj.2g.}{Que se adiciona, acrescenta.}{a.di.ci.o.nal}{0}
\verb{adicional}{}{}{"-ais}{}{s.m.}{Aquilo que se acrescenta.}{a.di.ci.o.nal}{0}
\verb{adicional}{}{}{"-ais}{}{}{Quantia suplementar que se soma a um imposto, taxa ou salário.}{a.di.ci.o.nal}{0}
\verb{adicionar}{}{}{}{}{v.t.}{Acrescentar alguma coisa a outra; juntar; aditar.}{a.di.ci.o.nar}{0}
\verb{adicionar}{}{}{}{}{}{Realizar adição; somar.}{a.di.ci.o.nar}{\verboinum{1}}
\verb{adido}{}{}{}{}{adj.}{Que foi adicionado, acrescentado.}{a.di.do}{0}
\verb{adido}{}{}{}{}{s.m.}{Funcionário agregado a outro, à corporação ou ao quadro, para auxiliar.}{a.di.do}{0}
\verb{adimplente}{}{Jur.}{}{}{adj.2g.}{Que cumpre as obrigações contratuais.}{a.dim.plen.te}{0}
\verb{adimplente}{}{Jur.}{}{}{s.2g.}{Indivíduo que realiza, em tempo hábil, as obrigações de contrato.}{a.dim.plen.te}{0}
\verb{adimplente}{}{Ant.}{}{}{}{inadimplente}{a.dim.plen.te}{0}
\verb{adiposidade}{}{Med.}{}{}{s.f.}{Excesso de gordura no organismo; obesidade.}{a.di.po.si.da.de}{0}
\verb{adiposo}{ô}{}{"-osos ⟨ó⟩}{"-osa ⟨ó⟩}{adj.}{Que tem gordura, muito gordo; obeso.}{a.di.po.so}{0}
\verb{adiposo}{ô}{}{"-osos ⟨ó⟩}{"-osa ⟨ó⟩}{}{Gorduroso.}{a.di.po.so}{0}
\verb{adir}{}{}{}{}{v.t.}{Acrescentar ou juntar uma coisa a outra.}{a.dir}{0}
\verb{adir}{}{}{}{}{}{Agregar; incorporar.}{a.dir}{\verboinum{18}\verboirregular{\emph{def.} adimos, adis}}
\verb{aditamento}{}{}{}{}{s.m.}{Ato ou efeito de aditar; suplemento; adicionamento.}{a.di.ta.men.to}{0}
\verb{aditar}{}{}{}{}{v.t.}{Fazer acréscimo; adicionar; juntar.}{a.di.tar}{\verboinum{1}}
\verb{aditiva}{}{Gram.}{}{}{s.f.}{Conjunção coordenativa que une duas orações com a mesma função.}{a.di.ti.va}{0}
\verb{aditivo}{}{}{}{}{adj.}{Que se adita; adicional.}{a.di.ti.vo}{0}
\verb{aditivo}{}{}{}{}{s.m.}{Aquilo que se adicionou.}{a.di.ti.vo}{0}
\verb{aditivo}{}{Quím.}{}{}{}{Substância adicionada a uma solução para melhorar"-lhe as propriedades.}{a.di.ti.vo}{0}
\verb{ádito}{}{}{}{}{s.m.}{Entrada; abertura; acesso.}{á.di.to}{0}
\verb{ádito}{}{}{}{}{}{Câmara secreta dos antigos templos gregos onde apenas os sacerdotes tinham acesso.}{á.di.to}{0}
\verb{adivinha}{}{}{}{}{s.f.}{Questão enigmática; adivinhação.}{a.di.vi.nha}{0}
\verb{adivinha}{}{}{}{}{}{Mulher que pratica a adivinhação, que diz adivinhar.}{a.di.vi.nha}{0}
\verb{adivinhação}{}{}{"-ões}{}{s.f.}{Ato ou efeito de adivinhar; descobrir o que está oculto.}{a.di.vi.nha.ção}{0}
\verb{adivinhação}{}{}{"-ões}{}{}{Brincadeira que apresenta enigmas para serem decifrados.}{a.di.vi.nha.ção}{0}
\verb{adivinhação}{}{}{"-ões}{}{}{Arte de predizer o futuro e conhecer o que se encontra oculto no presente ou no passado.}{a.di.vi.nha.ção}{0}
\verb{adivinhador}{ô}{}{}{}{adj.}{Que adivinha.}{a.di.vi.nha.dor}{0}
\verb{adivinhador}{ô}{}{}{}{s.m.}{Indivíduo que adivinha, que descobre o que está oculto.}{a.di.vi.nha.dor}{0}
\verb{adivinhador}{ô}{}{}{}{}{Indivíduo que pratica a arte da adivinhação.}{a.di.vi.nha.dor}{0}
\verb{adivinhão}{}{Pop.}{"-ões}{}{s.m.}{Indivíduo que supostamente prediz ou prevê o futuro; adivinho.}{a.di.vi.nhão}{0}
\verb{adivinhão}{}{}{"-ões}{}{}{Bruxo; feiticeiro.}{a.di.vi.nhão}{0}
\verb{adivinhar}{}{}{}{}{v.t.}{Predizer, prever o futuro ou descobrir, por meios sobrenaturais, coisas presentes ou passadas. }{a.di.vi.nhar}{0}
\verb{adivinhar}{}{}{}{}{}{Descobrir; decifrar; deduzir; interpretar.}{a.di.vi.nhar}{\verboinum{1}}
\verb{adivinho}{}{}{}{}{s.m.}{Indivíduo que pratica a arte da adivinhação, que se propõe a predizer o futuro e a desvendar coisas ocultas.}{a.di.vi.nho}{0}
\verb{adjacência}{}{}{}{}{s.f.}{Situação adjacente, condição do que é próximo; contiguidade; vizinhança.}{ad.ja.cên.cia}{0}
\verb{adjacente}{}{}{}{}{adj.2g.}{Que está posto ao lado de; próximo; contíguo; vizinho.}{ad.ja.cen.te}{0}
\verb{adjetivação}{}{Gram.}{"-ões}{}{s.f.}{Ação ou efeito de atribuir adjetivos, qualidades.}{ad.je.ti.va.ção}{0}
\verb{adjetivação}{}{}{"-ões}{}{}{Emprego de adjetivos; qualificação.}{ad.je.ti.va.ção}{0}
\verb{adjetivado}{}{Gram.}{}{}{adj.}{Que está junto de adjetivo.}{ad.je.ti.va.do}{0}
\verb{adjetivado}{}{}{}{}{}{Tornado adjetivo.}{ad.je.ti.va.do}{0}
\verb{adjetivar}{}{Gram.}{}{}{v.t.}{Atribuir adjetivo ou qualidade.}{ad.je.ti.var}{0}
\verb{adjetivar}{}{}{}{}{}{Tornar adjetivo; dar forma e função de adjetivo.}{ad.je.ti.var}{\verboinum{1}}
\verb{adjetivo}{}{}{}{}{adj.}{Que se junta.}{ad.je.ti.vo}{0}
\verb{adjetivo}{}{}{}{}{s.m.}{Atributo, qualificador.}{ad.je.ti.vo}{0}
\verb{adjetivo}{}{Gram.}{}{}{}{Classe de palavra usada junto do substantivo para atribuir"-lhe qualidades.}{ad.je.ti.vo}{0}
\verb{adjudicação}{}{Jur.}{"-ões}{}{s.f.}{Ato ou efeito de adjudicar, dar a alguém a posse e a prioridade de determinados bens. }{ad.ju.di.ca.ção}{0}
\verb{adjudicar}{}{Jur.}{}{}{v.t.}{Efetuar a adjudicação, conceder a alguém determinados bens, por sentença judicial.}{ad.ju.di.car}{0}
\verb{adjudicar}{}{}{}{}{v.pron.}{Chamar a si; atribuir; arrogar.}{ad.ju.di.car}{\verboinum{2}}
\verb{adjudicatório}{}{Jur.}{}{}{adj.}{Diz"-se de ato ou sentença em que há adjudicação.}{ad.ju.di.ca.tó.rio}{0}
\verb{adjunção}{}{}{"-ões}{}{s.f.}{Ato ou efeito de juntar; reunião.}{ad.jun.ção}{0}
\verb{adjunção}{}{}{"-ões}{}{}{Associação de uma pessoa ou coisa a outra para coadjuvar, ajudar.}{ad.jun.ção}{0}
\verb{adjunto}{}{}{}{}{adj.}{Que está próximo; contíguo; unido.}{ad.jun.to}{0}
\verb{adjunto}{}{}{}{}{s.m.}{Indivíduo que é ajudante; auxiliar; assistente.}{ad.jun.to}{0}
\verb{adjunto}{}{}{}{}{}{Substituto; suplente.}{ad.jun.to}{0}
\verb{adjunto}{}{Gram.}{}{}{}{Termo acessório de uma oração que modifica um nome ou um verbo.}{ad.jun.to}{0}
\verb{adjutório}{}{}{}{}{s.m.}{Prestação de ajuda; auxílio; socorro; mutirão; ajutório.}{ad.ju.tó.rio}{0}
\verb{adminículo}{}{}{}{}{s.m.}{Pequena ajuda; contribuição; subsídio.}{ad.mi.ní.cu.lo}{0}
\verb{administração}{}{}{"-ões}{}{s.f.}{Ato ou efeito de administrar.}{ad.mi.nis.tra.ção}{0}
\verb{administração}{}{}{"-ões}{}{}{Governo, gestão de negócios públicos ou particulares.}{ad.mi.nis.tra.ção}{0}
\verb{administração}{}{}{"-ões}{}{}{Local onde se administra.}{ad.mi.nis.tra.ção}{0}
\verb{administração}{}{}{"-ões}{}{}{Gerência, conjunto de pessoal que administra.}{ad.mi.nis.tra.ção}{0}
\verb{administrador}{ô}{}{}{}{adj.}{Que administra.}{ad.mi.nis.tra.dor}{0}
\verb{administrador}{ô}{}{}{}{s.m.}{Pessoa incumbida de uma administração.}{ad.mi.nis.tra.dor}{0}
\verb{administrar}{}{}{}{}{v.t.}{Dirigir, governar, gerenciar um negócio, um estabelecimento público ou privado.}{ad.mi.nis.trar}{0}
\verb{administrar}{}{}{}{}{}{Dar; conferir; aplicar.}{ad.mi.nis.trar}{\verboinum{1}}
\verb{administrativo}{}{}{}{}{adj.}{Relativo à prática da administração.}{ad.mi.nis.tra.ti.vo}{0}
\verb{admiração}{}{}{"-ões}{}{s.f.}{Ato ou efeito de admirar.}{ad.mi.ra.ção}{0}
\verb{admiração}{}{}{"-ões}{}{}{Disposição emocional que traduz respeito, estima, consideração.}{ad.mi.ra.ção}{0}
\verb{admiração}{}{}{"-ões}{}{}{Sentimento que exprime espanto, surpresa, estranheza.}{ad.mi.ra.ção}{0}
\verb{admirado}{}{}{}{}{adj.}{Que exprime um sentimento de admiração, veneração.}{ad.mi.ra.do}{0}
\verb{admirado}{}{}{}{}{}{Espantado, surpreso.}{ad.mi.ra.do}{0}
\verb{admirador}{ô}{}{}{}{adj.}{Que admira, dá valor, aprecia.}{ad.mi.ra.dor}{0}
\verb{admirador}{ô}{}{}{}{s.m.}{Indivíduo que admira, venera; fã.}{ad.mi.ra.dor}{0}
\verb{admirar}{}{}{}{}{v.t.}{Olhar ou considerar com admiração, espanto, surpresa.}{ad.mi.rar}{0}
\verb{admirar}{}{}{}{}{}{Sentir admiração por; venerar; apreciar.}{ad.mi.rar}{0}
\verb{admirar}{}{}{}{}{}{Causar admiração; ser admirável.}{ad.mi.rar}{\verboinum{1}}
\verb{admirativo}{}{}{}{}{adj.}{Que provoca ou exprime admiração.}{ad.mi.ra.ti.vo}{0}
\verb{admirável}{}{}{"-eis}{}{adj.2g.}{Que causa admiração; digno de ser admirado.}{ad.mi.rá.vel}{0}
\verb{admissão}{}{}{"-ões}{}{s.f.}{Ato ou efeito de admitir, de aceitar como verdadeiro.}{ad.mis.são}{0}
\verb{admissão}{}{}{"-ões}{}{}{Aceitação; aprovação; consentimento.}{ad.mis.são}{0}
\verb{admissão}{}{}{"-ões}{}{}{Introdução; entrada; ingresso.}{ad.mis.são}{0}
\verb{admissível}{}{}{"-eis}{}{adj.2g.}{Que se pode admitir; aceitável; possível.}{ad.mis.sí.vel}{0}
\verb{admitância}{}{Fís.}{}{}{s.f.}{Propriedade na qual uma corrente elétrica percorre um circuito elétrico sob ação de certa diferença de potencial.}{ad.mi.tân.cia}{0}
\verb{admitir}{}{}{}{}{v.t.}{Aceitar como verdadeiro.}{ad.mi.tir}{0}
\verb{admitir}{}{}{}{}{}{Tolerar um fato, consentir.}{ad.mi.tir}{0}
\verb{admitir}{}{}{}{}{}{Aceitar a participação em atividades pessoais ou profissionais.}{ad.mi.tir}{\verboinum{18}}
\verb{admoestação}{}{}{"-ões}{}{s.f.}{Ato ou efeito de admoestar; repreensão leve; reprimenda; advertência.}{ad.mo.es.ta.ção}{0}
\verb{admoestação}{}{}{"-ões}{}{}{Aviso, conselho.}{ad.mo.es.ta.ção}{0}
\verb{admoestar}{}{}{}{}{v.t.}{Chamar a atenção, advertir, repreender.}{ad.mo.es.tar}{0}
\verb{admoestar}{}{Por ext.}{}{}{}{Lembrar, avisar.}{ad.mo.es.tar}{\verboinum{1}}
\verb{ADN}{}{}{}{}{s.m.}{Sigla de \textit{ácido desoxirribonucleico}. Usa"-se mais comumente \textsc{dna} (\textit{desoxyribonucleic acid}). }{adn}{0}
\verb{adnominal}{}{Gram.}{"-ais}{}{adj.2g.}{Diz"-se da palavra que vem junto ao substantivo, para complementar"-lhe o sentido.}{ad.no.mi.nal}{0}
\verb{adobe}{ô}{}{}{}{s.m.}{Tijolo feito de argila crua, seco ao sol, às vezes acrescido de palha para torná"-lo mais resistente.}{a.do.be}{0}
\verb{adoçamento}{}{}{}{}{s.m.}{Ato ou efeito de adoçar.}{a.do.ça.men.to}{0}
\verb{adoçamento}{}{Fig.}{}{}{}{Abrandamento, moderação.}{a.do.ça.men.to}{0}
\verb{adoçante}{}{}{}{}{adj.2g.}{Que adoça.}{a.do.çan.te}{0}
\verb{adoçante}{}{}{}{}{s.m.}{Qualquer substância, natural ou artificial, que adoça.}{a.do.çan.te}{0}
\verb{adoção}{}{}{"-ões}{}{s.f.}{Ato ou efeito de adotar; aceitação; perfilhamento.}{a.do.ção}{0}
\verb{adoção}{}{Jur.}{"-ões}{}{}{Processo legal pelo qual se aceita voluntariamente uma criança como filho.}{a.do.ção}{0}
\verb{adoçar}{}{}{}{}{v.t.}{Tornar doce.}{a.do.çar}{0}
\verb{adoçar}{}{Fig.}{}{}{}{Abrandar, suavizar, aliviar.}{a.do.çar}{\verboinum{3}}
\verb{adocicado}{}{}{}{}{adj.}{Que se adocicou; levemente doce.}{a.do.ci.ca.do}{0}
\verb{adocicado}{}{Fig.}{}{}{}{Terno, brando.}{a.do.ci.ca.do}{0}
\verb{adocicar}{}{}{}{}{v.t.}{Tornar levemente doce.}{a.do.ci.car}{0}
\verb{adocicar}{}{Fig.}{}{}{}{Abrandar, atenuar.}{a.do.ci.car}{\verboinum{2}}
\verb{adoecer}{ê}{}{}{}{v.i.}{Ficar doente; enfermar.}{a.do.e.cer}{\verboinum{15}}
\verb{adoentado}{}{}{}{}{adj.}{Que se adoentou; um tanto doente.}{a.do.en.ta.do}{0}
\verb{adoentado}{}{}{}{}{}{Fraco, debilitado, abatido.}{a.do.en.ta.do}{0}
\verb{adoentar}{}{}{}{}{v.t.}{Tornar doente ou um pouco doente.}{a.do.en.tar}{\verboinum{1}}
\verb{adoidado}{}{}{}{}{adj.}{Que age como doido; desatinado.}{a.doi.da.do}{0}
\verb{adoidado}{}{}{}{}{}{Estouvado, desvairado.}{a.doi.da.do}{0}
\verb{adoidado}{}{}{}{}{adv.}{Demasiadamente; à beça.}{a.doi.da.do}{0}
\verb{adolescência}{}{}{}{}{s.f.}{Fase da vida humana entre a puberdade e a idade adulta, caracterizada por mudanças corporais e psicológicas.}{a.do.les.cên.cia}{0}
\verb{adolescente}{}{}{}{}{adj.2g.}{Relativo à adolescência.}{a.do.les.cen.te}{0}
\verb{adolescente}{}{}{}{}{s.2g.}{Indivíduo que está na fase da adolescência.}{a.do.les.cen.te}{0}
\verb{adônis}{}{Mit.}{}{}{s.m.}{Deus grego possuidor de grande beleza.}{a.dô.nis}{0}
\verb{adônis}{}{Por ext.}{}{}{}{Jovem elegante, bonito.}{a.dô.nis}{0}
\verb{adoração}{}{}{"-ões}{}{s.f.}{Ato ou efeito de adorar.}{a.do.ra.ção}{0}
\verb{adoração}{}{}{"-ões}{}{}{Culto a uma divindade.}{a.do.ra.ção}{0}
\verb{adoração}{}{}{"-ões}{}{}{Veneração, idolatria, amor excessivo.}{a.do.ra.ção}{0}
\verb{adorador}{ô}{}{}{}{adj.}{Que adora.}{a.do.ra.dor}{0}
\verb{adorador}{ô}{}{}{}{}{Admirador, apreciador.}{a.do.ra.dor}{0}
\verb{adorar}{}{}{}{}{v.t.}{Prestar culto a.}{a.do.rar}{0}
\verb{adorar}{}{}{}{}{}{Venerar, idolatrar, amar extremosamente.}{a.do.rar}{0}
\verb{adorar}{}{}{}{}{}{Gostar muito de.}{a.do.rar}{\verboinum{1}}
\verb{adorável}{}{}{"-eis}{}{adj.2g.}{Que se pode adorar.}{a.do.rá.vel}{0}
\verb{adorável}{}{}{"-eis}{}{}{Digno de adoração, de culto.}{a.do.rá.vel}{0}
\verb{adorável}{}{}{"-eis}{}{}{Encantador, fascinante.}{a.do.rá.vel}{0}
\verb{adormecer}{ê}{}{}{}{v.t.}{Fazer dormir, causar sono.}{a.dor.me.cer}{0}
\verb{adormecer}{ê}{}{}{}{}{Entorpecer, anestesiar.}{a.dor.me.cer}{0}
\verb{adormecer}{ê}{}{}{}{v.i.}{Cair no sono; dormir.}{a.dor.me.cer}{0}
\verb{adormecer}{ê}{}{}{}{}{Serenar, aquietar.}{a.dor.me.cer}{\verboinum{15}}
\verb{adormecimento}{}{}{}{}{s.m.}{Ato ou efeito de adormecer.}{a.dor.me.ci.men.to}{0}
\verb{adormecimento}{}{}{}{}{}{Ausência de sensibilidade; entorpecimento.}{a.dor.me.ci.men.to}{0}
\verb{adornar}{}{}{}{}{v.t.}{Colocar adorno; enfeitar, ornamentar.}{a.dor.nar}{\verboinum{1}}
\verb{adorno}{ô}{}{}{}{s.m.}{Aquilo que adorna; enfeite, ornamento.}{a.dor.no}{0}
\verb{adotar}{}{}{}{}{v.t.}{Escolher, optar.}{a.do.tar}{0}
\verb{adotar}{}{}{}{}{}{Admitir, seguir.}{a.do.tar}{0}
\verb{adotar}{}{Jur.}{}{}{}{Aceitar legalmente como filho; perfilhar.}{a.do.tar}{\verboinum{1}}
\verb{adotivo}{}{}{}{}{adj.}{Que se adotou.}{a.do.ti.vo}{0}
\verb{adotivo}{}{}{}{}{}{Relativo a adoção.}{a.do.ti.vo}{0}
\verb{adquirente}{}{}{}{}{adj.2g.}{Que adquire.}{ad.qui.ren.te}{0}
\verb{adquirente}{}{Jur.}{}{}{s.2g.}{Indivíduo que se torna proprietário de um bem.}{ad.qui.ren.te}{0}
\verb{adquirir}{}{}{}{}{v.t.}{Obter, conseguir, alcançar.}{ad.qui.rir}{0}
\verb{adquirir}{}{}{}{}{}{Comprar.}{ad.qui.rir}{\verboinum{18}}
\verb{adrede}{ê}{}{}{}{adv.}{De propósito; intencionalmente.}{a.dre.de}{0}
\verb{adrenalina}{}{Bioquím.}{}{}{s.f.}{Hormônio produzido pelas glândulas suprarrenais, responsável pela elevação da pressão sanguínea; epinefrina.}{a.dre.na.li.na}{0}
\verb{adrenalina}{}{Pop.}{}{}{}{Energia, excitação, vigor.}{a.dre.na.li.na}{0}
\verb{adriático}{}{}{}{}{adj.}{Relativo ao Mar Adriático, na Europa.}{a.dri.á.ti.co}{0}
\verb{adriça}{}{}{}{}{s.f.}{Cabo ou corda utilizado para içar velas ou bandeiras.}{a.dri.ça}{0}
\verb{adro}{}{}{}{}{s.m.}{Pátio externo localizado em frente ou em torno de uma igreja.}{a.dro}{0}
\verb{adro}{}{}{}{}{}{Antigo cemitério situado no terreno de algumas igrejas.}{a.dro}{0}
\verb{ad"-rogar}{}{}{}{}{v.t.}{Aceitar por adoção; adotar.}{ad"-ro.gar}{\verboinum{5}}
\verb{adscrito}{}{}{}{}{adj.}{Acrescentado, aumentado.}{ads.cri.to}{0}
\verb{adscrito}{}{}{}{}{}{Inscrito, registrado.}{ads.cri.to}{0}
\verb{adscrito}{}{}{}{}{}{Submetido, sujeito a.}{ads.cri.to}{0}
\verb{adsorção}{}{Fís.}{"-ões}{}{s.f.}{Retenção ou fixação de moléculas na superfície de substâncias sólidas. }{ad.sor.ção}{0}
\verb{adstringência}{}{}{}{}{s.f.}{Qualidade do que é adstringente; contração, constrição.}{ads.trin.gên.cia}{0}
\verb{adstringente}{}{}{}{}{adj.2g.}{Que adstringe.}{ads.trin.gen.te}{0}
\verb{adstringente}{}{}{}{}{s.m.}{Substância que produz constrição dos tecidos orgânicos.}{ads.trin.gen.te}{0}
\verb{adstringir}{}{}{}{}{v.t.}{Comprimir, estreitar.}{ads.trin.gir}{0}
\verb{adstringir}{}{}{}{}{}{Diminuir, reduzir.}{ads.trin.gir}{0}
\verb{adstringir}{}{}{}{}{}{Coagir, obrigar.}{ads.trin.gir}{0}
\verb{adstringir}{}{Med.}{}{}{}{Produzir contração nos tecidos orgânicos.}{ads.trin.gir}{\verboinum{22}}
\verb{adstrito}{}{}{}{}{adj.}{Que se adstringiu; constrito.}{ads.tri.to}{0}
\verb{adstrito}{}{}{}{}{}{Apertado, exíguo.}{ads.tri.to}{0}
\verb{adstrito}{}{}{}{}{}{Limitado, restrito.}{ads.tri.to}{0}
\verb{aduana}{}{}{}{}{s.f.}{Alfândega.}{a.du.a.na}{0}
\verb{aduana}{}{}{}{}{}{Antigo bairro habitado por cristãos, situado nas terras mouras.}{a.du.a.na}{0}
\verb{aduaneiro}{ê}{}{}{}{adj.}{Relativo à aduana; alfandegário.}{a.du.a.nei.ro}{0}
\verb{adubação}{}{}{"-ões}{}{s.f.}{Ato ou efeito de adubar a terra para o cultivo da lavoura.}{a.du.ba.ção}{0}
\verb{adubar}{}{}{}{}{v.t.}{Fertilizar a terra com adubo; estrumar.}{a.du.bar}{\verboinum{1}}
\verb{adubo}{}{}{}{}{s.m.}{Produto orgânico ou mineral utilizado para fertilizar a terra; fertilizante; estrume.}{a.du.bo}{0}
\verb{adução}{}{}{"-ões}{}{s.f.}{Ato ou efeito de aduzir, conduzir.}{a.du.ção}{0}
\verb{adução}{}{}{"-ões}{}{}{Condução de água, em sistemas de abastecimento, da captação à distribuição.}{a.du.ção}{0}
\verb{aduela}{é}{}{}{}{s.f.}{Tábua encurvada utilizada na fabricação de barris, tonéis etc.}{a.du.e.la}{0}
\verb{aduela}{é}{}{}{}{}{Pedra ou bloco empregado na composição de arcos e abóbadas.}{a.du.e.la}{0}
\verb{aduela}{é}{}{}{}{}{Tábua que reveste o vão de portas e janelas.}{a.du.e.la}{0}
\verb{adufe}{}{Mús.}{}{}{s.m.}{Pandeiro quadrado, feito de madeira leve e com pele retesada de ambos os lados.}{a.du.fe}{0}
\verb{adulação}{}{}{"-ões}{}{s.f.}{Ato ou efeito de adular; bajulação.}{a.du.la.ção}{0}
\verb{adulador}{ô}{}{}{}{adj.}{Que adula; bajulador.}{a.du.la.dor}{0}
\verb{adular}{}{}{}{}{v.t.}{Bajular ou lisonjear de modo excessivo e servil.}{a.du.lar}{\verboinum{1}}
\verb{adulteração}{}{}{"-ões}{}{s.f.}{Ato ou efeito de adulterar.}{a.dul.te.ra.ção}{0}
\verb{adulteração}{}{}{"-ões}{}{}{Falsificação, corrupção.}{a.dul.te.ra.ção}{0}
\verb{adulterado}{}{}{}{}{adj.}{Que sofreu adulteração.}{a.dul.te.ra.do}{0}
\verb{adulterado}{}{}{}{}{}{Falsificado, corrompido.}{a.dul.te.ra.do}{0}
\verb{adulterador}{ô}{}{}{}{adj.}{Que adultera; falsificador.}{a.dul.te.ra.dor}{0}
\verb{adulterar}{}{}{}{}{v.t.}{Falsificar, alterar.}{a.dul.te.rar}{0}
\verb{adulterar}{}{}{}{}{}{Corromper, viciar.}{a.dul.te.rar}{0}
\verb{adulterar}{}{}{}{}{v.i.}{Cometer adultério.}{a.dul.te.rar}{\verboinum{1}}
\verb{adulterino}{}{}{}{}{adj.}{Nascido de adultério.}{a.dul.te.ri.no}{0}
\verb{adultério}{}{}{}{}{s.m.}{Infidelidade conjugal; prevaricação.}{a.dul.té.rio}{0}
\verb{adúltero}{}{}{}{}{adj.}{Que pratica adultério.}{a.dúl.te.ro}{0}
\verb{adulto}{}{}{}{}{adj.}{Que atingiu o desenvolvimento pleno.}{a.dul.to}{0}
\verb{adulto}{}{}{}{}{}{Que chegou à maioridade legal, após a adolescência.}{a.dul.to}{0}
\verb{adulto}{}{}{}{}{}{Maduro, equilibrado.}{a.dul.to}{0}
\verb{adunco}{}{}{}{}{adj.}{Que tem a forma de gancho ou garra; curvo.}{a.dun.co}{0}
\verb{adunco}{}{}{}{}{}{Aquilino (nariz).}{a.dun.co}{0}
\verb{adustão}{}{}{"-ões}{}{s.f.}{Abrasamento.}{a.dus.tão}{0}
\verb{adustão}{}{Med.}{"-ões}{}{}{Cauterização. }{a.dus.tão}{0}
\verb{adusto}{}{}{}{}{adj.}{Queimado por calor intenso.}{a.dus.to}{0}
\verb{adusto}{}{}{}{}{}{Abrasador, muito quente.}{a.dus.to}{0}
\verb{adutor}{ô}{}{}{}{adj.}{Que aduz, transporta.}{a.du.tor}{0}
\verb{adutor}{ô}{}{}{}{s.m.}{Pessoa que conduz, que leva. }{a.du.tor}{0}
\verb{adutor}{ô}{Med.}{}{}{adj.}{Diz"-se do músculo que aproxima um órgão do eixo ou linha média do corpo.}{a.du.tor}{0}
\verb{adutora}{ô}{}{}{}{s.f.}{Canal, cano ou galeria que conduz água do manancial para o reservatório ou de um reservatório para outro.}{a.du.to.ra}{0}
\verb{aduzir}{}{}{}{}{v.t.}{Trazer, conduzir.}{a.du.zir}{0}
\verb{aduzir}{}{}{}{}{}{Apresentar, expor.}{a.du.zir}{\verboinum{21}}
\verb{ádvena}{}{}{}{}{adj.2g.}{Que veio de fora, de outro lugar; estrangeiro, forasteiro, adventício.}{ád.ve.na}{0}
\verb{ádvena}{}{}{}{}{s.2g.}{Pessoa que chega de fora, adventício.}{ád.ve.na}{0}
\verb{adventício}{}{}{}{}{adj.}{Que veio de fora, de outro lugar; estrangeiro, forasteiro, ádvena.}{ad.ven.tí.cio}{0}
\verb{adventício}{}{}{}{}{}{Inesperado, acidental.}{ad.ven.tí.cio}{0}
\verb{adventício}{}{}{}{}{s.m.}{Pessoa que chega de fora, ádvena.}{ad.ven.tí.cio}{0}
\verb{adventismo}{}{}{}{}{s.m.}{Doutrina ou seita protestante, que acredita numa segunda vinda de Jesus Cristo à Terra.}{ad.ven.tis.mo}{0}
\verb{adventista}{}{}{}{}{adj.2g.}{Que segue o adventismo.}{ad.ven.tis.ta}{0}
\verb{adventista}{}{}{}{}{}{Relativo ou pertencente ao adventismo ou aos adventistas.}{ad.ven.tis.ta}{0}
\verb{adventista}{}{}{}{}{s.2g.}{Indivíduo que é seguidor do adventismo.}{ad.ven.tis.ta}{0}
\verb{advento}{}{}{}{}{s.m.}{Chegada, vinda.}{ad.ven.to}{0}
\verb{advento}{}{}{}{}{}{Aparecimento, surgimento, começo.}{ad.ven.to}{0}
\verb{advento}{}{Relig.}{}{}{}{Período das quatro semanas que antecedem o Natal.}{ad.ven.to}{0}
\verb{adverbial}{}{Gram.}{"-ais}{}{adj.2g.}{Relativo a advérbio.}{ad.ver.bi.al}{0}
\verb{adverbial}{}{}{"-ais}{}{}{Que tem valor de advérbio.}{ad.ver.bi.al}{0}
\verb{adverbializar}{}{Gram.}{}{}{v.t.}{Dar função ou forma de advérbio; transformar em advérbio. }{ad.ver.bi.a.li.zar}{\verboinum{1}}
\verb{advérbio}{}{Gram.}{}{}{s.m.}{Classe de palavra invariável, que não tem lugar definido na oração e funciona como modificador tanto de verbo, como de adjetivo, de outro advérbio ou de frase.   }{ad.vér.bio}{0}
\verb{adversário}{}{}{}{}{adj.}{Que se opõe, que combate.}{ad.ver.sá.rio}{0}
\verb{adversário}{}{}{}{}{s.m.}{Opositor, antagonista.}{ad.ver.sá.rio}{0}
\verb{adversativa}{}{Gram.}{}{}{adj.}{Diz"-se da conjunção que une dois termos ou duas orações, estabelecendo entre eles uma relação de contraste ou oposição.}{ad.ver.sa.ti.va}{0}
\verb{adversativo}{}{}{}{}{adj.}{Que expressa oposição; adverso, contrário a.}{ad.ver.sa.ti.vo}{0}
\verb{adversidade}{}{}{}{}{s.f.}{Qualidade do que é adverso.}{ad.ver.si.da.de}{0}
\verb{adversidade}{}{}{}{}{}{Infortúnio, má sorte, revés.}{ad.ver.si.da.de}{0}
\verb{adversidade}{}{}{}{}{}{Contrariedade, transtorno.}{ad.ver.si.da.de}{0}
\verb{adverso}{é}{}{}{}{adj.}{Que apresenta oposição; contrário.}{ad.ver.so}{0}
\verb{adverso}{é}{}{}{}{}{Que traz infortúnio; prejudicial.}{ad.ver.so}{0}
\verb{adverso}{é}{}{}{}{}{Antagônico, hostil.}{ad.ver.so}{0}
\verb{advertência}{}{}{}{}{s.f.}{Ato ou efeito de advertir.}{ad.ver.tên.cia}{0}
\verb{advertência}{}{}{}{}{}{Repreensão leve; admoestação.}{ad.ver.tên.cia}{0}
\verb{advertência}{}{}{}{}{}{Observação, aviso.}{ad.ver.tên.cia}{0}
\verb{advertir}{}{}{}{}{v.t.}{Informar, avisar.}{ad.ver.tir}{0}
\verb{advertir}{}{}{}{}{}{Censurar ou repreender levemente; admoestar.}{ad.ver.tir}{0}
\verb{advertir}{}{}{}{}{}{Atentar em, reparar, observar.}{ad.ver.tir}{0}
\verb{advertir}{}{}{}{}{}{Aconselhar, prevenir.}{ad.ver.tir}{\verboinum{29}}
\verb{advindo}{}{}{}{}{adj.}{Que adveio ou sobreveio.}{ad.vin.do}{0}
\verb{advir}{}{}{}{}{v.i.}{Suceder, ocorrer, sobrevir.}{ad.vir}{0}
\verb{advir}{}{}{}{}{}{Resultar, provir.}{ad.vir}{\verboinum{56}}
\verb{advocacia}{}{}{}{}{s.f.}{Ação ou processo de advogar.}{ad.vo.ca.ci.a}{0}
\verb{advocacia}{}{Jur.}{}{}{}{Profissão de advogado.}{ad.vo.ca.ci.a}{0}
\verb{advocatício}{}{}{}{}{adj.}{Relativo à advocacia ou a advogado.}{ad.vo.ca.tí.cio}{0}
\verb{advogado}{}{}{}{}{s.m.}{Indivíduo que advoga; defensor, mediador.}{ad.vo.ga.do}{0}
\verb{advogado}{}{Jur.}{}{}{}{Indivíduo habilitado legalmente a exercer a advocacia.}{ad.vo.ga.do}{0}
\verb{advogar}{}{}{}{}{v.t.}{Fazer a defesa de.}{ad.vo.gar}{0}
\verb{advogar}{}{}{}{}{}{Interceder a favor de.}{ad.vo.gar}{0}
\verb{advogar}{}{Jur.}{}{}{}{Defender em juízo.}{ad.vo.gar}{0}
\verb{advogar}{}{}{}{}{v.i.}{Exercer a profissão de advogado.}{ad.vo.gar}{\verboinum{5}}
\verb{aedo}{é}{}{}{}{s.m.}{Poeta, na Grécia antiga, que recitava ao som da lira.}{a.e.do}{0}
\verb{aedo}{é}{Por ext.}{}{}{}{Poeta, trovador.}{a.e.do}{0}
\verb{aeração}{}{}{"-ões}{}{s.f.}{Ato ou efeito de aerar; ventilação.}{a.e.ra.ção}{0}
\verb{aeração}{}{}{"-ões}{}{}{Processo que consiste na purificação da água por meio do ar.}{a.e.ra.ção}{0}
\verb{aerado}{}{}{}{}{adj.}{Exposto à ação do ar; arejado, ventilado.}{a.e.ra.do}{0}
\verb{aerado}{}{}{}{}{}{Diz"-se do material que recebeu bolhas de ar em seu interior.}{a.e.ra.do}{0}
\verb{aerar}{}{}{}{}{v.t.}{Expor à ação do ar; ventilar, arejar.}{a.e.rar}{0}
\verb{aerar}{}{}{}{}{}{Encher de bolhas de ar.}{a.e.rar}{\verboinum{1}}
\verb{aéreo}{}{}{}{}{adj.}{Relativo ao ar.}{a.é.re.o}{0}
\verb{aéreo}{}{}{}{}{}{Que está suspenso ou se desloca no ar.}{a.é.re.o}{0}
\verb{aéreo}{}{}{}{}{}{Que vive ou se desenvolve no ar.}{a.é.re.o}{0}
\verb{aéreo}{}{}{}{}{}{Relativo à aviação.}{a.é.re.o}{0}
\verb{aéreo}{}{Fig.}{}{}{}{Elevado, superior.}{a.é.re.o}{0}
\verb{aéreo}{}{Fig.}{}{}{}{Leve, vaporoso, diáfano.}{a.é.re.o}{0}
\verb{aéreo}{}{Fig.}{}{}{}{Distraído, desatento.}{a.é.re.o}{0}
\verb{aerícola}{}{}{}{}{adj.2g.}{Que vive no ar.}{a.e.rí.co.la}{0}
\verb{aerobarco}{}{}{}{}{s.m.}{Barco a motor, dotado de asas, que o elevam e sustentam acima do nível da água, fazendo"-o deslocar"-se rapidamente.}{a.e.ro.bar.co}{0}
\verb{aeróbica}{}{}{}{}{adj.}{Diz"-se da ginástica cujos movimentos ativam a respiração e a oxigenação dos tecidos. }{a.e.ró.bi.ca}{0}
\verb{aeróbio}{}{Biol.}{}{}{adj.}{Diz"-se do organismo que necessita do oxigênio retirado do ar.}{a.e.ró.bio}{0}
\verb{aeróbio}{}{}{}{}{s.m.}{Organismo que se desenvolve somente na presença de ar ou de oxigênio.}{a.e.ró.bio}{0}
\verb{aerobionte}{}{Biol.}{}{}{s.m.}{Organismo aeróbio.}{a.e.ro.bi.on.te}{0}
\verb{aeroclube}{}{}{}{}{s.m.}{Centro de reunião ou de treinamento de pilotos da aviação civil.}{a.e.ro.clu.be}{0}
\verb{aerodinâmica}{}{Fís.}{}{}{s.f.}{Ciência que estuda o ar e os gases em movimento, especialmente em relação às propriedades dos corpos sólidos que interagem com eles.}{a.e.ro.di.nâ.mi.ca}{0}
\verb{aerodinâmico}{}{Fís.}{}{}{adj.}{Relativo à aerodinâmica.}{a.e.ro.di.nâ.mi.co}{0}
\verb{aerodinâmico}{}{}{}{}{}{Diz"-se do corpo sólido que, pela sua forma, apresenta baixa resistência na sua passagem pelo ar.}{a.e.ro.di.nâ.mi.co}{0}
\verb{aerodinâmico}{}{Por ext.}{}{}{}{Que apresenta desenho moderno, contemporâneo.}{a.e.ro.di.nâ.mi.co}{0}
\verb{aeródino}{}{}{}{}{s.m.}{Designação genérica de qualquer aparelho de voo mais pesado que o ar, como aviões, helicópteros etc.}{a.e.ró.di.no}{0}
\verb{aeródromo}{}{}{}{}{s.m.}{Área destinada a pouso e decolagem de aeronaves; campo de aviação, aeroporto.}{a.e.ró.dro.mo}{0}
\verb{aeroduto}{}{}{}{}{s.m.}{Conduto de ar utilizado para renovação de ar em edificações modernas.}{a.e.ro.du.to}{0}
\verb{aeroespacial}{}{}{"-ais}{}{adj.2g.}{Relativo a aeroespaço.}{a.e.ro.es.pa.ci.al}{0}
\verb{aeroespaço}{}{}{}{}{s.m.}{Região destinada a lançamento e controle de foguetes e satélites artificiais; espaço aéreo.}{a.e.ro.es.pa.ço}{0}
\verb{aerofagia}{}{Med.}{}{}{s.f.}{Deglutição exagerada de ar.}{a.e.ro.fa.gi.a}{0}
\verb{aerofólio}{}{}{}{}{s.m.}{Peça adaptada na traseira de um veículo, destinada a lhe dar maior estabilidade.}{a.e.ro.fó.lio}{0}
\verb{aerofoto}{ó}{}{}{}{s.f.}{Fotografia da Terra tirada de uma aeronave; aerofotografia.}{a.e.ro.fo.to}{0}
\verb{aerofotografia}{}{}{}{}{s.f.}{Método que consiste em fotografar utilizando câmeras instaladas em aeronaves; aerofoto.}{a.e.ro.fo.to.gra.fi.a}{0}
\verb{aerofotogrametria}{}{}{}{}{s.f.}{Técnica de levantamento fotográfico e geodésico da superfície terrestre, que utiliza câmeras instaladas em aeronaves.}{a.e.ro.fo.to.gra.me.tri.a}{0}
\verb{aerografia}{}{}{}{}{s.f.}{Estudo do ar atmosférico e de seus gases.}{a.e.ro.gra.fi.a}{0}
\verb{aerografia}{}{Art.}{}{}{}{Técnica de pintar ou envernizar utilizando"-se aerógrafo.}{a.e.ro.gra.fi.a}{0}
\verb{aerógrafo}{}{}{}{}{s.m.}{Instrumento a ar comprimido, utilizado na pintura e no envernizamento de materiais.}{a.e.ró.gra.fo}{0}
\verb{aerograma}{}{}{}{}{s.m.}{Papel de carta já franqueada que, dobrado, adquire a forma de envelope.}{a.e.ro.gra.ma}{0}
\verb{aerólito}{}{Geol.}{}{}{s.m.}{Tipo de meteorito composto principalmente de silicatos.}{a.e.ró.li.to}{0}
\verb{aeromoça}{ô}{}{}{}{s.f.}{Funcionária de empresas aéreas encarregada de cuidar do conforto e da segurança dos passageiros; comissária de bordo.}{a.e.ro.mo.ça}{0}
\verb{aeromodelismo}{}{}{}{}{s.m.}{Técnica de projetar e construir aeromodelos.}{a.e.ro.mo.de.lis.mo}{0}
\verb{aeromodelismo}{}{}{}{}{}{Esporte que se pratica com aeromodelos.}{a.e.ro.mo.de.lis.mo}{0}
\verb{aeromodelo}{ê}{}{}{}{s.m.}{Miniatura ou modelo de aeronaves utilizado para fins recreativos ou de estudo.}{a.e.ro.mo.de.lo}{0}
\verb{aeronauta}{}{}{}{}{s.2g.}{Pessoa que comanda ou tripula aeronaves.}{a.e.ro.nau.ta}{0}
\verb{aeronáutica}{}{}{}{}{s.f.}{Ciência ou prática da navegação aérea; aeronavegação.}{a.e.ro.náu.ti.ca}{0}
\verb{aeronáutica}{}{}{}{}{}{Força aérea de um país.}{a.e.ro.náu.ti.ca}{0}
\verb{aeronáutico}{}{}{}{}{adj.}{Relativo à aeronáutica.}{a.e.ro.náu.ti.co}{0}
\verb{aeronaval}{}{}{"-ais}{}{adj.2g.}{Relativo às forças aérea e naval, ou à força aérea da marinha de guerra.}{a.e.ro.na.val}{0}
\verb{aeronave}{}{}{}{}{s.f.}{Denominação comum dos aparelhos voadores, tais como aviões, planadores, helicópteros etc.}{a.e.ro.na.ve}{0}
\verb{aeronavegação}{}{}{"-ões}{}{s.f.}{Navegação humana que se faz pelo ar, com a ajuda de aparelhos; aeronáutica.}{a.e.ro.na.ve.ga.ção}{0}
\verb{aeropista}{}{}{}{}{s.f.}{Pista destinada a pouso e decolagem de aviões.}{a.e.ro.pis.ta}{0}
\verb{aeroplano}{}{}{}{}{s.m.}{Veículo aéreo mais pesado que o ar, propulsionado por hélice ou a jato; avião.}{a.e.ro.pla.no}{0}
\verb{aeroporto}{ô}{}{"-s ⟨ó⟩}{}{s.m.}{Campo de pouso e decolagem de aviões, com instalações técnicas e comerciais necessárias ao tráfego aéreo, à manutenção das aeronaves e ao embarque e desembarque de passageiros e de carga; campo de aviação, aeródromo.}{a.e.ro.por.to}{0}
\verb{aerossol}{ó}{Quím.}{"-óis}{}{s.m.}{Suspensão de partículas sólidas ou líquidas em ar ou gás.}{a.e.ros.sol}{0}
\verb{aerossol}{ó}{Por ext.}{"-óis}{}{}{Embalagem usada para borrifar essas partículas.}{a.e.ros.sol}{0}
\verb{aerostática}{}{}{}{}{s.f.}{Estudo das técnicas de manejo e direção de aeróstatos, como os balões.}{a.e.ros.tá.ti.ca}{0}
\verb{aerostática}{}{Fís.}{}{}{}{Ciência que estuda os gases em equilíbrio.}{a.e.ros.tá.ti.ca}{0}
\verb{aerostático}{}{}{}{}{adj.}{Relativo à aerostática ou aos aeróstatos.}{a.e.ros.tá.ti.co}{0}
\verb{aeróstato}{}{}{}{}{s.m.}{Veículo aéreo ascensionado por um gás mais leve que o ar, como os balões e os dirigíveis.}{a.e.rós.ta.to}{0}
\verb{aeroterrestre}{é}{}{}{}{adj.2g.}{Relativo ao ar e à terra.}{a.e.ro.ter.res.tre}{0}
\verb{aerotropismo}{}{Bot.}{}{}{s.m.}{Influência do ar na orientação do crescimento de um vegetal.}{a.e.ro.tro.pis.mo}{0}
\verb{aerovia}{}{}{}{}{s.f.}{Espaço aéreo reservado para o tráfego de aeronaves.}{a.e.ro.vi.a}{0}
\verb{aerovia}{}{}{}{}{}{Rota ou percurso regular dos aviões comerciais.}{a.e.ro.vi.a}{0}
\verb{aerovia}{}{}{}{}{}{Companhia de aviação.}{a.e.ro.vi.a}{0}
\verb{aeroviário}{}{}{}{}{adj.}{Relativo à aerovia. }{a.e.ro.vi.á.rio}{0}
\verb{aeroviário}{}{}{}{}{s.m.}{Indivíduo que trabalha em companhia de aviação. }{a.e.ro.vi.á.rio}{0}
\verb{aético}{}{}{}{}{adj.}{Contrário à ética; antiético.}{a.é.ti.co}{0}
\verb{afã}{}{}{}{}{s.m.}{Ansiedade, sofreguidão.}{a.fã}{0}
\verb{afã}{}{}{}{}{}{Trabalho intenso; lida.}{a.fã}{0}
\verb{afã}{}{}{}{}{}{Cuidado, zelo.}{a.fã}{0}
\verb{afabilidade}{}{}{}{}{s.f.}{Qualidade de quem é afável; amabilidade, cortesia, polidez.}{a.fa.bi.li.da.de}{0}
\verb{afadigar}{}{}{}{}{v.t.}{Causar fadiga a; cansar, fatigar.}{a.fa.di.gar}{0}
\verb{afadigar}{}{}{}{}{}{Importunar, incomodar.}{a.fa.di.gar}{0}
\verb{afadigar}{}{}{}{}{}{Perseguir, acossar.}{a.fa.di.gar}{\verboinum{5}}
\verb{afagar}{}{}{}{}{v.t.}{Fazer afago, carinho; mimar, acariciar.}{a.fa.gar}{0}
\verb{afagar}{}{}{}{}{}{Acalentar, nutrir, alentar.}{a.fa.gar}{\verboinum{5}}
\verb{afago}{}{}{}{}{s.m.}{Ato ou efeito de afagar; carinho, mimo.}{a.fa.go}{0}
\verb{afamado}{}{}{}{}{adj.}{Que se afamou; notável, célebre, famoso.}{a.fa.ma.do}{0}
\verb{afamar}{}{}{}{}{v.t.}{Dar fama a; notabilizar, celebrizar.}{a.fa.mar}{\verboinum{1}}
\verb{afanar}{}{}{}{}{v.t.}{Trabalhar, buscar com afã.}{a.fa.nar}{0}
\verb{afanar}{}{Pop.}{}{}{}{Roubar, furtar.}{a.fa.nar}{0}
\verb{afanar}{}{}{}{}{v.pron.}{Cansar"-se, fatigar"-se.}{a.fa.nar}{\verboinum{1}}
\verb{afanoso}{ô}{}{"-osos ⟨ó⟩}{"-osa ⟨ó⟩}{adj.}{Cheio de afã; trabalhoso, cansativo.}{a.fa.no.so}{0}
\verb{afasia}{}{Med.}{}{}{s.f.}{Perda quase total da capacidade de expressão ou compreensão de palavras, gestos ou símbolos por causa de lesões nas regiões cerebrais responsáveis por essas funções, sem alteração nos órgãos vocais.    }{a.fa.si.a}{0}
\verb{afásico}{}{}{}{}{adj.}{Relativo à afasia.}{a.fá.si.co}{0}
\verb{afásico}{}{}{}{}{s.m.}{Indivíduo que sofre de afasia.}{a.fá.si.co}{0}
\verb{afastado}{}{}{}{}{adj.}{Que se afastou.}{a.fas.ta.do}{0}
\verb{afastado}{}{}{}{}{}{Distante, remoto, longínquo.}{a.fas.ta.do}{0}
\verb{afastado}{}{}{}{}{}{Separado, apartado.}{a.fas.ta.do}{0}
\verb{afastamento}{}{}{}{}{s.m.}{Ato ou efeito de afastar.}{a.fas.ta.men.to}{0}
\verb{afastamento}{}{}{}{}{}{Distanciamento, separação.}{a.fas.ta.men.to}{0}
\verb{afastar}{}{}{}{}{v.t.}{Colocar de lado; distanciar.}{a.fas.tar}{0}
\verb{afastar}{}{}{}{}{}{Separar, apartar, isolar.}{a.fas.tar}{0}
\verb{afastar}{}{}{}{}{}{Demitir, exonerar.}{a.fas.tar}{0}
\verb{afastar}{}{}{}{}{v.pron.}{Tomar uma direção; dirigir"-se.}{a.fas.tar}{\verboinum{1}}
\verb{afável}{}{}{"-eis}{}{adj.2g.}{Amável, cortês, agradável, polido.}{a.fá.vel}{0}
\verb{afazer}{ê}{}{}{}{v.t.}{Acostumar, habituar.}{a.fa.zer}{0}
\verb{afazer}{ê}{}{}{}{}{Adaptar a um novo ambiente; aclimatar.}{a.fa.zer}{\verboinum{42}}
\verb{afazeres}{ê}{}{}{}{s.m.pl.}{Trabalhos, ocupações, negócios, quefazeres.}{a.fa.ze.res}{0}
\verb{afear}{}{}{}{}{v.t.}{Tornar feio; enfear.}{a.fe.ar}{\verboinum{4}}
\verb{afecção}{}{Med.}{"-ões}{}{s.f.}{Alteração patológica considerada sob seu aspecto atual, independente do diagnóstico; doença.}{a.fec.ção}{0}
\verb{afegane}{}{}{}{}{adj.2g. e s.2g.}{Afegão.}{a.fe.ga.ne}{0}
\verb{afegão}{}{}{"-ãos}{afegã}{adj.}{Relativo ou pertencente ao Afeganistão.}{a.fe.gão}{0}
\verb{afegão}{}{}{"-ãos}{afegã}{s.m.}{Indivíduo natural ou habitante do Afeganistão; afegane.}{a.fe.gão}{0}
\verb{afeição}{}{}{"-ões}{}{s.f.}{Afeto, amor, amizade.}{a.fei.ção}{0}
\verb{afeição}{}{}{"-ões}{}{}{Inclinação, pendor.}{a.fei.ção}{0}
\verb{afeiçoado}{}{}{}{}{adj.}{Que se afeiçoou a; que sente amor, amizade.}{a.fei.ço.a.do}{0}
\verb{afeiçoado}{}{}{}{}{}{Inclinado, dedicado.}{a.fei.ço.a.do}{0}
\verb{afeiçoado}{}{}{}{}{adj.}{Que tem feição ou forma de; amoldado.}{a.fei.ço.a.do}{0}
\verb{afeiçoar}{}{}{}{}{v.t.}{Inspirar afeição, amor, amizade.}{a.fei.ço.ar}{\verboinum{7}}
\verb{afeiçoar}{}{}{}{}{v.t.}{Dar feição ou forma a; amoldar, modelar.}{a.fei.ço.ar}{\verboinum{7}}
\verb{afeito}{ê}{}{}{}{adj.}{Acostumado, habituado, adaptado.}{a.fei.to}{0}
\verb{afélio}{}{Astron.}{}{}{s.m.}{Ponto da órbita de um planeta que fica à distância máxima do Sol.}{a.fé.lio}{0}
\verb{afélio}{}{Ant.}{}{}{}{periélio}{a.fé.lio}{0}
\verb{afeminado}{}{}{}{}{adj.}{Que se afeminou; efeminado.}{a.fe.mi.na.do}{0}
\verb{afeminado}{}{}{}{}{}{Diz"-se do homossexual masculino.}{a.fe.mi.na.do}{0}
\verb{afeminado}{}{Fig.}{}{}{}{Que aparenta fraqueza ou covardia.}{a.fe.mi.na.do}{0}
\verb{afeminado}{}{}{}{}{}{Que é excessivamente delicado, sensual, voluptuoso.}{a.fe.mi.na.do}{0}
\verb{afeminar}{}{}{}{}{v.t.}{Tornar semelhante ao que é feminino; efeminar.}{a.fe.mi.nar}{0}
\verb{afeminar}{}{Fig.}{}{}{}{Enfraquecer, acovardar.}{a.fe.mi.nar}{\verboinum{1}}
\verb{aferente}{}{}{}{}{adj.2g.}{Que conduz ou leva.}{a.fe.ren.te}{0}
\verb{aferente}{}{Anat.}{}{}{}{Relativo ao vaso que leva líquido da periferia para um órgão ou ao nervo que conduz um impulso a um centro nervoso.}{a.fe.ren.te}{0}
\verb{aférese}{}{Gram.}{}{}{s.f.}{Supressão de fonema ou sílaba inicial de uma palavra.}{a.fé.re.se}{0}
\verb{aferição}{}{}{"-ões}{}{s.f.}{Ato ou efeito de aferir.}{a.fe.ri.ção}{0}
\verb{aferição}{}{}{"-ões}{}{}{Marca de que algo foi aferido.}{a.fe.ri.ção}{0}
\verb{aferido}{}{}{}{}{adj.}{Que se aferiu; conferido com os padrões.}{a.fe.ri.do}{0}
\verb{aferido}{}{}{}{}{s.m.}{Calha por onde cai a água para mover uma roda hidráulica.}{a.fe.ri.do}{0}
\verb{aferidor}{ô}{}{}{}{adj.}{Que afere.}{a.fe.ri.dor}{0}
\verb{aferidor}{ô}{}{}{}{s.m.}{Instrumento para aferir.}{a.fe.ri.dor}{0}
\verb{aferir}{}{}{}{}{v.t.}{Conferir pesos e medidas com os respectivos padrões; afilar.}{a.fe.rir}{0}
\verb{aferir}{}{}{}{}{}{Colocar a marca de aferição.}{a.fe.rir}{0}
\verb{aferir}{}{}{}{}{}{Cotejar, comparar, avaliar.}{a.fe.rir}{\verboinum{29}}
\verb{aferrado}{}{}{}{}{adj.}{Preso com ferro.}{a.fer.ra.do}{0}
\verb{aferrado}{}{}{}{}{}{Obstinado, teimoso.}{a.fer.ra.do}{0}
\verb{aferrar}{}{}{}{}{v.t.}{Prender com ferro.}{a.fer.rar}{0}
\verb{aferrar}{}{}{}{}{}{Agarrar com força; segurar.}{a.fer.rar}{0}
\verb{aferrar}{}{}{}{}{}{Atacar, investir contra.}{a.fer.rar}{0}
\verb{aferrar}{}{}{}{}{}{Firmar, fincar.}{a.fer.rar}{0}
\verb{aferrar}{}{}{}{}{v.pron.}{Teimar, obstinar"-se.}{a.fer.rar}{\verboinum{1}}
\verb{aferro}{ê}{}{}{}{s.m.}{Ato ou efeito de aferrar"-se, de apegar"-se excessivamente.}{a.fer.ro}{0}
\verb{aferro}{ê}{}{}{}{}{Teimosia, insistência, afinco.}{a.fer.ro}{0}
\verb{aferroar}{}{}{}{}{v.t.}{Ferir com ponta perfurante; picar, aguilhoar.}{a.fer.ro.ar}{0}
\verb{aferroar}{}{Por ext.}{}{}{}{Ferir com o órgão pontiagudo de certos insetos, como o da abelha, do marimbondo etc.; espicaçar.}{a.fer.ro.ar}{0}
\verb{aferroar}{}{}{}{}{}{Magoar, afligir, incomodar.}{a.fer.ro.ar}{0}
\verb{aferroar}{}{}{}{}{}{Incitar, provocar.}{a.fer.ro.ar}{\verboinum{7}}
\verb{aferrolhar}{}{}{}{}{v.t.}{Fechar com tranca corrediça de ferro,  ferrolho.}{a.fer.ro.lhar}{0}
\verb{aferrolhar}{}{}{}{}{}{Colocar na prisão; prender.}{a.fer.ro.lhar}{0}
\verb{aferrolhar}{}{}{}{}{}{Guardar em lugar seguro com muito cuidado.}{a.fer.ro.lhar}{\verboinum{1}}
\verb{aferventado}{}{}{}{}{adj.}{Que se ferveu ligeiramente.}{a.fer.ven.ta.do}{0}
\verb{aferventado}{}{}{}{}{}{Levemente cozido.}{a.fer.ven.ta.do}{0}
\verb{aferventado}{}{Fig.}{}{}{}{Estimulado, excitado; inquieto.}{a.fer.ven.ta.do}{0}
\verb{aferventado}{}{Cul.}{}{}{s.m.}{Prato preparado com carne ou peixe, batatas e legumes ligeiramente cozidos.  }{a.fer.ven.ta.do}{0}
\verb{aferventar}{}{}{}{}{v.t.}{Submeter a uma rápida fervura.}{a.fer.ven.tar}{0}
\verb{aferventar}{}{Fig.}{}{}{}{Estimular, incentivar.}{a.fer.ven.tar}{\verboinum{1}}
\verb{afervorar}{}{}{}{}{v.t.}{Pôr em fervura, em ebulição; ferver; aferventar.}{a.fer.vo.rar}{0}
\verb{afervorar}{}{}{}{}{}{Excitar o fervor; tornar ardoroso.}{a.fer.vo.rar}{0}
\verb{afervorar}{}{}{}{}{}{Intensificar; enfatizar.}{a.fer.vo.rar}{\verboinum{1}}
\verb{afetação}{}{}{"-ões}{}{s.f.}{Ato ou efeito de afetar, de fazer crer; maneira artificial de ser; amaneiramento.}{a.fe.ta.ção}{0}
\verb{afetação}{}{}{"-ões}{}{}{Atitude falsa; fingimento.}{a.fe.ta.ção}{0}
\verb{afetação}{}{}{"-ões}{}{}{Vaidade; pedantismo; presunção.}{a.fe.ta.ção}{0}
\verb{afetado}{}{}{}{}{adj.}{Que se afetou; sem naturalidade; fingido.}{a.fe.ta.do}{0}
\verb{afetado}{}{}{}{}{}{Que sofreu uma afecção ou foi acometido por uma doença}{a.fe.ta.do}{0}
\verb{afetado}{}{}{}{}{}{Tocado por um sentimento.}{a.fe.ta.do}{0}
\verb{afetar}{}{}{}{}{v.t.}{Fazer crer; fingir; simular.}{a.fe.tar}{0}
\verb{afetar}{}{}{}{}{}{Apresentar ou imitar a forma de alguma coisa ou de um ser; representar.}{a.fe.tar}{0}
\verb{afetar}{}{}{}{}{}{Provocar lesão em; atingir; acometer.}{a.fe.tar}{0}
\verb{afetar}{}{}{}{}{}{Impressionar afetivamente; comover; abalar.}{a.fe.tar}{0}
\verb{afetar}{}{}{}{}{}{Dizer respeito a; concernir; interessar.}{a.fe.tar}{\verboinum{1}}
\verb{afetividade}{}{}{}{}{s.f.}{Qualidade ou caráter de quem é afetivo; afeição; carinho.}{a.fe.ti.vi.da.de}{0}
\verb{afetividade}{}{}{}{}{}{Fenômenos psíquicos de um indivíduo que são experimentados e vivenciados em forma de emoções e sentimentos.}{a.fe.ti.vi.da.de}{0}
\verb{afetivo}{}{}{}{}{adj.}{Que tem afeto; dedicado, carinhoso. }{a.fe.ti.vo}{0}
\verb{afeto}{é}{}{}{}{s.m.}{Sentimento terno por uma pessoa ou um animal; afeição; simpatia.}{a.fe.to}{0}
\verb{afeto}{é}{}{}{}{}{Objeto de afeição.}{a.fe.to}{0}
\verb{afeto}{é}{}{}{}{adj.}{Que é afeiçoado; dedicado.}{a.fe.to}{0}
\verb{afeto}{é}{}{}{}{}{Inclinado ao partido, às opiniões de alguém; partidário.}{a.fe.to}{0}
\verb{afetuoso}{ô}{}{"-osos ⟨ó⟩}{"-osa ⟨ó⟩}{adj.}{Que tem afeto; carinhoso; afável; terno.}{a.fe.tu.o.so}{0}
\verb{afiado}{}{}{}{}{adj.}{Que tem o gume amolado; cortante, aguçado.}{a.fi.a.do}{0}
\verb{afiado}{}{Fig.}{}{}{}{Que tem perspicácia; penetrante.}{a.fi.a.do}{0}
\verb{afiador}{ô}{}{}{}{adj.}{Que afia, que torna cortante.}{a.fi.a.dor}{0}
\verb{afiador}{ô}{}{}{}{s.m.}{Indivíduo ou objeto que afia; amolador.}{a.fi.a.dor}{0}
\verb{afiançar}{}{}{}{}{v.t.}{Responsabilizar"-se por; ser fiador de; abonar.}{a.fi.an.çar}{0}
\verb{afiançar}{}{}{}{}{}{Apresentar como digno de confiança; assegurar; afirmar.}{a.fi.an.çar}{\verboinum{3}}
\verb{afiar}{}{}{}{}{v.t.}{Tornar cortante; amolar.}{a.fi.ar}{0}
\verb{afiar}{}{}{}{}{}{Afinar a ponta; tornar agudo; aguçar.}{a.fi.ar}{0}
\verb{afiar}{}{Fig.}{}{}{}{Refinar; aperfeiçoar.}{a.fi.ar}{\verboinum{1}}
\verb{aficionado}{}{}{}{}{adj.}{Que é entusiasta; simpatizante.}{a.fi.ci.o.na.do}{0}
\verb{aficionado}{}{}{}{}{s.m.}{Pessoa propensa a determinada atividade; amador.}{a.fi.ci.o.na.do}{0}
\verb{afidalgado}{}{}{}{}{adj.}{Que tem aparência, hábitos ou costumes próprios de fidalgo; nobre; distinto.}{a.fi.dal.ga.do}{0}
\verb{afidalgar}{}{}{}{}{v.t.}{Tornar fidalgo ou adquirir hábitos de fidalgo; enobrecer.}{a.fi.dal.gar}{\verboinum{5}}
\verb{afigurar}{}{}{}{}{v.t.}{Apresentar forma ou figura de; assemelhar.}{a.fi.gu.rar}{0}
\verb{afigurar}{}{}{}{}{}{Imaginar algo; idear.}{a.fi.gu.rar}{0}
\verb{afigurar}{}{}{}{}{}{Dar a impressão de; parecer.}{a.fi.gu.rar}{\verboinum{1}}
\verb{afilado}{}{}{}{}{adj.}{Que corta; amolado.}{a.fi.la.do}{0}
\verb{afilado}{}{}{}{}{}{Que é fino; delicado.}{a.fi.la.do}{0}
\verb{afilar}{}{}{}{}{v.t.}{Reduzir a fio; tornar fino, delgado ou aguçado; afinar.}{a.fi.lar}{0}
\verb{afilar}{}{}{}{}{v.i.}{Tornar fino, delgado ou aguçado; adelgaçar.}{a.fi.lar}{\verboinum{1}}
\verb{afilhadismo}{}{}{}{}{s.m.}{Proteção dada aos afilhados ou aos favoritos; nepotismo, favoritismo.}{a.fi.lha.dis.mo}{0}
\verb{afilhado}{}{}{}{}{s.m.}{Pessoa que recebeu o batismo ou a confirmação em relação ao padrinho e à madrinha.}{a.fi.lha.do}{0}
\verb{afilhado}{}{}{}{}{}{Pessoa que recebe proteção como se fosse filho, favorito.}{a.fi.lha.do}{0}
\verb{afiliação}{}{}{"-ões}{}{s.f.}{Ato ou efeito de afiliar, de agregar a uma corporação.}{a.fi.li.a.ção}{0}
\verb{afiliar}{}{}{}{}{v.t.}{Agregar a uma corporação ou sociedade; tornar membro ou sócio.}{a.fi.li.ar}{\verboinum{1}}
\verb{afim}{}{}{"-ins}{}{adj.2g.}{Que apresenta afinidade, semelhança ou ligação.}{a.fim}{0}
\verb{afim}{}{Jur.}{"-ins}{}{s.2g.}{Parente por afinidade.}{a.fim}{0}
\verb{afinação}{}{}{"-ões}{}{s.f.}{Ato ou afeito de afinar.}{a.fi.na.ção}{0}
\verb{afinação}{}{}{"-ões}{}{}{Aprimoramento, ajuste; harmonia, acordo; acabamento perfeito.}{a.fi.na.ção}{0}
\verb{afinação}{}{Mús.}{"-ões}{}{}{Ajuste entre todas as notas de um instrumento, de uma orquestra, de um grupo vocal, de um conjunto musical ou da voz humana. }{a.fi.na.ção}{0}
\verb{afinado}{}{}{}{}{adj.}{Que se afinou.}{a.fi.na.do}{0}
\verb{afinado}{}{}{}{}{}{Em concordância com algo.}{a.fi.na.do}{0}
\verb{afinado}{}{Ant.}{}{}{}{desafinado}{a.fi.na.do}{0}
\verb{afinador}{ô}{}{}{}{adj.}{Que afina instrumentos musicais ou metais.}{a.fi.na.dor}{0}
\verb{afinador}{ô}{}{}{}{s.m.}{Indivíduo especialista em afinar instrumentos.}{a.fi.na.dor}{0}
\verb{afinador}{ô}{Mús.}{}{}{}{Chave usada para afinar alguns instrumentos de corda.}{a.fi.na.dor}{0}
\verb{afinal}{}{}{}{}{adv.}{Por fim; finalmente, enfim.}{a.fi.nal}{0}
\verb{afinar}{}{}{}{}{v.t.}{Tornar fino ou mais fino; afilar; apurar.}{a.fi.nar}{0}
\verb{afinar}{}{Mús.}{}{}{}{Ajustar o som de instrumentos musicais ou vozes, entre si ou isoladamente.}{a.fi.nar}{0}
\verb{afinar}{}{}{}{}{}{Purificar metais; acrisolar.}{a.fi.nar}{0}
\verb{afinar}{}{Fig.}{}{}{}{Tornar melhor; aprimorar; aperfeiçoar.}{a.fi.nar}{0}
\verb{afinar}{}{Pop.}{}{}{v.i.}{Acovardar"-se.}{a.fi.nar}{\verboinum{1}}
\verb{afincar}{}{}{}{}{v.t.}{Fixar com ferro ou cravo; fincar; cravar.}{a.fin.car}{0}
\verb{afincar}{}{Fig.}{}{}{}{Concentrar pensamento, atenção em; fitar.}{a.fin.car}{0}
\verb{afincar}{}{Fig.}{}{}{}{Insistir em; teimar.}{a.fin.car}{\verboinum{2}}
\verb{afinco}{}{}{}{}{s.m.}{Ato ou efeito de afincar; de fixar, cravar.}{a.fin.co}{0}
\verb{afinco}{}{Fig.}{}{}{}{Conduta firme; apego; perseverança; persistência.}{a.fin.co}{0}
\verb{afinidade}{}{}{}{}{s.f.}{Vínculo de parentesco que se contrai pelo matrimônio.}{a.fi.ni.da.de}{0}
\verb{afinidade}{}{}{}{}{}{Coincidência ou semelhança de gostos ou sentimentos.}{a.fi.ni.da.de}{0}
\verb{afinidade}{}{}{}{}{}{Pontos comuns entre duas ou mais espécies.}{a.fi.ni.da.de}{0}
\verb{afinidade}{}{Quím.}{}{}{}{Tendência que duas substâncias têm de se combinarem.}{a.fi.ni.da.de}{0}
\verb{afirmação}{}{}{"-ões}{}{s.f.}{Ato ou efeito de afirmar, de dizer sim.}{a.fir.ma.ção}{0}
\verb{afirmação}{}{}{"-ões}{}{}{Aquilo que se afirma ou se sustenta como verdade; afirmativa; confirmação; asseveração.}{a.fir.ma.ção}{0}
\verb{afirmar}{}{}{}{}{v.t.}{Tornar ou fazer firme; estabelecer; consolidar.}{a.fir.mar}{0}
\verb{afirmar}{}{}{}{}{}{Dizer ou declarar algo assumindo o caráter de verdade do que é dito; assegurar; asseverar.}{a.fir.mar}{0}
\verb{afirmar}{}{}{}{}{}{Garantir a veracidade ou a existência de algo; atestar; comprovar.}{a.fir.mar}{0}
\verb{afirmar}{}{}{}{}{}{Olhar fixamente; firmar.}{a.fir.mar}{\verboinum{1}}
\verb{afirmativa}{}{}{}{}{s.f.}{Declaração que afirma; afirmação; confirmação.}{a.fir.ma.ti.va}{0}
\verb{afirmativo}{}{}{}{}{adj.}{Que afirma ou confirma; que revela certeza; positivo.}{a.fir.ma.ti.vo}{0}
\verb{afivelar}{}{}{}{}{v.t.}{Prender, segurar ou atar com fivela.}{a.fi.ve.lar}{\verboinum{1}}
\verb{afixar}{cs}{}{}{}{v.t.}{Tornar fixo; prender; firmar; fixar.}{a.fi.xar}{0}
\verb{afixar}{cs}{}{}{}{}{Pregar ou prender em lugar público avisos, editais etc.}{a.fi.xar}{\verboinum{1}}
\verb{afixo}{cs}{}{}{}{adj.}{Que se afixou; preso, fixo, unido.}{a.fi.xo}{0}
\verb{afixo}{cs}{}{}{}{s.m.}{Qualquer coisa que se afixa.}{a.fi.xo}{0}
\verb{afixo}{cs}{Gram.}{}{}{}{Designação genérica dos prefixos e dos sufixos. }{a.fi.xo}{0}
\verb{aflautado}{}{}{}{}{adj.}{Que tem aparência ou som de flauta.}{a.flau.ta.do}{0}
\verb{aflautado}{}{}{}{}{}{Som alto e em tom agudo.}{a.flau.ta.do}{0}
\verb{aflição}{}{}{"-ões}{}{s.f.}{Sofrimento intenso, físico ou moral; agonia, angústia.}{a.fli.ção}{0}
\verb{aflição}{}{}{"-ões}{}{}{Preocupação, inquietação.}{a.fli.ção}{0}
\verb{afligir}{}{}{}{}{v.t.}{Causar aflição, preocupação.}{a.fli.gir}{0}
\verb{afligir}{}{}{}{}{}{Atingir, devastar.}{a.fli.gir}{\verboinum{22}}
\verb{aflitivo}{}{}{}{}{adj.}{Que causa aflição; preocupante, angustiante.}{a.fli.ti.vo}{0}
\verb{aflito}{}{}{}{}{adj.}{Que se afligiu.}{a.fli.to}{0}
\verb{aflito}{}{}{}{}{}{Angustiado, agoniado.}{a.fli.to}{0}
\verb{aflito}{}{}{}{}{}{Preocupado, inquieto.}{a.fli.to}{0}
\verb{afloração}{}{}{"-ões}{}{s.f.}{Ato ou efeito de aflorar; afloramento, surgimento.}{a.flo.ra.ção}{0}
\verb{afloração}{}{}{"-ões}{}{}{Nivelamento de uma superfície.}{a.flo.ra.ção}{0}
\verb{afloramento}{}{Geol.}{}{}{s.m.}{Camada de rocha ou de minério que chega à superfície do solo por processos naturais ou artificiais.}{a.flo.ra.men.to}{0}
\verb{aflorar}{}{}{}{}{v.i.}{Vir à superfície; assomar, surgir.}{a.flo.rar}{0}
\verb{aflorar}{}{}{}{}{v.t.}{Pôr uma superfície no mesmo nível de outra; nivelar.}{a.flo.rar}{0}
\verb{aflorar}{}{}{}{}{}{Tocar levemente; acariciar.}{a.flo.rar}{\verboinum{1}}
\verb{afluência}{}{}{}{}{s.f.}{Corrente abundante; afluxo.}{a.flu.ên.cia}{0}
\verb{afluência}{}{}{}{}{}{Ponto onde os rios convergem.}{a.flu.ên.cia}{0}
\verb{afluência}{}{}{}{}{}{Grande reunião de pessoas ou de coisas.}{a.flu.ên.cia}{0}
\verb{afluente}{}{}{}{}{adj.2g.}{Que aflui, chega em grande quantidade; caudaloso, abundante.}{a.flu.en.te}{0}
\verb{afluente}{}{}{}{}{s.m.}{Diz"-se do curso de água que deságua em outro ou em um lago, aumentando"-lhe o volume.}{a.flu.en.te}{0}
\verb{afluir}{}{}{}{}{v.i.}{Correr para; vir.}{a.flu.ir}{0}
\verb{afluir}{}{}{}{}{}{Chegar em grande quantidade; convergir.}{a.flu.ir}{\verboinum{26}}
\verb{afluxo}{cs}{}{}{}{s.m.}{Ato ou efeito de afluir, correr para; afluência.}{a.flu.xo}{0}
\verb{afluxo}{cs}{}{}{}{}{Grande quantidade de fluido em movimentação para um ponto.}{a.flu.xo}{0}
\verb{afobação}{}{}{"-ões}{}{s.f.}{Ato ou efeito de afobar; afobamento.}{a.fo.ba.ção}{0}
\verb{afobação}{}{}{"-ões}{}{}{Muita pressa, precipitação.}{a.fo.ba.ção}{0}
\verb{afobação}{}{}{"-ões}{}{}{Perturbação, atrapalhação.}{a.fo.ba.ção}{0}
\verb{afobação}{}{}{"-ões}{}{}{Cansaço, fadiga.}{a.fo.ba.ção}{0}
\verb{afobado}{}{}{}{}{adj.}{Que tem muita pressa; precipitado.}{a.fo.ba.do}{0}
\verb{afobado}{}{}{}{}{}{Atrapalhado; perturbado.}{a.fo.ba.do}{0}
\verb{afobado}{}{}{}{}{}{Cansado; fatigado.}{a.fo.ba.do}{0}
\verb{afobamento}{}{}{}{}{s.m.}{Afobação.}{a.fo.ba.men.to}{0}
\verb{afobar}{}{}{}{}{v.t.}{Causar pressa, precipitação.}{a.fo.bar}{0}
\verb{afobar}{}{}{}{}{}{Perturbar, atrapalhar.}{a.fo.bar}{0}
\verb{afobar}{}{Pop.}{}{}{}{Cansar, fatigar.}{a.fo.bar}{\verboinum{1}}
\verb{afofar}{}{}{}{}{v.t.}{Tornar fofo, macio.}{a.fo.far}{0}
\verb{afofar}{}{Fig.}{}{}{}{Tornar vaidoso; vangloriar.}{a.fo.far}{\verboinum{1}}
\verb{afogadilho}{}{}{}{}{s.m.}{Pressa, precipitação, afobação.}{a.fo.ga.di.lho}{0}
\verb{afogado}{}{}{}{}{adj.}{Que se afogou; sufocado.}{a.fo.ga.do}{0}
\verb{afogado}{}{}{}{}{}{Diz"-se do motor enguiçado por excesso de combustível ou falta de ar no carburador.}{a.fo.ga.do}{0}
\verb{afogado}{}{}{}{}{s.m.}{Indivíduo que morreu por afogamento.}{a.fo.ga.do}{0}
\verb{afogador}{ô}{}{}{}{adj.}{Que afoga ou sufoca.}{a.fo.ga.dor}{0}
\verb{afogador}{ô}{}{}{}{s.m.}{Adereço para o pescoço; colar, gargantilha.}{a.fo.ga.dor}{0}
\verb{afogador}{ô}{}{}{}{}{Válvula do carro que limita a entrada de ar para o carburador, enriquecendo a mistura do combustível a fim de facilitar a partida quando o motor está frio.}{a.fo.ga.dor}{0}
\verb{afogamento}{}{}{}{}{s.m.}{Ato ou efeito de afogar; sufocação, asfixia.}{a.fo.ga.men.to}{0}
\verb{afogamento}{}{}{}{}{}{Morte por submersão em líquido.}{a.fo.ga.men.to}{0}
\verb{afogar}{}{}{}{}{v.t.}{Impedir a respiração; sufocar.}{a.fo.gar}{0}
\verb{afogar}{}{}{}{}{}{Matar ou morrer por submersão em um líquido.}{a.fo.gar}{0}
\verb{afogar}{}{}{}{}{}{Fazer parar o motor de um veículo por excesso de combustível ou por falta de ar no carburador.}{a.fo.gar}{\verboinum{5}}
\verb{afogueado}{}{}{}{}{adj.}{Submetido a muito fogo ou a muito calor; ardente.}{a.fo.gue.a.do}{0}
\verb{afogueado}{}{}{}{}{}{Corado, avermelhado, enrubescido.}{a.fo.gue.a.do}{0}
\verb{afogueado}{}{Fig.}{}{}{}{Cheio de entusiasmo; excitado.}{a.fo.gue.a.do}{0}
\verb{afoguear}{}{}{}{}{v.t.}{Pôr fogo em; queimar.}{a.fo.gue.ar}{0}
\verb{afoguear}{}{}{}{}{}{Fazer corar; avermelhar, ruborizar.}{a.fo.gue.ar}{0}
\verb{afoguear}{}{}{}{}{}{Excitar, entusiasmar.}{a.fo.gue.ar}{\verboinum{4}}
\verb{afoitar"-se}{}{}{}{}{v.pron.}{Tornar"-se afoito; apressar"-se.}{a.foi.tar"-se}{0}
\verb{afoitar"-se}{}{}{}{}{}{Atrever"-se, arriscar"-se.}{a.foi.tar"-se}{\verboinum{1}}
\verb{afoiteza}{ê}{}{}{}{s.f.}{Qualidade do que é afoito; coragem, atrevimento, audácia.}{a.foi.te.za}{0}
\verb{afoito}{ô}{}{}{}{adj.}{Que tem coragem; destemido, audacioso.}{a.foi.to}{0}
\verb{afoito}{ô}{}{}{}{}{Apressado; ansioso; precipitado.}{a.foi.to}{0}
\verb{afonia}{}{Med.}{}{}{s.f.}{Perda total ou parcial da voz causada por lesão nos órgãos vocais.}{a.fo.ni.a}{0}
\verb{afônico}{}{}{}{}{adj.}{Indivíduo que sofre de afonia, que está sem voz.}{a.fô.ni.co}{0}
\verb{afora}{ó}{}{}{}{adv.}{Para fora; para o exterior; adiante. }{a.fo.ra}{0}
\verb{afora}{ó}{}{}{}{prep.}{Exceto, salvo, fora.}{a.fo.ra}{0}
\verb{aforamento}{}{Jur.}{}{}{s.m.}{Cessão ou transferência de domínio útil de um imóvel, mediante pagamento de uma pensão anual; enfiteuse.}{a.fo.ra.men.to}{0}
\verb{aforar}{}{Jur.}{}{}{v.t.}{Dar ou tomar por aforamento.}{a.fo.rar}{0}
\verb{aforar}{}{}{}{}{}{Conceder privilégios ou direitos.}{a.fo.rar}{\verboinum{1}}
\verb{aforar}{}{Pop.}{}{}{v.t.}{Tirar ou retirar; excetuar.}{a.fo.rar}{\verboinum{1}}
\verb{aforismo}{}{}{}{}{s.m.}{Frase ou sentença breve que encerra um pensamento ou um princípio moral; máxima, ditado.}{a.fo.ris.mo}{0}
\verb{aforístico}{}{}{}{}{adj.}{Relativo a aforismo, à máxima moral.}{a.fo.rís.ti.co}{0}
\verb{aformoseamento}{}{}{}{}{s.m.}{Ato ou efeito de aformosear; embelezamento.}{a.for.mo.se.a.men.to}{0}
\verb{aformosear}{}{}{}{}{v.t.}{Tornar formoso; embelezar.}{a.for.mo.se.ar}{0}
\verb{aformosear}{}{}{}{}{}{Colocar enfeite; adornar.}{a.for.mo.se.ar}{\verboinum{4}}
\verb{afortunado}{}{}{}{}{adj.}{Que foi abençoado com a boa sorte; feliz, venturoso, ditoso.}{a.for.tu.na.do}{0}
\verb{afoxé}{ch}{}{}{}{s.m.}{Grupo negro que desfila no carnaval da Bahia, cantando canções de candomblé em nagô ou iorubá.}{a.fo.xé}{0}
\verb{afrancesado}{}{}{}{}{adj.}{Que tem modos ou feição de francês.}{a.fran.ce.sa.do}{0}
\verb{afrancesar}{}{}{}{}{v.t.}{Tornar semelhante ao francês, nos costumes e usos.}{a.fran.ce.sar}{\verboinum{1}}
\verb{afreguesado}{}{}{}{}{adj.}{Que tem muitos fregueses; que é frequentado.}{a.fre.gue.sa.do}{0}
\verb{afreguesado}{}{}{}{}{}{Que se tornou freguês, cliente.}{a.fre.gue.sa.do}{0}
\verb{afreguesar}{}{}{}{}{v.t.}{Tornar freguês ou cliente de um estabelecimento.}{a.fre.gue.sar}{0}
\verb{afreguesar}{}{}{}{}{}{Atrair ou adquirir fregueses.}{a.fre.gue.sar}{\verboinum{1}}
\verb{afresco}{ê}{Art.}{}{}{s.m.}{Técnica de pintura mural que consiste na aplicação de pigmentos diluídos em água sobre um revestimento recente, ainda fresco.}{a.fres.co}{0}
\verb{afresco}{ê}{}{}{}{}{Pintura feita com essa técnica.}{a.fres.co}{0}
\verb{afretamento}{}{}{}{}{s.m.}{Ato ou efeito de afretar, alugar um veículo; fretamento.}{a.fre.ta.men.to}{0}
\verb{afretamento}{}{}{}{}{}{Contrato através do qual se adquire o direito de utilização total ou parcial de um navio.}{a.fre.ta.men.to}{0}
\verb{africâner}{ê}{}{}{}{s.m.}{Uma das línguas oficiais da África do Sul, originada do holandês do século \textsc{xvii}, falada pelos bôeres; africânder.}{a.fri.câ.ner}{0}
\verb{africanismo}{}{}{}{}{s.m.}{Conjunto dos estudos acerca da África.}{a.fri.ca.nis.mo}{0}
\verb{africanismo}{}{}{}{}{}{Palavra ou expressão originada de alguma língua africana.}{a.fri.ca.nis.mo}{0}
\verb{africanista}{}{}{}{}{adj.2g.}{Relativo ao africanismo.}{a.fri.ca.nis.ta}{0}
\verb{africanista}{}{}{}{}{s.2g.}{Indivíduo que se dedica aos estudos a respeito da África.}{a.fri.ca.nis.ta}{0}
\verb{africanizar}{}{}{}{}{v.t.}{Tornar africano; adquirir formas, costumes ou usos africanos.}{a.fri.ca.ni.zar}{\verboinum{1}}
\verb{africano}{}{}{}{}{adj.}{Relativo ou pertencente à África.}{a.fri.ca.no}{0}
\verb{africano}{}{}{}{}{s.m.}{Indivíduo natural ou habitante da África.}{a.fri.ca.no}{0}
\verb{afro}{}{}{}{}{adj.}{Que é inspirado em modelos e costumes da África negra; africano.}{a.fro}{0}
\verb{afro"-brasileiro}{ê}{}{afro"-brasileiros}{afro"-brasileira}{adj.}{Diz"-se das características brasileiras cuja origem é africana.}{a.fro"-bra.si.lei.ro}{0}
\verb{afro"-brasileiro}{ê}{}{afro"-brasileiros}{afro"-brasileira}{s.m.}{Indivíduo natural do Brasil com ascendência africana.}{a.fro"-bra.si.lei.ro}{0}
\verb{afrodisia}{}{}{}{}{s.f.}{Excitação sexual exagerada.}{a.fro.di.si.a}{0}
\verb{afrodisíaco}{}{}{}{}{adj.}{Relativo a Afrodite, deusa da mitologia grega que representava o amor.}{a.fro.di.sí.a.co}{0}
\verb{afrodisíaco}{}{}{}{}{}{Que excita ou restaura os desejos sexuais; excitante, libidinoso.}{a.fro.di.sí.a.co}{0}
\verb{afroixar}{ch}{}{}{}{}{Var. de \textit{afrouxar}.}{a.froi.xar}{0}
\verb{afronta}{}{}{}{}{s.f.}{Ato de ofensa pública; insulto, ultraje.}{a.fron.ta}{0}
\verb{afrontado}{}{}{}{}{adj.}{Que sofreu ofensa; ultrajado, insultado.}{a.fron.ta.do}{0}
\verb{afrontado}{}{}{}{}{}{Cansado, fatigado, ofegante.}{a.fron.ta.do}{0}
\verb{afrontamento}{}{}{}{}{s.m.}{Ato ou efeito de afrontar, ofender publicamente.}{a.fron.ta.men.to}{0}
\verb{afrontar}{}{}{}{}{v.t.}{Colocar frente a frente; confrontar, enfrentar.}{a.fron.tar}{0}
\verb{afrontar}{}{}{}{}{}{Dirigir insulto; ofender.}{a.fron.tar}{0}
\verb{afrontar}{}{}{}{}{v.i.}{Sentir cansaço ou aflição.}{a.fron.tar}{0}
\verb{afrontar}{}{}{}{}{v.pron.}{Comparar"-se, medir"-se.}{a.fron.tar}{\verboinum{1}}
\verb{afrontoso}{ô}{}{"-osos ⟨ó⟩}{"-osa ⟨ó⟩}{adj.}{Que causa afronta; injurioso, ofensivo.}{a.fron.to.so}{0}
\verb{afrontoso}{ô}{}{"-osos ⟨ó⟩}{"-osa ⟨ó⟩}{}{Que provoca asfixia; sufocante.}{a.fron.to.so}{0}
\verb{afrouxamento}{ch}{}{}{}{s.m.}{Ato ou efeito de afrouxar, de desapertar.}{a.frou.xa.men.to}{0}
\verb{afrouxamento}{ch}{}{}{}{}{Relaxamento dos músculos.}{a.frou.xa.men.to}{0}
\verb{afrouxamento}{ch}{}{}{}{}{Abrandamento; suavização.}{a.frou.xa.men.to}{0}
\verb{afrouxar}{ch}{}{}{}{v.t.}{Tornar frouxo, flexível, brando.}{a.frou.xar}{0}
\verb{afrouxar}{ch}{}{}{}{}{Reduzir a intensidade ou a velocidade.}{a.frou.xar}{0}
\verb{afrouxar}{ch}{}{}{}{}{Deixar mais largo; desapertar.}{a.frou.xar}{0}
\verb{afrouxar}{ch}{Fig.}{}{}{v.i.}{Diminuir o empenho, o entusiasmo; relaxar.}{a.frou.xar}{\verboinum{1}}
\verb{afta}{}{Med.}{}{}{s.f.}{Pequena ulceração superficial e dolorosa que aparece nas mucosas, especialmente da boca, dos lábios e da língua.}{af.ta}{0}
\verb{aftose}{ó}{}{}{}{s.f.}{Doença do gado, de origem virótica, contagiosa para a espécie humana. Nos gados bovino e suíno, determina bolhas na pele e nas mucosas, acompanhadas de febre e definhamento progressivo que pode levar à morte.}{af.to.se}{0}
\verb{aftose}{ó}{}{}{}{}{Febre aftosa.}{af.to.se}{0}
\verb{aftoso}{ô}{Med.}{"-osos ⟨ó⟩}{"-osa ⟨ó⟩}{adj.}{Que tem afta.}{af.to.so}{0}
\verb{aftoso}{ô}{Quím.}{"-osos ⟨ó⟩}{"-osa ⟨ó⟩}{}{Que causa afta.}{af.to.so}{0}
\verb{afugentar}{}{}{}{}{v.t.}{Pôr em fuga; afastar.}{a.fu.gen.tar}{0}
\verb{afugentar}{}{}{}{}{}{Fazer desaparecer}{a.fu.gen.tar}{\verboinum{1}}
\verb{afundamento}{}{}{}{}{s.m.}{Ato ou efeito de afundar, de fazer ir ao fundo.}{a.fun.da.men.to}{0}
\verb{afundamento}{}{}{}{}{}{Depressão em algumas partes do corpo, provocada por pancadas ou quedas fortes.}{a.fun.da.men.to}{0}
\verb{afundar}{}{}{}{}{v.t.}{Fazer ir ao fundo.}{a.fun.dar}{0}
\verb{afundar}{}{}{}{}{}{Tornar profundo; fazer penetrar profundamente.}{a.fun.dar}{0}
\verb{afundar}{}{Fig.}{}{}{}{Sair"-se mal em prova ou empreendimento; arruinar"-se.}{a.fun.dar}{0}
\verb{afundar}{}{}{}{}{v.i.}{Ir ao fundo.}{a.fun.dar}{\verboinum{1}}
\verb{afunilado}{}{}{}{}{adj.}{Que tem aspecto ou forma de funil.}{a.fu.ni.la.do}{0}
\verb{afunilado}{}{}{}{}{}{Que é aguçado.}{a.fu.ni.la.do}{0}
\verb{afunilar}{}{}{}{}{v.t.}{Dar ou tomar a forma de funil.}{a.fu.ni.lar}{0}
\verb{afunilar}{}{}{}{}{}{Estreitar.}{a.fu.ni.lar}{\verboinum{1}}
\verb{Ag}{}{Quím.}{}{}{}{Símb. da \textit{prata}.}{Ag}{0}
\verb{agá}{}{}{}{}{s.m.}{Nome da letra \textit{h}.}{a.gá}{0}
\verb{agachado}{}{}{}{}{adj.}{Que se agachou; acocorado; abaixado.}{a.ga.cha.do}{0}
\verb{agachado}{}{Fig.}{}{}{}{Que se encontra abatido; humilhado.}{a.ga.cha.do}{0}
\verb{agachamento}{}{}{}{}{s.m.}{Ato ou efeito de agachar"-se; abaixamento.}{a.ga.cha.men.to}{0}
\verb{agachamento}{}{Fig.}{}{}{}{Humilhação; abatimento.}{a.ga.cha.men.to}{0}
\verb{agachar"-se}{}{}{}{}{v.pron.}{Dobrar os joelhos e encolher o corpo; ficar de cócoras.}{a.ga.char"-se}{0}
\verb{agachar"-se}{}{}{}{}{}{Abaixar"-se.}{a.ga.char"-se}{0}
\verb{agachar"-se}{}{Fig.}{}{}{}{Humilhar"-se, submeter"-se.}{a.ga.char"-se}{\verboinum{1}}
\verb{agadanhar}{}{}{}{}{v.t.}{Lançar o gadanho a.}{a.ga.da.nhar}{0}
\verb{agadanhar}{}{}{}{}{}{Machucar ou agarrar com unhas ou garras.}{a.ga.da.nhar}{0}
\verb{agadanhar}{}{}{}{}{}{Praticar roubo; furtar.}{a.ga.da.nhar}{\verboinum{1}}
\verb{agalactia}{}{Med.}{}{}{s.f.}{Falta de secreção láctea no período pós"-parto.}{a.ga.lac.ti.a}{0}
\verb{agaloado}{}{}{}{}{adj.}{Que está guarnecido com galões.}{a.ga.lo.a.do}{0}
\verb{agaloado}{}{}{}{}{s.m.}{A guarnição de galões.}{a.ga.lo.a.do}{0}
\verb{ágape}{}{Relig.}{}{}{s.2g.}{Refeição em comum dos antigos cristãos para celebrar o rito eucarístico.}{á.ga.pe}{0}
\verb{ágape}{}{}{}{}{}{Banquete, almoço de confraternização entre amigos.}{á.ga.pe}{0}
\verb{ágar}{}{Quím.}{}{}{s.m.}{Ágar"-ágar.}{á.gar}{0}
\verb{ágar"-ágar}{}{Quím.}{ágar"-ágares}{}{s.m.}{Substância obtida de uma alga vermelha muito comum no Oriente, usada na indústria para dar consistência gelatinosa a alimentos, cosméticos etc.; ágar.}{á.gar"-á.gar}{0}
\verb{agárico}{}{Bot.}{}{}{s.m.}{Nome genérico de vários cogumelos do gênero \textit{Agaricus}, com inúmeras espécies comestíveis. }{a.gá.ri.co}{0}
\verb{agarração}{}{}{"-ões}{}{s.f.}{Afeição exagerada; apego; agarramento.}{a.gar.ra.ção}{0}
\verb{agarradiço}{}{}{}{}{adj.}{Que se agarra facilmente; pegajoso.}{a.gar.ra.di.ço}{0}
\verb{agarradiço}{}{Fig.}{}{}{}{Que importuna.}{a.gar.ra.di.ço}{0}
\verb{agarrado}{}{}{}{}{adj.}{Que está preso ou aprisionado com força.}{a.gar.ra.do}{0}
\verb{agarrado}{}{Fig.}{}{}{}{Apegado com firmeza; convicto.}{a.gar.ra.do}{0}
\verb{agarrado}{}{Fig.}{}{}{}{Muito ligado; muito unido; enlaçado.}{a.gar.ra.do}{0}
\verb{agarramento}{}{}{}{}{s.m.}{Ato ou efeito de agarrar, de segurar com força.}{a.gar.ra.men.to}{0}
\verb{agarramento}{}{Fig.}{}{}{}{Ligação, união estreita entre duas ou mais pessoas. }{a.gar.ra.men.to}{0}
\verb{agarrar}{}{}{}{}{v.t.}{Prender, segurar com força.}{a.gar.rar}{0}
\verb{agarrar}{}{}{}{}{}{Pegar em; apanhar.}{a.gar.rar}{0}
\verb{agarrar}{}{}{}{}{v.pron.}{Deixar prender a; aferrar;  persistir.}{a.gar.rar}{\verboinum{1}}
\verb{agasalhar}{}{}{}{}{v.t.}{Dar agasalho; hospedar; abrigar.}{a.ga.sa.lhar}{0}
\verb{agasalhar}{}{}{}{}{}{Cobrir com agasalho; enroupar.}{a.ga.sa.lhar}{\verboinum{1}}
\verb{agasalho}{}{}{}{}{s.m.}{Ato ou efeito de agasalhar.}{a.ga.sa.lho}{0}
\verb{agasalho}{}{}{}{}{}{Qualquer lugar que abrigue; alojamento; hospedagem.}{a.ga.sa.lho}{0}
\verb{agasalho}{}{}{}{}{}{Peça de vestuário que resguarda do frio; das quedas de temperatura.}{a.ga.sa.lho}{0}
\verb{agastado}{}{}{}{}{adj.}{Que se agastou; irado; encolerizado.}{a.gas.ta.do}{0}
\verb{agastado}{}{}{}{}{}{Que está aborrecido; amuado.}{a.gas.ta.do}{0}
\verb{agastamento}{}{}{}{}{s.m.}{ato ou efeito de agastar, de irritar; cólera; ira.}{a.gas.ta.men.to}{0}
\verb{agastamento}{}{}{}{}{}{Aquilo que aborrece; enfado; zanga.}{a.gas.ta.men.to}{0}
\verb{agastar}{}{}{}{}{v.t.}{Provocar irritação ou deixar"-se irritar; encolerizar.}{a.gas.tar}{0}
\verb{agastar}{}{}{}{}{}{Causar aborrecimento ou ficar aborrecido; zangar.}{a.gas.tar}{\verboinum{1}}
\verb{ágata}{}{Geol.}{}{}{s.f.}{Pedra semipreciosa, translúcida, de camadas distintas e multicolores, que serve para a manufatura de joias e objetos de arte.}{á.ga.ta}{0}
\verb{agatanhar}{}{}{}{}{v.t.}{Ferir com as unhas; unhar; arranhar. }{a.ga.ta.nhar}{0}
\verb{agatanhar}{}{}{}{}{}{Andar como  gato.}{a.ga.ta.nhar}{\verboinum{1}}
\verb{ágate}{}{}{}{}{s.m.}{Ferro esmaltado.}{á.ga.te}{0}
\verb{agauchado}{}{}{}{}{adj.}{Que tem aparência, modos ou hábitos de gaúcho.}{a.ga.u.cha.do}{0}
\verb{agauchar"-se}{}{}{}{}{v.pron.}{Adquirir modos ou hábitos de gaúcho.}{a.ga.u.char"-se}{\verboinum{1}}
\verb{agave}{}{Bot.}{}{}{s.m.}{Planta de cujas folhas se extraem fibras próprias para a fabricação de tapetes, cordas etc.; sisal.}{a.ga.ve}{0}
\verb{agave}{}{}{}{}{}{A fibra extraída dessa planta.}{a.ga.ve}{0}
\verb{agência}{}{}{}{}{s.f.}{Função ou cargo de agente.}{a.gên.cia}{0}
\verb{agência}{}{}{}{}{}{Capacidade de agir; diligência; atividade.}{a.gên.cia}{0}
\verb{agência}{}{}{}{}{}{Empresa que se encarrega de trabalhos por conta de terceiros.}{a.gên.cia}{0}
\verb{agência}{}{}{}{}{}{Filial de banco, casa bancária ou comercial, e de repartição pública.}{a.gên.cia}{0}
\verb{agenciador}{ô}{}{}{}{adj.}{Que agencia, que trabalha com afinco para obter algo.}{a.gen.ci.a.dor}{0}
\verb{agenciador}{ô}{}{}{}{}{Ativo; trabalhador; diligente.}{a.gen.ci.a.dor}{0}
\verb{agenciador}{ô}{}{}{}{s.m.}{Indivíduo que agencia; que trata de negócios alheios.}{a.gen.ci.a.dor}{0}
\verb{agenciar}{}{}{}{}{v.t.}{Tratar de negócios como representante ou agente.}{a.gen.ci.ar}{0}
\verb{agenciar}{}{}{}{}{}{Trabalhar com afinco para obter algo; diligenciar.}{a.gen.ci.ar}{0}
\verb{agenciar}{}{}{}{}{}{Solicitar; requerer; promover.}{a.gen.ci.ar}{\verboinum{1}}
%%\verb{agenciar}{}{}{}{}{}{0}{a.gen.ci.ar}{0}
\verb{agenda}{}{}{}{}{s.f.}{Caderneta, caderno ou livro no qual se anotam, dia a dia, os compromissos, os lembretes etc. }{a.gen.da}{0}
\verb{agenda}{}{Por ext.}{}{}{}{Compromissos que devem ser cumpridos em prazo determinado; programação.}{a.gen.da}{0}
\verb{agendado}{}{}{}{}{adj.}{Marcado em agenda; combinado.}{a.gen.da.do}{0}
\verb{agendar}{}{}{}{}{v.t.}{Marcar, incluir em agenda.}{a.gen.dar}{\verboinum{1}}
\verb{agente}{}{}{}{}{adj.2g.}{Que opera, agencia.}{a.gen.te}{0}
\verb{agente}{}{}{}{}{}{Que é agente.}{a.gen.te}{0}
\verb{agente}{}{}{}{}{s.2g.}{Pessoa que trata de negócios por conta alheia.}{a.gen.te}{0}
\verb{agente}{}{}{}{}{}{Membro de corporação policial ou de informações.}{a.gen.te}{0}
\verb{agente}{}{}{}{}{s.m.}{Aquilo que produz ou é capaz de produzir determinado efeito.}{a.gen.te}{0}
\verb{agente}{}{}{}{}{}{Causa, motivo.}{a.gen.te}{0}
\verb{agente}{}{Gram.}{}{}{}{Termo da oração que pratica a ação expressa pelo verbo.}{a.gen.te}{0}
\verb{agigantado}{}{}{}{}{adj.}{Que tem dimensões de gigante; enorme.}{a.gi.gan.ta.do}{0}
\verb{agigantado}{}{}{}{}{}{Que não tem medida; desmesurado.}{a.gi.gan.ta.do}{0}
\verb{agigantado}{}{}{}{}{}{Que é dotado de grande força; muito forte.}{a.gi.gan.ta.do}{0}
\verb{agigantamento}{}{}{}{}{s.m.}{Ato ou efeito de agigantar, de tornar gigantesco, aumentar muito.}{a.gi.gan.ta.men.to}{0}
\verb{agigantar}{}{}{}{}{v.t.}{Tornar gigantesco, muito maior.}{a.gi.gan.tar}{0}
\verb{agigantar}{}{}{}{}{v.pron.}{Ter grande destaque; sobressair; distinguir.}{a.gi.gan.tar}{\verboinum{1}}
\verb{ágil}{}{}{"-eis}{}{adj.2g.}{Que se move com facilidade; hábil; ligeiro.}{á.gil}{0}
\verb{agilidade}{}{}{}{}{s.f.}{Qualidade ou caráter de ágil; habilidade, ligeireza, desembaraço. }{a.gi.li.da.de}{0}
\verb{agilização}{}{}{"-ões}{}{s.f.}{Ato ou efeito de agilizar, de imprimir rapidez.}{a.gi.li.za.ção}{0}
\verb{agilizar}{}{}{}{}{v.t.}{Fazer de maneira ágil; imprimir maior agilidade, rapidez, eficiência.}{a.gi.li.zar}{\verboinum{1}}
\verb{ágio}{}{Econ.}{}{}{s.m.}{Diferença entre o valor nominal de um produto ou mercadoria e o preço cobrado.}{á.gio}{0}
\verb{ágio}{}{}{}{}{}{Lucro sobre a diferença do valor real da moeda, nas taxas de câmbio.}{á.gio}{0}
\verb{ágio}{}{}{}{}{}{Juro, superior à taxa legal, de dinheiro emprestado.}{á.gio}{0}
\verb{agiota}{ó}{}{}{}{adj.2g.}{Que se entrega à agiotagem; usurário.}{a.gi.o.ta}{0}
\verb{agiota}{ó}{}{}{}{s.2g.}{Indivíduo que faz agiotagem.}{a.gi.o.ta}{0}
\verb{agiotagem}{}{}{"-ens}{}{s.f.}{Especulação sobre fundos, mercadorias ou câmbios para obter lucros exagerados.}{a.gi.o.ta.gem}{0}
\verb{agiotagem}{}{}{"-ens}{}{}{Lucro advindo dessa especulação.}{a.gi.o.ta.gem}{0}
\verb{agiotagem}{}{}{"-ens}{}{}{Empréstimo de dinheiro a juros altos.}{a.gi.o.ta.gem}{0}
\verb{agiotar}{}{}{}{}{v.i.}{Entregar"-se à agiotagem; especular.}{a.gi.o.tar}{0}
\verb{agiotar}{}{}{}{}{}{Rebater ou descontar título de crédito.}{a.gi.o.tar}{\verboinum{1}}
\verb{agir}{}{}{}{}{v.i.}{Praticar ou efetuar na qualidade de agente; pôr em ação; realizar; atuar; operar.}{a.gir}{\verboinum{22}}
\verb{agitação}{}{}{"-ões}{}{s.f.}{Ato ou efeito de agitar, de mover com frequência.}{a.gi.ta.ção}{0}
\verb{agitação}{}{}{"-ões}{}{}{Movimento irregular e repetido; oscilação; abalo.}{a.gi.ta.ção}{0}
\verb{agitação}{}{}{"-ões}{}{}{Perturbação moral e psíquica; excitação.}{a.gi.ta.ção}{0}
\verb{agitação}{}{}{"-ões}{}{}{Comoção política; subversão; desordem.}{a.gi.ta.ção}{0}
\verb{agitação}{}{}{"-ões}{}{}{Alvoroço; barulho; tumulto. }{a.gi.ta.ção}{0}
\verb{agitadiço}{}{}{}{}{adj.}{Que se agita com facilidade ou frequência.}{a.gi.ta.di.ço}{0}
\verb{agitado}{}{}{}{}{adj.}{Que se movimenta muito; inquieto.}{a.gi.ta.do}{0}
\verb{agitado}{}{}{}{}{s.m.}{Indivíduo agitado, perturbado.}{a.gi.ta.do}{0}
\verb{agitador}{ô}{}{}{}{adj.}{Que agita, que movimenta.}{a.gi.ta.dor}{0}
\verb{agitador}{ô}{}{}{}{s.m.}{Indivíduo que promove agitação política, social; que divulga, em discursos, panfletos, as ideias de um grupo político; revolucionário.	 }{a.gi.ta.dor}{0}
\verb{agitador}{ô}{}{}{}{}{Dispositivo que, nas manteigueiras, serve para agitar o leite e separar dele a nata ou a manteiga.}{a.gi.ta.dor}{0}
\verb{agitar}{}{}{}{}{v.t.}{Fazer mover com frequência; abalar.}{a.gi.tar}{0}
\verb{agitar}{}{}{}{}{}{Comover fortemente; excitar.}{a.gi.tar}{0}
\verb{agitar}{}{}{}{}{}{Incitar à revolta; sublevar.}{a.gi.tar}{0}
\verb{agitar}{}{}{}{}{v.pron.}{Mover"-se; mexer"-se.}{a.gi.tar}{0}
\verb{agitar}{}{}{}{}{}{Perturbar"-se; inquietar"-se.}{a.gi.tar}{\verboinum{1}}
\verb{agito}{}{Pop.}{}{}{s.m.}{Estado de agitação; excitação; muvuca.}{a.gi.to}{0}
\verb{aglomeração}{}{}{"-ões}{}{s.f.}{Ato ou efeito de aglomerar, reunir; ajuntamento, amontoamento.}{a.glo.me.ra.ção}{0}
\verb{aglomeração}{}{}{"-ões}{}{}{Grande agrupamento de pessoas ou coisas; multidão.}{a.glo.me.ra.ção}{0}
\verb{aglomerado}{}{}{}{}{adj.}{Que se aglomerou, reuniu; amontoado, acumulado.}{a.glo.me.ra.do}{0}
\verb{aglomerado}{}{}{}{}{s.m.}{Diz"-se do material formado por partículas de uma substância, ligadas por prensagem.}{a.glo.me.ra.do}{0}
\verb{aglomerante}{}{}{}{}{adj.2g.}{Que aglomera, reúne.}{a.glo.me.ran.te}{0}
\verb{aglomerante}{}{}{}{}{}{Diz"-se do produto, como o cimento, que é usado para ligar outros materiais, como a areia, o cascalho etc.}{a.glo.me.ran.te}{0}
\verb{aglomerar}{}{}{}{}{v.t.}{Pôr junto; reunir, agrupar, amontoar.}{a.glo.me.rar}{0}
\verb{aglomerar}{}{Ant.}{}{}{}{desagromerar}{a.glo.me.rar}{\verboinum{1}}
\verb{aglutinação}{}{}{"-ões}{}{s.f.}{Ato ou efeito de aglutinar, fundir; união; junção.}{a.glu.ti.na.ção}{0}
\verb{aglutinação}{}{Med.}{"-ões}{}{}{União de tecidos da pele no processo de cicatrização.}{a.glu.ti.na.ção}{0}
\verb{aglutinante}{}{}{}{}{adj.2g.}{Que aglutina, cola, reúne.}{a.glu.ti.nan.te}{0}
\verb{aglutinante}{}{}{}{}{s.m.}{Substância que provoca aglutinação; cola, grude, adesivo.}{a.glu.ti.nan.te}{0}
\verb{aglutinar}{}{}{}{}{v.t.}{Unir com material aderente; colar, grudar.}{a.glu.ti.nar}{0}
\verb{aglutinar}{}{Med.}{}{}{}{Unir as bordas de uma ferida no processo de cicatrização.}{a.glu.ti.nar}{\verboinum{1}}
\verb{agnosticismo}{}{Filos.}{}{}{s.m.}{Doutrina que aceita apenas verdades comprovadas pela ciência, por considerar que questões como deus, vida após a morte e a finalidade última da vida estão além das capacidades de compreensão do intelecto humano.}{ag.nos.ti.cis.mo}{0}
\verb{agnóstico}{}{}{}{}{adj.}{Relativo ao agnosticismo.}{ag.nós.ti.co}{0}
\verb{agnóstico}{}{}{}{}{s.m.}{Indivíduo que segue os princípios do agnosticismo.}{ag.nós.ti.co}{0}
\verb{agogô}{}{Mús.}{}{}{s.m.}{Instrumento de percussão de origem africana, constituído por duas campainhas de metal em forma de \textsc{u} tocadas com varetas também de metal.}{a.go.gô}{0}
\verb{agonia}{}{}{}{}{s.f.}{Momento de padecimento que antecede a morte.}{a.go.ni.a}{0}
\verb{agonia}{}{}{}{}{}{Estado de diminuição das forças vitais que caracteriza esse momento.}{a.go.ni.a}{0}
\verb{agonia}{}{}{}{}{}{Angústia, aflição, ansiedade.}{a.go.ni.a}{0}
\verb{agoniado}{}{}{}{}{adj.}{Que sente agonia, aflição.}{a.go.ni.a.do}{0}
\verb{agoniado}{}{}{}{}{}{Inquieto, aflito, angustiado.}{a.go.ni.a.do}{0}
\verb{agoniar}{}{}{}{}{v.t.}{Causar agonia; afligir, angustiar.}{a.go.ni.ar}{0}
\verb{agoniar}{}{}{}{}{}{Provocar irritação; importunar.}{a.go.ni.ar}{\verboinum{1}}
\verb{agônico}{}{}{}{}{adj.}{Relativo à agonia, aflição.}{a.gô.ni.co}{0}
\verb{agônico}{}{}{}{}{}{Diz"-se do momento que antecede a morte.}{a.gô.ni.co}{0}
\verb{agonizante}{}{}{}{}{adj.2g.}{Que está em agonia, em aflição.}{a.go.ni.zan.te}{0}
\verb{agonizante}{}{}{}{}{s.m.}{Indivíduo que padece no momento próximo à morte; moribundo.}{a.go.ni.zan.te}{0}
\verb{agonizar}{}{}{}{}{v.t.}{Causar agonia; aflição.}{a.go.ni.zar}{0}
\verb{agonizar}{}{}{}{}{v.i.}{Estar em agonia; padecer no momento que antecede a morte.}{a.go.ni.zar}{0}
\verb{agonizar}{}{Fig.}{}{}{}{Estar próximo do fim; declinar.}{a.go.ni.zar}{\verboinum{1}}
\verb{agora}{ó}{}{}{}{adv.}{Neste momento, nesta hora; atualmente, presentemente.}{a.go.ra}{0}
\verb{agora}{ó}{Pop.}{}{}{conj.}{Mas, porém.}{a.go.ra}{0}
\verb{ágora}{}{}{}{}{s.f.}{Praça pública das antigas cidades gregas, onde se instalava o mercado e se realizavam as assembleias políticas.}{á.go.ra}{0}
\verb{agorafobia}{}{Med.}{}{}{s.f.}{Medo doentio de estar em lugares públicos e abertos como praças, ruas etc.}{a.go.ra.fo.bi.a}{0}
\verb{agorinha}{}{Pop.}{}{}{adv.}{Há pouco tempo; agora mesmo.}{a.go.ri.nha}{0}
\verb{agostiniano}{}{}{}{}{adj.}{Relativo a Santo Agostinho ou pertencente à Ordem religiosa fundada por ele.}{a.gos.ti.ni.a.no}{0}
\verb{agostiniano}{}{}{}{}{s.m.}{Frade pertencente a essa Ordem.}{a.gos.ti.ni.a.no}{0}
\verb{agosto}{ô}{}{}{}{s.m.}{O oitavo mês do ano civil.}{a.gos.to}{0}
\verb{agourar}{}{}{}{}{v.t.}{Fazer agouro, presságio.}{a.gou.rar}{0}
\verb{agourar}{}{}{}{}{}{Profetizar, predizer, adivinhar.}{a.gou.rar}{0}
\verb{agourar}{}{}{}{}{}{Desejar mau agouro, má sorte.}{a.gou.rar}{\verboinum{1}}
\verb{agoureiro}{ê}{}{}{}{adj. e s.m.  }{Agourento.}{a.gou.rei.ro}{0}
\verb{agourento}{}{}{}{}{adj.}{Que agoura, prediz.}{a.gou.ren.to}{0}
\verb{agourento}{}{}{}{}{s.m.}{Indivíduo que faz mau agouro; que anuncia desgraças.}{a.gou.ren.to}{0}
\verb{agouro}{ô}{}{}{}{s.m.}{Profecia, presságio.}{a.gou.ro}{0}
\verb{agouro}{ô}{}{}{}{}{Predição de algo ruim, desagradável; mau agouro.}{a.gou.ro}{0}
\verb{agraciamento}{}{}{}{}{s.m.}{Ato ou efeito de agraciar; condecoração.}{a.gra.ci.a.men.to}{0}
\verb{agraciar}{}{}{}{}{v.t.}{Conceder graça ou mercê.}{a.gra.ci.ar}{0}
\verb{agraciar}{}{}{}{}{}{Honrar com título honorífico; condecorar.}{a.gra.ci.ar}{\verboinum{1}}
\verb{agradar}{}{}{}{}{v.t.}{Satisfazer o gosto; causar prazer; contentar.}{a.gra.dar}{0}
\verb{agradar}{}{}{}{}{}{Fazer carinho; acariciar, afagar.}{a.gra.dar}{0}
\verb{agradar}{}{}{}{}{v.i.}{Parecer bem; causar boa impressão; encantar.}{a.gra.dar}{\verboinum{1}}
\verb{agradável}{}{}{"-eis}{}{adj.2g.}{Que agrada, que satisfaz; ameno, prazeroso.}{a.gra.dá.vel}{0}
\verb{agradecer}{ê}{}{}{}{v.t.}{Mostrar"-se grato; reconhecer.}{a.gra.de.cer}{\verboinum{15}}
\verb{agradecido}{}{}{}{}{adj.}{Que agradeceu; reconhecido, grato por algum favor.}{a.gra.de.ci.do}{0}
\verb{agradecimento}{}{}{}{}{s.m.}{Ato ou efeito de agradecer; reconhecimento, gratidão.}{a.gra.de.ci.men.to}{0}
\verb{agrado}{}{}{}{}{s.m.}{Ato ou efeito de agradar.}{a.gra.do}{0}
\verb{agrado}{}{}{}{}{}{Contentamento, satisfação.}{a.gra.do}{0}
\verb{agrado}{}{}{}{}{}{Demonstração de carinho, afago.}{a.gra.do}{0}
\verb{agrado}{}{}{}{}{}{Gratificação, gorjeta, presente.}{a.gra.do}{0}
\verb{ágrafo}{}{}{}{}{adj.}{Diz"-se da língua, do povo ou da cultura que não apresenta escrita, grafia.}{á.gra.fo}{0}
\verb{agrário}{}{}{}{}{adj.}{Relativo ao campo ou à terra.}{a.grá.rio}{0}
\verb{agrário}{}{}{}{}{}{Relativo à agricultura; agrícola.}{a.grá.rio}{0}
\verb{agravamento}{}{}{}{}{s.m.}{Ato ou efeito de agravar; piora, agravo.}{a.gra.va.men.to}{0}
\verb{agravamento}{}{Med.}{}{}{}{Aumento da gravidade de uma doença ou de um sintoma da doença.}{a.gra.va.men.to}{0}
\verb{agravante}{}{}{}{}{adj.2g.}{Que agrava; que aumenta a gravidade.}{a.gra.van.te}{0}
\verb{agravante}{}{Jur.}{}{}{}{Diz"-se da circunstância que revela maior gravidade, aumentando a pena a ser aplicada.}{a.gra.van.te}{0}
\verb{agravante}{}{}{}{}{s.m.}{Aquilo que agrava.}{a.gra.van.te}{0}
\verb{agravante}{}{Jur.}{}{}{s.2g.}{Pessoa que interpõe o agravo. }{a.gra.van.te}{0}
\verb{agravar}{}{}{}{}{v.t.}{Tornar grave ou mais grave; piorar.}{a.gra.var}{0}
\verb{agravar}{}{}{}{}{}{Ofender, magoar, injuriar.}{a.gra.var}{0}
\verb{agravar}{}{Jur.}{}{}{v.i.}{Interpor recurso de agravo.}{a.gra.var}{\verboinum{1}}
\verb{agravo}{}{}{}{}{s.m.}{Ofensa, afronta, injúria.}{a.gra.vo}{0}
\verb{agravo}{}{Jur.}{}{}{}{Denominação genérica a vários recursos cabíveis contra decisões anteriores.}{a.gra.vo}{0}
\verb{agredir}{}{}{}{}{v.t.}{Praticar agressão contra; atacar, assaltar.}{a.gre.dir}{0}
\verb{agredir}{}{}{}{}{}{Ofender, insultar, injuriar.}{a.gre.dir}{\verboinum{30}}
\verb{agregação}{}{}{"-ões}{}{s.f.}{Ato ou efeito de agregar, reunir.}{a.gre.ga.ção}{0}
\verb{agregação}{}{}{"-ões}{}{}{Reunião, aglomeração, associação.}{a.gre.ga.ção}{0}
\verb{agregado}{}{}{}{}{adj.}{Que está anexo, unido.}{a.gre.ga.do}{0}
\verb{agregado}{}{}{}{}{s.m.}{Reunião, conjunto.}{a.gre.ga.do}{0}
\verb{agregado}{}{}{}{}{}{Indivíduo que vive numa casa, como membro da família.}{a.gre.ga.do}{0}
\verb{agregado}{}{}{}{}{}{Trabalhador do campo que cultiva terra alheia.}{a.gre.ga.do}{0}
\verb{agregar}{}{}{}{}{v.t.}{Reunir, juntar.}{a.gre.gar}{0}
\verb{agregar}{}{}{}{}{}{Anexar, associar.}{a.gre.gar}{\verboinum{5}}
\verb{agremiação}{}{}{"-ões}{}{s.f.}{Ato ou efeito de agremiar, reunir, associar.}{a.gre.mi.a.ção}{0}
\verb{agremiação}{}{}{"-ões}{}{}{Agrupamento, associação, sociedade.}{a.gre.mi.a.ção}{0}
\verb{agremiar}{}{}{}{}{v.t.}{Reunir em grêmio, associação ou assembleia. }{a.gre.mi.ar}{0}
\verb{agremiar}{}{}{}{}{}{Tornar associado; agregar, ligar.}{a.gre.mi.ar}{\verboinum{1}}
\verb{agressão}{}{}{"-ões}{}{s.f.}{Ato ou efeito de agredir; atitude hostil.}{a.gres.são}{0}
\verb{agressão}{}{}{"-ões}{}{}{Ataque, investida.}{a.gres.são}{0}
\verb{agressão}{}{}{"-ões}{}{}{Insulto, ofensa.}{a.gres.são}{0}
\verb{agressividade}{}{}{}{}{s.f.}{Qualidade de quem é agressivo, hostil.}{a.gres.si.vi.da.de}{0}
\verb{agressividade}{}{}{}{}{}{Tendência para agredir ou provocar.}{a.gres.si.vi.da.de}{0}
\verb{agressividade}{}{}{}{}{}{Combatividade, dinamismo, energia.}{a.gres.si.vi.da.de}{0}
\verb{agressivo}{}{}{}{}{adj.}{Que envolve ou revela agressão.}{a.gres.si.vo}{0}
\verb{agressivo}{}{}{}{}{}{Que está voltado para o ataque; lutador.}{a.gres.si.vo}{0}
\verb{agressor}{ô}{}{}{}{adj.}{Que agride, ataca, provoca.}{a.gres.sor}{0}
\verb{agressor}{ô}{}{}{}{s.m.}{Indivíduo que agride, ataca, provoca. }{a.gres.sor}{0}
\verb{agreste}{é}{}{}{}{adj.2g.}{Relativo ao campo, ao agro; silvestre, selvagem.}{a.gres.te}{0}
\verb{agreste}{é}{}{}{}{}{Não cultivado, rústico, tosco.}{a.gres.te}{0}
\verb{agreste}{é}{}{}{}{}{Rude, indelicado, áspero.}{a.gres.te}{0}
\verb{agreste}{é}{Geogr.}{}{}{s.m.}{Zona do Nordeste brasileiro, situada entre a mata e a caatinga, de solo pedregoso e vegetação escassa.}{a.gres.te}{0}
\verb{agrião}{}{Bot.}{"-ões}{}{s.m.}{Erva comestível, rica em sais minerais e vitaminas, cujas folhas e talo, de sabor amargo e picante, são servidos como salada.}{a.gri.ão}{0}
\verb{agrião}{}{Zool.}{"-ões}{}{}{Tumor duro e indolor que se forma no tendão da perna de bois e cavalos.}{a.gri.ão}{0}
\verb{agrícola}{}{}{}{}{adj.2g.}{Relativo ao campo, à agricultura.}{a.grí.co.la}{0}
\verb{agrícola}{}{}{}{}{}{Que se dedica à agricultura ou que é baseado nela.}{a.grí.co.la}{0}
\verb{agricultor}{ô}{}{}{}{s.m.}{Indivíduo que cultiva ou lavra a terra; lavrador.}{a.gri.cul.tor}{0}
\verb{agricultura}{}{}{}{}{s.f.}{Arte de cultivar a terra; lavoura, cultura.}{a.gri.cul.tu.ra}{0}
\verb{agricultura}{}{}{}{}{}{Conjunto de técnicas de cultivo da terra que visam à produção de vegetais para consumo humano.}{a.gri.cul.tu.ra}{0}
\verb{agridoce}{ô}{}{}{}{adj.2g.}{Que tem sabor doce e azedo ao mesmo tempo.}{a.gri.do.ce}{0}
\verb{agrilhoar}{}{}{}{}{v.t.}{Prender com grilhões ou correntes de ferro; acorrentar, amarrar.}{a.gri.lho.ar}{0}
\verb{agrilhoar}{}{}{}{}{}{Impor constrangimento; reprimir, refrear.}{a.gri.lho.ar}{\verboinum{7}}
\verb{agrimensor}{ô}{}{}{}{s.m.}{Profissional habilitado para medir e demarcar terras.}{a.gri.men.sor}{0}
\verb{agrimensura}{}{}{}{}{s.f.}{Técnica da medição de terras, campos etc.}{a.gri.men.su.ra}{0}
\verb{agrisalhado}{}{}{}{}{adj.}{Que tem cabelos grisalhos, mesclados de fios brancos.}{a.gri.sa.lha.do}{0}
\verb{agrisalhar}{}{}{}{}{v.t.}{Tornar grisalho; proporcionar ou adquirir cabelos mesclados de fios brancos.}{a.gri.sa.lhar}{\verboinum{1}}
\verb{agro}{}{Desus.}{}{}{adj.}{Que é acre; azedo; ácido}{a.gro}{0}
\verb{agro}{}{Fig.}{}{}{}{Dificultoso; árduo.}{a.gro}{0}
\verb{agro}{}{Desus.}{}{}{s.m.}{Terra cultivada ou cultivável; campo.}{a.gro}{0}
\verb{agroecologia}{}{Ecol.}{}{}{s.f.}{Ramo da ecologia que estuda as relações entre a agricultura e o meio ambiente. }{a.gro.e.co.lo.gi.a}{0}
\verb{agroindústria}{}{}{}{}{s.f.}{Indústria que se baseia na agricultura ou no beneficiamento do produto agrícola.}{a.gro.in.dús.tria}{0}
\verb{agrologia}{}{}{}{}{s.f.}{Parte da agronomia que estuda os solos nas relações com a agricultura.}{a.gro.lo.gi.a}{0}
\verb{agronomia}{}{}{}{}{s.f.}{Conjunto das ciências, técnicas e dos princípios que regem a prática da agricultura.}{a.gro.no.mi.a}{0}
\verb{agronômico}{}{}{}{}{adj.}{Que se refere ou diz respeito à agronomia.}{a.gro.nô.mi.co}{0}
\verb{agrônomo}{}{}{}{}{s.m.}{Indivíduo que é diplomado ou especialista em agronomia.}{a.grô.no.mo}{0}
\verb{agropecuária}{}{}{}{}{s.f.}{Atividade econômica que envolve agricultura e pecuária.}{a.gro.pe.cu.á.ria}{0}
\verb{agropecuário}{}{}{}{}{adj.}{Que diz respeito à agropecuária.}{a.gro.pe.cu.á.rio}{0}
\verb{agrotóxico}{cs}{}{}{}{s.m.}{Qualquer composto químico, como pesticida, herbicida, hormônios vegetais, utilizado nas lavouras para aumentar a produtividade e melhorar a qualidade.}{a.gro.tó.xi.co}{0}
\verb{agrovia}{}{}{}{}{s.f.}{Via de ligação terrestre, marítima ou fluvial, entre centros agrícolas de produção e armazenagem e centros de consumo.  }{a.gro.vi.a}{0}
\verb{agrupamento}{}{}{}{}{s.m.}{Ato ou efeito de agrupar, de juntar ou reunir em grupo; ajuntamento; aglomeração.}{a.gru.pa.men.to}{0}
\verb{agrupar}{}{}{}{}{v.t.}{Reunir em grupo; juntar.}{a.gru.par}{0}
\verb{agrupar}{}{}{}{}{}{Dispor em grupos; organizar; ordenar.}{a.gru.par}{\verboinum{1}}
\verb{agrura}{}{}{}{}{s.f.}{Dificuldade; obstáculo.}{a.gru.ra}{0}
\verb{agrura}{}{}{}{}{}{Aspereza; escabrosidade.}{a.gru.ra}{0}
\verb{agrura}{}{Desus.}{}{}{}{Qualidade de agro; acidez.}{a.gru.ra}{0}
\verb{água}{}{}{}{}{s.f.}{Líquido incolor, inodoro, insípido, essencial à vida.}{á.gua}{0}
\verb{água}{}{}{}{}{}{A parte líquida da superfície terrestre.}{á.gua}{0}
\verb{água}{}{}{}{}{}{Cada uma das superfícies planas que formam um telhado.}{á.gua}{0}
\verb{aguaceiro}{ê}{}{}{}{s.m.}{Chuva forte, repentina e de pouca duração.}{a.gua.cei.ro}{0}
\verb{aguaceiro}{ê}{Fig.}{}{}{}{Contratempo, infelicidade inesperada.}{a.gua.cei.ro}{0}
\verb{aguacento}{}{}{}{}{adj.}{Que é semelhante à água.}{a.gua.cen.to}{0}
\verb{aguacento}{}{}{}{}{}{Encharcado de água.}{a.gua.cen.to}{0}
\verb{aguacento}{}{}{}{}{}{Diluído em água; aquoso.}{a.gua.cen.to}{0}
\verb{água"-com"-açúcar}{}{}{}{}{adj.2g.}{Diz"-se do romântico ingênuo; simples; piegas.}{á.gua"-com"-a.çú.car}{0}
\verb{aguada}{}{}{}{}{s.f.}{Abastecimento de água potável, especialmente para viagens marítimas.}{a.gua.da}{0}
\verb{aguada}{}{}{}{}{}{Lugar onde se faz tal abastecimento.}{a.gua.da}{0}
\verb{aguada}{}{}{}{}{}{Bebedouro natural.}{a.gua.da}{0}
\verb{água"-de"-cheiro}{ê}{Pop.}{águas"-de"-cheiro}{}{s.f.}{Perfume; água"-de"-colônia.}{á.gua"-de"-chei.ro}{0}
\verb{água"-de"-coco}{ô}{Bras.}{águas"-de"-coco}{}{s.f.}{Albume líquido do coco"-da"-baía, usado por suas propriedades medicinais e nutritivas, e como refresco.}{á.gua"-de"-co.co}{0}
\verb{água"-de"-colônia}{}{}{águas"-de"-colônia}{}{s.f.}{Solução composta de álcool e óleos aromáticos, usada como perfume e em medicina.}{á.gua"-de"-co.lô.nia}{0}
\verb{aguadeiro}{ê}{}{}{}{s.m.}{Pessoa que vende, fornece ou transporta água.}{a.gua.dei.ro}{0}
\verb{aguado}{}{}{}{}{adj.}{Que está cheio de água; aguacento.}{a.gua.do}{0}
\verb{aguado}{}{}{}{}{}{Imperfeito de gosto; insosso.}{a.gua.do}{0}
\verb{aguado}{}{}{}{}{}{Diz"-se do animal que sofre de aguamento.}{a.gua.do}{0}
\verb{aguado}{}{Pop.}{}{}{}{Que está desejando muito algo que não pode obter.}{a.gua.do}{0}
\verb{água"-forte}{ó}{}{águas"-fortes ⟨ó⟩}{}{s.f.}{Designação primitiva, mas ainda usada, do ácido nítrico dissolvido em água.}{á.gua"-for.te}{0}
\verb{água"-forte}{ó}{}{águas"-fortes ⟨ó⟩}{}{}{Técnica de gravura utilizada pela ação corrosiva do ácido nítrico. }{á.gua"-for.te}{0}
\verb{água"-forte}{ó}{}{águas"-fortes ⟨ó⟩}{}{}{A gravura que se obtém dessa técnica.}{á.gua"-for.te}{0}
\verb{água"-furtada}{}{}{águas"-furtadas}{}{s.f.}{Janelas que se abrem sobre o telhado, modificando o curso das águas; desvão.}{á.gua"-fur.ta.da}{0}
\verb{água"-marinha}{}{}{águas"-marinhas}{}{s.f.}{Pedra semipreciosa, transparente e brilhante, de cor azulada, que lembra a água do mar.}{á.gua"-ma.ri.nha}{0}
\verb{aguamento}{}{}{}{}{s.m.}{Ato ou efeito de aguar, de molhar.}{a.gua.men.to}{0}
\verb{aguamento}{}{Veter.}{}{}{}{Inflamação da membrana tegumentar da pata dos animais de carga ou de tração, causada por excesso de trabalho ou por resfriamento. }{a.gua.men.to}{0}
\verb{água"-morna}{ó/ ou /ô}{Pop.}{águas"-mornas ⟨ó/ ou /ô⟩}{}{s.2g.}{Indivíduo que é pacato; inofensivo; mole; indeciso.}{á.gua"-mor.na}{0}
\verb{aguapé}{}{}{}{}{s.f.}{Bebida de baixo teor alcoólico; vinho fraco.}{a.gua.pé}{0}
\verb{aguapé}{}{Bot.}{}{}{}{Designação de plantas aquáticas flutuantes que crescem na superfície de rios, lagos e pantanais.}{a.gua.pé}{0}
\verb{aguapezal}{}{}{"-ais}{}{s.m.}{Local em que há grande extensão de água coberta de aguapés.}{a.gua.pe.zal}{0}
\verb{aguar}{}{}{}{}{v.t.}{Molhar com água ou outro líquido.}{a.guar}{0}
\verb{aguar}{}{}{}{}{}{Misturar com água.}{a.guar}{0}
\verb{aguar}{}{}{}{}{}{Adulterar um líquido pela adição de água.}{a.guar}{0}
\verb{aguar}{}{}{}{}{}{Frustrar o prazer; atrapalhar.}{a.guar}{0}
\verb{aguar}{}{Pop.}{}{}{v.i.}{Ficar desejoso de algo que não se pode obter.}{a.guar}{0}
\verb{aguar}{}{}{}{}{}{Sofrer de aguamento (a cavalgadura).}{a.guar}{\verboinum{9}\verboirregular[aguo]{águo}}
\verb{aguardar}{}{}{}{}{v.t.}{Ficar à espera; permanecer na expectativa; esperar.}{a.guar.dar}{0}
\verb{aguardar}{}{}{}{}{}{Velar, vigiar, guardar.}{a.guar.dar}{\verboinum{1}}
\verb{aguardente}{}{}{}{}{s.f.}{Bebida de alto teor alcoólico, obtida pela destilação de várias frutas, plantas, raízes, cereais e sobretudo da cana"-de"-açúcar; cachaça. }{a.guar.den.te}{0}
\verb{aguardenteiro}{ê}{}{}{}{s.m.}{Indivíduo que fabrica ou vende aguardente.}{a.guar.den.tei.ro}{0}
\verb{aguardenteiro}{ê}{}{}{}{}{Pessoa que bebe com frequência; ébrio, bêbado.}{a.guar.den.tei.ro}{0}
\verb{água"-régia}{}{Quím.}{águas"-régias}{}{s.f.}{Reagente corrosivo, proveniente da mistura de uma parte de ácido nítrico concentrado e duas ou três partes de ácido clorídrico. }{á.gua"-ré.gia}{0}
\verb{aguarela}{é}{}{}{}{}{Var. de \textit{aquarela}.}{a.gua.re.la}{0}
\verb{aguarrás}{}{}{}{}{s.f.}{Essência de terebintina usada como solvente em pintura e esmaltação.}{a.guar.rás}{0}
\verb{águas}{}{}{}{}{s.f.pl.}{Grandes extensões de água (mares, rios, lagos).}{á.guas}{0}
\verb{águas}{}{}{}{}{}{Chuvas.}{á.guas}{0}
\verb{águas}{}{}{}{}{}{Águas minerais e medicinais de uma região.}{á.guas}{0}
\verb{água"-viva}{}{Zool.}{águas"-vivas}{}{s.f.}{Nome comum às espécies de medusa, animal marinho de corpo mole, gelatinoso e transparente, com tentáculos providos de células urticantes, que causam queimadura na pele humana.}{á.gua"-vi.va}{0}
\verb{aguçado}{}{}{}{}{adj.}{Que termina em ponta ou bico; fino; agudo.}{a.gu.ça.do}{0}
\verb{aguçado}{}{}{}{}{}{Que é afiado; cortante.}{a.gu.ça.do}{0}
\verb{aguçado}{}{Fig.}{}{}{}{Que fere; incomoda.}{a.gu.ça.do}{0}
\verb{aguçado}{}{}{}{}{s.m.}{Pessoa perspicaz, sagaz.}{a.gu.ça.do}{0}
\verb{aguçar}{}{}{}{}{v.t.}{Tornar agudo; adelgaçar na ponta.}{a.gu.çar}{0}
\verb{aguçar}{}{}{}{}{}{Dar fio; amolar.}{a.gu.çar}{0}
\verb{aguçar}{}{Fig.}{}{}{}{Provocar o surgimento ou a intensificação; estimular, excitar.}{a.gu.çar}{0}
\verb{aguçar}{}{Fig.}{}{}{}{Tornar perspicaz; apurar.}{a.gu.çar}{0}
\verb{aguçar}{}{}{}{}{v.pron.}{Agir com diligência; apressar"-se, esforçar"-se.}{a.gu.çar}{\verboinum{3}}
\verb{agudez}{ê}{}{}{}{s.f.}{Agudeza.}{a.gu.dez}{0}
\verb{agudeza}{ê}{}{}{}{s.f.}{Qualidade do que é agudo ou cortante; agudez.}{a.gu.de.za}{0}
\verb{agudeza}{ê}{}{}{}{}{Qualidade do que é afilado, pontiagudo.}{a.gu.de.za}{0}
\verb{agudeza}{ê}{}{}{}{}{Intensidade ou estado agudo de uma doença.}{a.gu.de.za}{0}
\verb{agudeza}{ê}{Fig.}{}{}{}{Sutileza, perspicácia.}{a.gu.de.za}{0}
\verb{agudeza}{ê}{}{}{}{}{Falta de delicadeza; aspereza.}{a.gu.de.za}{0}
\verb{agudo}{}{}{}{}{adj.}{Que é terminado em ponta ou em gume.}{a.gu.do}{0}
\verb{agudo}{}{}{}{}{}{Que é intenso; violento.}{a.gu.do}{0}
\verb{agudo}{}{Fig.}{}{}{}{Que é arguto; perspicaz.}{a.gu.do}{0}
\verb{agudo}{}{Gram.}{}{}{}{Diz"-se de acento que indica vogal tônica.}{a.gu.do}{0}
\verb{agudo}{}{Mús.}{}{}{}{Diz"-se do som alto, ao contrário do som baixo ou grave.}{a.gu.do}{0}
\verb{aguentar}{}{}{}{}{v.t.}{Sustentar, manter, equilibrar.}{a.guen.tar}{0}
\verb{aguentar}{}{}{}{}{}{Ter resistência; suportar, sofrer.}{a.guen.tar}{0}
\verb{aguentar}{}{}{}{}{v.pron.}{Conseguir manter"-se; conservar"-se.}{a.guen.tar}{\verboinum{1}}
\verb{aguerrido}{}{}{}{}{adj.}{Que está preparado, treinado para a guerra.}{a.guer.ri.do}{0}
\verb{aguerrido}{}{Por ext.}{}{}{}{Que demonstra destemor; valente; corajoso.}{a.guer.ri.do}{0}
\verb{aguerrir}{}{}{}{}{v.t.}{Habituar ou acostumar a lutas, trabalhos, fadigas.}{a.guer.rir}{0}
\verb{aguerrir}{}{}{}{}{}{Tornar energético, valoroso.}{a.guer.rir}{0}
\verb{aguerrir}{}{}{}{}{v.pron.}{Exercitar"-se nas armas, na guerra.}{a.guer.rir}{\verboinum{18}\verboirregular{\emph{def.} aguerrimos, aguerris}}
\verb{águia}{}{Zool.}{}{}{s.f.}{Designação comum às grandes aves de rapina, predadoras, dotadas de bico e garras de considerável força.}{á.guia}{0}
\verb{águia}{}{}{}{}{}{Indivíduo notável, de grande talento e perspicácia.}{á.guia}{0}
\verb{aguilhada}{}{}{}{}{s.f.}{Vara comprida, com ferrão na ponta, usada para conduzir ou estimular os bois no trabalho.}{a.gui.lha.da}{0}
\verb{aguilhão}{}{}{"-ões}{}{s.m.}{A ponta de ferro da aguilhada.}{a.gui.lhão}{0}
\verb{aguilhão}{}{Por ext.}{"-ões}{}{}{Ponta acerada e perfurante.}{a.gui.lhão}{0}
\verb{aguilhão}{}{Fig.}{"-ões}{}{}{Fator estimulante; incentivo.}{a.gui.lhão}{0}
\verb{aguilhoada}{}{}{}{}{s.f.}{A picada com aguilhão.}{a.gui.lho.a.da}{0}
\verb{aguilhoada}{}{Fig.}{}{}{}{Dor forte e súbita; pontada.}{a.gui.lho.a.da}{0}
\verb{aguilhoada}{}{}{}{}{}{Ação ou ocorrência com efeito estimulante; instigação, provocação.}{a.gui.lho.a.da}{0}
\verb{aguilhoar}{}{}{}{}{v.t.}{Picar ou ferir com aguilhão.}{a.gui.lho.ar}{0}
\verb{aguilhoar}{}{}{}{}{}{Causar grande sofrimento físico ou moral a; pungir.}{a.gui.lho.ar}{0}
\verb{aguilhoar}{}{}{}{}{}{Estimular, incitar.}{a.gui.lho.ar}{\verboinum{7}}
\verb{agulha}{}{}{}{}{s.f.}{Peça cilíndrica, com diferentes espessuras em sua extensão, normalmente de aço temperado e cromado, aguçada numa das extremidades, e com um orifício na outra, por onde passa lã, linha, barbante etc., para coser, bordar ou tecer.}{a.gu.lha}{0}
\verb{agulha}{}{}{}{}{}{Varinha de aço, metal, marfim ou outro material, com gancho próprio, usada para fazer meia, renda ou malha.}{a.gu.lha}{0}
\verb{agulha}{}{}{}{}{}{Qualquer extremidade aguda.}{a.gu.lha}{0}
\verb{agulha}{}{}{}{}{}{O ponteiro do relógio ou da bússola.}{a.gu.lha}{0}
\verb{agulha}{}{}{}{}{}{Maquinismo nas vias férreas que faz os trens mudarem de trilhos.}{a.gu.lha}{0}
\verb{agulha}{}{}{}{}{}{Pino de aço de algumas armas de fogo.}{a.gu.lha}{0}
\verb{agulha}{}{Zool.}{}{}{}{Designação comum a algumas espécies de peixe.}{a.gu.lha}{0}
\verb{agulhada}{}{}{}{}{s.f.}{Ferimento ou picada com agulha.}{a.gu.lha.da}{0}
\verb{agulhada}{}{Por ext.}{}{}{}{Dor forte, semelhante à provocada por picada de agulha; pontada.}{a.gu.lha.da}{0}
\verb{agulheiro}{ê}{}{}{}{s.m.}{Pequeno estojo ou almofada usado para guardar agulhas.}{a.gu.lhei.ro}{0}
\verb{agulheiro}{ê}{}{}{}{}{Fabricante de agulhas.}{a.gu.lhei.ro}{0}
\verb{agulheiro}{ê}{}{}{}{}{Ferroviário que movimenta os trilhos.}{a.gu.lhei.ro}{0}
\verb{agulheta}{ê}{}{}{}{s.f.}{Agulha grossa e de fundo largo, para passar fitas ou cordões por orifícios ou bainhas. }{a.gu.lhe.ta}{0}
\verb{agulheta}{ê}{}{}{}{}{Remate metálico de cadarço.}{a.gu.lhe.ta}{0}
\verb{agulheta}{ê}{}{}{}{}{Peça de metal atarraxada à saída de mangueiras de grande pressão.}{a.gu.lhe.ta}{0}
\verb{ah}{}{}{}{}{interj.}{Expressão que pode denotar alegria, tristeza, admiração, espanto ou surpresa.}{ah}{0}
\verb{ai}{}{}{}{}{interj.}{Expressão que geralmente denota dor ou lamento, mas também pode mostrar alegria.}{ai}{0}
\verb{ai}{}{}{}{}{s.m.}{Lamento,  reclamação. (\textit{A nós nos basta um sim ou um} ai \textit{dito de coração.} Pe. Vieira.)}{ai}{0}
\verb{aí}{}{}{}{}{adv.}{No lugar em que está a pessoa a quem se fala; nesse lugar.}{a.í}{0}
\verb{aí}{}{}{}{}{}{Nesse momento; então.}{a.í}{0}
\verb{aí}{}{}{}{}{}{Nesse caso, nesse ponto.}{a.í}{0}
\verb{aí}{}{Pop.}{}{}{conj.}{Expressão utilizada para dar continuidade ou conclusão ao assunto.}{a.í}{0}
\verb{aí}{}{}{}{}{interj.}{Expressão que denota incentivo, aplauso ou ironia.}{a.í}{0}
\verb{aia}{}{Desus.}{}{}{s.f.}{Dama de companhia.}{ai.a}{0}
\verb{aia}{}{Desus.}{}{}{}{Pessoa encarregada da educação de crianças de família nobre ou rica.}{ai.a}{0}
\verb{aia}{}{Desus.}{}{}{}{Criada de dama nobre; camareira.}{ai.a}{0}
\verb{aiatolá}{}{}{}{}{s.m.}{Entre os muçulmanos xiitas, o máximo líder espiritual e religioso, interpretador da lei islâmica.}{ai.a.to.lá}{0}
\verb{aidético}{}{}{}{}{adj.}{Diz"-se daquele que é portador do vírus da \textsc{aids}.}{ai.dé.ti.co}{0}
\verb{aidético}{}{}{}{}{s.m.}{Indivíduo que contraiu o vírus da \textsc{aids}.}{ai.dé.ti.co}{0}
\verb{AIDS}{}{Med.}{}{}{s.f.}{Síndrome de deficiência imunológica adquirida, doença de origem viral, letal e contagiosa, transmitida por via sexual ou sanguínea, que se manifesta pela deficiência do sistema imunológico do organismo. Abrev. de \textit{Acquired Immunological Deficiency Syndrome}.}{AIDS}{0}
\verb{ainda}{}{}{}{}{adv.}{Até agora, até o momento.}{a.in.da}{0}
\verb{ainda}{}{}{}{}{}{Também.}{a.in.da}{0}
\verb{ainda}{}{}{}{}{}{Mais uma vez, de novo.}{a.in.da}{0}
\verb{aio}{}{Desus.}{}{}{s.m.}{Pessoa encarregada da educação doméstica de crianças de família nobre ou rica.}{ai.o}{0}
\verb{aio}{}{Desus.}{}{}{}{Criado particular; camareiro, escudeiro.}{ai.o}{0}
\verb{aipim}{}{Bot.}{"-ins}{}{s.m.}{Arbusto de folhas partidas, flores amarelas, frutos capsulares e raízes alimentícias, também chamado de macaxeira ou mandioca.}{ai.pim}{0}
\verb{aipim}{}{}{"-ins}{}{}{A raiz dessa planta.}{ai.pim}{0}
\verb{aipo}{}{Bot.}{}{}{s.m.}{Erva nativa da Europa, de folhas decompostas, com longos pecíolos carnosos, estriados e de cor variada, usada em saladas e sopas, ou como condimento; salsão.}{ai.po}{0}
\verb{airado}{}{}{}{}{adj.}{Relativo ao ar; aéreo.}{ai.ra.do}{0}
\verb{airado}{}{Fig.}{}{}{}{Tomado por desvario; alucinado; louco.}{ai.ra.do}{0}
\verb{airado}{}{Fig.}{}{}{}{Que não tem seriedade; irresponsável, vadio.}{ai.ra.do}{0}
\verb{airoso}{ô}{}{"-osos ⟨ó⟩}{"-osa ⟨ó⟩}{adj.}{Que tem boa aparência; elegante, gracioso.}{ai.ro.so}{0}
\verb{airoso}{ô}{}{"-osos ⟨ó⟩}{"-osa ⟨ó⟩}{}{Que é gentil; delicado.}{ai.ro.so}{0}
\verb{aistórico}{}{}{}{}{adj.}{Diz"-se do fato que, por sua natureza ou definição, não participa ou não pode participar da história temporal do homem; anistórico.}{a.is.tó.ri.co}{0}
\verb{ajaezado}{}{}{}{}{adj.}{Diz"-se de cavalos e muares com todos os seus arreios e ornatos.}{a.ja.e.za.do}{0}
\verb{ajaezado}{}{Fig.}{}{}{}{Que está cheio de enfeites; ornado.}{a.ja.e.za.do}{0}
\verb{ajaezar}{}{}{}{}{v.t.}{Adornar, especialmente cavalos e muares, com jaezes; arrear, selar.}{a.ja.e.zar}{0}
\verb{ajaezar}{}{Fig.}{}{}{}{Pôr enfeites; adornar; ataviar.}{a.ja.e.zar}{\verboinum{1}}
\verb{ajantarado}{}{}{}{}{adj.}{Que se assemelha ao jantar.}{a.jan.ta.ra.do}{0}
\verb{ajantarado}{}{Bras.}{}{}{s.m.}{Refeição substanciosa e abundante, servida aos domingos e aos feriados, um pouco mais tarde que a hora habitual do almoço, para suprir o jantar; almoço ajantarado.}{a.jan.ta.ra.do}{0}
\verb{ajardinado}{}{}{}{}{adj.}{Que tem jardim.}{a.jar.di.na.do}{0}
\verb{ajardinado}{}{}{}{}{}{Que apresenta aspecto de jardim.}{a.jar.di.na.do}{0}
\verb{ajardinar}{}{}{}{}{v.t.}{Dispor em forma de jardim.}{a.jar.di.nar}{0}
\verb{ajardinar}{}{}{}{}{}{Converter ou transformar em jardim.}{a.jar.di.nar}{\verboinum{1}}
\verb{ajeitar}{}{}{}{}{v.t.}{Colocar de maneira adequada; acomodar, adaptar. }{a.jei.tar}{0}
\verb{ajeitar}{}{}{}{}{}{Conseguir por meios hábeis.}{a.jei.tar}{0}
\verb{ajeitar}{}{}{}{}{}{Dar jeito; elaborar, formar.}{a.jei.tar}{0}
\verb{ajeitar}{}{}{}{}{v.pron.}{Relacionar"-se bem; integrar"-se.}{a.jei.tar}{0}
\verb{ajeitar}{}{}{}{}{}{Tornar"-se evidente; deparar"-se.}{a.jei.tar}{\verboinum{1}}
\verb{ajoelhar"-se}{}{}{}{}{v.pron.}{Pôr"-se de joelhos.}{a.jo.e.lhar"-se}{0}
\verb{ajoelhar"-se}{}{Fig.}{}{}{}{Agir de forma submissa; humilhar"-se.}{a.jo.e.lhar"-se}{\verboinum{1}}
\verb{ajoujar}{}{}{}{}{v.t.}{Prender ou ligar com ajoujo.}{a.jou.jar}{0}
\verb{ajoujar}{}{Fig.}{}{}{v.pron.}{Deixar"-se dominar; submeter"-se.}{a.jou.jar}{\verboinum{1}}
\verb{ajoujo}{ô}{}{}{}{s.m.}{Cordão ou correia usada para prender ou jungir animais pelo pescoço.}{a.jou.jo}{0}
\verb{ajoujo}{ô}{}{}{}{}{Tira de couro com que se prende dois bois pelos chifres.}{a.jou.jo}{0}
\verb{ajoujo}{ô}{Fig.}{}{}{}{União forçada; indesejável.}{a.jou.jo}{0}
\verb{ajuda}{}{}{}{}{s.f.}{Ato ou efeito de ajudar; amparo, auxílio, socorro.}{a.ju.da}{0}
\verb{ajuda}{}{}{}{}{}{Favor que se presta a alguém; obséquio.}{a.ju.da}{0}
\verb{ajudador}{ô}{}{}{}{adj.}{Que ajuda, que auxilia; ajudante.}{a.ju.da.dor}{0}
\verb{ajudador}{ô}{}{}{}{s.m.}{Indivíduo que ajuda.}{a.ju.da.dor}{0}
\verb{ajudante}{}{}{}{}{adj.2g.}{Que ajuda; ajudador.}{a.ju.dan.te}{0}
\verb{ajudante}{}{}{}{}{s.2g.}{Indivíduo que ajuda outro; auxiliar.}{a.ju.dan.te}{0}
\verb{ajudar}{}{}{}{}{v.t.}{Prestar socorro; dar ajuda; auxiliar.}{a.ju.dar}{0}
\verb{ajudar}{}{}{}{}{}{Tornar mais fácil; propiciar.}{a.ju.dar}{0}
\verb{ajudar}{}{}{}{}{v.pron.}{Valer"-se, aproveitar"-se.}{a.ju.dar}{\verboinum{1}}
\verb{ajuizado}{}{}{}{}{adj.}{Que tem juízo; sensato, prudente.}{a.ju.i.za.do}{0}
\verb{ajuizado}{}{Jur.}{}{}{}{Que está posto em juízo.}{a.ju.i.za.do}{0}
\verb{ajuizar}{}{}{}{}{v.t.}{Formar juízo ou conceito; julgar, avaliar, ponderar, considerar.}{a.ju.i.zar}{0}
\verb{ajuizar}{}{Jur.}{}{}{}{Tornar objeto de demanda.}{a.ju.i.zar}{0}
\verb{ajuizar}{}{}{}{}{v.i.}{Ponderar, refletir, cogitar.}{a.ju.i.zar}{\verboinum{8}}
\verb{ajuntamento}{}{}{}{}{s.m.}{Ato ou efeito de ajuntar, de aproximar.}{a.jun.ta.men.to}{0}
\verb{ajuntamento}{}{}{}{}{}{Agrupamento de pessoas; aglomeração.}{a.jun.ta.men.to}{0}
\verb{ajuntar}{}{}{}{}{v.t.}{Pôr junto, aproximar, unir.}{a.jun.tar}{0}
\verb{ajuntar}{}{}{}{}{}{Coligir, colecionar.}{a.jun.tar}{0}
\verb{ajuntar}{}{}{}{}{}{Dizer em seguida; acrescentar.}{a.jun.tar}{0}
\verb{ajuntar}{}{}{}{}{v.i.}{Juntar dinheiro; economizar.}{a.jun.tar}{0}
\verb{ajuntar}{}{}{}{}{v.pron.}{Amasiar"-se, amigar"-se.}{a.jun.tar}{\verboinum{1}}
\verb{ajuramentado}{}{}{}{}{adj.}{Que prestou juramento; juramentado.}{a.ju.ra.men.ta.do}{0}
\verb{ajuramentar}{}{}{}{}{v.t.}{Conceder juramento a.}{a.ju.ra.men.tar}{0}
\verb{ajuramentar}{}{}{}{}{}{Fazer jurar.}{a.ju.ra.men.tar}{0}
\verb{ajuramentar}{}{}{}{}{v.pron.}{Obrigar"-se por juramento.}{a.ju.ra.men.tar}{\verboinum{1}}
\verb{ajustado}{}{}{}{}{adj.}{Que é exato; que está na medida certa.}{a.jus.ta.do}{0}
\verb{ajustado}{}{}{}{}{}{Que foi objeto de ajuste; combinado, contratado.}{a.jus.ta.do}{0}
\verb{ajustado}{}{}{}{}{}{Que está em conformidade, harmonia.}{a.jus.ta.do}{0}
\verb{ajustado}{}{}{}{}{}{Que se ajustou.}{a.jus.ta.do}{0}
\verb{ajustagem}{}{Bras.}{"-ens}{}{s.f.}{Ato ou efeito de ajustar; regulagem, ajuste.}{a.jus.ta.gem}{0}
\verb{ajustamento}{}{}{}{}{s.m.}{Ato ou efeito de ajustar, de tornar justo ou exato; ajuste.}{a.jus.ta.men.to}{0}
\verb{ajustamento}{}{}{}{}{}{Integração em um determinado contexto; adaptação, amoldamento.}{a.jus.ta.men.to}{0}
\verb{ajustamento}{}{}{}{}{}{Integridade moral; retidão, justiça.}{a.jus.ta.men.to}{0}
\verb{ajustar}{}{}{}{}{v.t.}{Tornar justo ou exato; acertar, igualar.}{a.jus.tar}{0}
\verb{ajustar}{}{}{}{}{}{Adaptar com precisão; acomodar, harmonizar.}{a.jus.tar}{0}
\verb{ajustar}{}{}{}{}{}{Estipular por acordo ou contrato.}{a.jus.tar}{0}
\verb{ajustar}{}{}{}{}{}{Apertar peça de vestuário.}{a.jus.tar}{0}
\verb{ajustar}{}{}{}{}{}{Regularizar, acertar contas.}{a.jus.tar}{\verboinum{1}}
\verb{ajuste}{}{}{}{}{s.m.}{Ato ou efeito de ajustar; ajustamento.}{a.jus.te}{0}
\verb{ajuste}{}{}{}{}{}{Estabelecimento de um trato; acordo; combinação.}{a.jus.te}{0}
\verb{ajuste}{}{}{}{}{}{Adaptação, amoldamento.}{a.jus.te}{0}
\verb{ajuste}{}{}{}{}{}{Liquidação de uma conta.}{a.jus.te}{0}
\verb{ajutório}{}{}{}{}{s.m.}{Prestação de ajuda; auxílio.}{a.ju.tó.rio}{0}
\verb{Al}{}{Quím.}{}{}{}{Símb. do \textit{alumínio}.}{Al}{0}
\verb{ala}{}{}{}{}{s.f.}{Fila de coisas ou de pessoas; fileira, renque. }{a.la}{0}
\verb{ala}{}{}{}{}{}{Cada um dos grupos internos de um partido político, associação etc.}{a.la}{0}
\verb{ala}{}{Bras.}{}{}{}{Cada uma das divisões de uma escola de samba. }{a.la}{0}
\verb{ala}{}{}{}{}{}{Asa.}{a.la}{0}
\verb{ala}{}{}{}{}{}{Parte de um edifício que se prolonga além do corpo principal.}{a.la}{0}
\verb{ala}{}{}{}{}{}{Cada um dos lados da linha de ataque, em certos jogos desportivos.}{a.la}{0}
\verb{alabarda}{}{}{}{}{s.f.}{Arma antiga, composta de uma longa haste de madeira terminada em ferro pontiagudo, atravessado por outra peça de ferro cortante, em forma de meia"-lua.}{a.la.bar.da}{0}
\verb{alabardeiro}{ê}{}{}{}{s.m.}{Indivíduo armado de alabarda.}{a.la.bar.dei.ro}{0}
\verb{alabastrino}{}{}{}{}{adj.}{Branco, da cor do alabastro, ou que apresenta outras propriedades dessa rocha.}{a.la.bas.tri.no}{0}
\verb{alabastro}{}{}{}{}{s.m.}{Rocha pouco dura, muito branca e translúcida, usada em trabalhos de escultura.}{a.la.bas.tro}{0}
\verb{alabastro}{}{}{}{}{}{O vaso feito dessa pedra.}{a.la.bas.tro}{0}
\verb{alabastro}{}{}{}{}{}{Qualidade do que é alvo; brancura.}{a.la.bas.tro}{0}
\verb{álacre}{}{}{}{}{adj.2g.}{Que é alegre, vivo, entusiasmado.}{á.la.cre}{0}
\verb{alacridade}{}{}{}{}{s.f.}{Qualidade de álacre; vivacidade, alegria, entusiasmo.}{a.la.cri.da.de}{0}
\verb{alado}{}{}{}{}{adj.}{Que tem asas.}{a.la.do}{0}
\verb{alado}{}{}{}{}{}{Que tem forma de asa.}{a.la.do}{0}
\verb{alado}{}{Fig.}{}{}{}{Que possui graça ou elegância.}{a.la.do}{0}
\verb{alado}{}{}{}{}{s.m.}{Ser vivo ou imaginário dotado de asas.}{a.la.do}{0}
\verb{alagação}{}{}{"-ões}{}{s.f.}{Ato ou efeito de alagar; enchente, inundação, alagamento.}{a.la.ga.ção}{0}
\verb{alagadiço}{}{}{}{}{adj.}{Que é sujeito a ser alagado.}{a.la.ga.di.ço}{0}
\verb{alagadiço}{}{}{}{}{}{Que é pantanoso, lodoso.}{a.la.ga.di.ço}{0}
\verb{alagadiço}{}{}{}{}{s.m.}{Terreno úmido e pesado, próprio para a plantação de arroz; alagadeiro.}{a.la.ga.di.ço}{0}
\verb{alagado}{}{}{}{}{adj.}{Que está coberto de água; encharcado.}{a.la.ga.do}{0}
\verb{alagado}{}{}{}{}{s.m.}{Pequeno lago, decorrente de chuva ou  inundação, de existência temporária.  }{a.la.ga.do}{0}
\verb{alagamento}{}{}{}{}{s.m.}{Ato ou efeito de alagar; enchente, alagação.}{a.la.ga.men.to}{0}
\verb{alagamento}{}{Fig.}{}{}{}{Ação de destruir; arrasamento, ruína.}{a.la.ga.men.to}{0}
\verb{alagar}{}{}{}{}{v.t.}{Cobrir de água; inundar.}{a.la.gar}{0}
\verb{alagar}{}{}{}{}{}{Encher ou cobrir de qualquer líquido.}{a.la.gar}{0}
\verb{alagar}{}{Fig.}{}{}{}{Alastrar"-se por; invadir.}{a.la.gar}{0}
\verb{alagar}{}{}{}{}{}{Causar a ruína de; destruir.}{a.la.gar}{\verboinum{5}}
\verb{alagoano}{}{}{}{}{adj.}{Relativo ao estado de Alagoas.}{a.la.go.a.no}{0}
\verb{alagoano}{}{}{}{}{s.m.}{Indivíduo natural ou habitante desse estado.}{a.la.go.a.no}{0}
\verb{alamar}{}{}{}{}{s.m.}{Adorno que guarnece uniformes militares de gala, feito de cordão trançado de seda, lã ou metal com alças, para abotoamento em tirinhas enviesadas, colocadas na parte inferior das mangas ou no abotoamento frontal do casaco; usado também só como efeito decorativo. }{a.la.mar}{0}
\verb{alambicado}{}{}{}{}{adj.}{Que se alambicou, destilado em alambique.}{a.lam.bi.ca.do}{0}
\verb{alambicado}{}{Fig.}{}{}{}{Que é presumido; afetado, pretensioso.}{a.lam.bi.ca.do}{0}
\verb{alambicar}{}{}{}{}{v.t.}{Destilar no alambique.}{a.lam.bi.car}{0}
\verb{alambicar}{}{Fig.}{}{}{}{Tornar afetado, pretensioso.}{a.lam.bi.car}{0}
\verb{alambicar}{}{}{}{}{}{Requintar, aprimorar com afetação.}{a.lam.bi.car}{\verboinum{2}}
\verb{alambique}{}{}{}{}{s.m.}{Aparelho de destilação, formado por uma caldeira onde se deposita o material a ser destilado, uma tubulação que conduz os vapores, e um condensador no qual os vapores passam para o estado líquido. }{a.lam.bi.que}{0}
\verb{alambrado}{}{}{}{}{adj.}{Que é cercado com arame.}{a.lam.bra.do}{0}
\verb{alambrado}{}{}{}{}{s.m.}{Cerca feita de fios de arame.}{a.lam.bra.do}{0}
\verb{alambrado}{}{}{}{}{}{Terreno com essa cerca.}{a.lam.bra.do}{0}
\verb{alameda}{ê}{}{}{}{s.f.}{Rua ou caminho marginado de álamos.}{a.la.me.da}{0}
\verb{alameda}{ê}{Por ext.}{}{}{}{Rua marginada por qualquer espécie de árvore.}{a.la.me.da}{0}
\verb{álamo}{}{Bot.}{}{}{s.m.}{Árvore ornamental de flores pequenas e madeira clara e leve; choupo.}{á.la.mo}{0}
\verb{alano}{}{}{}{}{adj.}{Relativo aos alanos, antigo povo asiático que invadiu a Gália e a Península Ibérica no século \textsc{v}.}{a.la.no}{0}
\verb{alano}{}{}{}{}{s.m.}{Grande cão de fila, utilizado na guarda e na caça.}{a.la.no}{0}
\verb{alar}{}{}{}{}{adj.2g.}{Relativo a asa; que tem forma de asa.}{a.lar}{0}
\verb{alar}{}{}{}{}{v.t.}{Dar asas; fazer voar.}{a.lar}{0}
\verb{alar}{}{}{}{}{}{Elevar, subir.}{a.lar}{0}
\verb{alar}{}{}{}{}{}{Suspender, içar.}{a.lar}{\verboinum{1}}
\verb{alaranjado}{}{}{}{}{adj.}{Diz"-se da cor próxima ao laranja, entre o vermelho e o amarelo.}{a.la.ran.ja.do}{0}
\verb{alaranjado}{}{}{}{}{}{Que é semelhante à laranja em gosto, cheiro e forma.}{a.la.ran.ja.do}{0}
\verb{alarde}{}{}{}{}{s.m.}{Ato ou efeito de alardear, exibir; ostentação, exibicionismo.}{a.lar.de}{0}
\verb{alardear}{}{}{}{}{v.t.}{Anunciar com ostentação; exibir, vangloriar.}{a.lar.de.ar}{\verboinum{4}}
\verb{alargamento}{}{}{}{}{s.m.}{Ato ou efeito de alargar; ampliação, aumento.}{a.lar.ga.men.to}{0}
\verb{alargar}{}{}{}{}{v.t.}{Tornar mais largo; afrouxar.}{a.lar.gar}{0}
\verb{alargar}{}{}{}{}{}{Tornar mais amplo, aumentar, prolongar.}{a.lar.gar}{\verboinum{5}}
\verb{alarido}{}{}{}{}{s.m.}{Ruído de vozes; gritaria, algazarra.}{a.la.ri.do}{0}
\verb{alarido}{}{}{}{}{}{Choradeira, lamentação.}{a.la.ri.do}{0}
\verb{alarma}{}{}{}{}{}{Var. de \textit{alarme}.}{a.lar.ma}{0}
\verb{alarmante}{}{}{}{}{adj.2g.}{Que causa alarme; que apresenta perigo, ameaça.}{a.lar.man.te}{0}
\verb{alarmante}{}{}{}{}{}{Que assusta, sobressalta; inquietante. }{a.lar.man.te}{0}
\verb{alarmar}{}{}{}{}{v.t.}{Dar sinal de alarme; alertar, avisar.}{a.lar.mar}{0}
\verb{alarmar}{}{}{}{}{}{Pôr em alarme; assustar, alvoroçar.}{a.lar.mar}{\verboinum{1}}
\verb{alarme}{}{}{}{}{s.m.}{Grito de alerta para chamar às armas; alarma.}{a.lar.me}{0}
\verb{alarme}{}{}{}{}{}{Sinal para avisar sobre algum perigo.}{a.lar.me}{0}
\verb{alarme}{}{}{}{}{}{Mecanismo disposto junto a portas e janelas para denunciar a presença de ladrões ou invasores.}{a.lar.me}{0}
\verb{alarme}{}{}{}{}{}{Gritaria, tumulto, confusão.}{a.lar.me}{0}
\verb{alarmista}{}{}{}{}{adj.2g.}{Que espalha notícias ou boatos inquietantes; pessimista.}{a.lar.mis.ta}{0}
\verb{alarve}{}{}{}{}{adj.2g.}{Relativo aos árabes do deserto.}{a.lar.ve}{0}
\verb{alarve}{}{}{}{}{}{Tolo, estúpido, vaidoso.}{a.lar.ve}{0}
\verb{alarve}{}{}{}{}{}{Comilão, glutão.}{a.lar.ve}{0}
\verb{alasquiano}{}{}{}{}{adj.}{Pertencente ao estado norte"-americano do Alasca.}{a.las.qui.a.no}{0}
\verb{alasquiano}{}{}{}{}{s.m.}{Indivíduo natural ou habitante desse estado.}{a.las.qui.a.no}{0}
\verb{alastramento}{}{}{}{}{s.m.}{Ato ou efeito de alastrar; propagação, difusão.}{a.las.tra.men.to}{0}
\verb{alastrar}{}{}{}{}{v.t.}{Pôr lastro; acrescentar peso; lastrar.}{a.las.trar}{0}
\verb{alastrar}{}{}{}{}{}{Encher, cobrir.}{a.las.trar}{0}
\verb{alastrar}{}{}{}{}{}{Derramar, espalhar.}{a.las.trar}{0}
\verb{alastrar}{}{}{}{}{}{Difundir, propagar, disseminar.}{a.las.trar}{\verboinum{1}}
\verb{alastrim}{}{Med.}{}{}{s.m.}{Doença infecto"-contagiosa, caracterizada por erupções na pele; forma benigna de varíola.}{a.las.trim}{0}
\verb{alatinado}{}{Gram.}{}{}{adj.}{Que segue a língua latina na forma ou na sintaxe.}{a.la.ti.na.do}{0}
\verb{alatinar}{}{Gram.}{}{}{v.t.}{Dar semelhança latina à morfologia ou à sintaxe de uma língua; latinizar.}{a.la.ti.nar}{\verboinum{1}}
\verb{alaúde}{}{Mús.}{}{}{s.m.}{Instrumento de cordas dedilhadas, com fundo abaulado e em forma de meia pera, muito utilizado no ocidente da Idade Média ao século século \textsc{xviii}.}{a.la.ú.de}{0}
\verb{alavanca}{}{}{}{}{s.f.}{Barra rígida, de ferro ou madeira, com ponto de apoio em uma das extremidades, utilizada para mover ou levantar objetos pesados.}{a.la.van.ca}{0}
\verb{alavanca}{}{Fig.}{}{}{}{Meio utilizado para se obter resultados; incentivo, expediente.}{a.la.van.ca}{0}
\verb{alavancagem}{}{}{"-ens}{}{s.f.}{Ato ou efeito de alavancar; levantamento.}{a.la.van.ca.gem}{0}
\verb{alavancar}{}{}{}{}{v.t.}{Levantar ou erguer com auxílio de uma alavanca.}{a.la.van.car}{0}
\verb{alavancar}{}{Fig.}{}{}{}{Fazer avançar; impulsionar, incentivar.}{a.la.van.car}{\verboinum{2}}
\verb{alazão}{}{}{"-ães \textit{ou} -ões}{"-ã}{adj.}{Diz"-se do cavalo que possui pelo avermelhado, cor de canela.}{a.la.zão}{0}
\verb{alazão}{}{}{"-ães \textit{ou} -ões}{"-ã}{s.m.}{Cavalo dessa cor.}{a.la.zão}{0}
\verb{alba}{}{}{}{}{s.f.}{Composição poética medieval em que se cantavam cenas ocorridas ao amanhecer; alva.}{al.ba}{0}
\verb{albanês}{}{}{}{}{adj.}{Relativo à Albânia.}{al.ba.nês}{0}
\verb{albanês}{}{}{}{}{s.m.}{Indivíduo natural ou habitante da Albânia.}{al.ba.nês}{0}
\verb{albanês}{}{}{}{}{}{Língua falada nesse país.}{al.ba.nês}{0}
\verb{albarda}{}{}{}{}{s.f.}{Sela rústica que se coloca nos animais de carga.}{al.bar.da}{0}
\verb{albarda}{}{Pop.}{}{}{}{Roupa malfeita, grosseira.}{al.bar.da}{0}
\verb{albarda}{}{Fig.}{}{}{}{Situação humilhante; vexame.}{al.bar.da}{0}
\verb{albardão}{}{}{"-ões}{}{s.m.}{Sela grosseira de grande tamanho, utilizada em animais de carga ou em cavalos.}{al.bar.dão}{0}
\verb{albardão}{}{Geogr.}{"-ões}{}{}{Terreno elevado, à margem de rios ou lagos.}{al.bar.dão}{0}
\verb{albatroz}{ó}{Zool.}{}{}{s.m.}{Ave marítima de grande porte, de corpo branco e asas longas e cinzas, usadas para planar.  }{al.ba.troz}{0}
\verb{albergar}{}{}{}{}{v.t.}{Dar hospedagem a; abrigar, asilar, alojar.}{al.ber.gar}{0}
\verb{albergar}{}{}{}{}{}{Conter, encerrar, guardar.}{al.ber.gar}{\verboinum{5}}
\verb{albergaria}{}{}{}{}{s.f.}{Local onde se recebem hóspedes; albergue, hospedaria.}{al.ber.ga.ri.a}{0}
\verb{albergaria}{}{}{}{}{}{Contrato de hospedagem.}{al.ber.ga.ri.a}{0}
\verb{albergue}{é}{}{}{}{s.m.}{Hospedaria, pousada, albergaria.}{al.ber.gue}{0}
\verb{albergue}{é}{}{}{}{}{Local onde se recolhem pessoas carentes; asilo, abrigo.}{al.ber.gue}{0}
\verb{albinismo}{}{Med.}{}{}{s.m.}{Doença hereditária, que acomete espécies animais e vegetais, caracterizada pela ausência total ou parcial de pigmentos da pele, dos pelos e da íris.}{al.bi.nis.mo}{0}
\verb{albino}{}{}{}{}{adj.}{Que não possui pigmentação da pele, dos pelos e da íris.}{al.bi.no}{0}
\verb{albor}{ô}{}{}{}{}{Var. de \textit{alvor}.}{al.bor}{0}
\verb{albornoz}{ó}{}{}{}{s.m.}{Manto de lã, comprido e com capuz, usado pelos árabes.}{al.bor.noz}{0}
\verb{álbum}{}{}{"-uns}{}{s.m.}{Caderno ou livro, de papel grosso, próprio para colagem de fotos, selos, recortes etc.}{ál.bum}{0}
\verb{álbum}{}{}{"-uns}{}{}{Tabuleta branca em que os antigos romanos publicavam anúncios, atos oficiais etc.}{ál.bum}{0}
\verb{albume}{}{}{}{}{s.m.}{Tecido nutritivo que envolve o embrião de certas sementes.}{al.bu.me}{0}
\verb{albume}{}{}{}{}{}{Clara do ovo.}{al.bu.me}{0}
\verb{albúmen}{}{}{}{}{}{Var. de \textit{albume}.}{al.bú.men}{0}
\verb{albumina}{}{Bioquím.}{}{}{s.f.}{Proteína viscosa, solúvel em água, encontrada na clara do ovo e na semente de certas plantas.}{al.bu.mi.na}{0}
\verb{alburno}{}{Bot.}{}{}{s.m.}{Parte nova da madeira, situada na periferia do tronco e dos galhos, que serve para conduzir água da raiz para o topo da árvore.}{al.bur.no}{0}
\verb{alça}{}{}{}{}{s.f.}{Peça que serve para prender, puxar ou levantar alguma coisa; asa, argola, presilha.}{al.ça}{0}
\verb{alça}{}{}{}{}{}{Tira de pano passada pelos ombros para segurar certas peças do vestuário; suspensório.}{al.ça}{0}
\verb{alça}{}{}{}{}{}{Peça de sola usada pelos sapateiros para ajustar as formas, tornando"-as mais altas.}{al.ça}{0}
\verb{alcácer}{ê}{}{}{}{s.m.}{Antiga fortaleza ou palácio fortificado, de origem árabe.}{al.cá.cer}{0}
\verb{alcachofra}{ô}{Bot.}{}{}{s.f.}{Planta hortense, cujas flores podem ser utilizadas como alimento.}{al.ca.cho.fra}{0}
\verb{alcaçuz}{}{Bot.}{}{}{s.m.}{Arbusto de flores azuis, cujo rizoma adocicado, em forma de xarope ou pasta, é usado na fabricação de remédios, doces e cervejas.}{al.ca.çuz}{0}
\verb{alcaçuz}{}{}{}{}{}{Suco ou xarope adocicado dessa planta.}{al.ca.çuz}{0}
\verb{alçada}{}{}{}{}{s.f.}{Competência, atribuição, limite da ação, campo de atuação.}{al.ça.da}{0}
\verb{alçada}{}{Jur.}{}{}{}{Jurisdição ou competência com que podem atuar juízes ou oficiais de justiça.}{al.ça.da}{0}
\verb{alcaguetar}{}{}{}{}{v.t.}{Denunciar como culpado; acusar, delatar, dedurar, caguetar.}{al.ca.gue.tar}{\verboinum{1}}
\verb{alcaguete}{ê}{}{}{}{s.2g.}{Indivíduo que serve de espião para a polícia.}{al.ca.gue.te}{0}
\verb{alcaguete}{ê}{Pop.}{}{}{}{Delator, dedo"-duro, cagueta.}{al.ca.gue.te}{0}
\verb{alcaide}{}{}{}{}{s.m.}{Antigo governador de província ou comarca.}{al.cai.de}{0}
\verb{alcaide}{}{}{}{}{}{Antigo oficial de justiça.}{al.cai.de}{0}
\verb{alcaide}{}{Pop.}{}{}{}{Objeto velho, desusado, imprestável.}{al.cai.de}{0}
\verb{alcaide}{}{Pop.}{}{}{}{Pessoa muito feia ou muito velha; gagá.}{al.cai.de}{0}
\verb{álcali}{}{Quím.}{}{}{s.m.}{Substância que mantém as mesmas propriedades da base.}{ál.ca.li}{0}
\verb{álcali}{}{}{}{}{}{Qualquer hidróxido ou óxido de um metal alcalino.}{ál.ca.li}{0}
\verb{alcalinidade}{}{Quím.}{}{}{s.f.}{Qualidade de substância alcalina ou básica.}{al.ca.li.ni.da.de}{0}
\verb{alcalinizar}{}{Quím.}{}{}{v.t.}{Tornar alcalino, básico.}{al.ca.li.ni.zar}{\verboinum{1}}
\verb{alcalino}{}{Quím.}{}{}{adj.}{Relativo a álcali ou a metais alcalinos; básico.}{al.ca.li.no}{0}
\verb{alcalino}{}{Por ext.}{}{}{}{Relativo a qualquer base forte em solução aquosa.}{al.ca.li.no}{0}
\verb{alcalino"-terroso}{ô}{Quím.}{alcalino"-terrosos ⟨ó⟩}{}{adj.}{Qualquer dos metais bivalentes do grupo \textsc{ii} da Tabela Periódica (cálcio, rádio, bário etc.).  }{al.ca.li.no"-ter.ro.so}{0}
\verb{alcaloide}{}{Quím.}{}{}{s.m.}{Nome genérico de compostos químicos orgânicos nitrogenados, com características básicas, muitos deles com aplicações terapêuticas, encontrados em vegetais, em alguns fungos ou obtidos por síntese. }{al.ca.loi.de}{0}
\verb{alcançar}{}{}{}{}{v.t.}{Chegar, vencendo uma distância.}{al.can.çar}{0}
\verb{alcançar}{}{}{}{}{}{Chegar até; atingir.}{al.can.çar}{0}
\verb{alcançar}{}{}{}{}{}{Conseguir, obter.}{al.can.çar}{0}
\verb{alcançar}{}{}{}{}{}{Compreender, entender.}{al.can.çar}{0}
\verb{alcançar}{}{}{}{}{}{Inteirar, abranger.}{al.can.çar}{\verboinum{3}}
\verb{alcance}{}{}{}{}{s.m.}{Ato ou efeito de alcançar, de chegar.}{al.can.ce}{0}
\verb{alcance}{}{}{}{}{}{Busca, encalço.}{al.can.ce}{0}
\verb{alcance}{}{}{}{}{}{Distância máxima que a vista ou um projétil consegue atingir.}{al.can.ce}{0}
\verb{alcance}{}{}{}{}{}{Obtenção, conquista.}{al.can.ce}{0}
\verb{alcance}{}{}{}{}{}{Importância, valor.}{al.can.ce}{0}
\verb{alcandorar"-se}{}{}{}{}{v.pron.}{Pousar em poleiro.}{al.can.do.rar"-se}{0}
\verb{alcandorar"-se}{}{Por ext.}{}{}{}{Colocar"-se no alto; elevar"-se, guindar"-se. }{al.can.do.rar"-se}{0}
\verb{alcandorar"-se}{}{Por ext.}{}{}{}{Tornar sublime; exaltar"-se.}{al.can.do.rar"-se}{\verboinum{1}}
\verb{alcanfor}{ô}{}{}{}{s.m.}{Substância cristalina, com odor característico, de largo emprego industrial e terapêutico, extraída da canforeira, e também obtida por via sintética; cânfora.}{al.can.for}{0}
\verb{alcantil}{}{}{}{}{s.m.}{Rocha talhada verticalmente.}{al.can.til}{0}
\verb{alcantil}{}{}{}{}{}{Lugar mais alto de uma elevação; cume.}{al.can.til}{0}
\verb{alcantilado}{}{}{}{}{adj.}{Que é talhado verticalmente.}{al.can.ti.la.do}{0}
\verb{alcantilado}{}{}{}{}{}{Que tem altura elevada; empinado.}{al.can.ti.la.do}{0}
\verb{alçapão}{}{}{"-ões}{}{s.m.}{Porta ou tampa horizontal, que se fecha de cima para baixo e comunica dois pavimentos.}{al.ça.pão}{0}
\verb{alçapão}{}{Bras.}{"-ões}{}{}{Armadilha para capturar pássaros.}{al.ça.pão}{0}
\verb{alcaparra}{}{}{}{}{s.f.}{Botão floral da alcaparreira, verde, aromático, usado como condimento, conservado em vinagre.}{al.ca.par.ra}{0}
\verb{alcaparra}{}{}{}{}{}{Alcaparreira.}{al.ca.par.ra}{0}
\verb{alcaparreira}{ê}{Bot.}{}{}{s.f.}{Arbusto espinhoso, cujos botões florais são usados como condimento.}{al.ca.par.rei.ra}{0}
\verb{alçaprema}{}{}{}{}{s.f.}{Alavanca ou barra para mover pesos consideráveis.}{al.ça.pre.ma}{0}
\verb{alçaprema}{}{}{}{}{}{Pinça resistente usada pelos dentistas.}{al.ça.pre.ma}{0}
\verb{alçar}{}{}{}{}{v.t.}{Tornar alto; erguer, levantar.}{al.çar}{0}
\verb{alçar}{}{Fig.}{}{}{}{Tornar conhecido, louvar.}{al.çar}{0}
\verb{alçar}{}{Por ext.}{}{}{v.pron.}{Sobressair"-se, distinguir"-se.}{al.çar}{\verboinum{3}}
\verb{alcateia}{é}{}{}{}{s.f.}{Bando ou grupo de lobos ou outros canídeos.}{al.ca.tei.a}{0}
\verb{alcateia}{é}{Por ext.}{}{}{}{Manada de animais ferozes.}{al.ca.tei.a}{0}
\verb{alcateia}{é}{Fig.}{}{}{}{Quadrilha de malfeitores, de desordeiros.}{al.ca.tei.a}{0}
\verb{alcatifa}{}{}{}{}{s.f.}{Tapete grande ou tecido de lã ou seda; alfombra.}{al.ca.ti.fa}{0}
\verb{alcatifa}{}{Por ext.}{}{}{}{Qualquer coisa que cubra o chão como um tapete.}{al.ca.ti.fa}{0}
\verb{alcatifar}{}{}{}{}{v.t.}{Revestir com alcatifa.}{al.ca.ti.far}{0}
\verb{alcatifar}{}{}{}{}{}{Cobrir à maneira de alcatifa; atapetar.}{al.ca.ti.far}{\verboinum{1}}
\verb{alcatra}{}{}{}{}{s.f.}{Parte do boi ou da vaca onde termina o lombo, à altura dos rins.}{al.ca.tra}{0}
\verb{alcatrão}{}{Quím.}{}{}{s.m.}{Substância líquida, escura e viscosa, obtida da destilação de matérias orgânicas como petróleo, carvão ou madeira.}{al.ca.trão}{0}
\verb{alcatraz}{}{Zool.}{}{}{s.m.}{Ave marinha, semelhante ao pelicano, cuja fêmea tem abdômen esbranquiçado e cauda longa, e o macho é negro, com o pescoço vermelho na época da procriação.}{al.ca.traz}{0}
\verb{alcatre}{}{}{}{}{}{Var. de \textit{alcatra}.}{al.ca.tre}{0}
\verb{alcatroar}{}{}{}{}{v.t.}{Revestir, misturar ou untar com alcatrão.}{al.ca.tro.ar}{\verboinum{7}}
\verb{alcatruz}{}{}{}{}{s.m.}{Vaso de barro, geralmente cilíndrico, usado para tirar água dos poços; caçamba.}{al.ca.truz}{0}
\verb{alce}{}{Zool.}{}{}{s.m.}{Mamífero ruminante, habitante de regiões frias, de pelagem escura, ponta do focinho larga e com chifres ramificados em galhadas. }{al.ce}{0}
\verb{alcear}{}{}{}{}{v.t.}{Pôr no alto; erguer, levantar.}{al.ce.ar}{0}
\verb{alcear}{}{}{}{}{}{Colocar alça.}{al.ce.ar}{\verboinum{4}}
\verb{alcíone}{}{Zool.}{}{}{s.f.}{Ave encontrada na Europa e na Ásia, com cerca de 16 cm de comprimento, plumagem dorsal azul e partes inferiores acastanhadas. }{al.cí.o.ne}{0}
\verb{alcíone}{}{Astron.}{}{}{}{Uma das setes estrelas visíveis à vista desarmada das Plêiades, grupo de estrelas da constelação de touro. }{al.cí.o.ne}{0}
\verb{alcoice}{}{}{}{}{}{Var. de \textit{alcouce}.}{al.coi.ce}{0}
\verb{álcool}{}{}{"-óis}{}{s.m.}{Líquido incolor, altamente inflamável, de cheiro e sabor característicos, obtido mediante a destilação de substâncias açucaradas e fermentadas, com largo uso como material de limpeza e como combustível.}{ál.co.ol}{0}
\verb{álcool}{}{Quím.}{"-óis}{}{}{Composto orgânico que deriva dos hidrocarbonetos, pela substituição de um átomo de hidrogênio ligado ao carbono por uma hidroxila. }{ál.co.ol}{0}
\verb{álcool}{}{Por ext.}{"-óis}{}{}{Qualquer bebida alcoólica.}{ál.co.ol}{0}
\verb{alcoólatra}{}{}{}{}{adj.2g.}{Referente àquele que se entrega ao alcoolismo.}{al.co.ó.la.tra}{0}
\verb{alcoólatra}{}{}{}{}{s.2g.}{Pessoa viciada na ingestão de bebidas alcoólicas; alcoólico.}{al.co.ó.la.tra}{0}
\verb{alcoólico}{}{}{}{}{adj.}{Relativo ao álcool.}{al.co.ó.li.co}{0}
\verb{alcoólico}{}{}{}{}{}{Que contém álcool.}{al.co.ó.li.co}{0}
\verb{alcoólico}{}{}{}{}{}{Que é viciado em bebidas alcoólicas.}{al.co.ó.li.co}{0}
\verb{alcoólico}{}{}{}{}{s.m.}{Pessoa que se entrega ao alcoolismo.}{al.co.ó.li.co}{0}
\verb{alcoolismo}{}{}{}{}{s.m.}{Vício de ingerir bebidas alcoólicas. }{al.co.o.lis.mo}{0}
\verb{alcoolismo}{}{}{}{}{}{Estado psíquico e fisiológico decorrente da ingestão demasiada de bebida alcoólica.}{al.co.o.lis.mo}{0}
\verb{alcoolizado}{}{}{}{}{adj.}{Que é tratado ou misturado com álcool.}{al.co.o.li.za.do}{0}
\verb{alcoolizado}{}{}{}{}{}{Que se embriagou; ébrio, bêbado.}{al.co.o.li.za.do}{0}
\verb{alcoolizado}{}{}{}{}{}{Próprio de ébrio.}{al.co.o.li.za.do}{0}
\verb{alcoolizar}{}{}{}{}{v.t.}{Adicionar álcool a uma substância.}{al.co.o.li.zar}{0}
\verb{alcoolizar}{}{}{}{}{}{Tornar embriagado; embebedar.}{al.co.o.li.zar}{\verboinum{1}}
\verb{alcoômetro}{}{}{}{}{s.m.}{Instrumento usado para medir a quantidade de álcool de uma bebida.}{al.co.ô.me.tro}{0}
\verb{Alcorão}{}{}{}{}{s.m.}{O livro sagrado do islamismo, que contém o código moral, político e religioso dos muçulmanos; Corão.}{Al.co.rão}{0}
\verb{alcouce}{}{Desus.}{}{}{s.m.}{Casa de prostituição; prostíbulo.}{al.cou.ce}{0}
\verb{alcova}{ô}{}{}{}{s.f.}{Pequeno quarto de dormir, sem janelas.}{al.co.va}{0}
\verb{alcova}{ô}{}{}{}{}{Quarto de casal.}{al.co.va}{0}
\verb{alcova}{ô}{Fig.}{}{}{}{Refúgio, esconderijo.}{al.co.va}{0}
\verb{alcovitar}{}{}{}{}{v.t.}{Intermediar relações amorosas; arranjar amantes para outrem.}{al.co.vi.tar}{0}
\verb{alcovitar}{}{}{}{}{v.i.}{Fazer intrigas; mexericar.}{al.co.vi.tar}{\verboinum{1}}
\verb{alcoviteira}{ê}{}{}{}{s.f.}{Mulher que alcovita, que atua como intermediária em relações amorosas.}{al.co.vi.tei.ra}{0}
\verb{alcoviteira}{ê}{}{}{}{}{Mulher que faz intrigas; mexeriqueira.}{al.co.vi.tei.ra}{0}
\verb{alcoviteiro}{ê}{}{}{}{s.m.}{Indivíduo que alcovita; intermediário de relações amorosas.}{al.co.vi.tei.ro}{0}
\verb{alcoviteiro}{ê}{}{}{}{}{Homem que mexerica; intrigante.}{al.co.vi.tei.ro}{0}
\verb{alcovitice}{}{}{}{}{s.f.}{Ofício de alcoviteiro.}{al.co.vi.ti.ce}{0}
\verb{alcunha}{}{}{}{}{s.f.}{Nome pelo qual alguém é popularmente conhecido; apelido, apodo.}{al.cu.nha}{0}
\verb{alcunhar}{}{}{}{}{v.t.}{Pôr alcunha; apelidar.}{al.cu.nhar}{\verboinum{1}}
\verb{aldeamento}{}{}{}{}{s.m.}{Ato ou efeito de aldear, de dividir em aldeias.}{al.de.a.men.to}{0}
\verb{aldeamento}{}{Desus.}{}{}{}{Povoado indígena chefiado por missionários ou autoridade leiga.}{al.de.a.men.to}{0}
\verb{aldeão}{}{}{"-ães, -ões \textit{ou} -ãos}{"-ã}{adj.}{Relativo ou próprio de aldeia; rústico, simples.}{al.de.ão}{0}
\verb{aldeão}{}{}{"-ães, -ões \textit{ou} -ãos}{"-ã}{}{Que é natural ou habitante de aldeia.}{al.de.ão}{0}
\verb{aldeão}{}{}{"-ães, -ões \textit{ou} -ãos}{"-ã}{s.m.}{Indivíduo natural ou habitante de aldeia.}{al.de.ão}{0}
\verb{aldear}{}{}{}{}{v.t.}{Dividir ou construir aldeias.}{al.de.ar}{0}
\verb{aldear}{}{}{}{}{}{Reunir, formando aldeia.}{al.de.ar}{\verboinum{4}}
\verb{aldeia}{ê}{}{}{}{s.f.}{Povoado de poucos habitantes, de categoria inferior à vila; povoação rústica.}{al.dei.a}{0}
\verb{aldeia}{ê}{Bras.}{}{}{}{Povoção habitada apenas por índios; maloca.}{al.dei.a}{0}
\verb{aldeído}{}{Quím.}{}{}{s.m.}{Composto orgânico que se forma como primeiro produto da oxidação de certos álcoois.  }{al.de.í.do}{0}
\verb{aldeola}{ó}{}{}{}{s.f.}{Aldeia pequena; aldeota.}{al.de.o.la}{0}
\verb{aldraba}{}{}{}{}{}{Var. de \textit{aldrava}.}{al.dra.ba}{0}
\verb{aldrava}{}{}{}{}{s.f.}{Argola ou peça de metal usada para bater às portas ou abri"-las.}{al.dra.va}{0}
\verb{aleá}{}{}{}{}{s.f.}{Aliá.}{a.le.á}{0}
\verb{aleatório}{}{}{}{}{adj.}{Que depende das circunstâncias, de acontecimentos incertos; casual, fortuito.}{a.le.a.tó.rio}{0}
\verb{alecrim}{}{Bot.}{}{}{s.m.}{Arbusto aromático, cujo óleo das sementes é usado como cicatrizante e em cosméticos, apresenta  folhas com várias propriedades medicinais, também usadas como condimento.  }{a.le.crim}{0}
\verb{alecrim}{}{}{}{}{}{O ramo, a folha ou a flor desse arbusto.}{a.le.crim}{0}
\verb{alegação}{}{}{"-ões}{}{s.f.}{Ato ou efeito de alegar, de mencionar como prova.}{a.le.ga.ção}{0}
\verb{alegação}{}{}{"-ões}{}{}{Aquilo que se alega; argumento, razão, arrazoado.}{a.le.ga.ção}{0}
\verb{alegado}{}{Jur.}{}{}{s.m.}{O que foi mencionado ou referido como argumento ou prova para favorecer algo.}{a.le.ga.do}{0}
\verb{alegante}{}{Jur.}{}{}{adj.2g.}{Que alega, que cita como prova.}{a.le.gan.te}{0}
\verb{alegante}{}{Jur.}{}{}{s.2g.}{Indivíduo que alega, que menciona em defesa de alguma causa.}{a.le.gan.te}{0}
\verb{alegar}{}{}{}{}{v.t.}{Apresentar explicação, desculpa ou pretexto para justificar algo.}{a.le.gar}{0}
\verb{alegar}{}{Jur.}{}{}{}{Citar, mencionar como prova.}{a.le.gar}{\verboinum{5}}
\verb{alegoria}{}{}{}{}{s.f.}{Expressão de uma ideia por uma sucessão de imagens figuradas daquilo que se quer exprimir.}{a.le.go.ri.a}{0}
\verb{alegoria}{}{}{}{}{}{Narrativa imaginária em que se personificam animais e coisas, em que cada pormenor tem valor simbólico.}{a.le.go.ri.a}{0}
\verb{alegoria}{}{Art.}{}{}{}{Representação visual (pintura ou escultura) que confere àquilo que foi representado um significado abstrato.}{a.le.go.ri.a}{0}
\verb{alegoria}{}{}{}{}{}{Carro ou ornamentação que ilustra um enredo de escola de samba.}{a.le.go.ri.a}{0}
\verb{alegórico}{}{}{}{}{adj.}{Relativo à alegoria; que contém alegoria.}{a.le.gó.ri.co}{0}
\verb{alegrão}{}{}{"-ões}{}{s.m.}{Grande, intensa alegria.}{a.le.grão}{0}
\verb{alegrão}{}{Pop.}{"-ões}{}{adj.}{Que está muito alegre.}{a.le.grão}{0}
\verb{alegrar}{}{}{}{}{v.t.}{Tornar alegre; causar alegria.}{a.le.grar}{0}
\verb{alegrar}{}{Pop.}{}{}{}{Dar aspecto alegre; embelezar.}{a.le.grar}{0}
\verb{alegrar}{}{Pop.}{}{}{}{Ficar ligeiramente embriagado.}{a.le.grar}{\verboinum{1}}
\verb{alegre}{é}{}{}{}{adj.2g.}{Que sente alegria; contente, satisfeito.}{a.le.gre}{0}
\verb{alegre}{é}{}{}{}{}{Que inspira ou causa alegria.}{a.le.gre}{0}
\verb{alegre}{é}{Pop.}{}{}{}{Diz"-se de quem está ligeiramente embriagado.}{a.le.gre}{0}
\verb{alegria}{}{}{}{}{s.f.}{Qualidade ou estado de alegre; contentamento, júbilo, satisfação.}{a.le.gri.a}{0}
\verb{alegria}{}{}{}{}{}{Tudo o que alegra, contenta, jubila.}{a.le.gri.a}{0}
\verb{alegria}{}{Por ext.}{}{}{}{Divertimento, festa.}{a.le.gri.a}{0}
\verb{alegro}{é}{Mús.}{}{}{adv.}{De maneira viva e alegre, em andamento animado.}{a.le.gro}{0}
\verb{alegro}{é}{Mús.}{}{}{s.m.}{Movimento, parte de uma composição que tem esse andamento.}{a.le.gro}{0}
\verb{aleia}{é}{}{}{}{s.f.}{Caminho ladeado de árvores ou arbustos; alameda.}{a.lei.a}{0}
\verb{aleia}{é}{}{}{}{}{Fileira, renque de árvores ou de arbustos.}{a.lei.a}{0}
\verb{aleia}{é}{}{}{}{}{Passagem ou caminho entre dois muros.}{a.lei.a}{0}
\verb{aleijado}{}{}{}{}{adj.}{Que apresenta alguma deficiência física, decorrente de problema congênito ou adquirido.}{a.lei.ja.do}{0}
\verb{aleijado}{}{}{}{}{s.m.}{Indivíduo portador dessa deficiência.}{a.lei.ja.do}{0}
\verb{aleijão}{}{Desus.}{"-ões}{}{s.m.}{Indivíduo portador de alguma deficiência física.}{a.lei.jão}{0}
\verb{aleijão}{}{}{"-ões}{}{}{A deficiência física de algum indivíduo.}{a.lei.jão}{0}
\verb{aleijão}{}{Por ext.}{"-ões}{}{}{Imperfeições em fatos ou processos.}{a.lei.jão}{0}
\verb{aleijar}{}{}{}{}{v.t.}{Causar deficiência física.}{a.lei.jar}{0}
\verb{aleijar}{}{Fig.}{}{}{}{Ferir moralmente; magoar, machucar.}{a.lei.jar}{0}
\verb{aleijar}{}{Fig.}{}{}{}{Causar dano financeiro; explorar.}{a.lei.jar}{\verboinum{1}}
\verb{aleitamento}{}{}{}{}{s.m.}{Ato ou efeito de aleitar; amamentação.}{a.lei.ta.men.to}{0}
\verb{aleitar}{}{}{}{}{v.t.}{Alimentar com leite; amamentar.}{a.lei.tar}{0}
\verb{aleitar}{}{Fig.}{}{}{}{Tornar branco ou claro como leite.}{a.lei.tar}{\verboinum{1}}
\verb{aleive}{}{}{}{}{s.m.}{Aleivosia.}{a.lei.ve}{0}
\verb{aleivosia}{}{}{}{}{s.f.}{Ato cometido com falsas demonstrações de amizade; traição, deslealdade; aleive.}{a.lei.vo.si.a}{0}
\verb{aleivosia}{}{}{}{}{}{Acusação falsa; calúnia.}{a.lei.vo.si.a}{0}
\verb{aleivosia}{}{}{}{}{}{Fraude.}{a.lei.vo.si.a}{0}
\verb{aleivoso}{ô}{}{"-osos ⟨ó⟩}{"-osa ⟨ó⟩}{adj.}{Que procede com aleivosia; desleal, traidor.}{a.lei.vo.so}{0}
\verb{aleivoso}{ô}{}{"-osos ⟨ó⟩}{"-osa ⟨ó⟩}{}{Caluniador.}{a.lei.vo.so}{0}
\verb{aleivoso}{ô}{}{"-osos ⟨ó⟩}{"-osa ⟨ó⟩}{}{Fraudulento, falso.}{a.lei.vo.so}{0}
\verb{aleluia}{}{Relig.}{}{}{s.f.}{Cântico de alegria e louvor judaico"-cristão.}{a.le.lui.a}{0}
\verb{aleluia}{}{}{}{}{}{Pequeno trecho litúrgico da missa que antecede à leitura do Evangelho.}{a.le.lui.a}{0}
\verb{aleluia}{}{}{}{}{}{Sábado do tempo da Páscoa cristã em que é celebrada a Ressurreição de Jesus Cristo.}{a.le.lui.a}{0}
\verb{aleluia}{}{Por ext.}{}{}{interj.}{Expressão que denota alegria e satisfação por um acontecimento ou pela conclusão de algo.}{a.le.lui.a}{0}
\verb{aleluia}{}{Zool.}{}{}{s.f.}{Denominação comum à forma alada de diversos insetos, como o cupim, quando saem do ninho para acasalar; siriri.}{a.le.lui.a}{0}
\verb{além}{}{}{}{}{adv.}{Lá; do lado de lá; acolá.}{a.lém}{0}
\verb{além}{}{}{}{}{}{Mais adiante; mais à frente.}{a.lém}{0}
\verb{além}{}{}{}{}{}{Em lugar longe; mais longe.}{a.lém}{0}
\verb{além}{}{}{}{}{}{Para fora; afora.}{a.lém}{0}
\verb{além}{}{}{}{}{s.m.}{O mundo após a morte; o desconhecido; o além"-túmulo.}{a.lém}{0}
\verb{além}{}{Ant.}{}{}{}{aquém}{a.lém}{0}
\verb{alemão}{}{}{"-ães}{"-ã}{adj.}{Relativo à Alemanha; germânico.}{a.le.mão}{0}
\verb{alemão}{}{}{"-ães}{"-ã}{s.m.}{Indivíduo natural ou habitante desse país.}{a.le.mão}{0}
\verb{alemão}{}{}{"-ães}{"-ã}{}{A língua oficial da Alemanha.}{a.le.mão}{0}
\verb{além"-mar}{}{}{}{}{adv.}{Que está do outro lado do mar; ultramar.}{a.lém"-mar}{0}
\verb{além"-mar}{}{}{}{}{s.m.}{Terras situadas além do mar continental.}{a.lém"-mar}{0}
\verb{além"-túmulo}{}{}{}{}{s.m.}{O que vem após a morte; o além.}{a.lém"-tú.mu.lo}{0}
\verb{alentado}{}{}{}{}{adj.}{Que tem alento; animado, esforçado.}{a.len.ta.do}{0}
\verb{alentado}{}{}{}{}{}{Valente, corajoso.}{a.len.ta.do}{0}
\verb{alentado}{}{}{}{}{}{Que possui corpo grande, volumoso.}{a.len.ta.do}{0}
\verb{alentador}{ô}{}{}{}{adj.}{Que alenta, anima, estimula.}{a.len.ta.dor}{0}
\verb{alentar}{}{}{}{}{v.t.}{Dar ânimo; encorajar.}{a.len.tar}{0}
\verb{alentar}{}{}{}{}{}{Dar sustento; alimentar, nutrir.}{a.len.tar}{0}
\verb{alentar}{}{Desus.}{}{}{v.i.}{Tomar fôlego; respirar.}{a.len.tar}{0}
\verb{alentar}{}{}{}{}{v.pron.}{Entusiasmar"-se.}{a.len.tar}{\verboinum{1}}
\verb{alento}{}{}{}{}{s.m.}{Respiração, fôlego.}{a.len.to}{0}
\verb{alento}{}{}{}{}{}{Ânimo, coragem.}{a.len.to}{0}
\verb{alento}{}{}{}{}{}{O que sustenta; alimento.}{a.len.to}{0}
\verb{alento}{}{}{}{}{}{Inspiração, entusiasmo.}{a.len.to}{0}
\verb{alento}{}{Zool.}{}{}{}{Cada um dos orifícios no interior das ventas do cavalo.}{a.len.to}{0}
\verb{alergia}{}{Med.}{}{}{s.f.}{Reação de hipersensibilidade do organismo a certas substâncias.}{a.ler.gi.a}{0}
\verb{alergia}{}{Fig.}{}{}{}{Sentimento de aversão, repulsa, antipatia.}{a.ler.gi.a}{0}
\verb{alérgico}{}{}{}{}{adj.}{Relativo à alergia.}{a.lér.gi.co}{0}
\verb{alérgico}{}{}{}{}{}{Que sofre de alergia.}{a.lér.gi.co}{0}
\verb{alergologia}{}{Med.}{}{}{s.f.}{Parte da medicina que estuda e trata das doenças alérgicas.}{a.ler.go.lo.gi.a}{0}
\verb{alerta}{é}{}{}{}{adj.2g.}{Que está acordado; atento, vigilante.}{a.ler.ta}{0}
\verb{alerta}{é}{}{}{}{s.m.}{Sinal ou aviso para prevenir de algum perigo.}{a.ler.ta}{0}
\verb{alerta}{é}{}{}{}{interj.}{Expressão que denota aviso diante de uma ameaça.}{a.ler.ta}{0}
\verb{alertar}{}{}{}{}{v.t.}{Dar o alerta; avisar, prevenir.}{a.ler.tar}{\verboinum{1}}
\verb{aleta}{ê}{}{}{}{s.f.}{Pequena ala ou asa.}{a.le.ta}{0}
\verb{aleta}{ê}{Anat.}{}{}{}{Cada uma das duas asas do nariz; narina.}{a.le.ta}{0}
\verb{aletria}{}{Cul.}{}{}{s.f.}{Tipo de massa de sêmola de trigo, em fios muito finos, usada em sopas ou em pratos doces.}{a.le.tri.a}{0}
\verb{alevantar}{}{}{}{}{}{Var. de \textit{levantar}.}{a.le.van.tar}{0}
\verb{alevim}{}{Zool.}{}{}{s.m.}{Alevino.}{a.le.vim}{0}
\verb{alevino}{}{Zool.}{}{}{s.m.}{Forma embrionária de peixe, de corpo fino e alongado, que ainda se alimenta das reservas da bolsa vitelínica; larva ou filhote de peixe.}{a.le.vi.no}{0}
\verb{alexandrino}{ch}{}{}{}{adj.}{Relativo à cidade de Alexandria, no Egito.}{a.le.xan.dri.no}{0}
\verb{alexandrino}{ch}{}{}{}{s.m.}{Indivíduo natural ou habitante dessa cidade.}{a.le.xan.dri.no}{0}
\verb{alexandrino}{ch}{}{}{}{adj.}{Relativo a Alexandre Magno, rei da Macedônia, ou à sua época, século \textsc{iv} a.C.}{a.le.xan.dri.no}{0}
\verb{alexandrino}{ch}{Gram.}{}{}{}{Diz"-se do verso heroico de doze sílabas, com acento na sexta.}{a.le.xan.dri.no}{0}
\verb{alfa}{}{}{}{}{s.m.}{Primeira letra do alfabeto grego.}{al.fa}{0}
\verb{alfa}{}{Fig.}{}{}{}{Princípio, início.}{al.fa}{0}
\verb{alfa}{}{}{}{}{}{A principal estrela de uma constelação.}{al.fa}{0}
\verb{alfabetação}{}{}{"-ões}{}{s.f.}{Ato ou efeito de colocar em ordem alfabética, alfabetar.}{al.fa.be.ta.ção}{0}
\verb{alfabetar}{}{}{}{}{v.t.}{Dispor em ordem alfabética; abecedar.}{al.fa.be.tar}{\verboinum{1}}
\verb{alfabético}{}{}{}{}{adj.}{Relativo ou pertencente ao alfabeto.}{al.fa.bé.ti.co}{0}
\verb{alfabético}{}{}{}{}{}{Que está disposto conforme a ordem das letras do alfabeto.}{al.fa.bé.ti.co}{0}
\verb{alfabetização}{}{}{"-ões}{}{s.f.}{Ato ou efeito de alfabetizar, de ensinar a ler e a escrever.}{al.fa.be.ti.za.ção}{0}
\verb{alfabetização}{}{Por ext.}{"-ões}{}{}{Ato de propagar a instrução primária.}{al.fa.be.ti.za.ção}{0}
\verb{alfabetizado}{}{}{}{}{adj.}{Que se alfabetizou, que aprendeu a ler e a escrever.}{al.fa.be.ti.za.do}{0}
\verb{alfabetizar}{}{}{}{}{v.t.}{Ensinar a ler e a escrever.}{al.fa.be.ti.zar}{0}
\verb{alfabetizar}{}{Por ext.}{}{}{}{Instruir alunos nos primeiros anos de escolarização.}{al.fa.be.ti.zar}{\verboinum{1}}
\verb{alfabeto}{é}{}{}{}{s.m.}{Conjunto de sinais gráficos utilizados para representar letras, fonemas, palavras ou imagens de uma língua.}{al.fa.be.to}{0}
\verb{alfabeto}{é}{}{}{}{}{Qualquer série de letras de uma língua, dispostas ordenadamente.}{al.fa.be.to}{0}
\verb{alface}{}{}{}{}{s.f.}{Planta hortense, de folhas verdes, consumida geralmente em saladas.}{al.fa.ce}{0}
\verb{alfafa}{}{}{}{}{s.f.}{Planta leguminosa, utilizada como forragem na alimentação do gado. }{al.fa.fa}{0}
\verb{alfaia}{}{}{}{}{s.f.}{Adorno, enfeite, joia etc.}{al.fai.a}{0}
\verb{alfaia}{}{}{}{}{}{Peça ornamental de igrejas.}{al.fai.a}{0}
\verb{alfaia}{}{Lus.}{}{}{}{Forquilha utilizada na agricultura.}{al.fai.a}{0}
\verb{alfaiataria}{}{}{}{}{s.f.}{Estabelecimento ou oficina onde se confeccionam roupas com feitio masculino.}{al.fai.a.ta.ri.a}{0}
\verb{alfaiate}{}{}{}{}{s.m.}{Indivíduo especializado em confeccionar roupas com feitio masculino para homens ou mulheres.}{al.fai.a.te}{0}
\verb{alfândega}{}{}{}{}{s.f.}{Repartição pública onde se fiscalizam bagagens e mercadorias em trânsito e são cobradas as taxas relativas à importação e à exportação; aduana.}{al.fân.de.ga}{0}
\verb{alfândega}{}{}{}{}{}{Recinto onde se instala essa repartição, geralmente na fronteira entre países ou em aeroportos e portos marítimos.}{al.fân.de.ga}{0}
\verb{alfândega}{}{Fig.}{}{}{}{Lugar onde há muito barulho e desordem.}{al.fân.de.ga}{0}
\verb{alfandegagem}{}{}{"-ens}{}{s.f.}{Depósito ou armazenamento de mercadorias na alfândega.}{al.fan.de.ga.gem}{0}
\verb{alfandegagem}{}{}{"-ens}{}{}{Cobrança das taxas alfandegárias.}{al.fan.de.ga.gem}{0}
\verb{alfandegário}{}{}{}{}{adj.}{Relativo à alfândega; aduaneiro.}{al.fan.de.gá.rio}{0}
\verb{alfanje}{}{}{}{}{s.m.}{Sabre de origem árabe, de lâmina curta e larga, com o fio no lado convexo da curva.}{al.fan.je}{0}
\verb{alfanje}{}{}{}{}{}{Foice de cabo comprido e grandes dentes de ferro usada para cortar feno.}{al.fan.je}{0}
\verb{alfanumérico}{}{Informát.}{}{}{adj.}{Diz"-se do sistema de codificação que combina letras do alfabeto e números.}{al.fa.nu.mé.ri.co}{0}
\verb{alfarrábio}{}{}{}{}{s.m.}{Livro valioso pela antiguidade, ao qual se atribui pouca importância. }{al.far.rá.bio}{0}
\verb{alfarrabista}{}{}{}{}{adj.2g.}{Que coleciona ou compra e vende livros velhos e usados; sebista, livreiro.}{al.far.ra.bis.ta}{0}
\verb{alfarroba}{ô}{}{}{}{s.f.}{Fruto comestível da alfarrobeira, de polpa adocicada e nutritiva.}{al.far.ro.ba}{0}
\verb{alfarrobeira}{ê}{Bot.}{}{}{s.f.}{Árvore frutífera, cuja madeira, vermelha e dura, é utilizada na marcenaria, tem como fruto a alfarroba.}{al.far.ro.bei.ra}{0}
\verb{alfavaca}{}{Bot.}{}{}{s.f.}{Arbusto lenhoso, cujas folhas aromáticas são utilizadas como condimento e na fabricação de licores.}{al.fa.va.ca}{0}
\verb{alfazema}{}{Bot.}{}{}{s.f.}{Arbusto aromático, do qual se extrai um óleo utilizado na fabricação de perfumes e na medicina; lavanda.}{al.fa.ze.ma}{0}
\verb{alféloa}{}{Cul.}{}{}{s.f.}{Massa de açúcar ou melaço, em ponto grosso, esbranquiçada, utilizada em confeitaria.}{al.fé.lo.a}{0}
\verb{alféloa}{}{}{}{}{}{Bala feita com essa massa.}{al.fé.lo.a}{0}
\verb{alfenim}{}{Cul.}{}{}{s.m.}{Massa de açúcar muito branca e consistente, com a qual se faz um doce de origem árabe.}{al.fe.nim}{0}
\verb{alfenim}{}{Fig.}{}{}{}{Indivíduo delicado, melindroso, efeminado.}{al.fe.nim}{0}
\verb{alferes}{é}{Desus.}{}{}{s.m.}{Antigo posto militar do oficial encarregado de levar a bandeira do regimento.}{al.fe.res}{0}
\verb{alferes}{é}{Por ext.}{}{}{}{Porta"-bandeira.}{al.fe.res}{0}
\verb{alfinetada}{}{}{}{}{s.f.}{Ato ou efeito de alfinetar; picada com alfinete.}{al.fi.ne.ta.da}{0}
\verb{alfinetada}{}{}{}{}{}{Dor muito aguda e rápida, semelhante à picada do alfinete.}{al.fi.ne.ta.da}{0}
\verb{alfinetada}{}{}{}{}{}{Dito ou insinuação picante, maliciosa, sarcástica ou ofensiva.}{al.fi.ne.ta.da}{0}
\verb{alfinetar}{}{}{}{}{v.t.}{Prender ou fixar com alfinete; marcar uma costura.}{al.fi.ne.tar}{0}
\verb{alfinetar}{}{}{}{}{}{Picar ou ferir com alfinete.}{al.fi.ne.tar}{0}
\verb{alfinetar}{}{Fig.}{}{}{}{Fazer provocações com ironia.}{al.fi.ne.tar}{0}
\verb{alfinetar}{}{Desus.}{}{}{}{Oferecer poeminhas ou epigramas.}{al.fi.ne.tar}{\verboinum{1}}
\verb{alfinete}{ê}{}{}{}{s.m.}{Pequena haste metálica, com ponta aguda em uma das extremidades e ponta arredondada em outra, utilizada para prender papéis, panos etc.}{al.fi.ne.te}{0}
\verb{alfinete}{ê}{}{}{}{}{Objeto de adorno ou joia semelhante ao alfinete, utilizado para prender gravatas ou enfeitar chapéus femininos.}{al.fi.ne.te}{0}
\verb{alfineteira}{ê}{}{}{}{s.f.}{Caixa onde se guardam alfinetes.}{al.fi.ne.tei.ra}{0}
\verb{alfineteira}{ê}{}{}{}{}{Pequena almofada em que se espetam alfinetes para não os perder.}{al.fi.ne.tei.ra}{0}
\verb{alfombra}{}{}{}{}{s.f.}{Tapete espesso e macio; alcatifa.}{al.fom.bra}{0}
\verb{alfombra}{}{Por ext.}{}{}{}{Terreno extenso coberto de relva ou musgo.}{al.fom.bra}{0}
\verb{alforje}{ó}{}{}{}{s.m.}{Saco fechado em ambas as extremidades e com abertura no centro formando duas bolsas, usado no ombro ou no lombo de animais de carga.}{al.for.je}{0}
\verb{alforria}{}{}{}{}{s.f.}{Liberdade concedida aos escravos.}{al.for.ri.a}{0}
\verb{alforria}{}{Por ext.}{}{}{}{Liberdade, emancipação.}{al.for.ri.a}{0}
\verb{alforriado}{}{}{}{}{adj.}{Que recebeu carta de alforria; forro, liberto, livre.}{al.for.ri.a.do}{0}
\verb{alforriado}{}{}{}{}{s.m.}{Indivíduo alforriado, liberto.}{al.for.ri.a.do}{0}
\verb{alforriar}{}{}{}{}{v.t.}{Conceder alforria; aforrar, forrar, libertar.}{al.for.ri.ar}{\verboinum{1}}
\verb{alga}{}{Bot.}{}{}{s.f.}{Organismo muito simples, clorofilado, com corpo constituído apenas por um talo, que nasce no fundo ou na superfície de água doce ou salgada; limo.}{al.ga}{0}
\verb{algaravia}{}{}{}{}{s.f.}{Língua árabe.}{al.ga.ra.vi.a}{0}
\verb{algaravia}{}{Fig.}{}{}{}{Linguagem incompreensível ou confusa.}{al.ga.ra.vi.a}{0}
\verb{algarismo}{}{}{}{}{s.m.}{Caractere que representa um número.}{al.ga.ris.mo}{0}
\verb{algarvio}{}{}{}{}{adj.}{Relativo ao Algarve (região ao sul de Portugal).}{al.gar.vi.o}{0}
\verb{algarvio}{}{Fig.}{}{}{}{Falador, tagarela.}{al.gar.vi.o}{0}
\verb{algarvio}{}{}{}{}{s.m.}{Dialeto da língua portuguesa.}{al.gar.vi.o}{0}
\verb{algazarra}{}{}{}{}{s.f.}{Barulheira, falatório.}{al.ga.zar.ra}{0}
\verb{algazarra}{}{}{}{}{}{Gritaria dos mouros durante os combates.}{al.ga.zar.ra}{0}
\verb{álgebra}{}{Mat.}{}{}{s.f.}{Parte da matemática que estuda as operações com grandezas abstratas e cálculos com variáveis.}{ál.ge.bra}{0}
\verb{álgebra}{}{}{}{}{}{Exemplar com estudos dessa matéria.}{ál.ge.bra}{0}
\verb{algébrico}{}{}{}{}{adj.}{Relativo à álgebra.}{al.gé.bri.co}{0}
\verb{algebrista}{}{}{}{}{s.2g.}{Pessoa versada em álgebra.}{al.ge.bris.ta}{0}
\verb{algema}{}{}{}{}{s.f.}{Instrumento de metal constituído por duas argolas unidas por uma corrente, utilizado para prender pessoas pelos pulsos ou pelos tornozelos.}{al.ge.ma}{0}
\verb{algema}{}{Fig.}{}{}{}{Obstáculo ou prisão moral; coerção, opressão.}{al.ge.ma}{0}
\verb{algemar}{}{}{}{}{v.t.}{Prender alguém com algemas.}{al.ge.mar}{0}
\verb{algemar}{}{Fig.}{}{}{}{Coagir, oprimir.}{al.ge.mar}{\verboinum{1}}
\verb{algia}{}{Med.}{}{}{s.f.}{Dor localizada em determinada parte do corpo.}{al.gi.a}{0}
\verb{algibeira}{ê}{}{}{}{s.f.}{Pequeno bolso costurado à roupa pelo lado interno.}{al.gi.bei.ra}{0}
\verb{algibeira}{ê}{Fig.}{}{}{}{Dinheiro, recurso financeiro.}{al.gi.bei.ra}{0}
\verb{álgido}{}{}{}{}{adj.}{Muito frio; algente, glacial.}{ál.gi.do}{0}
\verb{algo}{}{}{}{}{pron.}{Coisa indeterminada; qualquer coisa.}{al.go}{0}
\verb{algo}{}{}{}{}{adv.}{Um tanto; um pouco.}{al.go}{0}
\verb{algo}{}{Desus.}{}{}{s.m.}{Quantidade de bens ou quantia de dinheiro.}{al.go}{0}
\verb{algo}{}{Desus.}{}{}{}{Indivíduo rico.}{al.go}{0}
\verb{algodão}{}{}{"-ões}{}{s.m.}{Conjunto de pelos longos e entrelaçados, geralmente brancos, que reveste as sementes de certas espécies.}{al.go.dão}{0}
\verb{algodão}{}{}{"-ões}{}{}{Fio ou tecido que se fabrica com esses pelos.}{al.go.dão}{0}
\verb{algodão}{}{}{"-ões}{}{}{Produto farmacêutico feito com esse fio.}{al.go.dão}{0}
\verb{algodão}{}{}{"-ões}{}{}{O pé que produz o algodão; algodoeiro.}{al.go.dão}{0}
\verb{algodão"-doce}{ô}{}{algodões"-doces ⟨ô⟩}{}{s.m.}{Doce feito de açúcar reduzido a finíssimos fios que formam flocos semelhantes ao algodão.}{al.go.dão"-do.ce}{0}
\verb{algodão"-pólvora}{}{Quím.}{algodões"-pólvoras \textit{ou} algodões"-pólvora}{}{s.m.}{Substância explosiva obtida pela ação do ácido nítrico sobre o algodão.}{al.go.dão"-pól.vo.ra}{0}
\verb{algodoal}{}{}{"-ais}{}{s.m.}{Coletivo de algodão; extenso aglomerado de algodoeiros em determinada área.}{al.go.do.al}{0}
\verb{algodoaria}{}{}{}{}{s.f.}{Fábrica de fios ou tecidos de algodão; cotonaria, cotonifício.}{al.go.do.a.ri.a}{0}
\verb{algodoeiro}{ê}{Bot.}{}{}{s.m.}{Planta que produz algodão.}{al.go.do.ei.ro}{0}
\verb{algodoeiro}{ê}{}{}{}{}{Fabricante de algodão.}{al.go.do.ei.ro}{0}
\verb{algodoeiro}{ê}{}{}{}{adj.}{Relativo ao algodão.}{al.go.do.ei.ro}{0}
\verb{algoritmo}{}{Mat.}{}{}{s.m.}{Sequência predefinida de regras, raciocínios ou operações que produz uma solução para determinado tipo de problema.}{al.go.rit.mo}{0}
\verb{algoritmo}{}{Informát.}{}{}{}{Conjunto predefinido de regras ou instruções destinado à realização de operações predeterminadas e composto de um número finito de etapas.}{al.go.rit.mo}{0}
\verb{algoz}{ó}{}{}{}{s.m.}{Executor da pena de morte ou de outras penas corporais; carrasco.}{al.goz}{0}
\verb{algoz}{ó}{Fig.}{}{}{}{Pessoa cruel e desumana; assassino.}{al.goz}{0}
\verb{algoz}{ó}{}{}{}{}{Coisa que magoa ou aflige.}{al.goz}{0}
\verb{alguém}{}{}{}{}{pron.}{Alguma pessoa.}{al.guém}{0}
\verb{alguém}{}{}{}{}{}{Pessoa importante, de relevo social. (\textit{É importante ser alguém na vida.})}{al.guém}{0}
\verb{alguidar}{}{}{}{}{s.m.}{Recipiente de barro, metal ou material plástico, baixo, cuja borda tem diâmetro muito maior que o fundo, com diversos usos domésticos; ababá.}{al.gui.dar}{0}
\verb{algum}{}{}{}{}{pron.}{Um qualquer entre dois ou mais.}{al.gum}{0}
\verb{algum}{}{}{}{}{}{Um certo, determinado.}{al.gum}{0}
\verb{algum}{}{}{}{}{}{Em determinado grau ou quantidade.}{al.gum}{0}
\verb{algum}{}{Pop.}{}{}{s.m.}{Algum dinheiro.}{al.gum}{0}
\verb{algures}{}{}{}{}{adv.}{Em algum lugar, em alguma parte.}{al.gu.res}{0}
\verb{alhada}{}{}{}{}{s.f.}{Porção de alhos.}{a.lha.da}{0}
\verb{alhada}{}{Fig.}{}{}{}{Situação difícil ou embaraçosa; intriga, embrulhada.}{a.lha.da}{0}
\verb{alhal}{}{}{}{}{s.m.}{Coletivo de alho; extenso aglomerado de alhos em determinada área.}{a.lhal}{0}
\verb{alheado}{}{}{}{}{adj.}{Que se transferiu; cedido}{a.lhe.a.do}{0}
\verb{alheado}{}{}{}{}{}{Absorto nos próprios pensamentos, distraído, desatento, arrebatado.}{a.lhe.a.do}{0}
\verb{alheamento}{}{}{}{}{s.m.}{Ato ou efeito de alhear; alheação, alienação.}{a.lhe.a.men.to}{0}
\verb{alhear}{}{}{}{}{v.t.}{Transferir para outrem o direito; tornar alheio; alienar.}{a.lhe.ar}{0}
\verb{alhear}{}{}{}{}{}{Distrair"-se.}{a.lhe.ar}{0}
\verb{alhear}{}{}{}{}{v.pron.}{Enlouquecer, alienar"-se.}{a.lhe.ar}{\verboinum{4}}
\verb{alheio}{ê}{}{}{}{adj.}{Que não nos pertence; que diz respeito a outrem.}{a.lhei.o}{0}
\verb{alheio}{ê}{}{}{}{}{Que se mantém distante; afastado, arredado.}{a.lhei.o}{0}
\verb{alheio}{ê}{}{}{}{}{Que carece; privado.}{a.lhei.o}{0}
\verb{alheio}{ê}{}{}{}{s.m.}{O que pertence a outrem.}{a.lhei.o}{0}
\verb{alheios}{}{}{}{}{s.m.}{Os que não são parentes; os estranhos.}{a.lhei.os}{0}
\verb{alheiro}{ê}{}{}{}{s.m.}{Cultivador ou negociante de alhos.}{a.lhei.ro}{0}
\verb{alheiro}{ê}{}{}{}{}{Viveiro de alhos.}{a.lhei.ro}{0}
\verb{alho}{}{Bot.}{}{}{s.m.}{Planta com flores brancas ou avermelhadas cujo bulbo é utilizado como condimento e possui propriedades analgésicas e antissépticas.}{a.lho}{0}
\verb{alho}{}{}{}{}{}{O bulbo dessa planta, constituído por vários dentes.}{a.lho}{0}
\verb{alho}{}{Fig.}{}{}{}{Indivíduo muito esperto e sagaz.}{a.lho}{0}
\verb{alho"-poró}{}{Bot.}{alhos"-porós}{}{s.m.}{Planta com flores de cor branca, rosa ou lilás, cujos bulbos e folhas, utilizados como condimento, têm propriedades estimulantes e diuréticas; alho"-porro.}{a.lho"-po.ró}{0}
\verb{alho"-poró}{}{}{alhos"-porós}{}{}{O bulbo e as folhas dessa planta.}{a.lho"-po.ró}{0}
\verb{alho"-porro}{ô}{Bot.}{alhos"-porros ⟨ô⟩}{}{s.m.}{Alho"-poró.}{a.lho"-por.ro}{0}
\verb{alhures}{}{}{}{}{adv.}{Em outro lugar, em outra parte.}{a.lhu.res}{0}
\verb{ali}{}{}{}{}{adv.}{Naquele lugar}{a.li}{0}
\verb{ali}{}{}{}{}{}{Àquele lugar.}{a.li}{0}
\verb{ali}{}{}{}{}{}{Naquele momento; então.}{a.li}{0}
\verb{ali}{}{}{}{}{}{Naquele ato; naquele assunto; naquela pessoa.}{a.li}{0}
\verb{aliá}{}{}{}{}{s.f.}{Denominação originária do Sri Lanka para a fêmea do elefante; aleá.}{a.li.á}{0}
\verb{aliado}{}{}{}{}{adj.}{Que se liga a outro através de aliança, pacto ou tratado, em uma causa comum.}{a.li.a.do}{0}
\verb{aliado}{}{}{}{}{s.m.}{Membro de uma aliança ou coligação; confederado, coligado.}{a.li.a.do}{0}
\verb{aliado}{}{}{}{}{}{Parente por afinidade.}{a.li.a.do}{0}
\verb{aliança}{}{}{}{}{s.f.}{Ato ou efeito de aliar.}{a.li.an.ça}{0}
\verb{aliança}{}{}{}{}{}{União pelo matrimônio.}{a.li.an.ça}{0}
\verb{aliança}{}{}{}{}{}{Anel que simboliza noivado ou casamento.}{a.li.an.ça}{0}
\verb{aliança}{}{}{}{}{}{Segundo as Escrituras católicas, pacto de Deus com os indivíduos ou com um povo.}{a.li.an.ça}{0}
\verb{aliar}{}{}{}{}{v.t.}{Estabelecer união; associar.}{a.li.ar}{0}
\verb{aliar}{}{}{}{}{}{Unir por pacto ou tratado.}{a.li.ar}{0}
\verb{aliar}{}{}{}{}{}{Unir pelo matrimônio.}{a.li.ar}{\verboinum{1}}
\verb{aliás}{}{}{}{}{adv.}{Além disso.}{a.li.ás}{0}
\verb{aliás}{}{}{}{}{}{De outra maneira; do contrário.}{a.li.ás}{0}
\verb{aliás}{}{}{}{}{}{No entanto; contudo; apesar disso.}{a.li.ás}{0}
\verb{aliás}{}{}{}{}{}{A propósito; verdade seja dita.}{a.li.ás}{0}
\verb{álibi}{}{Jur.}{}{}{s.m.}{Defesa apresentada pelo réu para provar sua presença, no momento do delito, em lugar diferente daquele onde o delito ocorreu.}{á.li.bi}{0}
\verb{álibi}{}{Pop.}{}{}{}{Justificativa aceitável.}{á.li.bi}{0}
\verb{alicate}{}{}{}{}{s.m.}{Ferramenta própria para prender, segurar ou cortar certos objetos, composta de duas alavancas de ferro ou aço unidas por um eixo.}{a.li.ca.te}{0}
\verb{alicate}{}{Pop.}{}{}{}{Aguardente de cana; cachaça.}{a.li.ca.te}{0}
\verb{alicerçar}{}{}{}{}{v.t.}{Construir ou colocar o(s) alicerce(s).}{a.li.cer.çar}{0}
\verb{alicerçar}{}{}{}{}{}{Fundamentar, basear; tornar firme; consolidar.}{a.li.cer.çar}{\verboinum{3}}
\verb{alicerce}{é}{}{}{}{s.m.}{Maciço de alvenaria enterrado que serve de apoio para as estruturas de uma construção.}{a.li.cer.ce}{0}
\verb{alicerce}{é}{Fig.}{}{}{}{Aquilo que serve de base ou fundamento para qualquer coisa.}{a.li.cer.ce}{0}
\verb{aliciador}{ô}{}{}{}{adj.}{Que alicia.}{a.li.ci.a.dor}{0}
\verb{aliciamento}{}{}{}{}{s.m.}{Ato ou efeito de aliciar.}{a.li.ci.a.men.to}{0}
\verb{aliciar}{}{}{}{}{v.t.}{Atrair para si; seduzir; tornar cúmplice.}{a.li.ci.ar}{0}
\verb{aliciar}{}{}{}{}{}{Subornar, corromper.}{a.li.ci.ar}{0}
\verb{aliciar}{}{}{}{}{}{Incitar, instigar.}{a.li.ci.ar}{\verboinum{1}}
\verb{alienação}{}{}{"-ões}{}{s.f.}{Ato ou efeito de alienar.}{a.li.e.na.ção}{0}
\verb{alienação}{}{}{"-ões}{}{}{Transferência para outra pessoa de um bem ou direito.}{a.li.e.na.ção}{0}
\verb{alienação}{}{}{"-ões}{}{}{Estado de perturbação da mente.}{a.li.e.na.ção}{0}
\verb{alienação}{}{}{"-ões}{}{}{No marxismo, processo em que o ser humano se afasta de sua natureza por não ter controle sobre sua atividade essencial (o trabalho), pois as coisas que produz estão separadas do seu interesse e fora do seu alcance, transformadas, indistintamente, em mercadorias.}{a.li.e.na.ção}{0}
\verb{alienação}{}{Por ext.}{"-ões}{}{}{Indiferença aos problemas sociais e políticos.}{a.li.e.na.ção}{0}
\verb{alienado}{}{}{}{}{adj.}{Cedido ou transferido.}{a.li.e.na.do}{0}
\verb{alienado}{}{}{}{}{}{Louco, doido, desvairado.}{a.li.e.na.do}{0}
\verb{alienado}{}{}{}{}{s.m.}{Indivíduo que sofre de alienação, vivendo sem conhecer a realidade política e social que condiciona sua vida.}{a.li.e.na.do}{0}
\verb{alienante}{}{}{}{}{adj.2g.}{Que aliena.}{a.li.e.nan.te}{0}
\verb{alienante}{}{}{}{}{}{Que contribui para manter um indivíduo ou um grupo em estado de alienação, de ignorância da realidade objetiva do mundo.}{a.li.e.nan.te}{0}
\verb{alienar}{}{}{}{}{v.t.}{Transferir para outra pessoa o domínio ou a propriedade de algo.}{a.li.e.nar}{0}
\verb{alienar}{}{}{}{}{}{Renunciar; abandonar um direito ou um privilégio.}{a.li.e.nar}{0}
\verb{alienar}{}{Fig.}{}{}{v.pron.}{Tornar"-se louco; alucinar"-se, perturbar"-se.}{a.li.e.nar}{\verboinum{1}}
\verb{alienígena}{}{}{}{}{adj.2g.}{Que é natural de outro país; estrangeiro, forasteiro.}{a.li.e.ní.ge.na}{0}
\verb{alienígena}{}{Fig.}{}{}{}{Que pertence a outros mundos.}{a.li.e.ní.ge.na}{0}
\verb{alienista}{}{}{}{}{s.2g.}{Médico especialista em doenças mentais.}{a.li.e.nis.ta}{0}
\verb{alienista}{}{}{}{}{adj.2g.}{Relativo ao tratamento dos alienados.}{a.li.e.nis.ta}{0}
\verb{aliforme}{ó}{}{}{}{adj.2g.}{Que tem formato de asa; alar, ansiforme.}{a.li.for.me}{0}
\verb{aligátor}{}{Zool.}{}{}{s.m.}{Designação comum aos répteis cujas espécies têm focinho mais curto e largo que o dos jacarés e dos crocodilos; vivem em algumas regiões temperadas da América do Norte e da China.}{a.li.gá.tor}{0}
\verb{aligeirar}{}{}{}{}{v.t.}{Tornar"-se ligeiro; apressar"-se.}{a.li.gei.rar}{0}
\verb{aligeirar}{}{}{}{}{}{Mitigar, atenuar, aliviar.}{a.li.gei.rar}{0}
\verb{aligeirar}{}{}{}{}{v.pron.}{Livrar"-se, aliviar"-se, desembaraçar"-se.}{a.li.gei.rar}{\verboinum{1}}
\verb{alijar}{}{}{}{}{v.t.}{Lançar carga ao mar para aliviar o navio.}{a.li.jar}{0}
\verb{alijar}{}{Por ext.}{}{}{}{Livrar; lançar fora.}{a.li.jar}{0}
\verb{alijar}{}{}{}{}{}{Desconhecer, negar (dever, responsabilidade, compromisso etc.).}{a.li.jar}{\verboinum{1}}
\verb{alimária}{}{}{}{}{s.f.}{Qualquer animal, especialmente quadrúpede.}{a.li.má.ria}{0}
\verb{alimária}{}{}{}{}{}{Besta de carga.}{a.li.má.ria}{0}
\verb{alimária}{}{Fig.}{}{}{}{Pessoa estúpida e grosseira.}{a.li.má.ria}{0}
\verb{alimentação}{}{}{"-ões}{}{s.f.}{Ato ou efeito de alimentar"-se.}{a.li.men.ta.ção}{0}
\verb{alimentação}{}{}{"-ões}{}{}{Conjunto de substâncias necessárias à conservação da vida.}{a.li.men.ta.ção}{0}
\verb{alimentação}{}{Por ext.}{"-ões}{}{}{Ato de fornecer a alguma coisa o necessário ao seu funcionamento.}{a.li.men.ta.ção}{0}
\verb{alimentar}{}{}{}{}{v.t.}{Dar alimento; nutrir.}{a.li.men.tar}{0}
\verb{alimentar}{}{Fig.}{}{}{}{Dar sustento ou sustentar"-se.}{a.li.men.tar}{0}
\verb{alimentar}{}{Fig.}{}{}{}{Fornecer a alguma coisa o necessário ao seu funcionamento.}{a.li.men.tar}{\verboinum{1}}
\verb{alimentar}{}{}{}{}{adj.2g.}{Relativo a alimento ou alimentação.}{a.li.men.tar}{0}
\verb{alimentício}{}{}{}{}{adj.}{Próprio para alimentar.}{a.li.men.tí.cio}{0}
\verb{alimentício}{}{}{}{}{}{Que alimenta.}{a.li.men.tí.cio}{0}
\verb{alimentício}{}{}{}{}{}{Relativo a alimento ou alimentação.}{a.li.men.tí.cio}{0}
\verb{alimento}{}{}{}{}{s.m.}{Toda substância que serve para alimentar um ser vivo.}{a.li.men.to}{0}
\verb{alimento}{}{Por ext.}{}{}{}{Aquilo que mantém, sustenta, faz subsistir.}{a.li.men.to}{0}
\verb{alimento}{}{Fig.}{}{}{}{Aquilo que serve para desenvolver as faculdades intelectuais e morais.}{a.li.men.to}{0}
\verb{alimpar}{}{}{}{}{}{Var. de \textit{limpar}.}{a.lim.par}{0}
\verb{alindar}{}{}{}{}{v.t.}{Tornar lindo, enfeitado; adornar.}{a.lin.dar}{\verboinum{1}}
\verb{alínea}{}{}{}{}{s.f.}{Primeira linha de um novo parágrafo.}{a.lí.nea}{0}
\verb{alínea}{}{Por ext.}{}{}{}{Cada uma das subdivisões de um artigo de lei, decreto, contrato, estatuto e similares, indicada por letra ou número.}{a.lí.nea}{0}
\verb{alinhado}{}{}{}{}{adj.}{Que se encontra em linha reta, em fila, enfileirado.}{a.li.nha.do}{0}
\verb{alinhado}{}{}{}{}{}{Trajado com apuro; elegante.}{a.li.nha.do}{0}
\verb{alinhado}{}{}{}{}{}{Que é correto nas maneiras ou no agir.}{a.li.nha.do}{0}
\verb{alinhamento}{}{}{}{}{s.m.}{Ato ou efeito de alinhar"-se.}{a.li.nha.men.to}{0}
\verb{alinhamento}{}{Fig.}{}{}{}{Esmero, correção, apuro.}{a.li.nha.men.to}{0}
\verb{alinhamento}{}{}{}{}{}{Direção do eixo de estrada, canal, pista, construção.}{a.li.nha.men.to}{0}
\verb{alinhamento}{}{}{}{}{}{Disposição das linhas de um texto de modo que suas extremidades ou seu centro formem uma linha imaginária vertical.}{a.li.nha.men.to}{0}
\verb{alinhamento}{}{Astron.}{}{}{}{Conjunção simultânea de três ou mais planetas do sistema solar, formando uma linha imaginária.}{a.li.nha.men.to}{0}
\verb{alinhamento}{}{Econ.}{}{}{}{Equiparação de valores.}{a.li.nha.men.to}{0}
\verb{alinhar}{}{}{}{}{v.t.}{Dispor em linha reta; enfileirar.}{a.li.nhar}{0}
\verb{alinhar}{}{Fig.}{}{}{}{Arrumar, enfeitar.}{a.li.nhar}{0}
\verb{alinhar}{}{}{}{}{}{Dispor as linhas de um texto de modo que suas extremidades ou seu centro formem uma linha imaginária vertical.}{a.li.nhar}{0}
\verb{alinhar}{}{}{}{}{v.pron.}{Aderir a determinado grupo, causa ou ideologia.}{a.li.nhar}{0}
\verb{alinhar}{}{Fig.}{}{}{}{Equiparar"-se, pôr"-se no mesmo nível.}{a.li.nhar}{\verboinum{1}}
\verb{alinhavar}{}{}{}{}{v.t.}{Fazer uma costura provisória.}{a.li.nha.var}{0}
\verb{alinhavar}{}{Por ext.}{}{}{}{Traçar as linhas gerais de; esboçar.}{a.li.nha.var}{0}
\verb{alinhavar}{}{}{}{}{}{Executar imperfeitamente, às pressas; improvisar.}{a.li.nha.var}{\verboinum{1}}
\verb{alinhavo}{}{}{}{}{s.m.}{Costura provisória.}{a.li.nha.vo}{0}
\verb{alinhavo}{}{Fig.}{}{}{}{Esboço, arranjo.}{a.li.nha.vo}{0}
\verb{alinho}{}{}{}{}{s.m.}{Alinhamento.}{a.li.nho}{0}
\verb{alinho}{}{Fig.}{}{}{}{Esmero, apuro.}{a.li.nho}{0}
\verb{alinho}{}{Fig.}{}{}{}{Enfeite, ornato.}{a.li.nho}{0}
\verb{alíquota}{}{}{}{}{s.f.}{Percentual de imposto sobre um bem ou um direito tributado.}{a.lí.quo.ta}{0}
\verb{alíquota}{}{Mat.}{}{}{}{Divisor que produz como resultado um número inteiro, sem resto.}{a.lí.quo.ta}{0}
\verb{alisar}{}{}{}{}{v.t.}{Tornar liso ou plano; desenrugar, aplanar, desencrespar.}{a.li.sar}{0}
\verb{alisar}{}{}{}{}{}{Passar a mão sobre; fazer carícia.}{a.li.sar}{0}
\verb{alisar}{}{Pop.}{}{}{}{Agradar, bajular, proteger.}{a.li.sar}{\verboinum{1}}
\verb{aliseu}{}{}{}{}{adj.}{Alísio.}{a.li.seu}{0}
\verb{alísio}{}{}{}{}{adj.}{Diz"-se de um vento que sopra sobre extensas regiões do globo terrestre, das regiões subtropicais em direção às regiões equatoriais; aliseu.}{a.lí.sio}{0}
\verb{alistado}{}{}{}{}{adj.}{Inscrito em lista; inventariado, relacionado.}{a.lis.ta.do}{0}
\verb{alistamento}{}{}{}{}{s.m.}{Ato ou efeito de alistar.}{a.lis.ta.men.to}{0}
\verb{alistamento}{}{}{}{}{}{Inscrição perante uma autoridade pública a fim de exercer um direito ou cumprir um dever.}{a.lis.ta.men.to}{0}
\verb{alistamento}{}{}{}{}{}{Recrutamento para o serviço militar.}{a.lis.ta.men.to}{0}
\verb{alistar}{}{}{}{}{v.t.}{Colocar em lista; relacionar.}{a.lis.tar}{0}
\verb{alistar}{}{}{}{}{}{Afiliar"-se, inscrever"-se.}{a.lis.tar}{0}
\verb{alistar}{}{}{}{}{}{Inscrever"-se para o serviço militar.}{a.lis.tar}{\verboinum{1}}
\verb{aliteração}{}{Gram.}{"-ões}{}{s.f.}{Recurso utilizado na poesia e na prosa poética, no qual se repetem sons foneticamente parecidos em palavras da mesma frase ou verso, visando obter efeito estilístico.}{a.li.te.ra.ção}{0}
\verb{aliviar}{}{}{}{}{v.t.}{Dar alívio; acalmar, tranquilizar.}{a.li.vi.ar}{0}
\verb{aliviar}{}{}{}{}{}{Tornar mais leve; descarregar.}{a.li.vi.ar}{0}
\verb{aliviar}{}{}{}{}{}{Diminuir a intensidade.}{a.li.vi.ar}{\verboinum{1}}
\verb{alívio}{}{}{}{}{s.m.}{Ato ou efeito de aliviar.}{a.lí.vio}{0}
\verb{alívio}{}{}{}{}{}{Descanso, desafogo, tranquilidade, refrigério.}{a.lí.vio}{0}
\verb{alívio}{}{}{}{}{}{Consolo.}{a.lí.vio}{0}
\verb{alizar}{}{}{}{}{s.m.}{Guarnição de madeira que reveste as ombreiras de portas e janelas.}{a.li.zar}{0}
\verb{aljava}{}{}{}{}{s.f.}{Estojo carregado nas costas utilizado para guardar as setas.}{al.ja.va}{0}
\verb{aljôfar}{}{}{}{}{s.m.}{Pérola miúda.}{al.jô.far}{0}
\verb{aljôfar}{}{Fig.}{}{}{}{Gota de água; orvalho da manhã; lágrima.}{al.jô.far}{0}
\verb{alma}{}{}{}{}{s.f.}{Princípio da vida; espírito.}{al.ma}{0}
\verb{alma}{}{}{}{}{}{Essência.}{al.ma}{0}
\verb{alma}{}{}{}{}{}{Sentimento.}{al.ma}{0}
\verb{almaço}{}{}{}{}{adj.}{Diz"-se de um tipo de papel próprio para registros e documentos, com formato determinado.}{al.ma.ço}{0}
\verb{almanaque}{}{}{}{}{s.m.}{Publicação periódica de conteúdo informativo, recreativo, humorístico, literário, científico etc.}{al.ma.na.que}{0}
\verb{almeirão}{}{}{"-ões}{}{s.m.}{Verdura de folhas largas e sabor amargo, própria para salada; chicória.}{al.mei.rão}{0}
\verb{almejar}{}{}{}{}{v.t.}{Desejar ardentemente; ansiar.}{al.me.jar}{0}
\verb{almejar}{}{}{}{}{v.i.}{Estar prestes a morrer.}{al.me.jar}{\verboinum{1}}
\verb{almejo}{ê}{}{}{}{s.m.}{Desejo ardente, ansioso.}{al.me.jo}{0}
\verb{almenara}{}{}{}{}{s.f.}{Sinal luminoso colocado em lugares elevados para que sejam vistos a distância.}{al.me.na.ra}{0}
\verb{almirantado}{}{}{}{}{s.m.}{Situação ou dignidade de almirante.}{al.mi.ran.ta.do}{0}
\verb{almirantado}{}{}{}{}{}{Conjunto de oficiais superiores da Marinha.}{al.mi.ran.ta.do}{0}
\verb{almirante}{}{}{}{}{s.m.}{Posto mais alto da Marinha.}{al.mi.ran.te}{0}
\verb{almirante}{}{}{}{}{}{Variedade de pera.}{al.mi.ran.te}{0}
\verb{almirante}{}{Zool.}{}{}{}{Espécie de borboleta que tem asas marrons com desenhos avermelhados e brancos.}{al.mi.ran.te}{0}
\verb{almíscar}{}{}{}{}{s.m.}{Substância de odor persistente, extraída de uma glândula localizada no abdômen do almiscareiro e utilizada como fixador em perfumaria.}{al.mís.car}{0}
\verb{almíscar}{}{Por ext.}{}{}{}{Odor muito forte.}{al.mís.car}{0}
\verb{almiscarado}{}{}{}{}{adj.}{Perfumado com almíscar.}{al.mis.ca.ra.do}{0}
\verb{almiscarar}{}{}{}{}{v.t.}{Perfumar com almíscar.}{al.mis.ca.rar}{\verboinum{1}}
\verb{almiscareiro}{ê}{Zool.}{}{}{s.m.}{Mamífero ruminante de pequeno porte, que possui uma glândula no abdome da qual é extraído o almíscar.}{al.mis.ca.rei.ro}{0}
\verb{almoçar}{}{}{}{}{v.i.}{ Alimentar"-se na refeição do início da tarde.}{al.mo.çar}{0}
\verb{almoçar}{}{}{}{}{v.t.}{Comer algo no almoço.}{al.mo.çar}{\verboinum{3}}
\verb{almoço}{ô}{}{}{}{s.m.}{Refeição substancial usualmente feita no princípio da tarde.}{al.mo.ço}{0}
\verb{almoço}{ô}{}{}{}{}{A comida que constitui essa refeição.}{al.mo.ço}{0}
\verb{almocreve}{é}{}{}{}{s.m.}{Indivíduo que conduz as bestas de carga.}{al.mo.cre.ve}{0}
\verb{almofada}{}{}{}{}{s.f.}{Saco estofado utilizado para encosto, assento ou ornato.}{al.mo.fa.da}{0}
\verb{almofada}{}{}{}{}{}{Caixa com superfície interna acolchoada para a tintagem dos carimbos.}{al.mo.fa.da}{0}
\verb{almofadado}{}{}{}{}{adj.}{Que tem almofada(s); forrado ou estofado com superfície macia.}{al.mo.fa.da.do}{0}
\verb{almofadinha}{}{}{}{}{s.f.}{Almofada pequena utilizada para guardar alfinetes e agulhas.}{al.mo.fa.di.nha}{0}
\verb{almofadinha}{}{Bras.}{}{}{}{Homem que se veste com apuro demasiado; janota, dândi.}{al.mo.fa.di.nha}{0}
\verb{almofariz}{}{}{}{}{s.m.}{Recipiente utilizado para triturar substâncias sólidas; pilão.}{al.mo.fa.riz}{0}
\verb{almôndega}{}{Cul.}{}{}{s.f.}{Bolinho feito de carne moída, miolo de pão e temperos, que é frito e geralmente servido com molhos.}{al.môn.de.ga}{0}
\verb{almotolia}{}{}{}{}{s.f.}{Recipiente de forma cônica e gargalo estreito utilizado para azeite e óleos.}{al.mo.to.li.a}{0}
\verb{almotolia}{}{}{}{}{}{Dispositivo para lubrificar pequenos mecanismos.}{al.mo.to.li.a}{0}
\verb{almoxarifado}{ch}{}{}{}{s.m.}{Depósito de objetos, materiais e matérias"-primas.}{al.mo.xa.ri.fa.do}{0}
\verb{almoxarifado}{ch}{}{}{}{}{Função ou área de jurisdição do almoxarife.}{al.mo.xa.ri.fa.do}{0}
\verb{almoxarife}{ch}{}{}{}{s.m.}{Indivíduo responsável pelo almoxarifado.}{al.mo.xa.ri.fe}{0}
\verb{almoxarife}{ch}{}{}{}{}{Funcionário que era encarregado de administrar as propriedades da casa real.}{al.mo.xa.ri.fe}{0}
\verb{almuadem}{}{}{"-ens}{}{s.m.}{Mouro que anuncia a hora das preces.}{al.mu.a.dem}{0}
\verb{alô}{}{}{}{}{interj.}{Saudação, especialmente ao telefone.}{a.lô}{0}
\verb{alô}{}{}{}{}{s.m.}{Cumprimento em que se diz \textit{alô}.}{a.lô}{0}
\verb{alô}{}{Bras.}{}{}{}{Tipo de narrativa oral trazida ao Brasil pelos iorubas durante a escravidão.}{a.lô}{0}
\verb{alocação}{}{}{"-ões}{}{s.f.}{Ato ou efeito de alocar.}{a.lo.ca.ção}{0}
\verb{alocar}{}{}{}{}{v.t.}{Colocar algo ou alguém num ponto determinado de uma sequência de lugares.}{a.lo.car}{0}
\verb{alocar}{}{Econ.}{}{}{}{Destinar verbas ou recursos a um fim específico.}{a.lo.car}{\verboinum{2}}
\verb{alóctone}{}{}{}{}{adj.2g.}{Que não é originário do país ou região onde habita.}{a.lóc.to.ne}{0}
\verb{alocução}{}{}{"-ões}{}{s.f.}{Discurso breve, geralmente proferido em ocasião solene.}{a.lo.cu.ção}{0}
\verb{aloés}{}{Bot.}{}{}{s.m.}{Planta suculenta, medicinal e ornamental; babosa; agave.}{a.lo.és}{0}
\verb{aloés}{}{}{}{}{}{A resina dessa planta, cicatrizante e emoliente.}{a.lo.és}{0}
\verb{alogia}{}{}{}{}{s.f.}{Falta de lógica; absurdo, contrassenso, despropósito. }{a.lo.gi.a}{0}
\verb{alógico}{}{}{}{}{adj.}{Aquilo que é alógico.}{a.ló.gi.co}{0}
\verb{aloirado}{}{}{}{}{adj.}{Que é um tanto loiro, ou tornado loiro.}{a.loi.ra.do}{0}
\verb{aloirado}{}{Cul.}{}{}{}{Dourado ao fogo; bem assado.}{a.loi.ra.do}{0}
\verb{aloirar}{}{}{}{}{v.t.}{Tornar loiro ou um tanto loiro.}{a.loi.rar}{0}
\verb{aloirar}{}{}{}{}{}{Dourar ao fogo.}{a.loi.rar}{\verboinum{1}}
\verb{alojamento}{}{}{}{}{s.m.}{Ato ou efeito de alojar, de pôr em loja.}{a.lo.ja.men.to}{0}
\verb{alojamento}{}{}{}{}{}{Lugar onde se mora temporariamente; aposento, hospedaria.}{a.lo.ja.men.to}{0}
\verb{alojamento}{}{}{}{}{}{Local onde alguém ou algo se aloja.}{a.lo.ja.men.to}{0}
\verb{alojamento}{}{}{}{}{}{Lugar em que se aquartelam tropas; quartel, aboletamento.}{a.lo.ja.men.to}{0}
\verb{alojar}{}{}{}{}{v.t.}{Pôr ou guardar em loja; acomodar.}{a.lo.jar}{0}
\verb{alojar}{}{}{}{}{}{Hospedar alguém por tempo determinado ou permanente; abrigar.}{a.lo.jar}{0}
\verb{alojar}{}{}{}{}{}{Aquartelar, aboletar.}{a.lo.jar}{0}
\verb{alojar}{}{}{}{}{}{Depositar, instalar, armazenar.}{a.lo.jar}{\verboinum{1}}
\verb{alongado}{}{}{}{}{adj.}{Que se tornou longo; comprido.}{a.lon.ga.do}{0}
\verb{alongado}{}{}{}{}{}{Diz"-se de prazo que se prolongou.}{a.lon.ga.do}{0}
\verb{alongado}{}{}{}{}{}{Que foi ou está estendido; esticado.}{a.lon.ga.do}{0}
\verb{alongado}{}{}{}{}{}{Muito afastado; distante.}{a.lon.ga.do}{0}
\verb{alongamento}{}{}{}{}{s.m.}{Ato ou efeito de alongar, de tornar longo.}{a.lon.ga.men.to}{0}
\verb{alongamento}{}{}{}{}{}{Forma de exercício físico para distender os músculos.}{a.lon.ga.men.to}{0}
\verb{alongamento}{}{}{}{}{}{Aumento de comprimento; prolongamento.}{a.lon.ga.men.to}{0}
\verb{alongamento}{}{}{}{}{}{Aumento de duração; demora.}{a.lon.ga.men.to}{0}
\verb{alongar}{}{}{}{}{v.t.}{Tornar longo, ou mais longo; encompridar.}{a.lon.gar}{0}
\verb{alongar}{}{}{}{}{}{Estender, estirar.}{a.lon.gar}{0}
\verb{alongar}{}{}{}{}{}{Tornar distanciado; afastar.}{a.lon.gar}{0}
\verb{alongar}{}{}{}{}{}{Dirigir para longe; olhar à distância.}{a.lon.gar}{\verboinum{5}}
\verb{alopata}{}{}{}{}{s.2g.}{Indivíduo que exerce a alopatia ou dela se utiliza.}{a.lo.pa.ta}{0}
\verb{alopatia}{}{Med.}{}{}{s.f.}{Sistema terapêutico em que se empregam remédios para tratar doenças.  }{a.lo.pa.ti.a}{0}
\verb{alopático}{}{}{}{}{adj.}{Relativo ou pertencente a alopatia ou a alopata.}{a.lo.pá.ti.co}{0}
\verb{alopecia}{}{Med.}{}{}{s.f.}{Queda ou ausência de pelos ou cabelos.}{a.lo.pe.ci.a}{0}
\verb{alopécico}{}{}{}{}{adj.}{Relativo à alopecia.}{a.lo.pé.ci.co}{0}
\verb{alopécico}{}{}{}{}{}{Que sofre de alopecia.}{a.lo.pé.ci.co}{0}
\verb{alopécico}{}{}{}{}{s.m.}{Indivíduo que tem alopecia.}{a.lo.pé.ci.co}{0}
\verb{aloprado}{}{}{}{}{adj.}{Que é muito agitado, muito inquieto.}{a.lo.pra.do}{0}
\verb{aloprado}{}{}{}{}{s.m.}{Indivíduo amalucado, adoidado.}{a.lo.pra.do}{0}
\verb{alotropia}{}{Quím.}{}{}{s.f.}{Particularidade que alguns elementos químicos têm de se apresentarem com propriedades físicas distintas, decorrentes das diferenças na estrutura espacial da molécula. }{a.lo.tro.pi.a}{0}
\verb{aloucado}{}{}{}{}{adj.}{Que tem tendência à loucura; insano, amalucado, enlouquecido.}{a.lou.ca.do}{0}
\verb{alourado}{}{}{}{}{}{Var. de \textit{aloirado}.}{a.lou.ra.do}{0}
\verb{alourar}{}{}{}{}{}{Var. de \textit{aloirar}.}{a.lou.rar}{0}
\verb{alpaca}{}{Zool.}{}{}{s.f.}{Mamífero ruminante, menor que a lhama, de cabeça pequena, pescoço comprido, pelagem longa e lanosa. É encontrado, sob domesticação, na Bolívia e no Peru, e possui importância econômica.}{al.pa.ca}{0}
\verb{alpaca}{}{}{}{}{}{A lã desse animal.}{al.pa.ca}{0}
\verb{alpaca}{}{}{}{}{}{O tecido feito dessa lã.}{al.pa.ca}{0}
\verb{alpaca}{}{Quím.}{}{}{s.f.}{Espécie de liga metálica de zinco, cobre, níquel e prata.}{al.pa.ca}{0}
\verb{alpargata}{}{}{}{}{s.f.}{Calçado de pano com sola de corda ou palha.}{al.par.ga.ta}{0}
\verb{alpargata}{}{}{}{}{}{Sandália com sola de borracha, couro ou outro material, presa ao pé por tiras.}{al.par.ga.ta}{0}
\verb{alpendrado}{}{}{}{}{adj.}{Que tem a forma ou o aspecto de alpendre.}{al.pen.dra.do}{0}
\verb{alpendrado}{}{}{}{}{}{Que tem alpendre; recoberto com alpendre.}{al.pen.dra.do}{0}
\verb{alpendre}{}{}{}{}{s.m.}{Cobertura saliente, inclinada, de uma só água, geralmente à entrada de um prédio, com a parte mais baixa suspensa por colunas ou pilastras, e a parte mais alta apoiada à parede do edifício.}{al.pen.dre}{0}
\verb{alpendre}{}{}{}{}{}{Pátio coberto.}{al.pen.dre}{0}
\verb{alpendre}{}{}{}{}{}{Varanda coberta.}{al.pen.dre}{0}
\verb{alpercata}{}{}{}{}{}{Var. de \textit{alpargata}.}{al.per.ca.ta}{0}
\verb{alpergata}{}{}{}{}{}{Var. de \textit{alpargata}.}{al.per.ga.ta}{0}
\verb{alpestre}{é}{}{}{}{adj.2g.}{Relativo aos Alpes; alpino.}{al.pes.tre}{0}
\verb{alpestre}{é}{}{}{}{}{Que tem aspecto pedregoso, áspero; rochoso.}{al.pes.tre}{0}
\verb{alpinismo}{}{}{}{}{s.m.}{Esporte que consiste em escalar montanhas, montes, picos, rochas escarpadas etc.; montanhismo.}{al.pi.nis.mo}{0}
\verb{alpinista}{}{}{}{}{adj.2g.}{Relativo a alpinismo.}{al.pi.nis.ta}{0}
\verb{alpinista}{}{}{}{}{s.2g.}{Indivíduo que pratica o alpinismo.}{al.pi.nis.ta}{0}
\verb{alpino}{}{}{}{}{adj.}{Relativo ou pertencente aos Alpes, cadeia montanhosa da Europa.}{al.pi.no}{0}
\verb{alpista}{}{}{}{}{}{Var. de \textit{alpiste}.}{al.pis.ta}{0}
\verb{alpiste}{}{}{}{}{s.m.}{Planta das gramíneas.}{al.pis.te}{0}
\verb{alpiste}{}{}{}{}{}{Grãos dessa planta utilizados como alimento para pássaros.}{al.pis.te}{0}
\verb{alquebrado}{}{}{}{}{adj.}{Que se apresenta abatido, cansado.}{al.que.bra.do}{0}
\verb{alquebrado}{}{}{}{}{}{Que anda curvado, por velhice, cansaço ou doença.}{al.que.bra.do}{0}
\verb{alquebramento}{}{}{}{}{s.m.}{Esgotamento de forças; enfraquecimento, abatimento.}{al.que.bra.men.to}{0}
\verb{alquebramento}{}{}{}{}{}{Ato ou efeito de alquebrar, dobrar, curvar.}{al.que.bra.men.to}{0}
\verb{alquebrar}{}{}{}{}{v.t.}{Tornar abatido; enfraquecer, prostrar.}{al.que.brar}{\verboinum{1}}
\verb{alqueire}{}{}{}{}{s.m.}{Unidade de medida de superfície agrária equivalente, em Minas Gerais, Rio de Janeiro e Goiás, a 4,84 hectares, e, em São Paulo, a 2,42 hectares.}{al.quei.re}{0}
\verb{alquimia}{}{}{}{}{s.f.}{A química da Antiguidade e da Idade Média, cujo objetivo era descobrir o remédio contra todos os males físicos e morais, e a pedra filosofal, que deveria transmutar os metais em ouro. }{al.qui.mi.a}{0}
\verb{alquimista}{}{}{}{}{s.2g.}{Indivíduo que se dedica à alquimia.}{al.qui.mis.ta}{0}
\verb{alta}{}{}{}{}{adj.}{Subida de preços ou de cotação.}{al.ta}{0}
\verb{alta}{}{}{}{}{}{A parte mais elevada de uma cidade, vila etc.}{al.ta}{0}
\verb{alta}{}{}{}{}{}{Fem. de \textit{alto}.}{al.ta}{0}
\verb{alta}{}{}{}{}{s.f.}{Autorização médica que declara terminado um tratamento ou uma internação.}{al.ta}{0}
\verb{alta}{}{Ant.}{}{}{}{baixa}{al.ta}{0}
\verb{alta"-costura}{}{}{alta"-costuras}{}{s.f.}{A arte de criar e confeccionar roupas sofisticadas e de valor elevado.}{al.ta"-cos.tu.ra}{0}
\verb{alta"-costura}{}{Por ext.}{alta"-costuras}{}{}{O conjunto dos grandes costureiros, dos mais famosos.}{al.ta"-cos.tu.ra}{0}
\verb{alta"-costura}{}{}{alta"-costuras}{}{}{Cada uma das roupas criadas por esses costureiros.}{al.ta"-cos.tu.ra}{0}
\verb{alta"-fidelidade}{}{}{altas"-fidelidades}{}{s.f.}{Conjunto de técnicas eletrônicas que permite a gravação e a reprodução de sons sem distorções consideráveis e com baixos níveis de ruído.}{al.ta"-fi.de.li.da.de}{0}
\verb{altanaria}{}{}{}{}{s.f.}{Capacidade que algumas aves têm de voar alto.}{al.ta.na.ri.a}{0}
\verb{altanaria}{}{}{}{}{}{Caça às aves que voam alto.}{al.ta.na.ri.a}{0}
\verb{altanaria}{}{Fig.}{}{}{}{Altivez, orgulho, soberba.}{al.ta.na.ri.a}{0}
\verb{altaneiro}{ê}{}{}{}{adj.}{Que se eleva muito.}{al.ta.nei.ro}{0}
\verb{altaneiro}{ê}{}{}{}{}{Que voa muito alto.}{al.ta.nei.ro}{0}
\verb{altaneiro}{ê}{}{}{}{}{Que é cheio de altivez; soberbo, altanado.}{al.ta.nei.ro}{0}
\verb{altar}{}{}{}{}{s.m.}{Espécie de mesa destinada a sacrifícios ou imolações, em diversas religiões.}{al.tar}{0}
\verb{altar}{}{}{}{}{}{Pedra retangular, com o formato de uma mesa, onde o sacerdote celebra a missa.}{al.tar}{0}
\verb{altar"-mor}{ó}{}{altares"-mores ⟨ó⟩}{}{s.m.}{O altar principal de uma igreja.}{al.tar"-mor}{0}
\verb{alta"-roda}{ó}{}{altas"-rodas ⟨ó⟩}{}{s.f.}{Alta sociedade; círculo social elevado.}{al.ta"-ro.da}{0}
\verb{altear}{}{}{}{}{v.t.}{Tornar alto ou mais alto; erguer, elevar.}{al.te.ar}{0}
\verb{altear}{}{}{}{}{}{Tornar algo mais sublime.}{al.te.ar}{0}
\verb{altear}{}{}{}{}{v.i.}{Crescer em volume; inchar.}{al.te.ar}{0}
\verb{altear}{}{}{}{}{v.pron.}{Subir de posto, posição social; elevar"-se.}{al.te.ar}{\verboinum{4}}
\verb{alteração}{}{}{ões}{}{s.f.}{Ato ou efeito de alterar"-se, de mudar; modificação.}{al.te.ra.ção}{0}
\verb{alteração}{}{}{ões}{}{}{Mudança das características habituais, produzindo um estado de deterioração; degeneração, decomposição.}{al.te.ra.ção}{0}
\verb{alteração}{}{}{ões}{}{}{Ato de falsificar. }{al.te.ra.ção}{0}
\verb{alteração}{}{}{ões}{}{}{Excitação, indignação, desassossego.}{al.te.ra.ção}{0}
\verb{alterado}{}{}{}{}{adj.}{Que sofreu alteração; modificado.}{al.te.ra.do}{0}
\verb{alterado}{}{}{}{}{}{Que se adulterou; falsificado.}{al.te.ra.do}{0}
\verb{alterado}{}{}{}{}{}{Que se encontra em estado de decomposição ou deterioração.}{al.te.ra.do}{0}
\verb{alterado}{}{}{}{}{}{Desassossegado, inquieto.}{al.te.ra.do}{0}
\verb{alterar}{}{}{}{}{v.t.}{Causar ou sofrer mudança ou alteração; modificar.}{al.te.rar}{0}
\verb{alterar}{}{}{}{}{}{Tirar as características originais; falsificar.}{al.te.rar}{0}
\verb{alterar}{}{}{}{}{}{Fazer com que se decomponha; deteriorar.}{al.te.rar}{0}
\verb{alterar}{}{Fig.}{}{}{}{Causar perturbação; desassossegar, inquietar.}{al.te.rar}{\verboinum{1}}
\verb{altercação}{}{}{"-ões}{}{s.f.}{Ato ou efeito de altercar, de discutir com ardor; contenda, discussão.}{al.ter.ca.ção}{0}
\verb{altercar}{}{}{}{}{v.i.}{Discutir com ardor; polemizar.}{al.ter.car}{0}
\verb{altercar}{}{}{}{}{}{Defender em polêmica.}{al.ter.car}{\verboinum{2}}
\verb{alteridade}{}{}{}{}{s.f.}{Qualidade ou caráter daquilo ou daquele que é outro, distinto, enquanto ser.}{al.te.ri.da.de}{0}
\verb{alternação}{}{}{"-ões}{}{s.f.}{Ato ou efeito de alternar, de fazer suceder repetida e regularmente; alternância.}{al.ter.na.ção}{0}
\verb{alternado}{}{}{}{}{adj.}{Disposto com alternância, que sucede cada qual por sua vez; revezado.}{al.ter.na.do}{0}
\verb{alternador}{ô}{}{}{}{adj.}{Que alterna, reveza.}{al.ter.na.dor}{0}
\verb{alternador}{ô}{}{}{}{s.m.}{O que alterna, intercala.}{al.ter.na.dor}{0}
\verb{alternador}{ô}{}{}{}{}{Aparelho elétrico capaz de fornecer corrente alternada.}{al.ter.na.dor}{0}
\verb{alternância}{}{}{}{}{s.f.}{Ato ou efeito de alternar; alternação.}{al.ter.nân.cia}{0}
\verb{alternância}{}{}{}{}{}{Repetição de dois motivos ou objetos diferentes, sempre obedecendo à mesma ordem.}{al.ter.nân.cia}{0}
\verb{alternar}{}{}{}{}{v.t.}{Fazer suceder repetidamente; suceder em alternância.}{al.ter.nar}{0}
\verb{alternar}{}{}{}{}{}{Dispor em ordem alternada; pôr de permeio; revezar.}{al.ter.nar}{\verboinum{1}}
\verb{alternativa}{}{}{}{}{s.f.}{Sucessão de coisas que se repetem com alternação, cada coisa por sua vez.}{al.ter.na.ti.va}{0}
\verb{alternativa}{}{}{}{}{}{Opção entre duas ou mais coisas.}{al.ter.na.ti.va}{0}
\verb{alternativa}{}{}{}{}{}{Na lógica, refere"-se a um sistema de duas proposições, das quais apenas uma é verdadeira.}{al.ter.na.ti.va}{0}
\verb{alternativo}{}{}{}{}{adj.}{Que se diz, faz ou ocorre com alternação.}{al.ter.na.ti.vo}{0}
\verb{alternativo}{}{}{}{}{}{Que permite escolha, opção.}{al.ter.na.ti.vo}{0}
\verb{alternativo}{}{Pop.}{}{}{s.m.}{Pessoa que se opõe a valores, costumes e ideias impostos pela sociedade.}{al.ter.na.ti.vo}{0}
\verb{alterno}{é}{}{}{}{adj.}{Que ocorre de modo alternado; revezado.}{al.ter.no}{0}
\verb{alteroso}{ô}{}{"-osos ⟨ó⟩}{"-osa ⟨ó⟩}{adj.}{De grande altura; elevado.}{al.te.ro.so}{0}
\verb{alteroso}{ô}{Fig.}{"-osos ⟨ó⟩}{"-osa ⟨ó⟩}{}{Cheio de altivez; soberbo.}{al.te.ro.so}{0}
\verb{alteza}{ê}{}{}{}{s.f.}{Tratamento dado a príncipes.}{al.te.za}{0}
\verb{alteza}{ê}{Desus.}{}{}{}{Qualidade do que é alto; altura.}{al.te.za}{0}
\verb{alteza}{ê}{Desus.}{}{}{}{Elevação moral; grandeza, nobreza.}{al.te.za}{0}
\verb{altibaixos}{ch}{}{}{}{s.m.pl.}{Irregularidades do terreno acidentado; desnivelamentos.}{al.ti.bai.xos}{0}
\verb{altibaixos}{ch}{Fig.}{}{}{}{Contratempos, vicissitudes, altos e baixos.}{al.ti.bai.xos}{0}
\verb{altiloquência}{}{}{}{}{s.f.}{Estilo grandioso de se expressar verbalmente; alta eloquência.}{al.ti.lo.quên.cia}{0}
\verb{altimetria}{}{}{}{}{s.f.}{Técnica utilizada na medição de altitudes de um terreno.}{al.ti.me.tri.a}{0}
\verb{altimetria}{}{}{}{}{}{Representação dessas altitudes em uma planta topográfica.}{al.ti.me.tri.a}{0}
\verb{altímetro}{}{}{}{}{s.m.}{Instrumento usado para medir altitudes.}{al.tí.me.tro}{0}
\verb{altiplano}{}{Geol.}{}{}{s.m.}{Grande extensão de terras planas e localizadas a uma certa altitude acima do nível do mar; planalto.}{al.ti.pla.no}{0}
\verb{altíssimo}{}{}{}{}{adj.}{Superlativo absoluto sintético de \textit{alto}; muito alto.}{al.tís.si.mo}{0}
\verb{altíssimo}{}{}{}{}{s.m.}{Designação conferida a Deus; a Divindade Suprema.}{al.tís.si.mo}{0}
\verb{altissonante}{}{}{}{}{adj.2g.}{Que soa muito alto; retumbante, sonoroso.}{al.tis.so.nan.te}{0}
\verb{altissonante}{}{}{}{}{}{Que revela pomposidade; magnificente.}{al.tis.so.nan.te}{0}
\verb{altista}{}{Mús.}{}{}{s.2g.}{Indivíduo cuja voz possui as características do contralto em um quarteto de vozes.}{al.tis.ta}{0}
\verb{altista}{}{}{}{}{}{Músico que executa a parte da viola ou alto em um quarteto de cordas.}{al.tis.ta}{0}
\verb{altista}{}{}{}{}{adj.2g.}{Que força a elevação do preço das mercadorias, apostando na alta do câmbio; especulador.}{al.tis.ta}{0}
\verb{altitude}{}{}{}{}{s.f.}{Elevação de um ponto qualquer da superfície terrestre em relação ao nível do mar.}{al.ti.tu.de}{0}
\verb{altitude}{}{Astron.}{}{}{}{Elevação de um corpo celeste em relação à linha do horizonte.}{al.ti.tu.de}{0}
\verb{altivez}{ê}{}{}{}{s.f.}{Qualidade do que é altivo; elevação, nobreza.}{al.ti.vez}{0}
\verb{altivez}{ê}{}{}{}{}{Atitude de orgulho, arrogância.}{al.ti.vez}{0}
\verb{altivo}{}{}{}{}{adj.}{Que demonstra dignidade; elevado, nobre.}{al.ti.vo}{0}
\verb{altivo}{}{}{}{}{}{Orgulhoso, arrogante, soberbo.}{al.ti.vo}{0}
\verb{alto}{}{}{}{}{adj.}{Que apresenta grande altura; elevado.}{al.to}{0}
\verb{alto}{}{}{}{}{}{Que está levantado ou erguido.}{al.to}{0}
\verb{alto}{}{}{}{}{}{Superior, eminente, notável.}{al.to}{0}
\verb{alto}{}{}{}{}{}{Ilustre, magnífico, excelente.}{al.to}{0}
\verb{alto}{}{}{}{}{}{Que apresenta muita importância; sério, grave.}{al.to}{0}
\verb{alto}{}{}{}{}{}{Que tem intensidade; forte.}{al.to}{0}
\verb{alto}{}{}{}{}{}{De preço elevado; caro.}{al.to}{0}
\verb{alto}{}{Pop.}{}{}{}{Embriagado, bêbado.}{al.to}{0}
\verb{alto}{}{Fig.}{}{}{s.m.}{Ponto mais elevado; topo, cimo.}{al.to}{0}
\verb{alto}{}{}{}{}{}{O céu; local em que habitam as divindades.}{al.to}{0}
\verb{alto}{}{}{}{}{adv.}{A grande altura; na parte mais elevada.}{al.to}{0}
\verb{alto}{}{}{}{}{}{Com som forte ou agudo.}{al.to}{0}
\verb{alto}{}{}{}{}{interj.}{Expressão usada para mandar os soldados suspenderem a marcha ou pararem de atirar.}{al.to}{0}
\verb{alto}{}{Mús.}{}{}{s.m.}{Em um quarteto vocal, a voz do contralto.}{al.to}{0}
\verb{alto}{}{}{}{}{}{Instrumento de cordas semelhante à viola.}{al.to}{0}
\verb{alto"-astral}{}{Astrol.}{altos"-astrais}{}{s.m.}{Situação ou ocasião favorável, atribuída à influência positiva dos astros.}{al.to"-as.tral}{0}
\verb{alto"-astral}{}{Por ext.}{altos"-astrais}{}{adj.2g.}{Que é ou está bem"-humorado, feliz, agradável, influenciado positivamente pelos astros.}{al.to"-as.tral}{0}
\verb{alto"-falante}{}{}{alto"-falantes}{}{s.m.}{Instrumento utilizado para ampliar o som dos aparelhos de rádio.}{al.to"-fa.lan.te}{0}
\verb{alto"-falante}{}{}{alto"-falantes}{}{}{Megafone; porta"-voz.}{al.to"-fa.lan.te}{0}
\verb{alto"-forno}{ô}{}{altos"-fornos ⟨ó/ ou /ô⟩}{}{s.m.}{Grande forno utilizado para fundir o minério de ferro e transformá"-lo em ferro"-gusa.}{al.to"-for.no}{0}
\verb{alto"-mar}{}{}{altos"-mares}{}{s.m.}{Porção de mar, situada fora dos limites das águas territoriais de um país, e liberada, portanto, para a livre navegação; mar alto, mar livre.}{al.to"-mar}{0}
\verb{alto"-relevo}{ê}{}{altos"-relevos ⟨ê⟩}{}{s.m.}{Gravura ou impressão em que algumas partes se salientam do fundo.}{al.to"-re.le.vo}{0}
\verb{altruísmo}{}{}{}{}{s.m.}{Amor desinteressado ao próximo; abnegação, desprendimento, filantropia.}{al.tru.ís.mo}{0}
\verb{altruísta}{}{}{}{}{adj.2g.}{Relativo a altruísmo.}{al.tru.ís.ta}{0}
\verb{altruísta}{}{}{}{}{}{Que ama o próximo de forma desinteressada; humanitário, abnegado, filantropo.}{al.tru.ís.ta}{0}
\verb{altruísta}{}{Ant.}{}{}{}{egoísta}{al.tru.ís.ta}{0}
\verb{altura}{}{}{}{}{s.f.}{Qualidade do que é alto; alteza.}{al.tu.ra}{0}
\verb{altura}{}{}{}{}{}{Distância entre o ponto mais baixo e o ponto mais alto de algo ereto.}{al.tu.ra}{0}
\verb{altura}{}{}{}{}{}{Tamanho, estatura de um corpo.}{al.tu.ra}{0}
\verb{altura}{}{}{}{}{}{Lugar elevado; cume, topo.}{al.tu.ra}{0}
\verb{altura}{}{}{}{}{}{Céu, firmamento.}{al.tu.ra}{0}
\verb{altura}{}{}{}{}{}{Intensidade de uma onda sonora.}{al.tu.ra}{0}
\verb{altura}{}{}{}{}{}{Elevação moral; importância, valia.}{al.tu.ra}{0}
\verb{aluá}{}{Cul.}{}{}{s.m.}{Bebida refrigerante feita de farinha de arroz ou milho e fermentada com açúcar ou caldo de cana e cascas de frutas.}{a.lu.á}{0}
\verb{aluado}{}{}{}{}{adj.}{Que está sob influência da Lua; lunático, doido.}{a.lu.a.do}{0}
\verb{aluado}{}{}{}{}{}{Diz"-se do animal que está no cio.}{a.lu.a.do}{0}
\verb{alucinação}{}{}{"-ões}{}{s.f.}{Ato ou efeito de alucinar.}{a.lu.ci.na.ção}{0}
\verb{alucinação}{}{}{"-ões}{}{}{Impressão falsa, ilusão, devaneio.}{a.lu.ci.na.ção}{0}
\verb{alucinação}{}{}{"-ões}{}{}{Desvario, delírio, loucura.}{a.lu.ci.na.ção}{0}
\verb{alucinado}{}{}{}{}{adj.}{Que se alucinou.}{a.lu.ci.na.do}{0}
\verb{alucinado}{}{}{}{}{}{Iludido, fascinado.}{a.lu.ci.na.do}{0}
\verb{alucinado}{}{}{}{}{}{Desvairado, louco.}{a.lu.ci.na.do}{0}
\verb{alucinante}{}{}{}{}{adj.2g.}{Que provoca alucinação; que faz perder a razão.}{a.lu.ci.nan.te}{0}
\verb{alucinante}{}{}{}{}{}{Que perturba; delirante, apaixonante.}{a.lu.ci.nan.te}{0}
\verb{alucinar}{}{}{}{}{v.t.}{Privar da razão; desvairar.}{a.lu.ci.nar}{0}
\verb{alucinar}{}{}{}{}{}{Causar delírio ou loucura.}{a.lu.ci.nar}{0}
\verb{alucinar}{}{}{}{}{v.pron.}{Apaixonar"-se, fascinar"-se, a ponto de perder a razão.}{a.lu.ci.nar}{\verboinum{1}}
\verb{alucinógeno}{}{Quím.}{}{}{adj.}{Diz"-se da substância ou do produto que provoca alucinações ou estados eufóricos como a maconha, o ácido lisérgico etc.}{a.lu.ci.nó.ge.no}{0}
\verb{alude}{}{}{}{}{s.m.}{Massa de neve que se desprende e se precipita do alto da montanha; avalancha. }{a.lu.de}{0}
\verb{aludir}{}{}{}{}{v.t.}{Fazer rápida referência; mencionar.}{a.lu.dir}{\verboinum{18}}
\verb{alugado}{}{}{}{}{adj.}{Que se alugou; cedido ou tomado por um período mediante pagamento; locado.}{a.lu.ga.do}{0}
\verb{alugar}{}{}{}{}{v.t.}{Tomar ou ceder algo por um período mediante pagamento; locar.}{a.lu.gar}{\verboinum{5}}
\verb{aluguel}{é}{}{"-éis}{}{s.m.}{Cessão de uso de imóvel, objeto, animal ou veículo durante um tempo determinado e mediante pagamento de um valor estabelecido; locação.}{a.lu.guel}{0}
\verb{aluguel}{é}{}{"-éis}{}{}{O valor que o locador recebe referente a essa locação.}{a.lu.guel}{0}
\verb{aluir}{}{}{}{}{v.t.}{Fazer oscilar; abalar, estremecer.}{a.lu.ir}{0}
\verb{aluir}{}{}{}{}{}{Desabar, desmoronar, ruir.}{a.lu.ir}{0}
\verb{aluir}{}{}{}{}{}{Causar a ruína; prejudicar.}{a.lu.ir}{\verboinum{26}}
\verb{alumbramento}{}{}{}{}{s.m.}{Ato ou efeito de alumbrar; iluminar.}{a.lum.bra.men.to}{0}
\verb{alumbramento}{}{}{}{}{}{Estado de quem se maravilha; deslumbramento.}{a.lum.bra.men.to}{0}
\verb{alumbramento}{}{}{}{}{}{Inspiração, arrebatamento, revelação.}{a.lum.bra.men.to}{0}
\verb{alumbrar}{}{}{}{}{v.t.}{Colocar sob a luz; iluminar.}{a.lum.brar}{0}
\verb{alumbrar}{}{}{}{}{}{Causar deslumbramento; maravilhar.}{a.lum.brar}{0}
\verb{alumbrar}{}{}{}{}{}{Provocar arrebatamento; inspirar.}{a.lum.brar}{\verboinum{1}}
\verb{alume}{}{Quím.}{}{}{s.m.}{Nome dado aos sulfatos duplos e aos metais alcalinos com propriedades adstringentes, usados na clarificação ou na purificação de materiais; alúmen.}{a.lu.me}{0}
\verb{alúmen}{}{}{}{}{}{Var. de \textit{alume}.}{a.lú.men}{0}
\verb{alumiar}{}{}{}{}{v.t.}{Dar luz ou lume a; iluminar, acender.}{a.lu.mi.ar}{0}
\verb{alumiar}{}{}{}{}{}{Fazer brilhar; reluzir, resplandecer.}{a.lu.mi.ar}{0}
\verb{alumiar}{}{}{}{}{}{Dar conhecimento; esclarecer, instruir.}{a.lu.mi.ar}{\verboinum{1}}
\verb{alumínio}{}{Quím.}{}{}{s.m.}{Elemento químico metálico abundante na crosta terrestre, branco prateado, leve, maleável, resistente à corrosão, com inúmeras aplicações. \elemento{13}{26.981539}{Al}.}{a.lu.mí.nio}{0}
\verb{alunagem}{}{Astron.}{}{}{s.f.}{Alunissagem.}{a.lu.na.gem}{0}
\verb{alunar}{}{}{}{}{v.i.}{Alunissar.}{a.lu.nar}{\verboinum{1}}
\verb{alunissagem}{}{Astron.}{"-ens}{}{s.f.}{Ato ou efeito de pousar na superfície da Lua; alunagem.}{a.lu.nis.sa.gem}{0}
\verb{alunissar}{}{}{}{}{v.i.}{Pousar na superfície da Lua; alunar.}{a.lu.nis.sar}{\verboinum{1}}
\verb{aluno}{}{}{}{}{s.m.}{Indivíduo que recebe instrução em estabelecimento de ensino sob a orientação de um professor; estudante, discípulo, aprendiz.}{a.lu.no}{0}
\verb{alusão}{}{}{"-ões}{}{s.f.}{Ato ou efeito de aludir, de mencionar.}{a.lu.são}{0}
\verb{alusão}{}{}{"-ões}{}{}{Referência ou menção vaga a respeito de algo ou alguém sem nomeá"-lo diretamente.}{a.lu.são}{0}
\verb{alusivo}{}{}{}{}{adj.}{Que alude ou contém alusão; relativo, referente.}{a.lu.si.vo}{0}
\verb{aluvial}{}{}{"-ais}{}{adj.2g.}{Relativo a aluvião.}{a.lu.vi.al}{0}
\verb{aluvial}{}{}{"-ais}{}{}{Diz"-se do terreno formado por depósito de matérias inorgânicas trazidas pelas águas.}{a.lu.vi.al}{0}
\verb{aluvião}{}{Geol.}{"-ões}{}{s.m.}{Depósito, às margens dos rios, de sedimentos (cascalho, areia e argila) transportados pelas correntes de água.}{a.lu.vi.ão}{0}
\verb{aluvião}{}{}{"-ões}{}{}{Enchente, enxurrada, inundação.}{a.lu.vi.ão}{0}
\verb{aluvião}{}{Jur.}{"-ões}{}{}{Acréscimo de propriedade resultante do recuo ou do desvio sucessivo das águas de um rio.}{a.lu.vi.ão}{0}
\verb{alva}{}{}{}{}{s.f.}{Primeira claridade da manhã; alvorada, aurora.}{al.va}{0}
\verb{alva}{}{}{}{}{}{Longa veste branca utilizada pelos sacerdotes nas cerimônias religiosas.}{al.va}{0}
\verb{alvacento}{}{}{}{}{adj.}{Quase branco; esbranquiçado, alvar, alvadio. }{al.va.cen.to}{0}
\verb{alvadio}{}{}{}{}{adj.}{Alvacento.}{al.va.di.o}{0}
\verb{alvaiade}{}{Quím.}{}{}{s.m.}{Carbonato de chumbo utilizado em pinturas graças a sua pigmentação branca.}{al.vai.a.de}{0}
\verb{alvar}{}{}{}{}{adj.2g.}{Quase branco; alvacento.}{al.var}{0}
\verb{alvar}{}{}{}{}{}{Ingênuo, tolo, estúpido.}{al.var}{0}
\verb{alvará}{}{}{}{}{s.m.}{Documento ou licença passada por autoridade judiciária ou administrativa em favor de alguém, autorizando a prática ou o exercício de atividades como comércio, construção etc.}{al.va.rá}{0}
\verb{alvarenga}{}{}{}{}{s.f.}{Embarcação utilizada para carga e descarga de navios fundeados.}{al.va.ren.ga}{0}
\verb{alvedrio}{}{}{}{}{s.m.}{Vontade própria e independente; arbítrio.}{al.ve.dri.o}{0}
\verb{alveitar}{}{}{}{}{s.m.}{Indivíduo que ferra cavalgaduras.}{al.vei.tar}{0}
\verb{alveitar}{}{}{}{}{}{Indivíduo que trata de doenças de animais, sem ser veterinário.}{al.vei.tar}{0}
\verb{alvejante}{}{}{}{}{adj.2g.}{Diz"-se da substância que alveja, torna branco; branqueador.}{al.ve.jan.te}{0}
\verb{alvejar}{}{}{}{}{v.t.}{Tornar alvo; branquear, descorar.}{al.ve.jar}{\verboinum{1}}
\verb{alvejar}{}{}{}{}{v.t.}{Tomar como alvo; mirar, apontar.}{al.ve.jar}{\verboinum{1}}
\verb{alvenaria}{}{}{}{}{s.f.}{Tipo de obra constituída de materiais como pedras, tijolos etc., ligados por argamassa ou cimento.}{al.ve.na.ri.a}{0}
\verb{alvenaria}{}{}{}{}{}{Conjunto de materiais que entram na composição de muros, alicerces, paredes etc.}{al.ve.na.ri.a}{0}
\verb{alvenaria}{}{}{}{}{}{Pedra tosca, não lavrada, sem acabamento.}{al.ve.na.ri.a}{0}
\verb{alvenaria}{}{}{}{}{}{Profissão de pedreiro.}{al.ve.na.ri.a}{0}
\verb{álveo}{}{}{}{}{s.m.}{Leito de um curso d'água.}{ál.ve.o}{0}
\verb{álveo}{}{}{}{}{}{Escavação, canal, sulco.}{ál.ve.o}{0}
\verb{alveolar}{}{}{}{}{adj.2g.}{Relativo ou semelhante a alvéolo.}{al.ve.o.lar}{0}
\verb{alvéolo}{}{}{}{}{s.m.}{Pequena cavidade ou depressão.}{al.vé.o.lo}{0}
\verb{alvéolo}{}{}{}{}{}{Célula do favo onde as abelhas depositam tanto os ovos quanto o mel.}{al.vé.o.lo}{0}
\verb{alvéolo}{}{Anat.}{}{}{}{Cavidade do maxilar onde se implanta a raiz do dente.}{al.vé.o.lo}{0}
\verb{alvéolo}{}{Anat.}{}{}{}{Cavidade do pulmão onde se realizam as trocas gasosas da respiração.}{al.vé.o.lo}{0}
\verb{alvéolo}{}{Bot.}{}{}{}{Cavidade diminuta, limitada por paredes retas, existente em certos fungos, sementes e polens.}{al.vé.o.lo}{0}
\verb{alvião}{}{}{"-ões}{}{s.m.}{Tipo de picareta ou enxadão utilizado tanto para escavar a terra dura quanto para arrancar pedras.}{al.vi.ão}{0}
\verb{alvinegro}{ê}{}{}{}{adj.}{Que apresenta as cores branca e preta.}{al.vi.ne.gro}{0}
\verb{alvinegro}{ê}{Bras.}{}{}{}{Diz"-se dos times de futebol cujo uniforme ou cuja flâmula apresenta as cores branca e preta.}{al.vi.ne.gro}{0}
\verb{alvinitente}{}{}{}{}{adj.2g.}{De cor branca e brilhante, sem qualquer mancha.}{al.vi.ni.ten.te}{0}
\verb{alvíssaras}{}{}{}{}{s.f.pl.}{Recompensa que se dá para quem anuncia boas notícias.}{al.vís.sa.ras}{0}
\verb{alvíssaras}{}{}{}{}{interj.}{Exclamação de alegria para saudar boas notícias.}{al.vís.sa.ras}{0}
\verb{alvissareiro}{ê}{}{}{}{adj.}{Que traz boas notícias.}{al.vi.s.sa.rei.ro}{0}
\verb{alvissareiro}{ê}{}{}{}{}{Promissor, auspicioso.}{al.vi.s.sa.rei.ro}{0}
\verb{alvitrar}{}{}{}{}{v.t.}{Dar sugestão; aconselhar, propor, lembrar.}{al.vi.trar}{\verboinum{1}}
\verb{alvitre}{}{}{}{}{s.m.}{Aquilo que é sugerido; opinião, conselho, proposta.}{al.vi.tre}{0}
\verb{alvo}{}{}{}{}{adj.}{Que é branco, claro.}{al.vo}{0}
\verb{alvo}{}{Fig.}{}{}{}{Que é puro, inocente.}{al.vo}{0}
\verb{alvo}{}{}{}{}{s.m.}{Qualquer ponto que se procura atingir com algo; ponto de mira.}{al.vo}{0}
\verb{alvo}{}{Por ext.}{}{}{}{Objetivo, meta, finalidade.}{al.vo}{0}
\verb{alvor}{ô}{}{}{}{s.m.}{Primeira claridade da manhã; alvorada, alva.}{al.vor}{0}
\verb{alvor}{ô}{}{}{}{}{Brancura, alvura.}{al.vor}{0}
\verb{alvorada}{}{}{}{}{s.f.}{Claridade que antecede o romper do Sol.}{al.vo.ra.da}{0}
\verb{alvorada}{}{Por ext.}{}{}{}{Canto das aves ao romper o dia.}{al.vo.ra.da}{0}
\verb{alvorada}{}{}{}{}{}{Toque de música militar, ao alvorecer, com o intuito de despertar os soldados.}{al.vo.ra.da}{0}
\verb{alvorada}{}{Fig.}{}{}{}{Os primeiros anos; mocidade.}{al.vo.ra.da}{0}
\verb{alvorecer}{ê}{}{}{}{v.i.}{Surgir o dia; amanhecer.}{al.vo.re.cer}{0}
\verb{alvorecer}{ê}{Fig.}{}{}{}{Aparecer, despertar, surgir.}{al.vo.re.cer}{\verboinum{15}}
\verb{alvorecer}{ê}{}{}{}{s.m.}{O romper do dia.}{al.vo.re.cer}{0}
\verb{alvoroçado}{}{}{}{}{adj.}{Que se alvoroçou; inquieto, agitado.}{al.vo.ro.ça.do}{0}
\verb{alvoroçado}{}{}{}{}{}{Espantado, assustado.}{al.vo.ro.ça.do}{0}
\verb{alvoroçado}{}{}{}{}{}{Entusiasmado, alegre.}{al.vo.ro.ça.do}{0}
\verb{alvoroçar}{}{}{}{}{v.t.}{Pôr em agitação; inquietar, tumultuar.}{al.vo.ro.çar}{0}
\verb{alvoroçar}{}{}{}{}{}{Dar susto; sobressaltar.}{al.vo.ro.çar}{0}
\verb{alvoroçar}{}{}{}{}{}{Provocar alegria; entusiasmar.}{al.vo.ro.çar}{\verboinum{3}}
\verb{alvoroço}{ô}{}{}{}{s.m.}{Estado de agitação, tumulto, inquietação.}{al.vo.ro.ço}{0}
\verb{alvoroço}{ô}{}{}{}{}{Sobressalto, susto, perturbação.}{al.vo.ro.ço}{0}
\verb{alvoroço}{ô}{}{}{}{}{Manifestação de alegria, entusiasmo, gritaria.}{al.vo.ro.ço}{0}
\verb{alvura}{}{}{}{}{s.f.}{Qualidade do que é alvo; brancura.}{al.vu.ra}{0}
\verb{alvura}{}{Fig.}{}{}{}{Pureza, candura, inocência.}{al.vu.ra}{0}
\verb{Am}{}{Quím.}{}{}{}{Símb. do \textit{amerício}.}{Am}{0}
\verb{AM}{}{}{}{}{}{Sigla do estado do Amazonas.}{AM}{0}
\verb{ama}{}{}{}{}{s.f.}{Pessoa, geralmente do sexo feminino, contratada para cuidar das crianças de uma casa; ama"-seca, governanta.}{a.ma}{0}
\verb{ama}{}{Desus.}{}{}{}{A dona da casa em relação aos criados; senhora.}{a.ma}{0}
\verb{ama}{}{Desus.}{}{}{}{Mulher que amamenta criança alheia; ama de leite.}{a.ma}{0}
\verb{amabilidade}{}{}{}{}{s.f.}{Qualidade de quem é amável; delicadeza, gentileza, cortesia.}{a.ma.bi.li.da.de}{0}
\verb{amaciante}{}{}{}{}{adj.2g.}{Que amacia, suaviza.}{a.ma.ci.an.te}{0}
\verb{amaciante}{}{}{}{}{s.m.}{Diz"-se da substância utilizada para tornar macios tecidos, cabelos, pele etc.}{a.ma.ci.an.te}{0}
\verb{amaciar}{}{}{}{}{v.t.}{Tornar macio, tenro; suavizar.}{a.ma.ci.ar}{0}
\verb{amaciar}{}{}{}{}{}{Tornar calmo; abrandar, serenar.}{a.ma.ci.ar}{\verboinum{1}}
\verb{amada}{}{}{}{}{s.f.}{Mulher a quem se ama.}{a.ma.da}{0}
\verb{ama"-de"-leite}{}{Desus.}{amas"-de"-leite}{}{s.f.}{Mulher que alimenta criança alheia com seu próprio leite; ama.}{a.ma"-de"-lei.te}{0}
\verb{amado}{}{}{}{}{adj.}{Que se ama; querido, estimado.}{a.ma.do}{0}
\verb{amador}{ô}{}{}{}{adj.}{Que ama; amante.}{a.ma.dor}{0}
\verb{amador}{ô}{}{}{}{}{Que se dedica a arte, ofício ou esporte, por prazer ou curiosidade, sem ser profissional; diletante.}{a.ma.dor}{0}
\verb{amador}{ô}{}{}{}{}{Diz"-se de esporte, arte ou ofício praticado por indivíduo amador.}{a.ma.dor}{0}
\verb{amadorismo}{}{}{}{}{s.m.}{Qualidade ou condição de quem é amador, não profissional; diletantismo.}{a.ma.do.ris.mo}{0}
\verb{amadorismo}{}{}{}{}{}{Prática ou sistema oposto ao profissionalismo.}{a.ma.do.ris.mo}{0}
\verb{amadrinhar}{}{}{}{}{v.t.}{Servir de madrinha; batizar, paraninfar.}{a.ma.dri.nhar}{0}
\verb{amadrinhar}{}{}{}{}{}{Ligar um animal bravo a um animal manso para habituar ao trabalho.}{a.ma.dri.nhar}{0}
\verb{amadrinhar}{}{}{}{}{}{Emparelhar um cavalo com égua ou burro; madrinhar.}{a.ma.dri.nhar}{\verboinum{1}}
\verb{amadurar}{}{}{}{}{v.t.}{Amadurecer.}{a.ma.du.rar}{\verboinum{1}}
\verb{amadurecer}{ê}{}{}{}{v.t.}{Tornar maduro; sazonar, amadurar.}{a.ma.du.re.cer}{0}
\verb{amadurecer}{ê}{}{}{}{}{Chegar a completo desenvolvimento; aperfeiçoar, aprimorar.}{a.ma.du.re.cer}{0}
\verb{amadurecer}{ê}{Fig.}{}{}{v.i.}{Tornar"-se consciente, experiente, ponderado.}{a.ma.du.re.cer}{\verboinum{15}}
\verb{amadurecimento}{}{}{}{}{s.m.}{Ato ou efeito de amadurecer; sazonamento, maturação, aprimoramento.}{a.ma.du.re.ci.men.to}{0}
\verb{âmago}{}{Bot.}{}{}{s.m.}{Parte interna do tronco da árvore, formada por células mortas, onde não há transporte de água; cerne.}{â.ma.go}{0}
\verb{âmago}{}{Por ext.}{}{}{}{Parte central de qualquer coisa ou pessoa.}{â.ma.go}{0}
\verb{âmago}{}{Fig.}{}{}{}{A parte mais profunda de um ser; essência, alma.}{â.ma.go}{0}
\verb{amainar}{}{}{}{}{v.t.}{Abaixar as velas de uma embarcação para diminuir a força do vento ou do mar.}{a.mai.nar}{0}
\verb{amainar}{}{Fig.}{}{}{}{Tornar sereno; acalmar, abrandar, sossegar.}{a.mai.nar}{\verboinum{1}}
\verb{amaldiçoado}{}{}{}{}{adj.}{Que foi alvo de maldição; maldito, perdido.}{a.mal.di.ço.a.do}{0}
\verb{amaldiçoado}{}{}{}{}{}{Castigado, abominado.}{a.mal.di.ço.a.do}{0}
\verb{amaldiçoar}{}{}{}{}{v.t.}{Desejar, através de palavras e expressões, o mal de alguém; praguejar, maldizer. }{a.mal.di.ço.ar}{0}
\verb{amaldiçoar}{}{}{}{}{}{Condenar, reprovar, castigar.}{a.mal.di.ço.ar}{\verboinum{7}}
\verb{amálgama}{}{}{}{}{s.2g.}{Nome dado a ligas de mercúrio com outro metal, usadas especialmente na obturação de dentes e no garimpo do ouro.}{a.mál.ga.ma}{0}
\verb{amálgama}{}{Por ext.}{}{}{}{Qualquer liga de outros metais.}{a.mál.ga.ma}{0}
\verb{amálgama}{}{Fig.}{}{}{}{Qualquer mistura de elementos diferentes entre si; confusão.}{a.mál.ga.ma}{0}
\verb{amalgamar}{}{}{}{}{v.t.}{Preparar liga de mercúrio com outro metal.}{a.mal.ga.mar}{0}
\verb{amalgamar}{}{}{}{}{}{Fazer combinações; ligar, mesclar, misturar.}{a.mal.ga.mar}{\verboinum{1}}
\verb{amalucado}{}{}{}{}{adj.}{Que se amalucou; um tanto maluco; adoidado.}{a.ma.lu.ca.do}{0}
\verb{amalucar}{}{}{}{}{v.t.}{Tornar maluco; endoidar, enlouquecer.}{a.ma.lu.car}{\verboinum{2}}
\verb{amamentação}{}{}{"-ões}{}{s.f.}{Ato ou efeito de amamentar, criar ao peito; aleitamento.}{a.ma.men.ta.ção}{0}
\verb{amamentadora}{ô}{}{}{}{adj.}{Diz"-se da mulher que amamenta, que cria ao peito.}{a.ma.men.ta.do.ra}{0}
\verb{amamentar}{}{}{}{}{v.t.}{Dar de mamar; alimentar ao peito; aleitar.}{a.ma.men.tar}{0}
\verb{amamentar}{}{Por ext.}{}{}{}{Alimentar, nutrir.}{a.ma.men.tar}{\verboinum{1}}
\verb{amancebado}{}{}{}{}{adj.}{Que vive com alguém sem estar casado; ligado por mancebia; amigado,  amante.}{a.man.ce.ba.do}{0}
\verb{amancebamento}{}{}{}{}{s.m.}{Ato ou efeito de amancebar"-se; estado de mancebia ou concubinato.}{a.man.ce.ba.men.to}{0}
\verb{amancebar"-se}{}{}{}{}{v.pron.}{Viver com alguém sem estar religiosa ou juridicamente casado; ligar"-se em mancebia; amigar"-se, amasiar"-se.}{a.man.ce.bar"-se}{\verboinum{1}}
\verb{amaneirado}{}{}{}{}{adj.}{Que é afetado ou exagerado em suas maneiras; rebuscado, artificial.}{a.ma.nei.ra.do}{0}
\verb{amanhã}{}{}{}{}{adv.}{No dia seguinte ao atual.}{a.ma.nhã}{0}
\verb{amanhã}{}{}{}{}{}{Futuramente, mais tarde.}{a.ma.nhã}{0}
\verb{amanhã}{}{}{}{}{s.m.}{O dia seguinte.}{a.ma.nhã}{0}
\verb{amanhã}{}{}{}{}{}{O futuro, o porvir.}{a.ma.nhã}{0}
\verb{amanhar}{}{}{}{}{v.t.}{Cultivar a terra; lavrar.}{a.ma.nhar}{0}
\verb{amanhar}{}{}{}{}{}{Colocar em ordem; preparar, arranjar.}{a.ma.nhar}{0}
\verb{amanhar}{}{}{}{}{}{Pôr enfeites, adornar.}{a.ma.nhar}{\verboinum{1}}
\verb{amanhecer}{ê}{}{}{}{v.i.}{Iniciar o dia; raiar a manhã.}{a.ma.nhe.cer}{0}
\verb{amanhecer}{ê}{}{}{}{}{Estar em algum lugar no início do dia; despertar, acordar.}{a.ma.nhe.cer}{0}
\verb{amanhecer}{ê}{}{}{}{s.m.}{O início do dia; o alvorecer.}{a.ma.nhe.cer}{\verboinum{15}}
\verb{amanho}{}{}{}{}{s.m.}{Ato ou efeito de amanhar, cultivar.}{a.ma.nho}{0}
\verb{amanho}{}{}{}{}{}{Cultivo da terra; lavoura.}{a.ma.nho}{0}
\verb{amanho}{}{}{}{}{}{Preparo, arranjo.}{a.ma.nho}{0}
\verb{amansamento}{}{}{}{}{s.m.}{Ato ou efeito de amansar, domesticar.}{a.man.sa.men.to}{0}
\verb{amansar}{}{}{}{}{v.t.}{Tornar manso ou dócil; domesticar.}{a.man.sar}{0}
\verb{amansar}{}{Fig.}{}{}{}{Tornar sereno; sossegar, apaziguar.}{a.man.sar}{0}
\verb{amansar}{}{Fig.}{}{}{}{Tornar menos intenso; diminuir, atenuar.}{a.man.sar}{0}
\verb{amansar}{}{Ant.}{}{}{}{enfurecer}{a.man.sar}{\verboinum{1}}
\verb{amante}{}{}{}{}{adj.2g.}{Que ama, gosta muito; apaixonado.}{a.man.te}{0}
\verb{amante}{}{}{}{}{}{Que aprecia ou tem inclinação por algo; amador.}{a.man.te}{0}
\verb{amante}{}{}{}{}{s.2g.}{Indivíduo que mantém relações extraconjugais com outro.}{a.man.te}{0}
\verb{amanteigado}{}{}{}{}{adj.}{Semelhante à manteiga em sabor ou consistência.}{a.man.tei.ga.do}{0}
\verb{amanteigado}{}{}{}{}{}{Feito ou untado com manteiga.}{a.man.tei.ga.do}{0}
\verb{amanteigado}{}{}{}{}{}{Diz"-se do biscoito feito com grande quantidade de manteiga.}{a.man.tei.ga.do}{0}
\verb{amanteigar}{}{}{}{}{v.t.}{Dar consistência ou sabor de manteiga.}{a.man.tei.gar}{0}
\verb{amanteigar}{}{}{}{}{}{Passar manteiga.}{a.man.tei.gar}{0}
\verb{amanteigar}{}{Fig.}{}{}{}{Tornar mole como manteiga; abrandar, amaciar.}{a.man.tei.gar}{\verboinum{5}}
\verb{amanuense}{}{}{}{}{s.2g.}{Indivíduo que escreve textos à mão; copista, escrevente.}{a.ma.nu.en.se}{0}
\verb{amapaense}{}{}{}{}{adj.2g.}{Relativo ao estado do Amapá.}{a.ma.pa.en.se}{0}
\verb{amapaense}{}{}{}{}{s.2g.}{Indivíduo natural ou habitante desse estado.}{a.ma.pa.en.se}{0}
\verb{amar}{}{}{}{}{v.t.}{Sentir amor, devoção.}{a.mar}{0}
\verb{amar}{}{}{}{}{}{Apreciar muito; gostar.}{a.mar}{0}
\verb{amar}{}{}{}{}{}{Ter relação sexual.}{a.mar}{\verboinum{1}}
\verb{amaragem}{}{}{"-ens}{}{s.f.}{Ato ou efeito de amarar; amerissagem.}{a.ma.ra.gem}{0}
\verb{amaranto}{}{}{}{}{s.m.}{Planta ornamental nativa de regiões tropicais e temperadas, com folhas e sementes comestíveis.}{a.ma.ran.to}{0}
\verb{amarar}{}{}{}{}{v.t.}{Pousar na água; amerissar.}{a.ma.rar}{\verboinum{1}}
\verb{amarar}{}{}{}{}{v.t.}{Causar amargura; amargurar.}{a.ma.rar}{\verboinum{1}}
\verb{amarelado}{}{}{}{}{adj.}{De cor semelhante ao amarelo.}{a.ma.re.la.do}{0}
\verb{amarelado}{}{Por ext.}{}{}{}{Pálido, descolorado, sem viço.}{a.ma.re.la.do}{0}
\verb{amarelão}{}{}{"-ões}{}{s.m.}{Certo tipo de arroz e de milho.}{a.ma.re.lão}{0}
\verb{amarelão}{}{Med.}{"-ões}{}{}{Infecção caracterizada por grave anemia; ancilostomose; ancilostomíase.}{a.ma.re.lão}{0}
\verb{amarelar}{}{}{}{}{v.t.}{Tornar amarelo.}{a.ma.re.lar}{0}
\verb{amarelar}{}{Por ext.}{}{}{}{Empalidecer; perder o viço.}{a.ma.re.lar}{\verboinum{1}}
\verb{amarelecer}{ê}{}{}{}{v.t.}{Amarelar.}{a.ma.re.le.cer}{\verboinum{15}}
\verb{amarelecido}{}{}{}{}{adj.}{Amarelado.}{a.ma.re.le.ci.do}{0}
\verb{amarelento}{}{}{}{}{adj.}{Amarelado.}{a.ma.re.len.to}{0}
\verb{amarelento}{}{}{}{}{}{Que sofre de febre amarela.}{a.ma.re.len.to}{0}
\verb{amarelidão}{}{}{"-ões}{}{s.f.}{Qualidade de amarelo.}{a.ma.re.li.dão}{0}
\verb{amarelidão}{}{Por ext.}{"-ões}{}{}{Palidez; ausência de viço.}{a.ma.re.li.dão}{0}
\verb{amarelinha}{}{}{}{}{s.f.}{Jogo infantil que consiste em pular, com um ou dois pés, sobre casas riscadas no chão, sem pisar nas linhas e jogando uma pedrinha.}{a.ma.re.li.nha}{0}
\verb{amarelo}{é}{}{}{}{adj.}{Que tem a cor da gema do ovo, do ouro, do enxofre.}{a.ma.re.lo}{0}
\verb{amarelo}{é}{}{}{}{s.m.}{A cor amarela, situada, no espectro solar, entre o verde e o alaranjado.}{a.ma.re.lo}{0}
\verb{amarelo}{é}{}{}{}{adj.}{Pálido.}{a.ma.re.lo}{0}
\verb{amarelo}{é}{Pop.}{}{}{}{Designação dada ao japonês.}{a.ma.re.lo}{0}
\verb{amarfanhado}{}{}{}{}{adj.}{Vincado com dobras irregulares por ter sido comprimido; amarrotado.}{a.mar.fa.nha.do}{0}
\verb{amarfanhado}{}{Fig.}{}{}{}{Maltratado, humilhado.}{a.mar.fa.nha.do}{0}
\verb{amarfanhar}{}{}{}{}{v.t.}{Comprimir produzindo vincos irregulares; amarrotar.}{a.mar.fa.nhar}{0}
\verb{amarfanhar}{}{Fig.}{}{}{}{Maltratar, humilhar.}{a.mar.fa.nhar}{\verboinum{1}}
\verb{amargar}{}{}{}{}{v.t.}{Tornar amargo.}{a.mar.gar}{0}
\verb{amargar}{}{}{}{}{}{Padecer, sofrer.}{a.mar.gar}{\verboinum{5}}
\verb{amargo}{}{}{}{}{adj.}{Que tem sabor áspero e penetrante, eventualmente desagradável, como boldo ou café sem açúcar; amaro.}{a.mar.go}{0}
\verb{amargo}{}{Fig.}{}{}{}{Desagradável, doloroso.}{a.mar.go}{0}
\verb{amargor}{ô}{}{}{}{s.m.}{Qualidade de amargo.}{a.mar.gor}{0}
\verb{amargoso}{ô}{}{"-osos ⟨ó⟩}{"-osa ⟨ó⟩}{adj.}{Que tem sabor amargo.}{a.mar.go.so}{0}
\verb{amargura}{}{}{}{}{s.f.}{Angústia, tristeza, sofrimento.}{a.mar.gu.ra}{0}
\verb{amargurado}{}{}{}{}{adj.}{Cheio de amargura.}{a.mar.gu.ra.do}{0}
\verb{amargurar}{}{}{}{}{v.t.}{Causar amargura; amarar.}{a.mar.gu.rar}{\verboinum{1}}
\verb{amarílico}{}{Bras.}{}{}{adj.}{Relativo à febre amarela.}{a.ma.rí.li.co}{0}
\verb{amarílis}{}{Bot.}{}{}{s.2g.}{Planta de flores avermelhadas e perfumadas, cultivada como ornamental; açucena.}{a.ma.rí.lis}{0}
\verb{amaríssimo}{}{}{}{}{adj.}{Superlativo absoluto sintético de \textit{amargo}; muito amargo.}{a.ma.rís.si.mo}{0}
\verb{amaro}{}{}{}{}{adj.}{Amargo.}{a.ma.ro}{0}
\verb{amarra}{}{}{}{}{s.f.}{Corrente ou cabo que prende o navio à âncora.}{a.mar.ra}{0}
\verb{amarra}{}{Fig.}{}{}{}{Aquilo que prende algo ou alguém a outra coisa ou a outrem.}{a.mar.ra}{0}
\verb{amarração}{}{}{"-ões}{}{s.f.}{Ato ou efeito de amarrar.}{a.mar.ra.ção}{0}
\verb{amarrado}{}{}{}{}{adj.}{Preso com amarra; atado.}{a.mar.ra.do}{0}
\verb{amarrado}{}{Fig.}{}{}{}{De semblante carrancudo, fechado.}{a.mar.ra.do}{0}
\verb{amarrar}{}{}{}{}{v.t.}{Ligar fortemente; atar, prender.}{a.mar.rar}{0}
\verb{amarrar}{}{}{}{}{}{Prender por vínculo abstrato ou moral.}{a.mar.rar}{\verboinum{1}}
\verb{amarrotado}{}{}{}{}{adj.}{Vincado com dobras irregulares.}{a.mar.ro.ta.do}{0}
\verb{amarrotar}{}{}{}{}{v.t.}{Vincar com dobras irregulares, geralmente por compressão.}{a.mar.ro.tar}{\verboinum{1}}
\verb{amarugem}{}{}{"-ens}{}{s.f.}{Sabor amargo.}{a.ma.ru.gem}{0}
\verb{ama"-seca}{ê}{}{amas"-secas ⟨ê⟩}{}{s.f.}{Pessoa, geralmente do sexo feminino, contratada para cuidar das crianças de uma casa; ama; governanta.}{a.ma"-se.ca}{0}
\verb{amásia}{}{}{}{}{s.f.}{Companheira, concubina, namorada.}{a.má.sia}{0}
\verb{amasiar"-se}{}{}{}{}{v.pron.}{Passar a viver com alguém sem estar religiosa ou juridicamente casado.}{a.ma.si.ar"-se}{\verboinum{1}}
\verb{amasio}{}{}{}{}{s.m.}{Concubinato.}{a.ma.si.o}{0}
\verb{amásio}{}{}{}{}{s.m.}{Indivíduo amancebado.}{a.má.sio}{0}
\verb{amassadeira}{ê}{}{}{}{s.f.}{Mulher que amassa farinha para fazer pão.}{a.mas.sa.dei.ra}{0}
\verb{amassadeira}{ê}{}{}{}{}{Recipiente no qual se prepara massa destinada a fazer pão ou tijolos.}{a.mas.sa.dei.ra}{0}
\verb{amassador}{ô}{}{}{}{s.m.}{Operário que prepara argamassa.}{a.mas.sa.dor}{0}
\verb{amassador}{ô}{}{}{}{}{Lugar onde se prepara argamassa.}{a.mas.sa.dor}{0}
\verb{amassadouro}{ô}{}{}{}{s.m.}{Lugar onde se amassa.}{a.mas.sa.dou.ro}{0}
\verb{amassadura}{}{}{}{}{s.f.}{Ato ou efeito de amassar.}{a.mas.sa.du.ra}{0}
\verb{amassadura}{}{}{}{}{}{Conjunto de pães, biscoitos etc. que são cozidos de uma vez no mesmo forno; fornada.}{a.mas.sa.du.ra}{0}
\verb{amassar}{}{}{}{}{v.t.}{Deformar esmagando ou batendo; esmagar, sovar, mesclar, amarrotar.}{a.mas.sar}{\verboinum{1}}
\verb{amatutado}{}{}{}{}{adj.}{Que tem ou adquiriu modos ou aparência de matuto.}{a.ma.tu.ta.do}{0}
\verb{amatutar"-se}{}{}{}{}{v.pron.}{Adquirir modos ou aparência de matuto.}{a.ma.tu.tar"-se}{\verboinum{1}}
\verb{amável}{}{}{"-eis}{}{adj.2g.}{Que merece amor, afeto.}{a.má.vel}{0}
\verb{amável}{}{}{"-eis}{}{}{Que demonstra cortesia, delicadeza; simpático, agradável.}{a.má.vel}{0}
\verb{amavios}{}{Desus.}{}{}{s.m.pl.}{Bebidas para suscitar o amor; afrodisíacos.}{a.ma.vi.os}{0}
\verb{amazona}{}{}{}{}{s.f.}{Mulher que monta a cavalo.}{a.ma.zo.na}{0}
\verb{amazona}{}{}{}{}{}{Mulher corajosa.}{a.ma.zo.na}{0}
\verb{amazona}{}{}{}{}{}{Vestido ou saia longa para montar a cavalo.}{a.ma.zo.na}{0}
\verb{amazonense}{}{}{}{}{adj.2g.}{Relativo ao estado do Amazonas.}{a.ma.zo.nen.se}{0}
\verb{amazonense}{}{}{}{}{s.2g.}{Indivíduo natural ou habitante desse estado.}{a.ma.zo.nen.se}{0}
\verb{amazônico}{}{}{}{}{adj.}{Relativo a amazona.}{a.ma.zô.ni.co}{0}
\verb{amazônico}{}{}{}{}{adj.}{Referente à Amazônia.}{a.ma.zô.ni.co}{0}
\verb{âmbar}{}{}{}{}{s.m.}{Substância sólida de cor parda ou preta e cheiro almiscarado.}{âm.bar}{0}
\verb{âmbar}{}{}{}{}{s.m.}{Resina fóssil proveniente de uma espécie extinta de pinheiro, usada na fabricação de vários objetos; âmbar"-amarelo.}{âm.bar}{0}
\verb{âmbar}{}{}{}{}{adj.2g.}{Que tem a cor dessa resina, de tom entre o acastanhado e o amarelo.}{âm.bar}{0}
\verb{ambarino}{}{}{}{}{adj.}{Relativo a âmbar.}{am.ba.ri.no}{0}
\verb{ambição}{}{}{"-ões}{}{s.f.}{Forte desejo de poder, de riquezas materiais, de glórias etc.; desejo veemente, pretensão.}{am.bi.ção}{0}
\verb{ambição}{}{}{"-ões}{}{}{Anseio de alcançar determinado objetivo; aspiração quanto ao futuro.}{am.bi.ção}{0}
\verb{ambicionar}{}{}{}{}{v.t.}{Ter ambição; desejar, cobiçar.}{am.bi.ci.o.nar}{\verboinum{1}}
\verb{ambicioso}{ô}{}{"-osos ⟨ó⟩}{"-osa ⟨ó⟩}{adj.}{Que tem ambição.}{am.bi.ci.o.so}{0}
\verb{ambicioso}{ô}{}{"-osos ⟨ó⟩}{"-osa ⟨ó⟩}{}{Diz"-se de projeto, plano etc., cuja realização exige bastante dedicação e competência; arrojado.}{am.bi.ci.o.so}{0}
\verb{ambidestro}{é}{}{}{}{adj.}{Que utiliza a mão esquerda e a direita com a mesma facilidade.}{am.bi.des.tro}{0}
\verb{ambiência}{}{}{}{}{s.f.}{Meio físico em que vive um animal ou um vegetal; meio ambiente.}{am.bi.ên.cia}{0}
\verb{ambientado}{}{}{}{}{adj.}{Adaptado, acostumado.}{am.bi.en.ta.do}{0}
\verb{ambiental}{}{}{"-ais}{}{adj.2g.}{Relativo a ambiente.}{am.bi.en.tal}{0}
\verb{ambientalismo}{}{}{}{}{s.m.}{Movimento ou ideologia em defesa da preservação e do uso adequado do meio ambiente.}{am.bi.en.ta.lis.mo}{0}
\verb{ambientalista}{}{}{}{}{adj.2g.}{Relativo a ambientalismo.}{am.bi.en.ta.lis.ta}{0}
\verb{ambientar}{}{}{}{}{v.t.}{Adaptar ou acostumar a determinado ambiente físico ou social.}{am.bi.en.tar}{\verboinum{1}}
\verb{ambiente}{}{}{}{}{adj.2g.}{Que envolve ou rodeia os seres vivos ou as coisas.}{am.bi.en.te}{0}
\verb{ambiente}{}{}{}{}{s.m.}{Aquilo que envolve ou rodeia determinado ser vivo ou coisa.}{am.bi.en.te}{0}
\verb{ambiente}{}{}{}{}{}{Conjunto de condições materiais, sociais, ideológicas e psicológicas que envolve um indivíduo ou um grupo de pessoas.}{am.bi.en.te}{0}
\verb{ambiguidade}{}{}{}{}{s.f.}{Característica ou condição do que é ambíguo.}{am.bi.gui.da.de}{0}
\verb{ambiguidade}{}{}{}{}{}{Incerteza, dúvida, indecisão.}{am.bi.gui.da.de}{0}
\verb{ambiguidade}{}{Gram.}{}{}{}{Duplicidade de sentido que permite mais de uma interpretação.}{am.bi.gui.da.de}{0}
\verb{ambíguo}{}{}{}{}{adj.}{Que tem mais de um sentido.}{am.bí.guo}{0}
\verb{ambíguo}{}{}{}{}{}{Que desperta dúvida, incerteza; indefinido.}{am.bí.guo}{0}
\verb{ambíguo}{}{}{}{}{}{Que admite várias interpretações.}{am.bí.guo}{0}
\verb{âmbito}{}{}{}{}{s.m.}{Espaço que circunda, rodeia; periferia.}{âm.bi.to}{0}
\verb{âmbito}{}{}{}{}{}{Espaço delimitado; recinto, ambitude.}{âm.bi.to}{0}
\verb{âmbito}{}{Fig.}{}{}{}{Campo de ação; espaço em que ocorre ou se exerce alguma atividade.}{âm.bi.to}{0}
\verb{ambivalência}{}{}{}{}{s.f.}{Estado, condição ou caráter do que é ambivalente, do que apresenta dois aspectos ou dois valores diferentes.}{am.bi.va.lên.cia}{0}
\verb{ambivalente}{}{}{}{}{adj.2g.}{Que apresenta ambivalência, carrega em si dois valores ou poderes contrários.}{am.bi.va.len.te}{0}
\verb{ambos}{}{}{}{}{num.}{Um e outro; os dois.}{am.bos}{0}
\verb{ambos}{}{}{}{}{pron.}{Os dois de quem se fala; eles dois.}{am.bos}{0}
\verb{ambrosia}{}{Mit.}{}{}{s.f.}{Alimento dos deuses que concedia e conservava a imortalidade.}{am.bro.si.a}{0}
\verb{ambrosia}{}{Por ext.}{}{}{}{Comida ou bebida deliciosa.}{am.bro.si.a}{0}
\verb{ambrósia}{}{Bot.}{}{}{s.f.}{Gênero de plantas de propriedades medicinais, usadas em licores ou para a extração de tintura.}{am.bró.sia}{0}
\verb{ambrosíaco}{}{}{}{}{adj.}{Relativo a ambrosia.}{am.bro.sí.a.co}{0}
\verb{ambrosíaco}{}{Por ext.}{}{}{}{Delicioso, saboroso, delicado.}{am.bro.sí.a.co}{0}
\verb{âmbula}{}{}{}{}{s.f.}{Garrafa de gargalo estreito e bojo redondo.}{âm.bu.la}{0}
\verb{âmbula}{}{}{}{}{}{Pequeno vaso em que se guardam os santos óleos.}{âm.bu.la}{0}
%\verb{}{}{}{}{}{}{}{}{0}
%\verb{}{}{}{}{}{}{}{}{0}
\verb{ambulância}{}{}{}{}{s.f.}{Veículo automóvel especialmente equipado para atender e conduzir doentes e feridos.}{am.bu.lân.cia}{0}
\verb{ambulante}{}{}{}{}{adj.2g.}{Que se transporta sempre de um lugar para outro.}{am.bu.lan.te}{0}
\verb{ambulante}{}{}{}{}{s.m.}{Indivíduo que não se fixa apenas em um lugar para exercer o seu comércio.}{am.bu.lan.te}{0}
\verb{ambulatorial}{}{}{"-ais}{}{adj.2g.}{Relativo ou pertencente a ambulatório.}{am.bu.la.to.ri.al}{0}
\verb{ambulatório}{}{}{}{}{adj.}{Que anda de um lado para outro.}{am.bu.la.tó.rio}{0}
\verb{ambulatório}{}{}{}{}{s.m.}{Departamento hospitalar em que são atendidos os doentes sem gravidade, que se podem locomover e que não necessitam de internação.}{am.bu.la.tó.rio}{0}
\verb{ameaça}{}{}{}{}{s.f.}{Ação, gesto ou palavra que intimida.}{a.me.a.ça}{0}
\verb{ameaça}{}{}{}{}{}{Promessa de castigo ou malefício.}{a.me.a.ça}{0}
\verb{ameaça}{}{}{}{}{}{Prenúncio ou indício de acontecimento desagradável ou maléfico.}{a.me.a.ça}{0}
\verb{ameaçador}{ô}{}{}{}{adj.}{Que ameaça, que procura intimidar.}{a.me.a.ça.dor}{0}
\verb{ameaçador}{ô}{}{}{}{}{Diz"-se do tempo quando está por vir uma tempestade.}{a.me.a.ça.dor}{0}
\verb{ameaçar}{}{}{}{}{v.t.}{Dirigir ameaças.}{a.me.a.çar}{0}
\verb{ameaçar}{}{}{}{}{}{Procurar intimidar.}{a.me.a.çar}{0}
\verb{ameaçar}{}{}{}{}{}{Prometer, anunciar castigo.}{a.me.a.çar}{0}
\verb{ameaçar}{}{}{}{}{}{Estar na iminência de acontecer.}{a.me.a.çar}{\verboinum{3}}
\verb{ameaço}{}{}{}{}{s.m.}{Ameaça, prenúncio de um mal.}{a.me.a.ço}{0}
\verb{amealhar}{}{}{}{}{v.t.}{Juntar aos poucos, economizar dinheiro; poupar. }{a.me.a.lhar}{\verboinum{1}}
\verb{ameba}{é}{Biol.}{}{}{s.f.}{Protozoário que se movimenta e se alimenta por meio de pseudópodes, encontrado geralmente em ambiente aquático. }{a.me.ba}{0}
\verb{amebiano}{}{}{}{}{adj.}{Relativo a ameba.}{a.me.bi.a.no}{0}
\verb{amebiano}{}{}{}{}{}{Em que há amebas.}{a.me.bi.a.no}{0}
\verb{amebíase}{}{Med.}{}{}{s.f.}{Infecção intestinal causada por amebas, caracterizada por disenteria com perda de sangue, e contraída pela ingestão de água e alimentos contaminados.}{a.me.bí.a.se}{0}
\verb{amedrontador}{ô}{}{}{}{adj.}{Que amedronta, que mete medo.}{a.me.dron.ta.dor}{0}
\verb{amedrontador}{ô}{}{}{}{s.m.}{Indivíduo que amedronta, atemoriza.}{a.me.dron.ta.dor}{0}
\verb{amedrontar}{}{}{}{}{v.t.}{Causar medo; assustar, atemorizar.}{a.me.dron.tar}{0}
\verb{amedrontar}{}{}{}{}{}{Levar, induzir alguém pelo medo.}{a.me.dron.tar}{\verboinum{1}}
\verb{ameia}{ê}{}{}{}{s.f.}{Cada uma das partes salientes retangulares, separadas por intervalos iguais, na parte superior das muralhas de fortalezas e castelos, que protegiam os atiradores.}{a.mei.a}{0}
\verb{ameigar}{}{}{}{}{v.t.}{Fazer carinhos; afagar, acarinhar.}{a.mei.gar}{0}
\verb{ameigar}{}{}{}{}{}{Tornar meigo; suavizar.}{a.mei.gar}{\verboinum{5}}
\verb{amêijoa}{}{}{}{}{s.f.}{Molusco bivalve, comestível, cuja concha é oval.}{a.mêi.jo.a}{0}
\verb{ameixa}{ch}{}{}{}{s.f.}{Fruto comestível da ameixeira, carnoso e suculento, de cor roxo"-escura e polpa doce. }{a.mei.xa}{0}
\verb{ameixa}{ch}{}{}{}{}{O pé desta fruta; ameixeira.}{a.mei.xa}{0}
\verb{ameixal}{ch}{}{"-ais}{}{s.m.}{Coletivo de ameixa; lugar onde se cultivam ameixas.}{a.mei.xal}{0}
\verb{ameixeira}{ch}{Bot.}{}{}{s.f.}{Árvore frutífera, nativa da Europa e do Cáucaso, de folhas arredondadas, flores branco"-esverdeadas, que produz a ameixa.}{a.mei.xei.ra}{0}
\verb{amém}{}{}{}{}{interj.}{Expressão usada na liturgia religiosa, que denota concordância perfeita; assim seja.}{a.mém}{0}
\verb{amém}{}{}{}{}{s.m.}{Concordância incondicional; consentimento, aprovação.}{a.mém}{0}
\verb{amêndoa}{}{}{}{}{s.f.}{O fruto da amendoeira.}{a.mên.do.a}{0}
\verb{amêndoa}{}{}{}{}{}{A semente da amendoeira, da qual se extraem essência e óleo com usos medicinais e na indústria de cosméticos; também comestível, muito consumida em doces e salgados.}{a.mên.do.a}{0}
\verb{amendoado}{}{}{}{}{adj.}{Diz"-se do que se assemelha à amêndoa na cor e na forma.}{a.men.do.a.do}{0}
\verb{amendoado}{}{}{}{}{}{Que é preparado com amêndoas.}{a.men.do.a.do}{0}
\verb{amendoal}{}{}{"-ais}{}{s.m.}{Plantação de amendoeiras.}{a.men.do.al}{0}
\verb{amendoeira}{ê}{Bot.}{}{}{s.f.}{Árvore frutífera, nativa da Europa e da Ásia, de folhas simples, flores róseas ou brancas, que produz a amêndoa.}{a.men.do.ei.ra}{0}
\verb{amendoeira"-da"-praia}{ê}{Bot.}{amendoeiras"-da"-praia}{}{s.f.}{Árvore ornamental, própria  para beira"-mar, com raiz e casca adstringentes e de cuja amêndoa se extrai um óleo doce, usado em emulsões peitorais; amêndoa, chapéu"-de"-sol.}{a.men.do.ei.ra"-da"-prai.a}{0}
\verb{amendoim}{}{Bot.}{"-ins}{}{s.m.}{Planta herbácea, nativa do Brasil, de flores amareladas e frutos cilíndricos, cujas sementes e óleo são utilizados em culinária.}{a.men.do.im}{0}
\verb{amendoim}{}{}{"-ins}{}{}{A semente dessa planta.}{a.men.do.im}{0}
\verb{amenidade}{}{}{}{}{s.f.}{Estado, caráter ou qualidade do que é ameno.}{a.me.ni.da.de}{0}
\verb{amenidade}{}{Fig.}{}{}{}{Graça, suavidade.}{a.me.ni.da.de}{0}
\verb{amenidade}{}{Por ext.}{}{}{}{Assuntos superficiais; trivialidade.}{a.me.ni.da.de}{0}
\verb{amenização}{}{}{"-ões}{}{s.f.}{Ato ou efeito de amenizar, de tornar ameno, brando.}{a.me.ni.za.ção}{0}
\verb{amenizar}{}{}{}{}{v.t.}{Tornar ameno, brando; suavizar.}{a.me.ni.zar}{0}
\verb{amenizar}{}{}{}{}{}{Tornar menos árduo ou difícil.}{a.me.ni.zar}{\verboinum{1}}
\verb{ameno}{}{}{}{}{adj.}{Que demonstra suavidade, delicadeza; terno, brando.}{a.me.no}{0}
\verb{ameno}{}{}{}{}{}{Que se processa de maneira fácil, simples, agradável.}{a.me.no}{0}
\verb{ameno}{}{}{}{}{}{Moderado, suave.}{a.me.no}{0}
\verb{amenorreia}{é}{Med.}{}{}{s.f.}{Ausência de fluxo menstrual.}{a.me.nor.rei.a}{0}
\verb{amercear"-se}{}{}{}{}{v.pron.}{Apiedar"-se, condoer"-se.}{a.mer.ce.ar"-se}{\verboinum{4}}
\verb{americanismo}{}{}{}{}{s.m.}{Admiração, apreço, imitação das coisas da América, especialmente dos Estados Unidos da América.}{a.me.ri.ca.nis.mo}{0}
\verb{americanismo}{}{}{}{}{}{Tudo o que diz respeito a cultura, tradição e instituições do continente americano.}{a.me.ri.ca.nis.mo}{0}
\verb{americanista}{}{}{}{}{s.2g.}{Especialista em assuntos relacionados  à América.}{a.me.ri.ca.nis.ta}{0}
\verb{americanista}{}{}{}{}{}{Admirador ou partidário dos usos ou dos costumes da América.}{a.me.ri.ca.nis.ta}{0}
\verb{americanizar}{}{}{}{}{v.t.}{Tornar semelhante aos americanos; adaptar aos modos, aos costumes ou ao estilo de vida americano.}{a.me.ri.ca.ni.zar}{\verboinum{1}}
\verb{americano}{}{}{}{}{adj.}{Relativo ou pertencente ao continente americano ou aos países desse continente.}{a.me.ri.ca.no}{0}
\verb{americano}{}{}{}{}{}{Relativo ou pertencente aos Estados Unidos da América.}{a.me.ri.ca.no}{0}
\verb{americano}{}{}{}{}{s.m.}{Indivíduo natural ou habitante do continente americano.}{a.me.ri.ca.no}{0}
\verb{americano}{}{}{}{}{}{Indivíduo natural ou habitante dos Estados Unidos da América; norte"-americano.}{a.me.ri.ca.no}{0}
\verb{amerício}{}{Quím.}{}{}{s.m.}{Elemento químico radioativo, com brilho prateado, do grupo dos actinídeos, obtido artificialmente. \elemento{95}{(243)}{Am}.}{a.me.rí.cio}{0}
\verb{ameríndio}{}{}{}{}{adj.}{Relativo ou pertencente ao indígena americano.}{a.me.rín.dio}{0}
\verb{ameríndio}{}{}{}{}{s.m.}{Denominação dada ao indígena americano, para distingui"-lo do asiático.}{a.me.rín.dio}{0}
\verb{amerissagem}{}{}{"-ens}{}{s.f.}{Ato ou efeito de amerissar, de pousar sobre a água.}{a.me.ris.sa.gem}{0}
\verb{amerissar}{}{}{}{}{v.i.}{Pousar na superfície da água.}{a.me.ris.sar}{\verboinum{1}}
\verb{amesquinhamento}{}{}{}{}{s.m.}{Ato ou efeito de amesquinhar, de tornar pequeno, de pouco valor.}{a.mes.qui.nha.men.to}{0}
\verb{amesquinhar}{}{}{}{}{v.t.}{Tornar mesquinho, insignificante; rebaixar, humilhar.}{a.mes.qui.nhar}{0}
\verb{amesquinhar}{}{}{}{}{}{Ficar deprimido; afligir.}{a.mes.qui.nhar}{\verboinum{1}}
\verb{amestrado}{}{}{}{}{adj.}{Que se amestrou, que se tornou mestre; industriado.}{a.mes.tra.do}{0}
\verb{amestrado}{}{}{}{}{}{Diz"-se de animal ao qual foram ensinadas  certas habilidades; adestrado, treinado.}{a.mes.tra.do}{0}
\verb{amestrado}{}{Por ext.}{}{}{}{Diz"-se de animal que se amansou; domesticado.}{a.mes.tra.do}{0}
\verb{amestramento}{}{}{}{}{s.m.}{Ato ou efeito de amestrar, de instruir, ensinar.}{a.mes.tra.men.to}{0}
\verb{amestrar}{}{}{}{}{v.t.}{Tornar mestre; ensinar, industriar.}{a.mes.trar}{0}
\verb{amestrar}{}{}{}{}{}{Ensinar o animal a desenvolver certas habilidades; adestrar.}{a.mes.trar}{0}
\verb{amestrar}{}{Por ext.}{}{}{}{Amansar o animal; domesticar.}{a.mes.trar}{\verboinum{1}}
\verb{ametal}{}{Quím.}{"-ais}{}{s.m.}{Elemento sem brilho metálico e geralmente mau condutor de calor e eletricidade; não metal. }{a.me.tal}{0}
\verb{ametista}{}{}{}{}{s.f.}{Pedra semipreciosa, variedade violeta do quartzo.}{a.me.tis.ta}{0}
\verb{amianto}{}{}{}{}{s.m.}{Silicato natural de cálcio e magnésio, composto de fibras finas e sedosas, refratário ao fogo e dificilmente fusível, usado na fabricação de produtos resistentes ao fogo.}{a.mi.an.to}{0}
\verb{amical}{}{}{"-ais}{}{adj.2g.}{Que demonstra amizade.}{a.mi.cal}{0}
\verb{amicíssimo}{}{}{}{}{adj.}{Superlativo absoluto sintético de \textit{amigo}; muito amigo.}{a.mi.cís.si.mo}{0}
\verb{amida}{}{Quím.}{}{}{s.f.}{Classe de compostos orgânicos derivados da desidratação da amônia.}{a.mi.da}{0}
\verb{amídala}{}{Anat.}{}{}{s.f.}{Órgão de tecido linfoide, com formato de amêndoa, localizado na base da língua; tonsila.}{a.mí.da.la}{0}
\verb{amídala}{}{Anat.}{}{}{}{Substância cinzenta situada nos hemisférios cerebrais, responsável por certas reações emocionais e pelo aprendizado.}{a.mí.da.la}{0}
\verb{amidalite}{}{Med.}{}{}{s.f.}{Inflamação infecciosa das amídalas; tonsilite.}{a.mi.da.li.te}{0}
\verb{amido}{}{Bioquím.}{}{}{s.m.}{Substância orgânica formada por moléculas de glicose, encontrada em numerosos vegetais, como nos cereais e na batata, e muito utilizada na indústria alimentícia.}{a.mi.do}{0}
\verb{amieiro}{ê}{}{}{}{s.m.}{Árvore ornamental, cuja madeira é utilizada na fabricação de instrumentos musicais e cuja casca auxilia no combate à angina.}{a.mi.ei.ro}{0}
\verb{amiga}{}{}{}{}{adj.}{Ligada a outrem por laços de amizade.}{a.mi.ga}{0}
\verb{amigação}{}{}{"-ões}{}{s.f.}{Ato ou efeito de amigar"-se.}{a.mi.ga.ção}{0}
\verb{amigado}{}{}{}{}{adj.}{Que vive com outra pessoa sem estar religiosa ou juridicamente casado; amancebado.}{a.mi.ga.do}{0}
\verb{amigar"-se}{}{}{}{}{v.pron.}{Ligar"-se por amizade; amasiar"-se.}{a.mi.gar"-se}{\verboinum{5}}
\verb{amigável}{}{}{"-eis}{}{adj.2g.}{Que é feito de maneira conciliatória e sem conflitos.}{a.mi.gá.vel}{0}
\verb{amígdala}{}{}{}{}{}{Var. de \textit{amídala}.}{a.míg.da.la}{0}
\verb{amigdalite}{}{}{}{}{}{Var. de \textit{amidalite}.}{a.mig.da.li.te}{0}
\verb{amigo}{}{}{}{}{adj.}{Ligado por laços de amizade e afeto.}{a.mi.go}{0}
\verb{amigo"-da"-onça}{}{}{amigos"-da"-onça}{}{s.m.}{Falso amigo; amigo"-urso.}{a.mi.go"-da"-on.ça}{0}
\verb{amigo"-urso}{}{}{amigos"-ursos}{}{s.m.}{Amigo falso; amigo"-da"-onça.}{a.mi.go"-ur.so}{0}
\verb{amiláceo}{}{}{}{}{adj.}{Semelhante a ou constituído de amido.}{a.mi.lá.ceo}{0}
\verb{amimar}{}{}{}{}{}{Var. de \textit{mimar}.}{a.mi.mar}{0}
\verb{amina}{}{Quím.}{}{}{s.f.}{Nome genérico dos compostos orgânicos derivados da amônia.}{a.mi.na}{0}
\verb{aminoácido}{}{Quím.}{}{}{s.m.}{Nome genérico de substâncias que possuem as duas funções amina e ácida, fundamental a todas as proteínas.}{a.mi.no.á.ci.do}{0}
\verb{amistoso}{ô}{}{"-osos ⟨ó⟩}{"-osa ⟨ó⟩}{adj.}{Que é próprio de amigo; amigável.}{a.mis.to.so}{0}
\verb{amiudado}{}{}{}{}{adj.}{Que é frequente; repetido.}{a.mi.u.da.do}{0}
\verb{amiudar}{}{}{}{}{v.t.}{Executar (algo) com frequência, amiúde.}{a.mi.u.dar}{\verboinum{8}}
\verb{amiudar}{}{}{}{}{v.t.}{Tornar miúdo.}{a.mi.u.dar}{0}
\verb{amiudar}{}{}{}{}{}{Examinar detalhadamente; esmiuçar.}{a.mi.u.dar}{\verboinum{8}}
\verb{amiúde}{}{}{}{}{adv.}{Frequentemente; a miúdo.}{a.mi.ú.de}{0}
\verb{amizade}{}{}{}{}{s.f.}{Sentimento de estima, simpatia e admiração entre pessoas.}{a.mi.za.de}{0}
\verb{amnésia}{}{Med.}{}{}{s.f.}{Perda total ou parcial, definitiva ou temporária da memória.}{am.né.sia}{0}
\verb{amnésico}{}{}{}{}{adj.}{Relativo a amnésia.}{am.né.si.co}{0}
\verb{âmnio}{}{Anat.}{}{}{s.m.}{A mais interna das membranas que envolvem o feto no útero.}{âm.nio}{0}
\verb{amniótico}{}{}{}{}{adj.}{Relativo a âmnio.}{am.ni.ó.ti.co}{0}
\verb{amo}{}{}{}{}{s.m.}{Patrão, chefe; dono da casa.}{a.mo}{0}
\verb{amodorrar}{}{}{}{}{v.t.}{Causar modorra.}{a.mo.dor.rar}{\verboinum{1}}
\verb{amoedar}{}{}{}{}{v.t.}{Transformar o metal em moedas; cunhar.}{a.mo.e.dar}{\verboinum{1}}
\verb{amofinação}{}{}{"-ões}{}{s.f.}{Ato ou efeito de amofinar; aborrecimento; apoquentação.}{a.mo.fi.na.ção}{0}
\verb{amofinar}{}{}{}{}{v.t.}{Aborrecer, apoquentar.}{a.mo.fi.nar}{\verboinum{1}}
\verb{amoitar"-se}{}{}{}{}{v.t.}{Esconder, ocultar.}{a.moi.tar"-se}{0}
\verb{amoitar"-se}{}{}{}{}{v.pron.}{Proteger"-se, abrigar"-se, agasalhar"-se.}{a.moi.tar"-se}{\verboinum{1}}
\verb{amojar}{}{}{}{}{v.t.}{Ordenhar.}{a.mo.jar}{\verboinum{1}}
\verb{amojo}{ô}{}{}{}{s.m.}{Ato ou efeito de amojar.}{a.mo.jo}{0}
\verb{amolação}{}{}{"-ões}{}{s.f.}{Ato ou efeito de amolar; estorvo, incômodo, aborrecimento.}{a.mo.la.ção}{0}
\verb{amoladeira}{ê}{}{}{}{s.f.}{Pedra utilizada para amolar; esmeril.}{a.mo.la.dei.ra}{0}
\verb{amolado}{}{}{}{}{adj.}{Afiado, aguçado.}{a.mo.la.do}{0}
\verb{amolado}{}{}{}{}{}{Aborrecido.}{a.mo.la.do}{0}
\verb{amolador}{ô}{}{}{}{adj.}{Que amola.}{a.mo.la.dor}{0}
\verb{amolador}{ô}{}{}{}{s.m.}{Indivíduo que amola facas, tesouras e outros instrumentos cortantes.}{a.mo.la.dor}{0}
\verb{amolador}{ô}{}{}{}{}{Instrumento utilizado para amolar facas, tesouras etc.}{a.mo.la.dor}{0}
\verb{amolante}{}{}{}{}{adj.2g.}{Que aborrece; enfadonho, maçante.}{a.mo.lan.te}{0}
\verb{amolar}{}{}{}{}{v.t.}{Tornar cortante; afiar.}{a.mo.lar}{0}
\verb{amolar}{}{}{}{}{}{Aborrecer, irritar.}{a.mo.lar}{\verboinum{1}}
\verb{amoldar}{}{}{}{}{v.t.}{Ajustar ao molde; adaptar, modelar.}{a.mol.dar}{\verboinum{1}}
\verb{amolecado}{}{}{}{}{adj.}{Que age como ou tem ar de moleque.}{a.mo.le.ca.do}{0}
\verb{amolecar}{}{}{}{}{v.t.}{Tratar de modo depreciativo; ridicularizar, rebaixar.}{a.mo.le.car}{0}
\verb{amolecar}{}{}{}{}{v.pron.}{Tornar"-se semelhante a moleque.}{a.mo.le.car}{\verboinum{2}}
\verb{amolecer}{ê}{}{}{}{v.t.}{Tornar mole, flexível.}{a.mo.le.cer}{\verboinum{15}}
\verb{amolecimento}{}{}{}{}{s.m.}{Ato ou efeito de amolecer.}{a.mo.le.ci.men.to}{0}
\verb{amolentar}{}{}{}{}{v.t.}{Tornar um pouco mole; amolecer.}{a.mo.len.tar}{\verboinum{1}}
\verb{amolgar}{}{}{}{}{v.t.}{Deformar por esmagamento ou pancada; amassar.}{a.mol.gar}{\verboinum{5}}
\verb{amônia}{}{Quím.}{}{}{s.f.}{Solução aquosa do amoníaco.}{a.mô.nia}{0}
\verb{amoníaco}{}{Quím.}{}{}{s.m.}{Gás incolor composto de nitrogênio e hidrogênio, utilizados em fertilizantes, detergentes etc.}{a.mo.ní.a.co}{0}
\verb{amontoado}{}{}{}{}{adj.}{Agrupado de forma desordenada.}{a.mon.to.a.do}{0}
\verb{amontoado}{}{}{}{}{s.m.}{Conjunto de coisas agrupadas de forma desordenada.}{a.mon.to.a.do}{0}
\verb{amontoar}{}{}{}{}{v.t.}{Ajuntar de maneira desordenada.}{a.mon.to.ar}{\verboinum{7}}
\verb{amor}{ô}{}{}{}{s.m.}{Sentimento de afeição profunda.}{a.mor}{0}
\verb{amor}{ô}{}{}{}{}{Apego, interesse, fascínio.}{a.mor}{0}
\verb{amor}{ô}{}{}{}{}{O indivíduo amado.}{a.mor}{0}
\verb{amora}{ó}{}{}{}{s.f.}{Fruto da amoreira.}{a.mo.ra}{0}
\verb{amoral}{}{}{"-ais}{}{adj.2g.}{Que é moralmente neutro, não sendo contrário nem conforme os preceitos morais.}{a.mo.ral}{0}
\verb{amoralidade}{}{}{}{}{s.f.}{Qualidade daquele ou daquilo que é amoral.}{a.mo.ra.li.da.de}{0}
\verb{amorável}{}{}{"-eis}{}{adj.2g.}{Inclinado ao amor; meigo, afetuoso.}{a.mo.rá.vel}{0}
\verb{amorável}{}{}{"-eis}{}{}{Digno de ser amado.}{a.mo.rá.vel}{0}
\verb{amordaçamento}{}{}{}{}{s.m.}{Ato ou efeito de amordaçar.}{a.mor.da.ça.men.to}{0}
\verb{amordaçar}{}{}{}{}{v.t.}{Colocar mordaça.}{a.mor.da.çar}{0}
\verb{amordaçar}{}{Por ext.}{}{}{}{Impedir de falar, de emitir opinião.}{a.mor.da.çar}{\verboinum{3}}
\verb{amoreira}{ê}{Bot.}{}{}{s.f.}{Árvore nativa do Irã com frutos vermelhos comestíveis.}{a.mo.rei.ra}{0}
\verb{amorenado}{}{}{}{}{adj.}{Que tem a cor tendendo a tons morenos.}{a.mo.re.na.do}{0}
\verb{amorenar}{}{}{}{}{v.t.}{Tornar moreno.}{a.mo.re.nar}{\verboinum{1}}
\verb{amorfo}{ó}{}{}{}{adj.}{Que não tem forma ou figura determinada, definida.}{a.mor.fo}{0}
\verb{amorico}{}{}{}{}{s.m.}{Amor breve, passageiro, sem firmeza.}{a.mo.ri.co}{0}
\verb{amornar}{}{}{}{}{v.t.}{Tornar morno, aquecer levemente.}{a.mor.nar}{\verboinum{1}}
\verb{amoroso}{ô}{}{"-osos ⟨ó⟩}{"-osa ⟨ó⟩}{adj.}{Relativo a amor.}{a.mo.ro.so}{0}
\verb{amoroso}{ô}{}{"-osos ⟨ó⟩}{"-osa ⟨ó⟩}{}{Que é propenso ao amor.}{a.mo.ro.so}{0}
\verb{amor"-perfeito}{ô}{}{amores"-perfeitos ⟨ô⟩}{}{s.m.}{Planta ornamental de flores brancas, roxas e amarelas.}{a.mor"-per.fei.to}{0}
\verb{amor"-próprio}{ô}{}{amores"-próprios ⟨ô⟩}{}{s.m.}{Sentimento de respeito e dignidade que se tem por si mesmo.}{a.mor"-pró.prio}{0}
\verb{amor"-próprio}{ô}{Por ext.}{amores"-próprios ⟨ô⟩}{}{}{Vaidade, orgulho.}{a.mor"-pró.prio}{0}
\verb{amortalhar}{}{}{}{}{v.t.}{Envolver em mortalha; cobrir com panos semelhantes à mortalha.}{a.mor.ta.lhar}{0}
\verb{amortalhar}{}{}{}{}{v.pron.}{Por penitência ou desprendimento, usar vestuários simples e grosseiros.}{a.mor.ta.lhar}{\verboinum{1}}
\verb{amortecedor}{ô}{}{}{}{adj.}{Que amortece; diminui a força, a intensidade.}{a.mor.te.ce.dor}{0}
\verb{amortecedor}{ô}{}{}{}{}{Que diminui ou abafa o som; silenciador.}{a.mor.te.ce.dor}{0}
\verb{amortecedor}{ô}{}{}{}{s.m.}{Qualquer dispositivo utilizado para diminuir vibrações ou choques de máquinas, automóveis etc.}{a.mor.te.ce.dor}{0}
\verb{amortecer}{ê}{}{}{}{v.t.}{Tornar menos violento; fazer perder a força, a intensidade.}{a.mor.te.cer}{0}
\verb{amortecer}{ê}{}{}{}{}{Tornar dormente; entorpecer.}{a.mor.te.cer}{0}
\verb{amortecer}{ê}{Fig.}{}{}{}{Diminuir o vigor, o ânimo; acalmar, abrandar.}{a.mor.te.cer}{0}
\verb{amortecer}{ê}{}{}{}{v.i.}{Desfalecer, desmaiar.}{a.mor.te.cer}{\verboinum{15}}
\verb{amortização}{}{}{"-ões}{}{s.f.}{Ato ou efeito de amortizar; pagamento gradual de uma dívida.}{a.mor.ti.za.ção}{0}
\verb{amortização}{}{}{"-ões}{}{}{Cada uma das parcelas pagas para saldar uma dívida.}{a.mor.ti.za.ção}{0}
\verb{amortizar}{}{}{}{}{v.t.}{Saldar uma dívida gradativamente ou em prestações.}{a.mor.ti.zar}{\verboinum{1}}
\verb{amostra}{ó}{}{}{}{s.f.}{Porção de produto ou substância usada para demonstrar ou exemplificar suas qualidades ou propriedades.}{a.mos.tra}{0}
\verb{amostra}{ó}{}{}{}{}{Fração ou parcela representativa de um todo estatístico; amostragem.}{a.mos.tra}{0}
\verb{amostragem}{}{}{"-ens}{}{s.f.}{Processo ou método de escolha de uma amostra da qual se coletam informações.}{a.mos.tra.gem}{0}
\verb{amostrar}{}{}{}{}{v.t.}{Expor um produto ou uma substância para análise; demonstrar, mostrar.}{a.mos.trar}{0}
\verb{amostrar}{}{}{}{}{}{Extrair ou colher uma amostra.}{a.mos.trar}{\verboinum{1}}
\verb{amotinado}{}{}{}{}{adj.}{Que se amotinou; rebelde, agitador.}{a.mo.ti.na.do}{0}
\verb{amotinar}{}{}{}{}{v.t.}{Incitar, provocar motim; revoltar, perturbar.}{a.mo.ti.nar}{0}
\verb{amotinar}{}{}{}{}{}{Agitar, alvoroçar.}{a.mo.ti.nar}{\verboinum{1}}
\verb{amovível}{}{}{"-eis}{}{adj.2g.}{Que se pode deslocar; removível, deslocável.}{a.mo.ví.vel}{0}
\verb{amovível}{}{}{"-eis}{}{}{Que é transitório; não vitalício.}{a.mo.ví.vel}{0}
\verb{amparar}{}{}{}{}{v.t.}{Dar apoio; escorar.}{am.pa.rar}{0}
\verb{amparar}{}{}{}{}{}{Dar proteção; defender, resguardar.}{am.pa.rar}{0}
\verb{amparar}{}{}{}{}{}{Favorecer, estimular, encorajar.}{am.pa.rar}{\verboinum{1}}
\verb{amparo}{}{}{}{}{s.m.}{Ato ou efeito de amparar.}{am.pa.ro}{0}
\verb{amparo}{}{}{}{}{}{Esteio, apoio, sustentação.}{am.pa.ro}{0}
\verb{amparo}{}{}{}{}{}{Abrigo, proteção, refúgio.}{am.pa.ro}{0}
\verb{amperagem}{}{Fís.}{"-ens}{}{s.f.}{Intensidade de uma corrente elétrica medida em \textit{ampères}.}{am.pe.ra.gem}{0}
\verb{ampère}{}{Fís.}{}{}{s.m.}{Unidade de medida de intensidade de corrente elétrica no sistema internacional.}{\textit{ampère}}{0}
\verb{amplexo}{écs}{}{}{}{s.m.}{Ato ou efeito de abraçar; abraço.}{am.ple.xo}{0}
\verb{ampliação}{}{}{"-ões}{}{s.f.}{Ato ou efeito de ampliar; aumento, dilatação.}{am.pli.a.ção}{0}
\verb{ampliador}{ô}{}{}{}{adj.}{Que amplia, aumenta.}{am.pli.a.dor}{0}
\verb{ampliador}{ô}{}{}{}{s.m.}{Aparelho utilizado na técnica fotográfica para obter reproduções ampliadas de negativos.}{am.pli.a.dor}{0}
\verb{ampliar}{}{}{}{}{v.t.}{Tornar maior; aumentar, dilatar.}{am.pli.ar}{0}
\verb{ampliar}{}{}{}{}{}{Alargar ou alongar o comprimento ou a área.}{am.pli.ar}{0}
\verb{ampliar}{}{}{}{}{}{Tornar mais extenso; desenvolver.}{am.pli.ar}{0}
\verb{ampliar}{}{}{}{}{}{Exagerar, exceder.}{am.pli.ar}{\verboinum{1}}
\verb{amplidão}{}{}{"-ões}{}{s.f.}{Qualidade daquilo que é amplo, extenso; vastidão, grandeza, amplitude.}{am.pli.dão}{0}
\verb{amplidão}{}{}{"-ões}{}{}{Espaço sem limites definidos, longínquo, remoto.}{am.pli.dão}{0}
\verb{amplificador}{ô}{}{}{}{adj.}{Que amplifica, aumenta, intensifica.}{am.pli.fi.ca.dor}{0}
\verb{amplificador}{ô}{}{}{}{s.m.}{Qualquer aparelho usado para aumentar um sinal de áudio ou de vídeo.}{am.pli.fi.ca.dor}{0}
\verb{amplificar}{}{}{}{}{v.t.}{Aumentar em tamanho, intensidade ou extensão.}{am.pli.fi.car}{0}
\verb{amplificar}{}{Fig.}{}{}{}{Tornar maior em dignidade ou valor; exaltar, engrandecer.}{am.pli.fi.car}{\verboinum{2}}
\verb{amplitude}{}{}{}{}{s.f.}{Grande extensão; vastidão, amplidão.}{am.pli.tu.de}{0}
\verb{amplo}{}{}{}{}{adj.}{Que tem grande extensão; vasto, espaçoso.}{am.plo}{0}
\verb{amplo}{}{}{}{}{}{Que não tem limites; irrestrito.}{am.plo}{0}
\verb{amplo}{}{}{}{}{}{Que é farto, abundante, rico.}{am.plo}{0}
\verb{ampola}{ô}{}{}{}{s.f.}{Pequeno recipiente de vidro, em forma de tubo, que, após receber um líquido, é fechado hermeticamente.}{am.po.la}{0}
\verb{ampola}{ô}{}{}{}{}{O conteúdo desse recipiente.}{am.po.la}{0}
\verb{ampulheta}{ê}{}{}{}{s.f.}{Instrumento composto de dois cones de vidro, que se comunicam pelo vértice, e sobre o qual escoa uma certa quantidade de areia, servindo para medir o tempo; relógio de areia.}{am.pu.lhe.ta}{0}
\verb{amputação}{}{}{"-ões}{}{s.f.}{Ato ou efeito de amputar, cortar.}{am.pu.ta.ção}{0}
\verb{amputação}{}{Med.}{"-ões}{}{}{Procedimento cirúrgico que visa à retirada de um membro ou de um órgão do corpo.}{am.pu.ta.ção}{0}
\verb{amputação}{}{Fig.}{"-ões}{}{}{Diminuição, restrição.}{am.pu.ta.ção}{0}
\verb{amputar}{}{}{}{}{v.t.}{Cortar um órgão ou um membro do corpo; decepar, mutilar.}{am.pu.tar}{0}
\verb{amputar}{}{}{}{}{}{Pôr limites; diminuir, reduzir.}{am.pu.tar}{0}
\verb{amputar}{}{Fig.}{}{}{}{Eliminar, suprimir.}{am.pu.tar}{\verboinum{1}}
\verb{amuado}{}{}{}{}{adj.}{Que se amuou.}{a.mu.a.do}{0}
\verb{amuado}{}{}{}{}{}{Mal"-humorado, enfadado.}{a.mu.a.do}{0}
\verb{amuado}{}{}{}{}{}{Que não amadureceu; não se desenvolveu.}{a.mu.a.do}{0}
\verb{amuar}{}{}{}{}{v.t.}{Provocar mau humor; aborrecer, enfadar.}{a.mu.ar}{0}
\verb{amuar}{}{}{}{}{}{Insistir, teimar, importunar.}{a.mu.ar}{0}
\verb{amuar}{}{}{}{}{v.i.}{Não se desenvolver; não amadurecer.}{a.mu.ar}{\verboinum{1}}
\verb{amulatado}{}{}{}{}{adj.}{Que se amulatou; que tem cor ou aparência de mulato. }{a.mu.la.ta.do}{0}
\verb{amulatar"-se}{}{}{}{}{v.pron.}{Tornar"-se mulato na cor ou na aparência.}{a.mu.la.tar"-se}{\verboinum{1}}
\verb{amuleto}{ê}{}{}{}{s.m.}{Objeto ao qual se atribuem poderes mágicos de defesa contra desgraças, doenças ou feitiços; talismã.}{a.mu.le.to}{0}
\verb{amuo}{}{}{}{}{s.m.}{Estado de mau humor ou enfado manifestado em gestos, palavras ou na própria fisionomia.}{a.mu.o}{0}
\verb{amurada}{}{}{}{}{s.f.}{Prolongamento do costado do navio, acima do convés, que serve de parapeito para a tripulação.}{a.mu.ra.da}{0}
\verb{amurada}{}{Por ext.}{}{}{}{Muro de arrimo; paredão.}{a.mu.ra.da}{0}
\verb{amuralhar}{}{}{}{}{v.t.}{Cercar de muralhas ou muros; amurar.}{a.mu.ra.lhar}{\verboinum{1}}
\verb{amurar}{}{}{}{}{v.t.}{Cercar de muros; amuralhar.}{a.mu.rar}{\verboinum{1}}
\verb{anã}{}{}{}{}{s.f.}{Feminino de \textit{anão}.}{a.nã}{0}
\verb{anã}{}{Astron.}{}{}{}{Diz"-se de estrela de pequeno volume e  baixa luminosidade.}{a.nã}{0}
\verb{anabatismo}{}{}{}{}{s.m.}{Seita protestante, do século \textsc{xvi}, que desaprovava o batismo das crianças e as rebatizava na idade adulta, quando podiam optar pela religião conscientemente.}{a.na.ba.tis.mo}{0}
\verb{anabatista}{}{}{}{}{adj.2g.}{Relativo a anabatismo.}{a.na.ba.tis.ta}{0}
\verb{anabatista}{}{}{}{}{s.2g.}{Membro dessa seita protestante.}{a.na.ba.tis.ta}{0}
\verb{anabólico}{}{Bioquím.}{}{}{adj.}{Relativo a anabolismo.}{a.na.bó.li.co}{0}
\verb{anabolismo}{}{Bioquím.}{}{}{s.m.}{Conjunto de processos metabólicos que transformam os nutrientes em tecidos vivos; assimilação orgânica.}{a.na.bo.lis.mo}{0}
\verb{anabolizante}{}{Bioquím.}{}{}{adj.2g.}{Diz"-se da substância que favorece a assimilação dos nutrientes digeridos.}{a.na.bo.li.zan.te}{0}
\verb{anacoluto}{}{Gram.}{}{}{s.m.}{Interrupção da ordem lógica da frase, por causa da presença de palavra ou expressão sem ligação sintática com os demais elementos do período.}{a.na.co.lu.to}{0}
\verb{anaconda}{}{Zool.}{}{}{s.f.}{Serpente de cor esverdeada que vive perto de rios e pântanos, não venenosa e que engole sua presa após triturá"-la por compressão; sucuri.}{a.na.con.da}{0}
\verb{anacoreta}{ê}{}{}{}{s.2g.}{Religioso cristão que se afasta da sociedade para viver em retiro solitário; eremita.}{a.na.co.re.ta}{0}
\verb{anacoreta}{ê}{Por ext.}{}{}{}{Pessoa que vive afastada do convívio social.}{a.na.co.re.ta}{0}
\verb{anacrônico}{}{}{}{}{adj.}{Que está em desacordo com os usos da época em que se encontra.}{a.na.crô.ni.co}{0}
\verb{anacrônico}{}{}{}{}{}{Contrário ao que é atual; retrógrado, antiquado, obsoleto.}{a.na.crô.ni.co}{0}
\verb{anacronismo}{}{}{}{}{s.m.}{Erro ou confusão quanto a referências de uma época.}{a.na.cro.nis.mo}{0}
\verb{anacronismo}{}{}{}{}{}{Acontecimento ou atitude que está em desacordo com os usos de uma dada época.}{a.na.cro.nis.mo}{0}
\verb{anacronizar}{}{}{}{}{v.t.}{Expor ou relatar informações cometendo erros ou confusões quanto a referências de época.}{a.na.cro.ni.zar}{\verboinum{1}}
\verb{anaeróbico}{}{Biol.}{}{}{adj.}{Anaeróbio.}{a.na.e.ró.bi.co}{0}
\verb{anaeróbico}{}{}{}{}{}{Diz"-se dos exercícios que envolvem movimentos rápidos e intensos seguidos de descanso.}{a.na.e.ró.bi.co}{0}
\verb{anaeróbio}{}{Biol.}{}{}{adj.}{Diz"-se do organismo que vive em ambientes sem a presença de ar ou oxigênio livre; anaeróbico.}{a.na.e.ró.bio}{0}
\verb{anafilático}{}{Med.}{}{}{adj.}{Relativo a anafilaxia; alérgico.}{a.na.fi.lá.ti.co}{0}
\verb{anafilaxia}{cs}{Med.}{}{}{s.f.}{Reação de hipersensibilidade de um organismo diante de um segundo contato com uma substância estranha.}{a.na.fi.la.xi.a}{0}
\verb{anáfora}{}{Gram.}{}{}{s.f.}{Repetição de palavras ou expressões no início de frases ou versos sucessivos.}{a.ná.fo.ra}{0}
\verb{anagrama}{}{}{}{}{s.m.}{Palavra ou frase criada a partir da troca de posição de letras ou sílabas de uma outra palavra.}{a.na.gra.ma}{0}
\verb{anágua}{}{}{}{}{s.f.}{Roupa íntima feminina que se usa por baixo do vestido ou da saia; saiote.}{a.ná.gua}{0}
\verb{anais}{}{}{}{}{s.m.pl.}{História ou narração contada ano a ano.}{a.nais}{0}
\verb{anais}{}{}{}{}{}{Publicação periódica de caráter científico, literário ou artístico.}{a.nais}{0}
\verb{anajá}{}{Bot.}{}{}{s.m.}{Palmeira nativa do Brasil, cujos palmito e fruto são comestíveis e de cuja amêndoa se extrai um óleo amarelo, também comestível.}{a.na.já}{0}
\verb{anal}{}{}{}{}{adj.2g.}{Relativo a ânus.}{a.nal}{0}
\verb{anal}{}{Desus.}{}{}{adj.2g.}{Anual.}{a.nal}{0}
\verb{analfabetismo}{}{}{}{}{s.m.}{Condição de analfabeto.}{a.nal.fa.be.tis.mo}{0}
\verb{analfabetismo}{}{}{}{}{}{Porcentagem de pessoas analfabetas em um universo determinado.}{a.nal.fa.be.tis.mo}{0}
\verb{analfabeto}{é}{}{}{}{adj.}{Que não sabe ler nem escrever.}{a.nal.fa.be.to}{0}
\verb{analfabeto}{é}{Por ext.}{}{}{}{Ignorante.}{a.nal.fa.be.to}{0}
\verb{analgesia}{}{Med.}{}{}{s.f.}{Ausência de sensibilidade à dor.}{a.nal.ge.si.a}{0}
\verb{analgésico}{}{}{}{}{adj.}{Que diminui ou suprime a dor.}{a.nal.gé.si.co}{0}
\verb{analgésico}{}{}{}{}{s.m.}{Substância que suprime a dor.}{a.nal.gé.si.co}{0}
\verb{analisar}{}{}{}{}{v.t.}{Decompor em partes.}{a.na.li.sar}{0}
\verb{analisar}{}{}{}{}{}{Examinar minuciosamente.}{a.na.li.sar}{\verboinum{1}}
\verb{análise}{}{}{}{}{s.f.}{Ato ou efeito de analisar.}{a.ná.li.se}{0}
\verb{análise}{}{}{}{}{}{Exame minucioso.}{a.ná.li.se}{0}
\verb{análise}{}{}{}{}{s.f.}{Forma reduzida de \textit{psicanálise}.}{a.ná.li.se}{0}
\verb{analista}{}{}{}{}{adj.2g.}{Que analisa.}{a.na.lis.ta}{0}
\verb{analista}{}{Desus.}{}{}{s.2g.}{Pessoa que redige anais.}{a.na.lis.ta}{0}
\verb{analista}{}{}{}{}{s.2g.}{Indivíduo especialista em psicanálise; psicanalista.}{a.na.lis.ta}{0}
\verb{analítico}{}{}{}{}{adj.}{Relativo a análise.}{a.na.lí.ti.co}{0}
\verb{analogia}{}{}{}{}{s.f.}{Relação de semelhança entre coisas ou fatos.}{a.na.lo.gi.a}{0}
\verb{analogia}{}{}{}{}{}{Comparação ou raciocínio baseado na semelhança.}{a.na.lo.gi.a}{0}
\verb{analógico}{}{}{}{}{adj.}{Relativo a analogia.}{a.na.ló.gi.co}{0}
\verb{analógico}{}{Mat.}{}{}{}{Representação de grandezas de modo contínuo (em oposição a discreto).}{a.na.ló.gi.co}{0}
\verb{análogo}{}{}{}{}{adj.}{Que apresenta analogia, que se baseia em analogia; semelhante, similar.}{a.ná.lo.go}{0}
\verb{ananás}{}{}{}{}{s.m.}{Abacaxi.}{a.na.nás}{0}
\verb{ananaseiro}{ê}{Bot.}{}{}{s.m.}{Abacaxi.}{a.na.na.sei.ro}{0}
\verb{anão}{}{}{"-ões \textit{ou} -ãos}{anã}{s.m.}{Pessoa que sofre de nanismo.}{a.não}{0}
\verb{anão}{}{}{"-ões \textit{ou} -ãos}{anã}{adj.}{De estatura muito abaixo do normal.}{a.não}{0}
\verb{anarquia}{}{}{}{}{s.f.}{Sistema político baseado na ausência de autoridade e de qualquer forma de coação sobre o indivíduo.}{a.nar.qui.a}{0}
\verb{anarquia}{}{Por ext.}{}{}{}{Falta de organização; confusão, desordem.}{a.nar.qui.a}{0}
\verb{anárquico}{}{}{}{}{adj.}{Relativo a anarquia.}{a.nár.qui.co}{0}
\verb{anarquismo}{}{}{}{}{s.m.}{Teoria política que preconiza a substituição do Estado pela cooperação entre grupos ou indivíduos, considerando aquele dispensável e prejudicial ao estabelecimento de uma autêntica comunidade humana.}{a.nar.quis.mo}{0}
\verb{anarquista}{}{}{}{}{adj.2g.}{Relativo a anarquismo.}{a.nar.quis.ta}{0}
\verb{anarquista}{}{}{}{}{s.2g.}{Partidário do anarquismo; libertário.}{a.nar.quis.ta}{0}
\verb{anarquizar}{}{}{}{}{v.t.}{Tornar anárquico.}{a.nar.qui.zar}{0}
\verb{anarquizar}{}{}{}{}{}{Pôr em desordem; sublevar. }{a.nar.qui.zar}{\verboinum{1}}
\verb{anástrofe}{}{Gram.}{}{}{s.f.}{Inversão da ordem natural das palavras.}{a.nás.tro.fe}{0}
\verb{anátema}{}{Relig.}{}{}{s.m.}{Expulsão da Igreja; excomunhão.}{a.ná.te.ma}{0}
\verb{anátema}{}{}{}{}{}{Maldição, execração.}{a.ná.te.ma}{0}
\verb{anatematizar}{}{}{}{}{v.t.}{Excomungar.}{a.na.te.ma.ti.zar}{\verboinum{1}}
\verb{anatídeo}{}{Zool.}{}{}{adj.}{Relativo aos anatídeos, família de aves aquáticas de bico largo, pernas curtas e pés com membranas natatórias, que inclui patos, cisnes, marrecos etc. }{a.na.tí.deo}{0}
\verb{anatifa}{}{Zool.}{}{}{s.m.}{Crustáceo que se fixa ao casco dos navios.}{a.na.ti.fa}{0}
\verb{anatomia}{}{Biol.}{}{}{s.f.}{Área das ciências biológicas que estuda a forma e a estrutura dos seres vivos ou de seus órgãos.}{a.na.to.mi.a}{0}
\verb{anatomia}{}{Fig.}{}{}{}{Aspecto externo do corpo humano; compleição.}{a.na.to.mi.a}{0}
\verb{anatômico}{}{}{}{}{adj.}{Relativo a anatomia.}{a.na.tô.mi.co}{0}
\verb{anatômico}{}{}{}{}{}{Diz"-se de objeto ou equipamento que se adapta às formas do corpo humano.}{a.na.tô.mi.co}{0}
\verb{anatomista}{}{}{}{}{s.2g.}{Especialista em anatomia.}{a.na.to.mis.ta}{0}
\verb{anatomizar}{}{}{}{}{v.t.}{Fazer dissecação anatômica.}{a.na.to.mi.zar}{0}
\verb{anatomizar}{}{}{}{}{}{Estudar minuciosamente.}{a.na.to.mi.zar}{\verboinum{1}}
\verb{anavalhar}{}{}{}{}{v.t.}{Ferir com navalha.}{a.na.va.lhar}{\verboinum{1}}
\verb{anca}{}{}{}{}{s.f.}{Quarto traseiro dos quadrúpedes.}{an.ca}{0}
\verb{anca}{}{}{}{}{}{Cadeiras, nádegas.}{an.ca}{0}
\verb{ancestral}{}{}{"-ais}{}{adj.2g.}{Próprio dos antepassados.}{an.ces.tral}{0}
\verb{ancestral}{}{}{"-ais}{}{}{Muito antigo; remoto.}{an.ces.tral}{0}
\verb{ancestral}{}{}{"-ais}{}{s.2g.}{Familiar antepassado.}{an.ces.tral}{0}
\verb{ancho}{}{}{}{}{adj.}{Largo, amplo.}{an.cho}{0}
\verb{anchova}{ô}{Zool.}{}{}{s.f.}{Peixe do Atlântico, semelhante à manjuba brasileira, de grande valor comercial.}{an.cho.va}{0}
\verb{ancião}{}{}{"-ãos, -ães \textit{ou} -ões}{anciã}{adj.}{De idade avançada.}{an.ci.ão}{0}
\verb{ancião}{}{}{"-ãos, -ães \textit{ou} -ões}{anciã}{s.m.}{Homem muito velho e respeitável.}{an.ci.ão}{0}
\verb{ancila}{}{}{}{}{s.f.}{Escrava, serva.}{an.ci.la}{0}
\verb{ancilose}{ó}{Med.}{}{}{s.f.}{Diminuição da mobilidade de uma articulação.}{an.ci.lo.se}{0}
\verb{ancilostomíase}{}{Med.}{}{}{s.f.}{Infecção caracterizada por grave anemia; amarelão; ancilostomose.}{an.ci.los.to.mí.a.se}{0}
\verb{ancilóstomo}{}{Biol.}{}{}{s.m.}{Verme parasita causador da ancilostomíase.}{an.ci.lós.to.mo}{0}
\verb{ancilostomose}{ó}{Med.}{}{}{s.f.}{Ancilostomíase.}{an.ci.los.to.mo.se}{0}
\verb{ancinho}{}{}{}{}{s.m.}{Ferramenta de cabo longo com travessa dentada usada para juntar folhas secas e para preparar a terra para o plantio; rastelo.}{an.ci.nho}{0}
\verb{âncora}{}{}{}{}{s.f.}{Peça de ferro presa a uma corrente utilizada para manter a embarcação parada.}{ân.co.ra}{0}
\verb{âncora}{}{Fig.}{}{}{}{Proteção, abrigo, arrimo, esteio.}{ân.co.ra}{0}
\verb{ancoradouro}{ô}{}{}{}{s.m.}{Lugar próprio para a ancoragem.}{an.co.ra.dou.ro}{0}
\verb{ancoragem}{}{}{"-ens}{}{s.f.}{Ato ou efeito de ancorar.}{an.co.ra.gem}{0}
\verb{ancorar}{}{}{}{}{v.i.}{Lançar âncora ao fundo para manter a embarcação parada.}{an.co.rar}{0}
\verb{ancorar}{}{Fig.}{}{}{}{Sustentar, fixar.}{an.co.rar}{\verboinum{1}}
\verb{ancoreta}{ê}{}{}{}{s.f.}{Pequeno barril para transportar água, aguardente ou vinho.}{an.co.re.ta}{0}
\verb{andaço}{}{}{}{}{s.m.}{Pequena epidemia, geralmente localizada e de pouca gravidade.}{an.da.ço}{0}
\verb{andaço}{}{Bras.}{}{}{}{Diarreia, disenteria.}{an.da.ço}{0}
\verb{andada}{}{}{}{}{s.f.}{Ato ou efeito de andar; caminhada.}{an.da.da}{0}
\verb{andador}{ô}{}{}{}{adj.}{Que anda muito ou anda depressa.}{an.da.dor}{0}
\verb{andador}{ô}{Bras.}{}{}{}{Diz"-se de cavalo ensinado ou bastante ágil.}{an.da.dor}{0}
\verb{andador}{ô}{}{}{}{s.m.}{Aparelho usado como apoio por crianças ou adultos com dificuldade de locomoção.}{an.da.dor}{0}
\verb{andadura}{}{}{}{}{s.f.}{Modo de andar, especialmente o das cavalgaduras.}{an.da.du.ra}{0}
\verb{andadura}{}{}{}{}{}{No hipismo, tipo de passo dado pelos animais.}{an.da.du.ra}{0}
\verb{andaime}{}{}{}{}{s.m.}{Estrado provisório sobre o qual trabalham os operários nas construções.}{an.dai.me}{0}
\verb{andaluz}{}{}{}{}{adj.2g.}{Relativo à Andaluzia, região no sul da Espanha.}{an.da.luz}{0}
\verb{andaluz}{}{}{}{}{s.2g.}{Indivíduo natural ou habitante dessa região.}{an.da.luz}{0}
\verb{andamento}{}{}{}{}{s.m.}{Modo de andar.}{an.da.men.to}{0}
\verb{andamento}{}{}{}{}{}{Prosseguimento, encaminhamento.}{an.da.men.to}{0}
\verb{andamento}{}{Mús.}{}{}{}{Velocidade com que se executa uma música.}{an.da.men.to}{0}
\verb{andança}{}{}{}{}{s.f.}{Ação de andar; viagem, peregrinação.}{an.dan.ça}{0}
\verb{andante}{}{}{}{}{adj.2g.}{Que anda, errante, aventureiro.}{an.dan.te}{0}
\verb{andante}{}{Mús.}{}{}{s.m.}{Andamento entre o adágio e o alegro.}{an.dan.te}{0}
\verb{andar}{}{}{}{}{v.i.}{Caminhar, dar passos.}{an.dar}{0}
\verb{andar}{}{}{}{}{}{Percorrer, mover"-se.}{an.dar}{0}
\verb{andar}{}{}{}{}{}{Proceder, comportar"-se.}{an.dar}{0}
\verb{andar}{}{}{}{}{v.pred.}{Estar, viver em determinada condição.}{an.dar}{\verboinum{1}}
\verb{andar}{}{}{}{}{s.m.}{Encaminhamento, transcurso.}{an.dar}{0}
\verb{andar}{}{}{}{}{}{Cada pavimento de um edifício.}{an.dar}{0}
\verb{andarilho}{}{}{}{}{s.m.}{Indivíduo que anda muito, que gosta muito de andar.}{an.da.ri.lho}{0}
\verb{andas}{}{}{}{}{s.f.pl.}{Pernas"-de"-pau.}{an.das}{0}
\verb{andejo}{ê}{}{}{}{adj.}{Que anda muito.}{an.de.jo}{0}
\verb{andejo}{ê}{Fig.}{}{}{}{Inconstante, desvairado.}{an.de.jo}{0}
\verb{andino}{}{}{}{}{adj.}{Relativo aos Andes.}{an.di.no}{0}
\verb{andino}{}{}{}{}{s.2g.}{Indivíduo natural ou habitante dos Andes.}{an.di.no}{0}
\verb{andiroba}{ó}{Bot.}{}{}{s.f.}{Árvore tropical de flores amarelas ou vermelhas, madeira de qualidade, casca com propriedades medicinais e de cujas sementes se extrai óleo com diversas aplicações.}{an.di.ro.ba}{0}
\verb{andor}{ô}{Relig.}{}{}{s.m.}{Tabuleiro ornamentado em que se transportam imagens nas procissões.}{an.dor}{0}
\verb{andorinha}{}{Zool.}{}{}{s.f.}{Ave migratória de pequeno porte, bico curto e asas longas e pontiagudas.}{an.do.ri.nha}{0}
\verb{andorinhão}{}{Bras.}{"-ões}{}{s.m.}{Ave de coloração parda, lustro metálico, peito branco e asas longas.}{an.do.ri.nhão}{0}
\verb{andorrano}{}{}{}{}{adj.}{Relativo a Andorra, principado localizado na região dos Pireneus, na Europa.}{an.dor.ra.no}{0}
\verb{andorrano}{}{}{}{}{s.2g.}{Indivíduo natural ou habitante de Andorra; andorrense.}{an.dor.ra.no}{0}
\verb{andorrense}{}{}{}{}{adj.2g.}{Relativo a Andorra.}{an.dor.ren.se}{0}
\verb{andorrense}{}{}{}{}{s.2g.}{Indivíduo habitante ou natural desse país; andorrano.}{an.dor.ren.se}{0}
\verb{andrajo}{}{}{}{}{s.m.}{Trapo, farrapo.}{an.dra.jo}{0}
\verb{andrajos}{}{}{}{}{s.m.pl.}{Roupas sujas ou rasgadas.}{an.dra.jos}{0}
\verb{andrajoso}{ô}{}{"-osos ⟨ó⟩}{"-osa ⟨ó⟩}{adj.}{Coberto de trapos; esfarrapado, maltrapilho.}{an.dra.jo.so}{0}
\verb{androceu}{}{Bot.}{}{}{s.m.}{Conjunto dos órgãos masculinos da flor, representados pelos estames.}{an.dro.ceu}{0}
\verb{androfobia}{}{Med.}{}{}{s.f.}{Aversão ou horror ao sexo masculino.}{an.dro.fo.bi.a}{0}
\verb{andrófobo}{}{}{}{}{adj.}{Que tem horror ao sexo masculino.}{an.dró.fo.bo}{0}
\verb{androginia}{}{Biol.}{}{}{s.f.}{Qualidade ou característica do que é andrógino; hermafroditismo.}{an.dro.gi.ni.a}{0}
\verb{andrógino}{}{Biol.}{}{}{adj.}{Que apresenta características masculinas e femininas; hermafrodita.}{an.dró.gi.no}{0}
\verb{andrógino}{}{Bot.}{}{}{}{Diz"-se da planta que possui os aparelhos reprodutores feminino e masculino no mesmo pedúnculo.}{an.dró.gi.no}{0}
\verb{androide}{}{}{}{}{adj.2g.}{Que apresenta forma semelhante à humana.}{an.droi.de}{0}
\verb{androide}{}{}{}{}{s.m.}{Robô que apresenta forma humana. }{an.droi.de}{0}
\verb{andrômeda}{}{Astron.}{}{}{s.f.}{Constelação vista no Hemisfério Norte, próxima às constelações de Peixes e Pégaso.}{an.drô.me.da}{0}
\verb{andrômeda}{}{Astron.}{}{}{}{A galáxia Andrômeda, única espiral gigante visível a olho nu.}{an.drô.me.da}{0}
\verb{andrômeda}{}{Bot.}{}{}{}{Nome comum às plantas do gênero \textit{Andromeda}, nativas das regiões temperadas do Hemisfério Norte, cultivadas como ornamentais.}{an.drô.me.da}{0}
\verb{andropausa}{}{Med.}{}{}{s.f.}{Período da vida do homem caracterizado pela diminuição das funções sexuais, decorrente de involução glandular.}{an.dro.pau.sa}{0}
\verb{anedota}{ó}{}{}{}{s.f.}{Particularidade ou episódio curioso de figura ou acontecimento histórico que é pouco divulgado.}{a.ne.do.ta}{0}
\verb{anedota}{ó}{}{}{}{}{Relato curto e divertido; piada.}{a.ne.do.ta}{0}
\verb{anedotário}{}{}{}{}{s.m.}{Coleção de anedotas.}{a.ne.do.tá.rio}{0}
\verb{anedótico}{}{}{}{}{adj.}{Relativo a anedota.}{a.ne.dó.ti.co}{0}
\verb{anedótico}{}{}{}{}{}{Engraçado, divertido.}{a.ne.dó.ti.co}{0}
\verb{anedotista}{}{}{}{}{s.2g.}{Indivíduo que conta ou coleciona anedotas; piadista.}{a.ne.do.tis.ta}{0}
\verb{anel}{é}{}{"-éis}{}{s.m.}{Pequeno objeto circular com que se prende algo; aro, elo, argola.}{a.nel}{0}
\verb{anel}{é}{}{"-éis}{}{}{Qualquer objeto, figura ou órgão de forma circular.}{a.nel}{0}
\verb{anel}{é}{}{"-éis}{}{}{Pequeno círculo de metal, plástico ou madeira, com ou sem ornamento, usado nos dedos da mão como enfeite ou símbolo.}{a.nel}{0}
\verb{anel}{é}{}{"-éis}{}{}{Cacho ou caracol do cabelo.}{a.nel}{0}
\verb{anelado}{}{}{}{}{adj.}{Que tem forma de anel; anelar, aneliforme.}{a.ne.la.do}{0}
\verb{anelado}{}{}{}{}{}{Diz"-se do cabelo disposto em anéis; encaracolado, cacheado.}{a.ne.la.do}{0}
\verb{anelante}{}{}{}{}{adj.}{Que respira com dificuldade; ofegante.}{a.ne.lan.te}{0}
\verb{anelante}{}{}{}{}{}{Que deseja ou anseia com ardor; ansioso.}{a.ne.lan.te}{0}
\verb{anelão}{}{}{"-ões}{}{s.m.}{Anel grande e grosso, de prata ou de ouro.}{a.ne.lão}{0}
\verb{anelar}{}{}{}{}{adj.2g.}{Que tem forma de anel; anelado, aneliforme.}{a.ne.lar}{0}
\verb{anelar}{}{}{}{}{v.t.}{Dar forma de anel; encaracolar, cachear.}{a.ne.lar}{\verboinum{1}}
\verb{anelar}{}{}{}{}{v.t.}{Respirar com dificuldade; ofegar.}{a.ne.lar}{0}
\verb{anelar}{}{}{}{}{}{Desejar com ardor; ansiar, almejar.}{a.ne.lar}{\verboinum{1}}
\verb{anelídeo}{}{Zool.}{}{}{adj.}{Diz"-se dos animais invertebrados, com forma de verme, pertencentes ao filo dos anelídeos, que possuem o corpo dividido em segmentos semelhantes a anéis, como a minhoca, a sanguessuga etc.}{a.ne.lí.deo}{0}
\verb{aneliforme}{ó}{}{}{}{adj.2g.}{Que tem forma de anel; anelar, anelado.}{a.ne.li.for.me}{0}
\verb{anelo}{é}{}{}{}{s.m.}{Desejo intenso; anseio, aspiração.}{a.ne.lo}{0}
\verb{anemia}{}{Med.}{}{}{s.f.}{Diminuição do volume dos glóbulos vermelhos do sangue que ocasiona sintomas como fraqueza, palidez etc.}{a.ne.mi.a}{0}
\verb{anemia}{}{Por ext.}{}{}{}{Fraqueza, abatimento, debilidade.}{a.ne.mi.a}{0}
\verb{anêmico}{}{}{}{}{adj.}{Relativo a anemia.}{a.nê.mi.co}{0}
\verb{anêmico}{}{Fig.}{}{}{}{Sem vigor, sem brilho, sem relevo.}{a.nê.mi.co}{0}
\verb{anêmico}{}{Fig.}{}{}{}{Sem intensidade; fraco, frouxo.}{a.nê.mi.co}{0}
\verb{anêmico}{}{}{}{}{s.m.}{Indivíduo que sofre de anemia.}{a.nê.mi.co}{0}
\verb{anemofilia}{}{Bot.}{}{}{s.f.}{Tipo de polinização realizada pela ação dos ventos.}{a.ne.mo.fi.li.a}{0}
\verb{anemômetro}{}{}{}{}{s.m.}{Instrumento que mede a velocidade e a intensidade do vento e, por vezes, registra sua direção.}{a.ne.mô.me.tro}{0}
\verb{anêmona}{}{Bot.}{}{}{s.f.}{Nome comum de certas ervas ricas em glicosídeos tóxicos, cultivadas como ornamentais e por suas propriedades medicinais. }{a.nê.mo.na}{0}
\verb{anêmona"-do"-mar}{}{Zool.}{anêmonas"-do"-mar}{}{s.f.}{Animal celenterado marinho, sem esqueleto calcário, que vive solitário e se assemelha às flores da anêmona.}{a.nê.mo.na"-do"-mar}{0}
\verb{anequim}{}{Zool.}{"-ins}{}{s.m.}{Espécie de tubarão acinzentado, grande e muito agressivo, que costuma acompanhar os navios.}{a.ne.quim}{0}
\verb{anestesia}{}{Med.}{}{}{s.f.}{Suspensão parcial ou total da sensibilidade, resultante de fatores naturais ou provocada por alguma substância anestésica, com o intuito de aliviar ou evitar a dor.}{a.nes.te.si.a}{0}
\verb{anestesiar}{}{}{}{}{v.t.}{Provocar anestesia; suspender a sensibilidade à dor.}{a.nes.te.si.ar}{0}
\verb{anestesiar}{}{Fig.}{}{}{}{Tornar insensível, alheio; distrair.}{a.nes.te.si.ar}{\verboinum{1}}
\verb{anestésico}{}{}{}{}{adj.}{Diz"-se da substância que provoca anestesia, reduzindo ou eliminando a sensibilidade à dor.}{a.nes.té.si.co}{0}
\verb{anestesista}{}{Med.}{}{}{s.2g.}{Profissional que prepara e aplica a anestesia.}{a.nes.te.sis.ta}{0}
\verb{anético}{}{}{}{}{adj.}{Contrário à ética; aético.}{a.né.ti.co}{0}
\verb{aneurisma}{}{Med.}{}{}{s.m.}{Dilatação anormal das paredes de uma artéria, decorrente da diminuição da resistência local.}{a.neu.ris.ma}{0}
\verb{anexação}{cs}{}{"-ões}{}{s.f.}{Ato ou efeito de anexar; junção, incorporação.}{a.ne.xa.ção}{0}
\verb{anexar}{cs}{}{}{}{v.t.}{Juntar algo àquilo considerado como principal; unir.}{a.ne.xar}{0}
\verb{anexar}{cs}{}{}{}{}{Incorporar, acrescentar.}{a.ne.xar}{\verboinum{1}}
\verb{anexim}{ch}{}{}{}{s.m.}{Sentença popular; provérbio, adágio.}{a.ne.xim}{0}
\verb{anexo}{écs}{}{}{}{adj.}{Que se anexou; incorporado, acrescentado.}{a.ne.xo}{0}
\verb{anexo}{écs}{}{}{}{}{Que está subordinado ou dependente.}{a.ne.xo}{0}
\verb{anexo}{écs}{}{}{}{s.m.}{Aquilo que está ligado a uma coisa principal; acessório, suplemento, aditamento.}{a.ne.xo}{0}
\verb{anfetamina}{}{Quím.}{}{}{s.f.}{Substância utilizada para contrair os vasos sanguíneos e estimular o sistema nervoso.}{an.fe.ta.mi.na}{0}
\verb{anfíbio}{}{}{}{}{adj.}{Diz"-se do ser vivo, animal ou vegetal, que pode viver tanto em terra quanto na água.}{an.fí.bio}{0}
\verb{anfíbio}{}{}{}{}{}{Diz"-se do veículo que pode se locomover tanto na água quanto no solo.}{an.fí.bio}{0}
\verb{anfíbio}{}{Fig.}{}{}{}{Que apresenta sentimentos opostos, opiniões diferentes, vida dupla.}{an.fí.bio}{0}
\verb{anfíbio}{}{Zool.}{}{}{s.m.}{Classe de vertebrados que apresentam fase larval aquática e fase adulta terrestre ou aquática como os sapos, as rãs e as salamandras.}{an.fí.bio}{0}
\verb{anfibologia}{}{Gram.}{}{}{s.f.}{Duplicidade de sentido em uma construção sintática; ambiguidade.}{an.fi.bo.lo.gi.a}{0}
\verb{anfiteatro}{}{}{}{}{s.m.}{Antigo edifício oval ou circular, com arquibancadas e uma arena no centro, destinado a espetáculos públicos.}{an.fi.te.a.tro}{0}
\verb{anfiteatro}{}{}{}{}{}{Sala ou espaço ao ar livre, oval ou semicircular, cujos assentos são dispostos em forma de arquibancada e em cujo centro se realizam espetáculos.}{an.fi.te.a.tro}{0}
\verb{anfitrião}{}{}{"-ões}{}{s.m.}{Pessoa que recebe convidados em sua casa para jantares, festas, banquetes.}{an.fi.tri.ão}{0}
\verb{anfitrião}{}{}{"-ões}{}{}{Pessoa que hospeda convidados em sua casa.}{an.fi.tri.ão}{0}
\verb{ânfora}{}{}{}{}{s.f.}{Grande jarro de cerâmica com gargalo estreito e duas asas, utilizado para transporte e armazenamento de líquidos e cereais.}{ân.fo.ra}{0}
\verb{anfractuosidade}{}{}{}{}{s.f.}{Sinuosidade, saliência ou depressão irregulares.}{an.frac.tu.o.si.da.de}{0}
\verb{anfractuoso}{ô}{}{"-osos ⟨ó⟩}{"-osa ⟨ó⟩}{adj.}{Que apresenta saliências ou depressões irregulares; sinuoso.}{an.frac.tu.o.so}{0}
\verb{anfractuoso}{ô}{Fig.}{"-osos ⟨ó⟩}{"-osa ⟨ó⟩}{}{Deformado, disforme, torto.}{an.frac.tu.o.so}{0}
\verb{angariação}{}{}{"-ões}{}{s.f.}{Ato ou efeito de angariar; obtenção, recrutamento.}{an.ga.ri.a.ção}{0}
\verb{angariar}{}{}{}{}{v.t.}{Conseguir ou obter após solicitação.}{an.ga.ri.ar}{0}
\verb{angariar}{}{}{}{}{}{Atrair para si; conquistar, alcançar.}{an.ga.ri.ar}{0}
\verb{angariar}{}{}{}{}{}{Recrutar, aliciar, alistar.}{an.ga.ri.ar}{\verboinum{1}}
\verb{angélica}{}{Bot.}{}{}{s.f.}{Grande arbusto, de origem europeia, cultivado para fabricação de medicamentos, licores e perfumes.}{an.gé.li.ca}{0}
\verb{angelical}{}{}{"-ais}{}{adj.2g.}{Relativo ou semelhante aos anjos; angélico.}{an.ge.li.cal}{0}
\verb{angelical}{}{Fig.}{"-ais}{}{}{Puro, belo, perfeito.}{an.ge.li.cal}{0}
\verb{angélico}{}{}{}{}{adj.}{Relativo ou semelhante aos anjos; angelical.}{an.gé.li.co}{0}
\verb{angelim}{}{Bot.}{"-ins}{}{s.m.}{Nome dado a várias árvores da família das leguminosas, nativas do Brasil, que apresentam flores roxas ou vermelho"-escuras.}{an.ge.lim}{0}
\verb{ângelus}{}{Relig.}{}{}{s.m.}{Oração, em latim, em saudação à Virgem Maria, rezada ao amanhecer, ao meio"-dia e ao anoitecer.}{ân.ge.lus}{0}
\verb{angico}{}{Bot.}{}{}{s.m.}{Nome dado a várias árvores da família das leguminosas, que produzem madeira de boa qualidade e de cuja casca se extrai o tanino.}{an.gi.co}{0}
\verb{angina}{}{Med.}{}{}{s.f.}{Nome genérico das inflamações agudas da garganta e da faringe, de origem infecciosa, e que se caracterizam pela dificuldade de engolir e, às vezes, de respirar; tonsilite.}{an.gi.na}{0}
\verb{angina}{}{Med.}{}{}{}{Dor precordial com duração máxima de 15 minutos produzida por isquemia do miocárdio; angina do peito.}{an.gi.na}{0}
\verb{angiograma}{}{Med.}{}{}{s.m.}{Exame radiológico, com contraste, de um vaso sanguíneo.}{an.gi.o.gra.ma}{0}
\verb{angiologia}{}{Anat.}{}{}{s.f.}{Parte da anatomia que estuda o aparelho circulatório.}{an.gi.o.lo.gi.a}{0}
\verb{angiologia}{}{Med.}{}{}{}{Parte da medicina que trata das doenças vasculares.}{an.gi.o.lo.gi.a}{0}
\verb{angiologista}{}{Med.}{}{}{s.2g.}{Especialista no estudo e no tratamento das doenças vasculares.}{an.gi.o.lo.gis.ta}{0}
\verb{angioma}{}{Med.}{}{}{s.m.}{Tumor causado pelo aumento dos vasos sanguíneos ou linfáticos.}{an.gi.o.ma}{0}
\verb{angiopatia}{}{Med.}{}{}{s.f.}{Nome genérico dado às doenças vasculares; angiose.}{an.gi.o.pa.ti.a}{0}
\verb{angioplastia}{}{Med.}{}{}{s.f.}{Procedimento cirúrgico realizado com a introdução de um cateter em um vaso periférico para desobstruí"-lo.}{an.gi.o.plas.ti.a}{0}
\verb{angiose}{ó}{Med.}{}{}{s.f.}{Angiopatia.}{an.gi.o.se}{0}
\verb{angiosperma}{é}{Bot.}{}{}{s.f.}{Espécime das angiospermas, subdivisão do reino vegetal que reúne as plantas cujas sementes são protegidas pelo pericarpo (o fruto).}{an.gi.os.per.ma}{0}
\verb{anglicanismo}{}{Relig.}{}{}{s.m.}{Religião oficial da Inglaterra, ramo do protestantismo, que teve início no reinado de Henrique \textsc{viii}, quando este rompeu com o Papa.}{an.gli.ca.nis.mo}{0}
\verb{anglicano}{}{}{}{}{adj.}{Relativo ao anglicanismo.}{an.gli.ca.no}{0}
\verb{anglicano}{}{}{}{}{s.m.}{Indivíduo que professa o anglicanismo.}{an.gli.ca.no}{0}
\verb{anglicismo}{}{}{}{}{s.m.}{Palavra ou expressão proveniente da língua inglesa.}{an.gli.cis.mo}{0}
\verb{anglo}{}{}{}{}{adj.}{Relativo aos anglos, povo de origem germânica, que constituiu um dos elementos formadores da população inglesa.}{an.glo}{0}
\verb{anglo}{}{Por ext.}{}{}{}{Relativo à Inglaterra; inglês.}{an.glo}{0}
\verb{anglo}{}{}{}{}{s.m.}{Indivíduo dos anglos.}{an.glo}{0}
\verb{anglo"-americano}{}{}{anglo"-americanos}{anglo"-americana}{adj.}{Relativo ou pertencente simultaneamente à Grã"-Bretanha e aos \textsc{e.u.a}.}{an.glo"-a.me.ri.ca.no}{0}
\verb{anglo"-americano}{}{}{anglo"-americanos}{anglo"-americana}{}{Relativo à América inglesa, principalmente aos \textsc{e.u.a}.}{an.glo"-a.me.ri.ca.no}{0}
\verb{anglo"-americano}{}{}{anglo"-americanos}{anglo"-americana}{s.m.}{Indivíduo que tem ascendência inglesa e americana.}{an.glo"-a.me.ri.ca.no}{0}
\verb{anglo"-americano}{}{}{anglo"-americanos}{anglo"-americana}{}{Indivíduo natural ou habitante da América inglesa.}{an.glo"-a.me.ri.ca.no}{0}
\verb{anglofilia}{}{}{}{}{s.f.}{Amor, preferência às coisas inglesas.}{an.glo.fi.li.a}{0}
\verb{anglófilo}{}{}{}{}{adj.}{Que tem afinidade, amor aos ingleses e aos seus costumes. }{an.gló.fi.lo}{0}
\verb{anglófilo}{}{}{}{}{s.m.}{Indivíduo que tem amor à Inglaterra, ao seu povo e aos seus costumes. }{an.gló.fi.lo}{0}
\verb{anglofobia}{}{}{}{}{s.f.}{Aversão aos ingleses ou à Inglaterra.}{an.glo.fo.bi.a}{0}
\verb{anglófobo}{}{}{}{}{adj.}{Que tem anglofobia, aversão à Inglaterra.}{an.gló.fo.bo}{0}
\verb{anglófobo}{}{}{}{}{s.m.}{Indivíduo que tem ódio à Inglaterra.}{an.gló.fo.bo}{0}
\verb{anglo"-saxão}{cs}{}{anglo"-saxões ⟨cs⟩}{anglo"-saxã ⟨cs⟩}{s.m.}{Indivíduo de alguns dos povos germânicos que invadiram a Inglaterra entre os séculos \textsc{v} e \textsc{vi} e lá se fixaram.}{an.glo"-sa.xão}{0}
\verb{anglo"-saxão}{cs}{Por ext.}{anglo"-saxões ⟨cs⟩}{anglo"-saxã ⟨cs⟩}{}{Que tem origem inglesa; inglês.}{an.glo"-sa.xão}{0}
\verb{anglo"-saxão}{cs}{}{anglo"-saxões ⟨cs⟩}{anglo"-saxã ⟨cs⟩}{adj.}{Relativo a esses povos.}{an.glo"-sa.xão}{0}
\verb{angola}{ó}{}{}{}{s.f.}{Galinha"-d'angola.}{an.go.la}{0}
\verb{angola}{ó}{Desus.}{}{}{adj.2g. e s.2g.}{Angolano.}{an.go.la}{0}
\verb{angolano}{}{}{}{}{adj.}{Relativo à Angola (África).}{an.go.la.no}{0}
\verb{angolano}{}{}{}{}{s.m.}{Indivíduo natural ou habitante desse país.}{an.go.la.no}{0}
\verb{angolense}{}{}{}{}{adj.2g. e s.2g.}{Angolano.  }{an.go.len.se}{0}
\verb{angorá}{}{}{}{}{adj.2g.}{Relativo ou pertencente a Angorá.}{an.go.rá}{0}
\verb{angorá}{}{}{}{}{s.2g.}{Diz"-se de certas raças de gatos, coelhos e cabras, de pelos finos, compridos e sedosos, e dos animais dessa raça. }{an.go.rá}{0}
\verb{angra}{}{}{}{}{s.f.}{Pequena baía ou enseada que aparece onde há costas elevadas; litoral escarpado.}{an.gra}{0}
\verb{angu}{}{Cul.}{}{}{s.m.}{Massa consistente de farinha de milho, de mandioca ou de arroz.}{an.gu}{0}
\verb{angu}{}{Fig.}{}{}{}{Confusão, complicação, intriga.}{an.gu}{0}
\verb{angu"-de"-caroço}{ô}{Pop.}{angus"-de"-caroço ⟨ô⟩}{}{s.m.}{Confusão, intriga, complicação.}{an.gu"-de"-ca.ro.ço}{0}
\verb{anguiliforme}{ó}{}{}{}{adj.2g.}{Que tem forma de enguia.}{an.gui.li.for.me}{0}
\verb{anguiliforme}{ó}{Zool.}{}{}{}{Relativo aos anguiliformes, ordem de peixes à qual pertencem a enguia e a moreia.}{an.gui.li.for.me}{0}
\verb{anguiliforme}{ó}{Zool.}{}{}{s.m.}{Espécime dos anguiliformes.}{an.gui.li.for.me}{0}
\verb{angulação}{}{}{"-ões}{}{s.f.}{Formação, posição ou forma angular.}{an.gu.la.ção}{0}
\verb{angulado}{}{}{}{}{adj.}{Que é dotado de ângulos.}{an.gu.la.do}{0}
\verb{angular}{}{}{}{}{adj.2g.}{Relativo a ângulo.}{an.gu.lar}{0}
\verb{angular}{}{}{}{}{}{Que é dotado de ângulo.}{an.gu.lar}{0}
\verb{angular}{}{}{}{}{}{Que tem a forma de ângulo.}{an.gu.lar}{0}
\verb{ângulo}{}{}{}{}{s.m.}{Parte saliente ou reentrante; esquina, canto, aresta.}{ân.gu.lo}{0}
\verb{ângulo}{}{Geom.}{}{}{}{Medida do afastamento entre essas retas.}{ân.gu.lo}{0}
\verb{ângulo}{}{Fig.}{}{}{}{Ponto de vista; aspecto.}{ân.gu.lo}{0}
\verb{anguloso}{ô}{}{"-osos ⟨ó⟩}{"-osa ⟨ó⟩}{adj.}{Que é dotado de ângulos.}{an.gu.lo.so}{0}
\verb{anguloso}{ô}{}{"-osos ⟨ó⟩}{"-osa ⟨ó⟩}{}{Que tem arestas ou saliências pronunciadas.}{an.gu.lo.so}{0}
\verb{angústia}{}{}{}{}{s.f.}{Estado de ansiedade; agonia, aflição, sofrimento. }{an.gús.tia}{0}
\verb{angústia}{}{}{}{}{}{Redução de espaço e de tempo; estreiteza, limite, carência.}{an.gús.tia}{0}
\verb{angustiante}{}{}{}{}{adj.2g.}{Que causa angústia; angustioso.}{an.gus.ti.an.te}{0}
\verb{angustiar}{}{}{}{}{v.t.}{Causar angústia; afligir, atormentar.}{an.gus.ti.ar}{0}
\verb{angustiar}{}{}{}{}{}{Reduzir o espaço ou tempo; comprimir, apertar. }{an.gus.ti.ar}{0}
\verb{angustiar}{}{}{}{}{v.pron.}{Sentir angústia; agoniar"-se.}{an.gus.ti.ar}{\verboinum{1}}
\verb{angustioso}{ô}{}{"-osos ⟨ó⟩}{"-osa ⟨ó⟩}{adj.}{Angustiante.}{an.gus.ti.o.so}{0}
\verb{angusto}{}{Desus.}{}{}{adj.}{Sem espaço; estreito, apertado.}{an.gus.to}{0}
\verb{anguzada}{}{}{}{}{s.f.}{Mistura de coisas.}{an.gu.za.da}{0}
\verb{anguzada}{}{}{}{}{}{Reunião desordenada de pessoas.}{an.gu.za.da}{0}
\verb{anguzada}{}{Pop.}{}{}{}{Confusão, intriga, angu"-de"-caroço.}{an.gu.za.da}{0}
\verb{anhangá}{}{}{}{}{s.m.}{Alma em língua tupi.}{a.nhan.gá}{0}
\verb{anhanguera}{é}{Bras.}{}{}{s.m.}{Diabo velho; gênio manhoso e velhaco, entre os tupis.}{a.nhan.gue.ra}{0}
\verb{anhanguera}{é}{}{}{}{adj.2g.}{Que é destemido, corajoso.}{a.nhan.gue.ra}{0}
\verb{anhanguera}{é}{}{}{}{}{Ousado, valentão.}{a.nhan.gue.ra}{0}
\verb{anho}{}{}{}{}{s.m.}{Filhote de ovelha; cordeiro.}{a.nho}{0}
\verb{anhuma}{}{Zool.}{}{}{s.f.}{Ave de porte semelhante ao do peru, dos charcos e pântanos, com um espinho córneo na testa, dedos longos, plumagem alvinegra, partes inferiores brancas e pernas negras, bastante encontrada na Amazônia.}{a.nhu.ma}{0}
\verb{aniagem}{}{}{"-ens}{}{s.f.}{Tecido de juta, linho cru ou outra fibra vegetal, usado na confecção de sacos e fardos.}{a.ni.a.gem}{0}
\verb{anídrico}{}{Quím.}{}{}{adj.}{Que não contém água.}{a.ní.dri.co}{0}
\verb{anídrico}{}{}{}{}{}{Relativo a anidrido.}{a.ní.dri.co}{0}
\verb{anidrido}{}{Quím.}{}{}{s.m.}{Classe dos compostos químicos derivados de um ácido pela remoção de uma ou mais moléculas de água.}{a.ni.dri.do}{0}
\verb{anidro}{}{Quím.}{}{}{adj.}{Que não contém água, líquidos orgânicos etc.}{a.ni.dro}{0}
\verb{anil}{}{}{"-is}{}{s.m.}{Substância corante azul, encontrada nas folhas da anileira e em outras plantas.}{a.nil}{0}
\verb{anil}{}{}{"-is}{}{adj.2g.}{Diz"-se do que tem a cor azul.}{a.nil}{0}
\verb{anilado}{}{}{}{}{adj.}{Diz"-se do que é azulado.}{a.ni.la.do}{0}
\verb{anilar}{}{}{}{}{v.t.}{Tingir com anil.}{a.ni.lar}{0}
\verb{anilar}{}{}{}{}{}{Dar a cor azul.}{a.ni.lar}{\verboinum{1}}
\verb{anileira}{ê}{Bot.}{}{}{s.f.}{Designação comum a várias plantas da família das leguminosas, que fornecem o anil.}{a.ni.lei.ra}{0}
\verb{anilho}{}{}{}{}{s.m.}{Pequena argola de metal para enfiar cabo de ferramenta, ou para proteger furos de ilhós.}{a.ni.lho}{0}
\verb{anilina}{}{Quím.}{}{}{s.f.}{Amina derivada do benzeno, líquido incolor, de odor característico, empregada na fabricação de corantes, resina, perfumes etc.}{a.ni.li.na}{0}
\verb{animação}{}{}{"-ões}{}{s.f.}{Ato ou efeito de animar, de dar alma ou vida.}{a.ni.ma.ção}{0}
\verb{animação}{}{Fig.}{"-ões}{}{}{Sentimento de exaltação, alegria, entusiasmo.}{a.ni.ma.ção}{0}
\verb{animação}{}{Fig.}{"-ões}{}{}{Atividade intensa; movimentação, rebuliço.}{a.ni.ma.ção}{0}
\verb{animação}{}{Art.}{"-ões}{}{}{Arte cinematográfica que consiste na produção de imagens em movimento a partir de desenhos, bonecos ou objetos, por meio da filmagem quadro a quadro.}{a.ni.ma.ção}{0}
\verb{animado}{}{}{}{}{adj.}{Dotado de vida e de movimento.}{a.ni.ma.do}{0}
\verb{animado}{}{Fig.}{}{}{}{Cheio de entusiasmo; muito alegre; bem"-disposto.}{a.ni.ma.do}{0}
\verb{animador}{ô}{}{}{}{adj.}{Que anima, encoraja, estimula.}{a.ni.ma.dor}{0}
\verb{animador}{ô}{Art.}{}{}{s.m.}{Indivíduo que faz animação de imagens.}{a.ni.ma.dor}{0}
\verb{animador}{ô}{}{}{}{}{}{a.ni.ma.dor}{0}
\verb{animadversão}{}{}{"-ões}{}{s.f.}{Expressão de desaprovação; advertência.}{a.ni.mad.ver.são}{0}
\verb{animadversão}{}{}{"-ões}{}{}{Sentimento de ódio; aversão.}{a.ni.mad.ver.são}{0}
\verb{animal}{}{Zool.}{"-ais}{}{s.2g.}{Ser vivo multicelular, caracterizado pela organização complexa em sistemas digestivo, excretor, respiratório, circulatório e nervoso e pela capacidade de locomoção.}{a.ni.mal}{0}
\verb{animal}{}{Por ext.}{"-ais}{}{}{Qualquer animal irracional, por oposição ao ser humano.}{a.ni.mal}{0}
\verb{animal}{}{Fig.}{"-ais}{}{}{Indivíduo desumano, cruel, estúpido.}{a.ni.mal}{0}
\verb{animal}{}{}{"-ais}{}{adj.2g.}{Relativo a animal.}{a.ni.mal}{0}
\verb{animal}{}{Fig.}{"-ais}{}{}{Sensual, lascivo, voluptuoso.}{a.ni.mal}{0}
\verb{animalejo}{ê}{}{}{}{s.m.}{Diminutivo irregular de \textit{animal}; animal pequeno, bichinho.}{a.ni.ma.le.jo}{0}
\verb{animalesco}{ê}{}{}{}{adj.}{Relativo a animal.}{a.ni.ma.les.co}{0}
\verb{animalesco}{ê}{Fig.}{}{}{}{Que apresenta características semelhantes às dos animais irracionais; bruto, estúpido.}{a.ni.ma.les.co}{0}
\verb{animalidade}{}{}{}{}{s.f.}{Qualidade ou condição do que é animal.}{a.ni.ma.li.da.de}{0}
\verb{animalidade}{}{}{}{}{}{Conjunto de características próprias do animal; bestialidade.}{a.ni.ma.li.da.de}{0}
\verb{animalizar}{}{}{}{}{v.t.}{Tornar semelhante a animal; embrutecer, bestializar. }{a.ni.ma.li.zar}{0}
\verb{animalizar}{}{}{}{}{}{Transformar alimentos em substância própria para sustento de animais.}{a.ni.ma.li.zar}{\verboinum{1}}
\verb{animar}{}{}{}{}{v.t.}{Dar vida, ânimo; encorajar.}{a.ni.mar}{0}
\verb{animar}{}{}{}{}{}{Tornar mais vivo; intensificar, avivar.}{a.ni.mar}{0}
\verb{animar}{}{}{}{}{}{Promover o desenvolvimento; estimular, fomentar.}{a.ni.mar}{0}
\verb{animar}{}{}{}{}{v.pron.}{Atrever"-se, decidir"-se.}{a.ni.mar}{\verboinum{1}}
\verb{anímico}{}{}{}{}{adj.}{Relativo ou pertencente à alma.}{a.ní.mi.co}{0}
\verb{animismo}{}{Filos.}{}{}{s.m.}{Crença primitiva que atribui alma a seres e fenômenos da natureza.}{a.ni.mis.mo}{0}
\verb{animista}{}{}{}{}{adj.2g.}{Relativo a animismo.}{a.ni.mis.ta}{0}
\verb{animista}{}{}{}{}{s.2g.}{Adepto do animismo.}{a.ni.mis.ta}{0}
\verb{ânimo}{}{}{}{}{s.m.}{Alma, espírito, mente.}{â.ni.mo}{0}
\verb{ânimo}{}{}{}{}{}{Disposição de espírito, gênio, humor.}{â.ni.mo}{0}
\verb{ânimo}{}{}{}{}{}{Determinação, coragem, valor.}{â.ni.mo}{0}
\verb{ânimo}{}{}{}{}{interj.}{Termo que exprime encorajamento, força.}{â.ni.mo}{0}
\verb{animosidade}{}{}{}{}{s.f.}{Estado hostil, ressentimento, má vontade.}{a.ni.mo.si.da.de}{0}
\verb{animosidade}{}{}{}{}{}{Exaltação ou violência em uma discussão; polêmica, excitação.}{a.ni.mo.si.da.de}{0}
\verb{animoso}{ô}{}{"-osos ⟨ó⟩}{"-osa ⟨ó⟩}{adj.}{Que tem ânimo; corajoso, decidido.}{a.ni.mo.so}{0}
\verb{animoso}{ô}{}{"-osos ⟨ó⟩}{"-osa ⟨ó⟩}{}{Que apresenta animosidade; hostil, mal"-humorado.}{a.ni.mo.so}{0}
\verb{aninhar}{}{}{}{}{v.t.}{Colocar ou recolher no ninho.}{a.ni.nhar}{0}
\verb{aninhar}{}{Fig.}{}{}{}{Abrigar, aconchegar, agasalhar.}{a.ni.nhar}{0}
\verb{aninhar}{}{}{}{}{v.i.}{Fazer ninho; nidificar.}{a.ni.nhar}{\verboinum{1}}
\verb{ânion}{}{Fís. e Quím.}{}{}{s.m.}{Átomo com carga elétrica negativa.}{â.nion}{0}
\verb{aniquilação}{}{}{"-ões}{}{s.f.}{Ato ou efeito de aniquilar; destruição, extermínio, aniquilamento.}{a.ni.qui.la.ção}{0}
\verb{aniquilação}{}{Fís.}{"-ões}{}{}{Colisão de uma partícula com sua antipartícula, resultando em uma desmaterialização, e gerando energia irradiada.}{a.ni.qui.la.ção}{0}
\verb{aniquilado}{}{}{}{}{adj.}{Que se aniquilou; destruído, exterminado.}{a.ni.qui.la.do}{0}
\verb{aniquilado}{}{}{}{}{}{Abatido, humilhado.}{a.ni.qui.la.do}{0}
\verb{aniquilamento}{}{}{}{}{s.m.}{Aniquilação.}{a.ni.qui.la.men.to}{0}
\verb{aniquilar}{}{}{}{}{v.t.}{Destruir por completo; exterminar.}{a.ni.qui.lar}{0}
\verb{aniquilar}{}{}{}{}{}{Abater física e moralmente; esgotar, deprimir, humilhar.}{a.ni.qui.lar}{\verboinum{1}}
\verb{anis}{}{Bot.}{}{}{s.m.}{Erva aromática, de flores brancas, cultivada para fins medicinais, culinários e para produção de licores.}{a.nis}{0}
\verb{anisete}{é}{}{}{}{s.m.}{Licor feito de anis.}{a.ni.se.te}{0}
\verb{anistia}{}{}{}{}{s.f.}{Medida legislativa que extingue o caráter criminoso de um ato individual ou coletivo.}{a.nis.ti.a}{0}
\verb{anistia}{}{Por ext.}{}{}{}{Perdão em sentido amplo; esquecimento, absolvição.}{a.nis.ti.a}{0}
\verb{anistiar}{}{}{}{}{v.t.}{Dar anistia; inocentar.}{a.nis.ti.ar}{0}
\verb{anistiar}{}{Por ext.}{}{}{}{Perdoar, desculpar, esquecer.}{a.nis.ti.ar}{\verboinum{1}}
\verb{anistórico}{}{}{}{}{adj.}{Não histórico; alheio à história; aistórico.}{a.nis.tó.ri.co}{0}
\verb{anistórico}{}{}{}{}{}{Contrário à história; anti"-histórico.}{a.nis.tó.ri.co}{0}
\verb{aniversariante}{}{}{}{}{adj.2g.}{Diz"-se da pessoa que faz anos, que aniversaria.}{a.ni.ver.sa.ri.an.te}{0}
\verb{aniversariar}{}{Bras.}{}{}{v.i.}{Fazer anos; comemorar aniversário.}{a.ni.ver.sa.ri.ar}{\verboinum{1}}
\verb{aniversário}{}{}{}{}{adj.}{Diz"-se do dia em que se completa um ano ou mais de um fato ocorrido.}{a.ni.ver.sá.rio}{0}
\verb{aniversário}{}{}{}{}{}{Relativo a esse acontecimento.}{a.ni.ver.sá.rio}{0}
\verb{aniversário}{}{}{}{}{s.m.}{Comemoração desse evento.}{a.ni.ver.sá.rio}{0}
\verb{anjinho}{}{}{}{}{s.m.}{Pequeno anjo.}{an.ji.nho}{0}
\verb{anjinho}{}{Fig.}{}{}{}{Criança morta no começo da infância.}{an.ji.nho}{0}
\verb{anjinho}{}{Fig.}{}{}{}{Pessoa que finge inocência ou ingenuidade.}{an.ji.nho}{0}
\verb{anjo}{}{}{}{}{s.m.}{Na tradição judaico"-cristã e islâmica, ser celeste intermediário entre Deus e os homens.}{an.jo}{0}
\verb{anjo}{}{}{}{}{}{Criança que se veste como anjo nas procissões e festas católicas.}{an.jo}{0}
\verb{anjo}{}{Fig.}{}{}{}{Pessoa que apresenta características atribuídas aos anjos, como bondade, pureza, caridade, proteção.}{an.jo}{0}
\verb{ano}{}{}{}{}{s.m.}{Espaço de tempo de 12 meses, contados a partir de 1º de janeiro até 31 de dezembro.}{a.no}{0}
\verb{ano}{}{}{}{}{}{Período de 365 dias contados a partir de qualquer dia do ano.}{a.no}{0}
\verb{ano}{}{}{}{}{}{Medida de idade ou existência.}{a.no}{0}
\verb{ano}{}{Astron.}{}{}{}{Intervalo de tempo que um planeta do sistema solar leva para fazer uma volta completa ao redor do Sol. }{a.no}{0}
\verb{ano"-bom}{}{}{anos"-bons}{}{s.m.}{Ano"-novo.}{a.no"-bom}{0}
\verb{anódino}{}{}{}{}{adj.}{Diz"-se do medicamento que abranda ou faz cessar a dor; paliativo.}{a.nó.di.no}{0}
\verb{anódino}{}{Fig.}{}{}{}{Que tem pouca importância; medíocre, insignificante.}{a.nó.di.no}{0}
\verb{ânodo}{}{Quím.}{}{}{s.m.}{Eletrodo de carga positiva para onde migram os íons negativos em uma eletrólise.}{â.no.do}{0}
\verb{anófele}{}{Zool.}{}{}{s.m.}{Gênero de mosquitos ao qual pertencem os responsáveis pela transmissão da malária.}{a.nó.fe.le}{0}
\verb{anoftalmia}{}{Med.}{}{}{s.f.}{Doença congênita caracterizada pela ausência de um ou de ambos os olhos.}{a.nof.tal.mi.a}{0}
\verb{anoitecer}{ê}{}{}{}{v.i.}{Fazer"-se noite; começar a noite.}{a.noi.te.cer}{0}
\verb{anoitecer}{ê}{}{}{}{}{Estar ou achar"-se em determinado local no começo da noite.}{a.noi.te.cer}{0}
\verb{anoitecer}{ê}{Fig.}{}{}{v.t.}{Tornar escuro; cobrir de trevas; escurecer.}{a.noi.te.cer}{\verboinum{15}}
\verb{anoitecer}{ê}{}{}{}{s.m.}{O início ou o cair da noite.}{a.noi.te.cer}{0}
\verb{anojado}{}{}{}{}{adj.}{Que está com nojo, náusea; enjoado.}{a.no.ja.do}{0}
\verb{anojado}{}{}{}{}{}{Que está de luto; enlutado, entristecido.}{a.no.ja.do}{0}
\verb{anojado}{}{}{}{}{}{Desgostoso, entediado, enfadado.}{a.no.ja.do}{0}
\verb{anojar}{}{}{}{}{v.t.}{Causar nojo, aversão; enojar.}{a.no.jar}{0}
\verb{anojar}{}{}{}{}{}{Causar desgosto; aborrecer, molestar.}{a.no.jar}{0}
\verb{anojar}{}{}{}{}{v.pron.}{Pôr"-se de luto; enlutar"-se.}{a.no.jar}{\verboinum{1}}
\verb{anojo}{ô}{}{}{}{s.m.}{Ato ou efeito de anojar; nojo, repulsa, aversão.}{a.no.jo}{0}
\verb{anojo}{ô}{}{}{}{}{Pesar, luto, abatimento.}{a.no.jo}{0}
\verb{anojo}{ô}{}{}{}{}{Desgosto, aborrecimento, enfado.}{a.no.jo}{0}
\verb{ano"-luz}{}{Astron.}{anos"-luzes}{}{s.m.}{Unidade equivalente à distância que a luz no vácuo percorre em um ano à velocidade de 300 mil km/s.}{a.no"-luz}{0}
\verb{anomalia}{}{}{}{}{s.f.}{Aquilo que se desvia da norma, do padrão; irregularidade, anormalidade.}{a.no.ma.li.a}{0}
\verb{anomalia}{}{}{}{}{}{Deformidade, monstruosidade.}{a.no.ma.li.a}{0}
\verb{anômalo}{}{}{}{}{adj.}{Que não segue a norma, o padrão; irregular, anormal, estranho.}{a.nô.ma.lo}{0}
\verb{anonácea}{}{Bot.}{}{}{adj.}{Planta da família das anonáceas, cujos frutos apresentam"-se em bagas comestíveis como a fruta"-do"-conde, a graviola etc.}{a.no.ná.cea}{0}
\verb{anonimato}{}{}{}{}{s.m.}{Estado ou condição do que não se conhece o nome.}{a.no.ni.ma.to}{0}
\verb{anonimato}{}{}{}{}{}{Costume ou sistema de escrever sem a identificação do autor.}{a.no.ni.ma.to}{0}
\verb{anônimo}{}{}{}{}{adj.}{Que não apresenta o nome ou a assinatura do autor.}{a.nô.ni.mo}{0}
\verb{anônimo}{}{}{}{}{}{Que não revela seu nome; desconhecido, obscuro.}{a.nô.ni.mo}{0}
\verb{ano"-novo}{ô}{}{anos"-novos ⟨ó⟩}{}{s.m.}{Meia"-noite do dia 31 de dezembro ou o dia 1º de janeiro; ano"-bom.}{a.no"-no.vo}{0}
\verb{anopluro}{}{Zool.}{}{}{s.m.}{Inseto pertencente à ordem dos anopluros, parasitas que se alimentam do sangue de aves e mamíferos, como o piolho e o chato.}{a.no.plu.ro}{0}
\verb{anorexia}{cs}{Med.}{}{}{s.f.}{Perda ou redução do apetite, de origem psíquica ou orgânica; inapetência.}{a.no.re.xi.a}{0}
\verb{anoréxico}{cs}{}{}{}{adj.}{Relativo à anorexia.}{a.no.ré.xi.co}{0}
\verb{anoréxico}{cs}{}{}{}{s.m.}{Pessoa que sofre de perda ou redução de apetite.}{a.no.ré.xi.co}{0}
\verb{anormal}{}{}{"-ais}{}{adj.2g.}{Que não segue a norma ou a regra; anômalo.}{a.nor.mal}{0}
\verb{anormal}{}{}{"-ais}{}{s.m.}{Indivíduo que apresenta comportamento fora do padrão; desequilibrado.}{a.nor.mal}{0}
\verb{anormalidade}{}{}{}{}{s.f.}{Estado ou característica do que é anormal; anomalia, irregularidade.}{a.nor.ma.li.da.de}{0}
\verb{anormalidade}{}{}{}{}{}{Situação que foge à regra; exceção.}{a.nor.ma.li.da.de}{0}
\verb{anoso}{ô}{}{"-osos ⟨ó⟩}{"-osa ⟨ó⟩}{adj.}{Que existe há muitos anos; velho, antigo.}{a.no.so}{0}
\verb{anotação}{}{}{"-ões}{}{s.f.}{Ato ou efeito de anotar; registro.}{a.no.ta.ção}{0}
\verb{anotação}{}{}{"-ões}{}{}{Apontamento, comentário, nota.}{a.no.ta.ção}{0}
\verb{anotador}{ô}{}{}{}{adj.}{Que anota, comenta; comentarista.}{a.no.ta.dor}{0}
\verb{anotar}{}{}{}{}{v.t.}{Tomar notas; fazer anotações; registrar.}{a.no.tar}{0}
\verb{anotar}{}{}{}{}{}{Colocar notas em textos; explicar, esclarecer.}{a.no.tar}{\verboinum{1}}
\verb{anquinhas}{}{}{}{}{s.f.pl.}{Armação de arame com que se elevavam os quadris, estufando as saias das mulheres para trás, usada até meados do século \textsc{xix}.}{an.qui.nhas}{0}
\verb{anseio}{ê}{}{}{}{s.m.}{Estado de angústia, aflição, sofrimento.}{an.sei.o}{0}
\verb{anseio}{ê}{}{}{}{}{Desejo intenso, ambição, aspiração.}{an.sei.o}{0}
\verb{anseriforme}{ó}{Zool.}{}{}{s.m.}{Ave aquática pertencente à ordem dos anseriformes, que apresenta, como características, bico largo e pernas curtas, como o pato, o marreco, o ganso etc.}{an.se.ri.fo.rme}{0}
\verb{anserino}{}{}{}{}{adj.}{Relativo ou semelhante a pato ou ganso.}{an.se.ri.no}{0}
\verb{ânsia}{}{}{}{}{s.f.}{Sensação de desconforto e mal"-estar provocada pela contração do abdômen.}{ân.sia}{0}
\verb{ânsia}{}{}{}{}{}{Sentimento de aflição, angústia, agonia.}{ân.sia}{0}
\verb{ânsia}{}{}{}{}{}{Desejo ardente, ambição, anseio.}{ân.sia}{0}
\verb{ansiado}{}{}{}{}{adj.}{Que está com ânsia, náusea; enjoado.}{an.si.a.do}{0}
\verb{ansiado}{}{}{}{}{}{Inquieto, aflito, angustiado.}{an.si.a.do}{0}
\verb{ansiado}{}{}{}{}{}{Desejado intensamente, almejado.}{an.si.a.do}{0}
\verb{ansiar}{}{}{}{}{v.t.}{Causar ou sentir ânsia, agonia; angustiar, afligir.}{an.si.ar}{0}
\verb{ansiar}{}{}{}{}{}{Sentir náusea; enjoar.}{an.si.ar}{0}
\verb{ansiar}{}{}{}{}{}{Desejar ardentemente; ambicionar, aspirar.}{an.si.ar}{\verboinum{1}}
\verb{ansiedade}{}{}{}{}{s.f.}{Mal"-estar intenso; inquietação, agonia, angústia.}{an.si.e.da.de}{0}
\verb{ansiolítico}{}{Med.}{}{}{adj.}{Diz"-se de droga que alivia a ansiedade; tranquilizante.}{an.si.o.lí.ti.co}{0}
\verb{ansioso}{ô}{}{"-osos ⟨ó⟩}{"-osa ⟨ó⟩}{adj.}{Que sente aflição; inquieto, angustiado.}{an.si.o.so}{0}
\verb{ansioso}{ô}{}{"-osos ⟨ó⟩}{"-osa ⟨ó⟩}{}{Desejoso, ávido.}{an.si.o.so}{0}
\verb{anspeçada}{}{}{}{}{s.m.}{Na Marinha, graduação de praça entre soldado e cabo.}{ans.pe.ça.da}{0}
\verb{anta}{}{Zool.}{}{}{s.f.}{Mamífero selvagem de pelos lisos e curtos, cor marrom"-escura e pequena tromba.}{an.ta}{0}
\verb{anta}{}{Pop.}{}{}{s.2g.}{Indivíduo pouco inteligente; tolo, burro.}{an.ta}{0}
\verb{anta}{}{Arquit.}{}{}{s.f.}{Pilastra angular de um edifício.}{an.ta}{0}
\verb{anta}{}{Desus.}{}{}{}{Tipo de monumento ou de marco de terra.}{an.ta}{0}
\verb{antagônico}{}{}{}{}{adj.}{Contrário, oposto, incompatível.}{an.ta.gô.ni.co}{0}
\verb{antagonismo}{}{}{}{}{s.m.}{Oposição de ideias; rivalidade, incompatibilidade.}{an.ta.go.nis.mo}{0}
\verb{antagonista}{}{}{}{}{adj.2g.}{Que age em sentido oposto; adversário.}{an.ta.go.nis.ta}{0}
\verb{antálgico}{}{}{}{}{adj.}{Diz"-se de medicamento que combate a dor.}{an.tál.gi.co}{0}
\verb{antanho}{}{}{}{}{adv.}{No ano passado.}{an.ta.nho}{0}
\verb{antanho}{}{}{}{}{}{Antigamente.}{an.ta.nho}{0}
\verb{antanho}{}{}{}{}{s.m.}{Tempo passado.}{an.ta.nho}{0}
\verb{antártico}{}{}{}{}{adj.}{Do polo sul.}{an.tár.ti.co}{0}
\verb{ante}{}{}{}{}{prep.}{Diante de; em frente a.}{an.te}{0}
\verb{anteato}{}{}{}{}{s.m.}{Curta representação teatral, que antecede a peça principal.}{an.te.a.to}{0}
\verb{antebraço}{}{}{}{}{s.m.}{Parte do membro superior do homem entre o cotovelo e o punho.}{an.te.bra.ço}{0}
\verb{antecâmara}{}{}{}{}{s.f.}{Aposento que antecede a sala principal; sala de espera, antessala.}{an.te.câ.ma.ra}{0}
\verb{antecedência}{}{}{}{}{s.f.}{Ato ou efeito de anteceder.}{an.te.ce.dên.cia}{0}
\verb{antecedência}{}{}{}{}{}{Precedência, anterioridade.}{an.te.ce.dên.cia}{0}
\verb{antecedente}{}{}{}{}{adj.2g.}{Que antecede; precedente.}{an.te.ce.den.te}{0}
\verb{antecedentes}{}{}{}{}{s.m.pl.}{Fatos do passado de um indivíduo que são relevantes em determinado contexto.}{an.te.ce.den.tes}{0}
\verb{anteceder}{ê}{}{}{}{v.t.}{Estar ou vir antes.}{an.te.ce.der}{0}
\verb{anteceder}{ê}{}{}{}{v.pron.}{Agir com antecipação; antecipar"-se.}{an.te.ce.der}{\verboinum{12}}
\verb{antecessor}{ô}{}{}{}{s.m.}{Que antecede.}{an.te.ces.sor}{0}
\verb{antecipação}{}{}{"-ões}{}{s.f.}{Ato ou efeito de antecipar.}{an.te.ci.pa.ção}{0}
\verb{antecipação}{}{}{"-ões}{}{}{Pagamento parcial ou total feito antes do vencimento.}{an.te.ci.pa.ção}{0}
\verb{antecipar}{}{}{}{}{v.t.}{Fazer ocorrer antes do tempo previsto.}{an.te.ci.par}{0}
\verb{antecipar}{}{}{}{}{}{Prever, prenunciar.}{an.te.ci.par}{0}
\verb{antecipar}{}{}{}{}{v.pron.}{Agir com antecipação.}{an.te.ci.par}{\verboinum{1}}
\verb{antecos}{é}{}{}{}{s.m.pl.}{Diz"-se de habitantes de lugares que têm mesma longitude e latitudes simétricas (de mesmo valor em hemisférios diferentes).}{an.te.cos}{0}
\verb{antedata}{}{}{}{}{s.f.}{Data falsa anterior àquela em que o documento foi elaborado.}{an.te.da.ta}{0}
\verb{antedatar}{}{}{}{}{v.t.}{Pôr em um documento uma data falsa anterior à de sua elaboração.}{an.te.da.tar}{\verboinum{1}}
\verb{antediluviano}{}{}{}{}{adj.}{Anterior ao dilúvio.}{an.te.di.lu.vi.a.no}{0}
\verb{antediluviano}{}{Fig.}{}{}{}{Muito antigo ou velho; antiquado.}{an.te.di.lu.vi.a.no}{0}
\verb{antedizer}{ê}{}{}{}{v.t.}{Profetizar, prognosticar, predizer.}{an.te.di.zer}{\verboinum{41}}
\verb{antegostar}{}{}{}{}{v.t.}{Gostar antecipadamente; antegozar.}{an.te.gos.tar}{\verboinum{1}}
\verb{antegosto}{ô}{}{}{}{s.m.}{Gosto imaginado, anterior à real experiência.}{an.te.gos.to}{0}
\verb{antegozar}{}{}{}{}{v.t.}{Gozar algo antes de acontecer.}{an.te.go.zar}{\verboinum{1}}
\verb{antegozo}{ô}{}{}{}{s.m.}{Deleite antecipado.}{an.te.go.zo}{0}
\verb{ante"-histórico}{}{}{ante"-históricos}{}{adj.}{Relativo à pré"-história; pré"-histórico.}{an.te"-his.tó.ri.co}{0}
\verb{antelóquio}{}{}{}{}{s.m.}{Texto que antecede uma obra, contendo explicações e comentários sobre ela; prefácio.}{an.te.ló.quio}{0}
\verb{antemanhã}{}{}{}{}{s.f.}{Momento anterior ao nascer do sol.}{an.te.ma.nhã}{0}
\verb{antemão}{}{}{}{}{adv.}{Usado na locução \textit{de antemão}: previamente, antecipadamente.}{an.te.mão}{0}
\verb{antemeridiano}{}{}{}{}{adj.}{Anterior ao meio"-dia.}{an.te.me.ri.di.a.no}{0}
\verb{antena}{}{}{}{}{s.f.}{Dispositivo que capta ou transmite ondas eletromagnéticas.}{an.te.na}{0}
\verb{antena}{}{}{}{}{}{Apêndice sensorial presente, em pares, na cabeça dos artrópodes.}{an.te.na}{0}
\verb{antenado}{}{}{}{}{adj.}{Que tem antena(s).}{an.te.na.do}{0}
\verb{antenado}{}{Fig.}{}{}{}{Bem informado.}{an.te.na.do}{0}
\verb{antenupcial}{}{}{"-ais}{}{adj.2g.}{Que ocorre antes do casamento; pré"-nupcial.}{an.te.nup.ci.al}{0}
\verb{anteontem}{}{}{}{}{adv.}{No dia anterior ao de ontem.}{an.te.on.tem}{0}
\verb{anteparar}{}{}{}{}{v.t.}{Pôr anteparo.}{an.te.pa.rar}{0}
\verb{anteparar}{}{}{}{}{}{Proteger, resguardar.}{an.te.pa.rar}{\verboinum{1}}
\verb{anteparo}{}{}{}{}{s.m.}{Objeto que serve de proteção ou resguardo a algo ou alguém.}{an.te.pa.ro}{0}
\verb{antepassado}{}{}{}{}{adj.}{Que aconteceu antes.}{an.te.pas.sa.do}{0}
\verb{antepassado}{}{}{}{}{s.m.}{Ascendente, ancestral.}{an.te.pas.sa.do}{0}
\verb{antepassados}{}{}{}{}{s.m.}{Ascendentes, ancestrais.}{an.te.pas.sa.dos}{0}
\verb{antepasto}{}{}{}{}{s.m.}{Alimento servido antes do primeiro prato em uma refeição.}{an.te.pas.to}{0}
\verb{antepenúltimo}{}{}{}{}{adj.}{Imediatamente anterior ao penúltimo.}{an.te.pe.núl.ti.mo}{0}
\verb{antepor}{}{}{}{}{v.t.}{Pôr antes.}{an.te.por}{0}
\verb{antepor}{}{}{}{}{}{Priorizar, preferir.}{an.te.por}{0}
\verb{antepor}{}{}{}{}{}{Pôr contra, contrapor.}{an.te.por}{0}
\verb{antepor}{}{Ant.}{}{}{}{pospor}{an.te.por}{\verboinum{60}}
\verb{anteposição}{}{}{"-ões}{}{s.f.}{Ato ou efeito de antepor.}{an.te.po.si.ção}{0}
\verb{anteposição}{}{Fig.}{"-ões}{}{}{Prioridade, preferência.}{an.te.po.si.ção}{0}
\verb{antepositivo}{}{}{}{}{adj.}{Relativo a anteposição.}{an.te.po.si.ti.vo}{0}
\verb{anteposto}{ô}{}{"-s ⟨ó⟩}{"-a ⟨ó⟩}{adj.}{Que se antepôs.}{an.te.pos.to}{0}
\verb{anteprojeto}{é}{}{}{}{s.m.}{Esboço ou estágio preliminar de um projeto.}{an.te.pro.je.to}{0}
\verb{antera}{é}{Bot.}{}{}{s.f.}{Estrutura localizada na extremidade do estame, na qual fica o pólen.}{an.te.ra}{0}
\verb{anterior}{ô}{}{}{}{adj.}{Que está adiante; que existiu, sucedeu ou se fez antes.}{an.te.ri.or}{0}
\verb{anterior}{ô}{Ant.}{}{}{}{posterior}{an.te.ri.or}{0}
\verb{anterioridade}{}{}{}{}{s.f.}{Qualidade ou estado do que é anterior.}{an.te.ri.o.ri.da.de}{0}
\verb{antes}{}{}{}{}{adv.}{Em tempo ou lugar anterior.}{an.tes}{0}
\verb{antes}{}{}{}{}{}{De preferência, melhor.}{an.tes}{0}
\verb{antes}{}{}{}{}{}{Pelo contrário, ao contrário.}{an.tes}{0}
\verb{antessala}{}{}{antessalas}{}{s.f.}{Aposento que antecede a sala principal; sala de espera.}{an.tes.sa.la}{0}
\verb{antever}{ê}{}{}{}{v.t.}{Observar com antecedência; ver antes.}{an.te.ver}{0}
\verb{antever}{ê}{Fig.}{}{}{}{Ver antecipadamente; prognosticar, adivinhar.}{an.te.ver}{\verboinum{46}}
\verb{antevéspera}{}{}{}{}{s.f.}{Diz"-se do dia anterior à véspera.}{an.te.vés.pe.ra}{0}
\verb{antevisão}{}{}{"-ões}{}{s.f.}{Ato ou efeito de antever; visão antecipada; previsão.}{an.te.vi.são}{0}
\verb{antiabortivo}{}{}{}{}{adj.}{Diz"-se da substância que evita o aborto.}{an.ti.a.bor.ti.vo}{0}
\verb{antiácido}{}{}{}{}{adj.}{Que atua contra os ácidos, neutralizando"-lhes a ação.}{an.ti.á.ci.do}{0}
\verb{antiácido}{}{}{}{}{s.m.}{Substância que combate a acidez.}{an.ti.á.ci.do}{0}
\verb{antiaderente}{}{}{}{}{adj.2g.}{Diz"-se de substância que impede a adesão de um objeto a outro.}{an.ti.a.de.ren.te}{0}
\verb{antiaderente}{}{}{}{}{s.m.}{Essa substância.}{an.ti.a.de.ren.te}{0}
\verb{antiaéreo}{}{}{}{}{adj.}{Que defende ou protege dos ataques aéreos.}{an.ti.a.é.re.o}{0}
\verb{antialcoólico}{}{}{}{}{adj.}{Diz"-se de substância que combate os efeitos produzidos pelo álcool.}{an.ti.al.co.ó.li.co}{0}
\verb{antialcoólico}{}{}{}{}{s.m.}{Essa substância.}{an.ti.al.co.ó.li.co}{0}
\verb{antialérgico}{}{}{}{}{adj.}{Diz"-se de medicamento que combate a alergia.}{an.ti.a.lér.gi.co}{0}
\verb{antialérgico}{}{}{}{}{s.m.}{Esse medicamento.}{an.ti.a.lér.gi.co}{0}
\verb{antibiótico}{}{Med.}{}{}{s.m.}{Substância produzida por seres vivos, ou sinteticamente, capaz de impedir o crescimento de micro"-organismos ou de destruí"-los, usado contra moléstias infecciosas.}{an.ti.bi.ó.ti.co}{0}
\verb{antibiótico}{}{}{}{}{adj.}{Relativo aos antibióticos.}{an.ti.bi.ó.ti.co}{0}
\verb{anticiclone}{}{}{}{}{s.m.}{Zona de alta pressão atmosférica em relação à das regiões circunvizinhas, num mesmo nível, e de onde os ventos sopram em forma de espiral.}{an.ti.ci.clo.ne}{0}
\verb{anticlerical}{}{}{"-ais}{}{adj.2g.}{Que é contrário ao clero ou à sua influência e participação política, social ou moral.}{an.ti.cle.ri.cal}{0}
\verb{anticoagulante}{}{}{}{}{adj.2g.}{Diz"-se da substância que diminui a capacidade de coagulação do sangue.}{an.ti.co.a.gu.lan.te}{0}
\verb{anticoagulante}{}{}{}{}{s.m.}{Essa substância.}{an.ti.co.a.gu.lan.te}{0}
\verb{anticoncepcional}{}{}{"-ais}{}{adj.2g.}{Contrário à concepção; que evita a fecundação; contraceptivo. }{an.ti.con.cep.ci.o.nal}{0}
\verb{anticoncepcional}{}{}{"-ais}{}{s.m.}{Meio, substância, prática ou ação que procura impedir a concepção de filhos. }{an.ti.con.cep.ci.o.nal}{0}
\verb{anticonstitucional}{}{}{"-ais}{}{adj.2g.}{Que é contrário à constituição política de um país.}{an.ti.cons.ti.tu.ci.o.nal}{0}
\verb{anticonvulsivo}{}{Med.}{}{}{adj.}{Diz"-se de toda substância capaz de deter ou evitar as convulsões.}{an.ti.con.vul.si.vo}{0}
\verb{anticonvulsivo}{}{}{}{}{s.m.}{Essa substância.}{an.ti.con.vul.si.vo}{0}
\verb{anticorpo}{ô}{}{"-s ⟨ó⟩}{}{s.m.}{Substância produzida pelo organismo como reação a substâncias nele introduzidas.}{an.ti.cor.po}{0}
\verb{anticorrosivo}{}{}{}{}{adj.}{Referente às técnicas e processos para evitar a corrosão de materiais metálicos e protegê"-los dela.}{an.ti.cor.ro.si.vo}{0}
\verb{anticorrosivo}{}{}{}{}{s.m.}{Produto que permite essa proteção.}{an.ti.cor.ro.si.vo}{0}
%\verb{anticristão}{}{}{}{}{}{0}{an.ti.cris.tão}{0}
\verb{anticristo}{}{}{}{}{s.m.}{Inimigo de Cristo.}{an.ti.cris.to}{0}
\verb{anticristo}{}{Por ext.}{}{}{}{Qualquer perseguidor feroz dos cristãos.}{an.ti.cris.to}{0}
\verb{antidemocrático}{}{}{}{}{adj.}{Contrário, hostil à democracia.}{an.ti.de.mo.crá.ti.co}{0}
\verb{antidepressivo}{}{}{}{}{adj.}{Diz"-se de substância que atenua ou evita a depressão, ou estimula o ânimo do paciente em depressão.}{an.ti.de.pres.si.vo}{0}
\verb{antidepressivo}{}{}{}{}{s.m.}{Essa substância.}{an.ti.de.pres.si.vo}{0}
\verb{antiderrapante}{}{}{}{}{adj.2g.}{Diz"-se de revestimento de estrada que impede a derrapagem de veículos.}{an.ti.der.ra.pan.te}{0}
\verb{antiderrapante}{}{}{}{}{}{Diz"-se de dispositivo que aumenta a aderência do pneu ao solo.}{an.ti.der.ra.pan.te}{0}
\verb{antidetonante}{}{}{}{}{adj.2g.}{Diz"-se de dispositivo que impede a combustão prematura de um combustível durante a compressão em um motor de combustão interna.}{an.ti.de.to.nan.te}{0}
\verb{antidetonante}{}{}{}{}{s.m.}{Esse dispositivo.}{an.ti.de.to.nan.te}{0}
\verb{antidivorcista}{}{}{}{}{adj.2g.}{Que é contrário ao divórcio.}{an.ti.di.vor.cis.ta}{0}
\verb{antidivorcista}{}{}{}{}{s.2g.}{Pessoa que se opõe ao divórcio.}{an.ti.di.vor.cis.ta}{0}
\verb{antidoping}{ó}{}{}{}{adj.}{Que se opõe à prática do uso de estimulantes nos esportes.  }{an.ti.do.ping}{0}
\verb{antidoping}{ó}{}{}{}{s.m.}{Exame  que visa a detectar a presença de substância estimulante no organismo de atleta ou animal em competições desportivas.}{an.ti.do.ping}{0}
\verb{antídoto}{}{}{}{}{s.m.}{Substância, medicamento ou soro que combate ou neutraliza o efeito dos venenos.}{an.tí.do.to}{0}
\verb{antiespasmódico}{}{}{}{}{adj.}{Que evita ou alivia os espasmos.}{an.ti.es.pas.mó.di.co}{0}
\verb{antiespasmódico}{}{}{}{}{s.m.}{Substância ou medicamento que combate os espasmos.}{an.ti.es.pas.mó.di.co}{0}
\verb{antiestético}{}{}{}{}{adj.}{Contrário à estética, destituído de beleza.}{an.ti.es.té.ti.co}{0}
\verb{antiético}{}{}{}{}{adj.}{Contrário à ética, oposto à moral.}{an.ti.é.ti.co}{0}
\verb{antifebril}{}{}{"-is}{}{adj.2g.}{Antipirético.}{an.ti.fe.bril}{0}
\verb{antífona}{}{}{}{}{s.f.}{Versículo cantado antes de um salmo ou cântico religioso.}{an.tí.fo.na}{0}
\verb{antífrase}{}{}{}{}{s.f.}{Emprego de uma palavra ou expressão com sentido oposto ao verdadeiro, como recurso estilístico ou de ironia.}{an.tí.fra.se}{0}
\verb{antígeno}{}{Med.}{}{}{s.m.}{Substância que provoca a formação de anticorpos.}{an.tí.ge.no}{0}
\verb{antigo}{}{}{}{}{adj.}{Que existe há muito tempo.}{an.ti.go}{0}
\verb{antigo}{}{}{}{}{}{Que precedeu o atual.}{an.ti.go}{0}
\verb{antigo}{}{}{}{}{}{Que ocupa um cargo, posto ou posição há mais tempo.}{an.ti.go}{0}
\verb{antiguano}{}{}{}{}{adj.}{Relativo a Antígua e Barbuda, ilhas do mar das Antilhas.}{an.ti.gua.no}{0}
\verb{antiguano}{}{}{}{}{s.m.}{Indivíduo natural ou habitante dessas ilhas.}{an.ti.gua.no}{0}
\verb{antiguidade}{}{}{}{}{s.f.}{Qualidade do que é antigo.}{an.ti.gui.da.de}{0}
\verb{antiguidade}{}{}{}{}{}{Objeto antigo.}{an.ti.gui.da.de}{0}
\verb{antiguidade}{}{}{}{}{}{Precedência cronológica em um cargo, posto ou posição.}{an.ti.gui.da.de}{0}
\verb{antiguidade}{}{Hist.}{}{}{}{Período da História anterior à queda do Império Romano do Ocidente.}{an.ti.gui.da.de}{0}
\verb{anti"-helmíntico}{}{}{anti"-helmínticos}{}{adj.}{Que afugenta ou destrói os vermes; vermífugo.}{an.ti"-hel.mín.ti.co}{0}
\verb{anti"-helmíntico}{}{}{anti"-helmínticos}{}{s.m.}{Aquilo que afasta ou mata os vermes.}{an.ti"-hel.mín.ti.co}{0}
\verb{anti"-herói}{}{}{anti"-heróis}{}{s.m.}{Personagem de ficção que não apresenta as qualidades físicas ou as virtudes características do herói clássico. }{an.ti"-he.rói}{0}
\verb{anti"-higiênico}{}{}{anti"-higiênicos}{}{adj.}{Contrário à higiene; que não impede ou facilita a proliferação de bactérias. }{an.ti"-hi.gi.ê.ni.co}{0}
\verb{anti"-histórico}{}{}{anti"-históricos}{}{adj.}{Que contraria a história, seus fatos e princípios.}{an.ti"-his.tó.ri.co}{0}
\verb{anti"-horário}{}{}{anti"-horários}{}{adj.}{Em sentido contrário ao dos ponteiros de um relógio.}{an.ti"-ho.rá.rio}{0}
\verb{anti"-humano}{}{}{anti"-humanos}{}{adj.}{Que é desprovido de humanidade; cruel, desumano.}{an.ti"-hu.ma.no}{0}
\verb{anti"-imperialismo}{}{}{}{}{s.m.}{Teoria, posição ou movimento contrário ao imperialismo.}{an.ti"-im.pe.ri.a.lis.mo}{0}
\verb{anti"-imperialista}{}{}{}{}{adj.2g.}{Relativo ao anti"-imperialismo.}{an.ti"-im.pe.ri.a.lis.ta}{0}
\verb{anti"-imperialista}{}{}{}{}{}{Que é partidário do anti"-imperialismo.}{an.ti"-im.pe.ri.a.lis.ta}{0}
\verb{anti"-imperialista}{}{}{}{}{s.2g.}{Indivíduo adepto do anti"-imperialismo.}{an.ti"-im.pe.ri.a.lis.ta}{0}
\verb{anti"-infeccioso}{ô}{}{"-osos ⟨ó⟩}{"-osa ⟨ó⟩}{adj.}{Que é contrário às infecções.}{an.ti"-in.fec.ci.o.so}{0}
\verb{anti"-infeccioso}{ô}{}{"-osos ⟨ó⟩}{"-osa ⟨ó⟩}{s.m.}{Aquilo que combate as infecções.}{an.ti"-in.fec.ci.o.so}{0}
\verb{antiinflacionário}{}{}{}{}{adj.}{Que previne ou combate a inflação, o aumento excessivo e injustificado de algo.}{an.ti.in.fla.ci.o.ná.rio}{0}
\verb{anti"-inflamatório}{}{}{}{}{adj.}{Diz"-se da substância ou medicamento que combate as inflamações, reações de proteção do corpo contra agentes estranhos a ele.}{an.ti"-in.fla.ma.tó.rio}{0}
\verb{antilhano}{}{}{}{}{adj.}{Relativo às Antilhas.}{an.ti.lha.no}{0}
\verb{antilhano}{}{}{}{}{s.m.}{Indivíduo natural ou habitante dessas ilhas.}{an.ti.lha.no}{0}
\verb{antilogaritmo}{}{}{}{}{s.m.}{Número que corresponde a um logaritmo dado.}{an.ti.lo.ga.rit.mo}{0}
\verb{antílope}{}{Zool.}{}{}{s.m.}{Mamífero ruminante, da família dos bovídeos, que apresenta chifres permanentes, direcionados para cima e para trás.}{an.tí.lo.pe}{0}
\verb{antimatéria}{}{Fís.}{}{}{s.f.}{Denominação dada à matéria formada por partículas cujas cargas elétricas são opostas às das partículas encontradas normalmente no universo.}{an.ti.ma.té.ria}{0}
\verb{antimilitarismo}{}{}{}{}{s.m.}{Posição ou sentimento contrário ao militarismo, em que prevalece o poder das forças armadas na resolução de conflitos.}{an.ti.mi.li.ta.ris.mo}{0}
\verb{antimilitarismo}{}{}{}{}{}{Opinião contrária às ações militares ou à guerra.}{an.ti.mi.li.ta.ris.mo}{0}
\verb{antimíssil}{}{}{"-eis}{}{s.m.}{Armamento utilizado para interceptar e destruir outro míssil antes de este atingir o alvo.}{an.ti.mís.sil}{0}
\verb{antimônio}{}{Quím.}{}{}{s.m.}{Elemento químico com brilho metálico branco"-azulado, sólido, cristalino, quebradiço, pouco maleável, mau condutor de calor e eletricidade, usado em várias ligas, compostos medicinais, tintas etc. \elemento{51}{121.76}{Sb}.}{an.ti.mô.nio}{0}
\verb{antinatural}{}{}{"-ais}{}{adj.2g.}{Que é contrário à natureza ou a suas leis e fenômenos.}{an.ti.na.tu.ral}{0}
\verb{antinomia}{}{}{}{}{s.f.}{Contradição ou oposição entre dois princípios ou leis.}{an.ti.no.mi.a}{0}
%\verb{antinômico}{}{}{}{}{}{0}{an.ti.nô.mi.co}{0}
\verb{antiofídico}{}{}{}{}{adj.}{Diz"-se do soro produzido a partir do veneno de cobras e utilizado para combater os efeitos desse veneno.}{an.ti.o.fí.di.co}{0}
\verb{antioxidante}{cs}{}{}{}{adj.2g.}{Diz"-se da substância que inibe ou retarda os efeitos da oxidação.}{an.ti.o.xi.dan.te}{0}
\verb{antipapa}{}{Relig.}{}{}{s.m.}{Indivíduo que disputa o cargo de chefe supremo da Igreja Católica através de meios não legitimados por essa; falso papa.}{an.ti.pa.pa}{0}
\verb{antipartícula}{}{Fís.}{}{}{s.f.}{Partícula que apresenta massa idêntica à de outra, mas com carga elétrica oposta.}{an.ti.par.tí.cu.la}{0}
\verb{antipatia}{}{}{}{}{s.f.}{Repulsão instintiva e espontânea por algo ou alguém; aversão, repugnância.}{an.ti.pa.ti.a}{0}
\verb{antipatia}{}{}{}{}{}{Incompatibilidade ou desarmonia entre corpos ou substâncias.}{an.ti.pa.ti.a}{0}
\verb{antipatia}{}{Ant.}{}{}{}{simpatia}{an.ti.pa.ti.a}{0}
\verb{antipático}{}{}{}{}{adj.}{Que provoca antipatia; repulsivo, repugnante.}{an.ti.pá.ti.co}{0}
\verb{antipático}{}{}{}{}{}{Incompatível, desarmônico, discordante.}{an.ti.pá.ti.co}{0}
\verb{antipatizar}{}{}{}{}{v.t.}{Ter antipatia, aversão; implicar, hostilizar.}{an.ti.pa.ti.zar}{\verboinum{1}}
\verb{antipatriota}{ó}{}{}{}{adj.2g.}{Que é contra ou não ama a pátria, a terra natal.}{an.ti.pa.tri.o.ta}{0}
\verb{antipatriótico}{}{}{}{}{adj.}{Que é contrário aos interesses da pátria, da terra natal.}{an.ti.pa.tri.ó.ti.co}{0}
\verb{antipedagógico}{}{}{}{}{adj.}{Que se opõe aos princípios da pedagogia, da educação de crianças e jovens.}{an.ti.pe.da.gó.gi.co}{0}
\verb{antiperspirante}{}{}{}{}{adj.2g.}{Diz"-se da substância que reduz o suor excessivo.}{an.ti.pers.pi.ran.te}{0}
\verb{antipirético}{}{}{}{}{adj.}{Diz"-se do medicamento que combate a febre, diminuindo a temperatura do corpo; febrífugo, antifebril, antitérmico.}{an.ti.pi.ré.ti.co}{0}
\verb{antípoda}{}{}{}{}{s.2g.}{Habitante que, em relação a outro do globo, encontra"-se em ponto diametralmente oposto.}{an.tí.po.da}{0}
\verb{antípoda}{}{Fig.}{}{}{adj.2g.}{Contrário, oposto.}{an.tí.po.da}{0}
\verb{antiquado}{}{}{}{}{adj.}{Que se tornou antigo, obsoleto, fora de uso, arcaico.}{an.ti.qua.do}{0}
\verb{antiqualha}{}{}{}{}{s.f.}{Ruínas de monumentos antigos com valor histórico.}{an.ti.qua.lha}{0}
\verb{antiqualha}{}{}{}{}{}{Conjunto de objetos antigos; antiguidades, velharias.}{an.ti.qua.lha}{0}
\verb{antiqualha}{}{}{}{}{}{Costumes e hábitos de tempos antigos.}{an.ti.qua.lha}{0}
\verb{antiquário}{}{}{}{}{s.m.}{Indivíduo que estuda, coleciona ou comercializa antiguidades.}{an.ti.quá.rio}{0}
\verb{antiquário}{}{}{}{}{}{Loja onde se vendem antiguidades.}{an.ti.quá.rio}{0}
\verb{antir"-rábico}{}{}{antirrábicos}{}{adj.}{Que evita ou combate a raiva.}{an.tir"-rá.bi.co}{0}
\verb{antir"-rábico}{}{}{antirrábicos}{}{s.m.}{Aplicável contra a raiva.}{an.tir"-rá.bi.co}{0}
\verb{anti"-semita}{}{}{}{}{adj.2g.}{Que se opõe aos semitas, especialmente aos judeus}{an.ti"-se.mi.ta}{0}
\verb{anti"-semita}{}{}{}{}{s.m.}{Inimigo dos semitas.}{an.ti"-se.mi.ta}{0}
\verb{anti"-semitismo}{}{}{}{}{s.m.}{Doutrina ou movimento racista contra os semitas, especialmente contra os judeus.}{an.ti"-se.mi.tis.mo}{0}
\verb{anti"-sepsia}{}{}{}{}{}{Var. de \textit{antissepsia}.}{an.ti"-sep.si.a}{0}
\verb{anti"-séptico}{}{}{}{}{}{Var. de \textit{antisséptico}.}{an.ti"-sép.ti.co}{0}
\verb{antissifilítico}{}{}{antissifilíticos}{}{adj.}{Diz"-se de substância que combate a sífilis.}{an.tis.si.fi.lí.ti.co}{0}
\verb{antissifilítico}{}{}{antissifilíticos}{}{s.m.}{Essa substância.}{an.tis.si.fi.lí.ti.co}{0}
\verb{antissocial}{}{}{antissociais}{}{adj.2g.}{Contrário às ideias, costumes ou interesses da sociedade.}{an.tis.so.ci.al}{0}
\verb{antissepsia}{}{}{}{}{s.f.}{Conjunto de métodos que têm por objetivo eliminar micróbios e prevenir infecções; desinfecção.}{an.tis.sep.si.a}{0}
\verb{antisséptico}{}{}{}{}{adj.}{Diz"-se de substância capaz de impedir a proliferação de micróbios; desinfetante.}{an.tis.sép.ti.co}{0}
\verb{antisséptico}{}{}{}{}{s.m.}{Essa substância.}{an.tis.sép.ti.co}{0}
\verb{antitérmico}{}{}{}{}{adj.}{Que se opõe ao calor; isolante.}{an.ti.tér.mi.co}{0}
\verb{antitérmico}{}{Med.}{}{}{}{Que combate a febre; antipirético.}{an.ti.tér.mi.co}{0}
\verb{antítese}{}{Gram.}{}{}{s.f.}{Recurso estilístico usado para aproximar duas palavras ou ideias com sentidos opostos.}{an.tí.te.se}{0}
\verb{antítese}{}{}{}{}{}{Qualquer pessoa ou coisa nitidamente contrária a outra.}{an.tí.te.se}{0}
\verb{antítese}{}{}{}{}{}{Na filosofia, oposição ou contradição entre duas ideias ou proposições.}{an.tí.te.se}{0}
\verb{antitetânico}{}{}{}{}{adj.}{Diz"-se da substância que combate ou previne o tétano, doença infecciosa que provoca contrações musculares.}{an.ti.te.tâ.ni.co}{0}
\verb{antitético}{}{}{}{}{adj.}{Que constitui ou contém uma antítese; contrário, antagônico.}{an.ti.té.ti.co}{0}
\verb{antitóxico}{cs}{}{}{}{adj.}{Diz"-se da substância que anula os efeitos de veneno ou toxina.}{an.ti.tó.xi.co}{0}
\verb{antitoxina}{cs}{Bioquím.}{}{}{s.f.}{Substância formada no organismo para combater as toxinas produzidas por certas bactérias.}{an.ti.to.xi.na}{0}
\verb{antitruste}{}{}{}{}{adj.2g.}{Diz"-se da medida destinada a restringir ou combater a formação de trustes ou monopólios.}{an.ti.trus.te}{0}
\verb{antivariólico}{}{}{}{}{adj.}{Diz"-se do medicamento ou vacina utilizada para combater ou prevenir a varíola, doença infecciosa e contagiosa.}{an.ti.va.ri.ó.li.co}{0}
\verb{antivenéreo}{}{}{}{}{adj.}{Diz"-se do medicamento que combate doenças venéreas, as que afetam os órgãos genitais.}{an.ti.ve.né.re.o}{0}
\verb{antiviral}{}{}{"-ais}{}{adj.2g.}{Antivirótico.}{an.ti.vi.ral}{0}
\verb{antivirótico}{}{}{}{}{adj.}{Diz"-se do medicamento que combate os vírus, atuando no tratamento de doenças causadas por eles; antiviral.}{an.ti.vi.ró.ti.co}{0}
\verb{antivírus}{}{Informát.}{}{}{s.m.}{Programa que protege o computador, detectando e eliminando programas danosos existentes nele.}{an.ti.ví.rus}{0}
%\verb{antojar}{}{}{}{}{}{0}{an.to.jar}{0}
\verb{antojo}{ô}{}{}{}{s.m.}{Ato de colocar diante dos olhos.}{an.to.jo}{0}
\verb{antojo}{ô}{}{}{}{}{Desejo ou apetite excêntrico que certas mulheres grávidas apresentam.}{an.to.jo}{0}
\verb{antolhos}{ô}{}{}{}{s.m.pl.}{Peça de couro ou pano que se coloca na cabeça de certos animais, como o cavalo, para proteger"-lhes os olhos e impedir"-lhes a visão lateral.}{an.to.lhos}{0}
\verb{antologia}{}{}{}{}{s.f.}{Coleção de textos, em verso ou em prosa, de autores consagrados; seleção.}{an.to.lo.gi.a}{0}
\verb{antologia}{}{Bot.}{}{}{}{Tratado ou estudo acerca das flores.}{an.to.lo.gi.a}{0}
\verb{antológico}{}{}{}{}{adj.}{Relativo a antologia.}{an.to.ló.gi.co}{0}
\verb{antológico}{}{}{}{}{}{Que merece ser registrado em antologia.}{an.to.ló.gi.co}{0}
\verb{antonímia}{}{Gram.}{}{}{s.f.}{Relação de oposição de sentido entre duas palavras.}{an.to.ní.mia}{0}
\verb{antonímia}{}{}{}{}{}{Emprego de palavras com significados opostos.}{an.to.ní.mia}{0}
\verb{antônimo}{}{}{}{}{adj.}{Que é oposto, contrário.}{an.tô.ni.mo}{0}
\verb{antônimo}{}{Gram.}{}{}{s.m.}{Palavra ou locução que denota significado oposto ao de outra.}{an.tô.ni.mo}{0}
\verb{antonomásia}{}{}{}{}{s.f.}{Apelido, alcunha, cognome.}{an.to.no.má.sia}{0}
\verb{antonomásia}{}{Gram.}{}{}{}{Substituição do nome próprio de uma pessoa por um qualificativo que a caracterize.}{an.to.no.má.sia}{0}
\verb{antozoário}{}{Zool.}{}{}{s.m.}{Classe de animais celenterados marinhos que não apresentam forma semelhante à medusa em seu desenvolvimento, como os corais e as anêmonas"-do"-mar.}{an.to.zo.á.rio}{0}
\verb{antraz}{}{Med.}{}{}{s.m.}{Doença infecciosa, fatal ao gado bovino e a carneiros, e transmissível ao homem; caracteriza"-se pela formação de furúnculos até a mortificação de uma parte dos tecidos, com sintomas gerais graves.}{an.traz}{0}
\verb{antro}{}{}{}{}{s.m.}{Caverna, cova ou gruta natural, escura e profunda.}{an.tro}{0}
\verb{antro}{}{Fig.}{}{}{}{Lugar em que há corrupção,   perdição, vícios, pessoas com esses hábitos.}{an.tro}{0}
\verb{antro}{}{Anat.}{}{}{}{Cavidade ou espaço, especialmente dentro de um osso.}{an.tro}{0}
\verb{antropocêntrico}{}{}{}{}{adj.}{Relativo a antropocentrismo.}{an.tro.po.cên.tri.co}{0}
\verb{antropocêntrico}{}{}{}{}{}{Que considera o homem como o centro ou a medida do Universo. }{an.tro.po.cên.tri.co}{0}
\verb{antropocentrismo}{}{Filos.}{}{}{s.m.}{Forma de pensamento que considera o homem o centro do Universo, organizando todo o conhecimento de acordo com esse pressuposto.}{an.tro.po.cen.tris.mo}{0}
\verb{antropofagia}{}{}{}{}{s.f.}{Condição ou ato de antropófago.}{an.tro.po.fa.gi.a}{0}
\verb{antropofagia}{}{}{}{}{}{Prática regular e institucionalizada de consumo de carne humana por seres humanos, com caráter ritual; canibalismo. }{an.tro.po.fa.gi.a}{0}
\verb{antropofágico}{}{}{}{}{adj.}{Relativo à antropofagia.}{an.tro.po.fá.gi.co}{0}
\verb{antropófago}{}{}{}{}{adj.}{Que come carne humana.}{an.tro.pó.fa.go}{0}
%\verb{antropogeografia}{}{}{}{}{}{0}{an.tro.po.ge.o.gra.fi.a}{0}
\verb{antropoide}{}{}{}{}{adj.2g.}{Que tem formas semelhantes às do homem.}{an.tro.poi.de}{0}
\verb{antropologia}{}{}{}{}{s.f.}{Ciência que estuda o homem, suas origens, evolução, cultura, costumes, crenças etc.}{an.tro.po.lo.gi.a}{0}
\verb{antropológico}{}{}{}{}{adj.}{Relativo à antropologia.}{an.tro.po.ló.gi.co}{0}
\verb{antropólogo}{}{}{}{}{s.m.}{Especialista ou estudioso de antropologia.}{an.tro.pó.lo.go}{0}
\verb{antropometria}{}{}{}{}{s.f.}{Processo de mensuração do corpo humano ou de suas partes.}{an.tro.po.me.tri.a}{0}
\verb{antropometria}{}{}{}{}{}{Técnica de identificação de indivíduos através das características ou medidas corporais.}{an.tro.po.me.tri.a}{0}
\verb{antropometria}{}{Por ext.}{}{}{}{Repartição ou seção onde se fazem serviços de antropometria.}{an.tro.po.me.tri.a}{0}
\verb{antropométrico}{}{}{}{}{adj.}{Relativo à antropometria.}{an.tro.po.mé.tri.co}{0}
\verb{antropomórfico}{}{}{}{}{adj.}{Que tem aspecto ou forma semelhante ao de um ser humano; antropomorfo.}{an.tro.po.mór.fi.co}{0}
\verb{antropomorfismo}{}{}{}{}{s.m.}{Forma de pensamento que atribui a entidades divinas ou seres sobrenaturais formas, comportamentos e pensamentos característicos dos seres humanos.}{an.tro.po.mor.fis.mo}{0}
\verb{antropomorfo}{ó}{}{}{}{adj.}{Antropomórfico.}{an.tro.po.mor.fo}{0}
\verb{antroponímia}{}{}{}{}{s.f.}{Área da onomástica que estuda os nomes próprios de pessoa.}{an.tro.po.ní.mia}{0}
\verb{antroponímia}{}{}{}{}{}{Lista, estudo ou teoria sobre antropônimos.}{an.tro.po.ní.mia}{0}
\verb{antropônimo}{}{}{}{}{s.m.}{Nome próprio de pessoa.}{an.tro.pô.ni.mo}{0}
\verb{antropopiteco}{é}{Paleo.}{}{}{s.m.}{Primata hipotético apresentado como o intermediário entre o macaco e o homem.}{an.tro.po.pi.te.co}{0}
\verb{antropopiteco}{é}{Desus.}{}{}{}{Designação imprecisa às formas mais primitivas de hominídeos.}{an.tro.po.pi.te.co}{0}
\verb{antroposofia}{}{Filos.}{}{}{s.f.}{Doutrina espiritual e mística que estuda a natureza do ser humano, conferindo igual importância a seus aspectos físicos, emocionais e intelectuais.     }{an.tro.po.so.fi.a}{0}
\verb{antúrio}{}{Bot.}{}{}{s.m.}{Planta ornamental com folhas nervadas e flores vermelhas, creme ou esverdeadas, constituídas por um eixo carnoso.}{an.tú.rio}{0}
\verb{anu}{}{Zool.}{}{}{s.m.}{Pequena ave de cor preta, bico forte e cauda longa; anum.}{a.nu}{0}
\verb{anual}{}{}{"-ais}{}{adj.2g.}{Que dura um ano.}{a.nu.al}{0}
\verb{anual}{}{}{"-ais}{}{}{Que ocorre uma vez por ano.}{a.nu.al}{0}
\verb{anuário}{}{}{}{}{s.m.}{Publicação anual com informações sobre determinado ramo de atividade ou instituição.}{a.nu.á.rio}{0}
\verb{anuência}{}{}{}{}{s.f.}{Ato ou efeito de anuir; consentimento.}{a.nu.ên.cia}{0}
\verb{anuidade}{}{}{}{}{s.f.}{Quantia que se paga anualmente a uma instituição.}{a.nu.i.da.de}{0}
\verb{anuir}{}{}{}{}{v.t.}{Consentir, estar de acordo.}{a.nu.ir}{\verboinum{26}}
\verb{anulação}{}{}{"-ões}{}{s.f.}{Ato ou efeito de anular; anulamento, invalidação.}{a.nu.la.ção}{0}
\verb{anular}{}{}{}{}{v.t.}{Tornar nulo ou sem efeito.}{a.nu.lar}{0}
\verb{anular}{}{}{}{}{}{Obter vitória; vencer.}{a.nu.lar}{0}
\verb{anular}{}{}{}{}{v.pron.}{Perder a identidade; desvalorizar"-se.}{a.nu.lar}{\verboinum{1}}
\verb{anular}{}{}{}{}{adj.2g.}{Relativo a anel.}{a.nu.lar}{0}
\verb{anular}{}{}{}{}{}{Que tem a forma de anel; aneliforme.}{a.nu.lar}{0}
\verb{anular}{}{}{}{}{s.m.}{Quarto dedo (a contar do polegar), no qual geralmente se usa aliança; dedo anular.}{a.nu.lar}{0}
\verb{anulável}{}{}{"-eis}{}{adj.2g.}{Que pode ser anulado.}{a.nu.lá.vel}{0}
\verb{anum}{}{Zool.}{}{}{s.m.}{Anu.}{a.num}{0}
\verb{anunciação}{}{}{"-ões}{}{s.f.}{Ato ou efeito de anunciar.}{a.nun.ci.a.ção}{0}
\verb{anunciação}{}{Relig.}{"-ões}{}{}{No cristianismo, notícia levada pelo anjo Gabriel à Virgem Maria de que ela seria mãe do Filho de Deus.}{a.nun.ci.a.ção}{0}
\verb{anunciação}{}{Relig.}{"-ões}{}{}{Comemoração desse evento pela Igreja Católica.}{a.nun.ci.a.ção}{0}
\verb{anunciante}{}{}{}{}{adj.2g.}{Que anuncia.}{a.nun.ci.an.te}{0}
\verb{anunciante}{}{}{}{}{s.2g.}{Indivíduo ou instituição que veicula anúncios nos meios de comunicação.}{a.nun.ci.an.te}{0}
\verb{anunciar}{}{}{}{}{v.t.}{Fazer saber; noticiar.}{a.nun.ci.ar}{0}
\verb{anunciar}{}{}{}{}{}{Promover a divulgação; tornar público; divulgar.}{a.nun.ci.ar}{0}
\verb{anunciar}{}{}{}{}{}{Inserir anúncio de produto ou serviço em determinado meio de comunicação.}{a.nun.ci.ar}{\verboinum{1}}
\verb{anúncio}{}{}{}{}{s.m.}{Mensagem pela qual se dá qualquer coisa ao conhecimento público.}{a.nún.cio}{0}
\verb{anúncio}{}{}{}{}{}{Sinal, indício, prognóstico.}{a.nún.cio}{0}
\verb{anúncio}{}{}{}{}{}{Mensagem inserida em qualquer meio de comunicação com a finalidade de transmitir ao público as qualidades de determinado produto, serviço ou instituição; propaganda.}{a.nún.cio}{0}
\verb{anuro}{}{Zool.}{}{}{adj.}{Que não possui cauda.}{a.nu.ro}{0}
\verb{anuro}{}{Zool.}{}{}{s.m.}{Espécime dos anuros, ordem dos anfíbios que não possuem cauda, como as rãs e os sapos. }{a.nu.ro}{0}
\verb{ânus}{}{Anat.}{}{}{s.m.}{Orifício terminal do tubo digestivo, na extremidade do reto, pelo qual se expelem os excrementos.}{â.nus}{0}
\verb{anuviar}{}{}{}{}{v.t.}{Cobrir de nuvens; nublar.}{a.nu.vi.ar}{\verboinum{1}}
\verb{anverso}{é}{}{}{}{s.m.}{Lado principal ou frontal de objetos que apresentem dois lados opostos.}{an.ver.so}{0}
\verb{anverso}{é}{}{}{}{}{Lado de moeda ou medalha em que está o emblema ou efígie.}{an.ver.so}{0}
\verb{anverso}{é}{Ant.}{}{}{}{anteverso}{an.ver.so}{0}
\verb{anzol}{ó}{}{"-óis}{}{s.m.}{Pequeno gancho metálico preso a uma linha no qual se coloca uma isca para pescar.}{an.zol}{0}
\verb{anzol}{ó}{Fig.}{"-óis}{}{}{Artifício para a obtenção de algo; artimanha, engodo.}{an.zol}{0}
\verb{ao}{}{}{}{}{}{Combinação da preposição \textit{a} com o artigo definido ou pronome demonstrativo \textit{o}.}{ao}{0}
\verb{aonde}{}{}{}{}{}{Combinação da preposição \textit{a} com o advérbio ou pronome interrogativo ou relativo \textit{onde}.}{a.on.de}{0}
\verb{aorta}{ó}{Anat.}{}{}{s.f.}{Grande artéria que parte do ventrículo esquerdo do coração e que distribui sangue arterial a todas as partes do organismo.}{a.or.ta}{0}
\verb{AP}{}{}{}{}{}{Sigla do estado do Amapá.}{AP}{0}
\verb{apache}{}{}{}{}{adj.2g.}{Relativo aos Apaches.}{a.pa.che}{0}
\verb{apache}{}{}{}{}{s.2g.}{Indivíduo dos apaches, grupo indígena norte"-americano que habita uma região no sudoeste dos \textsc{eua}.}{a.pa.che}{0}
\verb{apadrinhado}{}{}{}{}{adj.}{Que tem padrinho.}{a.pa.dri.nha.do}{0}
\verb{apadrinhado}{}{}{}{}{}{Que está sob a proteção ou amparo de alguém.}{a.pa.dri.nha.do}{0}
\verb{apadrinhado}{}{}{}{}{}{Fundamentado, abonado.}{a.pa.dri.nha.do}{0}
\verb{apadrinhamento}{}{}{}{}{s.m.}{Ato ou efeito de apadrinhar.}{a.pa.dri.nha.men.to}{0}
\verb{apadrinhar}{}{}{}{}{v.t.}{Ser padrinho.}{a.pa.dri.nhar}{0}
\verb{apadrinhar}{}{}{}{}{}{Oferecer patrocínio ou proteção; favorecer, defender.}{a.pa.dri.nhar}{\verboinum{1}}
\verb{apagado}{}{}{}{}{adj.}{Que não tem fogo ou luz; extinto.}{a.pa.ga.do}{0}
\verb{apagado}{}{Fig.}{}{}{}{Que não se destaca; sem brilho; insignificante.}{a.pa.ga.do}{0}
\verb{apagador}{ô}{}{}{}{adj.}{Que apaga.}{a.pa.ga.dor}{0}
\verb{apagador}{ô}{}{}{}{s.m.}{Utensílio que serve para apagar o que foi escrito no quadro"-negro ou lousa.}{a.pa.ga.dor}{0}
\verb{apagão}{}{Bras.}{"-ões}{}{s.m.}{Interrupção no fornecimento de energia elétrica; blecaute.}{a.pa.gão}{0}
\verb{apagar}{}{}{}{}{v.t.}{Fazer desaparecer (fogo, luz ou brilho); extinguir.}{a.pa.gar}{0}
\verb{apagar}{}{}{}{}{}{Limpar sinais gráficos escritos.}{a.pa.gar}{0}
\verb{apagar}{}{Fig.}{}{}{}{Obscurecer, tirar a importância.}{a.pa.gar}{0}
\verb{apagar}{}{}{}{}{}{Matar.}{a.pa.gar}{0}
\verb{apagar}{}{}{}{}{v.i.}{Adormecer.}{a.pa.gar}{0}
\verb{apagar}{}{}{}{}{}{Morrer.}{a.pa.gar}{\verboinum{5}}
\verb{apaixonado}{ch}{}{}{}{adj.}{Que está dominado pela paixão; enamorado.}{a.pai.xo.na.do}{0}
\verb{apaixonante}{ch}{}{}{}{adj.2g.}{Que apaixona, cativa, prende a atenção.}{a.pai.xo.nan.te}{0}
\verb{apaixonar}{ch}{}{}{}{v.t.}{Inspirar paixão.}{a.pai.xo.nar}{0}
\verb{apaixonar}{ch}{}{}{}{}{Entusiasmar, arrebatar.}{a.pai.xo.nar}{\verboinum{1}}
\verb{apalacetado}{}{}{}{}{adj.}{Com aparência ou feitio de palacete.}{a.pa.la.ce.ta.do}{0}
\verb{apalavrado}{}{}{}{}{adj.}{Combinado verbalmente; acertado, marcado.}{a.pa.la.vra.do}{0}
\verb{apalavrar}{}{}{}{}{v.t.}{Contratar ou combinar verbalmente.}{a.pa.la.vrar}{\verboinum{1}}
\verb{apalermado}{}{}{}{}{adj.}{Que tem modos ou aparência de palerma.}{a.pa.ler.ma.do}{0}
\verb{apalermar}{}{}{}{}{v.t.}{Tornar palerma.}{a.pa.ler.mar}{\verboinum{1}}
\verb{apalpadela}{é}{}{}{}{s.f.}{Ato de apalpar levemente ou uma única vez.}{a.pal.pa.de.la}{0}
\verb{apalpar}{}{}{}{}{v.t.}{Tocar ou examinar com a(s) mão(s).}{a.pal.par}{\verboinum{1}}
\verb{apanágio}{}{}{}{}{s.m.}{Propriedade característica; atributo.}{a.pa.ná.gio}{0}
\verb{apanágio}{}{Jur.}{}{}{}{Pensão alimentícia oriunda dos bens do falecido.}{a.pa.ná.gio}{0}
\verb{apanha}{}{}{}{}{s.f.}{Ato ou efeito de apanhar.}{a.pa.nha}{0}
\verb{apanha}{}{}{}{}{}{Colheita de determinado produto agrícola.}{a.pa.nha}{0}
\verb{apanhado}{}{}{}{}{adj.}{Que se apanhou.}{a.pa.nha.do}{0}
\verb{apanhado}{}{}{}{}{s.m.}{Seleção, resumo, sinopse.}{a.pa.nha.do}{0}
\verb{apanhar}{}{}{}{}{v.t.}{Pegar, segurar.}{a.pa.nhar}{0}
\verb{apanhar}{}{}{}{}{}{Colher, recolher.}{a.pa.nhar}{0}
\verb{apanhar}{}{}{}{}{v.i.}{Receber pancada ou surra.}{a.pa.nhar}{\verboinum{1}}
\verb{apaniguado}{}{}{}{}{adj.}{Protegido, favorito, afilhado.}{a.pa.ni.gua.do}{0}
\verb{apaniguado}{}{}{}{}{}{Seguidor de determinada ideologia, pessoa ou partido.}{a.pa.ni.gua.do}{0}
\verb{apaniguar}{}{}{}{}{v.t.}{Dar proteção, favorecer.}{a.pa.ni.guar}{\verboinum{10}\verboirregular[apaniguo]{apaníguo}}
\verb{apara}{}{}{}{}{s.f.}{Fragmento de objeto que se apara ou se desbasta.}{a.pa.ra}{0}
\verb{aparadeira}{ê}{}{}{}{s.f.}{Parteira.}{a.pa.ra.dei.ra}{0}
\verb{aparadeira}{ê}{}{}{}{}{Comadre (urinol).}{a.pa.ra.dei.ra}{0}
\verb{aparadeira}{ê}{}{}{}{}{Recipiente para vomitar.}{a.pa.ra.dei.ra}{0}
\verb{aparadela}{é}{}{}{}{s.f.}{Ato ou efeito de aparar levemente.}{a.pa.ra.de.la}{0}
\verb{aparador}{ô}{}{}{}{adj.}{Que apara.}{a.pa.ra.dor}{0}
\verb{aparador}{ô}{}{}{}{s.m.}{Móvel de sala de jantar constituído por tampo e armário na parte inferior, utilizado para apoiar ou guardar aparelhagem de jantar.}{a.pa.ra.dor}{0}
\verb{aparafusar}{}{}{}{}{v.t.}{Fixar ou apertar com parafuso.}{a.pa.ra.fu.sar}{0}
\verb{aparafusar}{}{Fig.}{}{}{}{Pensar fixamente, meditar, matutar.}{a.pa.ra.fu.sar}{\verboinum{1}}
\verb{aparar}{}{}{}{}{v.t.}{Segurar, sustentar (aquilo que cai ou que é atirado).}{a.pa.rar}{0}
\verb{aparar}{}{}{}{}{}{Cortar partes excessivas; tornar plano; alisar.}{a.pa.rar}{\verboinum{1}}
\verb{aparato}{}{}{}{}{s.m.}{Luxo, pompa, ostentação.}{a.pa.ra.to}{0}
\verb{aparato}{}{}{}{}{}{Preparativos ou equipamentos necessários à realização de determinado evento ou objetivo.}{a.pa.ra.to}{0}
\verb{aparatoso}{ô}{}{"-osos ⟨ó⟩}{"-osa ⟨ó⟩}{adj.}{Feito com ou que exibe aparato.}{a.pa.ra.to.so}{0}
\verb{aparceirar}{}{}{}{}{v.t.}{Admitir como parceiro ou sócio.}{a.par.cei.rar}{0}
\verb{aparceirar}{}{}{}{}{v.pron.}{Entrar em parceria; associar"-se.}{a.par.cei.rar}{\verboinum{1}}
\verb{aparecer}{ê}{}{}{}{v.i.}{Tornar"-se visível.}{a.pa.re.cer}{0}
\verb{aparecer}{ê}{}{}{}{}{Destacar"-se, sobressair.}{a.pa.re.cer}{0}
\verb{aparecer}{ê}{}{}{}{}{Comparecer, apresentar"-se.}{a.pa.re.cer}{\verboinum{15}}
\verb{aparecimento}{}{}{}{}{s.m.}{Ato ou efeito de aparecer.}{a.pa.re.ci.men.to}{0}
\verb{aparecimento}{}{}{}{}{}{Princípio, origem, primeira aparição.}{a.pa.re.ci.men.to}{0}
\verb{aparelhado}{}{}{}{}{adj.}{Munido de equipamentos, provisões, alimentos etc.; preparado.}{a.pa.re.lha.do}{0}
\verb{aparelhagem}{}{}{"-ens}{}{s.f.}{Conjunto de aparelhos ou equipamentos.}{a.pa.re.lha.gem}{0}
\verb{aparelhamento}{}{}{}{}{s.m.}{Ato ou efeito de aparelhar.}{a.pa.re.lha.men.to}{0}
\verb{aparelhar}{}{}{}{}{v.t.}{Preparar convenientemente; organizar, equipar.}{a.pa.re.lhar}{0}
\verb{aparelhar}{}{}{}{}{}{Desbastar, aplainar (madeira ou pedra).}{a.pa.re.lhar}{\verboinum{1}}
\verb{aparelho}{ê}{}{}{}{s.m.}{Máquina, instrumento, conjunto de mecanismos para uma finalidade ou uso específico.}{a.pa.re.lho}{0}
\verb{aparelho}{ê}{}{}{}{}{Conjunto de arreios das cavalgaduras.}{a.pa.re.lho}{0}
\verb{aparelho}{ê}{Anat.}{}{}{}{Conjunto de órgãos que cooperam em determinada função.}{a.pa.re.lho}{0}
\verb{aparência}{}{}{}{}{s.f.}{Aspecto exterior ou visível.}{a.pa.rên.cia}{0}
\verb{aparência}{}{}{}{}{}{Mostra enganosa; falso indício, fingimento.}{a.pa.rên.cia}{0}
\verb{aparentado}{}{}{}{}{adj.}{Que tem parentes, geralmente influentes.}{a.pa.ren.ta.do}{0}
\verb{aparentado}{}{Fig.}{}{}{}{Semelhante.}{a.pa.ren.ta.do}{0}
\verb{aparentar}{}{}{}{}{v.t.}{Mostrar na aparência.}{a.pa.ren.tar}{0}
\verb{aparentar}{}{}{}{}{}{Simular, fingir.}{a.pa.ren.tar}{\verboinum{1}}
\verb{aparentar}{}{}{}{}{v.t.}{Estabelecer parentesco, tornar parente.}{a.pa.ren.tar}{\verboinum{1}}
\verb{aparente}{}{}{}{}{adj.2g.}{Que aparece; visível.}{a.pa.ren.te}{0}
\verb{aparente}{}{}{}{}{}{Verossímil, provável.}{a.pa.ren.te}{0}
\verb{aparente}{}{}{}{}{}{Que dá aparência de ser sem sê"-lo na realidade; falso, fingido, simulado.}{a.pa.ren.te}{0}
\verb{aparição}{}{}{"-ões}{}{s.f.}{Ato ou efeito de aparecer.}{a.pa.ri.ção}{0}
\verb{aparição}{}{}{"-ões}{}{}{Manifestação repentina e sobrenatural de um fantasma, santo ou entidade divina.}{a.pa.ri.ção}{0}
\verb{apartado}{}{}{}{}{adj.}{Que se apartou; separado.}{a.par.ta.do}{0}
\verb{apartado}{}{}{}{}{}{Que vive retirado; solitário, isolado.}{a.par.ta.do}{0}
\verb{apartado}{}{}{}{}{}{Longínquo.}{a.par.ta.do}{0}
\verb{apartamento}{}{}{}{}{s.m.}{Separação, apartação.}{a.par.ta.men.to}{0}
\verb{apartamento}{}{}{}{}{s.m.}{Residência particular em um edifício com diversas unidades.}{a.par.ta.men.to}{0}
\verb{apartar}{}{}{}{}{v.t.}{Separar, desunir, afastar.}{a.par.tar}{\verboinum{1}}
\verb{aparte}{}{}{}{}{s.m.}{Interrupção que se faz a alguém que está com a palavra.}{a.par.te}{0}
\verb{aparte}{}{Por ext.}{}{}{}{Autorização para fazer tal interrupção.}{a.par.te}{0}
\verb{apartear}{}{}{}{}{v.t.}{Interromper a fala de alguém com apartes.}{a.par.te.ar}{\verboinum{1}}
\verb{apartheid}{}{}{}{}{s.m.}{Segregação racial institucionalizada.}{\textit{apartheid}}{0}
\verb{apartheid}{}{}{}{}{}{Sistema oficial de segregação racial praticado na África do Sul para proteger a minoria branca.}{\textit{apartheid}}{0}
\verb{apart"-hotel}{}{}{apart"-hotéis}{}{s.m.}{Prédio de apartamentos com serviços de hotelaria.}{\textit{apart"-hotel}}{0}
\verb{apartidário}{}{}{}{}{adj.}{Que não está ligado a partidos políticos.}{a.par.ti.dá.rio}{0}
\verb{aparvalhado}{}{}{}{}{adj.}{Que age como parvo; tolo, idiota.}{a.par.va.lha.do}{0}
\verb{aparvalhar}{}{}{}{}{v.t.}{Tornar parvo, idiota, tolo.}{a.par.va.lhar}{\verboinum{1}}
\verb{apascentar}{}{}{}{}{v.t.}{Levar ao pasto.}{a.pas.cen.tar}{0}
\verb{apascentar}{}{}{}{}{}{Doutrinar, pastorear, ensinar.}{a.pas.cen.tar}{\verboinum{1}}
\verb{apassivado}{}{Gram.}{}{}{adj.}{Usado na voz passiva.}{a.pas.si.va.do}{0}
\verb{apassivador}{ô}{Gram.}{}{}{adj.}{Diz"-se de elemento gramatical que põe o verbo na voz passiva.}{a.pas.si.va.dor}{0}
\verb{apassivar}{}{}{}{}{v.t.}{Tornar passivo, inerte, apático.}{a.pas.si.var}{0}
\verb{apassivar}{}{Gram.}{}{}{}{Pôr o verbo na voz passiva.}{a.pas.si.var}{\verboinum{1}}
\verb{apatetado}{}{}{}{}{adj.}{Que age como pateta; abobalhado.}{a.pa.te.ta.do}{0}
\verb{apatetar}{}{}{}{}{v.t.}{Tornar pateta.}{a.pa.te.tar}{\verboinum{1}}
\verb{apatia}{}{}{}{}{s.f.}{Estado de insensibilidade emocional; indiferença; falta de interesse.}{a.pa.ti.a}{0}
\verb{apático}{}{}{}{}{adj.}{Indiferente, insensível.}{a.pá.ti.co}{0}
\verb{apátrida}{}{}{}{}{adj.}{Que não tem pátria.}{a.pá.tri.da}{0}
\verb{apavorado}{}{}{}{}{adj.}{Cheio de pavor; aterrorizado, pávido.}{a.pa.vo.ra.do}{0}
\verb{apavoramento}{}{}{}{}{s.m.}{Ato ou efeito de apavorar.}{a.pa.vo.ra.men.to}{0}
\verb{apavorante}{}{}{}{}{adj.2g.}{Que provoca pavor.}{a.pa.vo.ran.te}{0}
\verb{apavorar}{}{}{}{}{v.t.}{Causar pavor; aterrorizar, assustar.}{a.pa.vo.rar}{\verboinum{1}}
\verb{apaziguar}{}{}{}{}{v.t.}{Pôr em paz; pacificar.}{a.pa.zi.guar}{0}
\verb{apaziguar}{}{}{}{}{}{Acalmar, aquietar, sossegar.}{a.pa.zi.guar}{\verboinum{10}\verboirregular[apaziguo]{apazíguo}}
\verb{apear}{}{}{}{}{v.t.}{Descer ou retirar de montaria ou veículo.}{a.pe.ar}{\verboinum{1}}
\verb{apedeuta}{}{}{}{}{s.2g.}{Pessoa sem instrução; ignorante, apedeuto.}{a.pe.deu.ta}{0}
\verb{apedrejar}{}{}{}{}{v.t.}{Atirar pedras.}{a.pe.dre.jar}{0}
\verb{apedrejar}{}{Por ext.}{}{}{}{Castigar ou executar alguém a pedradas; lapidar.}{a.pe.dre.jar}{\verboinum{1}}
\verb{apegado}{}{}{}{}{adj.}{Que tem apego; afeiçoado.}{a.pe.ga.do}{0}
\verb{apegar}{}{}{}{}{v.t.}{Fazer sentir ou sentir apego; afeiçoar"-se.}{a.pe.gar}{0}
\verb{apegar}{}{}{}{}{}{Contagiar, contaminar.}{a.pe.gar}{0}
\verb{apegar}{}{}{}{}{v.pron.}{Agarrar"-se, amparar"-se. }{a.pe.gar}{\verboinum{5}}
\verb{apego}{ê}{}{}{}{s.m.}{Ligação afetiva; estima, afeição.}{a.pe.go}{0}
\verb{apelação}{}{}{"-ões}{}{s.f.}{Ato ou efeito de apelar.}{a.pe.la.ção}{0}
\verb{apelação}{}{Jur.}{"-ões}{}{}{Recurso das decisões judiciais definitivas.}{a.pe.la.ção}{0}
\verb{apelante}{}{}{}{}{adj.2g.}{Que apela judicialmente.}{a.pe.lan.te}{0}
\verb{apelar}{}{}{}{}{v.t.}{Invocar proteção; pedir auxílio.}{a.pe.lar}{0}
\verb{apelar}{}{Jur.}{}{}{}{Recorrer de decisão judicial através de apelação.}{a.pe.lar}{0}
\verb{apelar}{}{Bras.}{}{}{}{Recorrer de meios ou argumentos heterodoxos, grosseiros ou rudes.}{a.pe.lar}{\verboinum{1}}
\verb{apelativo}{}{}{}{}{adj.}{Que denomina ou designa algo ou alguém.}{a.pe.la.ti.vo}{0}
\verb{apelativo}{}{Bras.}{}{}{}{Que usa de recursos heterodoxos, violentos ou rudes.}{a.pe.la.ti.vo}{0}
\verb{apelável}{}{}{"-eis}{}{adj.2g.}{De que se pode apelar.}{a.pe.lá.vel}{0}
\verb{apelidar}{}{}{}{}{v.t.}{Chamar por apelido, alcunha.}{a.pe.li.dar}{\verboinum{1}}
\verb{apelido}{}{}{}{}{s.m.}{Designação diferente do nome civil pelo qual alguém é conhecido; alcunha.}{a.pe.li.do}{0}
\verb{apelo}{ê}{}{}{}{s.m.}{Ato ou efeito de apelar.}{a.pe.lo}{0}
\verb{apelo}{ê}{}{}{}{}{Solicitação ou convocação com determinada finalidade.}{a.pe.lo}{0}
\verb{apenas}{}{}{}{}{adv.}{Somente, unicamente.}{a.pe.nas}{0}
\verb{apenas}{}{}{}{}{}{Com dificuldade.}{a.pe.nas}{0}
\verb{apenas}{}{}{}{}{conj.}{Assim que, logo que.}{a.pe.nas}{0}
\verb{apêndice}{}{}{}{}{s.m.}{Suplemento, acessório.}{a.pên.di.ce}{0}
\verb{apêndice}{}{Biol.}{}{}{}{Parte saliente de um corpo, órgão ou estrutura.}{a.pên.di.ce}{0}
\verb{apendicite}{}{Med.}{}{}{s.f.}{Inflamação do apêndice ileocecal, estrutura do intestino.}{a.pen.di.ci.te}{0}
\verb{apendoar}{}{}{}{}{v.i.}{Criar pendões (diz"-se do milho).}{a.pen.do.ar}{\verboinum{7}}
\verb{apensar}{}{}{}{}{v.t.}{Pôr (algo) em apenso; anexar.}{a.pen.sar}{\verboinum{1}}
\verb{apenso}{}{}{}{}{adj.}{Que se juntou; anexo.}{a.pen.so}{0}
\verb{apequenar}{}{}{}{}{v.t.}{Tornar pequeno; encolher.}{a.pe.que.nar}{\verboinum{1}}
\verb{aperceber}{ê}{}{}{}{v.t.}{Perceber, dar"-se conta.}{a.per.ce.ber}{0}
\verb{aperceber}{ê}{}{}{}{}{Preparar, aparelhar, aprestar.}{a.per.ce.ber}{\verboinum{12}}
\verb{aperfeiçoamento}{}{}{}{}{s.m.}{Ato ou efeito de aperfeiçoar; melhoramento.}{a.per.fei.ço.a.men.to}{0}
\verb{aperfeiçoar}{}{}{}{}{v.t.}{Tornar perfeito ou mais perfeito.}{a.per.fei.ço.ar}{0}
\verb{aperfeiçoar}{}{}{}{}{}{Adquirir maior grau de conhecimento ou especialização.}{a.per.fei.ço.ar}{\verboinum{7}}
\verb{apergaminhado}{}{}{}{}{adj.}{Que tem aspecto semelhante ao pergaminho; pergaminhoso.}{a.per.ga.mi.nha.do}{0}
\verb{apergaminhado}{}{Por ext.}{}{}{}{Amarelento, ressequido.}{a.per.ga.mi.nha.do}{0}
\verb{aperitivo}{}{}{}{}{adj.}{Que estimula o apetite.}{a.pe.ri.ti.vo}{0}
\verb{aperitivo}{}{}{}{}{s.m.}{Comida leve ou bebida, geralmente alcoólica, servida antes de uma refeição.}{a.pe.ri.ti.vo}{0}
\verb{aperolar}{}{}{}{}{v.t.}{Dar aspecto de pérola.}{a.pe.ro.lar}{0}
\verb{aperolar}{}{}{}{}{}{Ornar com pérola.}{a.pe.ro.lar}{\verboinum{1}}
\verb{aperos}{é}{}{}{}{s.m.pl.}{Conjunto de peças para encilhar um cavalo.}{a.pe.ros}{0}
\verb{aperrar}{}{}{}{}{v.t.}{Engatilhar (arma de fogo).}{a.per.rar}{\verboinum{1}}
\verb{aperreação}{}{}{"-ões}{}{s.f.}{Aperreio.}{a.per.re.a.ção}{0}
\verb{aperreado}{}{}{}{}{adj.}{Que está contrariado, aborrecido.}{a.per.re.a.do}{0}
\verb{aperrear}{}{}{}{}{v.t.}{Aborrecer, impacientar, apoquentar.}{a.per.re.ar}{\verboinum{4}}
\verb{aperreio}{ê}{}{}{}{s.m.}{Sentimento causado por contrariedade ou aborrecimento; aperreação.}{a.per.rei.o}{0}
\verb{apertado}{}{}{}{}{adj.}{Que se apertou; unido fortemente.}{a.per.ta.do}{0}
\verb{apertado}{}{}{}{}{}{Comprimido, espremido.}{a.per.ta.do}{0}
\verb{apertado}{}{}{}{}{}{Muito justo, a ponto de causar desconforto.}{a.per.ta.do}{0}
\verb{apertado}{}{Fig.}{}{}{}{Aflito, angustiado.}{a.per.ta.do}{0}
\verb{apertado}{}{}{}{}{}{Pressionado, coagido.}{a.per.ta.do}{0}
\verb{apertado}{}{}{}{}{}{Que passa por dificuldades financeiras.}{a.per.ta.do}{0}
\verb{apertão}{}{}{"-ões}{}{s.m.}{Aperto forte.}{a.per.tão}{0}
\verb{apertão}{}{}{"-ões}{}{}{Aperto dado com intenção libidinosa.}{a.per.tão}{0}
\verb{apertão}{}{}{"-ões}{}{}{Situação difícil; aflição, angústia.}{a.per.tão}{0}
\verb{apertar}{}{}{}{}{v.t.}{Juntar ou unir muito; aproximar.}{a.per.tar}{0}
\verb{apertar}{}{}{}{}{}{Comprimir com força; espremer.}{a.per.tar}{0}
\verb{apertar}{}{}{}{}{}{Tornar mais estreito; ajustar.}{a.per.tar}{0}
\verb{apertar}{}{Fig.}{}{}{}{Afligir, angustiar, atormentar.}{a.per.tar}{0}
\verb{apertar}{}{Fig.}{}{}{}{Pressionar, coagir, insistir.}{a.per.tar}{0}
\verb{apertar}{}{Fig.}{}{}{v.i.}{Estar em dificuldades financeiras.}{a.per.tar}{\verboinum{1}}
\verb{aperto}{ê}{}{}{}{s.m.}{Ato ou efeito de apertar; compressão, ajuste, estreitamento.}{a.per.to}{0}
\verb{aperto}{ê}{}{}{}{}{Angústia, aflição, embaraço.}{a.per.to}{0}
\verb{aperto}{ê}{Fig.}{}{}{}{Intimidação, coação.}{a.per.to}{0}
\verb{aperto}{ê}{Fig.}{}{}{}{Situação financeira difícil; penúria.}{a.per.to}{0}
\verb{aperto}{ê}{Fig.}{}{}{}{Multidão compacta de pessoas.}{a.per.to}{0}
\verb{apertura}{}{}{}{}{s.f.}{Situação difícil; aflição, aperto.}{a.per.tu.ra}{0}
\verb{apesar}{}{}{}{}{adv.}{Usado na locução \textit{apesar de}: a despeito de; não obstante.}{a.pe.sar}{0}
\verb{apessoado}{}{}{}{}{adj.}{Diz"-se daquele que tem boa aparência; bem"-apessoado.}{a.pes.so.a.do}{0}
\verb{apetecer}{ê}{}{}{}{v.t.}{Ter ou despertar apetite; desejar.}{a.pe.te.cer}{0}
\verb{apetecer}{ê}{}{}{}{}{Desejar ardentemente, cobiçar.}{a.pe.te.cer}{0}
\verb{apetecer}{ê}{}{}{}{}{Despertar interesse; agradar.}{a.pe.te.cer}{\verboinum{15}}
\verb{apetecível}{}{}{"-eis}{}{adj.2g.}{Passível de apetecer; desejável, apetitoso.}{a.pe.te.cí.vel}{0}
\verb{apetência}{}{}{}{}{s.f.}{Desejo natural de comer; apetite.}{a.pe.tên.cia}{0}
\verb{apetite}{}{}{}{}{s.m.}{Desejo de comer; apetência.}{a.pe.ti.te}{0}
\verb{apetite}{}{Por ext.}{}{}{}{Ânimo, vontade, disposição.}{a.pe.ti.te}{0}
\verb{apetite}{}{}{}{}{}{Forte desejo sexual; sensualidade, volúpia.}{a.pe.ti.te}{0}
\verb{apetitoso}{ô}{}{"-s ⟨ó⟩}{"-a ⟨ó⟩}{adj.}{Que desperta a vontade de comer; apetecível, saboroso, gostoso.}{a.pe.ti.to.so}{0}
\verb{apetitoso}{ô}{}{"-s ⟨ó⟩}{"-a ⟨ó⟩}{}{Que excita o desejo; tentador, provocador.}{a.pe.ti.to.so}{0}
\verb{apetitoso}{ô}{}{"-s ⟨ó⟩}{"-a ⟨ó⟩}{}{Ambicioso, cobiçoso.}{a.pe.ti.to.so}{0}
\verb{apetrechar}{}{}{}{}{v.t.}{Munir de apetrechos, utensílios, ferramentas; petrechar.}{a.pe.tre.char}{\verboinum{1}}
\verb{apetrecho}{ê}{}{}{}{s.m.}{Ferramenta ou utensílio necessário à execução de algum ofício ou atividade; petrecho. }{a.pe.tre.cho}{0}
\verb{apiário}{}{}{}{}{adj.}{Relativo às abelhas.}{a.pi.á.rio}{0}
\verb{apiário}{}{}{}{}{s.m.}{Local próprio para a criação de abelhas.}{a.pi.á.rio}{0}
\verb{ápice}{}{}{}{}{s.m.}{O ponto mais alto; cume, vértice.}{á.pi.ce}{0}
\verb{ápice}{}{}{}{}{}{Ponto máximo; o grau mais elevado; apogeu, auge.}{á.pi.ce}{0}
\verb{apícola}{}{}{}{}{adj.2g.}{Relativo à criação de abelhas.}{a.pí.co.la}{0}
\verb{apícola}{}{}{}{}{s.2g.}{Criador de abelhas; apicultor.}{a.pí.co.la}{0}
\verb{apicultor}{ô}{}{}{}{s.m.}{Indivíduo que cria ou trata de abelhas; apícola, abelheiro.}{a.pi.cul.tor}{0}
\verb{apicultura}{}{}{}{}{s.f.}{Arte ou técnica de criar abelhas para obter mel, própolis etc.}{a.pi.cul.tu.ra}{0}
\verb{apiedar}{}{}{}{}{v.t.}{Tratar com piedade, compaixão.}{a.pi.e.dar}{0}
\verb{apiedar}{}{}{}{}{v.pron.}{Sentir piedade; sensibilizar"-se, condoer"-se, compadecer"-se.}{a.pi.e.dar}{\verboinum{1}}
\verb{apimentado}{}{}{}{}{adj.}{Temperado com pimenta.}{a.pi.men.ta.do}{0}
\verb{apimentado}{}{}{}{}{}{Estimulado, excitado.}{a.pi.men.ta.do}{0}
\verb{apimentado}{}{Fig.}{}{}{}{Picante, malicioso.}{a.pi.men.ta.do}{0}
\verb{apimentar}{}{}{}{}{v.t.}{Temperar com pimenta.}{a.pi.men.tar}{0}
\verb{apimentar}{}{}{}{}{}{Estimular, excitar.}{a.pi.men.tar}{0}
\verb{apimentar}{}{Fig.}{}{}{}{Tornar picante, malicioso.}{a.pi.men.tar}{\verboinum{1}}
\verb{apinhado}{}{}{}{}{adj.}{Completamente cheio; repleto, abarrotado.}{a.pi.nha.do}{0}
\verb{apinhado}{}{}{}{}{}{Muito junto; amontoado.}{a.pi.nha.do}{0}
\verb{apinhar}{}{}{}{}{v.t.}{Encher completamente; ajuntar, aglomerar, amontoar.}{a.pi.nhar}{0}
\verb{apinhar}{}{}{}{}{}{Unir estreitamente como pinhões em uma pinha.}{a.pi.nhar}{\verboinum{1}}
\verb{apitar}{}{}{}{}{v.i.}{Tocar com apito; dar sinal com apito.}{a.pi.tar}{0}
\verb{apitar}{}{Pop.}{}{}{v.t.}{Arbitrar; dirigir um jogo como árbitro.}{a.pi.tar}{\verboinum{1}}
\verb{apito}{}{}{}{}{s.m.}{Instrumento de metal ou outro material que produz um som agudo, usado para avisar, orientar, pedir auxílio, dirigir jogos etc.}{a.pi.to}{0}
\verb{apito}{}{}{}{}{}{O som produzido por esse instrumento.}{a.pi.to}{0}
\verb{aplacar}{}{}{}{}{v.t.}{Tornar plácido, calmo; tranquilizar, serenar.}{a.pla.car}{0}
\verb{aplacar}{}{}{}{}{}{Fazer diminuir de intensidade; abrandar, moderar.}{a.pla.car}{\verboinum{2}}
\verb{aplainar}{}{}{}{}{v.t.}{Alisar com plaina; remover arestas; nivelar, igualar.}{a.plai.nar}{0}
\verb{aplainar}{}{Fig.}{}{}{}{Remover as dificuldades; superar, simplificar.}{a.plai.nar}{\verboinum{1}}
\verb{aplanar}{}{}{}{}{v.t.}{Tornar plano; nivelar, aplainar.}{a.pla.nar}{0}
\verb{aplanar}{}{Fig.}{}{}{}{Superar dificuldades; facilitar, simplificar.}{a.pla.nar}{\verboinum{1}}
\verb{aplaudir}{}{}{}{}{v.i.}{Bater palmas como manifestação de aprovação.}{a.plau.dir}{0}
\verb{aplaudir}{}{}{}{}{v.t.}{Aprovar alguém ou algo com aplausos; aclamar, louvar.}{a.plau.dir}{\verboinum{18}}
\verb{aplauso}{}{}{}{}{s.m.}{Ato ou efeito de aplaudir, bater palmas em sinal de aprovação; aclamação.}{a.plau.so}{0}
\verb{aplauso}{}{}{}{}{}{Demonstração pública de aprovação ou louvor por meio de palmas, assobios, palavras etc.}{a.plau.so}{0}
\verb{aplicabilidade}{}{}{}{}{s.f.}{Caráter ou qualidade do que é aplicável; praticabilidade.}{a.pli.ca.bi.li.da.de}{0}
\verb{aplicação}{}{}{"-ões}{}{s.f.}{Ato ou efeito de aplicar.}{a.pli.ca.ção}{0}
\verb{aplicação}{}{}{"-ões}{}{}{Execução de uma tarefa; emprego, utilização.}{a.pli.ca.ção}{0}
\verb{aplicação}{}{}{"-ões}{}{}{Investimento ou emprego de capital com a finalidade de produzir juros.}{a.pli.ca.ção}{0}
\verb{aplicação}{}{}{"-ões}{}{}{Dedicação, afinco, devotamento.}{a.pli.ca.ção}{0}
\verb{aplicação}{}{}{"-ões}{}{}{Adorno ou enfeite sobreposto a uma peça de costura.}{a.pli.ca.ção}{0}
\verb{aplicado}{}{}{}{}{adj.}{Que se aplicou; posto, justaposto.}{a.pli.ca.do}{0}
\verb{aplicado}{}{}{}{}{}{Posto em prática; executado, empregado.}{a.pli.ca.do}{0}
\verb{aplicado}{}{}{}{}{}{Estudioso, diligente, esforçado.}{a.pli.ca.do}{0}
\verb{aplicar}{}{}{}{}{v.t.}{Pôr em cima ou junto; sobrepor, justapor.}{a.pli.car}{0}
\verb{aplicar}{}{}{}{}{}{Colocar em prática; empregar, investir.}{a.pli.car}{0}
\verb{aplicar}{}{}{}{}{}{Infligir, impor.}{a.pli.car}{0}
\verb{aplicar}{}{}{}{}{}{Ministrar, prescrever.}{a.pli.car}{0}
\verb{aplicar}{}{}{}{}{}{Acomodar, adaptar.}{a.pli.car}{0}
\verb{aplicar}{}{}{}{}{v.pron.}{Entregar"-se com afinco; dedicar"-se.}{a.pli.car}{\verboinum{2}}
\verb{aplicativo}{}{}{}{}{adj.}{Que pode ser aplicado; aplicável.}{a.pli.ca.ti.vo}{0}
\verb{aplicativo}{}{Informát.}{}{}{s.m.}{Programa de computador concebido para executar tarefas específicas, como textos, desenhos, diagramação etc.}{a.pli.ca.ti.vo}{0}
\verb{aplicável}{}{}{"-eis}{}{adj.2g.}{Que pode ser aplicado; aplicativo, executável, praticável.}{a.pli.cá.vel}{0}
\verb{aplique}{}{}{}{}{s.m.}{Qualquer enfeite colocado sobre roupas ou sapatos femininos.}{a.pli.que}{0}
\verb{aplique}{}{}{}{}{}{Enfeite ou conjunto de fios de cabelo sobrepostos à cabeça para fins de embelezamento.}{a.pli.que}{0}
\verb{aplique}{}{}{}{}{}{Objeto colocado na parede como ornamento ou foco de iluminação.}{a.pli.que}{0}
%\verb{aplomb}{}{}{}{}{}{0}{aplomb}{0}
\verb{apneia}{é}{Med.}{}{}{s.f.}{Parada súbita e momentânea da respiração.}{ap.nei.a}{0}
\verb{apocalipse}{}{Relig.}{}{}{s.m.}{O último livro do Novo Testamento que contém as revelações feitas ao evangelista São João sobre o destino da humanidade e o fim do mundo.}{a.po.ca.lip.se}{0}
\verb{apocalipse}{}{Fig.}{}{}{}{Grande cataclismo, flagelo terrível.}{a.po.ca.lip.se}{0}
\verb{apocalíptico}{}{}{}{}{adj.}{Relativo ao apocalipse.}{a.po.ca.líp.ti.co}{0}
\verb{apocalíptico}{}{}{}{}{}{Difícil de compreender; obscuro.}{a.po.ca.líp.ti.co}{0}
\verb{apocopado}{}{}{}{}{adj.}{Diz"-se do vocábulo que sofreu apócope.}{a.po.co.pa.do}{0}
\verb{apócope}{}{Gram.}{}{}{s.f.}{Supressão de fonema ou sílaba no final de vocábulo.}{a.pó.co.pe}{0}
\verb{apócrifo}{}{}{}{}{adj.}{Diz"-se de obra ou fato que não é autêntico; falso.}{a.pó.cri.fo}{0}
\verb{apócrifo}{}{Relig.}{}{}{}{Diz"-se de textos bíblicos que não foram incluídos pela igreja no cânone das Escrituras.}{a.pó.cri.fo}{0}
\verb{apodar}{}{}{}{}{v.t.}{Dirigir gracejos; zombar.}{a.po.dar}{0}
\verb{apodar}{}{}{}{}{}{Apelidar, alcunhar, qualificar.}{a.po.dar}{0}
\verb{apodar}{}{}{}{}{}{Comparar de modo depreciativo.}{a.po.dar}{\verboinum{1}}
\verb{ápode}{}{}{}{}{adj.}{Desprovido de pés.}{á.po.de}{0}
\verb{ápode}{}{Zool.}{}{}{}{Relativo aos ápodes, ordem de anfíbios desprovidos de pés.}{á.po.de}{0}
\verb{apoderar"-se}{}{}{}{}{v.pron.}{Apossar"-se de uma coisa pela força; tomar.}{a.po.de.rar"-se}{0}
\verb{apoderar"-se}{}{}{}{}{}{Exercer domínio sobre uma pessoa.}{a.po.de.rar"-se}{\verboinum{1}}
\verb{apodo}{ô}{}{}{}{s.m.}{Zombaria, mofa.}{a.po.do}{0}
\verb{apodo}{ô}{}{}{}{}{Comparação depreciativa.}{a.po.do}{0}
\verb{apodo}{ô}{}{}{}{}{Alcunha, apelido.}{a.po.do}{0}
\verb{apodrecer}{ê}{}{}{}{v.t.}{Tornar podre; estragar.}{a.po.dre.cer}{0}
\verb{apodrecer}{ê}{}{}{}{}{Deteriorar moralmente; corromper.}{a.po.dre.cer}{0}
\verb{apodrecer}{ê}{Fig.}{}{}{v.i.}{Permanecer muito tempo; ficar esquecido.}{a.po.dre.cer}{\verboinum{15}}
\verb{apodrecimento}{}{}{}{}{s.m.}{Ato ou efeito de apodrecer, de tornar podre.}{a.po.dre.ci.men.to}{0}
\verb{apófise}{}{Anat.}{}{}{s.f.}{Saliência ou proeminência natural de um osso.}{a.pó.fi.se}{0}
\verb{apofonia}{}{Gram.}{}{}{s.f.}{Troca de uma vogal que se opera na estrutura fonológica de um elemento vocabular, podendo resultar em mudança de significação ou de função gramatical.}{a.po.fo.ni.a}{0}
\verb{apogeu}{}{Fig.}{}{}{s.m.}{O mais alto grau; o auge.}{a.po.geu}{0}
\verb{apogeu}{}{Astron.}{}{}{}{Posição orbital mais afastada, de um corpo que gravita em torno da Terra.}{a.po.geu}{0}
\verb{apógrafo}{}{}{}{}{s.m.}{Reprodução de um escrito original.}{a.pó.gra.fo}{0}
\verb{apógrafo}{}{}{}{}{}{Instrumento usado para reproduzir desenhos.}{a.pó.gra.fo}{0}
\verb{apoiado}{}{}{}{}{adj.}{Que recebeu aprovação, apoio, favorecimento.}{a.poi.a.do}{0}
\verb{apoiado}{}{}{}{}{}{Sustentado, encostado.}{a.poi.a.do}{0}
\verb{apoiado}{}{}{}{}{}{Fundamentado, baseado.}{a.poi.a.do}{0}
\verb{apoiado}{}{}{}{}{interj.}{Expressão que denota aprovação; muito bem.}{a.poi.a.do}{0}
\verb{apoiar}{}{}{}{}{v.t.}{Dar apoio; aprovar, favorecer.}{a.poi.ar}{0}
\verb{apoiar}{}{}{}{}{}{Sustentar, encostar.}{a.poi.ar}{0}
\verb{apoiar}{}{}{}{}{}{Fundar, fundamentar.}{a.poi.ar}{\verboinum{1}}
\verb{apoio}{ô}{}{}{}{s.m.}{Tudo o que serve para sustentar, segurar, firmar.}{a.poi.o}{0}
\verb{apoio}{ô}{}{}{}{}{Auxílio, amparo, ajuda.}{a.poi.o}{0}
\verb{apoio}{ô}{}{}{}{}{Aprovação, aplauso.}{a.poi.o}{0}
\verb{apoio}{ô}{}{}{}{}{Fundamento, argumento.}{a.poi.o}{0}
\verb{apojadura}{}{}{}{}{s.f.}{Afluxo volumoso de leite na mulher ou fêmea animal que amamenta.}{a.po.ja.du.ra}{0}
\verb{apojar}{}{}{}{}{v.i.}{Intumescer"-se ou encher"-se de leite ou de outro líquido (fêmea que amamenta).}{a.po.jar}{\verboinum{1}}
\verb{apojatura}{}{Mús.}{}{}{s.f.}{Ornamento melódico representado por uma pequena nota que, precedendo a nota real, lhe tira parte do valor e da acentuação.}{a.po.ja.tu.ra}{0}
\verb{apojo}{ô}{Bras.}{}{}{s.m.}{O leite mais consistente extraído da vaca após o mais ralo, que se tira inicialmente. }{a.po.jo}{0}
\verb{apólice}{}{}{}{}{s.f.}{Certificado de obrigação mercantil.}{a.pó.li.ce}{0}
\verb{apólice}{}{Por ext.}{}{}{}{Título de dívida pública.}{a.pó.li.ce}{0}
\verb{apólice}{}{}{}{}{}{Ação de companhia ou de sociedade anônima.}{a.pó.li.ce}{0}
\verb{apolíneo}{}{}{}{}{adj.}{Relativo ou pertencente a Apolo, deus da mitologia grega que personificava o Sol.}{a.po.lí.neo}{0}
\verb{apolíneo}{}{Por ext.}{}{}{}{Belo, formoso.}{a.po.lí.neo}{0}
\verb{apolítico}{}{}{}{}{adj.}{Que não é político, que não apresenta significado político.}{a.po.lí.ti.co}{0}
\verb{apolítico}{}{}{}{}{}{Que não se interessa por política.}{a.po.lí.ti.co}{0}
\verb{apolítico}{}{}{}{}{s.m.}{Indivíduo que não se envolve em política.}{a.po.lí.ti.co}{0}
\verb{apologético}{}{}{}{}{adj.}{Em que há apologia; que defende e justifica; laudatório.}{a.po.lo.gé.ti.co}{0}
\verb{apologia}{}{}{}{}{s.f.}{Discurso ou texto de defesa; louvor, elogio.}{a.po.lo.gi.a}{0}
\verb{apologista}{}{}{}{}{adj.2g.}{Que faz apologia, que faz elogios inflamados.}{a.po.lo.gis.ta}{0}
\verb{apologista}{}{}{}{}{s.2g.}{Indivíduo que faz apologia.}{a.po.lo.gis.ta}{0}
\verb{apólogo}{}{}{}{}{s.m.}{Narrativa curta, de cunho moral, em que são personagens  e têm vida animais e seres inanimados; fábula. }{a.pó.lo.go}{0}
\verb{apoltronar"-se}{}{}{}{}{v.pron.}{Sentar"-se em poltrona.}{a.pol.tro.nar"-se}{\verboinum{1}}
\verb{apoltronar"-se}{}{}{}{}{v.pron.}{Tornar"-se poltrão; acovardar"-se.}{a.pol.tro.nar"-se}{\verboinum{1}}
\verb{apontado}{}{}{}{}{adj.}{Que tem ponta.}{a.pon.ta.do}{0}
\verb{apontado}{}{}{}{}{adj.}{Marcado com pontos ou sinais.}{a.pon.ta.do}{0}
\verb{apontador}{ô}{}{}{}{s.m.}{Objeto usado para fazer a ponta do lápis.}{a.pon.ta.dor}{0}
\verb{apontador}{ô}{}{}{}{s.m.}{Indivíduo que serve de ponto no teatro.}{a.pon.ta.dor}{0}
\verb{apontador}{ô}{}{}{}{}{Livro em que se apontam serviços ou faltas de funcionários.}{a.pon.ta.dor}{0}
\verb{apontamento}{}{}{}{}{s.m.}{Registro escrito do que se leu ou observou; anotação.}{a.pon.ta.men.to}{0}
\verb{apontar}{}{}{}{}{v.t.}{Marcar com ponto; assinalar; mencionar; anotar, registrar.}{a.pon.tar}{\verboinum{1}}
\verb{apontar}{}{}{}{}{v.t.}{Fazer a ponta; aguçar.}{a.pon.tar}{0}
\verb{apontar}{}{}{}{}{}{Indicar, mostrar, fazer pontaria.}{a.pon.tar}{\verboinum{1}}
\verb{apoplético}{}{}{}{}{adj.}{Relativo à apoplexia;  predisposto a ela.}{a.po.plé.ti.co}{0}
\verb{apoplético}{}{Fig.}{}{}{}{Acalorado, entusiasmado; irritado.}{a.po.plé.ti.co}{0}
\verb{apoplexia}{cs}{Med.}{}{}{s.f.}{Suspensão súbita e completa das funções do cérebro, provocada por hemorragia cerebral, trombose ou embolia.}{a.po.ple.xi.a}{0}
\verb{apoquentação}{}{}{"-ões}{}{s.f.}{Ato ou efeito de apoquentar; aborrecimento, chateação.}{a.po.quen.ta.ção}{0}
\verb{apoquentar}{}{}{}{}{v.t.}{Importunar, com poucas ou pequenas coisas, insistentemente, levando à irritação.}{a.po.quen.tar}{0}
\verb{apoquentar}{}{}{}{}{}{Aborrecer, aporrinhar, chatear, azucrinar.}{a.po.quen.tar}{\verboinum{1}}
\verb{apor}{ô}{}{}{}{v.t.}{Pôr junto ou em cima; juntar; justapor; sobrepor.}{a.por}{0}
\verb{apor}{ô}{}{}{}{}{Acrescentar, inserir.}{a.por}{\verboinum{60}}
\verb{aporrinhação}{}{Pop.}{"-ões}{}{s.f.}{Chateação, aborrecimento, apoquentação.}{a.por.ri.nha.ção}{0}
\verb{aporrinhar}{}{}{}{}{v.t.}{Aborrecer insistentemente; chatear, azucrinar, apoquentar.}{a.por.ri.nhar}{\verboinum{1}}
\verb{aportar}{}{}{}{}{v.i.}{Chegar ao porto; ancorar.}{a.por.tar}{0}
\verb{aportar}{}{}{}{}{v.t.}{Encaminhar ou levar a algum lugar.}{a.por.tar}{0}
\verb{aportar}{}{}{}{}{}{Entrar, chegar.}{a.por.tar}{\verboinum{1}}
\verb{aportuguesamento}{}{}{}{}{s.m.}{Ato ou efeito de aportuguesar.}{a.por.tu.gue.sa.men.to}{0}
\verb{aportuguesamento}{}{}{}{}{}{Tornar"-se semelhante à língua ou ao povo português.}{a.por.tu.gue.sa.men.to}{0}
\verb{aportuguesar}{}{}{}{}{v.t.}{Dar ou adquirir hábitos e costumes próprios do povo português.}{a.por.tu.gue.sar}{0}
\verb{aportuguesar}{}{}{}{}{}{Adaptar, fonética e morfologicamente, vocábulos estrangeiros para a língua portuguesa.}{a.por.tu.gue.sar}{\verboinum{1}}
\verb{após}{}{}{}{}{prep.}{Depois.}{a.pós}{0}
\verb{após}{}{}{}{}{}{Atrás de (noção de espaço).}{a.pós}{0}
\verb{após}{}{}{}{}{adv.}{Depois, em seguida, em outro momento.}{a.pós}{0}
\verb{aposentado}{}{}{}{}{adj.}{Que goza de aposentadoria.}{a.po.sen.ta.do}{0}
\verb{aposentado}{}{}{}{}{s.m.}{Indivíduo que obteve aposentadoria.}{a.po.sen.ta.do}{0}
\verb{aposentadoria}{}{}{}{}{s.f.}{Ato ou efeito de aposentar.}{a.po.sen.ta.do.ri.a}{0}
\verb{aposentadoria}{}{}{}{}{}{Hospedagem, albergaria, alojamento.}{a.po.sen.ta.do.ri.a}{0}
\verb{aposentadoria}{}{}{}{}{}{Estado de inatividade de quem se aposentou, ao fim de certo tempo de serviço, ou por invalidez.}{a.po.sen.ta.do.ri.a}{0}
\verb{aposentadoria}{}{}{}{}{}{Quantia recebida mensalmente pelo beneficiário aposentado.}{a.po.sen.ta.do.ri.a}{0}
\verb{aposentar}{}{}{}{}{v.t.}{Conceder aposentadoria.}{a.po.sen.tar}{0}
\verb{aposentar}{}{Fig.}{}{}{}{Pôr de lado; inutilizar.}{a.po.sen.tar}{0}
\verb{aposentar}{}{Desus.}{}{}{}{Hospedar em aposento.}{a.po.sen.tar}{0}
\verb{aposentar}{}{}{}{}{v.pron.}{Atingir a idade ou o tempo de serviço legalmente necessários para receber proventos mensais, sem dar a contrapartida de trabalho.}{a.po.sen.tar}{\verboinum{1}}
\verb{aposento}{}{}{}{}{s.m.}{Local onde se mora; residência.}{a.po.sen.to}{0}
\verb{aposento}{}{}{}{}{}{Compartimento de moradia; especialmente o quarto de dormir.}{a.po.sen.to}{0}
\verb{aposição}{}{}{"-ões}{}{s.f.}{Ato ou efeito de apor; adjunção, colocação. }{a.po.si.ção}{0}
\verb{aposição}{}{Gram.}{"-ões}{}{}{Emprego de um substantivo ou locução substantiva como aposto.}{a.po.si.ção}{0}
\verb{apossar}{}{}{}{}{v.t.}{Dar posse; empossar.}{a.pos.sar}{0}
\verb{apossar}{}{}{}{}{v.pron.}{Tomar posse; apoderar"-se.}{a.pos.sar}{\verboinum{1}}
\verb{aposta}{ó}{}{}{}{s.f.}{Acordo firmado entre pessoas de opinião diferente, no qual aquela que perder ou errar em seu julgamento deverá pagar à outra um valor previamente estipulado.}{a.pos.ta}{0}
\verb{aposta}{ó}{}{}{}{}{Quantia ou coisa apostada.}{a.pos.ta}{0}
\verb{apostador}{ô}{}{}{}{adj.}{Que aposta.}{a.pos.ta.dor}{0}
\verb{apostador}{ô}{}{}{}{s.m.}{Indivíduo que faz aposta.}{a.pos.ta.dor}{0}
\verb{apostar}{}{}{}{}{v.t.}{Fazer uma aposta. }{a.pos.tar}{0}
\verb{apostar}{}{}{}{}{}{Afirmar com convicção.}{a.pos.tar}{0}
\verb{apostar}{}{}{}{}{}{Jogar, arriscar.}{a.pos.tar}{\verboinum{1}}
\verb{apostasia}{}{}{}{}{s.f.}{Abandono, mudança ou renúncia de crença, doutrina, partido ou opinião.}{a.pos.ta.si.a}{0}
\verb{apostasia}{}{Relig.}{}{}{}{Abjuração.}{a.pos.ta.si.a}{0}
\verb{apostasia}{}{}{}{}{}{Deserção.}{a.pos.ta.si.a}{0}
\verb{apóstata}{}{}{}{}{adj.2g.}{Que renunciou a crenças, partidos, ideias.}{a.pós.ta.ta}{0}
\verb{apóstata}{}{}{}{}{s.2g.}{Indivíduo que cometeu apostasia; desertor.}{a.pós.ta.ta}{0}
\verb{apostatar}{}{}{}{}{v.t.}{Abandonar religião ou partido; abjurar.}{a.pos.ta.tar}{\verboinum{1}}
\verb{apostema}{}{Med.}{}{}{s.m.}{Abscesso; tumor; lesão supurada.}{a.pos.te.ma}{0}
\verb{apostemar}{}{}{}{}{v.i.}{Formar abscesso.}{a.pos.te.mar}{\verboinum{1}}
\verb{a posteriori}{ó}{Filos.}{}{}{}{Diz"-se de conhecimento ou ideia resultante de experiência ou que dela dependa.}{\textit{a posteriori}}{0}
\verb{a posteriori}{ó}{}{}{}{loc. adv.}{Posteriormente; depois de. Antôn.: \textit{a priori}.}{\textit{a posteriori}}{0}
\verb{apostila}{}{}{}{}{s.f.}{Síntese ou matérias de aulas, para distribuição, em cópias, para os alunos; apostilha.}{a.pos.ti.la}{0}
\verb{apostila}{}{}{}{}{}{Anotação ou comentário à margem de um texto ou documento.}{a.pos.ti.la}{0}
\verb{apostilar}{}{}{}{}{v.t.}{Fazer apostilas, pôr em apostilas.}{a.pos.ti.lar}{0}
\verb{apostilar}{}{}{}{}{}{Anotar, emendar, fazer notas à margem.}{a.pos.ti.lar}{\verboinum{1}}
\verb{apostilha}{}{}{}{}{s.f.}{Apostila.}{a.pos.ti.lha}{0}
\verb{aposto}{ô}{}{"-s ⟨ó⟩}{"-a ⟨ó⟩}{adj.}{Que se apôs; acrescentado, adjunto.}{a.pos.to}{0}
\verb{aposto}{ô}{Gram.}{"-s ⟨ó⟩}{"-a ⟨ó⟩}{s.m.}{Nome, pronome, expressão ou oração de valor substantivo, que explica, compara, identifica, enumera ou desenvolve a ideia do termo antecedente.}{a.pos.to}{0}
\verb{apostolado}{}{}{}{}{adj.}{Anunciado, doutrinado ou ensinado por apóstolo.}{a.pos.to.la.do}{0}
\verb{apostolado}{}{}{}{}{s.m.}{Atividade, missão de apóstolo; conjunto de apóstolos.}{a.pos.to.la.do}{0}
\verb{apostolado}{}{}{}{}{}{Trabalho religioso de propagação, divulgação ou ensino de uma doutrina, de um credo.}{a.pos.to.la.do}{0}
\verb{apostolar}{}{}{}{}{adj.2g.}{Que é próprio de apóstolo; apostólico.}{a.pos.to.lar}{0}
\verb{apostolar}{}{}{}{}{v.t.}{Pregar o Evangelho, fazer a divulgação de uma doutrina; evangelizar.}{a.pos.to.lar}{\verboinum{1}}
\verb{apostólico}{}{}{}{}{adj.}{Que se refere aos apóstolos, ao seu trabalho ou à Santa Sé; apostolar.}{a.pos.tó.li.co}{0}
\verb{apóstolo}{}{Relig.}{}{}{s.m.}{Cada um dos doze homens escolhidos por Jesus para serem seus discípulos e continuadores de sua missão.}{a.pós.to.lo}{0}
\verb{apóstolo}{}{}{}{}{}{Indivíduo que trabalha na propagação de uma doutrina;  aquele que evangeliza.}{a.pós.to.lo}{0}
\verb{apostrofar}{}{Gram.}{}{}{v.t.}{Dirigir"-se ou referir"-se a alguém, inesperadamente, de maneira direta  e provocativa ou insultuosa.}{a.pos.tro.far}{\verboinum{1}}
\verb{apostrofar}{}{Gram.}{}{}{v.t.}{Pôr apóstrofo.}{a.pos.tro.far}{\verboinum{1}}
\verb{apóstrofe}{}{Gram.}{}{}{s.f.}{Figura de retórica que consiste em o orador interromper seu discurso e se dirigir a alguém ou a alguma coisa, de maneira direta, imprevista e insultuosa; interpelação repentina, direta e veemente.}{a.pós.tro.fe}{0}
\verb{apóstrofo}{}{Gram.}{}{}{s.m.}{Sinal gráfico (`) que serve para indicar a supressão de vogal na palavra. }{a.pós.tro.fo}{0}
\verb{apótema}{}{Mat.}{}{}{s.m.}{Perpendicular que une o centro de um polígono regular ao centro de um de seus lados.}{a.pó.te.ma}{0}
\verb{apoteose}{ó}{}{}{}{s.f.}{Momento mais importante e glorioso de uma apresentação ou de um acontecimento.}{a.po.te.o.se}{0}
\verb{apoteose}{ó}{}{}{}{}{Homenagem pública a alguém de destaque; consagração.}{a.po.te.o.se}{0}
\verb{apoteótico}{}{}{}{}{adj.}{Relativo a apoteose.}{a.po.te.ó.ti.co}{0}
\verb{apoteótico}{}{Fig.}{}{}{}{Deslumbrante, grandioso, elogioso.}{a.po.te.ó.ti.co}{0}
\verb{apoucado}{}{}{}{}{adj.}{Reduzido a pouco; restrito, escasso.}{a.pou.ca.do}{0}
\verb{apoucado}{}{}{}{}{}{Que tem inteligência ou entendimento limitados. }{a.pou.ca.do}{0}
\verb{apoucado}{}{}{}{}{}{Humilhado, rebaixado, depreciado.}{a.pou.ca.do}{0}
\verb{apoucar}{}{}{}{}{v.t.}{Reduzir a pouco ou a poucos; restringir; diminuir.}{a.pou.car}{0}
\verb{apoucar}{}{}{}{}{}{Humilhar, rebaixar.}{a.pou.car}{\verboinum{2}}
\verb{aprazar}{}{}{}{}{v.t.}{Marcar prazo ou tempo para que se faça algo.}{a.pra.zar}{0}
\verb{aprazar}{}{}{}{}{}{Combinar, ajustar.}{a.pra.zar}{\verboinum{1}}
\verb{aprazer}{ê}{}{}{}{v.t.}{Causar prazer; ser agradável;  agradar.}{a.pra.zer}{0}
\verb{aprazer}{ê}{}{}{}{v.pron.}{Contentar"-se; deliciar"-se; deleitar"-se.}{a.pra.zer}{\verboinum{14}}
\verb{aprazível}{}{}{"-eis}{}{adj.2g.}{Que causa prazer; agradável, alegre, deleitoso, encantador.}{a.pra.zí.vel}{0}
\verb{apre}{}{}{}{}{interj.}{Expressão que pode denotar aversão, tédio, ira, desprezo.}{a.pre}{0}
\verb{apreçar}{}{}{}{}{v.t.}{Dar o preço de algo.}{a.pre.çar}{0}
\verb{apreçar}{}{}{}{}{}{Perguntar o preço.}{a.pre.çar}{0}
\verb{apreçar}{}{}{}{}{}{Avaliar; ajustar. }{a.pre.çar}{\verboinum{3}}
\verb{apreciação}{}{}{"-ões}{}{s.f.}{Ato ou efeito de apreciar, analisar, avaliar. }{a.pre.ci.a.ção}{0}
\verb{apreciação}{}{}{"-ões}{}{}{Opinião, conceito, julgamento.}{a.pre.ci.a.ção}{0}
\verb{apreciador}{ô}{}{}{}{adj.}{Que aprecia, que julga.}{a.pre.ci.a.dor}{0}
\verb{apreciador}{ô}{}{}{}{s.m.}{Avaliador, conhecedor.}{a.pre.ci.a.dor}{0}
\verb{apreciar}{}{}{}{}{v.t.}{Dar apreço, valor, merecimento;  considerar, estimar, prezar.}{a.pre.ci.ar}{0}
\verb{apreciar}{}{}{}{}{}{Analisar, julgar, avaliar.}{a.pre.ci.ar}{\verboinum{1}}
\verb{apreciativo}{}{}{}{}{adj.}{Que realiza uma avaliação.}{a.pre.ci.a.ti.vo}{0}
\verb{apreciável}{}{}{"-eis}{}{adj.2g.}{Que pode ser apreciado; digno de apreço, consideração, estima.}{a.pre.ci.á.vel}{0}
\verb{apreciável}{}{}{"-eis}{}{}{Ponderável, notável.}{a.pre.ci.á.vel}{0}
\verb{apreço}{ê}{}{}{}{s.m.}{Estima, consideração que se tem por alguém ou  alguma coisa.}{a.pre.ço}{0}
\verb{apreender}{ê}{}{}{}{v.t.}{Fazer apreensão; apropriar"-se judicialmente de bens ou rendimentos; tomar posse.}{a.pre.en.der}{0}
\verb{apreender}{ê}{}{}{}{}{Assimilar mentalmente; compreender.}{a.pre.en.der}{\verboinum{12}}
\verb{apreensão}{}{}{"-ões}{}{s.f.}{Ato ou efeito de apreender; tomada; prisão.}{a.pre.en.são}{0}
\verb{apreensão}{}{}{"-ões}{}{}{Preocupação, receio, cisma.}{a.pre.en.são}{0}
\verb{apreensão}{}{}{"-ões}{}{}{Compreensão, percepção.}{a.pre.en.são}{0}
\verb{apreensível}{}{}{"-eis}{}{adj.2g.}{Que se pode apreender.}{a.pre.en.sí.vel}{0}
\verb{apreensivo}{}{}{}{}{adj.}{Que apreende; que sente ou em que há apreensão. }{a.pre.en.si.vo}{0}
\verb{apreensivo}{}{}{}{}{}{Preocupado, receoso, cismático.}{a.pre.en.si.vo}{0}
\verb{apreensor}{ô}{}{}{}{adj.}{Diz"-se do que prende, retém.}{a.pre.en.sor}{0}
\verb{apreensor}{ô}{}{}{}{s.m.}{Indivíduo que apreende, prende, retém.}{a.pre.en.sor}{0}
\verb{apreensório}{}{}{}{}{adj.}{Que serve para apreender; apreensor.}{a.pre.en.só.rio}{0}
\verb{apregoar}{}{}{}{}{v.t.}{Fazer uma proclamação em público.}{a.pre.go.ar}{0}
\verb{apregoar}{}{}{}{}{}{Declarar, anunciar em voz alta.}{a.pre.go.ar}{\verboinum{7}}
\verb{aprender}{ê}{}{}{}{v.t.}{Adquirir conhecimento; ficar sabendo.}{a.pren.der}{0}
\verb{aprender}{ê}{}{}{}{}{Reter na memória.}{a.pren.der}{\verboinum{12}}
\verb{aprendiz}{}{}{}{}{s.m.}{Indivíduo que está aprendendo ofício ou arte; principiante, novato.}{a.pren.diz}{0}
\verb{aprendiz}{}{}{}{}{}{Pessoa inexperiente, pouco hábil.}{a.pren.diz}{0}
\verb{aprendizado}{}{}{}{}{s.m.}{Ato ou efeito de aprender; aprendizagem.}{a.pren.di.za.do}{0}
\verb{aprendizado}{}{}{}{}{}{Experiência.}{a.pren.di.za.do}{0}
\verb{aprendizagem}{}{}{"-ens}{}{s.f.}{Ato de aprender um ofício; aprendizado.}{a.pren.di.za.gem}{0}
\verb{aprendizagem}{}{}{"-ens}{}{}{Tempo durante o qual se aprende.}{a.pren.di.za.gem}{0}
%\verb{apresamento}{}{}{}{}{}{0}{a.pre.sa.men.to}{0}
\verb{apresar}{}{}{}{}{v.t.}{Tomar como presa; capturar.}{a.pre.sar}{\verboinum{1}}
\verb{apresentação}{}{}{"-ões}{}{s.f.}{Ato, efeito ou modo de apresentar.}{a.pre.sen.ta.ção}{0}
\verb{apresentação}{}{}{"-ões}{}{}{Aparência externa; aspecto; presença, porte pessoal.}{a.pre.sen.ta.ção}{0}
\verb{apresentação}{}{}{"-ões}{}{}{Prefácio de um livro.}{a.pre.sen.ta.ção}{0}
\verb{apresentação}{}{}{"-ões}{}{}{Mostra; exibição; representação.}{a.pre.sen.ta.ção}{0}
%\verb{apresentado}{}{}{}{}{}{0}{a.pre.sen.ta.do}{0}
\verb{apresentador}{ô}{}{}{}{adj.}{Que apresenta, que mostra.}{a.pre.sen.ta.dor}{0}
\verb{apresentador}{ô}{}{}{}{s.m.}{Indivíduo que apresenta ou conduz programas, shows, debates ou entrevistas, principalmente em televisão e rádio.}{a.pre.sen.ta.dor}{0}
\verb{apresentar}{}{}{}{}{v.t.}{Pôr na presença, à vista.}{a.pre.sen.tar}{0}
\verb{apresentar}{}{}{}{}{}{Exibir, mostrar, alegar, ostentar, expor.}{a.pre.sen.tar}{0}
\verb{apresentar}{}{}{}{}{v.pron.}{Comparecer, aparecer, afigurar"-se, ir à presença de alguém.}{a.pre.sen.tar}{\verboinum{1}}
\verb{apresentável}{}{}{"-eis}{}{adj.2g.}{Que é digno ou capaz de ser apresentado, de ser mostrado.}{a.pre.sen.tá.vel}{0}
\verb{apresentável}{}{}{"-eis}{}{}{Que tem boa aparência.}{a.pre.sen.tá.vel}{0}
\verb{apressado}{}{}{}{}{adj.}{Que tem pressa; acelerado.}{a.pres.sa.do}{0}
\verb{apressado}{}{}{}{}{}{Rápido, precipitado, açodado.}{a.pres.sa.do}{0}
\verb{apressar}{}{}{}{}{v.t.}{Dar pressa; tornar mais rápido, acelerar; antecipar.}{a.pres.sar}{0}
\verb{apressar}{}{}{}{}{}{Instigar, estimular.}{a.pres.sar}{\verboinum{1}}
%\verb{apressurado}{}{}{}{}{}{0}{a.pres.su.ra.do}{0}
\verb{apressurar}{}{}{}{}{v.t.}{Tornar apressado, apressar; acelerar.}{a.pres.su.rar}{0}
\verb{apressurar}{}{}{}{}{}{Aprontar com precipitação.}{a.pres.su.rar}{\verboinum{1}}
\verb{aprestar}{}{}{}{}{v.t.}{Preparar com prontidão;  fazer preparativos. }{a.pres.tar}{0}
\verb{aprestar}{}{}{}{}{}{Aprontar; aparelhar.}{a.pres.tar}{\verboinum{1}}
\verb{aprestos}{ê}{}{}{}{s.m.pl.}{Preparativos, providências.}{a.pres.tos}{0}
\verb{aprestos}{ê}{}{}{}{}{Utensílios, petrechos.}{a.pres.tos}{0}
\verb{aprimoramento}{}{}{}{}{s.m.}{Ato ou efeito de aprimorar; aperfeiçoamento.}{a.pri.mo.ra.men.to}{0}
\verb{aprimorar}{}{}{}{}{v.t.}{Tornar primoroso; aperfeiçoar, refinar.}{a.pri.mo.rar}{\verboinum{1}}
\verb{a priori}{ó}{Filos.}{}{}{}{Diz"-se do conhecimento ou de ideia anterior à experiência ou independente dela.}{\textit{a priori}}{0}
\verb{a priori}{ó}{}{}{}{loc. adv.}{Anteriormente; antes de. Antôn.: \textit{a posteriori}.}{\textit{a priori}}{0}
\verb{apriorismo}{}{Filos.}{}{}{s.m.}{Doutrina que atribui um papel fundamental a conceitos e raciocínios \textit{a priori}.}{a.pri.o.ris.mo}{0}
\verb{apriorismo}{}{}{}{}{}{Aceitação, na ordem do conhecimento, de fatores independentes da experiência.}{a.pri.o.ris.mo}{0}
\verb{aprisco}{}{}{}{}{s.m.}{Curral; lugar onde se recolhe o gado, especialmente as ovelhas.}{a.pris.co}{0}
\verb{aprisco}{}{}{}{}{}{Cabana rústica, que serve de abrigo, albergue. }{a.pris.co}{0}
\verb{aprisionamento}{}{}{}{}{s.m.}{Ato ou efeito de aprisionar, prender. }{a.pri.si.o.na.men.to}{0}
\verb{aprisionar}{}{}{}{}{v.t.}{Fazer prisioneiro; colocar em prisão, encarcerar.}{a.pri.si.o.nar}{0}
\verb{aprisionar}{}{}{}{}{}{Prender.}{a.pri.si.o.nar}{\verboinum{1}}
\verb{aproar}{}{}{}{}{v.t.}{Virar a proa da embarcação para determinado rumo; dirigir.}{a.pro.ar}{0}
\verb{aproar}{}{Por ext.}{}{}{}{Chegar, aportar; ancorar.}{a.pro.ar}{\verboinum{7}}
\verb{aprobativo}{}{}{}{}{adj.}{Que aprova; aprobatório.}{a.pro.ba.ti.vo}{0}
\verb{aprobatório}{}{}{}{}{adj.}{Que aprova; aprobativo.}{a.pro.ba.tó.rio}{0}
\verb{aprofundamento}{}{}{}{}{s.m.}{Ato ou efeito de aprofundar.}{a.pro.fun.da.men.to}{0}
\verb{aprofundar}{}{}{}{}{v.t.}{Tornar mais profundo.}{a.pro.fun.dar}{0}
\verb{aprofundar}{}{}{}{}{}{Examinar ou investigar minuciosamente.}{a.pro.fun.dar}{\verboinum{1}}
\verb{aprontar}{}{}{}{}{v.t.}{Tornar pronto, preparar.}{a.pron.tar}{0}
\verb{aprontar}{}{}{}{}{}{Vestir, arrumar, trajar.}{a.pron.tar}{0}
\verb{aprontar}{}{Pop.}{}{}{}{Fazer o que não deve.}{a.pron.tar}{\verboinum{1}}
\verb{apronto}{}{}{}{}{s.m.}{Ato ou efeito de aprontar; preparo.}{a.pron.to}{0}
\verb{apropositado}{}{}{}{}{adj.}{Que vem a propósito; pertinente, oportuno.}{a.pro.po.si.ta.do}{0}
\verb{apropriação}{}{}{"-ões}{}{s.f.}{Ato ou efeito de apropriar.}{a.pro.pri.a.ção}{0}
\verb{apropriado}{}{}{}{}{adj.}{Próprio para determinado fim; adequado, útil.}{a.pro.pri.a.do}{0}
\verb{apropriado}{}{}{}{}{}{Que se apropriou.}{a.pro.pri.a.do}{0}
\verb{apropriar}{}{}{}{}{v.t.}{Tomar como propriedade; apoderar.}{a.pro.pri.ar}{0}
\verb{apropriar}{}{}{}{}{}{Adequar, adaptar.}{a.pro.pri.ar}{\verboinum{1}}
\verb{aprovação}{}{}{"-ões}{}{s.f.}{Ato ou efeito de aprovar.}{a.pro.va.ção}{0}
\verb{aprovado}{}{}{}{}{adj.}{Que obteve aprovação; julgado bom.}{a.pro.va.do}{0}
\verb{aprovar}{}{}{}{}{v.t.}{Considerar bom, adequado, justo, acertado.}{a.pro.var}{0}
\verb{aprovar}{}{}{}{}{}{Autorizar, sancionar, consentir, ratificar.}{a.pro.var}{\verboinum{1}}
\verb{aproveitador}{ô}{}{}{}{adj.}{Que aproveita.}{a.pro.vei.ta.dor}{0}
\verb{aproveitador}{ô}{Pop.}{}{}{}{Que explora ou tira proveito de uma situação de modo pouco ético.}{a.pro.vei.ta.dor}{0}
\verb{aproveitamento}{}{}{}{}{s.m.}{Ato ou efeito de aproveitar.}{a.pro.vei.ta.men.to}{0}
\verb{aproveitamento}{}{}{}{}{}{Progresso em habilidades físicas ou intelectuais.}{a.pro.vei.ta.men.to}{0}
\verb{aproveitamento}{}{}{}{}{}{Utilização adequada.}{a.pro.vei.ta.men.to}{0}
\verb{aproveitar}{}{}{}{}{v.t.}{Tirar proveito de pessoa, coisa, situação etc.}{a.pro.vei.tar}{0}
\verb{aproveitar}{}{}{}{}{}{Tornar algo proveitoso, útil.}{a.pro.vei.tar}{0}
\verb{aproveitar}{}{}{}{}{}{Não desperdiçar.}{a.pro.vei.tar}{0}
\verb{aproveitar}{}{}{}{}{}{Fazer progresso em determinada habilidade física ou intelectual.}{a.pro.vei.tar}{0}
\verb{aproveitar}{}{}{}{}{v.pron.}{Abusar da ingenuidade de alguém para determinada finalidade, geralmente em benefício próprio.}{a.pro.vei.tar}{\verboinum{1}}
\verb{aproveitável}{}{}{"-eis}{}{adj.2g.}{Que se pode aproveitar, utilizar; utilizável.}{a.pro.vei.tá.vel}{0}
\verb{aprovisionamento}{}{}{}{}{s.m.}{Ato ou efeito de aprovisionar; abastecimento.}{a.pro.vi.si.o.na.men.to}{0}
\verb{aprovisionar}{}{}{}{}{v.t.}{Abastecer de provisões; prover.}{a.pro.vi.si.o.nar}{\verboinum{1}}
\verb{aproximação}{s}{}{"-ões}{}{s.f.}{Ato de aproximar.}{a.pro.xi.ma.ção}{0}
\verb{aproximação}{s}{}{"-ões}{}{}{Resultado aproximado; estimativa.}{a.pro.xi.ma.ção}{0}
\verb{aproximado}{s}{}{}{}{adj.}{Que se aproximou; próximo.}{a.pro.xi.ma.do}{0}
\verb{aproximar}{s}{}{}{}{v.t.}{Pôr ou tornar próximo.}{a.pro.xi.mar}{0}
\verb{aproximar}{s}{}{}{}{}{Pôr duas ou mais pessoas em contato.}{a.pro.xi.mar}{0}
\verb{aproximar}{s}{Fig.}{}{}{}{Estabelecer relações ou comparações entre ideias, fatos, objetos etc.}{a.pro.xi.mar}{\verboinum{1}}
\verb{aproximativo}{s}{}{}{}{adj.}{Que (se) aproxima.}{a.pro.xi.ma.ti.vo}{0}
\verb{aproximativo}{s}{}{}{}{}{Aproximado.}{a.pro.xi.ma.ti.vo}{0}
\verb{aprumado}{}{}{}{}{adj.}{Posto a prumo; vertical.}{a.pru.ma.do}{0}
\verb{aprumado}{}{Fig.}{}{}{}{Diz"-se de indivíduo correto e digno nas relações sociais.}{a.pru.ma.do}{0}
\verb{aprumar}{}{}{}{}{v.t.}{Pôr a prumo, em linha vertical; endireitar.}{a.pru.mar}{0}
\verb{aprumar}{}{Fig.}{}{}{}{Vestir"-se com apuro.}{a.pru.mar}{\verboinum{1}}
\verb{aprumo}{}{}{}{}{s.m.}{Efeito de aprumar.}{a.pru.mo}{0}
\verb{áptero}{}{}{}{}{adj.}{Que não tem asas.}{áp.te.ro}{0}
\verb{aptidão}{}{}{"-ões}{}{s.f.}{Qualidade do que é apto.}{ap.ti.dão}{0}
\verb{aptidão}{}{}{"-ões}{}{}{Capacidade inata ou adquirida para determinada habilidade ou tarefa.}{ap.ti.dão}{0}
\verb{apto}{}{}{}{}{adj.}{Que tem capacidade inata ou adquirida; hábil, capaz.}{ap.to}{0}
\verb{apunhalar}{}{}{}{}{v.t.}{Golpear com punhal ou com objeto semelhante.}{a.pu.nha.lar}{0}
\verb{apunhalar}{}{Fig.}{}{}{}{Ferir moralmente; trair, magoar.}{a.pu.nha.lar}{\verboinum{1}}
\verb{apupada}{}{}{}{}{s.f.}{Vaia.}{a.pu.pa.da}{0}
\verb{apupar}{}{}{}{}{v.t.}{Dirigir vaias, apupos, assobios; escarnecer.}{a.pu.par}{\verboinum{1}}
\verb{apupo}{}{}{}{}{s.m.}{Vaia.}{a.pu.po}{0}
\verb{apuração}{}{}{"-ões}{}{s.f.}{Ato ou efeito de apurar.}{a.pu.ra.ção}{0}
\verb{apuração}{}{}{"-ões}{}{}{Contagem de votos de uma eleição.}{a.pu.ra.ção}{0}
\verb{apuração}{}{}{"-ões}{}{}{Investigação, averiguação.}{a.pu.ra.ção}{0}
\verb{apurado}{}{}{}{}{adj.}{Que se apurou.}{a.pu.ra.do}{0}
\verb{apurado}{}{}{}{}{}{Requintado, elegante, selecionado.}{a.pu.ra.do}{0}
\verb{apurar}{}{}{}{}{v.t.}{Tornar puro; purificar.}{a.pu.rar}{0}
\verb{apurar}{}{}{}{}{}{Investigar, averiguar.}{a.pu.rar}{0}
\verb{apurar}{}{}{}{}{}{Tornar mais fino, sensível (diz"-se de um gosto ou sentido fisiológico); aguçar.}{a.pu.rar}{0}
\verb{apurar}{}{Bras.}{}{}{}{Acelerar a marcha.}{a.pu.rar}{\verboinum{1}}
\verb{apuro}{}{}{}{}{s.m.}{Ato ou efeito de apurar; aperfeiçoamento.}{a.pu.ro}{0}
\verb{apuro}{}{}{}{}{}{Elegância, requinte.}{a.pu.ro}{0}
\verb{aquacultura}{}{}{}{}{s.f.}{Criação de animais ou plantas aquáticas.}{a.qua.cul.tu.ra}{0}
\verb{aqualouco}{ô}{Bras.}{}{}{s.m.}{Acrobata que faz demonstrações cômicas no trampolim de uma piscina, vestido à antiga.}{a.qua.lou.co}{0}
\verb{aquaplanagem}{}{}{"-ens}{}{s.f.}{Pouso sobre a água; amerissagem.}{a.qua.pla.na.gem}{0}
\verb{aquaplanagem}{}{Por ext.}{"-ens}{}{}{Perda de controle de um veículo pela falta de aderência dos pneus ao solo, causada por uma poça ou camada de água na pista.}{a.qua.pla.na.gem}{0}
\verb{aquarela}{é}{}{}{}{s.f.}{Tinta em forma de massa solúvel em água.}{a.qua.re.la}{0}
\verb{aquarela}{é}{}{}{}{}{Técnica de pintura em que se emprega essa tinta.}{a.qua.re.la}{0}
\verb{aquarela}{é}{}{}{}{}{A pintura feita com essa técnica.}{a.qua.re.la}{0}
\verb{aquarelista}{}{}{}{}{s.2g.}{Pintor(a) de aquarelas.}{a.qua.re.lis.ta}{0}
\verb{aquariano}{}{Astrol.}{}{}{s.m.}{Indivíduo que nasceu sob o signo de aquário.}{a.qua.ri.a.no}{0}
\verb{aquariano}{}{Astrol.}{}{}{adj.}{Relativo ou pertencente a esse signo.}{a.qua.ri.a.no}{0}
\verb{aquário}{}{}{}{}{s.m.}{Tanque de água para criar animais ou plantas aquáticas.}{a.quá.rio}{0}
\verb{aquário}{}{Astron.}{}{}{s.m.}{A décima primeira constelação zodiacal.}{a.quá.rio}{0}
\verb{aquário}{}{Astrol.}{}{}{}{O signo do zodíaco referente a essa constelação.}{a.quá.rio}{0}
\verb{aquartelado}{}{}{}{}{adj.}{Alojado ou recolhido em quartel\textsuperscript{1}.}{a.quar.te.la.do}{0}
\verb{aquartelado}{}{}{}{}{adj.}{Dividido em quartéis, em quatro partes.}{a.quar.te.la.do}{0}
\verb{aquartelamento}{}{}{}{}{s.m.}{Ato ou efeito de aquartelar, de colocar em um quartel do Exército, da Marinha ou da Aeronáutica.}{a.quar.te.la.men.to}{0}
\verb{aquartelamento}{}{}{}{}{s.m.}{Divisão em quartéis, em quatro partes.}{a.quar.te.la.men.to}{0}
\verb{aquartelar}{}{}{}{}{v.t.}{Alojar ou recolher em quartel (diz"-se de soldado, tropa etc.).}{a.quar.te.lar}{0}
\verb{aquartelar}{}{Por ext.}{}{}{}{Alojar, hospedar.}{a.quar.te.lar}{\verboinum{1}}
\verb{aquartelar}{}{}{}{}{v.t.}{Dividir em quartéis, em quatro partes (diz"-se de escudo).}{a.quar.te.lar}{\verboinum{1}}
\verb{aquático}{}{}{}{}{adj.}{Relativo a água.}{a.quá.ti.co}{0}
\verb{aquático}{}{}{}{}{}{Que vive dentro da água ou em sua superfície.}{a.quá.ti.co}{0}
\verb{aquavia}{}{}{}{}{s.f.}{Hidrovia.}{a.qua.vi.a}{0}
\verb{aquecedor}{ô}{}{}{}{s.m.}{Aparelho que aquece água ou o ar de ambientes ou veículos.}{a.que.ce.dor}{0}
\verb{aquecer}{ê}{}{}{}{v.t.}{Aumentar a temperatura; tornar (mais) quente.}{a.que.cer}{0}
\verb{aquecer}{ê}{}{}{}{}{Preparar (especialmente a musculatura) para exercer atividades físicas intensas.}{a.que.cer}{0}
\verb{aquecer}{ê}{Fig.}{}{}{}{Dar conforto, afeto, consolo; tornar menos triste ou carente.}{a.que.cer}{0}
\verb{aquecer}{ê}{}{}{}{}{Animar, estimular, entusiasmar, excitar.}{a.que.cer}{\verboinum{15}}
\verb{aquecimento}{}{}{}{}{s.m.}{Ato ou efeito de aquecer.}{a.que.ci.men.to}{0}
\verb{aquecimento}{}{}{}{}{}{Sistema de equipamentos destinado a aquecer água ou o ar de um ambiente ou veículo.}{a.que.ci.men.to}{0}
\verb{aquecimento}{}{}{}{}{}{Série de exercícios físicos destinada a preparar a musculatura de um atleta ou praticante para o desempenho de atividades físicas intensas.}{a.que.ci.men.to}{0}
\verb{aqueduto}{}{}{}{}{s.m.}{Sistema de canalização destinado a conduzir água.}{a.que.du.to}{0}
\verb{aquela}{é}{}{}{}{pron.}{Forma feminina de \textit{aquele}.}{a.que.la}{0}
\verb{àquela}{é}{}{}{}{pron.}{Contração da preposição \textit{a} com o pronome \textit{aquela}.}{à.que.la}{0}
\verb{aquele}{ê}{}{}{}{pron.}{Demonstrativo que indica algo ou alguém que está afastado, no espaço ou no tempo, do falante e do ouvinte.}{a.que.le}{0}
\verb{àquele}{ê}{}{}{}{}{Contração da preposição \textit{a} com o pronome \textit{aquele}.}{à.que.le}{0}
\verb{aquém}{}{}{}{}{adv.}{Do lado de cá (de algo).}{a.quém}{0}
\verb{aquém}{}{}{}{}{}{Abaixo de; menos de.}{a.quém}{0}
%\verb{aquém"-mar}{}{}{}{}{}{0}{a.quém"-mar}{0}
\verb{aquênio}{}{Bot.}{}{}{s.m.}{Fruto pequeno, simples e seco, com a semente presa à parede do pericarpo por um único ponto.}{a.quê.nio}{0}
\verb{aquentar}{}{}{}{}{v.t.}{Tornar quente; aquecer, esquentar.}{a.quen.tar}{0}
\verb{aquentar}{}{Fig.}{}{}{}{Animar, estimular, encorajar.}{a.quen.tar}{\verboinum{1}}
\verb{aqui}{}{}{}{}{adv.}{Em lugar próximo ao falante.}{a.qui}{0}
\verb{aqui}{}{}{}{}{}{Neste lugar; a este lugar; até este lugar.}{a.qui}{0}
\verb{aqui}{}{}{}{}{}{Neste momento; nesta ocasião; neste ponto.}{a.qui}{0}
\verb{aqui}{}{Fig.}{}{}{}{Nesta vida; neste mundo.}{a.qui}{0}
\verb{aquicultura}{}{}{}{}{s.f.}{Atividade de criar animais e plantas aquáticas, e também de cultivar produtos diversos em água com a finalidade de obter gêneros de melhor qualidade.}{a.qui.cul.tu.ra}{0}
\verb{aquiescência}{}{}{}{}{s.f.}{Ato ou efeito de aquiescer; consentimento, anuência.}{a.qui.es.cên.cia}{0}
\verb{aquiescer}{ê}{}{}{}{v.i.}{Consentir, anuir.}{a.qui.es.cer}{0}
\verb{aquiescer}{ê}{}{}{}{v.t.}{Concordar.}{a.qui.es.cer}{\verboinum{15}}
\verb{aquietar}{}{}{}{}{v.t.}{Tornar quieto; tranquilizar.}{a.qui.e.tar}{\verboinum{1}}
\verb{aquífero}{}{}{}{}{adj.}{Que contém ou conduz água.}{a.quí.fe.ro}{0}
\verb{aquilão}{}{}{"-ões}{}{s.m.}{O vento do norte.}{a.qui.lão}{0}
\verb{aquilão}{}{}{}{}{s.m.}{Unguento semelhante ao basilicão.}{a.qui.lão}{0}
%\verb{aquilatado}{}{}{}{}{}{0}{a.qui.la.ta.do}{0}
\verb{aquilatar}{}{}{}{}{v.t.}{Determinar o quilate.}{a.qui.la.tar}{0}
\verb{aquilatar}{}{Por ext.}{}{}{}{Julgar o valor; avaliar.}{a.qui.la.tar}{0}
\verb{aquilatar}{}{Fig.}{}{}{}{Aperfeiçoar, apurar.}{a.qui.la.tar}{0}
\verb{aquilino}{}{}{}{}{adj.}{Relativo a águia.}{a.qui.li.no}{0}
\verb{aquilino}{}{Fig.}{}{}{}{Perspicaz, penetrante.}{a.qui.li.no}{0}
\verb{aquilo}{}{}{}{}{pron.}{Demonstrativo que indica algo que está afastado, no espaço ou no tempo, do falante e do ouvinte.}{a.qui.lo}{0}
\verb{àquilo}{}{}{}{}{}{Contração da preposição \textit{a} com o pronome \textit{aquilo}.}{à.qui.lo}{0}
%\verb{aquinhoado}{}{}{}{}{}{0}{a.qui.nho.a.do}{0}
\verb{aquinhoar}{}{}{}{}{v.t.}{Repartir em quinhões.}{a.qui.nho.ar}{\verboinum{7}}
\verb{aquisição}{}{}{"-ões}{}{s.f.}{Ato ou efeito de adquirir.}{a.qui.si.ção}{0}
\verb{aquisição}{}{Por ext.}{"-ões}{}{}{Aquilo que se adquire.}{a.qui.si.ção}{0}
\verb{aquisitivo}{}{}{}{}{adj.}{Relativo a aquisição.}{a.qui.si.ti.vo}{0}
\verb{aquoso}{ô}{}{"-osos ⟨ó⟩}{"-osa ⟨é⟩}{adj.}{Relativo a água.}{a.quo.so}{0}
\verb{ar}{}{}{}{}{s.m.}{Camada gasosa que envolve a Terra, constituída predominantemente de nitrogênio (78\%) e oxigênio (21\%); atmosfera.}{ar}{0}
\verb{ar}{}{}{}{}{}{Vento, brisa.}{ar}{0}
\verb{ar}{}{}{}{}{}{Aparência, aspecto, semblante, fisionomia.}{ar}{0}
\verb{ar}{}{}{}{}{}{Modo de ser; maneira.}{ar}{0}
\verb{Ar}{}{Quím.}{}{}{}{Símb. do \textit{argônio}. }{Ar}{0}
\verb{ara}{}{Relig.}{}{}{s.f.}{Tipo de altar.}{a.ra}{0}
\verb{ara}{}{Bras.}{}{}{interj.}{Expressão que denota impaciência, espanto, menosprezo.}{a.ra}{0}
\verb{árabe}{}{}{}{}{adj.}{Relativo à Península Arábica e regiões adjacentes.}{á.ra.be}{0}
\verb{árabe}{}{}{}{}{s.m.}{Indivíduo natural ou habitante dessa região.}{á.ra.be}{0}
\verb{árabe"-saudita}{}{}{árabes"-sauditas}{}{adj.2g.}{Relativo à Arábia Saudita. }{á.ra.be"-sau.di.ta}{0}
\verb{árabe"-saudita}{}{}{árabes"-sauditas}{}{s.2g.}{Indivíduo natural ou habitante desse país; saudita.}{á.ra.be"-sau.di.ta}{0}
\verb{arabesco}{ê}{}{}{}{s.m.}{Ornamento de origem árabe com linhas, ramagens, folhas etc.}{a.ra.bes.co}{0}
\verb{arabesco}{ê}{}{}{}{adj.}{Relativo aos árabes.}{a.ra.bes.co}{0}
\verb{arábico}{}{}{}{}{adj.}{Relativo aos árabes.}{a.rá.bi.co}{0}
\verb{arábico}{}{Mat.}{}{}{}{Diz"-se dos algarismos que representam os números, na notação usualmente adotada para o sistema decimal de numeração: 0, 1, 2, 3, 4, 5, 6, 7, 8 e 9.}{a.rá.bi.co}{0}
\verb{arabismo}{}{Gram.}{}{}{s.m.}{Palavra, expressão ou construção próprias da língua árabe.}{a.ra.bis.mo}{0}
\verb{arabismo}{}{}{}{}{}{Movimento cultural ou político em favor da divulgação dos valores, interesses ou posições políticas do mundo árabe.}{a.ra.bis.mo}{0}
\verb{arabista}{}{}{}{}{adj.2g.}{Relativo a arabismo.}{a.ra.bis.ta}{0}
\verb{arabista}{}{}{}{}{s.2g.}{Indivíduo especialista em língua, cultura ou literatura árabe.}{a.ra.bis.ta}{0}
\verb{arabizar}{}{}{}{}{v.t.}{Dar ou adquirir características árabes; tornar árabe.}{a.ra.bi.zar}{\verboinum{1}}
\verb{araçá}{}{Bot.}{}{}{s.m.}{Planta em forma de arbusto ou árvore com tronco malhado e frutos bacáceos e comestíveis; araçazeiro.}{a.ra.çá}{0}
\verb{araçá}{}{}{}{}{}{O fruto dessas plantas.}{a.ra.çá}{0}
\verb{aracajuano}{}{}{}{}{adj.}{Relativo a Aracaju, capital do estado de Sergipe.}{a.ra.ca.ju.a.no}{0}
\verb{aracajuano}{}{}{}{}{s.m.}{Indivíduo natural ou habitante dessa cidade.}{a.ra.ca.ju.a.no}{0}
\verb{aracajuense}{}{}{}{}{adj.2g. e s.2g.}{Aracajuano.}{a.ra.ca.ju.en.se}{0}
\verb{araçazeiro}{ê}{Bot.}{}{}{s.m.}{Planta em forma de arbusto ou árvore com tronco malhado e frutos bacáceos e comestíveis;  araçá.}{a.ra.ça.zei.ro}{0}
\verb{aracnídeo}{}{Zool.}{}{}{s.m.}{Espécime dos aracnídeos, classe de animais artrópodes com quatro pares de patas, um par de palpos e sem antenas, como aranhas, ácaros e escorpiões.}{a.rac.ní.deo}{0}
\verb{aracnoide}{}{}{}{}{adj.2g.}{Semelhante à aranha ou à teia de aranha.}{a.rac.noi.de}{0}
\verb{aracnoide}{}{Anat.}{}{}{s.f.}{Membrana delgada e transparente que envolve o cérebro e a medula espinhal; é uma das três \textit{meninges}.}{a.rac.noi.de}{0}
\verb{arado}{}{}{}{}{s.m.}{Instrumento agrícola para lavrar a terra.}{a.ra.do}{0}
\verb{arado}{}{Por ext.}{}{}{}{A vida agrícola.}{a.ra.do}{0}
\verb{arado}{}{}{}{}{adj.}{Esfomeado, faminto, esfaimado.}{a.ra.do}{0}
\verb{aragem}{}{}{"-ens}{}{s.f.}{Vento brando; brisa.}{a.ra.gem}{0}
\verb{aragem}{}{Fig.}{"-ens}{}{}{Oportunidade; momento favorável.}{a.ra.gem}{0}
\verb{aragonês}{}{}{}{}{adj.}{Relativo a Aragão, região da Espanha.}{a.ra.go.nês}{0}
\verb{aragonês}{}{}{}{}{s.m.}{Indivíduo natural ou habitante dessa região.}{a.ra.go.nês}{0}
\verb{aramado}{}{}{}{}{adj.}{Que é fechado por cercas de arame.}{a.ra.ma.do}{0}
\verb{aramado}{}{}{}{}{s.m.}{Cerca de arame.}{a.ra.ma.do}{0}
\verb{aramaico}{}{}{}{}{adj.}{Relativo aos arameus ou à sua língua.}{a.ra.mai.co}{0}
\verb{aramaico}{}{}{}{}{s.m.}{Língua semítica falada no Aram (antiga Síria) e na Mesopotâmia, ainda hoje falada em algumas localidades dessa região.}{a.ra.mai.co}{0}
\verb{aramaico}{}{}{}{}{}{Indivíduo natural ou habitante do Aram.}{a.ra.mai.co}{0}
\verb{aramar}{}{}{}{}{v.t.}{Circundar com fio ou cerca de arame.}{a.ra.mar}{\verboinum{1}}
\verb{arame}{}{}{}{}{s.m.}{Fio de metal flexível, geralmente latão, ferro ou cobre.}{a.ra.me}{0}
\verb{arame}{}{Desus.}{}{}{}{Liga metálica, geralmente composta de cobre e zinco.}{a.ra.me}{0}
\verb{arameu}{}{}{}{}{s.m.}{Indivíduo dos arameus, povo semita que vivia em Aram (a antiga Síria) e na Mesopotâmia.}{a.ra.meu}{0}
\verb{aramista}{}{}{}{}{s.2g.}{Equilibrista que anda na corda bamba ou arame; funâmbulo.}{a.ra.mis.ta}{0}
\verb{arandela}{é}{}{}{}{s.f.}{Luminária presa à parede, com vela ou lâmpada elétrica.}{a.ran.de.la}{0}
\verb{aranha}{}{Zool.}{}{}{s.f.}{Designação comum a certas espécies de aracnídeos, algumas venenosas, que têm no abdômen glândulas que segregam seda com a qual fazem as teias.}{a.ra.nha}{0}
\verb{aranha}{}{}{}{}{}{Carruagem leve puxada por um cavalo.}{a.ra.nha}{0}
\verb{aranha}{}{Pop.}{}{}{}{Designação dada ao orgão sexual das mulheres.}{a.ra.nha}{0}
\verb{aranha"-caranguejeira}{ê}{Zool.}{aranhas"-caranguejeiras}{}{s.f.}{Aranha grande, com o corpo coberto de pelos urticantes, que não tece teia, e se alimenta de pequenos vertebrados; aranhaçu; caranguejeira.}{a.ra.nha"-ca.ran.gue.jei.ra}{0}
\verb{aranhaçu}{}{Zool.}{}{}{s.f.}{Aranha grande, com o corpo coberto de pelos urticantes, que não tece teia, e se alimenta de pequenos vertebrados; caranguejeira; aranha"-caranguejeira.}{a.ra.nha.çu}{0}
\verb{aranhol}{ó}{}{}{}{s.m.}{Lugar onde há teias de aranha.}{a.ra.nhol}{0}
\verb{aranhol}{ó}{}{}{}{}{Tipo de armadilha, semelhante a teia de aranha, usada para caçar pássaros.}{a.ra.nhol}{0}
\verb{aranzel}{é}{}{"-éis}{}{s.m.}{Discurso prolixo e maçante.}{a.ran.zel}{0}
\verb{aranzel}{é}{}{"-éis}{}{}{Discussão confusa e estéril; gritaria, confusão.}{a.ran.zel}{0}
\verb{araponga}{}{Zool.}{}{}{s.f.}{Ave de várias cores e canto estridente e metálico.}{a.ra.pon.ga}{0}
\verb{araponga}{}{Fig.}{}{}{}{Pessoa de voz estridente ou que fala muito alto.}{a.ra.pon.ga}{0}
\verb{araponga}{}{Bras.}{}{}{}{Agente de serviços de informação; espião.}{a.ra.pon.ga}{0}
\verb{arapuca}{}{}{}{}{s.f.}{Tipo de armadilha, geralmente em forma de pirâmide, para capturar pássaros pequenos.}{a.ra.pu.ca}{0}
\verb{arapuca}{}{Por ext.}{}{}{}{Armadilha, cilada, embuste.}{a.ra.pu.ca}{0}
\verb{araque}{}{}{}{}{s.m.}{Acaso, casualidade, contingência.}{a.ra.que}{0}
\verb{araque}{}{}{}{}{s.m.}{Bebida alcoólica de origem árabe, preparada com anis.}{a.ra.que}{0}
\verb{arar}{}{}{}{}{v.t.}{Sulcar a terra, preparando"-a para o plantio; lavrar.}{a.rar}{\verboinum{1}}
\verb{arar}{}{Bras.}{}{}{v.i.}{Estar com muita fome.}{a.rar}{\verboinum{1}}
\verb{arará}{}{Zool.}{}{}{s.m.}{Designação de várias espécies de insetos alados que saem em revoada no final da tarde para acasalar.}{a.ra.rá}{0}
\verb{arará}{}{}{}{}{}{Nome dado pelos índios urubus a certas pulseiras e adornos feitos de plumagem.}{a.ra.rá}{0}
%\verb{}{}{}{}{}{}{}{}{0}
%\verb{}{}{}{}{}{}{}{}{0}
\verb{araruta}{}{Bot.}{}{}{s.f.}{Planta herbácea de cuja raiz se produz farinha.}{a.ra.ru.ta}{0}
\verb{araruta}{}{}{}{}{}{A farinha obtida dessa planta.}{a.ra.ru.ta}{0}
\verb{araticum}{}{Bot.}{}{}{s.m.}{Designação comum a várias árvores e arbustos do cerrado, que produzem frutos grandes, doces, perfumados e comestíveis.}{a.ra.ti.cum}{0}
\verb{araticum}{}{}{}{}{}{O fruto dessas plantas.}{a.ra.ti.cum}{0}
\verb{aratu}{}{Zool.}{}{}{s.m.}{Designação de várias espécies de caranguejos habitantes do mangue, que sobem em árvores e arbustos.}{a.ra.tu}{0}
%\verb{araucano}{}{}{}{}{}{0}{a.rau.ca.no}{0}
\verb{araucária}{}{Bot.}{}{}{s.f.}{Designação de certas espécies de árvores, algumas com sementes comestíveis e que produzem madeira de qualidade; pinheiro"-do"-paraná.}{a.rau.cá.ria}{0}
\verb{araúna}{}{Zool.}{}{}{s.f.}{Espécie de arara com plumagem azul esverdeado, garganta escura e bico muito grande.}{a.ra.ú.na}{0}
\verb{arauto}{}{}{}{}{s.m.}{Na Idade Média, porta"-voz do governo responsável pelas proclamações solenes, mensagens oficiais e anúncios de guerra e de paz.}{a.rau.to}{0}
\verb{arauto}{}{Por ext.}{}{}{}{Mensageiro, pregoeiro, propugnador.}{a.rau.to}{0}
\verb{arável}{}{}{"-eis}{}{adj.2g.}{Que pode ser arado ou lavrado; cultivável.}{a.rá.vel}{0}
\verb{arbitragem}{}{}{"-ens}{}{s.f.}{Ato ou efeito de arbitrar.}{ar.bi.tra.gem}{0}
\verb{arbitragem}{}{}{"-ens}{}{}{Decisão ou julgamento feito por árbitro ou perito.}{ar.bi.tra.gem}{0}
\verb{arbitragem}{}{}{"-ens}{}{}{Ato de atuar como árbitro em uma competição esportiva.}{ar.bi.tra.gem}{0}
\verb{arbitragem}{}{Por ext.}{"-ens}{}{}{O árbitro ou o conjunto dos árbitros de uma competição esportiva.}{ar.bi.tra.gem}{0}
\verb{arbitral}{}{Jur.}{"-ais}{}{adj.2g.}{Relativo a árbitro.}{ar.bi.tral}{0}
\verb{arbitral}{}{Desus.}{"-ais}{}{}{Sem fundamento; arbitrário.}{ar.bi.tral}{0}
\verb{arbitramento}{}{}{}{}{s.m.}{Ato ou efeito de arbitrar.}{ar.bi.tra.men.to}{0}
\verb{arbitramento}{}{}{}{}{}{Parecer emitido por árbitros ou peritos.}{ar.bi.tra.men.to}{0}
\verb{arbitrar}{}{}{}{}{v.i.}{Atuar como árbitro.}{ar.bi.trar}{0}
\verb{arbitrar}{}{}{}{}{}{Tomar uma decisão.}{ar.bi.trar}{\verboinum{1}}
\verb{arbitrariedade}{}{}{}{}{s.f.}{Qualidade de arbitrário; ação ou comportamento arbitrário.}{ar.bi.tra.ri.e.da.de}{0}
\verb{arbitrário}{}{}{}{}{adj.}{Que não segue lei, regra ou motivação natural.}{ar.bi.trá.rio}{0}
\verb{arbitrário}{}{}{}{}{}{Abusivo, despótico.}{ar.bi.trá.rio}{0}
\verb{arbitrário}{}{}{}{}{}{Em Linguística, mesmo que \textit{imotivado}.}{ar.bi.trá.rio}{0}
\verb{arbítrio}{}{}{}{}{s.m.}{Decisão que depende apenas da vontade.}{ar.bí.trio}{0}
\verb{árbitro}{}{}{}{}{s.m.}{Indivíduo autorizado a dirigir uma competição esportiva e fazer cumprir as regras, com direito de decidir e de aplicar as punições previstas no regulamento.}{ár.bi.tro}{0}
\verb{árbitro}{}{Jur.}{}{}{}{Indivíduo designado por juiz ou acordo entre as partes para solucionar um litígio ou mediar uma negociação.}{ár.bi.tro}{0}
\verb{arbóreo}{}{}{}{}{adj.}{Relativo a árvore.}{ar.bó.re.o}{0}
\verb{arbóreo}{}{}{}{}{}{Semelhante a árvore, geralmente por apresentar ramificações.}{ar.bó.re.o}{0}
\verb{arborescente}{}{}{}{}{adj.2g.}{Diz"-se de uma planta herbácea quando adquire consistência lenhosa.}{ar.bo.res.cen.te}{0}
\verb{arborescente}{}{}{}{}{}{Que tem forma ou porte de árvore.}{ar.bo.res.cen.te}{0}
\verb{arborescer}{ê}{}{}{}{v.i.}{Transformar"-se em árvore; arvorescer.}{ar.bo.res.cer}{0}
\verb{arborescer}{ê}{}{}{}{}{Desenvolver"-se como árvore.}{ar.bo.res.cer}{\verboinum{15}}
\verb{arborícola}{}{}{}{}{adj.2g.}{Diz"-se de animal ou vegetal que vive nas árvores.}{ar.bo.rí.co.la}{0}
\verb{arboricultor}{ô}{}{}{}{s.m.}{Especialista em ou praticante de arboricultura.}{ar.bo.ri.cul.tor}{0}
\verb{arboricultura}{}{}{}{}{s.f.}{Cultivo de árvores, especialmente as ornamentais e frutíferas.}{ar.bo.ri.cul.tu.ra}{0}
\verb{arborização}{}{}{"-ões}{}{s.f.}{Ato ou efeito de arborizar.}{ar.bo.ri.za.ção}{0}
\verb{arborização}{}{}{"-ões}{}{}{O conjunto das árvores cultivadas.}{ar.bo.ri.za.ção}{0}
\verb{arborizado}{}{}{}{}{adj.}{Diz"-se de local ou ambiente em que há muitas árvores.}{ar.bo.ri.za.do}{0}
\verb{arborizar}{}{}{}{}{v.t.}{Plantar árvores.}{ar.bo.ri.zar}{\verboinum{1}}
\verb{arbustivo}{}{}{}{}{adj.}{Relativo a arbusto.}{ar.bus.ti.vo}{0}
\verb{arbusto}{}{}{}{}{s.m.}{Vegetal lenhoso de caule ramificado desde a base e, portanto, sem tronco.}{ar.bus.to}{0}
\verb{arca}{}{}{}{}{s.f.}{Grande caixa, geralmente de madeira, usada para guardar roupas ou objetos.}{ar.ca}{0}
\verb{arca}{}{Desus.}{}{}{}{Cofre, tesouro.}{ar.ca}{0}
\verb{arcabouço}{}{}{}{}{s.m.}{Esqueleto, ossatura.}{ar.ca.bou.ço}{0}
\verb{arcabouço}{}{Fig.}{}{}{}{Esboço, traços gerais.}{ar.ca.bou.ço}{0}
\verb{arcabuz}{}{}{}{}{s.m.}{Antiga arma de fogo de cano curto e largo.}{ar.ca.buz}{0}
\verb{arcada}{}{}{}{}{s.f.}{Conjunto ou sequência de arcos de alvenaria em uma edificação.}{ar.ca.da}{0}
\verb{arcada}{}{Mús.}{}{}{}{Movimento do arco em um instrumento de cordas.}{ar.ca.da}{0}
\verb{árcade}{}{}{}{}{adj.2g.}{Relativo à Arcádia, região do Peloponeso na Grécia antiga.}{ár.ca.de}{0}
\verb{árcade}{}{}{}{}{s.2g.}{Indivíduo natural ou habitante dessa região.}{ár.ca.de}{0}
\verb{árcade}{}{}{}{}{adj.2g.}{Diz"-se do estilo literário dos membros das arcádias.}{ár.ca.de}{0}
\verb{árcade}{}{}{}{}{s.2g.}{Indivíduo que é membro de uma arcádia.}{ár.ca.de}{0}
\verb{arcádia}{}{}{}{}{s.f.}{Tipo de sociedade literária dos séculos \textsc{xvii} e \textsc{xviii} em que se cultivava o classicismo.}{ar.cá.dia}{0}
\verb{arcádico}{}{}{}{}{adj.}{Referente às arcádias.}{ar.cá.di.co}{0}
\verb{arcádico}{}{}{}{}{}{Que tem caráter bucólico.}{ar.cá.di.co}{0}
\verb{arcadismo}{}{}{}{}{s.m.}{Corrente literária representada pelas arcádias, cuja estética idealizava a vida no campo em contato direto com a natureza. [Obs.: \textit{inicial maiúscula}]}{ar.ca.dis.mo}{0}
\verb{arcadismo}{}{}{}{}{}{Estilo ou influência exercida por essa corrente.}{ar.ca.dis.mo}{0}
\verb{arcaico}{}{}{}{}{adj.}{Relativo a épocas remotas; antigo.}{ar.cai.co}{0}
\verb{arcaico}{}{}{}{}{}{Relativo a fase(s) anterior(es) a um período considerado de maturidade de determinada cultura, língua, povo etc.}{ar.cai.co}{0}
\verb{arcaico}{}{Por ext.}{}{}{}{Que deixou de ser usado; obsoleto, antiquado.}{ar.cai.co}{0}
\verb{arcaísmo}{}{}{}{}{s.m.}{Palavra, expressão, construção, acepção ou estilo que não mais é usado no estado atual da língua.}{ar.ca.ís.mo}{0}
\verb{arcaísmo}{}{Por ext.}{}{}{}{Aquilo que é antiquado, que está fora de moda.}{ar.ca.ís.mo}{0}
\verb{arcaizar}{}{}{}{}{v.t.}{Dar ou adquirir características arcaicas; tornar arcaico.}{ar.cai.zar}{0}
\verb{arcaizar}{}{}{}{}{v.i.}{Empregar arcaísmos.}{ar.cai.zar}{\verboinum{1}}
\verb{arcangélico}{}{}{}{}{adj.}{Relativo a arcanjo.}{ar.can.gé.li.co}{0}
\verb{arcanjo}{}{}{}{}{s.m.}{Anjo de ordem superior.}{ar.can.jo}{0}
\verb{arcano}{}{}{}{}{s.m.}{Mistério, segredo.}{ar.ca.no}{0}
\verb{arcano}{}{}{}{}{}{Lugar oculto.}{ar.ca.no}{0}
\verb{arcano}{}{}{}{}{adj.}{Misterioso, secreto, oculto.}{ar.ca.no}{0}
\verb{arção}{}{}{"-ões}{}{s.m.}{Parte saliente da sela.}{ar.ção}{0}
\verb{arcar}{}{}{}{}{v.t.}{Dar forma de arco; curvar, vergar.}{ar.car}{\verboinum{2}}
\verb{arcar}{}{}{}{}{v.t.}{Carregar, aguentar, enfrentar.}{ar.car}{0}
\verb{arcar}{}{}{}{}{v.i.}{Respirar com dificuldade; ofegar.}{ar.car}{\verboinum{2}}
\verb{arcaria}{}{}{}{}{s.f.}{Arcada.}{ar.ca.ri.a}{0}
\verb{arcaz}{}{}{}{}{s.m.}{Arca grande com gavetões, usada para guardar objetos sagrados nas sacristias.}{ar.caz}{0}
%\verb{arce}{}{}{}{}{}{0}{ar.ce}{0}
\verb{arcebispado}{}{}{}{}{s.m.}{Território sob a jurisdição de um arcebispo.}{ar.ce.bis.pa.do}{0}
\verb{arcebispado}{}{}{}{}{}{Período de tempo em que um arcebispo exerce seu cargo.}{ar.ce.bis.pa.do}{0}
\verb{arcebispado}{}{}{}{}{}{Função de arcebispo.}{ar.ce.bis.pa.do}{0}
\verb{arcebispado}{}{Por ext.}{}{}{}{Residência oficial do arcebispo.}{ar.ce.bis.pa.do}{0}
\verb{arcebispo}{}{}{}{}{s.m.}{Bispo investido de maiores poderes, responsável por uma arquidiocese.}{ar.ce.bis.po}{0}
\verb{arcediago}{}{}{}{}{s.m.}{Dignatário eclesiástico que recebe do bispo maiores poderes.}{ar.ce.di.a.go}{0}
\verb{archeiro}{ê}{}{}{}{s.m.}{Guarda do palácio que usava archa como arma.}{ar.chei.ro}{0}
\verb{archote}{ó}{}{}{}{s.m.}{Corda untada de breu que se acende para iluminar.}{ar.cho.te}{0}
\verb{arcipreste}{é}{}{}{}{s.m.}{Pároco mais antigo e idoso, investido de maiores poderes.}{ar.ci.pres.te}{0}
\verb{arco}{}{}{}{}{s.m.}{Arma formada por vara flexível e corda presa às suas extremidades, utilizada para atirar flechas (ou setas).}{ar.co}{0}
\verb{arco}{}{}{}{}{}{Curvatura de abóboda.}{ar.co}{0}
\verb{arco}{}{Geom.}{}{}{}{Segmento de uma curva ou circunferência.}{ar.co}{0}
\verb{arco}{}{Mús.}{}{}{}{Parte avulsa do violino, constituída por vara com fios de crina de cavalo presos à extremidade, utilizada para fazer vibrar as cordas.}{ar.co}{0}
\verb{arco"-da"-velha}{é}{Pop.}{arcos"-da"-velha ⟨é⟩}{}{s.m.}{Coisa extraordinária, espantosa, antiga (diz"-se \textit{coisa do arco"-da"-velha}).}{ar.co"-da"-ve.lha}{0}
\verb{arco"-da"-velha}{é}{}{arcos"-da"-velha ⟨é⟩}{}{}{Arco"-íris.}{ar.co"-da"-ve.lha}{0}
\verb{arco"-íris}{}{}{arco"-íris}{}{s.m.}{Fenômeno óptico que produz, pela decomposição da luz solar, um arco de luz colorido.}{ar.co"-í.ris}{0}
\verb{ar"-condicionado}{}{}{ares"-condicionados}{}{s.m.}{Equipamento que controla a temperatura e a umidade do ar, em ambientes fechados.}{ar"-con.di.ci.o.na.do}{0}
\verb{arconte}{}{}{}{}{s.m.}{Magistrado da antiga Grécia.}{ar.con.te}{0}
\verb{árdego}{}{}{}{}{adj.}{Ardente, impetuoso.}{ár.de.go}{0}
\verb{árdego}{}{}{}{}{}{Que se irrita facilmente; irascível.}{ár.de.go}{0}
\verb{árdego}{}{}{}{}{}{Árduo, trabalhoso.}{ár.de.go}{0}
\verb{ardeídeo}{}{Zool.}{}{}{s.m.}{Espécime dos ardeídos, família de aves com pernas e dedos compridos, pescoço fino, bico longo e pontiagudo e penas impermeabilizadas, como garças e socós.}{ar.de.í.deo}{0}
\verb{ardeídeo}{}{}{}{}{adj.}{Relativo a essa família de aves.}{ar.de.í.deo}{0}
\verb{ardência}{}{}{}{}{s.f.}{Qualidade ou estado daquilo que arde.}{ar.dên.cia}{0}
\verb{ardência}{}{Por ext.}{}{}{}{Sensação do ardor da queimadura ou semelhante a ela.}{ar.dên.cia}{0}
\verb{ardência}{}{Por ext.}{}{}{}{Sensação de sabor picante, ácido, azedo etc.}{ar.dên.cia}{0}
\verb{ardência}{}{Fig.}{}{}{}{Entusiasmo, vivacidade, intensidade de uma sensação.}{ar.dên.cia}{0}
\verb{ardente}{}{}{}{}{adj.2g.}{Que arde, queima.}{ar.den.te}{0}
\verb{ardente}{}{Por ext.}{}{}{}{Que causa sensação gustativa ardente; picante, ácido, azedo.}{ar.den.te}{0}
\verb{ardente}{}{Fig.}{}{}{}{Cheio de entusiasmo, intensidade, paixão, vivacidade.}{ar.den.te}{0}
\verb{ardentia}{}{}{}{}{s.f.}{Calor intenso.}{ar.den.ti.a}{0}
\verb{ardentia}{}{}{}{}{}{Brilho, fosforescência.}{ar.den.ti.a}{0}
\verb{arder}{ê}{}{}{}{v.i.}{Estar em chamas.}{ar.der}{0}
\verb{arder}{ê}{Por ext.}{}{}{}{Provocar sensação tátil de ardor; queimar.}{ar.der}{0}
\verb{arder}{ê}{Por ext.}{}{}{}{Provocar sensação gustativa de ardor; ter sabor picante, ácido, azedo.}{ar.der}{\verboinum{12}\verboirregular{ardo, ardes}}
\verb{ardido}{}{Bras.}{}{}{adj.}{Picante, azedo, ácido.}{ar.di.do}{0}
\verb{ardido}{}{Pop.}{}{}{}{Rançoso.}{ar.di.do}{0}
\verb{ardido}{}{Bras.}{}{}{s.m.}{Irritação na pele; assadura.}{ar.di.do}{0}
\verb{ardido}{}{}{}{}{adj.}{Corajoso, audacioso, valente.}{ar.di.do}{0}
\verb{ardil}{}{}{"-is}{}{s.m.}{Astúcia, sagacidade, artimanha.}{ar.dil}{0}
\verb{ardil}{}{}{"-is}{}{}{Armadilha, armação, cilada.}{ar.dil}{0}
\verb{ardileza}{ê}{}{}{}{s.f.}{Ardil.}{ar.di.le.za}{0}
\verb{ardiloso}{ô}{}{"-osos ⟨ó⟩}{"-osa ⟨ó⟩}{adj.}{Astucioso, esperto, manhoso.}{ar.di.lo.so}{0}
\verb{ardimento}{}{}{}{}{s.m.}{Ardência.}{ar.di.men.to}{0}
\verb{ardimento}{}{}{}{}{s.m.}{Esforço que leva a não recuar; coragem.}{ar.di.men.to}{0}
\verb{ardor}{ô}{}{}{}{s.m.}{Calor intenso; ardência.}{ar.dor}{0}
\verb{ardoroso}{ô}{}{"-osos ⟨ó⟩}{"-osa ⟨ó⟩}{adj.}{Que tem ardor, paixão; entusiasmado.}{ar.do.ro.so}{0}
\verb{ardósia}{}{}{}{}{s.f.}{Rocha de cor cinza ou verde e granulação finíssima, utilizada para revestimento em pisos e paredes.}{ar.dó.sia}{0}
\verb{ardume}{}{Bras.}{}{}{s.m.}{Qualidade do que arde; ardor.}{ar.du.me}{0}
\verb{árduo}{}{}{}{}{adj.}{Íngreme, escarpado (diz"-se de caminhos).}{ár.du.o}{0}
\verb{árduo}{}{Fig.}{}{}{}{Cansativo, fatigante.}{ár.du.o}{0}
\verb{árduo}{}{Fig.}{}{}{}{Que causa dor ou sofrimento; penoso.}{ár.du.o}{0}
\verb{are}{}{}{}{}{s.m.}{Unidade de medida de superfície, equivalente a 100 m\textsuperscript{2}.}{a.re}{0}
\verb{área}{}{}{}{}{s.f.}{Determinada extensão de território ou superfície.}{á.re.a}{0}
\verb{área}{}{}{}{}{}{A medida de uma superfície.}{á.re.a}{0}
\verb{área}{}{}{}{}{}{Campo do conhecimento ou de atividade.}{á.re.a}{0}
\verb{areação}{}{}{"-ões}{}{s.f.}{Ato ou efeito de arear.}{a.re.a.ção}{0}
\verb{areal}{}{}{"-ais}{}{s.m.}{Grande extensão de superfície coberta de areia; areão.}{a.re.al}{0}
\verb{areão}{}{}{"-ões}{}{s.m.}{Areal.}{a.re.ão}{0}
\verb{areão}{}{Bras.}{"-ões}{}{}{Areia de grãos grossos ou misturada com pedras.}{a.re.ão}{0}
\verb{arear}{}{}{}{}{v.t.}{Cobrir com areia ou material semelhante.}{a.re.ar}{0}
\verb{arear}{}{}{}{}{}{Limpar ou polir, esfregando com areia ou substância apropriada.}{a.re.ar}{\verboinum{1}}
\verb{arear}{}{}{}{}{v.i.}{Perder o rumo; desorientar"-se, desnortear"-se.}{a.re.ar}{0}
\verb{arear}{}{}{}{}{}{Perder o juízo; perder a cabeça; pasmar"-se.}{a.re.ar}{\verboinum{1}}
\verb{areento}{}{}{}{}{adj.}{Coberto com areia; arenoso.}{a.re.en.to}{0}
\verb{areia}{ê}{}{}{}{s.f.}{Material granulado formado por partículas resultantes da desagregação de rochas, decorrente de erosão, e geralmente encontradas nas praias, desertos, leitos dos rios.}{a.rei.a}{0}
\verb{areia}{ê}{Por ext.}{}{}{}{A praia.}{a.rei.a}{0}
\verb{areia}{ê}{Por ext.}{}{}{}{Qualquer material em pó.}{a.rei.a}{0}
\verb{areia}{ê}{}{}{}{adj.}{Diz"-se da cor bege semelhante à tonalidade da areia.}{a.rei.a}{0}
\verb{arejamento}{}{}{}{}{s.m.}{Ato ou efeito de arejar, ventilar.}{a.re.ja.men.to}{0}
\verb{arejamento}{}{}{}{}{}{Renovação do ar de um ambiente; ventilação.}{a.re.ja.men.to}{0}
\verb{arejar}{}{}{}{}{v.t.}{Expor ao ar; ventilar.}{a.re.jar}{0}
\verb{arejar}{}{}{}{}{v.i.}{Tomar ar novo; refrescar.}{a.re.jar}{0}
\verb{arejar}{}{Fig.}{}{}{}{Distrair, espairecer.}{a.re.jar}{0}
\verb{arejar}{}{}{}{}{}{Renovar as ideias; liberalizar, esclarecer.}{a.re.jar}{\verboinum{1}}
\verb{arena}{ê}{}{}{}{s.f.}{Espaço fechado, coberto de areia, nos antigos circos romanos, onde se realizavam lutas, jogos, festividades etc. }{a.re.na}{0}
\verb{arena}{ê}{}{}{}{}{Área central de um circo ou teatro onde ocorrem exibições de artistas ou representações de peças; picadeiro, anfiteatro.}{a.re.na}{0}
\verb{arena}{ê}{}{}{}{}{Terreno circular fechado destinado à realização de touradas e outros espetáculos.}{a.re.na}{0}
\verb{arenga}{}{}{}{}{s.f.}{Discurso ou pregação enfadonha, cansativa.}{a.ren.ga}{0}
\verb{arenga}{}{}{}{}{}{Discussão, disputa, altercação.}{a.ren.ga}{0}
\verb{arenga}{}{}{}{}{}{Intriga, mexerico.}{a.ren.ga}{0}
\verb{arengar}{}{}{}{}{v.t.}{Discursar ou pregar de forma enfadonha ou cansativa.}{a.ren.gar}{0}
\verb{arengar}{}{}{}{}{v.i.}{Discutir, polemizar.}{a.ren.gar}{0}
\verb{arengar}{}{}{}{}{}{Fazer intrigas; mexericar.}{a.ren.gar}{\verboinum{5}}
\verb{arengueiro}{ê}{}{}{}{adj.}{Que discursa ou prega de forma enfadonha, cansativa.}{a.ren.guei.ro}{0}
\verb{arengueiro}{ê}{}{}{}{}{Que disputa, polemiza.}{a.ren.guei.ro}{0}
\verb{arengueiro}{ê}{}{}{}{}{Mexeriqueiro, intrigante.}{a.ren.guei.ro}{0}
\verb{arenito}{}{Geol.}{}{}{s.m.}{Rocha sedimentar composta de grãos de quartzo, feldspato ou calcário, unidos por argila ou por outro cimento qualquer.}{a.re.ni.to}{0}
\verb{arenoso}{ô}{}{"-osos ⟨ó⟩}{"-osa ⟨ó⟩}{adj.}{Que está coberto de areia.}{a.re.no.so}{0}
\verb{arenoso}{ô}{}{"-osos ⟨ó⟩}{"-osa ⟨ó⟩}{}{Misturado com areia.}{a.re.no.so}{0}
\verb{arenoso}{ô}{}{"-osos ⟨ó⟩}{"-osa ⟨ó⟩}{}{Semelhante a areia no aspecto ou na cor.}{a.re.no.so}{0}
\verb{arenque}{}{Zool.}{}{}{s.m.}{Peixe migratório, de aproximadamente 30 cm de comprimento, comum no Pacífico e no Atlântico norte, que forma cardume e possui grande valor comercial.}{a.ren.que}{0}
\verb{aréola}{}{}{}{}{s.f.}{Pequena área.}{a.ré.o.la}{0}
\verb{aréola}{}{Anat.}{}{}{}{Círculo, de coloração escura, que rodeia o mamilo.}{a.ré.o.la}{0}
\verb{aréola}{}{Astron.}{}{}{}{Circunferência brilhante entre o halo e o disco do Sol ou da Lua.}{a.ré.o.la}{0}
\verb{areômetro}{}{Fís.}{}{}{s.m.}{Instrumento usado para medição da massa ou da densidade de líquidos ou sólidos, que consiste em um flutuador, o qual, imerso na substância, indica a grandeza que se quer medir.}{a.re.ô.me.tro}{0}
\verb{areópago}{}{}{}{}{s.m.}{Reunião ou assembleia de sábios, intelectuais, políticos etc.}{a.re.ó.pa.go}{0}
\verb{areópago}{}{}{}{}{}{Em Atenas, na Grécia antiga, tribunal supremo em que se discutiam questões políticas e religiosas.}{a.re.ó.pa.go}{0}
\verb{ares}{}{}{}{}{s.m.pl.}{Condições climáticas; atmosfera.}{a.res}{0}
\verb{ares}{}{}{}{}{}{Aparência, aspecto.}{a.res}{0}
\verb{aresta}{é}{}{}{}{s.f.}{Ângulo formado pela intersecção de dois planos ou duas superfícies.}{a.res.ta}{0}
\verb{aresta}{é}{}{}{}{}{Qualquer saliência natural de aparência angulosa; quina, esquina.}{a.res.ta}{0}
\verb{aresta}{é}{}{}{}{}{Coisa insignificante, sem importância; bagatela.}{a.res.ta}{0}
\verb{aresta}{é}{Geom.}{}{}{}{Intersecção de duas faces de um poliedro.}{a.res.ta}{0}
\verb{aresto}{ê}{Jur.}{}{}{s.m.}{Acórdão.}{a.res.to}{0}
\verb{arfante}{}{}{}{}{adj.2g.}{Que respira com dificuldade; ofegante.}{ar.fan.te}{0}
\verb{arfar}{}{}{}{}{v.i.}{Respirar com dificuldade; ofegar, arquejar.}{ar.far}{0}
\verb{arfar}{}{}{}{}{}{Balançar, oscilar.}{ar.far}{\verboinum{1}}
\verb{argamassa}{}{}{}{}{s.f.}{Mistura aglutinante, preparada com areia, água, cal ou cimento, utilizada no assentamento ou no revestimento de alvenarias.}{ar.ga.mas.sa}{0}
\verb{argeliano}{}{}{}{}{adj. e s.m.  }{Argelino.}{ar.ge.li.a.no}{0}
\verb{argelino}{}{}{}{}{adj.}{Relativo à Argélia.}{ar.ge.li.no}{0}
\verb{argelino}{}{}{}{}{s.m.}{Indivíduo natural ou habitante desse país; argeliano.}{ar.ge.li.no}{0}
\verb{argentar}{}{}{}{}{v.t.}{Banhar ou cobrir de prata; pratear.}{ar.gen.tar}{\verboinum{1}}
\verb{argentaria}{}{}{}{}{s.f.}{Objetos de prata utilizados na mesa, como talheres, baixelas, guarnições etc.}{ar.gen.ta.ri.a}{0}
\verb{argentário}{}{}{}{}{s.m.}{Móvel utilizado para guardar objetos de prata.}{ar.gen.tá.rio}{0}
\verb{argentário}{}{}{}{}{}{Homem muito rico, milionário.}{ar.gen.tá.rio}{0}
\verb{argênteo}{}{}{}{}{adj.}{Que é feito de prata.}{ar.gên.teo}{0}
\verb{argênteo}{}{}{}{}{}{Que tem a cor da prata; prateado.}{ar.gên.teo}{0}
\verb{argênteo}{}{}{}{}{}{Que tem o timbre fino como o da prata.}{ar.gên.teo}{0}
\verb{argentífero}{}{}{}{}{adj.}{Diz"-se do mineral que contém prata.}{ar.gen.tí.fe.ro}{0}
\verb{argentino}{}{}{}{}{adj.}{Relativo à Argentina.}{ar.gen.ti.no}{0}
\verb{argentino}{}{}{}{}{s.m.}{Indivíduo natural ou habitante desse país.}{ar.gen.ti.no}{0}
\verb{argentino}{}{Desus.}{}{}{adj.}{Argênteo.}{ar.gen.ti.no}{0}
\verb{argila}{}{}{}{}{s.f.}{Rocha sedimentar, de tonalidades variadas que vão do branco ao avermelhado, e que, ao receber água, pode ser modelada e utilizada na cerâmica; barro.}{ar.gi.la}{0}
\verb{argiloso}{ô}{}{"-osos ⟨ó⟩}{"-osa ⟨ó⟩}{adj.}{Que contém argila ou que apresenta suas características; barrento.}{ar.gi.lo.so}{0}
\verb{argola}{ó}{}{}{}{s.f.}{Anel de metal, madeira ou plástico com que se prende ou se puxa alguma coisa; aro.}{ar.go.la}{0}
\verb{argola}{ó}{}{}{}{}{Qualquer objeto de forma circular e vazio no centro.}{ar.go.la}{0}
\verb{argola}{ó}{}{}{}{}{Aparelho de ginástica esportiva que consiste em dois aros de aço fixos às extremidades de cordas suspensas.}{ar.go.la}{0}
\verb{argonauta}{}{Mit.}{}{}{s.2g.}{Na antiga Grécia, designação dada a cada um dos lendários navegadores da nau Argo, que buscavam o velocino de ouro.}{ar.go.nau.ta}{0}
\verb{argonauta}{}{Por ext.}{}{}{}{Navegador ousado, explorador dos mares.}{ar.go.nau.ta}{0}
\verb{argônio}{}{Quím.}{}{}{s.m.}{Elemento químico da família dos gases nobres, incolor e inodoro, encontrado na atmosfera terrestre e utilizado em válvulas eletrônicas, tubos de descarga, em lâmpadas especiais etc. \elemento{18}{39.948}{Ar}.}{ar.gô.nio}{0}
\verb{argúcia}{}{}{}{}{s.f.}{Senso aguçado de observação; raciocínio sutil; perspicácia.}{ar.gú.cia}{0}
\verb{argúcia}{}{}{}{}{}{Dito espirituoso, ardiloso.}{ar.gú.cia}{0}
\verb{argueiro}{ê}{}{}{}{s.m.}{Partícula minúscula; cisco, grânulo.}{ar.guei.ro}{0}
\verb{argueiro}{ê}{Fig.}{}{}{}{Coisa mínima, insignificante.}{ar.guei.ro}{0}
\verb{arguição}{}{}{"-ões}{}{s.f.}{Ato ou efeito de arguir, questionar; interrogatório.}{ar.gui.ção}{0}
\verb{arguição}{}{}{"-ões}{}{}{Argumentação, alegação.}{ar.gui.ção}{0}
\verb{arguição}{}{}{"-ões}{}{}{Exame ou teste oral.}{ar.gui.ção}{0}
%\verb{arguidor}{}{}{}{}{}{0}{ar.gu.i.dor}{0}
\verb{arguir}{}{}{}{}{v.t.}{Questionar ou interrogar, examinando provas ou fatos.}{ar.guir}{0}
\verb{arguir}{}{}{}{}{}{Oferecer argumentos; discutir.}{ar.guir}{0}
\verb{arguir}{}{Jur.}{}{}{}{Acusar, julgar, condenar.}{ar.guir}{0}
\verb{arguir}{}{}{}{}{}{Contestar, impugnar, opor"-se.}{ar.guir}{\verboinum{28}}
\verb{argumentação}{}{}{"-ões}{}{s.f.}{Ato ou efeito de argumentar; discussão, disputa, controvérsia.}{ar.gu.men.ta.ção}{0}
\verb{argumentação}{}{}{"-ões}{}{}{Conjunto de fatos, ideias, provas usados para comprovar uma tese.}{ar.gu.men.ta.ção}{0}
\verb{argumentar}{}{}{}{}{v.t.}{Apresentar razões, fatos, provas que comprovem uma afirmação.}{ar.gu.men.tar}{0}
\verb{argumentar}{}{}{}{}{}{Entrar em debate; discutir, objetar.}{ar.gu.men.tar}{0}
\verb{argumentar}{}{}{}{}{}{Tirar conclusões; deduzir, inferir.}{ar.gu.men.tar}{0}
\verb{argumentar}{}{}{}{}{}{Apresentar como pretexto ou desculpa; alegar.}{ar.gu.men.tar}{\verboinum{1}}
\verb{argumento}{}{}{}{}{s.m.}{Raciocínio que leva à indução ou à dedução de algo que se quer provar.}{ar.gu.men.to}{0}
\verb{argumento}{}{}{}{}{}{Recurso ou prova usada para negar ou afirmar um fato.}{ar.gu.men.to}{0}
\verb{argumento}{}{}{}{}{}{Sumário, resumo, temática.}{ar.gu.men.to}{0}
\verb{arguto}{}{}{}{}{adj.}{Que tem espírito sagaz,  penetrante; perspicaz, engenhoso.}{ar.gu.to}{0}
\verb{arguto}{}{}{}{}{}{De som afinado; canoro.}{ar.gu.to}{0}
\verb{ária}{}{Mús.}{}{}{s.f.}{Peça destinada ao canto de uma só voz; canção, melodia.}{á.ria}{0}
\verb{ária}{}{}{}{}{adj.}{Relativo aos árias; ariano.}{á.ria}{0}
\verb{ária}{}{}{}{}{s.2g.}{Indivíduo pertencente a antigo povo indo"-europeu, originário da Índia.}{á.ria}{0}
\verb{arianismo}{}{}{}{}{s.m.}{Doutrina de Ário, famoso padre cristão de Alexandria (280--336), segundo a qual era Cristo uma criatura de natureza intermediária entre a divindade e a humanidade.  }{a.ri.a.nis.mo}{0}
\verb{ariano}{}{}{}{}{s.m.}{Seguidor do arianismo.}{a.ri.a.no}{0}
\verb{ariano}{}{}{}{}{adj.}{Entre os teóricos do racismo alemão, dizia"-se dos europeus de raça supostamente pura, descendentes dos árias.   }{a.ri.a.no}{0}
\verb{ariano}{}{}{}{}{s.m.}{A língua ancestral de toda a família indo"-europeia. }{a.ri.a.no}{0}
\verb{ariano}{}{Astrol.}{}{}{s.m.}{Indivíduo que nasceu sob o signo de áries.}{a.ri.a.no}{0}
\verb{ariano}{}{Astrol.}{}{}{adj.}{Relativo ou pertencente a esse signo.}{a.ri.a.no}{0}
\verb{aridez}{ê}{}{}{}{s.f.}{Estado ou característica do que é árido, seco.}{a.ri.dez}{0}
\verb{aridez}{ê}{}{}{}{}{Esterilidade, infertilidade, infecundidade.}{a.ri.dez}{0}
\verb{aridez}{ê}{Fig.}{}{}{}{Aspereza, insensibilidade, rudeza.}{a.ri.dez}{0}
\verb{árido}{}{}{}{}{adj.}{Que não apresenta umidade; seco.}{á.ri.do}{0}
\verb{árido}{}{}{}{}{}{Que não produz; estéril, infecundo.}{á.ri.do}{0}
\verb{árido}{}{Fig.}{}{}{}{Insensível, rude, duro.}{á.ri.do}{0}
\verb{áries}{}{Astron.}{}{}{s.m.}{Primeira constelação zodiacal.}{á.ri.es}{0}
\verb{áries}{}{Astrol.}{}{}{}{O signo do zodíaco referente a essa constelação.}{á.ri.es}{0}
\verb{aríete}{}{}{}{}{s.m.}{Bomba hidráulica utilizada para elevar água acima do nível da queda.}{a.rí.e.te}{0}
\verb{aríete}{}{}{}{}{}{Antiga máquina de guerra com que se arrombavam portas e muralhas.}{a.rí.e.te}{0}
\verb{arigó}{}{}{}{}{adj.}{Que trabalha na construção de estradas e nas usinas de açúcar.}{a.ri.gó}{0}
\verb{arigó}{}{Pop.}{}{}{}{Caipira, matuto, simplório.}{a.ri.gó}{0}
\verb{ariranha}{}{Zool.}{}{}{s.f.}{Mamífero carnívoro, habitante dos grandes rios, semelhante à lontra, cuja pele, macia e cinzenta, é muito procurada pelos caçadores. }{a.ri.ra.nha}{0}
\verb{arisco}{}{}{}{}{adj.}{Que possui comportamento arredio, esquivo, desconfiado.}{a.ris.co}{0}
\verb{arisco}{}{}{}{}{}{Que possui muita areia; arenoso.}{a.ris.co}{0}
\verb{aristocracia}{}{}{}{}{s.f.}{Grupo ou classe de pessoas que detém o poder político por meio de concessões ou herança; nobreza, fidalguia.}{a.ris.to.cra.ci.a}{0}
\verb{aristocracia}{}{}{}{}{}{Forma de governo que concentra o poder em um número reduzido de pessoas.}{a.ris.to.cra.ci.a}{0}
\verb{aristocrata}{}{}{}{}{adj.2g.}{Que pertence à aristocracia; nobre, fidalgo.}{a.ris.to.cra.ta}{0}
\verb{aristocrata}{}{}{}{}{}{Que apresenta maneiras distintas, sofisticadas, requintadas.}{a.ris.to.cra.ta}{0}
\verb{aristocrático}{}{}{}{}{adj.}{Relativo à aristocracia; nobre, fidalgo.}{a.ris.to.crá.ti.co}{0}
\verb{aristocrático}{}{}{}{}{}{Distinto, requintado, sofisticado.}{a.ris.to.crá.ti.co}{0}
\verb{aristotélico}{}{}{}{}{adj.}{Relativo a Aristóteles ou a sua doutrina.}{a.ris.to.té.li.co}{0}
\verb{aristotélico}{}{}{}{}{s.m.}{Indivíduo partidário do aristotelismo.}{a.ris.to.té.li.co}{0}
\verb{aristotelismo}{}{Filos.}{}{}{s.m.}{Doutrina do filósofo grego Aristóteles, século \textsc{iii} a.C., considerada a primeira formalização da lógica.}{a.ris.to.te.lis.mo}{0}
\verb{aritmética}{}{}{}{}{s.f.}{Ramo da matemática que estuda os números, suas propriedades e operações.}{a.rit.mé.ti.ca}{0}
\verb{aritmética}{}{}{}{}{}{Qualquer cálculo, conta, operação.}{a.rit.mé.ti.ca}{0}
\verb{aritmética}{}{}{}{}{}{Livro ou compêndio que contém os princípios da aritmética.}{a.rit.mé.ti.ca}{0}
\verb{arlequim}{}{}{"-ins}{}{s.m.}{Personagem de comédia que usava traje colorido e divertia o público.}{ar.le.quim}{0}
\verb{arlequim}{}{}{"-ins}{}{}{Fantasia carnavalesca inspirada no traje desse personagem.}{ar.le.quim}{0}
\verb{arlequim}{}{Fig.}{"-ins}{}{}{Indivíduo irresponsável.}{ar.le.quim}{0}
\verb{arlequim}{}{Zool.}{"-ins}{}{}{Inseto de cores preta e vermelha, antenas longas e espinhos no tórax.}{ar.le.quim}{0}
\verb{arlequim}{}{Bras.}{"-ins}{}{}{Personagem do bumba"-meu"-boi.}{ar.le.quim}{0}
\verb{arma}{}{}{}{}{s.f.}{Utensílio, mecanismo ou aparelho utilizado como meio de ataque ou defesa em uma luta.}{ar.ma}{0}
\verb{arma}{}{Por ext.}{}{}{}{Qualquer coisa que sirva a esses fins.}{ar.ma}{0}
\verb{arma}{}{Fig.}{}{}{}{Qualquer argumento que proporcione vantagem ou sirva de defesa em uma discussão.}{ar.ma}{0}
\verb{arma}{}{Fig.}{}{}{}{Recurso, meio de favorecer ou prejudicar alguém.}{ar.ma}{0}
\verb{arma}{}{}{}{}{}{Cada um dos tipos de tropas que constituem um exército, como infantaria, artilharia, cavalaria etc.}{ar.ma}{0}
\verb{armação}{}{}{"-ões}{}{s.f.}{Ato ou efeito de armar.}{ar.ma.ção}{0}
\verb{armação}{}{}{"-ões}{}{}{Conjunto de peças que dão sustentação, reforço ou união às diversas partes de um todo.}{ar.ma.ção}{0}
\verb{armação}{}{}{"-ões}{}{}{Conjunto dos armários, prateleiras, vitrinas e balcões de uma loja.}{ar.ma.ção}{0}
\verb{armação}{}{Bras.}{"-ões}{}{}{Situação montada artificialmente para prejudicar ou proteger alguém.}{ar.ma.ção}{0}
\verb{armada}{}{}{}{}{s.f.}{Conjunto das forças navais de um país; frota.}{ar.ma.da}{0}
\verb{armadilha}{}{}{}{}{s.f.}{Dispositivo ou artifício para capturar animais.}{ar.ma.di.lha}{0}
\verb{armadilha}{}{Fig.}{}{}{}{Artifício para enganar ou ludibriar alguém; embuste, cilada.}{ar.ma.di.lha}{0}
\verb{armado}{}{}{}{}{adj.}{Munido de arma(s).}{ar.ma.do}{0}
\verb{armado}{}{Fig.}{}{}{}{Preparado ou disposto para enfrentar determinado problema.}{ar.ma.do}{0}
\verb{armador}{ô}{}{}{}{adj.}{Que arma; armeiro.}{ar.ma.dor}{0}
\verb{armador}{ô}{}{}{}{s.m.}{Pessoa que prepara a igreja com enfeites para festas.}{ar.ma.dor}{0}
\verb{armador}{ô}{Bras.}{}{}{}{Gancho em que se prende a rede de dormir.}{ar.ma.dor}{0}
\verb{armador}{ô}{Esport.}{}{}{}{Jogador que articula e prepara jogadas.}{ar.ma.dor}{0}
\verb{armadura}{}{}{}{}{s.f.}{Conjunto das peças protetoras do corpo dos antigos guerreiros, como elmo, cota de malha, couraça etc.}{ar.ma.du.ra}{0}
\verb{armadura}{}{Por ext.}{}{}{}{Tudo aquilo que serve para sustentar ou reforçar uma obra; armação.}{ar.ma.du.ra}{0}
\verb{armamentismo}{}{}{}{}{s.m.}{Doutrina que defende o aumento do material bélico de um país ou conjunto de países.}{ar.ma.men.tis.mo}{0}
\verb{armamentista}{}{}{}{}{adj.2g.}{Relativo a armamentismo.}{ar.ma.men.tis.ta}{0}
\verb{armamentista}{}{}{}{}{s.2g.}{Indivíduo adepto do armamentismo.}{ar.ma.men.tis.ta}{0}
\verb{armamento}{}{}{}{}{s.m.}{Ato ou efeito de armar.}{ar.ma.men.to}{0}
\verb{armamento}{}{}{}{}{}{Conjunto das armas ou equipamentos de combate de um indivíduo ou de um país.}{ar.ma.men.to}{0}
\verb{armar}{}{}{}{}{v.t.}{Munir de armas.}{ar.mar}{0}
\verb{armar}{}{}{}{}{}{Preparar um mecanismo ou aparelho para funcionar.}{ar.mar}{0}
\verb{armar}{}{}{}{}{}{Montar ou dispor as peças de um todo de maneira adequada (diz"-se geralmente de brinquedos).}{ar.mar}{0}
\verb{armar}{}{Bras.}{}{}{}{Preparar uma armação; tramar.}{ar.mar}{0}
\verb{armar}{}{}{}{}{v.i.}{Preparar armadilha.}{ar.mar}{\verboinum{1}}
\verb{armaria}{}{}{}{}{s.f.}{Depósito de armas; arsenal.}{ar.ma.ri.a}{0}
\verb{armaria}{}{}{}{}{}{Conjunto de armas; armamento.}{ar.ma.ri.a}{0}
\verb{armaria}{}{}{}{}{}{Arte ou ciência que estuda a origem, a composição e a significação dos brasões; heráldica.}{ar.ma.ri.a}{0}
\verb{armarinho}{}{Bras.}{}{}{s.m.}{Loja em que se vendem materiais para costura, tecidos e miudezas afins.}{ar.ma.ri.nho}{0}
\verb{armário}{}{}{}{}{s.m.}{Móvel de madeira ou metal com prateleiras ou gavetas para guardar roupas, louças, remédios ou quaisquer objetos.}{ar.má.rio}{0}
\verb{armário}{}{Bras.}{}{}{}{Homem muito forte e corpulento.}{ar.má.rio}{0}
\verb{armas}{}{}{}{}{s.f.pl.}{As forças armadas de um país.}{ar.mas}{0}
\verb{armas}{}{}{}{}{}{A profissão ou carreira militar.}{ar.mas}{0}
\verb{armas}{}{}{}{}{}{Insígnias de brasão.}{ar.mas}{0}
\verb{armazém}{}{}{"-éns}{}{s.m.}{Depósito de mercadorias, munição etc.}{ar.ma.zém}{0}
\verb{armazém}{}{}{"-éns}{}{}{Estabelecimento comercial onde se vendem gêneros alimentícios e utensílios diversos; mercearia.}{ar.ma.zém}{0}
\verb{armazenado}{}{}{}{}{adj.}{Que se armazenou; que foi guardado em armazém.}{ar.ma.ze.na.do}{0}
\verb{armazenagem}{}{}{"-ens}{}{s.f.}{Ato ou efeito de armazenar.}{ar.ma.ze.na.gem}{0}
\verb{armazenagem}{}{}{"-ens}{}{}{Quantia paga pela permanência de mercadorias em alfândegas, cais, postos ferroviários etc.}{ar.ma.ze.na.gem}{0}
\verb{armazenamento}{}{}{}{}{s.m.}{Ato ou efeito de armazenar.}{ar.ma.ze.na.men.to}{0}
\verb{armazenar}{}{}{}{}{v.t.}{Pôr em armazém; guardar.}{ar.ma.ze.nar}{0}
\verb{armazenar}{}{}{}{}{}{Manter ou conservar em depósito.}{ar.ma.ze.nar}{0}
\verb{armazenar}{}{}{}{}{}{Acumular, juntar.}{ar.ma.ze.nar}{0}
\verb{armazenar}{}{Informát.}{}{}{}{Introduzir informações em dispositivo de memória permanente ou temporária para que seja possível recuperá"-las posteriormente.}{ar.ma.ze.nar}{\verboinum{1}}
\verb{armeiro}{ê}{}{}{}{s.m.}{Indivíduo que fabrica, vende ou faz manutenção de armas.}{ar.mei.ro}{0}
\verb{armeiro}{ê}{}{}{}{}{Indivíduo responsável pela guarda e distribuição das armas de uma tropa ou quartel.}{ar.mei.ro}{0}
\verb{armênio}{}{}{}{}{adj.}{Relativo à Armênia.}{ar.mê.nio}{0}
\verb{armênio}{}{}{}{}{s.m.}{Indivíduo natural ou habitante desse país.}{ar.mê.nio}{0}
\verb{armênio}{}{}{}{}{}{Língua falada na Armênia, Geórgia e Turquia.}{ar.mê.nio}{0}
\verb{armento}{}{}{}{}{s.m.}{Rebanho, especialmente de gado vacum ou cavalos.}{ar.men.to}{0}
\verb{armila}{}{}{}{}{s.f.}{Conjunto de anéis ou molduras que cercam a base de certos tipos de coluna.}{ar.mi.la}{0}
\verb{armilar}{}{}{}{}{adj.2g.}{Formado de armilas.}{ar.mi.lar}{0}
\verb{arminho}{}{Zool.}{}{}{s.m.}{Mamífero das regiões polares de pelagem macia avermelhada no verão e branca no inverno.}{ar.mi.nho}{0}
\verb{arminho}{}{Por ext.}{}{}{}{A pele ou o pelo desse animal.}{ar.mi.nho}{0}
\verb{arminho}{}{Fig.}{}{}{}{Aquilo que é muito branco; brancura.}{ar.mi.nho}{0}
\verb{armistício}{}{}{}{}{s.m.}{Acordo que suspende os combates entre as partes envolvidas numa guerra; trégua.}{ar.mis.tí.cio}{0}
\verb{armoriado}{}{}{}{}{adj.}{Ornado com armas ou brasão, pintados, esculpidos ou aplicados.}{ar.mo.ri.a.do}{0}
\verb{armorial}{}{}{"-ais}{}{adj.2g.}{Relativo a heráldica, armaria ou brasões.}{ar.mo.ri.al}{0}
\verb{armorial}{}{}{"-ais}{}{s.m.}{Livro em que se registram brasões.}{ar.mo.ri.al}{0}
\verb{arnês}{}{}{}{}{s.m.}{Antiga armadura completa de um guerreiro.}{ar.nês}{0}
\verb{arnês}{}{Por ext.}{}{}{}{Arreios de cavalo.}{ar.nês}{0}
\verb{arnês}{}{Fig.}{}{}{}{Aquilo que protege; abrigo, amparo.}{ar.nês}{0}
\verb{arnica}{}{Bot.}{}{}{s.f.}{Erva alpestre cultivada como ornamental e com propriedades medicinais.}{ar.ni.ca}{0}
\verb{arnica}{}{}{}{}{}{Tintura obtida dessa planta, de uso local em contusões e edemas e de uso interno em medicamentos homeopáticos.}{ar.ni.ca}{0}
\verb{aro}{}{}{}{}{s.m.}{Forma circular; anel.}{a.ro}{0}
\verb{aro}{}{Por ext.}{}{}{}{Armação de óculos ou luneta.}{a.ro}{0}
\verb{aro}{}{}{}{}{}{A medida interna ou externa da roda de veículos como automóvel, bicicleta etc.}{a.ro}{0}
\verb{aroeira}{ê}{Bot.}{}{}{s.f.}{Árvore de flores brancas ou amareladas e folhas penadas, cuja casca tem propriedades medicinais.}{a.ro.ei.ra}{0}
\verb{aroma}{}{}{}{}{s.m.}{Odor agradável exalado por certas substâncias; perfume; fragrância.}{a.ro.ma}{0}
\verb{aroma}{}{}{}{}{}{Essência odorífera.}{a.ro.ma}{0}
\verb{aromático}{}{}{}{}{adj.}{Relativo a aroma.}{a.ro.má.ti.co}{0}
\verb{aromático}{}{}{}{}{}{Que tem odor agradável.}{a.ro.má.ti.co}{0}
\verb{aromatizado}{}{}{}{}{adj.}{Que ganhou um odor agradável.}{a.ro.ma.ti.za.do}{0}
\verb{aromatizante}{}{}{}{}{adj.2g.}{Que aromatiza, perfuma ou condimenta.}{a.ro.ma.ti.zan.te}{0}
\verb{aromatizar}{}{}{}{}{v.t.}{Impregnar de aroma; perfumar, condimentar.}{a.ro.ma.ti.zar}{\verboinum{1}}
\verb{arpão}{}{}{"-ões}{}{s.m.}{Instrumento de ferro em forma de flecha e preso a um cabo usado para capturar peixes de grande porte.}{ar.pão}{0}
\verb{arpear}{}{}{}{}{v.t.}{Arpoar.}{ar.pe.ar}{\verboinum{4}}
\verb{arpejar}{}{Mús.}{}{}{v.i.}{Executar arpejos.}{ar.pe.jar}{\verboinum{1}}
\verb{arpejo}{ê}{Mús.}{}{}{s.m.}{Execução sucessiva das notas de um acorde.}{ar.pe.jo}{0}
\verb{arpéu}{}{}{}{}{s.m.}{Pequeno arpão; gancho.}{ar.péu}{0}
\verb{arpoador}{ô}{}{}{}{adj.}{Que arpoa.}{ar.po.a.dor}{0}
\verb{arpoar}{}{}{}{}{v.t.}{Cravar o arpão; fisgar com arpão.}{ar.po.ar}{0}
\verb{arpoar}{}{}{}{}{v.i.}{Arremessar o arpão; arpear.}{ar.po.ar}{0}
\verb{arpoar}{}{Fig.}{}{}{v.t.}{Agarrar ou segurar de forma segura; impedir que escape.}{ar.po.ar}{\verboinum{7}}
\verb{arqueação}{}{}{"-ões}{}{s.f.}{Ato ou efeito de arquear; arqueadura.}{ar.que.a.ção}{0}
\verb{arqueação}{}{}{"-ões}{}{}{A curvatura de um arco.}{ar.que.a.ção}{0}
\verb{arqueação}{}{}{"-ões}{}{}{Medida de capacidade de recipientes arqueados.}{ar.que.a.ção}{0}
\verb{arqueado}{}{}{}{}{adj.}{Que tem ou adquiriu forma de arco; dobrado.}{ar.que.a.do}{0}
\verb{arqueadura}{}{}{}{}{s.f.}{Curvatura em arco; arqueação; arqueamento.}{ar.que.a.du.ra}{0}
\verb{arquear}{}{}{}{}{v.t.}{Curvar em forma de arco; dobrar.}{ar.que.ar}{0}
\verb{arquear}{}{}{}{}{}{Medir a capacidade de recipientes arqueados ou cilíndricos.}{ar.que.ar}{\verboinum{4}}
\verb{arqueiro}{ê}{}{}{}{s.m.}{Indivíduo que fabrica ou vende arcos.}{ar.quei.ro}{0}
\verb{arqueiro}{ê}{}{}{}{}{Guerreiro que tem o arco como arma.}{ar.quei.ro}{0}
\verb{arqueiro}{ê}{Esport.}{}{}{}{Jogador que defende o gol; goleiro.}{ar.quei.ro}{0}
\verb{arqueiro}{ê}{}{}{}{s.m.}{Indivíduo que fabrica ou vende arcas.}{ar.quei.ro}{0}
\verb{arquejante}{}{}{}{}{adj.2g.}{Que arqueja; ofegante.}{ar.que.jan.te}{0}
\verb{arquejar}{}{}{}{}{v.t.}{Respirar com dificuldade; ofegar.}{ar.que.jar}{\verboinum{1}}
\verb{arquejo}{ê}{}{}{}{s.m.}{Ato de arquejar; respiração difícil.}{ar.que.jo}{0}
\verb{arqueologia}{}{}{}{}{s.f.}{Ciência que estuda as culturas e costumes dos povos antigos através de seus vestígios, como documentos, artefatos, fósseis, construções etc.}{ar.que.o.lo.gi.a}{0}
\verb{arqueológico}{}{}{}{}{adj.}{Referente à arqueologia.}{ar.que.o.ló.gi.co}{0}
\verb{arqueológico}{}{Por ext.}{}{}{}{Muito antigo.}{ar.que.o.ló.gi.co}{0}
\verb{arqueólogo}{}{}{}{}{s.m.}{Indivíduo que se dedica à arqueologia.}{ar.que.ó.lo.go}{0}
\verb{arquétipo}{}{}{}{}{s.m.}{Modelo, padrão.}{ar.qué.ti.po}{0}
\verb{arquétipo}{}{}{}{}{}{Na psicologia de Jung, imagem psíquica do inconsciente coletivo, a qual é compartilhada por toda a espécie humana.}{ar.qué.ti.po}{0}
\verb{arquibancada}{}{}{}{}{s.f.}{Série de assentos dispostos em níveis, semelhante a uma escada, para acomodar a audiência em um estádio, auditório, circo, anfiteatro etc.}{ar.qui.ban.ca.da}{0}
\verb{arquidiocese}{é}{Relig.}{}{}{s.f.}{Diocese que tem jurisdição sobre outras e que se encontra sob o comando de um arcebispo; arcebispado.}{ar.qui.di.o.ce.se}{0}
\verb{arquiducado}{}{}{}{}{s.m.}{Território sob o domínio de um arquiduque.}{ar.qui.du.ca.do}{0}
\verb{arquiducado}{}{}{}{}{}{Dignidade de arquiduque.}{ar.qui.du.ca.do}{0}
\verb{arquiduque}{}{}{}{arquiduquesa ⟨ê⟩}{s.m.}{Título nobiliárquico imediatamente superior ao de duque.}{ar.qui.du.que}{0}
\verb{arquiduque}{}{}{}{arquiduquesa ⟨ê⟩}{}{Título honorífico usado principalmente pelos príncipes austríacos.}{ar.qui.du.que}{0}
\verb{arquiduquesa}{ê}{}{}{}{s.f.}{Esposa de arquiduque.}{ar.qui.du.que.sa}{0}
\verb{arquiepiscopado}{}{}{}{}{s.m.}{Arcebispado.}{ar.qui.e.pis.co.pa.do}{0}
\verb{arquiepiscopal}{}{}{"-ais}{}{adj.2g.}{Relativo a arcebispo.}{ar.qui.e.pis.co.pal}{0}
\verb{arqui"-inimigo}{}{}{}{}{s.m.}{Inimigo no mais alto grau; arquirrival.}{ar.qui"-i.ni.mi.go}{0}
\verb{arquimilionário}{}{}{}{}{adj.}{Que é extremamente rico; multimilionário, bilionário.}{ar.qui.mi.li.o.ná.rio}{0}
\verb{arquipélago}{}{}{}{}{s.m.}{Conjunto de ilhas dispostas em grupo numa determinada região do oceano.}{ar.qui.pé.la.go}{0}
\verb{arquitetar}{}{}{}{}{v.t.}{Conceber ou elaborar projeto ou elemento arquitetônico.}{ar.qui.te.tar}{0}
\verb{arquitetar}{}{}{}{}{}{Construir, edificar.}{ar.qui.te.tar}{0}
\verb{arquitetar}{}{Por ext.}{}{}{}{Planejar detalhadamente a realização de algo.}{ar.qui.te.tar}{0}
\verb{arquitetar}{}{Fig.}{}{}{}{Tramar, urdir.}{ar.qui.te.tar}{\verboinum{1}}
\verb{arquiteto}{é}{}{}{}{s.m.}{Indivíduo que se dedica à arquitetura.}{ar.qui.te.to}{0}
\verb{arquiteto}{é}{}{}{}{}{Profissional legalmente habilitado para o exercício da arquitetura.}{ar.qui.te.to}{0}
\verb{arquiteto}{é}{}{}{}{}{Indivíduo responsável pela idealização ou realização de qualquer feito relevante.}{ar.qui.te.to}{0}
\verb{arquitetônico}{}{}{}{}{adj.}{Relativo a arquitetura.}{ar.qui.te.tô.ni.co}{0}
\verb{arquitetura}{}{}{}{}{s.f.}{Técnica e arte de criar espaços e ambientes em edificações, de acordo com critérios funcionais e estéticos.}{ar.qui.te.tu.ra}{0}
\verb{arquitetura}{}{}{}{}{}{Conjunto das edificações executadas em um dado local e determinado contexto histórico.}{ar.qui.te.tu.ra}{0}
\verb{arquitetura}{}{}{}{}{}{Disposição dos vários elementos de uma edificação ou espaço urbano.}{ar.qui.te.tu.ra}{0}
\verb{arquitetura}{}{Fig.}{}{}{}{Plano, projeto, planejamento.}{ar.qui.te.tu.ra}{0}
\verb{arquitrave}{}{}{}{}{s.f.}{Viga horizontal que descansa sobre o capitel das colunas ou pilares.}{ar.qui.tra.ve}{0}
\verb{arquivar}{}{}{}{}{v.t.}{Guardar em arquivo.}{ar.qui.var}{0}
\verb{arquivar}{}{}{}{}{}{Interromper o andamento jurídico de (diz"-se de inquérito ou processo).}{ar.qui.var}{0}
\verb{arquivar}{}{Fig.}{}{}{}{Deixar de lado; esquecer.}{ar.qui.var}{\verboinum{1}}
\verb{arquivista}{}{}{}{}{s.2g.}{Indivíduo responsável por um arquivo.}{ar.qui.vis.ta}{0}
\verb{arquivo}{}{}{}{}{s.m.}{Conjunto de documentos produzidos durante as atividades de uma instituição e destinados a serem conservados para um possível uso futuro.}{ar.qui.vo}{0}
\verb{arquivo}{}{Por ext.}{}{}{}{Lugar onde se recolhem e mantêm esses documentos.}{ar.qui.vo}{0}
\verb{arquivo}{}{}{}{}{}{Móvel de escritório, geralmente de metal e com gavetas, para guardar documentos.}{ar.qui.vo}{0}
\verb{arquivo}{}{Informát.}{}{}{}{Conjunto organizado de informações digitalizadas que pode ser gravado em um dispositivo de armazenamento.}{ar.qui.vo}{0}
\verb{arrabalde}{}{}{}{}{s.m.}{Área que fica em volta dos limites de uma cidade ou povoação; arredor, cercania.}{ar.ra.bal.de}{0}
\verb{arrabalde}{}{Por ext.}{}{}{}{Lugar muito afastado do centro de uma cidade ou povoação; subúrbio.}{ar.ra.bal.de}{0}
\verb{arraia}{}{Zool.}{}{}{s.f.}{Peixe de corpo achatado e nadadeiras em forma de asas; raia.}{ar.rai.a}{0}
\verb{arraia}{}{}{}{}{s.f.}{A camada social mais baixa da sociedade; plebe, ralé, arraia"-miúda.}{ar.rai.a}{0}
\verb{arraial}{}{}{"-ais}{}{s.m.}{Lugarejo de caráter provisório; acampamento.}{ar.rai.al}{0}
\verb{arraial}{}{}{"-ais}{}{}{Pequena aldeia ou povoação; lugarejo.}{ar.rai.al}{0}
\verb{arraial}{}{}{"-ais}{}{}{Local onde são realizadas festas populares.}{ar.rai.al}{0}
\verb{arraia"-miúda}{}{}{arraias"-miúdas}{}{s.f.}{Arraia.}{ar.rai.a"-mi.ú.da}{0}
\verb{arraigado}{}{}{}{}{adj.}{Que se arraigou; radicado, enraizado.}{ar.rai.ga.do}{0}
\verb{arraigado}{}{Fig.}{}{}{}{Obstinado, teimoso, pertinaz.}{ar.rai.ga.do}{0}
\verb{arraigar}{}{}{}{}{v.t.}{Fixar pela raiz; enraizar, germinar.}{ar.rai.gar}{0}
\verb{arraigar}{}{Fig.}{}{}{}{Firmar de maneira definitiva; assentar.}{ar.rai.gar}{0}
\verb{arraigar}{}{}{}{}{v.i.}{Lançar ou criar raízes.}{ar.rai.gar}{0}
\verb{arraigar}{}{}{}{}{v.pron.}{Estabelecer"-se, fixar"-se, fixar moradia.}{ar.rai.gar}{\verboinum{5}}
\verb{arrais}{}{}{}{}{s.m.}{Comandante de navio.}{ar.rais}{0}
\verb{arrancada}{}{}{}{}{s.f.}{Ato ou efeito de arrancar.}{ar.ran.ca.da}{0}
\verb{arrancada}{}{}{}{}{}{Partida ou movimento inesperado ou violento.}{ar.ran.ca.da}{0}
\verb{arrancada}{}{}{}{}{}{Início do funcionamento de um motor ou do movimento de um veículo.}{ar.ran.ca.da}{0}
\verb{arrancar}{}{}{}{}{v.t.}{Tirar, extrair pelo uso da força.}{ar.ran.car}{0}
\verb{arrancar}{}{}{}{}{}{Desprender da terra; desenraizar (diz"-se de vegetal).}{ar.ran.car}{0}
\verb{arrancar}{}{}{}{}{}{Obter com dificuldade.}{ar.ran.car}{0}
\verb{arrancar}{}{}{}{}{}{Pôr"-se em movimento de maneira violenta.}{ar.ran.car}{\verboinum{2}}
\verb{arranca"-rabo}{}{}{arranca"-rabos}{}{s.m.}{Conflito entre indivíduos; desavença, discussão, briga.}{ar.ran.ca"-ra.bo.}{0}
\verb{arranchar}{}{}{}{}{v.t.}{Reunir em rancho.}{ar.ran.char}{0}
\verb{arranchar}{}{}{}{}{}{Dar pousada.}{ar.ran.char}{0}
\verb{arranchar}{}{}{}{}{v.i.}{Tomar o rancho.}{ar.ran.char}{0}
\verb{arranchar}{}{}{}{}{}{Estabelecer"-se provisoriamente em rancho.}{ar.ran.char}{\verboinum{1}}
\verb{arranco}{}{}{}{}{s.m.}{Movimento inesperado e violento.}{ar.ran.co}{0}
\verb{arranha"-céu}{}{}{arranha"-céus}{}{s.m.}{Edifício muito alto.}{ar.ra.nha"-céu}{0}
\verb{arranhadura}{}{}{}{}{s.f.}{Arranhão.}{ar.ra.nha.du.ra}{0}
\verb{arranhão}{}{}{"-ões}{}{s.m.}{Ferida leve e superficial.}{ar.ra.nhão}{0}
\verb{arranhão}{}{}{"-ões}{}{}{Risca ou estria que prejudica o aspecto de uma superfície polida.}{ar.ra.nhão}{0}
\verb{arranhar}{}{}{}{}{v.t.}{Produzir ferimento superficial na pele; ferir.}{ar.ra.nhar}{0}
\verb{arranhar}{}{}{}{}{}{Produzir risco ou ranhura em uma superfície qualquer.}{ar.ra.nhar}{0}
\verb{arranhar}{}{Fig.}{}{}{}{Ferir moralmente; ofender.}{ar.ra.nhar}{0}
\verb{arranhar}{}{Fig.}{}{}{}{Conhecer de maneira precária uma língua, uma matéria, a técnica de um instrumento musical.}{ar.ra.nhar}{0}
\verb{arranhar}{}{}{}{}{v.i.}{Ter a propriedade de produzir arranhadura.}{ar.ra.nhar}{\verboinum{1}}
\verb{arranjado}{}{}{}{}{adj.}{Que se arranjou.}{ar.ran.ja.do}{0}
\verb{arranjado}{}{Bras.}{}{}{}{Que tem situação financeira satisfatória ou suficiente para manter um padrão mínimo de vida.}{ar.ran.ja.do}{0}
\verb{arranjar}{}{}{}{}{v.t.}{Dispor convenientemente; pôr em ordem; arrumar.}{ar.ran.jar}{0}
\verb{arranjar}{}{}{}{}{}{Resolver amigavelmente; conciliar.}{ar.ran.jar}{0}
\verb{arranjar}{}{}{}{}{}{Consertar, reparar.}{ar.ran.jar}{0}
\verb{arranjar}{}{}{}{}{}{Colocar enfeites; adornar.}{ar.ran.jar}{0}
\verb{arranjar}{}{}{}{}{}{Obter, conseguir.}{ar.ran.jar}{0}
\verb{arranjar}{}{Mús.}{}{}{}{Fazer arranjo.}{ar.ran.jar}{0}
\verb{arranjar}{}{}{}{}{v.pron.}{Alcançar uma situação boa ou satisfatória.}{ar.ran.jar}{0}
\verb{arranjar}{}{}{}{}{}{Conseguir namorado(a), cônjuge, amante etc.}{ar.ran.jar}{\verboinum{1}}
\verb{arranjo}{}{}{}{}{s.m.}{Ato ou efeito de arranjar.}{ar.ran.jo}{0}
\verb{arranjo}{}{}{}{}{}{Disposição harmoniosa ou estética.}{ar.ran.jo}{0}
\verb{arranjo}{}{Mat.}{}{}{}{Agrupamento ordenado de um conjunto finito de elementos.}{ar.ran.jo}{0}
\verb{arranjo}{}{Mús.}{}{}{}{Redistribuição ou reelaboração da harmonia, melodia ou ritmo de uma música, geralmente para instrumentos diferentes daqueles para os quais a peça foi originalmente escrita.}{ar.ran.jo}{0}
\verb{arranjo}{}{Bras.}{}{}{}{Conchavo, conluio.}{ar.ran.jo}{0}
\verb{arranque}{}{}{}{}{s.m.}{Ato ou efeito de arrancar.}{ar.ran.que}{0}
\verb{arranque}{}{}{}{}{}{Ato ou efeito de se pôr em funcionamento um motor ou máquina.}{ar.ran.que}{0}
\verb{arras}{}{}{}{}{s.f.pl.}{Quantia paga para garantir o cumprimento de um contrato; sinal.}{ar.ras}{0}
\verb{arras}{}{Por ext.}{}{}{}{Provas, demonstrações, evidências.}{ar.ras}{0}
\verb{arrasado}{}{}{}{}{adj.}{Que se arrasou.}{ar.ra.sa.do}{0}
\verb{arrasado}{}{}{}{}{}{Devastado, assolado, destruído.}{ar.ra.sa.do}{0}
\verb{arrasado}{}{}{}{}{}{Muito deprimido, física ou moralmente; esgotado, prostrado.}{ar.ra.sa.do}{0}
\verb{arrasado}{}{}{}{}{}{Humilhado, vexado.}{ar.ra.sa.do}{0}
\verb{arrasado}{}{}{}{}{}{Sem nenhum dinheiro; falido.}{ar.ra.sa.do}{0}
\verb{arrasador}{ô}{}{}{}{adj.}{Que arrasa.}{ar.ra.sa.dor}{0}
\verb{arrasar}{}{}{}{}{v.t.}{Tornar raso; aplainar, nivelar.}{ar.ra.sar}{0}
\verb{arrasar}{}{}{}{}{}{Destruir, devastar, derrubar, demolir, arruinar.}{ar.ra.sar}{0}
\verb{arrasar}{}{}{}{}{}{Humilhar, abater, vexar.}{ar.ra.sar}{0}
\verb{arrasar}{}{}{}{}{}{Abater física ou moralmente; esgotar, prostrar.}{ar.ra.sar}{\verboinum{1}}
\verb{arrastado}{}{}{}{}{adj.}{Que se arrasta.}{ar.ras.ta.do}{0}
\verb{arrastado}{}{}{}{}{}{Levado contra a vontade, obrigado; compelido.}{ar.ras.ta.do}{0}
\verb{arrastado}{}{}{}{}{}{Lento, demorado, moroso.}{ar.ras.ta.do}{0}
\verb{arrastão}{}{}{"-ões}{}{s.m.}{Ato ou efeito de arrastar com violência.}{ar.ras.tão}{0}
\verb{arrastão}{}{}{"-ões}{}{}{Tipo de rede de pesca que se arrasta pelo fundo do mar.}{ar.ras.tão}{0}
\verb{arrastão}{}{Bras.}{"-ões}{}{}{Ato de recolher do mar essa rede de pesca.}{ar.ras.tão}{0}
\verb{arrastão}{}{Bras.}{"-ões}{}{}{Assalto realizado por um bando que aborda as pessoas ou estabelecimentos comerciais que encontra pelo caminho.}{ar.ras.tão}{0}
\verb{arrasta"-pé}{}{Bras.}{arrasta"-pés}{}{s.m.}{Baile popular; forró.}{ar.ras.ta"-pé}{0}
\verb{arrastar}{}{}{}{}{v.t.}{Puxar ou conduzir sem afastar do chão.}{ar.ras.tar}{0}
\verb{arrastar}{}{}{}{}{}{Conduzir contra a vontade.}{ar.ras.tar}{0}
\verb{arrastar}{}{}{}{}{}{Falar lentamente, morosamente.}{ar.ras.tar}{0}
\verb{arrastar}{}{}{}{}{v.pron.}{Deslocar"-se com dificuldade.}{ar.ras.tar}{0}
\verb{arrastar}{}{}{}{}{}{Passar (diz"-se do tempo) de maneira mais lenta que o esperado ou desejado.}{ar.ras.tar}{\verboinum{1}}
\verb{arrasto}{}{}{}{}{s.m.}{Ato ou efeito de arrastar.}{ar.ras.to}{0}
\verb{arrasto}{}{}{}{}{}{Força de resistência exercida pelo ar sobre uma aeronave ou veículo espacial.}{ar.ras.to}{0}
\verb{arrátel}{}{}{"-eis}{}{s.m.}{Unidade de medida de peso correspondente a 459 gramas ou 16 onças; libra.}{ar.rá.tel}{0}
\verb{arrazoado}{}{}{}{}{adj.}{Conforme à razão; sensato, razoável.}{ar.ra.zo.a.do}{0}
\verb{arrazoado}{}{}{}{}{s.m.}{Discurso em que se expõem razões em defesa de uma causa.}{ar.ra.zo.a.do}{0}
\verb{arrazoar}{}{}{}{}{v.t.}{Expor (argumentos, motivos); argumentar, discorrer.}{ar.ra.zo.ar}{0}
\verb{arrazoar}{}{}{}{}{}{Censurar, repreender.}{ar.ra.zo.ar}{0}
\verb{arrazoar}{}{}{}{}{v.i.}{Falar, discorrer.}{ar.ra.zo.ar}{\verboinum{7}}
\verb{arre}{}{}{}{}{interj.}{Expressão que denota enfado ou raiva.}{ar.re}{0}
\verb{arre}{}{}{}{}{}{Expressão usada para tocar animais de carga.}{ar.re}{0}
\verb{arreamento}{}{}{}{}{s.m.}{Ato ou efeito de arrear.}{ar.re.a.men.to}{0}
\verb{arreamento}{}{}{}{}{}{Conjunto dos móveis e adornos de uma casa.}{ar.re.a.men.to}{0}
\verb{arreamento}{}{Bras.}{}{}{}{Conjunto de peças necessárias para se utilizar uma montaria; arreios.}{ar.re.a.men.to}{0}
\verb{arrear}{}{}{}{}{v.t.}{Pôr arreios; encilhar, selar.}{ar.re.ar}{0}
\verb{arrear}{}{}{}{}{}{Colocar enfeites; ornamentar.}{ar.re.ar}{0}
\verb{arrear}{}{}{}{}{}{Colocar móveis; mobiliar.}{ar.re.ar}{\verboinum{4}}
\verb{arreata}{}{}{}{}{s.f.}{Correia ou corda com que se conduzem ou prendem as bestas.}{ar.re.a.ta}{0}
\verb{arreata}{}{Por ext.}{}{}{}{Grupo de animais ligados por arreata.}{ar.re.a.ta}{0}
\verb{arrebanhar}{}{}{}{}{v.t.}{Reunir em rebanho.}{ar.re.ba.nhar}{0}
\verb{arrebanhar}{}{}{}{}{}{Reunir, juntar, recolher.}{ar.re.ba.nhar}{\verboinum{1}}
\verb{arrebatado}{}{}{}{}{adj.}{Veemente, impetuoso.}{ar.re.ba.ta.do}{0}
\verb{arrebatado}{}{}{}{}{}{Precipitado, irrefletido.}{ar.re.ba.ta.do}{0}
\verb{arrebatado}{}{}{}{}{}{Dominado por paixão ou irritação; exaltado.}{ar.re.ba.ta.do}{0}
\verb{arrebatado}{}{}{}{}{}{Maravilhado, deslumbrado, extasiado.}{ar.re.ba.ta.do}{0}
\verb{arrebatador}{ô}{}{}{}{adj.}{Que arrebata.}{ar.re.ba.ta.dor}{0}
\verb{arrebatamento}{}{}{}{}{s.m.}{Ato ou efeito de arrebatar.}{ar.re.ba.ta.men.to}{0}
\verb{arrebatamento}{}{}{}{}{}{Êxtase, exaltação, arroubo, irritação.}{ar.re.ba.ta.men.to}{0}
\verb{arrebatamento}{}{}{}{}{}{Precipitação.}{ar.re.ba.ta.men.to}{0}
\verb{arrebatar}{}{}{}{}{v.t.}{Puxar ou tomar com violência ou inesperadamente; arrancar.}{ar.re.ba.tar}{0}
\verb{arrebatar}{}{}{}{}{}{Impelir, conduzir.}{ar.re.ba.tar}{0}
\verb{arrebatar}{}{}{}{}{}{Atrair, encantar, extasiar.}{ar.re.ba.tar}{0}
\verb{arrebatar}{}{}{}{}{}{Sequestrar.}{ar.re.ba.tar}{\verboinum{1}}
\verb{arrebentação}{}{}{"-ões}{}{s.f.}{Ato ou efeito de arrebentar.}{ar.re.ben.ta.ção}{0}
\verb{arrebentação}{}{}{"-ões}{}{}{Choque das ondas sobre a praia, rochedo, banco etc.}{ar.re.ben.ta.ção}{0}
\verb{arrebentação}{}{Por ext.}{"-ões}{}{}{O lugar, próximo à costa, onde as ondas quebram.}{ar.re.ben.ta.ção}{0}
\verb{arrebenta"-cavalo}{}{Bras.}{arrebenta"-cavalos}{}{s.m.}{Erva com folhas geminadas, flores brancas e bagas grandes e comestíveis, utilizada por suas propriedades medicinais.}{ar.re.ben.ta"-ca.va.lo}{0}
\verb{arrebentado}{}{}{}{}{adj.}{Que se arrebentou.}{ar.re.ben.ta.do}{0}
\verb{arrebentado}{}{}{}{}{}{Muito machucado.}{ar.re.ben.ta.do}{0}
\verb{arrebentado}{}{}{}{}{}{Muito cansado; exausto.}{ar.re.ben.ta.do}{0}
\verb{arrebentado}{}{}{}{}{}{Sem nenhum dinheiro; falido, quebrado.}{ar.re.ben.ta.do}{0}
\verb{arrebentar}{}{}{}{}{v.t.}{Estourar, explodir, rebentar.}{ar.re.ben.tar}{\verboinum{1}}
\verb{arrebicar}{}{}{}{}{v.t.}{Pôr arrebiques; enfeitar.}{ar.re.bi.car}{0}
\verb{arrebicar}{}{}{}{}{}{Maquiar em excesso.}{ar.re.bi.car}{\verboinum{2}}
\verb{arrebique}{}{}{}{}{s.m.}{Cosmético avermelhado para pintar o rosto.}{ar.re.bi.que}{0}
\verb{arrebique}{}{}{}{}{}{Enfeite exagerado ou de mau gosto.}{ar.re.bi.que}{0}
\verb{arrebitado}{}{}{}{}{adj.}{Que tem a ponta virada para cima (diz"-se, geralmente, de nariz).}{ar.re.bi.ta.do}{0}
\verb{arrebitado}{}{Bras.}{}{}{}{Esperto, vivo, espevitado.}{ar.re.bi.ta.do}{0}
\verb{arrebitado}{}{}{}{}{}{Petulante, insolente.}{ar.re.bi.ta.do}{0}
\verb{arrebitar}{}{}{}{}{v.t.}{Revirar a ponta para cima.}{ar.re.bi.tar}{0}
\verb{arrebitar}{}{}{}{}{v.pron.}{Levantar"-se.}{ar.re.bi.tar}{0}
\verb{arrebitar}{}{}{}{}{}{Irritar"-se.}{ar.re.bi.tar}{\verboinum{1}}
\verb{arrebite}{}{}{}{}{s.m.}{Parafuso ou prego de duas cabeças que liga chapas de metal; rebite.}{ar.re.bi.te}{0}
\verb{arrebol}{ó}{}{"-óis}{}{s.m.}{Cor avermelhada do céu no amanhecer ou no pôr do sol.}{ar.re.bol}{0}
\verb{arrecadação}{}{}{"-ões}{}{s.f.}{Ato ou efeito de arrecadar.}{ar.re.ca.da.ção}{0}
\verb{arrecadação}{}{}{"-ões}{}{}{Cobrança de taxas ou impostos.}{ar.re.ca.da.ção}{0}
\verb{arrecadação}{}{}{"-ões}{}{}{Local onde se guarda algo; depósito.}{ar.re.ca.da.ção}{0}
\verb{arrecadador}{ô}{}{}{}{adj.}{Que arrecada; recebedor.}{ar.re.ca.da.dor}{0}
\verb{arrecadar}{}{}{}{}{v.t.}{Fazer cobrança; recolher, receber.}{ar.re.ca.dar}{0}
\verb{arrecadar}{}{}{}{}{}{Reunir, juntar, angariar.}{ar.re.ca.dar}{0}
\verb{arrecadar}{}{}{}{}{}{Colocar em segurança; guardar.}{ar.re.ca.dar}{\verboinum{1}}
\verb{arrecife}{}{}{}{}{s.m.}{Rocha que fica perto da costa, pouco acima ou abaixo da superfície do mar; recife.}{ar.re.ci.fe}{0}
\verb{arredar}{}{}{}{}{v.t.}{Provocar o recuo; demover, desviar.}{ar.re.dar}{0}
\verb{arredar}{}{}{}{}{v.i.}{Pôr"-se longe; afastar"-se.}{ar.re.dar}{\verboinum{1}}
\verb{arredio}{}{}{}{}{adj.}{Que se afasta do convívio social ou dos lugares que frequentava; apartado, isolado.}{ar.re.di.o}{0}
\verb{arredio}{}{}{}{}{}{Diz"-se da rês desgarrada do rebanho ou da manada.}{ar.re.di.o}{0}
\verb{arredondado}{}{}{}{}{adj.}{Que sofreu arredondamento.}{ar.re.don.da.do}{0}
\verb{arredondado}{}{}{}{}{}{Que possui forma redonda, circular.}{ar.re.don.da.do}{0}
\verb{arredondamento}{}{}{}{}{s.m.}{Ato ou efeito de arredondar.}{ar.re.don.da.men.to}{0}
\verb{arredondamento}{}{Mat.}{}{}{}{Aproximação do resultado de uma operação, em que se desprezam os últimos algarismos de um número.}{ar.re.don.da.men.to}{0}
\verb{arredondar}{}{}{}{}{v.t.}{Tornar redondo ou circular.}{ar.re.don.dar}{0}
\verb{arredondar}{}{Mat.}{}{}{}{Aproximar o total de uma operação aritmética, ignorando"-se os algarismos finais de um número.}{ar.re.don.dar}{0}
\verb{arredondar}{}{Fig.}{}{}{}{Tornar mais harmonioso, sonoro.}{ar.re.don.dar}{\verboinum{1}}
\verb{arredor}{ó}{}{}{}{adv.}{Ao redor, em volta, em torno.}{ar.re.dor}{0}
\verb{arredor}{ó}{}{}{}{adj.}{Adjacente, circunvizinho.}{ar.re.dor}{0}
\verb{arredores}{ó}{}{}{}{s.m.pl.}{Conjunto das localidades próximas; imediações, cercanias, vizinhança.}{ar.re.do.res}{0}
\verb{arrefecer}{ê}{}{}{}{v.i.}{Perder calor; esfriar.}{ar.re.fe.cer}{0}
\verb{arrefecer}{ê}{Fig.}{}{}{}{Perder a energia, o fervor; desanimar.}{ar.re.fe.cer}{\verboinum{15}}
\verb{arrefecer}{ê}{Fig.}{}{}{}{Abrandar, ceder, amenizar.}{ar.re.fe.cer}{\verboinum{15}}
\verb{arrefecimento}{}{}{}{}{s.m.}{Ato ou efeito de arrefecer.}{ar.re.fe.ci.men.to}{0}
\verb{arrefecimento}{}{}{}{}{}{Perda de calor; esfriamento.}{ar.re.fe.ci.men.to}{0}
\verb{arrefecimento}{}{}{}{}{}{Perda de entusiasmo, de energia; desânimo, apatia.}{ar.re.fe.ci.men.to}{0}
\verb{arregaçar}{}{}{}{}{v.t.}{Dobrar para cima, puxar.}{ar.re.ga.çar}{0}
\verb{arregaçar}{}{}{}{}{}{Colher a borda ou a barra de uma vestimenta.}{ar.re.ga.çar}{\verboinum{3}}
\verb{arregalar}{}{}{}{}{v.t.}{Abrir muito os olhos, por admiração, espanto, perplexidade etc.}{ar.re.ga.lar}{\verboinum{1}}
\verb{arreganhar}{}{}{}{}{v.t.}{Mostrar os dentes, expressando riso ou raiva.}{ar.re.ga.nhar}{0}
\verb{arreganhar}{}{}{}{}{}{Fazer zombaria; escarnecer, troçar.}{ar.re.ga.nhar}{\verboinum{1}}
\verb{arreganho}{}{}{}{}{s.m.}{Ato ou efeito de arreganhar, abrir a boca mostrando os dentes.}{ar.re.ga.nho}{0}
\verb{arreganho}{}{}{}{}{}{Atitude de desafio, audácia, ameaça.}{ar.re.ga.nho}{0}
\verb{arregimentar}{}{}{}{}{v.t.}{Reunir ou associar em partido ou grupo.}{ar.re.gi.men.tar}{0}
\verb{arregimentar}{}{}{}{}{}{Agrupar em regimento militar; ordenar, reunir.}{ar.re.gi.men.tar}{\verboinum{1}}
\verb{arreglo}{ê}{}{}{}{s.m.}{Adaptação de peça teatral estrangeira.}{ar.re.glo}{0}
\verb{arreio}{ê}{}{}{}{s.m.}{Conjunto de peças necessárias para o trabalho de carga ou para a montaria em cavalos.}{ar.rei.o}{0}
\verb{arreio}{ê}{}{}{}{}{Adorno, enfeite, ornamento.}{ar.rei.o}{0}
\verb{arrelia}{}{}{}{}{s.f.}{Mau pressentimento; mau agouro.}{ar.re.li.a}{0}
\verb{arrelia}{}{}{}{}{}{Falta de paciência; zanga, irritação.}{ar.re.li.a}{0}
\verb{arreliado}{}{}{}{}{adj.}{Que gosta de rixas; briguento, brigão.}{ar.re.li.a.do}{0}
\verb{arreliado}{}{}{}{}{}{Atrevido, insolente.}{ar.re.li.a.do}{0}
\verb{arreliar}{}{}{}{}{v.t.}{Causar aborrecimento; irritar, azucrinar, apoquentar.}{ar.re.li.ar}{\verboinum{1}}
\verb{arreliento}{}{}{}{}{adj.}{Que provoca rixas; briguento, provocante, atrevido.}{ar.re.li.en.to}{0}
%\verb{arremangar}{}{}{}{}{}{0}{ar.re.man.gar}{0}
\verb{arrematado}{}{}{}{}{adj.}{Que se concluiu; terminado, acabado.}{ar.re.ma.ta.do}{0}
\verb{arrematado}{}{}{}{}{}{Que foi adquirido em leilão.}{ar.re.ma.ta.do}{0}
\verb{arrematar}{}{}{}{}{v.t.}{Dar acabamento, completar com detalhes; concluir, rematar.}{ar.re.ma.tar}{0}
\verb{arrematar}{}{}{}{}{}{Finalizar uma conversa ou discussão repentinamente.}{ar.re.ma.tar}{\verboinum{1}}
\verb{arrematar}{}{}{}{}{v.t.}{Comprar em leilão ou hasta pública.}{ar.re.ma.tar}{\verboinum{1}}
\verb{arremate}{}{}{}{}{s.m.}{Ato ou efeito de arrematar; remate.}{ar.re.ma.te}{0}
\verb{arremate}{}{}{}{}{}{Conclusão, finalização, acabamento.}{ar.re.ma.te}{0}
\verb{arremate}{}{}{}{}{}{Conjunto de pontos que acabam o trabalho de uma costura, impedindo que o tecido desfie.}{ar.re.ma.te}{0}
\verb{arremate}{}{Esport.}{}{}{}{No futebol, lançamento para o gol, concluindo uma jogada.}{ar.re.ma.te}{0}
\verb{arremedar}{}{}{}{}{v.t.}{Imitar a fala ou os gestos de uma pessoa de forma caricatural, grosseira; macaquear, remedar.}{ar.re.me.dar}{0}
\verb{arremedar}{}{}{}{}{}{Ter semelhança; parecer.}{ar.re.me.dar}{\verboinum{1}}
\verb{arremedo}{ê}{}{}{}{s.m.}{Ato ou efeito de arremedar.}{ar.re.me.do}{0}
\verb{arremedo}{ê}{}{}{}{}{Imitação grotesca, ridícula; paródia.}{ar.re.me.do}{0}
\verb{arremessador}{ô}{}{}{}{adj.}{Que arremessa; lançador.}{ar.re.mes.sa.dor}{0}
\verb{arremessão}{}{}{"-ões}{}{s.m.}{Impulso de lançar para longe; arremesso.}{ar.re.mes.são}{0}
\verb{arremessão}{}{}{"-ões}{}{}{Arma própria para arremesso, como o dardo, a lança etc.}{ar.re.mes.são}{0}
\verb{arremessar}{}{}{}{}{v.t.}{Lançar longe e com força; atirar.}{ar.re.mes.sar}{0}
\verb{arremessar}{}{}{}{}{}{Fazer caminhar com ímpeto; impulsionar, impelir.}{ar.re.mes.sar}{0}
\verb{arremessar}{}{}{}{}{v.pron.}{Atrever"-se, arriscar"-se, aventurar"-se.}{ar.re.mes.sar}{\verboinum{1}}
\verb{arremesso}{ê}{}{}{}{s.m.}{Ato ou efeito de arremessar; lançamento.}{ar.re.mes.so}{0}
\verb{arremesso}{ê}{}{}{}{}{Ação rápida e violenta; investida, ataque.}{ar.re.mes.so}{0}
\verb{arremesso}{ê}{Esport.}{}{}{}{No basquetebol, o lance livre da bola à cesta, em cobrança de falta ou em conclusão de uma jogada.}{ar.re.mes.so}{0}
\verb{arremeter}{ê}{}{}{}{v.t.}{Atacar com ímpeto e fúria; investir, acometer.}{ar.re.me.ter}{0}
\verb{arremeter}{ê}{}{}{}{v.i.}{Lançar"-se audaciosamente; atrever"-se, arrojar"-se.}{ar.re.me.ter}{\verboinum{12}}
\verb{arremetida}{}{}{}{}{s.f.}{Ato ou efeito de arremeter; ataque, investida.}{ar.re.me.ti.da}{0}
\verb{arremetida}{}{}{}{}{}{Ação impetuosa, arrojada.}{ar.re.me.ti.da}{0}
\verb{arrendado}{}{}{}{}{adj.}{Que é dado ou tomado em arrendamento, pelo qual se recebe um valor contratado.}{ar.ren.da.do}{0}
\verb{arrendado}{}{}{}{}{}{Que apresenta bons rendimentos.}{ar.ren.da.do}{0}
\verb{arrendado}{}{}{}{}{adj.}{Enfeitado com rendas; rendado.}{ar.ren.da.do}{0}
\verb{arrendador}{ô}{}{}{}{s.m.}{Indivíduo que cede algo em sistema de arrendamento; locador.}{ar.ren.da.dor}{0}
\verb{arrendamento}{}{}{}{}{s.m.}{Ato ou efeito de arrendar.}{ar.ren.da.men.to}{0}
\verb{arrendamento}{}{Jur.}{}{}{}{Contrato pelo qual o dono de um imóvel cede seu uso ou usufruto a outrem, por um certo tempo, mediante o pagamento de uma quantia estabelecida.}{ar.ren.da.men.to}{0}
\verb{arrendamento}{}{}{}{}{}{O instrumento desse contrato.}{ar.ren.da.men.to}{0}
\verb{arrendar}{}{}{}{}{v.t.}{Dar ou tomar em arrendamento; alugar.}{ar.ren.dar}{\verboinum{1}}
\verb{arrendar}{}{}{}{}{v.t.}{Enfeitar com rendas; rendilhar, rendar.}{ar.ren.dar}{\verboinum{1}}
\verb{arrendatário}{}{}{}{}{s.m.}{Indivíduo que toma algo em arrendamento; locatário, inquilino.}{ar.ren.da.tá.rio}{0}
\verb{arrenegar}{}{}{}{}{v.t.}{Renegar, renunciar.}{ar.re.ne.gar}{0}
\verb{arrenegar}{}{}{}{}{v.pron.}{Sentir"-se irritado; zangar"-se, encolerizar"-se.}{ar.re.ne.gar}{\verboinum{5}}
\verb{arrepanhar}{}{}{}{}{v.t.}{Puxar, recolher, fazendo dobras; repuxar, enrugar.}{ar.re.pa.nhar}{0}
\verb{arrepanhar}{}{}{}{}{}{Poupar, economizar com sovinice.}{ar.re.pa.nhar}{0}
\verb{arrepanhar}{}{Fig.}{}{}{}{Tirar com violência; arrebatar.}{ar.re.pa.nhar}{\verboinum{1}}
\verb{arrepelar}{}{}{}{}{v.t.}{Puxar, arrancar cabelos, pelos, penas etc.}{ar.re.pe.lar}{0}
\verb{arrepelar}{}{}{}{}{v.pron.}{Demonstrar arrependimento; lastimar"-se, lamentar"-se.}{ar.re.pe.lar}{\verboinum{1}}
\verb{arrepender"-se}{}{}{}{}{v.pron.}{Lamentar"-se por faltas ou erros cometidos; sentir pesar.}{ar.re.pen.der"-se}{0}
\verb{arrependerse}{}{}{}{}{}{Mudar de atitudes ou opinião; voltar atrás, retroceder.}{ar.re.pen.der"-se}{\verboinum{12}}
\verb{arrependido}{}{}{}{}{adj.}{Que se arrependeu; penitente.}{ar.re.pen.di.do}{0}
\verb{arrependimento}{}{}{}{}{s.m.}{Ato ou efeito de arrepender"-se.}{ar.re.pen.di.men.to}{0}
\verb{arrependimento}{}{}{}{}{}{Contrição, remorso, pesar.}{ar.re.pen.di.men.to}{0}
\verb{arrependimento}{}{}{}{}{}{Mudança de opinião ou de atitude; desistência.}{ar.re.pen.di.men.to}{0}
\verb{arrepiado}{}{}{}{}{adj.}{Diz"-se do cabelo ou pelo eriçado, levantado.}{ar.re.pi.a.do}{0}
\verb{arrepiado}{}{}{}{}{}{Espantado, assustado, apavorado.}{ar.re.pi.a.do}{0}
\verb{arrepiante}{}{}{}{}{adj.2g.}{Que arrepia, aterroriza; pavoroso, medonho, assustador.}{ar.re.pi.an.te}{0}
\verb{arrepiar}{}{}{}{}{v.t.}{Levantar ou eriçar os cabelos, pelos etc.}{ar.re.pi.ar}{0}
\verb{arrepiar}{}{}{}{}{}{Fazer tremer de medo, de horror; causar calafrios.}{ar.re.pi.ar}{0}
\verb{arrepiar}{}{}{}{}{}{Fazer tremer ou tiritar de frio.}{ar.re.pi.ar}{0}
\verb{arrepiar}{}{}{}{}{}{Desmanchar, desordenar o cabelo.}{ar.re.pi.ar}{\verboinum{1}}
\verb{arrepio}{}{}{}{}{s.m.}{Estremecimento causado por frio, medo ou emoção violenta e repentina; calafrio.}{ar.re.pi.o}{0}
\verb{arrestar}{}{Jur.}{}{}{v.t.}{Apreender judicialmente; embargar, confiscar.}{ar.res.tar}{\verboinum{1}}
\verb{arresto}{é}{Jur.}{}{}{s.m.}{Medida preventiva de apreensão de bens de um devedor para assegurar o pagamento de dívidas; confisco, embargo.}{ar.res.to}{0}
\verb{arrevesado}{}{}{}{}{adj.}{Colocado às avessas; invertido.}{ar.re.ve.sa.do}{0}
\verb{arrevesado}{}{}{}{}{}{De difícil compreensão; obscuro, ininteligível.}{ar.re.ve.sa.do}{0}
\verb{arrevesado}{}{}{}{}{}{De comportamento ríspido, áspero, rude.}{ar.re.ve.sa.do}{0}
\verb{arrevesar}{}{}{}{}{v.t.}{Colocar ao revés, às avessas.}{ar.re.ve.sar}{0}
\verb{arrevesar}{}{}{}{}{}{Dar sentido contrário; inverter.}{ar.re.ve.sar}{0}
\verb{arrevesar}{}{}{}{}{}{Tornar obscuro, confuso.}{ar.re.ve.sar}{\verboinum{1}}
\verb{arriar}{}{}{}{}{v.t.}{Fazer descer; abaixar, colocar no chão.}{ar.ri.ar}{0}
\verb{arriar}{}{}{}{}{v.i.}{Cair, desabar sob o próprio peso.}{ar.ri.ar}{0}
\verb{arriar}{}{}{}{}{}{Perder as forças; desanimar, desistir.}{ar.ri.ar}{\verboinum{1}}
\verb{arriba}{}{}{}{}{adv.}{Acima, para cima, adiante.}{ar.ri.ba}{0}
\verb{arriba}{}{}{}{}{interj.}{Expressão que denota movimento para cima e adiante.}{ar.ri.ba}{0}
\verb{arribação}{}{}{"-ões}{}{s.f.}{Ato ou efeito de arribar, chegar a algum lugar; arribada.}{ar.ri.ba.ção}{0}
\verb{arribação}{}{}{"-ões}{}{}{Deslocamento de animais de uma região a outra em determinadas épocas do ano; migração.}{ar.ri.ba.ção}{0}
\verb{arribada}{}{}{}{}{s.f.}{Arribação.}{ar.ri.ba.da}{0}
\verb{arribada}{}{}{}{}{}{Recuperação da saúde; convalescença, melhora.}{ar.ri.ba.da}{0}
\verb{arribar}{}{}{}{}{v.i.}{Subir, chegar ao alto, ao cume.}{ar.ri.bar}{0}
\verb{arribar}{}{}{}{}{}{Mudar de uma região para outra; migrar.}{ar.ri.bar}{0}
\verb{arribar}{}{}{}{}{}{Chegar o navio ao porto ou costa; aportar.}{ar.ri.bar}{0}
\verb{arribar}{}{Fig.}{}{}{}{Melhorar de saúde ou de sorte; restabelecer"-se.}{ar.ri.bar}{\verboinum{1}}
\verb{arrieiro}{ê}{}{}{}{s.m.}{Indivíduo que conduz bestas de carga; tropeiro.}{ar.ri.ei.ro}{0}
\verb{arrieiro}{ê}{Fig.}{}{}{}{Indivíduo rude, grosseiro.}{ar.ri.ei.ro}{0}
\verb{arrimar}{}{}{}{}{v.t.}{Servir de apoio; escorar, encostar.}{ar.ri.mar}{0}
\verb{arrimar}{}{}{}{}{}{Amparar, socorrer, proteger.}{ar.ri.mar}{\verboinum{1}}
\verb{arrimo}{}{}{}{}{s.m.}{Peça ou lugar que serve de apoio; escora, encosto.}{ar.ri.mo}{0}
\verb{arrimo}{}{}{}{}{}{Amparo, sustento, auxílio.}{ar.ri.mo}{0}
\verb{arriscado}{}{}{}{}{adj.}{Que envolve risco ou perigo; perigoso.}{ar.ris.ca.do}{0}
\verb{arriscado}{}{}{}{}{}{Que se aventura; audacioso, ousado.}{ar.ris.ca.do}{0}
\verb{arriscar}{}{}{}{}{v.t.}{Pôr em risco ou perigo.}{ar.ris.car}{0}
\verb{arriscar}{}{}{}{}{}{Expor à boa ou má sorte; aventurar.}{ar.ris.car}{\verboinum{2}}
\verb{arritmia}{}{}{}{}{s.f.}{Perturbação ou variação do ritmo.}{ar.rit.mi.a}{0}
\verb{arritmia}{}{Med.}{}{}{}{Irregularidade do ritmo cardíaco.}{ar.rit.mi.a}{0}
\verb{arrítmico}{}{}{}{}{adj.}{Que não apresenta ritmo. }{ar.rít.mi.co}{0}
\verb{arrítmico}{}{}{}{}{}{Relativo à arritmia.}{ar.rít.mi.co}{0}
\verb{arrítmico}{}{Med.}{}{}{}{Diz"-se das contrações cardíacas irregulares.}{ar.rít.mi.co}{0}
\verb{arrivismo}{}{}{}{}{s.m.}{Característica ou comportamento daquele que quer ter êxito a todo custo; ambição.}{ar.ri.vis.mo}{0}
\verb{arrivista}{}{}{}{}{adj.2g.}{Diz"-se da pessoa que deseja vencer a todo custo; ambicioso.}{ar.ri.vis.ta}{0}
\verb{arrizotônico}{}{Gram.}{}{}{adj.}{Diz"-se do vocábulo cujo acento recai fora da raiz, em um sufixo derivacional ou flexional.}{ar.ri.zo.tô.ni.co}{0}
\verb{arroba}{ô}{}{}{}{s.f.}{Unidade de peso utilizada no Brasil na medida de produtos agropecuários equivalente a 15 kg.}{ar.ro.ba}{0}
\verb{arroba}{ô}{Informát.}{}{}{}{Nome do sinal gráfico @ usado em endereçamento eletrônico com sentido locativo.}{ar.ro.ba}{0}
\verb{arrochar}{}{}{}{}{v.t.}{Fixar carga com pedaço de madeira, apertando"-a.}{ar.ro.char}{0}
\verb{arrochar}{}{}{}{}{}{Prender com muita força; atar, apertar.}{ar.ro.char}{0}
\verb{arrochar}{}{Fig.}{}{}{}{Pressionar, exigir muito, oprimir.}{ar.ro.char}{0}
\verb{arrochar}{}{}{}{}{}{Não fazer a correção monetária dos salários.}{ar.ro.char}{\verboinum{1}}
\verb{arrocho}{ô}{}{}{}{s.m.}{Pau curto e torto utilizado para torcer as cordas que prendem cargas, fardos etc.}{ar.ro.cho}{0}
\verb{arrocho}{ô}{Fig.}{}{}{}{Situação difícil, penosa; aperto.}{ar.ro.cho}{0}
\verb{arrocho}{ô}{}{}{}{}{Repressão violenta da polícia.}{ar.ro.cho}{0}
\verb{arrocho}{ô}{}{}{}{}{Não correção monetária dos salários.}{ar.ro.cho}{0}
\verb{arrogância}{}{}{}{}{s.f.}{Atitude orgulhosa; altivez, soberba.}{ar.ro.gân.cia}{0}
\verb{arrogância}{}{}{}{}{}{Atitude desrespeitosa; insolência, atrevimento.}{ar.ro.gân.cia}{0}
\verb{arrogante}{}{}{}{}{adj.2g.}{Que revela arrogância; soberbo, orgulhoso, altivo.}{ar.ro.gan.te}{0}
\verb{arrogante}{}{}{}{}{}{Que apresenta atitude desrespeitosa; insolente, atrevido, pretensioso.}{ar.ro.gan.te}{0}
\verb{arrogar}{}{}{}{}{v.t.}{Ter como próprio; apropriar"-se.}{ar.ro.gar}{0}
\verb{arrogar}{}{}{}{}{}{Atribuir, imputar.}{ar.ro.gar}{\verboinum{5}}
\verb{arroio}{ô}{}{}{}{s.m.}{Pequeno curso de água, permanente ou não; regato, córrego, riacho.}{ar.roi.o}{0}
\verb{arrojado}{}{}{}{}{adj.}{Que apresenta audácia; valente, intrépido, temerário.}{ar.ro.ja.do}{0}
\verb{arrojado}{}{}{}{}{}{Que revela características inovadoras; ousado, moderno.}{ar.ro.ja.do}{0}
\verb{arrojar}{}{}{}{}{v.t.}{Atirar com ímpeto ou força; arremessar, lançar.}{ar.ro.jar}{0}
\verb{arrojar}{}{}{}{}{v.pron.}{Lançar"-se de uma grande altura; despenhar"-se.}{ar.ro.jar}{0}
\verb{arrojar}{}{}{}{}{}{Atrever"-se, aventurar"-se, arriscar"-se.}{ar.ro.jar}{0}
\verb{arrojar}{}{}{}{}{}{Andar de rastos; arrastar"-se, rebaixar"-se.}{ar.ro.jar}{\verboinum{1}}
\verb{arrojo}{ô}{}{}{}{s.m.}{Ato ou efeito de arrojar, arremessar.}{ar.ro.jo}{0}
\verb{arrojo}{ô}{}{}{}{}{Ousadia, audácia, atrevimento.}{ar.ro.jo}{0}
\verb{arrolamento}{}{}{}{}{s.m.}{Ato ou efeito de arrolar; levantamento.}{ar.ro.la.men.to}{0}
\verb{arrolamento}{}{Jur.}{}{}{}{Relação de bens, inventário.}{ar.ro.la.men.to}{0}
\verb{arrolar}{}{}{}{}{v.t.}{Pôr em rol ou lista; inventariar, classificar.}{ar.ro.lar}{\verboinum{1}}
\verb{arrolar}{}{}{}{}{v.t.}{Dar formato de rolo; enrolar.}{ar.ro.lar}{\verboinum{1}}
\verb{arrolhar}{}{}{}{}{v.t.}{Tapar com rolha; rolhar.}{ar.ro.lhar}{\verboinum{1}}
\verb{arromba}{}{Mús.}{}{}{s.f.}{Canção viva e ruidosa, tocada na viola.}{ar.rom.ba}{0}
\verb{arromba}{}{}{}{}{}{Usado na expressão \textit{de arromba}: excelente, assombroso, espantoso.}{ar.rom.ba}{0}
\verb{arrombador}{ô}{}{}{}{adj.}{Que arromba, que abre à força.}{ar.rom.ba.dor}{0}
\verb{arrombamento}{}{}{}{}{s.m.}{Ato ou efeito de arrombar; abertura forçada, rombo.}{ar.rom.ba.men.to}{0}
\verb{arrombar}{}{}{}{}{v.t.}{Fazer um grande buraco, um rombo; romper.}{ar.rom.bar}{0}
\verb{arrombar}{}{}{}{}{}{Abrir à força, com violência.}{ar.rom.bar}{0}
\verb{arrombar}{}{Fig.}{}{}{}{Derrotar, abater, humilhar.}{ar.rom.bar}{\verboinum{1}}
\verb{arrostar}{}{}{}{}{v.t.}{Olhar de frente, sem medo; encarar.}{ar.ros.tar}{0}
\verb{arrostar}{}{}{}{}{}{Pôr frente a frente; afrontar, resistir.}{ar.ros.tar}{\verboinum{1}}
\verb{arrotar}{}{}{}{}{v.i.}{Soltar gases estomacais pela boca; eructar.}{ar.ro.tar}{0}
\verb{arrotar}{}{Fig.}{}{}{v.t.}{Ostentar, alardear, bravatear.}{ar.ro.tar}{\verboinum{1}}
\verb{arrotear}{}{}{}{}{v.t.}{Cultivar um terreno pela primeira vez; preparar o terreno para semear.}{ar.ro.te.ar}{0}
\verb{arrotear}{}{}{}{}{}{Dar educação; instruir.}{ar.ro.te.ar}{\verboinum{4}}
\verb{arroto}{ô}{}{}{}{s.m.}{Emissão sonora de gases estomacais pela boca; eructação.}{ar.ro.to}{0}
\verb{arroubamento}{}{}{}{}{s.m.}{Arroubo.}{ar.rou.ba.men.to}{0}
\verb{arroubar}{}{}{}{}{v.t.}{Tornar extasiado; enlevar, arrebatar.}{ar.rou.bar}{\verboinum{1}}
\verb{arroubo}{ô}{}{}{}{s.m.}{Ato ou efeito de arroubar; enlevo, arrebatamento, encanto, arroubamento.}{ar.rou.bo}{0}
\verb{arroxeado}{ch}{}{}{}{adj.}{De cor semelhante ou próxima a roxo.}{ar.ro.xe.a.do}{0}
\verb{arroxear}{ch}{}{}{}{v.t.}{Tornar roxo; roxear.}{ar.ro.xe.ar}{\verboinum{4}}
\verb{arroz}{ô}{Bot.}{}{}{s.m.}{Planta da família das gramíneas, de origem asiática, com muitas variedades cultiváveis, cujos grãos constituem a base alimentar de grande parte da população mundial.}{ar.roz}{0}
\verb{arroz}{ô}{}{}{}{}{O grão dessa planta.}{ar.roz}{0}
\verb{arroz}{ô}{Cul.}{}{}{}{Qualquer prato preparado com esses grãos, descascados e cozidos. }{ar.roz}{0}
\verb{arrozal}{}{}{"-ais}{}{s.m.}{Terreno plantado com pés de arroz.}{ar.ro.zal}{0}
\verb{arroz"-doce}{ô}{Cul.}{}{}{s.m.}{Doce preparado com arroz cozido no leite adoçado, polvilhado com canela.}{ar.roz"-do.ce}{0}
\verb{arruaça}{}{}{}{}{s.f.}{Desordem ou tumulto de rua; motim, briga.}{ar.ru.a.ça}{0}
\verb{arruaceiro}{ê}{}{}{}{adj.}{Que promove ou participa de arruaça, confusão de rua; baderneiro.}{ar.ru.a.cei.ro}{0}
\verb{arruado}{}{}{}{}{s.m.}{Ato ou efeito de arruar, projetar a abertura de ruas; arruamento.}{ar.ru.a.do}{0}
\verb{arruado}{}{Bras.}{}{}{}{Pequeno povoado à beira da estrada.}{ar.ru.a.do}{0}
\verb{arruamento}{}{}{}{}{s.m.}{Ato ou efeito de arruar, projetar a abertura de ruas.}{ar.ru.a.men.to}{0}
\verb{arruamento}{}{}{}{}{}{Disposição das ruas de um loteamento ou bairro.}{ar.ru.a.men.to}{0}
\verb{arruamento}{}{}{}{}{}{Instalação de estabelecimentos de uma mesma área profissional em uma determinada rua.}{ar.ru.a.men.to}{0}
\verb{arruar}{}{}{}{}{v.t.}{Traçar ou projetar ruas, loteamentos, caminhos.}{ar.ru.ar}{0}
\verb{arruar}{}{}{}{}{v.i.}{Passear pelas ruas como vadio, desocupado.}{ar.ru.ar}{\verboinum{1}}
\verb{arruda}{}{Bot.}{}{}{s.f.}{Planta originária da Europa, cultivada em jardins e hortas, de cujas folhas, verde"-acinzentadas, extrai"-se um óleo para fins medicinais e aromáticos.}{ar.ru.da}{0}
\verb{arruela}{é}{}{}{}{s.f.}{Pequeno anel de metal, couro ou borracha, por onde passa um parafuso, servindo de base para a porca, a fim de evitar o desgaste da peça a ser ajustada.}{ar.ru.e.la}{0}
\verb{arrufar}{}{}{}{}{v.t.}{Tornar amuado, agastado; irritar, zangar.}{ar.ru.far}{0}
\verb{arrufar}{}{}{}{}{}{Tornar crespo; arrepiar, ouriçar.}{ar.ru.far}{\verboinum{1}}
\verb{arrufo}{}{}{}{}{s.m.}{Ato ou efeito de arrufar; amuo, irritação.}{ar.ru.fo}{0}
\verb{arrufo}{}{}{}{}{}{Mau humor ou ressentimento passageiro entre pessoas que se estimam.}{ar.ru.fo}{0}
\verb{arruinado}{}{}{}{}{adj.}{Reduzido a ruína; devastado, destruído.}{ar.rui.na.do}{0}
\verb{arruinado}{}{}{}{}{}{Reduzido à miséria; empobrecido.}{ar.rui.na.do}{0}
\verb{arruinado}{}{}{}{}{}{Infeccionado, inflamado.}{ar.rui.na.do}{0}
\verb{arruinar}{}{}{}{}{v.t.}{Causar ruína; fazer desmoronar; destruir.}{ar.rui.nar}{0}
\verb{arruinar}{}{}{}{}{}{Causar estrago; danificar.}{ar.rui.nar}{0}
\verb{arruinar}{}{}{}{}{}{Reduzir à miséria física, moral ou financeira; arrasar, empobrecer.}{ar.rui.nar}{0}
\verb{arruinar}{}{}{}{}{v.i.}{Infeccionar; gangrenar.}{ar.rui.nar}{\verboinum{1}}
\verb{arruivado}{}{}{}{}{adj.}{Que tem cor próxima ao ruivo; avermelhado.}{ar.rui.va.do}{0}
\verb{arruivar}{}{}{}{}{v.t.}{Tornar ruivo; tingir de ruivo.}{ar.rui.var}{\verboinum{1}}
\verb{arrulhar}{}{}{}{}{v.i.}{Produzir sons como os pombos e as rolas.}{ar.ru.lhar}{0}
\verb{arrulhar}{}{Fig.}{}{}{}{Dizer palavras doces e meigas.}{ar.ru.lhar}{0}
\verb{arrulhar}{}{Fig.}{}{}{}{Cantar e embalar, fazendo adormecer.}{ar.ru.lhar}{\verboinum{1}}
\verb{arrulho}{}{}{}{}{s.m.}{Som emitido por pombas e rolas.}{ar.ru.lho}{0}
\verb{arrulho}{}{Fig.}{}{}{}{Fala meiga e carinhosa.}{ar.ru.lho}{0}
\verb{arrulho}{}{Fig.}{}{}{}{Cantiga ou toada para fazer adormecer.}{ar.ru.lho}{0}
\verb{arrumação}{}{}{"-ões}{}{s.f.}{Ato ou efeito de arrumar; ordenação, arranjo.}{ar.ru.ma.ção}{0}
\verb{arrumação}{}{}{"-ões}{}{}{Escrituração comercial em ordem.}{ar.ru.ma.ção}{0}
\verb{arrumação}{}{}{"-ões}{}{}{Emprego, colocação, trabalho.}{ar.ru.ma.ção}{0}
\verb{arrumadeira}{ê}{}{}{}{s.f.}{Empregada incumbida da limpeza e arrumação de casa, quarto de hotel, escritório etc.}{ar.ru.ma.dei.ra}{0}
\verb{arrumadela}{é}{}{}{}{s.f.}{Arrumação rápida e superficial para dar uma certa organização ao ambiente.}{ar.ru.ma.de.la}{0}
\verb{arrumado}{}{}{}{}{adj.}{Que foi posto em ordem; organizado.}{ar.ru.ma.do}{0}
\verb{arrumado}{}{}{}{}{}{Devidamente vestido; pronto.}{ar.ru.ma.do}{0}
\verb{arrumado}{}{}{}{}{}{Resolvido, acertado, combinado.}{ar.ru.ma.do}{0}
\verb{arrumar}{}{}{}{}{v.t.}{Pôr em certa ordem; arranjar, dispor.}{ar.ru.mar}{0}
\verb{arrumar}{}{}{}{}{}{Fazer a organização; preparar, aprontar.}{ar.ru.mar}{0}
\verb{arrumar}{}{}{}{}{}{Conseguir, obter, alcançar.}{ar.ru.mar}{0}
\verb{arrumar}{}{}{}{}{}{Dar rumo a; direcionar.}{ar.ru.mar}{0}
\verb{arrumar}{}{}{}{}{v.pron.}{Conseguir boa situação; colocar"-se, empregar"-se.}{ar.ru.mar}{0}
\verb{arrumar}{}{}{}{}{}{Vestir"-se, ajeitar"-se, aprontar"-se.}{ar.ru.mar}{\verboinum{1}}
\verb{arsenal}{}{}{"-ais}{}{s.m.}{Estabelecimento oficial onde se fabricam ou guardam armamentos.}{ar.se.nal}{0}
\verb{arsenal}{}{Por ext.}{"-ais}{}{}{Lugar onde há muitas armas.}{ar.se.nal}{0}
\verb{arsenal}{}{Fig.}{"-ais}{}{}{Grande quantidade; conjunto, porção.}{ar.se.nal}{0}
\verb{arsênico}{}{Quím.}{}{}{s.m.}{Nome comum de alguns compostos venenosos em que entra o arsênio;  arsênio.}{ar.sê.ni.co}{0}
\verb{arsênio}{}{Quím.}{}{}{s.m.}{Elemento químico sólido cristalino, acinzentado, usado em transistores, em ligas especiais, em pirotecnia e, sob a forma de compostos, em medicina. \elemento{33}{74.9216}{As}.}{ar.sê.nio}{0}
\verb{arte}{}{}{}{}{s.f.}{Atividade humana de criação plástica, literária, musical etc.}{ar.te}{0}
\verb{arte}{}{}{}{}{}{As habilidades necessárias à boa execução de algo.}{ar.te}{0}
\verb{arte}{}{}{}{}{}{O conjunto das obras de arte de um povo ou país.}{ar.te}{0}
\verb{arte}{}{Bras.}{}{}{}{Travessura, traquinagem.}{ar.te}{0}
\verb{arte}{}{}{}{}{}{Redução de \textit{arte"-final}.}{ar.te}{0}
\verb{artefato}{}{}{}{}{s.m.}{Artigo manufaturado; objeto, peça.}{ar.te.fa.to}{0}
\verb{artefato}{}{}{}{}{}{Aparelho ou mecanismo construído para determinada finalidade.}{ar.te.fa.to}{0}
\verb{artefato}{}{Fig.}{}{}{}{Raciocínio ou procedimento utilizado para se contornar uma dificuldade ou produzir uma conclusão falsa.}{ar.te.fa.to}{0}
\verb{arte"-final}{}{}{artes"-finais}{}{s.f.}{A forma final de um trabalho gráfico ou artístico, pronto para ser reproduzido.}{ar.te"-fi.nal}{0}
\verb{arte"-finalista}{}{}{artes"-finalistas}{}{s.2g.}{Profissional responsável por executar ou avaliar as artes"-finais.}{ar.te"-fi.na.lis.ta}{0}
\verb{arteirice}{}{}{}{}{s.f.}{Ação de arteiro; esperteza, manha.}{ar.tei.ri.ce}{0}
\verb{arteirice}{}{Bras.}{}{}{}{Travessura, traquinada.}{ar.tei.ri.ce}{0}
\verb{arteiro}{ê}{}{}{}{adj.}{Que é esperto, sagaz, ardiloso.}{ar.tei.ro}{0}
\verb{arteiro}{ê}{Bras.}{}{}{}{Travesso, traquina.}{ar.tei.ro}{0}
\verb{artelho}{ê}{Anat.}{}{}{s.m.}{Articulação por onde o pé se prende com a perna; junta, nó, tornozelo, dedo do pé.}{ar.te.lho}{0}
\verb{artéria}{}{Anat.}{}{}{s.f.}{Vaso que transporta sangue oxigenado pelo corpo.}{ar.té.ria}{0}
\verb{artéria}{}{}{}{}{}{Via de comunicação de alta capacidade, por onde circula grande parte do tráfego.}{ar.té.ria}{0}
\verb{arterial}{}{Anat.}{"-ais}{}{adj.2g.}{Relativo a artéria.}{ar.te.ri.al}{0}
\verb{arteriografia}{}{}{}{}{s.f.}{Exame radiológico que possibilita a visualização das artérias.}{ar.te.ri.o.gra.fi.a}{0}
\verb{arteriosclerose}{ó}{Med.}{}{}{s.f.}{Doença caracterizada pelo endurecimento do tecido das artérias.}{ar.te.ri.os.cle.ro.se}{0}
\verb{artesanal}{}{}{"-ais}{}{adj.2g.}{Relativo a artesão ou a artesanato.}{ar.te.sa.nal}{0}
\verb{artesanal}{}{Fig.}{"-ais}{}{}{Que é feito sem muita sofisticação ou acabamento; rústico.}{ar.te.sa.nal}{0}
\verb{artesanato}{}{}{}{}{s.m.}{A técnica do trabalho manual não industrializado.}{ar.te.sa.na.to}{0}
\verb{artesanato}{}{}{}{}{}{O produto final desse trabalho.}{ar.te.sa.na.to}{0}
\verb{artesanato}{}{}{}{}{}{Conjunto das peças produzidas nessas condições.}{ar.te.sa.na.to}{0}
\verb{artesão}{}{}{"-ãos}{artesã}{s.m.}{Indivíduo que exerce por conta própria uma arte, um ofício manual.}{ar.te.são}{0}
\verb{artesiano}{}{}{}{}{adj.}{Diz"-se do lençol de água subterrâneo escoado por um poço artificial.}{ar.te.si.a.no}{0}
\verb{artesiano}{}{}{}{}{}{Diz"-se de poço cavado perpendicularmente ao solo visando atingir um desses lençóis.}{ar.te.si.a.no}{0}
\verb{artesiano}{}{}{}{}{}{Relativo a Artésia, região da França.}{ar.te.si.a.no}{0}
\verb{artesiano}{}{}{}{}{s.2g.}{Indivíduo natural ou habitante dessa região.}{ar.te.si.a.no}{0}
\verb{ártico}{}{}{}{}{adj.}{Que é do norte; setentrional, boreal.}{ár.ti.co}{0}
\verb{articulação}{}{}{"-ões}{}{s.f.}{Ato ou efeito de articular.}{ar.ti.cu.la.ção}{0}
\verb{articulação}{}{Anat.}{"-ões}{}{}{Ponto de contato entre duas ou mais partes do corpo.}{ar.ti.cu.la.ção}{0}
\verb{articulação}{}{Gram.}{"-ões}{}{}{Produção das palavras ou dos sons da fala.}{ar.ti.cu.la.ção}{0}
\verb{articulação}{}{}{"-ões}{}{}{Junção entre duas peças que permite mobilidade.}{ar.ti.cu.la.ção}{0}
\verb{articulação}{}{}{"-ões}{}{}{Arranjo de pessoas ou tomada de providências visando a consecução de objetivos definidos.}{ar.ti.cu.la.ção}{0}
\verb{articulado}{}{}{}{}{adj.}{Que se articulou.}{ar.ti.cu.la.do}{0}
\verb{articulado}{}{}{}{}{}{Que apresenta uma ou mais articulações.}{ar.ti.cu.la.do}{0}
\verb{articulado}{}{Fig.}{}{}{}{Diz"-se de pessoa hábil no que faz ou no modo de expressar"-se.}{ar.ti.cu.la.do}{0}
\verb{articular}{}{}{}{}{v.t.}{Unir através de articulações; juntar, unir.}{ar.ti.cu.lar}{0}
\verb{articular}{}{Gram.}{}{}{}{Pronunciar as palavras ou sons da fala.}{ar.ti.cu.lar}{0}
\verb{articular}{}{}{}{}{}{Tomar as providências para a consecução de objetivos definidos.}{ar.ti.cu.lar}{\verboinum{1}}
\verb{articular}{}{Med.}{}{}{adj.2g.}{Relativo às articulações.}{ar.ti.cu.lar}{0}
\verb{articular}{}{Gram.}{}{}{}{Relativo ao artigo.}{ar.ti.cu.lar}{0}
\verb{articulista}{}{}{}{}{s.2g.}{Indivíduo que escreve artigos de jornal, revista etc.}{ar.ti.cu.lis.ta}{0}
\verb{artífice}{}{}{}{}{s.2g.}{Artesão ou operário que trabalha em certos ofícios.}{ar.tí.fi.ce}{0}
\verb{artífice}{}{Fig.}{}{}{}{Inventor, autor, criador.}{ar.tí.fi.ce}{0}
\verb{artificial}{}{}{"-ais}{}{adj.2g.}{Que é produzido pelo homem e não pela natureza.}{ar.ti.fi.ci.al}{0}
\verb{artificial}{}{}{"-ais}{}{}{Fingido, forçado. (\textit{Aquela atriz interpreta de uma forma muito artificial.})}{ar.ti.fi.ci.al}{0}
\verb{artifício}{}{}{}{}{s.m.}{Meio para se obter um artefato.}{ar.ti.fí.cio}{0}
\verb{artifício}{}{Por ext.}{}{}{}{Recurso ou raciocínio engenhoso.}{ar.ti.fí.cio}{0}
\verb{artifício}{}{Por ext.}{}{}{}{Sagacidade, astúcia.}{ar.ti.fí.cio}{0}
\verb{artificioso}{ô}{}{"-osos ⟨ó⟩}{"-osa ⟨ó⟩}{adj.}{Que procura enganar; astucioso.}{ar.ti.fi.ci.o.so}{0}
\verb{artificioso}{ô}{}{"-osos ⟨ó⟩}{"-osa ⟨ó⟩}{}{Habilidoso, engenhoso}{ar.ti.fi.ci.o.so}{0}
\verb{artigo}{}{}{}{}{s.m.}{Mercadoria; objeto de comércio.}{ar.ti.go}{0}
\verb{artigo}{}{}{}{}{}{Cada uma das partes numeradas de decretos, leis, códigos.}{ar.ti.go}{0}
\verb{artigo}{}{}{}{}{}{Texto de jornal ou revista, com assinatura do autor, no qual expressa opinião ou se comenta um assunto a partir de determinado ponto de vista.}{ar.ti.go}{0}
\verb{artigo}{}{Gram.}{}{}{}{Classe de palavra variável que antecede palavras indicando"-lhes o gênero e o número.}{ar.ti.go}{0}
\verb{artilharia}{}{}{}{}{s.f.}{Uma das armas do exército; a tropa que opera com esse tipo de armamento.}{ar.ti.lha.ri.a}{0}
\verb{artilharia}{}{Fig.}{}{}{}{Qualquer recurso de argumentação para ataque ou defesa numa discussão.}{ar.ti.lha.ri.a}{0}
\verb{artilheiro}{ê}{}{}{}{s.m.}{Militar da artilharia.}{ar.ti.lhei.ro}{0}
\verb{artilheiro}{ê}{Esport.}{}{}{}{Jogador que faz maior número de gols, numa partida ou campeonato.}{ar.ti.lhei.ro}{0}
\verb{artimanha}{}{}{}{}{s.f.}{Astúcia, estratagema, ardil.}{ar.ti.ma.nha}{0}
\verb{artista}{}{}{}{}{s.2g.}{Indivíduo que se dedica à arte em geral.}{ar.tis.ta}{0}
\verb{artista}{}{}{}{}{}{Indivíduo que representa papéis em teatro, cinema, rádio, televisão; ator.}{ar.tis.ta}{0}
\verb{artista}{}{}{}{}{}{Indivíduo habilidoso e meticuloso no seu ofício.}{ar.tis.ta}{0}
\verb{artístico}{}{}{}{}{adj.}{Relativo a arte.}{ar.tís.ti.co}{0}
\verb{artístico}{}{}{}{}{}{Relativo aos artistas, às obras de arte.}{ar.tís.ti.co}{0}
\verb{artístico}{}{}{}{}{}{Que foi trabalhado ou fabricado com arte.}{ar.tís.ti.co}{0}
\verb{artrite}{}{Med.}{}{}{s.f.}{Inflamação em uma articulação, caracterizada por dor, vermelhidão, calor e, às vezes, inchaço.}{ar.tri.te}{0}
\verb{artrítico}{}{}{}{}{adj.}{Relativo a artrite.}{ar.trí.ti.co}{0}
\verb{artrítico}{}{}{}{}{}{Que sofre de artrite.}{ar.trí.ti.co}{0}
\verb{artrítico}{}{}{}{}{s.m.}{Indivíduo que sofre de artrite.}{ar.trí.ti.co}{0}
%\verb{artritismo}{}{}{}{}{}{0}{ar.tri.tis.mo}{0}
\verb{artrópode}{}{}{}{}{adj.2g.}{Relativo aos artrópodes, filo de animais invertebrados de corpo segmentado e membros articulados como insetos, crustáceos, aracnídeos etc.}{ar.tró.po.de}{0}
\verb{artrose}{ó}{Med.}{}{}{s.f.}{Distúrbio articular crônico, doloroso, deformante, inflamatório ou não, que se caracteriza pela destruição da cartilagem e de seus constituintes ósseos; processo degenerativo de uma articulação.}{ar.tro.se}{0}
\verb{aruaque}{}{}{}{}{}{}{a.ru.a.que}{0}
\verb{aruaque}{}{}{}{}{}{Var. de  \textit{aruak}}{a.ru.a.que}{0}
\verb{arúspice}{}{}{}{}{s.m.}{Sacerdote romano que adivinhava o futuro pelo exame das entranhas das pessoas sacrificadas.}{a.rús.pi.ce}{0}
\verb{arvoado}{}{}{}{}{adj.}{Que perdeu o juízo; desatinado, desarvoado.}{ar.vo.a.do}{0}
\verb{arvorado}{}{}{}{}{adj.}{Que foi plantado, arborizado.}{ar.vo.ra.do}{0}
\verb{arvorado}{}{}{}{}{}{Erguido, levantado, hasteado.}{ar.vo.ra.do}{0}
\verb{arvorado}{}{}{}{}{}{Ostentado, alardeado.}{ar.vo.ra.do}{0}
\verb{arvorar}{}{}{}{}{v.t.}{Plantar árvores em; arborizar.}{ar.vo.rar}{0}
\verb{arvorar}{}{}{}{}{}{Elevar, hastear, levantar.}{ar.vo.rar}{0}
\verb{arvorar}{}{}{}{}{}{Ostentar, alardear.}{ar.vo.rar}{\verboinum{1}}
\verb{árvore}{}{}{}{}{s.f.}{Vegetal lenhoso com caule principal (tronco) ereto e que se ramifica bem acima do solo, formando uma copa.}{ár.vo.re}{0}
\verb{arvoredo}{ê}{}{}{}{s.m.}{Extenso aglomerado de árvores.}{ar.vo.re.do}{0}
\verb{arvoreta}{ê}{}{}{}{s.f.}{Diminutivo irregular de \textit{árvore}; árvore pequena.}{ar.vo.re.ta}{0}
\verb{arvorismo}{}{Esport.}{}{}{s.m.}{Modalidade esportiva que consiste na travessia de um percurso acrobático realizado próximo à copa das árvores, utilizando"-se diversas técnicas verticais, como a tirolesa e o rapel.}{ar.vo.ris.mo}{0}
\verb{ás}{}{}{}{}{s.m.}{Carta do baralho marcada por um só ponto.}{ás}{0}
\verb{ás}{}{Fig.}{}{}{}{Indivíduo muito habilidoso em determinada atividade.}{ás}{0}
\verb{às}{}{}{}{}{}{Combinação da preposição \textit{a} com o artigo \textit{as}.}{às}{0}
\verb{às}{}{}{}{}{}{Combinação da preposição \textit{a} com o pronome demonstrativo \textit{as}.}{às}{0}
\verb{As}{}{Quím.}{}{}{}{Símb. do \textit{arsênio}.}{As}{0}
\verb{asa}{}{}{}{}{s.f.}{Membro anterior das aves e morcegos especialmente modificado para o voo.}{a.sa}{0}
\verb{asa}{}{}{}{}{}{Apêndice membranoso fixado ao tórax dos insetos.}{a.sa}{0}
\verb{asa}{}{}{}{}{}{Nadadeira peitoral de certos peixes.}{a.sa}{0}
\verb{asa}{}{}{}{}{}{Superfície horizontal de um avião a qual lhe dá sustentação.}{a.sa}{0}
\verb{asa}{}{Por ext.}{}{}{}{Parte saliente de certos objetos e utensílios, a qual facilita seu manuseio.}{a.sa}{0}
\verb{asa}{}{}{}{}{}{Nave central de uma igreja.}{a.sa}{0}
\verb{asa}{}{Por ext.}{}{}{}{Qualquer coisa que se assemelha a uma asa.}{a.sa}{0}
\verb{asa"-delta}{é}{Esport.}{}{}{s.f.}{Equipamento constituído de uma estrutura triangular revestida de tecido fino e não poroso, utilizado na prática de voo livre.}{a.sa"-del.ta}{0}
%\verb{asado}{}{}{}{}{}{0}{a.sa.do}{0}
%\verb{asado}{}{}{}{}{}{0}{a.sa.do}{0}
\verb{asa"-negra}{ê}{}{}{}{s.2g.}{Indivíduo que costuma prejudicar os outros.}{a.sa"-ne.gra}{0}
\verb{asbesto}{é}{}{}{}{s.m.}{Material fibroso incombustível e infusível utilizado como isolante térmico, acústico, elétrico; amianto é um tipo de asbesto.}{as.bes.to}{0}
\verb{ascendência}{}{}{}{}{s.f.}{Série de gerações anteriores a um indivíduo; antepassados.}{as.cen.dên.cia}{0}
\verb{ascendência}{}{}{}{}{}{Influência ou hegemonia exercida sobre outrem.}{as.cen.dên.cia}{0}
\verb{ascendência}{}{Desus.}{}{}{}{Ação de subir.}{as.cen.dên.cia}{0}
\verb{ascendente}{}{}{}{}{adj.2g.}{Que sobe, eleva"-se, cresce ou aumenta.}{as.cen.den.te}{0}
\verb{ascendente}{}{}{}{}{s.2g.}{Antepassado.}{as.cen.den.te}{0}
\verb{ascendente}{}{Astrol.}{}{}{s.m.}{Astro do zodíaco que se encontra na posição do horizonte oriental no momento do nascimento de alguém.}{as.cen.den.te}{0}
\verb{ascender}{ê}{}{}{}{v.t.}{Subir, elevar, aumentar.}{as.cen.der}{0}
\verb{ascender}{ê}{Fig.}{}{}{}{Elevar"-se em posição hierárquica ou social. (\textit{O rapaz ascendeu rapidamente na carreira profissional.})}{as.cen.der}{\verboinum{12}}
\verb{ascensão}{}{}{"-ões}{}{s.f.}{Ato ou efeito de ascender física, social ou hierarquicamente.}{as.cen.são}{0}
\verb{ascensão}{}{Relig.}{"-ões}{}{}{No cristianismo, subida ao céu e glorificação de Jesus Cristo após sua morte.}{as.cen.são}{0}
\verb{ascensão}{}{Por ext.}{"-ões}{}{}{A festa comemorativa desse evento; o dia dessa festa.}{as.cen.são}{0}
\verb{ascensional}{}{}{"-ais}{}{adj.2g.}{Relativo a ascensão.}{as.cen.si.o.nal}{0}
\verb{ascensional}{}{}{"-ais}{}{}{Que tende a ou faz subir.}{as.cen.si.o.nal}{0}
\verb{ascensionista}{}{}{}{}{adj.2g.}{Que anda de balão ou sobe montanhas.}{as.cen.si.o.nis.ta}{0}
\verb{ascensionista}{}{}{}{}{s.2g.}{Indivíduo que anda de balão ou sobe montanhas.}{as.cen.si.o.nis.ta}{0}
\verb{ascensor}{ô}{}{}{}{adj.}{Que ascende.}{as.cen.sor}{0}
\verb{ascensor}{ô}{}{}{}{s.m.}{Máquina com cabina ou plataforma que se desloca na vertical para levar pessoas ou cargas aos diversos pavimentos de um edifício; elevador.}{as.cen.sor}{0}
\verb{ascensorista}{}{}{}{}{adj.2g.}{Que maneja o elevador ou ascensor.}{as.cen.so.ris.ta}{0}
\verb{ascensorista}{}{}{}{}{s.2g.}{Indivíduo encarregado de manejar o elevador ou ascensor.}{as.cen.so.ris.ta}{0}
\verb{ascese}{é}{}{}{}{s.m.}{Exercício prático de devoção, disciplina e autocontrole do corpo e do espírito, que leva à efetiva realização da virtude, à plenitude da vida moral.}{as.ce.se}{0}
\verb{asceta}{é}{}{}{}{s.2g.}{Pessoa que se consagra à ascese, vive em práticas de devoção.}{as.ce.ta}{0}
\verb{ascético}{}{}{}{}{adj.}{Relativo aos ascetas ou ao ascetismo.}{as.cé.ti.co}{0}
\verb{ascético}{}{}{}{}{}{Que é devoto, contemplativo, místico.}{as.cé.ti.co}{0}
\verb{ascetismo}{}{}{}{}{s.m.}{Doutrina que considera a ascese, isto é, a disciplina e o autocontrole do corpo e do espírito, o essencial da vida moral, um caminho em direção a Deus, à verdade ou à virtude.}{as.ce.tis.mo}{0}
\verb{ascetismo}{}{}{}{}{}{Conjunto de práticas e comportamentos determinados por essa doutrina. }{as.ce.tis.mo}{0}
\verb{ascite}{}{}{}{}{s.f.}{Acúmulo de líquido na cavidade abdominal; barriga"-d'água.}{as.ci.te}{0}
\verb{asco}{}{}{}{}{s.m.}{Repugnância, nojo, enjoo.}{as.co}{0}
\verb{asco}{}{Biol.}{}{}{}{Órgão em forma de saco, no interior dos quais se formam esporos sexuais nos fungos e liquens.}{as.co}{0}
\verb{ascórbico}{}{Quím.}{}{}{adj.}{Diz"-se de ácido encontrado em vegetais verdes e frutas cítricas, usado na prevenção do escorbuto e da deficiência de vitamina \textsc{c}.}{as.cór.bi.co}{0}
\verb{aselha}{ê}{}{}{}{s.f.}{Pequena alça de cadarço que, colocada na parte superior traseira do cano das botinas, é usada como puxadeira.}{a.se.lha}{0}
\verb{aselha}{ê}{}{}{}{}{Aro pequeno que serve, na roupa, para o encaixe de botão ou colchete; casa. }{a.se.lha}{0}
\verb{aselha}{ê}{}{}{}{}{Presilha de cordel ou fita; alça. }{a.se.lha}{0}
\verb{asfaltamento}{}{}{}{}{s.m.}{Ato ou efeito de asfaltar, de cobrir ou revestir de asfalto.}{as.fal.ta.men.to}{0}
\verb{asfaltar}{}{}{}{}{v.t.}{Cobrir ou revestir de asfalto.}{as.fal.tar}{\verboinum{1}}
\verb{asfalto}{}{}{}{}{s.m.}{Variedade de betume, que resulta da destilação do petróleo bruto, usado na pavimentação de estradas e impermeabilização.}{as.fal.to}{0}
\verb{asfalto}{}{Fig.}{}{}{}{A rua, a estrada.}{as.fal.to}{0}
\verb{asfixia}{cs}{Med.}{}{}{s.f.}{Dificuldade ou impossibilidade de respirar; sufocação.}{as.fi.xi.a}{0}
\verb{asfixiante}{cs}{}{}{}{adj.2g.}{Que asfixia, sufoca.}{as.fi.xi.an.te}{0}
\verb{asfixiar}{cs}{}{}{}{v.t.}{Causar asfixia; sufocar.}{as.fi.xi.ar}{0}
\verb{asfixiar}{cs}{}{}{}{}{Matar por asfixia.}{as.fi.xi.ar}{0}
\verb{asfixiar}{cs}{Fig.}{}{}{}{Impor limites; tolher.}{as.fi.xi.ar}{0}
\verb{asfixiar}{cs}{}{}{}{v.i.}{Não poder respirar livremente.}{as.fi.xi.ar}{\verboinum{1}}
\verb{asiático}{}{}{}{}{adj.}{Relativo à Ásia.}{a.si.á.ti.co}{0}
\verb{asiático}{}{}{}{}{}{Diz"-se do luxo excessivo.}{a.si.á.ti.co}{0}
\verb{asiático}{}{}{}{}{s.m.}{Indivíduo natural ou habitante do continente asiático.}{a.si.á.ti.co}{0}
\verb{asilado}{}{}{}{}{adj.}{Diz"-se de pessoa que está recolhida ou abrigada em asilo.}{a.si.la.do}{0}
\verb{asilado}{}{}{}{}{s.m.}{Indivíduo a quem se deu asilo.}{a.si.la.do}{0}
\verb{asilar}{}{}{}{}{v.t.}{Recolher em asilo; abrigar.}{a.si.lar}{\verboinum{1}}
\verb{asilo}{}{}{}{}{s.m.}{Instituição de assistência social que abriga pobres e desamparados, como crianças, mendigos, velhos.}{a.si.lo}{0}
\verb{asilo}{}{Por ext.}{}{}{}{Proteção, amparo.}{a.si.lo}{0}
\verb{asinino}{}{}{}{}{adj.}{Relativo a asno.}{a.si.ni.no}{0}
\verb{asinino}{}{Fig.}{}{}{}{Desprovido de inteligência; estúpido.}{a.si.ni.no}{0}
\verb{asma}{}{Med.}{}{}{s.f.}{Doença respiratória caracterizada por crises de apneia e espasmos dos brônquios.}{as.ma}{0}
\verb{asmático}{}{}{}{}{adj.}{Relativo a asma.}{as.má.ti.co}{0}
\verb{asmático}{}{}{}{}{s.m.}{Indivíduo que sofre de asma.}{as.má.ti.co}{0}
\verb{asnático}{}{}{}{}{adj.}{Asinino.}{as.ná.ti.co}{0}
\verb{asnear}{}{}{}{}{v.i.}{Dizer ou fazer asneiras; bobear.}{as.ne.ar}{0}
\verb{asnear}{}{}{}{}{}{Mostrar"-se presunçoso.}{as.ne.ar}{\verboinum{4}}
\verb{asneira}{ê}{}{}{}{s.f.}{Ato ou dito tolo ou impensado; bobagem, tolice, disparate, burrice.}{as.nei.ra}{0}
\verb{asneira}{ê}{}{}{}{}{Ato ou palavra obscena.}{as.nei.ra}{0}
\verb{asneirão}{}{}{"-ões}{}{s.m.}{Grande asno; parvalhão.}{as.nei.rão}{0}
\verb{asneirento}{}{}{}{}{adj.}{Que diz asneiras.}{as.nei.ren.to}{0}
\verb{asnice}{}{}{}{}{s.f.}{Asneira. }{as.ni.ce}{0}
\verb{asno}{}{}{}{}{s.m.}{Jumento.}{as.no}{0}
\verb{asno}{}{Fig.}{}{}{}{Indivíduo desprovido de inteligência; burro.}{as.no}{0}
%\verb{aspa}{}{}{}{}{}{0}{as.pa}{0}
\verb{aspar}{}{}{}{}{v.t.}{Pôr entre aspas; aspear.}{as.par}{\verboinum{1}}
\verb{aspargo}{}{Bot.}{}{}{s.m.}{Planta originária da África e da Ásia, cultivada pelas raízes medicinais e pelos brotos carnosos, usados como alimento.}{as.par.go}{0}
\verb{aspas}{}{}{}{}{s.f.pl.}{Sinais de pontuação com que se abre e fecha uma citação, expressões pouco usadas, sentido figurado, título de obra.}{as.pas}{0}
\verb{aspear}{}{}{}{}{v.t.}{Colocar entre aspas; aspar.}{as.pe.ar}{\verboinum{4}}
\verb{aspecto}{é}{}{}{}{s.m.}{Maneira pela qual alguém ou alguma coisa se apresenta à vista; aparência, semblante, fisionomia.}{as.pec.to}{0}
\verb{aspecto}{é}{}{}{}{}{Maneira de ver, ponto de vista; ângulo.}{as.pec.to}{0}
\verb{aspereza}{ê}{}{}{}{s.f.}{Qualidade do que é áspero, do que não é liso e uniforme.}{as.pe.re.za}{0}
\verb{aspereza}{ê}{Fig.}{}{}{}{Severidade, rispidez.}{as.pe.re.za}{0}
\verb{aspergir}{}{}{}{}{v.t.}{Borrifar, molhar superficialmente com gotas de água ou de outro líquido.}{as.per.gir}{\verboinum{25}}
\verb{áspero}{}{}{}{}{adj.}{Que tem a superfície desigual; acidentado, irregular.}{ás.pe.ro}{0}
\verb{áspero}{}{}{}{}{}{Desagradável ao paladar; acre, azedo.}{ás.pe.ro}{0}
\verb{áspero}{}{Fig.}{}{}{}{Rude, ríspido, duro.}{ás.pe.ro}{0}
\verb{aspersão}{}{}{"-ões}{}{s.f.}{Ato ou efeito de aspergir; respingo, borrifo.}{as.per.são}{0}
\verb{aspersório}{}{}{}{}{s.m.}{Instrumento de metal ou de madeira usado para aspergir água benta; hissope.}{as.per.só.rio}{0}
\verb{aspeto}{é}{}{}{}{}{Var. de \textit{aspecto}.}{as.pe.to}{0}
\verb{áspide}{}{Zool.}{}{}{s.2g.}{Pequena cobra venenosa, semelhante à víbora, de corpo marrom com estrias negras no dorso.}{ás.pi.de}{0}
\verb{aspiração}{}{}{"-ões}{}{s.f.}{Ato ou efeito de aspirar; absorção.}{as.pi.ra.ção}{0}
\verb{aspiração}{}{Fig.}{"-ões}{}{}{Desejo profundo de alcançar um objetivo.}{as.pi.ra.ção}{0}
\verb{aspirado}{}{}{}{}{adj.}{Que se aspirou; sorvido, absorvido.}{as.pi.ra.do}{0}
\verb{aspirado}{}{Gram.}{}{}{}{Diz"-se do fonema que se pronuncia de modo gutural.}{as.pi.ra.do}{0}
\verb{aspirador}{ô}{}{}{}{adj.}{Que produz aspiração.}{as.pi.ra.dor}{0}
\verb{aspirador}{ô}{}{}{}{s.m.}{Aparelho eletrodoméstico usado para aspirar pó, partículas de sujeira, areia.}{as.pi.ra.dor}{0}
\verb{aspirante}{}{}{}{}{adj.2g.}{Que aspira, suga, absorve.}{as.pi.ran.te}{0}
\verb{aspirante}{}{}{}{}{s.2g.}{Pessoa que aspira a um cargo, título ou dignidade.}{as.pi.ran.te}{0}
\verb{aspirante}{}{}{}{}{}{Militar que detém a posição hierárquica de aspirante.}{as.pi.ran.te}{0}
\verb{aspirante}{}{}{}{}{}{Aluno da Escola Naval.}{as.pi.ran.te}{0}
\verb{aspirar}{}{}{}{}{v.t.}{Atrair o ar aos pulmões; inspirar, respirar.}{as.pi.rar}{0}
\verb{aspirar}{}{}{}{}{}{Sugar, chupar, absorver.}{as.pi.rar}{0}
\verb{aspirar}{}{}{}{}{}{Desejar profundamente; pretender.}{as.pi.rar}{0}
\verb{aspirar}{}{Gram.}{}{}{}{Pronunciar um som de modo gutural.}{as.pi.rar}{\verboinum{1}}
\verb{aspirina}{}{Quím.}{}{}{s.f.}{Medicamento analgésico e antitérmico; ácido acetilsalicílico.}{as.pi.ri.na}{0}
\verb{asqueroso}{ô}{}{"-osos ⟨ó⟩}{"-osa ⟨ó⟩}{adj.}{Que causa asco, nojo; repugnante.}{as.que.ro.so}{0}
\verb{assacar}{}{}{}{}{v.t.}{Atribuir sem fundamento, inventar ou espalhar calúnias.}{as.sa.car}{\verboinum{2}}
\verb{assadeira}{ê}{}{}{}{s.f.}{Forma onde se põe alimento para assar.}{as.sa.dei.ra}{0}
\verb{assado}{}{}{}{}{adj.}{Que se assou.}{as.sa.do}{0}
\verb{assado}{}{}{}{}{}{Diz"-se de pele que apresenta assadura.}{as.sa.do}{0}
\verb{assado}{}{}{}{}{s.m.}{Qualquer alimento assado, principalmente carne.}{as.sa.do}{0}
\verb{assadura}{}{}{}{}{s.f.}{Ato ou efeito de assar, de submeter à ação do fogo.}{as.sa.du.ra}{0}
\verb{assadura}{}{}{}{}{}{Irritação na pele devido a atrito, calor.}{as.sa.du.ra}{0}
\verb{assalariado}{}{}{}{}{adj.}{Que é pago ou remunerado com salário.}{as.sa.la.ri.a.do}{0}
\verb{assalariado}{}{}{}{}{s.m.}{Indivíduo que trabalha mediante salário.}{as.sa.la.ri.a.do}{0}
\verb{assalariar}{}{}{}{}{v.t.}{Dar salário; contratar.}{as.sa.la.ri.ar}{0}
\verb{assalariar}{}{Pop.}{}{}{}{Subornar, corromper.}{as.sa.la.ri.ar}{\verboinum{1}}
\verb{assaltante}{}{}{}{}{adj.2g.}{Que assalta, que ataca de repente.}{as.sal.tan.te}{0}
\verb{assaltante}{}{}{}{}{s.2g.}{Indivíduo que ataca para roubar.}{as.sal.tan.te}{0}
\verb{assaltar}{}{}{}{}{v.t.}{Atacar de repente; investir de súbito.}{as.sal.tar}{0}
\verb{assaltar}{}{}{}{}{}{Atacar para roubar.}{as.sal.tar}{0}
\verb{assaltar}{}{}{}{}{}{Lembrar repentinamente; ocorrer.}{as.sal.tar}{0}
\verb{assaltar}{}{}{}{}{}{Investir, assediar, surpreender.}{as.sal.tar}{\verboinum{1}}
\verb{assalto}{}{}{}{}{s.m.}{Ato ou efeito de assaltar; investida impetuosa.}{as.sal.to}{0}
\verb{assalto}{}{}{}{}{}{Ataque inesperado, violento, com o  intuito de roubar.}{as.sal.to}{0}
\verb{assalto}{}{Esport.}{}{}{}{No boxe, cada um dos períodos de tempo em que se divide uma luta.}{as.sal.to}{0}
\verb{ássana}{}{Filos.}{}{}{s.m.}{Cada uma das posturas corporais da ioga.}{ás.sa.na}{0}
\verb{assanhado}{}{}{}{}{adj.}{Que tem muita sanha; enfurecido, furioso.}{as.sa.nha.do}{0}
\verb{assanhado}{}{}{}{}{}{Diz"-se do cabelo despenteado, desgrenhado.}{as.sa.nha.do}{0}
\verb{assanhado}{}{Pop.}{}{}{}{Namorador, erótico, fogoso.}{as.sa.nha.do}{0}
\verb{assanhamento}{}{}{}{}{s.m.}{Ato ou efeito de assanhar; irritação, fúria, sanha.}{as.sa.nha.men.to}{0}
\verb{assanhamento}{}{Pop.}{}{}{}{Excitação erótica.}{as.sa.nha.men.to}{0}
\verb{assanhar}{}{}{}{}{v.t.}{Provocar a sanha; irritar, enfurecer.}{as.sa.nhar}{0}
\verb{assanhar}{}{}{}{}{}{Tornar emaranhado; descabelar.}{as.sa.nhar}{0}
\verb{assanhar}{}{}{}{}{v.pron.}{Proceder com falta de compostura; alvoroçar.}{as.sa.nhar}{0}
\verb{assanhar}{}{Pop.}{}{}{}{Oferecer"-se eroticamente.}{as.sa.nhar}{\verboinum{1}}
\verb{assar}{}{}{}{}{v.t.}{Submeter à ação do fogo até ficar cozido ou tostado.}{as.sar}{0}
\verb{assar}{}{}{}{}{}{Consumir em chamas; queimar.}{as.sar}{0}
\verb{assar}{}{}{}{}{}{Causar assadura, irritar a pele.}{as.sar}{\verboinum{1}}
\verb{assassinar}{}{}{}{}{v.t.}{Matar com violência.}{as.sas.si.nar}{0}
\verb{assassinar}{}{Fig.}{}{}{}{Destruir, aniquilar.}{as.sas.si.nar}{\verboinum{1}}
\verb{assassinato}{}{}{}{}{s.m.}{Ato ou efeito de matar alguém, homicídio voluntário; assassínio.}{as.sas.si.na.to}{0}
\verb{assassínio}{}{}{}{}{s.m.}{Assassinato.}{as.sas.sí.nio}{0}
\verb{assassino}{}{}{}{}{s.m.}{Indivíduo que comete homicídio, que tira a vida de alguém.}{as.sas.si.no}{0}
\verb{assassino}{}{}{}{}{}{Indivíduo que causa perda ou ruína.}{as.sas.si.no}{0}
\verb{assaz}{}{}{}{}{adv.}{Bastante, suficientemente; muito.}{as.saz}{0}
\verb{asseado}{}{}{}{}{adj.}{Limpo, claro.}{as.se.a.do}{0}
\verb{assear}{}{}{}{}{v.t.}{Limpar, cuidar, enfeitar, vestir com esmero.}{as.se.ar}{\verboinum{4}}
\verb{assecla}{é}{}{}{}{s.m.}{Indivíduo que segue uma seita, um partido, uma ideia, cega e subservientemente; partidário.}{as.se.cla}{0}
\verb{assediar}{}{}{}{}{v.t.}{Cercar, rodear, sitiar.}{as.se.di.ar}{0}
\verb{assediar}{}{}{}{}{}{Importunar com perguntas, propostas, pedidos insistentes.}{as.se.di.ar}{\verboinum{1}}
\verb{assédio}{}{}{}{}{s.m.}{Cerco, sítio.}{as.sé.dio}{0}
\verb{assédio}{}{}{}{}{}{Perseguição, insistência importuna.}{as.sé.dio}{0}
\verb{assegurado}{}{}{}{}{adj.}{Que se tornou seguro, garantido, firme. }{as.se.gu.ra.do}{0}
\verb{assegurar}{}{}{}{}{v.t.}{Declarar com certeza; afirmar, garantir.}{as.se.gu.rar}{0}
\verb{assegurar}{}{}{}{}{v.pron.}{Certificar"-se, tornar"-se seguro.}{as.se.gu.rar}{\verboinum{1}}
\verb{asseio}{ê}{}{}{}{s.m.}{Limpeza, higiene, esmero, cuidado no vestir"-se, pentear"-se.}{as.sei.o}{0}
\verb{asselvajado}{}{}{}{}{adj.}{Que tem modos ou hábitos de selvagem.}{as.sel.va.ja.do}{0}
\verb{asselvajar}{}{}{}{}{v.t.}{Tornar selvagem, rude.}{as.sel.va.jar}{\verboinum{1}}
\verb{assembleia}{é}{}{}{}{s.f.}{Reunião de muitas pessoas para um fim determinado.}{as.sem.blei.a}{0}
\verb{assembleia}{é}{}{}{}{}{Corpo político e legislativo; congresso, parlamento.}{as.sem.blei.a}{0}
\verb{assemelhar}{}{}{}{}{v.t.}{Ser ou tornar semelhante, parecido; ter semelhança, parecer,}{as.se.me.lhar}{\verboinum{1}}
\verb{assenhorear"-se}{}{}{}{}{v.pron.}{Dominar como senhor ou dono; apoderar"-se, apossar"-se, ocupar.}{as.se.nho.re.ar"-se}{\verboinum{4}}
\verb{assentada}{}{Jur.}{}{}{s.f.}{Sessão de um tribunal para discussão das causas ou audiências de testemunhas.}{as.sen.ta.da}{0}
\verb{assentada}{}{}{}{}{}{Ato ou efeito de assentar, acomodar.}{as.sen.ta.da}{0}
\verb{assentado}{}{}{}{}{adj.}{Que está colocado de maneira firme;  sentado; estável.}{as.sen.ta.do}{0}
\verb{assentado}{}{}{}{}{}{Sossegado, ajuizado, acomodado.}{as.sen.ta.do}{0}
\verb{assentador}{ô}{}{}{}{s.m.}{Indivíduo que faz assentamentos; que dispõe peças de um aparelho ou construção no devido lugar.   }{as.sen.ta.dor}{0}
\verb{assentamento}{}{}{}{}{s.m.}{Ato ou efeito de assentar.}{as.sen.ta.men.to}{0}
\verb{assentamento}{}{}{}{}{}{Registro, nota por escrito, averbação.}{as.sen.ta.men.to}{0}
\verb{assentamento}{}{}{}{}{}{Ajustamento, colocação de peças de um aparelho ou construção no devido lugar.}{as.sen.ta.men.to}{0}
\verb{assentamento}{}{}{}{}{}{Fixação ou estabelecimento de residência em determinado lugar; acomodação.}{as.sen.ta.men.to}{0}
\verb{assentar}{}{}{}{}{v.t.}{Colocar, dispor, aplicar de maneira firme, adaptada, segura.}{as.sen.tar}{0}
\verb{assentar}{}{}{}{}{}{Anotar, registrar, inscrever, averbar.}{as.sen.tar}{0}
\verb{assentar}{}{}{}{}{}{Instalar, acomodar, fixar residência em determinado lugar; basear.}{as.sen.tar}{\verboinum{1}}
\verb{assente}{}{}{}{}{adj.2g.}{Que está decidido, resolvido, estabelecido, firme, assentado.}{as.sen.te}{0}
\verb{assentimento}{}{}{}{}{s.m.}{Ato ou efeito de consentir, assentir.}{as.sen.ti.men.to}{0}
\verb{assentimento}{}{}{}{}{}{Anuência, aprovação.}{as.sen.ti.men.to}{0}
\verb{assentir}{}{}{}{}{v.t.}{Dar consentimento; concordar, anuir, aprovar, convir.}{as.sen.tir}{\verboinum{29}}
\verb{assento}{}{}{}{}{s.m.}{Lugar ou móvel em que se senta}{as.sen.to}{0}
\verb{assento}{}{}{}{}{}{Lugar em que algo está assentado; base, apoio.}{as.sen.to}{0}
\verb{assento}{}{}{}{}{}{Tampo de cadeira, banco, vaso sanitário.}{as.sen.to}{0}
\verb{assento}{}{}{}{}{}{Anotação, registro.}{as.sen.to}{0}
\verb{assento}{}{Pop.}{}{}{}{Nádegas.}{as.sen.to}{0}
\verb{assepsia}{}{Med.}{}{}{s.f.}{Ausência de germes patogênicos, para evitar infecção.}{as.sep.si.a}{0}
\verb{assepsia}{}{Med.}{}{}{}{Conjunto de processos que tem por fim impedir a penetração, no organismo, de germes capazes de produzir infecção.}{as.sep.si.a}{0}
\verb{asséptico}{}{}{}{}{adj.}{Relativo a assepsia.}{as.sép.ti.co}{0}
\verb{asséptico}{}{}{}{}{}{Isento de germes; extremamente limpo; desinfetado.}{as.sép.ti.co}{0}
\verb{asserção}{}{}{"-ões}{}{s.f.}{Afirmação, declaração; frase considerada verdadeira.}{as.ser.ção}{0}
\verb{assertar}{}{}{}{}{v.t.}{Fazer asserto, afirmação; asseverar.}{as.ser.tar}{\verboinum{1}}
\verb{assertiva}{}{}{}{}{s.f.}{Afirmativa; asserção, asserto. }{as.ser.ti.va}{0}
\verb{assertivo}{}{}{}{}{adj.}{Que afirma; afirmativo, assertivo.}{as.ser.ti.vo}{0}
\verb{assertivo}{}{}{}{}{}{Em que há afirmação.}{as.ser.ti.vo}{0}
\verb{asserto}{ê}{}{}{}{s.m.}{Afirmação, asserção.}{as.ser.to}{0}
\verb{asserto}{ê}{Filos.}{}{}{}{Proposição afirmativa.}{as.ser.to}{0}
\verb{assessor}{ô}{}{}{}{s.m.}{Indivíduo que auxilia um chefe ou órgão; assistente, adjunto.}{as.ses.sor}{0}
\verb{assessorar}{}{}{}{}{v.t.}{Dar assistência; servir de assessor; auxiliar.}{as.ses.so.rar}{\verboinum{1}}
\verb{assessoria}{}{}{}{}{s.f.}{Ato ou afeito de assessorar, dar auxílio técnico.}{as.ses.so.ri.a}{0}
\verb{assessoria}{}{}{}{}{}{Conjunto de pessoas que dá assistência, fornece informações, dá apoio técnico a um chefe.}{as.ses.so.ri.a}{0}
\verb{assessoria}{}{}{}{}{}{Escritório especialista em um assunto e que presta serviço de assistência a pessoas ou instituições.}{as.ses.so.ri.a}{0}
\verb{assessório}{}{}{}{}{adj.}{Relativo a assessor ou a assessoria.}{as.ses.só.rio}{0}
\verb{assestar}{}{}{}{}{v.t.}{Apontar, dirigir.}{as.ses.tar}{\verboinum{1}}
\verb{asseveração}{}{}{"-ões}{}{s.f.}{Ato ou efeito de asseverar, assegurar, atestar, dar como certo; afirmação.}{as.se.ve.ra.ção}{0}
\verb{asseverar}{}{}{}{}{v.t.}{Dar como certo; afirmar com segurança, assegurar; atestar.}{as.se.ve.rar}{\verboinum{1}}
\verb{assexuado}{cs}{}{}{}{adj.}{Que não tem órgãos sexuais.}{as.se.xu.a.do}{0}
\verb{assexuado}{cs}{Biol.}{}{}{}{Diz"-se da reprodução que se efetua sem a intervenção dos dois sexos.}{as.se.xu.a.do}{0}
\verb{assexuado}{cs}{Fig.}{}{}{}{Diz"-se do indivíduo que aparentemente não tem vida sexual, ou por ela não tem interesse.}{as.se.xu.a.do}{0}
\verb{assexual}{cs}{}{"-ais}{}{adj.2g.}{Diz"-se da pessoa que se abstém de toda a atividade sexual quer por decisão própria quer por necessidade; assexuado.}{as.se.xu.al}{0}
\verb{assiduidade}{}{}{}{}{s.f.}{Qualidade de assíduo; frequência, constância.}{as.si.dui.da.de}{0}
\verb{assíduo}{}{}{}{}{adj.}{Que se faz presente constantemente em determinado lugar.}{as.sí.du.o}{0}
\verb{assíduo}{}{}{}{}{}{Que não falta às suas obrigações.}{as.sí.du.o}{0}
\verb{assíduo}{}{}{}{}{}{Que não sofre interrupção; contínuo.}{as.sí.du.o}{0}
\verb{assim}{}{}{}{}{adv.}{Deste, desse, ou daquele modo.}{as.sim}{0}
\verb{assim}{}{}{}{}{}{Com características semelhantes; de natureza igual.}{as.sim}{0}
\verb{assim}{}{}{}{}{}{Do mesmo modo; igualmente.}{as.sim}{0}
\verb{assim}{}{}{}{}{}{Deste tamanho, desta quantidade.}{as.sim}{0}
\verb{assim}{}{}{}{}{conj.}{Portanto, assim sendo.}{as.sim}{0}
\verb{assimetria}{}{}{}{}{s.f.}{Ausência de proporções regulares.}{as.si.me.tri.a}{0}
\verb{assimétrico}{}{}{}{}{adj.}{Que não tem simetria; desigual.}{as.si.mé.tri.co}{0}
\verb{assimilação}{}{}{"-ões}{}{s.f.}{Ato ou efeito de assimilar, de tornar semelhante ou igual.}{as.si.mi.la.ção}{0}
\verb{assimilação}{}{Biol.}{"-ões}{}{}{Propriedade que possui o organismo vivo de regenerar sua matéria viva a partir de substâncias simples, obtendo, assim, a energia necessária ao seu funcionamento.}{as.si.mi.la.ção}{0}
\verb{assimilação}{}{Gram.}{"-ões}{}{}{Comunicação de traços de um fonema a outro vizinho.}{as.si.mi.la.ção}{0}
\verb{assimilação}{}{}{"-ões}{}{}{Processo de fusão de culturas em um tipo cultural comum.}{as.si.mi.la.ção}{0}
\verb{assimilado}{}{}{}{}{adj.}{Que sofreu assimilação.}{as.si.mi.la.do}{0}
\verb{assimilado}{}{}{}{}{}{Em que ocorreu assimilação.}{as.si.mi.la.do}{0}
\verb{assimilador}{ô}{}{}{}{adj.}{Que produz assimilação.}{as.si.mi.la.dor}{0}
\verb{assimilador}{ô}{}{}{}{}{Que assimila.}{as.si.mi.la.dor}{0}
\verb{assimilar}{}{}{}{}{v.t.}{Tornar semelhante ou igual; identificar.}{as.si.mi.lar}{0}
\verb{assimilar}{}{}{}{}{}{Compenetrar"-se; fixar.}{as.si.mi.lar}{0}
\verb{assimilar}{}{}{}{}{}{Estabelecer comparação.}{as.si.mi.lar}{0}
\verb{assimilar}{}{Biol.}{}{}{}{Regenerar, reconstituir a própria matéria, substância.}{as.si.mi.lar}{0}
\verb{assimilar}{}{Gram.}{}{}{}{Adotar traços do fonema vizinho.}{as.si.mi.lar}{0}
\verb{assimilar}{}{}{}{}{}{Fazer fusão de culturas; apropriar"-se.}{as.si.mi.lar}{\verboinum{1}}
\verb{assinado}{}{}{}{}{adj.}{Em que há assinatura.}{as.si.na.do}{0}
\verb{assinado}{}{}{}{}{s.m.}{Documento autenticado com assinatura.}{as.si.na.do}{0}
\verb{assinalado}{}{}{}{}{adj.}{Que tem ou leva sinal; marcado.}{as.si.na.la.do}{0}
\verb{assinalado}{}{}{}{}{}{Designado, apontado.}{as.si.na.la.do}{0}
\verb{assinalado}{}{}{}{}{}{Que se distinguiu; ilustre, notável.}{as.si.na.la.do}{0}
\verb{assinalar}{}{}{}{}{v.t.}{Marcar com sinal.}{as.si.na.lar}{0}
\verb{assinalar}{}{}{}{}{}{Dar sinal, notícia ou conhecimento.}{as.si.na.lar}{0}
\verb{assinalar}{}{}{}{}{}{Designar, apontar.}{as.si.na.lar}{0}
\verb{assinalar}{}{}{}{}{v.pron.}{Distinguir"-se, notabilizar"-se.}{as.si.na.lar}{\verboinum{1}}
\verb{assinante}{}{}{}{}{s.2g.}{Pessoa que faz assinatura; subscritor.}{as.si.nan.te}{0}
\verb{assinar}{}{}{}{}{v.t.}{Pôr seu próprio nome ou sinal; firmar com a própria assinatura.}{as.si.nar}{0}
\verb{assinar}{}{}{}{}{}{Adquirir assinatura, ser assinante.}{as.si.nar}{\verboinum{1}}
\verb{assinatura}{}{}{}{}{s.f.}{Ato ou efeito de assinar; de pôr o próprio nome ou sinal em.}{as.si.na.tu.ra}{0}
\verb{assinatura}{}{}{}{}{}{Nome assinado; firma.}{as.si.na.tu.ra}{0}
\verb{assinatura}{}{}{}{}{}{Contrato pelo qual se adquire o direito de receber por tempo determinado uma publicação, frequentar temporada de espetáculos etc.                   }{as.si.na.tu.ra}{0}
\verb{assincronismo}{}{}{}{}{s.m.}{Falta de sincronismo, de concomitância de coisas ou fenômenos no tempo.}{as.sin.cro.nis.mo}{0}
\verb{assíncrono}{}{}{}{}{adj.}{Que não é sincrônico, não é simultâneo.}{as.sín.cro.no}{0}
\verb{assindético}{}{Gram.}{}{}{adj.}{Em que há ausência de conjuções coordenativas entre frases ou partes da mesma frase.}{as.sin.dé.ti.co}{0}
\verb{assíndeto}{}{Gram.}{}{}{s.m.}{Ausência de conjunção coordenativa entre palavras, termos da oração, ou orações de um período.}{as.sín.de.to}{0}
\verb{assírio}{}{}{}{}{adj.}{Relativo à antiga Assíria (atual Iraque).}{as.sí.rio}{0}
\verb{assírio}{}{}{}{}{s.m.}{Indivíduo natural ou habitante da Assíria.}{as.sí.rio}{0}
\verb{assírio}{}{}{}{}{}{A língua dos assírios.}{as.sí.rio}{0}
\verb{assisado}{}{}{}{}{adj.}{Que tem siso; ajuizado, prudente, sensato.}{as.si.sa.do}{0}
\verb{assistência}{}{}{}{}{s.f.}{Ato ou efeito de assistir.}{as.sis.tên.cia}{0}
\verb{assistência}{}{}{}{}{}{Proteção, amparo, ajuda.}{as.sis.tên.cia}{0}
\verb{assistência}{}{}{}{}{}{Conjunto de assistentes.}{as.sis.tên.cia}{0}
\verb{assistência}{}{}{}{}{}{Instituição governamental que presta socorros médicos ou cirúrgicos, quase sempre gratuitamente; pronto"-socorro.}{as.sis.tên.cia}{0}
\verb{assistencial}{}{}{"-ais}{}{adj.2g.}{Relativo a assistência.}{as.sis.ten.ci.al}{0}
\verb{assistencial}{}{}{"-ais}{}{}{Em que há assistência.}{as.sis.ten.ci.al}{0}
\verb{assistente}{}{}{}{}{adj.2g.}{Que assiste ou dá assistência.}{as.sis.ten.te}{0}
\verb{assistente}{}{}{}{}{s.2g.}{Indivíduo que presencia um evento, cerimônia, ato.}{as.sis.ten.te}{0}
\verb{assistente}{}{}{}{}{}{Indivíduo que dá assistência a doente.}{as.sis.ten.te}{0}
\verb{assistente}{}{}{}{}{}{Auxiliar, adjunto.}{as.sis.ten.te}{0}
\verb{assistente}{}{}{}{}{}{Morador, residente.}{as.sis.ten.te}{0}
\verb{assistir}{}{}{}{}{v.t.}{Ver e ouvir, observar, presenciar.}{as.sis.tir}{0}
\verb{assistir}{}{}{}{}{}{Residir, morar.}{as.sis.tir}{0}
\verb{assistir}{}{}{}{}{}{Ajudar, socorrer (enfermo, pobre, parturiente).}{as.sis.tir}{0}
\verb{assistir}{}{}{}{}{}{Acompanhar na qualidade de ajudante, assistente ou assessor.}{as.sis.tir}{\verboinum{18}}
\verb{assoalhado}{}{}{}{}{adj.}{Que tem soalho.}{as.so.a.lha.do}{0}
\verb{assoalhado}{}{}{}{}{}{Que esteve exposto ao sol.}{as.so.a.lha.do}{0}
\verb{assoalhado}{}{}{}{}{s.m.}{Piso constituído por tábuas, tacos ou outro tipo de revestimento que imite esses materiais; soalho.}{as.so.a.lha.do}{0}
\verb{assoalhar}{}{}{}{}{v.t.}{Pôr soalho de tábuas, tacos ou materiais semelhantes.}{as.so.a.lhar}{0}
\verb{assoalhar}{}{}{}{}{}{Expor ao sol.}{as.so.a.lhar}{\verboinum{1}}
\verb{assoalho}{}{}{}{}{s.m.}{Piso de madeira; soalho.}{as.so.a.lho}{0}
\verb{assoar}{}{}{}{}{v.t.}{Soprar o ar pelo nariz para limpar a mucosidade.}{as.so.ar}{\verboinum{7}}
\verb{assoberbado}{}{}{}{}{adj.}{Que tem modos soberbos; altivo.}{as.so.ber.ba.do}{0}
\verb{assoberbado}{}{}{}{}{}{Que está sobrecarregado de serviços.}{as.so.ber.ba.do}{0}
\verb{assoberbar}{}{}{}{}{v.t.}{Tratar com soberba, desprezo ou arrogância; humilhar.}{as.so.ber.bar}{0}
\verb{assoberbar}{}{}{}{}{}{Tornar orgulhoso.}{as.so.ber.bar}{0}
\verb{assoberbar}{}{}{}{}{}{Sobrecarregar de serviço.}{as.so.ber.bar}{\verboinum{1}}
\verb{assobiar}{}{}{}{}{v.i.}{Produzir um som agudo, soprando o ar entre os lábios.}{as.so.bi.ar}{0}
\verb{assobiar}{}{}{}{}{v.t.}{Executar assobiando.}{as.so.bi.ar}{0}
\verb{assobiar}{}{}{}{}{}{Dirigir apupos; vaiar.}{as.so.bi.ar}{\verboinum{1}}
\verb{assobio}{}{}{}{}{s.m.}{Som agudo e prolongado produzido pelo ar comprimido entre os lábios.}{as.so.bi.o}{0}
\verb{assobio}{}{}{}{}{}{Som agudo produzido por alguns animais; sibilo.}{as.so.bi.o}{0}
\verb{assobio}{}{}{}{}{}{Instrumento para assobiar; apito.}{as.so.bi.o}{0}
\verb{assobradado}{}{}{}{}{adj.}{Diz"-se da casa que tem um outro pavimento por cima do térreo.}{as.so.bra.da.do}{0}
\verb{associação}{}{}{"-ões}{}{s.f.}{Ato ou efeito de associar, de combinar, de aproximar.}{as.so.ci.a.ção}{0}
\verb{associação}{}{}{"-ões}{}{}{Entidade ou agrupamentos de pessoas com objetivos e interesses comuns.}{as.so.ci.a.ção}{0}
\verb{associação}{}{Biol.}{"-ões}{}{}{Conjunto das espécies animais ou vegetais que vivem no mesmo \textit{habitat}.}{as.so.ci.a.ção}{0}
\verb{associação}{}{}{"-ões}{}{}{Qualquer ligação entre dois ou mais elementos psíquicos cujo encadeamento constitui uma cadeia associativa.}{as.so.ci.a.ção}{0}
\verb{associado}{}{}{}{}{adj.}{Diz"-se de pessoa ligada a outra(s) por objetivos ou interesses comuns; membro.}{as.so.ci.a.do}{0}
\verb{associado}{}{}{}{}{s.m.}{Sócio.}{as.so.ci.a.do}{0}
\verb{associar}{}{}{}{}{v.t.}{Pôr junto; reunir, agregar.}{as.so.ci.ar}{0}
\verb{associar}{}{}{}{}{}{Reunir em sociedade.}{as.so.ci.ar}{0}
\verb{associar}{}{}{}{}{}{Tomar como sócio.}{as.so.ci.ar}{\verboinum{1}}
\verb{associativo}{}{}{}{}{adj.}{Relativo a associação.}{as.so.ci.a.ti.vo}{0}
\verb{associativo}{}{}{}{}{}{Que se associa, liga, une.}{as.so.ci.a.ti.vo}{0}
\verb{assolação}{}{}{"-ões}{}{s.f.}{Ato ou efeito de assolar; devastação, destruição.}{as.so.la.ção}{0}
\verb{assolador}{ô}{}{}{}{adj.}{Que assola, devastador, destruidor.}{as.so.la.dor}{0}
\verb{assolar}{}{}{}{}{v.t.}{Devastar, arruinar, destruir.}{as.so.lar}{0}
\verb{assolar}{}{Fig.}{}{}{}{Pôr em grande aflição; agoniar.}{as.so.lar}{\verboinum{1}}
\verb{assoldadar}{}{}{}{}{v.t.}{Contratar a soldo.}{as.sol.da.dar}{0}
\verb{assoldadar}{}{}{}{}{v.pron.}{Pôr"-se a serviço de alguém mediante soldo ou soldada; assalariar"-se.}{as.sol.da.dar}{\verboinum{1}}
\verb{assomar}{}{}{}{}{v.t.}{Subir a lugar elevado ou extremo.}{as.so.mar}{0}
\verb{assomar}{}{}{}{}{}{Manifestar, revelar.}{as.so.mar}{0}
\verb{assomar}{}{}{}{}{v.pron.}{Irar"-se, irritar"-se.}{as.so.mar}{0}
\verb{assomar}{}{}{}{}{}{Animar"-se com a bebida.}{as.so.mar}{\verboinum{1}}
\verb{assombração}{}{}{"-ões}{}{s.f.}{Ato ou efeito de assombrar.}{as.som.bra.ção}{0}
\verb{assombração}{}{}{"-ões}{}{}{Sentimento de terror causado por coisas inexplicáveis.}{as.som.bra.ção}{0}
\verb{assombração}{}{}{"-ões}{}{}{Alma do outro mundo; fantasma.}{as.som.bra.ção}{0}
\verb{assombrado}{}{}{}{}{adj.}{Coberto de sombra; sombrio.}{as.som.bra.do}{0}
\verb{assombrado}{}{}{}{}{}{Cheio de terror; apavorado.}{as.som.bra.do}{0}
\verb{assombrado}{}{}{}{}{}{Muito admirado; espantado.}{as.som.bra.do}{0}
\verb{assombrado}{}{}{}{}{s.m.}{Lugar onde aparecem assombrações.}{as.som.bra.do}{0}
\verb{assombramento}{}{}{}{}{s.m.}{Ato ou efeito de assombrar.}{as.som.bra.men.to}{0}
\verb{assombrar}{}{}{}{}{v.t.}{Fazer sombra; cobrir de sombra.}{as.som.brar}{0}
\verb{assombrar}{}{}{}{}{}{Causar susto; aterrorizar.}{as.som.brar}{0}
\verb{assombrar}{}{}{}{}{}{Encher de admiração; maravilhar.}{as.som.brar}{\verboinum{1}}
\verb{assombro}{ô}{}{}{}{s.m.}{Grande espanto ou admiração.}{as.som.bro}{0}
\verb{assombro}{ô}{}{}{}{}{Susto, pavor.}{as.som.bro}{0}
\verb{assombro}{ô}{}{}{}{}{Fantasma.}{as.som.bro}{0}
\verb{assombroso}{ô}{}{"-osos ⟨ó⟩}{"-osa ⟨ó⟩}{adj.}{Que produz assombro; espantoso, impressionante.}{as.som.bro.so}{0}
\verb{assomo}{ô}{}{}{}{s.m.}{Ato ou efeito de assomar ou aparecer.}{as.so.mo}{0}
\verb{assomo}{ô}{}{}{}{}{Aparência, indício.}{as.so.mo}{0}
\verb{assomo}{ô}{}{}{}{}{Vontade forte; ímpeto.}{as.so.mo}{0}
\verb{assomo}{ô}{}{}{}{}{Irritação, zanga.}{as.so.mo}{0}
\verb{assonância}{}{Gram.}{}{}{s.f.}{Identidade ou semelhança fonética entre vogais tônicas de palavras próximas, em repetição ritmada ou não.}{as.so.nân.cia}{0}
\verb{assonante}{}{}{}{}{adj.2g.}{Em que há assonância.}{as.so.nan.te}{0}
\verb{assopradela}{é}{}{}{}{s.f.}{Ato ou efeito de assoprar ligeiramente; sopro.}{as.so.pra.de.la}{0}
\verb{assoprar}{}{}{}{}{v.t.}{Expelir o ar pela boca ou nariz; soprar.}{as.so.prar}{\verboinum{1}}
\verb{assopro}{ô}{}{}{}{s.m.}{Ato ou efeito de soprar; exalação, expiração, sopro.}{as.so.pro}{0}
\verb{assoreamento}{}{}{}{}{s.m.}{Acúmulo de sedimentos pelo depósito de terra, argila etc, num rio, canal ou estuário, causado por enchentes pluviais ou pelo vento, devido ao mau uso do solo e à degradação da bacia hidrográfica.}{as.so.re.a.men.to}{0}
\verb{assorear}{}{}{}{}{v.t.}{Produzir assoreamento; obstruir de areia, terra etc.}{as.so.re.ar}{\verboinum{4}}
\verb{assoviador}{ô}{}{}{}{}{Var. de \textit{assobiador}.}{as.so.vi.a.dor}{0}
\verb{assoviar}{}{}{}{}{}{Var. de \textit{assobiar}.}{as.so.vi.ar}{0}
\verb{assovio}{}{}{}{}{}{Var. de \textit{assobio}.}{as.so.vi.o}{0}
\verb{assuada}{}{}{}{}{s.f.}{Reunião de pessoas visando promover algazarra.}{as.su.a.da}{0}
\verb{assuada}{}{}{}{}{}{Arruaça, motim, balbúrdia.}{as.su.a.da}{0}
\verb{assuada}{}{}{}{}{}{Vaia.}{as.su.a.da}{0}
\verb{assuar}{}{}{}{}{v.t.}{Manifestar publicamente desagrado ou despeito; apupar, zombar, vaiar.}{as.su.ar}{\verboinum{8}}
\verb{assumido}{}{}{}{}{adj.}{Que assume sua ideologia, opção sexual, valores etc.}{as.su.mi.do}{0}
\verb{assumido}{}{}{}{}{s.m.}{Indivíduo que assume essas posições.}{as.su.mi.do}{0}
\verb{assumir}{}{}{}{}{v.t.}{Tomar para si; apropriar"-se.}{as.su.mir}{0}
\verb{assumir}{}{}{}{}{}{Tornar"-se responsável.}{as.su.mir}{0}
\verb{assumir}{}{}{}{}{}{Tomar aparência; adotar, ostentar.}{as.su.mir}{\verboinum{18}}
\verb{assunção}{}{}{"-ões}{}{s.f.}{Ato ou efeito de assumir.}{as.sun.ção}{0}
\verb{assunção}{}{}{"-ões}{}{}{Na lógica, proposição aceita, independentemente de sua veracidade ou falsidade, por possibilitar, a partir dela, uma série de inferências.}{as.sun.ção}{0}
\verb{assunção}{}{Relig.}{"-ões}{}{}{Subida do corpo de Maria ao céu.}{as.sun.ção}{0}
\verb{assunção}{}{Relig.}{"-ões}{}{}{Festa católica em celebração a esse episódio.}{as.sun.ção}{0}
\verb{assuntar}{}{}{}{}{v.t.}{Prestar atenção; observar, olhar, reparar.}{as.sun.tar}{0}
\verb{assuntar}{}{}{}{}{}{Verificar em busca de detalhes; apurar, investigar.}{as.sun.tar}{0}
\verb{assuntar}{}{}{}{}{v.i.}{Pensar, refletir.}{as.sun.tar}{0}
\verb{assuntar}{}{}{}{}{}{Tomar conta, vigiar.}{as.sun.tar}{\verboinum{1}}
\verb{assunto}{}{}{}{}{s.m.}{Tema de que se trata ou sobre o qual se conversa.}{as.sun.to}{0}
\verb{assustadiço}{}{}{}{}{adj.}{Que se assusta facilmente.}{as.sus.ta.di.ço}{0}
\verb{assustado}{}{}{}{}{adj.}{Que se assustou; amedrontado.}{as.sus.ta.do}{0}
\verb{assustado}{}{}{}{}{}{Medroso, hesitante, inseguro, tímido, hesitante.}{as.sus.ta.do}{0}
\verb{assustado}{}{Bras.}{}{}{s.m.}{Festa com dança; arrasta"-pé.}{as.sus.ta.do}{0}
\verb{assustador}{ô}{}{}{}{adj.}{Que assusta; aterrador.}{as.sus.ta.dor}{0}
\verb{assustar}{}{}{}{}{v.t.}{Dar ou sofrer medo ou susto; amedrontar, atemorizar.}{as.sus.tar}{0}
\verb{assustar}{}{}{}{}{v.i.}{Ter a propriedade de causar medo ou susto.}{as.sus.tar}{\verboinum{1}}
\verb{astatínio}{}{Quím.}{}{}{s.m.}{Elemento químico do grupo dos halogênios, raro na natureza, radioativo; astato. \elemento{85}{(210)}{At}.}{as.ta.tí.nio}{0}
\verb{astato}{}{Quím.}{}{}{s.m.}{Astatínio. }{as.ta.to}{0}
\verb{asteca}{é}{}{}{}{adj.2g.}{Relativo aos astecas, povo que antes da invasão espanhola habitava a região hoje correspondente ao México.}{as.te.ca}{0}
\verb{asteca}{é}{}{}{}{s.2g.}{Indivíduo pertencente a esse povo.}{as.te.ca}{0}
\verb{asteca}{é}{}{}{}{s.m.}{A língua falada por esse povo, também conhecida como náuatle e hoje extinta.}{as.te.ca}{0}
\verb{astenia}{}{Med.}{}{}{s.f.}{Fraqueza, debilidade.}{as.te.ni.a}{0}
\verb{asterisco}{}{}{}{}{s.m.}{Sinal gráfico em forma de estrela (*) utilizado para marcar uma chamada de nota de rodapé ou de um comentário qualquer, que geralmente fica em letras pequenas em um canto da página, também acompanhado de um asterisco; usado também para indicar a supressão de uma ou mais letras por tratar"-se de dúvida ou de uso inadequado.}{as.te.ris.co}{0}
\verb{asteroide}{}{Astron.}{}{}{s.m.}{Cada um dos pequenos corpos celestes que gravitam em torno do Sol, sendo que a maioria deles tem órbita entre Marte e Júpiter.}{as.te.roi.de}{0}
\verb{asteroide}{}{}{}{}{adj.2g.}{Semelhante a estrela.}{as.te.roi.de}{0}
\verb{asteroide}{}{}{}{}{}{Relativo aos asteroides.}{as.te.roi.de}{0}
\verb{astigmático}{}{}{}{}{adj.}{Relativo a astigmatismo.}{as.tig.má.ti.co}{0}
\verb{astigmático}{}{}{}{}{}{Que tem astigmatismo.}{as.tig.má.ti.co}{0}
\verb{astigmático}{}{}{}{}{s.m.}{Indivíduo portador de astigmatismo.}{as.tig.má.ti.co}{0}
\verb{astigmatismo}{}{Med.}{}{}{s.m.}{Deformação na curvatura da córnea, que resulta em distúrbio da visão; erro de refração.}{as.tig.ma.tis.mo}{0}
\verb{astracã}{}{}{}{}{s.m.}{Pele escura e de pelos macios do cordeiro caracul (variedade asiática) recém"-nascido, utilizada como agasalho ou enfeite de peças de vestuário.}{as.tra.cã}{0}
\verb{astrágalo}{}{Anat.}{}{}{s.m.}{Um dos ossos do tarso.}{as.trá.ga.lo}{0}
\verb{astral}{}{}{"-ais}{}{adj.2g.}{Relativo aos astros.}{as.tral}{0}
\verb{astral}{}{}{"-ais}{}{}{Relativo ao mundo espiritual, que coexiste com o mundo físico.}{as.tral}{0}
\verb{astral}{}{Bras.}{"-ais}{}{}{Estado de espírito; humor, disposição, ânimo.}{as.tral}{0}
\verb{astral}{}{Bras.}{"-ais}{}{}{Conjunto de características, geralmente estéticas e psicossociais, que tornam um ambiente agradável ou desagradável para um indivíduo ou grupo.}{as.tral}{0}
\verb{astral}{}{Pop.}{"-ais}{}{}{Redução de mapa"-astral; horóscopo.}{as.tral}{0}
\verb{astro}{}{}{}{}{s.m.}{Nome comum a todos os corpos celestes.}{as.tro}{0}
\verb{astro}{}{Fig.}{}{}{}{Indivíduo eminente, notável.}{as.tro}{0}
\verb{astro}{}{Fig.}{}{}{}{Ator de destaque, especialmente o protagonista de um filme ou espetáculo.}{as.tro}{0}
\verb{astro}{}{Astrol.}{}{}{}{Cada um dos corpos celestes que, segundo a astrologia, influenciam os destinos humanos.}{as.tro}{0}
\verb{astrofísica}{}{}{}{}{s.f.}{Ramo da física que estuda os astros.}{as.tro.fí.si.ca}{0}
\verb{astrofísico}{}{}{}{}{adj.}{Relativo a astrofísica.}{as.tro.fí.si.co}{0}
\verb{astrofísico}{}{}{}{}{s.m.}{Indivíduo dedicado à astrofísica.}{as.tro.fí.si.co}{0}
\verb{astrolábio}{}{}{}{}{s.m.}{Instrumento náutico que media a altura dos astros e determinava a posição (latitude e longitude) do observador.}{as.tro.lá.bio}{0}
\verb{astrologia}{}{Astrol.}{}{}{s.f.}{Estudo da influência dos astros na personalidade e no destino das pessoas.}{as.tro.lo.gi.a}{0}
\verb{astrológico}{}{}{}{}{adj.}{Relativo a astrologia.}{as.tro.ló.gi.co}{0}
\verb{astrólogo}{}{}{}{}{s.m.}{Indivíduo que se dedica à astrologia.}{as.tró.lo.go}{0}
\verb{astronauta}{}{}{}{}{s.2g.}{Indivíduo que navega em veículo espacial, fora da atmosfera terrestre.}{as.tro.nau.ta}{0}
\verb{astronáutica}{}{}{}{}{s.f.}{Ciência e técnica que trata do projeto, construção e operação de veículos espaciais.}{as.tro.náu.ti.ca}{0}
\verb{astronáutico}{}{}{}{}{adj.}{Relativo a astronáutica ou a astronauta.}{as.tro.náu.ti.co}{0}
\verb{astronave}{}{}{}{}{s.f.}{Veículo projetado para viajar fora da atmosfera terrestre.}{as.tro.na.ve}{0}
\verb{astrônimo}{}{Astron.}{}{}{s.m.}{Designação comum dada aos nomes próprios de astros.}{as.trô.ni.mo}{0}
\verb{astronomia}{}{}{}{}{s.f.}{Ciência que estuda o espaço sideral e os corpos celestes.}{as.tro.no.mi.a}{0}
\verb{astronômico}{}{}{}{}{adj.}{Relativo a astronomia.}{as.tro.nô.mi.co}{0}
\verb{astronômico}{}{Fig.}{}{}{}{De valor muito elevado.}{as.tro.nô.mi.co}{0}
\verb{astrônomo}{}{}{}{}{s.m.}{Indivíduo dedicado à astronomia.}{as.trô.no.mo}{0}
\verb{astúcia}{}{}{}{}{s.f.}{Habilidade em enganar ou em não se deixar enganar. }{as.tú.cia}{0}
\verb{astucioso}{}{}{"-osos ⟨ó⟩}{"-osa ⟨ó⟩}{adj.}{Que usa de astúcia; cheio de astúcia; muito astuto.}{as.tu.ci.o.so}{0}
\verb{astuto}{}{}{}{}{adj.}{Que age com astúcia.}{as.tu.to}{0}
\verb{At}{}{Quím.}{}{}{}{Símb. do \textit{astato}. }{At}{0}
\verb{ata}{}{}{}{}{s.f.}{Registro escrito do que se passou em uma reunião de trabalho.}{a.ta}{0}
\verb{ata}{}{}{}{}{s.f.}{Fruto da ateira; fruta"-do"-conde, pinha.}{a.ta}{0}
\verb{atabalhoado}{}{}{}{}{adj.}{Que é feito às pressas.}{a.ta.ba.lho.a.do}{0}
\verb{atabalhoado}{}{}{}{}{}{Que faz as coisas de maneira confusa, atrapalhada.}{a.ta.ba.lho.a.do}{0}
\verb{atabalhoar}{}{}{}{}{v.t.}{Fazer ou dizer coisas às pressas.}{a.ta.ba.lho.ar}{0}
\verb{atabalhoar}{}{}{}{}{}{Fazer ou dizer coisas sem propósito.}{a.ta.ba.lho.ar}{\verboinum{7}}
\verb{atabaque}{}{Mús.}{}{}{s.m.}{Tambor feito de pele esticada sobre um pau oco usado em danças de origem africana.}{a.ta.ba.que}{0}
\verb{atacadista}{}{}{}{}{adj.2g.}{Diz"-se do comércio de mercadorias em grandes quantidades.}{a.ta.ca.dis.ta}{0}
\verb{atacadista}{}{}{}{}{}{Que atua nesse tipo de comércio.}{a.ta.ca.dis.ta}{0}
\verb{atacadista}{}{}{}{}{s.2g.}{Profissional que atua nesse tipo de comércio.}{a.ta.ca.dis.ta}{0}
\verb{atacado}{}{}{}{}{adj.}{Que sofreu ataque.}{a.ta.ca.do}{0}
\verb{atacado}{}{Bras.}{}{}{}{De mau humor; irritado, antipático.}{a.ta.ca.do}{0}
\verb{atacado}{}{}{}{}{s.m.}{Comércio de bens em grandes quantidades.}{a.ta.ca.do}{0}
\verb{atacador}{ô}{}{}{}{s.m.}{Cordão com que se prende uma peça de roupa a outra.}{a.ta.ca.dor}{0}
\verb{atacante}{}{}{}{}{adj.2g.}{Que toma a iniciativa do ataque.}{a.ta.can.te}{0}
\verb{atacante}{}{Esport.}{}{}{s.2g.}{Jogador que joga no ataque, geralmente finalizando as jogadas ofensivas.}{a.ta.can.te}{0}
\verb{atacar}{}{}{}{}{v.t.}{Cometer uma ação ofensiva.}{a.ta.car}{0}
\verb{atacar}{}{}{}{}{}{Lançar agressão moral; ofender, injuriar.}{a.ta.car}{0}
\verb{atacar}{}{}{}{}{}{Acometer, contagiar.}{a.ta.car}{\verboinum{2}}
\verb{atadura}{}{}{}{}{s.f.}{Ato ou efeito de atar.}{a.ta.du.ra}{0}
\verb{atadura}{}{}{}{}{}{Faixa de tecido apropriada para fazer curativos.}{a.ta.du.ra}{0}
\verb{atafulhar}{}{}{}{}{v.t.}{Encher em demasia; abarrotar.}{a.ta.fu.lhar}{\verboinum{1}}
\verb{atalaia}{}{Bot.}{}{}{s.f.}{Planta nativa da África e Austrália. }{a.ta.lai.a}{0}
\verb{atalaia}{}{}{}{}{}{Lugar elevado, de onde se observa.}{a.ta.lai.a}{0}
\verb{atalaia}{}{}{}{}{s.2g.}{Sentinela, vigia, guarda.}{a.ta.lai.a}{0}
\verb{atalhar}{}{}{}{}{v.t.}{Abrir caminho ou atalho, abreviando o percurso.}{a.ta.lhar}{0}
\verb{atalhar}{}{}{}{}{}{Tornar mais breve; encurtar, resumir.}{a.ta.lhar}{0}
\verb{atalhar}{}{}{}{}{}{Obstruir, impedir, sustar.}{a.ta.lhar}{0}
\verb{atalhar}{}{}{}{}{}{Interromper a fala de alguém}{a.ta.lhar}{0}
\verb{atalhar}{}{}{}{}{v.i.}{Ficar confuso, embaraçado, indeciso.}{a.ta.lhar}{\verboinum{1}}
\verb{atalho}{}{}{}{}{s.m.}{Caminho secundário usado para encurtar distâncias e diminuir o tempo gasto em um percurso.}{a.ta.lho}{0}
\verb{atalho}{}{}{}{}{}{Situação difícil; empecilho, obstáculo, embaraço.}{a.ta.lho}{0}
\verb{atalho}{}{}{}{}{}{Corte, interrupção, remate.}{a.ta.lho}{0}
\verb{atalho}{}{Informát.}{}{}{}{Ponto de entrada em um sistema, acessado por meio do teclado, de ícones etc.}{a.ta.lho}{0}
\verb{atamancar}{}{}{}{}{v.t.}{Consertar ou remendar grosseiramente.}{a.ta.man.car}{0}
\verb{atamancar}{}{}{}{}{}{Fazer algo às pressas e sem cuidado.}{a.ta.man.car}{\verboinum{2}}
\verb{atanazar}{}{}{}{}{v.t.}{Apertar com tenaz; atenazar, atazanar. }{a.ta.na.zar}{0}
\verb{atanazar}{}{}{}{}{}{Supliciar, torturar, mortificar. }{a.ta.na.zar}{0}
\verb{atanazar}{}{}{}{}{}{Importunar, aborrecer, azucrinar. }{a.ta.na.zar}{\verboinum{1}}
\verb{atapetado}{}{}{}{}{adj.}{Coberto com tapete; acarpetado.}{a.ta.pe.ta.do}{0}
\verb{atapetar}{}{}{}{}{v.t.}{Forrar com tapete; acarpetar, alcatifar.}{a.ta.pe.tar}{\verboinum{1}}
\verb{ataque}{}{}{}{}{s.m.}{Ato ou efeito de atacar; investida, assalto.}{a.ta.que}{0}
\verb{ataque}{}{}{}{}{}{Agressão, ofensa, crítica.}{a.ta.que}{0}
\verb{ataque}{}{}{}{}{}{Ação militar com o fim de conquista de uma posição ou da destruição de forças inimigas.}{a.ta.que}{0}
\verb{ataque}{}{Med.}{}{}{}{Acesso súbito de uma moléstia.}{a.ta.que}{0}
\verb{ataque}{}{Esport.}{}{}{}{Lance ofensivo em que os jogadores tentam dominar os adversários com o intuito de marcar pontos.}{a.ta.que}{0}
\verb{atar}{}{}{}{}{v.t.}{Prender ou apertar com nó ou laçada.}{a.tar}{0}
\verb{atar}{}{}{}{}{}{Unir, ligar, vincular.}{a.tar}{0}
\verb{atar}{}{}{}{}{}{Sujeitar, subordinar.}{a.tar}{0}
\verb{atar}{}{}{}{}{}{Formar relações; contratar, combinar.}{a.tar}{0}
\verb{atar}{}{}{}{}{}{Prender, conter, reprimir.}{a.tar}{\verboinum{1}}
\verb{atarantado}{}{}{}{}{adj.}{Que se atarantou; aturdido, atrapalhado, confuso.}{a.ta.ran.ta.do}{0}
\verb{atarantar}{}{}{}{}{v.t.}{Causar atrapalhação; confundir, perturbar, desnortear.}{a.ta.ran.tar}{\verboinum{1}}
\verb{atarefado}{}{}{}{}{adj.}{Sobrecarregado de trabalho; muito ocupado.}{a.ta.re.fa.do}{0}
\verb{atarefado}{}{}{}{}{}{Muito dedicado ao trabalho.}{a.ta.re.fa.do}{0}
\verb{atarefar}{}{}{}{}{v.t.}{Encarregar de tarefa.}{a.ta.re.far}{0}
\verb{atarefar}{}{}{}{}{}{Encher, cumular de trabalho.}{a.ta.re.far}{0}
\verb{atarefar}{}{}{}{}{v.pron.}{Entregar"-se muito ao trabalho; aplicar"-se.}{a.ta.re.far}{\verboinum{1}}
\verb{atarracado}{}{}{}{}{adj.}{Que está muito apertado; prensado.}{a.tar.ra.ca.do}{0}
\verb{atarracado}{}{}{}{}{}{Diz"-se do indivíduo baixo e gordo. }{a.tar.ra.ca.do}{0}
\verb{atarracar}{}{}{}{}{v.t.}{Apertar com muita força; arrochar.}{a.tar.ra.car}{0}
\verb{atarracar}{}{}{}{}{}{Subjugar, achatar, arrasar.}{a.tar.ra.car}{0}
\verb{atarracar}{}{}{}{}{}{Preparar a ferradura para ajustá"-la ao casco do animal.}{a.tar.ra.car}{\verboinum{2}}
\verb{atarraxar}{ch}{}{}{}{v.t.}{Apertar com tarraxa, rosca; parafusar.}{a.tar.ra.xar}{0}
\verb{atarraxar}{ch}{}{}{}{}{Prender bem; unir, firmar.}{a.tar.ra.xar}{\verboinum{1}}
\verb{atascadeiro}{ê}{}{}{}{s.m.}{Local onde há muita lama; lamaçal, atoleiro.}{a.tas.ca.dei.ro}{0}
\verb{atascar}{}{}{}{}{v.t.}{Meter em atascadeiro, lamaçal; enlamear, atolar.}{a.tas.car}{0}
\verb{atascar}{}{}{}{}{v.pron.}{Apegar"-se a vícios; degradar"-se.}{a.tas.car}{\verboinum{2}}
\verb{atassalhar}{}{}{}{}{v.t.}{Fazer em pedaços; dilacerar, rasgar, retalhar.}{a.tas.sa.lhar}{0}
\verb{atassalhar}{}{}{}{}{}{Deixar em destroços; derrotar.}{a.tas.sa.lhar}{0}
\verb{atassalhar}{}{Fig.}{}{}{}{Fazer calúnias; difamar, desacreditar.}{a.tas.sa.lhar}{0}
\verb{atassalhar}{}{}{}{}{}{Dar dentadas; morder, abocanhar.}{a.tas.sa.lhar}{\verboinum{1}}
\verb{ataúde}{}{}{}{}{s.m.}{Caixa longa e com tampa onde é colocado o corpo de um morto; esquife, caixão.}{a.ta.ú.de}{0}
\verb{ataviar}{}{}{}{}{v.t.}{Colocar adornos; enfeitar, embelezar.}{a.ta.vi.ar}{0}
\verb{ataviar}{}{}{}{}{v.pron.}{Vestir"-se bem; aprontar"-se.}{a.ta.vi.ar}{\verboinum{1}}
\verb{atávico}{}{}{}{}{adj.}{Relativo a atavismo.}{a.tá.vi.co}{0}
\verb{atávico}{}{Biol.}{}{}{}{Diz"-se do caráter de um ascendente remoto que reaparece em um organismo, mas que não está presente em seus ascendentes mais próximos.}{a.tá.vi.co}{0}
\verb{atavio}{}{}{}{}{s.m.}{Enfeite, adorno, ornamento.}{a.ta.vi.o}{0}
\verb{atavio}{}{}{}{}{}{Conjunto de aparelhos, apetrechos.}{a.ta.vi.o}{0}
\verb{atavismo}{}{Biol.}{}{}{s.m.}{Reaparecimento de um caráter de um ascendente remoto em um organismo e que não está presente nos descendentes mais próximos.}{a.ta.vis.mo}{0}
\verb{atazanar}{}{}{}{}{v.t.}{Apertar com tenaz; atenazar, atanazar.}{a.ta.za.nar}{0}
\verb{atazanar}{}{}{}{}{}{Supliciar, torturar, mortificar. }{a.ta.za.nar}{0}
\verb{atazanar}{}{}{}{}{}{Importunar, aborrecer, azucrinar. }{a.ta.za.nar}{\verboinum{1}}
\verb{até}{}{}{}{}{prep.}{Indica um limite de tempo, de espaço, de ação, quantidade.}{a.té}{0}
\verb{até}{}{}{}{}{adv.}{Ainda; também; mesmo.}{a.té}{0}
\verb{atear}{}{}{}{}{v.t.}{Pôr, lançar, tocar fogo.}{a.te.ar}{0}
\verb{atear}{}{}{}{}{}{Avivar, soprar o fogo, a chama, o lume etc.}{a.te.ar}{0}
\verb{atear}{}{}{}{}{}{Provocar, fomentar, desencadear paixões, ódios, guerras etc.}{a.te.ar}{0}
\verb{atear}{}{}{}{}{}{Alastrar, propagar, espalhar.}{a.te.ar}{\verboinum{4}}
\verb{ateira}{ê}{Bot.}{}{}{s.f.}{Pequena árvore, nativa das regiões tropicas americanas, cujo fruto é a pinha;  pinheira.}{a.tei.ra}{0}
\verb{ateísmo}{}{}{}{}{s.m.}{Doutrina que nega categoricamente a existência de Deus.}{a.te.ís.mo}{0}
\verb{ateísmo}{}{}{}{}{}{Atitude daquele que não crê em Deus. }{a.te.ís.mo}{0}
\verb{ateliê}{}{}{}{}{s.m.}{Oficina ou estúdio onde trabalham artistas, como pintores, escultores, fotógrafos etc.}{a.te.li.ê}{0}
\verb{atemorizador}{ô}{}{}{}{adj.}{Que provoca temor; assustador, aterrador.}{a.te.mo.ri.za.dor}{0}
\verb{atemorizar}{}{}{}{}{v.t.}{Causar temor, susto; apavorar, assustar.}{a.te.mo.ri.zar}{\verboinum{1}}
\verb{atenazar}{}{}{}{}{v.t.}{Apertar com tenaz.}{a.te.na.zar}{0}
\verb{atenazar}{}{}{}{}{}{Supliciar, torturar, mortificar.}{a.te.na.zar}{0}
\verb{atenazar}{}{}{}{}{}{Importunar, aborrecer, azucrinar.}{a.te.na.zar}{\verboinum{1}}
\verb{atenção}{}{}{"-ões}{}{s.f.}{Ação de aplicar a mente em algo; meditação, concentração.}{a.ten.ção}{0}
\verb{atenção}{}{}{"-ões}{}{}{Atitude de consideração, solicitude, boa vontade.}{a.ten.ção}{0}
\verb{atenção}{}{}{"-ões}{}{}{Ato de se ocupar de algo, atentar; reparo, caso.}{a.ten.ção}{0}
\verb{atenção}{}{}{"-ões}{}{interj.}{Expressão que denota advertência, cuidado.}{a.ten.ção}{0}
\verb{atencioso}{ô}{}{"-osos ⟨ó⟩}{"-osa ⟨ó⟩}{adj.}{Que presta atenção; atento.}{a.ten.ci.o.so}{0}
\verb{atencioso}{ô}{}{"-osos ⟨ó⟩}{"-osa ⟨ó⟩}{}{Que revela cuidados; polido, cortês, solícito.}{a.ten.ci.o.so}{0}
\verb{atendente}{}{}{}{}{adj.2g.}{Que atende, que dá atenção.}{a.ten.den.te}{0}
\verb{atendente}{}{}{}{}{s.2g.}{Pessoa encarregada de cuidar dos doentes em hospitais e consultórios, auxiliando nos serviços de enfermagem.}{a.ten.den.te}{0}
\verb{atender}{ê}{}{}{}{v.t.}{Dar ou prestar atenção; ouvir atentamente.}{a.ten.der}{0}
\verb{atender}{ê}{}{}{}{}{Levar em consideração; ter em vista; considerar.}{a.ten.der}{0}
\verb{atender}{ê}{}{}{}{}{Atentar, observar, notar.}{a.ten.der}{0}
\verb{atender}{ê}{}{}{}{}{Prestar socorro; cuidar, acudir.}{a.ten.der}{\verboinum{12}}
\verb{atendimento}{}{}{}{}{s.m.}{Ato ou efeito de atender, prestar atenção.}{a.ten.di.men.to}{0}
\verb{atendimento}{}{}{}{}{}{Local onde se atende o público; recepção.}{a.ten.di.men.to}{0}
\verb{ateneu}{}{Desus.}{}{}{s.m.}{Nome dado a algumas instituições de ensino fundamental e médio.}{a.te.neu}{0}
\verb{ateneu}{}{}{}{}{}{Em Atenas, na antiga Grécia, templo onde oradores e poetas liam suas obras.}{a.te.neu}{0}
\verb{ateniense}{}{}{}{}{adj.2g.}{Relativo a Atenas, capital da Grécia.}{a.te.ni.en.se}{0}
\verb{ateniense}{}{}{}{}{s.2g.}{Indivíduo natural ou habitante dessa cidade.}{a.te.ni.en.se}{0}
\verb{atentado}{}{}{}{}{s.m.}{Ato de agressão ou violação dos direitos das pessoas.}{a.ten.ta.do}{0}
\verb{atentado}{}{}{}{}{}{Tentativa ou execução de um crime.}{a.ten.ta.do}{0}
\verb{atentar}{}{}{}{}{v.t.}{Pôr em execução; empreender.}{a.ten.tar}{0}
\verb{atentar}{}{}{}{}{}{Cometer um atentado, um crime contra a pessoa ou um patrimônio.}{a.ten.tar}{\verboinum{1}}
\verb{atentar}{}{}{}{}{v.t.}{Olhar com atenção; observar com tento; reparar.}{a.ten.tar}{0}
\verb{atentar}{}{}{}{}{}{Prestar atenção; considerar, refletir.}{a.ten.tar}{\verboinum{1}}
\verb{atentatório}{}{}{}{}{adj.}{Que atenta contra; que infringe. }{a.ten.ta.tó.rio}{0}
\verb{atento}{}{}{}{}{adj.}{Que presta atenção; interessado, concentrado.}{a.ten.to}{0}
\verb{atento}{}{}{}{}{}{Tomado em consideração; ponderado}{a.ten.to}{0}
\verb{atento}{}{}{}{}{}{Que expressa respeito; cortês, atencioso.}{a.ten.to}{0}
\verb{atenuação}{}{}{"-ões}{}{s.f.}{Ato ou efeito de atenuar, minorar.}{a.te.nu.a.ção}{0}
\verb{atenuação}{}{}{"-ões}{}{}{Perda da intensidade; abrandamento, enfraquecimento.}{a.te.nu.a.ção}{0}
\verb{atenuante}{}{}{}{}{adj.2g.}{Que atenua, torna menos grave.}{a.te.nu.an.te}{0}
\verb{atenuante}{}{Jur.}{}{}{}{Diz"-se da circunstância que diminui a gravidade de um crime, promovendo a redução da pena de um réu.}{a.te.nu.an.te}{0}
\verb{atenuar}{}{}{}{}{v.t.}{Tornar tênue, menos intenso; reduzir, diminuir.}{a.te.nu.ar}{0}
\verb{atenuar}{}{}{}{}{}{Tornar menos vivo; abrandar, suavizar, amenizar.}{a.te.nu.ar}{\verboinum{1}}
\verb{aterrador}{ô}{}{}{}{adj.}{Que aterroriza; amedrontador, ameaçador.}{a.ter.ra.dor}{0}
\verb{aterragem}{}{}{"-ens}{}{s.f.}{Aterrissagem.}{a.ter.ra.gem}{0}
\verb{aterrar}{}{}{}{}{v.t.}{Causar terror; aterrorizar, amedrontar.}{a.ter.rar}{\verboinum{1}}
\verb{aterrar}{}{}{}{}{v.t.}{Cobrir com terra ou entulho.}{a.ter.rar}{\verboinum{1}}
\verb{aterrissagem}{}{}{"-ens}{}{s.f.}{Ato ou efeito de aterrissar, tocar a terra; pouso, aterragem.}{a.ter.ris.sa.gem}{0}
\verb{aterrissar}{}{}{}{}{v.t.}{Tocar a terra; pousar.}{a.ter.ris.sar}{\verboinum{1}}
\verb{aterro}{ê}{}{}{}{s.m.}{Ato ou efeito de aterrar; cobrir com terra ou entulho.}{a.ter.ro}{0}
\verb{aterro}{ê}{}{}{}{}{Obra que consiste em depositar terra ou entulho para nivelar ou elevar um terreno. }{a.ter.ro}{0}
\verb{aterrorizador}{ô}{}{}{}{adj.}{Que causa medo, terror; aterrador, amedrontador.}{a.ter.ro.ri.za.dor}{0}
\verb{aterrorizar}{}{}{}{}{v.t.}{Provocar terror; amedrontar, assustar.}{a.ter.ro.ri.zar}{\verboinum{1}}
\verb{ater}{}{}{}{}{v.t.}{Fazer parar; reter.}{a.ter}{0}
\verb{ater}{}{}{}{}{v.pron.}{Buscar um encosto; apoiar"-se.}{a.ter}{0}
\verb{ater}{}{}{}{}{}{Limitar"-se, restringir"-se.}{a.ter}{0}
\verb{aterse}{}{}{}{}{}{Fiar"-se; confiar"-se.}{a.ter}{\verboinum{39}}
\verb{atestação}{}{}{"-ões}{}{s.f.}{Confirmação da veracidade de algo; atestado, certidão.}{a.tes.ta.ção}{0}
\verb{atestado}{}{}{}{}{adj.}{Que se confirmou; comprovado, certificado.}{a.tes.ta.do}{0}
\verb{atestado}{}{}{}{}{}{Documento em que se confirma a veracidade de um fato; certidão, atestação.}{a.tes.ta.do}{0}
\verb{atestador}{ô}{}{}{}{adj.}{Que confirma; declarante; atestante.}{a.tes.ta.dor}{0}
\verb{atestante}{}{}{}{}{adj.2g.}{Que confirma; declarante; atestador.}{a.tes.tan.te}{0}
\verb{atestar}{}{}{}{}{v.t.}{Certificar por escrito; afirmar oficialmente.}{a.tes.tar}{0}
\verb{atestar}{}{}{}{}{}{Servir de testumunha; depor.}{a.tes.tar}{\verboinum{1}}
\verb{atestatório}{}{}{}{}{adj.}{Que comprova ou certifica.}{a.tes.ta.tó.rio}{0}
\verb{ateu}{}{}{}{ateia}{adj.}{Que não crê em Deus.}{a.teu}{0}
\verb{ateu}{}{}{}{ateia}{s.m.}{Indivíduo descrente.}{a.teu}{0}
%\verb{atiçador}{}{}{}{}{}{0}{a.ti.ça.dor}{0}
\verb{atiçar}{}{}{}{}{v.t.}{Provocar, promover, instigar, excitar.}{a.ti.çar}{0}
\verb{atiçar}{}{}{}{}{}{Avivar (o fogo), tornar mais aceso, soprando ou colocando combustível.}{a.ti.çar}{\verboinum{3}}
\verb{ático}{}{}{}{}{adj.}{Relativo a Ática, a Atenas ou  aos atenienses.}{á.ti.co}{0}
\verb{ático}{}{}{}{}{}{Estilo, de falar ou escrever, elegante, sóbrio, puro.}{á.ti.co}{0}
\verb{ático}{}{}{}{}{s.m.}{Indivíduo natural ou habitante da Ática ou de Atenas.}{á.ti.co}{0}
\verb{atijolado}{}{}{}{}{adj.}{Que é feito com tijolos.}{a.ti.jo.la.do}{0}
\verb{atijolar}{}{}{}{}{v.t.}{Revestir de tijolo.}{a.ti.jo.lar}{0}
\verb{atijolar}{}{}{}{}{}{Dar cor ou semelhança de tijolo.}{a.ti.jo.lar}{\verboinum{1}}
\verb{atilado}{}{}{}{}{adj.}{Que é esperto, inteligente, culto, perspicaz.}{a.ti.la.do}{0}
\verb{atilado}{}{}{}{}{}{Escrupuloso, correto.}{a.ti.la.do}{0}
\verb{atilado}{}{}{}{}{}{Polido, elegante, fino.}{a.ti.la.do}{0}
\verb{atilamento}{}{}{}{}{s.m.}{Ato ou efeito de atilar; esperteza, sagacidade; elegância.}{a.ti.la.men.to}{0}
\verb{atilamento}{}{}{}{}{}{Pontualidade, exatidão, prudência.}{a.ti.la.men.to}{0}
\verb{atilamento}{}{}{}{}{}{Escrúpulo, correção.}{a.ti.la.men.to}{0}
\verb{atilar}{}{}{}{}{v.t.}{Executar com cuidado; aperfeiçoar.}{a.ti.lar}{0}
\verb{atilar}{}{}{}{}{}{Tornar esperto, hábil.}{a.ti.lar}{\verboinum{1}}
\verb{atilho}{}{}{}{}{s.m.}{Tira estreita de pano, couro, cordão, barbante, fita, que serve para prender, atar, amarrar.}{a.ti.lho}{0}
\verb{atilho}{}{}{}{}{}{Feixe de espigas de milho ou outros objetos atados.}{a.ti.lho}{0}
\verb{atilho}{}{}{}{}{}{Estopim.}{a.ti.lho}{0}
\verb{átimo}{}{}{}{}{s.m.}{Instante, momento.}{á.ti.mo}{0}
\verb{átimo}{}{}{}{}{}{Porção mínima.}{á.ti.mo}{0}
\verb{atinado}{}{}{}{}{adj.}{Que tem tino, que é atento, astuto, prudente, ajuizado.}{a.ti.na.do}{0}
\verb{atinar}{}{}{}{}{v.t.}{Descobrir pelo tino, pela intuição.}{a.ti.nar}{0}
\verb{atinar}{}{}{}{}{}{Perceber, atentar, reparar, notar.}{a.ti.nar}{\verboinum{1}}
\verb{atinente}{}{}{}{}{adj.2g.}{Que é pertinente, relativo, concernente; que diz respeito.}{a.ti.nen.te}{0}
\verb{atingir}{}{}{}{}{v.t.}{Alcançar, tocar, acertar.}{a.tin.gir}{0}
\verb{atingir}{}{}{}{}{}{Chegar.}{a.tin.gir}{0}
\verb{atingir}{}{}{}{}{}{Abranger, incluir.}{a.tin.gir}{0}
\verb{atingir}{}{}{}{}{}{Afetar, abalar.}{a.tin.gir}{\verboinum{22}}
\verb{atipicidade}{}{}{}{}{s.f.}{Característica do que é atípico, incomum.}{a.ti.pi.ci.da.de}{0}
\verb{atípico}{}{}{}{}{adj.}{Que não é comum, típico; que difere do normal; irregular.}{a.tí.pi.co}{0}
\verb{atiradeira}{ê}{}{}{}{s.f.}{Brinquedo infantil, que consiste numa forquilha de madeira, provida de dois elásticos que seguram uma rodelinha de couro, com a qual se atiram pequenas pedras; estilingue.}{a.ti.ra.dei.ra}{0}
\verb{atiradiço}{}{Pop.}{}{}{adj.}{Que é dado a aventuras amorosas, galanteios; atirado, conquistador, petulante, atrevido.}{a.ti.ra.di.ço}{0}
\verb{atirado}{}{Pop.}{}{}{adj.}{Que ousa; atrevido, petulante.}{a.ti.ra.do}{0}
\verb{atirador}{ô}{}{}{}{adj.}{Que que atira.}{a.ti.ra.dor}{0}
\verb{atirador}{ô}{}{}{}{s.m.}{Indivíduo que tem habilidade e destreza em atirar com arma de fogo ou de arremesso.}{a.ti.ra.dor}{0}
\verb{atirar}{}{}{}{}{v.t.}{Arremessar, lançar.}{a.ti.rar}{0}
\verb{atirar}{}{}{}{}{}{Disparar arma de fogo ou de arremesso; alvejar.}{a.ti.rar}{\verboinum{1}}
\verb{atitar}{}{}{}{}{v.t.}{Soltar gritos estridentes, atitos; piar, assobiar.}{a.ti.tar}{\verboinum{1}}
\verb{atito}{}{}{}{}{s.m.}{Grito agudo que soltam certas aves quando enfurecidas; pio, assobio.}{a.ti.to}{0}
\verb{atitude}{}{}{}{}{s.f.}{Posição do corpo; postura, porte.}{a.ti.tu.de}{0}
\verb{atitude}{}{}{}{}{}{Modo de proceder ou agir em face de determinada situação; comportamento.}{a.ti.tu.de}{0}
\verb{ativa}{}{}{}{}{s.f.}{Parte principal em qualquer ato.}{a.ti.va}{0}
\verb{ativa}{}{}{}{}{}{Serviço ativo.}{a.ti.va}{0}
\verb{ativa}{}{}{}{}{}{Exercício efetivo de um serviço, de uma atividade.}{a.ti.va}{0}
\verb{ativação}{}{}{"-ões}{}{s.f.}{Ato ou efeito de ativar, de dar atividade.}{a.ti.va.ção}{0}
\verb{ativação}{}{}{"-ões}{}{}{Aumento das propriedades físicas, químicas ou biológicas de um corpo.}{a.ti.va.ção}{0}
\verb{ativado}{}{}{}{}{adj.}{Que sofreu ativação.}{a.ti.va.do}{0}
\verb{ativado}{}{}{}{}{}{Que se tornou ativo ou mais ativo.}{a.ti.va.do}{0}
\verb{ativar}{}{}{}{}{v.t.}{Dar atividade.}{a.ti.var}{0}
\verb{ativar}{}{}{}{}{}{Tornar ativo ou aumentar a atividade; impulsionar, intensificar.}{a.ti.var}{\verboinum{1}}
\verb{atividade}{}{}{}{}{s.f.}{Qualidade do que é ativo; ação.}{a.ti.vi.da.de}{0}
\verb{atividade}{}{}{}{}{}{Vivacidade e energia na ação; diligência.}{a.ti.vi.da.de}{0}
\verb{atividade}{}{}{}{}{}{Modo de vida; ocupação, profissão.}{a.ti.vi.da.de}{0}
\verb{atividade}{}{}{}{}{}{Faculdade ou possibilidade de agir, de fazer, de se mover.}{a.ti.vi.da.de}{0}
\verb{ativismo}{}{Filos.}{}{}{s.m.}{Doutrina que avalia a verdade de acordo com a função prática do conhecimento; pragmatismo.}{a.ti.vis.mo}{0}
\verb{ativismo}{}{}{}{}{}{Atuação dedicada a serviço de uma doutrina ou ideal de cunho político"-social.}{a.ti.vis.mo}{0}
\verb{ativismo}{}{}{}{}{}{Ação política ativa; militância.}{a.ti.vis.mo}{0}
\verb{ativista}{}{}{}{}{adj.2g.}{Relativo a ativismo.}{a.ti.vis.ta}{0}
\verb{ativista}{}{}{}{}{s.2g.}{Partidário do ativismo.}{a.ti.vis.ta}{0}
\verb{ativista}{}{}{}{}{}{Militante político.}{a.ti.vis.ta}{0}
\verb{ativo}{}{}{}{}{adj.}{Que exerce ação; que atua, opera, move"-se, funciona.}{a.ti.vo}{0}
\verb{ativo}{}{}{}{}{}{Que se caracteriza pela ação, pela diligência; vivo, ágil, enérgico.}{a.ti.vo}{0}
\verb{ativo}{}{}{}{}{}{Que exerce ação intensa; forte na ação}{a.ti.vo}{0}
\verb{ativo}{}{}{}{}{}{Que participa ou influencia; atuante, participante.}{a.ti.vo}{0}
\verb{ativo}{}{Ant.}{}{}{}{passivo}{a.ti.vo}{0}
\verb{atlântico}{}{}{}{}{adj.}{Relativo ao Oceano Atlântico ou aos locais por ele banhados.}{a.tlân.ti.co}{0}
\verb{atlântico}{}{}{}{}{}{Diz"-se do domínio florístico, que compreende a cordilheira marítima brasileira, cuja principal formação é a floresta atlântica.}{a.tlân.ti.co}{0}
\verb{atlântico}{}{}{}{}{s.m.}{O Oceano Atlântico.}{a.tlân.ti.co}{0}
\verb{atlas}{}{}{}{}{s.m.}{Coleção de mapas ou cartas geográficas em volume.}{a.tlas}{0}
\verb{atlas}{}{Anat.}{}{}{}{A primeira vértebra cervical que sustenta a cabeça.}{a.tlas}{0}
\verb{atleta}{é}{Hist.}{}{}{s.m.}{Indivíduo que combatia nos jogos solenes da Grécia e Roma antigas; lutador.}{a.tle.ta}{0}
\verb{atleta}{é}{}{}{}{s.2g.}{Desportista dedicado à prática de exercícios ou jogos que requerem grande esforço e habilidade física.}{a.tle.ta}{0}
\verb{atleta}{é}{}{}{}{}{Pessoa forte, musculosa.}{a.tle.ta}{0}
\verb{atlético}{}{}{}{}{adj.}{Relativo a atleta.}{a.tlé.ti.co}{0}
\verb{atlético}{}{}{}{}{}{Fisicamente vigoroso; musculoso, forte.}{a.tlé.ti.co}{0}
\verb{atletismo}{}{}{}{}{s.m.}{Designação comum aos exercícios físicos, individuais ou entre equipes, de caráter competitivo.}{a.tle.tis.mo}{0}
\verb{atletismo}{}{}{}{}{}{A prática de esportes atléticos.}{a.tle.tis.mo}{0}
\verb{atmosfera}{é}{}{}{}{s.f.}{Camada gasosa dos astros em geral.}{at.mos.fe.ra}{0}
\verb{atmosfera}{é}{}{}{}{}{Camada de ar que envolve a Terra.}{at.mos.fe.ra}{0}
\verb{atmosfera}{é}{Por ext.}{}{}{}{O estado atmosférico; o tempo, o céu.}{at.mos.fe.ra}{0}
\verb{atmosfera}{é}{Fig.}{}{}{}{Ambiente social ou espiritual; clima.}{at.mos.fe.ra}{0}
\verb{atmosférico}{}{}{}{}{adj.}{Relativo à atmosfera.}{at.mos.fé.ri.co}{0}
\verb{ato}{}{}{}{}{s.m.}{Tudo o que se faz ou se pode fazer. }{a.to}{0}
\verb{ato}{}{}{}{}{}{Procedimento, conduta.}{a.to}{0}
\verb{ato}{}{}{}{}{}{Cerimônia, solenidade.}{a.to}{0}
\verb{ato}{}{}{}{}{}{Cada uma das partes em que se divide uma peça de teatro, um balé, uma ópera.}{a.to}{0}
\verb{atoalhado}{}{}{}{}{adj.}{Coberto com toalha.}{a.to.a.lha.do}{0}
\verb{atoalhado}{}{}{}{}{s.m.}{Pano ou toalha de mesa.}{a.to.a.lha.do}{0}
\verb{atoalhar}{}{}{}{}{v.t.}{Cobrir com toalha.}{a.to.a.lhar}{0}
\verb{atoalhar}{}{}{}{}{}{Tecer como toalha.}{a.to.a.lhar}{\verboinum{1}}
\verb{atoarda}{}{Lus.}{}{}{s.f.}{Notícia vaga, imprecisa; boato.}{a.to.ar.da}{0}
\verb{atobá}{}{Zool.}{}{}{s.m.}{Ave que habita o Atlântico tropical e subtropical, de bico reto, corpo marrom e barriga branca.}{a.to.bá}{0}
\verb{atocaiar}{}{}{}{}{}{Var. de \textit{tocaiar}.}{a.to.cai.ar}{0}
\verb{atochado}{}{}{}{}{adj.}{Que está muito cheio; atulhado.}{a.to.cha.do}{0}
\verb{atochado}{}{}{}{}{}{Impedido de mover"-se; apertado, entalado.}{a.to.cha.do}{0}
\verb{atochar}{}{}{}{}{v.t.}{Encher em excesso; atulhar.}{a.to.char}{0}
\verb{atochar}{}{}{}{}{}{Fazer entrar com força; entalar.}{a.to.char}{\verboinum{1}}
\verb{atol}{ó}{}{"-óis}{}{s.m.}{Tipo de recife de coral, de forma elíptica, erigido sobre vulcões submersos, com uma laguna central. }{a.tol}{0}
\verb{atolado}{}{}{}{}{adj.}{Que se meteu em atoleiro, em lodo; atascado.}{a.to.la.do}{0}
\verb{atoladouro}{ô}{}{}{}{s.m.}{Lugar de solo mole, pantanoso; atoleiro.}{a.to.la.dou.ro}{0}
\verb{atolar}{}{}{}{}{v.t.}{Meter ou enterrar em atoleiro; atascar.}{a.to.lar}{0}
\verb{atolar}{}{Fig.}{}{}{v.pron.}{Entregar"-se em excesso aos prazeres, ao trabalho, ao vício.}{a.to.lar}{0}
\verb{atolar}{}{}{}{}{}{Meter"-se em dificuldades.}{a.to.lar}{\verboinum{1}}
\verb{atoleimado}{}{}{}{}{adj.}{Que é um tanto tolo; aparvalhado.}{a.to.lei.ma.do}{0}
\verb{atoleimar}{}{}{}{}{v.t.}{Tornar tolo, apatetar.}{a.to.lei.mar}{0}
\verb{atoleimar}{}{}{}{}{v.pron.}{Fazer"-se de tolo.}{a.to.lei.mar}{\verboinum{1}}
\verb{atoleiro}{ê}{}{}{}{s.m.}{Lugar de solo mole, pantanoso; atoladeiro.}{a.to.lei.ro}{0}
\verb{atoleiro}{ê}{Fig.}{}{}{}{Dificuldade, apuros.}{a.to.lei.ro}{0}
\verb{atômico}{}{}{}{}{adj.}{Relativo a átomo.}{a.tô.mi.co}{0}
\verb{atômico}{}{}{}{}{}{Referente ao núcleo atômico; nuclear.}{a.tô.mi.co}{0}
\verb{atomizador}{ô}{}{}{}{s.m.}{Instrumento usado para aspergir líquido em gotículas; vaporizador, borrifador, nebulizador.}{a.to.mi.za.dor}{0}
\verb{atomizar}{}{}{}{}{v.t.}{Reduzir a átomo ou a dimensão mínima.}{a.to.mi.zar}{0}
\verb{atomizar}{}{}{}{}{}{Borrifar com o atomizador; nebulizar.}{a.to.mi.zar}{\verboinum{1}}
\verb{átomo}{}{Quím.}{}{}{s.m.}{A menor fração de um elemento suscetível de combinação química.}{á.to.mo}{0}
\verb{átomo}{}{Por ext.}{}{}{}{Coisa excessivamente pequena; parte mínima; momento, instante.}{á.to.mo}{0}
\verb{atonalidade}{}{Mús.}{}{}{s.f.}{Sistema moderno de composição que não segue as funções tonais clássicas, as normas tradicionais da harmonia.}{a.to.na.li.da.de}{0}
\verb{atonia}{}{}{}{}{s.f.}{Falta de força, debilidade geral, fraqueza.}{a.to.ni.a}{0}
\verb{atonia}{}{Fig.}{}{}{}{Abatimento moral ou intelectual.}{a.to.ni.a}{0}
\verb{atônito}{}{}{}{}{adj.}{Que foi tomado de assombro ou de grande admiração; espantado, pasmado.}{a.tô.ni.to}{0}
\verb{atônito}{}{}{}{}{}{Confuso, perturbado.}{a.tô.ni.to}{0}
\verb{átono}{}{Gram.}{}{}{adj.}{Diz"-se da vogal, sílaba ou palavra sem acento tônico; atônico.}{á.to.no}{0}
\verb{átono}{}{Gram.}{}{}{}{Diz"-se do pronome pessoal oblíquo.}{á.to.no}{0}
\verb{atopetar}{}{}{}{}{v.t.}{Içar até o tope do mastro.}{a.to.pe.tar}{0}
\verb{atopetar}{}{}{}{}{}{Encher muito; abarrotar.}{a.to.pe.tar}{\verboinum{1}}
\verb{ator}{ô}{}{}{atriz}{s.m.}{Indivíduo que representa um personagem em peças teatrais, novelas, filmes.}{a.tor}{0}
\verb{ator}{ô}{}{}{atriz}{}{Indivíduo que tem um papel ativo em algum acontecimento.}{a.tor}{0}
\verb{ator}{ô}{Fig.}{}{atriz}{}{Pessoa que sabe fingir.}{a.tor}{0}
\verb{atordoado}{}{}{}{}{adj.}{Que se atordoou, que sente a cabeça rodar, que quase perdeu os sentidos ou desmaiou.}{a.tor.do.a.do}{0}
\verb{atordoado}{}{}{}{}{s.m.}{Indivíduo atordoado.}{a.tor.do.a.do}{0}
\verb{atordoamento}{}{}{}{}{s.m.}{Ato ou efeito de atordoar, de perturbar os sentidos.}{a.tor.do.a.men.to}{0}
\verb{atordoante}{}{}{}{}{adj.2g.}{Que atordoa.}{a.tor.do.an.te}{0}
\verb{atordoar}{}{}{}{}{v.t.}{Causar perturbação dos sentidos por efeito de pancada, queda, bebida, estrondo, surpresa etc.; aturdir, estontear.}{a.tor.do.ar}{0}
\verb{atordoar}{}{}{}{}{}{Tornar menos sensível; amortecer, adormentar.}{a.tor.do.ar}{0}
\verb{atordoar}{}{}{}{}{}{Surpreender muito; maravilhar.}{a.tor.do.ar}{\verboinum{7}}
\verb{atormentado}{}{}{}{}{adj.}{Que sofre tormentos ou torturas; flagelado, torturado.}{a.tor.men.ta.do}{0}
\verb{atormentado}{}{Fig.}{}{}{}{Que padece de sofrimento moral; angustiado.}{a.tor.men.ta.do}{0}
\verb{atormentado}{}{}{}{}{s.m.}{Indivíduo atormentado.}{a.tor.men.ta.do}{0}
\verb{atormentar}{}{}{}{}{v.t.}{Causar aflição ou tormento; torturar, flagelar.}{a.tor.men.tar}{0}
\verb{atormentar}{}{Fig.}{}{}{}{Causar sofrimento moral; angustiar.}{a.tor.men.tar}{0}
\verb{atormentar}{}{Fig.}{}{}{}{Causar incômodo; importunar, azucrinar.}{a.tor.men.tar}{0}
\verb{atormentar}{}{}{}{}{}{Agitar com violência; fustigar.}{a.tor.men.tar}{\verboinum{1}}
\verb{atóxico}{cs}{}{}{}{adj.}{Que não é tóxico, que não tem veneno.}{a.tó.xi.co}{0}
\verb{atrabiliário}{}{}{}{}{adj.}{Que está triste, de mau humor, melancólico.}{a.tra.bi.li.á.rio}{0}
\verb{atrabiliário}{}{}{}{}{}{Que vive tomado pela cólera; irascível, violento.}{a.tra.bi.li.á.rio}{0}
\verb{atracação}{}{}{"-ões}{}{s.f.}{Ato ou efeito de atracar, de aproximar ou encostar embarcação ao cais ou a outra embarcação.}{a.tra.ca.ção}{0}
\verb{atracadouro}{ô}{}{}{}{s.m.}{Lugar onde se amarram as embarcações.}{a.tra.ca.dou.ro}{0}
\verb{atração}{}{}{"-ões}{}{s.f.}{Ato ou efeito de atrair, de puxar para si.}{a.tra.ção}{0}
\verb{atração}{}{Fig.}{"-ões}{}{}{Poder de encantar; simpatia, sedução.}{a.tra.ção}{0}
\verb{atração}{}{Fig.}{"-ões}{}{}{Inclinaçao afetiva; interesse, afeição.}{a.tra.ção}{0}
\verb{atração}{}{Por ext.}{"-ões}{}{}{Aquilo que tem a finalidade de entreter, distrair; divertimento.}{a.tra.ção}{0}
\verb{atração}{}{}{"-ões}{}{}{Pessoa ou coisa que desperta grande interesse.}{a.tra.ção}{0}
\verb{atracar}{}{}{}{}{v.t.}{Amarrar uma embarcação à terra.}{a.tra.car}{0}
\verb{atracar}{}{}{}{}{}{Encostar um barco a outro.}{a.tra.car}{0}
\verb{atracar}{}{Pop.}{}{}{}{Agarrar alguém com intenções amorosas.}{a.tra.car}{0}
\verb{atracar}{}{Fig.}{}{}{v.pron.}{Pegar"-se corpo a corpo, engalfinhar"-se.}{a.tra.car}{\verboinum{2}}
\verb{atraente}{}{}{}{}{adj.2g.}{Que exerce atração; sedução, encantador.}{a.tra.en.te}{0}
\verb{atraente}{}{}{}{}{}{Que tem a virtude de atrair; convidativo, acolhedor.}{a.tra.en.te}{0}
\verb{atraiçoado}{}{}{}{}{adj.}{Que é vítima de traição; traído}{a.trai.ço.a.do}{0}
\verb{atraiçoado}{}{}{}{}{}{Que não é leal; traiçoeiro.}{a.trai.ço.a.do}{0}
\verb{atraiçoar}{}{}{}{}{v.t.}{Cometer traição; trair.}{a.trai.ço.ar}{0}
\verb{atraiçoar}{}{}{}{}{}{Cometer infidelidade; enganar.}{a.trai.ço.ar}{0}
\verb{atraiçoar}{}{}{}{}{}{Revelar segredo íntimo.}{a.trai.ço.ar}{0}
\verb{atraiçoar}{}{}{}{}{}{Deturpar, desvirtuar.}{a.trai.ço.ar}{\verboinum{7}}
\verb{atrair}{}{}{}{}{v.t.}{Trazer para si, fazer aproximar"-se.}{a.tra.ir}{0}
\verb{atrair}{}{}{}{}{}{Exercer atração sobre; seduzir, fascinar.}{a.tra.ir}{0}
\verb{atrair}{}{}{}{}{}{Chamar; incitar a aproximar"-se.}{a.tra.ir}{0}
\verb{atrair}{}{}{}{}{}{Fazer surgir ideia, opinião, sentimento etc.}{a.tra.ir}{0}
\verb{atrair}{}{}{}{}{}{Levar alguém a adesão de algo; aliciar.}{a.tra.ir}{\verboinum{19}}
\verb{atrapalhação}{}{}{"-ões}{}{s.f.}{Ato ou efeito de atrapalhar; confusão, embaraço.}{a.tra.pa.lha.ção}{0}
\verb{atrapalhado}{}{}{}{}{adj.}{Perturbado, embaraçado.}{a.tra.pa.lha.do}{0}
\verb{atrapalhado}{}{}{}{}{}{Em situação embaraçosa, em dificuldades, especialmente financeiras.}{a.tra.pa.lha.do}{0}
\verb{atrapalhado}{}{}{}{}{}{Confuso, desordenado.}{a.tra.pa.lha.do}{0}
\verb{atrapalhar}{}{}{}{}{v.t.}{Ser um obstáculo; perturbar, estorvar, impedir.}{a.tra.pa.lhar}{0}
\verb{atrapalhar}{}{}{}{}{}{Desempenhar mal.}{a.tra.pa.lhar}{\verboinum{1}}
\verb{atrás}{}{}{}{}{adv.}{Na parte posterior; após, depois.}{a.trás}{0}
\verb{atrás}{}{Ant.}{}{}{}{adiante}{a.trás}{0}
\verb{atrasado}{}{}{}{}{adj.}{Que ficou para trás; retardatário.}{a.tra.sa.do}{0}
\verb{atrasado}{}{}{}{}{}{Pouco desenvolvido mental ou fisicamente; retardado.}{a.tra.sa.do}{0}
\verb{atrasado}{}{}{}{}{}{Ultrapassado, antiquado, retrógrado.}{a.tra.sa.do}{0}
\verb{atrasar}{}{}{}{}{v.t.}{Fazer demorar; retardar.}{a.tra.sar}{0}
\verb{atrasar}{}{}{}{}{}{Pôr para trás; recuar.}{a.tra.sar}{0}
\verb{atrasar}{}{}{}{}{}{Impedir o progresso; prejudicar.}{a.tra.sar}{\verboinum{1}}
\verb{atraso}{}{}{}{}{s.m.}{Ato ou efeito de atrasar, retardar; demora.}{a.tra.so}{0}
\verb{atraso}{}{Fig.}{}{}{}{Falta de desenvolvimento; ignorância.}{a.tra.so}{0}
\verb{atrativo}{}{}{}{}{adj.}{Que tem o poder de atrair, seduzir.}{a.tra.ti.vo}{0}
\verb{atrativo}{}{}{}{}{s.m.}{Encanto, simpatia, estímulo.}{a.tra.ti.vo}{0}
\verb{atravancamento}{}{}{}{}{s.m.}{Ato ou efeito de atravancar, de impedir com obstáculo.}{a.tra.van.ca.men.to}{0}
\verb{atravancar}{}{}{}{}{v.t.}{Colocar obstáculos; atrapalhar, dificultar.}{a.tra.van.car}{0}
\verb{atravancar}{}{}{}{}{}{Acumular, sobrecarregar.}{a.tra.van.car}{0}
\verb{atravancar}{}{}{}{}{v.pron.}{Pôr"-se no meio; intrometer"-se.}{a.tra.van.car}{\verboinum{2}}
\verb{através}{}{}{}{}{adv.}{De lado a lado, transversalmente.}{a.tra.vés}{0}
\verb{atravessado}{}{}{}{}{adj.}{Posto no sentido transversal.}{a.tra.ves.sa.do}{0}
\verb{atravessado}{}{}{}{}{}{Em cruz; cruzado, oblíquo.}{a.tra.ves.sa.do}{0}
\verb{atravessado}{}{}{}{}{}{Passado de lado a lado; transpassado, varado.}{a.tra.ves.sa.do}{0}
\verb{atravessado}{}{Fig.}{}{}{}{Cheio de irritação, de má vontade.}{a.tra.ves.sa.do}{0}
\verb{atravessador}{ô}{}{}{}{s.m.}{Indivíduo que compra mercadorias do produtor e as revende para o comércio varejista com uma margem de lucro; intermediário.}{a.tra.ves.sa.dor}{0}
\verb{atravessar}{}{}{}{}{v.t.}{Posicionar transversalmente ou de través.}{a.tra.ves.sar}{0}
\verb{atravessar}{}{}{}{}{}{Passar através; cruzar, transpassar.}{a.tra.ves.sar}{0}
\verb{atravessar}{}{}{}{}{}{Passar além; transpor.}{a.tra.ves.sar}{0}
\verb{atravessar}{}{Fig.}{}{}{}{Passar dificuldades; sofrer, experimentar.}{a.tra.ves.sar}{0}
\verb{atravessar}{}{}{}{}{}{Resistir, prolongar, subsistir.}{a.tra.ves.sar}{0}
\verb{atravessar}{}{}{}{}{}{Comprar produtos por atacado para revender com lucro.}{a.tra.ves.sar}{0}
\verb{atravessar}{}{}{}{}{}{Pôr em oposição; contrapor, interpor.}{a.tra.ves.sar}{\verboinum{1}}
\verb{atrelar}{}{}{}{}{v.t.}{Prender pela trela ou correia.}{a.tre.lar}{0}
\verb{atrelar}{}{}{}{}{}{Prender um animal a um veículo.}{a.tre.lar}{0}
\verb{atrelar}{}{}{}{}{}{Engatar, unir um veículo ou vagão a outro.}{a.tre.lar}{0}
\verb{atrelar}{}{Fig.}{}{}{}{Dominar, submeter, reprimir.}{a.tre.lar}{0}
\verb{atrelar}{}{Fig.}{}{}{v.pron.}{Seguir insistentemente alguém.}{a.tre.lar}{\verboinum{1}}
\verb{atrever"-se}{}{}{}{}{v.pron.}{Ter coragem, audácia; ousar, arriscar"-se.}{a.tre.ver"-se}{0}
\verb{atreverse}{}{}{}{}{}{Enfrentar, afrontar, opor"-se.}{a.tre.ver"-se}{\verboinum{12}}
\verb{atrevido}{}{}{}{}{adj.}{Que se atreve; ousado, destemido.}{a.tre.vi.do}{0}
\verb{atrevido}{}{}{}{}{}{Que procede com insolência; petulante, malcriado.}{a.tre.vi.do}{0}
\verb{atrevimento}{}{}{}{}{s.m.}{Ato corajoso; ousadia, audácia.}{a.tre.vi.men.to}{0}
\verb{atrevimento}{}{}{}{}{}{Falta de respeito; insolência, petulância.}{a.tre.vi.men.to}{0}
\verb{atribuição}{}{}{"-ões}{}{s.f.}{Ato ou efeito de atribuir.}{a.tri.bui.ção}{0}
\verb{atribuição}{}{}{"-ões}{}{}{Responsabilidade de um cargo; competência, obrigação.}{a.tri.bui.ção}{0}
\verb{atribuição}{}{}{"-ões}{}{}{Direito, privilégio, prerrogativa.}{a.tri.bui.ção}{0}
\verb{atribuir}{}{}{}{}{v.t.}{Considerar como autor ou possuidor; imputar.}{a.tri.bu.ir}{0}
\verb{atribuir}{}{}{}{}{}{Conceder, conferir, outorgar.}{a.tri.bu.ir}{0}
\verb{atribuirse}{}{}{}{}{v.pron.}{Tomar para si; reivindicar.}{a.tri.bu.ir}{\verboinum{26}}
\verb{atribulação}{}{}{"-ões}{}{s.f.}{Sofrimento, aflição, tormento.}{a.tri.bu.la.ção}{0}
\verb{atribulação}{}{}{"-ões}{}{}{Acontecimento desagradável; adversidade, infelicidade.}{a.tri.bu.la.ção}{0}
\verb{atribular}{}{}{}{}{v.t.}{Causar aflição; angustiar, atormentar, inquietar.}{a.tri.bu.lar}{\verboinum{1}}
\verb{atributivo}{}{}{}{}{adj.}{Que atribui, confere um cargo, poder, privilégio a alguém.}{a.tri.bu.ti.vo}{0}
\verb{atributivo}{}{}{}{}{}{Que indica ou enuncia um atributo.}{a.tri.bu.ti.vo}{0}
\verb{atributo}{}{}{}{}{s.m.}{Aquilo que é próprio ou inerente a um ser.}{a.tri.bu.to}{0}
\verb{atributo}{}{}{}{}{}{Qualidade, atrativo, condição.}{a.tri.bu.to}{0}
\verb{atributo}{}{}{}{}{}{Emblema distintivo, símbolo, signo.}{a.tri.bu.to}{0}
\verb{átrio}{}{}{}{}{s.m.}{Pátio interno de uma construção; vestíbulo.}{á.trio}{0}
\verb{átrio}{}{}{}{}{}{Sala de estar, saguão.}{á.trio}{0}
\verb{átrio}{}{Anat.}{}{}{}{Cada uma das duas aurículas do coração.}{á.trio}{0}
\verb{atritar}{}{}{}{}{v.t.}{Fazer pressão, provocando fricção entre corpos; roçar, friccionar.}{a.tri.tar}{0}
\verb{atritar}{}{}{}{}{}{Magoar, penalizar, atormentar.}{a.tri.tar}{0}
\verb{atritar}{}{}{}{}{v.pron.}{Ter desentendimento; irritar"-se, indispor"-se.}{a.tri.tar}{\verboinum{1}}
\verb{atrito}{}{}{}{}{s.m.}{Fricção entre dois corpos.}{a.tri.to}{0}
\verb{atrito}{}{Fig.}{}{}{}{Dificuldade de entendimento; desavença, conflito.}{a.tri.to}{0}
\verb{atrito}{}{Fís.}{}{}{}{Resistência que um corpo desenvolve quando outro corpo se move sobre ele ou através dele.}{a.tri.to}{0}
\verb{atriz}{}{}{}{}{s.f.}{Feminino de \textit{ator}.}{a.triz}{0}
\verb{atriz}{}{}{}{}{}{Mulher que representa papéis em teatro, cinema, televisão.}{a.triz}{0}
\verb{atro}{}{}{}{}{adj.}{Que tem a cor negra, escura.}{a.tro}{0}
\verb{atro}{}{}{}{}{}{Que causa tristeza; sombrio, lúgubre.}{a.tro}{0}
\verb{atroada}{}{}{}{}{s.f.}{Grande barulho; estrondo.}{a.tro.a.da}{0}
\verb{atroador}{ô}{}{}{}{adj.}{Que faz muito barulho; estrondoso.}{a.tro.a.dor}{0}
\verb{atroar}{}{}{}{}{v.t.}{Fazer estremecer com o estrondo; fazer retumbar.}{a.tro.ar}{0}
\verb{atroar}{}{Fig.}{}{}{}{Aturdir com barulho; atordoar, ensurdecer.}{a.tro.ar}{0}
\verb{atroar}{}{}{}{}{v.i.}{Trovejar, retumbar.}{a.tro.ar}{\verboinum{7}}
\verb{atrocidade}{}{}{}{}{s.f.}{Qualidade do que é atroz, cruel.}{a.tro.ci.da.de}{0}
\verb{atrocidade}{}{}{}{}{}{Ato desumano; barbaridade, crueldade.}{a.tro.ci.da.de}{0}
\verb{atrofia}{}{Med.}{}{}{s.f.}{Definhamento ou diminuição de peso ou volume de uma célula, tecido ou órgão, causados por falta de nutrição, falta de uso ou doença.}{a.tro.fi.a}{0}
\verb{atrofia}{}{}{}{}{}{Enfraquecimento ou degeneração de alguma faculdade mental.}{a.tro.fi.a}{0}
\verb{atrofiado}{}{}{}{}{adj.}{Que padece de atrofia.}{a.tro.fi.a.do}{0}
\verb{atrofiado}{}{}{}{}{}{Que não se desenvolve; degenerado, enfraquecido.}{a.tro.fi.a.do}{0}
\verb{atrofiamento}{}{}{}{}{s.m.}{Ato ou efeito de atrofiar; enfraquecimento, degeneração.}{a.tro.fi.a.men.to}{0}
\verb{atrofiar}{}{}{}{}{v.t.}{Impedir o desenvolvimento; tolher, acanhar.}{a.tro.fi.ar}{0}
\verb{atrofiar}{}{}{}{}{v.pron.}{Definhar"-se, enfraquecer"-se.}{a.tro.fi.ar}{\verboinum{1}}
\verb{atroo}{}{}{}{}{s.m.}{Ato ou efeito de atroar; estrondo.}{a.tro.o}{0}
\verb{atropelação}{}{}{"-ões}{}{s.f.}{Ato ou efeito de atropelar; atropelo.}{a.tro.pe.la.ção}{0}
\verb{atropelado}{}{}{}{}{adj.}{Que sofreu atropelamento.}{a.tro.pe.la.do}{0}
\verb{atropelado}{}{}{}{}{}{Desorganizado, confuso.}{a.tro.pe.la.do}{0}
\verb{atropelado}{}{}{}{}{}{Precipitado, apressado.}{a.tro.pe.la.do}{0}
\verb{atropelamento}{}{}{}{}{s.m.}{Ato ou efeito de atropelar.}{a.tro.pe.la.men.to}{0}
\verb{atropelamento}{}{}{}{}{}{Colisão, choque.}{a.tro.pe.la.men.to}{0}
\verb{atropelamento}{}{}{}{}{}{Acidente de trânsito causado pelo choque de veículos.}{a.tro.pe.la.men.to}{0}
\verb{atropelar}{}{}{}{}{v.t.}{Colidir, derrubar, passando por cima ou não.}{a.tro.pe.lar}{0}
\verb{atropelar}{}{}{}{}{}{Dar encontrão; empurrar, esbarrar violentamente.}{a.tro.pe.lar}{0}
\verb{atropelar}{}{}{}{}{}{Precipitar, apressar.}{a.tro.pe.lar}{0}
\verb{atropelar}{}{}{}{}{}{Causar tormento; torturar, afligir.}{a.tro.pe.lar}{0}
\verb{atropelar}{}{Fig.}{}{}{}{Não respeitar as leis; infringir, transgredir.}{a.tro.pe.lar}{0}
\verb{atropelar}{}{}{}{}{v.pron.}{Confundir"-se, embaralhar"-se.}{a.tro.pe.lar}{\verboinum{1}}
\verb{atropelo}{ê}{}{}{}{s.m.}{Falta de organização; confusão, desordem.}{a.tro.pe.lo}{0}
\verb{atropelo}{ê}{}{}{}{}{Precipitação, pressa.}{a.tro.pe.lo}{0}
\verb{atropina}{}{Quím.}{}{}{s.f.}{Substância alcaloide extraída da beladona e usada na medicina como sedativo, midriático e antiespasmódico.}{a.tro.pi.na}{0}
\verb{atroz}{ó}{}{}{}{adj.}{Que não tem piedade; cruel, desumano.}{a.troz}{0}
\verb{atroz}{ó}{}{}{}{}{Horrível de suportar; intolerável, lancinante.}{a.troz}{0}
\verb{atroz}{ó}{}{}{}{}{Que causa muito espanto; assombroso, monstruoso.}{a.troz}{0}
\verb{atuação}{}{}{"-ões}{}{s.f.}{Ato ou efeito de atuar; ação, representação.}{a.tu.a.ção}{0}
\verb{atual}{}{}{"-ais}{}{adj.2g.}{Que se realiza no momento presente.}{a.tu.al}{0}
\verb{atual}{}{}{"-ais}{}{}{Relativo a sua época; contemporâneo.}{a.tu.al}{0}
\verb{atual}{}{}{"-ais}{}{}{Real, efetivo, imediato.}{a.tu.al}{0}
\verb{atualidade}{}{}{}{}{s.f.}{Qualidade do que se realiza no momento presente.}{a.tu.a.li.da.de}{0}
\verb{atualidade}{}{}{}{}{}{O momento ou a época presente.}{a.tu.a.li.da.de}{0}
\verb{atualidades}{}{}{}{}{s.f.pl.}{Notícias ou novidades acerca do momento presente.}{a.tu.a.li.da.des}{0}
\verb{atualização}{}{}{"-ões}{}{s.f.}{Ato ou efeito de atualizar; modernização.}{a.tu.a.li.za.ção}{0}
\verb{atualizado}{}{}{}{}{adj.}{Que se atualizou; modernizado.}{a.tu.a.li.za.do}{0}
\verb{atualizado}{}{}{}{}{}{Que está bem informado do que sucede no momento presente.}{a.tu.a.li.za.do}{0}
\verb{atualizar}{}{}{}{}{v.t.}{Tornar atual, contemporâneo; modernizar.}{a.tu.a.li.zar}{0}
\verb{atualizar}{}{}{}{}{}{Acrescentar elementos ou informações mais recentes.}{a.tu.a.li.zar}{0}
\verb{atualizar}{}{Informát.}{}{}{}{Substituir total ou parcialmente um programa de computador, instalando uma versão mais recente.}{a.tu.a.li.zar}{\verboinum{1}}
\verb{atuante}{}{}{}{}{adj.2g.}{Que está em exercício de uma atividade.}{a.tu.an.te}{0}
\verb{atuante}{}{}{}{}{}{Que participa ativamente; ativo.}{a.tu.an.te}{0}
\verb{atuar}{}{}{}{}{v.i.}{Exercer ação ou atividade; agir.}{a.tu.ar}{0}
\verb{atuar}{}{}{}{}{v.t.}{Ter influência; influir.}{a.tu.ar}{0}
\verb{atuar}{}{}{}{}{}{Fazer pressão; constranger.}{a.tu.ar}{0}
\verb{atuar}{}{}{}{}{v.i.}{Desempenhar um papel dramático; representar.}{a.tu.ar}{\verboinum{1}}
\verb{atuária}{}{Mat.}{}{}{s.f.}{Parte da estatística que estuda as bases teóricas dos seguros em geral.}{a.tu.á.ria}{0}
\verb{atuário}{}{}{}{}{s.m.}{Especialista em matemática estatística que atua em companhias de seguro.}{a.tu.á.rio}{0}
\verb{atulhar}{}{}{}{}{v.t.}{Encher totalmente; abarrotar.}{a.tu.lhar}{\verboinum{1}}
\verb{atum}{}{Zool.}{"-uns}{}{s.m.}{Peixe marinho, cujo comprimento pode chegar a 2,40 m, cuja carne é muito apreciada, principalmente em conserva.}{a.tum}{0}
\verb{aturado}{}{}{}{}{adj.}{Que persevera; persistente, constante.}{a.tu.ra.do}{0}
\verb{aturado}{}{}{}{}{}{Suportado com resignação.}{a.tu.ra.do}{0}
\verb{aturar}{}{}{}{}{v.t.}{Suportar com paciência; sofrer, tolerar.}{a.tu.rar}{0}
\verb{aturar}{}{}{}{}{}{Aguentar, sustentar.}{a.tu.rar}{0}
\verb{aturar}{}{}{}{}{}{Ser capaz de atender; satisfazer, comportar.}{a.tu.rar}{0}
\verb{aturar}{}{}{}{}{}{Ter perseverança; persistir, continuar.}{a.tu.rar}{\verboinum{1}}
\verb{aturdido}{}{}{}{}{adj.}{Perturbado, atordoado, pasmado.}{a.tur.di.do}{0}
\verb{aturdimento}{}{}{}{}{s.m.}{Estado de perturbação da mente ou dos sentidos.}{a.tur.di.men.to}{0}
\verb{aturdir}{}{}{}{}{v.t.}{Perturbar a mente ou os sentidos; atordoar.}{a.tur.dir}{\verboinum{34}}
\verb{Au}{}{Quím.}{}{}{}{Símb. do \textit{ouro}.}{Au}{0}
\verb{audácia}{}{}{}{}{s.f.}{Impulso que leva a realizar ações difíceis; ousadia, intrepidez, coragem.}{au.dá.cia}{0}
\verb{audacioso}{ô}{}{"-osos ⟨ó⟩}{"-osa ⟨ó⟩}{adj.}{Que realiza ações difíceis; corajoso, valente, ousado.}{au.da.ci.o.so}{0}
\verb{audaz}{}{}{}{}{adj.}{Que tem audácia;  audacioso, ousado.}{au.daz}{0}
\verb{audibilidade}{}{}{}{}{s.f.}{Qualidade do que é audível.}{au.di.bi.li.da.de}{0}
\verb{audição}{}{}{"-ões}{}{s.f.}{Sentido pelo qual se percebem os sons.}{au.di.ção}{0}
\verb{audição}{}{Mús.}{"-ões}{}{}{Apresentação ou transmissão de peça musical.}{au.di.ção}{0}
\verb{audiência}{}{}{}{}{s.f.}{Conjunto das pessoas que ouvem ou assistem determinado programa de rádio ou televisão.}{au.di.ên.cia}{0}
\verb{audiência}{}{}{}{}{}{Recepção dada a alguém com objetivo de ouvi"-la.}{au.di.ên.cia}{0}
\verb{audiência}{}{Jur.}{}{}{}{Sessão de tribunal.}{au.di.ên.cia}{0}
\verb{áudio}{}{}{}{}{s.m.}{Sinal sonoro; som.}{áu.dio}{0}
\verb{áudio}{}{}{}{}{}{Componente sonoro de um programa audiovisual, como filme, transmissão de televisão etc.}{áu.dio}{0}
\verb{audiograma}{}{}{}{}{s.m.}{Representação gráfica da variação do limiar de audição de acordo com a frequência.}{au.di.o.gra.ma}{0}
\verb{audiometria}{}{}{}{}{s.f.}{Exame da capacidade auditiva realizado por meio de instrumentos.}{au.di.o.me.tri.a}{0}
\verb{audiômetro}{}{}{}{}{s.m.}{Equipamento para avaliar a capacidade auditiva.}{au.di.ô.me.tro}{0}
\verb{audiovisual}{}{}{}{}{adj.2g.}{Que utiliza informação auditiva e visual para compor a mensagem (diz"-se dos sistemas de comunicação ou da mensagem veiculada por eles).}{au.di.o.vi.su.al}{0}
\verb{auditivo}{}{}{}{}{adj.}{Relativo à audição ou ao ouvido.}{au.di.ti.vo}{0}
\verb{auditivo}{}{}{}{}{}{Que é mais sensível às informações auditivas do que às visuais.}{au.di.ti.vo}{0}
\verb{auditivo}{}{}{}{}{s.m.}{Indivíduo com essa característica.}{au.di.ti.vo}{0}
\verb{auditor}{ô}{}{}{}{s.m.}{Perito encarregado de examinar as operações contábeis de uma empresa ou instituição.}{au.di.tor}{0}
\verb{auditor}{ô}{}{}{}{}{Magistrado com jurisdição na Justiça Militar.}{au.di.tor}{0}
\verb{auditoria}{}{}{}{}{s.f.}{Repartição ou departamento onde o auditor exerce suas funções.}{au.di.to.ri.a}{0}
\verb{auditoria}{}{}{}{}{}{Exame das atividades contábeis de uma empresa ou instituição.}{au.di.to.ri.a}{0}
\verb{auditório}{}{}{}{}{s.m.}{Lugar devidamente equipado para a realização de espetáculos, conferências, solenidades etc.}{au.di.tó.rio}{0}
\verb{audível}{}{}{"-eis}{}{adj.2g.}{Que pode ser ouvido.}{au.dí.vel}{0}
\verb{auê}{}{Pop.}{}{}{s.m.}{Confusão, tumulto, agitação, falatório.}{au.ê}{0}
\verb{auferir}{}{}{}{}{v.t.}{Obter, conseguir, colher.}{au.fe.rir}{\verboinum{29}}
\verb{auge}{}{}{}{}{s.m.}{O ponto mais elevado; cume.}{au.ge}{0}
\verb{auge}{}{}{}{}{}{O estado de maior desenvolvimento; apogeu.}{au.ge}{0}
\verb{auge}{}{}{}{}{}{O nível mais alto de um sentimento; apogeu.}{au.ge}{0}
\verb{augurar}{}{}{}{}{v.t.}{Fazer prognósticos; prever, pressagiar.}{au.gu.rar}{0}
\verb{augurar}{}{}{}{}{}{Fazer votos; desejar.}{au.gu.rar}{\verboinum{1}}
\verb{áugure}{}{}{}{}{s.m.}{Antigo sacerdote romano que adivinhava o futuro pela observação do voo e do canto das aves.}{áu.gu.re}{0}
\verb{áugure}{}{Por ext.}{}{}{}{Indivíduo que prevê o futuro; adivinho.}{áu.gu.re}{0}
\verb{augúrio}{}{}{}{}{s.m.}{Profecia feita pelo áugure.}{au.gú.rio}{0}
\verb{augúrio}{}{Por ext.}{}{}{}{Prognóstico, agouro, presságio.}{au.gú.rio}{0}
\verb{augusto}{}{}{}{}{adj.}{Que é respeitável, venerado.}{au.gus.to}{0}
\verb{augusto}{}{}{}{}{}{Elevado, sublime, majestoso.}{au.gus.to}{0}
\verb{aula}{}{}{}{}{s.f.}{Atividade de ensino formal, envolvendo professor e aluno(s).}{au.la}{0}
\verb{áulico}{}{}{}{}{adj.}{Relativo a corte ⟨ô⟩.}{áu.li.co}{0}
\verb{aumentar}{}{}{}{}{v.t.}{Tornar maior em tamanho, extensão, massa, intensidade etc.; ampliar, amplificar.}{au.men.tar}{0}
\verb{aumentar}{}{}{}{}{}{Fazer parecer maior.}{au.men.tar}{0}
\verb{aumentar}{}{}{}{}{}{Agravar, piorar.}{au.men.tar}{\verboinum{1}}
\verb{aumentativo}{}{}{}{}{adj.}{Que aumenta.}{au.men.ta.ti.vo}{0}
\verb{aumentativo}{}{Gram.}{}{}{s.m.}{Palavra que designa um grau maior em relação à palavra de que deriva.}{au.men.ta.ti.vo}{0}
\verb{aumento}{}{}{}{}{s.m.}{Ato ou efeito de aumentar.}{au.men.to}{0}
\verb{aumento}{}{}{}{}{}{Acréscimo ou majoração (diz"-se de salários ou preços de bens).}{au.men.to}{0}
\verb{aumento}{}{Fís.}{}{}{}{Quociente entre o tamanho da imagem de um objeto produzida por um sistema óptico e o tamanho real do objeto.}{au.men.to}{0}
\verb{aura}{}{}{}{}{s.f.}{Vento brando; brisa.}{au.ra}{0}
\verb{aura}{}{}{}{}{}{Manifestação que irradia de todos os seres vivos, perceptível somente por pessoas de sensibilidade especial.}{au.ra}{0}
\verb{aura}{}{}{}{}{}{Manifestação perceptível de um estado de espírito.}{au.ra}{0}
\verb{áureo}{}{}{}{}{adj.}{Relativo a ouro.}{áu.re.o}{0}
\verb{áureo}{}{}{}{}{}{Feito de ouro ou recoberto de ouro.}{áu.re.o}{0}
\verb{áureo}{}{Fig.}{}{}{}{Brilhante, nobre; muito valioso.}{áu.re.o}{0}
\verb{auréola}{}{}{}{}{s.f.}{Círculo luminoso que envolve a cabeça das imagens de santos.}{au.ré.o.la}{0}
\verb{auréola}{}{}{}{}{}{Círculo luminoso que envolve qualquer objeto, especialmente os astros; halo.}{au.ré.o.la}{0}
\verb{auréola}{}{Fig.}{}{}{}{Brilho moral; glória, prestígio.}{au.ré.o.la}{0}
\verb{aureolar}{}{}{}{}{adj.2g.}{Semelhante a auréola.}{au.re.o.lar}{0}
\verb{aureolar}{}{}{}{}{v.t.}{Cingir com auréola.}{au.re.o.lar}{0}
\verb{aureolar}{}{}{}{}{}{Glorificar.}{au.re.o.lar}{\verboinum{1}}
\verb{aurícula}{}{Anat.}{}{}{s.f.}{Parte externa e cartilaginosa do ouvido; orelha.}{au.rí.cu.la}{0}
\verb{aurícula}{}{Anat.}{}{}{}{Cada uma das duas cavidades da parte superior do coração.}{au.rí.cu.la}{0}
\verb{auricular}{}{}{}{}{adj.2g.}{Relativo a aurícula ou ao ouvido.}{au.ri.cu.lar}{0}
\verb{auricular}{}{}{}{}{}{Localizado em ou relacionado com o ouvido.}{au.ri.cu.lar}{0}
\verb{auricular}{}{}{}{}{}{Que tem forma de aurícula.}{au.ri.cu.lar}{0}
\verb{aurífero}{}{}{}{}{adj.}{Que contém ou produz ouro.}{au.rí.fe.ro}{0}
\verb{auriverde}{ê}{}{}{}{adj.2g.}{De cor dourada e verde ou amarela e verde.}{au.ri.ver.de}{0}
\verb{aurora}{ó}{}{}{}{s.f.}{Momento imediatamente anterior ao nascer do Sol, em que apenas o céu se ilumina.}{au.ro.ra}{0}
\verb{aurora}{ó}{Fig.}{}{}{}{Os primeiros anos da vida; infância.}{au.ro.ra}{0}
\verb{aurora}{ó}{Fig.}{}{}{}{Princípio, começo, origem.}{au.ro.ra}{0}
\verb{auscultação}{}{}{"-ões}{}{s.f.}{Ato ou efeito de auscultar.}{aus.cul.ta.ção}{0}
\verb{auscultador}{ô}{}{}{}{s.m.}{Instrumento que serve para auscultar; estetoscópio.}{aus.cul.ta.dor}{0}
\verb{auscultar}{}{Med.}{}{}{v.t.}{Escutar atenciosamente, com ou sem auxílio de equipamento, determinada parte do organismo como parte de um exame clínico.}{aus.cul.tar}{0}
\verb{auscultar}{}{}{}{}{}{Investigar, sondar, inquirir.}{aus.cul.tar}{\verboinum{1}}
\verb{ausência}{}{}{}{}{s.f.}{Afastamento de uma pessoa de seu domicílio ou lugar habitual.}{au.sên.cia}{0}
\verb{ausência}{}{}{}{}{}{O não comparecimento a determinado compromisso.}{au.sên.cia}{0}
\verb{ausência}{}{}{}{}{}{Período em que alguém está afastado.}{au.sên.cia}{0}
\verb{ausência}{}{}{}{}{}{Inexistência, falta, carência.}{au.sên.cia}{0}
\verb{ausentar}{}{}{}{}{v.t.}{Afastar de determinado local por tempo indeterminado.}{au.sen.tar}{0}
\verb{ausentar}{}{}{}{}{v.pron.}{Deixar de comparecer em determinado compromisso ou de participar de determinada atividade.}{au.sen.tar}{\verboinum{1}}
\verb{ausente}{}{}{}{}{adj.2g.}{Que se afastou do domicílio ou lugar habitual.}{au.sen.te}{0}
\verb{ausente}{}{}{}{}{}{Que deixou de comparecer a determinado compromisso.}{au.sen.te}{0}
\verb{ausente}{}{}{}{}{}{Que não se envolve; distante.}{au.sen.te}{0}
\verb{ausente}{}{}{}{}{}{Distraído, alheio, abstraído.}{au.sen.te}{0}
\verb{auspiciar}{}{}{}{}{v.t.}{Fazer auspício; predizer, augurar.}{aus.pi.ci.ar}{\verboinum{1}}
\verb{auspício}{}{}{}{}{s.m.}{Pressentimento, presságio, prognóstico.}{aus.pí.cio}{0}
\verb{auspicioso}{ô}{}{"-osos ⟨ó⟩}{"-osa ⟨ó⟩}{adj.}{De bom agouro; promissor.}{aus.pi.ci.o.so}{0}
\verb{austeridade}{}{}{}{}{s.f.}{Qualidade de austero.}{aus.te.ri.da.de}{0}
\verb{austero}{é}{}{}{}{adj.}{Que é rígido de caráter, hábitos e opiniões.}{aus.te.ro}{0}
\verb{austero}{é}{}{}{}{}{Pouco flexível; rígido, rigoroso.}{aus.te.ro}{0}
\verb{austero}{é}{}{}{}{}{Sério, formal.}{aus.te.ro}{0}
\verb{austral}{}{}{"-ais}{}{adj.2g.}{Que está situado no hemisfério limitado pelo equador e contendo o polo sul; meridional.}{aus.tral}{0}
\verb{austral}{}{}{"-ais}{}{}{Que é próprio do sul.}{aus.tral}{0}
\verb{austral}{}{}{"-ais}{}{}{Unidade monetária da Argentina.}{aus.tral}{0}
\verb{austral}{}{Ant.}{"-ais}{}{}{boreal}{aus.tral}{0}
\verb{australiano}{}{}{}{}{adj.}{Relativo à Austrália.}{aus.tra.li.a.no}{0}
\verb{australiano}{}{}{}{}{s.m.}{Indivíduo natural ou habitante desse país.}{aus.tra.li.a.no}{0}
\verb{austríaco}{}{}{}{}{adj.}{Relativo à Áustria.}{aus.trí.a.co}{0}
\verb{austríaco}{}{}{}{}{s.m.}{Indivíduo natural ou habitante desse país.}{aus.trí.a.co}{0}
\verb{austro}{}{}{}{}{s.m.}{O sul; vento sul. }{aus.tro}{0}
\verb{autarquia}{}{}{}{}{s.f.}{Poder absoluto de um indivíduo ou grupo.}{au.tar.qui.a}{0}
\verb{autarquia}{}{}{}{}{}{Governo de um Estado conduzido por seus próprios cidadãos.}{au.tar.qui.a}{0}
\verb{autarquia}{}{Jur.}{}{}{}{Entidade de direito público com autonomia administrativa e econômica que presta serviços ou fornece recursos ao Estado, sendo tutelada por ele.}{au.tar.qui.a}{0}
\verb{autárquico}{}{}{}{}{adj.}{Relativo a autarquia.}{au.tár.qui.co}{0}
\verb{autenticado}{}{}{}{}{adj.}{Que se autenticou.}{au.ten.ti.ca.do}{0}
\verb{autenticar}{}{}{}{}{v.t.}{Garantir a autenticidade; reconhecer como verdadeiro.}{au.ten.ti.car}{0}
\verb{autenticar}{}{}{}{}{}{Autorizar, legalizar, certificar.}{au.ten.ti.car}{\verboinum{2}}
\verb{autenticidade}{}{}{}{}{s.f.}{Qualidade de autêntico.}{au.ten.ti.ci.da.de}{0}
\verb{autenticidade}{}{}{}{}{}{Legitimidade.}{au.ten.ti.ci.da.de}{0}
\verb{autêntico}{}{}{}{}{adj.}{Que tem origem ou qualidade comprovada; genuíno, legítimo, verdadeiro.}{au.tên.ti.co}{0}
\verb{autêntico}{}{}{}{}{}{De autoria comprovada.}{au.tên.ti.co}{0}
\verb{autêntico}{}{}{}{}{}{Fidedigno.}{au.tên.ti.co}{0}
\verb{autismo}{}{Med.}{}{}{s.m.}{Fenômeno patológico caracterizado pela tendência à introspecção, alheando"-se do mundo exterior.}{au.tis.mo}{0}
\verb{autista}{}{}{}{}{adj.2g.}{Relativo ao autismo.}{au.tis.ta}{0}
\verb{autista}{}{}{}{}{s.2g.}{Indivíduo que sofre de autismo.}{au.tis.ta}{0}
\verb{auto}{}{}{}{}{s.m.}{Ação pública, solenidade, ato.}{au.to}{0}
\verb{auto}{}{Jur.}{}{}{}{Narração circunstanciada de qualquer ato, administrativo ou judiciário, escrita e autenticada pelo respectivo escrivão e por testemunhas.}{au.to}{0}
\verb{auto}{}{}{}{}{}{Composição dramática com origem na Idade Média, que apresenta argumento bíblico, alegórico ou moralizante.}{au.to}{0}
\verb{auto}{}{}{}{}{s.m.}{Forma reduzida de \textit{automóvel}.}{au.to}{0}
\verb{autoafirmação}{}{}{autoafirmações}{}{s.f.}{Necessidade íntima que o indivíduo tem de se impor à aceitação do meio; afirmação.}{au.to.a.fir.ma.ção}{0}
\verb{autobiografar"-se}{}{}{}{}{v.pron.}{Escrever a própria biografia.}{au.to.bi.o.gra.far"-se}{\verboinum{1}}
\verb{autobiografia}{}{}{}{}{s.f.}{Narração sobre a vida de um indivíduo, escrita por ele mesmo.}{au.to.bi.o.gra.fi.a}{0}
\verb{autobiográfico}{}{}{}{}{adj.}{Relativo a autobiografia.}{au.to.bi.o.grá.fi.co}{0}
\verb{autocensura}{}{}{}{}{s.f.}{Repressão censória praticada por um indivíduo contra si mesmo.}{au.to.cen.su.ra}{0}
\verb{autoclave}{}{Med.}{}{}{s.f.}{Aparelho de desinfecção por meio do vapor a alta pressão e temperatura; esterilizador.}{au.to.cla.ve}{0}
\verb{autocolante}{}{}{}{}{adj.2g.}{Que tem um dos lados recoberto de substância adesiva. }{au.to.co.lan.te}{0}
\verb{autocolante}{}{}{}{}{s.m.}{Etiqueta, papel ou impresso autocolante.}{au.to.co.lan.te}{0}
\verb{autoconfiança}{}{}{}{}{s.f.}{Confiança em si mesmo.}{au.to.con.fi.an.ça}{0}
\verb{autoconsumo}{}{Econ.}{}{}{s.m.}{Consumo realizado pelo próprio produtor.}{au.to.con.su.mo}{0}
\verb{autocontrole}{ô}{}{}{}{s.m.}{Controle sobre si mesmo; autodomínio, equilíbrio.}{au.to.con.tro.le}{0}
\verb{autocracia}{}{}{}{}{s.f.}{Governo cujo chefe tem poderes ilimitados e absolutos.}{au.to.cra.ci.a}{0}
\verb{autocrata}{}{}{}{}{adj.2g.}{Diz"-se de governante cujo poder é absoluto e independente.}{au.to.cra.ta}{0}
\verb{autocrata}{}{}{}{}{s.2g.}{Pessoa que exerce governo absoluto.}{au.to.cra.ta}{0}
\verb{autocrítica}{}{}{}{}{s.f.}{Ato de o indivíduo reconhecer as qualidades e defeitos do próprio caráter; crítica de si mesmo.}{au.to.crí.ti.ca}{0}
\verb{autóctone}{}{}{}{}{adj.2g.}{Que é originário da região onde habita; nativo.}{au.tóc.to.ne}{0}
\verb{autóctone}{}{}{}{}{}{Que vive ou se situa no seu local de origem; aborígene.}{au.tóc.to.ne}{0}
\verb{autóctone}{}{}{}{}{s.2g.}{Aborígene, indígena, nativo.}{au.tóc.to.ne}{0}
\verb{auto da fé}{}{}{autos da fé}{}{s.m.}{Auto de fé.}{au.to da fé}{0}
\verb{auto de fé}{}{}{autos de fé}{}{s.m.}{Solenidade da Inquisição à qual compareciam os penitenciados para ouvir a leitura das sentenças e, depois, serem abolvidos ou condenados ao fogo.}{au.to de fé}{0}
\verb{auto de fé}{}{Fig.}{autos de fé}{}{}{Ato de destruir algo pelo fogo.}{au.to de fé}{0}
\verb{autodefesa}{ê}{}{}{}{s.f.}{Ato de o indivíduo defender"-se contra agressão de qualquer ordem.}{au.to.de.fe.sa}{0}
\verb{autodefesa}{ê}{Jur.}{}{}{}{Defesa de um direito feita pelo seu próprio titular.}{au.to.de.fe.sa}{0}
\verb{autodestruição}{}{}{"-ões}{}{s.f.}{Destruição, aniquilamento de si mesmo.}{au.to.des.trui.ção}{0}
\verb{autodeterminação}{}{}{"-ões}{}{s.f.}{Ato ou efeito de decidir por si mesmo.}{au.to.de.ter.mi.na.ção}{0}
\verb{autodeterminação}{}{}{"-ões}{}{}{Faculdade que tem uma nação de escolher seu próprio destino político.}{au.to.de.ter.mi.na.ção}{0}
\verb{autodidata}{}{}{}{}{adj.2g.}{Diz"-se daquele que se instrui por esforço próprio, sem ajuda de mestres.}{au.to.di.da.ta}{0}
\verb{autodidata}{}{}{}{}{s.2g.}{Indivíduo autodidata.}{au.to.di.da.ta}{0}
\verb{autodidatismo}{}{}{}{}{s.m.}{Ato de adquirir instrução por si mesmo, sem ajuda de professores.}{au.to.di.da.tis.mo}{0}
\verb{autodomínio}{}{}{}{}{s.m.}{Controle sobre si mesmo; autocontrole, equilíbrio.}{au.to.do.mí.nio}{0}
\verb{autódromo}{}{Esport.}{}{}{s.m.}{Conjunto de pistas e instalações diversas para corridas automobilísticas.}{au.tó.dro.mo}{0}
\verb{autoescola}{ó}{}{autoescolas ⟨ó⟩}{}{s.f.}{Escola para treinamento, habilitação e formação de motoristas.}{au.to.es.co.la}{0}
\verb{autoestima}{}{}{autoestimas}{}{s.f.}{Aceitação que o indivíduo tem de si;  consideração por si mesmo; amor"-próprio.}{au.to.es.ti.ma}{0}
\verb{autoestrada}{}{}{autoestradas}{}{s.f.}{Estrada, sem interseções, destinada ao tráfego de automóveis; autopista.}{au.to.es.tra.da}{0}
\verb{autofagia}{}{}{}{}{s.f.}{Ato de um organismo nutrir"-se da sua própria carne ou substância.}{au.to.fa.gi.a}{0}
\verb{autofagia}{}{Biol.}{}{}{}{Processo de autodestruição da célula.}{au.to.fa.gi.a}{0}
\verb{autófago}{}{}{}{}{adj.}{Que comete autofagia.}{au.tó.fa.go}{0}
\verb{autófago}{}{}{}{}{s.m.}{Indivíduo que passa pelo processo de autofagia.}{au.tó.fa.go}{0}
\verb{autofecundação}{}{}{"-ões}{}{s.f.}{Fecundação realizada sem a participação de outro indivíduo; autogamia.}{au.to.fe.cun.da.ção}{0}
\verb{autofinanciamento}{}{}{}{}{s.m.}{Ato de financiar através da aplicação dos próprios recursos.}{au.to.fi.nan.ci.a.men.to}{0}
\verb{autogamia}{}{Biol.}{}{}{s.f.}{Fecundação realizada através da fusão dos gametas masculinos e femininos de um mesmo indivíduo; autofecundação.}{au.to.ga.mi.a}{0}
\verb{autógeno}{}{Biol.}{}{}{adj.}{Que se gera a si mesmo, produzido sem influência externa.         }{au.tó.ge.no}{0}
\verb{autogestão}{}{Econ.}{"-ões}{}{s.f.}{Gerência de uma empresa pelos próprios trabalhadores.}{au.to.ges.tão}{0}
\verb{autogiro}{}{}{}{}{s.m.}{Avião que, além da hélice de propulsão, possui, na parte superior, uma hélice horizontal,  formada de grandes pás, articuladas num eixo vertical e dispostas de modo que girem com a ação do vento, o que lhe permite subir e baixar em direção vertical; o mesmo que helicóptero.}{au.to.gi.ro}{0}
\verb{autografar}{}{}{}{}{v.t.}{Pôr autógrafo ou assinatura.}{au.to.gra.far}{\verboinum{1}}
\verb{autógrafo}{}{}{}{}{adj.}{Que foi escrito pelo próprio autor; original.}{au.tó.gra.fo}{0}
\verb{autógrafo}{}{}{}{}{s.m.}{Escrito do próprio punho do autor.}{au.tó.gra.fo}{0}
\verb{autógrafo}{}{}{}{}{}{Assinatura de próprio punho.}{au.tó.gra.fo}{0}
\verb{autolocadora}{ô}{}{}{}{s.f.}{Firma que aluga veículos automóveis.}{au.to.lo.ca.do.ra}{0}
\verb{automação}{}{}{"-ões}{}{s.f.}{Sistema automático pelo qual os mecanismos controlam seu próprio funcionamento, sem a intervenção direta do homem.}{au.to.ma.ção}{0}
\verb{automático}{}{}{}{}{adj.}{Que se move, regula ou opera por si mesmo.}{au.to.má.ti.co}{0}
\verb{automático}{}{}{}{}{}{Que se realiza por meios mecânicos.}{au.to.má.ti.co}{0}
\verb{automático}{}{Fig.}{}{}{}{Involuntário, inconsciente.}{au.to.má.ti.co}{0}
\verb{automatismo}{}{}{}{}{s.m.}{Qualidade ou estado do que é maquinal, automático.}{au.to.ma.tis.mo}{0}
\verb{automatismo}{}{}{}{}{}{Falta de vontade própria, movimento inconsciente, involuntário.}{au.to.ma.tis.mo}{0}
\verb{automatização}{}{}{"-ões}{}{s.f.}{Ato ou efeito de automatizar, de tornar automático.}{au.to.ma.ti.za.ção}{0}
\verb{automatização}{}{}{"-ões}{}{}{Ver: \textit{automação}.}{au.to.ma.ti.za.ção}{0}
\verb{automatizar}{}{}{}{}{v.t.}{Tornar automático.}{au.to.ma.ti.zar}{\verboinum{1}}
\verb{autômato}{}{}{}{}{s.m.}{Máquina ou engenho que se põe em movimento por meios mecânicos.}{au.tô.ma.to}{0}
\verb{autômato}{}{}{}{}{}{Aparelho que imita os movimentos humanos; robô.}{au.tô.ma.to}{0}
\verb{autômato}{}{Fig.}{}{}{}{Indivíduo de comportamento maquinal, sem raciocínio e sem vontade própria.}{au.tô.ma.to}{0}
\verb{automedicação}{}{Med.}{"-ões}{}{s.f.}{Consumo de medicamentos por conta própria, sem prescrição médica.}{au.to.me.di.ca.ção}{0}
\verb{automedicar"-se}{}{}{}{}{v.pron.}{Efetuar automedicação; tomar medicamento por conta própria.}{au.to.me.di.car"-se}{\verboinum{2}}
\verb{automobilismo}{}{}{}{}{s.m.}{Fabricação e uso de veículos automotores.}{au.to.mo.bi.lis.mo}{0}
\verb{automobilismo}{}{Esport.}{}{}{}{Esporte cuja prática consiste em corridas de automóveis.}{au.to.mo.bi.lis.mo}{0}
\verb{automobilista}{}{}{}{}{s.2g.}{Pessoa que se dedica ao automobilismo.}{au.to.mo.bi.lis.ta}{0}
\verb{automobilístico}{}{}{}{}{adj.}{Relativo ao automobilismo.}{au.to.mo.bi.lís.ti.co}{0}
\verb{automotivo}{}{}{}{}{adj.}{Relativo a automóvel ou à indústria automobilística.}{au.to.mo.ti.vo}{0}
\verb{automotivo}{}{}{}{}{}{Diz"-se de sistemas ou de materiais usados em veículos automotores.}{au.to.mo.ti.vo}{0}
\verb{automotriz}{}{}{}{}{s.f.}{Veículo ferroviário provido de motor próprio.}{au.to.mo.triz}{0}
\verb{automóvel}{}{}{"-eis}{}{adj.2g.}{Que se locomove por seus próprios meios, sem intervenção de força exterior.}{au.to.mó.vel}{0}
\verb{automóvel}{}{}{"-eis}{}{s.m.}{Qualquer veículo, movido a motor de explosão, destinado ao transporte de passageiros ou carga.}{au.to.mó.vel}{0}
\verb{autonomia}{}{}{}{}{s.f.}{Faculdade de se governar por si próprio.}{au.to.no.mi.a}{0}
\verb{autonomia}{}{}{}{}{}{Liberdade ou independência moral ou intelectual.}{au.to.no.mi.a}{0}
\verb{autonomia}{}{}{}{}{}{Capacidade de um veículo de fazer um percurso sem reabastecer"-se.}{au.to.no.mi.a}{0}
\verb{autônomo}{}{}{}{}{adj.}{Que tem autonomia, que é governado por suas próprias leis.}{au.tô.no.mo}{0}
\verb{autônomo}{}{}{}{}{}{Livre, independente.}{au.tô.no.mo}{0}
\verb{autônomo}{}{}{}{}{s.m.}{Pessoa que trabalha por conta própria.}{au.tô.no.mo}{0}
\verb{auto"-ônibus}{}{}{}{}{s.m.}{Ônibus.}{au.to"-ô.ni.bus}{0}
\verb{autopeça}{é}{}{}{}{s.f.}{Peça ou acessório de veículo automóvel.}{au.to.pe.ça}{0}
\verb{autopeça}{é}{Por ext.}{}{}{}{Local onde se vendem peças de automóvel.}{au.to.pe.ça}{0}
\verb{autopista}{}{}{}{}{s.f.}{Autoestrada.}{au.to.pis.ta}{0}
\verb{autopreservação}{}{}{"-ões}{}{s.f.}{Preservação de si próprio.}{au.to.pre.ser.va.ção}{0}
\verb{autopromoverse}{}{}{}{}{v.pron.}{Vangloriar"-se dos próprios feitos, atributos; promover"-se.}{au.to.pro.mo.ver"-se}{\verboinum{12}}
\verb{autópsia}{}{}{}{}{s.f.}{Exame de si próprio.}{au.tóp.sia}{0}
\verb{autópsia}{}{Med.}{}{}{}{Exame minucioso de um cadáver, para determinar o momento e a causa da morte; necropsia.}{au.tóp.sia}{0}
\verb{autopsiar}{}{}{}{}{v.t.}{Fazer autópsia.}{au.top.si.ar}{\verboinum{6}}
\verb{autopunição}{}{}{"-ões}{}{s.f.}{Punição que alguém dá a si próprio.}{au.to.pu.ni.ção}{0}
\verb{autor}{ô}{}{}{}{s.m.}{Indivíduo que é a causa, a origem.}{au.tor}{0}
\verb{autor}{ô}{}{}{}{}{Indivíduo responsável pela criação de algo; inventor, fundador.}{au.tor}{0}
\verb{autor}{ô}{}{}{}{}{Escritor de obra literária, artística ou científica.}{au.tor}{0}
\verb{autor}{ô}{}{}{}{}{O praticante de uma ação; agente.}{au.tor}{0}
\verb{autor}{ô}{Jur.}{}{}{}{Indivíduo que executa um crime ou delito.}{au.tor}{0}
\verb{autoral}{}{}{"-ais}{}{adj.2g.}{Relativo ou próprio de autor de obra literária, artística ou científica.}{au.to.ral}{0}
\verb{autorama}{}{}{}{}{s.m.}{Autódromo em miniatura para corridas de carro de brinquedo, geralmente elétricos.}{au.to.ra.ma}{0}
\verb{autorretrato}{}{}{autorretratos}{}{s.m.}{Retrato de um indivíduo feito por ele mesmo.}{au.tor.re.tra.to}{0}
\verb{autoria}{}{}{}{}{s.f.}{Qualidade ou condição de autor.}{au.to.ri.a}{0}
\verb{autoria}{}{}{}{}{}{Responsabilidade por um ato.}{au.to.ri.a}{0}
\verb{autoridade}{}{}{}{}{s.f.}{Poder ou direito legítimo de dar ordens, tomar decisões etc.}{au.to.ri.da.de}{0}
\verb{autoridade}{}{}{}{}{}{Superioridade de quem tem esse poder ou direito.}{au.to.ri.da.de}{0}
\verb{autoridade}{}{}{}{}{}{Influência, ascendência, prestígio.}{au.to.ri.da.de}{0}
\verb{autoridade}{}{}{}{}{}{Especialista que tem mérito reconhecido sobre determinado assunto ou campo de conhecimento.}{au.to.ri.da.de}{0}
\verb{autoritário}{}{}{}{}{adj.}{Que usa sua autoridade com rigor.}{au.to.ri.tá.rio}{0}
\verb{autoritário}{}{}{}{}{}{Que não admite oposição.}{au.to.ri.tá.rio}{0}
\verb{autoritário}{}{}{}{}{}{Despótico, dominador, ditador.}{au.to.ri.tá.rio}{0}
\verb{autoritarismo}{}{}{}{}{s.m.}{Qualidade ou caráter do que é autoritário; despotismo.}{au.to.ri.ta.ris.mo}{0}
\verb{autoritarismo}{}{}{}{}{}{Sistema político que limita as liberdades individuais e públicas.}{au.to.ri.ta.ris.mo}{0}
\verb{autorização}{}{}{"-ões}{}{s.f.}{Ato ou efeito de autorizar.}{au.to.ri.za.ção}{0}
\verb{autorização}{}{}{"-ões}{}{}{Consentimento ou permissão para se fazer algo.}{au.to.ri.za.ção}{0}
\verb{autorização}{}{}{"-ões}{}{}{Documento ou registro escrito que expressa essa permissão.}{au.to.ri.za.ção}{0}
\verb{autorizado}{}{}{}{}{adj.}{Que recebeu direito, poder, autorização.}{au.to.ri.za.do}{0}
\verb{autorizado}{}{}{}{}{}{Permitido, expressamente concedido.}{au.to.ri.za.do}{0}
\verb{autorizado}{}{}{}{}{}{Que tem autoridade; digno de crédito; respeitável.}{au.to.ri.za.do}{0}
\verb{autorizar}{}{}{}{}{v.t.}{Dar poder ou autoridade.}{au.to.ri.zar}{0}
\verb{autorizar}{}{}{}{}{}{Conceder licença; consentir, permitir.}{au.to.ri.zar}{0}
\verb{autorizar}{}{}{}{}{}{Tornar possível; dar direito; dar motivo.}{au.to.ri.zar}{0}
\verb{autorizar}{}{}{}{}{}{Justificar, abonar, confirmar.}{au.to.ri.zar}{\verboinum{1}}
\verb{autos}{}{Jur.}{}{}{s.m.pl.}{Conjunto ordenado de todas as peças de um processo judicial.}{au.tos}{0}
\verb{autosserviço}{}{}{autosserviços}{}{s.m.}{Sistema de comercialização em que o próprio cliente se serve, sem o auxílio de um vendedor.}{au.tos.ser.vi.ço}{0}
\verb{autossuficiência}{}{}{autossuficiências}{}{s.f.}{Qualidade ou estado de autossuficiente.}{au.tos.su.fi.ci.ên.cia}{0}
\verb{autossuficiente}{}{}{autossuficientes}{}{adj.2g.}{Que se basta a si mesmo, que é capaz de viver sem depender de outrem; independente.}{au.tos.su.fi.ci.en.te}{0}
\verb{autossugestão}{}{}{autossugestões}{}{s.f.}{Ato ou efeito de autossugestionar.}{au.tos.su.ges.tão}{0}
\verb{autosugestão}{}{}{autossugestões}{}{}{Sugestão que alguém exerce sobre si mesmo.}{au.to.su.ges.tão}{0}
\verb{autosugestionar"-se}{}{}{}{}{v.pron.}{Sugerir a si próprio; convencer"-se.}{au.to.su.ges.ti.o.nar"-se}{\verboinum{1}}
\verb{autosustentável}{}{}{autossustentáveis}{}{adj.2g.}{Que é capaz de assegurar sua própria sobrevivência.}{au.to.sus.ten.tá.vel}{0}
\verb{autuação}{}{}{"-ões}{}{s.f.}{Ato ou efeito de autuar; auto.}{au.tu.a.ção}{0}
\verb{autuação}{}{}{"-ões}{}{}{Termo inicial de um processo.}{au.tu.a.ção}{0}
%\verb{autuado}{}{}{}{}{}{0}{au.tu.a.do}{0}
\verb{autuar}{}{}{}{}{v.t.}{Reunir documentos para formar um processo; processar.}{au.tu.ar}{0}
\verb{autuar}{}{}{}{}{}{Lavrar um auto contra alguém.}{au.tu.ar}{\verboinum{1}}
%\verb{auxiliador}{ss\ldots{}ô}{}{}{}{}{0}{au.xi.li.a.dor}{0}
\verb{auxiliar}{s}{}{}{}{adj.2g.}{Que auxilia, colabora; assistente; ajudante.}{au.xi.li.ar}{0}
\verb{auxiliar}{s}{}{}{}{}{Que tem papel secundário em certa atividade.}{au.xi.li.ar}{0}
\verb{auxiliar}{s}{}{}{}{v.t.}{Prestar auxílio; ajudar, socorrer.}{au.xi.li.ar}{0}
\verb{auxiliar}{s}{}{}{}{}{Contribuir, colaborar, facilitar.}{au.xi.li.ar}{\verboinum{1}}
\verb{auxílio}{s}{}{}{}{s.m.}{Ato ou efeito de auxiliar; socorro, ajuda, cooperação.}{au.xí.lio}{0}
\verb{auxílio}{s}{}{}{}{}{Amparo, proteção, assistência.}{au.xí.lio}{0}
\verb{auxílio}{s}{}{}{}{}{Ajuda material; esmola, óbolo, caridade.}{au.xí.lio}{0}
\verb{avacalhação}{}{}{"-ões}{}{s.f.}{Ato ou efeito de avacalhar.}{a.va.ca.lha.ção}{0}
\verb{avacalhação}{}{}{"-ões}{}{}{Desleixo, bagunça, desorganização.}{a.va.ca.lha.ção}{0}
\verb{avacalhado}{}{}{}{}{adj.}{Que se presta ao escárnio; ridicularizado, desmoralizado.}{a.va.ca.lha.do}{0}
\verb{avacalhado}{}{}{}{}{}{De aspecto descuidado; desleixado, desmazelado.}{a.va.ca.lha.do}{0}
\verb{avacalhar}{}{}{}{}{v.t.}{Pôr em ridículo; desmoralizar.}{a.va.ca.lhar}{0}
\verb{avacalhar}{}{}{}{}{}{Fazer com desleixo, sem cuidado.}{a.va.ca.lhar}{\verboinum{1}}
\verb{aval}{}{Jur.}{avais \textit{ou} avales}{}{s.m.}{Garantia de pagamento de uma letra de câmbio feita por terceiros.}{a.val}{0}
\verb{aval}{}{}{avais \textit{ou} avales}{}{}{Apoio moral ou intelectual; aprovação.}{a.val}{0}
\verb{avalancha}{}{}{}{}{s.f.}{Grande massa de neve ou gelo que despenca rápida e violentamente da encosta de montanhas altas; avalanche.}{a.va.lan.cha}{0}
\verb{avalancha}{}{Fig.}{}{}{}{Grande acúmulo de coisas que incomodam.}{a.va.lan.cha}{0}
\verb{avalancha}{}{Fig.}{}{}{}{Invasão súbita de pessoas ou animais.}{a.va.lan.cha}{0}
\verb{avalanche}{}{}{}{}{}{Var. de \textit{avalancha}.}{a.va.lan.che}{0}
\verb{avaliação}{}{}{"-ões}{}{s.f.}{Ato ou efeito de avaliar.}{a.va.li.a.ção}{0}
\verb{avaliação}{}{}{"-ões}{}{}{Cálculo do valor de um bem.}{a.va.li.a.ção}{0}
\verb{avaliação}{}{}{"-ões}{}{}{Apreciação, análise, consideração.}{a.va.li.a.ção}{0}
\verb{avaliação}{}{}{"-ões}{}{}{Verificação para se determinar a competência de um aluno, um profissional etc.}{a.va.li.a.ção}{0}
\verb{avaliador}{ô}{}{}{}{s.m.}{Profissional encarregado de calcular valores de bens.}{a.va.li.a.dor}{0}
\verb{avaliar}{}{}{}{}{v.t.}{Estabelecer a valia ou o valor de algo.}{a.va.li.ar}{0}
\verb{avaliar}{}{}{}{}{}{Determinar a quantidade; calcular, computar.}{a.va.li.ar}{0}
\verb{avaliar}{}{}{}{}{}{Apreciar o mérito; estimar.}{a.va.li.ar}{0}
\verb{avaliar}{}{}{}{}{}{Ter ideia; conjecturar, imaginar.}{a.va.li.ar}{\verboinum{1}}
\verb{avalista}{}{}{}{}{s.2g.}{Pessoa que dá garantia de pagamento em nome de outra; fiador.}{a.va.lis.ta}{0}
\verb{avalizado}{}{}{}{}{adj.}{Que foi afiançado, garantido, abonado.}{a.va.li.za.do}{0}
\verb{avalizar}{}{}{}{}{v.t.}{Obrigar"-se por aval sobre os títulos de alguém.}{a.va.li.zar}{0}
\verb{avalizar}{}{Fig.}{}{}{}{Abonar, apoiar, afiançar.}{a.va.li.zar}{\verboinum{1}}
\verb{avançado}{}{}{}{}{adj.}{Que está à frente dos outros; adiantado.}{a.van.ça.do}{0}
\verb{avançado}{}{}{}{}{}{Que revela avanço ou progresso em relação à época em que se situa; inovador, liberal.}{a.van.ça.do}{0}
\verb{avançar}{}{}{}{}{v.t.}{Fazer que ande para diante; adiantar.}{a.van.çar}{0}
\verb{avançar}{}{}{}{}{}{Investir com ímpeto; atacar.}{a.van.çar}{0}
\verb{avançar}{}{}{}{}{}{Expor, proferir ideia; aventar.}{a.van.çar}{0}
\verb{avançar}{}{}{}{}{}{Estender"-se, prolongar"-se, expandir"-se.}{a.van.çar}{0}
\verb{avançar}{}{}{}{}{}{Apoderar"-se com avidez ou cobiça.}{a.van.çar}{0}
\verb{avançar}{}{}{}{}{v.i.}{Estar à frente dos outros; progredir, inovar"-se.}{a.van.çar}{0}
\verb{avançar}{}{}{}{}{}{Decorrer, passar o tempo.}{a.van.çar}{\verboinum{3}}
\verb{avanço}{}{}{}{}{s.m.}{Ato ou efeito de avançar; adiantamento.}{a.van.ço}{0}
\verb{avanço}{}{}{}{}{}{Vantagem, dianteira.}{a.van.ço}{0}
\verb{avanço}{}{}{}{}{}{Melhora de estado ou qualidade; progresso.}{a.van.ço}{0}
\verb{avantajado}{}{}{}{}{adj.}{Que tem vantagem; é superior.}{a.van.ta.ja.do}{0}
\verb{avantajado}{}{}{}{}{}{De grandes dimensões; corpulento, volumoso.}{a.van.ta.ja.do}{0}
\verb{avantajar}{}{}{}{}{v.t.}{Levar vantagem; exceder, sobressair.}{a.van.ta.jar}{0}
\verb{avantajar}{}{}{}{}{}{Obter progressos; elevar, promover.}{a.van.ta.jar}{\verboinum{1}}
\verb{avante}{}{}{}{}{adv.}{Adiante, diante.}{a.van.te}{0}
\verb{avante}{}{}{}{}{}{Em direção para frente; para diante}{a.van.te}{0}
\verb{avante}{}{}{}{}{interj.}{Expressão que incita movimento para a frente.}{a.van.te}{0}
\verb{avarandado}{}{}{}{}{adj.}{Que tem varanda, sacada.}{a.va.ran.da.do}{0}
\verb{avarandado}{}{}{}{}{s.m.}{Prédio ou casa que tem varanda.}{a.va.ran.da.do}{0}
\verb{avarandado}{}{}{}{}{}{Espécie de alpendre ou varanda coberta que circunda a casa.}{a.va.ran.da.do}{0}
\verb{avarento}{}{}{}{}{adj.}{Que é extremamente apegado a dinheiro; sovina, mesquinho, pão"-duro, avaro.}{a.va.ren.to}{0}
\verb{avareza}{ê}{}{}{}{s.f.}{Apego excessivo ao dinheiro; sovinice, mesquinhez.}{a.va.re.za}{0}
\verb{avaria}{}{}{}{}{s.f.}{Estrago, perda, dano, prejuízo.}{a.va.ri.a}{0}
\verb{avaria}{}{}{}{}{}{Dano causado a um navio ou a sua carga.}{a.va.ri.a}{0}
\verb{avariar}{}{}{}{}{v.t.}{Causar avaria; danificar, estragar, deteriorar.}{a.va.ri.ar}{\verboinum{1}}
\verb{avaro}{}{}{}{}{adj.}{Que é extremamente apegado a dinheiro; sovina; avarento; mesquinho.}{a.va.ro}{0}
\verb{avascular}{}{Anat.}{}{}{adj.2g.}{Que não apresenta vasos sanguíneos ou linfáticos.}{a.vas.cu.lar}{0}
\verb{avassalador}{ô}{}{}{}{adj.}{Que avassala; dominador, devastador.}{a.vas.sa.la.dor}{0}
\verb{avassalar}{}{}{}{}{v.t.}{Reduzir à condição de vassalo.}{a.vas.sa.lar}{0}
\verb{avassalar}{}{}{}{}{}{Escravizar, dominar, subjugar.}{a.vas.sa.lar}{\verboinum{1}}
\verb{avatar}{}{}{}{}{s.m.}{Metamorfose, transformação, mutação.}{a.va.tar}{0}
\verb{avatar}{}{Relig.}{}{}{}{No Hinduísmo, reencarnação de uma divindade, especialmente de Vixnu.}{a.va.tar}{0}
\verb{ave}{}{Zool.}{}{}{s.f.}{Classe de animais vertebrados ovíparos, de corpo revestido de penas, bico córneo, sem dentes, cujos membros anteriores apresentam forma de asas e servem para voar.}{a.ve}{0}
\verb{ave}{}{}{}{}{interj.}{Expressão empregada como saudação; salve.}{a.ve}{0}
\verb{ave"-do"-paraíso}{}{Zool.}{aves"-do"-paraíso}{}{s.f.}{Nome comum dado a várias espécies de aves da Nova Guiné, famosas pelas longas e belas plumas de um colorido vivo.}{a.ve"-do"-pa.ra.í.so}{0}
\verb{aveia}{ê}{Bot.}{}{}{s.f.}{Erva da família das gramíneas, cujos grãos são ricos em substâncias nutritivas, sendo utilizados na alimentação humana e animal.}{a.vei.a}{0}
\verb{aveia}{ê}{}{}{}{}{O grão dessa planta.}{a.vei.a}{0}
\verb{avejão}{}{}{"-ões}{}{s.m.}{Visão fantasmagórica; espectro, assombração.}{a.ve.jão}{0}
\verb{avejão}{}{}{"-ões}{}{}{Homem muito alto e feio.}{a.ve.jão}{0}
\verb{avelã}{}{}{}{}{s.f.}{O fruto da aveleira.}{a.ve.lã}{0}
\verb{avelaneira}{ê}{}{}{}{s.f.}{Aveleira.}{a.ve.la.nei.ra}{0}
\verb{avelãzeira}{ê}{}{}{}{s.f.}{Aveleira.}{a.ve.lã.zei.ra}{0}
\verb{aveleira}{ê}{Bot.}{}{}{s.f.}{Árvore com inflorescências bulbosas e vermelhas, frutos em forma de noz comestíveis e raízes com propriedades medicinais, cultivada também como ornamental; avelã, avelaneira, avelãzeira.}{a.ve.lei.ra}{0}
\verb{aveludado}{}{}{}{}{adj.}{Que tem aspecto ou textura de veludo.}{a.ve.lu.da.do}{0}
\verb{aveludado}{}{Fig.}{}{}{}{Brando, suave, melodioso (diz"-se de som ou voz).}{a.ve.lu.da.do}{0}
\verb{aveludar}{}{}{}{}{v.t.}{Dar ou adquirir aspecto ou textura de veludo.}{a.ve.lu.dar}{0}
\verb{aveludar}{}{Fig.}{}{}{}{Tornar macio, suave.}{a.ve.lu.dar}{\verboinum{1}}
\verb{ave"-maria}{}{Relig.}{ave"-marias}{}{s.f.}{Oração católica consagrada à Virgem Maria.}{a.ve"-ma.ri.a}{0}
%\verb{ave"-marias}{}{}{}{}{}{0}{a.ve"-ma.ri.as}{0}
\verb{avena}{}{Bot.}{}{}{s.f.}{Aveia.}{a.ve.na}{0}
\verb{avena}{}{Mús.}{}{}{}{Antiga flauta pastoril feita com talo da planta avena.}{a.ve.na}{0}
\verb{avenca}{}{Bot.}{}{}{s.f.}{Designação comum a várias plantas criptogâmicas que vivem em ambientes úmidos e sombrios.}{a.ven.ca}{0}
\verb{avença}{}{}{}{}{s.f.}{Acordo entre litigantes.}{a.ven.ça}{0}
\verb{avenida}{}{}{}{}{s.f.}{Via urbana de circulação, em geral com várias pistas para veículos.}{a.ve.ni.da}{0}
\verb{avental}{}{}{"-ais}{}{s.m.}{Peça de tecido, plástico ou couro usada para proteger a roupa em certos tipos de trabalho, presa pela cintura e eventualmente também pelo pescoço.}{a.ven.tal}{0}
\verb{aventar}{}{}{}{}{v.t.}{Expor ou agitar ao vento; ventilar.}{a.ven.tar}{0}
\verb{aventar}{}{}{}{}{}{Jogar fora; atirar.}{a.ven.tar}{0}
\verb{aventar}{}{}{}{}{}{Expor, enunciar, sugerir, lembrar.}{a.ven.tar}{0}
\verb{aventar}{}{}{}{}{}{Pressentir, adivinhar.}{a.ven.tar}{0}
\verb{aventar}{}{}{}{}{v.i.}{Sentir pelo olfato.}{a.ven.tar}{\verboinum{1}}
\verb{aventar}{}{}{}{}{v.t.}{Segurar (animal) apertando"-lhe o septo nasal.}{a.ven.tar}{\verboinum{1}}
\verb{aventura}{}{}{}{}{s.f.}{Experiência ou empreendimento de caráter incomum e desfecho imprevisível, geralmente envolvendo risco.}{a.ven.tu.ra}{0}
\verb{aventura}{}{}{}{}{}{Circunstância inesperada, acontecimento surpreendente.}{a.ven.tu.ra}{0}
\verb{aventurar}{}{}{}{}{v.t.}{Ousar dizer; fazer declaração de repercussão incerta.}{a.ven.tu.rar}{0}
\verb{aventurar}{}{}{}{}{}{Aventar, sugerir.}{a.ven.tu.rar}{0}
\verb{aventurar}{}{}{}{}{v.pron.}{Arriscar"-se.}{a.ven.tu.rar}{\verboinum{1}}
\verb{aventureiro}{ê}{}{}{}{adj.}{Que vive de aventuras.}{a.ven.tu.rei.ro}{0}
\verb{aventuroso}{ô}{}{"-osos ⟨ó⟩}{"-osa ⟨ó⟩}{adj.}{Que se aventura; arriscado, temerário.}{a.ven.tu.ro.so}{0}
\verb{averbação}{}{}{"-ões}{}{s.f.}{Ato ou efeito de averbar; averbamento.}{a.ver.ba.ção}{0}
\verb{averbamento}{}{}{}{}{s.m.}{Ato ou efeito de averbar; averbação.}{a.ver.ba.men.to}{0}
\verb{averbar}{}{}{}{}{v.t.}{Anotar (algo) na margem de um documento ou registro público.}{a.ver.bar}{0}
\verb{averbar}{}{}{}{}{}{Registrar (depoimento ou termo).}{a.ver.bar}{\verboinum{1}}
\verb{averiguação}{}{}{"-ões}{}{s.f.}{Ato ou efeito de averiguar; investigação.}{a.ve.ri.gua.ção}{0}
\verb{averiguar}{}{}{}{}{v.t.}{Examinar cuidadosamente; inquirir, investigar.}{a.ve.ri.guar}{0}
\verb{averiguar}{}{}{}{}{}{Determinar a veracidade; verificar, apurar, certificar"-se.}{a.ve.ri.guar}{\verboinum{10}\verboirregular[averiguo]{averíguo}}
\verb{avermelhado}{}{}{}{}{adj.}{Que tem cor semelhante à vermelha.}{a.ver.me.lha.do}{0}
\verb{avermelhar}{}{}{}{}{v.t.}{Tornar vermelho ou avermelhado.}{a.ver.me.lhar}{\verboinum{1}}
\verb{aversão}{}{}{"-ões}{}{s.f.}{Sentimento de repulsa; antipatia, repugnância.}{a.ver.são}{0}
\verb{aversão}{}{}{"-ões}{}{}{Ódio, rancor.}{a.ver.são}{0}
\verb{avesso}{ê}{}{}{}{adj.}{Inverso, contrário.}{a.ves.so}{0}
\verb{avesso}{ê}{}{}{}{}{Que é contra algo; hostil.}{a.ves.so}{0}
\verb{avesso}{ê}{}{}{}{s.m.}{O lado oposto ao principal; o reverso.}{a.ves.so}{0}
\verb{avesso}{ê}{}{}{}{}{O lado mal, negativo.}{a.ves.so}{0}
\verb{avestruz}{}{Zool.}{}{}{s.2g.}{Ave de grande porte com pernas longas, pés com dois dedos, asas atrofiadas e onívora.}{a.ves.truz}{0}
\verb{avexado}{ch}{}{}{}{}{Var. de \textit{vexado}.}{a.ve.xa.do}{0}
\verb{avexar}{ch}{}{}{}{}{Var. de \textit{vexar}.}{a.ve.xar}{0}
\verb{avezar}{}{}{}{}{v.t.}{Acostumar, habituar.}{a.ve.zar}{\verboinum{1}}
\verb{aviação}{}{}{"-ões}{}{s.f.}{Navegação aérea.}{a.vi.a.ção}{0}
\verb{aviação}{}{}{"-ões}{}{}{Conjunto de aeronaves.}{a.vi.a.ção}{0}
\verb{aviado}{}{}{}{}{adj.}{Que se aviou.}{a.vi.a.do}{0}
\verb{aviado}{}{}{}{}{}{Preparado, executado, encaminhado.}{a.vi.a.do}{0}
\verb{aviado}{}{}{}{}{}{Apressado.}{a.vi.a.do}{0}
\verb{aviador}{ô}{}{}{}{adj.}{Que avia.}{a.vi.a.dor}{0}
\verb{aviador}{ô}{}{}{}{s.m.}{Piloto de avião.}{a.vi.a.dor}{0}
\verb{aviamento}{}{}{}{}{s.m.}{Ato ou efeito de aviar.}{a.vi.a.men.to}{0}
\verb{aviamento}{}{}{}{}{}{O material ou equipamento necessário para se concluir algo.}{a.vi.a.men.to}{0}
\verb{aviamento}{}{}{}{}{}{Conjunto dos materiais usados para dar acabamento em costura ou bordado, como botões, fechos, linhas, viés etc.}{a.vi.a.men.to}{0}
\verb{avião}{}{}{"-ões}{}{s.m.}{Veículo dotado de propulsão e asas, que lhe dão sustentação no ar.}{a.vi.ão}{0}
\verb{aviar}{}{}{}{}{v.t.}{Executar, efetuar, concluir.}{a.vi.ar}{0}
\verb{aviar}{}{}{}{}{}{Preparar medicamento.}{a.vi.ar}{0}
\verb{aviar}{}{}{}{}{}{Providenciar o necessário para determinada finalidade.}{a.vi.ar}{\verboinum{1}}
\verb{aviário}{}{}{}{}{s.m.}{Viveiro de aves.}{a.vi.á.rio}{0}
\verb{aviário}{}{}{}{}{}{Estabelecimento onde se vendem aves.}{a.vi.á.rio}{0}
\verb{aviário}{}{}{}{}{adj.}{Relativo a aves; avícola.}{a.vi.á.rio}{0}
\verb{aviatório}{}{}{}{}{adj.}{Referente a aviação.}{a.vi.a.tó.rio}{0}
\verb{avícola}{}{}{}{}{s.2g.}{Indivíduo que se dedica à avicultura; avicultor.}{a.ví.co.la}{0}
\verb{avícola}{}{}{}{}{adj.2g.}{Referente a aves; aviário.}{a.ví.co.la}{0}
\verb{avícula}{}{}{}{}{s.f.}{Ave pequena.}{a.ví.cu.la}{0}
\verb{avicultor}{ô}{}{}{}{s.m.}{Indivíduo que se dedica à avicultura; avícola.}{a.vi.cul.tor}{0}
\verb{avicultura}{}{}{}{}{s.f.}{Técnica de criar aves.}{a.vi.cul.tu.ra}{0}
\verb{avicultura}{}{}{}{}{}{Criação de aves.}{a.vi.cul.tu.ra}{0}
\verb{avidez}{ê}{}{}{}{s.f.}{Qualidade de quem é ávido.}{a.vi.dez}{0}
\verb{avidez}{ê}{}{}{}{}{Desejo intenso.}{a.vi.dez}{0}
\verb{avidez}{ê}{}{}{}{}{Ansiedade.}{a.vi.dez}{0}
\verb{ávido}{}{}{}{}{adj.}{Que deseja ardentemente.}{á.vi.do}{0}
\verb{ávido}{}{}{}{}{}{Ansioso.}{á.vi.do}{0}
\verb{ávido}{}{}{}{}{}{Ambicioso, avarento.}{á.vi.do}{0}
\verb{ávido}{}{}{}{}{}{Voraz, sedento, esfomeado.}{á.vi.do}{0}
\verb{avigorar}{}{}{}{}{v.t.}{Dar vigor; fortalecer.}{a.vi.go.rar}{\verboinum{1}}
\verb{aviltado}{}{}{}{}{adj.}{Que se aviltou.}{a.vil.ta.do}{0}
\verb{aviltado}{}{}{}{}{}{Desonrado, humilhado.}{a.vil.ta.do}{0}
\verb{aviltado}{}{}{}{}{}{Desvalorizado.}{a.vil.ta.do}{0}
%\verb{aviltante}{}{}{}{}{}{0}{a.vil.tan.te}{0}
\verb{aviltar}{}{}{}{}{v.t.}{Tornar vil.}{a.vil.tar}{0}
\verb{aviltar}{}{}{}{}{}{Desonrar, humilhar.}{a.vil.tar}{0}
\verb{aviltar}{}{}{}{}{}{Baixar o preço; desvalorizar.}{a.vil.tar}{\verboinum{1}}
\verb{avinagrado}{}{}{}{}{adj.}{Que tem gosto ou cheiro de vinagre; acre, azedo.}{a.vi.na.gra.do}{0}
\verb{avinagrado}{}{Fig.}{}{}{}{Irritado, azedo, zangado.}{a.vi.na.gra.do}{0}
\verb{avindo}{}{}{}{}{adj.}{Que se aveio; combinado, ajustado.}{a.vin.do}{0}
\verb{avinhado}{}{}{}{}{adj.}{Que tem sabor, cheiro ou cor de vinho.}{a.vi.nha.do}{0}
\verb{avinhado}{}{}{}{}{}{Embriagado.}{a.vi.nha.do}{0}
\verb{avios}{}{}{}{}{s.m.pl.}{Os materiais e equipamentos necessários à execução de algo; aviamentos.}{a.vi.os}{0}
\verb{avir}{}{}{}{}{v.t.}{Pôr em harmonia; conciliar. (\textit{Se não fizer como eu falo, vai se avir comigo.})}{a.vir}{0}
\verb{avir}{}{}{}{}{}{Combinar, arranjar. (\textit{Vamo"-nos avir para a viagem.})}{a.vir}{\verboinum{56}}
\verb{avisado}{}{}{}{}{adj.}{Que recebeu aviso.}{a.vi.sa.do}{0}
\verb{avisado}{}{}{}{}{}{Cauteloso, prudente.}{a.vi.sa.do}{0}
\verb{avisado}{}{}{}{}{}{Discreto, sensato, moderado.}{a.vi.sa.do}{0}
\verb{avisar}{}{}{}{}{v.t.}{Fazer saber; informar, comunicar.}{a.vi.sar}{0}
\verb{avisar}{}{}{}{}{}{Prevenir; comunicar com antecedência.}{a.vi.sar}{0}
\verb{avisar}{}{}{}{}{}{Aconselhar, recomendar.}{a.vi.sar}{\verboinum{1}}
\verb{aviso}{}{}{}{}{s.m.}{Ato ou efeito de avisar.}{a.vi.so}{0}
\verb{aviso}{}{}{}{}{}{Informação, comunicação, declaração.}{a.vi.so}{0}
\verb{aviso}{}{Por ext.}{}{}{}{O documento pelo qual se dá a informação.}{a.vi.so}{0}
\verb{aviso}{}{}{}{}{}{Advertência, conselho.}{a.vi.so}{0}
\verb{avistar}{}{}{}{}{v.t.}{Ver, enxergar.}{a.vis.tar}{0}
\verb{avistar}{}{}{}{}{}{Entrever, vislumbrar.}{a.vis.tar}{0}
\verb{avistar}{}{}{}{}{}{Pressentir, perceber.}{a.vis.tar}{0}
\verb{avistar}{}{}{}{}{v.pron.}{Encontrar"-se por acaso; topar.}{a.vis.tar}{\verboinum{1}}
\verb{avitaminose}{ó}{Med.}{}{}{s.f.}{Doença causada por falta de vitamina(s).}{a.vi.ta.mi.no.se}{0}
\verb{avivar}{}{}{}{}{v.t.}{Tornar mais vivo; estimular, animar.}{a.vi.var}{\verboinum{1}}
\verb{aviventar}{}{}{}{}{v.t.}{Avivar, reanimar, vivificar.}{a.vi.ven.tar}{0}
\verb{aviventar}{}{}{}{}{}{Realçar.}{a.vi.ven.tar}{\verboinum{1}}
%\verb{avizinhação}{}{}{}{}{}{0}{a.vi.zi.nha.ção}{0}
\verb{avizinhar}{}{}{}{}{v.t.}{Tornar vizinho, aproximar, pôr perto.}{a.vi.zi.nhar}{0}
\verb{avizinhar}{}{}{}{}{}{Tocar, confiar.}{a.vi.zi.nhar}{\verboinum{1}}
\verb{avo}{}{}{}{}{s.m.}{Avos.}{a.vo}{0}
\verb{avó}{}{}{}{}{s.f.}{A mãe do pai ou da mãe.}{a.vó}{0}
\verb{avô}{}{}{}{avó}{s.m.}{O pai do pai ou da mãe.}{a.vô}{0}
\verb{avoado}{}{Pop.}{}{}{adj.}{Que anda com a cabeça no ar, aéreo; distraído.}{a.vo.a.do}{0}
\verb{avoante}{}{}{}{}{adj.2g.}{Que voa.}{a.vo.an.te}{0}
\verb{avoante}{}{Zool.}{}{}{s.m.}{Pomba que vive em bandos migratórios, reunindo"-se em certas regiões do Nordeste em março e abril, quando são então caçadas.}{a.vo.an.te}{0}
\verb{avocação}{}{}{"-ões}{}{s.f.}{Ato ou efeito de avocar.}{a.vo.ca.ção}{0}
\verb{avocação}{}{Jur.}{"-ões}{}{}{Chamamento de uma causa a juízo superior.}{a.vo.ca.ção}{0}
\verb{avocar}{}{}{}{}{v.t.}{Atrair, chamar a si; fazer voltar.}{a.vo.car}{0}
\verb{avocar}{}{}{}{}{}{Atribuir"-se, arrogar"-se.}{a.vo.car}{0}
\verb{avocar}{}{Jur.}{}{}{}{Deslocar uma causa para outro tribunal.}{a.vo.car}{\verboinum{2}}
\verb{avoengo}{}{}{}{}{adj.}{Procedente ou herdado de avós.}{a.vo.en.go}{0}
\verb{avoengos}{}{}{}{}{s.m.pl.}{Antepassados; avós.}{a.vo.en.gos}{0}
\verb{avolumar}{}{}{}{}{v.t.}{Aumentar o volume, crescer.}{a.vo.lu.mar}{\verboinum{1}}
\verb{à"-vontade}{}{}{à"-vontades}{}{s.m.}{Condição, estado de quem está à vontade, sem embaraço; naturalidade, desinibição.}{à"-von.ta.de}{0}
\verb{avos}{}{}{}{}{s.m.pl.}{Partes iguais em que a unidade ou o inteiro foi dividido.}{a.vos}{0}
\verb{avós}{}{}{}{}{s.m.pl.}{Antepassados, avoengos.}{a.vós}{0}
\verb{avós}{}{}{}{}{}{O avô e a avó. }{a.vós}{0}
\verb{avulsão}{}{Med.}{"-ões}{}{s.f.}{Ação de extrair ou arrancar com violência; extração.}{a.vul.são}{0}
\verb{avulsão}{}{Med.}{"-ões}{}{}{Extração de um orgão ou de parte dele.}{a.vul.são}{0}
\verb{avulso}{}{}{}{}{adj.}{Que não faz parte de coleção; solto.}{a.vul.so}{0}
\verb{avulso}{}{}{}{}{}{Desligado daquilo a que pertence. }{a.vul.so}{0}
\verb{avultado}{}{}{}{}{adj.}{Que avulta; volumoso, corpulento, vultoso.}{a.vul.ta.do}{0}
\verb{avultado}{}{}{}{}{}{Grande, considerável.}{a.vul.ta.do}{0}
\verb{avultar}{}{}{}{}{v.t.}{Ganhar vulto; fazer parecer maior; exagerar.}{a.vul.tar}{0}
\verb{avultar}{}{}{}{}{}{Acrescentar, engrandecer.}{a.vul.tar}{0}
\verb{avultar}{}{}{}{}{}{Representar em vulto ou em relevo.}{a.vul.tar}{0}
\verb{avultar}{}{}{}{}{v.i.}{Sobressair, realçar.}{a.vul.tar}{0}
\verb{avultar}{}{}{}{}{}{Crescer, aumentar.}{a.vul.tar}{\verboinum{1}}
\verb{avuncular}{}{}{}{}{adj.2g.}{Relativo ou pertencente ao tio materno.}{a.vun.cu.lar}{0}
\verb{axadrezado}{ch}{}{}{}{adj.}{Diz"-se de tecido, papel ou outro material que apresenta quadrados, desenhados ou pintados, de duas ou mais cores que se alternam, imitando um tabuleiro de xadrez; xadrez.  }{a.xa.dre.za.do}{0}
\verb{axial}{cs}{}{"-ais}{}{adj.2g.}{Relativo a eixo, que forma eixo, que serve de eixo.}{a.xi.al}{0}
\verb{axila}{cs}{Anat.}{}{}{s.f.}{Cavidade na parte inferior na junção do braço com o tronco; sovaco.}{a.xi.la}{0}
\verb{axiologia}{cs}{Filos.}{}{}{s.f.}{Estudo ou teoria dos valores.}{a.xi.o.lo.gi.a}{0}
\verb{axioma}{cs}{}{}{}{s.m.}{Na lógica aristotélica,  ponto de partida de um raciocínio, considerado como indemonstrável, evidente.}{a.xi.o.ma}{0}
\verb{axioma}{cs}{Por ext.}{}{}{}{Verdade ou princípio evidente por si mesmo; máxima, sentença; verdade intuitiva.}{a.xi.o.ma}{0}
\verb{axiomático}{cs}{}{}{}{adj.}{Relativo ao axioma; que tem valor, caráter de axioma; evidente, incontestável.}{a.xi.o.má.ti.co}{0}
\verb{azado}{}{}{}{}{adj.}{Conveniente, adequado, propício, oportuno, próprio.}{a.za.do}{0}
\verb{azado}{}{}{}{}{}{Ágil, jeitoso.}{a.za.do}{0}
\verb{azáfama}{}{}{}{}{s.f.}{Pressa acompanhada de esforço e atrapalhação.}{a.zá.fa.ma}{0}
\verb{azáfama}{}{}{}{}{}{Muita pressa, urgência, grande afã; corre"-corre.}{a.zá.fa.ma}{0}
\verb{azafamado}{}{}{}{}{adj.}{Que tem muita pressa; apressado.}{a.za.fa.ma.do}{0}
\verb{azafamado}{}{}{}{}{}{Muito ocupado; sobrecarregado de trabalho.}{a.za.fa.ma.do}{0}
\verb{azafamar}{}{}{}{}{v.t.}{Sobrecarregar de trabalho; pôr em azáfama.}{a.za.fa.mar}{0}
\verb{azafamar}{}{}{}{}{}{Apressar.}{a.za.fa.mar}{\verboinum{1}}
\verb{azagaia}{}{}{}{}{s.f.}{Lança curta de arremesso, usada pelos mouros e por certos povos africanos.}{a.za.gai.a}{0}
\verb{azálea}{}{Bot.}{}{}{s.f.}{Arbusto ornamental, originário da China, de flores que vão do branco ao vermelho.}{a.zá.le.a}{0}
\verb{azaleia}{é}{}{}{}{}{Var. de \textit{azálea}.}{a.za.lei.a}{0}
\verb{azar}{}{}{}{}{s.m.}{Má sorte, desdita.}{a.zar}{0}
\verb{azar}{}{}{}{}{}{Infelicidade, fatalidade.}{a.zar}{0}
\verb{azar}{}{}{}{}{}{Acaso, casualidade.}{a.zar}{0}
\verb{azarado}{}{}{}{}{adj.}{Que tem azar, má sorte; infeliz; caipora.}{a.za.ra.do}{0}
\verb{azarão}{}{Pop.}{"-ões}{}{s.m.}{O cavalo menos cotado para vencer uma corrida.}{a.za.rão}{0}
\verb{azarão}{}{Por ext.}{"-ões}{}{}{Qualquer indivíduo, animal ou veículo menos cotado para vencer uma competição.}{a.za.rão}{0}
\verb{azarar}{}{}{}{}{v.t.}{Dar azar ou má sorte.}{a.za.rar}{0}
\verb{azarar}{}{Pop.}{}{}{}{Paquerar, tentar conquistar alguém.}{a.za.rar}{\verboinum{1}}
\verb{azarento}{}{}{}{}{adj.}{Que tem azar; azarado, infeliz.}{a.za.ren.to}{0}
\verb{azarento}{}{}{}{}{}{Que dá azar; aziago, nefasto.}{a.za.ren.to}{0}
%\verb{azedamento}{}{}{}{}{}{0}{a.ze.da.men.to}{0}
\verb{azedar}{}{}{}{}{v.t.}{Tornar azedo; coalhar (leite), acidificar, avinagrar.}{a.ze.dar}{0}
\verb{azedar}{}{Fig.}{}{}{}{Causar mau humor; irritar, exasperar.}{a.ze.dar}{\verboinum{1}}
\verb{azedia}{}{}{}{}{s.f.}{Azedume.}{a.ze.di.a}{0}
\verb{azedo}{ê}{}{}{}{adj.}{Que tem o sabor semelhante ao do vinagre ou do limão; ácido, acre.}{a.ze.do}{0}
\verb{azedo}{ê}{}{}{}{}{Diz"-se do alimento que a fermentação estragou.}{a.ze.do}{0}
\verb{azedo}{ê}{Fig.}{}{}{s.m.}{Indivíduo ríspido, irritado, de mau humor, exacerbado.}{a.ze.do}{0}
\verb{azedume}{}{}{}{}{s.m.}{Estado daquilo que é azedo, ácido; acidez, amargor.}{a.ze.du.me}{0}
\verb{azedume}{}{Fig.}{}{}{}{Aspereza, animosidade.}{a.ze.du.me}{0}
\verb{azeitar}{}{}{}{}{v.t.}{Temperar ou untar com azeite.}{a.zei.tar}{0}
\verb{azeitar}{}{}{}{}{}{Lubrificar.}{a.zei.tar}{0}
\verb{azeitar}{}{Pop.}{}{}{}{Entrar em acordo, afinar.}{a.zei.tar}{\verboinum{1}}
\verb{azeite}{}{}{}{}{s.m.}{Óleo extraído da azeitona.}{a.zei.te}{0}
\verb{azeite}{}{Por ext.}{}{}{}{Óleo comestível extraído de diversos vegetais ou da gordura de certos animais.}{a.zei.te}{0}
\verb{azeite"-de"-dendê}{}{}{azeites"-de"-dendê}{}{s.m.}{Azeite extraído do fruto da palmeira do dendê.}{a.zei.te"-de"-den.dê}{0}
\verb{azeiteira}{ê}{}{}{}{s.f.}{Pequeno recipiente de vidro usado para servir azeite; azeiteiro, galheta, almotolia. }{a.zei.tei.ra}{0}
\verb{azeiteiro}{ê}{}{}{}{adj.}{Relativo a azeite.}{a.zei.tei.ro}{0}
\verb{azeiteiro}{ê}{}{}{}{s.m.}{Indivíduo que fabrica ou vende azeite.}{a.zei.tei.ro}{0}
\verb{azeiteiro}{ê}{}{}{}{}{Pequeno recipiente de vidro usado para servir azeite; azeiteira, almotolia, galheta. }{a.zei.tei.ro}{0}
\verb{azeitona}{}{}{}{}{s.f.}{Fruto da oliveira; oliva.}{a.zei.to.na}{0}
\verb{azeitonado}{}{}{}{}{adj.}{Que é semelhante a azeitona na cor, aspecto ou sabor.}{a.zei.to.na.do}{0}
%\verb{azeitoneira}{ê}{}{}{}{}{0}{a.zei.to.nei.ra}{0}
\verb{azêmola}{}{}{}{}{s.f.}{Animal de carga que forma récua com outras.}{a.zê.mo.la}{0}
\verb{azêmola}{}{}{}{}{}{Cavalo velho e cansado.}{a.zê.mo.la}{0}
\verb{azêmola}{}{Fig.}{}{}{}{Pessoa estúpida ou inútil.}{a.zê.mo.la}{0}
\verb{azenha}{}{}{}{}{s.f.}{Moinho de roda movido a água.}{a.ze.nha}{0}
\verb{azerbaidjano}{}{}{}{}{adj.}{Relativo ao Azerbaidjão.}{a.zer.baid.ja.no}{0}
\verb{azerbaidjano}{}{}{}{}{s.m.}{Indivíduo natural ou habitante desse país; azeri.}{a.zer.baid.ja.no}{0}
\verb{azerbaidjano}{}{}{}{}{}{Língua falada no Azerbaidjão, no noroeste do Irã, em outras ex"-repúblicas soviéticas do Cáucaso, na Turquia e no Iraque; azeri.}{a.zer.baid.ja.no}{0}
\verb{azeri}{}{}{}{}{adj. e s.m.  }{Azerbaidjano.}{a.ze.ri}{0}
\verb{azeviche}{}{}{}{}{s.m.}{Carvão fóssil, de cor muito negra.}{a.ze.vi.che}{0}
\verb{azeviche}{}{}{}{}{}{Essa cor.}{a.ze.vi.che}{0}
\verb{azeviche}{}{Fig.}{}{}{}{Coisa muito preta.}{a.ze.vi.che}{0}
\verb{azia}{}{}{}{}{s.f.}{Sensação de ardor, de queimação  no estômago; pirose.}{a.zi.a}{0}
\verb{aziago}{}{}{}{}{adj.}{Que traz mal agouro; que anuncia desgraça; agourento.}{a.zi.a.go}{0}
\verb{aziago}{}{}{}{}{}{Infeliz, infausto.}{a.zi.a.go}{0}
\verb{ázimo}{}{}{}{}{adj.}{Que não é fermentado; sem levedura.}{á.zi.mo}{0}
\verb{ázimo}{}{Relig.}{}{}{s.m.}{Espécie de pão feito sem fermento.}{á.zi.mo}{0}
\verb{azimute}{}{}{}{}{s.m.}{}{a.zi.mu.te}{0}
\verb{azinhavrar}{}{}{}{}{v.i.}{Cobrir"-se de azinhavre.}{a.zi.nha.vrar}{\verboinum{1}}
\verb{azinhavre}{}{}{}{}{s.m.}{Substância verde que se forma em objetos de cobre expostos ao ar ou à umidade; zinabre.}{a.zi.nha.vre}{0}
\verb{azo}{}{}{}{}{s.m.}{Ocasião conveniente, cômoda. }{a.zo}{0}
\verb{azo}{}{}{}{}{}{Pretexto, motivo, ensejo.}{a.zo}{0}
\verb{azorrague}{}{}{}{}{s.m.}{Açoite com uma só ou várias tiras de couro atada(s) a um pedaço de pau, com que os cocheiros incitam os animais; chicote.}{a.zor.ra.gue}{0}
\verb{azorrague}{}{Fig.}{}{}{}{Castigo, flagelo.}{a.zor.ra.gue}{0}
%\verb{azotado}{}{}{}{}{}{0}{a.zo.ta.do}{0}
%\verb{azotar}{}{}{}{}{}{0}{a.zo.tar}{0}
\verb{azótico}{}{Quím.}{}{}{adj.}{Diz"-se de um ácido muito reativo, formado pela combinação de um átomo de hidrogênio, um de nitrogênio e três de oxigênio, muito usado na indústria; nítrico.}{a.zó.ti.co}{0}
\verb{azoto}{ô}{Quím.}{}{}{s.m.}{Elemento químico do grupo dos não metais, gasoso, incolor, inodoro, abundante na atmosfera, usado na indústria de explosivos e em indústrias que necessitam de atmosfera inerte; nitrogênio.}{a.zo.to}{0}
\verb{azougado}{}{}{}{}{adj.}{Que é muito agitado; muito esperto, vivo ou ladino.}{a.zou.ga.do}{0}
\verb{azougar}{}{}{}{}{v.t.}{Cobrir ou juntar com azougue; amalgamar.}{a.zou.gar}{0}
\verb{azougar}{}{Fig.}{}{}{}{Tornar esperto, vivo.}{a.zou.gar}{\verboinum{5}}
\verb{azougue}{}{Quím.}{}{}{s.m.}{O mesmo que \textit{mercúrio}.}{a.zou.gue}{0}
\verb{azougue}{}{Fig.}{}{}{}{Indivíduo muito esperto ou muito inquieto; pessoa azougada.}{a.zou.gue}{0}
\verb{AZT}{}{Med.}{}{}{s.m.}{Sigla de azidotimidina, medicamento usado no tratamento da \textsc{aids}, que impede a multiplicação do vírus.}{AZT}{0}
%\verb{azucrinado}{}{}{}{}{}{0}{a.zu.cri.na.do}{0}
\verb{azucrinante}{}{}{}{}{adj.2g.}{Que azucrina; irritante, importuno, maçante.}{a.zu.cri.nan.te}{0}
\verb{azucrinar}{}{}{}{}{v.t.}{Importunar, perseguir com lamúrias ou choro; irritar; apoquentar.}{a.zu.cri.nar}{\verboinum{1}}
\verb{azul}{}{}{"-uis}{}{adj.2g.}{Que tem uma cor semelhante à do céu sem nuvens.}{a.zul}{0}
\verb{azul}{}{}{"-uis}{}{s.m.}{A cor azul, em todas as suas gradações, entre o violeta e o verde no espectro da luz solar.}{a.zul}{0}
\verb{azulado}{}{}{}{}{adj.}{Que tem cor semelhante ao azul; anilado.}{a.zu.la.do}{0}
\verb{azulão}{}{Zool.}{"-ões}{}{s.m.}{Nome comum a várias espécies de pássaros azuis.}{a.zu.lão}{0}
\verb{azulão}{}{Zool.}{"-ões}{}{}{Certa espécie de siri.}{a.zu.lão}{0}
\verb{azulão}{}{}{"-ões}{}{}{Tecido grosseiro de algodão, de cor azul; zuarte.}{a.zu.lão}{0}
\verb{azulão}{}{}{"-ões}{}{s.m.}{A cor azul bem escura.}{a.zu.lão}{0}
\verb{azular}{}{}{}{}{v.t.}{Tornar azul; anilar.}{a.zu.lar}{0}
\verb{azular}{}{Pop.}{}{}{}{Sair ou fugir apressadamente; desaparecer.}{a.zu.lar}{\verboinum{1}}
\verb{azul"-celeste}{é}{}{\textit{do s.m.: }azuis"-celestes ⟨é⟩}{}{adj.}{Diz"-se do azul da cor do céu.}{a.zul"-ce.les.te}{0}
\verb{azul"-celeste}{é}{}{\textit{do s.m.: }azuis"-celestes ⟨é⟩}{}{s.m.}{A cor azul"-celeste.}{a.zul"-ce.les.te}{0}
\verb{azul"-de"-metileno}{}{Quím.}{azuis"-de"-metileno}{}{s.m.}{Corante azul, derivado do alcatrão mineral, usado em tinturaria e como antisséptico.}{a.zul"-de"-me.ti.le.no}{0}
\verb{azulejador}{ô}{}{}{}{s.m.}{Indivíduo que faz ou coloca azulejos; azulejista; ladrilheiro.}{a.zu.le.ja.dor}{0}
\verb{azulejar}{}{}{}{}{v.t.}{Assentar azulejos; ladrilhar.}{a.zu.le.jar}{\verboinum{1}}
\verb{azulejista}{}{}{}{}{s.2g.}{Indivíduo que fabrica ou assenta azulejos; azulejador; ladrilheiro.}{a.zu.le.jis.ta}{0}
\verb{azulejo}{ê}{}{}{}{s.m.}{Ladrilho vitrificado, próprio para revestir ou guarnecer paredes.}{a.zu.le.jo}{0}
\verb{azul"-ferrete}{ê}{}{\textit{do s.m.: }azuis"-ferretes \textit{ou} azuis"-ferrete ⟨ê⟩}{}{adj.2g.}{Diz"-se de um azul muito escuro, quase preto.}{a.zul"-fer.re.te}{0}
\verb{azul"-ferrete}{ê}{}{\textit{do s.m.: }azuis"-ferretes \textit{ou} azuis"-ferrete ⟨ê⟩}{}{s.m.}{Essa cor; azul"-turquesa.}{a.zul"-fer.re.te}{0}
\verb{azul"-marinho}{}{}{\textit{do s.m.: }azuis"-marinhos }{}{adj.2g.}{Diz"-se do azul muito escuro, da cor do mar profundo.}{a.zul"-ma.ri.nho}{0}
\verb{azul"-marinho}{}{}{\textit{do s.m.: }azuis"-marinhos }{}{s.m.}{A cor azul"-marinho.}{a.zul"-ma.ri.nho}{0}
\verb{azul"-piscina}{}{}{\textit{do s.m.: }azuis"-piscina \textit{e} azuis"-piscinas}{}{adj.2g.}{Diz"-se de uma cor semelhante à da água clorada da piscina com fundo azul. }{a.zul"-pis.ci.na}{0}
\verb{azul"-piscina}{}{}{\textit{do s.m.: }azuis"-piscina \textit{e} azuis"-piscinas}{}{s.m.}{Essa cor.}{a.zul"-pis.ci.na}{0}
\verb{azul"-turquesa}{ê}{}{\textit{do s.m.: }azuis"-turquesa \textit{e} azuis"-turquesas ⟨ê⟩}{}{adj.2g.}{Diz"-se da cor azul do tom da turquesa, mineral azul"-esverdeado.}{a.zul"-tur.que.sa}{0}
\verb{azul"-turquesa}{ê}{}{\textit{do s.m.: }azuis"-turquesa \textit{e} azuis"-turquesas ⟨ê⟩}{}{s.m.}{Essa cor; azul"-ferrete.}{a.zul"-tur.que.sa}{0}
