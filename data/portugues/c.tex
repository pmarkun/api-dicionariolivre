\verb{c}{}{}{}{}{s.m.}{Terceira letra do alfabeto português.}{c}{0}
\verb{C}{}{}{}{}{}{Algarismo romano equivalente a \textit{100}.}{C}{0}
\verb{C}{}{Mat.}{}{}{}{No sistema hexadecimal, representa o décimo terceiro algarismo, equivalente ao número decimal 12.}{C}{0}
\verb{C}{}{Mús.}{}{}{}{A nota ou acorde referente ao \textit{dó}, ou à primeira nota da escala de \textit{dó}.}{C}{0}
\verb{C}{}{Quím.}{}{}{}{Símb. do \textit{carbono}.}{C}{0}
\verb{Ca}{}{Quím.}{}{}{}{Símb. do \textit{cálcio}.}{Ca}{0}
\verb{cã}{}{}{}{}{s.f.}{Cabelo branco.}{cã}{0}
\verb{cã}{}{}{}{}{s.m.}{Título de alguns chefes ou soberanos orientais.}{cã}{0}
\verb{cá}{}{}{}{}{adv.}{Neste lugar; aqui. (\textit{Cá está você!})}{cá}{0}
\verb{cá}{}{}{}{}{s.m.}{Nome da letra \textit{k}.}{cá}{0}
\verb{cá}{}{}{}{}{}{Para este lugar. (\textit{Venha cá, por favor.})}{cá}{0}
\verb{caaba}{}{}{}{}{s.f.}{Templo muçulmano, em Meca, particularmente venerado pelos maometanos.}{ca.a.ba}{0}
\verb{caaba}{}{}{}{}{}{A pedra sagrada encontrada nesse templo.}{ca.a.ba}{0}
\verb{caapora}{ó}{Mit.}{}{}{s.2g.}{O mesmo que caipora.}{ca.a.po.ra}{0}
\verb{caatinga}{}{}{}{}{s.f.}{Vegetação típica do nordeste brasileiro e do norte de Minas, formada por pequenas árvores, comumente espinhosas, que perdem as folhas na estação seca.}{ca.a.tin.ga}{0}
\verb{caatinga}{}{}{}{}{}{Região onde existe essa vegetação.}{ca.a.tin.ga}{0}
\verb{caba}{}{}{}{}{s.f.}{Nome dado ao marimbondo.}{ca.ba}{0}
\verb{cabaça}{}{}{}{}{s.f.}{Fruto da cabaceira, de cor amarela e comestível quando pequeno.}{ca.ba.ça}{0}
\verb{cabaça}{}{}{}{}{}{A casca desse fruto, que depois de seca é utilizada como cuia ou tigela, ou para confecção de instrumentos de percussão.}{ca.ba.ça}{0}
\verb{cabaça}{}{}{}{}{s.m.}{A criança gêmea que nasce em segundo lugar.}{ca.ba.ça}{0}
\verb{cabaceira}{ê}{Bot.}{}{}{s.f.}{Árvore baixa, de caule tortuoso, flores solitárias e grandes, fornece madeira dura e forte, própria para marcenaria, e cujo fruto é usado como tigela, cuia ou instrumento musical; cuieira, coité.}{ca.ba.cei.ra}{0}
\verb{cabaceiro}{ê}{Bot.}{}{}{s.m.}{Árvore de flores alvas ou amareladas e de madeira branca e leve, própria para caixotaria e fabrico de papel.}{ca.ba.cei.ro}{0}
%\verb{}{}{}{}{}{}{}{}{0}
\verb{cabaço}{}{}{}{}{s.m.}{O fruto da cabaceira.}{ca.ba.ço}{0}
\verb{cabaço}{}{}{}{}{}{Recipiente que se faz com esse fruto.}{ca.ba.ço}{0}
\verb{cabaço}{}{Pop.}{}{}{}{O hímem.}{ca.ba.ço}{0}
\verb{cabaço}{}{Pop.}{}{}{}{O homem ou a mulher virgem.}{ca.ba.ço}{0}
%\verb{}{}{}{}{}{}{}{}{0}
\verb{cabal}{}{}{"-ais}{}{adj.2g.}{Que é pleno, completo, suficiente.}{ca.bal}{0}
\verb{cabal}{}{}{"-ais}{}{}{Que é rigoroso, severo.}{ca.bal}{0}
\verb{cabala}{}{}{}{}{s.f.}{Interpretação mística e misteriosa da Bíblia, entre os judeus.}{ca.ba.la}{0}
\verb{cabala}{}{}{}{}{}{Maquinagens secretas entre indivíduos que têm o mesmo objetivo; conluio.}{ca.ba.la}{0}
\verb{cabala}{}{}{}{}{}{Ciência oculta; magia, esoterismo, misticismo, simbolismo dos números e das letras. }{ca.ba.la}{0}
\verb{cabalar}{}{}{}{}{v.i.}{Fazer ou participar de cabala, de intriga; conspirar, tramar.}{ca.ba.lar}{0}
\verb{cabalar}{}{}{}{}{v.t.}{Aliciar eleitores, ou obter votos por meios ilícitos.}{ca.ba.lar}{\verboinum{1}}
\verb{cabalístico}{}{}{}{}{adj.}{Relativo à cabala.}{ca.ba.lís.ti.co}{0}
\verb{cabalístico}{}{}{}{}{}{Que é misterioso, oculto, enigmático.}{ca.ba.lís.ti.co}{0}
\verb{cabana}{}{}{}{}{s.f.}{Habitação pequena ou simples, geralmente campestre ou em local afastado, feita com materiais rústicos ou de pouco valor, e sem acabamento; barraca.}{ca.ba.na}{0}
\verb{cabano}{}{}{}{}{adj.}{Diz"-se do cavalo de orelhas caídas, ou do boi que tem os chifres inclinados para baixo. }{ca.ba.no}{0}
\verb{cabano}{}{Por ext.}{}{}{s.m.}{Chapéu de palha de abas largas.}{ca.ba.no}{0}
\verb{cabano}{}{}{}{}{s.m.}{Nome dado a partidários ou simplesmente a facções políticas do Norte e do Nordeste à época imediatamente anterior às regências e durante estas (1831--1840), e ligadas às revoltas da Cabanada e da Cabanagem.  }{ca.ba.no}{0}
\verb{cabaré}{}{}{}{}{s.m.}{Casa de diversões onde se bebe e dança, e, em geral, são apresentados números de dança, música e variedades.}{ca.ba.ré}{0}
\verb{cabaz}{}{}{}{}{s.m.}{Cesto de verga, vime ou junco, de variadas formas, geralmente com tampa e asa.}{ca.baz}{0}
\verb{cabaz}{}{}{}{}{}{Caixa cilíndrica de lata, para transporte de alimentos.}{ca.baz}{0}
\verb{cabeça}{ê}{}{}{}{s.f.}{Parte do corpo onde se encontram o cérebro, os olhos, o nariz, a boca e as orelhas. }{ca.be.ça}{0}
\verb{cabeça}{ê}{}{}{}{}{Parte do corpo com que se pensa; cérebro.}{ca.be.ça}{0}
\verb{cabeça}{ê}{}{}{}{}{Parte arredondada e mais larga de alguma coisa. (\textit{Cabeça de prego.})}{ca.be.ça}{0}
\verb{cabeça}{ê}{}{}{}{s.2g.}{Pessoa que dirige um grupo de outras; chefe, líder. (\textit{O amigo sempre fora o cabeça da classe.})}{ca.be.ça}{0}
\verb{cabeça"-chata}{ê}{Bras.}{cabeças"-chatas ⟨ê⟩}{}{s.m.}{Apelido dado aos nordestinos, especialmente aos cearenses.}{ca.be.ça"-cha.ta}{0}
\verb{cabeça"-chata}{ê}{Zool.}{cabeças"-chatas ⟨ê⟩}{}{}{Tubarão costeiro, de ampla distribuição nas águas quentes do mundo, de cor marrom a cinza, olhos pequenos e circulares, e fendas branquiais moderadamente longas.}{ca.be.ça"-cha.ta}{0}
\verb{cabeçada}{}{}{}{}{s.f.}{Pancada dada com a cabeça. (\textit{O garoto deu uma cabeçada na parede da sala.})}{ca.be.ça.da}{0}
\verb{cabeçada}{}{}{}{}{}{Ato que se pratica sem pensar e traz maus resultados; burrada. (\textit{O tio não cansava de dar cabeçadas, tentando encontrar um emprego a todo custo.})}{ca.be.ça.da}{0}
\verb{cabeça"-de"-negro}{ê\ldots{}ê}{Bot.}{cabeças"-de"-negro ⟨ê\ldots{}ê⟩}{}{s.f.}{Arbusto de flores amarelas e carnosas, com propriedades medicinais.}{ca.be.ça"-de"-ne.gro}{0}
\verb{cabeça"-de"-negro}{ê\ldots{}ê}{}{cabeças"-de"-negro ⟨ê\ldots{}ê⟩}{}{}{Bomba utilizada nos festejos juninos, de alto poder de detonação.}{ca.be.ça"-de"-ne.gro}{0}
\verb{cabeça"-de"-ponte}{ê}{}{cabeças"-de"-ponte ⟨ê⟩}{}{s.f.}{Posição ocupada por uma força militar em território inimigo, no lado oposto de um rio ou do mar, para garantir o desembarque posterior do restante da tropa.}{ca.be.ça"-de"-pon.te}{0}
\verb{cabeça"-de"-porco}{ê\ldots{}ô}{Pop.}{cabeças"-de"-porco ⟨ê\ldots{}ô⟩}{}{s.f.}{Habitação coletiva das classes pobres; cortiço.}{ca.be.ça"-de"-por.co}{0}
\verb{cabeça"-de"-prego}{ê\ldots{}é}{}{cabeças"-de"-prego ⟨ê\ldots{}é⟩}{}{s.f.}{Em certas regiões, larvas de mosquito em águas estagnadas.}{ca.be.ça"-de"-pre.go}{0}
\verb{cabeça"-de"-prego}{ê\ldots{}é}{Bras.}{cabeças"-de"-prego ⟨ê\ldots{}é⟩}{}{}{Pequeno abscesso cutâneo; espinha.}{ca.be.ça"-de"-pre.go}{0}
\verb{cabeça"-de"-vento}{ê}{}{cabeças"-de"-vento ⟨ê⟩}{}{s.2g.}{Indivíduo que não age com atenção, bom"-senso, prudência ou responsabilidade.}{ca.be.ça"-de"-ven.to}{0}
\verb{cabeça"-dura}{ê}{}{cabeças"-duras ⟨ê⟩}{}{s.2g.}{Indivíduo que não tem a capacidade de entender ou de aprender; rude, estúpido.}{ca.be.ça"-du.ra}{0}
\verb{cabeça"-dura}{ê}{}{cabeças"-duras ⟨ê⟩}{}{}{Indivíduo teimoso, insistente ou obstinado.}{ca.be.ça"-du.ra}{0}
\verb{cabeça"-inchada}{ê}{Pop.}{cabeças"-inchadas ⟨ê⟩}{}{s.f.}{Dor"-de"-cotovelo.}{ca.be.ça"-in.cha.da}{0}
\verb{cabeça"-inchada}{ê}{}{cabeças"-inchadas ⟨ê⟩}{}{}{Desgosto causado por derrota.}{ca.be.ça"-in.cha.da}{0}
\verb{cabeçalho}{}{}{}{}{s.m.}{Parte de cima de um texto, separada do que segue.}{ca.be.ça.lho}{0}
\verb{cabeçalho}{}{}{}{}{}{Título e primeiros dizeres de qualquer texto.}{ca.be.ça.lho}{0}
\verb{cabeção}{}{}{"-ões}{}{s.m.}{Cabeça grande.}{ca.be.ção}{0}
\verb{cabeção}{}{}{"-ões}{}{}{Gola larga que pende de certas peças do vestuário.}{ca.be.ção}{0}
\verb{cabeção}{}{}{"-ões}{}{}{A parte superior da camisa do traje da baiana.}{ca.be.ção}{0}
\verb{cabecear}{}{}{}{}{v.i.}{Mover a cabeça.}{ca.be.ce.ar}{0}
\verb{cabecear}{}{}{}{}{}{Menear a cabeça involuntariamente e de maneira repetida, deixando"-a pender.}{ca.be.ce.ar}{0}
\verb{cabecear}{}{}{}{}{}{Atirar ou rebater a bola com a cabeça.}{ca.be.ce.ar}{\verboinum{4}}
\verb{cabeceira}{ê}{}{}{}{s.f.}{Parte da cama onde se deita a cabeça.}{ca.be.cei.ra}{0}
\verb{cabeceira}{ê}{}{}{}{}{Cada uma das extremidades de uma mesa retangular ou oval.}{ca.be.cei.ra}{0}
\verb{cabeceira}{ê}{}{}{}{}{Nascente de um rio ou riacho.}{ca.be.cei.ra}{0}
\verb{cabecilha}{}{}{}{}{s.2g.}{Chefe de um grupo; líder, cabeça.}{ca.be.ci.lha}{0}
\verb{cabeço}{ê}{}{}{}{s.m.}{Cume arredondado de um monte.}{ca.be.ço}{0}
\verb{cabeço}{ê}{}{}{}{}{Monte pequeno e arredondado.}{ca.be.ço}{0}
\verb{cabeçorra}{ô}{Pop.}{}{}{s.f.}{Cabeça grande; cabeção.}{ca.be.çor.ra}{0}
\verb{cabeçote}{ó}{}{}{}{s.m.}{Cada uma das duas peças de ferro que fixam o objeto que se torneia.}{ca.be.ço.te}{0}
\verb{cabeçote}{ó}{}{}{}{}{Cabeça magnética de reprodução, gravação e apagamento de um aparelho gravador.}{ca.be.ço.te}{0}
\verb{cabeçote}{ó}{}{}{}{}{Parte dianteira e superior da sela.}{ca.be.ço.te}{0}
\verb{cabeçudo}{}{}{}{}{adj.}{Que tem cabeça grande.}{ca.be.çu.do}{0}
\verb{cabeçudo}{}{Fig.}{}{}{}{Que é teimoso, obstinado.}{ca.be.çu.do}{0}
\verb{cabeçudo}{}{}{}{}{s.m.}{Indivíduo que tem cabeça grande.}{ca.be.çu.do}{0}
\verb{cabeçudo}{}{Fig.}{}{}{}{Indivíduo teimoso, persistente.}{ca.be.çu.do}{0}
\verb{cabedal}{}{}{"-ais}{}{s.m.}{Patrimônio constituído em dinheiro; capital.}{ca.be.dal}{0}
\verb{cabedal}{}{}{"-ais}{}{}{O conjunto dos bens que formam o patrimônio de alguém.}{ca.be.dal}{0}
\verb{cabedal}{}{Fig.}{"-ais}{}{}{O conjunto dos bens intelectuais ou morais.}{ca.be.dal}{0}
\verb{cabeleira}{ê}{}{}{}{s.f.}{O conjunto dos cabelos da cabeça, quando compridos. }{ca.be.lei.ra}{0}
\verb{cabeleira}{ê}{}{}{}{}{O conjunto dos cabelos postiços dispostos como os naturais; peruca.}{ca.be.lei.ra}{0}
\verb{cabeleira}{ê}{}{}{}{s.m.}{Indivíduo que usa cabelos muito compridos.}{ca.be.lei.ra}{0}
\verb{cabeleireiro}{ê}{}{}{}{s.m.}{Profissional que corta e trata cabelos.}{ca.be.lei.rei.ro}{0}
\verb{cabeleireiro}{ê}{Por ext.}{}{}{}{Local de trabalho desse profissional.}{ca.be.lei.rei.ro}{0}
\verb{cabelo}{ê}{}{}{}{s.m.}{Pelo ou conjunto de pelos da cabeça humana.}{ca.be.lo}{0}
\verb{cabelo}{ê}{Por ext.}{}{}{}{Pelos que nascem em qualquer parte do corpo.}{ca.be.lo}{0}
\verb{cabeludo}{}{}{}{}{adj.}{Que tem cabelo longo ou abundante.}{ca.be.lu.do}{0}
\verb{cabeludo}{}{Pop.}{}{}{}{Que é difícil de lidar ou resolver; complicado.}{ca.be.lu.do}{0}
\verb{cabeludo}{}{Pop.}{}{}{}{Que é obsceno, imoral.}{ca.be.lu.do}{0}
\verb{cabeludo}{}{}{}{}{s.m.}{Indivíduo que tem muito cabelo ou que os tem longos.}{ca.be.lu.do}{0}
\verb{caber}{ê}{}{}{}{v.t.}{Poder ser contido; poder estar dentro.}{ca.ber}{0}
\verb{caber}{ê}{}{}{}{}{Ser compatível.}{ca.ber}{0}
\verb{caber}{ê}{}{}{}{}{Competir, pertencer.}{ca.ber}{0}
\verb{caber}{ê}{}{}{}{}{Pertencer como partilha ou quinhão.}{ca.ber}{0}
\verb{caber}{ê}{}{}{}{v.i.}{Vir a propósito; ter cabimento.}{ca.ber}{\verboinum{44}}
\verb{cabida}{}{}{}{}{s.f.}{Maneira de receber; acolhimento.}{ca.bi.da}{0}
\verb{cabida}{}{}{}{}{}{Relações amigáveis; confiança.}{ca.bi.da}{0}
\verb{cabide}{}{}{}{}{s.m.}{Móvel, haste ou gancho em que se penduram roupas, chapéus etc.}{ca.bi.de}{0}
\verb{cabidela}{é}{}{}{}{s.f.}{O conjunto das extremidades das aves, além do fígado, moela e miúdos em geral.}{ca.bi.de.la}{0}
\verb{cabidela}{é}{Cul.}{}{}{}{Prato que consiste nesses miúdos refogados no sangue da ave.}{ca.bi.de.la}{0}
\verb{cabido}{}{Relig.}{}{}{s.m.}{Conjunto dos clérigos de uma catedral.}{ca.bi.do}{0}
\verb{cabido}{}{}{}{}{adj.}{Que é compatível, adequado, apropriado.}{ca.bi.do}{0}
\verb{cabilda}{}{}{}{}{s.f.}{Designação comum a diversas tribos nômades da África setentrional.}{ca.bil.da}{0}
\verb{cabilda}{}{Por ext.}{}{}{}{Tribo.}{ca.bil.da}{0}
\verb{cabimento}{}{}{}{}{s.m.}{Aceitação, valimento, recebimento.}{ca.bi.men.to}{0}
\verb{cabimento}{}{}{}{}{}{Oportunidade, conveniência, propriedade.}{ca.bi.men.to}{0}
\verb{cabina}{}{}{}{}{s.f.}{Quarto de dormir de navio; camarote.}{ca.bi.na}{0}
\verb{cabina}{}{}{}{}{}{Quarto de dormir do vagão de um trem.}{ca.bi.na}{0}
\verb{cabina}{}{}{}{}{}{Lugar em que ficam o piloto e o copiloto de um avião; carlinga.}{ca.bi.na}{0}
\verb{cabina}{}{}{}{}{}{Lugar onde ficam o motorista de ônibus e caminhões.}{ca.bi.na}{0}
\verb{cabina}{}{}{}{}{}{Pequena construção feita para a atividade de uma pessoa.}{ca.bi.na}{0}
\verb{cabine}{}{}{}{}{s.f.}{Cabina.}{ca.bi.ne}{0}
\verb{cabineiro}{ê}{}{}{}{s.m.}{Sinaleiro de ferrovia.}{ca.bi.nei.ro}{0}
\verb{cabineiro}{ê}{}{}{}{}{Ascensorista.}{ca.bi.nei.ro}{0}
\verb{cabisbaixo}{ch}{}{}{}{adj.}{Que está de cabeça baixa, curvada.}{ca.bis.bai.xo}{0}
\verb{cabisbaixo}{ch}{Fig.}{}{}{}{Que está triste, abatido, humilhado.}{ca.bis.bai.xo}{0}
\verb{cabiúna}{}{Bot.}{}{}{s.f.}{Árvore de madeira nobre e cor escura, usada em obras hidráulicas, construção naval e civil, marcenaria e carpintaria.}{ca.bi.ú.na}{0}
\verb{cabiúna}{}{Bras.}{}{}{s.2g.}{O negro desembarcado clandestinamente no litoral brasileiro, após a lei que proibira o tráfico.}{ca.bi.ú.na}{0}
\verb{cabível}{}{}{"-eis}{}{adj.2g.}{Que cabe, que tem cabimento.}{ca.bí.vel}{0}
\verb{cabo}{}{}{}{}{s.m.}{Graduação militar imediatamente acima de soldado raso e abaixo de sargento.}{ca.bo}{0}
\verb{cabo}{}{}{}{}{s.m.}{Parte por onde se segura ou objeto ou instrumento.}{ca.bo}{0}
\verb{cabo}{}{}{}{}{}{Feixe de fios metálicos para transmissão de telégrafo, rádio, televisão.}{ca.bo}{0}
\verb{cabo}{}{}{}{}{}{Corda grossa de embarcação.}{ca.bo}{0}
\verb{cabochão}{}{}{"-ões}{}{s.m.}{Pedra talhada e polida, comumente arredondada, mas não facetada como o diamante.}{ca.bo.chão}{0}
\verb{caboclada}{}{}{}{}{s.f.}{Bando de caboclos.}{ca.bo.cla.da}{0}
\verb{caboclada}{}{}{}{}{}{Ação própria de caboclo.}{ca.bo.cla.da}{0}
\verb{caboclada}{}{}{}{}{}{Traição, deslealdade.}{ca.bo.cla.da}{0}
\verb{caboclinho}{}{Zool.}{}{}{s.m.}{Pássaro brasileiro.}{ca.bo.cli.nho}{0}
\verb{caboclo}{ô}{}{}{}{s.m.}{Mestiço de branco com índio, de cor acobreada.}{ca.bo.clo}{0}
\verb{caboclo}{ô}{}{}{}{}{O sertanejo caipira.}{ca.bo.clo}{0}
\verb{caboclo}{ô}{Fig.}{}{}{}{Indivíduo desconfiado ou traiçoeiro.}{ca.bo.clo}{0}
\verb{caboclo}{ô}{}{}{}{adj.}{Cor de cobre.}{ca.bo.clo}{0}
\verb{caboclo}{ô}{}{}{}{}{Relativo a caboclo.}{ca.bo.clo}{0}
\verb{cabo"-de"-esquadra}{}{}{cabos"-de"-esquadra}{}{s.m.}{No Brasil colonial e imperial, graduação militar inferior à de furriel.}{ca.bo"-de"-es.qua.dra}{0}
\verb{cabo"-de"-esquadra}{}{Por ext.}{cabos"-de"-esquadra}{}{}{O militar que detinha essa graduação.}{ca.bo"-de"-es.qua.dra}{0}
\verb{cabo"-de"-guerra}{é}{}{cabos"-de"-guerra ⟨é⟩}{}{s.m.}{Antigo oficial superior do Exército.}{ca.bo"-de"-guer.ra}{0}
\verb{cabo"-de"-guerra}{é}{}{cabos"-de"-guerra ⟨é⟩}{}{}{Jogo ou competição em que duas equipes puxam em direções opostas as pontas de uma corda grossa, vencendo a que conseguir arrastar a outra.}{ca.bo"-de"-guer.ra}{0}
\verb{cabograma}{}{}{}{}{s.m.}{Telegrama transmitido através de cabo submarino.}{ca.bo.gra.ma}{0}
\verb{caboré}{}{}{}{}{}{Var. de \textit{caburé}.}{ca.bo.ré}{0}
\verb{cabotagem}{}{}{"-ens}{}{s.f.}{Navegação entre portos de um mesmo país ou a distâncias pequenas, dentro de águas costeiras.}{ca.bo.ta.gem}{0}
\verb{cabotinagem}{}{}{"-ens}{}{s.f.}{Cabotinismo.}{ca.bo.ti.na.gem}{0}
\verb{cabotinismo}{}{}{}{}{s.m.}{Modo de vida, ato ou comportamento de cabotino.}{ca.bo.ti.nis.mo}{0}
\verb{cabotino}{}{}{}{}{s.m.}{Cômico ambulante.}{ca.bo.ti.no}{0}
\verb{cabotino}{}{}{}{}{}{Ator ou comediante de categoria inferior.}{ca.bo.ti.no}{0}
\verb{cabotino}{}{Fig.}{}{}{}{Indivíduo vaidoso, que faz questão de aparecer, que faz alarde em torno do próprio nome; exibicionista.}{ca.bo.ti.no}{0}
\verb{cabo"-verdiano}{}{}{cabo"-verdianos}{}{adj.}{Relativo a Cabo Verde.}{ca.bo"-ver.di.a.no}{0}
\verb{cabo"-verdiano}{}{}{cabo"-verdianos}{}{s.m.}{Indivíduo natural ou habitante desse país.}{ca.bo"-ver.di.a.no}{0}
\verb{cabra}{}{Zool.}{}{}{s.f.}{Mamífero ruminante, fêmea do bode.}{ca.bra}{0}
\verb{cabra}{}{Bras.}{}{}{s.m.}{Mestiço de mulato e negro.}{ca.bra}{0}
\verb{cabra}{}{Pop.}{}{}{}{Indivíduo, sujeito, cara.}{ca.bra}{0}
\verb{cabra"-cega}{é}{}{cabras"-cegas ⟨é⟩}{}{s.f.}{Jogo infantil em que uma criança fica vendada e deve conseguir agarrar outra criança, para ser por esta substituída.}{ca.bra"-ce.ga}{0}
\verb{cabra"-macho}{}{Pop.}{cabras"-machos}{}{s.m.}{Indivíduo corajoso, decidido, valente.}{ca.bra"-ma.cho}{0}
\verb{cabrão}{}{}{"-ões}{}{s.m.}{Macho da cabra; bode.}{ca.brão}{0}
\verb{cabrão}{}{Fig.}{"-ões}{}{}{Indivíduo que consente ou ignora que sua mulher tenha relações sexuais com outro homem; corno.}{ca.brão}{0}
\verb{cábrea}{}{}{}{}{s.f.}{Espécie de guindaste para suspender e deslocar grandes pesos, sobretudo nos portos.}{cá.brea}{0}
\verb{cabreiro}{ê}{}{}{}{s.m.}{Pastor que guarda cabras.}{ca.brei.ro}{0}
\verb{cabreiro}{ê}{}{}{}{}{Indivíduo ativo, diligente.}{ca.brei.ro}{0}
\verb{cabreiro}{ê}{}{}{}{adj.}{Que guarda cabras.}{ca.brei.ro}{0}
\verb{cabreiro}{ê}{}{}{}{}{Que é esperto, ativo.}{ca.brei.ro}{0}
\verb{cabreiro}{ê}{Pop.}{}{}{}{Que é desconfiado, cismado.}{ca.brei.ro}{0}
\verb{cabrestante}{}{}{}{}{s.m.}{Máquina ou mecanismo para içar âncoras, suspender vergas e levantar grandes pesos.}{ca.bres.tan.te}{0}
\verb{cabresto}{ê}{}{}{}{s.m.}{Correia que se prende à cabeça do animal e que serve para amarrá"-lo ou dirigi"-lo.}{ca.bres.to}{0}
\verb{cabresto}{ê}{}{}{}{}{Boi manso que serve de guia aos touros.}{ca.bres.to}{0}
\verb{cabresto}{ê}{Fig.}{}{}{}{Algo que controla, subjuga, especialmente reprimindo, contendo ou prendendo.}{ca.bres.to}{0}
\verb{cabril}{}{}{"-is}{}{s.m.}{Curral de cabras.}{ca.bril}{0}
\verb{cabril}{}{}{"-is}{}{adj.}{Relativo a cabra.}{ca.bril}{0}
\verb{cabril}{}{}{"-is}{}{}{Diz"-se de terreno agreste ou hostil, por ser montanhoso, rochoso, escarpado.}{ca.bril}{0}
\verb{cabriola}{ó}{}{}{}{s.f.}{Salto de cabra.}{ca.bri.o.la}{0}
\verb{cabriola}{ó}{}{}{}{}{Cambalhota.}{ca.bri.o.la}{0}
\verb{cabriola}{ó}{Fig.}{}{}{}{Mudança repentina de opinião ou de partido.}{ca.bri.o.la}{0}
\verb{cabriolar}{}{}{}{}{v.i.}{Dar cabriolas; saltar.}{ca.bri.o.lar}{\verboinum{1}}
\verb{cabriolé}{}{}{}{}{s.m.}{Carruagem leve, de duas rodas, com capota móvel, puxada por um cavalo.}{ca.bri.o.lé}{0}
\verb{cabriolé}{}{}{}{}{}{Carroceria de automóvel conversível, com lugar para dois ou três passageiros.}{ca.bri.o.lé}{0}
\verb{cabrita}{}{}{}{}{s.f.}{Cabra pequena.}{ca.bri.ta}{0}
\verb{cabrita}{}{Pop.}{}{}{}{Moça muito nova; adolescente.}{ca.bri.ta}{0}
\verb{cabrita}{}{}{}{}{}{Cabo ou empunhadura de serra manual.}{ca.bri.ta}{0}
\verb{cabritar}{}{}{}{}{v.i.}{Saltar ou andar saltando como os cabritos; pular.}{ca.bri.tar}{\verboinum{1}}
\verb{cabrito}{}{}{}{}{s.m.}{Pequeno bode, filhote da cabra.}{ca.bri.to}{0}
\verb{cabrito}{}{Pop.}{}{}{}{Menino irrequieto, travesso.}{ca.bri.to}{0}
\verb{cabriúva}{}{Bot.}{}{}{s.f.}{Árvore da floresta atlântica, de grande porte, que fornece madeira aromática e resistente, utilizada em construções em geral, móveis e canoas.}{ca.bri.ú.va}{0}
\verb{cabriúva}{}{}{}{}{}{Certa bebida feita com açúcar, gengibre e aguardente.}{ca.bri.ú.va}{0}
\verb{cabrocha}{ó}{Bras.}{}{}{s.f.}{Mulata jovem.}{ca.bro.cha}{0}
\verb{cabrocha}{ó}{}{}{}{s.m.}{Mulato novo.}{ca.bro.cha}{0}
\verb{cabrocha}{ó}{Por ext.}{}{}{}{Qualquer mestiço escuro, de cabelos grossos e cabelo pixaim.}{ca.bro.cha}{0}
\verb{cábula}{}{}{}{}{s.f.}{Falta ou pouca assiduidade às aulas.}{cá.bu.la}{0}
\verb{cábula}{}{}{}{}{s.m.}{Estudante pouco assíduo.}{cá.bu.la}{0}
\verb{cabular}{}{}{}{}{v.i.}{Não comparecer à aula; comportar"-se como cábula.}{ca.bu.lar}{\verboinum{1}}
\verb{cabuloso}{ô}{}{"-osos ⟨ó⟩}{"-osa ⟨ó⟩}{adj.}{Que aborrece; maçante.}{ca.bu.lo.so}{0}
\verb{cabuloso}{ô}{}{"-osos ⟨ó⟩}{"-osa ⟨ó⟩}{}{Que é complicado, obscuro.}{ca.bu.lo.so}{0}
\verb{caburé}{}{Bras.}{}{}{s.m.}{Mestiço de negro com índio; cafuzo.}{ca.bu.ré}{0}
\verb{caburé}{}{}{}{}{}{Caipira, sertanejo.}{ca.bu.ré}{0}
\verb{caburé}{}{}{}{}{}{Indivíduo feio e melancólico.}{ca.bu.ré}{0}
\verb{caburé}{}{}{}{}{}{Indivíduo que só sai à noite.}{ca.bu.ré}{0}
\verb{caca}{}{Pop.}{}{}{s.f.}{Excremento, fezes.}{ca.ca}{0}
\verb{caca}{}{Por ext.}{}{}{}{Qualquer sujeira, porcaria.}{ca.ca}{0}
\verb{caça}{}{}{}{}{s.f.}{Ato de caçar.}{ca.ça}{0}
\verb{caça}{}{}{}{}{}{O objeto da caça.}{ca.ça}{0}
\verb{caça}{}{Fig.}{}{}{}{Busca insistente; perseguição.}{ca.ça}{0}
\verb{caça}{}{}{}{}{s.m.}{Avião de caça.}{ca.ça}{0}
\verb{caçada}{}{}{}{}{s.f.}{Ato ou efeito de caçar.}{ca.ça.da}{0}
\verb{caçador}{ô}{}{}{}{s.m.}{Indivíduo que caça.}{ca.ça.dor}{0}
\verb{caçador}{ô}{}{}{}{adj.}{Que pratica a caça.}{ca.ça.dor}{0}
\verb{caçador}{ô}{}{}{}{}{Soldado da infantaria ou da cavalaria ligeira.}{ca.ça.dor}{0}
\verb{caça"-dotes}{ó}{Bras.}{}{}{s.2g.}{Indivíduo pobre que busca enriquecer casando com pessoa rica.}{ca.ça"-do.tes}{0}
\verb{caçamba}{}{}{}{}{s.f.}{Balde amarrado a uma corda usado para tirar água de poços.}{ca.çam.ba}{0}
\verb{caçamba}{}{}{}{}{}{Lata ou balde em que se leva a argamassa para os pedreiros.}{ca.çam.ba}{0}
\verb{caçamba}{}{}{}{}{}{Qualquer tipo de recipiente ou depósito.}{ca.çam.ba}{0}
\verb{caçamba}{}{}{}{}{}{Receptáculo de caminhões, guindastes, escavadeiras, dragas etc.}{ca.çam.ba}{0}
\verb{caça"-minas}{}{}{}{}{s.m.}{Tipo de navio de guerra destinado a localizar e destruir minas submarinas.}{ca.ça"-mi.nas}{0}
\verb{caça"-níqueis}{}{}{}{}{s.m.}{Máquina de jogo que funciona por meio da introdução de moedas e que paga um prêmio, igualmente em moedas, àquele que acertar as combinações previstas.}{ca.ça"-ní.queis}{0}
%\verb{}{}{}{}{}{}{}{}{0}
\verb{caçanje}{}{Por ext.}{}{}{}{Português mal falado ou mal escrito.}{ca.çan.je}{0}
\verb{caçanje}{}{}{}{}{s.m.}{Dialeto crioulo do português, falado na Angola.}{ca.çan.je}{0}
\verb{cação}{}{Zool.}{"-ões}{}{s.m.}{Peixe com fendas branquiais laterais e corpo de forma alongada, de tamanho médio ou pequeno, e cuja carne é muito consumida.}{ca.ção}{0}
\verb{caçapa}{}{}{}{}{s.f.}{Cada uma das seis redes onde caem as bolas, no jogo de sinuca.}{ca.ça.pa}{0}
\verb{caçapo}{}{Desus.}{}{}{s.m.}{Filhote de coelho.}{ca.ça.po}{0}
\verb{caçar}{}{}{}{}{v.t.}{Perseguir animais silvestres, para os aprisionar ou matar.}{ca.çar}{0}
\verb{caçar}{}{Fig.}{}{}{}{Perseguir, procurar, buscar.}{ca.çar}{\verboinum{3}}
\verb{cacareco}{é}{}{}{}{s.m.}{Objeto velho ou sem valor.}{ca.ca.re.co}{0}
%\verb{}{}{}{}{}{}{}{}{0}
\verb{cacarejar}{}{}{}{}{v.i.}{Cantar como a galinha ou outras aves de canto semelhante.}{ca.ca.re.jar}{0}
\verb{cacarejar}{}{Fig.}{}{}{}{Falar muito sem dizer nada; tagarelar.}{ca.ca.re.jar}{\verboinum{1}}
\verb{cacarejo}{ê}{}{}{}{s.m.}{Ato ou efeito de cacarejar.}{ca.ca.re.jo}{0}
\verb{cacarejo}{ê}{}{}{}{}{O canto da galinha ou de ave semelhante.}{ca.ca.re.jo}{0}
\verb{cacarejo}{ê}{}{}{}{}{Qualidade de quem fala muito; tagarelice.}{ca.ca.re.jo}{0}
\verb{cacaréus}{}{}{}{}{s.m.pl.}{Cacarecos.}{ca.ca.réus}{0}
\verb{caçarola}{ó}{}{}{}{s.f.}{Panela de metal com bordas altas, cabo e tampa.}{ca.ça.ro.la}{0}
\verb{cacatua}{}{Zool.}{}{}{s.f.}{Ave da ordem dos papagaios, de porte vultoso, de plumagem colorida, bico volumoso, cauda curta e um penacho grande e erétil.}{ca.ca.tu.a}{0}
\verb{cacau}{}{}{}{}{s.m.}{O fruto do cacaueiro, com polpa adocicada, comestível, usada em refrescos e doces.}{ca.cau}{0}
\verb{cacau}{}{Por ext.}{}{}{}{A semente desse fruto.}{ca.cau}{0}
\verb{cacau}{}{Por ext.}{}{}{}{O pó solúvel feito com essa semente, usado na alimentação, e que é matéria"-prima para a fabricação do chocolate.}{ca.cau}{0}
\verb{cacaual}{}{}{"-ais}{}{s.m.}{Coletivo de cacaueiro.}{ca.cau.al}{0}
\verb{cacaueiro}{ê}{Bot.}{}{}{s.m.}{Árvore cujas pequenas flores se inserem sobre o tronco, onde também surgem os frutos, de polpa doce e sabor agradável, com sementes que contêm gordura, a manteiga de cacau, e que servem depois de torradas para preparar o chocolate.}{ca.cau.ei.ro}{0}
\verb{cacauicultor}{ô}{}{}{}{s.m.}{Indivíduo que cultiva ou possui plantação de cacau.}{ca.cau.i.cul.tor}{0}
\verb{cacauicultura}{}{}{}{}{s.f.}{Plantação ou cultura de cacau.}{ca.cau.i.cul.tu.ra}{0}
\verb{cacauzeiro}{ê}{Bot.}{}{}{s.m.}{Cacaueiro.}{ca.cau.zei.ro}{0}
\verb{cacetada}{}{}{}{}{s.f.}{Pancada com cacete; bordoada, paulada.}{ca.ce.ta.da}{0}
\verb{cacetada}{}{}{}{}{}{Algo que é maçante, que importuna ou causa incômodo.}{ca.ce.ta.da}{0}
\verb{cacetada}{}{}{}{}{}{Grande quantidade.}{ca.ce.ta.da}{0}
\verb{cacete}{ê}{}{}{}{s.m.}{Pedaço de pau com uma das extremidades mais longa que a outra.}{ca.ce.te}{0}
\verb{cacete}{ê}{Pop.}{}{}{}{O pênis.}{ca.ce.te}{0}
\verb{cacete}{ê}{}{}{}{adj.}{Que provoca tédio, enfado, aborrecimento; maçante.}{ca.ce.te}{0}
\verb{cacete}{ê}{}{}{}{interj.}{Expressão que denota apreensão, lembrança repentina, desagrado, impaciência.}{ca.ce.te}{0}
\verb{caceteação}{}{}{"-ões}{}{s.f.}{Ato ou efeito de cacetear, estado causado por algo maçante, importunador, aborrecedor.}{ca.ce.te.a.ção}{0}
\verb{cacetear}{}{}{}{}{v.t.}{Desferir cacetadas; bater; cacetar.}{ca.ce.te.ar}{0}
\verb{cacetear}{}{}{}{}{}{Causar aborrecimento; maçar, chatear.}{ca.ce.te.ar}{\verboinum{4}}
\verb{cachaça}{}{}{}{}{s.f.}{Aguardente destilada da cana"-de"-açúcar; pinga.}{ca.cha.ça}{0}
\verb{cachaça}{}{Fig.}{}{}{}{Paixão, vício.}{ca.cha.ça}{0}
\verb{cachação}{}{}{"-ões}{}{s.m.}{Pancada ou empurrão no cachaço, parte superior do pescoço.}{ca.cha.ção}{0}
\verb{cachaceiro}{ê}{}{}{}{adj.}{Diz de quem costuma beber cachaça ou outra bebida alcoólica em grandes quantidades.}{ca.cha.cei.ro}{0}
\verb{cachaceiro}{ê}{}{}{}{s.m.}{Indivíduo que é dado ao uso exagerado da cachaça ou de outra bebida alcoólica.}{ca.cha.cei.ro}{0}
\verb{cachaço}{}{}{}{}{s.m.}{A parte posterior do pescoço.}{ca.cha.ço}{0}
\verb{cachaço}{}{}{}{}{}{Corte de carne bovina, correspondente à parte posterior do pescoço do animal. }{ca.cha.ço}{0}
\verb{cachaço}{}{}{}{}{}{Pescoço largo e grosso.}{ca.cha.ço}{0}
\verb{cachaço}{}{}{}{}{}{Porco não castrado, que serve como reprodutor.}{ca.cha.ço}{0}
\verb{cachalote}{ó}{Zool.}{}{}{s.m.}{Baleia encontrada em todos os oceanos e mares, com até vinte metros de comprimento, coloração cinzenta ou preta, de cabeça muito grande e quase quadrangular, da qual se extrai o espermacete.}{ca.cha.lo.te}{0}
\verb{cachão}{}{}{"-ões}{}{s.m.}{Grande quantidade de água que sai com muita força; borbotão, jorro.}{ca.chão}{0}
\verb{cachê}{}{}{}{}{s.m.}{Remuneração que ator, músico ou outro artista recebe por apresentação.}{ca.chê}{0}
\verb{cachê}{}{Por ext.}{}{}{}{Pagamento feito a qualquer pessoa que se apresente em espetáculo público.}{ca.chê}{0}
\verb{cacheado}{}{}{}{}{adj.}{Diz"-se de cabelo enrolado que forma cachos.}{ca.che.a.do}{0}
\verb{cachear}{}{}{}{}{v.i.}{Dar cachos.}{ca.che.ar}{0}
\verb{cachear}{}{}{}{}{}{Encrespar o cabelo, tornando"-o cacheado.}{ca.che.ar}{\verboinum{4}}
\verb{cachecol}{ó}{}{"-óis}{}{s.m.}{Manta longa e estreita, de lã, seda ou outro tecido flexível, usada enrolada em torno do pescoço, para proteger do frio ou como acessório.}{ca.che.col}{0}
\verb{cachenê}{}{}{}{}{s.m.}{Manta comprida e estreita para agasalhar o rosto até o nariz.}{ca.che.nê}{0}
\verb{cachimbada}{}{}{}{}{s.f.}{Porção de tabaco que se põe no cachimbo.}{ca.chim.ba.da}{0}
\verb{cachimbada}{}{}{}{}{}{Ato de aspirar a fumaça do cachimbo.}{ca.chim.ba.da}{0}
\verb{cachimbada}{}{}{}{}{}{Fumaça de cachimbo.}{ca.chim.ba.da}{0}
\verb{cachimbar}{}{}{}{}{v.i.}{Fumar cachimbo.}{ca.chim.bar}{0}
\verb{cachimbar}{}{Fig.}{}{}{v.t.}{Matutar, meditar, ponderar.}{ca.chim.bar}{\verboinum{1}}
\verb{cachimbo}{}{}{}{}{s.m.}{Utensílio para fumar feito de madeira, barro ou outros materiais, que consiste num tubo delgado que tem, numa das extremidades, um recipiente onde se coloca tabaco ou outro produto, e, na outra extremidade, uma abertura ou bocal por onde se aspira a fumaça.}{ca.chim.bo}{0}
\verb{cachimbo}{}{}{}{}{}{Buraco do castiçal onde se encaixa a vela.}{ca.chim.bo}{0}
\verb{cachimbo}{}{}{}{}{}{Bebida preparada com cachaça e mel de abelha.}{ca.chim.bo}{0}
\verb{cachimônia}{}{Pop.}{}{}{s.f.}{Cabeça, inteligência, mente, juízo.}{ca.chi.mô.nia}{0}
\verb{cachimônia}{}{Pop.}{}{}{}{Paciência, calma.}{ca.chi.mô.nia}{0}
\verb{cacho}{}{}{}{}{s.m.}{Conjunto de flores ou frutos que brotam muito próximos entre si.}{ca.cho}{0}
\verb{cacho}{}{}{}{}{}{Mecha pendente de cabelo, enrolada em aspiral ou em anéis.}{ca.cho}{0}
\verb{cacho}{}{Pop.}{}{}{}{Caso amoroso.}{ca.cho}{0}
\verb{cachoeira}{ê}{}{}{}{s.f.}{Torrente de água que corre ou cai formando cachão, turbilhão.}{ca.cho.ei.ra}{0}
\verb{cachola}{ó}{Pop.}{}{}{s.f.}{Cabeça, bestunto, cachimônia. (\textit{Quebrei a cachola procurando meus brincos.})}{ca.cho.la}{0}
\verb{cacholeta}{ê}{}{}{}{s.f.}{Pancada que se dá na cabeça com a mão ou com uma vara.}{ca.cho.le.ta}{0}
\verb{cacholeta}{ê}{Fig.}{}{}{}{Censura, repreensão.}{ca.cho.le.ta}{0}
\verb{cachopa}{ô}{}{}{}{s.f.}{Mulher jovem; moça.}{ca.cho.pa}{0}
\verb{cachopa}{ô}{}{}{}{}{Cacho de flores na extremidade de um ramo.}{ca.cho.pa}{0}
\verb{cachorra}{ô}{}{}{}{s.f.}{Cadela ainda nova.}{ca.chor.ra}{0}
\verb{cachorra}{ô}{Pop.}{}{}{}{Mulher devassa, indecente, imoral.}{ca.chor.ra}{0}
\verb{cachorrada}{}{}{}{}{s.f.}{Bando de cachorros.}{ca.chor.ra.da}{0}
\verb{cachorrada}{}{Fig.}{}{}{}{Grupo de indivíduos ordinários, maus.}{ca.chor.ra.da}{0}
\verb{cachorrada}{}{Fig.}{}{}{}{Ato ou comportamento indecente, vil; canalhice.}{ca.chor.ra.da}{0}
\verb{cachorrice}{}{Fig.}{}{}{s.f.}{Ação má; canalhice.}{ca.chor.ri.ce}{0}
\verb{cachorro}{ô}{}{}{}{s.m.}{Cão novo e pequeno.}{ca.chor.ro}{0}
\verb{cachorro}{ô}{Por ext.}{}{}{}{Qualquer cão.}{ca.chor.ro}{0}
\verb{cachorro}{ô}{}{}{}{}{Cria de lobo, leão, onça, hiena etc.}{ca.chor.ro}{0}
\verb{cachorro}{ô}{Fig.}{}{}{}{Indivíduo indigno; canalha, cafajeste.}{ca.chor.ro}{0}
\verb{cachorro"-quente}{ô}{}{cachorros"-quentes ⟨ô⟩}{}{s.m.}{Sanduíche feito com pão pequeno e alongado, e salsicha quente, servida com ou sem molho; \textit{hot dog}.}{ca.chor.ro"-quen.te}{0}
\verb{cacife}{}{}{}{}{s.m.}{Quantia correspondente, no jogo, à entrada de cada jogador.}{ca.ci.fe}{0}
\verb{cacife}{}{Fig.}{}{}{}{Qualidade ou atributo pessoal que habilita ou capacita alguém para algo.}{ca.ci.fe}{0}
\verb{cacimba}{}{}{}{}{s.f.}{Cova aberta em terreno úmido ou pantanoso, para recolher a água presente no solo que nela se acumula.}{ca.cim.ba}{0}
\verb{cacimba}{}{}{}{}{}{Poço cavado até um lençol de água.}{ca.cim.ba}{0}
\verb{cacimba}{}{}{}{}{}{Fonte de água potável.}{ca.cim.ba}{0}
\verb{cacique}{}{}{}{}{s.m.}{Chefe indígena entre diversas tribos da América.}{ca.ci.que}{0}
\verb{cacique}{}{Fig.}{}{}{}{Aquele que dita as regras ou impõe sua vontade num lugar ou sobre um grupo de indivíduos.}{ca.ci.que}{0}
\verb{cacique}{}{Fig.}{}{}{}{Indivíduo de muita influência política, eleitoral ou administrativa.}{ca.ci.que}{0}
\verb{caco}{}{}{}{}{s.m.}{Fragmento ou pedaço quebrado de louça, barro, vidro, mármore etc.}{ca.co}{0}
\verb{caco}{}{Fig.}{}{}{}{A parte ou porção restante de algo já gasto pelo uso ou desgastado pelo tempo.}{ca.co}{0}
\verb{caco}{}{Fig.}{}{}{}{Pessoa consumida pela idade ou por doença.}{ca.co}{0}
\verb{caçoada}{}{}{}{}{s.f.}{Ato de caçoar; zombaria, brincadeira, troça.}{ca.ço.a.da}{0}
\verb{caçoar}{}{}{}{}{v.i.}{Fazer caçoada; escarnecer, zombar, troçar.}{ca.ço.ar}{\verboinum{7}}
\verb{cacoete}{ê}{}{}{}{s.m.}{Movimentos ou contrações repetidas e involuntárias dos músculos do corpo; tique.}{ca.co.e.te}{0}
\verb{cacoete}{ê}{}{}{}{}{Mau hábito, mania.}{ca.co.e.te}{0}
\verb{cacófato}{}{Gram.}{}{}{s.m.}{Som desagradável ou palavra obscena, proveniente da união de sílabas finais de uma palavra com as iniciais da seguinte.}{ca.có.fa.to}{0}
\verb{cacófaton}{}{}{}{}{}{Var. de \textit{cacófato}.}{ca.có.fa.ton}{0}
\verb{cacofonia}{}{}{}{}{s.f.}{Cacófato.}{ca.co.fo.ni.a}{0}
\verb{cacografar}{}{Gram.}{}{}{v.t.}{Escrever com erro de ortografia.}{ca.co.gra.far}{\verboinum{1}}
\verb{cacto}{}{Bot.}{}{}{s.m.}{Planta de caule esférico ou anguloso, coberto de espinhos, que dão flores, algumas grandes e de cores vivas.}{cac.to}{0}
\verb{caçula}{}{}{}{}{adj.2g.}{Diz"-se do mais moço dos filhos, ou dos irmãos.}{ca.çu.la}{0}
\verb{caçula}{}{}{}{}{s.2g.}{O irmão, ou filho mais novo.}{ca.çu.la}{0}
\verb{cacunda}{}{Bras.}{}{}{s.f.}{Costas, dorso.}{ca.cun.da}{0}
\verb{cacunda}{}{}{}{}{}{Corcunda.}{ca.cun.da}{0}
\verb{cada}{}{}{}{}{pron.}{Qualquer dos elementos de um grupo, destacando um por um; todo.}{ca.da}{0}
\verb{cada}{}{}{}{}{}{Qualquer grupo de pessoas ou coisas, destacando grupo por grupo.}{ca.da}{0}
\verb{cadafalso}{}{}{}{}{s.m.}{Tablado ou estrado erguido em lugar público, para sobre ele se executarem condenados; forca.}{ca.da.fal.so}{0}
\verb{cadarço}{}{}{}{}{s.m.}{Cordão ou fita estreita de linho, algodão, seda ou lã; barbilho.}{ca.dar.ço}{0}
\verb{cadastrar}{}{}{}{}{v.t.}{Fazer o cadastro; reunir e organizar informações acerca de bens, pessoas ou itens diversos, em forma de cadastro.}{ca.das.trar}{\verboinum{1}}
\verb{cadastro}{}{}{}{}{s.m.}{Registro público dos bens imóveis de determinado território.}{ca.das.tro}{0}
\verb{cadastro}{}{}{}{}{}{Registro que bancos ou casas comerciais mantêm de seus clientes.}{ca.das.tro}{0}
\verb{cadastro}{}{}{}{}{}{Registro policial de criminosos ou contraventores.}{ca.das.tro}{0}
\verb{cadastro}{}{}{}{}{}{Censo, recenseamento.}{ca.das.tro}{0}
\verb{cadáver}{}{}{}{}{s.m.}{O corpo morto de um animal ou de um ser humano; defunto.}{ca.dá.ver}{0}
\verb{cadavérico}{}{}{}{}{adj.}{Relativo a cadáver.}{ca.da.vé.ri.co}{0}
\verb{cadê}{}{Pop.}{}{}{adv.}{Palavra interrogativa que significa \textit{onde está?}}{ca.dê}{0}
\verb{cadeado}{}{}{}{}{s.m.}{Fechadura portátil, cujo aro, móvel, é introduzido em duas argolas fixas às peças que se quer unir ou fechar.}{ca.de.a.do}{0}
\verb{cadeia}{ê}{}{}{}{s.f.}{Cada uma das partes de uma corrente de metal, ou essa corrente inteira.}{ca.dei.a}{0}
\verb{cadeia}{ê}{}{}{}{}{Conjunto de coisas semelhantes. (\textit{Cadeia de montanhas.})}{ca.dei.a}{0}
\verb{cadeia}{ê}{}{}{}{}{Lugar em que se prendem os criminosos; prisão, cárcere, xadrez.}{ca.dei.a}{0}
\verb{cadeira}{ê}{}{}{}{s.f.}{Peça de mobília que consiste num assento com costas, e, às vezes com braços, dobrável ou não, para uma pessoa.}{ca.dei.ra}{0}
\verb{cadeira}{ê}{}{}{}{}{Disciplina ministrada em estabelecimento escolar; matéria.}{ca.dei.ra}{0}
\verb{cadeira}{ê}{}{}{}{}{Lugar ocupado por membro de corporação política, científica ou literária.}{ca.dei.ra}{0}
\verb{cadeira}{ê}{}{}{}{}{Lugar privilegiado em teatros, estádios, ginásios etc.}{ca.dei.ra}{0}
\verb{cadeira}{ê}{}{}{}{}{Bilhete de ingresso próprio para esse lugar.}{ca.dei.ra}{0}
\verb{cadeirinha}{}{}{}{}{s.f.}{Cadeira pequena.}{ca.dei.ri.nha}{0}
\verb{cadeirinha}{}{}{}{}{}{Brinquedo que consiste em duas pessoas formarem com as mãos uma cruzeta, para que outra nela se sente.}{ca.dei.ri.nha}{0}
\verb{cadeirinha}{}{}{}{}{}{Antigo meio de transporte que era uma cadeira pequena com duas traves longas, carregada por dois homens ou dois animais, um à frente e outro atrás.}{ca.dei.ri.nha}{0}
\verb{cadela}{é}{}{}{}{s.f.}{A fêmea do cão.}{ca.de.la}{0}
\verb{cadela}{é}{Pop.}{}{}{}{Mulher vulgar, desavergonhada.}{ca.de.la}{0}
\verb{cadência}{}{}{}{}{s.f.}{Compasso e harmonia na disposição das palavras.}{ca.dên.cia}{0}
\verb{cadência}{}{}{}{}{}{Regularidade de movimentos ou de sons; ritmo.}{ca.dên.cia}{0}
\verb{cadência}{}{}{}{}{}{Vocação, tendência.}{ca.dên.cia}{0}
\verb{cadenciado}{}{}{}{}{adj.}{Que tem cadência; compassado, ritmado.}{ca.den.ci.a.do}{0}
\verb{cadenciar}{}{}{}{}{v.t.}{Dar cadência ou ritmo.}{ca.den.ci.ar}{\verboinum{6}}
\verb{cadente}{}{}{}{}{adj.2g.}{Que cai ou está a cair.}{ca.den.te}{0}
\verb{cadente}{}{}{}{}{}{Que tem cadência; ritmado.}{ca.den.te}{0}
\verb{caderneta}{ê}{}{}{}{s.f.}{Pequeno caderno em que se anotam ou escrevem lembretes e informações diversas.}{ca.der.ne.ta}{0}
\verb{caderno}{é}{}{}{}{s.m.}{Conjunto de folhas de papel cortadas, coladas ou cosidas, formando livro de anotações, de exercícios escolares etc.}{ca.der.no}{0}
\verb{caderno}{é}{}{}{}{}{Suplemento de jornal, revista etc.}{ca.der.no}{0}
\verb{cadete}{ê}{}{}{}{s.m.}{Aluno de escola militar superior do Exército e da Aeronáutica; aspirante a oficial. }{ca.de.te}{0}
\verb{cadinho}{}{}{}{}{s.m.}{Grande vaso para fundir ou calcinar minérios; crisol.}{ca.di.nho}{0}
\verb{cádmio}{}{Quím.}{}{}{s.m.}{Elemento químico metálico, branco azulado, mole, utilizado em ligas, soldas, recobrimentos protetores, baterias, pilhas especiais etc. \elemento{48}{112.411}{Cd}.}{cád.mi.o}{0}
\verb{caducar}{}{}{}{}{v.i.}{Tornar"-se caduco, perder a força, decair, envelhecer.}{ca.du.car}{0}
\verb{caducar}{}{Jur.}{}{}{}{Tornar"-se nulo, deixar de ter valor, de estar em vigor.}{ca.du.car}{\verboinum{2}}
\verb{caducar}{}{Bras.}{}{}{}{Perder parcialmente a razão, o tino, por efeito de idade avançada.}{ca.du.car}{0}
\verb{caduceu}{ê}{}{}{}{s.m.}{Bastão com duas serpentes enroscadas e com duas asas na extremidade superior, símbolo da medicina.}{ca.du.ceu}{0}
\verb{caducidade}{}{}{}{}{s.f.}{Qualidade de caduco; velhice, decadência.}{ca.du.ci.da.de}{0}
\verb{caduco}{}{}{}{}{adj.}{Que cai; que está prestes a cair.}{ca.du.co}{0}
\verb{caduco}{}{}{}{}{}{Que perdeu as forças, o viço ou a capacidade mental; decrépito.}{ca.du.co}{0}
\verb{caduco}{}{}{}{}{}{Que se anulou.}{ca.du.co}{0}
\verb{caduco}{}{Bras.}{}{}{}{Que perdeu em parte a razão, o tino, por consequência de idade avançada.}{ca.du.co}{0}
\verb{caduquice}{}{}{}{}{s.f.}{Ato ou comportamento de quem está caduco, pouco lúcido.}{ca.du.qui.ce}{0}
\verb{cafajestada}{}{Bras.}{}{}{s.f.}{Grupo de cafajestes.}{ca.fa.jes.ta.da}{0}
\verb{cafajestada}{}{}{}{}{}{Procedimento de cafajeste.}{ca.fa.jes.ta.da}{0}
\verb{cafajeste}{é}{Bras.}{}{}{s.m.}{Indivíduo sem maneiras, vulgar; mau"-caráter, canalha.}{ca.fa.jes.te}{0}
\verb{cafarnaum}{}{Bras.}{}{}{s.m.}{Depósito de coisas velhas.}{ca.far.na.um}{0}
\verb{cafarnaum}{}{}{}{}{}{Lugar de tumulto ou de desordem.}{ca.far.na.um}{0}
\verb{café}{}{}{}{}{s.m.}{Fruto do cafeeiro, drupa em forma de elipse ou de globo, vermelha, com escassa polpa adocicada e duas grandes sementes, que depois de torradas e moídas são consumidas como bebida.}{ca.fé}{0}
\verb{café}{}{Por ext.}{}{}{}{A bebida feita com esse fruto.}{ca.fé}{0}
\verb{café}{}{}{}{}{}{Estabelecimento comercial onde se servem café e outras bebidas.}{ca.fé}{0}
\verb{café"-concerto}{ê}{}{cafés"-concertos \textit{ou} cafés"-concerto ⟨ê⟩}{}{s.m.}{Casa de diversões onde o público bebe assistindo a números musicais e de variedades.}{ca.fé"-con.cer.to}{0}
\verb{cafeeiro}{ê}{Bot.}{}{}{s.m.}{Arbusto originário da Arábia, de flores pequenas, alvas e perfumadas, muito cultivado no Brasil, e cujo fruto é o café.}{ca.fe.ei.ro}{0}
\verb{cafeeiro}{ê}{}{}{}{adj.}{Relativo ao café.}{ca.fe.ei.ro}{0}
\verb{cafeicultor}{ô}{}{}{}{s.m.}{Indivíduo que planta café.}{ca.fe.i.cul.tor}{0}
\verb{cafeicultura}{}{}{}{}{s.f.}{Cultivo ou plantação de café.}{ca.fe.i.cul.tu.ra}{0}
\verb{cafeína}{}{}{}{}{s.f.}{Alcaloide branco, cristalino e amargo, estimulante do sistema nervoso central, encontrado no café, no chá e no guaraná.}{ca.fe.í.na}{0}
\verb{cafetã}{}{}{}{}{s.f.}{Espécie de túnica larga, usada sobretudo pelos povos árabes e turcos, ricamente bordada, podendo ser forrada de peles.}{ca.fe.tã}{0}
\verb{cafetã}{}{}{}{}{}{Túnica longa e sem cinto, geralmente bordada no decote e nas barras da manga e da saia.}{ca.fe.tã}{0}
\verb{cafetão}{}{Bras.}{"-ões}{"-ina}{s.m.}{Indivíduo que vive da prostituição, seja explorando meretrizes, seja estabelecido como dono de prostíbulo.}{ca.fe.tão}{0}
\verb{cafeteira}{ê}{}{}{}{s.f.}{Recipiente em que se prepara e serve o café.}{ca.fe.tei.ra}{0}
\verb{cafetina}{}{Bras.}{}{}{s.f.}{Mulher que vive da exploração de meretrizes, especialmente como administradora ou proprietária de prostíbulo.}{ca.fe.ti.na}{0}
\verb{cafezal}{}{}{"-ais}{}{s.m.}{Coletivo de cafeeiro.}{ca.fe.zal}{0}
\verb{cafezinho}{}{}{}{}{s.m.}{Café servido em xícaras pequenas; café pequeno.}{ca.fe.zi.nho}{0}
\verb{cáfila}{}{}{}{}{s.f.}{Grande quantidade de camelos que transportam mercadorias; caravana.}{cá.fi.la}{0}
\verb{cáfila}{}{Fig.}{}{}{}{Bando, corja, malta.}{cá.fi.la}{0}
\verb{cafona}{}{Pop.}{}{}{adj.}{Diz"-se daquele que tem mau gosto, ou é muito apegado a convenções; brega.}{ca.fo.na}{0}
\verb{cafona}{}{}{}{}{s.m.}{Indivíduo que revela mau gosto, convencionalismo e pouca sofisticação.}{ca.fo.na}{0}
\verb{cafonice}{}{}{}{}{s.f.}{Qualidade de cafona.}{ca.fo.ni.ce}{0}
\verb{cáften}{}{}{}{}{s.m.}{Cafetão.}{cáf.ten}{0}
\verb{caftina}{}{}{}{}{}{Var. de \textit{cafetina}.}{caf.ti.na}{0}
\verb{cafua}{}{}{}{}{s.f.}{Antro, caverna, esconderijo.}{ca.fu.a}{0}
\verb{cafua}{}{}{}{}{}{Habitação miserável.}{ca.fu.a}{0}
\verb{cafua}{}{}{}{}{}{Quarto escuro onde se prendiam os alunos que eram deixados de castigo.}{ca.fu.a}{0}
\verb{cafua}{}{}{}{}{}{Quarto ou lugar fechado usado como  castigo.}{ca.fu.a}{0}
\verb{cafundó}{}{Bras.}{}{}{s.m.}{Lugar ermo e afastado, de difícil acesso.}{ca.fun.dó}{0}
\verb{cafuné}{}{}{}{}{s.m.}{Ato de coçar levemente a cabeça de alguém para fazê"-lo adormecer.}{ca.fu.né}{0}
\verb{cafuné}{}{Por ext.}{}{}{}{Carícia, afago, mimo.}{ca.fu.né}{0}
\verb{cafuzo}{}{Bras.}{}{}{adj.}{Diz"-se de filho de índio e negro; caburé.}{ca.fu.zo}{0}
\verb{cafuzo}{}{}{}{}{}{Diz"-se de mestiço de pele escura ou quase negra, cabelo corrido e grosso.}{ca.fu.zo}{0}
\verb{cafuzo}{}{}{}{}{s.m.}{Mestiço de negro e índio; caburé.}{ca.fu.zo}{0}
\verb{cafuzo}{}{}{}{}{}{Mestiço de pele escura ou negra e cabelos lisos e cheios.}{ca.fu.zo}{0}
\verb{cágado}{}{Zool.}{}{}{s.m.}{Réptil semelhante à tartaruga, que vive em lagoas rasas e terrenos pantanosos, de pescoço longo e carapaça achatada.}{cá.ga.do}{0}
\verb{cágado}{}{Fig.}{}{}{}{Indivíduo lerdo, vagaroso.}{cá.ga.do}{0}
\verb{caganeira}{ê}{Pop.}{}{}{s.f.}{Diarreia.}{ca.ga.nei.ra}{0}
\verb{cagão}{}{Pop.}{"-ões}{"-ona}{adj.}{Que defeca com frequência.}{ca.gão}{0}
\verb{cagão}{}{Pop.}{"-ões}{"-ona}{}{Que tem muito medo, que é tímido em excesso.}{ca.gão}{0}
\verb{cagão}{}{Pop.}{"-ões}{"-ona}{}{Que tem muita sorte.}{ca.gão}{0}
\verb{cagão}{}{Pop.}{"-ões}{"-ona}{s.m.}{Indivíduo que defeca muito ou tem diarreia.}{ca.gão}{0}
\verb{cagão}{}{Pop.}{"-ões}{"-ona}{}{Indivíduo que não tem coragem; medroso.}{ca.gão}{0}
\verb{cagão}{}{Pop.}{"-ões}{"-ona}{}{Indivíduo que tem sorte.}{ca.gão}{0}
\verb{cagar}{}{Pop.}{}{}{v.i.}{Defecar.}{ca.gar}{\verboinum{5}}
\verb{cagueta}{ê}{}{}{}{}{Var. de \textit{alcaguete}.}{ca.gue.ta}{0}
\verb{caguetar}{}{}{}{}{}{Var. de \textit{alcaguetar}.}{ca.gue.tar}{\verboinum{1}}
\verb{caiação}{}{}{"-ões}{}{s.f.}{Ato ou efeito de caiar; caiadura.}{cai.a.ção}{0}
\verb{caiana}{}{}{}{}{s.f.}{Cana"-caiana.}{cai.a.na}{0}
\verb{caiaque}{}{}{}{}{s.m.}{Pequena embarcação esquimó, feita de peles de foca envolvendo uma armação de ossos de baleias, impulsionada com remo de duas pás.}{cai.a.que}{0}
\verb{caiaque}{}{}{}{}{}{Embarcação semelhante à anterior na forma, geralmente com dois lugares, para prática de esporte e lazer.}{cai.a.que}{0}
\verb{caiar}{}{}{}{}{v.t.}{Cobrir com cal.}{cai.ar}{0}
\verb{caiar}{}{}{}{}{}{Pintar com água de cal.}{cai.ar}{0}
\verb{caiar}{}{}{}{}{}{Dar cor branca a algo.}{cai.ar}{0}
\verb{caiar}{}{}{}{}{}{Revestir, pintar.}{cai.ar}{\verboinum{1}}
\verb{cãibra}{}{Med.}{}{}{s.f.}{Contração muscular involuntária, espasmódica e dolorosa.}{cãi.bra}{0}
\verb{caibro}{}{}{}{}{s.m.}{Peça de madeira que sustenta as ripas do telhado ou as tábuas do soalho.}{cai.bro}{0}
\verb{caiçara}{}{}{}{}{s.f.}{Cerca indígena, feita com varas, galhos etc.}{cai.ça.ra}{0}
\verb{caiçara}{}{}{}{}{s.2g.}{Caipira, matuto.}{cai.ça.ra}{0}
\verb{caiçara}{}{}{}{}{}{Indivíduo natural ou habitante do litoral; praiano.}{cai.ça.ra}{0}
\verb{caída}{}{}{}{}{s.f.}{Ato ou efeito de cair; queda.}{ca.í.da}{0}
\verb{caída}{}{Fig.}{}{}{}{Ruína, declínio.}{ca.í.da}{0}
\verb{caída}{}{Fig.}{}{}{}{Inclinação, propensão, tendência.}{ca.í.da}{0}
\verb{caído}{}{}{}{}{adj.}{Prostrado, derrubado.}{ca.í.do}{0}
\verb{caído}{}{}{}{}{}{Abatido, enfraquecido, fatigado.}{ca.í.do}{0}
\verb{caído}{}{Fig.}{}{}{}{Enamorado, apaixonado.}{ca.í.do}{0}
\verb{caieira}{ê}{}{}{}{s.f.}{Forno onde se calcina o calcário para a obtenção de cal.}{cai.ei.ra}{0}
\verb{caieira}{ê}{}{}{}{}{Fábrica de cal.}{cai.ei.ra}{0}
\verb{caimão}{}{Zool.}{"-ões}{}{s.m.}{Nome comum aos jacarés encontrados nos rios e lagoas da América do Sul e América Central.}{cai.mão}{0}
\verb{câimbra}{}{Med.}{}{}{s.f.}{Cãibra.}{câim.bra}{0}
\verb{caimento}{}{}{}{}{s.m.}{Ato ou efeito de cair; queda.}{ca.i.men.to}{0}
\verb{caimento}{}{Fig.}{}{}{}{Abatimento, prostração.}{ca.i.men.to}{0}
\verb{caimento}{}{Fig.}{}{}{}{Inclinação amorosa muito forte.  }{ca.i.men.to}{0}
\verb{caimento}{}{}{}{}{}{Flexibilidade de um tecido que o faz cair verticalmente com elegância.}{ca.i.men.to}{0}
\verb{cainçada}{}{}{}{}{s.f.}{Grande ajuntamento de cães; canzoada, cainçalha.}{ca.in.ça.da}{0}
\verb{cainçalha}{}{}{}{}{s.f.}{Cainçada.}{ca.in.ça.lha}{0}
\verb{cainhar}{}{}{}{}{v.i.}{Latir (o cão) de modo doloroso.  }{ca.i.nhar}{\verboinum{1}}
\verb{cainho}{}{}{}{}{adj.}{Relativo ou próprio de cão.}{ca.i.nho}{0}
\verb{cainho}{}{}{}{}{}{Diz"-se daquele que é avaro, mesquinho.}{ca.i.nho}{0}
\verb{caipira}{}{}{}{}{s.2g.}{Habitante do campo, do interior.}{cai.pi.ra}{0}
\verb{caipira}{}{}{}{}{}{Roceiro, caboclo, capiau.}{cai.pi.ra}{0}
\verb{caipira}{}{}{}{}{}{Indivíduo tímido, acanhado.}{cai.pi.ra}{0}
\verb{caipira}{}{}{}{}{adj.}{Relativo ao homem do campo.  }{cai.pi.ra}{0}
\verb{caipirada}{}{}{}{}{s.f.}{Grupo ou ajuntamento de caipiras.}{cai.pi.ra.da}{0}
\verb{caipirada}{}{}{}{}{}{Atitude, hábito ou comportamento próprio de caipira; caipirismo.}{cai.pi.ra.da}{0}
\verb{caipirinha}{}{}{}{}{s.f.}{Bebida preparada com rodelas de limão macerados com casca, misturados e batidos com açúcar, gelo e cachaça.}{cai.pi.ri.nha}{0}
\verb{caipirinha}{}{Por ext.}{}{}{}{Bebida semelhante à anterior, mas feita com uma fruta qualquer em vez de limão e outra aguardente no lugar da cachaça.}{cai.pi.ri.nha}{0}
\verb{caipirismo}{}{}{}{}{s.m.}{Acanhamento, tolice.}{cai.pi.ris.mo}{0}
\verb{caipiríssima}{}{Bras.}{}{}{s.f.}{Caipirinha feita com vodca no lugar da cachaça.}{cai.pi.rís.si.ma}{0}
\verb{caipora}{ó}{Mit.}{}{}{s.2g.}{Personagem mítico tupi das florestas brasileiras, representado de diversas formas, conforme a região: como um menino escuro ou um pequeno indígena, ágil, nu ou de tanga, fumando cachimbo, como uma mulher perneta que anda aos saltos e outros, que sempre traz azar a quem o encontra; caapora.}{cai.po.ra}{0}
\verb{caipora}{ó}{}{}{}{adj.}{Que não tem sorte nos negócios e em outras coisas; infeliz, azarado, azarento.}{cai.po.ra}{0}
\verb{caipora}{ó}{}{}{}{s.m.}{Infelicidade.}{cai.po.ra}{0}
\verb{caiporismo}{}{}{}{}{s.m.}{Má sorte; desditas e malogros  constantes; azar.}{cai.po.ris.mo}{0}
\verb{cair}{}{}{}{}{v.i.}{Mover"-se para baixo levado pelo próprio peso; tombar. (\textit{O rapaz caiu da escada.})}{ca.ir}{0}
\verb{cair}{}{}{}{}{}{Sair de um ponto para outro mais baixo; baixar, descer.}{ca.ir}{0}
\verb{cair}{}{}{}{}{}{Chegar ao fim. (\textit{A tarde caía.})}{ca.ir}{0}
\verb{cair}{}{}{}{}{}{Perder o cargo.}{ca.ir}{0}
\verb{cair}{}{}{}{}{}{Ser apanhado por alguma coisa. (\textit{Os policiais caíram numa emboscada. })}{ca.ir}{0}
\verb{cair}{}{}{}{}{}{Ter determinada aceitação; ficar, soar. (\textit{Não cai bem convidar só um dos professores para a festa.})}{ca.ir}{0}
\verb{cair}{}{}{}{}{}{Dar determinada apresentação a pessoa ou coisa; assentar, ficar, ir. (\textit{Essa roupa cai muito bem em você.})}{ca.ir}{\verboinum{19}}
\verb{cairota}{ó}{}{}{}{s.2g.}{Indivíduo natural ou habitante dessa cidade.}{cai.ro.ta}{0}
\verb{cairota}{ó}{}{}{}{adj.}{Relativo ao Cairo, capital do Egito.}{cai.ro.ta}{0}
\verb{cais}{}{}{}{}{s.m.}{Plataforma do porto onde os navios ficam para embarque e desembarque de passageiros e cargas.}{cais}{0}
\verb{cáiser}{}{}{}{}{s.m.}{Título de imperador, na Alemanha, depois de sua unificação, no séc. \textsc{xix}, até a instituição da República, pelo fim da \textsc{i} Guerra Mundial.}{cái.ser}{0}
\verb{caititu}{}{Zool.}{}{}{s.m.}{Mamífero de pelagem branca e preta, com uma faixa branca no pescoço, em forma de colar; cateto, porco"-do"-mato.}{cai.ti.tu}{0}
\verb{caititu}{}{}{}{}{}{Cilindro manual do aparelho de ralar mandioca, nas casas de farinha.}{cai.ti.tu}{0}
\verb{caixa}{ch}{}{}{}{s.f.}{Objeto feito de madeira, papelão ou outro material, com ou sem tampa, próprio para guardar ou transportar coisas.}{cai.xa}{0}
\verb{caixa}{ch}{}{}{}{s.m.}{Lugar em que se recebem e se fazem pagamentos em um estabelecimento comercial.}{cai.xa}{0}
\verb{caixa}{ch}{}{}{}{s.2g.}{Pessoa que trabalha recebendo e fazendo pagamentos em um estabelecimento comercial.}{cai.xa}{0}
%\verb{}{}{}{}{}{}{}{}{0}
%\verb{}{}{}{}{}{}{}{}{0}
\verb{caixa"-d'água}{ch}{}{caixas"-d'água ⟨ch⟩}{}{s.f.}{Recipiente em que se guarda a água para uma casa, um bairro ou uma cidade; reservatório.}{cai.xa"-d'á.gua}{0}
\verb{caixa"-forte}{ch\ldots{}ó}{}{caixas"-fortes ⟨ch\ldots{}ó⟩}{}{s.m.}{Cofre à prova de roubo e de fogo, geralmente em dependência de casa bancária ou estabelecimento comercial, ou em repartição pública, para guardar valores, joias, documentos.}{cai.xa"-for.te}{0}
\verb{caixão}{ch}{}{"-ões}{}{s.m.}{Caixa grande.}{cai.xão}{0}
\verb{caixão}{ch}{}{"-ões}{}{s.m.}{Caixa oblonga, de madeira, dentro da qual se enterram os mortos; ataúde, esquife, féretro.}{cai.xão}{0}
\verb{caixa"-pregos}{ch\ldots{}é}{}{}{}{s.m.}{Lugar muito afastado, muito longínquo; cafundó.}{cai.xa"-pre.gos}{0}
\verb{caixa"-preta}{ch\ldots{}ê}{}{caixas"-pretas ⟨ch\ldots{}ê⟩}{}{s.f.}{Caixa metálica eletrônica que registra todas as circunstâncias de um voo.}{cai.xa"-pre.ta}{0}
\verb{caixeiral}{ch}{}{"-ais}{}{adj.2g.}{Relativo a caixeiro (balconista) ou à sua ocupação. }{cai.xei.ral}{0}
\verb{caixeiro}{ch}{}{}{}{s.m.}{Indivíduo que atende os fregueses no comércio; balconista.}{cai.xei.ro}{0}
\verb{caixeiro"-viajante}{ch}{}{caixeiros"-viajantes ⟨ch⟩}{}{s.m.}{Vendedor que exerce suas atividades viajando.  }{cai.xei.ro"-vi.a.jan.te}{0}
\verb{caixeta}{chê}{}{}{}{s.f.}{Caixinha.}{cai.xe.ta}{0}
\verb{caixilho}{ch}{}{}{}{s.m.}{Parte da esquadria onde se fixam os vidros.}{cai.xi.lho}{0}
\verb{caixilho}{ch}{}{}{}{}{Moldura de quadro.}{cai.xi.lho}{0}
\verb{caixinha}{ch}{}{}{}{s.f.}{Caixa pequena.}{cai.xi.nha}{0}
\verb{caixinha}{ch}{}{}{}{}{Coleta de dinheiro com determinado propósito.}{cai.xi.nha}{0}
\verb{caixinha}{ch}{}{}{}{}{Gorjeta.}{cai.xi.nha}{0}
\verb{caixote}{chó}{}{}{}{s.m.}{Caixa tosca, de tamanho mediano, para embalagem de produtos, mercadorias etc. }{cai.xo.te}{0}
\verb{cajá}{}{}{}{}{s.m.}{Fruto da cajazeira, amarelo, muito aromático, suculento e azedo, de que se fazem refrescos e sorvetes.}{ca.já}{0}
\verb{cajadada}{}{}{}{}{s.f.}{Golpe ou pancada de cajado.}{ca.ja.da.da}{0}
\verb{cajado}{}{}{}{}{s.m.}{Bastão de madeira com a parte superior arqueada, usado como apoio.}{ca.ja.do}{0}
\verb{cajarana}{}{}{}{}{s.f.}{Cajá"-manga.}{ca.ja.ra.na}{0}
\verb{cajazeira}{ê}{Bot.}{}{}{s.f.}{Árvore de até 25 m, cujo fruto, amarelo, suculento e azedo, é o cajá.}{ca.ja.zei.ra}{0}
\verb{cajazeiro}{ê}{Bot.}{}{}{s.m.}{Cajazeira.}{ca.ja.zei.ro}{0}
\verb{caju}{}{}{}{}{s.m.}{Pseudofruto do cajueiro, na forma de pedúnculo carnoso e suculento, amarelo, rosado ou vermelho, comestível, que pende do verdadeiro fruto (a castanha"-de"-caju), e que é muito apreciado em refrescos, doces etc. }{ca.ju}{0}
\verb{cajuada}{}{}{}{}{s.f.}{Refresco feito com suco de caju.}{ca.ju.a.da}{0}
\verb{cajuada}{}{Cul.}{}{}{}{Doce de caju.}{ca.ju.a.da}{0}
\verb{cajual}{}{}{"-ais}{}{s.m.}{Aglomerado de cajueiros numa determinada área; cajueiral.}{ca.ju.al}{0}
\verb{cajueiral}{}{}{"-ais}{}{s.m.}{Cajual.}{ca.ju.ei.ral}{0}
\verb{cajueiro}{ê}{Bot.}{}{}{s.m.}{Árvore cujo pseudofruto é o caju, e o fruto, a castanha"-de"-caju, nativa do Norte e Nordeste do Brasil, de caule tortuoso, folhas grandes, coriáceas, e flores pequenas e vermelhas.}{ca.ju.ei.ro}{0}
\verb{cajuína}{}{}{}{}{s.f.}{Tipo de vinho preparado com caju.}{ca.ju.í.na}{0}
\verb{cajuzeiro}{ê}{Bot.}{}{}{s.m.}{Cajueiro.}{ca.ju.zei.ro}{0}
\verb{cal}{}{}{}{}{s.f.}{Pó branco constituído principalmente de cálcio com diversas utilizações na construção, indústria, proteção contra fogo, tratamento de água.}{cal}{0}
\verb{calabouço}{}{}{}{}{s.m.}{Prisão subterrânea; cárcere, cadeia.}{ca.la.bou.ço}{0}
\verb{calabre}{}{}{}{}{s.m.}{Corda grossa.}{ca.la.bre}{0}
\verb{calabrês}{}{}{}{}{adj.}{Relativo à Calábria, região do sul da Itália.}{ca.la.brês}{0}
\verb{calabrês}{}{}{}{}{s.m.}{Indivíduo natural ou habitante dessa região.}{ca.la.brês}{0}
\verb{calaceiro}{ê}{}{}{}{adj.}{Preguiçoso, vadio.}{ca.la.cei.ro}{0}
\verb{calada}{}{}{}{}{s.f.}{Ausência de ruído; silêncio.}{ca.la.da}{0}
\verb{calado}{}{}{}{}{adj.}{Diz"-se de indivíduo que fala pouco ou não fala.}{ca.la.do}{0}
\verb{calado}{}{}{}{}{s.m.}{Distância vertical entre a face inferior da quilha de uma embarcação e a superfície da água.}{ca.la.do}{0}
\verb{calado}{}{}{}{}{adv.}{Em silêncio.}{ca.la.do}{0}
\verb{calafate}{}{}{}{}{s.m.}{Indivíduo especializado em calafetação.}{ca.la.fa.te}{0}
\verb{calafetar}{}{}{}{}{v.t.}{Tapar ou vedar frestas, buracos, fendas.}{ca.la.fe.tar}{\verboinum{1}}
\verb{calafrio}{}{}{}{}{s.m.}{Sensação de frio acompanhada de tremor, contrações ou bater de queixo.}{ca.la.fri.o}{0}
\verb{calafrio}{}{}{}{}{}{Sensação de frio relacionada com doença ou febre.}{ca.la.fri.o}{0}
\verb{calamidade}{}{}{}{}{s.f.}{Grande desgraça; catástrofe, infortúnio.}{ca.la.mi.da.de}{0}
\verb{calamidade}{}{Fig.}{}{}{}{Coisa ou pessoa com características muito inconvenientes.}{ca.la.mi.da.de}{0}
\verb{calamitoso}{ô}{}{"-osos ⟨ó⟩}{"-osa ⟨ó⟩}{adj.}{Que envolve calamidade; catastrófico, infeliz.}{ca.la.mi.to.so}{0}
\verb{calandra}{}{}{}{}{s.f.}{Prensa que produz matrizes de estereotipia.}{ca.lan.dra}{0}
\verb{calandra}{}{}{}{}{}{Máquina para acetinar papel.}{ca.lan.dra}{0}
\verb{calandrar}{}{}{}{}{v.t.}{Produzir matriz estereotípica na calandra.}{ca.lan.drar}{0}
\verb{calandrar}{}{}{}{}{}{Acetinar na calandra.}{ca.lan.drar}{\verboinum{1}}
\verb{calango}{}{Zool.}{}{}{s.m.}{Lagarto de pequeno porte de cores verde e branca.}{ca.lan.go}{0}
\verb{calão}{}{}{"-ões}{}{s.m.}{Linguajar caracterizado pelo uso de termos grosseiros.}{ca.lão}{0}
\verb{calão}{}{}{"-ões}{}{s.m.}{Tipo de embarcação comprida e larga utilizada na pesca de atum.}{ca.lão}{0}
\verb{calão}{}{}{"-ões}{}{s.m.}{Recipiente de barro ou cobre, de origem indiana, para líquidos em geral.}{ca.lão}{0}
\verb{calão}{}{}{"-ões}{}{}{Unidade de medida de capacidade, equivalente a aproximadamente 8 litros.}{ca.lão}{0}
\verb{calar}{}{}{}{}{v.i.}{Permanecer em silêncio.}{ca.lar}{0}
\verb{calar}{}{}{}{}{v.t.}{Fazer cessar de falar ou de produzir ruído; silenciar.}{ca.lar}{0}
\verb{calar}{}{}{}{}{}{Impedir de reclamar ou de manifestar"-se; reprimir.}{ca.lar}{\verboinum{1}}
\verb{calça}{}{}{}{}{s.f.}{Peça de vestuário que cobre cada uma das pernas e vai até a cintura.}{cal.ça}{0}
\verb{calça}{}{}{}{}{}{Peça íntima do vestuário feminino; calcinha.}{cal.ça}{0}
\verb{calçada}{}{}{}{}{s.f.}{Caminho pavimentado para os pedestres, situado às margens das ruas e avenidas e geralmente em um nível pouco mais alto que elas; passeio.}{cal.ça.da}{0}
\verb{calçada}{}{}{}{}{}{Caminho revestido de pedras, paralelepípedos ou outro tipo de pavimentação semelhante.}{cal.ça.da}{0}
\verb{calçadão}{}{Bras.}{"-ões}{}{s.m.}{Calçada ampla e longa, geralmente com elementos paisagísticos.}{cal.ça.dão}{0}
\verb{calçadão}{}{Bras.}{"-ões}{}{}{Rua para circulação exclusiva de pedestres.}{cal.ça.dão}{0}
\verb{calçadeira}{ê}{}{}{}{s.f.}{Utensílio em forma de canaleta curta que se coloca entre o calcanhar e a parte de trás do sapato para facilitar a entrada do pé no calçado.}{cal.ça.dei.ra}{0}
\verb{calçado}{}{}{}{}{adj.}{Diz"-se de rua ou terreno coberto por camada regular de pedras ou outro material.}{cal.ça.do}{0}
\verb{calçado}{}{}{}{}{}{Diz"-se de pessoa que tem os pés cobertos ou protegidos por algum calçado.}{cal.ça.do}{0}
\verb{calçado}{}{}{}{}{}{Diz"-se de móvel ou utensílio em cujos pés de apoio se colocou um calço.}{cal.ça.do}{0}
\verb{calçado}{}{}{}{}{s.m.}{Peça de vestuário feita de couro ou material resistente para proteger os pés; sapato.}{cal.ça.do}{0}
\verb{calçamento}{}{}{}{}{s.m.}{Ato ou efeito de calçar.}{cal.ça.men.to}{0}
\verb{calçamento}{}{}{}{}{}{Aquilo que calça.}{cal.ça.men.to}{0}
\verb{calçamento}{}{}{}{}{}{Camada de pedras ou outro material que reveste uma rua ou terreno.}{cal.ça.men.to}{0}
\verb{calcâneo}{}{Anat.}{}{}{s.m.}{O maior osso do tarso, o qual forma a saliência do calcanhar.}{cal.câ.neo}{0}
\verb{calcanhar}{}{Anat.}{}{}{s.m.}{Parte posterior do pé, formada pelo calcâneo e pelos tendões e músculos que unem o pé à perna.}{cal.ca.nhar}{0}
\verb{calcanhar}{}{}{}{}{}{Parte do calçado ou da meia que corresponde ou que cobre o calcanhar.}{cal.ca.nhar}{0}
\verb{calcanhar"-de"-aquiles}{}{}{calcanhares"-de"-aquiles}{}{s.m.}{Parte em que alguém ou algo é vulnerável.}{cal.ca.nhar"-de"-a.qui.les}{0}
\verb{calção}{}{}{"-ões}{}{s.m.}{Calça curta que chega somente até o joelho ou um pouco acima; bermuda.}{cal.ção}{0}
\verb{calção}{}{}{"-ões}{}{}{Peça de vestuário muito curta e aderente ao corpo usada por banhistas e desportistas.}{cal.ção}{0}
\verb{calcar}{}{}{}{}{v.t.}{Comprimir com os pés; pisar.}{cal.car}{0}
\verb{calcar}{}{}{}{}{}{Comprimir com força; machucar.}{cal.car}{0}
\verb{calcar}{}{Fig.}{}{}{}{Humilhar, vexar.}{cal.car}{0}
\verb{calcar}{}{}{}{}{}{Tratar com menosprezo.}{cal.car}{0}
\verb{calcar}{}{}{}{}{}{Reproduzir um desenho; decalcar.}{cal.car}{\verboinum{2}}
\verb{calçar}{}{}{}{}{v.t.}{Pôr os pés em calçado, as mãos em luvas, ou as pernas em calças etc.}{cal.çar}{0}
\verb{calçar}{}{}{}{}{}{Fazer o calçamento; empedrar, pavimentar.}{cal.çar}{0}
\verb{calçar}{}{}{}{}{}{Pôr calço ou cunha.}{cal.çar}{\verboinum{3}}
\verb{calcário}{}{}{}{}{adj.}{Relativo a cálcio ou que contém cálcio.}{cal.cá.rio}{0}
\verb{calcário}{}{}{}{}{}{Diz"-se de rocha sedimentar que tem carbonato de cálcio entre seus constituintes.}{cal.cá.rio}{0}
\verb{calcário}{}{Geol.}{}{}{s.m.}{Designação comum às rochas sedimentares cujo constituinte principal é carbonato de cálcio e de magnésio, como calcita, dolomita, mármore, giz.}{cal.cá.rio}{0}
\verb{calças}{}{}{}{}{s.f.}{Calça.}{cal.ças}{0}
\verb{calceiro}{ê}{}{}{}{s.m.}{Indivíduo que faz calças.}{cal.cei.ro}{0}
\verb{calceta}{ê}{}{}{}{s.f.}{Argola de ferro fixada ao tornozelo do prisioneiro e ligada a outro prisioneiro por uma corrente.}{cal.ce.ta}{0}
\verb{calceta}{ê}{Por ext.}{}{}{}{A pena de trabalhos forçados.}{cal.ce.ta}{0}
\verb{calceta}{ê}{}{}{}{s.2g.}{Indivíduo condenado à calceta.}{cal.ce.ta}{0}
\verb{calceteiro}{ê}{}{}{}{s.m.}{Indivíduo que faz o calçamento de ruas e caminhos.}{cal.ce.tei.ro}{0}
\verb{calcificação}{}{Med.}{"-ões}{}{s.f.}{Depósito de sais de cálcio, como parte de um processo que pode ser normal ou patológico.}{cal.ci.fi.ca.ção}{0}
\verb{calcificar}{}{}{}{}{v.t.}{Tornar rígido pelo acúmulo de sais de cálcio.}{cal.ci.fi.car}{0}
\verb{calcificar}{}{}{}{}{}{Dar ou tomar consistência de cal.}{cal.ci.fi.car}{0}
\verb{calcificar}{}{}{}{}{}{Tornar"-se calcário pela absorção de substâncias que contêm cálcio ou cal.}{cal.ci.fi.car}{\verboinum{2}}
\verb{calcinação}{}{}{"-ões}{}{s.f.}{Ato ou efeito de calcinar.}{cal.ci.na.ção}{0}
\verb{calcinação}{}{Quím.}{"-ões}{}{}{Processo térmico pelo qual se provoca a decomposição de algumas de suas substâncias.}{cal.ci.na.ção}{0}
\verb{calcinar}{}{}{}{}{v.t.}{Aquecer muito; incinerar, carbonizar.}{cal.ci.nar}{0}
\verb{calcinar}{}{}{}{}{}{Cauterizar, abrasar.}{cal.ci.nar}{\verboinum{1}}
\verb{calcinha}{}{}{}{}{s.f.}{Peça íntima do vestuário feminino que cobre a região genital.}{cal.ci.nha}{0}
%\verb{}{}{}{}{}{}{}{}{0}
\verb{cálcio}{}{Quím.}{}{}{s.m.}{Elemento químico metálico, prateado, do grupo dos alcalino"-terrosos, utilizado na forma de compostos em diversas aplicações industriais e tecnológicas. \elemento{20}{40.078}{Ca}.}{cál.ci.o}{0}
\verb{calço}{}{}{}{}{s.m.}{Objeto, geralmente de pequena dimensão, colocado debaixo de outro, para mantê"-lo aprumado ou firme em determinada posição; cunha.}{cal.ço}{0}
\verb{calçudo}{}{}{}{}{adj.}{Que usa calças muito compridas ou muito grandes.}{cal.çu.do}{0}
\verb{calçudo}{}{Zool.}{}{}{s.m.}{Aves que têm as pernas cobertas de penas até os pés.}{cal.çu.do}{0}
\verb{calculador}{ô}{}{}{}{adj.}{Que faz cálculo.}{cal.cu.la.dor}{0}
\verb{calculador}{ô}{}{}{}{s.m.}{Nos observatórios, indivíduo encarregado dos cálculos matemáticos.}{cal.cu.la.dor}{0}
\verb{calculador}{ô}{Fig.}{}{}{}{Indivíduo que nada faz sem um fim interesseiro ou previdente; calculista.}{cal.cu.la.dor}{0}
\verb{calculadora}{ô}{}{}{}{s.f.}{Máquina que efetua operações aritméticas.}{cal.cu.la.do.ra}{0}
\verb{calcular}{}{}{}{}{v.t.}{Determinar por meio de cálculo.}{cal.cu.lar}{0}
\verb{calcular}{}{}{}{}{}{Contar, computar.}{cal.cu.lar}{0}
\verb{calcular}{}{}{}{}{}{Avaliar com maior ou menor precisão; estimar.}{cal.cu.lar}{0}
\verb{calcular}{}{}{}{}{}{Imaginar, avaliar, presumir.}{cal.cu.lar}{\verboinum{1}}
\verb{calculista}{}{}{}{}{adj.2g.}{Que calcula.}{cal.cu.lis.ta}{0}
\verb{calculista}{}{}{}{}{s.2g.}{Indivíduo cujas ações premeditadas visam a seus interesses pessoais.}{cal.cu.lis.ta}{0}
\verb{cálculo}{}{}{}{}{s.m.}{Execução de operação ou processo matemático ou algébrico.}{cál.cu.lo}{0}
\verb{cálculo}{}{}{}{}{}{Estimativa, conjectura, previsão.}{cál.cu.lo}{0}
\verb{cálculo}{}{}{}{}{}{Sentimento de cobiça.}{cál.cu.lo}{0}
\verb{cálculo}{}{Med.}{}{}{}{Concreção pétrea que se forma em certas partes do organismo pela agregação de diversas substâncias.}{cál.cu.lo}{0}
\verb{calda}{}{}{}{}{s.f.}{Solução fervida de açúcar e água; xarope.}{cal.da}{0}
\verb{calda}{}{}{}{}{}{Suco de qualquer fruta fervido com açúcar, formando um líquido viscoso.}{cal.da}{0}
\verb{calda}{}{Bras.}{}{}{}{Resíduo da destilação do caldo da cana.}{cal.da}{0}
\verb{caldeamento}{}{}{}{}{s.m.}{Ato ou efeito de caldear; caldeação.}{cal.de.a.men.to}{0}
\verb{caldeamento}{}{Fig.}{}{}{}{Mistura, miscigenação.}{cal.de.a.men.to}{0}
\verb{caldear}{}{}{}{}{v.t.}{Tornar incandescente.}{cal.de.ar}{0}
\verb{caldear}{}{}{}{}{}{Soldar, ligar metais por meio de aquecimento.}{cal.de.ar}{0}
\verb{caldear}{}{}{}{}{}{Dissolver na água criando uma pasta.}{cal.de.ar}{0}
\verb{caldear}{}{Fig.}{}{}{}{Mestiçar, miscigenar.}{cal.de.ar}{\verboinum{4}}
\verb{caldeira}{ê}{}{}{}{s.f.}{Recipiente grande e de metal para aquecer líquidos ou produzir vapor.}{cal.dei.ra}{0}
\verb{caldeirada}{}{}{}{}{s.f.}{A quantidade de líquido que cabe em uma caldeira.}{cal.dei.ra.da}{0}
\verb{caldeirada}{}{Por ext.}{}{}{}{A porção de líquido que se derrama de qualquer recipiente.}{cal.dei.ra.da}{0}
\verb{caldeirada}{}{Cul.}{}{}{}{Cozido feito com diversos tipos de peixe.}{cal.dei.ra.da}{0}
\verb{caldeirão}{}{}{"-ões}{}{s.m.}{Tipo de panela grande e com alças, geralmente semelhante a uma esfera, utilizada principalmente para ferver água ou cozinhar em água fervente.}{cal.dei.rão}{0}
\verb{caldeireiro}{ê}{}{}{}{s.m.}{Indivíduo que faz caldeiras e outros utensílios de metal.}{cal.dei.rei.ro}{0}
\verb{caldeireiro}{ê}{Pop.}{}{}{}{Indivíduo que anuncia a chuva.}{cal.dei.rei.ro}{0}
\verb{caldeirinha}{}{}{}{}{s.f.}{Pequeno recipiente para água benta.}{cal.dei.ri.nha}{0}
\verb{caldeu}{}{}{}{caldeia}{adj.}{Relativo à Caldeia, antiga região da Ásia.}{cal.deu}{0}
\verb{caldeu}{}{}{}{caldeia}{s.m.}{Indivíduo natural ou habitante dessa região.}{cal.deu}{0}
\verb{caldo}{}{Cul.}{}{}{s.m.}{Alimento líquido ou molho no qual são cozidos carne, peixe, legumes.}{cal.do}{0}
\verb{caldo}{}{Bras.}{}{}{}{Suco extraído de frutas ou de certas partes de algumas plantas.}{cal.do}{0}
\verb{caldo}{}{Pop.}{}{}{}{Brincadeira que consiste em afundar os outros na água do mar ou de piscina.}{cal.do}{0}
\verb{caleça}{é}{}{}{}{s.f.}{Caleche.}{ca.le.ça}{0}
\verb{caleche}{é}{}{}{}{s.m.}{Carro de dois assentos e quatro rodas, descoberto na parte dianteira e puxado por uma parelha de cavalos; caleça.}{ca.le.che}{0}
\verb{calefação}{}{}{"-ões}{}{s.f.}{Sistema de aquecimento ambiente.}{ca.le.fa.ção}{0}
\verb{caleidoscópio}{}{}{}{}{s.m.}{Tubo cilíndrico com espelhos e um furo central na extremidade, por meio do qual se observam as imagens formadas pelos pequenos objetos coloridos em seu interior; calidoscópio.}{ca.lei.dos.có.pio}{0}
\verb{caleidoscópio}{}{Por ext.}{}{}{}{Imagem composta de diversas cores e formas em constante mutação.}{ca.lei.dos.có.pio}{0}
\verb{caleidoscópio}{}{Por ext.}{}{}{}{Sucessão vertiginosa de sensações.}{ca.lei.dos.có.pio}{0}
\verb{calejado}{}{}{}{}{adj.}{Que tem ou desenvolveu calos; caloso.}{ca.le.ja.do}{0}
\verb{calejado}{}{Fig.}{}{}{}{Que tem experiência, prática, habilidade.}{ca.le.ja.do}{0}
\verb{calejado}{}{Fig.}{}{}{}{Diz"-se de indivíduo que se tornou insensível, empedernido.}{ca.le.ja.do}{0}
\verb{calejar}{}{}{}{}{v.t.}{Criar ou adquirir calos.}{ca.le.jar}{0}
\verb{calejar}{}{Fig.}{}{}{}{Tornar insensível, empedernido.}{ca.le.jar}{\verboinum{1}}
\verb{calendário}{}{}{}{}{s.m.}{Tabela na qual se dispõem os dias, as semanas e os meses do ano, indicando feriados, festas e fases da Lua.}{ca.len.dá.rio}{0}
\verb{calendário}{}{Por ext.}{}{}{}{Conjunto de datas estabelecidas para a realização de determinados eventos.}{ca.len.dá.rio}{0}
\verb{calendário}{}{}{}{}{}{Sistema de divisão do tempo, geralmente baseado em fatores astronômicos (órbita do Sol ou da Lua) e em convenções específicas.}{ca.len.dá.rio}{0}
\verb{calendas}{}{}{}{}{s.f.pl.}{Primeiro dia do mês romano.}{ca.len.das}{0}
\verb{calêndula}{}{Bot.}{}{}{s.f.}{Planta ornamental de flores amarelas e brancas, usada em medicamentos ou  cosméticos.}{ca.lên.du.la}{0}
\verb{calêndula}{}{}{}{}{}{A flor dessa planta.}{ca.lên.du.la}{0}
\verb{calha}{}{}{}{}{s.f.}{Cano aberto na parte de cima, feito de zinco ou outro material metálico, para coletar e escoar água da chuva nos telhados.}{ca.lha}{0}
\verb{calhamaço}{}{Pop.}{}{}{s.m.}{Livro ou caderno volumoso, com muitas páginas.}{ca.lha.ma.ço}{0}
\verb{calhambeque}{é}{}{}{}{s.m.}{Barco velho ou de mau aspecto, que não inspira segurança.}{ca.lham.be.que}{0}
\verb{calhambeque}{é}{}{}{}{}{Automóvel velho e maltratado.}{ca.lham.be.que}{0}
\verb{calhambeque}{é}{}{}{}{}{Objeto velho e sem valor; traste.}{ca.lham.be.que}{0}
\verb{calhandra}{}{Zool.}{}{}{s.f.}{Ave encontrada na Europa, Ásia e África, semelhante à cotovia.}{ca.lhan.dra}{0}
\verb{calhar}{}{}{}{}{v.i.}{Entrar ou caber em calha; encaixar"-se.}{ca.lhar}{0}
\verb{calhar}{}{}{}{}{}{Vir a tempo; ser oportuno.}{ca.lhar}{0}
\verb{calhar}{}{}{}{}{v.t.}{Acontecer, coincidir, suceder.}{ca.lhar}{0}
\verb{calhar}{}{}{}{}{}{Adaptar"-se; cair bem.}{ca.lhar}{\verboinum{1}}
\verb{calhau}{}{}{}{}{s.m.}{Pedra de pequenas dimensões; seixo.}{ca.lhau}{0}
\verb{calhau}{}{Bras.}{}{}{}{Na gíria jornalística, texto de pouco interesse, aproveitado para preencher claros na paginação de jornal ou de revista.  }{ca.lhau}{0}
\verb{calhorda}{ó}{}{}{}{s.m.}{Indivíduo desprezível, patife, pulha.}{ca.lhor.da}{0}
\verb{calibrador}{ô}{}{}{}{s.m.}{Instrumento para calibrar; calibre.}{ca.li.bra.dor}{0}
\verb{calibragem}{}{}{"-ens}{}{s.f.}{Ato ou efeito de calibrar.}{ca.li.bra.gem}{0}
\verb{calibrar}{}{}{}{}{v.t.}{Dar o calibre adequado.}{ca.li.brar}{0}
\verb{calibrar}{}{}{}{}{}{Medir o calibre.}{ca.li.brar}{0}
\verb{calibrar}{}{Bras.}{}{}{}{Dar a pressão de ar adequada a pneu, câmara"-de"-ar.}{ca.li.brar}{\verboinum{1}}
\verb{calibre}{}{}{}{}{s.m.}{Diâmetro interno do cano de arma de fogo ou de qualquer cano.}{ca.li.bre}{0}
\verb{calibre}{}{}{}{}{}{Diâmetro de um projétil.}{ca.li.bre}{0}
\verb{calibre}{}{}{}{}{}{Capacidade de um recipiente.}{ca.li.bre}{0}
\verb{calibre}{}{Fig.}{}{}{}{Tamanho, valor, dimensão, volume.}{ca.li.bre}{0}
\verb{calibre}{}{Fig.}{}{}{}{Valor socialmente atribuído; merecimento.}{ca.li.bre}{0}
\verb{calibre}{}{}{}{}{}{Calibrador.}{ca.li.bre}{0}
\verb{caliça}{}{}{}{}{s.f.}{Pó e argamassa seca que sobram de uma obra; entulho.}{ca.li.ça}{0}
\verb{cálice}{}{}{}{}{s.m.}{Taça de dimensões reduzidas usada geralmente para licores e vinhos finos.}{cá.li.ce}{0}
\verb{cálice}{}{Relig.}{}{}{}{O vaso utilizado na missa para consagrar o vinho.}{cá.li.ce}{0}
\verb{cálice}{}{Bot.}{}{}{}{Invólucro externo de uma flor, formado pelas sépalas.}{cá.li.ce}{0}
\verb{calicida}{}{}{}{}{adj.}{Que remove calos e calosidades.}{ca.li.ci.da}{0}
\verb{calidez}{ê}{}{}{}{s.f.}{Qualidade de cálido.}{ca.li.dez}{0}
\verb{cálido}{}{}{}{}{adj.}{Que tem calor; quente.}{cá.li.do}{0}
\verb{cálido}{}{Fig.}{}{}{}{Entusiasmado, ardente, fogoso.}{cá.li.do}{0}
\verb{cálido}{}{}{}{}{adj.}{Astuto, sagaz, fino.}{cá.li.do}{0}
\verb{calidoscópio}{}{}{}{}{}{Var. de \textit{caleidoscópio}.}{ca.li.dos.có.pio}{0}
\verb{califa}{}{}{}{}{s.m.}{Título de soberano muçulmano.}{ca.li.fa}{0}
\verb{califado}{}{}{}{}{s.m.}{Dignidade ou jurisdição de califa.}{ca.li.fa.do}{0}
\verb{califado}{}{}{}{}{}{Território governado por um califa.}{ca.li.fa.do}{0}
\verb{califado}{}{}{}{}{}{Tempo durante o qual um califa governa.}{ca.li.fa.do}{0}
\verb{califórnio}{}{Quím.}{}{}{s.m.}{Elemento químico radioativo, do grupo dos actinídeos, obtido artificialmente, e usado como fonte de radiação em medicina. \elemento{98}{(251)}{Cf}.}{ca.li.fór.nio}{0}
\verb{caligrafia}{}{}{}{}{s.f.}{Técnica de escrita à mão formando letras bem legíveis e elegantes segundo um padrão determinado de beleza.}{ca.li.gra.fi.a}{0}
\verb{caligrafia}{}{Por ext.}{}{}{}{Desenho ou estilo peculiar de escrita à mão.}{ca.li.gra.fi.a}{0}
\verb{calígrafo}{}{}{}{}{s.m.}{Indivíduo que se dedica ao ensino ou à prática da caligrafia.}{ca.lí.gra.fo}{0}
\verb{calipígio}{}{}{}{}{adj.}{Que tem belas nádegas.}{ca.li.pí.gio}{0}
\verb{calista}{}{}{}{}{s.2g.}{Indivíduo especializado no tratamento dos pés, especialmente de calos; pedicuro.}{ca.lis.ta}{0}
\verb{calma}{}{}{}{}{s.f.}{Calor atmosférico, geralmente com ausência de ventos.}{cal.ma}{0}
\verb{calma}{}{Por ext.}{}{}{}{O momento mais quente do dia.}{cal.ma}{0}
\verb{calma}{}{}{}{}{}{Calmaria.}{cal.ma}{0}
\verb{calma}{}{Fig.}{}{}{}{Ausência de perturbação mental; tranquilidade, sossego.}{cal.ma}{0}
\verb{calmante}{}{}{}{}{adj.2g.}{Que acalma.}{cal.man.te}{0}
\verb{calmar}{}{}{}{}{v.t.}{Tornar calmo.}{cal.mar}{0}
\verb{calmar}{}{}{}{}{}{Fazer perder a intensidade.}{cal.mar}{0}
\verb{calmar}{}{}{}{}{}{Pacificar, reprimir.}{cal.mar}{\verboinum{1}}
\verb{calmaria}{}{}{}{}{s.f.}{Ausência de ventos ou do movimento das ondas.}{cal.ma.ri.a}{0}
\verb{calmaria}{}{}{}{}{}{Grande calor sem vento.}{cal.ma.ri.a}{0}
\verb{calmaria}{}{Fig.}{}{}{}{Tranquilidade, sossego.}{cal.ma.ri.a}{0}
\verb{calmo}{}{}{}{}{adj.}{Em que há calma.}{cal.mo}{0}
\verb{calmo}{}{}{}{}{}{Sossegado, tranquilo.}{cal.mo}{0}
\verb{calmo}{}{}{}{}{}{Em estado de calmaria.}{cal.mo}{0}
\verb{calmoso}{ô}{}{"-osos ⟨ó⟩}{"-osa ⟨ó⟩}{adj.}{Em que há calma; quente, abafado.}{cal.mo.so}{0}
\verb{calo}{}{}{}{}{s.m.}{Endurecimento da pele formado em região que recebe compressão ou fricção contínua ou frequente; calosidade.}{ca.lo}{0}
\verb{calo}{}{Fig.}{}{}{}{Falta de sensibilidade moral; embrutecimento.}{ca.lo}{0}
\verb{calombo}{}{}{}{}{s.m.}{Inchaço na superfície do corpo, aparente ou que pode ser sentido pelo tato.}{ca.lom.bo}{0}
\verb{calombo}{}{Por ext.}{}{}{}{Qualquer inchaço ou protuberância, especialmente as arredondadas.}{ca.lom.bo}{0}
\verb{calombo}{}{}{}{}{}{Leite ou sangue coagulado; coágulo, coalho.}{ca.lom.bo}{0}
\verb{calor}{ô}{}{}{}{}{Sensação produzida pelo contato ou proximidade com um corpo quente.}{ca.lor}{0}
\verb{calor}{ô}{}{}{}{s.m.}{Qualidade, estado ou condição daquilo que é quente.}{ca.lor}{0}
\verb{calor}{ô}{}{}{}{}{Temperatura elevada, quentura.}{ca.lor}{0}
\verb{calor}{ô}{Fig.}{}{}{}{Ardor, vivacidade, animação.}{ca.lor}{0}
\verb{calor}{ô}{Fís. e Quím.}{}{}{}{Forma de energia que se transfere de um sistema para outro devido à diferença de temperatura entre eles.}{ca.lor}{0}
\verb{calorão}{}{}{"-ões}{}{s.m.}{Calor forte, intenso.}{ca.lo.rão}{0}
\verb{calorento}{}{}{}{}{adj.}{Que apresenta ou produz elevação da temperatura.}{ca.lo.ren.to}{0}
\verb{calorento}{}{}{}{}{}{Diz"-se do indivíduo muito sensível ao calor.}{ca.lo.ren.to}{0}
\verb{caloria}{}{}{}{}{s.f.}{Unidade de medida do valor nutritivo dos alimentos.}{ca.lo.ri.a}{0}
\verb{caloria}{}{Fís.}{}{}{}{Unidade de medida de calor que equivale a 4,18 joules.}{ca.lo.ri.a}{0}
\verb{calórico}{}{}{}{}{adj.}{Relativo a calor ou a caloria.}{ca.ló.ri.co}{0}
\verb{calórico}{}{}{}{}{}{Diz"-se do alimento que tem muitas calorias.}{ca.ló.ri.co}{0}
\verb{calorífero}{}{}{}{}{adj.}{Que possui ou produz calor.}{ca.lo.rí.fe.ro}{0}
\verb{calorífico}{}{}{}{}{adj.}{Relativo a trocas de energia sob a forma de calor.}{ca.lo.rí.fi.co}{0}
\verb{calorífico}{}{}{}{}{}{Diz"-se do aparelho que produz e transmite calor.}{ca.lo.rí.fi.co}{0}
\verb{calorimetria}{}{Fís.}{}{}{s.f.}{Parte da física dedicada à medição das quantidades de calor absorvidas ou emitidas num processo físico ou químico.}{ca.lo.ri.me.tri.a}{0}
\verb{caloroso}{ô}{}{"-osos ⟨ó⟩}{"-osa ⟨ó⟩}{adj.}{Que provoca calor; quente, calorento.}{ca.lo.ro.so}{0}
\verb{caloroso}{ô}{}{"-osos ⟨ó⟩}{"-osa ⟨ó⟩}{}{Que demonstra animação, vivacidade; cordial, entusiástico.}{ca.lo.ro.so}{0}
\verb{calosidade}{}{}{}{}{s.f.}{Espessamento da pele devido a atrito ou irritação.}{ca.lo.si.da.de}{0}
\verb{calosidade}{}{}{}{}{}{Qualquer formação saliente, dura e áspera, na superfície de algo.}{ca.lo.si.da.de}{0}
\verb{caloso}{ô}{}{"-osos ⟨ó⟩}{"-osa ⟨ó⟩}{adj.}{Que apresenta calos; calejado.}{ca.lo.so}{0}
\verb{caloso}{ô}{}{"-osos ⟨ó⟩}{"-osa ⟨ó⟩}{}{Que tem textura dura e áspera, semelhante a calo.}{ca.lo.so}{0}
\verb{calota}{ó}{}{}{}{s.f.}{Peça de metal abaulada que se prende à parte externa das rodas dos automóveis para proteger as extremidades do eixo e das porcas que fixam a roda.}{ca.lo.ta}{0}
\verb{calota}{ó}{Anat.}{}{}{}{Parte superior da caixa craniana.}{ca.lo.ta}{0}
\verb{calota}{ó}{Geom.}{}{}{}{Parte de uma esfera limitada por dois planos paralelos.}{ca.lo.ta}{0}
\verb{calote}{ó}{}{}{}{s.m.}{Dívida não paga ou contraída por quem não tinha intenção de pagá"-la.}{ca.lo.te}{0}
\verb{calotear}{}{}{}{}{v.t.}{Contrair dívidas sem ter intenção ou possibilidade de pagá"-las; passar calotes.}{ca.lo.te.ar}{\verboinum{4}}
\verb{caloteiro}{ê}{}{}{}{s.m.}{Indivíduo que tem o costume de não pagar dívidas, de forma intencional ou sistemática.}{ca.lo.tei.ro}{0}
\verb{calouro}{ô}{}{}{}{s.m.}{Estudante novato de qualquer curso, especialmente de curso universitário; primeiranista.}{ca.lou.ro}{0}
\verb{calouro}{ô}{}{}{}{}{Indivíduo inexperiente em qualquer ramo.}{ca.lou.ro}{0}
\verb{calouro}{ô}{}{}{}{}{Artista amador que se apresenta em programa de auditório veiculado em rádio ou televisão.}{ca.lou.ro}{0}
\verb{caluda}{}{}{}{}{interj.}{Expressão usada para impor silêncio.}{ca.lu.da}{0}
\verb{calundu}{}{}{}{}{s.m.}{Estado de humor caracterizado pela irritabilidade e enfado.}{ca.lun.du}{0}
\verb{calunga}{}{}{}{}{s.f.}{Imagem ou boneco levado à frente nos blocos de maracatu.}{ca.lun.ga}{0}
\verb{calunga}{}{}{}{}{}{Divindade ou entidade espiritual de cultos de origem banto.}{ca.lun.ga}{0}
\verb{calunga}{}{}{}{}{}{Desenho que representa forma humana usado como alvo em exercícios de tiro.}{ca.lun.ga}{0}
\verb{calúnia}{}{}{}{}{s.f.}{Acusação falsa contra alguém; mentira, falsidade.}{ca.lú.nia}{0}
\verb{calúnia}{}{Jur.}{}{}{}{Ato de imputar falsamente a alguém um fato definido como crime.}{ca.lú.nia}{0}
\verb{caluniar}{}{}{}{}{v.t.}{Atingir alguém com acusações falsas; difamar.}{ca.lu.ni.ar}{0}
\verb{caluniar}{}{}{}{}{}{Imputar a alguém fato definido como crime.}{ca.lu.ni.ar}{\verboinum{1}}
\verb{calunioso}{ô}{}{"-osos ⟨ó⟩}{"-osa ⟨ó⟩}{adj.}{Que faz calúnia; difamador.}{ca.lu.ni.o.so}{0}
\verb{calunioso}{ô}{}{"-osos ⟨ó⟩}{"-osa ⟨ó⟩}{}{Que contém calúnia; injurioso.}{ca.lu.ni.o.so}{0}
\verb{calva}{}{}{}{}{s.f.}{Parte da cabeça onde os cabelos deixaram de crescer ou de onde caíram; careca.}{cal.va}{0}
\verb{calvário}{}{}{}{}{s.m.}{Nome da colina, nas cercanias de Jerusalém, onde Jesus Cristo foi crucificado.}{cal.vá.rio}{0}
\verb{calvário}{}{Por ext.}{}{}{}{Martírio, sofrimento, tormento.}{cal.vá.rio}{0}
\verb{calvície}{}{}{}{}{s.f.}{Ausência total ou parcial de cabelos na cabeça, uma das formas de alopecia.}{cal.ví.cie}{0}
\verb{calvinismo}{}{Relig.}{}{}{s.m.}{Conjunto das ideias e doutrinas religiosas propostas por João Calvino, um dos principais responsáveis pela Reforma Protestante, no século \textsc{xvi}.}{cal.vi.nis.mo}{0}
\verb{calvinista}{}{}{}{}{adj.2g.}{Relativo ao calvinismo.}{cal.vi.nis.ta}{0}
\verb{calvinista}{}{}{}{}{s.2g.}{Adepto ou seguidor do calvinismo.}{cal.vi.nis.ta}{0}
\verb{calvo}{}{}{}{}{adj.}{Diz"-se do indivíduo que não tem cabelos na cabeça ou em parte dela; careca.}{cal.vo}{0}
\verb{calvo}{}{Fig.}{}{}{}{Diz"-se do terreno sem vegetação, árido, descalvado.}{cal.vo}{0}
\verb{calvo}{}{}{}{}{}{Diz"-se de mentira evidente, descarada.}{cal.vo}{0}
\verb{cama}{}{}{}{}{s.f.}{Móvel próprio para se dormir; leito.}{ca.ma}{0}
\verb{camada}{}{}{}{}{s.f.}{Porção de material colocado ou espalhado uniformemente sobre uma superfície.}{ca.ma.da}{0}
\verb{camada}{}{}{}{}{}{Qualquer substância aplicada sobre outra ou entre duas outras.}{ca.ma.da}{0}
\verb{camada}{}{}{}{}{}{Categoria, classe ou estrato social.}{ca.ma.da}{0}
\verb{camada}{}{Geol.}{}{}{}{Unidade individual de rocha sedimentar estratificada.}{ca.ma.da}{0}
\verb{cama"-de"-gato}{}{}{camas"-de"-gato}{}{s.f.}{Brincadeira infantil feita com um barbante  atado em uma ponta aos dedos de um participante e em outra ponta aos dedos de  um segundo participante, os quais criam várias disposições ou desenhos.}{ca.ma"-de"-ga.to}{0}
\verb{cama"-de"-gato}{}{Esport.}{camas"-de"-gato}{}{}{No futebol, lance em que um jogador, abaixado, usa o corpo para desequilibrar outro jogador que salta para alcançar a bola.}{ca.ma"-de"-ga.to}{0}
\verb{camafeu}{}{}{}{}{s.m.}{Pedra preciosa ou semipreciosa, trabalhada de modo a formar uma figura em relevo.}{ca.ma.feu}{0}
\verb{camafeu}{}{Fig.}{}{}{}{Mulher de feições delicadas.}{ca.ma.feu}{0}
\verb{camaleão}{}{Zool.}{"-ões}{}{s.m.}{Réptil arborícola com capacidade de mudar de coloração de acordo com o ambiente em que se encontra e dotado de língua longa e pegajosa e olhos com movimentos independentes.}{ca.ma.le.ão}{0}
\verb{camaleão}{}{Fig.}{"-ões}{}{}{Indivíduo que muda de opinião ou de atitude conforme lhe convenha; hipócrita, volúvel.}{ca.ma.le.ão}{0}
\verb{camará}{}{Bot.}{}{}{s.m.}{Arbusto ornamental de folhas opostas e flores amarelas com bagas roxo escuro; cambará.}{ca.ma.rá}{0}
\verb{câmara}{}{}{}{}{s.f.}{Espaço fechado. (\textit{Câmara de máquina fotográfica.})}{câ.ma.ra}{0}
\verb{câmara}{}{}{}{}{}{Máquina fotográfica. (\textit{Peguei minha câmara e tirei belas fotos.})}{câ.ma.ra}{0}
\verb{câmara}{}{}{}{}{}{Aparelho que filma e transmite imagens. (\textit{Trouxeram a câmara para iniciarmos as filmagens.})}{câ.ma.ra}{0}
\verb{câmara}{}{}{}{}{s.2g.}{Indivíduo que trabalha com um aparelho que filma e transmite imagens. (\textit{O câmara trabalha muito bem.})}{câ.ma.ra}{0}
\verb{câmara}{}{}{}{}{s.f.}{Conjunto de vereadores ou de deputados. (\textit{A câmara entrará em recesso no final do ano.})}{câ.ma.ra}{0}
\verb{camarada}{}{}{}{}{s.2g.}{Pessoa que convive bem com outra; companheiro, colega.}{ca.ma.ra.da}{0}
\verb{camarada}{}{}{}{}{}{Uma pessoa qualquer; sujeito, indivíduo.}{ca.ma.ra.da}{0}
\verb{camarada}{}{}{}{}{}{Companheiro de militância política de esquerda.}{ca.ma.ra.da}{0}
\verb{camarada}{}{}{}{}{}{Trabalhador temporário em uma propriedade rural; peão, tropeiro.}{ca.ma.ra.da}{0}
\verb{camaradagem}{}{}{"-ens}{}{s.f.}{Situação de entendimento, familiaridade, entre pessoas que convivem.}{ca.ma.ra.da.gem}{0}
\verb{camaradagem}{}{}{"-ens}{}{}{Sentimento de solidariedade, companheirismo, amizade.}{ca.ma.ra.da.gem}{0}
\verb{câmara"-de"-ar}{}{}{câmaras"-de"-ar}{}{s.f.}{Tubo de borracha vulcanizada, cheio de ar comprimido, que se coloca dentro do pneu de automóveis, bicicletas e bolas de couro.}{câ.ma.ra"-de"-ar}{0}
\verb{camarão}{}{Zool.}{"-ões}{}{s.m.}{Nome comum a vários crustáceos marinhos ou de água doce que apresentam corpo comprimido lateralmente e cinco pares de patas.}{ca.ma.rão}{0}
\verb{camarão}{}{}{"-ões}{}{}{Vaso de louça antigo.}{ca.ma.rão}{0}
\verb{camarão}{}{}{"-ões}{}{}{Gancho com que se suspendem lustres do teto.}{ca.ma.rão}{0}
\verb{camareira}{ê}{}{}{}{s.f.}{Arrumadeira de quartos de hotéis ou de camarotes de navios.}{ca.ma.rei.ra}{0}
\verb{camareira}{ê}{}{}{}{}{Mulher encarregada de organizar e conservar figurinos dos artistas de televisão, teatro, cinema etc.}{ca.ma.rei.ra}{0}
\verb{camareiro}{ê}{}{}{}{s.m.}{Empregado que arruma quartos de hotéis, camarotes de navios etc.}{ca.ma.rei.ro}{0}
\verb{camarilha}{}{}{}{}{s.f.}{Grupo de pessoas que cercam autoridades ou pessoas eminentes, procurando influir direta ou indiretamente em suas decisões.}{ca.ma.ri.lha}{0}
\verb{camarim}{}{}{"-ins}{}{s.m.}{Recinto, nos bastidores do teatro, onde os atores se vestem.}{ca.ma.rim}{0}
\verb{camarinha}{}{}{}{}{s.f.}{Quarto de dormir; alcova, aposentos.}{ca.ma.ri.nha}{0}
\verb{camarinha}{}{}{}{}{}{Pequena prateleira em um canto ou nicho da sala.}{ca.ma.ri.nha}{0}
\verb{camarinha}{}{}{}{}{}{Pingos ou gotas de suor.}{ca.ma.ri.nha}{0}
\verb{camaroeiro}{ê}{}{}{}{s.m.}{Rede utilizada para pescar camarões.}{ca.ma.ro.ei.ro}{0}
\verb{camaronense}{}{}{}{}{adj.2g.}{Relativo a Camarões.}{ca.ma.ro.nen.se}{0}
\verb{camaronense}{}{}{}{}{s.2g.}{Indivíduo natural ou habitante desse país.}{ca.ma.ro.nen.se}{0}
\verb{camaronês}{}{}{}{}{adj. e s.m.  }{Camaronense.}{ca.ma.ro.nês}{0}
\verb{camarote}{ó}{}{}{}{s.m.}{Compartimento fechado, em uma sala de espetáculos, de onde os espectadores podem assistir às apresentações.}{ca.ma.ro.te}{0}
\verb{camarote}{ó}{}{}{}{}{Cabina onde se alojam os passageiros de navio.}{ca.ma.ro.te}{0}
\verb{camaroteiro}{ê}{}{}{}{s.m.}{Vendedor de bilhetes de entrada para camarotes de salas de espetáculo e também para outros lugares da plateia.}{ca.ma.ro.tei.ro}{0}
\verb{camaroteiro}{ê}{}{}{}{}{Criado encarregado de arrumar os camarotes dos navios.}{ca.ma.ro.tei.ro}{0}
\verb{camartelo}{é}{}{}{}{s.m.}{Espécie de martelo usado por pedreiros para desbastar, cortar e picar pedras.}{ca.mar.te.lo}{0}
\verb{cambada}{}{}{}{}{s.f.}{Agrupamento de pessoas; bando, multidão.}{cam.ba.da}{0}
\verb{cambada}{}{}{}{}{}{Porção de coisas enfiadas ou amarradas em algum suporte.}{cam.ba.da}{0}
\verb{cambado}{}{}{}{}{adj.}{Inclinado para o lado; curvado, torto.}{cam.ba.do}{0}
\verb{cambado}{}{}{}{}{}{Que manca; coxo, cambaio.}{cam.ba.do}{0}
\verb{cambaio}{}{}{}{}{adj.}{Diz"-se do indivíduo que tem as pernas fracas ou tortas.}{cam.bai.o}{0}
\verb{cambaio}{}{}{}{}{}{Aquele que manca; trôpego, coxo, cambado, zambo.}{cam.bai.o}{0}
\verb{cambalacho}{}{}{}{}{s.m.}{Plano com intenção de enganar alguém; tramoia, conluio.}{cam.ba.la.cho}{0}
\verb{cambalacho}{}{}{}{}{}{Transação fraudulenta; negociata, trapaça.}{cam.ba.la.cho}{0}
\verb{cambaleante}{}{}{}{}{adj.2g.}{Que cambaleia; vacilante, inseguro.}{cam.ba.le.an.te}{0}
\verb{cambalear}{}{}{}{}{v.i.}{Não se firmar sobre os pés; não ter apoio;  vacilar.}{cam.ba.le.ar}{0}
\verb{cambalear}{}{}{}{}{}{Mostrar"-se inseguro; hesitar.}{cam.ba.le.ar}{\verboinum{4}}
\verb{cambaleio}{ẽ}{}{}{}{s.m.}{Ato de cambalear.}{cam.ba.lei.o}{0}
\verb{cambalhota}{ó}{}{}{}{s.f.}{Salto para a frente ou para trás no qual se gira o corpo e se cai sobre os pés; cambota, pirueta.}{cam.ba.lho.ta}{0}
%\verb{}{}{}{}{}{}{}{}{0}
\verb{cambapé}{}{}{}{}{s.m.}{Rasteira.}{cam.ba.pé}{0}
\verb{cambapé}{}{Fig.}{}{}{}{Armadilha, cilada.}{cam.ba.pé}{0}
\verb{cambar}{}{}{}{}{v.i.}{Entortar as pernas ao andar.}{cam.bar}{0}
\verb{cambar}{}{}{}{}{}{Andar cambaio.}{cam.bar}{\verboinum{1}}
\verb{cambará}{}{Bot.}{}{}{s.m.}{Camará.}{cam.ba.rá}{0}
\verb{cambaxirra}{ch}{Zool.}{}{}{s.f.}{Pássaro pequeno, canoro, de coloração parda, com listras pretas nas asas, no dorso e na cauda, que se alimenta de insetos e larvas; garrincha, corruíra.}{cam.ba.xir.ra}{0}
\verb{cambeta}{ê}{Bras.}{}{}{adj.}{Cambaio, coxo.}{cam.be.ta}{0}
\verb{cambial}{}{}{"-ais}{}{adj.2g.}{Relativo ou pertencente a câmbio.}{cam.bi.al}{0}
\verb{cambiante}{}{}{}{}{adj.2g.}{Que cambia.}{cam.bi.an.te}{0}
\verb{cambiante}{}{}{}{}{}{De cor indistinta, imprecisa.}{cam.bi.an.te}{0}
\verb{cambiar}{}{}{}{}{v.t.}{Fazer uma operação cambial, trocar dinheiro de um país pelo de outro.}{cam.bi.ar}{0}
\verb{cambiar}{}{}{}{}{}{Trocar uma coisa por outra; permutar, mudar, alterar.}{cam.bi.ar}{\verboinum{1}}
\verb{câmbio}{}{}{}{}{s.m.}{Ato de cambiar.}{câm.bio}{0}
\verb{câmbio}{}{}{}{}{}{Barra de metal que serve para trocar as marchas de um carro; alavanca de marcha.}{câm.bio}{0}
\verb{cambista}{}{Bras.}{}{}{s.2g.}{Indivíduo que revende ingressos de espetáculos a preço mais elevado do que o oficial.}{cam.bis.ta}{0}
\verb{cambista}{}{}{}{}{}{Indivíduo que faz negócios ou tem estabelecimento de câmbio.}{cam.bis.ta}{0}
\verb{cambito}{}{}{}{}{s.m.}{Pernil de porco.}{cam.bi.to}{0}
\verb{cambito}{}{Pop.}{}{}{}{Perna, de homem ou mulher, muito fina; gambito.}{cam.bi.to}{0}
\verb{cambojano}{}{}{}{}{adj.}{Relativo ao Camboja.}{cam.bo.ja.no}{0}
\verb{cambojano}{}{}{}{}{s.m.}{Indivíduo natural ou habitante desse país.}{cam.bo.ja.no}{0}
\verb{cambota}{ó}{}{}{}{s.f.}{Cambalhota.}{cam.bo.ta}{0}
\verb{cambraia}{}{}{}{}{s.f.}{Tecido fino, quase transparente, de linho ou algodão.}{cam.brai.a}{0}
\verb{cambriano}{}{}{}{}{adj.}{Diz"-se do tempo geológico da Terra compreendido entre 570 e 510 milhões de anos atrás, com seus terrenos e conjuntos de fósseis.  }{cam.bri.a.no}{0}
\verb{cambriano}{}{}{}{}{s.m.}{Esse tempo geológico.  }{cam.bri.a.no}{0}
\verb{cambucá}{}{}{}{}{s.m.}{Fruto comestível do cambucazeiro, com bagas amarelas, de polpa avermelhada.}{cam.bu.cá}{0}
\verb{cambucazeiro}{ê}{Bot.}{}{}{s.m.}{Árvore cujo fruto é o cambucá.}{cam.bu.ca.zei.ro}{0}
\verb{cambuci}{}{Bot.}{}{}{s.m.}{Árvore de pequeno porte, nativa do Brasil, de flores brancas e frutos comestíveis, com bagas esféricas, muito aromáticas.}{cam.bu.ci}{0}
\verb{cambuci}{}{}{}{}{}{O fruto dessa árvore.}{cam.bu.ci}{0}
\verb{cambulhada}{}{}{}{}{s.f.}{Conjunto de coisas diversas; cambada, molho, enfiada.}{cam.bu.lha.da}{0}
\verb{cambuquira}{}{}{}{}{s.f.}{O grelo da aboboreira.}{cam.bu.qui.ra}{0}
\verb{cambuquira}{}{Cul.}{}{}{}{Guisado preparado com esses grelos.}{cam.bu.qui.ra}{0}
\verb{camburão}{}{Bras.}{"-ões}{}{s.m.}{Carro de polícia usado para transportar os presos.}{cam.bu.rão}{0}
\verb{cameleiro}{ê}{}{}{}{s.m.}{Pessoa que conduz camelos.}{ca.me.lei.ro}{0}
\verb{camélia}{}{}{}{}{s.f.}{Arbusto originário do Japão, de flores de tamanho e cores diversas, muito ornamental.}{ca.mé.lia}{0}
\verb{camélia}{}{}{}{}{}{A flor desse arbusto.}{ca.mé.lia}{0}
\verb{camelo}{ê}{Zool.}{}{}{s.m.}{Mamífero ruminante, de pescoço comprido e encurvado, com duas corcovas nas costas, capaz de resistir a longos períodos sem água, usado no transporte de cargas e como montaria no deserto.}{ca.me.lo}{0}
\verb{camelô}{}{Bras.}{}{}{s.2g.}{Vendedor que expõe suas mercadorias nas ruas e calçadas.}{ca.me.lô}{0}
\verb{câmera}{}{}{}{}{s.f.}{Espaço fechado; câmara.}{câ.me.ra}{0}
\verb{câmera}{}{}{}{}{}{Máquina fotográfica.}{câ.me.ra}{0}
\verb{câmera}{}{}{}{}{}{Aparelho que filma e transmite imagens.}{câ.me.ra}{0}
\verb{câmera}{}{}{}{}{s.2g.}{Pessoa que opera esse aparelho de filmagem.}{câ.me.ra}{0}
\verb{camerlengo}{}{}{}{}{s.m.}{Cardeal que toma conta da Igreja no período compreendido entre a morte de um papa e a eleição do seguinte.}{ca.mer.len.go}{0}
\verb{camicaze}{}{}{}{}{adj.2g. e s.2g.}{Forma aportuguesada de \textit{kamikaze}.}{ca.mi.ca.ze}{0}
\verb{caminhada}{}{}{}{}{s.f.}{Ação de caminhar; passeio.}{ca.mi.nha.da}{0}
\verb{caminhada}{}{}{}{}{}{Grande distância para andar; jornada.}{ca.mi.nha.da}{0}
\verb{caminhante}{}{}{}{}{adj.2g.}{Que caminha; caminhador.}{ca.mi.nhan.te}{0}
\verb{caminhante}{}{}{}{}{s.2g.}{Pessoa que caminha; caminhador.}{ca.mi.nhan.te}{0}
\verb{caminhão}{}{}{"-ões}{}{s.m.}{Veículo grande, motorizado, próprio para o transporte de cargas pesadas.}{ca.mi.nhão}{0}
\verb{caminhar}{}{}{}{}{v.i.}{Andar, percorrer um caminho a pé.}{ca.mi.nhar}{0}
\verb{caminhar}{}{Fig.}{}{}{}{Ir para a frente; avançar, seguir, progredir.}{ca.mi.nhar}{\verboinum{1}}
\verb{caminheiro}{ê}{}{}{}{adj.}{Caminhador.}{ca.mi.nhei.ro}{0}
\verb{caminho}{}{}{}{}{s.m.}{Faixa de terreno que leva de um lugar a outro; via, estrada.}{ca.mi.nho}{0}
\verb{caminho}{}{}{}{}{}{Espaço que se percorre andando de um ponto a outro.}{ca.mi.nho}{0}
\verb{caminho}{}{}{}{}{}{Direção, caminho, rumo, destino.}{ca.mi.nho}{0}
\verb{caminhoneiro}{ê}{Bras.}{}{}{s.m.}{Motorista de caminhão.}{ca.mi.nho.nei.ro}{0}
\verb{caminhonete}{é}{}{}{}{s.f.}{Veículo motorizado, menor que um caminhão, que serve para transportar pequenas cargas ou passageiros.}{ca.mi.nho.ne.te}{0}
\verb{camioneta}{ê}{}{}{}{s.f.}{Caminhonete.}{ca.mi.o.ne.ta}{0}
\verb{camionete}{é}{}{}{}{s.f.}{Caminhonete.}{ca.mi.o.ne.te}{0}
\verb{camisa}{}{}{}{}{s.f.}{Roupa com colarinho e mangas, que se veste sobre a pele ou sobre uma camiseta, e que vai do pescoço até as coxas.}{ca.mi.sa}{0}
\verb{camisa"-de"-força}{ô}{}{camisas"-de"-força ⟨ô⟩}{}{s.f.}{Tipo de camisa, de tecido muito resistente, apertada, que serve para imobilizar os pacientes de hospitais psiquiátricos, impedindo"-os de machucarem a si mesmos ou outras pessoas.}{ca.mi.sa"-de"-for.ça}{0}
\verb{camisa"-de"-meia}{ê}{}{camisas"-de"-meia}{}{s.f.}{Camiseta de malha.}{ca.mi.sa"-de"-mei.a}{0}
\verb{camisa"-de"-vênus}{}{}{camisas"-de"-vênus}{}{s.f.}{Camisinha; preservativo.}{ca.mi.sa"-de"-vê.nus}{0}
\verb{camisaria}{}{}{}{}{s.f.}{Estabelecimento onde se fazem ou se vendem camisas.}{ca.mi.sa.ri.a}{0}
\verb{camiseira}{ê}{}{}{}{s.f.}{Mulher que faz ou vende camisas.}{ca.mi.sei.ra}{0}
\verb{camiseira}{ê}{Bras.}{}{}{}{Móvel onde se guardam camisas; camiseiro.}{ca.mi.sei.ra}{0}
\verb{camiseiro}{ê}{}{}{}{s.m.}{Pessoa que fabrica ou vende camisas.}{ca.mi.sei.ro}{0}
\verb{camiseiro}{ê}{}{}{}{}{Armário para guardar camisas.}{ca.mi.sei.ro}{0}
\verb{camiseta}{ê}{}{}{}{s.f.}{Camisa sem gola nem botão, usada sobre a pele.}{ca.mi.se.ta}{0}
\verb{camisinha}{}{}{}{}{s.f.}{Camisa pequena.}{ca.mi.si.nha}{0}
\verb{camisinha}{}{Pop.}{}{}{}{Espécie de capa anatômica, fina, de borracha resistente, que serve para recobrir o pênis durante uma relação sexual, seja para evitar a concepção, seja para evitar a transmissão de doenças; preservativo, camisa"-de"-vênus.}{ca.mi.si.nha}{0}
\verb{camisola}{ó}{}{}{}{s.f.}{Camisa comprida, feminina, para dormir.}{ca.mi.so.la}{0}
\verb{camomila}{}{Bot.}{}{}{s.f.}{Erva, nativa da Europa, de flores brancas e amarelas, de largo uso medicinal, com as quais se faz chá de propriedades calmantes e digestivas.}{ca.mo.mi.la}{0}
\verb{camoniano}{}{}{}{}{adj.}{Relativo a Luís de Camões, poeta português.}{ca.mo.ni.a.no}{0}
\verb{camoniano}{}{}{}{}{}{Que segue ou imita o estilo desse poeta.}{ca.mo.ni.a.no}{0}
\verb{camonista}{}{}{}{}{s.2g.}{Pessoa que conhece a fundo ou admira a obra de Luís de Camões.}{ca.mo.nis.ta}{0}
\verb{camorra}{ô}{}{}{}{s.f.}{Antigo grupo de malfeitores de Nápoles.}{ca.mor.ra}{0}
\verb{camorra}{ô}{Por ext.}{}{}{}{Qualquer grupo de malfeitores, criminosos.}{ca.mor.ra}{0}
\verb{campa}{}{}{}{}{s.f.}{Pedra que se coloca sobre a sepultura para tampá"-la.}{cam.pa}{0}
\verb{campa}{}{Por ext.}{}{}{}{Sepultura, túmulo.}{cam.pa}{0}
\verb{campainha}{}{}{}{}{s.f.}{Pequena sineta de mão.}{cam.pa.i.nha}{0}
\verb{campainha}{}{}{}{}{}{Aparelho elétrico ou mecânico, instalado em portas, telefones etc., que emite um som de chamada.}{cam.pa.i.nha}{0}
\verb{campal}{}{}{"-ais}{}{adj.2g.}{Relativo ou pertencente a campo.}{cam.pal}{0}
\verb{campal}{}{}{"-ais}{}{}{Que se realiza ao ar livre, num espaço aberto.}{cam.pal}{0}
\verb{campana}{}{}{}{}{s.f.}{Campainha, sino.}{cam.pa.na}{0}
\verb{campanário}{}{}{}{}{s.m.}{Parte aberta da torre da igreja onde estão os sinos.}{cam.pa.ná.rio}{0}
\verb{campanário}{}{}{}{}{}{Torre de sinos.}{cam.pa.ná.rio}{0}
\verb{campanário}{}{Por ext.}{}{}{}{A aldeia, vila ou lugar onde se ergue a torre dos sinos.}{cam.pa.ná.rio}{0}
\verb{campanha}{}{}{}{}{s.f.}{Conjunto de ações, de esforços, para se atingir um fim determinado.}{cam.pa.nha}{0}
\verb{campanha}{}{}{}{}{}{Conjunto de operações militares contra as forças inimigas.}{cam.pa.nha}{0}
\verb{campanha}{}{}{}{}{}{Campo de grande extensão; planície.}{cam.pa.nha}{0}
\verb{campanudo}{}{}{}{}{adj.}{Que tem forma ou aspecto de campânula ou sino.}{cam.pa.nu.do}{0}
\verb{campanudo}{}{Fig.}{}{}{}{Que se mostra pomposo, bombástico.}{cam.pa.nu.do}{0}
\verb{campânula}{}{}{}{}{s.f.}{Qualquer objeto em forma de sino.}{cam.pâ.nu.la}{0}
\verb{campânula}{}{}{}{}{}{Variedade de redoma usada para proteger certos objetos ou alimentos da ação da poeira, de insetos etc.}{cam.pâ.nu.la}{0}
\verb{campânula}{}{Bot.}{}{}{}{Planta ornamental, com flores vistosas, em forma de campainha.}{cam.pâ.nu.la}{0}
\verb{campeão}{}{}{"-ões}{"-ã}{s.m.}{Vencedor de prova ou torneio.}{cam.pe.ão}{0}
\verb{campear}{}{}{}{}{}{Procurar alguma coisa. (\textit{Estou campeando meus óculos há dois dias.})}{cam.pe.ar}{0}
\verb{campear}{}{}{}{}{v.t.}{Andar à cavalo pelo campo à procura de animal. (\textit{O fazendeiro campeia o bezerro.})}{cam.pe.ar}{0}
\verb{campear}{}{}{}{}{}{Ser comum em algum lugar; dominar, prevalecer. (\textit{A impunidade campeia no país.})}{cam.pe.ar}{\verboinum{4}}
\verb{campeche}{é}{Bot.}{}{}{s.m.}{Árvore cujo tronco espinhoso fornece um cerne vermelho escuro de que se extrai um corante, que é usado em tinturaria e nos laboratórios para corar preparações histológicas.}{cam.pe.che}{0}
\verb{campeiro}{ê}{}{}{}{adj.}{Relativo a campo.}{cam.pei.ro}{0}
\verb{campeiro}{ê}{}{}{}{s.m.}{Trabalhador do campo.}{cam.pei.ro}{0}
\verb{campeiro}{ê}{}{}{}{}{Empregado que trata do gado, e que vive habitualmente nos campos gerais.}{cam.pei.ro}{0}
\verb{campeonato}{}{Esport.}{}{}{s.m.}{Torneio ou disputa em que se concede o título de campeão ao vencedor.}{cam.pe.o.na.to}{0}
\verb{campesinho}{}{}{}{}{adj.}{Campestre.}{cam.pe.si.nho}{0}
\verb{campesino}{}{}{}{}{adj.}{Campestre.}{cam.pe.si.no}{0}
\verb{campestre}{é}{}{}{}{adj.2g.}{Relativo ao campo; rural, rústico, campesino, campesinho, camponês.}{cam.pes.tre}{0}
\verb{campestre}{é}{Bot.}{}{}{}{Diz"-se de planta que habita lugares abertos.}{cam.pes.tre}{0}
\verb{campestre}{é}{Bras.}{}{}{s.m.}{Pequeno campo alto, de área diminuta no meio da mata.}{cam.pes.tre}{0}
\verb{campina}{}{}{}{}{s.f.}{Campo extenso, pouco acidentado e sem árvores, geralmente coberto de ervas; prado.}{cam.pi.na}{0}
\verb{campina}{}{}{}{}{}{Planície.}{cam.pi.na}{0}
\verb{camping}{}{}{}{}{s.m.}{Atividade coletiva, turística ou esportiva, que consiste em viajar e acampar ao ar livre, em lugar apropriado, fazendo uso de barracas e de outros equipamentos necessários.}{\textit{camping}}{0}
\verb{campismo}{}{}{}{}{s.m.}{\textit{Camping}.}{cam.pis.mo}{0}
\verb{campo}{}{}{}{}{s.m.}{Região que fica fora de uma cidade. (\textit{Compramos uma casa no campo.})}{cam.po}{0}
\verb{campo}{}{}{}{}{}{Terreno em que se plantou; plantação. (\textit{Os campos de trigo estão em flor.})}{cam.po}{0}
\verb{campo}{}{}{}{}{}{Região usada para plantar ou criar gado.}{cam.po}{0}
\verb{campo}{}{}{}{}{}{Terreno próprio para praticar esportes. (\textit{Os meninos estão jogando bola no campo.})}{cam.po}{0}
\verb{campo}{}{}{}{}{}{Área de conhecimento ou atividade; assunto, matéria, ramo. (\textit{A psicologia faz parte do seu campo de atuação.})}{cam.po}{0}
\verb{campo"-grandense}{}{}{campo"-grandenses}{}{adj.2g.}{Relativo a Campo Grande, capital do Mato Grosso do Sul.}{cam.po"-gran.den.se}{0}
\verb{campo"-grandense}{}{}{campo"-grandenses}{}{s.2g.}{Indivíduo natural ou habitante dessa cidade.}{cam.po"-gran.den.se}{0}
\verb{camponês}{}{}{}{}{adj.}{Relativo ao campo, campestre.}{cam.po.nês}{0}
\verb{camponês}{}{}{}{}{s.m.}{Indivíduo que vive ou trabalha no campo.}{cam.po.nês}{0}
\verb{camponês}{}{Fig.}{}{}{}{Indivíduo rude, rústico.}{cam.po.nês}{0}
\verb{campônio}{}{}{}{}{s.m.}{Indivíduo que habita ou trabalha no campo.}{cam.pô.nio}{0}
\verb{campo"-santo}{}{}{campos"-santos}{}{s.m.}{Cemitério.}{cam.po"-san.to}{0}
\verb{campus}{}{}{}{}{s.m.}{Conjunto de edifícios e terrenos de uma universidade.}{\textit{campus}}{0}
\verb{camuflagem}{}{}{"-ens}{}{s.f.}{Ato ou efeito de camuflar.}{ca.mu.fla.gem}{0}
\verb{camuflagem}{}{}{"-ens}{}{}{Aquilo que serve para camuflar ou disfarçar.}{ca.mu.fla.gem}{0}
\verb{camuflar}{}{}{}{}{v.t.}{Esconder ou disfarçar por meio de camuflagem, especialmente na guerra.}{ca.mu.flar}{0}
\verb{camuflar}{}{}{}{}{}{Disfarçar ou dissimular sob falsas aparências.}{ca.mu.flar}{\verboinum{1}}
\verb{camundongo}{}{Zool.}{}{}{s.m.}{Pequeno mamífero roedor, comum em todas as regiões de habitação humana, de pelagem macia, cinza amarelado, mais clara nas partes inferiores, orelhas grandes e arredondadas, e cauda nua e longa.}{ca.mun.don.go}{0}
\verb{camurça}{}{Zool.}{}{}{s.f.}{Mamífero ruminante, semelhante às cabras, encontrado nos Alpes ou em montanhas da Europa e da Ásia, de chifres negros unidos nas bases e com as pontas curvadas para trás, e pelagem amarronzada.}{ca.mur.ça}{0}
\verb{camurça}{}{Por ext.}{}{}{}{A pele curtida desse animal, usada para fabricar calçados, luvas etc.}{ca.mur.ça}{0}
\verb{camurça}{}{}{}{}{}{Tecido que imita essas peles por ter uma espécie de penugem numa das faces.}{ca.mur.ça}{0}
\verb{cana}{}{}{}{}{s.f.}{Cana"-de"-açúcar.}{ca.na}{0}
\verb{cana}{}{}{}{}{}{Caule muito comprido, cheio de nós, como o bambu e a taquara.}{ca.na}{0}
\verb{cana"-de"-açúcar}{}{Bot.}{canas"-de"-açúcar}{}{s.f.}{Planta herbácea nativa da Ásia e muito cultivada no Brasil, cujo colmo é espesso e repleto de suco açucarado, fornece forragem, açúcar e álcool combustível.}{ca.na"-de"-a.çú.car}{0}
\verb{canadense}{}{}{}{}{adj.2g.}{Relativo ao Canadá.}{ca.na.den.se}{0}
\verb{canadense}{}{}{}{}{s.2g.}{Indivíduo natural ou habitante desse país.}{ca.na.den.se}{0}
\verb{canal}{}{}{"-ais}{}{s.m.}{Escavação feita pelo homem, por onde corre a água.}{ca.nal}{0}
\verb{canal}{}{}{"-ais}{}{}{Curso de água cavado pelo homem, ligando rios, lagos ou mares. (\textit{O Canal do Panamá liga o Oceano Atlântico ao Pacífico.})}{ca.nal}{0}
\verb{canal}{}{}{"-ais}{}{}{Tubo dentro do organismo; conduto. (\textit{Tratamentos de canal dentário são geralmente muito demorados e trabalhosos.})}{ca.nal}{0}
\verb{canal}{}{}{"-ais}{}{}{Faixa de frequência de uma estação de rádio ou de televisão. (\textit{Há uma série de bons programas nos canais de \textsc{tv}) a cabo.}}{ca.nal}{0}
\verb{canal}{}{}{"-ais}{}{}{Braço de mar ou rio; estreito. (\textit{A travessia do Canal da Mancha é um grande desafio para os nadadores.})}{ca.nal}{0}
\verb{canaleta}{ê}{}{}{}{s.f.}{Pequeno canal ou rego por onde escorre água.}{ca.na.le.ta}{0}
\verb{canalete}{ê}{}{}{}{s.m.}{Canaleta.}{ca.na.le.te}{0}
\verb{canalha}{}{}{}{}{adj.2g.}{Relativo a pessoa vil, reles.}{ca.na.lha}{0}
\verb{canalha}{}{}{}{}{}{Diz"-se daquele que é infame, reles.}{ca.na.lha}{0}
\verb{canalha}{}{}{}{}{s.2g.}{Indivíduo vil, velhaco.}{ca.na.lha}{0}
\verb{canalha}{}{}{}{}{s.f.}{Conjunto de pessoas infames, desprezíveis.}{ca.na.lha}{0}
\verb{canalhada}{}{}{}{}{s.f.}{Ato, procedimento de canalha.}{ca.na.lha.da}{0}
\verb{canalhada}{}{}{}{}{}{Grupo de canalhas.}{ca.na.lha.da}{0}
\verb{canalhice}{}{}{}{}{s.f.}{Ato próprio de canalha; canalhada.}{ca.na.lhi.ce}{0}
\verb{canalículo}{}{}{}{}{s.m.}{Canal de pequeno tamanho.}{ca.na.lí.cu.lo}{0}
\verb{canalização}{}{}{"-ões}{}{s.f.}{Ato ou efeito de canalizar.}{ca.na.li.za.ção}{0}
\verb{canalização}{}{}{"-ões}{}{}{Certa quantidade de canos ou canais.}{ca.na.li.za.ção}{0}
\verb{canalizar}{}{}{}{}{v.t.}{Fazer avançar e escorrer por meio de canos, canais, valas etc.}{ca.na.li.zar}{0}
\verb{canalizar}{}{}{}{}{}{Colocar redes de água e de esgoto.}{ca.na.li.zar}{0}
\verb{canalizar}{}{}{}{}{}{Construir canais.}{ca.na.li.zar}{0}
\verb{canalizar}{}{Fig.}{}{}{}{Conduzir, encaminhar.}{ca.na.li.zar}{\verboinum{1}}
\verb{cananeu}{}{}{}{}{adj.}{Relativo à terra de Canaã (Palestina).}{ca.na.neu}{0}
\verb{cananeu}{}{}{}{}{s.m.}{Indivíduo natural ou habitante de Canaã.}{ca.na.neu}{0}
\verb{cananeu}{}{}{}{}{}{Grupo de línguas semíticas ao qual pertencem o fenício e o hebraico.}{ca.na.neu}{0}
\verb{cananeu}{}{}{}{}{}{A língua falada na terra de Canaã antes de ser conquistada pelos israelitas.}{ca.na.neu}{0}
\verb{canapé}{}{}{}{}{s.m.}{Espécie de sofá com costas e braços.}{ca.na.pé}{0}
\verb{canapé}{}{}{}{}{}{Pequena fatia de pão sobre a qual se colocam iguarias variadas, como presunto, queijo, ovo etc., geralmente servido como aperitivo.}{ca.na.pé}{0}
\verb{canarês}{}{}{}{}{adj. e s.m.  }{Canarino.}{ca.na.rês}{0}
\verb{canarino}{}{}{}{}{adj.}{Relativo ao arquipélago das Canárias ( costa atlântica da África).}{ca.na.ri.no}{0}
\verb{canarino}{}{}{}{}{s.m.}{Indivíduo natural ou habitante desse arquipélago; canarês.}{ca.na.ri.no}{0}
\verb{canário}{}{Zool.}{}{}{s.m.}{Ave de plumagem predominantemente amarela, bico curto e grosso, originária das ilhas Canárias.}{ca.ná.rio}{0}
\verb{canário}{}{Fig.}{}{}{}{Indivíduo que canta bem.}{ca.ná.rio}{0}
\verb{canário}{}{}{}{}{adj. e s.m.  }{Canarino.}{ca.ná.rio}{0}
\verb{canastra}{}{}{}{}{s.f.}{Variedade de cesta quadrangular, entrelaçada com ripas flexíveis de madeira, larga e baixa, com tampa ou não.}{ca.nas.tra}{0}
\verb{canastra}{}{}{}{}{}{Parte posterior do corpo humano, acima da cintura e abaixo dos ombros; costas, corcunda.}{ca.nas.tra}{0}
\verb{canastra}{}{}{}{}{}{Jogo de cartas, jogado por quatro pessoas com duas parcerias, que tem como principal objetivo a formação de uma combinação de sete cartas também chamada canastra.}{ca.nas.tra}{0}
\verb{canastrão}{}{}{"-ões}{}{s.m.}{Canastra grande.}{ca.nas.trão}{0}
\verb{canastrão}{}{}{"-ões}{}{}{Ator medíocre; mau ator.}{ca.nas.trão}{0}
\verb{canavial}{}{}{"-ais}{}{s.m.}{Plantação de cana"-de"-açúcar.}{ca.na.vi.al}{0}
\verb{canavieiro}{ê}{}{}{}{adj.}{Relativo a cana"-de"-açúcar.}{ca.na.vi.ei.ro}{0}
\verb{canavieiro}{ê}{}{}{}{s.m.}{Indivíduo que planta cana"-de"-açúcar.}{ca.na.vi.ei.ro}{0}
\verb{cancã}{}{}{}{}{s.m.}{Dança típica dos cabarés de Paris, ruidosa e de movimentos muito rápidos, na qual as dançarinas lançam as pernas para o alto, enquanto erguem e sacodem as saias com as mãos.}{can.cã}{0}
\verb{canção}{}{}{"-ões}{}{s.f.}{Composição musical para ser cantada; cantiga popular, modinha.}{can.ção}{0}
\verb{cancela}{é}{}{}{}{s.f.}{Porta gradeada, em geral de madeira e de pequena altura; porteira.}{can.ce.la}{0}
\verb{cancela}{é}{}{}{}{}{Grande porta, geralmente metálica e de pouca altura, usada em passagens de nível, para abrir ou fechar o trânsito.}{can.ce.la}{0}
\verb{cancelamento}{}{}{}{}{s.m.}{Ato ou efeito de cancelar, de tornar sem valor, de eliminar.}{can.ce.la.men.to}{0}
\verb{cancelar}{}{}{}{}{v.t.}{Eliminar ou riscar o que está escrito para tornar sem efeito.}{can.ce.lar}{0}
\verb{cancelar}{}{}{}{}{}{Declarar ou dar como nulo, ou sem valor.}{can.ce.lar}{0}
\verb{cancelar}{}{}{}{}{}{Desistir, suspender, suprimir.}{can.ce.lar}{\verboinum{1}}
\verb{câncer}{}{Astron.}{}{}{s.m.}{Quarta constelação zodiacal.}{cân.cer}{0}
\verb{câncer}{}{Astrol.}{}{}{}{O signo do zodíaco referente a essa constelação.}{cân.cer}{0}
\verb{câncer}{}{Med.}{}{}{s.m.}{Doença grave que espalha tumores pelo corpo e pode provocar a morte.}{cân.cer}{0}
\verb{canceriano}{}{Astrol.}{}{}{s.m.}{Indivíduo que nasceu sob o signo de câncer.}{can.ce.ri.a.no}{0}
\verb{canceriano}{}{Astrol.}{}{}{adj.}{Relativo ou pertencente a esse signo.}{can.ce.ri.a.no}{0}
\verb{cancerígeno}{}{}{}{}{adj.}{Diz"-se de substância que gera câncer.}{can.ce.rí.ge.no}{0}
\verb{cancerologia}{}{Med.}{}{}{s.f.}{Parte da medicina que estuda o câncer.}{can.ce.ro.lo.gi.a}{0}
\verb{canceroso}{ô}{}{"-osos ⟨ó⟩}{"-osa ⟨ó⟩}{adj.}{Que é da natureza do câncer.}{can.ce.ro.so}{0}
\verb{canceroso}{ô}{}{"-osos ⟨ó⟩}{"-osa ⟨ó⟩}{s.m.}{Indivíduo que tem câncer.}{can.ce.ro.so}{0}
\verb{cancha}{}{Bras.}{}{}{s.f.}{Pista preparada para corridas de cavalo; raia.}{can.cha}{0}
\verb{cancha}{}{Bras.}{}{}{}{Terreno preparado para diversas modalidades de jogos, como futebol, tênis, basquete etc.}{can.cha}{0}
\verb{cancha}{}{Pop.}{}{}{}{Larga experiência; conhecimento.}{can.cha}{0}
\verb{cancioneiro}{ê}{Mús.}{}{}{s.m.}{Coleção de canções.}{can.ci.o.nei.ro}{0}
\verb{cancioneiro}{ê}{Liter.}{}{}{}{Coleção de canções, poemas ou cantigas portuguesas ou espanholas.}{can.ci.o.nei.ro}{0}
\verb{cancioneiro}{ê}{Liter.}{}{}{}{Coleção de poemas ou canções da tradição popular.}{can.ci.o.nei.ro}{0}
\verb{cançoneta}{ê}{}{}{}{s.f.}{Pequena canção popular, sobre tema leve, espirituoso ou satírico.}{can.ço.ne.ta}{0}
\verb{cançonetista}{}{}{}{}{adj.2g.}{Relativo a cançoneta.}{can.ço.ne.tis.ta}{0}
\verb{cançonetista}{}{Mús.}{}{}{s.2g.}{Autor ou cantor de cançonetas.}{can.ço.ne.tis.ta}{0}
\verb{cancro}{}{Med.}{}{}{s.m.}{Ulceração isolada da pele ou mucosas que constitui o estágio inicial de várias doenças infecciosas, em geral sexualmente transmissíveis.}{can.cro}{0}
\verb{cancro}{}{Med.}{}{}{}{Câncer.}{can.cro}{0}
\verb{cancro}{}{Por ext.}{}{}{}{Doença ou qualquer mal que gradativamente enfraqueça ou destrua um organismo.}{can.cro}{0}
\verb{candango}{}{Bras.}{}{}{s.m.}{Designação que os africanos davam aos portugueses.}{can.dan.go}{0}
\verb{candango}{}{Bras.}{}{}{}{Designação dada aos trabalhadores pioneiros vindos do Nordeste.}{can.dan.go}{0}
\verb{cande}{}{}{}{}{s.m.}{Açúcar que resulta da cristalização da sacarose e que apresenta grandes cristais em forma de prismas.}{can.de}{0}
\verb{candeeiro}{ê}{}{}{}{s.m.}{Aparelho de iluminação, de formatos variados, alimentado por querosene, óleo ou gás.}{can.de.ei.ro}{0}
\verb{candeia}{ê}{}{}{}{s.f.}{Pequeno aparelho de iluminação, de folha"-de"-flandres ou de barro, abastecido com óleo ou gás inflamável; usa"-se geralmente no alto, pendente de um prego preso à parede.}{can.dei.a}{0}
\verb{candela}{é}{}{}{}{s.f.}{Candeia, vela, lamparina.}{can.de.la}{0}
\verb{candela}{é}{Quím.}{}{}{}{Unidade de medida de intensidade luminosa. Símb.: cd.}{can.de.la}{0}
\verb{candelabro}{}{}{}{}{s.m.}{Grande castiçal com ramificações, a cada uma das quais corresponde um foco de luz; lustre, lampadário.}{can.de.la.bro}{0}
\verb{candelária}{}{Relig.}{}{}{s.f.}{Festa religiosa da Purificação da Virgem Maria, em 2 de fevereiro, durante a qual os participantes carregam círios ou candeias.}{can.de.lá.ria}{0}
\verb{candente}{}{}{}{}{adj.2g.}{Que está ardendo em brasa.}{can.den.te}{0}
\verb{cândida}{}{}{}{}{s.f.}{Água sanitária.}{cân.di.da}{0}
\verb{cândida}{}{}{}{}{}{Aguardente de cana.}{cân.di.da}{0}
\verb{cândida}{}{}{}{}{}{Fungo causador da micose.}{cân.di.da}{0}
\verb{candidatar}{}{}{}{}{v.t.}{Apresentar ou indicar alguém como candidato.}{can.di.da.tar}{\verboinum{1}}
\verb{candidato}{}{}{}{}{s.m.}{Indivíduo que aspira a um cargo, honraria, emprego etc.}{can.di.da.to}{0}
\verb{candidato}{}{}{}{}{}{Indivíduo que precisa de votos que o elejam para um cargo ou função.}{can.di.da.to}{0}
\verb{candidatura}{}{}{}{}{s.f.}{Pretensão ou aspiração de candidato.}{can.di.da.tu.ra}{0}
\verb{candidatura}{}{}{}{}{}{Apresentação ou indicação de candidato para um processo de eleição por votos.}{can.di.da.tu.ra}{0}
\verb{candidez}{ê}{}{}{}{s.f.}{Candura.}{can.di.dez}{0}
\verb{cândido}{}{}{}{}{adj.}{Que é muito branco; de grande alvura.}{cân.di.do}{0}
\verb{cândido}{}{}{}{}{}{Que apresenta pureza, inocência.}{cân.di.do}{0}
\verb{candomblé}{}{}{}{}{s.m.}{Religião e culto afro"-brasileiros, em que são celebrados os orixás, por meio de cantos e danças, acompanhados de oferendas.}{can.dom.blé}{0}
\verb{candomblé}{}{Por ext.}{}{}{}{Local onde são celebrados os cultos.}{can.dom.blé}{0}
\verb{candonga}{}{}{}{}{s.f.}{Carinho fingido; adulação.}{can.don.ga}{0}
\verb{candonga}{}{}{}{}{}{Pessoa querida, namorada, amor.}{can.don.ga}{0}
\verb{candor}{ô}{}{}{}{s.m.}{Candura.}{can.dor}{0}
\verb{candura}{}{}{}{}{s.f.}{Qualidade de cândido; candidez, candura.}{can.du.ra}{0}
\verb{caneca}{é}{}{}{}{s.f.}{Recipiente pequeno para conter líquidos, de formato cilíndrico, dotado de asa para facilitar o manuseio.}{ca.ne.ca}{0}
\verb{caneco}{é}{}{}{}{s.m.}{Caneca estreita e alta.}{ca.ne.co}{0}
\verb{caneco}{é}{Pop.}{}{}{}{Taça que se disputa numa competição esportiva.}{ca.ne.co}{0}
\verb{canela}{é}{Bot.}{}{}{s.f.}{Árvore originária do Sri Lanka, cuja casca aromática se usa como especiaria.}{ca.ne.la}{0}
\verb{canela}{é}{Anat.}{}{}{s.f.}{A parte da perna entre o joelho e o pé.}{ca.ne.la}{0}
\verb{canela}{é}{Por ext.}{}{}{}{O pó obtido com a trituração da casca dessa árvore.}{ca.ne.la}{0}
\verb{canelada}{}{}{}{}{s.f.}{Pancada desferida ou sofrida na canela da perna.}{ca.ne.la.da}{0}
\verb{canelado}{}{}{}{}{adj.}{Que tem caneluras.}{ca.ne.la.do}{0}
\verb{canelado}{}{}{}{}{s.m.}{Conjunto de caneluras.}{ca.ne.la.do}{0}
\verb{caneleira}{ê}{Bot.}{}{}{s.f.}{Árvore que dá canela.}{ca.ne.lei.ra}{0}
\verb{caneleira}{ê}{Esport.}{}{}{}{Proteção acolchoada usada na parte frontal da perna para resguardar de pancadas. }{ca.ne.lei.ra}{0}
\verb{canelone}{}{Cul.}{}{}{s.m.}{Tipo de massa alimentícia que, depois de cozida, é recheada, enrolada em cilindro e levada ao forno para gratinar.}{ca.ne.lo.ne}{0}
\verb{caneludo}{}{}{}{}{adj.}{Diz"-se daquele que tem as canelas das pernas longas ou grossas.}{ca.ne.lu.do}{0}
\verb{canelura}{}{}{}{}{s.f.}{Cada uma das ranhuras verticais em colunas ou outras partes da construção civil.}{ca.ne.lu.ra}{0}
\verb{canelura}{}{}{}{}{}{Estrias, sulcos que se fazem nos vidros de portas ou janelas para torná"-los menos transparentes.}{ca.ne.lu.ra}{0}
\verb{caneta}{ê}{}{}{}{s.f.}{Pequeno tubo em que se encaixa a pena ou a ponta com que se escreve à tinta.}{ca.ne.ta}{0}
\verb{caneta}{ê}{}{}{}{}{Cabo com que os cirurgiões seguram o cautério.}{ca.ne.ta}{0}
\verb{caneta"-tinteiro}{ê}{}{canetas"-tinteiros \textit{ou} canetas"-tinteiro ⟨ê⟩}{}{s.f.}{Caneta provida de reservatório para tinta.}{ca.ne.ta"-tin.tei.ro}{0}
\verb{cânfora}{}{}{}{}{s.f.}{Substância cristalina, com odor característico, de largo emprego industrial e terapêutico, extraída da canforeira, e também obtida por via sintética.}{cân.fo.ra}{0}
\verb{canforado}{}{}{}{}{adj.}{Que tem ou que é preparado com cânfora.}{can.fo.ra.do}{0}
\verb{canforeira}{ê}{Bot.}{}{}{s.f.}{Árvore ornamental, originária da Ásia, de madeira resistente, da qual se extrai um óleo que produz a cânfora.}{can.fo.rei.ra}{0}
\verb{canga}{}{}{}{}{s.f.}{Peça de madeira que prende os bois pelo pescoço e os liga ao carro ou ao arado.}{can.ga}{0}
\verb{canga}{}{Fig.}{}{}{}{Opressão, domínio.}{can.ga}{0}
\verb{canga}{}{}{}{}{}{Retângulo ou triângulo de tecido que se enrola geralmente da cintura para baixo, por cima da roupa de banho.}{can.ga}{0}
\verb{cangaceiro}{ê}{}{}{}{s.m.}{Bandido do sertão nordestino, que anda fortemente armado.}{can.ga.cei.ro}{0}
\verb{cangaço}{}{}{}{}{s.m.}{Banditismo social que ocorreu no sertão nordestino.}{can.ga.ço}{0}
\verb{cangaço}{}{}{}{}{}{O conjunto das armas dos cangaceiros.}{can.ga.ço}{0}
\verb{cangaço}{}{}{}{}{}{O comportamento ou o modo de viver dos cangaceiros.}{can.ga.ço}{0}
\verb{cangalha}{}{}{}{}{s.f.}{Armação para sustentar e equilibrar a carga das bestas, distribuindo"-a em duas metades.}{can.ga.lha}{0}
\verb{cangalha}{}{}{}{}{}{Peça de três paus, unidos em triângulo, que se enfia no pescoço do porco para não destruir hortas cultivadas.}{can.ga.lha}{0}
\verb{cangambá}{}{Zool.}{}{}{s.m.}{Mamífero carnívoro, de coloração preta, com uma faixa branca dorsal, que excreta, pela glândula anal, um líquido fétido e nauseante, como defesa; jaritataca.}{can.gam.bá}{0}
\verb{cangapé}{}{}{}{}{s.m.}{Pontapé dado súbita e maldosamente na barriga da perna de alguém.}{can.ga.pé}{0}
\verb{cangote}{ó}{Pop.}{}{}{s.m.}{A parte posterior do pescoço; nuca.}{can.go.te}{0}
\verb{canguçu}{}{Zool.}{}{}{s.m.}{Onça"-pintada.}{can.gu.çu}{0}
\verb{cangulo}{}{Zool.}{}{}{s.m.}{Peixe marinho, do Atlântico tropical e do Mediterrâneo, cuja pele é provida de escamas com pequenos tubérculos espinhosos com coloração cinza esverdeado e manchas coloridas.}{can.gu.lo}{0}
\verb{canguru}{}{Zool.}{}{}{s.m.}{Nome de várias espécies de mamíferos marsupiais da Austrália, Nova Guiné etc., cujas pernas traseiras são fortemente desenvolvidas, o que lhes permite dar grandes saltos.  }{can.gu.ru}{0}
\verb{canhada}{}{}{}{}{s.f.}{Terreno plano e baixo entre duas colinas pequenas.}{ca.nha.da}{0}
\verb{canhada}{}{}{}{}{}{Vale profundo e estreito entre montanhas; depressão.}{ca.nha.da}{0}
\verb{cânhamo}{}{Bot.}{}{}{s.m.}{Arbusto nativo da Ásia, cultivado há mais de quatro mil anos, fornece fibras com aplicações industriais, e também a maconha e o haxixe.}{câ.nha.mo}{0}
\verb{cânhamo}{}{Por ext.}{}{}{}{Fibra, fio ou tecido de cânhamo.}{câ.nha.mo}{0}
\verb{cânhamo}{}{Pop.}{}{}{}{Maconha de boa qualidade.}{câ.nha.mo}{0}
\verb{canhão}{}{}{"-ões}{}{s.m.}{Peça de artilharia de grosso calibre, cano longo, grande velocidade inicial, e campo de tiro vertical limitado.}{ca.nhão}{0}
\verb{canhão}{}{}{"-ões}{}{}{Peça metálica que forma a entrada de certas fechaduras.}{ca.nhão}{0}
\verb{canhão}{}{}{"-ões}{}{}{Extremidade de manga de veste, de bota, de luva etc.}{ca.nhão}{0}
\verb{canhão}{}{Pop.}{"-ões}{}{}{Mulher muito feia.}{ca.nhão}{0}
\verb{canhenho}{}{}{}{}{s.m.}{Caderno de notas, de apontamentos.}{ca.nhe.nho}{0}
\verb{canhenho}{}{}{}{}{}{Registro de lembranças.}{ca.nhe.nho}{0}
\verb{canhenho}{}{Fig.}{}{}{}{Capacidade de lembrar; memória.}{ca.nhe.nho}{0}
\verb{canhestro}{ê}{}{}{}{adj.}{Diz"-se daquele que não tem habilidade nem destreza; desajeitado.               }{ca.nhes.tro}{0}
\verb{canhestro}{ê}{}{}{}{}{Que tem vergonha; tímido, ressabiado.    }{ca.nhes.tro}{0}
\verb{canhonaço}{}{}{}{}{s.m.}{Disparo de canhão.}{ca.nho.na.ço}{0}
\verb{canhonada}{}{}{}{}{s.f.}{Descarga de canhões.}{ca.nho.na.da}{0}
\verb{canhonear}{}{}{}{}{v.t.}{Disparar tiros de canhão.}{ca.nho.ne.ar}{0}
\verb{canhonear}{}{}{}{}{}{Atacar com censuras; criticar.}{ca.nho.ne.ar}{\verboinum{4}}
\verb{canhoneio}{ê}{}{}{}{s.m.}{Canhonada.}{ca.nho.nei.o}{0}
\verb{canhoneira}{ê}{}{}{}{s.f.}{Navio de pequeno tamanho, empregado em operação de defesa costeira e fluvial.}{ca.nho.nei.ra}{0}
\verb{canhoneiro}{ê}{}{}{}{s.m.}{Que tem canhões; guarnecido de canhões.}{ca.nho.nei.ro}{0}
\verb{canhota}{ó/ ou /ô}{}{}{}{s.f.}{A mão esquerda.}{ca.nho.ta}{0}
\verb{canhota}{ó/ ou /ô}{}{}{}{}{Feminino de \textit{canhoto}.}{ca.nho.ta}{0}
\verb{canhoto}{ô}{}{"-s ⟨ô⟩}{do adj.: -a  ⟨ó }ou\textit{ ô⟩}{adj.}{Que usa mais a mão esquerda.}{ca.nho.to}{0}
\verb{canhoto}{ô}{}{"-s ⟨ô⟩}{do adj.: -a  ⟨ó }ou\textit{ ô⟩}{s.m.}{Parte que não se destaca de um carnê ou talão.}{ca.nho.to}{0}
\verb{canibal}{}{}{"-ais}{}{adj.2g.}{Que come carne de outro ser humano; antropófago. (\textit{Foi preso, na semana passada, o alemão canibal.})}{ca.ni.bal}{0}
\verb{canibal}{}{}{"-ais}{}{}{Que come carne de animal da mesma espécie.}{ca.ni.bal}{0}
\verb{canibalesco}{ê}{}{}{}{adj.}{Próprio de canibal.}{ca.ni.ba.les.co}{0}
\verb{canibalismo}{}{}{}{}{s.m.}{Prática regular e institucionalizada de consumo de carne humana por seres humanos, com caráter ritual; antropofagia.}{ca.ni.ba.lis.mo}{0}
\verb{canibalismo}{}{Por ext.}{}{}{}{Ato de um animal devorar outro da mesma espécie ou da mesma família.  }{ca.ni.ba.lis.mo}{0}
%\verb{}{}{}{}{}{}{}{}{0}
\verb{canibalizar}{}{}{}{}{v.t.}{Retirar peças de máquina para utilizar na reparação ou na construção de outra máquina.}{ca.ni.ba.li.zar}{0}
\verb{canibalizar}{}{Por ext.}{}{}{}{Reaproveitar.}{ca.ni.ba.li.zar}{\verboinum{1}}
\verb{canície}{}{}{}{}{s.f.}{Brancura dos cabelos.}{ca.ní.cie}{0}
\verb{canície}{}{}{}{}{}{Idade do aparecimento das cãs; velhice.}{ca.ní.cie}{0}
\verb{caniço}{}{}{}{}{s.m.}{Cana fina.}{ca.ni.ço}{0}
\verb{caniço}{}{}{}{}{}{Vara fina de cana, usada na pesca.}{ca.ni.ço}{0}
\verb{caniço}{}{Pop.}{}{}{}{Pessoa magra; magricela.}{ca.ni.ço}{0}
\verb{canícula}{}{}{}{}{s.f.}{Grande calor do verão.}{ca.ní.cu.la}{0}
\verb{canicular}{}{}{}{}{adj.2g.}{Relativo ao tempo da canícula.}{ca.ni.cu.lar}{0}
\verb{canicular}{}{}{}{}{}{Muito quente; abrasador.}{ca.ni.cu.lar}{0}
\verb{canicultor}{ô}{}{}{}{s.m.}{Criador de cães.}{ca.ni.cul.tor}{0}
\verb{canicultura}{}{}{}{}{s.f.}{Criação de cães.}{ca.ni.cul.tu.ra}{0}
\verb{canídeo}{}{Zool.}{}{}{s.m.}{Espécime dos canídeos, mamíferos com pernas longas, garras fortes não retráteis, cauda longa e focinho delgado, como os cães, os lobos, as raposas.}{ca.ní.deo}{0}
\verb{canídeo}{}{Zool.}{}{}{adj.}{Relativo aos canídeos.}{ca.ní.deo}{0}
\verb{canil}{}{}{"-is}{}{s.m.}{Local onde se criam, alojam ou comercializam cães.}{ca.nil}{0}
\verb{canil}{}{}{"-is}{}{}{Abrigo para cães de caça.}{ca.nil}{0}
\verb{caninana}{}{Zool.}{}{}{s.f.}{Cobra não venenosa com comprimento entre 2 e 3 metros e cores negra e amarela.}{ca.ni.na.na}{0}
\verb{caninana}{}{Fig.}{}{}{}{Pessoa irritadiça, geniosa.}{ca.ni.na.na}{0}
\verb{canindé}{}{Bras.}{}{}{s.m.}{Espécie de arara azul.}{ca.nin.dé}{0}
\verb{canindé}{}{Bras.}{}{}{}{Tipo de faca comprida usada no sertão cearense.}{ca.nin.dé}{0}
\verb{caninha}{}{Bras.}{}{}{s.f.}{Aguardente de cana; cachaça.}{ca.ni.nha}{0}
\verb{canino}{}{}{}{}{adj.}{Relativo a cão.}{ca.ni.no}{0}
\verb{canino}{}{Anat.}{}{}{}{Diz"-se dos dentes que se localizam entre os incisivos e os pré"-molares na dentição da espécie humana.}{ca.ni.no}{0}
\verb{canitar}{}{Bras.}{}{}{s.m.}{Adorno de cabeça, fartamente enfeitado com penas, usado por alguns povos indígenas em certas solenidades.}{ca.ni.tar}{0}
\verb{canivetada}{}{}{}{}{s.f.}{Golpe de canivete.}{ca.ni.ve.ta.da}{0}
\verb{canivete}{é}{}{}{}{s.m.}{Pequena faca de bolso com lâmina retrátil ou dobrável.}{ca.ni.ve.te}{0}
\verb{canivete}{é}{Zool.}{}{}{}{Certo peixe de cores cinza e prata com uma faixa escura na região mediana; peixe"-charuto.}{ca.ni.ve.te}{0}
\verb{canivete}{é}{Fig.}{}{}{}{Cavalo pequeno, magro e feio.}{ca.ni.ve.te}{0}
\verb{canja}{}{Cul.}{}{}{s.f.}{Sopa de galinha com arroz.}{can.ja}{0}
\verb{canja}{}{Bras.}{}{}{}{Tarefa que se faz sem esforço.}{can.ja}{0}
\verb{canja}{}{Bras.}{}{}{}{Apresentação improvisada ou não prevista de um músico, geralmente a pedido do público.}{can.ja}{0}
\verb{canjerana}{}{Bot.}{}{}{s.f.}{Árvore de madeira vermelha, rosada ou amarelada e flores brancas ou esverdeadas, cultivada pela madeira nobre e aromática e pela tintura dela extraída.}{can.je.ra.na}{0}
\verb{canjerê}{}{Bras.}{}{}{s.m.}{Reunião de pessoas para prática de feitiçaria.}{can.je.rê}{0}
\verb{canjerê}{}{}{}{}{}{Bruxaria, feitiço, mandinga.}{can.je.rê}{0}
\verb{canjica}{}{Cul.}{}{}{s.f.}{Sopa doce de milho branco cozido com leite.}{can.ji.ca}{0}
\verb{canjiquinha}{}{}{}{}{s.f.}{Milho picado para alimentação de aves.}{can.ji.qui.nha}{0}
\verb{canjiquinha}{}{Cul.}{}{}{}{Papa feita com milho verde ralado, leite, açúcar e canela; curau.}{can.ji.qui.nha}{0}
\verb{canjirão}{}{}{"-ões}{}{s.m.}{Jarro de boca larga, usado geralmente para vinho.}{can.ji.rão}{0}
\verb{cano}{}{}{}{}{s.m.}{Cilindro oco, feito geralmente de metal ou plástico, para transportar líquidos e gases.}{ca.no}{0}
\verb{canoa}{ô}{}{}{}{s.f.}{Embarcação de pequeno porte sem quilha e sem leme e movida a remo.}{ca.no.a}{0}
\verb{canoagem}{}{Esport.}{"-ens}{}{s.f.}{Modalidade na qual os praticantes descem rios ou corredeiras usando canoas ou similares.}{ca.no.a.gem}{0}
\verb{canoeiro}{ê}{}{}{}{s.m.}{Indivíduo que dirige, fabrica ou lida com canoas.}{ca.no.ei.ro}{0}
\verb{cânon}{}{}{}{}{s.m.}{Cânone.}{câ.non}{0}
\verb{cânone}{}{Relig.}{}{}{s.m.}{Regra, decreto referente à disciplina religiosa.}{câ.no.ne}{0}
\verb{cânone}{}{Por ext.}{}{}{}{Conjunto de regras.}{câ.no.ne}{0}
\verb{cânone}{}{Relig.}{}{}{}{Lista de santos reconhecidos pela Igreja.}{câ.no.ne}{0}
\verb{cânone}{}{Relig.}{}{}{}{Parte central da missa católica.}{câ.no.ne}{0}
\verb{canonical}{}{}{"-ais}{}{adj.2g.}{Relativo a cônego ou a canonicato.}{ca.no.ni.cal}{0}
\verb{canonicato}{}{}{}{}{s.m.}{Dignidade de cônego.}{ca.no.ni.ca.to}{0}
\verb{canonicidade}{}{}{}{}{s.f.}{Qualidade de canônico.}{ca.no.ni.ci.da.de}{0}
\verb{canônico}{}{}{}{}{adj.}{Relativo a cânone.}{ca.nô.ni.co}{0}
\verb{canônico}{}{}{}{}{}{De acordo com os cânones.}{ca.nô.ni.co}{0}
\verb{canonisa}{}{}{}{}{s.f.}{Mulher com cargo equivalente ao de cônego.}{ca.no.ni.sa}{0}
\verb{canonização}{}{}{"-ões}{}{s.f.}{Ato ou efeito de canonizar.}{ca.no.ni.za.ção}{0}
\verb{canonização}{}{}{"-ões}{}{}{Decisão papal que reconhece as virtudes excepcionais de uma pessoa e a incorpora no conjunto dos santos.}{ca.no.ni.za.ção}{0}
\verb{canonização}{}{}{"-ões}{}{}{Cerimônia que acompanha essa decisão.}{ca.no.ni.za.ção}{0}
\verb{canonização}{}{Por ext.}{"-ões}{}{}{Glorificação.}{ca.no.ni.za.ção}{0}
\verb{canonizar}{}{}{}{}{v.t.}{Incluir no conjunto dos santos.}{ca.no.ni.zar}{0}
\verb{canonizar}{}{Fig.}{}{}{}{Exaltar, elogiar.}{ca.no.ni.zar}{0}
\verb{canonizar}{}{}{}{}{}{Tornar prática corrente; consagrar, autorizar.}{ca.no.ni.zar}{\verboinum{1}}
\verb{canoro}{ó}{}{}{}{adj.}{De canto agradável e harmonioso; melodioso, sonoro, suave.}{ca.no.ro}{0}
\verb{cansaço}{}{}{}{}{s.m.}{Falta de forças ou sensação semelhante provocada por esforço físico ou mental ou por doença; fadiga.}{can.sa.ço}{0}
\verb{cansaço}{}{Fig.}{}{}{}{Aborrecimento.}{can.sa.ço}{0}
\verb{cansado}{}{}{}{}{adj.}{Que se cansou; cheio de cansaço; fatigado.}{can.sa.do}{0}
\verb{cansado}{}{}{}{}{}{Aborrecido, enfastiado.}{can.sa.do}{0}
\verb{cansado}{}{}{}{}{}{Diz"-se da terra que se tornou pouco produtiva por já haver suportado muitas culturas; infértil.}{can.sa.do}{0}
\verb{cansanção}{}{Bot.}{"-ões}{}{s.m.}{Designação comum a várias plantas com pelos urticantes que queimam a pele.}{can.san.ção}{0}
\verb{cansar}{}{}{}{}{v.t.}{Causar cansaço físico ou mental; deixar sem forças para mais atividade; esgotar, fatigar. (\textit{O jogo acirrado cansou as jogadoras.})}{can.sar}{0}
\verb{cansar}{}{}{}{}{}{Aborrecer, enfastiar. (\textit{Essa conversa toda me cansa.})}{can.sar}{0}
\verb{cansar}{}{}{}{}{}{Desistir de fazer alguma coisa. (\textit{Cansamos de esperá"-lo para a reunião e decidimos iniciá"-la.})}{can.sar}{\verboinum{1}}
\verb{cansativo}{}{}{}{}{adj.}{Que causa cansaço; fatigante.}{can.sa.ti.vo}{0}
\verb{cansativo}{}{}{}{}{}{Entediante, monótono, enfadonho.}{can.sa.ti.vo}{0}
\verb{canseira}{ê}{}{}{}{s.f.}{Cansaço.}{can.sei.ra}{0}
\verb{canseira}{ê}{}{}{}{}{Esforço ou espera demasiada para se conseguir algo.}{can.sei.ra}{0}
\verb{cantada}{}{}{}{}{s.f.}{Canto.}{can.ta.da}{0}
\verb{cantada}{}{Pop.}{}{}{}{Conversa cheia de esperteza para conseguir alguma coisa de alguém. (\textit{O rapaz passou uma cantada no pai para conseguir o carro.})}{can.ta.da}{0}
\verb{cantador}{ô}{}{}{}{adj.}{Que canta.}{can.ta.dor}{0}
\verb{cantador}{ô}{}{}{}{s.m.}{Cantor popular.}{can.ta.dor}{0}
\verb{cantador}{ô}{}{}{}{}{Poeta popular que conta histórias por meio de versos cantados geralmente acompanhado de viola ou outro instrumento.}{can.ta.dor}{0}
\verb{cantante}{}{}{}{}{adj.2g.}{Que canta.}{can.tan.te}{0}
\verb{cantante}{}{}{}{}{}{Próprio para canto.}{can.tan.te}{0}
\verb{cantão}{}{}{"-ões}{}{s.m.}{Tipo de divisão territorial utilizado em diversos países.}{can.tão}{0}
\verb{cantar}{}{}{}{}{v.t.}{Produzir sons musicais com a voz.}{can.tar}{0}
\verb{cantar}{}{}{}{}{}{Reproduzir sons musicais para as palavras de um texto.}{can.tar}{0}
\verb{cantar}{}{}{}{}{}{Falar em louvor de pessoa ou coisa em uma poesia; celebrar, exaltar.}{can.tar}{0}
\verb{cantar}{}{Pop.}{}{}{}{Falar com esperteza para conseguir alguma coisa de alguém.}{can.tar}{\verboinum{1}}
\verb{cantaria}{}{}{}{}{s.f.}{Pedra lavrada para uso em construções.}{can.ta.ri.a}{0}
\verb{cantaria}{}{}{}{}{}{Obra de alvenaria feita com essas pedras.}{can.ta.ri.a}{0}
\verb{cantárida}{}{Zool.}{}{}{s.f.}{Besouro de cor verde brilhante do qual se extrai substância com propriedades diuréticas e afrodisíacas, muito utilizada na Antiguidade.}{can.tá.ri.da}{0}
\verb{cântaro}{}{}{}{}{s.m.}{Vaso bojudo de barro ou metal, com uma ou duas asas, para líquidos.}{cân.ta.ro}{0}
\verb{cântaro}{}{}{}{}{}{Usado na locução \textit{a cântaros}: fartamente, copiosamente, torrencialmente. (\textit{Ontem choveu a cântaros.})}{cân.ta.ro}{0}
\verb{cantarolar}{}{}{}{}{v.t.}{Cantar a meia voz, para si mesmo.}{can.ta.ro.lar}{0}
\verb{cantarolar}{}{}{}{}{}{Cantar desafinadamente.}{can.ta.ro.lar}{0}
\verb{cantarolar}{}{Fig.}{}{}{}{Emitir som melodioso ou cadenciado.}{can.ta.ro.lar}{\verboinum{1}}
\verb{cantata}{}{}{}{}{s.f.}{Antiga forma de poema lírico.}{can.ta.ta}{0}
\verb{cantata}{}{}{}{}{}{Composição poética para ser cantada.}{can.ta.ta}{0}
\verb{cantata}{}{Mús.}{}{}{}{Peça musical para solista(s), coro e orquestra.}{can.ta.ta}{0}
%\verb{}{}{}{}{}{}{}{}{0}
\verb{canteiro}{ê}{}{}{}{s.m.}{Porção de terreno para cultivo de plantas, geralmente flores e hortaliças.}{can.tei.ro}{0}
\verb{canteiro}{ê}{}{}{}{}{Operário que lavra pedra de cantaria.}{can.tei.ro}{0}
\verb{canteiro}{ê}{}{}{}{}{Escultor que trabalha em pedra.}{can.tei.ro}{0}
\verb{cântico}{}{}{}{}{s.m.}{Canto em honra a uma divindade.}{cân.ti.co}{0}
\verb{cântico}{}{Por ext.}{}{}{}{Qualquer hino ou poema em louvor a alguém ou algo.}{cân.ti.co}{0}
\verb{cantiga}{}{}{}{}{s.f.}{Poesia cantada, geralmente composta em redondilhas menores.}{can.ti.ga}{0}
\verb{cantiga}{}{}{}{}{}{Quadra para cantar.}{can.ti.ga}{0}
\verb{cantiga}{}{}{}{}{}{Canção, modinha, ária.}{can.ti.ga}{0}
\verb{cantil}{}{}{"-is}{}{s.m.}{Recipiente com tampa para transportar líquidos em viagens.}{can.til}{0}
\verb{cantilena}{}{}{}{}{s.f.}{Cantiga suave.}{can.ti.le.na}{0}
\verb{cantilena}{}{}{}{}{}{Cantiga monótona; melopeia.}{can.ti.le.na}{0}
\verb{cantilena}{}{Pop.}{}{}{}{Conversação enfadonha; ladainha.}{can.ti.le.na}{0}
\verb{cantina}{}{}{}{}{s.f.}{Pequena lanchonete em escolas, hospitais, quartéis, na qual se servem bebidas e comidas rápidas.}{can.ti.na}{0}
\verb{cantina}{}{}{}{}{}{Restaurante rústico.}{can.ti.na}{0}
\verb{cantina}{}{}{}{}{}{Restaurante que serve especialmente comida italiana e bons vinhos.}{can.ti.na}{0}
\verb{canto}{}{}{}{}{s.m.}{Parte de fora do encontro de duas superfícies; aresta, quina.}{can.to}{0}
\verb{canto}{}{}{}{}{s.m.}{Conjunto de sons musicais produzidos pela voz do homem ou de animal.}{can.to}{0}
\verb{canto}{}{}{}{}{}{Parte de dentro do encontro de duas superfícies.}{can.to}{0}
\verb{canto}{}{}{}{}{}{Lugar quieto ou retirado.}{can.to}{0}
\verb{canto}{}{}{}{}{}{Arte de produzir sons musicais com a voz.}{can.to}{0}
\verb{cantochão}{}{}{"-ãos}{}{s.m.}{Canto litúrgico da Igreja Católica Ocidental; canto gregoriano.}{can.to.chão}{0}
\verb{cantoneira}{ê}{}{}{}{s.f.}{Móvel ou prateleira próprio para ser instalado no canto de um cômodo.}{can.to.nei.ra}{0}
\verb{cantoneira}{ê}{}{}{}{}{Peça metálica em forma de \textsc{l} para reforçar quinas ou fixar junções nos cantos de móveis.}{can.to.nei.ra}{0}
\verb{cantor}{ô}{}{}{}{adj.}{Que canta.}{can.tor}{0}
\verb{cantor}{ô}{}{}{}{s.m.}{Indivíduo que canta por profissão ou por lazer.}{can.tor}{0}
\verb{cantor}{ô}{Liter.}{}{}{}{Poeta épico.}{can.tor}{0}
\verb{cantoria}{}{}{}{}{s.f.}{Ato ou efeito de cantar; canto.}{can.to.ri.a}{0}
\verb{cantoria}{}{}{}{}{}{Conjunto de vozes cantando.}{can.to.ri.a}{0}
\verb{cantoria}{}{Bras.}{}{}{}{Desafio de cantadores.}{can.to.ri.a}{0}
\verb{canudo}{}{}{}{}{s.m.}{Tubo estreito e comprido.}{ca.nu.do}{0}
\verb{canudo}{}{}{}{}{}{Estojo cilíndrico para guardar e transportar papéis enrolados.}{ca.nu.do}{0}
\verb{canudo}{}{Pop.}{}{}{}{O diploma universitário.}{ca.nu.do}{0}
\verb{cânula}{}{Med.}{}{}{s.f.}{Tubo de plástico, metal ou vidro, adaptável a seringas e instrumentos cirúrgicos, utilizado em procedimentos ou após eles.}{câ.nu.la}{0}
\verb{canutilho}{}{}{}{}{s.m.}{Miçanga longa para enfeite e guarnição de vestuário.}{ca.nu.ti.lho}{0}
%\verb{}{}{}{}{}{}{}{}{0}
\verb{canzarrão}{}{}{"-ões}{}{s.m.}{Cão muito grande.}{can.zar.rão}{0}
\verb{canzoada}{}{}{}{}{s.f.}{Grupo de cães.}{can.zo.a.da}{0}
\verb{canzoada}{}{}{}{}{}{Barulho produzido por muitos cães.}{can.zo.a.da}{0}
\verb{canzoada}{}{Pop.}{}{}{}{Grupo de patifes.}{can.zo.a.da}{0}
\verb{cão}{}{Zool.}{cães}{}{s.m.}{Mamífero carnívoro criado no mundo todo como animal doméstico.}{cão}{0}
\verb{cão}{}{Fig.}{cães}{}{}{Pessoa muito má.}{cão}{0}
\verb{cão}{}{Bras.}{cães}{}{}{O diabo.}{cão}{0}
\verb{cão}{}{}{cães}{}{}{Peça de arma de fogo que faz a percussão sobre o estopim da cápsula.}{cão}{0}
\verb{caolho}{ô}{}{}{}{adj.}{Que não tem um dos olhos.}{ca.o.lho}{0}
\verb{caolho}{ô}{}{}{}{}{Estrábico, vesgo.}{ca.o.lho}{0}
\verb{caos}{}{}{}{}{s.m.}{Estado de confusão de todos os elementos antes da estruturação do mundo.}{ca.os}{0}
\verb{caos}{}{Fig.}{}{}{}{Grande confusão; desordem.}{ca.os}{0}
\verb{caótico}{}{}{}{}{adj.}{Que está em estado de caos.}{ca.ó.ti.co}{0}
\verb{cão"-tinhoso}{ô}{Bras.}{cães"-tinhosos ⟨ó⟩}{}{s.m.}{O diabo, o tinhoso.}{cão"-ti.nho.so}{0}
\verb{capa}{}{}{}{}{s.m.}{Décima letra do alfabeto grego.}{ca.pa}{0}
\verb{capa}{}{}{}{}{s.f.}{Peça de vestuário usada sobre outra para protegê"-la ou proteger quem a veste contra a chuva ou o frio.}{ca.pa}{0}
\verb{capa}{}{}{}{}{}{Peça de tecido, de plástico ou de outro material usada para proteger objetos; cobertura.}{ca.pa}{0}
\verb{capa}{}{}{}{}{}{Cobertura de papel ou de outro material, flexível ou rígida, que cobre ou protege um livro, um folheto etc.}{ca.pa}{0}
\verb{capa}{}{}{}{}{s.f.}{Capação, castração.}{ca.pa}{0}
\verb{capação}{}{}{"-ões}{}{s.f.}{Ato ou efeito de capar animais.}{ca.pa.ção}{0}
\verb{capação}{}{}{"-ões}{}{}{Período em que se pratica esse ato.}{ca.pa.ção}{0}
\verb{capação}{}{}{"-ões}{}{}{Corte dos brotos de uma planta.}{ca.pa.ção}{0}
\verb{capacete}{ê}{}{}{}{s.m.}{Armadura de proteção, em forma oval, para a cabeça.}{ca.pa.ce.te}{0}
\verb{capacho}{}{}{}{}{s.m.}{Tipo de tapete espesso e de fibras grossas, que se coloca à porta externa das casas para limpar os pés.}{ca.pa.cho}{0}
\verb{capacho}{}{Fig.}{}{}{}{Indivíduo submisso, servil.}{ca.pa.cho}{0}
\verb{capacidade}{}{}{}{}{s.f.}{Conjunto de qualidades que permitem a alguém fazer alguma coisa.}{ca.pa.ci.da.de}{0}
\verb{capacidade}{}{}{}{}{}{Pessoa que faz bem alguma coisa; sumidade, talento.}{ca.pa.ci.da.de}{0}
\verb{capacidade}{}{}{}{}{}{Espaço que pode ser ocupado dentro de alguma coisa. (\textit{A capacidade do tanque de gasolina desse carro é de 50 litros.})}{ca.pa.ci.da.de}{0}
\verb{capacitância}{}{Fís.}{}{}{s.f.}{Propriedade de alguns sistemas de armazenar energia elétrica.}{ca.pa.ci.tân.cia}{0}
\verb{capacitância}{}{Fís.}{}{}{}{Quociente entre carga elétrica do capacitor e a tensão elétrica nesse ponto do circuito.}{ca.pa.ci.tân.cia}{0}
\verb{capacitar}{}{}{}{}{v.t.}{Tornar apto; habilitar.}{ca.pa.ci.tar}{0}
\verb{capacitar}{}{}{}{}{v.pron.}{Convencer"-se; persuadir"-se.}{ca.pa.ci.tar}{\verboinum{1}}
\verb{capacitor}{ô}{Fís.}{}{}{s.m.}{Sistema formado por dois condutores elétricos separados por um isolador.}{ca.pa.ci.tor}{0}
\verb{capado}{}{}{}{}{adj.}{Que se capou; castrado.}{ca.pa.do}{0}
\verb{capado}{}{}{}{}{s.m.}{Animal castrado.}{ca.pa.do}{0}
\verb{capado}{}{}{}{}{}{Porco castrado, designado para engorda.}{ca.pa.do}{0}
\verb{capadócio}{}{}{}{}{adj.}{Relativo à Capadócia, província da Ásia Menor.}{ca.pa.dó.cio}{0}
\verb{capadócio}{}{}{}{}{s.m.}{Indivíduo natural ou habitante dessa província.}{ca.pa.dó.cio}{0}
\verb{capadócio}{}{Pejor.}{}{}{adj.}{Pouco inteligente; ignorante, burro.}{ca.pa.dó.cio}{0}
\verb{capadócio}{}{Bras.}{}{}{}{Impostor, trapaceiro.}{ca.pa.dó.cio}{0}
\verb{capanga}{}{Bras.}{}{}{s.m.}{Indivíduo contratado para prestar serviços de guarda"-costas ou relacionados com defesa da propriedade, acerto de contas.}{ca.pan.ga}{0}
\verb{capanga}{}{Bras.}{}{}{s.f.}{Tipo de bolsa que se usa a tiracolo para carregar pequenos objetos.}{ca.pan.ga}{0}
\verb{capanga}{}{Bras.}{}{}{s.m.}{Bolsa pequena usada por comerciantes de pedras preciosas.}{ca.pan.ga}{0}
\verb{capão}{}{}{"-ões}{}{s.m.}{Porção de mato ou de vegetação arbórea isolada ou realçada no meio da paisagem.}{ca.pão}{0}
\verb{capão}{}{}{"-ões}{}{s.m.}{Frango capado e designado para engorda e abate.}{ca.pão}{0}
\verb{capão}{}{}{"-ões}{}{s.m.}{Animal capado.}{ca.pão}{0}
\verb{capão}{}{Fig.}{"-ões}{}{}{Indivíduo covarde, medroso, fraco.}{ca.pão}{0}
\verb{capar}{}{}{}{}{v.t.}{Retirar ou inutilizar o órgão reprodutor.}{ca.par}{0}
\verb{capar}{}{}{}{}{}{Podar flor ou broto de planta.}{ca.par}{0}
\verb{capar}{}{}{}{}{}{Remover fragmento de texto, filme, obra; mutilar.}{ca.par}{\verboinum{1}}
\verb{capataz}{}{}{}{}{s.m.}{Chefe de um grupo de trabalhadores braçais.}{ca.pa.taz}{0}
\verb{capataz}{}{}{}{}{}{Administrador de propriedade rural.}{ca.pa.taz}{0}
\verb{capatazia}{}{}{}{}{s.f.}{Função de capataz.}{ca.pa.ta.zi.a}{0}
\verb{capatazia}{}{}{}{}{}{Grupo chefiado por capataz.}{ca.pa.ta.zi.a}{0}
\verb{capaz}{}{}{}{}{adj.}{Que pode conter algo em si.}{ca.paz}{0}
\verb{capaz}{}{}{}{}{}{Que tem competência ou aptidão; competente, apto.}{ca.paz}{0}
\verb{capcioso}{ô}{}{"-osos ⟨ó⟩}{"-osa ⟨ó⟩}{adj.}{Que engana; caviloso, manhoso. (\textit{É um homem malvado e capcioso.})}{cap.ci.o.so}{0}
\verb{capcioso}{ô}{}{"-osos ⟨ó⟩}{"-osa ⟨ó⟩}{}{Que procura confundir, para levar ao erro; ardiloso, astucioso. (\textit{Na prova só havia questões capciosas.})}{cap.ci.o.so}{0}
\verb{capear}{}{}{}{}{v.t.}{Revestir com capa.}{ca.pe.ar}{0}
\verb{capear}{}{}{}{}{}{Ocultar, encobrir, disfarçar.}{ca.pe.ar}{\verboinum{4}}
\verb{capeirão}{}{}{"-ões}{}{s.m.}{Capa grande.}{ca.pei.rão}{0}
\verb{capela}{é}{}{}{}{s.f.}{Pequena igreja de um só andar.}{ca.pe.la}{0}
\verb{capela}{é}{}{}{}{}{Cada um dos locais, em uma igreja, reservados para oração, meditação ou pequenos serviços religiosos, onde fica um altar de santo.}{ca.pe.la}{0}
\verb{capela}{é}{}{}{}{}{Espaço destinado ao culto religioso dentro de alguns estabelecimentos.}{ca.pe.la}{0}
\verb{capelania}{}{}{}{}{s.f.}{Cargo, dignidade ou ofício de capelão.}{ca.pe.la.ni.a}{0}
\verb{capelão}{}{}{"-ões}{}{s.m.}{Padre responsável pelos ofícios religiosos de uma capela.}{ca.pe.lão}{0}
\verb{capelão}{}{}{"-ões}{}{}{Padre encarregado de dar assistência espiritual às pessoas de uma organização.}{ca.pe.lão}{0}
\verb{capelo}{ê}{}{}{}{s.m.}{Capuz de frade.}{ca.pe.lo}{0}
\verb{capelo}{ê}{}{}{}{}{Chapéu de cardeal.}{ca.pe.lo}{0}
\verb{capelo}{ê}{}{}{}{}{Pequena capa usada por doutores em cerimônias acadêmicas.}{ca.pe.lo}{0}
\verb{capenga}{}{}{}{}{adj.}{Que capenga; coxo, manco.}{ca.pen.ga}{0}
\verb{capengar}{}{}{}{}{v.i.}{Andar com dificuldade, balançando o corpo, por defeito físico ou ferimento em uma das pernas; coxear.}{ca.pen.gar}{\verboinum{5}}
\verb{capeta}{ê}{Pop.}{}{}{s.m.}{O diabo.}{ca.pe.ta}{0}
\verb{capeta}{ê}{}{}{}{adj.}{Diz"-se daquele que é levado, travesso.}{ca.pe.ta}{0}
\verb{capetagem}{}{}{"-ens}{}{s.f.}{Ato ou procedimento de capeta; traquinagem, travessura.}{ca.pe.ta.gem}{0}
\verb{capetice}{}{}{}{}{s.f.}{Capetagem.}{ca.pe.ti.ce}{0}
\verb{capetinha}{}{}{}{}{s.2g.}{Criança travessa, levada.}{ca.pe.ti.nha}{0}
\verb{capiau}{}{}{}{capioa}{s.m.}{Caipira.}{ca.pi.au}{0}
\verb{capilar}{}{}{}{}{adj.2g.}{Relativo a cabelo.}{ca.pi.lar}{0}
\verb{capilar}{}{}{}{}{}{Que é fino como um cabelo.}{ca.pi.lar}{0}
\verb{capilária}{}{Bot.}{}{}{s.f.}{Designação comum a diversas avencas.}{ca.pi.lá.ria}{0}
\verb{capilaridade}{}{}{}{}{s.f.}{Qualidade do que é capilar.}{ca.pi.la.ri.da.de}{0}
\verb{capilaridade}{}{}{}{}{}{Parte da física que estuda os fenômenos capilares.}{ca.pi.la.ri.da.de}{0}
\verb{capilé}{}{}{}{}{s.m.}{Calda ou xarope feito com suco de avenca.}{ca.pi.lé}{0}
\verb{capim}{}{Bot.}{"-ins}{}{s.m.}{Erva fina e comprida, geralmente usada na alimentação do gado.}{ca.pim}{0}
\verb{capina}{}{}{}{}{s.f.}{Ato ou efeito de capinar; retirada de capim.}{ca.pi.na}{0}
\verb{capinação}{}{}{"-ões}{}{s.f.}{Capina.}{ca.pi.na.ção}{0}
\verb{capinadeira}{ê}{}{}{}{s.f.}{Máquina para capina mecânica.}{ca.pi.na.dei.ra}{0}
\verb{capinar}{}{}{}{}{v.t.}{Limpar terreno ou plantação de capim ou outra erva.}{ca.pi.nar}{\verboinum{1}}
\verb{capincho}{}{Zool.}{}{}{s.m.}{Capivara.}{ca.pin.cho}{0}
\verb{capineiro}{ê}{}{}{}{s.m.}{Indivíduo que capina.}{ca.pi.nei.ro}{0}
\verb{capinzal}{}{}{"-ais}{}{s.m.}{Terreno onde cresce capim de qualquer espécie.}{ca.pin.zal}{0}
\verb{capiscar}{}{}{}{}{v.t.}{Entender pouco ou mal de língua, ofício ou arte.}{ca.pis.car}{0}
\verb{capiscar}{}{Por ext.}{}{}{}{Entender, perceber o sentido de algo.}{ca.pis.car}{\verboinum{2}}
\verb{capista}{}{}{}{}{s.2g.}{Indivíduo que desenha ou projeta capas para livros.}{ca.pis.ta}{0}
\verb{capitação}{}{}{"-ões}{}{s.f.}{Imposto, tributo ou contribuição que se paga por cabeça.}{ca.pi.ta.ção}{0}
\verb{capital}{}{}{"-ais}{}{adj.2g.}{Que se considera em primeiro lugar; essencial, fundamental, principal.}{ca.pi.tal}{0}
\verb{capital}{}{}{"-ais}{}{s.f.}{Cidade onde funciona o Governo de um Estado ou país.}{ca.pi.tal}{0}
\verb{capital}{}{}{"-ais}{}{s.m.}{Dinheiro que uma pessoa ou empresa tem para produzir mais riqueza.}{ca.pi.tal}{0}
\verb{capitalismo}{}{}{}{}{s.m.}{Sistema econômico que se baseia na propriedade privada dos meios de produção, visando ao lucro. }{ca.pi.ta.lis.mo}{0}
\verb{capitalista}{}{}{}{}{adj.2g.}{Relativo a capital ou ao capitalismo.}{ca.pi.ta.lis.ta}{0}
\verb{capitalista}{}{}{}{}{s.2g.}{Indivíduo que vive do rendimento de um capital.}{ca.pi.ta.lis.ta}{0}
\verb{capitalista}{}{Por ext.}{}{}{}{Indivíduo que tem muito dinheiro.}{ca.pi.ta.lis.ta}{0}
\verb{capitalização}{}{}{"-ões}{}{s.f.}{Ato ou efeito de capitalizar; acumulação de capitais.}{ca.pi.ta.li.za.ção}{0}
\verb{capitalizar}{}{}{}{}{v.t.}{Transformar alguma propriedade em capital.}{ca.pi.ta.li.zar}{0}
\verb{capitalizar}{}{}{}{}{v.i.}{Formar ou acumular capital, bens, riquezas.}{ca.pi.ta.li.zar}{\verboinum{1}}
\verb{capitanear}{}{}{}{}{v.t.}{Dirigir como capitão; comandar.}{ca.pi.ta.ne.ar}{0}
\verb{capitanear}{}{}{}{}{}{Dirigir, governar.}{ca.pi.ta.ne.ar}{\verboinum{4}}
\verb{capitânia}{}{}{}{}{adj.}{Diz"-se do navio em que se acha embarcado o comandante de uma força naval.}{ca.pi.tâ.nia}{0}
\verb{capitania}{}{}{}{}{s.f.}{Dignidade ou cargo de capitão.}{ca.pi.ta.ni.a}{0}
\verb{capitânia}{}{}{}{}{s.m.}{Navio que comanda uma esquadra.}{ca.pi.tâ.nia}{0}
\verb{capitania}{}{}{}{}{}{Cada uma das grandes extensões de terra que, no Brasil colonial, eram dadas pelo rei de Portugal a pessoas de sua escolha, para as colonizar e defender.}{ca.pi.ta.ni.a}{0}
\verb{capitão}{}{}{"-ães}{}{s.m.}{Militar que tem o posto entre o de primeiro"-tenente e o de major.}{ca.pi.tão}{0}
\verb{capitão}{}{}{"-ães}{}{}{Chefe de tropa militar; caudilho, comandante.}{ca.pi.tão}{0}
\verb{capitão}{}{}{"-ães}{}{}{Comandante de navio mercante.}{ca.pi.tão}{0}
\verb{capitão"-aviador}{ô}{}{capitães"-aviadores ⟨ô⟩}{}{s.m.}{Posto da Aeronáutica imediatamente superior ao de primeiro"-tenente"-aviador e imediatamente inferior ao de major"-aviador.}{ca.pi.tão"-a.vi.a.dor}{0}
\verb{capitão"-aviador}{ô}{}{capitães"-aviadores ⟨ô⟩}{}{}{Militar que ocupa esse posto.}{ca.pi.tão"-a.vi.a.dor}{0}
\verb{capitão"-de"-corveta}{ê}{}{capitães"-de"-corveta ⟨ê⟩}{}{s.m.}{Posto da Armada, entre o de capitão"-tenente e o de capitão"-de"-fragata.}{ca.pi.tão"-de"-cor.ve.ta}{0}
\verb{capitão"-de"-corveta}{ê}{}{capitães"-de"-corveta ⟨ê⟩}{}{}{Militar que ocupa esse posto.}{ca.pi.tão"-de"-cor.ve.ta}{0}
\verb{capitão"-de"-fragata}{}{}{capitães"-de"-fragata}{}{s.m.}{Posto da Armada, entre o de capitão"-de"-corveta e o de capitão"-de"-mar"-e"-guerra.}{ca.pi.tão"-de"-fra.ga.ta}{0}
\verb{capitão"-de"-fragata}{}{}{capitães"-de"-fragata}{}{}{Militar que ocupa esse posto.}{ca.pi.tão"-de"-fra.ga.ta}{0}
\verb{capitão"-de"-mar"-e"-guerra}{é}{}{capitães"-de"-mar"-e"-guerra ⟨é⟩}{}{s.m.}{Oficial que ocupa o posto acima de capitão"-de"-fragata e abaixo de contra"-almirante.}{ca.pi.tão"-de"-mar"-e"-guer.ra}{0}
\verb{capitão"-de"-mar"-e"-guerra}{é}{}{capitães"-de"-mar"-e"-guerra ⟨é⟩}{}{}{Militar que ocupa esse posto.}{ca.pi.tão"-de"-mar"-e"-guer.ra}{0}
\verb{capitão"-do"-mato}{}{}{capitães"-do"-mato}{}{s.m.}{Agente de polícia que tinha a seu cargo a captura de escravos fugidos.}{ca.pi.tão"-do"-ma.to}{0}
\verb{capitão"-mor}{ó}{}{capitães"-mores ⟨ó⟩}{}{}{Título que tinham os donatários das capitanias hereditárias.}{ca.pi.tão"-mor}{0}
\verb{capitão"-mor}{ó}{}{capitães"-mores ⟨ó⟩}{}{s.m.}{Autoridade militar que comandava, numa cidade ou vila, a milícia chamada ordenança.}{ca.pi.tão"-mor}{0}
\verb{capitão"-tenente}{}{}{capitães"-tenentes}{}{s.m.}{Posto da Armada, entre o de primeiro"-tenente e o de capitão"-de"-corveta.}{ca.pi.tão"-te.nen.te}{0}
\verb{capitão"-tenente}{}{}{capitães"-tenentes}{}{}{Militar que ocupa esse posto.}{ca.pi.tão"-te.nen.te}{0}
\verb{capitari}{}{Zool.}{}{}{s.m.}{Macho da tartaruga.}{ca.pi.ta.ri}{0}
\verb{capitel}{é}{}{"-éis}{}{s.m.}{Parte superior de coluna, balaústre ou pilastra.}{ca.pi.tel}{0}
\verb{capitólio}{}{}{}{}{s.m.}{Uma das colinas de Roma.}{ca.pi.tó.lio}{0}
\verb{capitólio}{}{}{}{}{}{Templo de Júpiter construído sobre essa colina.}{ca.pi.tó.lio}{0}
\verb{capitólio}{}{Por ext.}{}{}{}{Edificação majestosa, geralmente sede administrativa ou política de aglomerado urbano.}{ca.pi.tó.lio}{0}
\verb{capitólio}{}{Fig.}{}{}{}{Glória, triunfo.}{ca.pi.tó.lio}{0}
\verb{capitoso}{ô}{}{"-osos ⟨ó⟩}{"-osa ⟨ó⟩}{adj.}{Que sobe à cabeça, que entontece, embriaga.}{ca.pi.to.so}{0}
\verb{capitoso}{ô}{Fig.}{"-osos ⟨ó⟩}{"-osa ⟨ó⟩}{}{Que teima; obstinado, presunçoso.}{ca.pi.to.so}{0}
\verb{capitulação}{}{}{"-ões}{}{s.f.}{Ato ou efeito de capitular.}{ca.pi.tu.la.ção}{0}
\verb{capitulação}{}{}{"-ões}{}{}{Convenção segundo a qual um chefe militar entrega ao inimigo o posto que defende ou as tropas que comanda; rendição.}{ca.pi.tu.la.ção}{0}
\verb{capitular}{}{}{}{}{adj.2g.}{Relativo a capítulo ou a assembleia de religiosos.}{ca.pi.tu.lar}{0}
\verb{capitular}{}{}{}{}{}{Diz"-se de letra maiúscula.}{ca.pi.tu.lar}{0}
\verb{capitular}{}{}{}{}{v.t.}{Reduzir a capítulos; numerar, ordenar.}{ca.pi.tu.lar}{0}
\verb{capitular}{}{}{}{}{v.i.}{Render"-se a inimigos mediante capitulação.}{ca.pi.tu.lar}{\verboinum{1}}
\verb{capítulo}{}{}{}{}{s.m.}{Cada uma das partes em que se divide um texto.}{ca.pí.tu.lo}{0}
\verb{capivara}{}{Zool.}{}{}{s.f.}{Maior mamífero roedor existente, de pelagem marrom, pernas curtas, pés anteriores com quatro dedos e posteriores com três. }{ca.pi.va.ra}{0}
\verb{capixaba}{ch}{}{}{}{adj.2g.}{Relativo ao estado do Espírito Santo; espírito"-santense.}{ca.pi.xa.ba}{0}
\verb{capixaba}{ch}{}{}{}{s.2g.}{Indivíduo natural ou habitante desse estado.}{ca.pi.xa.ba}{0}
\verb{capô}{}{}{}{}{s.m.}{Cobertura móvel que protege o motor de automóvel.}{ca.pô}{0}
\verb{capoeira}{ê}{}{}{}{s.f.}{Mato que cresce em um terreno depois da queimada.}{ca.po.ei.ra}{0}
\verb{capoeira}{ê}{Esport.}{}{}{}{Luta corporal, originária entre os escravos brasileiros, em que se usam os pés nos movimentos de ataque e defesa.}{ca.po.ei.ra}{0}
\verb{capoeira}{ê}{}{}{}{s.2g.}{Indivíduo que pratica essa luta.}{ca.po.ei.ra}{0}
\verb{capoeiragem}{}{Esport.}{"-ens}{}{s.f.}{Sistema de luta dos capoeiras.}{ca.po.ei.ra.gem}{0}
\verb{caporal}{}{}{"-ais}{}{s.m.}{Variedade de tabaco picado, de má qualidade.}{ca.po.ral}{0}
\verb{capota}{ó}{}{}{}{s.f.}{Cobertura de automóveis e outros veículos.}{ca.po.ta}{0}
\verb{capotar}{}{}{}{}{v.i.}{Tombar o veículo, ficando de lado, de rodas para cima, ou mesmo voltando a ficar sobre as rodas, depois de girar sobre si.}{ca.po.tar}{\verboinum{1}}
\verb{capote}{ó}{}{}{}{s.m.}{Capa de tecido grosso, aberta na frente, que se veste sobre a roupa e vai até abaixo dos joelhos; casacão, sobretudo.}{ca.po.te}{0}
\verb{capote}{ó}{Zool.}{}{}{}{Ave da família da galinha, de penas pretas com pintas brancas; galinha"-d'angola.}{ca.po.te}{0}
\verb{capoteiro}{ê}{}{}{}{s.m.}{Indivíduo que fabrica, vende ou conserta capotas de automóvel.}{ca.po.tei.ro}{0}
\verb{caprichar}{}{}{}{}{v.t.}{Fazer alguma coisa com muito cuidado; esforçar"-se, apurar.}{ca.pri.char}{\verboinum{1}}
\verb{capricho}{}{}{}{}{s.m.}{Cuidado que se toma ao fazer alguma coisa; aplicação, apuro, esmero.}{ca.pri.cho}{0}
\verb{capricho}{}{}{}{}{}{Desejo de fazer alguma coisa só por teimosia.}{ca.pri.cho}{0}
\verb{caprichoso}{ô}{}{"-osos ⟨ó⟩}{"-osa ⟨ó⟩}{adj.}{Que faz tudo com muito cuidado; aplicado.}{ca.pri.cho.so}{0}
\verb{caprichoso}{ô}{}{"-osos ⟨ó⟩}{"-osa ⟨ó⟩}{}{Que age por capricho; inconstante, volúvel.}{ca.pri.cho.so}{0}
\verb{capricorniano}{}{Astrol.}{}{}{s.m.}{Indivíduo que nasceu sob o signo de capricórnio.}{ca.pri.cor.ni.a.no}{0}
\verb{capricorniano}{}{Astrol.}{}{}{adj.}{Relativo ou pertencente a esse signo.}{ca.pri.cor.ni.a.no}{0}
\verb{capricórnio}{}{Astron.}{}{}{s.m.}{Décima constelação do zodíaco.}{ca.pri.cór.nio}{0}
\verb{capricórnio}{}{Astrol.}{}{}{}{O signo do zodíaco referente a essa constelação.}{ca.pri.cór.nio}{0}
\verb{caprino}{}{}{}{}{adj.}{Relativo a cabras e ovelhas.}{ca.pri.no}{0}
\verb{caprino}{}{}{}{}{s.m.}{Animal da família das cabras e ovelhas.}{ca.pri.no}{0}
\verb{cápsula}{}{}{}{}{s.f.}{Pequeno objeto, de paredes muito finas e geralmente de forma cilíndrica, usado para guardar e proteger alguma coisa que se coloca dentro dele.}{cáp.su.la}{0}
\verb{cápsula}{}{}{}{}{}{Compartimento estanque para astronautas ou instrumentos de medida, numa missão espacial.}{cáp.su.la}{0}
\verb{capsular}{}{}{}{}{adj.2g.}{Relativo a cápsula ou que tem forma de cápsula.}{cap.su.lar}{0}
\verb{capsular}{}{}{}{}{v.t.}{Colocar algo em cápsula.}{cap.su.lar}{\verboinum{1}}
\verb{captar}{}{}{}{}{v.t.}{Fazer alguma coisa vir em sua direção; atrair.}{cap.tar}{0}
\verb{captar}{}{}{}{}{}{Receber ou registrar algo que se transmite ou se mostra; pegar. (\textit{Captou o ruído.})}{cap.tar}{0}
\verb{captar}{}{}{}{}{}{Perceber o significado de algo. (\textit{Os alunos captaram o sentido da palavra.})}{cap.tar}{0}
\verb{captar}{}{}{}{}{}{Recolher alguma coisa de vários pontos para alguma finalidade.}{cap.tar}{\verboinum{1}}
\verb{captor}{ô}{}{}{}{s.m.}{Indivíduo que captura; capturador.}{cap.tor}{0}
\verb{captura}{}{}{}{}{s.f.}{Ato ou efeito de capturar; prisão, apreensão.}{cap.tu.ra}{0}
\verb{capturar}{}{}{}{}{v.t.}{Prender, aprisionar uma pessoa ou um animal, tirando"-lhe a liberdade; pegar, deter, tomar, arrestar.}{cap.tu.rar}{\verboinum{1}}
\verb{capuchinha}{}{Bot.}{}{}{s.f.}{Trepadeira, nativa do Peru, de folhas comestíveis, e cujos frutos, depois de secos, são purgativos.}{ca.pu.chi.nha}{0}
\verb{capuchinho}{}{}{}{}{s.m.}{Capucho pequeno.}{ca.pu.chi.nho}{0}
\verb{capuchinho}{}{}{}{}{}{Religioso da Ordem de São Francisco.}{ca.pu.chi.nho}{0}
\verb{capucho}{}{}{}{}{adj.}{Diz"-se de frade da Ordem de São Francisco; franciscano.}{ca.pu.cho}{0}
\verb{capucho}{}{}{}{}{s.m.}{Esse religioso.}{ca.pu.cho}{0}
\verb{capulho}{}{Bot.}{}{}{s.m.}{Cápsula onde se forma o floco do algodão.}{ca.pu.lho}{0}
\verb{capuz}{}{}{}{}{s.m.}{Cobertura para a cabeça, geralmente presa à gola de uma peça de roupa.}{ca.puz}{0}
\verb{caquético}{}{}{}{}{adj.}{Que sofre de caquexia.}{ca.qué.ti.co}{0}
\verb{caquexia}{cs}{Med.}{}{}{s.f.}{Abatimento, fraqueza causada por desnutrição grave, câncer e outras doenças  consuntivas.}{ca.que.xi.a}{0}
\verb{cáqui}{}{}{}{}{s.m.}{Cor de barro.}{cá.qui}{0}
\verb{cáqui}{}{}{}{}{adj.}{Que tem essa cor.}{cá.qui}{0}
\verb{caqui}{}{}{}{}{s.m.}{Fruto de casca fina, polpa amarela ou vermelha, bem doce e com sementes grandes.}{ca.qui}{0}
\verb{caquizeiro}{ê}{Bot.}{}{}{s.m.}{Árvore nativa do Japão, muito cultivada no Brasil, cujo fruto é o caqui.}{ca.qui.zei.ro}{0}
\verb{cara}{}{}{}{}{s.f.}{Parte anterior da cabeça; rosto, face.}{ca.ra}{0}
\verb{cara}{}{}{}{}{}{Expressão do rosto; fisionomia, semblante. }{ca.ra}{0}
\verb{cara}{}{Fig.}{}{}{}{Aparência das pessoas ou coisas; aspecto.}{ca.ra}{0}
\verb{cara}{}{}{}{}{}{Um dos lados da moeda, oposto à coroa. }{ca.ra}{0}
\verb{cara}{}{Pop.}{}{}{s.m.}{Sujeito, indivíduo.}{ca.ra}{0}
\verb{cará}{}{Bot.}{}{}{s.m.}{Nome comum a diversas plantas rasteiras que produzem tubérculos comestíveis.}{ca.rá}{0}
\verb{cará}{}{}{}{}{}{Tubérculo comestível dessa planta.}{ca.rá}{0}
\verb{carabina}{}{}{}{}{s.f.}{Espingarda curta; fuzil.}{ca.ra.bi.na}{0}
\verb{carabineiro}{ê}{}{}{}{s.m.}{Soldado armado de carabina.}{ca.ra.bi.nei.ro}{0}
\verb{carabineiro}{ê}{}{}{}{}{Pessoa que fabrica ou vende carabinas.}{ca.ra.bi.nei.ro}{0}
\verb{caraça}{}{}{}{}{s.f.}{Cara larga e cheia; carão.}{ca.ra.ça}{0}
\verb{caraça}{}{}{}{}{}{Máscara de papelão; carranca.}{ca.ra.ça}{0}
\verb{caracará}{}{Zool.}{}{}{s.m.}{Carcará.}{ca.ra.ca.rá}{0}
\verb{caracol}{ó}{Zool.}{"-óis}{}{s.m.}{Nome comum a todos os pequenos moluscos gastrópodes, pulmonados, terrestres, providos de concha fina em forma de espiral.}{ca.ra.col}{0}
\verb{caracol}{ó}{}{"-óis}{}{}{Cacho de cabelo enrolado em espiral.}{ca.ra.col}{0}
\verb{caracol}{ó}{}{"-óis}{}{}{Espiral.}{ca.ra.col}{0}
\verb{caracolar}{}{}{}{}{v.t.}{Fazer mover em círculos, em espiral.}{ca.ra.co.lar}{0}
\verb{caracolar}{}{}{}{}{v.i.}{Fazer a cavalgadura curvetear.}{ca.ra.co.lar}{\verboinum{1}}
\verb{caractere}{é}{}{}{}{s.m.}{Traço que distingue uma pessoa, animal ou coisa; característica, marca, peculiaridade. }{ca.rac.te.re}{0}
\verb{caractere}{é}{}{}{}{}{Qualquer número, letra do alfabeto, ou símbolo usados graficamente.}{ca.rac.te.re}{0}
\verb{característica}{}{}{}{}{s.f.}{Aquilo que caracteriza, que particulariza; peculiaridade, particularidade.}{ca.rac.te.rís.ti.ca}{0}
\verb{característico}{}{}{}{}{adj.}{Que caracteriza, que distingue uma pessoa ou coisa; peculiar.}{ca.rac.te.rís.ti.co}{0}
\verb{caracterizador}{ô}{}{}{}{adj.}{Que caracteriza ou serve para caracterizar; caracterizante.  }{ca.rac.te.ri.za.dor}{0}
\verb{caracterizador}{ô}{}{}{}{s.m.}{Indivíduo que faz caracterizações.}{ca.rac.te.ri.za.dor}{0}
\verb{caracterizar}{}{}{}{}{v.t.}{Descrever, apontando as propriedades do caráter, as propriedades características.}{ca.rac.te.ri.zar}{0}
\verb{caracterizar}{}{}{}{}{}{Vestir e pintar o ator para compor a personagem que representará em cena. }{ca.rac.te.ri.zar}{\verboinum{1}}
\verb{caracu}{}{}{}{}{s.m.}{Raça de boi de pelo curto e avermelhado.}{ca.ra.cu}{0}
\verb{caracu}{}{}{}{}{}{Tutano; medula dos ossos do boi.}{ca.ra.cu}{0}
\verb{cara"-de"-pau}{}{Pejor.}{caras"-de"-pau}{}{adj.2g.}{Caradura, descarado, sem"-vergonha, cínico.}{ca.ra"-de"-pau}{0}
\verb{caradura}{}{}{}{}{adj.2g.}{Diz"-se de pessoa descarada, sem"-vergonha.}{ca.ra.du.ra}{0}
\verb{caradurismo}{}{}{}{}{s.m.}{Qualidade de quem é caradura.}{ca.ra.du.ris.mo}{0}
\verb{caraguatá}{}{Bot.}{}{}{s.m.}{Nome de várias plantas ornamentais; gravatá.}{ca.ra.gua.tá}{0}
\verb{caraíba}{}{Bras.}{}{}{s.m.}{Entre os índios do século \textsc{xvi}, de língua tupi, feiticeiro indígena; pajé.}{ca.ra.í.ba}{0}
\verb{caraíba}{}{}{}{}{}{Denominação que os índios do século \textsc{xvi}, de língua tupi, davam aos homens brancos, aos europeus.}{ca.ra.í.ba}{0}
%\verb{}{}{}{}{}{}{}{}{0}
\verb{caralho}{}{Chul.}{}{}{s.m.}{O pênis.}{ca.ra.lho}{0}
\verb{caralho}{}{}{}{}{}{Palavra usada na expressão \textit{pra caralho}: em grande quantidade ou intensidade, muito, à beça.}{ca.ra.lho}{0}
\verb{caralho}{}{}{}{}{interj.}{Expressa surpresa, admiração, entusiasmo.}{ca.ra.lho}{0}
\verb{caralho}{}{}{}{}{}{Expressa raiva, indignação.}{ca.ra.lho}{0}
\verb{caramanchão}{}{}{"-ões}{}{s.m.}{Estrutura simples e leve, usada em parques e jardins, construída com ripas ou estacas entrelaçadas, na qual se apoiam trepadeiras, que formam cobertura sombreada.}{ca.ra.man.chão}{0}
\verb{caramba}{}{}{}{}{interj.}{Expressa admiração, espanto, surpresa ou ironia.}{ca.ram.ba}{0}
\verb{carambola}{ó}{}{}{}{s.f.}{Fruto da caramboleira.}{ca.ram.bo.la}{0}
\verb{carambola}{ó}{}{}{}{}{Batida de bola de bilhar sucessivamente em outras duas.}{ca.ram.bo.la}{0}
\verb{carambolar}{}{}{}{}{v.i.}{Fazer carambola no bilhar.}{ca.ram.bo.lar}{\verboinum{1}}
\verb{caramboleira}{ê}{Bot.}{}{}{s.f.}{Árvore cujo fruto, comestível, é a carambola.}{ca.ram.bo.lei.ra}{0}
\verb{caramboleiro}{ê}{Bot.}{}{}{s.m.}{Caramboleira.}{ca.ram.bo.lei.ro}{0}
\verb{caramboleiro}{ê}{Fig.}{}{}{adj.}{Que faz carambola.}{ca.ram.bo.lei.ro}{0}
%\verb{}{}{}{}{}{}{}{}{0}
\verb{caramelar}{}{Cul.}{}{}{v.t.}{Cobrir com caramelo.}{ca.ra.me.lar}{\verboinum{1}}
\verb{caramelo}{é}{Cul.}{}{}{s.m.}{Calda de açúcar queimado, usada em doces, como cobertura de pudins etc.}{ca.ra.me.lo}{0}
\verb{caramelo}{é}{}{}{}{}{Bala feita com essa calda.}{ca.ra.me.lo}{0}
\verb{cara"-metade}{}{Pop.}{caras"-metades}{}{s.f.}{Parceiro com quem se tem mais afinidades, amorosas ou não.}{ca.ra"-me.ta.de}{0}
\verb{caraminguá}{}{Pop.}{}{}{s.m.}{Dinheiro.}{ca.ra.min.guá}{0}
\verb{caraminguás}{}{Pop.}{}{}{s.m.pl.}{Cacarecos.}{ca.ra.min.guás}{0}
\verb{caraminhola}{ó}{Pop.}{}{}{s.f.}{Cabeleira desalinhada; grenha.}{ca.ra.mi.nho.la}{0}
\verb{caraminhola}{ó}{Fig.}{}{}{}{Intriga, mexerico, mentira.}{ca.ra.mi.nho.la}{0}
\verb{caraminhola}{ó}{}{}{}{}{Fantasia, sonho, imaginação.}{ca.ra.mi.nho.la}{0}
\verb{caramujo}{}{Zool.}{}{}{s.m.}{Nome comum aos moluscos gastrópodes, que vivem na água doce ou salgada, e cuja concha é mais pesada e resistente do que a dos caracóis terrestres.}{ca.ra.mu.jo}{0}
\verb{caramunha}{}{}{}{}{s.f.}{Choro continuado das crianças.}{ca.ra.mu.nha}{0}
\verb{caramunha}{}{}{}{}{}{Careta de criança quando chora.}{ca.ra.mu.nha}{0}
\verb{carancho}{}{Zool.}{}{}{s.m.}{Carcará.}{ca.ran.cho}{0}
\verb{caranga}{}{Pop.}{}{}{s.f.}{Carango, automóvel.}{ca.ran.ga}{0}
\verb{carango}{}{Pop.}{}{}{s.m.}{Qualquer carro; automóvel, caranga.}{ca.ran.go}{0}
\verb{carango}{}{Zool.}{}{}{}{Inseto anopluro, que vive principalmente na região pubiana, cujos ovos são postos na base dos pelos; chato.}{ca.ran.go}{0}
\verb{caranguejeira}{ê}{Zool.}{}{}{s.f.}{Forma reduzida de \textit{aranha"-caranguejeira}.}{ca.ran.gue.jei.ra}{0}
\verb{caranguejo}{ê}{Zool.}{}{}{s.m.}{Nome comum aos crustáceos com cinco pares de patas terminadas em pinças, que vivem em água doce ou salgada, em tocas, e que se alimentam de restos orgânicos.}{ca.ran.gue.jo}{0}
\verb{caranguejola}{ó}{Zool.}{}{}{s.f.}{Grande crustáceo, parecido com o caranguejo, encontrado na costa atlântica da Europa, e que se pesca com alçapão. }{ca.ran.gue.jo.la}{0}
\verb{carantonha}{}{}{}{}{s.f.}{Cara feia; cara fechada; caraça, esgar, carranca.}{ca.ran.to.nha}{0}
\verb{carão}{}{}{"-ões}{}{s.m.}{Cara grande ou feia; carranca, caraça.}{ca.rão}{0}
\verb{carão}{}{Pop.}{"-ões}{}{}{Repreensão, pito, ralho ou reprimenda sofridos.}{ca.rão}{0}
\verb{carapaça}{}{}{}{}{s.f.}{Casca dura, resistente, que protege o corpo de certos animais, como o tatu, a tartaruga etc.}{ca.ra.pa.ça}{0}
\verb{carapanã}{}{Zool.}{}{}{s.m.}{Mosquito. }{ca.ra.pa.nã}{0}
\verb{carapau}{}{Zool.}{}{}{s.m.}{Peixe marinho, encontrado no Atlântico.}{ca.ra.pau}{0}
\verb{carapeta}{ê}{}{}{}{s.f.}{Peão pequeno que se faz rodopiar com os dedos.}{ca.ra.pe.ta}{0}
\verb{carapetão}{}{}{ões}{}{s.m.}{Grande mentira, mentira deslavada; patranha, balão, peta.}{ca.ra.pe.tão}{0}
%\verb{}{}{}{}{}{}{}{}{0}
\verb{carapina}{}{Bras.}{}{}{s.m.}{Carpinteiro.}{ca.ra.pi.na}{0}
\verb{carapinha}{}{}{}{}{s.f.}{Cabelo crespo dos negros; pixaim.}{ca.ra.pi.nha}{0}
\verb{carapinhada}{}{}{}{}{s.f.}{Bebida congelada, em flocos de gelo, feita de xarope ou suco de fruta.}{ca.ra.pi.nha.da}{0}
\verb{cara"-pintada}{}{Pop.}{caras"-pintadas}{}{s.2g.}{Jovem que participa de uma manifestação ou passeata de protesto com o rosto pintado.}{ca.ra"-pin.ta.da}{0}
\verb{carapuça}{}{}{}{}{s.f.}{Tipo de gorro, em forma de funil, que se ajusta à cabeça.}{ca.ra.pu.ça}{0}
\verb{caratê}{}{Esport.}{}{}{s.m.}{Tipo de luta marcial, difundida pelos japoneses, em que se usam apenas as mãos e os pés.}{ca.ra.tê}{0}
\verb{caráter}{}{}{}{}{s.m.}{Característica particular que distingue algo ou alguém; marca, cunho, especificidade. }{ca.rá.ter}{0}
\verb{caráter}{}{}{}{}{}{Conjunto de traços psicológicos e morais que caracterizam uma pessoa; índole, personalidade.}{ca.rá.ter}{0}
\verb{caraterística}{}{}{}{}{}{Var. de \textit{característica}.}{ca.ra.te.rís.ti.ca}{0}
\verb{caraterístico}{}{}{}{}{}{Var. de \textit{característico}.}{ca.ra.te.rís.ti.co}{0}
\verb{caraterizar}{}{}{}{}{}{Var. de \textit{caracterizar}.}{ca.ra.te.ri.zar}{\verboinum{1}}
\verb{caratinga}{}{Zool.}{}{}{s.m.}{Peixe de coloração prateada e estrias esverdeadas, encontrado principalmente no litoral do Sudeste brasileiro e cuja carne é utilizada como alimento.}{ca.ra.tin.ga}{0}
\verb{caratinga}{}{Bot.}{}{}{}{Nome comum a várias espécies de trepadeiras, com folhas verde"-escuras e manchas vermelhas, e cujo tubérculo é comestível; cará.}{ca.ra.tin.ga}{0}
\verb{caravana}{}{}{}{}{s.f.}{Grupo de viajantes ou peregrinos que viajam juntos pelo deserto ou por regiões pouco seguras.}{ca.ra.va.na}{0}
\verb{caravana}{}{Por ext.}{}{}{}{Conjunto de veículos que viajam juntos.}{ca.ra.va.na}{0}
\verb{caravançará}{}{}{}{}{s.m.}{No Oriente Médio, estalagem ou hospedaria para abrigar caravanas.}{ca.ra.van.ça.rá}{0}
\verb{caravançará}{}{Fig.}{}{}{}{Confusão, mistura, desordem.}{ca.ra.van.ça.rá}{0}
\verb{caravaneiro}{ê}{}{}{}{s.m.}{Indivíduo que conduz caravanas; guia.}{ca.ra.va.nei.ro}{0}
\verb{caravela}{é}{}{}{}{s.f.}{Antiga embarcação de tonelagem média, rápida, movida a vela.}{ca.ra.ve.la}{0}
\verb{caravela}{é}{Zool.}{}{}{}{Nome comum dado a animais celenterados que vivem em colônias, possuem um órgão flutuador e cujos tentáculos são urticantes; água"-viva.}{ca.ra.ve.la}{0}
\verb{caraveleiro}{ê}{}{}{}{s.m.}{Indivíduo que fazia parte da tripulação das caravelas.}{ca.ra.ve.lei.ro}{0}
\verb{carbeto}{ê}{Quím.}{}{}{s.m.}{Qualquer composto binário do carbono com outro elemento; carboneto, carbureto.}{car.be.to}{0}
\verb{carboidrato}{}{Bioquím.}{}{}{s.m.}{Cada um dos compostos orgânicos de carbono, hidrogênio e oxigênio, como o açúcar, o amido, a celulose etc.; glicídio.}{car.bo.i.dra.to}{0}
\verb{carbonato}{}{Quím.}{}{}{s.m.}{Sal, éster ou ânion derivado do ácido carbônico.}{car.bo.na.to}{0}
\verb{carboneto}{ê}{Quím.}{}{}{s.m.}{Carbeto.}{car.bo.ne.to}{0}
\verb{carbônico}{}{Quím.}{}{}{adj.}{Diz"-se de qualquer estrutura química composta de átomos de carbono.}{car.bô.ni.co}{0}
\verb{carbonífero}{}{}{}{}{adj.}{Que tem ou produz carvão.}{car.bo.ní.fe.ro}{0}
\verb{carbonífero}{}{Geol.}{}{}{}{Diz"-se do período da era paleozoica compreendido entre o devoniano e o permiano, aproximadamente entre 360 e 290 milhões de anos.}{car.bo.ní.fe.ro}{0}
\verb{carbonizar}{}{}{}{}{v.t.}{Reduzir a carvão ou a resíduo de carvão; queimar completamente; calcinar.}{car.bo.ni.zar}{\verboinum{1}}
\verb{carbono}{}{Quím.}{}{}{s.m.}{Elemento químico do grupo dos não metais, que constitui um componente essencial dos compostos orgânicos, usado sob a forma de diamante em joias, grafites, na siderurgia e na metalurgia. \elemento{6}{12.0107}{C}.}{car.bo.no}{0}
\verb{carbúnculo}{}{Med.}{}{}{s.m.}{Infecção extensa e profunda da pele que produz lesão com secreção purulenta, causada por um bacilo.}{car.bún.cu.lo}{0}
\verb{carburação}{}{}{"-ões}{}{s.f.}{Ato ou efeito de carburar.}{car.bu.ra.ção}{0}
\verb{carburação}{}{}{"-ões}{}{}{Processo de misturar vapores do combustível com o ar para provocar combustão em motores de explosão.}{car.bu.ra.ção}{0}
\verb{carburador}{ô}{}{}{}{s.m.}{Dispositivo que mistura combustível com o ar, para alimentar os motores de combustão interna de ignição por faísca.}{car.bu.ra.dor}{0}
\verb{carburante}{}{}{}{}{adj.2g.}{Que produz carburação.}{car.bu.ran.te}{0}
\verb{carburante}{}{}{}{}{s.m.}{Produto ou combustível usado nos motores a explosão.}{car.bu.ran.te}{0}
\verb{carburar}{}{}{}{}{v.t.}{Misturar os vapores do combustível com o ar, em proporções adequadas à combustão.}{car.bu.rar}{\verboinum{1}}
\verb{carbureto}{ê}{Quím.}{}{}{s.m.}{Carbeto.}{car.bu.re.to}{0}
\verb{carcaça}{}{}{}{}{s.f.}{Esqueleto de animal; ossada, arcabouço.}{car.ca.ça}{0}
\verb{carcaça}{}{Fig.}{}{}{}{Casco velho de navio arruinado, lançado à costa pelo mar.}{car.ca.ça}{0}
\verb{carcaça}{}{Fig.}{}{}{}{Mulher velha, magra e feia; bruxa.}{car.ca.ça}{0}
\verb{carcamano}{}{Pop.}{}{}{s.m.}{Forma pejorativa que se dá aos italianos.}{car.ca.ma.no}{0}
\verb{carcará}{}{Zool.}{}{}{s.m.}{Ave de plumagem alvinegra com cabeça branca e penacho preto na nuca, que habita regiões abertas desde o sul dos Estados Unidos até o sul da Argentina; caracará, carancho.}{car.ca.rá}{0}
\verb{carcás}{}{}{}{}{s.m.}{Estojo sem tampa em que se guardavam as flechas, sendo carregado nas costas, pendente do ombro; aljava.}{car.cás}{0}
\verb{carceragem}{}{}{ ⟨"-ens⟩}{}{s.f.}{Ato ou efeito de encarcerar, prender; prisão.}{car.ce.ra.gem}{0}
\verb{carceragem}{}{}{ ⟨"-ens⟩}{}{}{Lugar, na delegacia, onde ficam os detentos.}{car.ce.ra.gem}{0}
\verb{carceragem}{}{}{ ⟨"-ens⟩}{}{}{Despesas com a manutenção dos presos.}{car.ce.ra.gem}{0}
\verb{carcerário}{}{}{}{}{adj.}{Relativo a cárcere, prisão.}{car.ce.rá.rio}{0}
\verb{cárcere}{}{}{}{}{s.m.}{Cela de cadeia; calabouço, prisão.}{cár.ce.re}{0}
\verb{carcereiro}{ê}{}{}{}{s.m.}{Funcionário responsável pela guarda dos presos.}{car.ce.rei.ro}{0}
\verb{carcinoma}{}{Med.}{}{}{s.m.}{Tumor maligno, de origem epitelial ou glandular, que tende a invadir as estruturas próximas e a provocar metástases; câncer.}{car.ci.no.ma}{0}
\verb{carcinose}{ó}{Med.}{}{}{s.f.}{Disseminação de um câncer pelo organismo, por metástases múltiplas.}{car.ci.no.se}{0}
\verb{carcoma}{}{}{}{}{s.f.}{Aquilo que devora, consome, arruína; podridão.}{car.co.ma}{0}
\verb{carcoma}{}{Zool.}{}{}{}{Caruncho.}{car.co.ma}{0}
\verb{carcoma}{}{}{}{}{}{Pó de madeira roída por caruncho.}{car.co.ma}{0}
\verb{carcomer}{ê}{}{}{}{v.t.}{Roer a madeira; reduzir a pó.}{car.co.mer}{0}
\verb{carcomer}{ê}{}{}{}{}{Provocar a ruína; destruir.}{car.co.mer}{\verboinum{12}}
\verb{carcomido}{}{}{}{}{adj.}{Roído por carcoma; corroído, carunchado.}{car.co.mi.do}{0}
\verb{carcomido}{}{}{}{}{}{Arruinado, destruído, apodrecido.}{car.co.mi.do}{0}
\verb{carda}{}{}{}{}{s.f.}{Ato ou efeito de cardar, desembaraçar.}{car.da}{0}
\verb{carda}{}{}{}{}{}{Máquina que desembaraça, destrinça e limpa as fibras têxteis.}{car.da}{0}
\verb{cardamomo}{ô}{Bot.}{}{}{s.m.}{Erva nativa da Índia, cujas sementes são utilizadas como condimento e das quais se extrai um óleo para aromatizar bebidas.}{car.da.mo.mo}{0}
\verb{cardápio}{}{}{}{}{s.m.}{Em um restaurante, lista dos pratos servidos com os respectivos preços.}{car.dá.pio}{0}
\verb{cardar}{}{}{}{}{v.t.}{Pentear ou desenredar a lã ou outro tipo de têxtil com a carda.}{car.dar}{0}
\verb{cardar}{}{Fig.}{}{}{}{Extorquir através de artimanhas; explorar.}{car.dar}{\verboinum{1}}
\verb{cardeal}{}{Zool.}{"-ais}{}{s.m.}{Ave de coloração vermelha e longo topete na cabeça com ocorrência no Brasil, na Argentina e no Paraguai.}{car.de.al}{0}
\verb{cardeal}{}{}{"-ais}{}{adj.2g.}{Principal, fundamental, cardinal.}{car.de.al}{0}
\verb{cardeal}{}{}{"-ais}{}{s.m.}{Na Igreja Católica, membro do Sacro Colégio, eleitor e seu mais próximo colaborador.}{car.de.al}{0}
\verb{cárdia}{}{Anat.}{}{}{s.f.}{Abertura superior do estômago que se comunica com o esôfago.}{cár.dia}{0}
\verb{cardíaco}{}{}{}{}{adj.}{Relativo ao coração.}{car.dí.a.co}{0}
\verb{cardíaco}{}{}{}{}{s.m.}{Pessoa que sofre de problemas do coração; cardiopata.}{car.dí.a.co}{0}
\verb{cardial}{}{}{"-ais}{}{adj.2g.}{Relativo à cárdia.}{car.di.al}{0}
\verb{cardigã}{}{}{}{}{s.m.}{Casaco ou suéter de malha, com decote redondo ou em \textsc{v}, abotoado até o pescoço ou aberto de cima a baixo.}{car.di.gã}{0}
\verb{cardinal}{}{}{"-ais}{}{adj.2g.}{Principal, fundamental, cardeal.}{car.di.nal}{0}
\verb{cardinal}{}{Mat.}{"-ais}{}{}{Diz"-se do número real, inteiro, que indica quantidade.}{car.di.nal}{0}
\verb{cardinal}{}{Gram.}{"-ais}{}{}{Diz"-se do numeral que corresponde ao número de elementos de um conjunto.}{car.di.nal}{0}
\verb{cardinalato}{}{}{}{}{s.m.}{Dignidade de cardeal.}{car.di.na.la.to}{0}
\verb{cardinalício}{}{}{}{}{adj.}{Relativo a cardeal ou a cardinalato.}{car.di.na.lí.cio}{0}
\verb{cardiografia}{}{}{}{}{s.f.}{Registro gráfico dos movimentos cardíacos por meio de um cardiógrafo.}{car.di.o.gra.fi.a}{0}
\verb{cardiógrafo}{}{}{}{}{s.m.}{Aparelho que faz o registro gráfico dos movimentos cardíacos.}{car.di.ó.gra.fo}{0}
\verb{cardiograma}{}{}{}{}{s.m.}{Traçado de registro gráfico dos movimentos cardíacos; eletrocardiograma.}{car.di.o.gra.ma}{0}
\verb{cardiologia}{}{Med.}{}{}{s.f.}{Ramo da medicina que trata do estudo do coração, dos vasos sanguíneos e de suas respectivas doenças.}{car.di.o.lo.gi.a}{0}
\verb{cardiologista}{}{Med.}{}{}{s.2g.}{Médico especializado no estudo e tratamento do coração, dos vasos e de suas afecções; cardiólogo.}{car.di.o.lo.gis.ta}{0}
\verb{cardiólogo}{}{Med.}{}{}{s.m.}{Cardiologista.}{car.di.ó.lo.go}{0}
\verb{cardiopata}{}{Med.}{}{}{s.2g.}{Pessoa que sofre de problemas do coração; cardíaco.}{car.di.o.pa.ta}{0}
\verb{cardiopatia}{}{Med.}{}{}{s.f.}{Designação comum de qualquer doença do coração.}{car.di.o.pa.ti.a}{0}
\verb{cardiovascular}{}{Anat.}{}{}{adj.2g.}{Relativo tanto ao coração quanto aos vasos sanguíneos.}{car.di.o.vas.cu.lar}{0}
\verb{cardo}{}{Bot.}{}{}{s.m.}{Planta de flores amarelas, folhas com espinhos e caule ereto, cheio de pelos.}{car.do}{0}
\verb{cardume}{}{}{}{}{s.m.}{Coletivo de peixes.}{car.du.me}{0}
\verb{cardume}{}{Fig.}{}{}{}{Multidão de pessoas; bando, aglomeração.}{car.du.me}{0}
\verb{cardume}{}{Fig.}{}{}{}{Ajuntamento de coisas, montão.}{car.du.me}{0}
\verb{careca}{é}{}{}{}{s.f.}{Calva, calvície.}{ca.re.ca}{0}
\verb{careca}{é}{}{}{}{adj.}{Diz"-se do indivíduo calvo, sem pelos na cabeça.}{ca.re.ca}{0}
\verb{careca}{é}{}{}{}{}{Diz"-se do pneu liso, cujos frisos estão gastos pelo uso.}{ca.re.ca}{0}
\verb{carecente}{}{}{}{}{adj.2g.}{Que não possui nada; despossuído, carente.}{ca.re.cen.te}{0}
\verb{carecer}{ê}{}{}{}{v.t.}{Ter necessidade; precisar.}{ca.re.cer}{0}
\verb{carecer}{ê}{}{}{}{}{Não possuir, não ter.}{ca.re.cer}{\verboinum{15}}
\verb{careiro}{ê}{}{}{}{adj.}{Que vende ou cobra preço elevado por algo.}{ca.rei.ro}{0}
\verb{carena}{ê}{}{}{}{s.f.}{Parte do casco do navio que fica submersa; quilha.}{ca.re.na}{0}
\verb{carena}{ê}{Veter.}{}{}{}{Crista em forma de quilha de certos ossos, como no esterno das aves.}{ca.re.na}{0}
\verb{carência}{}{}{}{}{s.f.}{Ausência de algo necessário; privação, falta.}{ca.rên.cia}{0}
\verb{carência}{}{}{}{}{}{Ausência de laços afetivos.}{ca.rên.cia}{0}
\verb{carência}{}{Jur.}{}{}{}{Prazo estabelecido em contrato, geralmente de plano de saúde, em que o segurado não pode usufruir de certas vantagens do conjunto do plano.}{ca.rên.cia}{0}
\verb{carente}{}{}{}{}{adj.2g.}{Que carece, que não tem; despossuído, necessitado, carecente.}{ca.ren.te}{0}
\verb{carepa}{é}{}{}{}{s.f.}{Pequenas escamas que se formam na pele, principalmente no couro cabeludo; caspa.}{ca.re.pa}{0}
\verb{carepa}{é}{}{}{}{}{Pó que aparece na superfície das frutas secas.}{ca.re.pa}{0}
\verb{carepa}{é}{}{}{}{}{Superfície áspera da madeira desbastada.}{ca.re.pa}{0}
\verb{carestia}{}{}{}{}{s.f.}{Aumento dos preços, acima do valor real.}{ca.res.ti.a}{0}
\verb{carestia}{}{}{}{}{}{Encarecimento do custo de vida.}{ca.res.ti.a}{0}
\verb{carestia}{}{}{}{}{}{Escassez, falta de alimentos ou de outro produto.}{ca.res.ti.a}{0}
\verb{careta}{ê}{}{}{}{s.f.}{Contração do rosto; trejeito, esgar.}{ca.re.ta}{0}
\verb{careta}{ê}{Pop.}{}{}{adj.2g.}{Que é conservador, tradicional, quadrado.}{ca.re.ta}{0}
\verb{careta}{ê}{Pop.}{}{}{}{Que não usa drogas.}{ca.re.ta}{0}
\verb{caretear}{}{}{}{}{v.i.}{Fazer caretas.}{ca.re.te.ar}{\verboinum{4}}
\verb{careteiro}{ê}{}{}{}{adj.}{Que faz caretas ou trejeitos.}{ca.re.tei.ro}{0}
\verb{caretice}{}{Pop.}{}{}{s.f.}{Comportamento ou dito de quem é conservador ou de quem não usa drogas.}{ca.re.ti.ce}{0}
\verb{carga}{}{}{}{}{s.f.}{Conjunto de coisas que um veículo ou um animal transporta; carregamento.}{car.ga}{0}
\verb{carga}{}{}{}{}{}{Coisa com que se carrega aparelho ou instrumeto. (\textit{A bateria perde rapidamente a carga.})}{car.ga}{0}
\verb{carga}{}{}{}{}{}{Conjunto de dificuldades que devem ser vencidas; fardo, peso.}{car.ga}{0}
\verb{carga"-d'água}{}{}{cargas"-d'água}{}{s.f.}{Chuva forte; tromba"-d'água, bátega.}{car.ga"-d'á.gua}{0}
\verb{cargas"-d'água}{}{Pop.}{}{}{s.f.pl.}{Motivo inexplicado, oculto; razão misteriosa para um acontecimento. (\textit{Gostaria de saber por que cargas"-d'água você não vem.})}{car.gas"-d'á.gua}{0}
\verb{cargo}{}{}{}{}{s.m.}{Função de uma pessoa que trabalha em uma organização. (\textit{Sua competência fez com que fosse promovido para um cargo superior.})}{car.go}{0}
\verb{cargueiro}{ê}{}{}{}{adj.}{Que transporta carga.}{car.guei.ro}{0}
\verb{cargueiro}{ê}{}{}{}{s.m.}{Navio de carga.}{car.guei.ro}{0}
\verb{cariar}{}{}{}{}{v.t.}{Produzir cárie.  }{ca.ri.ar}{0}
\verb{cariar}{}{}{}{}{v.i.}{Criar cárie.}{ca.ri.ar}{\verboinum{1}}
\verb{cariátide}{}{}{}{}{s.f.}{Estátua humana, geralmente feminina, usada como coluna na Grécia antiga.}{ca.ri.á.ti.de}{0}
\verb{caribé}{}{Cul.}{}{}{s.m.}{Prato preparado com polpa de abacate.  	 }{ca.ri.bé}{0}
\verb{caribé}{}{}{}{}{}{Refresco de beiju de tapioca.  }{ca.ri.bé}{0}
\verb{caribé}{}{}{}{}{}{Mingau de farinha fina.  }{ca.ri.bé}{0}
\verb{caribe}{}{}{}{}{s.m.}{Família linguística que abrange cerca de 20 línguas vivas, distribuída, no Brasil, por Roraima, Amapá, Norte do Pará e do Amazonas e Mato Grosso, e compreendendo também a Guiana Francesa, Guiana e Venezuela.}{ca.ri.be}{0}
\verb{caribe}{}{}{}{}{s.2g.}{Indivíduo pertencente a qualquer dos povos Caribe.}{ca.ri.be}{0}
\verb{caribenho}{}{}{}{}{s.m.}{Indivíduo natural ou habitante dos países do mar do Caribe.}{ca.ri.be.nho}{0}
\verb{caribenho}{}{}{}{}{adj.}{Relativo ou pertencente aos países do mar do Caribe.}{ca.ri.be.nho}{0}
\verb{cariboca}{ó}{Bras.}{}{}{s.2g.}{Caboclo.}{ca.ri.bo.ca}{0}
\verb{caricato}{}{}{}{}{adj.}{Que é semelhante a uma caricatura; ridículo, burlesco, grotesco, caricaturesco.   }{ca.ri.ca.to}{0}
\verb{caricato}{}{}{}{}{adj.}{Diz"-se de ator cômico que interpreta papéis grotescos, ridículos.}{ca.ri.ca.to}{0}
\verb{caricatura}{}{}{}{}{s.f.}{Desenho que exagera os traços, as características de alguém ou de alguma coisa.}{ca.ri.ca.tu.ra}{0}
\verb{caricatura}{}{Fig.}{}{}{}{Pessoa ridícula por sua aparência ou pelos seus modos.}{ca.ri.ca.tu.ra}{0}
\verb{caricaturar}{}{}{}{}{v.t.}{Fazer caricaturas.}{ca.ri.ca.tu.rar}{\verboinum{1}}
\verb{caricaturista}{}{}{}{}{s.2g.}{Pessoa cuja profissão é fazer caricaturas.}{ca.ri.ca.tu.ris.ta}{0}
\verb{carícia}{}{}{}{}{s.f.}{Qualquer demonstração física de afeto ou de amor carnal; afago, carinho, meiguice.  }{ca.rí.cia}{0}
\verb{caridade}{}{}{}{}{s.f.}{Ato ou efeito de fazer o bem, ajudando ou dando algo a alguém sem qualquer interesse ou recompensa; benevolência, complacência, compaixão.}{ca.ri.da.de}{0}
\verb{caridade}{}{}{}{}{}{Ato de ajudar alguém necessitado.}{ca.ri.da.de}{0}
\verb{caridoso}{ô}{}{}{}{adj.}{Que faz ou demonstra caridade; caritativo.}{ca.ri.do.so}{0}
\verb{cárie}{}{Med.}{}{}{s.f.}{Pequeno furo no dente, formado pelos ácidos que as bactérias liberam ao se alimentarem dos restos de comida que ficam nos dentes quando estes não são devidamente escovados. }{cá.rie}{0}
\verb{carijó}{}{}{}{}{adj.}{Diz"-se de galináceo que tem penas malhadas de branco e preto; pedrês.}{ca.ri.jó}{0}
\verb{caril}{}{Cul.}{}{}{s.m.}{Condimento de origem indiana, picante, preparado com açafrão, pimenta, cúrcuma, entre outras especiarias pulverizadas, ou o molho obtido com essa base, usado em inúmeros pratos.}{ca.ril}{0}
\verb{carimã}{}{Bras.}{}{}{s.2g.}{Massa mole e azeda de mandioca, com a qual se fazem bolos.}{ca.ri.mã}{0}
\verb{carimbar}{}{}{}{}{v.t.}{Marcar com carimbo.}{ca.rim.bar}{\verboinum{1}}
\verb{carimbo}{}{}{}{}{s.m.}{Instrumento de madeira, metal ou borracha, que apresenta uma placa gravada com letras, números, figuras, símbolos etc., em relevo, usado para identificar, marcar, autenticar, à tinta, documentos, cartas, livros etc.; sinete. }{ca.rim.bo}{0}
\verb{carimbo}{}{}{}{}{}{Marca, impressão feita por esse instrumento.}{ca.rim.bo}{0}
\verb{carinho}{}{}{}{}{s.m.}{Sentimento de amor ou amizade por alguém; meiguice, afago.}{ca.ri.nho}{0}
\verb{carinho}{}{}{}{}{}{Cuidado, desvelo.}{ca.ri.nho}{0}
\verb{carinhoso}{ô}{}{"-osos ⟨ó⟩}{"-osa ⟨ó⟩}{adj.}{Que tem ou faz carinho; terno, meigo.}{ca.ri.nho.so}{0}
\verb{carioca}{ó}{}{}{}{adj.2g.}{Relativo à cidade do Rio de Janeiro, capital do estado do Rio de Janeiro.}{ca.ri.o.ca}{0}
\verb{carioca}{ó}{}{}{}{}{Diz"-se do café já preparado, ao qual se adiciona água para que fique mais fraco.}{ca.ri.o.ca}{0}
\verb{carioca}{ó}{}{}{}{s.2g.}{Indivíduo natural ou habitante da cidade do Rio de Janeiro.}{ca.ri.o.ca}{0}
\verb{carisma}{}{}{}{}{s.m.}{Qualidade que algumas pessoas têm de seduzir, fascinar ou liderar sem dificuldades um grupo de pessoas.}{ca.ris.ma}{0}
\verb{carisma}{}{Relig.}{}{}{}{Dom da graça divina, atribuído a uma pessoa.}{ca.ris.ma}{0}
\verb{carismático}{}{}{}{}{adj.}{Que tem carisma.}{ca.ris.má.ti.co}{0}
\verb{caritativo}{}{}{}{}{adj.}{Caridoso.}{ca.ri.ta.ti.vo}{0}
\verb{carlinga}{}{}{}{}{s.f.}{Nas aeronaves, espaço onde se acomoda o piloto; cabina.}{car.lin.ga}{0}
\verb{carma}{}{Relig.}{}{}{s.m.}{Nas religiões hinduísta e budista, dogma segundo o qual o destino de uma pessoa é determinado pela totalidade de suas ações passadas, em vidas (encarnações) anteriores. }{car.ma}{0}
\verb{carme}{}{}{}{}{s.m.}{Poema, verso, canto.   }{car.me}{0}
\verb{carme}{}{Mús.}{}{}{}{Nos séculos \textsc{xiv} e \textsc{xv}, a voz (a parte) mais aguda de uma composição.}{car.me}{0}
\verb{carmelita}{}{Relig.}{}{}{s.2g.}{Frade ou freira pertencente à ordem religiosa de Nossa Senhora do Carmo; carmelitano.}{car.me.li.ta}{0}
\verb{carmelita}{}{}{}{}{adj.2g.}{Relativo ou pertencente a essa ordem.}{car.me.li.ta}{0}
\verb{carmesim}{}{}{}{}{s.m.}{A cor vermelha viva do carmim; carmim, escarlate.}{car.me.sim}{0}
\verb{carmesim}{}{}{}{}{adj.2g.}{Que tem a cor semelhante à do carmim.}{car.me.sim}{0}
\verb{carmim}{}{}{}{}{adj.2g.}{Que tem cor vermelha viva e intensa; carmesim, rubro.}{car.mim}{0}
\verb{carmim}{}{}{}{}{s.m.}{Essa cor.}{car.mim}{0}
\verb{carmona}{}{}{}{}{s.f.}{Cremona.}{car.mo.na}{0}
\verb{carnação}{}{}{"-ões}{}{s.f.}{Cor da pele humana.}{car.na.ção}{0}
\verb{carnação}{}{}{"-ões}{}{}{Representação do corpo humano nu, em sua cor natural.}{car.na.ção}{0}
\verb{carnadura}{}{}{}{}{s.f.}{Aspecto, aparência exterior do corpo humano.}{car.na.du.ra}{0}
\verb{carnadura}{}{}{}{}{}{Constituição física; musculatura, compleição.}{car.na.du.ra}{0}
\verb{carnadura}{}{}{}{}{}{Parte carnuda do corpo.}{car.na.du.ra}{0}
\verb{carnal}{}{}{"-ais}{}{adj.2g.}{Relativo a ou próprio da carne.}{car.nal}{0}
\verb{carnal}{}{}{"-ais}{}{}{Relativo ao corpo, por oposição ao que é espiritual.}{car.nal}{0}
\verb{carnal}{}{}{"-ais}{}{}{Lascivo, concupiscente, sensual.}{car.nal}{0}
\verb{carnal}{}{}{"-ais}{}{}{Consanguíneo.}{car.nal}{0}
\verb{carnaúba}{}{}{}{}{s.f.}{Carnaubeira.}{car.na.ú.ba}{0}
\verb{carnaúba}{}{}{}{}{}{A cera obtida das folhas dessa palmeira.}{car.na.ú.ba}{0}
\verb{carnaubal}{}{}{"-ais}{}{s.m.}{Grande quantidade de carnaúbas próximas umas das outras.}{car.na.u.bal}{0}
\verb{carnaubeira}{ê}{Bot.}{}{}{s.f.}{Palmeira que pode atingir 40 m de altura, de estipe reto, nativa do Brasil, cujas folhas fornecem cera; carnaúba.}{car.na.u.bei.ra}{0}
\verb{carnaval}{}{}{"-ais}{}{s.m.}{Grande festa popular, realizada nos três dias anteriores à quarta"-feira de cinzas, e na qual os foliões dançam e desfilam fantasiados.}{car.na.val}{0}
\verb{carnaval}{}{Pop.}{"-ais}{}{}{Desordem, confusão, algazarra.}{car.na.val}{0}
\verb{carnavalesco}{ê}{}{}{}{adj.}{Relativo ou pertencente ao carnaval.}{car.na.va.les.co}{0}
\verb{carnavalesco}{ê}{}{}{}{s.m.}{Folião de carnaval.}{car.na.va.les.co}{0}
\verb{carnê}{}{}{}{}{s.m.}{Pequeno bloco de folhas destacáveis em que se imprimem os dados relativos às prestações de uma compra feita a prazo.}{car.nê}{0}
\verb{carnê}{}{}{}{}{}{Pequeno caderno de apontamentos, no qual se anotam endereços, telefones, compromissos etc.}{car.nê}{0}
\verb{carne}{}{}{}{}{s.f.}{Nos homens, o tecido muscular que se encontra logo abaixo da pele.}{car.ne}{0}
\verb{carne}{}{}{}{}{}{A parte comestível do corpo de certos animais.}{car.ne}{0}
\verb{carnear}{}{}{}{}{v.i.}{Abater e cortar as carnes do gado para secar; charquear. }{car.ne.ar}{\verboinum{4}}
\verb{carne"-de"-sol}{ó}{Bras.}{carnes"-de"-sol ⟨ó⟩}{}{s.f.}{Carne bovina, salgada e seca ao sol.}{car.ne"-de"-sol}{0}
\verb{carnegão}{}{}{"-ões}{}{s.m.}{Carnicão.}{car.ne.gão}{0}
\verb{carneira}{ê}{}{}{}{s.f.}{Pele de carneiro que se prepara para diversos fins.}{car.nei.ra}{0}
\verb{carneira}{ê}{}{}{}{}{Ovelha.}{car.nei.ra}{0}
\verb{carneirada}{}{}{}{}{s.f.}{Rebanho de carneiros.}{car.nei.ra.da}{0}
\verb{carneirada}{}{Fig.}{}{}{}{Grupo de pessoas submissas, que obedecem ou seguem a opinião alheia sem questionar.}{car.nei.ra.da}{0}
\verb{carneiro}{ê}{}{}{}{s.m.}{Mamífero ruminante, que tem o corpo coberto de lã, e cuja fêmea é a ovelha.}{car.nei.ro}{0}
\verb{carneiro}{ê}{Fig.}{}{}{}{Maria"-vai"-com"-as"-outras.}{car.nei.ro}{0}
\verb{carne"-seca}{ê}{Bras.}{carnes"-secas ⟨ê⟩}{}{s.f.}{Charque.}{car.ne"-se.ca}{0}
\verb{carniça}{}{}{}{}{s.f.}{Animal morto, em estado de putrefação.}{car.ni.ça}{0}
\verb{carnicão}{}{}{"-ões}{}{s.m.}{Nos furúnculos e tumores, a região purulenta e necrosada do tecido; carnegão.}{car.ni.cão}{0}
\verb{carniceiro}{ê}{}{}{}{adj.}{Carnívoro.}{car.ni.cei.ro}{0}
\verb{carniceiro}{ê}{Fig.}{}{}{}{Diz"-se daquele que é cruel, sanguinário.}{car.ni.cei.ro}{0}
\verb{carniceiro}{ê}{Bras.}{}{}{s.m.}{Cirurgião que opera mal ou negligentemente.}{car.ni.cei.ro}{0}
\verb{carniceiro}{ê}{}{}{}{}{Aquele que abate as reses ou as retalha para vender; açougueiro.}{car.ni.cei.ro}{0}
\verb{carnificina}{}{}{}{}{s.f.}{Grande massacre; mortandade, chacina, extermínio, carniçaria. }{car.ni.fi.ci.na}{0}
\verb{carnívoro}{}{Ecol.}{}{}{adj.}{Diz"-se de organismo que se alimenta exclusivamente ou principalmente de animais (de carne).}{car.ní.vo.ro}{0}
\verb{carnívoro}{}{Bot.}{}{}{}{Diz"-se de planta que, apesar de fazer fotossíntese e absorver nutrientes do solo, possui órgãos adaptados para a captura de pequenos insetos, com os quais complementa sua alimentação; insetívoro.}{car.ní.vo.ro}{0}
\verb{carnívoro}{}{Por ext.}{}{}{}{Diz"-se de indivíduos que se alimentam preferencialmente ou exclusivamente de carne vermelha.}{car.ní.vo.ro}{0}
\verb{carnívoro}{}{Zool.}{}{}{s.m.}{Espécime dos carnívoros, ordem de mamíferos caracterizados principalmente pela dentição, com caninos cônicos e pontiagudos, e incisivos cortantes, adaptados para trinchar carne, base de sua alimentação. }{car.ní.vo.ro}{0}
\verb{carnosidade}{}{}{}{}{s.f.}{Qualidade de carnoso.}{car.no.si.da.de}{0}
\verb{carnosidade}{}{}{}{}{}{Formação anormal de tecido; calombo.}{car.no.si.da.de}{0}
\verb{carnoso}{ô}{}{"-osos ⟨ó⟩}{"-osa ⟨ó⟩}{adj.}{Que tem muita carne; carnudo.}{car.no.so}{0}
\verb{carnoso}{ô}{Bot.}{"-osos ⟨ó⟩}{"-osa ⟨ó⟩}{}{Diz"-se de planta que apresenta tecido espesso e suculento, no qual armazena água; suculenta.}{car.no.so}{0}
\verb{carnoso}{ô}{Fig.}{"-osos ⟨ó⟩}{"-osa ⟨ó⟩}{}{Substancial, nutritivo, sucoso, suculento.}{car.no.so}{0}
\verb{carnudo}{}{}{}{}{adj.}{Que tem muita carne ou polpa; carnoso.}{car.nu.do}{0}
\verb{caro}{}{}{}{}{adj.}{De preço alto.}{ca.ro}{0}
\verb{caro}{}{}{}{}{}{Querido, estimado.}{ca.ro}{0}
\verb{caroá}{}{Bot.}{}{}{s.m.}{Planta da família das bromeliáceas, nativa do Brasil, acaule, terrestre, de cujas folhas se obtém fibra; gravatá.}{ca.ro.á}{0}
\verb{caroá}{}{}{}{}{}{A fibra dessa planta, da qual se fabrica barbante, tecido grosseiro e papel.}{ca.ro.á}{0}
\verb{caroável}{}{}{"-eis}{}{adj.2g.}{Afável, carinhoso, afetuoso, meigo.}{ca.ro.á.vel}{0}
\verb{caroba}{ó}{Bot.}{}{}{s.f.}{Nome comum a várias árvores do gênero \textit{jacarandá}, com propriedades medicinais, e das quais se utiliza a madeira em marcenaria; barbatimão.}{ca.ro.ba}{0}
\verb{carochinha}{}{}{}{}{s.f.}{Bruxa, ou mulher muito feia e velha.}{ca.ro.chi.nha}{0}
\verb{caroço}{ô}{}{"-s ⟨ó⟩}{}{s.m.}{Parte dura, interna de um fruto que constitui ou protege sua semente.}{ca.ro.ço}{0}
\verb{caroço}{ô}{Pop.}{"-s ⟨ó⟩}{}{}{Qualquer tumor ou erupção da pele; calombo.}{ca.ro.ço}{0}
\verb{caroçudo}{ô}{}{}{}{adj.}{Que possui caroços; encaroçado.}{ca.ro.çu.do}{0}
\verb{carola}{ó}{}{}{}{adj.2g.}{Diz"-se de indivíduo que frequenta assiduamente igrejas, missas, procissões etc., que é muito beato.}{ca.ro.la}{0}
\verb{carola}{ó}{}{}{}{s.2g.}{Esse indivíduo.}{ca.ro.la}{0}
\verb{carolice}{}{}{}{}{s.f.}{Qualidade ou ato de carola; carolismo.}{ca.ro.li.ce}{0}
\verb{carolíngio}{}{Hist.}{}{}{adj.}{Referente à época de Carlos Magno (742--814), rei dos francos, ou à sua dinastia.}{ca.ro.lín.gio}{0}
\verb{carolismo}{}{}{}{}{s.m.}{Atitude própria de carola; carolice.}{ca.ro.lis.mo}{0}
\verb{carolo}{ô}{}{}{}{s.m.}{Pancada na cabeça aplicada com vara ou com o nó dos dedos.}{ca.ro.lo}{0}
\verb{carolo}{ô}{}{}{}{}{Espiga de milho debulhada.}{ca.ro.lo}{0}
\verb{carona}{}{}{}{}{s.f.}{Condução gratuita, ou de favor, em qualquer veículo.}{ca.ro.na}{0}
\verb{carona}{}{}{}{}{s.2g.}{Pessoa que viaja, num veículo, sem pagar passagem, ou de favor.}{ca.ro.na}{0}
\verb{caroteno}{}{Bioquím.}{}{}{s.m.}{Substância precursora da vitamina \textsc{a} encontrada em vegetais, usada como corante, entre outras aplicações.}{ca.ro.te.no}{0}
\verb{carótida}{}{Anat.}{}{}{s.f.}{Cada uma das duas grandes artérias que conduzem o sangue da aorta para a cabeça.}{ca.ró.ti.da}{0}
\verb{carpa}{}{Zool.}{}{}{s.f.}{Peixe de água doce, de cor prateada e barbas curtas, muito utilizado para criação em pequenos tanques.}{car.pa}{0}
\verb{carpa}{}{}{}{}{s.f.}{Ato ou efeito de capinar; capina.}{car.pa}{0}
\verb{carpelo}{é}{Bot.}{}{}{s.m.}{Folha transformada que forma o gineceu.}{car.pe.lo}{0}
\verb{carpetar}{}{}{}{}{v.t.}{Revestir com carpete; acarpetar.}{car.pe.tar}{\verboinum{1}}
\verb{carpete}{é}{}{}{}{s.m.}{Tipo de tapete, geralmente menos espesso, que reveste todo o cômodo.}{car.pe.te}{0}
\verb{carpete}{é}{Lus.}{}{}{}{Tapete grande e solto no chão.}{car.pe.te}{0}
\verb{carpideira}{ê}{}{}{}{s.f.}{Mulher contratada para chorar os mortos em funerais.}{car.pi.dei.ra}{0}
\verb{carpideira}{ê}{Por ext.}{}{}{}{Mulher que chora ou se lamenta com frequência.}{car.pi.dei.ra}{0}
\verb{carpideira}{ê}{Bras.}{}{}{}{Máquina para capinar.}{car.pi.dei.ra}{0}
\verb{carpina}{}{}{}{}{s.f.}{Ato ou efeito de capinar.}{car.pi.na}{0}
\verb{carpintaria}{}{}{}{}{s.f.}{Oficina de carpinteiro.}{car.pin.ta.ri.a}{0}
\verb{carpintaria}{}{}{}{}{}{Ofício de carpinteiro.}{car.pin.ta.ri.a}{0}
\verb{carpintaria}{}{}{}{}{}{Obra de carpinteiro.}{car.pin.ta.ri.a}{0}
\verb{carpinteiro}{ê}{}{}{}{s.m.}{Profissional que faz trabalhos, especialmente pesados, em madeira, como vigamentos, andaimes, tabuados.}{car.pin.tei.ro}{0}
\verb{carpintejar}{}{}{}{}{v.i.}{Exercer profissão de carpinteiro.}{car.pin.te.jar}{0}
\verb{carpintejar}{}{}{}{}{}{Preparar a madeira para obra.}{car.pin.te.jar}{\verboinum{1}}
\verb{carpir}{}{}{}{}{v.t.}{Arrancar mato; capinar.}{car.pir}{0}
\verb{carpir}{}{}{}{}{}{Lamentar, chorar, lastimar.}{car.pir}{\verboinum{34}\verboirregular{\emph{def.} carpimos, carpis}}
\verb{carpo}{}{Bot.}{}{}{s.m.}{Fruto.}{car.po}{0}
\verb{carpo}{}{Anat.}{}{}{}{Conjunto de ossos que articulam a mão com o antebraço.}{car.po}{0}
\verb{carpófago}{}{}{}{}{adj.}{Diz"-se de animal que se alimenta de frutos.}{car.pó.fa.go}{0}
\verb{carqueja}{ê}{Bot.}{}{}{s.f.}{Erva arbustiva com propriedades medicinais, especialmente para o estômago; erva"-santa.}{car.que.ja}{0}
\verb{carquilha}{}{}{}{}{s.f.}{Ruga na pele; prega, dobra.}{car.qui.lha}{0}
\verb{carrada}{}{}{}{}{s.f.}{Porção de carga que um carro pode transportar a cada viagem.}{car.ra.da}{0}
\verb{carrada}{}{}{}{}{}{Grande quantidade.}{car.ra.da}{0}
\verb{carranca}{}{}{}{}{s.f.}{Fisionomia de raiva, tristeza, mau humor.}{car.ran.ca}{0}
\verb{carranca}{}{}{}{}{}{Máscara.}{car.ran.ca}{0}
\verb{carranca}{}{}{}{}{}{Figura que se coloca na proa do navio.}{car.ran.ca}{0}
\verb{carranca}{}{}{}{}{}{Peça fixada na parede externa da casa, próximo à janela, que mantém a veneziana aberta.}{car.ran.ca}{0}
\verb{carrança}{}{}{}{}{adj.}{Diz"-se de pessoa apegada ao passado.}{car.ran.ça}{0}
\verb{carrancudo}{}{}{}{}{adj.}{Que demonstra mau humor; emburrado.}{car.ran.cu.do}{0}
\verb{carrapateira}{ê}{}{}{}{s.f.}{Mamoneira.}{car.ra.pa.tei.ra}{0}
\verb{carrapaticida}{}{}{}{}{s.m.}{Substância que serve para matar carrapatos.}{car.ra.pa.ti.ci.da}{0}
\verb{carrapato}{}{}{}{}{s.m.}{Parasita que se fixa na pele e suga o sangue de pessoas e animais.}{car.ra.pa.to}{0}
\verb{carrapato}{}{Fig.}{}{}{}{Indivíduo inconveniente que importuna os outros.}{car.ra.pa.to}{0}
\verb{carrapato}{}{}{}{}{}{Mamona.}{car.ra.pa.to}{0}
\verb{carrapeta}{ê}{Bras.}{}{}{s.f.}{Árvore pequena de flores brancas e frutos capsulares.}{car.ra.pe.ta}{0}
\verb{carrapicho}{}{Bot.}{}{}{s.m.}{Planta cujos frutos têm pequenos espinhos ou pelos que aderem às roupas das pessoas e aos pelos dos animais.}{car.ra.pi.cho}{0}
\verb{carrapicho}{}{}{}{}{}{Porção de cabelos amarrada no alto da cabeça.}{car.ra.pi.cho}{0}
\verb{carrapicho}{}{}{}{}{}{Cabelo muito embaraçado.}{car.ra.pi.cho}{0}
\verb{carrascal}{}{Bot.}{"-ais}{}{s.m.}{Formação vegetal que ocorre em algumas regiões do Nordeste brasileiro, constituída de arbustos duros entrelaçados.}{car.ras.cal}{0}
\verb{carrasco}{}{}{}{}{s.m.}{Indivíduo que executa a pena de morte.}{car.ras.co}{0}
\verb{carrasco}{}{Por ext.}{}{}{}{Indivíduo cruel, desumano.}{car.ras.co}{0}
\verb{carrasco}{}{Geogr.}{}{}{}{Carrascal; nome que o sertanejo nordestino dá à flora comum às regiões montanhosas, onde predominam arbustos com espinhos.}{car.ras.co}{0}
\verb{carraspana}{}{Pop.}{}{}{s.f.}{Embriaguez, bebedeira.}{car.ras.pa.na}{0}
\verb{carraspana}{}{Pop.}{}{}{}{Bronca, reprimenda.}{car.ras.pa.na}{0}
\verb{carrear}{}{}{}{}{v.i.}{Dirigir carro.}{car.re.ar}{0}
\verb{carrear}{}{}{}{}{v.t.}{Fazer frete; carregar.}{car.re.ar}{0}
\verb{carrear}{}{}{}{}{}{Causar, ocasionar.}{car.re.ar}{0}
\verb{carrear}{}{}{}{}{}{Arrastar, levar.}{car.re.ar}{0}
\verb{carrear}{}{Fig.}{}{}{}{Acumular, juntar.}{car.re.ar}{\verboinum{4}}
\verb{carreata}{}{Bras.}{}{}{s.f.}{Caravana de veículos em passeata, como manifestação pública com fins políticos ou comemorativos.}{car.re.a.ta}{0}
\verb{carregação}{}{}{"-ões}{}{s.f.}{Ato ou efeito de carregar.}{car.re.ga.ção}{0}
\verb{carregação}{}{Bras.}{"-ões}{}{}{Doença, afecção.}{car.re.ga.ção}{0}
\verb{carregadeira}{ê}{}{}{}{s.f.}{Mulher que transporta cargas na cabeça.}{car.re.ga.dei.ra}{0}
\verb{carregadeira}{ê}{Bras.}{}{}{}{Saúva.}{car.re.ga.dei.ra}{0}
\verb{carregado}{}{}{}{}{adj.}{Que recebeu carga, especialmente em grande quantidade, ou que está completo; cheio, repleto.}{car.re.ga.do}{0}
\verb{carregado}{}{}{}{}{}{Diz"-se de céu nublado que anuncia chuva, especialmente de grande intensidade.}{car.re.ga.do}{0}
\verb{carregado}{}{}{}{}{}{Diz"-se de expressão ou estado de espírito que denota tristeza, nervosismo, preocupação ou mau humor.}{car.re.ga.do}{0}
\verb{carregador}{ô}{}{}{}{adj.}{Que carrega.}{car.re.ga.dor}{0}
\verb{carregador}{ô}{}{}{}{s.m.}{Indivíduo que transporta carga ou carrega bagagem.}{car.re.ga.dor}{0}
\verb{carregador}{ô}{}{}{}{}{Indivíduo que remete carga para transporte.}{car.re.ga.dor}{0}
\verb{carregador}{ô}{}{}{}{}{Peça do armamento que acondiciona a munição e alimenta a arma de fogo.}{car.re.ga.dor}{0}
\verb{carregamento}{}{}{}{}{s.m.}{Ato ou efeito de carregar.}{car.re.ga.men.to}{0}
\verb{carregamento}{}{}{}{}{}{A etapa de arrumação da carga para transporte em navio ou avião.}{car.re.ga.men.to}{0}
\verb{carregamento}{}{}{}{}{}{O conjunto dos objetos que constituem a carga.}{car.re.ga.men.to}{0}
\verb{carregamento}{}{}{}{}{}{A ação de colocar munição em artefato bélico.}{car.re.ga.men.to}{0}
\verb{carregar}{}{}{}{}{v.t.}{Pôr sobre ou dentro, para fins de transporte ou armazenamento; encher.}{car.re.gar}{0}
\verb{carregar}{}{}{}{}{}{Levar de um lugar para outro; transportar.}{car.re.gar}{0}
\verb{carregar}{}{}{}{}{}{Trazer consigo; portar, conduzir, levar.}{car.re.gar}{\verboinum{5}}
\verb{carreira}{ê}{}{}{}{s.f.}{Sequência de pessoas ou coisas dispostas uma depois da outra; fila, fileira. (\textit{As formigas formavam uma carreira que não tinha fim.})}{car.rei.ra}{0}
\verb{carreira}{ê}{}{}{}{}{Caminhos feitos nas plantações de café, milho etc.; carreiro. (\textit{Quando crianças, corríamos pelas carreiras nas plantações.})}{car.rei.ra}{0}
\verb{carreira}{ê}{}{}{}{}{Profissão em que se pode progredir ou em que há promoção. (\textit{Ele seguiu a carreira militar.})}{car.rei.ra}{0}
\verb{carreira}{ê}{}{}{}{}{Risca feita pelo penteado de cabelo. (\textit{Depois de lavar os cabelos, ela ficava horas arrumando as carreiras para trançá"-los. })}{car.rei.ra}{0}
\verb{carreirismo}{}{}{}{}{s.m.}{Tendência ou prática de lançar mão de procedimentos condenáveis para obter ganhos materiais ou alcançar posições rapidamente; oportunismo. }{car.rei.ris.mo}{0}
\verb{carreirista}{}{}{}{}{adj.2g.}{Relativo a carreirismo.}{car.rei.ris.ta}{0}
\verb{carreirista}{}{}{}{}{}{Diz"-se de pessoa que lança mão de procedimentos condenáveis para obter ganhos materiais ou alcançar posições rapidamente. }{car.rei.ris.ta}{0}
\verb{carreiro}{ê}{}{}{}{adj.}{Relativo a carro.}{car.rei.ro}{0}
\verb{carreiro}{ê}{}{}{}{s.m.}{Indivíduo que guia carros de boi.}{car.rei.ro}{0}
\verb{carreiro}{ê}{}{}{}{}{Caminhos feitos nas plantações de café, milho etc.; carreira.}{car.rei.ro}{0}
\verb{carreiro}{ê}{}{}{}{}{Trilha por onde passam habitualmente animais de caça.}{car.rei.ro}{0}
\verb{carreiro}{ê}{}{}{}{}{Caminho estreito; trilha; atalho.}{car.rei.ro}{0}
\verb{carreiro}{ê}{}{}{}{}{Caminho de formigas.}{car.rei.ro}{0}
\verb{carreta}{ê}{}{}{}{s.f.}{Caminhão de grande porte para transporte de carga.}{car.re.ta}{0}
\verb{carreta}{ê}{}{}{}{}{Veículo puxado por animais de carga.}{car.re.ta}{0}
\verb{carreta}{ê}{}{}{}{}{Pequeno carro com duas rodas.}{car.re.ta}{0}
\verb{carreta}{ê}{}{}{}{}{Carro empurrado à mão para levar caixão mortuário.}{car.re.ta}{0}
\verb{carreteira}{ê}{}{}{}{s.f.}{Estrada carroçável.}{car.re.tei.ra}{0}
\verb{carreteiro}{ê}{}{}{}{s.m.}{Condutor profissional de carro ou carreta.}{car.re.tei.ro}{0}
\verb{carreteiro}{ê}{}{}{}{}{Proprietário de caminhão que faz transporte de cargas.}{car.re.tei.ro}{0}
\verb{carreteiro}{ê}{}{}{}{adj.}{Diz"-se do barco usado para carregar e descarregar navios de grande porte.}{car.re.tei.ro}{0}
\verb{carretel}{é}{}{"-éis}{}{s.m.}{Cilindro com bordas para enrolar fios, cordas, linhas, arames.}{car.re.tel}{0}
\verb{carretel}{é}{}{"-éis}{}{}{Molinete de pesca.}{car.re.tel}{0}
\verb{carretilha}{}{}{}{}{s.f.}{Pequena roldana.}{car.re.ti.lha}{0}
\verb{carretilha}{}{}{}{}{}{Utensílio formado por um cabo e um disco móvel que serve para cortar massas diversas em culinária.}{car.re.ti.lha}{0}
\verb{carreto}{ê}{}{}{}{s.m.}{Ato ou efeito de carretar; serviço de frete.}{car.re.to}{0}
\verb{carreto}{ê}{}{}{}{}{Importância paga pelo serviço de frete.}{car.re.to}{0}
\verb{carriça}{}{Zool.}{}{}{s.f.}{Ave de cor avermelhada com listras negras no dorso e asas, que se alimenta de insetos e aranhas e tem canto agradável; curupuruí.}{car.ri.ça}{0}
\verb{carril}{}{}{"-is}{}{s.m.}{Rastro deixado na estrada pelas rodas dos veículos.}{car.ril}{0}
\verb{carril}{}{Lus.}{"-is}{}{}{Cada um dos trilhos em uma estrada de ferro.}{car.ril}{0}
\verb{carrilhão}{}{Mús.}{"-ões}{}{s.m.}{Instrumento formado por vários sinos afinados.}{car.ri.lhão}{0}
\verb{carrilhão}{}{}{"-ões}{}{}{Relógio que anuncia as horas com som semelhante ao de um carrilhão.}{car.ri.lhão}{0}
\verb{carrinho}{}{}{}{}{s.m.}{Carro para transportar bebês ou crianças.}{car.ri.nho}{0}
\verb{carrinho}{}{}{}{}{}{Qualquer carro metálico para ser empurrado e transportar compras, bagagem, materiais de construção.}{car.ri.nho}{0}
\verb{carrinho}{}{}{}{}{}{Viatura de duas rodas puxada por um só cavalo.}{car.ri.nho}{0}
\verb{carrinho}{}{Esport.}{}{}{}{No futebol, lance em que um jogador procura retirar a bola do adversário deslizando pelo chão com os pés estendidos.}{car.ri.nho}{0}
\verb{carro}{}{}{}{}{s.m.}{Veículo que se locomove sobre rodas.}{car.ro}{0}
\verb{carro}{}{Por ext.}{}{}{}{Qualquer peça que se utiliza para fazer transporte de material ou de pessoas. (\textit{Carro do elevador. Carro da máquina.})}{car.ro}{0}
\verb{carroça}{ó}{}{}{}{s.f.}{Carro rudimentar com duas rodas, geralmente feito de madeira, puxado por tração animal.}{car.ro.ça}{0}
\verb{carroça}{ó}{Bras.}{}{}{}{Veículo velho ou em mau estado de conservação; calhambeque.}{car.ro.ça}{0}
\verb{carroça}{ó}{Fig.}{}{}{}{Pessoa lerda, vagarosa.}{car.ro.ça}{0}
\verb{carroçada}{}{}{}{}{s.f.}{Porção de carga que uma carroça pode transportar de uma vez.}{car.ro.ça.da}{0}
\verb{carroção}{}{}{"-ões}{}{s.m.}{Carro coberto e puxado por tração animal, usado antigamente para transportar pessoas.}{car.ro.ção}{0}
\verb{carroçaria}{}{}{}{}{}{Var de \textit{carroceria}.}{car.ro.ça.ri.a}{0}
\verb{carroçável}{}{}{"-eis}{}{adj.2g.}{Próprio para o tráfego de veículos como carros, carroças, caminhões.}{car.ro.çá.vel}{0}
\verb{carroceiro}{ê}{}{}{}{s.m.}{Condutor de carroça.}{car.ro.cei.ro}{0}
\verb{carroceiro}{ê}{}{}{}{}{Indivíduo que faz fretes usando uma carroça.}{car.ro.cei.ro}{0}
\verb{carroceiro}{ê}{Fig.}{}{}{}{Indivíduo grosseiro, rude, mal"-educado.}{car.ro.cei.ro}{0}
\verb{carroceria}{}{}{}{}{s.f.}{Nos veículos de passeio, a estrutura metálica fechada dentro da qual ficam os passageiros e a bagagem.}{car.ro.ce.ri.a}{0}
\verb{carroceria}{}{}{}{}{}{Nos veículos utilitários e caminhões, a parte traseira, geralmente aberta, onde a carga é transportada.}{car.ro.ce.ri.a}{0}
\verb{carro"-chefe}{é}{}{}{}{s.m.}{O principal carro alegórico de um desfile.}{car.ro"-che.fe}{0}
\verb{carro"-chefe}{é}{Fig.}{}{}{}{Elemento (música, filme, obra) que se destaca em um conjunto, por ser de maior interesse ou destaque, geralmente servindo como recurso de divulgação de todo o conjunto.}{car.ro"-che.fe}{0}
\verb{carrocinha}{}{}{}{}{s.f.}{Tipo de carroça utilizada em terraplenagem, com uma caixa que se inclina para descarga.}{car.ro.ci.nha}{0}
\verb{carrocinha}{}{Bras.}{}{}{}{Veículo do serviço de controle de zoonoses que passa pelas ruas da cidade recolhendo animais abandonados.}{car.ro.ci.nha}{0}
\verb{carro"-forte}{ó}{}{carros"-fortes ⟨ó⟩}{}{s.m.}{Veículo automotor blindado, utilizado para transportar grandes valores, tripulado por guardas armados.}{car.ro"-for.te}{0}
\verb{carro"-pipa}{}{}{carros"-pipas \textit{ou} carros"-pipa}{}{s.m.}{Caminhão equipado com grande tanque, usado para transporte de água.}{car.ro"-pi.pa}{0}
\verb{carrossel}{é}{}{"-éis}{}{s.m.}{Aparelho de parques de diversões no qual as pessoas montadas em cavalos de madeira giram em círculos.}{car.ros.sel}{0}
\verb{carruagem}{}{}{"-ens}{}{s.f.}{Veículo de tração animal, com quatro rodas e suspensão, usado para o transporte de pessoas.}{car.rua.gem}{0}
\verb{carta}{}{}{}{}{s.f.}{Comunicação escrita em folha solta, geralmente acomodada num envelope, enviada a uma pessoa ou a uma empresa especificada no texto. (\textit{Eu já enviei uma carta a meu cunhado, mas não tive resposta.})}{car.ta}{0}
\verb{carta}{}{}{}{}{}{Lista impressa dos alimentos ou de bebidas que são oferecidos a clientes de um restaurante; menu, cardápio. (\textit{Logo que chegamos, o garçom entregou"-nos a carta de vinhos.})}{car.ta}{0}
\verb{carta}{}{}{}{}{}{Representação em duas dimensões do espaço geográfico; mapa.}{car.ta}{0}
\verb{carta}{}{}{}{}{}{Cada uma das unidades que compõem fisicamente um jogo de baralho. (\textit{Ela estava com poucas cartas na mão, e continuava arriscando nas jogadas.})}{car.ta}{0}
\verb{carta}{}{}{}{}{}{Documento que se obtém para a condução de veículos; carta de motorista. (\textit{Ele foi pego dirigindo sem carta, mais de uma vez.})}{car.ta}{0}
\verb{carta}{}{}{}{}{}{O conjunto das leis máximas do país; a Constituição. (Nessa acepção, usa"-se apenas com maiúscula).}{car.ta}{0}
\verb{cartada}{}{}{}{}{s.f.}{Em jogos de cartas, cada lance de um jogador.}{car.ta.da}{0}
\verb{cartada}{}{Fig.}{}{}{}{Ação ou empreendimento ousado e decisivo.}{car.ta.da}{0}
\verb{cartaginês}{}{}{}{}{adj.}{Relativo a Cartago, cidade no norte da Tunísia.}{car.ta.gi.nês}{0}
\verb{cartaginês}{}{}{}{}{s.m.}{Indivíduo natural ou habitante dessa cidade.}{car.ta.gi.nês}{0}
\verb{cartaginês}{}{}{}{}{}{Língua falada em Cartago.}{car.ta.gi.nês}{0}
\verb{cartão}{}{}{"-ões}{}{s.m.}{Papel grosso e resistente composto de várias camadas.}{car.tão}{0}
\verb{cartão}{}{}{"-ões}{}{}{Pequeno retângulo de papel, normalmente de boa qualidade, em que são colocadas informações pessoais ou profissionais para apresentação; cartão de visita. (\textit{Logo que chegamos, ele nos deu o seu cartão para que pudéssemos encontrá"-lo.})}{car.tão}{0}
\verb{cartão}{}{Por ext.}{"-ões}{}{}{Qualquer retângulo de papel ou de plástico usado para operações em aparelhos eletrônicos. (\textit{Cartão de crédito. Cartão de telefone. Cartão bancário.})}{car.tão}{0}
\verb{cartão"-postal}{}{}{cartões"-postais}{}{s.m.}{Cartão com foto ou ilustração em uma das faces, ficando a outra reservada para escrita e endereçamento.}{car.tão"-pos.tal}{0}
\verb{cartão"-postal}{}{}{cartões"-postais}{}{}{Paisagem específica fortemente associada a uma cidade ou país.}{car.tão"-pos.tal}{0}
\verb{cartão"-resposta}{ó}{}{cartões"-resposta ⟨ó⟩}{}{s.m.}{Impresso com porte pago, geralmente usado para correspondência entre consumidores e empresas.}{car.tão"-res.pos.ta}{0}
\verb{cartapácio}{}{}{}{}{s.m.}{Livro grande, antigo ou mal conservado; calhamaço.}{car.ta.pá.cio}{0}
\verb{cartapácio}{}{}{}{}{}{Pasta de papéis avulsos.}{car.ta.pá.cio}{0}
\verb{cartas}{}{}{}{}{s.f.pl.}{Baralho.}{car.tas}{0}
\verb{cartas}{}{}{}{}{}{Qualquer jogo realizado com baralho.}{car.tas}{0}
\verb{cartaz}{}{}{}{}{s.m.}{Anúncio feito em uma folha de papel de formato grande, próprio para locais públicos abertos ou amplos.}{car.taz}{0}
\verb{cartaz}{}{}{}{}{}{Fama, popularidade.}{car.taz}{0}
\verb{cartazista}{}{}{}{}{s.2g.}{Indivíduo que projeta ou desenha cartazes.}{car.ta.zis.ta}{0}
\verb{carteado}{}{}{}{}{s.m.}{Qualquer jogo realizado com baralho.}{car.te.a.do}{0}
\verb{carteamento}{}{}{}{}{s.m.}{Ato ou efeito de cartear.}{car.te.a.men.to}{0}
\verb{cartear}{}{}{}{}{v.t.}{Lançar informação em carta de navegação.}{car.te.ar}{0}
\verb{cartear}{}{}{}{}{v.i.}{Jogar cartas.}{car.te.ar}{0}
\verb{cartear}{}{}{}{}{}{Distribuir as cartas em um jogo.}{car.te.ar}{0}
\verb{cartear}{}{}{}{}{}{Dar informação sem ter certeza; chutar.}{car.te.ar}{0}
\verb{cartear}{}{}{}{}{v.pron.}{Corresponder"-se por carta.}{car.te.ar}{\verboinum{4}}
\verb{carteio}{ê}{}{}{}{s.m.}{Carteamento.}{car.tei.o}{0}
\verb{carteira}{ê}{}{}{}{s.f.}{Pequena bolsa de couro, dobrável e com divisões internas, própria para guardar cédulas e documentos.}{car.tei.ra}{0}
\verb{carteira}{ê}{Bras.}{}{}{}{Invólucro geralmente de papel ou papelão para guardar cigarros.}{car.tei.ra}{0}
\verb{carteira}{ê}{}{}{}{}{Mesa pequena própria para escrever e desenhar, eventualmente acoplada a uma cadeira.}{car.tei.ra}{0}
\verb{carteira}{ê}{}{}{}{}{Documento oficial em forma de caderneta.}{car.tei.ra}{0}
\verb{carteira}{ê}{}{}{}{}{Nome de certos setores de instituições financeiras.}{car.tei.ra}{0}
\verb{carteiro}{ê}{}{}{}{s.m.}{Funcionário que distribui cartas, mensagens e outras correspondências.}{car.tei.ro}{0}
\verb{cartel}{é}{}{"-éis}{}{s.m.}{Carta de desafio; provocação.}{car.tel}{0}
\verb{cartel}{é}{}{"-éis}{}{}{Acordo entre empresas de um mesmo setor para restringir concorrência, permitindo determinar os preços.}{car.tel}{0}
\verb{cartela}{é}{}{}{}{s.f.}{Mostruário de tecidos, fitas e similares.}{car.te.la}{0}
\verb{cartela}{é}{}{}{}{}{Cada um dos cartões numerados do jogo de tômbola.}{car.te.la}{0}
\verb{cárter}{}{}{}{}{s.m.}{Invólucro metálico localizado na parte inferior de um motor, destinado a protegê"-lo de corpos estranhos, e onde é recolhido e esfriado o óleo aquecido proveniente de outras partes do motor. }{cár.ter}{0}
\verb{cartesianismo}{}{}{}{}{s.m.}{Doutrina do filósofo, matemático e físico francês René Descartes, e de seus seguidores, caracterizada pelo racionalismo e pelo dualismo metafísico.}{car.te.si.a.nis.mo}{0}
\verb{cartesiano}{}{}{}{}{adj.}{Relativo a Descartes ou ao cartesianismo.}{car.te.si.a.no}{0}
\verb{cartesiano}{}{}{}{}{}{Que é partidário do cartesianismo.}{car.te.si.a.no}{0}
\verb{cartesiano}{}{}{}{}{s.m.}{Indivíduo que partilha das ideias de Descartes, especialista ou continuador do cartesianismo.}{car.te.si.a.no}{0}
\verb{cartilagem}{}{}{"-ens}{}{s.f.}{Tecido conjuntivo fibroso, resistente e flexível, que constitui a totalidade do esqueleto de certos animais e uma parte do esqueleto dos vertebrados superiores.}{car.ti.la.gem}{0}
\verb{cartilaginoso}{ô}{}{"-osos  ⟨ó⟩}{"-osa ⟨ó⟩}{adj.}{Relativo à cartilagem, ou que tem cartilagem.}{car.ti.la.gi.no.so}{0}
\verb{cartilaginoso}{ô}{Biol.}{"-osos  ⟨ó⟩}{"-osa ⟨ó⟩}{}{Diz"-se de órgão vegetal que apresenta consistência semelhante à cartilagem animal.}{car.ti.la.gi.no.so}{0}
\verb{cartilha}{}{}{}{}{s.f.}{Livro para ensinar a ler.}{car.ti.lha}{0}
\verb{cartilha}{}{}{}{}{}{Compêndio elementar ou rudimentos de arte, ciência ou doutrina.}{car.ti.lha}{0}
\verb{cartilha}{}{Fig.}{}{}{}{Padrão de comportamento ou maneira de ser.}{car.ti.lha}{0}
\verb{cartografia}{}{}{}{}{s.f.}{Técnica ou ciência de compor cartas geográficas.}{car.to.gra.fi.a}{0}
\verb{cartografia}{}{}{}{}{}{Descrição ou tratado sobre mapas.}{car.to.gra.fi.a}{0}
\verb{cartógrafo}{}{}{}{}{s.m.}{Indivíduo que trabalha na confecção de cartas geográficas.}{car.tó.gra.fo}{0}
\verb{cartola}{ó}{}{}{}{s.f.}{Chapéu masculino de aba estreita, copa alta e cilíndrica, geralmente de cor preta, usada em ocasiões solenes.}{car.to.la}{0}
\verb{cartola}{ó}{Pop.}{}{}{s.m.}{Indivíduo de posição elevada, que despreza as opiniões e tendências populares.}{car.to.la}{0}
\verb{cartola}{ó}{Pop.}{}{}{}{Dirigente de clube ou entidade esportiva.}{car.to.la}{0}
\verb{cartolina}{}{}{}{}{s.f.}{Cartão de espessura mediana, intermediário entre o papel consistente e o papelão.}{car.to.li.na}{0}
\verb{cartomancia}{}{}{}{}{s.f.}{Arte de adivinhar por meio da leitura e interpretação de cartas de jogar.}{car.to.man.ci.a}{0}
\verb{cartomante}{}{}{}{}{s.2g.}{Indivíduo que pratica a cartomancia.}{car.to.man.te}{0}
\verb{cartonado}{}{}{}{}{adj.}{Diz"-se do livro coberto com capa de cartão, revestido de papel com os nomes da obra, autor e editor impressos.}{car.to.na.do}{0}
\verb{cartonagem}{}{}{"-ens}{}{s.f.}{Confecção de artefatos de cartão.}{car.to.na.gem}{0}
\verb{cartonagem}{}{}{"-ens}{}{}{Oficina que se destina a essa confecção.}{car.to.na.gem}{0}
\verb{cartonagem}{}{}{"-ens}{}{}{Encadernação de livros com cartão.}{car.to.na.gem}{0}
\verb{cartorário}{}{}{}{}{s.m.}{Escrevente de cartório.}{car.to.rá.rio}{0}
\verb{cartorário}{}{}{}{}{}{Livro de registros de documentos públicos ou cartas, títulos, escrituras, certidões etc.}{car.to.rá.rio}{0}
\verb{cartorário}{}{}{}{}{adj.}{Relativo a cartório.}{car.to.rá.rio}{0}
\verb{cartório}{}{}{}{}{s.m.}{Lugar onde se arquivam documentos.}{car.tó.rio}{0}
\verb{cartório}{}{}{}{}{}{Repartição onde funcionam os tabelionatos, os ofícios de notas, os registros públicos, e se mantêm os respectivos arquivos.}{car.tó.rio}{0}
\verb{cartuchame}{}{}{}{}{s.m.}{Porção de cartuchos para arma de fogo.}{car.tu.cha.me}{0}
\verb{cartucheira}{ê}{}{}{}{s.f.}{Artefato de couro ou de lona, geralmente usado à cintura ou a tiracolo, e onde se guardam cartuchos para arma de fogo.}{car.tu.chei.ra}{0}
\verb{cartucho}{}{}{}{}{s.m.}{Canudo de papel ou papelão.}{car.tu.cho}{0}
\verb{cartucho}{}{Informát.}{}{}{}{Dispositivo de material plástico duro que se usa na impressora.}{car.tu.cho}{0}
\verb{cartucho}{}{}{}{}{}{Pequeno saco de papel.}{car.tu.cho}{0}
\verb{cartucho}{}{}{}{}{}{Tubo metálico que contém a bala de uma arma de fogo.}{car.tu.cho}{0}
\verb{cartum}{}{}{"-uns}{}{s.m.}{Desenho ou narrativa gráfica, caricatural, que apresenta uma situação humorística, utilizando, ou não, legendas.}{car.tum}{0}
\verb{cartunista}{}{}{}{}{s.2g.}{Desenhista de cartum.}{car.tu.nis.ta}{0}
\verb{cartuxa}{ch}{}{}{}{s.f.}{Ordem religiosa, de regime misto de solidão e vida em comum, fundada por São Bruno, no século \textsc{xi}.}{car.tu.xa}{0}
\verb{cartuxa}{ch}{}{}{}{}{Convento dessa ordem.}{car.tu.xa}{0}
\verb{cartuxo}{ch}{}{}{}{adj.}{Relativo à ordem cartuxa.}{car.tu.xo}{0}
\verb{cartuxo}{ch}{}{}{}{s.m.}{Religioso que pertence a essa ordem.}{car.tu.xo}{0}
\verb{carunchado}{}{}{}{}{adj.}{Que está cheio de caruncho; apodrecido.}{ca.run.cha.do}{0}
\verb{carunchar}{}{}{}{}{v.i.}{Encher"-se de caruncho; ser atacado pelo caruncho.}{ca.run.char}{\verboinum{1}}
\verb{caruncho}{}{Zool.}{}{}{s.m.}{Inseto que perfura sobretudo madeira, livros e cereais.}{ca.run.cho}{0}
\verb{caruncho}{}{}{}{}{}{O pó que resulta da ação desses insetos.}{ca.run.cho}{0}
\verb{caruncho}{}{Fig.}{}{}{}{Coisa velha.}{ca.run.cho}{0}
\verb{carunchoso}{ô}{}{"-osos ⟨ó⟩}{"-osa ⟨ó⟩}{adj.}{Cheio de caruncho.}{ca.run.cho.so}{0}
\verb{carunchoso}{ô}{}{"-osos ⟨ó⟩}{"-osa ⟨ó⟩}{}{Danificado pelo caruncho; podre.}{ca.run.cho.so}{0}
\verb{carunchoso}{ô}{}{"-osos ⟨ó⟩}{"-osa ⟨ó⟩}{}{Que está em mau estado de conservação; deteriorado, gasto.}{ca.run.cho.so}{0}
\verb{caruru}{}{Bot.}{}{}{s.m.}{Designação comum a certas plantas, a maioria invasora de plantações, cujas folhas, verdes, são saborosas e nutritivas, e por isso muito usadas na culinária.}{ca.ru.ru}{0}
\verb{carvalhal}{}{}{"-ais}{}{s.m.}{Plantação de carvalhos.}{car.va.lhal}{0}
\verb{carvalho}{}{Bot.}{}{}{s.m.}{Árvore ornamental, de rápido crescimento, típica da Europa, que fornece madeira resistente, usada na construção em geral.}{car.va.lho}{0}
\verb{carvalho}{}{}{}{}{}{A madeira dessa árvore.}{car.va.lho}{0}
\verb{carvão}{}{}{"-ões}{}{s.m.}{Material sólido, de origem mineral ou vegetal, que principalmente consiste em carbono com pequeno percentual de hidrogênio, compostos orgânicos complexos e materiais inorgânicos, muito usado industrialmente como combustível.}{car.vão}{0}
\verb{carvão}{}{}{"-ões}{}{}{Brasa extinta; tição.}{car.vão}{0}
\verb{carvão}{}{}{"-ões}{}{}{Lápis de carvão para desenho.}{car.vão}{0}
\verb{carvão}{}{}{"-ões}{}{}{Desenho feito com esse lápis.}{car.vão}{0}
\verb{carvão"-de"-pedra}{é}{}{carvões"-de"-pedra ⟨é⟩}{}{s.m.}{Produto utilizado como combustível; hulha.}{car.vão"-de"-pe.dra}{0}
\verb{carvoaria}{}{}{}{}{s.f.}{Lugar onde se produz ou vende carvão.}{car.vo.a.ri.a}{0}
\verb{carvoeiro}{ê}{}{}{}{s.m.}{Indivíduo que fabrica ou vende carvão.}{car.vo.ei.ro}{0}
\verb{carvoeiro}{ê}{}{}{}{adj.}{Relativo ao carvão.}{car.vo.ei.ro}{0}
\verb{cãs}{}{}{}{}{s.f.pl.}{Cabelos brancos.}{cãs}{0}
\verb{casa}{}{}{}{}{s.f.}{Construção que serve para morar; habitação, moradia, residência.}{ca.sa}{0}
\verb{casa}{}{}{}{}{}{Lar, família.}{ca.sa}{0}
\verb{casa}{}{}{}{}{}{Conjunto de auxiliares de um governante.}{ca.sa}{0}
\verb{casa}{}{}{}{}{}{Construção onde funciona uma empresa, firma ou estabelecimento.}{ca.sa}{0}
\verb{casa}{}{}{}{}{}{Abertura de roupa por onde passa o botão.}{ca.sa}{0}
\verb{casaca}{}{}{}{}{s.f.}{Peça de vestuário de cerimônia masculino, curta na frente, ficando à altura da cintura, e com abas compridas atrás. }{ca.sa.ca}{0}
\verb{casacão}{}{}{"-ões}{}{s.m.}{Casaco longo, cujo comprimento e feitio variam segundo a moda, feito, em geral, de tecido grosso e consistente, e usado como agasalho contra o frio.}{ca.sa.cão}{0}
\verb{casaco}{}{}{}{}{s.m.}{Peça de vestuário de mangas compridas e aberta na frente, mas que geralmente se pode fechar com botões, zíper, colchetes etc., e que cobre o tronco, descendo um pouco abaixo da cintura.}{ca.sa.co}{0}
\verb{casado}{}{}{}{}{adj.}{Diz"-se de pessoa que está ligada a outra por casamento.}{ca.sa.do}{0}
\verb{casado}{}{}{}{}{}{Que está ligado; harmonizado.}{ca.sa.do}{0}
\verb{casadoiro}{ô}{}{}{}{}{Var. de \textit{casadouro}.}{ca.sa.doi.ro}{0}
\verb{casadouro}{ô}{}{}{}{adj.}{Que está em idade de casar.}{ca.sa.dou.ro}{0}
\verb{casadouro}{ô}{}{}{}{}{Que deseja casar.}{ca.sa.dou.ro}{0}
\verb{casa"-forte}{ó}{}{casas"-fortes ⟨ó⟩}{}{s.f.}{Compartimento, numa casa bancária, geralmente no subsolo, com paredes espessas, refratárias a fogo, e portas de aço, para guarda de dinheiro e objetos preciosos; caixa"-forte.}{ca.sa"-for.te}{0}
\verb{casa"-grande}{}{}{casas"-grandes}{}{s.f.}{Residência do senhor, nas fazendas e engenhos de açúcar do Brasil colonial e imperial.}{ca.sa"-gran.de}{0}
\verb{casa"-grande}{}{Por ext.}{casas"-grandes}{}{}{Casa de proprietário de engenho ou de fazenda.}{ca.sa"-gran.de}{0}
\verb{casal}{}{}{"-ais}{}{s.m.}{Pequeno povoado; lugarejo de poucas casas.}{ca.sal}{0}
\verb{casal}{}{}{"-ais}{}{}{Pequena propriedade rústica; granja.}{ca.sal}{0}
\verb{casal}{}{}{"-ais}{}{}{Par composto de macho e fêmea, ou homem e mulher.}{ca.sal}{0}
\verb{casamata}{}{}{}{}{s.f.}{Abrigo subterrâneo blindado, numa fortificação, para estocar material bélico.}{ca.sa.ma.ta}{0}
\verb{casamenteiro}{ê}{}{}{}{adj.}{Relativo a casamento; matrimonial.}{ca.sa.men.tei.ro}{0}
\verb{casamenteiro}{ê}{}{}{}{}{Que promove ou arranja casamentos.}{ca.sa.men.tei.ro}{0}
\verb{casamenteiro}{ê}{}{}{}{}{Que anima ou excita ao casamento.}{ca.sa.men.tei.ro}{0}
\verb{casamento}{}{}{}{}{s.m.}{Ato realizado diante de autoridades e testemunhas, em que um homem e uma mulher prometem construir uma família; matrimônio.}{ca.sa.men.to}{0}
\verb{casar}{}{}{}{}{v.t.}{Unir por casamento; matrimoniar.}{ca.sar}{0}
\verb{casar}{}{}{}{}{}{Promover o casamento.}{ca.sar}{0}
\verb{casar}{}{}{}{}{}{Combinar, harmonizar.}{ca.sar}{\verboinum{1}}
\verb{casarão}{}{}{"-ões}{}{s.m.}{Casa grande.}{ca.sa.rão}{0}
\verb{casarão}{}{Fig.}{"-ões}{}{}{Casa rica, opulenta.}{ca.sa.rão}{0}
\verb{casario}{}{}{}{}{s.m.}{Fileira ou aglomeração de casas.}{ca.sa.ri.o}{0}
\verb{casca}{}{}{}{}{s.f.}{Invólucro exterior de vários órgãos vegetais, como tronco, caule, raiz, fruto e semente.}{cas.ca}{0}
\verb{casca}{}{}{}{}{}{Tegumento endurecido dos crustáceos, moluscos e répteis.}{cas.ca}{0}
\verb{cascabulho}{}{}{}{}{s.m.}{Casca de vários frutos e sementes, especialmente a das castanhas e a da glande dos carvalhos.}{cas.ca.bu.lho}{0}
\verb{cascabulho}{}{}{}{}{}{Monte de cascas.}{cas.ca.bu.lho}{0}
\verb{cascading}{}{Esport.}{}{}{s.m.}{Modalidade esportiva que consiste na descida de cascatas ou cachoeiras, utilizando"-se técnicas verticais como o \textit{rapel}.}{\textit{cascading}}{0}
\verb{casca"-grossa}{ó}{}{cascas"-grossas ⟨ó⟩}{}{adj.2g.}{Diz"-se daquele que é grosseiro, rude.}{cas.ca"-gros.sa}{0}
\verb{casca"-grossa}{ó}{}{cascas"-grossas ⟨ó⟩}{}{s.2g.}{Indivíduo rude, grosseiro e de mau gosto.}{cas.ca"-gros.sa}{0}
\verb{cascalho}{}{}{}{}{s.m.}{O conjunto das lascas de pedra que saltam quando se lavra a cantaria.}{cas.ca.lho}{0}
\verb{cascalho}{}{}{}{}{}{Escórias de ferro.}{cas.ca.lho}{0}
\verb{cascão}{}{}{"-ões}{}{s.m.}{Casca grossa ou endurecida; crosta.}{cas.cão}{0}
\verb{cascão}{}{}{"-ões}{}{}{Crosta de sujidade na pele do corpo.}{cas.cão}{0}
\verb{cascata}{}{}{}{}{s.f.}{Pequena queda"-d'água entre pedras ou rochedos.}{cas.ca.ta}{0}
\verb{cascata}{}{Pop.}{}{}{}{Conversa fiada, mentira.}{cas.ca.ta}{0}
\verb{cascateiro}{ê}{Pop.}{}{}{adj.}{Diz"-se daquele que cascateia, que mente ou conta vantagem.}{cas.ca.tei.ro}{0}
\verb{cascateiro}{ê}{Pop.}{}{}{s.m.}{Indivíduo mentiroso, ou que fica a contar vantagem.}{cas.ca.tei.ro}{0}
\verb{cascavel}{é}{Zool.}{"-éis}{}{s.f.}{Réptil semelhante à serpente, comum nas zonas secas, de coloração pardo"-escura, com losangos claros no dorso, facilmente reconhecível pela presença de guizo ou chocalho na ponta da cauda; boicininga.}{cas.ca.vel}{0}
\verb{casco}{}{}{}{}{s.m.}{Unha dos mamíferos ungulados, como o boi e o cavalo.}{cas.co}{0}
\verb{casco}{}{}{}{}{}{Garrafa vazia de refrigerante, cerveja, águas minerais etc.}{cas.co}{0}
\verb{casco}{}{}{}{}{}{O conjunto formado pelos ossos do crânio.}{cas.co}{0}
\verb{casco}{}{Fig.}{}{}{}{Inteligência, tino, cabeça.}{cas.co}{0}
\verb{casco}{}{}{}{}{}{Carcaça de navio.}{cas.co}{0}
\verb{cascudo}{}{}{}{}{adj.}{Que tem casca grossa ou pele dura.}{cas.cu.do}{0}
\verb{cascudo}{}{Zool.}{}{}{s.m.}{Peixe caracterizado pelo corpo delgado, revestido de placas ósseas, e pela cabeça grande, que vive no fundo dos rios de lugares rochosos, e alimenta"-se de lodo, vegetais e restos orgânicos; acari, uacari.}{cas.cu.do}{0}
\verb{cascudo}{}{Pop.}{}{}{}{Pancada na cabeça com o nó dos dedos dobrados.}{cas.cu.do}{0}
\verb{casear}{}{}{}{}{v.t.}{Fazer casas para botões.}{ca.se.ar}{\verboinum{4}}
\verb{casebre}{é}{}{}{}{s.m.}{Casa pequena e humilde, sem conforto.}{ca.se.bre}{0}
\verb{casebre}{é}{}{}{}{}{Pequena habitação rústica; choupana, cabana.}{ca.se.bre}{0}
\verb{caseína}{}{}{}{}{s.f.}{Proteína existente no leite, do qual pode ser extraída para fins medicinais ou industriais.}{ca.se.í.na}{0}
\verb{caseiro}{ê}{}{}{}{s.m.}{Indivíduo que toma conta da casa de alguém, especialmente casa de campo.}{ca.sei.ro}{0}
\verb{caseiro}{ê}{}{}{}{adj.}{Que é feito em casa. }{ca.sei.ro}{0}
\verb{caseiro}{ê}{}{}{}{}{Que é usado em casa.}{ca.sei.ro}{0}
\verb{caseiro}{ê}{}{}{}{}{Relativo a casa.}{ca.sei.ro}{0}
\verb{caseiro}{ê}{}{}{}{}{Diz"-se de pessoa que gosta muito de ficar em casa, que evita sair à rua.}{ca.sei.ro}{0}
\verb{caserna}{é}{}{}{}{s.f.}{Edifício ou alojamento para moradia de soldados; quartel.}{ca.ser.na}{0}
\verb{casimira}{}{}{}{}{s.f.}{Tecido consistente de lã, usado em geral para vestuário masculino.}{ca.si.mi.ra}{0}
\verb{casinha}{}{}{}{}{s.f.}{Diminutivo de casa.}{ca.si.nha}{0}
\verb{casinha}{}{Pop.}{}{}{}{Lugar destinado a dejeções; latrina.}{ca.si.nha}{0}
\verb{casinhola}{ó}{}{}{}{s.f.}{Casa pequena ou humilde.}{ca.si.nho.la}{0}
\verb{casinholo}{ô}{}{}{}{s.m.}{Casinhola.}{ca.si.nho.lo}{0}
\verb{casmurro}{}{}{}{}{adj.}{Que é teimoso, implicante.}{cas.mur.ro}{0}
\verb{casmurro}{}{}{}{}{}{Que é fechado em si mesmo; sorumbático.}{cas.mur.ro}{0}
\verb{casmurro}{}{}{}{}{s.m.}{Indivíduo teimoso, obstinado.}{cas.mur.ro}{0}
\verb{casmurro}{}{}{}{}{}{Indivíduo ensimesmado, triste.}{cas.mur.ro}{0}
\verb{caso}{}{}{}{}{s.m.}{Acontecimento, fato, ocorrência.}{ca.so}{0}
\verb{caso}{}{}{}{}{}{Circunstância, casualidade.}{ca.so}{0}
\verb{caso}{}{}{}{}{}{História, conto, narrativa.}{ca.so}{0}
\verb{caso}{}{Gram.}{}{}{}{Em línguas declináveis, a flexão que indica a função sintática da palavra na frase.}{ca.so}{0}
\verb{caso}{}{Pop.}{}{}{}{Relação amorosa.}{ca.so}{0}
\verb{caso}{}{}{}{}{conj.}{Na hipótese de; dado que; se.}{ca.so}{0}
\verb{casório}{}{Pop.}{}{}{s.m.}{Casamento, matrimônio.}{ca.só.rio}{0}
\verb{caspa}{}{}{}{}{s.f.}{Escamas que se formam na superfície da pele, especialmente no couro cabeludo.}{cas.pa}{0}
\verb{caspento}{}{}{}{}{adj.}{Que tem muita caspa.}{cas.pen.to}{0}
\verb{cáspite}{}{}{}{}{interj.}{Expressão que denota admiração ou espanto, em geral com um pouco de ironia.}{cás.pi.te}{0}
\verb{casquento}{}{}{}{}{adj.}{Que tem muita casca ou casca grossa.}{cas.quen.to}{0}
\verb{casquete}{é}{}{}{}{s.m.}{Boné.}{cas.que.te}{0}
\verb{casquete}{é}{}{}{}{}{Espécie de boné flexível, sem abas, muitas vezes usado como complemento de uniforme; carapuça.}{cas.que.te}{0}
\verb{casquete}{é}{}{}{}{}{Chapéu velho.}{cas.que.te}{0}
\verb{casquilho}{}{}{}{}{adj.}{Diz"-se de indivíduo que se veste com apuro excessivo, no rigor da moda.}{cas.qui.lho}{0}
\verb{casquilho}{}{}{}{}{s.m.}{Indivíduo que se enfeita com exagero; almofadinha.}{cas.qui.lho}{0}
\verb{casquilho}{}{}{}{}{}{Cilindro oco que remata a lança dos carros.}{cas.qui.lho}{0}
\verb{casquinada}{}{}{}{}{s.f.}{Risada de escárnio.}{cas.qui.na.da}{0}
\verb{casquinada}{}{}{}{}{}{Gargalhada, risada.}{cas.qui.na.da}{0}
\verb{casquinar}{}{}{}{}{v.i.}{Rir com escárnio.}{cas.qui.nar}{0}
\verb{casquinar}{}{}{}{}{}{Gargalhar.}{cas.qui.nar}{\verboinum{1}}
\verb{casquinha}{}{}{}{}{s.f.}{Casca delgada; película.}{cas.qui.nha}{0}
\verb{casquinha}{}{Bras.}{}{}{}{Cone ou copinho feito de massa de biscoito para sorvete.}{cas.qui.nha}{0}
\verb{casquinha}{}{Pop.}{}{}{}{Vantagem, proveito.}{cas.qui.nha}{0}
\verb{cassa}{}{}{}{}{s.f.}{Tecido muito fino, de linho ou de algodão.}{cas.sa}{0}
\verb{cassação}{}{}{"-ões}{}{s.f.}{Ato ou efeito de cassar; anulação.}{cas.sa.ção}{0}
\verb{cassar}{}{}{}{}{v.t.}{Impedir que produza efeitos; anular, revogar.}{cas.sar}{0}
\verb{cassar}{}{}{}{}{}{Apreender, recolher.}{cas.sar}{\verboinum{1}}
\verb{cassata}{}{Cul.}{}{}{s.f.}{Sorvete de origem napolitana, em camadas de diferentes sabores, com discreto recheio de bolo e frutas cristalizadas.}{cas.sa.ta}{0}
\verb{cassete}{é}{}{}{}{s.m.}{Pequena caixa contendo fita magnética ou filme pronto para entrar em funcionamento ao ser introduzido em um gravador ou câmara.}{cas.se.te}{0}
\verb{cassete}{é}{}{}{}{}{Aparelho de gravação e de reprodução sonora, que funciona com a introdução de um cassete.}{cas.se.te}{0}
\verb{cassetete}{é}{}{}{}{s.m.}{Cacete ou bastão de tamanhos variados, de madeira ou de borracha, com alça em uma das extremidades, usado por policiais em situação de confronto.}{cas.se.te.te}{0}
\verb{cassino}{}{}{}{}{s.m.}{Casa de diversões que trabalha basicamente com jogos de azar e geralmente oferece outras diversões aos frequentadores, com espetáculo de música e dança, representações teatrais etc.}{cas.si.no}{0}
\verb{cassiterita}{}{}{}{}{s.f.}{Óxido de estanho tetragonal, marrom ou preto, de brilho adamantino; principal minério de estanho.}{cas.si.te.ri.ta}{0}
\verb{casta}{}{}{}{}{s.f.}{Camada social hereditária e endógama, cujos membros pertencem à mesma raça, etnia, profissão ou religião.}{cas.ta}{0}
\verb{casta}{}{}{}{}{}{Espécie, gênero.}{cas.ta}{0}
\verb{casta}{}{Fig.}{}{}{}{Raça, classe.}{cas.ta}{0}
\verb{castanha}{}{}{}{}{s.f.}{O fruto do castanheiro, ou do cajueiro.}{cas.ta.nha}{0}
\verb{castanha"-do"-pará}{}{}{castanhas"-do"-pará}{}{s.f.}{Semente do fruto do castanheiro"-do"-pará, comestível, nutritiva, muito apreciada assada, crua, ou na composição de doces, além de fornecer óleo. }{cas.ta.nha"-do"-pa.rá}{0}
\verb{castanhal}{}{}{"-ais}{}{s.m.}{Coletivo de castanheiro.}{cas.ta.nhal}{0}
\verb{castanheira}{ê}{Bot.}{}{}{s.f.}{Castanheiro.}{cas.ta.nhei.ra}{0}
\verb{castanheiro}{ê}{Bot.}{}{}{s.m.}{Nome comum a plantas com sementes comestíveis, conhecidas como castanhas.}{cas.ta.nhei.ro}{0}
\verb{castanheiro"-do"-pará}{}{Bot.}{}{}{s.m.}{Árvore frondosa, de até 50 m, nativa da Amazônia, com flores grandes, brancas ou amareladas, e fruto esférico, lenhoso, contendo de 12 a 24 sementes, conhecidas como castanhas"-do"-pará, de grande importância econômica.}{cas.ta.nhei.ro"-do"-pa.rá}{0}
\verb{castanheta}{ê}{}{}{}{s.f.}{Estalido produzido pela ponta do dedo médio ao roçar rapidamente o polegar.}{cas.ta.nhe.ta}{0}
\verb{castanho}{}{}{}{}{adj.}{Que tem a cor da casca da castanha.}{cas.ta.nho}{0}
\verb{castanho}{}{}{}{}{}{Diz"-se dessa cor.}{cas.ta.nho}{0}
\verb{castanho}{}{}{}{}{s.m.}{Essa cor.}{cas.ta.nho}{0}
\verb{castanholas}{ó}{}{}{}{s.f.pl.}{Instrumento de percussão, muito popular na Espanha, constituído por duas peças de madeira ou de marfim, que ligadas entre si, e aos dedos ou pulsos do tocador, por um cordel, batem uma contra outra.}{cas.ta.nho.las}{0}
\verb{castão}{}{}{"-ões}{}{s.m.}{Remate superior das bengalas.}{cas.tão}{0}
\verb{castelã}{}{}{}{}{s.f.}{Senhora ou dona de um castelo.}{cas.te.lã}{0}
\verb{castelão}{}{}{"-ães, -ãos \textit{ou } -ões}{"-ã, -oa \textit{ou } -ona}{s.m.}{Senhor proprietário de um castelo.}{cas.te.lão}{0}
\verb{castelhanismo}{}{}{}{}{s.m.}{Qualidade, caráter do natural ou habitante de Castela, região da Espanha.}{cas.te.lha.nis.mo}{0}
\verb{castelhano}{}{}{}{}{s.m.}{Indivíduo natural ou habitante dessa região.}{cas.te.lha.no}{0}
\verb{castelhano}{}{}{}{}{}{Língua falada em Castela, que se estendeu por toda a Espanha e por muitos países americanos.}{cas.te.lha.no}{0}
\verb{castelhano}{}{}{}{}{adj.}{Relativo a Castela, região da Espanha.}{cas.te.lha.no}{0}
\verb{castelo}{é}{}{}{}{s.m.}{Residência real ou senhorial dotada de fortificações.}{cas.te.lo}{0}
\verb{castelo}{é}{}{}{}{}{Fortaleza medieval.}{cas.te.lo}{0}
\verb{castelo}{é}{}{}{}{}{Parte mais elevada do convés do navio.}{cas.te.lo}{0}
\verb{castiçal}{}{}{"-ais}{}{s.m.}{Utensílio em cuja parte superior há um bocal onde se coloca uma vela para iluminar.}{cas.ti.çal}{0}
\verb{castiço}{}{}{}{}{adj.}{Que é de boa casta, de boa raça.}{cas.ti.ço}{0}
\verb{castiço}{}{}{}{}{}{Próprio para reproduzir a casta, a raça.}{cas.ti.ço}{0}
\verb{castiço}{}{Fig.}{}{}{}{Vernáculo.}{cas.ti.ço}{0}
\verb{castidade}{}{}{}{}{s.f.}{Qualidade de casto.}{cas.ti.da.de}{0}
\verb{castidade}{}{}{}{}{}{Abstinência dos prazeres sensuais.}{cas.ti.da.de}{0}
\verb{castigar}{}{}{}{}{v.t.}{Aplicar castigo; punir severamente.}{cas.ti.gar}{0}
\verb{castigar}{}{}{}{}{}{Repreender, advertir, corrigir.}{cas.ti.gar}{0}
\verb{castigar}{}{}{}{}{}{Apurar, aperfeiçoar.}{cas.ti.gar}{\verboinum{5}}
\verb{castigo}{}{}{}{}{s.m.}{Pena que se inflige a um culpado; punição, correção.}{cas.ti.go}{0}
\verb{castigo}{}{}{}{}{}{Repreensão, advertência.}{cas.ti.go}{0}
\verb{casto}{}{}{}{}{adj.}{Diz"-se daquele que se abstém de quaisquer relações sexuais.}{cas.to}{0}
\verb{casto}{}{}{}{}{}{Que é puro, inocente.}{cas.to}{0}
\verb{castor}{ô}{Zool.}{}{}{s.m.}{Mamífero roedor, de pelagem macia e densa, cauda achatada coberta por grandes escamas e cinco dedos com garras em cada pata.}{cas.tor}{0}
\verb{castor}{ô}{}{}{}{}{O pelo desse animal.}{cas.tor}{0}
\verb{castração}{}{}{"-ões}{}{s.f.}{Ato ou efeito de castrar; capadura.}{cas.tra.ção}{0}
\verb{castrar}{}{}{}{}{v.t.}{Cortar ou destruir os órgãos reprodutores.}{cas.trar}{0}
\verb{castrar}{}{}{}{}{}{Anular a personalidade de alguém.}{cas.trar}{\verboinum{1}}
\verb{castrense}{}{}{}{}{adj.2g.}{Relativo à classe ou ao acampamento militar.}{cas.tren.se}{0}
\verb{castro}{}{}{}{}{s.m.}{Castelo fortificado de origem pré"-romana ou romana.}{cas.tro}{0}
\verb{casual}{}{}{"-ais}{}{adj.2g.}{Que depende do acaso; acidental, eventual.}{ca.su.al}{0}
\verb{casualidade}{}{}{}{}{s.f.}{Qualidade de casual; acaso, eventualidade.}{ca.su.a.li.da.de}{0}
\verb{casuar}{}{Zool.}{}{}{s.m.}{Ave corredora, australiana, que lembra o avestruz.}{ca.su.ar}{0}
\verb{casuarina}{}{Bot.}{}{}{s.f.}{Nome comum às árvores da família das casuarináceas, nativas da Austrália, de crescimento rápido, há muito introduzidas no Brasil, cultivadas como ornamentais e pela madeira. }{ca.su.a.ri.na}{0}
\verb{casuísmo}{}{}{}{}{s.m.}{Aceitação passiva de ideias, doutrinas ou princípios.}{ca.su.ís.mo}{0}
\verb{casuísmo}{}{}{}{}{}{Sistema moral, atitude ou conduta hipócrita, de acomodação.}{ca.su.ís.mo}{0}
\verb{casuísta}{}{}{}{}{adj.2g.}{Casuístico.}{ca.su.ís.ta}{0}
\verb{casuísta}{}{}{}{}{s.2g.}{Indivíduo que pratica o casuísmo.}{ca.su.ís.ta}{0}
\verb{casuístico}{}{}{}{}{adj.}{Relativo ao casuísmo.}{ca.su.ís.ti.co}{0}
\verb{casula}{}{}{}{}{s.f.}{Vestimenta sacerdotal, que se põe sobre a alva e a estola, para a celebração da missa.}{ca.su.la}{0}
\verb{casulo}{}{Zool.}{}{}{s.m.}{Envoltório formado por um longo fio de seda enrolado no qual as larvas de alguns insetos, como as mariposas, passam a fase intermediária de seu desenvolvimento.}{ca.su.lo}{0}
\verb{casulo}{}{Bot.}{}{}{}{Invólucro de fibra que envolve a semente de algumas plantas, como o algodão.}{ca.su.lo}{0}
\verb{cata}{}{}{}{}{s.f.}{Ato ou efeito de catar; busca, procura.}{ca.ta}{0}
\verb{cata}{}{}{}{}{}{Separação dos grãos enegrecidos e mirrados do café.}{ca.ta}{0}
\verb{cata}{}{}{}{}{}{Escavação para mineração.}{ca.ta}{0}
\verb{catabolismo}{}{Bioquím.}{}{}{s.m.}{Fase do metabolismo em que ocorre a degradação pelo organismo das macromoléculas nutritivas, com liberação de energia.}{ca.ta.bo.lis.mo}{0}
\verb{catacego}{é}{Pop.}{}{}{adj.}{Que enxerga mal; de pouca visão.}{ca.ta.ce.go}{0}
\verb{cataclisma}{}{}{}{}{s.f.}{Cataclismo.}{ca.ta.clis.ma}{0}
\verb{cataclismo}{}{}{}{}{s.m.}{Grande inundação; dilúvio.}{ca.ta.clis.mo}{0}
\verb{cataclismo}{}{Geol.}{}{}{}{Transformação brusca e de grande amplitude da crosta terrestre.}{ca.ta.clis.mo}{0}
\verb{cataclismo}{}{Fig.}{}{}{}{Convulsão social; revolta.}{ca.ta.clis.mo}{0}
\verb{cataclismo}{}{Fig.}{}{}{}{Grande desastre; calamidade.}{ca.ta.clis.mo}{0}
\verb{catacrese}{é}{Gram.}{}{}{s.f.}{Aplicação de um termo figurado por falta de um termo próprio. (\textit{Pé da mesa. Asa da xícara.})}{ca.ta.cre.se}{0}
\verb{catacumba}{}{}{}{}{s.f.}{Construção subterrânea que serve de sepultura ou de ossuário.}{ca.ta.cum.ba}{0}
\verb{catacumba}{}{}{}{}{}{Lugar onde se refugiavam os cristãos fugitivos.}{ca.ta.cum.ba}{0}
\verb{catacumba}{}{}{}{}{}{Lugar subterrâneo e retirado.}{ca.ta.cum.ba}{0}
\verb{catadupa}{}{}{}{}{s.f.}{Queda"-d'água de altura considerável, em grande quantidade e com estrondo.}{ca.ta.du.pa}{0}
\verb{catadupa}{}{Fig.}{}{}{}{Jorro, derramamento.}{ca.ta.du.pa}{0}
\verb{catadura}{}{}{}{}{s.f.}{Semblante, aspecto, aparência.}{ca.ta.du.ra}{0}
\verb{catadura}{}{}{}{}{}{Disposição de ânimo.}{ca.ta.du.ra}{0}
\verb{catafalco}{}{}{}{}{s.m.}{Estrado alto sobre o qual se coloca o ataúde ou a representação de um morto a quem se deseja prestar honras.}{ca.ta.fal.co}{0}
\verb{catalão}{}{}{"-ães}{"-ã}{adj.}{Relativo à Catalunha, região da Espanha.}{ca.ta.lão}{0}
\verb{catalão}{}{}{"-ães}{"-ã}{s.m.}{Indivíduo natural ou habitante dessa região.}{ca.ta.lão}{0}
\verb{catalão}{}{}{"-ães}{"-ã}{}{Língua românica falada na Catalunha e em Valença, nas ilhas Baleares e em Andorra.}{ca.ta.lão}{0}
\verb{catalepsia}{}{Med.}{}{}{s.f.}{Incapacidade transitória ou permanente da manifestação espontânea da vontade, que pode ocorrer na histeria, na hipnose e em algumas doenças infecciosas graves.}{ca.ta.lep.si.a}{0}
\verb{cataléptico}{}{}{}{}{adj.}{Relativo à catalepsia.}{ca.ta.lép.ti.co}{0}
\verb{cataléptico}{}{}{}{}{}{Diz"-se daquele que sofre de catalepsia.}{ca.ta.lép.ti.co}{0}
\verb{catalisador}{ô}{Fig.}{}{}{adj.}{Que é estimulante, dinamizador, incentivador.}{ca.ta.li.sa.dor}{0}
\verb{catalisador}{ô}{Fís. e Quím.}{}{}{s.m.}{Substância cuja presença acelera uma reação química.}{ca.ta.li.sa.dor}{0}
\verb{catalisar}{}{Fig.}{}{}{}{Estimular, dinamizar, incentivar.}{ca.ta.li.sar}{\verboinum{1}}
\verb{catalisar}{}{Fís. e Quím.}{}{}{v.t.}{Provocar catálise.}{ca.ta.li.sar}{0}
\verb{catálise}{}{Fís. e Quím.}{}{}{s.f.}{Modificação da velocidade de uma reação química provocada por uma substância que geralmente está presente em pequenas quantidades e pode ser recuperada no final.}{ca.tá.li.se}{0}
\verb{catalítico}{}{}{}{}{adj.}{Relativo à catálise.}{ca.ta.lí.ti.co}{0}
\verb{catalogar}{}{}{}{}{v.t.}{Relacionar em catálogo; inventariar.}{ca.ta.lo.gar}{0}
\verb{catalogar}{}{Pop.}{}{}{}{Classificar, qualificar.}{ca.ta.lo.gar}{\verboinum{5}}
\verb{catálogo}{}{}{}{}{s.m.}{Relação ou lista sumária, metódica, geralmente por ordem alfabética, de pessoas ou coisas.}{ca.tá.lo.go}{0}
\verb{catana}{}{}{}{}{s.f.}{Espada japonesa curva e curta.}{ca.ta.na}{0}
\verb{catanduba}{}{}{}{}{}{Var. de \textit{catanduva}.}{ca.tan.du.ba}{0}
\verb{catanduva}{}{Bot.}{}{}{s.f.}{Árvore com flores amarelas, vagens e madeira de boa qualidade.}{ca.tan.du.va}{0}
\verb{catanduva}{}{}{}{}{}{Terreno arenoso e pouco fértil.}{ca.tan.du.va}{0}
\verb{catanduva}{}{}{}{}{}{Mato rasteiro e espinhento.}{ca.tan.du.va}{0}
\verb{catão}{}{}{"-ões}{}{adj.}{Diz"-se de indivíduo de costumes e princípios muito rígidos.}{ca.tão}{0}
\verb{catão}{}{}{"-ões}{}{}{Diz"-se de indivíduo que aparenta austeridade.}{ca.tão}{0}
\verb{cata"-piolho}{ô}{Bras.}{cata"-piolhos ⟨ô⟩}{}{s.m.}{O dedo mais grosso e curto da mão; polegar, mata"-piolho.}{ca.ta"-pi.o.lho}{0}
\verb{cataplasma}{}{}{}{}{s.2g.}{Pasta medicamentosa feita com polpas, raízes ou folhas que se aplica sobre ferimentos ou inflamações.}{ca.ta.plas.ma}{0}
\verb{cataplasma}{}{Fig.}{}{}{}{Pessoa muito frágil.}{ca.ta.plas.ma}{0}
\verb{catapora}{ó}{Med.}{}{}{s.f.}{Doença infecciosa provocada por vírus e caracterizada por febre e erupções na pele, sendo muito comum na infância; varicela.}{ca.ta.po.ra}{0}
\verb{catapulta}{}{}{}{}{s.f.}{Máquina de guerra que era usada para arremessar projéteis contra as tropas inimigas.}{ca.ta.pul.ta}{0}
\verb{catar}{}{}{}{}{v.t.}{Pegar, colher, apanhar.}{ca.tar}{0}
\verb{catar}{}{}{}{}{}{Pesquisar, procurar, buscar.}{ca.tar}{\verboinum{1}}
\verb{catarata}{}{}{}{}{s.f.}{Queda"-d'água, cachoeira, especialmente de grande tamanho e volume de água.}{ca.ta.ra.ta}{0}
\verb{catarata}{}{Med.}{}{}{}{Doença da visão caracterizada por perda total ou parcial da transparência do cristalino ou de sua membrana.}{ca.ta.ra.ta}{0}
\verb{catarinense}{}{}{}{}{adj.2g.}{Relativo a Santa Catarina; barriga"-verde.}{ca.ta.ri.nen.se}{0}
\verb{catarinense}{}{}{}{}{s.2g.}{Indivíduo natural ou habitante desse estado.}{ca.ta.ri.nen.se}{0}
\verb{catarral}{}{}{"-ais}{}{adj.2g.}{Relativo a catarro.}{ca.tar.ral}{0}
\verb{catarral}{}{Pop.}{"-ais}{}{s.2g.}{Bronquite aguda.}{ca.tar.ral}{0}
\verb{catarreira}{ê}{Pop.}{}{}{s.f.}{Secreção nasal abundante.}{ca.tar.rei.ra}{0}
\verb{catarrento}{}{}{}{}{adj.}{Que tem muito catarro.}{ca.tar.ren.to}{0}
\verb{catarrento}{}{}{}{}{}{Propenso a ficar cheio de catarro.}{ca.tar.ren.to}{0}
\verb{catarrino}{}{Zool.}{}{}{adj.}{Relativo aos catarrinos.}{ca.tar.ri.no}{0}
\verb{catarrino}{}{Zool.}{}{}{s.m.}{Espécime dos primatas que inclui os monos, o homem e alguns macacos, todos com as narinas próximas e voltadas para a frente.}{ca.tar.ri.no}{0}
\verb{catarro}{}{Med.}{}{}{s.m.}{Secreção patológica das mucosas nasais.}{ca.tar.ro}{0}
\verb{catarse}{}{}{}{}{s.f.}{Purificação, purgação.}{ca.tar.se}{0}
\verb{catarse}{}{Med.}{}{}{}{Técnica usada para remissão dos sintomas por meio da exteriorização verbal e emocional dos traumas afetivos reprimidos.}{ca.tar.se}{0}
\verb{catártico}{}{}{}{}{adj.}{Relativo a catarse.}{ca.tár.ti.co}{0}
\verb{catártico}{}{Farm.}{}{}{}{Diz"-se de purgativo mais forte que o laxante e menos forte que o drástico.}{ca.tár.ti.co}{0}
\verb{catástrofe}{}{}{}{}{s.f.}{Grande desastre, de conseqûências calamitosas.}{ca.tás.tro.fe}{0}
\verb{catastrófico}{}{}{}{}{adj.}{Relativo a catástrofe.}{ca.tas.tró.fi.co}{0}
\verb{catatau}{}{Pop.}{}{}{s.m.}{Indivíduo de baixa estatura.}{ca.ta.tau}{0}
\verb{catatau}{}{}{}{}{}{Castigo físico; tapa, pancada.}{ca.ta.tau}{0}
\verb{cata"-vento}{}{}{cata"-ventos}{}{}{Aparelho que utiliza a força do vento para puxar água de poços.}{ca.ta"-ven.to}{0}
\verb{cata"-vento}{}{}{cata"-ventos}{}{s.m.}{Aparelho que mede velocidade e direção do vento.}{ca.ta"-ven.to}{0}
\verb{cata"-vento}{}{}{cata"-ventos}{}{}{Brinquedo com papel em forma de velas de moinho que gira conforme o sopro do vento.}{ca.ta"-ven.to}{0}
\verb{cata"-vento}{}{Fig.}{cata"-ventos}{}{}{Indivíduo volúvel e inconstante.}{ca.ta"-ven.to}{0}
\verb{catchup}{}{}{}{}{s.m.}{Molho de tomate temperado com vinagre e outros temperos, com sabor levemente adocicado; \textit{ketchup}.}{\textit{catchup}}{0}
\verb{catecismo}{}{}{}{}{s.m.}{Conjunto de informações dispostas de maneira didática sobre dogmas e preceitos de doutrina religiosa.}{ca.te.cis.mo}{0}
\verb{catecismo}{}{}{}{}{}{Curso, sessão ou reunião em que essas informações são ministradas.}{ca.te.cis.mo}{0}
\verb{catecúmeno}{}{}{}{}{s.m.}{Pessoa que se prepara para receber o batismo.}{ca.te.cú.me.no}{0}
\verb{cátedra}{}{}{}{}{s.f.}{Cadeira pontifícia.}{cá.te.dra}{0}
\verb{cátedra}{}{}{}{}{}{Cadeira professoral.}{cá.te.dra}{0}
\verb{cátedra}{}{}{}{}{}{Denominação utilizada antes da reforma universitária de 1968 para o cargo de professor titular.}{cá.te.dra}{0}
\verb{cátedra}{}{}{}{}{}{A disciplina ministrada por professor dessa categoria.}{cá.te.dra}{0}
\verb{catedral}{}{}{"-ais}{}{s.f.}{Principal igreja de uma diocese ou arquidiocese, sendo sua sede; igreja episcopal.}{ca.te.dral}{0}
\verb{catedrático}{}{}{}{}{adj.}{Relativo a cátedra.}{ca.te.drá.ti.co}{0}
\verb{catedrático}{}{}{}{}{}{Constituído por professores catedráticos.}{ca.te.drá.ti.co}{0}
\verb{catedrático}{}{}{}{}{s.m.}{Denominação usada para professor titular antes da reforma universitária de 1968.}{ca.te.drá.ti.co}{0}
\verb{catedrático}{}{Por ext.}{}{}{}{Indivíduo com muitos conhecimentos em determinado assunto; conhecedor, especialista.}{ca.te.drá.ti.co}{0}
\verb{categoria}{}{}{}{}{s.f.}{Espécie, natureza, qualidade.}{ca.te.go.ri.a}{0}
\verb{categoria}{}{}{}{}{}{Conjunto de coisas ou pessoas agrupadas segundo determinado critério; classe.}{ca.te.go.ri.a}{0}
\verb{categoria}{}{}{}{}{}{Posição na hierarquia.}{ca.te.go.ri.a}{0}
\verb{categórico}{}{}{}{}{adj.}{Relativo a categoria.}{ca.te.gó.ri.co}{0}
\verb{categórico}{}{}{}{}{}{Claro, indiscutível, explícito.}{ca.te.gó.ri.co}{0}
\verb{categorizado}{}{}{}{}{adj.}{Organizado em categorias; classificado.}{ca.te.go.ri.za.do}{0}
\verb{categorizar}{}{}{}{}{v.t.}{Organizar em categorias.}{ca.te.go.ri.zar}{\verboinum{1}}
\verb{categute}{}{Med.}{}{}{s.m.}{Fio, geralmente feito de tripa de carneiro, utilizado em cirurgias para fazer suturas e ligaduras.}{ca.te.gu.te}{0}
\verb{catenária}{}{Geom.}{}{}{s.f.}{Curva segundo a qual se estende, sob a influência do peso, um fio homogêneo, suspenso pelas extremidades a dois pontos fixos.   }{ca.te.ná.ria}{0}
\verb{catequese}{é}{}{}{}{s.f.}{Instrução religiosa dada de maneira metódica.}{ca.te.que.se}{0}
\verb{catequese}{é}{Por ext.}{}{}{}{Doutrinação.}{ca.te.que.se}{0}
\verb{catequista}{}{}{}{}{s.2g.}{Pessoa que ensina o catecismo.}{ca.te.quis.ta}{0}
\verb{catequização}{}{}{"-ões}{}{s.f.}{Ato ou efeito de catequizar.}{ca.te.qui.za.ção}{0}
\verb{catequizador}{ô}{}{}{}{adj.}{Diz"-se de quem profere o catecismo; catequista.}{ca.te.qui.za.dor}{0}
\verb{catequizar}{}{}{}{}{v.t.}{Dar instrução religiosa.}{ca.te.qui.zar}{0}
\verb{catequizar}{}{}{}{}{}{Fazer doutrinação, especialmente sobre questões políticas e sociais.}{ca.te.qui.zar}{\verboinum{1}}
\verb{cateretê}{}{Bras.}{}{}{s.m.}{Dança bastante difundida em regiões rurais do Brasil na qual homens e mulheres formam filas separadas, sapateiam e batem palmas ao som da música; catira.}{ca.te.re.tê}{0}
\verb{caterva}{é}{}{}{}{s.f.}{Multidão, bando, corja.}{ca.ter.va}{0}
\verb{catete}{ê}{Bras.}{}{}{s.m.}{Milho de espiga curta e grão pequeno.}{ca.te.te}{0}
\verb{cateter}{é}{Med.}{}{}{s.m.}{Tubo ou sonda que se introduz através de canais do corpo para retirar líquidos, injetar medicamentos ou fazer exames.}{ca.te.ter}{0}
\verb{cateterismo}{}{Med.}{}{}{s.m.}{Procedimento no qual se introduz um cateter em canais ou cavidades do corpo.}{ca.te.te.ris.mo}{0}
\verb{cateto}{ê}{}{}{}{s.m.}{Cada um dos lados do ângulo reto no triângulo retângulo.}{ca.te.to}{0}
\verb{cateto}{ê}{Zool.}{}{}{}{Mamífero de pelagem branca e preta, com uma faixa branca no pescoço, em forma de colar; caititu, porco"-do"-mato.}{ca.te.to}{0}
\verb{catilinária}{}{}{}{}{s.f.}{Acusação violenta feita contra alguém.}{ca.ti.li.ná.ria}{0}
\verb{catimbau}{}{}{}{}{s.m.}{Catimbó.}{ca.tim.bau}{0}
\verb{catimbó}{}{Bras.}{}{}{s.m.}{Prática de feitiçaria; macumba, catimbau.}{ca.tim.bó}{0}
\verb{catimplora}{ó}{}{}{}{s.f.}{Vasilha de metal para resfriar água.}{ca.tim.plo.ra}{0}
\verb{catimplora}{ó}{}{}{}{}{Tubo ou funil para passar um líquido de uma vasilha para outra, sem o agitar.}{ca.tim.plo.ra}{0}
\verb{catimplora}{ó}{}{}{}{}{Regador de jardim.}{ca.tim.plo.ra}{0}
\verb{catimplora}{ó}{}{}{}{}{Almotolia de bico estreito e comprido.}{ca.tim.plo.ra}{0}
\verb{catimplora}{ó}{}{}{}{}{Chapéu alto; cartola.}{ca.tim.plo.ra}{0}
\verb{catimplora}{ó}{Bras.}{}{}{}{Sorveteira manual de folha"-de"-flandres.}{ca.tim.plo.ra}{0}
\verb{catimplora}{ó}{}{}{}{}{Bueiro.}{ca.tim.plo.ra}{0}
\verb{catinga}{}{}{}{}{s.f.}{Odor forte e desagradável.}{ca.tin.ga}{0}
\verb{catinga}{}{}{}{}{}{Var. de \textit{caatinga}.}{ca.tin.ga}{0}
\verb{catinga}{}{}{}{}{adj.2g.}{Não generoso; sovina.}{ca.tin.ga}{0}
\verb{catinga}{}{}{}{}{s.f.}{Apego excessivo a dinheiro; sovinice.}{ca.tin.ga}{0}
\verb{catingar}{}{Bras.}{}{}{v.i.}{Cheirar mal.}{ca.tin.gar}{\verboinum{5}}
\verb{catingar}{}{Bras.}{}{}{v.i.}{Ser ou mostrar"-se sovina, avarento.}{ca.tin.gar}{\verboinum{5}}
\verb{catingoso}{ô}{Bras.}{"-osos ⟨ó⟩}{"-osa ⟨ó⟩}{adj.}{Que exala catinga, cheiro desagradável; malcheiroso; catinguento.}{ca.tin.go.so}{0}
\verb{catingueiro}{ê}{Bras.}{}{}{s.m.}{Habitante da caatinga.}{ca.tin.guei.ro}{0}
\verb{catinguento}{}{}{}{}{adj.}{Que exala cheiro ruim; malcheiroso.}{ca.tin.guen.to}{0}
\verb{cátion}{}{Fís. e Quím.}{}{}{s.m.}{Íon com carga positiva.}{cá.tion}{0}
\verb{catira}{}{Bras.}{}{}{s.2g.}{Cateretê.}{ca.ti.ra}{0}
\verb{catita}{}{}{}{}{adj.2g.}{Vestido com esmero; elegante, bonito.}{ca.ti.ta}{0}
\verb{cativante}{}{}{}{}{adj.2g.}{Que cativa; sedutor.}{ca.ti.van.te}{0}
\verb{cativar}{}{}{}{}{v.t.}{Tornar cativo; prender, capturar.}{ca.ti.var}{0}
\verb{cativar}{}{}{}{}{}{Ganhar a simpatia; atrair.}{ca.ti.var}{0}
\verb{cativar}{}{}{}{}{}{Seduzir, atrair, granjear.}{ca.ti.var}{0}
\verb{cativar}{}{}{}{}{v.pron.}{Render"-se, penhorar"-se.}{ca.ti.var}{0}
\verb{cativar}{}{}{}{}{}{Apaixonar"-se, enamorar"-se.}{ca.ti.var}{\verboinum{1}}
\verb{cativeiro}{ê}{}{}{}{s.m.}{Qualidade ou estado de cativo.}{ca.ti.vei.ro}{0}
\verb{cativeiro}{ê}{}{}{}{}{Prisão, clausura.}{ca.ti.vei.ro}{0}
\verb{cativeiro}{ê}{}{}{}{}{Escravidão, servidão.}{ca.ti.vei.ro}{0}
\verb{cativeiro}{ê}{Fig.}{}{}{}{Prisão moral ou espiritual; opressão, tirania, domínio.}{ca.ti.vei.ro}{0}
\verb{cativo}{}{}{}{}{adj.}{Que não goza de liberdade; preso.}{ca.ti.vo}{0}
\verb{cativo}{}{}{}{}{}{Forçado à escravidão.}{ca.ti.vo}{0}
\verb{cativo}{}{}{}{}{}{Seduzido, dominado.}{ca.ti.vo}{0}
\verb{cativo}{}{}{}{}{s.m.}{Prisioneiro de guerra.}{ca.ti.vo}{0}
\verb{cativo}{}{}{}{}{}{Escravo.}{ca.ti.vo}{0}
\verb{catódio}{}{Fís.}{}{}{s.m.}{Catodo.}{ca.tó.dio}{0}
\verb{cátodo}{}{}{}{}{}{Var. de \textit{catodo}.}{cá.to.do}{0}
\verb{catodo}{}{Fís.}{}{}{s.m.}{Eletrodo negativo de uma pilha ou tubo eletrônico.}{ca.to.do}{0}
\verb{catolicidade}{}{}{}{}{s.f.}{Qualidade de católico.}{ca.to.li.ci.da.de}{0}
\verb{catolicidade}{}{}{}{}{}{A doutrina católica.}{ca.to.li.ci.da.de}{0}
\verb{catolicidade}{}{}{}{}{}{O conjunto dos indivíduos católicos.}{ca.to.li.ci.da.de}{0}
\verb{catolicismo}{}{Relig.}{}{}{s.m.}{A doutrina da Igreja Católica, religião que reconhece o papa como autoridade máxima.}{ca.to.li.cis.mo}{0}
\verb{catolicismo}{}{}{}{}{}{A universalidade da religião, reivindicada para si pela Igreja Católica.}{ca.to.li.cis.mo}{0}
\verb{catolicismo}{}{}{}{}{}{O conjunto dos seguidores da Igreja Católica.}{ca.to.li.cis.mo}{0}
\verb{católico}{}{}{}{}{adj.}{Relativo a ou seguidor do catolicismo.}{ca.tó.li.co}{0}
\verb{catorze}{ô}{}{}{}{num.}{Nome dado à quantidade expressa pelo número 14.}{ca.tor.ze}{0}
\verb{catraca}{}{}{}{}{s.f.}{Dispositivo giratório que permite a passagem de uma pessoa por vez, geralmente equipado com um contador, que fica no interior de ônibus e na entrada de estações e outros lugares de grande circulação; roleta, torniquete, borboleta.}{ca.tra.ca}{0}
\verb{catraia}{}{}{}{}{s.f.}{Pequena embarcação com duas proas e tripulada por um só homem.}{ca.trai.a}{0}
\verb{catraieiro}{ê}{}{}{}{s.f.}{Condutor de uma catraia; barqueiro.}{ca.trai.ei.ro}{0}
\verb{catrapus}{}{}{}{}{s.m.}{O galopar do cavalo.}{ca.tra.pus}{0}
\verb{catrapus}{}{}{}{}{interj.}{Expressão que tenta imitar o som de queda repentina e ruidosa.}{ca.tra.pus}{0}
\verb{catre}{}{}{}{}{s.m.}{Cama dobrável de viagem.}{ca.tre}{0}
\verb{catre}{}{}{}{}{}{Leito de lona usado nos navios.}{ca.tre}{0}
\verb{catuaba}{}{Bot.}{}{}{s.f.}{Arbusto de flores grandes e cápsulas de cor ocre, cultivado como ornamental e por suas propriedades medicinais e afrodisíacas.}{ca.tu.a.ba}{0}
\verb{catuaba}{}{}{}{}{}{A bebida feita com essa planta.}{ca.tu.a.ba}{0}
\verb{catual}{}{}{"-ais}{}{s.m.}{Funcionário público que exerce cargos de juiz, governador ou de inspetor na Índia.}{ca.tu.al}{0}
\verb{catucada}{}{}{}{}{}{Var. de \textit{cutucada}.}{ca.tu.ca.da}{0}
\verb{catucão}{}{}{}{}{}{Var. de \textit{cutucão}.}{ca.tu.cão}{0}
\verb{catucar}{}{}{}{}{}{Var. de \textit{cutucar}.}{ca.tu.car}{0}
\verb{catulé}{}{Bot.}{}{}{s.m.}{Palmeira de caule reto e liso, a qual produz óleo comestível; indaiá.}{ca.tu.lé}{0}
\verb{caturra}{}{}{}{}{adj.2g.}{Diz"-se de indivíduo teimoso, apegado a ideias e hábitos ultrapassados e que tende a iniciar polêmicas e discussões.}{ca.tur.ra}{0}
\verb{caturrice}{}{}{}{}{s.f.}{Ato ou dito de indivíduo caturra; teimosia.}{ca.tur.ri.ce}{0}
\verb{caturrismo}{}{}{}{}{s.m.}{Caturrice, teimosia.}{ca.tur.ris.mo}{0}
\verb{caubói}{}{}{}{}{s.m.}{Personagem heroico de filmes de faroeste.}{cau.bói}{0}
\verb{caubói}{}{}{}{}{}{Tocador de boiada; vaqueiro.}{cau.bói}{0}
\verb{caubói}{}{}{}{}{}{Competidor de rodeios; vaqueiro.}{cau.bói}{0}
\verb{caução}{}{}{"-ões}{}{s.f.}{Cuidado, cautela, precaução.}{cau.ção}{0}
\verb{caução}{}{}{"-ões}{}{}{Aquilo que assegura uma obrigação; garantia.}{cau.ção}{0}
\verb{caução}{}{Jur.}{"-ões}{}{}{Bem, valor ou conjunto de valores dado ou depositado como garantia de contrato ou dívida.}{cau.ção}{0}
\verb{caucasiano}{}{}{}{}{adj.}{Relativo ao Cáucaso, cordilheira localizada na Europa oriental.}{cau.ca.si.a.no}{0}
\verb{caucasiano}{}{}{}{}{s.m.}{Indivíduo natural ou habitante da região dessa cordilheira.}{cau.ca.si.a.no}{0}
\verb{caucasiano}{}{}{}{}{adj.}{Relativo à divisão étnica da espécie humana que inclui os povos originários da Europa, Norte da África e Sudoeste da Ásia; caucasoide.}{cau.ca.si.a.no}{0}
\verb{caucasiano}{}{Gram.}{}{}{}{Diz"-se das línguas não indo"-europeias faladas na região do Cáucaso.}{cau.ca.si.a.no}{0}
\verb{caucho}{}{Bot.}{}{}{s.m.}{Árvore de grande porte com folhas oblongas e da qual se extraem substâncias utilizadas na fabricação de papel e borracha.}{cau.cho}{0}
\verb{caucionar}{}{}{}{}{v.t.}{Dar em caução; garantir, afiançar.}{cau.ci.o.nar}{\verboinum{1}}
\verb{cauda}{}{}{}{}{s.f.}{Apêndice posterior do corpo de alguns animais, constituído pela porção terminal da coluna vertebral; rabo.}{cau.da}{0}
\verb{cauda}{}{Por ext.}{}{}{}{Qualquer coisa que por sua forma ou posição posterior em relação a algo se assemelhe à cauda dos animais.}{cau.da}{0}
\verb{cauda}{}{Astron.}{}{}{}{Rastro luminoso dos cometas, formado por poeira e gás.}{cau.da}{0}
\verb{caudado}{}{Zool.}{}{}{s.m.}{Espécime dos caudados, ordem composta pelos anfíbios de corpo alongado, cauda persistente e pernas curtas; são representados pelas salamandras e tritões; urodelo.}{cau.da.do}{0}
\verb{caudado}{}{}{}{}{adj.}{Que tem cauda.}{cau.da.do}{0}
\verb{caudal}{}{}{"-ais}{}{adj.2g.}{Relativo a cauda ou à parte posterior de um corpo.}{cau.dal}{0}
\verb{caudal}{}{}{"-ais}{}{adj.2g.}{Que jorra ou ocorre em abundância.}{cau.dal}{0}
\verb{caudaloso}{ô}{}{"-osos ⟨ó⟩}{"-osa ⟨ó⟩}{adj.}{Que possui fluxo intenso; abundante, copioso, rico.}{cau.da.lo.so}{0}
\verb{caudatário}{}{}{}{}{s.m.}{Indivíduo que carrega a cauda dos mantos de autoridades nas solenidades.}{cau.da.tá.rio}{0}
\verb{caudatário}{}{Fig.}{}{}{}{Indivíduo bajulador, servil, sem personalidade.}{cau.da.tá.rio}{0}
\verb{caudilhismo}{}{}{}{}{s.m.}{Sistema político em que predomina o poder dos caudilhos.}{cau.di.lhis.mo}{0}
\verb{caudilhismo}{}{}{}{}{}{Ato, dito ou procedimento de caudilho.}{cau.di.lhis.mo}{0}
\verb{caudilho}{}{}{}{}{s.m.}{Chefe militar ou político ligado a grupo de poder não oficial.}{cau.di.lho}{0}
\verb{caudilho}{}{}{}{}{}{Ditador.}{cau.di.lho}{0}
\verb{caudilho}{}{}{}{}{}{Homem muito influente; figurão.}{cau.di.lho}{0}
\verb{cauim}{}{}{}{}{s.m.}{Bebida fermentada feita da mandioca, de origem indígena.}{cau.im}{0}
\verb{caule}{}{Bot.}{}{}{s.m.}{Haste das plantas, que parte da raiz e geralmente apresenta folhas e ramificações.}{cau.le}{0}
\verb{caulim}{}{}{}{}{s.m.}{Argila de cor branca; barro branco.}{cau.lim}{0}
\verb{causa}{}{}{}{}{}{Aquilo que produz determinado efeito ou que determina um acontecimento.}{cau.sa}{0}
\verb{causa}{}{}{}{}{s.f.}{Aquilo que faz com que algo exista.}{cau.sa}{0}
\verb{causa}{}{}{}{}{}{Origem, motivo, razão.}{cau.sa}{0}
\verb{causa}{}{}{}{}{}{Partido, facção, interesse.}{cau.sa}{0}
\verb{causa}{}{}{}{}{}{Ação judicial.}{cau.sa}{0}
\verb{causador}{ô}{}{}{}{adj.}{Que determina um acontecimento; ocasionador.}{cau.sa.dor}{0}
\verb{causal}{}{}{"-ais}{}{}{Relativo a causa.}{cau.sal}{0}
\verb{causal}{}{Gram.}{"-ais}{}{adj.2g.}{Diz"-se de conjunção ou oração que exprime ideia de causa.}{cau.sal}{0}
\verb{causalidade}{}{}{}{}{s.f.}{Relação de causa e efeito.}{cau.sa.li.da.de}{0}
\verb{causar}{}{}{}{}{v.t.}{Ser causa; motivar, originar.}{cau.sar}{\verboinum{1}}
\verb{causídico}{}{}{}{}{s.m.}{Defensor de causas, ações judiciais; advogado.}{cau.sí.di.co}{0}
\verb{causo}{}{}{}{}{s.m.}{Narrativa oral que relata um acontecimento interessante ou pitoresco vivido pelo narrador.}{cau.so}{0}
\verb{causticar}{}{}{}{}{v.t.}{Aplicar substâncias cáusticas; queimar.}{caus.ti.car}{\verboinum{2}}
\verb{causticidade}{}{}{}{}{s.f.}{Qualidade de cáustico.}{caus.ti.ci.da.de}{0}
\verb{causticidade}{}{}{}{}{}{Mordacidade, sarcasmo.}{caus.ti.ci.da.de}{0}
\verb{cáustico}{}{}{}{}{adj.}{Que corrói ou queima tecidos orgânicos.}{cáus.ti.co}{0}
\verb{cáustico}{}{}{}{}{}{Sarcástico, irônico, maldoso.}{cáus.ti.co}{0}
\verb{cautela}{é}{}{}{}{s.f.}{Cuidado, precaução.}{cau.te.la}{0}
\verb{cautela}{é}{}{}{}{}{Documento de penhor ou depósito em forma de recibo.}{cau.te.la}{0}
\verb{cautela}{é}{}{}{}{}{Título provisório que representa ação ou debênture.}{cau.te.la}{0}
\verb{cauteloso}{ô}{}{"-osos ⟨ó⟩}{"-osa ⟨ó⟩}{adj.}{Que tem cautela; cuidadoso, prudente.}{cau.te.lo.so}{0}
\verb{cautério}{}{Med.}{}{}{s.m.}{Agente químico ou aparelho usado para extinguir lesões.}{cau.té.rio}{0}
\verb{cauterizar}{}{}{}{}{v.t.}{Aplicar cautérios; extinguir, neutralizar.}{cau.te.ri.zar}{\verboinum{1}}
\verb{cauto}{}{}{}{}{adj.}{Cauteloso, previdente.}{cau.to}{0}
\verb{cava}{}{}{}{}{s.f.}{Ação de cavar.}{ca.va}{0}
\verb{cava}{}{}{}{}{}{Escavação, buraco.}{ca.va}{0}
\verb{cava}{}{}{}{}{}{Abertura da roupa, com ou sem mangas, por onde passam os braços.}{ca.va}{0}
\verb{cavação}{}{}{"-ões}{}{s.f.}{Ato ou efeito de cavar.}{ca.va.ção}{0}
\verb{cavação}{}{Pop.}{"-ões}{}{}{Procedimento ilícito para obter lucros ou vantagens.}{ca.va.ção}{0}
\verb{cavaco}{}{}{}{}{s.m.}{Lasca ou pedaço de madeira.}{ca.va.co}{0}
\verb{cavaco}{}{}{}{}{}{Conversa informal e amigável; bate"-papo.}{ca.va.co}{0}
\verb{cavaco}{}{Bras.}{}{}{}{Cavaquinho.}{ca.va.co}{0}
\verb{cavadeira}{ê}{}{}{}{s.f.}{Peça de ferro presa a um cabo de madeira que serve para cavar.}{ca.va.dei.ra}{0}
\verb{cavado}{}{}{}{}{s.m.}{Lugar que se cavou.}{ca.va.do}{0}
\verb{cavado}{}{}{}{}{}{Cava de vestuário.}{ca.va.do}{0}
\verb{cavador}{ô}{}{}{}{s.m.}{Trabalhador de enxada.}{ca.va.dor}{0}
\verb{cavador}{ô}{Pop.}{}{}{}{Indivíduo que age por meios ilícitos para construir vantagens ou lucro.}{ca.va.dor}{0}
\verb{cavala}{}{Zool.}{}{}{s.f.}{Peixe de corpo alongado e focinho pontudo.}{ca.va.la}{0}
\verb{cavalão}{}{Fig.}{"-ões}{}{s.m.}{Indivíduo muito forte e corpulento.}{ca.va.lão}{0}
\verb{cavalar}{}{}{}{}{adj.2g.}{Relativo a cavalo.}{ca.va.lar}{0}
\verb{cavalar}{}{}{}{}{}{Acima do normal; excessivo.}{ca.va.lar}{0}
\verb{cavalaria}{}{}{}{}{s.f.}{Tropa militar a cavalo.}{ca.va.la.ri.a}{0}
\verb{cavalaria}{}{}{}{}{}{Ordem de cavalaria.}{ca.va.la.ri.a}{0}
\verb{cavalaria}{}{}{}{}{}{Multidão de cavalos ou homens montados a cavalo.}{ca.va.la.ri.a}{0}
\verb{cavalariano}{}{Bras.}{}{}{s.m.}{Membro de cavalaria; cavaleiro.}{ca.va.la.ri.a.no}{0}
\verb{cavalariano}{}{Bras.}{}{}{}{Indivíduo que negocia com cavalos.}{ca.va.la.ri.a.no}{0}
\verb{cavalariça}{}{}{}{}{s.f.}{Alojamento de cavalos; cocheira, estrebaria.}{ca.va.la.ri.ça}{0}
\verb{cavalariço}{}{}{}{}{s.m.}{Empregado de cavalaria.}{ca.va.la.ri.ço}{0}
%\verb{}{}{}{}{}{}{}{}{0}
\verb{cavaleiro}{ê}{}{}{cavaleira \textit{ou} amazona}{adj.}{Que anda a cavalo.}{ca.va.lei.ro}{0}
\verb{cavaleiro}{ê}{}{}{cavaleira \textit{ou} amazona}{s.m.}{Título que era atribuído aos que usavam espada. }{ca.va.lei.ro}{0}
\verb{cavalete}{ê}{}{}{}{s.m.}{Armação que serve de suporte para tela de pintar ou lousa.}{ca.va.le.te}{0}
\verb{cavalete}{ê}{}{}{}{}{Peça com quatro pés que sustenta mesa, bancada de trabalho, andaime etc.}{ca.va.le.te}{0}
\verb{cavalgada}{}{}{}{}{s.f.}{Grupo de pessoas a cavalo.}{ca.val.ga.da}{0}
\verb{cavalgadura}{}{}{}{}{s.f.}{Animal que se pode montar e cavalgar.}{ca.val.ga.du.ra}{0}
\verb{cavalgar}{}{}{}{}{v.t.}{Montar, encavalgar.}{ca.val.gar}{0}
\verb{cavalgar}{}{}{}{}{}{Andar ou percorrer a cavalo.}{ca.val.gar}{\verboinum{5}}
\verb{cavalgata}{}{}{}{}{s.f.}{Cavalgada.}{ca.val.ga.ta}{0}
\verb{cavalhada}{}{Bras.}{}{}{s.f.}{Grande quantidade de cavalos; manada, cavalaria.}{ca.va.lha.da}{0}
\verb{cavalhada}{}{Bras.}{}{}{}{Gado cavalar.}{ca.va.lha.da}{0}
\verb{cavalhada}{}{Fig.}{}{}{}{Façanha, proeza.}{ca.va.lha.da}{0}
\verb{cavalhada}{}{}{}{}{}{Tipo de evento popular em que cavaleiros ricamente trajados fazem certas encenações.}{ca.va.lha.da}{0}
\verb{cavalheiresco}{ê}{}{}{}{adj.}{Próprio de cavalheiro.}{ca.va.lhei.res.co}{0}
\verb{cavalheirismo}{}{}{}{}{s.m.}{Qualidade ou atitude de cavalheiro.}{ca.va.lhei.ris.mo}{0}
\verb{cavalheirismo}{}{}{}{}{}{Nobreza, delicadeza, gentileza.}{ca.va.lhei.ris.mo}{0}
\verb{cavalheiro}{ê}{}{}{}{adj.}{Nobre, gentil, distinto.}{ca.va.lhei.ro}{0}
\verb{cavalheiro}{ê}{}{}{}{}{Homem educado, delicado e de nobres sentimentos.}{ca.va.lhei.ro}{0}
\verb{cavalheiro}{ê}{}{}{}{}{Nobre, fidalgo.}{ca.va.lhei.ro}{0}
\verb{cavalheiro}{ê}{}{}{}{}{Homem, por oposição a dama, especialmente no contexto de um baile.}{ca.va.lhei.ro}{0}
\verb{cavalo}{}{Zool.}{}{}{s.m.}{Animal mamífero de grande porte, criado e domesticado para ser cavalgado como meio de transporte ou em competições desportivas.}{ca.va.lo}{0}
\verb{cavalo}{}{Fig.}{}{}{}{Indivíduo grosseiro ou violento; estúpido, animal.}{ca.va.lo}{0}
\verb{cavalo}{}{}{}{}{}{Peça do jogo de xadrez que se movimenta em "L" e que pode pular posições ocupadas por outras peças.}{ca.va.lo}{0}
\verb{cavalo}{}{}{}{}{}{Ramo ou tronco em que se faz um enxerto.}{ca.va.lo}{0}
\verb{cavalo"-de"-pau}{}{}{cavalos"-de"-pau}{}{s.m.}{Manobra rápida que consiste em fazer um veículo inverter a direção, mediante a aplicação súbita dos freios, com o fim de fazê"-lo parar.}{ca.va.lo"-de"-pau}{0}
\verb{cavalo"-marinho}{}{Zool.}{cavalos"-marinhos}{}{s.m.}{Peixe que tem a cabeça parecida com a do cavalo e anda em posição vertical; hipocampo.}{ca.va.lo"-ma.ri.nho}{0}
\verb{cavalo"-vapor}{ô}{Fís.}{cavalos"-vapor ⟨ô⟩}{}{s.m.}{Unidade de medida de potência igual a 745,7 watts. }{ca.va.lo"-va.por}{0}
\verb{cavanhaque}{}{}{}{}{s.m.}{Barba do queixo aparada em ponta.}{ca.va.nha.que}{0}
\verb{cavaquear}{}{Pop.}{}{}{v.t.}{Conversar informalmente; bater papo.}{ca.va.que.ar}{0}
\verb{cavaquear}{}{Pop.}{}{}{}{Irritar"-se (com brincadeira ou provocação).}{ca.va.que.ar}{\verboinum{4}}
\verb{cavaqueira}{ê}{Pop.}{}{}{s.f.}{Bate"-papo.}{ca.va.quei.ra}{0}
\verb{cavaquinho}{}{Mús.}{}{}{s.m.}{Instrumento em forma de um pequeno violão com quatro cordas dedilháveis, utilizado sobretudo no samba e no choro.}{ca.va.qui.nho}{0}
\verb{cavar}{}{}{}{}{v.t.}{Abrir um buraco na terra; furar, escavar. (\textit{Alguns animais gostam de cavar a terra.})}{ca.var}{0}
\verb{cavar}{}{}{}{}{}{Fazer cavas em roupas. (\textit{A costureira cavou demais o meu vestido.})}{ca.var}{0}
\verb{cavar}{}{Fig.}{}{}{}{Esforçar"-se para conseguir alguma coisa. (\textit{Tive de cavar para conseguir esse emprego na loja.})}{ca.var}{\verboinum{1}}
\verb{cavatina}{}{Mús.}{}{}{s.f.}{Pequena ária simples geralmente intercalada em um recitativo.}{ca.va.ti.na}{0}
\verb{caveira}{ê}{}{}{}{s.f.}{Crânio descarnado; esqueleto da cabeça.}{ca.vei.ra}{0}
\verb{caveira}{ê}{Fig.}{}{}{}{Rosto muito magro.}{ca.vei.ra}{0}
\verb{caverna}{é}{}{}{}{s.f.}{Cavidade no interior da terra, especialmente em terrenos rochosos.}{ca.ver.na}{0}
\verb{cavername}{}{}{}{}{s.m.}{Esqueleto do casco de uma embarcação.}{ca.ver.na.me}{0}
\verb{cavernoso}{ô}{}{"-osos ⟨ó⟩}{"-osa ⟨ó⟩}{adj.}{Que tem cavernas.}{ca.ver.no.so}{0}
\verb{cavernoso}{ô}{}{"-osos ⟨ó⟩}{"-osa ⟨ó⟩}{}{Semelhante a caverna.}{ca.ver.no.so}{0}
\verb{cavernoso}{ô}{}{"-osos ⟨ó⟩}{"-osa ⟨ó⟩}{}{Diz"-se de som profundo ou rouco.}{ca.ver.no.so}{0}
\verb{caviar}{}{}{}{}{s.m.}{Ova de esturjão conservada em sal ou molho condimentado.}{ca.vi.ar}{0}
\verb{cavidade}{}{}{}{}{s.f.}{Espaço oco dentro de um sólido, um corpo biológico ou um de seus órgãos.}{ca.vi.da.de}{0}
\verb{cavilação}{}{}{"-ões}{}{s.f.}{Argumento falso e enganoso; sofisma.}{ca.vi.la.ção}{0}
\verb{cavilação}{}{}{"-ões}{}{}{Astúcia, manha.}{ca.vi.la.ção}{0}
\verb{cavilha}{}{}{}{}{s.f.}{Pino de madeira ou de metal usado para tapar orifícios ou para juntar peças.}{ca.vi.lha}{0}
\verb{caviloso}{ô}{}{"-osos ⟨ó⟩}{"-osa ⟨ó⟩}{adj.}{Em que há cavilação; capcioso, fraudulento.}{ca.vi.lo.so}{0}
\verb{caviúna}{}{}{}{}{}{Var. de \textit{cabiúna}.}{ca.vi.ú.na}{0}
\verb{cavo}{}{}{}{}{adj.}{Que tem cavidade; côncavo, fundo.}{ca.vo}{0}
\verb{cavo}{}{}{}{}{}{Que é oco, vazio.}{ca.vo}{0}
\verb{cavo}{}{Fig.}{}{}{}{Rouco, cavernoso.}{ca.vo}{0}
\verb{cavoucar}{}{}{}{}{v.t.}{Abrir cavoucos.}{ca.vou.car}{\verboinum{2}}
\verb{cavouco}{ô}{}{}{}{s.m.}{Escavação aberta para alicerces de uma construção.}{ca.vou.co}{0}
\verb{cavouco}{ô}{}{}{}{}{Cova, fosso.}{ca.vou.co}{0}
\verb{cavouco}{ô}{}{}{}{}{O vão em que gira a roda do moinho.}{ca.vou.co}{0}
\verb{cavouqueiro}{ê}{}{}{}{s.m.}{Indivíduo que cavouca, que abre cavoucos.}{ca.vou.quei.ro}{0}
\verb{cavouqueiro}{ê}{}{}{}{}{Indivíduo que trabalha em minas ou pedreiras.}{ca.vou.quei.ro}{0}
\verb{cavucar}{}{}{}{}{v.t.}{Revolver ou escavar a terra; cavar.}{ca.vu.car}{0}
\verb{cavucar}{}{Fig.}{}{}{}{Trabalhar com intenso esforço e obstinação.}{ca.vu.car}{\verboinum{2}}
\verb{caxambu}{ch}{}{}{}{s.m.}{Dança de roda de origem africana, com acompanhamento de tambores, semelhante ao samba; jongo.}{ca.xam.bu}{0}
\verb{caxambu}{ch}{}{}{}{}{Grande tambor usado na dança de mesmo nome.}{ca.xam.bu}{0}
\verb{caxangá}{ch}{Zool.}{}{}{s.m.}{Espécie de siri, encontrado nas Américas e na Europa, de carne muito saborosa.}{ca.xan.gá}{0}
\verb{caxemira}{ch}{}{}{}{s.f.}{Tecido fino e macio de lã, fabricado na Índia.}{ca.xe.mi.ra}{0}
\verb{caxerenguengue}{ch}{Bras.}{}{}{s.m.}{Caxirenguengue.}{ca.xe.ren.guen.gue}{0}
\verb{caxeta}{chê}{Bot.}{}{}{s.f.}{Árvore de raízes esponjosas, das quais se faz boias, salva"-vidas etc., e cuja madeira branca, porosa e mole tem inúmeros usos.  }{ca.xe.ta}{0}
\verb{caxias}{ch}{}{}{}{adj.2g.}{Diz"-se de pessoa muito apegada ao cumprimento de seus deveres.}{ca.xi.as}{0}
\verb{caxinguelê}{ch}{Zool.}{}{}{s.m.}{Espécie de esquilo florestal, de cauda longa e cores marrom e laranja, encontrado em regiões do Norte e Nordeste do Brasil; serelepe, esquilo.}{ca.xin.gue.lê}{0}
\verb{caxirenguengue}{ch}{}{}{}{s.m.}{Faca desgastada e inútil, às vezes com a lâmina partida ou sem cabo; caxerenguengue.}{ca.xi.ren.guen.gue}{0}
\verb{caxumba}{ch}{Med.}{}{}{s.f.}{Inflamação contagiosa e infecciosa das parótidas; papeira.}{ca.xum.ba}{0}
\verb{cazaquistanês}{}{}{}{}{adj.}{Relativo ao Cazaquistão.}{ca.za.quis.ta.nês}{0}
\verb{cazaquistanês}{}{}{}{}{s.m.}{Indivíduo natural ou habitante desse país.}{ca.za.quis.ta.nês}{0}
\verb{Cd}{}{Quím.}{}{}{}{Símb. do \textit{cádmio}.}{Cd}{0}
\verb{CD}{}{}{}{}{s.m.}{Disco plástico pequeno no qual os sons são gravados sob forma de sinais numéricos e lidos por um sistema a \textit{laser}; abrev. de \textit{compact disc}.}{CD}{0}
\verb{CDB}{}{Econ.}{}{}{s.m.}{Abrev. de Certificado de Depósito Bancário; comprovante bancário de depósito a prazo fixo, que rende juros e correção monetária.}{CDB}{0}
\verb{CD"-ROM}{}{Informát.}{}{}{s.m.}{Disco ótico que contém grande quantidade de informações para serem lidas pelo computador.}{CD"-ROM}{0}
\verb{Ce}{}{Quím.}{}{}{}{Símb. do \textit{cério}. }{Ce}{0}
\verb{CE}{}{}{}{}{}{Sigla do estado do Ceará.}{CE}{0}
\verb{cê}{}{}{}{}{s.m.}{Nome da letra \textit{c}.  }{cê}{0}
\verb{cear}{}{}{}{}{v.t.}{Comer a ceia.}{ce.ar}{\verboinum{4}}
\verb{cearense}{}{}{}{}{adj.2g.}{Relativo ao Ceará.}{ce.a.ren.se}{0}
\verb{cearense}{}{}{}{}{s.2g.}{Indivíduo natural ou habitante desse estado.}{ce.a.ren.se}{0}
\verb{ceata}{}{}{}{}{s.f.}{Ceia farta, abundante.}{ce.a.ta}{0}
\verb{cebídeo}{}{Zool.}{}{}{s.m.}{Espécime dos cebídeos, família de primatas arborícolas, encontrado nas regiões neotropicais, com até 1 m de comprimento, cauda longa e preênsil, dedos providos de unhas chatas, e tendo o primeiro dedo do pé opondo"-se aos demais dedos.}{ce.bí.deo}{0}
\verb{cebola}{ô}{Bot.}{}{}{s.f.}{Planta cujo bulbo, de cheiro forte e sabor picante, é muito usado como alimento e tempero.}{ce.bo.la}{0}
\verb{cebolada}{}{Cul.}{}{}{s.f.}{Molho preparado com cebolas refogadas ou fritas.}{ce.bo.la.da}{0}
\verb{cebolada}{}{Cul.}{}{}{}{Iguaria preparada com esse molho.}{ce.bo.la.da}{0}
\verb{cebolão}{}{Pop.}{"-ões}{}{s.m.}{Relógio antigo, que se usava no bolso ou no pulso, grande, redondo e grosso.}{ce.bo.lão}{0}
\verb{cebolinha}{}{Bot.}{}{}{s.f.}{Erva cujas folhas cilíndricas e verdes"-escuras são muito usadas como tempero.}{ce.bo.li.nha}{0}
\verb{cebolinha}{}{}{}{}{}{Pequena cebola, usada na preparação de conservas.}{ce.bo.li.nha}{0}
\verb{cecear}{}{}{}{}{v.i.}{Pronunciar o \textit{s} e o \textit{z} com som de \textit{ss}, com a ponta da língua entre os dentes.}{ce.ce.ar}{\verboinum{4}}
\verb{cê"-cedilha}{}{}{cês"-cedilhas}{}{s.m.}{Cê em que se pôs cedilha [ç].}{cê"-ce.di.lha}{0}
\verb{ceceio}{ê}{}{}{}{s.m.}{Ato ou efeito de cecear; pronúncia de \textit{s} e \textit{z} como interdentais.}{ce.cei.o}{0}
\verb{ceco}{é}{Anat.}{}{}{s.m.}{A primeira parte do intestino grosso.}{ce.co}{0}
\verb{cê"-dê"-efe}{é}{Pop.}{cê"-dê"-efes ⟨é⟩}{}{s.2g.}{Pessoa que se dedica excessivamente aos estudos, trabalhos, compromissos etc.}{cê"-dê"-efe}{0}
\verb{cedência}{}{}{}{}{s.f.}{Ato ou efeito de ceder; transferência de direitos a outra pessoa.}{ce.dên.cia}{0}
\verb{cedente}{}{}{}{}{adj.2g.}{Que cede ou faz cessão.}{ce.den.te}{0}
\verb{ceder}{ê}{}{}{}{v.t.}{Transferir a alguém posse ou direito sobre algo; renunciar.}{ce.der}{0}
\verb{ceder}{ê}{}{}{}{}{Conceder, outorgar.}{ce.der}{0}
\verb{ceder}{ê}{}{}{}{}{Concordar, sucumbir.}{ce.der}{0}
\verb{ceder}{ê}{}{}{}{v.i.}{Tornar"-se menos intenso; abrandar, diminuir.}{ce.der}{\verboinum{12}\verboirregular{cedo, cedes}}
\verb{cediço}{}{}{}{}{adj.}{Que está estagnado, quase podre.}{ce.di.ço}{0}
\verb{cedilha}{}{}{}{}{s.f.}{Sinal gráfico em forma de pequena vírgula, colocado sob a letra \textit{c}, que lhe confere um som de \textit{s}, quando seguida de \textit{a, o} e \textit{u}.}{ce.di.lha}{0}
\verb{cedilhar}{}{}{}{}{v.i.}{Colocar cedilha na letra \textit{c}.}{ce.di.lhar}{\verboinum{1}}
\verb{cedinho}{}{}{}{}{adv.}{Muito cedo; logo de manhã.}{ce.di.nho}{0}
\verb{cedível}{}{}{"-eis}{}{adj.2g.}{Que se pode ceder.}{ce.dí.vel}{0}
\verb{cedo}{ê}{}{}{}{}{Antes do tempo ou da condição própria, combinada.}{ce.do}{0}
\verb{cedo}{ê}{}{}{}{adv.}{Ao alvorecer, de madrugada.}{ce.do}{0}
\verb{cedo}{ê}{}{}{}{}{Em breve tempo; depressa, rapidamente.}{ce.do}{0}
\verb{cedro}{é}{Bot.}{}{}{s.m.}{Árvore de grande porte, de flores grandes, brancas, e cuja madeira é própria para marcenaria.}{ce.dro}{0}
\verb{cedro}{é}{}{}{}{}{A madeira dessa árvore.}{ce.dro}{0}
\verb{cédula}{}{}{}{}{s.f.}{Documento escrito; apontamento.}{cé.du.la}{0}
\verb{cédula}{}{}{}{}{}{Confissão de dívida, escrita, mas não legalizada.}{cé.du.la}{0}
\verb{cédula}{}{}{}{}{}{Papel ou nota que representa a moeda de curso legal; dinheiro.}{cé.du.la}{0}
\verb{cédula}{}{Bras.}{}{}{}{Papel com nome de candidato a cargo eletivo, e próprio para votação.}{cé.du.la}{0}
\verb{cefaleia}{é}{Med.}{}{}{s.f.}{Dor de cabeça.}{ce.fa.lei.a}{0}
\verb{cefálico}{}{}{}{}{adj.}{Relativo à cabeça.}{ce.fá.li.co}{0}
\verb{cefalópode}{}{Zool.}{}{}{s.m.}{Espécime dos cefalópodes, classe de moluscos marinhos cujos representantes mais comuns são as lulas, os polvos e os náutilos, dotados de olhos bastante desenvolvidos, cabeça grande, boca provida de bico córneo e rádula, com oito, dez ou mais braços ou tentáculos.}{ce.fa.ló.po.de}{0}
\verb{cefalotórax}{cs}{Zool.}{}{}{s.m.}{Região anterior do corpo dos crustáceos e aracnídeos, formada pela fusão da cabeça e do tórax.}{ce.fa.lo.tó.rax}{0}
\verb{cegar}{}{}{}{}{}{Ofuscar.}{ce.gar}{0}
\verb{cegar}{}{}{}{}{v.t.}{Privar da vista; perder a visão.}{ce.gar}{0}
\verb{cegar}{}{}{}{}{}{Deslumbrar, fascinar.}{ce.gar}{0}
\verb{cegar}{}{}{}{}{}{Tirar o fio ou gume de facas.}{ce.gar}{\verboinum{5}}
\verb{cega"-rega}{é\ldots{}é}{Zool.}{cega"-regas ⟨é\ldots{}é⟩}{}{s.f.}{Cigarra.}{ce.ga"-re.ga}{0}
\verb{cega"-rega}{é\ldots{}é}{Fig.}{cega"-regas ⟨é\ldots{}é⟩}{}{}{Indivíduo muito tagarela, de voz desagradável.}{ce.ga"-re.ga}{0}
\verb{cegas}{é}{}{}{}{s.f.pl.}{Usado na expressão \textit{às cegas}: sem ver. }{ce.gas}{0}
\verb{cego}{é}{}{}{}{adj.}{Privado da vista.}{ce.go}{0}
\verb{cego}{é}{Fig.}{}{}{}{Alucinado, obcecado.}{ce.go}{0}
\verb{cego}{é}{}{}{}{}{Que impede a reflexão, o raciocínio.}{ce.go}{0}
\verb{cego}{é}{}{}{}{}{Absoluto, irrestrito.}{ce.go}{0}
\verb{cego}{é}{}{}{}{}{Diz"-se do instrumento cortante que tem o fio gasto.}{ce.go}{0}
\verb{cego}{é}{}{}{}{s.m.}{Indivíduo privado da visão.}{ce.go}{0}
\verb{cegonha}{}{Zool.}{}{}{s.f.}{Grande ave europeia de pernas compridas, bico longo e plumagem branca.}{ce.go.nha}{0}
\verb{cegueira}{ê}{}{}{}{s.f.}{Estado de cego; incapacidade de ver.}{ce.guei.ra}{0}
\verb{cegueira}{ê}{Fig.}{}{}{}{Afeição extrema, exagerada, a alguém ou alguma coisa.}{ce.guei.ra}{0}
\verb{cegueira}{ê}{}{}{}{}{Falta de lucidez, ou de inteligência.}{ce.guei.ra}{0}
\verb{ceia}{ê}{}{}{}{s.f.}{Refeição da noite.}{cei.a}{0}
\verb{ceifa}{ê}{}{}{}{s.f.}{Ato ou efeito de ceifar; sega.}{cei.fa}{0}
\verb{ceifa}{ê}{}{}{}{}{O tempo de ceifar.}{cei.fa}{0}
\verb{ceifa}{ê}{Fig.}{}{}{}{Grande mortandade ou destruição.}{cei.fa}{0}
\verb{ceifadeira}{ê}{}{}{}{s.f.}{Máquina agrícola para executar a ceifa das plantações; ceifeira.}{cei.fa.dei.ra}{0}
\verb{ceifar}{}{}{}{}{v.t.}{Cortar ou abater cereais, ervas etc., com foice ou instrumento apropriado.}{cei.far}{0}
\verb{ceifar}{}{Fig.}{}{}{}{Aniquilar, destruir.}{cei.far}{\verboinum{1}}
\verb{ceifeira}{ê}{}{}{}{s.f.}{Máquina para ceifar; ceifadeira.}{cei.fei.ra}{0}
\verb{ceifeira}{ê}{}{}{}{}{Mulher que trabalha na ceifa de cereais.}{cei.fei.ra}{0}
\verb{ceifeiro}{ê}{}{}{}{adj.}{Diz"-se de indivíduo que trabalha na ceifa.}{cei.fei.ro}{0}
\verb{ceitil}{}{}{"-is}{}{s.m.}{Antiga moeda portuguesa.}{cei.til}{0}
\verb{ceitil}{}{Fig.}{"-is}{}{}{Insignificância, ninharia.}{cei.til}{0}
\verb{cela}{é}{}{}{}{s.f.}{Pequeno quarto.}{ce.la}{0}
\verb{cela}{é}{}{}{}{}{Aposento de frades ou de freiras, nos conventos.}{ce.la}{0}
\verb{cela}{é}{}{}{}{}{Aposento de condenados, em penitenciárias.}{ce.la}{0}
\verb{celebração}{}{}{"-ões}{}{s.f.}{Ato ou efeito de celebrar; comemoração.}{ce.le.bra.ção}{0}
\verb{celebrado}{}{}{}{}{adj.}{Que é exaltado, louvado.}{ce.le.bra.do}{0}
\verb{celebrado}{}{}{}{}{}{Que é muito conhecido ou admirado; célebre, famoso.}{ce.le.bra.do}{0}
\verb{celebrado}{}{}{}{}{}{Realizado com solenidade.}{ce.le.bra.do}{0}
\verb{celebrante}{}{}{}{}{adj.2g.}{Que celebra.}{ce.le.bran.te}{0}
\verb{celebrante}{}{}{}{}{s.m.}{Sacerdote que celebra a missa.}{ce.le.bran.te}{0}
\verb{celebrar}{}{}{}{}{v.t.}{Lembrar algum acontecimento por meio de uma festa; comemorar, festejar.}{ce.le.brar}{0}
\verb{celebrar}{}{}{}{}{}{Lembrar a importância de pessoa ou acontecimento; enaltecer, exaltar.}{ce.le.brar}{0}
\verb{celebrar}{}{}{}{}{}{Realizar algum ato por ter autoridade.}{ce.le.brar}{\verboinum{1}}
\verb{célebre}{}{}{}{}{adj.2g.}{Que tem fama; famoso.}{cé.le.bre}{0}
\verb{célebre}{}{}{}{}{}{Que é notável, ilustre.}{cé.le.bre}{0}
\verb{célebre}{}{}{}{}{}{Que foge ao comum; estranho, singular.}{cé.le.bre}{0}
\verb{celebridade}{}{}{}{}{s.f.}{Qualidade de célebre; fama, notoriedade.}{ce.le.bri.da.de}{0}
\verb{celebridade}{}{}{}{}{}{Indivíduo célebre, inútil.}{ce.le.bri.da.de}{0}
\verb{celebrizar}{}{}{}{}{v.t.}{Tornar célebre, notável.}{ce.le.bri.zar}{0}
\verb{celebrizar}{}{}{}{}{}{Comemorar, celebrar.}{ce.le.bri.zar}{\verboinum{1}}
\verb{celeiro}{ê}{}{}{}{s.m.}{Grande depósito para cereais ou provisões.}{ce.lei.ro}{0}
\verb{celenterado}{}{Zool.}{}{}{s.m.}{Espécime dos celenterados, filo que inclui os pólipos, as anêmonas"-do"-mar, os corais e as águas"-vivas.}{ce.len.te.ra.do}{0}
\verb{celerado}{}{}{}{}{adj.}{Que é criminoso; fascínora.}{ce.le.ra.do}{0}
\verb{celerado}{}{}{}{}{}{Perverso, mau.}{ce.le.ra.do}{0}
\verb{célere}{}{}{}{}{adj.}{Que é ligeiro, veloz.}{cé.le.re}{0}
\verb{celeste}{é}{}{}{}{adj.2g.}{Relativo ao céu.}{ce.les.te}{0}
\verb{celeste}{é}{}{}{}{}{Que existe no céu.}{ce.les.te}{0}
\verb{celeste}{é}{}{}{}{}{Concernente à divindade; divino.}{ce.les.te}{0}
\verb{celeste}{é}{Fig.}{}{}{}{Superior, perfeito.}{ce.les.te}{0}
\verb{celestial}{}{}{"-ais}{}{adj.2g.}{Celeste.}{ce.les.ti.al}{0}
\verb{celeuma}{}{}{}{}{s.f.}{Vozearia de pessoas que trabalham.}{ce.leu.ma}{0}
\verb{celeuma}{}{}{}{}{}{Canto de barqueiros.}{ce.leu.ma}{0}
\verb{celeuma}{}{}{}{}{}{Barulho, algazarra.}{ce.leu.ma}{0}
\verb{celhas}{ê}{}{}{}{s.f.pl.}{Pelo da orla das pálpebras; cílios.}{ce.lhas}{0}
\verb{celhas}{ê}{Por ext.}{}{}{}{Sobrancelhas.}{ce.lhas}{0}
\verb{celibatário}{}{}{}{}{adj.}{Diz"-se de indivíduo que não se casou.}{ce.li.ba.tá.rio}{0}
\verb{celibatário}{}{}{}{}{s.m.}{Indivíduo solteiro.}{ce.li.ba.tá.rio}{0}
\verb{celibato}{}{}{}{}{s.m.}{Estado de celibatário, condição de pessoa que se mantém casada.}{ce.li.ba.to}{0}
\verb{celofane}{}{}{}{}{s.m.}{Denominação comercial de folhas delgadas e transparentes, obtidas da viscose, usadas como papel de embrulho e para adorno.}{ce.lo.fa.ne}{0}
\verb{celso}{é}{}{}{}{adj.}{Dotado de grande altura; alto, elevado.}{cel.so}{0}
\verb{celso}{é}{Fig.}{}{}{}{Sublime. excelso.}{cel.so}{0}
\verb{celta}{é}{}{}{}{s.2g.}{Indivíduo dos celtas, povos da Antiguidade, que se distribuíram pela Europa ocidental, habitando um território que vai desde a Turquia central até as Ilhas Britânicas.}{cel.ta}{0}
\verb{celta}{é}{}{}{}{s.m.}{Grupo de línguas faladas por esse povo.}{cel.ta}{0}
\verb{celta}{é}{}{}{}{adj.}{Relativo aos celtas ou ao seu grupo de línguas.}{cel.ta}{0}
\verb{celtibero}{é}{}{}{}{adj.}{Relativo à Celtibéria, nome antigo da parte da Espanha que corresponde hoje ao Aragão e parte de Castela.}{cel.ti.be.ro}{0}
\verb{celtibero}{é}{}{}{}{s.m.}{Indivíduo natural ou habitante da Celtibéria.}{cel.ti.be.ro}{0}
\verb{céltico}{}{}{}{}{adj.}{Relativo a celta.}{cél.ti.co}{0}
\verb{célula}{}{Biol.}{}{}{s.f.}{Unidade básica, fundamental, microscópica, que forma o corpo de todos os seres vivos, animais e vegetais, com exceção dos vírus,   e é a menor unidade de matéria viva, capaz de viver e se reproduzir de forma independente. }{cé.lu.la}{0}
\verb{célula}{}{}{}{}{}{Grupo de pessoas com atividades afins, geralmente políticas.}{cé.lu.la}{0}
\verb{célula"-ovo}{ô}{Biol.}{células"-ovos \textit{ou} células"-ovo ⟨ó⟩}{}{s.f.}{A célula sexual feminina dos animais e dos vegetais.}{cé.lu.la"-o.vo}{0}
\verb{celular}{}{}{}{}{adj.2g.}{Relativo a célula.}{ce.lu.lar}{0}
\verb{celular}{}{}{}{}{}{Que é formado por célula.}{ce.lu.lar}{0}
\verb{celular}{}{}{}{}{}{Relativo a cadeias penitenciárias.}{ce.lu.lar}{0}
\verb{celular}{}{}{}{}{s.m.}{Redução de \textit{telefone celular}: telefone portátil, pessoal, utilizado em telefonia celular.}{ce.lu.lar}{0}
\verb{celulite}{}{Med.}{}{}{s.f.}{Alteração visível da pele ou do tecido subcutâneo, devido à inflamação do tecido celular.}{ce.lu.li.te}{0}
\verb{celuloide}{}{}{}{}{s.m.}{Substância obtida da mistura da cânfora e do algodão"-pólvora, é sólida, transparente e elástica, torna"-se maleável pelo aquecimento, e usa"-se para fins industriais.}{ce.lu.loi.de}{0}
\verb{celulose}{ó}{Bioquím.}{}{}{s.f.}{Carboidrato formado por uma longa cadeia de resíduos de molécula de glicose, que consiste no principal componente da parede celular dos vegetais.}{ce.lu.lo.se}{0}
\verb{celulose}{ó}{}{}{}{}{Substância branca, fibrosa, utilizada na fabricação do papel.}{ce.lu.lo.se}{0}
\verb{cem}{}{}{}{}{num.}{Nome dado à quantidade expressa pelo número 100.}{cem}{0}
\verb{cementar}{}{}{}{}{v.t.}{Endurecer, por meio do calor, as camadas externas de metais para diminuir seu desgaste.}{ce.men.tar}{\verboinum{1}}
\verb{cemento}{}{}{}{}{s.m.}{Substância com que se rodeia um corpo a fim de cementá"-lo.}{ce.men.to}{0}
\verb{cemento}{}{}{}{}{}{Qualquer substância com poder adesivo.}{ce.men.to}{0}
\verb{cemento}{}{Anat.}{}{}{}{Camada de tecido ósseo que recobre a raiz dentária.}{ce.men.to}{0}
\verb{cemitério}{}{}{}{}{s.m.}{Espaço, terreno ou recinto em que se enterram e guardam mortos.}{ce.mi.té.rio}{0}
\verb{cena}{}{}{}{}{s.f.}{Espaço em que os atores representam a peça teatral.}{ce.na}{0}
\verb{cena}{}{}{}{}{}{Cada uma das menores unidades de uma peça teatral, de um filme ou de uma telenovela.}{ce.na}{0}
\verb{cena}{}{}{}{}{}{Acontecimento que desperta interesse.}{ce.na}{0}
\verb{cenáculo}{}{}{}{}{s.m.}{Lugar onde Cristo teve a última ceia com seus discípulos.}{ce.ná.cu.lo}{0}
\verb{cenáculo}{}{}{}{}{}{Reunião de pessoas que professam as mesmas ideias.}{ce.ná.cu.lo}{0}
\verb{cenário}{}{}{}{}{s.m.}{Conjunto de elementos visuais que compõem o espaço onde se apresenta um espetáculo teatral, cinematográfico, televisivo; cena.}{ce.ná.rio}{0}
\verb{cenário}{}{}{}{}{}{Lugar onde ocorre algum fato, ou onde decorre a ação, ou parte da ação de uma peça, romance, filme.}{ce.ná.rio}{0}
\verb{cenário}{}{}{}{}{}{Panorama, paisagem.}{ce.ná.rio}{0}
\verb{cenarista}{}{}{}{}{s.2g.}{Técnico que faz cenários.}{ce.na.ris.ta}{0}
\verb{cenho}{}{}{}{}{s.m.}{Aspecto severo ou semblante carrancudo.}{ce.nho}{0}
\verb{cenho}{}{Zool.}{}{}{}{Doença entre o casco e o pelo dos equídeos.}{ce.nho}{0}
\verb{cênico}{}{}{}{}{adj.}{Relativo à cena ou ao teatro.}{cê.ni.co}{0}
\verb{cenóbio}{}{}{}{}{s.m.}{Habitação de monges.}{ce.nó.bio}{0}
\verb{cenóbio}{}{Bot.}{}{}{}{Grupo de células de origem comum, que ocorre nas algas inferiores.}{ce.nó.bio}{0}
\verb{cenobita}{}{}{}{}{s.m.}{Monge que vive em comunidade, em cenóbio.}{ce.no.bi.ta}{0}
\verb{cenobita}{}{Por ext.}{}{}{}{Indivíduo que vive retirado do mundo, geralmente em comunidade com interesses comuns.}{ce.no.bi.ta}{0}
\verb{cenografia}{}{}{}{}{s.f.}{Arte e técnica de projetar e dirigir a execução de cenários para espetáculos teatrais.}{ce.no.gra.fi.a}{0}
\verb{cenógrafo}{}{}{}{}{s.m.}{Especialista em cenografia.}{ce.nó.gra.fo}{0}
\verb{cenotáfio}{}{}{}{}{s.m.}{Monumento fúnebre em memória de alguém cujo corpo não jaz ali sepultado.}{ce.no.tá.fio}{0}
\verb{cenoura}{ô}{}{}{}{s.f.}{Planta hortícola, de raiz alongada, aromática, comestível, de cor vermelho"-alaranjada e rica em açúcar.}{ce.nou.ra}{0}
\verb{cenoura}{ô}{}{}{}{}{A raiz dessa planta.}{ce.nou.ra}{0}
\verb{cenozoico}{ó}{}{}{}{s.m.}{A era geológica mais moderna, que prossegue até o período atual.}{ce.no.zoi.co}{0}
\verb{cenozoico}{ó}{}{}{}{}{Sistema de rochas que caracteriza essa era.}{ce.no.zoi.co}{0}
\verb{cenozoico}{ó}{}{}{}{adj.}{Relativo a essa era.}{ce.no.zoi.co}{0}
\verb{censitário}{}{}{}{}{adj.}{Relativo a censo.}{cen.si.tá.rio}{0}
\verb{censo}{}{}{}{}{}{Conjunto dos dados estatísticos dos habitantes de uma cidade, província, estado ou nação, com todas as suas características.}{cen.so}{0}
\verb{censo}{}{}{}{}{s.m.}{Rendimento que serve de base ao exercício de certos direitos.}{cen.so}{0}
\verb{censor}{ô}{}{}{}{s.m.}{Indivíduo que censura.}{cen.sor}{0}
\verb{censor}{ô}{}{}{}{}{Crítico.}{cen.sor}{0}
\verb{censor}{ô}{}{}{}{}{Funcionário público encarregado da revisão e censura de obras literárias e artísticas.}{cen.sor}{0}
\verb{censor}{ô}{}{}{}{}{Entre os romanos, funcionário que recenseava e velava pelos bons costumes.}{cen.sor}{0}
\verb{censório}{}{}{}{}{adj.}{Relativo a censor ou a censura.}{cen.só.rio}{0}
\verb{censual}{}{}{"-ais}{}{adj.2g.}{Relativo a censo; censitário.}{cen.su.al}{0}
\verb{censura}{}{}{}{}{s.f.}{Ato ou efeito de censurar.}{cen.su.ra}{0}
\verb{censura}{}{}{}{}{}{Cargo ou dignidade de censor.}{cen.su.ra}{0}
\verb{censura}{}{}{}{}{}{Exame crítico de obras literárias ou artísticas; crítica.}{cen.su.ra}{0}
\verb{censura}{}{}{}{}{}{Repreensão, reprovação, crítica.}{cen.su.ra}{0}
\verb{censurar}{}{}{}{}{v.t.}{Exercer censura moral, política, religiosa etc., sobre alguém ou algo.}{cen.su.rar}{0}
\verb{censurar}{}{}{}{}{}{Criticar, notar.}{cen.su.rar}{0}
\verb{censurar}{}{}{}{}{}{Desaprovar, repreender.}{cen.su.rar}{\verboinum{8}}
\verb{centauro}{}{}{}{}{s.m.}{Ser da mitologia grega, metade homem e metade cavalo.}{cen.tau.ro}{0}
\verb{centavo}{}{}{}{}{s.m.}{A centésima parte; centésimo.}{cen.ta.vo}{0}
\verb{centavo}{}{}{}{}{}{Moeda divisionária que representa a centésima parte da unidade monetária de diversos países.}{cen.ta.vo}{0}
\verb{centeio}{ê}{}{}{}{s.m.}{Planta da família das gramíneas, de colmo ereto e flores dispostas em espigas, cujos grãos são usados na fabricação de pães, na produção de álcool, e substitui a cevada na fabricação da cerveja.}{cen.tei.o}{0}
\verb{centelha}{ê}{}{}{}{s.f.}{Partícula ígnea ou luminosa, que se desprende de um corpo incandescente; fagulha.}{cen.te.lha}{0}
\verb{centelha}{ê}{Fig.}{}{}{}{Inspiração, lampejo.}{cen.te.lha}{0}
\verb{centelha}{ê}{Fig.}{}{}{}{Aquilo que brilha momentaneamente.}{cen.te.lha}{0}
\verb{centena}{}{}{}{}{s.f.}{Conjunto de cem unidades.}{cen.te.na}{0}
\verb{centena}{}{}{}{}{}{Dez dezenas.}{cen.te.na}{0}
\verb{centenário}{}{}{}{}{adj.}{Relativo a cem.}{cen.te.ná.rio}{0}
\verb{centenário}{}{}{}{}{}{Que tem cem anos.}{cen.te.ná.rio}{0}
\verb{centenário}{}{}{}{}{s.m.}{Homem que atingiu cem ou mais anos.}{cen.te.ná.rio}{0}
\verb{centenário}{}{}{}{}{}{Espaço de cem anos; século.}{cen.te.ná.rio}{0}
\verb{centesimal}{}{}{"-ais}{}{adj.2g.}{Relativo a centésimo.}{cen.te.si.mal}{0}
\verb{centesimal}{}{}{"-ais}{}{}{Diz"-se de divisão em cem partes iguais.}{cen.te.si.mal}{0}
\verb{centésimo}{}{}{}{}{num.}{Ordinal e fracionário correspondente a cem.}{cen.té.si.mo}{0}
\verb{centésimo}{}{}{}{}{s.m.}{A centésima parte; centavo.}{cen.té.si.mo}{0}
\verb{centiare}{}{}{}{}{s.m.}{Unidade de medida para superfícies agrárias que corresponde a um metro quadrado ou à centésima parte do are.}{cen.ti.a.re}{0}
\verb{centígrado}{}{Fís.}{}{}{s.m.}{Um grau, na escala centesimal de temperatura.}{cen.tí.gra.do}{0}
\verb{centigrama}{}{}{}{}{s.m.}{Unidade de massa, equivalente à centésima parte do grama.}{cen.ti.gra.ma}{0}
\verb{centilitro}{}{}{}{}{s.m.}{Unidade de volume, equivalente à centésima parte do litro.}{cen.ti.li.tro}{0}
\verb{centímetro}{}{}{}{}{s.m.}{A centésima parte do metro.}{cen.tí.me.tro}{0}
\verb{cêntimo}{}{}{}{}{s.m.}{A centésima parte da moeda de diversos países, tais como Bolívia, França, Paraguai, Holanda, Quênia etc.}{cên.ti.mo}{0}
\verb{cento}{}{}{}{}{num.}{Conjunto de dez dezenas.}{cen.to}{0}
\verb{cento}{}{}{}{}{s.m.}{Centena.}{cen.to}{0}
\verb{centopeia}{é}{}{}{}{}{Var. de \textit{centopeia}.}{cen.to.pei.a}{0}
\verb{centopeia}{é}{Zool.}{}{}{s.f.}{Animal invertebrado que apresenta um par de antenas na cabeça, corpo alongado e achatado, e vários pares de patas.}{cen.to.pei.a}{0}
\verb{central}{}{}{"-ais}{}{adj.2g.}{Relativo a centro.}{cen.tral}{0}
\verb{central}{}{}{"-ais}{}{}{Situado no centro.}{cen.tral}{0}
\verb{central}{}{Fig.}{"-ais}{}{}{Principal, fundamental, essencial.}{cen.tral}{0}
\verb{central}{}{}{"-ais}{}{s.f.}{Local ou edifício onde se acham centralizadas certas instalações.}{cen.tral}{0}
\verb{centralização}{}{}{"-ões}{}{s.f.}{Ato ou efeito de centralizar.}{cen.tra.li.za.ção}{0}
\verb{centralização}{}{}{"-ões}{}{}{Reunião, concentração em um mesmo centro ou local.}{cen.tra.li.za.ção}{0}
\verb{centralização}{}{}{"-ões}{}{}{Acumulação de atribuições no poder central.}{cen.tra.li.za.ção}{0}
\verb{centralizar}{}{}{}{}{v.t.}{Tornar central; reunir em um centro.}{cen.tra.li.zar}{0}
\verb{centralizar}{}{}{}{}{}{Fazer convergir a um centro; atrair.}{cen.tra.li.zar}{\verboinum{1}}
\verb{centrar}{}{}{}{}{v.t.}{Colocar no centro; centralizar.}{cen.trar}{0}
\verb{centrar}{}{}{}{}{}{Fazer coincidir uma série de centros para formar um eixo.}{cen.trar}{0}
\verb{centrar}{}{Fig.}{}{}{}{Fazer convergir; concentrar.}{cen.trar}{\verboinum{1}}
\verb{centrífuga}{}{}{}{}{s.f.}{Aparelho com que se efetuam centrifugações, provido de um rotor capaz de girar com velocidade elevada.}{cen.trí.fu.ga}{0}
\verb{centrífugo}{}{}{}{}{adj.}{Que se afasta ou procura afastar"-se do centro.}{cen.trí.fu.go}{0}
\verb{centrífugo}{}{}{}{}{}{Que funciona por meio de força centrífuga.}{cen.trí.fu.go}{0}
\verb{centrípeto}{}{}{}{}{adj.}{Que se dirige para o centro; que procura aproximar"-se do centro.}{cen.trí.pe.to}{0}
\verb{centrista}{}{}{}{}{adj.2g.}{Diz"-se daquele que é partidário da posição política de centro, nem de esquerda, nem de direita.}{cen.tris.ta}{0}
\verb{centro}{}{}{}{}{s.m.}{Ponto que se encontra a igual distância dos limites de alguma coisa; meio.}{cen.tro}{0}
\verb{centro}{}{}{}{}{}{Local onde habitualmente se tratam certos negócios ou executam certas atividades.}{cen.tro}{0}
\verb{centro}{}{}{}{}{}{Zona de maior movimento de pessoas e de comércio em uma cidade.}{cen.tro}{0}
\verb{centro}{}{}{}{}{}{Ponto comum para onde as coisas se voltam.}{cen.tro}{0}
\verb{centro}{}{}{}{}{}{Casa de oração dos espíritas.}{cen.tro}{0}
\verb{centro"-africano}{}{}{centro"-africanos}{centro"-africana}{adj.}{Relativo à República Centro"-Africana.}{cen.tro"-a.fri.ca.no}{0}
\verb{centro"-africano}{}{}{centro"-africanos}{centro"-africana}{s.m.}{Indivíduo natural ou habitante desse país.}{cen.tro"-a.fri.ca.no}{0}
\verb{centro"-americano}{}{}{centro"-americanos}{centro"-americana}{adj.}{Relativo à América Central.}{cen.tro"-a.me.ri.ca.no}{0}
\verb{centro"-americano}{}{}{centro"-americanos}{centro"-americana}{s.m.}{Indivíduo natural ou habitante de qualquer um dos países da América Cental.}{cen.tro"-a.me.ri.ca.no}{0}
\verb{centroavante}{}{Esport.}{}{}{s.2g.}{No futebol, jogador que se coloca, principalmente, no centro da linha de ataque.}{cen.tro.a.van.te}{0}
\verb{centromédio}{}{Esport.}{}{}{s.2g.}{No futebol, jogador que ocupa a posição central da linha intermediária em um time.}{cen.tro.mé.dio}{0}
\verb{Centro"-Oeste}{é}{}{Centro"-Oestes ⟨é⟩}{}{s.m.}{Região geográfica e administrativa do Brasil que inclui os estados de Goiás, Mato Grosso, Mato Grosso do Sul e o Distrito Federal.}{Cen.tro"-O.es.te}{0}
\verb{centuplicar}{}{}{}{}{v.t.}{Multiplicar por cem.}{cen.tu.pli.car}{0}
\verb{centuplicar}{}{Fig.}{}{}{}{Aumentar muito.}{cen.tu.pli.car}{\verboinum{2}}
\verb{cêntuplo}{}{}{}{}{num.}{Cem vezes maior.}{cên.tu.plo}{0}
\verb{cêntuplo}{}{}{}{}{s.m.}{O produto da multiplicação por cem.}{cên.tu.plo}{0}
\verb{centúria}{}{}{}{}{s.f.}{Conjunto de cem unidades; centena.}{cen.tú.ria}{0}
\verb{centúria}{}{}{}{}{}{Tempo de cem anos; século.}{cen.tú.ria}{0}
\verb{centúria}{}{}{}{}{}{No exército romano, companhia de cem soldados.}{cen.tú.ria}{0}
\verb{centurião}{}{}{"-ões}{}{s.m.}{No exército romano, comandante de uma centúria.}{cen.tu.ri.ão}{0}
\verb{CEP}{}{}{}{}{s.m.}{Sigla de \textit{Código de Endereçamento Postal}, código numérico que identifica ruas e bairros em todo o território nacional para ser usado no endereçamento das cartas.}{CEP}{0}
\verb{cepa}{ê}{}{}{}{s.f.}{Tronco de videira.}{ce.pa}{0}
\verb{cepa}{ê}{}{}{}{}{Parte da planta que permanece viva no solo depois de lhe podarem o caule.}{ce.pa}{0}
\verb{cepa}{ê}{Fig.}{}{}{}{Em genealogia, tronco de família ou linhagem.}{ce.pa}{0}
\verb{cepilho}{}{}{}{}{s.m.}{Tipo de plaina para alisar madeira.}{ce.pi.lho}{0}
\verb{cepilho}{}{}{}{}{}{Tipo de lima fina para polir metais.}{ce.pi.lho}{0}
\verb{cepo}{ê}{}{}{}{s.m.}{Toro ou pedaço de toro cortado transversalmente.}{ce.po}{0}
\verb{cepticismo}{}{}{}{}{}{Var. de \textit{ceticismo}.}{cep.ti.cis.mo}{0}
\verb{céptico}{}{}{}{}{}{Var. de \textit{cético}.}{cép.ti.co}{0}
\verb{cera}{ê}{}{}{}{s.f.}{Substância mole e amarelada produzida por certas glândulas das abelhas, que a utilizam para construir favos.}{ce.ra}{0}
\verb{cera}{ê}{}{}{}{}{Substância de origem vegetal semelhante à cera das abelhas.}{ce.ra}{0}
\verb{cera}{ê}{}{}{}{}{Qualquer substância sintética com iguais características, com diversas aplicações.}{ce.ra}{0}
\verb{cera}{ê}{}{}{}{}{Secreção das glândulas do conduto auditivo externo; cerume.}{ce.ra}{0}
\verb{cera}{ê}{}{}{}{}{Ação vagarosa e desleixada, geralmente visando retardar a execução de uma tarefa ou, no caso de esportes, de uma jogada.}{ce.ra}{0}
\verb{cerâmica}{}{}{}{}{s.f.}{Técnica de fabricação de objetos e utensílios de argila.}{ce.râ.mi.ca}{0}
\verb{cerâmica}{}{}{}{}{}{Os artefatos produzidos com essa técnica.}{ce.râ.mi.ca}{0}
\verb{cerâmica}{}{}{}{}{}{A fábrica desses artefatos; olaria.}{ce.râ.mi.ca}{0}
\verb{cerâmica}{}{}{}{}{}{A matéria"-prima utilizada na fabricação desses artefatos; argila.}{ce.râ.mi.ca}{0}
\verb{ceramista}{}{}{}{}{s.2g.}{Indivíduo que domina a técnica da cerâmica.}{ce.ra.mis.ta}{0}
\verb{cerca}{ê}{}{}{}{s.f.}{Obra feita de madeira ou arame para delimitar uma área de terreno.}{cer.ca}{0}
\verb{cercado}{}{}{}{}{adj.}{Sitiado, rodeado.}{cer.ca.do}{0}
\verb{cercado}{}{}{}{}{s.m.}{Porção de terreno rodeada por cerca, muro, estacaria para fins específicos.}{cer.ca.do}{0}
\verb{cercadura}{}{}{}{}{s.f.}{Aquilo que guarnece ou adorna as bordas ou contornos de objeto ou desenho.}{cer.ca.du.ra}{0}
\verb{cercania}{}{}{}{}{s.f.}{Cercanias.}{cer.ca.ni.a}{0}
\verb{cercanias}{}{}{}{}{s.f.pl.}{Região localizada nos arredores de cidade ou povoado; imediações, vizinhança.}{cer.ca.ni.as}{0}
\verb{cercar}{}{}{}{}{v.t.}{Pôr ou fazer cerca.}{cer.car}{0}
\verb{cercar}{}{}{}{}{}{Rodear, circundar com muro ou equivalente.}{cer.car}{0}
\verb{cercar}{}{}{}{}{}{Fazer cerco; sitiar.}{cer.car}{\verboinum{2}}
\verb{cerce}{é}{}{}{}{adv.}{Pela parte mais baixa; rente.}{cer.ce}{0}
\verb{cerceamento}{}{}{}{}{s.m.}{Ato ou efeito de cercear.}{cer.ce.a.men.to}{0}
\verb{cercear}{}{}{}{}{v.t.}{Cortar pela base, pela raiz.}{cer.ce.ar}{0}
\verb{cercear}{}{}{}{}{}{Aparar as extremidades.}{cer.ce.ar}{0}
\verb{cercear}{}{Fig.}{}{}{}{Impedir, suprimir, desfazer.}{cer.ce.ar}{0}
\verb{cercear}{}{Fig.}{}{}{}{Impor limite; restringir, diminuir.}{cer.ce.ar}{\verboinum{4}}
\verb{cerco}{ê}{}{}{}{s.m.}{Ato ou efeito de cercear.}{cer.co}{0}
\verb{cerco}{ê}{}{}{}{}{Aquilo que cerca.}{cer.co}{0}
\verb{cerco}{ê}{}{}{}{}{Local cercado, geralmente para animais; cercado.}{cer.co}{0}
\verb{cerco}{ê}{}{}{}{}{Ação militar ou de caçadores para cercar determinado objetivo ou caça.}{cer.co}{0}
\verb{cerda}{é}{Zool.}{}{}{s.f.}{Pelo espesso, rígido e áspero de certos animais.}{cer.da}{0}
\verb{cerda}{é}{}{}{}{}{Fibra natural ou sintética de escovas, pincéis e utensílios semelhantes.}{cer.da}{0}
\verb{cerdo}{ê}{}{}{}{s.m.}{Porco.}{cer.do}{0}
\verb{cereal}{}{}{"-ais}{}{s.m.}{Nome comum de diversas plantas que produzem grãos alimentícios, como trigo, aveia, arroz.}{ce.re.al}{0}
\verb{cereal}{}{}{"-ais}{}{}{Os grãos produzidos por essas plantas.}{ce.re.al}{0}
\verb{cereal}{}{}{"-ais}{}{}{O alimento produzido com alguns desses grãos, inteiros ou moídos, como aveia, trigo.}{ce.re.al}{0}
\verb{cerealífero}{}{}{}{}{adj.}{Relativo a cereais.}{ce.re.a.lí.fe.ro}{0}
\verb{cerealífero}{}{}{}{}{}{Que produz cereais.}{ce.re.a.lí.fe.ro}{0}
\verb{cerealista}{}{}{}{}{adj.2g.}{Relativo ao comércio ou à produção de cereais.}{ce.re.a.lis.ta}{0}
\verb{cerealista}{}{}{}{}{s.2g.}{Indivíduo que comercializa ou produz cereais.}{ce.re.a.lis.ta}{0}
\verb{cerebelo}{ê}{Anat.}{}{}{s.m.}{Parte do encéfalo situada na parte posterior do tronco encefálico, responsável pela coordenação dos movimentos e pelo equilíbrio do corpo.}{ce.re.be.lo}{0}
\verb{cerebral}{}{}{"-ais}{}{adj.2g.}{Relativo ao cérebro.}{ce.re.bral}{0}
\verb{cerebral}{}{Fig.}{"-ais}{}{}{Diz"-se de pessoa ou atitude orientada predominantemente pelo raciocínio, em detrimento dos fatores emocionais e da sensibilidade.}{ce.re.bral}{0}
\verb{cerebrino}{}{}{}{}{adj.}{Imaginário, fantasioso, abstrato.}{ce.re.bri.no}{0}
\verb{cérebro}{}{Anat.}{}{}{s.m.}{Órgão do sistema nervoso central situado na caixa craniana, sendo parte do encéfalo, e responsável pela coordenação neural e pelo pensamento.}{cé.re.bro}{0}
\verb{cérebro}{}{Fig.}{}{}{}{Capacidade intelectual, inteligência, talento, raciocínio.}{cé.re.bro}{0}
\verb{cerebrospinal}{}{Anat.}{"-ais}{}{adj.2g.}{Relativo ao cérebro e à medula espinhal.}{ce.re.bros.pi.nal}{0}
\verb{cereja}{ê}{}{}{}{s.f.}{Fruto da cerejeira, vermelho, comestível e de sabor agridoce.}{ce.re.ja}{0}
\verb{cereja}{ê}{}{}{}{adj.2g.}{De tonalidade vermelha semelhante à cereja.}{ce.re.ja}{0}
\verb{cerejeira}{ê}{Bot.}{}{}{s.f.}{Árvore com casca lisa e cinzenta, flores brancas e frutos vermelhos comestíveis.}{ce.re.jei.ra}{0}
\verb{cerejeira}{ê}{Por ext.}{}{}{}{A madeira dessa árvore, muito utilizada em marcenaria.}{ce.re.jei.ra}{0}
\verb{cerífero}{}{}{}{}{adj.}{Que produz cera.}{ce.rí.fe.ro}{0}
\verb{cerimônia}{}{}{}{}{s.f.}{Conjunto de atos formais que compõem a execução de um rito.}{ce.ri.mô.nia}{0}
\verb{cerimônia}{}{}{}{}{}{A execução de qualquer solenidade.}{ce.ri.mô.nia}{0}
\verb{cerimônia}{}{}{}{}{}{Padrão de comportamento formal entre as pessoas; etiqueta.}{ce.ri.mô.nia}{0}
\verb{cerimônia}{}{}{}{}{}{Cortesia excessiva em determinado contexto ou ambiente.}{ce.ri.mô.nia}{0}
\verb{cerimonial}{}{}{"-ais}{}{adj.2g.}{Relativo a cerimônia.}{ce.ri.mo.ni.al}{0}
\verb{cerimonial}{}{}{"-ais}{}{s.m.}{Conjunto de formalidades a serem seguidas em ato solene.}{ce.ri.mo.ni.al}{0}
\verb{cerimonial}{}{}{"-ais}{}{}{Regra ou livro que contém essas formalidades.}{ce.ri.mo.ni.al}{0}
\verb{cerimonioso}{ô}{}{"-osos ⟨ó⟩}{"-osa ⟨ó⟩}{adj.}{Cheio de cerimônias.}{ce.ri.mo.ni.o.so}{0}
\verb{cerimonioso}{ô}{}{"-osos ⟨ó⟩}{"-osa ⟨ó⟩}{}{Que se comporta com cerimônia, eventualmente de maneira excessiva.}{ce.ri.mo.ni.o.so}{0}
\verb{cério}{}{Quím.}{}{}{s.m.}{Elemento químico metálico, cinzento, mole, maleável e dúctil, o mais abundante da família dos lantanídeos (terras"-raras); usado em  ligas, em camisas de lampião, na indústria de vidros e porcelanas etc. \elemento{58}{140.116}{Ce}.}{cé.rio}{0}
\verb{cernambi}{}{Zool.}{}{}{s.m.}{Pequeno molusco que vive enterrado na areia da praia, e é usado como alimento. }{cer.nam.bi}{0}
\verb{cerne}{é}{}{}{}{s.m.}{Parte mais interna e dura do lenho das árvores.}{cer.ne}{0}
\verb{cerne}{é}{Fig.}{}{}{}{Miolo, âmago, essência.}{cer.ne}{0}
\verb{cerol}{ó}{}{"-óis}{}{s.m.}{Massa de cera com que os sapateiros untam a linha para coser a sola.}{ce.rol}{0}
\verb{cerol}{ó}{Bras.}{"-óis}{}{}{Substância feita com cola de madeira e vidro moído, que é aplicada à linha do papagaio para torná"-la cortante.}{ce.rol}{0}
\verb{ceroma}{}{Zool.}{}{}{s.f.}{Membrana que reveste a base do bico de algumas aves.}{ce.ro.ma}{0}
\verb{ceroplastia}{}{}{}{}{s.f.}{Técnica de moldar figuras em cera.}{ce.ro.plas.ti.a}{0}
\verb{ceroplástica}{}{}{}{}{s.f.}{Ceroplastia.}{ce.ro.plás.ti.ca}{0}
\verb{ceroula}{ô}{}{}{}{s.f.}{Ceroulas.}{ce.rou.la}{0}
\verb{ceroulas}{}{}{}{}{s.f.pl.}{Peça do vestuário masculino que cobre o ventre e as pernas, usada por baixo das calças.}{ce.rou.las}{0}
\verb{cerração}{}{}{"-ões}{}{s.f.}{Nevoeiro espesso que impede a visibilidade; bruma.}{cer.ra.ção}{0}
\verb{cerrado}{}{}{}{}{adj.}{Fechado, denso, espesso, encoberto.}{cer.ra.do}{0}
\verb{cerrado}{}{}{}{}{}{Apertado, unido.}{cer.ra.do}{0}
\verb{cerrado}{}{Bot.}{}{}{s.m.}{Tipo de formação vegetal com vegetação herbácea abundante e árvores pequenas e tortuosas.}{cer.ra.do}{0}
\verb{cerrar}{}{}{}{}{v.t.}{Fechar.}{cer.rar}{0}
\verb{cerrar}{}{}{}{}{}{Unir, apertar.}{cer.rar}{0}
\verb{cerrar}{}{}{}{}{}{Cobrir"-se de nuvens (diz"-se do céu ou do tempo).}{cer.rar}{\verboinum{1}}
\verb{cerro}{ê}{}{}{}{s.m.}{Pequeno morro, geralmente íngreme e pedregoso.}{cer.ro}{0}
\verb{certa}{é}{}{}{}{adj.}{Usado na locução \textit{na certa}: com certeza; certamente; sem dúvida.}{cer.ta}{0}
\verb{certame}{}{}{}{}{s.m.}{Combate, luta, briga.}{cer.ta.me}{0}
\verb{certame}{}{}{}{}{}{Discussão, debate.}{cer.ta.me}{0}
\verb{certame}{}{}{}{}{}{Concurso, competição, torneio.}{cer.ta.me}{0}
\verb{certâmen}{}{}{}{}{}{Var. de \textit{certame}.}{cer.tâ.men}{0}
\verb{certeiro}{ê}{}{}{}{adj.}{Que acerta com precisão os seus objetivos.}{cer.tei.ro}{0}
\verb{certeiro}{ê}{}{}{}{}{Bem dirigido; perspicaz, sagaz.}{cer.tei.ro}{0}
\verb{certeiro}{ê}{}{}{}{}{Preciso, sensato, correto, acertado.}{cer.tei.ro}{0}
\verb{certeza}{ê}{}{}{}{s.f.}{Qualidade do que é certo.}{cer.te.za}{0}
\verb{certeza}{ê}{}{}{}{}{Conhecimento indiscutível.}{cer.te.za}{0}
\verb{certeza}{ê}{}{}{}{}{Convicção.}{cer.te.za}{0}
\verb{certeza}{ê}{}{}{}{}{Estabilidade, segurança nas ações.}{cer.te.za}{0}
\verb{certidão}{}{}{"-ões}{}{s.f.}{Documento legal com o qual se certifica determinado fato; atestado.}{cer.ti.dão}{0}
\verb{certificado}{}{}{}{}{adj.}{Dado como certo.}{cer.ti.fi.ca.do}{0}
\verb{certificado}{}{}{}{}{}{Contido em certidão.}{cer.ti.fi.ca.do}{0}
\verb{certificado}{}{}{}{}{s.m.}{O conteúdo de uma certidão.}{cer.ti.fi.ca.do}{0}
\verb{certificado}{}{}{}{}{}{Documento que certifica alguma coisa.}{cer.ti.fi.ca.do}{0}
\verb{certificar}{}{}{}{}{v.t.}{Afirmar a certeza.}{cer.ti.fi.car}{0}
\verb{certificar}{}{}{}{}{}{Dar por certo.}{cer.ti.fi.car}{0}
\verb{certificar}{}{}{}{}{}{Emitir certidão.}{cer.ti.fi.car}{0}
\verb{certificar}{}{}{}{}{}{Confirmar.}{cer.ti.fi.car}{0}
\verb{certificar}{}{}{}{}{v.pron.}{Convencer"-se da certeza.}{cer.ti.fi.car}{\verboinum{2}}
\verb{certo}{é}{}{}{}{adj.}{Em que não há erro; correto, exato, preciso.}{cer.to}{0}
\verb{certo}{é}{}{}{}{}{Que há de ocorrer necessariamente; infalível, seguro.}{cer.to}{0}
\verb{certo}{é}{}{}{}{}{Previamente determinado.}{cer.to}{0}
\verb{certo}{é}{}{}{}{}{Persuadido, convencido.}{cer.to}{0}
\verb{certo}{é}{}{}{}{}{Diz"-se de relógio ajustado de acordo com a hora oficial.}{cer.to}{0}
\verb{certo}{é}{}{}{}{pron.}{Não determinado; algum, um.}{cer.to}{0}
\verb{certo}{é}{}{}{}{s.m.}{A coisa ou escolha certa.}{cer.to}{0}
\verb{certo}{é}{}{}{}{adv.}{Com certeza; certamente.}{cer.to}{0}
\verb{certo}{é}{}{}{}{}{Com exatidão; de maneira precisa.}{cer.to}{0}
\verb{cerúleo}{}{}{}{}{adj.}{Que tem a cor do céu.}{ce.rú.leo}{0}
\verb{cerume}{}{Med.}{}{}{s.m.}{Secreção das glândulas do canal auditivo; cera do ouvido.}{ce.ru.me}{0}
\verb{cerúmen}{}{}{}{}{}{Var. de \textit{cerume}.}{ce.rú.men}{0}
\verb{cerva}{é}{}{}{}{s.f.}{Redução de \textit{cerveja}.}{cer.va}{0}
\verb{cervantesco}{ê}{}{}{}{adj.}{Relativo ao escritor espanhol Miguel de Cervantes.}{cer.van.tes.co}{0}
\verb{cerveja}{ê}{}{}{}{s.f.}{Bebida fermentada feita de cereais, especialmente cevada e lúpulo.}{cer.ve.ja}{0}
\verb{cervejada}{}{}{}{}{s.f.}{Festa em que se bebe muita cerveja ou que se oferece cerveja aos convidados.}{cer.ve.ja.da}{0}
\verb{cervejaria}{}{}{}{}{s.f.}{Local onde se fabrica, vende ou consome cerveja.}{cer.ve.ja.ri.a}{0}
\verb{cervejeiro}{ê}{}{}{}{adj.}{Relativo a cerveja.}{cer.ve.jei.ro}{0}
\verb{cervejeiro}{ê}{}{}{}{s.m.}{Indivíduo que fabrica ou comercializa cerveja.}{cer.ve.jei.ro}{0}
\verb{cervical}{}{}{"-ais}{}{adj.2g.}{Relativo a cerviz.}{cer.vi.cal}{0}
\verb{cervídeo}{}{Zool.}{}{}{s.m.}{Espécime dos cervídeos, família de mamíferos ruminantes que inclui os cervos, veados, alces e renas.}{cer.ví.deo}{0}
\verb{cervídeo}{}{Zool.}{}{}{adj.}{Relativo aos cervídeos.}{cer.ví.deo}{0}
\verb{cerviz}{}{Anat.}{}{}{s.f.}{Parte posterior do pescoço; nuca, cachaço.}{cer.viz}{0}
\verb{cervo}{ê}{Zool.}{}{}{s.m.}{Mamífero quadrúpede com pernas longas e cauda curta, sendo que os machos apresentam cornos simples ou ramificados; veado.}{cer.vo}{0}
\verb{cerzideira}{ê}{}{}{}{s.f.}{Agulha própria para cerzir.}{cer.zi.dei.ra}{0}
\verb{cerzideira}{ê}{}{}{}{}{Mulher que cerze.}{cer.zi.dei.ra}{0}
\verb{cerzidor}{ô}{}{}{}{adj.}{Que cerze, remenda.}{cer.zi.dor}{0}
\verb{cerzidor}{ô}{Pop.}{}{}{}{Diz"-se de quem escreve copiando ou compilando textos alheios.}{cer.zi.dor}{0}
\verb{cerzidura}{}{}{}{}{s.f.}{Ato ou efeito de cerzir.}{cer.zi.du.ra}{0}
\verb{cerzidura}{}{}{}{}{}{Pequena costura ou remendo.}{cer.zi.du.ra}{0}
\verb{cerzimento}{}{}{}{}{s.m.}{Cerzidura.}{cer.zi.men.to}{0}
\verb{cerzir}{}{}{}{}{v.t.}{Cozer com pontos miúdos.}{cer.zir}{\verboinum{30}}
\verb{cesárea}{}{}{}{}{s.f.}{Forma reduzida de \textit{cesariana}.}{ce.sá.re.a}{0}
\verb{cesáreo}{}{}{}{}{adj.}{Relativo a César, imperador romano; cesariano.}{ce.sá.re.o}{0}
\verb{cesariana}{}{}{}{}{s.f.}{Cirurgia em que se abre o ventre materno para retirar o feto, fazendo o parto.}{ce.sa.ri.a.na}{0}
\verb{cesarismo}{}{Hist.}{}{}{s.m.}{Governo dos césares romanos.}{ce.sa.ris.mo}{0}
\verb{cesarismo}{}{}{}{}{}{Governo em que uma só pessoa detém muito poder; despotismo.}{ce.sa.ris.mo}{0}
\verb{césio}{}{Quím.}{}{}{s.m.}{Elemento químico radioativo, cor de prata, mole e dúctil, do grupo dos metais alcalinos, utilizado em válvulas eletrônicas, células fotoelétricas, relógios atômicos, no tratamento do câncer etc. \elemento{55}{132.90546}{Cs}.}{cé.sio}{0}
\verb{cessação}{}{}{"-ões}{}{adj.}{Ato ou efeito de cessar; interrupção.}{ces.sa.ção}{0}
\verb{cessão}{}{}{"-ões}{}{s.f.}{Ato ou efeito de ceder.}{ces.são}{0}
\verb{cessão}{}{}{"-ões}{}{}{Transferência de posse.}{ces.são}{0}
\verb{cessão}{}{}{"-ões}{}{}{Desistência.}{ces.são}{0}
\verb{cessar}{}{}{}{}{v.t.}{Dar ou ter fim; parar.}{ces.sar}{\verboinum{1}}
\verb{cessar"-fogo}{ô}{}{}{}{s.m.}{Interrupção temporária ou definitiva das atividades de guerra.}{ces.sar"-fo.go}{0}
\verb{cessionário}{}{Jur.}{}{}{adj.}{Que se beneficia de uma cessão.}{ces.si.o.ná.rio}{0}
\verb{cesta}{ê}{}{}{}{s.f.}{Utensílio, geralmente de palha ou vime, para guardar ou transportar coisas.}{ces.ta}{0}
\verb{cesta}{ê}{}{}{}{}{A quantidade de objetos que uma cesta pode conter.}{ces.ta}{0}
\verb{cesteiro}{ê}{}{}{}{s.m.}{Fabricante ou vendedor de cestos ou cestas.}{ces.tei.ro}{0}
\verb{cestinha}{}{Esport.}{}{}{s.2g.}{No basquetebol, jogador que faz o maior número de pontos.}{ces.ti.nha}{0}
\verb{cesto}{ê}{}{}{}{s.m.}{Pequena cesta, geralmente sem asas.}{ces.to}{0}
\verb{cesura}{}{}{}{}{s.f.}{Incisão, corte, abertura.}{ce.su.ra}{0}
\verb{cesura}{}{}{}{}{}{Cicatriz.}{ce.su.ra}{0}
\verb{cetáceo}{}{Zool.}{}{}{adj.}{Relativo aos cetáceos, ordem de mamíferos aquáticos com nadadeira caudal horizontal e que inclui as baleias, golfinhos e botos.}{ce.tá.ceo}{0}
\verb{cetáceo}{}{}{}{}{s.m.}{Espécime dessa ordem.}{ce.tá.ceo}{0}
\verb{ceticismo}{}{}{}{}{s.m.}{Doutrina filosófica segundo a qual não existe certeza, e adota a dúvida como procedimento intelectual sistemático.}{ce.ti.cis.mo}{0}
\verb{ceticismo}{}{Por ext.}{}{}{}{Falta de crença; incredulidade, dúvida.}{ce.ti.cis.mo}{0}
\verb{cético}{}{}{}{}{adj.}{Adepto do ceticismo.}{cé.ti.co}{0}
\verb{cético}{}{Por ext.}{}{}{}{Que duvida de tudo; descrente.}{cé.ti.co}{0}
\verb{cetim}{}{}{}{}{s.m.}{Tecido de seda macio, liso e brilhante.}{ce.tim}{0}
\verb{cetinoso}{ô}{}{"-osos ⟨ó⟩}{"-osa ⟨ó⟩}{adj.}{Macio como o cetim; acetinado.}{ce.ti.no.so}{0}
%\verb{}{}{}{}{}{}{}{}{0}
\verb{cetro}{é}{}{}{}{s.m.}{Bastão curto e ornamentado que representa o poder real e é usado pelos reis.}{ce.tro}{0}
\verb{cetro}{é}{Fig.}{}{}{}{O poder do rei, do soberano ou de qualquer autoridade.}{ce.tro}{0}
\verb{cetro}{é}{Fig.}{}{}{}{Despotismo, tirania.}{ce.tro}{0}
\verb{céu}{}{}{}{}{s.m.}{Lugar onde estão e se movem os astros.}{céu}{0}
\verb{céu}{}{}{}{}{}{A parte do espaço visível da Terra; firmamento.}{céu}{0}
\verb{céu}{}{Relig.}{}{}{}{Lugar onde habita Deus, os anjos, as boas almas.}{céu}{0}
\verb{céu}{}{Fig.}{}{}{}{Lugar onde existe a plenitude, a harmonia, a felicidade.}{céu}{0}
\verb{céus}{}{}{}{}{interj.}{Expressão que denota surpresa, espanto, assombro.}{céus}{0}
\verb{ceva}{é}{}{}{}{s.f.}{Ato ou efeito de cevar; cevagem.}{ce.va}{0}
\verb{ceva}{é}{}{}{}{}{O alimento com que se cevam os animais.}{ce.va}{0}
\verb{ceva}{é}{}{}{}{}{Isca, engodo, chamariz.}{ce.va}{0}
\verb{cevada}{}{Bot.}{}{}{s.f.}{Planta gramínea com flores dispostas em espigas e frutos amarelados em forma de grão.}{ce.va.da}{0}
\verb{cevada}{}{Por ext.}{}{}{}{O grão dessas plantas, com o qual se fabrica cerveja.}{ce.va.da}{0}
\verb{cevado}{}{}{}{}{adj.}{Que se cevou; alimentado.}{ce.va.do}{0}
\verb{cevado}{}{}{}{}{}{Submetido à engorda.}{ce.va.do}{0}
\verb{cevadura}{}{}{}{}{s.f.}{Ato ou efeito de cevar; ceva.}{ce.va.du.ra}{0}
\verb{cevar}{}{}{}{}{v.t.}{Dar alimento a; nutrir.}{ce.var}{0}
\verb{cevar}{}{}{}{}{}{Tornar gordo; engordar.}{ce.var}{0}
\verb{cevar}{}{}{}{}{}{Pôr isca em.}{ce.var}{0}
\verb{cevar}{}{}{}{}{v.pron.}{Fartar"-se, saciar"-se.}{ce.var}{\verboinum{1}}
\verb{Cf}{}{Quím.}{}{}{}{Símb. do \textit{califórnio}.}{Cf}{0}
\verb{CFC}{}{Quím.}{}{}{s.m.}{Sigla de \textit{clorofluorcarboneto}, substância usada como propelente em aerossóis.}{CFC}{0}
\verb{chá}{}{}{}{}{s.m.}{Infusão preparada com ervas.}{chá}{0}
\verb{chã}{}{}{}{}{s.f.}{Extensão plana de terra; planície.}{chã}{0}
\verb{chacal}{}{Zool.}{"-ais}{}{s.m.}{Mamífero carnívoro encontrado na Ásia e na África.}{cha.cal}{0}
\verb{chácara}{}{}{}{}{s.f.}{Propriedade rural em que se criam aves, animais e se plantam frutas e legumes.}{chá.ca.ra}{0}
\verb{chácara}{}{}{}{}{}{Pequena propriedade em área não urbana, geralmente destinada ao lazer.}{chá.ca.ra}{0}
\verb{chacareiro}{ê}{}{}{}{s.m.}{Proprietário de chácara.}{cha.ca.rei.ro}{0}
\verb{chacareiro}{ê}{}{}{}{}{Indivíduo que cuida de hortas e jardins; jardineiro.}{cha.ca.rei.ro}{0}
\verb{chacina}{}{}{}{}{s.f.}{Matança de muitas pessoas, geralmente com crueldade.}{cha.ci.na}{0}
\verb{chacina}{}{}{}{}{}{Abate e esquartejamento de animais.}{cha.ci.na}{0}
\verb{chacina}{}{}{}{}{}{Carne cortada em postas, salgada e curada.}{cha.ci.na}{0}
\verb{chacinar}{}{}{}{}{v.t.}{Matar, massacrar.}{cha.ci.nar}{0}
\verb{chacinar}{}{}{}{}{}{Esquartejar carne de animais.}{cha.ci.nar}{0}
\verb{chacinar}{}{}{}{}{}{Salgar e curar carne.}{cha.ci.nar}{\verboinum{1}}
\verb{chacoalhar}{}{}{}{}{v.t.}{Balançar, sacudir, chocalhar.}{cha.co.a.lhar}{\verboinum{1}}
\verb{chacota}{ó}{}{}{}{s.f.}{Zombaria, escárnio, troça.}{cha.co.ta}{0}
\verb{chacotear}{}{}{}{}{v.t.}{Fazer troça; escarnecer.}{cha.co.te.ar}{\verboinum{4}}
\verb{chá"-da"-índia}{}{Bot.}{chás"-da"-índia}{}{s.f.}{Erva de folhas verdes escuras de que se faz chá.}{chá"-da"-ín.dia}{0}
\verb{chã"-de"-dentro}{}{}{chãs"-de"-dentro}{}{s.f.}{Músculo da coxa do boi com fibras macias.}{chã"-de"-den.tro}{0}
\verb{chã"-de"-fora}{}{}{chãs"-de"-fora}{}{s.f.}{Carne da coxa do boi de fibras grossas e razoavelmente macias.}{chã"-de"-fo.ra}{0}
\verb{chadiano}{}{}{}{}{adj.}{Relativo à República do Chade, país da África.}{cha.di.a.no}{0}
\verb{chadiano}{}{}{}{}{s.m.}{Indivíduo natural ou habitante desse país.}{cha.di.a.no}{0}
\verb{chafariz}{}{}{}{}{s.m.}{Construção, geralmente em locais públicos, com fontes de água.}{cha.fa.riz}{0}
\verb{chafurda}{}{}{}{}{s.f.}{Local em que vivem os porcos; chiqueiro.}{cha.fur.da}{0}
\verb{chafurda}{}{}{}{}{}{Qualidade do que é muito sujo; imundície, sujeira.}{cha.fur.da}{0}
\verb{chafurda}{}{}{}{}{}{Residência muito suja.}{cha.fur.da}{0}
\verb{chafurdar}{}{}{}{}{v.i.}{Revolver"-se na lama.}{cha.fur.dar}{0}
\verb{chafurdar}{}{Fig.}{}{}{v.t.}{Comprometer por motivo desonroso; macular.}{cha.fur.dar}{0}
\verb{chafurdar}{}{Fig.}{}{}{}{Envolver"-se em vícios; corromper.}{cha.fur.dar}{\verboinum{1}}
\verb{chaga}{}{}{}{}{s.f.}{Ferida aberta; úlcera.}{cha.ga}{0}
\verb{chaga}{}{}{}{}{}{Incisão na casca das árvores.}{cha.ga}{0}
\verb{chaga}{}{Fig.}{}{}{}{Coisa que penaliza.}{cha.ga}{0}
\verb{chaga}{}{}{}{}{}{Desgraça, infortúnio.}{cha.ga}{0}
\verb{chagado}{}{}{}{}{adj.}{Coberto de chagas.}{cha.ga.do}{0}
\verb{chagado}{}{Fig.}{}{}{}{Aflito, sofrido.}{cha.ga.do}{0}
\verb{chagásico}{}{}{}{}{adj.}{Relativo à doença de Chagas.}{cha.gá.si.co}{0}
\verb{chagásico}{}{}{}{}{s.m.}{Indivíduo que sofre da doença de Chagas.}{cha.gá.si.co}{0}
\verb{chairel}{é}{}{"-éis}{}{s.m.}{Revestimento de tecido ou couro, anteposto à sela ou albarda, que cobre a anca da cavalgadura.}{chai.rel}{0}
\verb{chalaça}{}{}{}{}{s.f.}{Dito zombeteiro; gracejo de mau gosto; caçoada, troça.}{cha.la.ça}{0}
\verb{chalacear}{}{}{}{}{v.i.}{Dizer chalaças; zombar, gracejar.}{cha.la.ce.ar}{\verboinum{4}}
\verb{chalaceiro}{ê}{}{}{}{adj.}{Diz"-se daquele que diz ou faz chalaças; gozador.}{cha.la.cei.ro}{0}
\verb{chalé}{}{}{}{}{s.m.}{Casa de campo, em estilo suíço, pequena, geralmente de madeira.}{cha.lé}{0}
\verb{chalé}{}{}{}{}{}{Casa rústica.}{cha.lé}{0}
\verb{chaleira}{ê}{}{}{}{s.f.}{Vasilha bojuda, de metal, com bico e tampa, onde se aquece água, especialmente para o preparo do chá.}{cha.lei.ra}{0}
\verb{chaleira}{ê}{Pop.}{}{}{adj.2g.}{Diz"-se daquele que é bajulador.}{cha.lei.ra}{0}
\verb{chaleirar}{}{Pop.}{}{}{v.t.}{Bajular, adular de modo servil.}{cha.lei.rar}{\verboinum{1}}
\verb{chalrar}{}{}{}{}{}{Emitir sons a pequenos intervalos (os pássaros).}{chal.rar}{\verboinum{1}}
\verb{chalrar}{}{}{}{}{}{Palrar, tagarelar.}{chal.rar}{0}
\verb{chalrar}{}{}{}{}{v.i.}{Falar, conversar à toa e alegremente.}{chal.rar}{0}
\verb{chalreada}{}{}{}{}{s.f.}{Ruído simultâneo de muitas vozes; falatório.}{chal.re.a.da}{0}
\verb{chalreada}{}{}{}{}{}{Chilreada, gorgeio de muitos pássaros.}{chal.re.a.da}{0}
\verb{chalrear}{}{}{}{}{}{Var. de \textit{chalrar}.}{chal.re.ar}{0}
\verb{chalupa}{}{}{}{}{s.f.}{Pequena embarcação a vela, de um só mastro, para cabotagem.}{cha.lu.pa}{0}
\verb{chalupa}{}{}{}{}{}{Barco de vela e remos.}{cha.lu.pa}{0}
\verb{chama}{}{}{}{}{s.f.}{Mistura gasosa e incandescente, acompanhada de luz e energia térmica, que se forma durante uma combustão.}{cha.ma}{0}
\verb{chamada}{}{Pop.}{}{}{}{Reprimenda, xingamento.}{cha.ma.da}{0}
\verb{chamada}{}{}{}{}{s.f.}{Ato ou efeito de chamar; chamamento.}{cha.ma.da}{0}
\verb{chamada}{}{}{}{}{}{Ato de chamar as pessoas pelo nome para constatar a sua presença.}{cha.ma.da}{0}
\verb{chamada}{}{}{}{}{}{Toque de reunir.}{cha.ma.da}{0}
\verb{chamada}{}{}{}{}{}{Sinal, sobretudo em textos, para chamar a atenção para notas de rodapé, citações etc.}{cha.ma.da}{0}
\verb{chamado}{}{}{}{}{adj.}{Que foi convidado, escolhido, designado.}{cha.ma.do}{0}
\verb{chamado}{}{}{}{}{}{Denominado, apelidado.}{cha.ma.do}{0}
\verb{chamado}{}{}{}{}{s.m.}{Ato ou efeito de chamar; chamada, chamamento.}{cha.ma.do}{0}
\verb{chamalote}{ó}{}{}{}{s.m.}{Tecido sedoso com reflexos ondulantes.}{cha.ma.lo.te}{0}
\verb{chamamento}{}{}{}{}{s.m.}{Ato ou efeito de chamar; chamado, convocação.}{cha.ma.men.to}{0}
\verb{chamar}{}{}{}{}{v.t.}{Mandar vir; convocar.}{cha.mar}{0}
\verb{chamar}{}{}{}{}{}{Dar nome; designar, qualificar.}{cha.mar}{0}
\verb{chamar}{}{}{}{}{}{Dizer o nome de alguém, aguardando comunicação, aproximação ou indicação de presença.}{cha.mar}{\verboinum{1}}
\verb{chamarisco}{}{}{}{}{s.m.}{O que chama a atenção ou que é usado com essa finalidade; atrativo, isca.}{cha.ma.ris.co}{0}
\verb{chamariz}{}{}{}{}{s.m.}{Coisa que chama, que atrai.}{cha.ma.riz}{0}
\verb{chamariz}{}{Fig.}{}{}{}{Engodo, isca.}{cha.ma.riz}{0}
\verb{chamariz}{}{}{}{}{}{Pássaro engaiolado ou amarrado a fim de atrair outro para a armadilha.}{cha.ma.riz}{0}
\verb{chá"-mate}{}{}{chás"-mate \textit{ou} chás"-mates}{}{s.m.}{As folhas secas da erva"-mate.}{chá"-ma.te}{0}
\verb{chá"-mate}{}{}{chás"-mate \textit{ou} chás"-mates}{}{}{A bebida feita com a infusão dessas folhas.}{chá"-ma.te}{0}
\verb{chá"-mate}{}{}{chás"-mate \textit{ou} chás"-mates}{}{}{Mate.}{chá"-ma.te}{0}
\verb{chamativo}{}{}{}{}{adj.}{Que desperta ou atrai a atenção por sua vivacidade; atraente.}{cha.ma.ti.vo}{0}
\verb{chambre}{}{}{}{}{s.m.}{Roupão caseiro.}{cham.bre}{0}
\verb{chamego}{ê}{}{}{}{s.m.}{Amizade íntima; aproximação estreita; apego.}{cha.me.go}{0}
\verb{chamego}{ê}{}{}{}{}{Namoro.}{cha.me.go}{0}
\verb{chamego}{ê}{}{}{}{}{Paixão violenta.}{cha.me.go}{0}
\verb{chamego}{ê}{}{}{}{}{Excitação para atos libidinosos.}{cha.me.go}{0}
\verb{chamejante}{}{}{}{}{adj.2g.}{Que chameja, que arde em chamas.}{cha.me.jan.te}{0}
\verb{chamejar}{}{}{}{}{v.i.}{Deitar chamas; arder.}{cha.me.jar}{0}
\verb{chamejar}{}{Por ext.}{}{}{}{Brilhar, resplandecer.}{cha.me.jar}{0}
\verb{chamejar}{}{}{}{}{}{Irritar"-se, encolerizar"-se.}{cha.me.jar}{0}
\verb{chamejar}{}{Por ext.}{}{}{v.t.}{Emitir como chamas; dardejar.}{cha.me.jar}{\verboinum{1}}
\verb{chaminé}{}{}{}{}{s.f.}{Conduto ou tubo para dar saída à fumaça de fogão, forno etc.}{cha.mi.né}{0}
\verb{chaminé}{}{}{}{}{}{Lareira, fogão para aquecer as salas.}{cha.mi.né}{0}
\verb{champanha}{}{}{}{}{s.f.}{Champanhe.}{cham.pa.nha}{0}
\verb{champanhe}{}{}{}{}{s.m.}{Vinho espumante, branco ou rosado, originário da região de Champagne, na França.}{cham.pa.nhe}{0}
\verb{chamusca}{}{}{}{}{s.f.}{Ato ou efeito de chamuscar; queimadura de baixo grau.}{cha.mus.ca}{0}
\verb{chamuscado}{}{}{}{}{adj.}{Que está levemente queimado.}{cha.mus.ca.do}{0}
\verb{chamuscar}{}{}{}{}{v.t.}{Passar alguma coisa pelo fogo; queimar de leve.}{cha.mus.car}{\verboinum{2}}
\verb{chamusco}{}{}{}{}{s.m.}{Queima ligeira daquilo que se passa pelo fogo.}{cha.mus.co}{0}
\verb{chamusco}{}{}{}{}{}{Cheiro de coisa queimada.}{cha.mus.co}{0}
\verb{chamusco}{}{Fig.}{}{}{}{Suspeita ou iminência de conflito, perigo.}{cha.mus.co}{0}
\verb{chanca}{}{Pop.}{}{}{s.f.}{Pé grande.}{chan.ca}{0}
\verb{chanca}{}{}{}{}{}{Calçado largo e grosseiro.}{chan.ca}{0}
\verb{chanca}{}{}{}{}{}{Perna masculina delgada e alta.}{chan.ca}{0}
\verb{chança}{}{}{}{}{s.f.}{Dito zombeteiro; troça.}{chan.ça}{0}
\verb{chança}{}{}{}{}{}{Vaidade, presunção.}{chan.ça}{0}
\verb{chance}{}{}{}{}{s.f.}{Ocasião favorável; oportunidade.}{chan.ce}{0}
\verb{chancela}{é}{}{}{}{s.f.}{Ato ou efeito de chancelar.}{chan.ce.la}{0}
\verb{chancela}{é}{}{}{}{}{Selo pendente, em alguns documentos oficiais.}{chan.ce.la}{0}
\verb{chancela}{é}{}{}{}{}{Rubrica.}{chan.ce.la}{0}
\verb{chancelar}{}{}{}{}{v.t.}{Pôr chancela; carimbar, assinar, selar.}{chan.ce.lar}{0}
\verb{chancelar}{}{}{}{}{}{Achar justo; aprovar.}{chan.ce.lar}{\verboinum{1}}
\verb{chancelaria}{}{}{}{}{s.f.}{Repartição que põe chancela em alguns documentos.}{chan.ce.la.ri.a}{0}
\verb{chancelaria}{}{}{}{}{}{O Ministério das Relações Exteriores, em alguns países.}{chan.ce.la.ri.a}{0}
\verb{chancelaria}{}{}{}{}{}{Sede administrativa de uma embaixada.}{chan.ce.la.ri.a}{0}
\verb{chanceler}{é}{}{}{}{s.m.}{Antigo magistrado a quem incumbia a guarda do selo real.}{chan.ce.ler}{0}
\verb{chanceler}{é}{}{}{}{}{Funcionário encarregado de chancelar documentos ou diplomas.}{chan.ce.ler}{0}
\verb{chanceler}{é}{}{}{}{}{Ministro das Relações Exteriores em alguns países, e chefe do governo em outros.}{chan.ce.ler}{0}
\verb{chanchada}{}{}{}{}{s.f.}{Peça teatral ou filme sem valor, em que predominam as graças vulgares ou a pornografia.}{chan.cha.da}{0}
\verb{chanchada}{}{}{}{}{}{Qualquer espetáculo de pouco ou nenhum valor.}{chan.cha.da}{0}
\verb{chanfalho}{}{}{}{}{s.m.}{Espada velha, ferrugenta e sem corte.}{chan.fa.lho}{0}
\verb{chanfalho}{}{}{}{}{}{Instrumento desafinado.}{chan.fa.lho}{0}
\verb{chanfradura}{}{}{}{}{s.f.}{Ato ou efeito de chanfrar.}{chan.fra.du.ra}{0}
\verb{chanfradura}{}{}{}{}{}{Recorte nas extremidades de um objeto ou terreno.}{chan.fra.du.ra}{0}
\verb{chanfrar}{}{}{}{}{v.t.}{Cortar em ângulo ou de esguelha; recortar em forma de meia"-lua.}{chan.frar}{0}
\verb{chanfrar}{}{Fig.}{}{}{}{Falar mal de alguém, na ausência.}{chan.frar}{\verboinum{1}}
\verb{chanfro}{}{}{}{}{s.m.}{Recorte em meia"-lua ou oblíquo.}{chan.fro}{0}
\verb{chantagear}{}{}{}{}{v.t.}{Fazer chantagem.}{chan.ta.ge.ar}{\verboinum{4}}
\verb{chantagem}{}{}{"-ens}{}{s.f.}{Ato de extorquir dinheiro, favores ou vantagens a alguém mediante ameaças.}{chan.ta.gem}{0}
\verb{chantagista}{}{}{}{}{s.2g.}{Indivíduo que pratica chantagem.}{chan.ta.gis.ta}{0}
\verb{chantili}{}{Cul.}{}{}{s.m.}{Creme branco, doce, de consistência leve e espumosa, feito de creme de leite batido.}{chan.ti.li}{0}
\verb{chantre}{}{}{}{}{s.m.}{Eclesiástico encarregado da direção dos coros nas igrejas e capelas.}{chan.tre}{0}
\verb{chão}{}{}{chãos}{}{adj.}{Que é plano, liso.}{chão}{0}
\verb{chão}{}{}{chãos}{}{}{Que é tranquilo, sereno.}{chão}{0}
\verb{chão}{}{}{chãos}{}{}{Que é singelo, sem enfeites.}{chão}{0}
\verb{chão}{}{}{chãos}{}{}{Vulgar, rasteiro.}{chão}{0}
\verb{chão}{}{}{chãos}{}{s.m.}{Solo.}{chão}{0}
\verb{chão}{}{}{chãos}{}{}{Terra plana.}{chão}{0}
\verb{chão}{}{}{chãos}{}{}{Pequena propriedade de terra.}{chão}{0}
\verb{chapa}{}{}{}{}{s.f.}{Qualquer peça lisa e pouco espessa, feita de material consistente, como metal, madeira, vidro etc.}{cha.pa}{0}
\verb{chapa}{}{}{}{}{}{Fotografia de alguma parte interna do corpo; radiografia.}{cha.pa}{0}
\verb{chapa}{}{}{}{}{}{Peça de metal colocada nos carros com o número do licenciamento; placa.}{cha.pa}{0}
\verb{chapa}{}{Pop.}{}{}{s.m.}{Camarada.}{cha.pa}{0}
\verb{chapada}{}{Geogr.}{}{}{s.f.}{Área de terra de dimensões consideráveis, situada a uma certa altitude, cujo topo é relativamente plano, limitada por nítidas rupturas de declive, formando às vezes escarpas abruptas.}{cha.pa.da}{0}
\verb{chapada}{}{Geogr.}{}{}{}{Extensão de terreno de superfície regular e horizontal; tipo de planalto extenso.}{cha.pa.da}{0}
\verb{chapada}{}{Pop.}{}{}{}{Pancada ou bofetada em cheio.}{cha.pa.da}{0}
\verb{chapadão}{}{}{"-ões}{}{s.m.}{Chapada extensa.}{cha.pa.dão}{0}
\verb{chapadão}{}{}{"-ões}{}{}{Série de chapadas.}{cha.pa.dão}{0}
\verb{chapado}{}{}{}{}{adj.}{Guarnecido de chapas ou lâminas de metal.}{cha.pa.do}{0}
\verb{chapado}{}{Pop.}{}{}{}{Que é completo, perfeito.}{cha.pa.do}{0}
\verb{chapado}{}{Pop.}{}{}{}{Diz"-se do indivíduo que se tornou cansado, prostrado e sonolento, devido à ingestão de tóxicos, bebidas e alimentos.}{cha.pa.do}{0}
\verb{chapar}{}{}{}{}{v.t.}{Pôr chapa.}{cha.par}{0}
\verb{chapar}{}{}{}{}{}{Segurar com chapas.}{cha.par}{0}
\verb{chapar}{}{}{}{}{}{Marcar, cunhar.}{cha.par}{0}
\verb{chapar}{}{}{}{}{}{Ornar, guarnecer.}{cha.par}{0}
\verb{chapar}{}{}{}{}{v.i.}{Tornar"-se cansado, prostrado e sonolento devido à ingestão de tóxicos, bebidas e alimentos.}{cha.par}{0}
\verb{chapar}{}{}{}{}{v.pron.}{Estatelar"-se.}{cha.par}{\verboinum{1}}
\verb{chapear}{}{}{}{}{v.t.}{Cobrir, revestir ou guarnecer de chapas.}{cha.pe.ar}{0}
\verb{chapear}{}{}{}{}{}{Fazer com argamassa revestimento áspero e desigual, em paredes e tetos.}{cha.pe.ar}{\verboinum{4}}
\verb{chapeirão}{}{}{"-ões}{}{s.m.}{Chapéu grande, de abas largas.}{cha.pei.rão}{0}
\verb{chapeirão}{}{Bras.}{"-ões}{}{}{Recife que aflora à superfície, com forma de cogumelo.}{cha.pei.rão}{0}
\verb{chapelão}{}{}{"-ões}{}{s.m.}{Chapéu grande; chapeirão.}{cha.pe.lão}{0}
\verb{chapelaria}{}{}{}{}{s.f.}{Estabelecimento onde se fabricam ou se vendem chapéus.}{cha.pe.la.ri.a}{0}
\verb{chapeleira}{ê}{}{}{}{s.f.}{Mulher que fabrica ou vende chapéus. }{cha.pe.lei.ra}{0}
\verb{chapeleira}{ê}{}{}{}{}{Caixa onde se guardam chapéus; porta"-chapéus.}{cha.pe.lei.ra}{0}
\verb{chapeleiro}{ê}{}{}{}{s.m.}{Pessoa que faz ou vende chapéus.}{cha.pe.lei.ro}{0}
\verb{chapeleta}{ê}{}{}{}{s.f.}{Chapéu pequeno; chapelete.}{cha.pe.le.ta}{0}
\verb{chapelete}{ê}{}{}{}{s.m.}{Chapeleta.}{cha.pe.le.te}{0}
\verb{chapéu}{}{}{}{}{s.m.}{Cobertura para a cabeça, feita de diversos materiais, como palha, feltro etc., com abas e copa. (\textit{Ele pôs o chapéu e foi embora.})}{cha.péu}{0}
\verb{chapéu}{}{Fig.}{}{}{}{No futebol, tipo de drible em que se chuta a bola por cima do adversário, recuperando"-a na frente. (\textit{Ele deu um chapéu no jogador e tirou"-o da jogada.})}{cha.péu}{0}
\verb{chapéu"-de"-chuva}{}{}{chapéus"-de"-chuva}{}{s.m.}{Guarda"-chuva.}{cha.péu"-de"-chu.va}{0}
\verb{chapéu"-de"-sol}{ó}{}{chapéus"-de"-sol ⟨ó⟩}{}{s.m.}{Amendoeira"-da"-praia.}{cha.péu"-de"-sol}{0}
\verb{chapéu"-de"-sol}{ó}{}{chapéus"-de"-sol ⟨ó⟩}{}{}{Guarda"-chuva, ou guarda"-sol.}{cha.péu"-de"-sol}{0}
\verb{chapinhar}{}{}{}{}{v.i.}{Agitar água ou lama com as mãos ou os pés. (\textit{As crianças gostam de chapinhar nas poças d'água, depois da chuva.})}{cha.pi.nhar}{\verboinum{1}}
\verb{chapisco}{}{}{}{}{s.m.}{Ato ou efeito de chapiscar.}{cha.pis.co}{0}
\verb{charada}{}{}{}{}{s.f.}{Brincadeira que consiste em adivinhar palavras ou expressões por meio de indicações sobre suas sílabas, seus sinônimos etc.}{cha.ra.da}{0}
\verb{charadista}{}{}{}{}{s.2g.}{Pessoa que faz ou decifra charadas.}{cha.ra.dis.ta}{0}
\verb{charadista}{}{}{}{}{adj.2g.}{Relativo a charada.}{cha.ra.dis.ta}{0}
\verb{charanga}{}{}{}{}{s.f.}{Pequena banda de música.}{cha.ran.ga}{0}
\verb{charão}{}{}{}{}{s.m.}{Verniz de laca originário da China e do Japão, ou obra envernizada com ele. }{cha.rão}{0}
\verb{charco}{}{}{}{}{s.m.}{Pântano coberto de vegetação; banhado, brejo.}{char.co}{0}
\verb{charge}{}{}{}{}{s.f.}{Desenho ou caricatura que critica ou ridiculariza, com humor, sarcasmo ou ironia, uma pessoa, um fato ou uma situação; caricatura. }{char.ge}{0}
\verb{chargista}{}{}{}{}{s.2g.}{Pessoa que faz charge.}{char.gis.ta}{0}
\verb{charlar}{}{}{}{}{v.i.}{Conversar à toa, sem assunto determinado; tagarelar.}{char.lar}{\verboinum{1}}
\verb{charlatanear}{}{}{}{}{v.i.}{Agir ou falar como um charlatão.}{char.la.ta.ne.ar}{\verboinum{4}}
\verb{charlatanesco}{ê}{}{}{}{adj.}{Referente a ou próprio de charlatão.}{char.la.ta.nes.co}{0}
\verb{charlatanice}{}{}{}{}{s.f.}{Qualidade, condição, modos, ação ou linguagem de charlatão; charlatanismo.}{char.la.ta.ni.ce}{0}
\verb{charlatanismo}{}{}{}{}{s.m.}{Charlatanice.}{char.la.ta.nis.mo}{0}
\verb{charlatão}{}{}{"-ães \textit{ou} -ões}{"-ã \textit{ou} -ona}{s.m.}{Vendedor ambulante de remédios, que anuncia com espalhafato propriedades que as suas drogas não têm, enganando as pessoas.}{char.la.tão}{0}
\verb{charlatão}{}{Por ext.}{"-ães \textit{ou} -ões}{"-ã \textit{ou} -ona}{}{Médico ignorante, incompetente e inescrupuloso.}{char.la.tão}{0}
\verb{charlatão}{}{Por ext.}{"-ães \textit{ou} -ões}{"-ã \textit{ou} -ona}{}{Pessoa que se faz passar por médico, exercendo a medicina sem estar legalmente autorizado, possuindo apenas alguns conhecimentos empíricos.}{char.la.tão}{0}
\verb{charlatão}{}{}{"-ães \textit{ou} -ões}{"-ã \textit{ou} -ona}{}{Pessoa esperta, desonesta, que explora a boa"-fé do público, anunciando habilidades e qualidades que não possui para obter dinheiro ou vantagens; impostor, trapaceiro, embusteiro, mistificador.}{char.la.tão}{0}
\verb{charleston}{}{Mús.}{}{}{s.m.}{Tipo de foxtrote sincopado, muito em voga em 1920.}{\textit{charleston}}{0}
\verb{charme}{}{}{}{}{s.m.}{Conjunto de características que dão a uma pessoa o poder de encantar, atrair ou seduzir; simpatia, encanto, atração, sedução.}{char.me}{0}
\verb{charmoso}{ô}{}{"-osos ⟨ó⟩}{"-osa ⟨ó⟩}{adj.}{Que tem charme; atraente, encantador, sedutor.}{char.mo.so}{0}
\verb{charneca}{é}{Bras.}{}{}{s.f.}{Pântano.}{char.ne.ca}{0}
\verb{charneca}{é}{}{}{}{}{Vegetação rasteira, resistente, que medra em terrenos áridos e arenosos.}{char.ne.ca}{0}
\verb{charola}{ó}{}{}{}{s.f.}{Andor de procissão.}{cha.ro.la}{0}
\verb{charola}{ó}{}{}{}{}{Corredor semicircular das igrejas. }{cha.ro.la}{0}
\verb{charque}{}{}{}{}{s.m.}{Carne de vaca, salgada e seca ao sol em mantas; carne"-seca, jabá.}{char.que}{0}
\verb{charqueada}{}{}{}{}{s.f.}{Estabelecimento onde se prepara o charque.}{char.que.a.da}{0}
\verb{charquear}{}{}{}{}{v.i.}{Preparar o charque.}{char.que.ar}{\verboinum{4}}
\verb{charrete}{é}{}{}{}{s.f.}{Carro de duas rodas puxado por um ou dois cavalos.}{char.re.te}{0}
\verb{charrua}{}{}{}{}{s.f.}{Arado grande, de ferro.}{char.ru.a}{0}
\verb{charter}{}{}{}{}{s.m.}{Fretamento ou aluguel de avião ou outro meio de transporte para viagens turísticas.}{\textit{charter}}{0}
\verb{charutaria}{}{Bras.}{}{}{s.f.}{Estabelecimento onde se vendem charutos, cigarros, fumo etc.; tabacaria.}{cha.ru.ta.ri.a}{0}
\verb{charuteira}{ê}{}{}{}{s.f.}{Pequena caixa para guardar charutos.}{cha.ru.tei.ra}{0}
\verb{charuteiro}{ê}{}{}{}{s.m.}{Proprietário de charutaria.}{cha.ru.tei.ro}{0}
\verb{charuteiro}{ê}{}{}{}{}{Operário que faz charutos.}{cha.ru.tei.ro}{0}
\verb{charuto}{}{}{}{}{s.m.}{Rolo de folhas secas de fumo, para fumar.   }{cha.ru.to}{0}
\verb{charuto}{}{Cul.}{}{}{}{Qualquer alimento em forma de charuto.}{cha.ru.to}{0}
\verb{chasco}{}{}{}{}{s.m.}{Zombaria, gracejo.}{chas.co}{0}
\verb{chasquear}{}{}{}{}{v.t.}{Dizer chasco; zombar, gracejar.}{chas.que.ar}{\verboinum{4}}
\verb{chassi}{}{}{}{}{s.m.}{Estrutura de aço sobre a qual se monta a carroceria de um automóvel.}{chas.si}{0}
\verb{chassis}{}{}{}{}{s.m.}{Chassi.}{chas.sis}{0}
\verb{chat}{}{Informát.}{}{}{s.m.}{Conversa, bate"-papo por meio de mensagens escritas, em tempo real, através da internet.}{\textit{chat}}{0}
\verb{chata}{}{}{}{}{s.f.}{Barcaça larga, pouco funda e de fundo chato.}{cha.ta}{0}
\verb{chateação}{}{Pop.}{"-ões}{}{s.f.}{Ato ou efeito de chatear.}{cha.te.a.ção}{0}
\verb{chateação}{}{}{"-ões}{}{}{Aquilo que chateia; aborrecimento; amolação.}{cha.te.a.ção}{0}
\verb{chatear}{}{}{}{}{v.t.}{Aborrecer, importunar, amolar.}{cha.te.ar}{\verboinum{4}}
\verb{chateza}{ê}{}{}{}{s.f.}{Qualidade do que é chato, plano.}{cha.te.za}{0}
\verb{chateza}{ê}{}{}{}{}{Atributo do que ou de quem é chato, maçante.}{cha.te.za}{0}
\verb{chatice}{}{}{}{}{s.f.}{Qualidade do que é chato, maçante.}{cha.ti.ce}{0}
\verb{chato}{}{}{}{}{adj.}{Que é sem relevo; liso, plano.}{cha.to}{0}
\verb{chato}{}{Pop.}{}{}{}{Que é maçante, importuno, enfadonho.}{cha.to}{0}
\verb{chato}{}{Pop.}{}{}{s.m.}{Aquilo que aborrece, irrita ou preocupa.}{cha.to}{0}
\verb{chato}{}{}{}{}{}{Inseto anopluro, que vive principalmente na região pubiana, cujos ovos são postos na base dos pelos.}{cha.to}{0}
\verb{chauvinismo}{}{}{}{}{s.m.}{Nacionalismo exagerado.}{chau.vi.ni.smo}{0}
\verb{chauvinismo}{}{}{}{}{}{Atitude própria de chauvinista.}{chau.vi.ni.smo}{0}
\verb{chauvinista}{}{}{}{}{adj.2g.}{Relativo ao chauvinismo.}{chau.vi.nis.ta}{0}
\verb{chauvinista}{}{}{}{}{}{Diz"-se de pessoa que tem patriotismo exagerado.}{chau.vi.nis.ta}{0}
\verb{chavão}{}{}{"-ões}{}{s.m.}{Forma ou molde para bolos e massas.}{cha.vão}{0}
\verb{chavão}{}{}{"-ões}{}{}{Modelo, padrão.}{cha.vão}{0}
\verb{chavão}{}{}{"-ões}{}{}{Dito ou frase que se repete abusivamente, perdendo o valor expressivo; clichê. }{cha.vão}{0}
\verb{chave}{}{}{}{}{s.f.}{Peça com que se abre e fecha alguma coisa.}{cha.ve}{0}
\verb{chave}{}{}{}{}{}{Instrumento próprio para fazer alguma coisa girar.}{cha.ve}{0}
\verb{chave}{}{}{}{}{}{Elemento decisivo.}{cha.ve}{0}
\verb{chave}{}{}{}{}{}{Sinal que abrange os elementos de um conjunto.}{cha.ve}{0}
\verb{chaveiro}{ê}{}{}{}{s.m.}{Indivíduo que guarda chaves.}{cha.vei.ro}{0}
\verb{chaveiro}{ê}{}{}{}{}{Indivíduo que faz ou conserta chaves.}{cha.vei.ro}{0}
\verb{chaveiro}{ê}{}{}{}{}{Objeto em que se prendem chaves.}{cha.vei.ro}{0}
\verb{chavelho}{ê}{}{}{}{s.m.}{Chifre.}{cha.ve.lho}{0}
\verb{chávena}{}{}{}{}{s.f.}{Xícara ou taça para chá, café e outras bebidas.}{chá.ve.na}{0}
\verb{chaveta}{ê}{}{}{}{s.f.}{Peça para ajudar a segurar a roda na ponta do eixo.}{cha.ve.ta}{0}
\verb{chaveta}{ê}{}{}{}{}{Peça para segurar a cavilha.}{cha.ve.ta}{0}
\verb{chaveta}{ê}{}{}{}{}{Haste que une as duas partes de uma dobradiça.}{cha.ve.ta}{0}
\verb{checagem}{}{}{}{}{s.f.}{Ato ou efeito de checar, conferir.}{che.ca.gem}{0}
\verb{checar}{}{}{}{}{v.t.}{Estabelecer a veracidade, correção ou realidade de algo; conferir, verificar.}{che.car}{0}
\verb{checar}{}{}{}{}{}{Estabelecer comparação; confrontar.}{che.car}{\verboinum{2}}
\verb{check"-in}{}{}{}{}{s.m.}{Ato de registrar"-se em hotel.}{\textit{check"-in}}{0}
\verb{check"-in}{}{}{}{}{}{Ato de apresentar"-se em aeroporto ou outro porto de embarque para mostrar a passagem e despachar a bagagem.}{\textit{check"-in}}{0}
\verb{check"-out}{}{}{}{}{s.m.}{O ato de pagar a conta e sair de hotel.}{\textit{check"-out}}{0}
\verb{check"-up}{}{}{}{}{s.m.}{Checape.}{\textit{check"-up}}{0}
\verb{checo}{é}{}{}{}{}{Var. de \textit{tcheco}.}{che.co}{0}
\verb{chefão}{}{}{"-ões}{}{s.m.}{O chefe mais poderoso; manda"-chuva; patrão.}{che.fão}{0}
\verb{chefatura}{}{}{}{}{s.f.}{Repartição onde o chefe trabalha; chefia.}{che.fa.tu.ra}{0}
\verb{chefe}{é}{}{}{}{s.2g.}{Indivíduo que exerce autoridade, que  dirige, comanda ou governa.}{che.fe}{0}
\verb{chefia}{}{}{}{}{s.f.}{Cargo ou dignidade de chefe.}{che.fi.a}{0}
\verb{chefia}{}{}{}{}{}{Repartição onde o chefe exerce suas funções.}{che.fi.a}{0}
\verb{chefiar}{}{}{}{}{v.t.}{Dirigir, comandar ou governar na qualidade de chefe.}{che.fi.ar}{\verboinum{6}}
\verb{chega}{ê}{}{}{}{s.m.}{Repreensão, censura, reprimenda.}{che.ga}{0}
\verb{chega}{ê}{}{}{}{interj.}{Indica que se pare, que já basta; basta.}{che.ga}{0}
\verb{chegada}{}{}{}{}{s.f.}{Ato ou efeito de chegar.}{che.ga.da}{0}
\verb{chegada}{}{}{}{}{}{Aproximação.}{che.ga.da}{0}
\verb{chegada}{}{}{}{}{}{Fim, termo, conclusão.}{che.ga.da}{0}
\verb{chegado}{}{}{}{}{adj.}{Que é propenso, dado.}{che.ga.do}{0}
\verb{chegado}{}{}{}{}{}{Que tem grau de parentesco bastante próximo ou que é ligado por amizade; íntimo.}{che.ga.do}{0}
\verb{chegança}{}{}{}{}{s.f.}{Folguedo popular das festas de Natal, com danças e representações de cenas marítimas entre cristãos e mouros.}{che.gan.ça}{0}
\verb{chegança}{}{}{}{}{}{Visitas feitas pelos festeiros às residências por ocasião das festas de Natal e Reis.}{che.gan.ça}{0}
\verb{chegar}{}{}{}{}{v.i.}{Alcançar um determinado ponto.}{che.gar}{0}
\verb{chegar}{}{}{}{}{}{Vir o momento de alguma coisa acontecer. (\textit{O verão deve chegar mais cedo esse ano.})}{che.gar}{0}
\verb{chegar}{}{}{}{}{}{Ser suficiente; bastar.}{che.gar}{0}
\verb{chegar}{}{}{}{}{v.pron.}{Ir mais perto de pessoa ou coisa; aproximar"-se. (\textit{Chegou"-se a ele e deu"-lhe um abraço.})}{che.gar}{\verboinum{5}}
\verb{cheia}{ê}{}{}{}{s.f.}{Enchente, inundação.}{chei.a}{0}
\verb{cheia}{ê}{Fig.}{}{}{}{Grande número; grande quantidade; porção.}{chei.a}{0}
\verb{cheio}{ê}{}{}{}{adj.}{Onde nada mais falta; completo.}{chei.o}{0}
\verb{cheio}{ê}{}{}{}{}{Onde nada mais cabe; repleto.}{chei.o}{0}
\verb{cheio}{ê}{Pop.}{}{}{}{Aborrecido, farto.}{chei.o}{0}
\verb{cheirar}{}{}{}{}{v.t.}{Sentir o cheiro.}{chei.rar}{0}
\verb{cheirar}{}{}{}{}{}{Exalar cheiro.}{chei.rar}{0}
\verb{cheirar}{}{Fig.}{}{}{}{Procurar, bisbilhotar.}{chei.rar}{0}
\verb{cheirar}{}{Fig.}{}{}{}{Calcular, suspeitar.}{chei.rar}{\verboinum{1}}
\verb{cheireta}{}{}{}{}{}{Var. de \textit{xereta}.}{chei.re.ta}{0}
\verb{cheiro}{ê}{}{}{}{s.m.}{Sensação produzida no olfato pelas partículas que exalam odor.}{chei.ro}{0}
\verb{cheiroso}{ô}{}{"-osos ⟨ó⟩}{"-osa ⟨ó⟩}{adj.}{Que tem cheiro agradável.}{chei.ro.so}{0}
\verb{cheiro"-verde}{ê}{}{cheiros"-verdes ⟨ê⟩}{}{s.m.}{Ervas aromáticas, verdes, usadas como tempero, tais como a cebolinha, a salsa etc.}{chei.ro"-ver.de}{0}
\verb{cheque}{é}{}{}{}{s.m.}{Documento, normalmente fórmula impressa, por meio do qual o titular de uma conta emite ordem para o banco pagar ou transmitir certa quantia a seu favor ou a favor de um beneficiário.}{che.que}{0}
\verb{chereta}{ê}{}{}{}{adj. e s.2g.}{Ver \textit{xereta}.}{che.re.ta}{0}
\verb{cherne}{é}{Zool.}{}{}{s.m.}{Peixe do Mediterrâneo e do Atlântico que, quando jovem apresenta manchas brancas pelo corpo e quando adulto possui coloração escura; frequenta lugares lodosos e tem a carne bastante estimada no mercado.}{cher.ne}{0}
\verb{chéster}{}{Zool.}{}{}{s.m.}{Ave semelhante ao frango, porém maior, comercializada já pronta para consumo.}{chés.ter}{0}
\verb{chi}{}{}{}{}{interj.}{Expressão que denota espanto, decepção, impaciência ou impressão de perigo iminente.}{chi}{0}
\verb{chiado}{}{}{}{}{s.m.}{Ato ou efeito de chiar.}{chi.a.do}{0}
\verb{chiado}{}{}{}{}{}{Ruído desagradável, agudo e sibilante.}{chi.a.do}{0}
\verb{chiar}{}{}{}{}{v.i.}{Emitir chiados.}{chi.ar}{0}
\verb{chiar}{}{}{}{}{}{Produzir ruído áspero; ranger.}{chi.ar}{0}
\verb{chiar}{}{}{}{}{}{Esbravejar, vociferar.}{chi.ar}{0}
\verb{chiar}{}{}{}{}{}{Protestar, reclamar.}{chi.ar}{\verboinum{6}}
\verb{chibante}{}{}{}{}{adj.2g.}{Diz"-se de indivíduo brigão, valentão.}{chi.ban.te}{0}
\verb{chibante}{}{}{}{}{}{Diz"-se de indivíduo orgulhoso, altivo.}{chi.ban.te}{0}
\verb{chibante}{}{}{}{}{}{Diz"-se de indivíduo vestido com apuro exagerado; casquilho, janota.}{chi.ban.te}{0}
\verb{chibata}{}{}{}{}{s.f.}{Vara comprida e delgada, de junco, cipó, marmeleiro, usada para castigar ou dirigir animais; chicote.}{chi.ba.ta}{0}
\verb{chibatada}{}{}{}{}{s.f.}{Pancada com chibata; chicotada, relhada.}{chi.ba.ta.da}{0}
\verb{chibatear}{}{}{}{}{v.t.}{Bater com a chibata; chicotear.}{chi.ba.te.ar}{\verboinum{4}}
\verb{chibé}{}{}{}{}{s.m.}{Refresco de limão, com açúcar e farinha.}{chi.bé}{0}
\verb{chicana}{}{}{}{}{s.f.}{Contestação capciosa, em questões judiciais.}{chi.ca.na}{0}
\verb{chicana}{}{}{}{}{}{Astúcia, trapaça, tramoia.}{chi.ca.na}{0}
\verb{chicanear}{}{}{}{}{v.i.}{Usar de malícia ou de má"-fé no curso de uma questão ou de um processo judicial.}{chi.ca.ne.ar}{\verboinum{4}}
\verb{chicaneiro}{ê}{}{}{}{adj.}{Diz"-se de indivíduo que faz chicana, que trapaceia.}{chi.ca.nei.ro}{0}
\verb{chicanista}{}{}{}{}{adj.2g.}{Chicaneiro.}{chi.ca.nis.ta}{0}
\verb{chicle}{}{}{}{}{s.m.}{Goma insolúvel e pagajosa que flui do tronco do sapotizeiro, que é empregada como ingrediente na fabricação da goma de mascar.}{chi.cle}{0}
\verb{chicle}{}{Por ext.}{}{}{}{Goma de mascar; chiclete.}{chi.cle}{0}
\verb{chiclete}{é}{}{}{}{s.m.}{Goma de mascar; chicle.}{chi.cle.te}{0}
\verb{chico}{}{}{}{}{s.m.}{Macaco doméstico.}{chi.co}{0}
\verb{chico}{}{Pop.}{}{}{}{Menstruação.}{chi.co}{0}
\verb{chicória}{}{Bot.}{}{}{s.f.}{Planta muito cultivada em hortas, cujas folhas, amargas, são comestíveis; almeirão. }{chi.có.ria}{0}
\verb{chicotada}{}{}{}{}{s.f.}{Pancada com chicote; chibatada.}{chi.co.ta.da}{0}
\verb{chicote}{ó}{}{}{}{s.m.}{Correia comprida de couro ou trançado de cordas, presos a um cabo geralmente de madeira, que serve para açoitar; azorrague, chibata.}{chi.co.te}{0}
\verb{chicotear}{}{}{}{}{v.t.}{Dar chicotadas em; açoitar, chibatar.}{chi.co.te.ar}{\verboinum{4}}
\verb{chicote"-queimado}{ó}{}{chicotes"-queimados ⟨ó⟩}{}{s.m.}{Brincadeira infantil em que uma criança esconde um lenço enrolado,  cinta ou qualquer outro objeto que possa fazer as vezes do chicote, para que aquele que o encontre saia  chicoteando os demais.}{chi.co.te"-quei.ma.do}{0}
\verb{chifrada}{}{}{}{}{s.f.}{Golpe de chifre.}{chi.fra.da}{0}
\verb{chifrar}{}{}{}{}{v.t.}{Atacar com os chifres.}{chi.frar}{0}
\verb{chifrar}{}{Pop.}{}{}{}{Cometer adultério; trair, cornear.}{chi.frar}{\verboinum{1}}
\verb{chifre}{}{}{}{}{s.m.}{Cada um dos dois apêndices ósseos presentes na parte superior da cabeça de muitos ungulados; corno.}{chi.fre}{0}
\verb{chifrudo}{}{}{}{}{adj.}{Diz"-se de animal que possui chifres.}{chi.fru.do}{0}
\verb{chifrudo}{}{Pop.}{}{}{s.m.}{Indivíduo traído por aquele com quem mantém um relacionamento amoroso.}{chi.fru.do}{0}
\verb{chifrudo}{}{}{}{}{}{O diabo.}{chi.fru.do}{0}
\verb{chilenas}{}{}{}{}{s.f.pl.}{Grandes esporas de largas rosetas.}{chi.le.nas}{0}
\verb{chileno}{}{}{}{}{adj.}{Relativo a Chile.}{chi.le.no}{0}
\verb{chileno}{}{}{}{}{s.m.}{Indivíduo natural ou habitante desse país.}{chi.le.no}{0}
\verb{chilique}{}{}{}{}{s.m.}{Ataque nervoso ou histérico; faniquito.}{chi.li.que}{0}
\verb{chilique}{}{}{}{}{}{Perda dos sentidos; desmaio.}{chi.li.que}{0}
\verb{chilrada}{}{}{}{}{s.f.}{O chilrar dos pássaros.}{chil.ra.da}{0}
\verb{chilrada}{}{}{}{}{}{Muitos chilreios.}{chil.ra.da}{0}
\verb{chilrar}{}{}{}{}{v.i.}{Pipilar, gorjear.}{chil.rar}{0}
\verb{chilrar}{}{}{}{}{}{Falar muito; tagarelar.}{chil.rar}{\verboinum{1}}
\verb{chilreada}{}{}{}{}{s.f.}{Chilrada.}{chil.re.a.da}{0}
\verb{chilrear}{}{}{}{}{v.i.}{Chilrar.}{chil.re.ar}{\verboinum{4}}
\verb{chilreio}{ê}{}{}{}{s.m.}{Ato de chilrar; gorjeio.}{chil.rei.o}{0}
\verb{chilro}{}{}{}{}{s.m.}{Chilreio.}{chil.ro}{0}
\verb{chim}{}{}{}{}{adj.2g.}{Relativo à China.}{chim}{0}
\verb{chim}{}{}{}{}{s.2g.}{Indivíduo natural ou habitante da China.}{chim}{0}
\verb{chimango}{}{}{}{}{s.m.}{Alcunha que os federalistas aplicavam aos adeptos do Partido Republicano. }{chi.man.go}{0}
\verb{chimango}{}{}{}{}{}{Membro ou partidário do Partido Liberal, no século \textsc{xix}, no Rio Grande do Sul.}{chi.man.go}{0}
\verb{chimarrão}{}{}{"-ões}{}{s.m.}{Bebida de mate que se toma com bombilha, preparada com água quente. }{chi.mar.rão}{0}
\verb{chimpanzé}{}{Zool.}{}{}{s.m.}{Primata de lábios protuberantes, braços mais longos que as pernas e sem cauda.}{chim.pan.zé}{0}
\verb{china}{}{}{}{}{s.2g.}{Indivíduo natural ou habitante da China; chinês.}{chi.na}{0}
\verb{china}{}{Bras.}{}{}{s.f.}{Mulher ou moça do campo.}{chi.na}{0}
\verb{china}{}{}{}{}{s.f.}{Certa raça de gado bovino.}{chi.na}{0}
\verb{china}{}{Pejor.}{}{}{}{Prostituta.}{chi.na}{0}
\verb{chincha}{}{}{}{}{s.f.}{Tira de couro ou tecido forte que prende a sela ao animal de montaria; cincha.}{chin.cha}{0}
\verb{chinchila}{}{Zool.}{}{}{s.f.}{Pequeno roedor andino cuja pelagem densa e macia foi muito visada por caçadores por seu alto valor de troca.}{chin.chi.la}{0}
\verb{chinela}{é}{}{}{}{s.f.}{Pequeno calçado que não protege o calcanhar.}{chi.ne.la}{0}
\verb{chinela}{é}{}{}{}{}{Chinelo.}{chi.ne.la}{0}
\verb{chinelada}{}{}{}{}{s.f.}{Golpe desferido com chinelo ou chinela.}{chi.ne.la.da}{0}
\verb{chinelo}{é}{}{}{}{s.m.}{Calçado macio e que não se fixa ao pé, geralmente usado em ambientes informais.}{chi.ne.lo}{0}
\verb{chinês}{}{}{}{}{adj.}{Relativo a China.}{chi.nês}{0}
\verb{chinês}{}{}{}{}{s.m.}{Indivíduo natural ou habitante desse país.}{chi.nês}{0}
\verb{chinfrim}{}{}{}{}{adj.}{Que não presta; ordinário, insignificante.}{chin.frim}{0}
\verb{chinfrim}{}{}{}{}{}{De mau gosto.}{chin.frim}{0}
\verb{chinfrim}{}{Pop.}{}{}{s.m.}{Grande confusão; algazarra.}{chin.frim}{0}
\verb{chinfrinada}{}{Pop.}{}{}{s.f.}{Coisa ridícula, risível.}{chin.fri.na.da}{0}
\verb{chinó}{}{}{}{}{s.m.}{Cabeleira postiça; peruca.}{chi.nó}{0}
\verb{chio}{}{}{}{}{s.m.}{Chiado.}{chi.o}{0}
\verb{chip}{}{}{}{}{s.m.}{Componente eletrônico que contém um circuito miniaturizado produzido em uma peça de silício.}{chip}{0}
\verb{chipanzé}{}{}{}{}{}{Var. de \textit{chimpanzé}.}{chi.pan.zé}{0}
\verb{chique}{}{}{}{}{adj.}{Que demonstra elegância e bom gosto.}{chi.que}{0}
\verb{chiquê}{}{Pop.}{}{}{s.m.}{Pretensa elegância ou elegância afetada.}{chi.quê}{0}
\verb{chiqueiro}{ê}{}{}{}{s.m.}{Lugar onde se criam e recolhem os porcos; pocilga.}{chi.quei.ro}{0}
\verb{chiqueiro}{ê}{Fig.}{}{}{}{Lugar muito sujo.}{chi.quei.ro}{0}
\verb{chispa}{}{}{}{}{s.f.}{Faísca, centelha, fagulha.}{chis.pa}{0}
\verb{chispa}{}{}{}{}{}{Brilho forte e momentâneo.}{chis.pa}{0}
\verb{chispa}{}{Fig.}{}{}{}{Ideia, lampejo.}{chis.pa}{0}
\verb{chispada}{}{}{}{}{s.f.}{Corrida, disparada; ato ou efeito de chispar.}{chis.pa.da}{0}
\verb{chispar}{}{}{}{}{v.i.}{Soltar faíscas.}{chis.par}{0}
\verb{chispar}{}{Fig.}{}{}{}{Arder de cólera.}{chis.par}{0}
\verb{chispar}{}{Bras.}{}{}{}{Correr repentina e velozmente.}{chis.par}{\verboinum{1}}
\verb{chispe}{}{}{}{}{s.m.}{Pé de porco.}{chis.pe}{0}
\verb{chispe}{}{}{}{}{}{Sapato feminino de salto alto e bico fino.}{chis.pe}{0}
\verb{chiste}{}{}{}{}{s.m.}{Dito de humor fino e sutil; piada.}{chis.te}{0}
\verb{chiste}{}{}{}{}{}{Qualidade do que é cômico; graça.}{chis.te}{0}
\verb{chistoso}{ô}{}{"-osos ⟨ó⟩}{"-osa ⟨ó⟩}{adj.}{Em que há chiste; engraçado, jocoso.}{chis.to.so}{0}
\verb{chita}{}{}{}{}{s.f.}{Tecido de algodão de baixo valor, geralmente estampado em cores.}{chi.ta}{0}
\verb{chitão}{}{}{"-ões}{}{s.m.}{Chita com estampas grandes.}{chi.tão}{0}
\verb{choça}{ó}{}{}{}{s.f.}{Casebre, choupana, cabana.}{cho.ça}{0}
\verb{chocadeira}{ê}{}{}{}{s.f.}{Aparelho com aquecimento elétrico para chocar ovos.}{cho.ca.dei.ra}{0}
\verb{chocagem}{}{}{"-ens}{}{s.f.}{Ato de chocar.}{cho.ca.gem}{0}
\verb{chocalhar}{}{}{}{}{v.i.}{Tocar chocalho.}{cho.ca.lhar}{0}
\verb{chocalhar}{}{}{}{}{v.t.}{Agitar produzindo ruído semelhante ao do chocalho.}{cho.ca.lhar}{\verboinum{1}}
\verb{chocalho}{}{}{}{}{s.m.}{Brinquedo que faz ruído ao ser agitado.}{cho.ca.lho}{0}
\verb{chocalho}{}{}{}{}{}{Tipo de campainha que se coloca no pescoço dos animais.}{cho.ca.lho}{0}
\verb{chocalho}{}{Mús.}{}{}{}{Tipo de instrumento musical de percussão que produz som ao ser agitado.}{cho.ca.lho}{0}
\verb{chocante}{}{}{}{}{adj.2g.}{Que choca, impressiona, abala.}{cho.can.te}{0}
\verb{chocante}{}{Pop.}{}{}{}{Muito bom, muito divertido.}{cho.can.te}{0}
\verb{chocar}{}{}{}{}{v.t.}{Ir de encontro a; embater.}{cho.car}{0}
\verb{chocar}{}{}{}{}{}{Causar impressão forte e desagradável.}{cho.car}{\verboinum{2}}
\verb{chocar}{}{}{}{}{v.t.}{Aquecer ou incubar ovos.}{cho.car}{0}
\verb{chocar}{}{}{}{}{v.i.}{Deteriorar"-se, estragar, apodrecer.}{cho.car}{\verboinum{2}}
\verb{chocarreiro}{ê}{}{}{}{adj.}{Relativo a chocarrice.}{cho.car.rei.ro}{0}
\verb{chocarrice}{}{}{}{}{s.f.}{Piada, chiste.}{cho.car.ri.ce}{0}
\verb{chocarrice}{}{}{}{}{}{Gracejo insolente.}{cho.car.ri.ce}{0}
\verb{chochar}{}{}{}{}{v.i.}{Não desenvolver"-se ou frutificar; ficar chocho; secar.}{cho.char}{0}
\verb{chochar}{}{}{}{}{}{Malograr, abortar, gorar.}{cho.char}{\verboinum{1}}
\verb{chocho}{ô}{}{}{}{adj.}{Diz"-se de fruto sem suco ou miolo.}{cho.cho}{0}
\verb{chocho}{ô}{}{}{}{}{Diz"-se de espiga ou vagem sem grão.}{cho.cho}{0}
\verb{chocho}{ô}{}{}{}{}{Diz"-se de ovo podre ou choco.}{cho.cho}{0}
\verb{chocho}{ô}{}{}{}{}{Sem graça, sem interesse, fútil, tolo.}{cho.cho}{0}
\verb{chocho}{ô}{Esport.}{}{}{}{Diz"-se de jogada, chute ou arremesso sem vigor ou precisão.}{cho.cho}{0}
\verb{choco}{ô}{}{}{}{s.m.}{Ato ou efeito de chocar ovos.}{cho.co}{0}
\verb{choco}{ô}{}{}{}{}{Período de incubação.}{cho.co}{0}
\verb{choco}{ô}{}{}{}{adj.}{Diz"-se de ave que está chocando.}{cho.co}{0}
\verb{choco}{ô}{}{}{}{}{Diz"-se de ovo que gorou ou apodreceu.}{cho.co}{0}
\verb{choco}{ô}{}{}{}{}{Diz"-se de bebida deteriorada ou que perdeu o gás.}{cho.co}{0}
\verb{chocolate}{}{}{}{}{s.m.}{Alimento feito de pasta de cacau, açúcar, leite e substâncias aromáticas.}{cho.co.la.te}{0}
\verb{chocolate}{}{}{}{}{}{Bebida à base de leite com chocolate em pó, geralmente servida quente.}{cho.co.la.te}{0}
\verb{chocolateira}{ê}{}{}{}{s.f.}{Vasilha para preparar ou servir bebida de chocolate.}{cho.co.la.tei.ra}{0}
\verb{chocolateiro}{ê}{}{}{}{s.m.}{Fabricante ou comerciante de chocolate e derivados.}{cho.co.la.tei.ro}{0}
\verb{chocolateiro}{ê}{}{}{}{}{Agricultor que produz cacau.}{cho.co.la.tei.ro}{0}
\verb{chofer}{é}{}{}{}{s.m.}{Motorista.}{cho.fer}{0}
\verb{chofre}{ô}{}{}{}{s.m.}{Choque repentino.}{cho.fre}{0}
\verb{chofre}{ô}{}{}{}{}{Usado na locução \textit{de chofre}: repentinamente.}{cho.fre}{0}
\verb{choldra}{ô}{Pop.}{}{}{s.f.}{Coisa sem utilidade, imprestável.}{chol.dra}{0}
\verb{choldra}{ô}{}{}{}{}{Escória, ralé.}{chol.dra}{0}
\verb{chopada}{}{}{}{}{s.f.}{Festa informal em que se serve chope.}{cho.pa.da}{0}
\verb{chope}{ô}{}{}{}{s.m.}{Tipo de cerveja fresca armazenada em barril sob pressão.}{cho.pe}{0}
\verb{choque}{ó}{}{}{}{s.m.}{Embate entre corpos em movimento.}{cho.que}{0}
\verb{choque}{ó}{}{}{}{}{Abalo emocional.}{cho.que}{0}
\verb{choque}{ó}{Fig.}{}{}{}{Conflito, briga, oposição, discordância.}{cho.que}{0}
\verb{choque}{ó}{}{}{}{}{Sensação causada pelo contato com corrente elétrica.}{cho.que}{0}
\verb{choradeira}{ê}{}{}{}{}{Choro longo e persistente.}{cho.ra.dei.ra}{0}
\verb{choradeira}{ê}{}{}{}{s.f.}{Lamentação, lamúria.}{cho.ra.dei.ra}{0}
\verb{choradeira}{ê}{}{}{}{}{Pedido lamuriento.}{cho.ra.dei.ra}{0}
\verb{chorado}{}{}{}{}{adj.}{Executado ou cantado em tom lastimoso.}{cho.ra.do}{0}
\verb{chorado}{}{}{}{}{}{Obtido a muito custo.}{cho.ra.do}{0}
\verb{choramigar}{}{}{}{}{}{Var. de \textit{choramingar}.}{cho.ra.mi.gar}{0}
\verb{choramigas}{}{}{}{}{}{Var. de \textit{choramingas}.}{cho.ra.mi.gas}{0}
\verb{choramingar}{}{}{}{}{v.i.}{Chorar por motivos fúteis.}{cho.ra.min.gar}{0}
\verb{choramingar}{}{}{}{}{}{Chorar lamentosamente, geralmente em tom baixo.}{cho.ra.min.gar}{\verboinum{5}}
\verb{choramingas}{}{}{}{}{s.2g.}{Indivíduo que vive choramingando.}{cho.ra.min.gas}{0}
\verb{chorão}{}{}{"-ões}{}{adj.}{Que chora muito.}{cho.rão}{0}
\verb{chorão}{}{}{"-ões}{}{}{Lastimoso, triste, choroso.}{cho.rão}{0}
\verb{chorão}{}{Bras.}{"-ões}{}{s.m.}{Instrumentista ou compositor de choro.}{cho.rão}{0}
\verb{chorão}{}{Bot.}{"-ões}{}{}{Árvore ornamental com pequenos ramos verdes em vez de folhas.}{cho.rão}{0}
\verb{chorar}{}{}{}{}{v.i.}{Derramar lágrimas.}{cho.rar}{0}
\verb{chorar}{}{}{}{}{}{Lastimar"-se, reclamar.}{cho.rar}{0}
\verb{chorar}{}{}{}{}{}{Manifestar dor ou tristeza.}{cho.rar}{0}
\verb{chorar}{}{}{}{}{v.t.}{Lamentar, deplorar.}{cho.rar}{\verboinum{1}}
\verb{chorinho}{}{Mús.}{}{}{s.m.}{Gênero musical executado em conjuntos geralmente com violão, flauta, cavaquinho e pandeiro.}{cho.ri.nho}{0}
\verb{choro}{ô}{}{}{}{s.m.}{Ato de chorar; pranto.}{cho.ro}{0}
\verb{choro}{ô}{}{}{}{}{Porção de bebida que se serve além da dose.}{cho.ro}{0}
\verb{choro}{ô}{}{}{}{}{Chorinho.}{cho.ro}{0}
\verb{choroso}{ô}{}{"-osos ⟨ó⟩}{"-osa ⟨ó⟩}{adj.}{Que chora ou que indica choro; lacrimoso.}{cho.ro.so}{0}
\verb{choroso}{ô}{}{"-osos ⟨ó⟩}{"-osa ⟨ó⟩}{}{Magoado, triste.}{cho.ro.so}{0}
\verb{chorrilho}{}{}{}{}{s.m.}{Conjunto ou sequência de pessoas ou coisas semelhantes.}{chor.ri.lho}{0}
\verb{chorumela}{é}{Bras.}{}{}{s.f.}{Coisa sem valor ou quantia insignificante; ninharia.}{cho.ru.me.la}{0}
\verb{choupal}{}{}{"-ais}{}{s.m.}{Plantação de choupos.}{chou.pal}{0}
\verb{choupana}{}{}{}{}{s.f.}{Casa rústica e eventualmente precária.}{chou.pa.na}{0}
\verb{choupo}{ô}{Bot.}{}{}{s.m.}{Árvore ornamental de flores pequenas e madeira leve e clara.}{chou.po}{0}
\verb{chouriço}{}{}{}{}{s.m.}{Produto alimentício feito de tripa de porco recheada de carne, gordura e sangue, com temperos e defumado.}{chou.ri.ço}{0}
\verb{chouto}{ô}{}{}{}{s.m.}{Trote miúdo e balançado, incômodo para o cavaleiro.}{chou.to}{0}
\verb{chove"-não"-molha}{ó\ldots{}ó}{}{}{}{s.2g.2n}{Situação que não se resolve ou em que não há decisão.}{cho.ve"-não"-mo.lha}{0}
\verb{chover}{ê}{}{}{}{v.i.}{Cair água da atmosfera.}{cho.ver}{0}
\verb{chover}{ê}{Fig.}{}{}{v.t.}{Aparecer em abundância. (\textit{Depois que a prefeitura enviou os impostos aos cidadãos, choveram reclamações.})}{cho.ver}{\verboinum{12}\verboirregular{\emph{def.} chove}}
\verb{chucha}{}{}{}{}{s.f.}{Ato de chuchar.}{chu.cha}{0}
\verb{chucha}{}{}{}{}{}{O seio que amamenta.}{chu.cha}{0}
\verb{chucha}{}{}{}{}{}{O leite do seio.}{chu.cha}{0}
\verb{chuchar}{}{}{}{}{v.t.}{Sugar, chupar, sorver.}{chu.char}{\verboinum{1}}
\verb{chuchu}{}{}{}{}{s.m.}{Planta herbácea de fruto verde comestível.}{chu.chu}{0}
\verb{chuchu}{}{}{}{}{}{O fruto dessa planta.}{chu.chu}{0}
\verb{chuchurrear}{}{}{}{}{v.t.}{Beber aos goles fazendo barulho.}{chu.chur.re.ar}{\verboinum{4}}
\verb{chuço}{}{}{}{}{s.m.}{Vara guarnecida com ponteira de ferro.}{chu.ço}{0}
\verb{chucrute}{}{}{}{}{s.m.}{Iguaria preparada com repolho picado e fermentado, servido como acompanhamento de outros pratos.}{chu.cru.te}{0}
\verb{chué}{}{}{}{}{adj.}{Reles, ordinário, chinfrim.}{chu.é}{0}
\verb{chué}{}{}{}{}{}{Vestido sem apuro.}{chu.é}{0}
\verb{chula}{}{}{}{}{s.f.}{Tipo de dança e música popular portuguesa.}{chu.la}{0}
\verb{chulé}{}{}{}{}{s.m.}{O mau cheiro ou a sujeira acumulada nos pés.}{chu.lé}{0}
\verb{chuleado}{}{}{}{}{s.m.}{Ato ou efeito de chulear; chuleio.}{chu.le.a.do}{0}
\verb{chulear}{}{}{}{}{v.t.}{Coser a borda de tecido para que não desfie.}{chu.le.ar}{\verboinum{4}}
\verb{chuleio}{ê}{}{}{}{s.m.}{Ato ou efeito de chulear.}{chu.lei.o}{0}
\verb{chuleio}{ê}{}{}{}{}{O ponto de chulear.}{chu.lei.o}{0}
\verb{chulismo}{}{}{}{}{s.m.}{Ato ou dito chulo.}{chu.lis.mo}{0}
\verb{chulo}{}{}{}{}{adj.}{Rude, ordinário, grosseiro.}{chu.lo}{0}
\verb{chumaço}{}{}{}{}{s.m.}{Porção de penas, algodão ou outro material mole que se coloca por baixo de algo para torná"-lo mais macio; enchimento.}{chu.ma.ço}{0}
\verb{chumbada}{}{}{}{}{s.f.}{Tiro de chumbo.}{chum.ba.da}{0}
\verb{chumbado}{}{}{}{}{adj.}{Soldado ou preso com chumbo.}{chum.ba.do}{0}
\verb{chumbado}{}{}{}{}{}{Fixado à parede de forma definitiva.}{chum.ba.do}{0}
\verb{chumbado}{}{}{}{}{}{Ferido por tiro de chumbo.}{chum.ba.do}{0}
\verb{chumbado}{}{Pop.}{}{}{}{Embriagado.}{chum.ba.do}{0}
\verb{chumbar}{}{}{}{}{v.t.}{Soldar com chumbo ou outro metal.}{chum.bar}{0}
\verb{chumbar}{}{}{}{}{}{Fixar à parede com cimento.}{chum.bar}{0}
\verb{chumbar}{}{}{}{}{}{Ferir com chumbo.}{chum.bar}{\verboinum{1}}
\verb{chumbo}{}{Quím.}{}{}{s.m.}{Elemento químico metálico, azulado, maleável, dúctil, usado principalmente em encanamentos e revestimentos protetores. \elemento{82}{207.2}{Pb}.}{chum.bo}{0}
\verb{chumbo}{}{Fig.}{}{}{adj.}{Que é muito pesado.}{chum.bo}{0}
\verb{chupada}{}{}{}{}{s.f.}{Ato de chupar uma vez.}{chu.pa.da}{0}
\verb{chupada}{}{Pop.}{}{}{}{Bronca, sermão, repreensão.}{chu.pa.da}{0}
\verb{chupadela}{é}{}{}{}{s.f.}{Ato de chupar uma vez; chupada.}{chu.pa.de.la}{0}
\verb{chupado}{}{}{}{}{adj.}{Muito magro (diz"-se especialmente do rosto).}{chu.pa.do}{0}
\verb{chupão}{}{}{"-ões}{}{adj.}{Que chupa.}{chu.pão}{0}
\verb{chupão}{}{}{"-ões}{}{s.m.}{Beijo ruidoso e demorado.}{chu.pão}{0}
\verb{chupão}{}{}{"-ões}{}{}{Marca na pele resultante de uma chupada.}{chu.pão}{0}
\verb{chupar}{}{}{}{}{v.t.}{Sugar, sorver.}{chu.par}{0}
\verb{chupar}{}{}{}{}{}{Aplicar os lábios sugando.}{chu.par}{\verboinum{1}}
\verb{chupeta}{ê}{}{}{}{s.f.}{Mamilo de borracha para distrair ou acalmar crianças.}{chu.pe.ta}{0}
\verb{chupim}{}{Zool.}{"-ins}{}{s.m.}{Ave encontrada em todo o Brasil, que põe ovos nos ninhos de outras espécies e que, quando em bandos, prejudica lavouras.}{chu.pim}{0}
\verb{chupim}{}{Pop.}{"-ins}{}{}{Pessoa que vive à custa de outrem.}{chu.pim}{0}
\verb{chupitar}{}{}{}{}{v.t.}{Chupar devagar; bebericar.}{chu.pi.tar}{\verboinum{1}}
\verb{churrascada}{}{}{}{}{s.f.}{Festa ou refeição em que se serve churrasco.}{chur.ras.ca.da}{0}
\verb{churrascaria}{}{}{}{}{s.f.}{Restaurante em que se serve churrasco.}{chur.ras.ca.ri.a}{0}
\verb{churrasco}{}{}{}{}{s.m.}{Carne assada na brasa, com tempero ou apenas sal grosso.}{chur.ras.co}{0}
\verb{churrasco}{}{}{}{}{}{A refeição ou festa em que se come churrasco.}{chur.ras.co}{0}
\verb{churrasquear}{}{}{}{}{v.i.}{Preparar ou comer churrasco.}{chur.ras.que.ar}{\verboinum{4}}
\verb{churrasqueira}{ê}{}{}{}{s.f.}{Utensílio no qual se acende brasa para preparar churrasco.}{chur.ras.quei.ra}{0}
\verb{churrasqueiro}{ê}{}{}{}{s.f.}{Indivíduo que prepara o churrasco.}{chur.ras.quei.ro}{0}
\verb{churro}{}{}{}{}{adj.}{Sujo, sórdido, miserável.}{chur.ro}{0}
\verb{churro}{}{}{}{}{}{Ordinário.}{chur.ro}{0}
\verb{churro}{}{Cul.}{}{}{s.m.}{Alimento feito de massa, frito, passado em açúcar ou canela e geralmente recheado com doce.}{chur.ro}{0}
\verb{chusma}{}{}{}{}{s.f.}{Grande quantidade de pessoas ou coisas.}{chus.ma}{0}
\verb{chutar}{}{}{}{}{v.t.}{Impelir com pontapé.}{chu.tar}{0}
\verb{chutar}{}{}{}{}{}{Dar pontapé.}{chu.tar}{0}
\verb{chutar}{}{}{}{}{}{Tentar acertar por adivinhação, sem conhecimento de causa.}{chu.tar}{\verboinum{1}}
\verb{chute}{}{}{}{}{s.m.}{Impulso dado com o pé; ato ou efeito de chutar.}{chu.te}{0}
\verb{chute}{}{}{}{}{}{Pontapé.}{chu.te}{0}
\verb{chute}{}{}{}{}{}{Palpite; tentativa de acerto sem conhecimento.}{chu.te}{0}
\verb{chuteira}{ê}{}{}{}{s.f.}{Calçado próprio para chutar a bola em certos esportes.}{chu.tei.ra}{0}
\verb{chuva}{}{}{}{}{s.f.}{Precipitação atmosférica de água resultante da condensação do vapor.}{chu.va}{0}
\verb{chuva}{}{Por ext.}{}{}{}{Precipitação de qualquer coisa.}{chu.va}{0}
\verb{chuva}{}{Fig.}{}{}{}{Grande porção de alguma coisa.}{chu.va}{0}
\verb{chuvada}{}{}{}{}{s.f.}{Chuvarada.}{chu.va.da}{0}
\verb{chuvarada}{}{}{}{}{s.f.}{Chuva forte e abundante.}{chu.va.ra.da}{0}
\verb{chuveirada}{}{}{}{}{s.f.}{Banho rápido e curto de chuveiro.}{chu.vei.ra.da}{0}
\verb{chuveiro}{ê}{}{}{}{s.m.}{Dispositivo em forma de recipiente com furos finos no fundo por onde passa a água, próprio para banho.}{chu.vei.ro}{0}
\verb{chuveiro}{ê}{}{}{}{}{Chuva repentina e abundante.}{chu.vei.ro}{0}
\verb{chuviscar}{}{}{}{}{v.i.}{Chover em pouca quantidade e em gotas miúdas.}{chu.vis.car}{\verboinum{2}}
\verb{chuvisco}{}{}{}{}{s.m.}{Chuva fina.}{chu.vis.co}{0}
\verb{chuvisqueiro}{ê}{Bras.}{}{}{s.m.}{Chuvisco.}{chu.vis.quei.ro}{0}
\verb{chuvoso}{ô}{}{"-osos ⟨ó⟩}{"-osa ⟨ó⟩}{adj.}{Diz"-se de região ou clima em que há chuva abundante. (\textit{O clima amazônico é chuvoso.})}{chu.vo.so}{0}
\verb{chuvoso}{ô}{}{"-osos ⟨ó⟩}{"-osa ⟨ó⟩}{}{Em que há chuva; de chuva. (\textit{Dia chuvoso.})}{chu.vo.so}{0}
\verb{cianeto}{ê}{Quím.}{}{}{s.m.}{Qualquer sal, éster ou ânion do ácido cianídrico.}{ci.a.ne.to}{0}
\verb{cianose}{ó}{Med.}{}{}{s.f.}{Coloração azulada da pele resultante da insuficiência de oxigênio no sangue.}{ci.a.no.se}{0}
\verb{cianótico}{}{}{}{}{adj.}{Relativo a cianose.}{ci.a.nó.ti.co}{0}
\verb{cianureto}{ê}{Quím.}{}{}{s.m.}{Cianeto.}{ci.a.nu.re.to}{0}
\verb{ciática}{}{Med.}{}{}{adj.}{Diz"-se da dor aguda do nervo ciático.}{ci.á.ti.ca}{0}
\verb{ciático}{}{Med.}{}{}{adj.}{Diz"-se do nervo maior da coxa.}{ci.á.ti.co}{0}
\verb{cibernética}{}{}{}{}{s.f.}{Ciência que estuda o sistema de controle e de comunicações nos organismos vivos e nas máquinas.}{ci.ber.né.ti.ca}{0}
\verb{cibório}{}{}{}{}{s.m.}{Vaso para guardar hóstias.}{ci.bó.rio}{0}
\verb{cica}{}{}{}{}{s.f.}{O gosto amargo e travoso de certas frutas quando ainda verdes.}{ci.ca}{0}
\verb{cicatriz}{}{}{}{}{s.f.}{Marca que permanece na pele após o fechamento de uma ferida.}{ci.ca.triz}{0}
\verb{cicatrização}{}{}{"-ões}{}{s.f.}{Ato ou efeito de cicatrizar.}{ci.ca.tri.za.ção}{0}
\verb{cicatrizante}{}{}{}{}{adj.2g.}{Diz"-se de substância, medicamento ou procedimento que facilita ou acelera a cicatrização.}{ci.ca.tri.zan.te}{0}
\verb{cicatrizar}{}{}{}{}{v.t.}{Fazer com que um ferimento se cure. (\textit{O remédio cicatrizou o corte.})}{ci.ca.tri.zar}{0}
\verb{cicatrizar}{}{Fig.}{}{}{}{Cessar, dissipar. (\textit{Aquela dor já cicatrizou.})}{ci.ca.tri.zar}{0}
\verb{cicatrizar}{}{}{}{}{v.i.}{Curar"-se.}{ci.ca.tri.zar}{\verboinum{1}}
\verb{cícero}{}{}{}{}{s.m.}{Unidade de medida utilizada em tipografia, equivalente a 12 pontos ou 4,511mm.}{cí.ce.ro}{0}
\verb{cicerone}{}{}{}{}{s.m.}{Indivíduo que guia turistas, mostrando locais e fornecendo informações de interesse.}{ci.ce.ro.ne}{0}
\verb{ciciar}{}{}{}{}{v.i.}{Fazer barulho ou falar muito baixo.}{ci.ci.ar}{\verboinum{6}}
\verb{cicio}{}{}{}{}{s.m.}{Som muito baixo ou sussurro.}{ci.ci.o}{0}
\verb{ciclagem}{}{Fís.}{"-ens}{}{s.f.}{A frequência de uma corrente alternada.}{ci.cla.gem}{0}
\verb{cíclico}{}{}{}{}{adj.}{Relativo a ciclo.}{cí.cli.co}{0}
\verb{cíclico}{}{}{}{}{}{Que se desenvolve em ciclos.}{cí.cli.co}{0}
\verb{cíclico}{}{}{}{}{}{Que se realiza ou repete de tempos em tempos.}{cí.cli.co}{0}
\verb{ciclismo}{}{}{}{}{s.m.}{A técnica ou o esporte de se andar de bicicleta.}{ci.clis.mo}{0}
\verb{ciclista}{}{}{}{}{s.2g.}{Indivíduo que anda de bicicleta ou que pratica o ciclismo como esporte.}{ci.clis.ta}{0}
\verb{ciclo}{}{}{}{}{s.m.}{Conjunto de eventos que se repetem com regularidade.}{ci.clo}{0}
\verb{ciclo}{}{}{}{}{}{Tempo em que acontece esse conjunto de eventos.}{ci.clo}{0}
\verb{ciclone}{}{}{}{}{s.m.}{Vento de alta velocidade que sopra em círculos em regiões de baixa pressão em relação a regiões vizinhas.}{ci.clo.ne}{0}
\verb{ciclope}{ó}{}{}{}{s.m.}{Ser mitológico gigante de um só olho, no meio da testa.}{ci.clo.pe}{0}
\verb{ciclópico}{}{}{}{}{adj.}{Relativo a ciclope.}{ci.cló.pi.co}{0}
\verb{ciclópico}{}{}{}{}{}{Gigantesco, extraordinário.}{ci.cló.pi.co}{0}
\verb{cicuta}{}{}{}{}{s.f.}{Planta venenosa que cresce em pântanos e montanhas.}{ci.cu.ta}{0}
\verb{cicuta}{}{}{}{}{}{Veneno extraído dessa planta.}{ci.cu.ta}{0}
\verb{cidadania}{}{}{}{}{s.f.}{Qualidade de cidadão.}{ci.da.da.ni.a}{0}
\verb{cidadão}{}{}{"-ãos}{"-ã}{s.m.}{Indivíduo com direitos e deveres civis e políticos perante um Estado.}{ci.da.dão}{0}
\verb{cidadão}{}{}{"-ãos}{"-ã}{}{Habitante de cidade.}{ci.da.dão}{0}
\verb{cidade}{}{}{}{}{s.f.}{Lugar em que se concentra uma população, repleto de casas, edifícios, escolas, lojas, indústrias e serviços que se organizam em ruas e se distribuem em bairros.}{ci.da.de}{0}
\verb{cidade"-dormitório}{}{}{cidades"-dormitórios \textit{ou} cidades"-dormitório}{}{s.f.}{Cidade sem atividade econômica, cujos habitantes trabalham em cidades vizinhas, retornando a ela somente para dormir.}{ci.da.de"-dor.mi.tó.rio}{0}
\verb{cidadela}{é}{}{}{}{s.f.}{Fortaleza defensiva de uma cidade.}{ci.da.de.la}{0}
\verb{cidade"-satélite}{}{}{cidades"-satélites \textit{ou} cidades"-satélite}{}{s.f.}{Cidade com ou sem autonomia administrativa e cuja vida depende de uma cidade vizinha mais desenvolvida.}{ci.da.de"-sa.té.li.te}{0}
\verb{cidra}{}{}{}{}{s.f.}{O fruto da cidreira.}{ci.dra}{0}
\verb{cidreira}{ê}{Bot.}{}{}{s.f.}{Arbusto com flores brancas, folhas aromáticas e frutos de cujas cascas se faz doce.}{ci.drei.ra}{0}
\verb{ciência}{}{}{}{}{s.f.}{Conjunto organizado de conhecimentos sobre um assunto.}{ci.ên.cia}{0}
\verb{ciência}{}{}{}{}{}{Conhecimento.}{ci.ên.cia}{0}
\verb{ciente}{}{}{}{}{adj.2g.}{Que tem ciência; sábio, erudito.}{ci.en.te}{0}
\verb{ciente}{}{}{}{}{}{Que tem conhecimento de alguma coisa.}{ci.en.te}{0}
\verb{ciente}{}{}{}{}{s.m.}{Assinatura aposta em documentos para comprovar que se tomou conhecimento de seu conteúdo.}{ci.en.te}{0}
\verb{cientificar}{}{}{}{}{v.t.}{Dar conhecimento; tornar ciente; informar.}{ci.en.ti.fi.car}{\verboinum{2}}
\verb{cientificismo}{}{}{}{}{s.m.}{Confiança ilimitada no potencial da ciência para resolver os problemas humanos.}{ci.en.ti.fi.cis.mo}{0}
\verb{científico}{}{}{}{}{adj.}{Relativo à ciência.}{ci.en.tí.fi.co}{0}
\verb{científico}{}{}{}{}{}{Que tem o rigor da ciência.}{ci.en.tí.fi.co}{0}
\verb{científico}{}{Desus.}{}{}{}{Diz"-se do curso de nível médio voltado para as disciplinas científicas, realizado em três anos, que vigorou até 1971.}{ci.en.tí.fi.co}{0}
\verb{cientismo}{}{}{}{}{s.m.}{Cientificismo.}{ci.en.tis.mo}{0}
\verb{cientista}{}{}{}{}{s.2g.}{Indivíduo que cultiva particularmente uma ciência; especialista numa ciência.}{ci.en.tis.ta}{0}
\verb{cifose}{ó}{Med.}{}{}{s.f.}{Curvatura anormal, para trás, da coluna vertebral.}{ci.fo.se}{0}
\verb{cifra}{}{}{}{}{s.f.}{Sinal gráfico representado pelo algarismo zero, que não tem valor absoluto.}{ci.fra}{0}
\verb{cifra}{}{}{}{}{}{Importância ou número total.}{ci.fra}{0}
\verb{cifra}{}{}{}{}{}{Escrita secreta ou a chave dessa escrita.}{ci.fra}{0}
\verb{cifrado}{}{}{}{}{adj.}{Escrito com caracteres secretos.}{ci.fra.do}{0}
\verb{cifrão}{}{}{"-ões}{}{s.m.}{Sinal gráfico representado por um \textsc{s} cortado por um ou dois traços verticais, e que indica as unidades monetárias de vários países.}{ci.frão}{0}
\verb{cifrar}{}{}{}{}{v.t.}{Escrever em cifra.}{ci.frar}{0}
\verb{cifrar}{}{}{}{}{}{Resumir, sintetizar.}{ci.frar}{\verboinum{1}}
\verb{cigano}{}{}{}{}{s.m.}{Indivíduo de um povo nômade, provavelmente originário da Índia, que tem um codigo ético próprio, dedica"-se à música, lê a sorte e pratica o comércio.}{ci.ga.no}{0}
\verb{cigano}{}{Pejor.}{}{}{}{Indivíduo de vida errante; boêmio.}{ci.ga.no}{0}
\verb{cigano}{}{Pejor.}{}{}{}{Indivíduo velhaco, trapaceiro.}{ci.ga.no}{0}
\verb{cigarra}{}{Zool.}{}{}{s.f.}{Nome comum a vários insetos da família dos cicadídeos, cuja característica marcante é o canto estridente dos machos.}{ci.gar.ra}{0}
\verb{cigarra}{}{Bras.}{}{}{}{Campainha elétrica.}{ci.gar.ra}{0}
\verb{cigarreira}{ê}{}{}{}{s.f.}{Caixinha ou estojo onde se guardam cigarros.}{ci.gar.rei.ra}{0}
\verb{cigarrilha}{}{}{}{}{s.f.}{Cigarro enrolado na folha do tabaco.}{ci.gar.ri.lha}{0}
\verb{cigarrilha}{}{}{}{}{}{Pequeno charuto de fumo ruim.}{ci.gar.ri.lha}{0}
\verb{cigarrinha}{}{}{}{}{s.f.}{Cigarra pequena.}{ci.gar.ri.nha}{0}
\verb{cigarrinha}{}{}{}{}{}{Nome comum a vários insetos pequenos, prejudiciais às lavouras, e que se alimentam de seiva das plantas.}{ci.gar.ri.nha}{0}
\verb{cigarro}{}{}{}{}{s.m.}{Pequena porção de fumo picado, enrolado em papel muito fino, para se fumar.}{ci.gar.ro}{0}
\verb{cilada}{}{}{}{}{s.f.}{Lugar oculto onde se espera a caça ou o inimigo.}{ci.la.da}{0}
\verb{cilada}{}{}{}{}{}{Emboscada, armadilha.}{ci.la.da}{0}
\verb{cilada}{}{}{}{}{}{Deslealdade, traição.}{ci.la.da}{0}
\verb{cilha}{}{}{}{}{s.f.}{Tira de pano ou de coro com que se aperta a sela ou a carga por baixo do ventre das cavalgaduras.}{ci.lha}{0}
\verb{ciliar}{}{}{}{}{adj.2g.}{Relativo aos cílios.}{ci.li.ar}{0}
\verb{ciliar}{}{}{}{}{}{Diz"-se de formação vegetal que margeia cursos de água.}{ci.li.ar}{0}
\verb{cilício}{}{}{}{}{s.m.}{Pequena túnica, cinto ou cordão, de crina, de lã áspera, às vezes com farpas de madeira, que por penitência se trazia vestido diretamente sobre a pele.}{ci.lí.cio}{0}
\verb{cilício}{}{Fig.}{}{}{}{Tortura, tormento.}{ci.lí.cio}{0}
\verb{cilindrada}{}{}{}{}{s.f.}{Volume máximo de gás admitido por um cilindro em um motor de explosão.}{ci.lin.dra.da}{0}
\verb{cilindrada}{}{}{}{}{}{Esse volume expresso em centímetros cúbicos ou em litros.}{ci.lin.dra.da}{0}
\verb{cilindrar}{}{}{}{}{v.t.}{Tornear em forma de cilindro.}{ci.lin.drar}{0}
\verb{cilindrar}{}{}{}{}{}{Submeter à pressão de cilindro.}{ci.lin.drar}{\verboinum{1}}
\verb{cilíndrico}{}{}{}{}{adj.}{Que tem forma de cilindro.}{ci.lín.dri.co}{0}
\verb{cilindro}{}{Geom.}{}{}{s.m.}{Sólido de base circular, alongado, cujo diâmetro é igual em todo o comprimento.}{ci.lin.dro}{0}
\verb{cilindro}{}{}{}{}{}{Peça oca do motor, onde se movimenta o pistão.}{ci.lin.dro}{0}
\verb{cílio}{}{}{}{}{s.m.}{Cada um dos pequenos pelos que guarnecem a borda externa das pálpebras; pestana.}{cí.lio}{0}
\verb{cílio}{}{Biol.}{}{}{}{Pelo que guarnece certos órgãos vegetais, tornando possíveis, em certos casos, movimentos na água.}{cí.lio}{0}
\verb{cima}{}{}{}{}{s.f.}{A parte mais elevada.}{ci.ma}{0}
\verb{cima}{}{}{}{}{}{Cume, cimo, topo.}{ci.ma}{0}
\verb{cimalha}{}{}{}{}{s.f.}{Parte mais alta da parede, em cuja saliência se assenta o beiral do telhado.}{ci.ma.lha}{0}
\verb{címbalo}{}{}{}{}{s.m.}{Antigo instrumento de cordas.}{cím.ba.lo}{0}
\verb{címbalo}{}{}{}{}{}{Instrumento de percussão formado por dois pratos metálicos, que se faz bater um contra o outro.}{cím.ba.lo}{0}
\verb{cimeira}{ê}{}{}{}{s.f.}{Cume, topo.}{ci.mei.ra}{0}
\verb{cimeira}{ê}{}{}{}{}{Ornato que fica na parte superior de um capacete ou elmo.}{ci.mei.ra}{0}
\verb{cimeiro}{ê}{}{}{}{adj.}{Que fica no cimo, no alto.}{ci.mei.ro}{0}
\verb{cimeiro}{ê}{Fig.}{}{}{}{Da mais alta importância; do mais alto nível.}{ci.mei.ro}{0}
\verb{cimentado}{}{}{}{}{adj.}{Que foi ligado ou unido com cimento.}{ci.men.ta.do}{0}
\verb{cimentado}{}{}{}{}{}{Firmado, consolidado.}{ci.men.ta.do}{0}
\verb{cimentar}{}{}{}{}{v.t.}{Ligar ou cobrir com cimento.}{ci.men.tar}{0}
\verb{cimentar}{}{}{}{}{}{Firmar, consolidar.}{ci.men.tar}{\verboinum{1}}
\verb{cimento}{}{}{}{}{s.m.}{Substância em pó obtida pelo cozimento de calcários naturais ou artificiais, utilizada como aglomerante ou para ligar certos materiais.}{ci.men.to}{0}
\verb{cimento}{}{}{}{}{}{Argamassa obtida pela mistura desse pó com cal e água.}{ci.men.to}{0}
\verb{cimento}{}{}{}{}{}{Chão revestido de cimento.}{ci.men.to}{0}
\verb{cimitarra}{}{}{}{}{s.f.}{Espada oriental de lâmina larga e recurva, e que tem um só gume, usada especialmente por guerreiros muçulmanos.}{ci.mi.tar.ra}{0}
\verb{cimo}{}{}{}{}{s.m.}{A parte superior de um objeto elevado; o ponto mais alto; cume.}{ci.mo}{0}
\verb{cinamomo}{ô}{Bot.}{}{}{s.m.}{Árvore nativa da Ásia, de flores pequenas e aromáticas, muito usada na arborização de ruas.}{ci.na.mo.mo}{0}
\verb{cincada}{}{}{}{}{s.f.}{Ato ou efeito de cincar; erro, gafe.}{cin.ca.da}{0}
\verb{cincar}{}{}{}{}{v.i.}{Cometer cincadas; errar, falhar.}{cin.car}{\verboinum{2}}
\verb{cincerro}{ê}{}{}{}{s.m.}{Sineta que pende do pescoço de certos animais e serve para reunir e guiar o rebanho.}{cin.cer.ro}{0}
\verb{cincha}{}{}{}{}{s.f.}{Faixa de couro ou de qualquer tecido forte que passa por baixo da barriga da cavalgadura para segurar a sela.}{cin.cha}{0}
\verb{cinco}{}{}{}{}{num.}{Nome dado à quantidade expressa pelo número 5.  }{cin.co}{0}
\verb{cindir}{}{}{}{}{v.t.}{Separar, afastar.}{cin.dir}{0}
\verb{cindir}{}{}{}{}{}{Cortar, fender.}{cin.dir}{0}
\verb{cindir}{}{Fig.}{}{}{}{Tornar nulo, romper.}{cin.dir}{\verboinum{18}}
\verb{cine}{}{}{}{}{s.m.}{Cinema.}{ci.ne}{0}
\verb{cineasta}{}{}{}{}{s.2g.}{Indivíduo que exerce atividade criadora e técnica relacionada com o cinema.}{ci.ne.as.ta}{0}
\verb{cineclube}{}{}{}{}{s.m.}{Associação que reúne apreciadores de cinema para estudar"-lhe a técnica e a história.}{ci.ne.clu.be}{0}
\verb{cinéfilo}{}{}{}{}{adj.}{Diz"-se de indivíduo que gosta muito de cinema.}{ci.né.fi.lo}{0}
\verb{cinegética}{}{}{}{}{s.f.}{A arte da caça, especialmente com a ajuda de cães.}{ci.ne.gé.ti.ca}{0}
\verb{cinegético}{}{}{}{}{adj.}{Relativo à caça.}{ci.ne.gé.ti.co}{0}
\verb{cinegrafista}{}{}{}{}{adj.2g.}{Diz"-se de indivíduo que opera uma câmara de cinema ou de televisão.}{ci.ne.gra.fis.ta}{0}
\verb{cinejornal}{}{}{"-ais}{}{s.m.}{Noticiário especialmente produzido para exibição em cinemas.}{ci.ne.jor.nal}{0}
\verb{cinema}{}{}{}{}{s.m.}{Arte de compor e realizar filmes cinematográficos.}{ci.ne.ma}{0}
\verb{cinema}{}{}{}{}{}{Estabelecimento para exibir filmes.}{ci.ne.ma}{0}
\verb{cinemateca}{é}{}{}{}{s.f.}{Local onde se conservam os filmes cinematográficos, especialmente os considerados de valor cultural ou artístico.}{ci.ne.ma.te.ca}{0}
\verb{cinemática}{}{Fís.}{}{}{s.f.}{Ramo da física que estuda os movimentos dos corpos, sem referência às forças que os produzem.}{ci.ne.má.ti.ca}{0}
\verb{cinemático}{}{}{}{}{adj.}{Relativo ao movimento.}{ci.ne.má.ti.co}{0}
\verb{cinematografar}{}{}{}{}{v.t.}{Registrar em imagens cinematográficas; filmar.}{ci.ne.ma.to.gra.far}{0}
\verb{cinematografar}{}{}{}{}{}{Projetar essas imagens na tela.}{ci.ne.ma.to.gra.far}{\verboinum{1}}
\verb{cinematografia}{}{}{}{}{s.f.}{Conjunto de métodos e processos empregados para registrar e projetar fotograficamente cenas animadas ou em movimento.}{ci.ne.ma.to.gra.fi.a}{0}
\verb{cinematográfico}{}{}{}{}{adj.}{Próprio de cinema.}{ci.ne.ma.to.grá.fi.co}{0}
\verb{cinematógrafo}{}{}{}{}{s.m.}{Aparelho capaz de reproduzir numa tela o movimento por meio de uma sequência de fotografias.}{ci.ne.ma.tó.gra.fo}{0}
\verb{cinerama}{}{}{}{}{s.m.}{Tipo de projeção cinematográfica, sobre tela côncava, o qual produz no espectador a impressão de relevo, como se as imagens tivessem três dimensões.}{ci.ne.ra.ma}{0}
\verb{cinerário}{}{}{}{}{adj.}{Relativo a cinzas.}{ci.ne.rá.rio}{0}
\verb{cinerário}{}{}{}{}{s.m.}{Urna ou caixão que contém restos mortais.}{ci.ne.rá.rio}{0}
\verb{cinéreo}{}{}{}{}{adj.}{Que tem cor cinzenta.}{ci.né.re.o}{0}
\verb{cinescópio}{}{}{}{}{s.m.}{Tubo usado nos receptores de televisão para produzir a imagem.}{ci.nes.có.pio}{0}
\verb{cinética}{}{Fís.}{}{}{s.f.}{Ramo da física que trata da ação das forças nas mudanças de movimento dos corpos.}{ci.né.ti.ca}{0}
\verb{cingalês}{}{}{}{}{adj.}{Relativo à República de Sri Lanka.}{cin.ga.lês}{0}
\verb{cingalês}{}{}{}{}{s.m.}{Indivíduo natural ou habitante dessa república.}{cin.ga.lês}{0}
\verb{cingalês}{}{}{}{}{}{Língua oficial de Sri Lanka.}{cin.ga.lês}{0}
\verb{cingapuriano}{}{}{}{}{adj.}{Relativo a Cingapura (sudeste da Ásia).}{cin.ga.pu.ri.a.no}{0}
\verb{cingapuriano}{}{}{}{}{s.m.}{Indivíduo natural ou habitante desse país.}{cin.ga.pu.ri.a.no}{0}
\verb{cingir}{}{}{}{}{v.t.}{Estar à volta de; rodear, circundar.}{cin.gir}{0}
\verb{cingir}{}{}{}{}{}{Colocar na cabeça como insígnia ou guarnição.}{cin.gir}{0}
\verb{cingir}{}{}{}{}{}{Pôr ou prender à cinta.}{cin.gir}{0}
\verb{cingir}{}{}{}{}{}{Unir, ligar.}{cin.gir}{0}
\verb{cingir}{}{}{}{}{}{Abraçar.}{cin.gir}{0}
\verb{cingir}{}{}{}{}{v.pron.}{Limitar"-se, restringir"-se.}{cin.gir}{\verboinum{22}}
\verb{cíngulo}{}{}{}{}{s.m.}{Cordão com que o sacerdote aperta a alva na cintura.}{cín.gu.lo}{0}
\verb{cínico}{}{}{}{}{s.m.}{Indivíduo descarado, desavergonhado.}{cí.ni.co}{0}
\verb{cinismo}{}{}{}{}{s.m.}{Falta de vergonha; descaramento.}{ci.nis.mo}{0}
\verb{cinocéfalo}{}{Zool.}{}{}{s.m.}{Gênero dos macacos de cabeça semelhante à do cão.}{ci.no.cé.fa.lo}{0}
\verb{cinografia}{}{}{}{}{s.f.}{Tratado sobre cães.}{ci.no.gra.fi.a}{0}
\verb{cinologia}{}{}{}{}{s.f.}{Estudo relativo aos cães.}{ci.no.lo.gi.a}{0}
\verb{cinquenta}{}{}{}{}{num.}{Nome dado à quantidade expressa pelo número 50.  }{cin.quen.ta}{0}
\verb{cinquentão}{}{Pop.}{"-ões}{"-ona}{adj.}{Diz"-se do indivíduo que completou cinquenta anos de idade; quinquagenário.}{cin.quen.tão}{0}
\verb{cinquentão}{}{Pop.}{"-ões}{"-ona}{s.m.}{Indivíduo que está nessa faixa etária.}{cin.quen.tão}{0}
\verb{cinquentenário}{}{}{}{}{adj.}{Que tem cinquenta anos.}{cin.quen.te.ná.rio}{0}
\verb{cinquentenário}{}{}{}{}{s.m.}{Quinquagésimo aniversário.}{cin.quen.te.ná.rio}{0}
\verb{cinta}{}{}{}{}{}{Faixa para apertar a cintura; cinto, cinturão.}{cin.ta}{0}
\verb{cinta}{}{}{}{}{s.f.}{Qualquer tira ou faixa de pano ou couro para apertar, cingir.}{cin.ta}{0}
\verb{cintar}{}{}{}{}{v.t.}{Pôr cinta ou faixa para prender ou envolver.}{cin.tar}{0}
\verb{cintar}{}{}{}{}{}{Formar cerca; orlar, rodear.}{cin.tar}{\verboinum{1}}
\verb{cintilação}{}{}{"-ões}{}{s.f.}{Ato ou efeito de cintilar.}{cin.ti.la.ção}{0}
\verb{cintilação}{}{}{"-ões}{}{}{Claridade excessiva; resplandecência.}{cin.ti.la.ção}{0}
\verb{cintilação}{}{}{"-ões}{}{}{Brilho intenso; fulgor, esplendor.}{cin.ti.la.ção}{0}
\verb{cintilante}{}{}{}{}{adj.2g.}{Que cintila; brilhante.}{cin.ti.lan.te}{0}
\verb{cintilante}{}{}{}{}{}{Que irradia muita luz; fulgurante, resplandecente.}{cin.ti.lan.te}{0}
\verb{cintilar}{}{}{}{}{v.i.}{Brilhar intermitentemente como as estrelas; faiscar, tremeluzir.}{cin.ti.lar}{0}
\verb{cintilar}{}{}{}{}{}{Resplandecer, fulgurar.}{cin.ti.lar}{\verboinum{1}}
\verb{cinto}{}{}{}{}{s.m.}{Faixa ou fita que aperta a cintura; cinturão, cinta, cós.}{cin.to}{0}
\verb{cinto}{}{}{}{}{}{Muro circular, cercado.}{cin.to}{0}
\verb{cintura}{}{}{}{}{s.f.}{A parte média do corpo humano, geralmente mais estreita, situada entre os quadris e o tórax.}{cin.tu.ra}{0}
\verb{cintura}{}{}{}{}{}{Parte do vestuário que rodeia a cintura; cós.}{cin.tu.ra}{0}
\verb{cinturado}{}{}{}{}{adj.}{Que tem cintura; acinturado.}{cin.tu.ra.do}{0}
\verb{cinturado}{}{}{}{}{}{Apertado na cintura; cintado.}{cin.tu.ra.do}{0}
\verb{cinturão}{}{}{"-ões}{}{s.m.}{Cinto largo, em geral de couro, em que se prendem armas, cartucheiras, em que se traz dinheiro etc.}{cin.tu.rão}{0}
\verb{cinturão}{}{}{"-ões}{}{}{Qualquer cinto largo; cinta.}{cin.tu.rão}{0}
\verb{cinturão}{}{}{"-ões}{}{}{Aquilo que circunda ou rodeia.}{cin.tu.rão}{0}
\verb{cinza}{}{}{}{}{s.f.}{Pó ou resíduo resultante da combustão de certas substâncias; borralho.}{cin.za}{0}
\verb{cinza}{}{}{}{}{adj.2g.}{Da cor da cinza; cinzento.}{cin.za}{0}
\verb{cinzas}{}{}{}{}{s.f.pl.}{Restos mortais; despojos.}{cin.zas}{0}
\verb{cinzas}{}{Relig.}{}{}{}{A Quarta"-Feira de Cinzas, o primeiro dia da Quaresma.}{cin.zas}{0}
\verb{cinzeiro}{ê}{}{}{}{s.m.}{Pequeno recipiente de metal, louça ou plástico em que se deita cinza dos cigarros, charutos ou cachimbos.}{cin.zei.ro}{0}
\verb{cinzeiro}{ê}{}{}{}{}{Monte de cinzas.}{cin.zei.ro}{0}
\verb{cinzel}{é}{}{"-éis}{}{s.m.}{Instrumento de aço cortante, usado por escultores e gravadores.}{cin.zel}{0}
\verb{cinzelar}{}{}{}{}{v.t.}{Trabalhar com cinzel; esculpir, gravar.}{cin.ze.lar}{0}
\verb{cinzelar}{}{}{}{}{}{Fazer com esmero; aprimorar.}{cin.ze.lar}{\verboinum{1}}
\verb{cinzento}{}{}{}{}{adj.}{Da cor da cinza.}{cin.zen.to}{0}
\verb{cinzento}{}{Fig.}{}{}{}{Sem luz; apagado, triste.}{cin.zen.to}{0}
\verb{cio}{}{Biol.}{}{}{s.m.}{Período de excitabilidade sexual dos animais, principalmente das fêmeas, favorável ao acasalamento.}{ci.o}{0}
\verb{cioso}{ô}{}{"-osos ⟨ó⟩}{"-osa ⟨ó⟩}{adj.}{Que cuida com muita atenção; diligente.}{ci.o.so}{0}
\verb{cioso}{ô}{}{"-osos ⟨ó⟩}{"-osa ⟨ó⟩}{}{Ciumento, zeloso.}{ci.o.so}{0}
\verb{cipó}{}{Bot.}{}{}{s.m.}{Nome comum às plantas trepadeiras, de ramos flexíveis que pendem das árvores e nelas se entrelaçam, e que ocorrem nas matas tropicais.}{ci.pó}{0}
\verb{cipó}{}{}{}{}{}{Chicote, chibata, vara.}{ci.pó}{0}
\verb{cipoal}{}{}{"-ais}{}{s.m.}{Mato onde se encontram muitos cipós emaranhados.}{ci.po.al}{0}
\verb{cipoal}{}{Fig.}{"-ais}{}{}{Negócio intrincado; complicação, dificuldade.}{ci.po.al}{0}
\verb{cipreste}{é}{Bot.}{}{}{s.m.}{Árvore que atinge até 45 m, esguia e estreita, de folhas muito ramificadas, verde"-escuras, cultivada como ornamental e pela madeira.}{ci.pres.te}{0}
\verb{cipriota}{ó}{}{}{}{adj.2g.}{Relativo ao Chipre, ilha do Mediterrâneo oriental.}{ci.pri.o.ta}{0}
\verb{cipriota}{ó}{}{}{}{s.2g.}{Indivíduo natural ou habitante desse país.}{ci.pri.o.ta}{0}
\verb{ciranda}{}{}{}{}{s.f.}{Dança e cantiga de roda infantil; cirandinha.}{ci.ran.da}{0}
\verb{ciranda}{}{}{}{}{}{Peneira grossa de palha.}{ci.ran.da}{0}
\verb{cirandar}{}{}{}{}{v.i.}{Dançar e cantar a ciranda.}{ci.ran.dar}{0}
\verb{cirandar}{}{}{}{}{}{Passar pela ciranda; peneirar.}{ci.ran.dar}{\verboinum{1}}
\verb{cirandinha}{}{}{}{}{s.f.}{Dança infantil; ciranda.}{ci.ran.di.nha}{0}
\verb{circense}{}{}{}{}{adj.}{Relativo a circo.}{cir.cen.se}{0}
\verb{circense}{}{Fig.}{}{}{}{Excêntrico, grotesco.}{cir.cen.se}{0}
\verb{circo}{}{}{}{}{}{Anfiteatro circular e coberto, onde são realizados espetáculos cômicos, de ginástica, de acrobacia etc.}{cir.co}{0}
\verb{circo}{}{}{}{}{}{Na Antiga Roma, grande recinto destinado aos jogos públicos.}{cir.co}{0}
\verb{circo}{}{}{}{}{s.m.}{Disposição circular; cinto, círculo.}{cir.co}{0}
\verb{circuito}{}{}{}{}{s.m.}{Caminho em volta; limite, contorno, giro.}{cir.cui.to}{0}
\verb{circuito}{}{Fís.}{}{}{}{Caminho que uma corrente elétrica percorre de uma ponta a outra.}{cir.cui.to}{0}
\verb{circulação}{}{}{"-ões}{}{s.f.}{Ato ou efeito de circular.}{cir.cu.la.ção}{0}
\verb{circulação}{}{}{"-ões}{}{}{Movimento de passagem; trânsito, marcha.}{cir.cu.la.ção}{0}
\verb{circulação}{}{}{"-ões}{}{}{Movimento ininterrupto; curso.}{cir.cu.la.ção}{0}
\verb{circulação}{}{Biol.}{"-ões}{}{}{Movimento contínuo de fluido interno, sangue nos animais e seiva nos vegetais, que lhes transmite alimento ou oxigênio e remove substâncias nocivas produzidas pelo organismo.}{cir.cu.la.ção}{0}
\verb{circulador}{ô}{}{}{}{adj.}{Que faz circular alguma coisa.}{cir.cu.la.dor}{0}
\verb{circulador}{ô}{}{}{}{s.m.}{Aparelho que faz circular o ar.}{cir.cu.la.dor}{0}
\verb{circular}{}{}{}{}{}{Que se movimenta por um caminho e volta por outro até atingir o ponto de partida. (\textit{Não se cobra tarifa nesse ônibus circular.})}{cir.cu.lar}{0}
\verb{circular}{}{}{}{}{v.t.}{Percorrer em volta; cercar, rodear.}{cir.cu.lar}{0}
\verb{circular}{}{}{}{}{adj.2g.}{Que tem forma ou aspecto de círculo.}{cir.cu.lar}{0}
\verb{circular}{}{}{}{}{}{Carta ou ofício que se envia a muitas pessoas.}{cir.cu.lar}{0}
\verb{circular}{}{}{}{}{v.i.}{Passar de mão em mão; propagar"-se.}{cir.cu.lar}{0}
\verb{circular}{}{}{}{}{}{Caminhar em círculo; girar.}{cir.cu.lar}{\verboinum{1}}
\verb{circulatório}{}{}{}{}{adj.}{Relativo a circulação ou a qualquer movimento circular.}{cir.cu.la.tó.rio}{0}
\verb{circulatório}{}{Anat.}{}{}{}{Diz"-se do sistema de órgãos que assegura a circulação do sangue e da linfa.}{cir.cu.la.tó.rio}{0}
\verb{círculo}{}{}{}{}{s.m.}{Espaço limitado por uma circunferência.}{cír.cu.lo}{0}
\verb{círculo}{}{}{}{}{}{Conjunto de pessoas reunidas para um determinado fim; assembleia, grêmio.}{cír.cu.lo}{0}
\verb{circumpolar}{}{}{}{}{adj.2g.}{Que está perto ou em volta do polo terrestre.}{cir.cum.po.lar}{0}
\verb{circum"-navegação}{}{}{"-ões}{}{s.f.}{Ato ou efeito de circum"-navegar; percurso circular ao redor da Terra, de um continente, de uma ilha etc.}{cir.cu.m"-na.ve.ga.ção}{0}
\verb{circum"-navegar}{}{}{}{}{v.t.}{Navegar em torno da Terra, de um continente, de uma ilha etc.}{cir.cu.m"-na.ve.gar}{\verboinum{5}}
\verb{circuncidar}{}{}{}{}{v.t.}{Praticar a circuncisão.}{cir.cun.ci.dar}{\verboinum{1}}
\verb{circuncisão}{}{}{"-ões}{}{s.f.}{Corte, em extensão variável, do prepúcio, realizado por motivos higiênicos ou religiosos.}{cir.cun.ci.são}{0}
\verb{circunciso}{}{}{}{}{adj.}{Diz"-se do homem que sofreu a circuncisão.}{cir.cun.ci.so}{0}
\verb{circundar}{}{}{}{}{v.t.}{Estar ou andar em volta; cercar, rodear, circular.}{cir.cun.dar}{\verboinum{1}}
\verb{circunferência}{}{Geom.}{}{}{s.f.}{Linha curva fechada e plana, regular ou não, com pontos que equidistam de um ponto interior.}{cir.cun.fe.rên.cia}{0}
\verb{circunferência}{}{}{}{}{}{Perímetro de uma área; contorno de um círculo.}{cir.cun.fe.rên.cia}{0}
\verb{circunferência}{}{}{}{}{}{Periferia, circuito, circunvizinhança.}{cir.cun.fe.rên.cia}{0}
\verb{circunflexo}{écs}{}{}{}{adj.}{Diz"-se do acento gráfico, constituído de dois traços formando um ângulo, que se sobrepõe às vogais tônicas \textit{a}, \textit{e} e \textit{o}, para indicar o timbre fechado.}{cir.cun.fle.xo}{0}
\verb{circunjacente}{}{}{}{}{adj.2g.}{Que jaz ou se estende em torno; circunvizinho.}{cir.cun.ja.cen.te}{0}
\verb{circunlocução}{}{}{"-ões}{}{s.f.}{Circunlóquio.}{cir.cun.lo.cu.ção}{0}
\verb{circunlóquio}{}{}{}{}{s.m.}{Rodeio de palavras; perífrase, circunlocução.}{cir.cun.ló.quio}{0}
\verb{circunscrever}{ê}{}{}{}{v.t.}{Traçar, escrever uma figura ao redor.}{cir.cuns.cre.ver}{0}
\verb{circunscrever}{ê}{}{}{}{}{Determinar os limites; conter, restringir.}{cir.cuns.cre.ver}{0}
\verb{circunscrever}{ê}{}{}{}{}{Conter no interior; abranger, encerrar.}{cir.cuns.cre.ver}{\verboinum{12}}
\verb{circunscrição}{}{}{"-ões}{}{s.f.}{Ato ou efeito de circunscrever.}{cir.cuns.cri.ção}{0}
\verb{circunscrição}{}{}{"-ões}{}{}{Divisão administrativa, política ou religiosa de um território.}{cir.cuns.cri.ção}{0}
\verb{circunscrito}{}{}{}{}{adj.}{Que tem limites marcados por uma linha ou superfície.}{cir.cuns.cri.to}{0}
\verb{circunscrito}{}{}{}{}{}{Limitado, restrito, localizado.}{cir.cuns.cri.to}{0}
\verb{circunspeção}{}{}{}{}{}{Var. de \textit{circunspecção}.}{cir.cuns.pe.ção}{0}
\verb{circunspecção}{}{}{"-ões}{}{s.f.}{Precaução, prudência, moderação no modo de falar ou de agir.}{cir.cuns.pec.ção}{0}
\verb{circunspecto}{é}{}{}{}{adj.}{Que se comporta com circunspecção; prudente, moderado, precavido.}{cir.cuns.pec.to}{0}
\verb{circunspeto}{é}{}{}{}{}{Var. de \textit{circunspecto}.}{cir.cuns.pe.to}{0}
\verb{circunstância}{}{}{}{}{s.f.}{Particularidades de tempo, lugar ou modo que acompanham um fato, uma situação.}{cir.cuns.tân.cia}{0}
\verb{circunstância}{}{}{}{}{}{Situação, caso, pretexto.}{cir.cuns.tân.cia}{0}
\verb{circunstancial}{}{}{"-ais}{}{adj.2g.}{Relativo a circunstância; incidental.}{cir.cuns.tan.ci.al}{0}
\verb{circunstancial}{}{Gram.}{"-ais}{}{}{Diz"-se do adjunto ou complemento de valor adverbial.}{cir.cuns.tan.ci.al}{0}
\verb{circunstanciar}{}{}{}{}{v.t.}{Expor as circunstâncias de um fato; esmiuçar, detalhar.}{cir.cuns.tan.ci.ar}{\verboinum{6}}
\verb{circunstante}{}{}{}{}{adj.2g.}{Que está ao redor; circunjacente.}{cir.cuns.tan.te}{0}
\verb{circunstante}{}{}{}{}{}{Que está presente em uma situação.}{cir.cuns.tan.te}{0}
\verb{circunvagar}{}{}{}{}{v.t.}{Andar em torno; vagar, rodear, contornar.}{cir.cun.va.gar}{0}
\verb{circunvagar}{}{}{}{}{}{Dar voltas; girar, circular.}{cir.cun.va.gar}{\verboinum{5}}
\verb{circunvizinhança}{}{}{}{}{s.f.}{Área ou população vizinha; arredores, cercanias.}{cir.cun.vi.zi.nhan.ça}{0}
\verb{circunvizinho}{}{}{}{}{adj.}{Que está próximo ou em redor; adjacente.}{cir.cun.vi.zi.nho}{0}
\verb{circunvolução}{}{}{"-ões}{}{s.f.}{Volta ao redor de um centro.}{cir.cun.vo.lu.ção}{0}
\verb{circunvolução}{}{Anat.}{"-ões}{}{}{Dobras sinuosas do córtex cerebral.}{cir.cun.vo.lu.ção}{0}
\verb{cireneu}{}{}{}{"-eia}{adj.}{Relativo a Cirene, antiga cidade e colônia grega da África.}{ci.re.neu}{0}
\verb{cireneu}{}{}{}{"-eia}{s.m.}{Natural ou habitante dessa cidade.}{ci.re.neu}{0}
\verb{cireneu}{}{Fig.}{}{"-eia}{adj.}{Que auxilia ou colabora em trabalhos difíceis.}{ci.re.neu}{0}
\verb{círio}{}{Relig.}{}{}{}{Procissão em que se leva uma dessas velas. (\textit{A festa do círio de Nazaré, em Belém do Pará, é uma das maiores procissões do mundo.})}{cí.rio}{0}
\verb{círio}{}{}{}{}{s.m.}{Grande vela de cera. (\textit{O padre acendeu o círio pascal na missa do Sábado de Aleluia.})}{cí.rio}{0}
\verb{cirrípede}{}{Zool.}{}{}{s.m.}{Espécime dos cirrípedes, grupo dos crustáceos marinhos, dotado de carapaça calcária, que vive preso a rochas, cascos de navios, algas etc.; são representados pelas cracas.}{cir.rí.pe.de}{0}
\verb{cirro}{}{Med.}{}{}{s.m.}{Respiração ruidosa, estertorosa; sarrido.}{cir.ro}{0}
\verb{cirro}{}{Med.}{}{}{}{Tumor canceroso de consistência dura.}{cir.ro}{0}
\verb{cirro}{}{}{}{}{s.m.}{Nuvem branca e muito alta, formada por diminutos cristais de gelo.}{cir.ro}{0}
\verb{cirrose}{ó}{Med.}{}{}{s.f.}{Inflamação crônica do fígado; cirrose hepática.}{cir.ro.se}{0}
\verb{cirrose}{ó}{Por ext.}{}{}{}{Endurecimento de um órgão resultante de inflamação de tecido conjuntivo.}{cir.ro.se}{0}
\verb{cirrótico}{}{}{}{}{adj.}{Relativo a cirrose.}{cir.ró.ti.co}{0}
\verb{cirurgia}{}{}{}{}{s.f.}{Parte da medicina que se dedica ao tratamento de doenças e traumatismos por meio de operações. }{ci.rur.gi.a}{0}
\verb{cirurgia}{}{}{}{}{}{Intervenção cirúrgica; operação.}{ci.rur.gi.a}{0}
\verb{cirurgião}{}{}{"-ões}{"-ã}{s.m.}{Especialista que se dedica à cirurgia, na área médica, odontológica ou veterinária.}{ci.rur.gi.ão}{0}
\verb{cirurgião"-dentista}{}{}{cirurgiões"-dentistas \textit{ou} cirurgiães"-dentistas}{cirurgiã"-dentista}{s.m.}{Dentista que se dedica à cirurgia.}{ci.rur.gi.ão"-den.tis.ta}{0}
\verb{cirúrgico}{}{}{}{}{adj.}{Relativo a cirurgia.}{ci.rúr.gi.co}{0}
\verb{cisalpino}{}{}{}{}{adj.}{Que está situado aquém dos Alpes, cadeia de montanhas da Europa.}{ci.sal.pi.no}{0}
\verb{cisandino}{}{}{}{}{adj.}{Que está situado aquém dos Andes, cordilheira da América do Sul.}{ci.san.di.no}{0}
\verb{cisão}{}{}{"-ões}{}{s.f.}{Ato ou efeito de cindir; divisão, separação.}{ci.são}{0}
\verb{cisão}{}{}{"-ões}{}{}{Divisão, dissidência de opiniões dentro de uma sociedade, um partido político etc.}{ci.são}{0}
\verb{cisatlântico}{}{}{}{}{adj.}{Que está situado aquém do Oceano Atlântico.}{ci.sa.tlân.ti.co}{0}
\verb{ciscar}{}{}{}{}{v.t.}{Limpar de ciscos, gravetos, folhas etc.}{cis.car}{0}
\verb{ciscar}{}{}{}{}{}{Revolver o solo em busca de comida (a galinha).}{cis.car}{0}
\verb{ciscar}{}{}{}{}{}{Incitar o cão a morder; açular.}{cis.car}{\verboinum{2}}
\verb{cisco}{}{}{}{}{s.m.}{Partícula ou grão de qualquer corpo que cai no olho; argueiro.}{cis.co}{0}
\verb{cisco}{}{}{}{}{}{Aparas miúdas; lixo, varredura.}{cis.co}{0}
\verb{cisco}{}{}{}{}{}{Pó ou miudezas de carvão.}{cis.co}{0}
\verb{cisma}{}{}{}{}{s.m.}{Divisão, separação de doutrina religiosa.}{cis.ma}{0}
\verb{cisma}{}{}{}{}{s.f.}{Ato ou efeito de cismar; ideia fixa; mania.}{cis.ma}{0}
\verb{cisma}{}{Por ext.}{}{}{}{Dissidência de opiniões dentro de uma sociedade, um partido político etc.; cisão.}{cis.ma}{0}
\verb{cisma}{}{}{}{}{}{Esperança vã; sonho, fantasia.}{cis.ma}{0}
\verb{cisma}{}{}{}{}{}{Desconfiança, suspeita, dúvida.}{cis.ma}{0}
\verb{cismado}{}{}{}{}{adj.}{Desconfiado de alguém; prevenido, receoso.}{cis.ma.do}{0}
\verb{cismar}{}{}{}{}{v.i.}{Ficar absorto em pensamentos; preocupar"-se.}{cis.mar}{0}
\verb{cismar}{}{}{}{}{v.t.}{Pensar insistentemente; refletir, cogitar.}{cis.mar}{0}
\verb{cismar}{}{}{}{}{}{Desconfiar, suspeitar.}{cis.mar}{\verboinum{1}}
\verb{cismático}{}{}{}{}{adj.}{Relativo a cisma.}{cis.má.ti.co}{0}
\verb{cismático}{}{}{}{}{adj.}{Que anda apreensivo; preocupado.}{cis.má.ti.co}{0}
\verb{cismático}{}{}{}{}{}{Que segue um cisma; dissidente.}{cis.má.ti.co}{0}
\verb{cismático}{}{}{}{}{}{Que devaneia; meditativo.}{cis.má.ti.co}{0}
\verb{cismativo}{}{}{}{}{adj.}{Que cisma; meditativo, absorto.}{cis.ma.ti.vo}{0}
\verb{cisne}{}{Zool.}{}{}{s.m.}{Ave de pescoço longo, aquática, de plumagem branca, e que percorre grandes distâncias no período de migração.}{cis.ne}{0}
\verb{cisplatino}{}{}{}{}{adj.}{Que está situado aquém do rio da Prata, no sul da América do Sul. }{cis.pla.ti.no}{0}
\verb{cissiparidade}{}{Biol.}{}{}{s.f.}{Reprodução que envolve divisão em duas partes; divisão binária.}{cis.si.pa.ri.da.de}{0}
\verb{cissura}{}{}{}{}{s.f.}{Abertura longa e estreita; fenda, fissura.}{cis.su.ra}{0}
\verb{cisterciense}{}{}{}{}{adj.2g.}{Relativo à Ordem de Cister, na França.}{cis.ter.ci.en.se}{0}
\verb{cisterciense}{}{}{}{}{}{Monge ou monja pertencente a essa ordem.}{cis.ter.ci.en.se}{0}
\verb{cisterna}{é}{}{}{}{s.f.}{Depósito, reservatório de águas pluviais.}{cis.ter.na}{0}
\verb{cístico}{}{}{}{}{adj.}{Relativo a cisto.}{cís.ti.co}{0}
\verb{cístico}{}{}{}{}{}{Relativo à vesícula biliar.}{cís.ti.co}{0}
\verb{cistite}{}{Med.}{}{}{s.f.}{Inflamação da bexiga, geralmente de origem infecciosa.}{cis.ti.te}{0}
\verb{cisto}{}{Med.}{}{}{s.m.}{Tumor formado por um saco ou vesícula que contém um líquido ou uma substância mole; quisto.}{cis.to}{0}
\verb{citação}{}{}{"-ões}{}{s.f.}{Ato ou efeito de citar, mencionar.}{ci.ta.ção}{0}
\verb{citação}{}{}{"-ões}{}{}{Trecho citado; referência.}{ci.ta.ção}{0}
\verb{citação}{}{Jur.}{"-ões}{}{}{Intimação judicial.}{ci.ta.ção}{0}
\verb{citadino}{}{}{}{}{adj.}{Referente a cidade; urbano.}{ci.ta.di.no}{0}
\verb{citadino}{}{}{}{}{s.m.}{Habitante da cidade. }{ci.ta.di.no}{0}
\verb{citado}{}{}{}{}{adj.}{Diz"-se do autor ou documento mencionado em algum texto.}{ci.ta.do}{0}
\verb{citado}{}{Jur.}{}{}{}{Aquele que recebeu intimação judicial.}{ci.ta.do}{0}
\verb{citar}{}{}{}{}{v.t.}{Fazer referência a um documento ou autor; mencionar.}{ci.tar}{0}
\verb{citar}{}{Jur.}{}{}{}{Intimar judicialmente.}{ci.tar}{\verboinum{1}}
\verb{cítara}{}{Mús.}{}{}{s.f.}{Instrumento de cordas, com forma geralmente trapezoidal, semelhante à lira.}{cí.ta.ra}{0}
\verb{citarista}{}{}{}{}{s.2g.}{Músico que toca cítara.}{ci.ta.ris.ta}{0}
\verb{citerior}{ô}{}{}{}{adj.}{Que está situado aquém, do lado de cá.}{ci.te.ri.or}{0}
\verb{citologia}{}{Biol.}{}{}{s.f.}{Parte da biologia que estuda o desenvolvimento e as funções das células.}{ci.to.lo.gi.a}{0}
\verb{citológico}{}{}{}{}{adj.}{Relativo à citologia.}{ci.to.ló.gi.co}{0}
\verb{citologista}{}{}{}{}{s.2g.}{Biólogo especializado em citologia.}{ci.to.lo.gis.ta}{0}
\verb{citoplasma}{}{Biol.}{}{}{s.f.}{Líquido de aparência gelatinosa, presente no interior das células, compreendido entre a membrana e o núcleo.}{ci.to.plas.ma}{0}
\verb{cítrico}{}{}{}{}{adj.}{Relativo a plantas do gênero \textit{Citrus}, ao qual pertencem as frutas como a laranja, o limão, a cidra, a tangerina etc.}{cí.tri.co}{0}
\verb{cítrico}{}{Quím.}{}{}{}{Diz"-se do ácido extraído de diversas plantas, especialmente nas frutas cítricas, e que tem propriedade antioxidante, anticoagulante etc.}{cí.tri.co}{0}
\verb{citricultor}{ô}{}{}{}{s.m.}{Agricultor que se dedica à plantação de frutas cítricas.}{ci.tri.cul.tor}{0}
\verb{citricultura}{}{}{}{}{s.f.}{Cultivo de plantas cítricas, como o limão, a laranja, a tangerina etc.}{ci.tri.cul.tu.ra}{0}
\verb{citrino}{}{}{}{}{adj.}{Que tem a cor e o sabor da cidra ou do limão.}{ci.tri.no}{0}
\verb{citrino}{}{Geol.}{}{}{s.m.}{Variedade de quartzo transparente, amarelado ou alaranjado, semelhante ao topázio.}{ci.tri.no}{0}
\verb{citronela}{é}{}{}{}{s.f.}{Essência retirada de uma planta de cheiro semelhante ao do limão, usada para espantar insetos.}{ci.tro.ne.la}{0}
\verb{ciumada}{}{}{}{}{s.f.}{Ciúme exagerado; explosão pública de ciúme; ciumeira.}{ci.u.ma.da}{0}
\verb{ciúme}{}{}{}{}{s.m.}{Sentimento penoso; medo de perder algo ou alguém amado.}{ci.ú.me}{0}
\verb{ciúme}{}{}{}{}{}{Despeito, inveja.}{ci.ú.me}{0}
\verb{ciumeira}{ê}{Pop.}{}{}{s.f.}{Ciumada.}{ci.u.mei.ra}{0}
\verb{ciumento}{}{}{}{}{adj.}{Que tem ciúme.}{ci.u.men.to}{0}
\verb{cível}{}{Jur.}{"-eis}{}{adj.2g.}{Relativo ao direito civil.}{cí.vel}{0}
\verb{cível}{}{Jur.}{"-eis}{}{s.m.}{Tribunal onde são julgados os processos de natureza civil.}{cí.vel}{0}
\verb{cívico}{}{}{}{}{adj.}{Relativo ao cidadão como membro do Estado.}{cí.vi.co}{0}
\verb{cívico}{}{}{}{}{}{Que se relaciona com a pátria; patriótico.}{cí.vi.co}{0}
\verb{civil}{}{}{"-is}{}{adj.2g.}{Relativo ao povo ou às relações dos cidadãos entre si.}{ci.vil}{0}
\verb{civil}{}{}{"-is}{}{}{Relativo ao cidadão, considerado em seu caráter, condições e relações privadas, reguladas por normas do direito civil.}{ci.vil}{0}
\verb{civil}{}{}{"-is}{}{}{Que não é militar ou religioso.}{ci.vil}{0}
\verb{civil}{}{}{"-is}{}{}{Civilizado, sociável, polido, cortês.}{ci.vil}{0}
\verb{civil}{}{}{"-is}{}{s.m.}{Indivíduo que não é militar; paisano.}{ci.vil}{0}
\verb{civilidade}{}{}{}{}{s.f.}{Observância das boas maneiras entre as pessoas para demonstrar respeito mútuo e consideração.}{ci.vi.li.da.de}{0}
\verb{civilidade}{}{}{}{}{}{Cortesia, polidez, delicadeza.}{ci.vi.li.da.de}{0}
\verb{civilismo}{}{}{}{}{s.m.}{Sistema ou doutrina dos que defendem o exercício do poder do Estado pelos civis.}{ci.vi.lis.mo}{0}
\verb{civilismo}{}{}{}{}{}{Ardor cívico; civismo.}{ci.vi.lis.mo}{0}
\verb{civilista}{}{}{}{}{adj.2g.}{Relativo ao civilismo.}{ci.vi.lis.ta}{0}
\verb{civilista}{}{}{}{}{s.2g.}{Indivíduo que é partidário do civilismo.}{ci.vi.lis.ta}{0}
\verb{civilização}{}{}{"-ões}{}{s.f.}{Ato ou efeito de civilizar.}{ci.vi.li.za.ção}{0}
\verb{civilização}{}{}{"-ões}{}{}{Conjunto de características culturais, políticas, sociais e econômicas de uma época, de uma região, de um país ou de uma sociedade.}{ci.vi.li.za.ção}{0}
\verb{civilização}{}{}{"-ões}{}{}{Condição de adiantamento cultural e social; progresso.}{ci.vi.li.za.ção}{0}
\verb{civilizado}{}{}{}{}{adj.}{Que tem civilização; aquele que saiu do estado de barbárie.}{ci.vi.li.za.do}{0}
\verb{civilizado}{}{}{}{}{}{Bem"-educado, polido, cortês.}{ci.vi.li.za.do}{0}
\verb{civilizar}{}{}{}{}{v.t.}{Converter ao estado de civilização; fazer sair do estado de barbárie.}{ci.vi.li.zar}{0}
\verb{civilizar}{}{}{}{}{}{Tornar polido, cortês, bem"-educado.}{ci.vi.li.zar}{\verboinum{1}}
\verb{civismo}{}{}{}{}{s.m.}{Dedicação e fidelidade ao interesse público, à pátria; patriotismo, civilismo.}{ci.vis.mo}{0}
\verb{cizânia}{}{Bot.}{}{}{s.f.}{Espécie de capim que cresce no meio do trigo; joio.}{ci.zâ.nia}{0}
\verb{cizânia}{}{Fig.}{}{}{}{Desarmonia, desavença, discórdia.}{ci.zâ.nia}{0}
\verb{Cl}{}{Quím.}{}{}{}{Símb. do \textit{cloro}. }{Cl}{0}
\verb{clã}{}{}{}{}{s.m.}{Agrupamento de famílias que têm um ascendente comum.}{clã}{0}
\verb{clã}{}{}{}{}{}{Na Escócia e na Irlanda, tribo formada por pessoas de origem comum.}{clã}{0}
\verb{clã}{}{}{}{}{}{Partido, facção.}{clã}{0}
\verb{clamar}{}{}{}{}{v.t.}{Proferir em voz alta; gritar, bradar.}{cla.mar}{0}
\verb{clamar}{}{}{}{}{}{Pedir insistentemente; rogar, suplicar.}{cla.mar}{0}
\verb{clamar}{}{}{}{}{}{Protestar veementemente; reclamar, exigir.}{cla.mar}{\verboinum{1}}
\verb{clamor}{ô}{}{}{}{s.m.}{Ato de clamar.}{cla.mor}{0}
\verb{clamor}{ô}{}{}{}{}{Gritaria de quem protesta, reclama; vozeria.}{cla.mor}{0}
\verb{clamor}{ô}{}{}{}{}{Rogo ou súplica proferida em voz alta.}{cla.mor}{0}
\verb{clamoroso}{ô}{}{"-oso ⟨ó⟩}{"-osa ⟨ó⟩}{adj.}{Em que há clamor; ruidoso, escandaloso.}{cla.mo.ro.so}{0}
\verb{clamoroso}{ô}{}{"-oso ⟨ó⟩}{"-osa ⟨ó⟩}{}{Muito evidente; indubitável, incontestável.}{cla.mo.ro.so}{0}
\verb{clandestinidade}{}{}{}{}{s.f.}{Qualidade ou caráter do que é clandestino; ilegalidade.}{clan.des.ti.ni.da.de}{0}
\verb{clandestino}{}{}{}{}{adj.}{Que se faz às escondidas; ilegal, ilegítimo.}{clan.des.ti.no}{0}
\verb{clandestino}{}{}{}{}{}{Aquele que se introduz furtivamente a bordo de um navio, avião ou trem para viajar sem documentos nem passagem.}{clan.des.ti.no}{0}
\verb{clangor}{ô}{}{}{}{s.m.}{Som forte, estridente, de trombeta ou instrumento semelhante.}{clan.gor}{0}
\verb{claque}{}{}{}{}{s.f.}{Grupo de espectadores pagos para aplaudir ou vaiar um espetáculo, um ator etc.}{cla.que}{0}
\verb{clara}{}{}{}{}{s.f.}{Substância transparente e albuminosa que envolve a gema do ovo.}{cla.ra}{0}
\verb{clara}{}{}{}{}{}{Abertura em bosque ou floresta; clareira.}{cla.ra}{0}
\verb{claraboia}{ó}{}{}{}{s.f.}{Abertura de um telhado, quase sempre coberta por cúpula envidraçada, para iluminar peça interior de um edifício.}{cla.ra.boi.a}{0}
\verb{claraboia}{ó}{}{}{}{}{Janela redonda por onde penetra luz em uma casa.}{cla.ra.boi.a}{0}
\verb{clarão}{}{}{"-ões}{}{s.m.}{Claridade intensa e repentina.}{cla.rão}{0}
\verb{clarão}{}{}{"-ões}{}{}{Cintilação rápida e viva; brilho.}{cla.rão}{0}
\verb{clarão}{}{Fig.}{"-ões}{}{}{Inspiração súbita.}{cla.rão}{0}
\verb{clarear}{}{}{}{}{v.t.}{Tornar claro; aclarar, iluminar.}{cla.re.ar}{0}
\verb{clarear}{}{}{}{}{v.i.}{Fazer"-se dia; romper a aurora.}{cla.re.ar}{0}
\verb{clarear}{}{}{}{}{v.t.}{Tornar mais inteligível; esclarecer.}{cla.re.ar}{\verboinum{4}}
\verb{clareira}{ê}{}{}{}{s.f.}{Espaço em bosque, mata ou floresta em que não há vegetação; clara.}{cla.rei.ra}{0}
\verb{clareira}{ê}{}{}{}{}{Espaço vazio; lacuna, claro.}{cla.rei.ra}{0}
\verb{clareza}{ê}{}{}{}{s.f.}{Qualidade do que é claro ou inteligível.}{cla.re.za}{0}
\verb{clareza}{ê}{}{}{}{}{Limpidez, transparência, nitidez.}{cla.re.za}{0}
\verb{claridade}{}{}{}{}{s.f.}{Qualidade do que é claro; luminosidade.}{cla.ri.da.de}{0}
\verb{claridade}{}{}{}{}{}{Luz intensa; fulgor.}{cla.ri.da.de}{0}
\verb{clarificar}{}{}{}{}{v.t.}{Tornar claro ou mais claro; purificar, limpar.}{cla.ri.fi.car}{0}
\verb{clarificar}{}{}{}{}{v.pron.}{Purificar"-se espiritualmente; arrepender"-se.}{cla.ri.fi.car}{\verboinum{2}}
\verb{clarim}{}{Mús.}{"-ins}{}{s.m.}{Pequeno instrumento de sopro, com bocal em forma de taça, de som claro e estridente.}{cla.rim}{0}
\verb{clarinada}{}{}{}{}{s.f.}{Toque ou som de clarim.}{cla.ri.na.da}{0}
\verb{clarineta}{ê}{Mús.}{}{}{s.f.}{Clarinete.}{cla.ri.ne.ta}{0}
\verb{clarinete}{ê}{Mús.}{}{}{s.m.}{Instrumento de sopro de palheta simples e chaves como as da flauta.}{cla.ri.ne.te}{0}
\verb{clarinetista}{}{}{}{}{s.2g.}{Músico que toca o clarinete.}{cla.ri.ne.tis.ta}{0}
\verb{clarividência}{}{}{}{}{s.f.}{Qualidade ou caráter de clarividente; perspicácia, sagacidade.}{cla.ri.vi.dên.cia}{0}
\verb{clarividência}{}{Relig.}{}{}{}{No Espiritismo, faculdade por meio da qual o médium, sem utilizar os sentidos, distingue formas em um plano extrafísico.}{cla.ri.vi.dên.cia}{0}
\verb{clarividente}{}{}{}{}{adj.2g.}{Que vê com clareza; sagaz, perspicaz.}{cla.ri.vi.den.te}{0}
\verb{clarividente}{}{Relig.}{}{}{}{Diz"-se do médium que tem a faculdade de distinguir formas em um plano extrafísico.}{cla.ri.vi.den.te}{0}
\verb{claro}{}{}{}{}{adj.}{Que clareia, ilumina; brilhante, luminoso.}{cla.ro}{0}
\verb{claro}{}{}{}{}{}{Que se vê ou se distingue bem.}{cla.ro}{0}
\verb{claro}{}{}{}{}{}{Diz"-se de indivíduo de cor branca ou quase branca.}{cla.ro}{0}
\verb{claro}{}{}{}{}{}{Que se entende facilmente; evidente, patente.}{cla.ro}{0}
\verb{claro}{}{}{}{}{s.m.}{Espaço vazio; lacuna, vaga.}{cla.ro}{0}
\verb{claro}{}{}{}{}{adv.}{Com clareza, claramente.}{cla.ro}{0}
\verb{claro"-escuro}{}{}{claro"-escuros \textit{ou} claros"-escuros}{}{s.m.}{Claridade atenuada; luz suave; penumbra.}{cla.ro"-es.cu.ro}{0}
\verb{claro"-escuro}{}{Art.}{claro"-escuros \textit{ou} claros"-escuros}{}{}{Na pintura, modulação de um efeito de luz difusa sobre um fundo de sombra.}{cla.ro"-es.cu.ro}{0}
\verb{claro"-escuro}{}{}{claro"-escuros \textit{ou} claros"-escuros}{}{}{Impressão produzida pelos contrastes dos claros com os escuros.}{cla.ro"-es.cu.ro}{0}
\verb{classe}{}{}{}{}{s.f.}{Grupo de indivíduos que se distinguem por suas particularidades dentro de um grupo maior; categoria. (\textit{A classe média é a que tem sentido mais os efeitos do desemprego ao longo dos anos.})}{clas.se}{0}
\verb{classe}{}{}{}{}{}{Conjunto de alunos de uma sala de aula.}{clas.se}{0}
\verb{classe}{}{}{}{}{}{Boas maneiras; educação, distinção.}{clas.se}{0}
\verb{Classicismo}{}{Art.}{}{}{s.m.}{Escola literária e artística do século \textsc{xvi}, caracterizada pela busca ao equilíbrio, à harmonia das formas e à idealização da realidade, inspirada na Antiguidade greco"-romana.}{Clas.si.cis.mo}{0}
\verb{clássico}{}{}{}{}{adj.}{Que se refere à arte ou à cultura greco"-romanas da Antiguidade.}{clás.si.co}{0}
\verb{clássico}{}{}{}{}{}{Que se tornou um modelo digno de apreciação e imitação; exemplar.}{clás.si.co}{0}
\verb{clássico}{}{}{}{}{s.m.}{Autor ou obra clássica ou consagrada.}{clás.si.co}{0}
\verb{clássico}{}{Esport.}{}{}{}{Partida de futebol disputada entre dois clubes famosos e tradicionalmente rivais.}{clás.si.co}{0}
\verb{classificação}{}{}{"-ões}{}{s.f.}{Ato ou efeito de classificar.}{clas.si.fi.ca.ção}{0}
\verb{classificação}{}{}{"-ões}{}{}{Distribuição sistemática em diferentes categorias de acordo com as características comuns.}{clas.si.fi.ca.ção}{0}
\verb{classificação}{}{}{"-ões}{}{}{Posição obtida por um candidato em um concurso, processo seletivo, competição etc. }{clas.si.fi.ca.ção}{0}
\verb{classificação}{}{Biol.}{"-ões}{}{}{Método de organização dos seres vivos em categorias e grupos.}{clas.si.fi.ca.ção}{0}
\verb{classificado}{}{}{}{}{adj.}{Que se classificou; distribuído por classes.}{clas.si.fi.ca.do}{0}
\verb{classificado}{}{}{}{}{}{Aquele que obteve número de pontos suficiente em concurso, processo seletivo, competição etc.}{clas.si.fi.ca.do}{0}
\verb{classificado}{}{}{}{}{}{Diz"-se do anúncio de jornal que obedece a determinada classificação.}{clas.si.fi.ca.do}{0}
\verb{classificar}{}{}{}{}{v.t.}{Dividir, distribuir em classes, segundo um sistema ou método.}{clas.si.fi.car}{0}
\verb{classificar}{}{}{}{}{}{Determinar a ordem dos candidatos aprovados em um concurso, processo seletivo, competição etc.}{clas.si.fi.car}{0}
\verb{classificar}{}{Biol.}{}{}{}{Determinar a classe, ordem, família, gênero e espécie dos seres vivos.}{clas.si.fi.car}{0}
\verb{classificar}{}{}{}{}{}{Qualificar, rotular, tachar.}{clas.si.fi.car}{\verboinum{2}}
\verb{classista}{}{}{}{}{adj.2g.}{Que representa uma classe ou defende seus direitos. }{clas.sis.ta}{0}
\verb{claudicar}{}{}{}{}{v.i.}{Arrastar de uma perna; coxear, mancar.}{clau.di.car}{0}
\verb{claudicar}{}{Fig.}{}{}{}{Cometer faltas; falhar, errar.}{clau.di.car}{\verboinum{2}}
\verb{claustro}{}{}{}{}{s.m.}{Em um convento, parte interior descoberta rodeada de um passeio.}{claus.tro}{0}
\verb{claustro}{}{Por ext.}{}{}{}{Convento, monastério, mosteiro.}{claus.tro}{0}
\verb{claustro}{}{}{}{}{}{A vida, a regra monástica.}{claus.tro}{0}
\verb{claustrofobia}{}{}{}{}{s.f.}{Medo doentio de permanecer em ambientes fechados.}{claus.tro.fo.bi.a}{0}
\verb{claustrofóbico}{}{}{}{}{adj.}{Relativo a claustrofobia.}{claus.tro.fó.bi.co}{0}
\verb{claustrófobo}{}{}{}{}{s.m.}{Indivíduo que tem medo de lugares fechados.}{claus.tró.fo.bo}{0}
\verb{cláusula}{}{Jur.}{}{}{s.f.}{Cada um dos artigos ou disposições que fazem parte de um contrato, de uma escritura, de um documento.}{cláu.su.la}{0}
\verb{cláusula}{}{}{}{}{}{Norma, preceito, condição.}{cláu.su.la}{0}
\verb{clausura}{}{}{}{}{s.f.}{Recinto, local fechado.}{clau.su.ra}{0}
\verb{clausura}{}{}{}{}{}{Parte do convento interditada a pessoas estranhas.}{clau.su.ra}{0}
\verb{clausura}{}{}{}{}{}{Estado ou situação de quem não pode sair do claustro; internamento, reclusão.}{clau.su.ra}{0}
\verb{clava}{}{}{}{}{s.f.}{Pedaço de pau grosso, mais volumoso numa das extremidades,  que se usa para ataque e defesa; maça.}{cla.va}{0}
\verb{clave}{}{Mús.}{}{}{s.f.}{Sinal colocado no início da pauta musical que indica o tom das notas.}{cla.ve}{0}
\verb{clavícula}{}{Anat.}{}{}{s.f.}{Cada um dos dois ossos longos e curvos que articulam o esterno com a escápula.}{cla.ví.cu.la}{0}
\verb{claviculário}{}{}{}{}{s.m.}{Indivíduo responsável pela chave de um cofre, arquivo etc.; chaveiro.}{cla.vi.cu.lá.rio}{0}
\verb{clavina}{}{}{}{}{s.f.}{Espécie de espingarda curta; carabina.}{cla.vi.na}{0}
\verb{clemência}{}{}{}{}{s.f.}{Sentimento ou disposição para perdoar ou minorar os castigos ou faltas; indulgência, bondade.}{cle.mên.cia}{0}
\verb{clemência}{}{Fig.}{}{}{}{Brandura, suavidade, amenidade.}{cle.mên.cia}{0}
\verb{clemente}{}{}{}{}{adj.2g.}{Que revela clemência; indulgente, benevolente.}{cle.men.te}{0}
\verb{clemente}{}{}{}{}{}{Suave, ameno, brando.}{cle.men.te}{0}
\verb{clepsidra}{}{}{}{}{s.f.}{Instrumento composto por dois cones, sendo um deles cheio de água, usado para medir o tempo baseando"-se na velocidade de escoamento da água de um cone a outro; relógio de água.}{clep.si.dra}{0}
\verb{cleptomania}{}{}{}{}{s.f.}{Impulso doentio que leva um indivíduo a roubar objetos, independentemente de seu valor.}{clep.to.ma.ni.a}{0}
\verb{cleptomaníaco}{}{}{}{}{s.m.}{Indivíduo que sofre de cleptomania; cleptômano.}{clep.to.ma.ní.a.co}{0}
\verb{cleptômano}{}{}{}{}{s.m.}{Cleptomaníaco.}{clep.tô.ma.no}{0}
\verb{clerical}{}{}{"-ais}{}{adj.2g.}{Relativo ao clero ou à Igreja.}{cle.ri.cal}{0}
\verb{clericalismo}{}{}{}{}{s.m.}{Doutrina que defende a influência da Igreja nos negócios do Estado.}{cle.ri.ca.lis.mo}{0}
\verb{clericalismo}{}{}{}{}{}{Atitude dos que apoiam o clero de forma incondicional.}{cle.ri.ca.lis.mo}{0}
\verb{clérigo}{}{}{}{}{s.m.}{Indivíduo que pertence à condição eclesiástica; sacerdote, padre.}{clé.ri.go}{0}
\verb{clero}{é}{}{}{}{s.m.}{O conjunto dos padres e bispos cristãos em sua totalidade ou limitados a uma igreja, região etc.}{cle.ro}{0}
\verb{clicar}{}{}{}{}{v.i.}{Produzir o som do clique, estalido.}{cli.car}{0}
\verb{clicar}{}{Informát.}{}{}{}{Apertar o botão do \textit{mouse}.}{cli.car}{\verboinum{2}}
\verb{clicar}{}{}{}{}{}{Fotografar.}{cli.car}{0}
\verb{clichê}{}{}{}{}{s.m.}{Chapa metálica onde se grava em relevo uma imagem ou um texto para ser reproduzido por meio de impressão.}{cli.chê}{0}
\verb{clichê}{}{Fig.}{}{}{}{Frase banalizada por ser muito repetida; lugar"-comum, chavão.}{cli.chê}{0}
\verb{clicheria}{}{}{}{}{s.f.}{Oficina onde são feitos os clichês; fotogravura.}{cli.che.ri.a}{0}
\verb{cliente}{}{}{}{}{s.2g.}{Pessoa que recorre aos serviços profissionais de outrem.}{cli.en.te}{0}
\verb{cliente}{}{}{}{}{}{Comprador assíduo de um estabelecimento comercial; freguês.}{cli.en.te}{0}
\verb{clientela}{é}{}{}{}{s.f.}{Conjunto de clientes; freguesia.}{cli.en.te.la}{0}
\verb{clientelismo}{}{}{}{}{s.m.}{Tipo de relação política praticada durante as eleições que consiste em privilegiar uma clientela em troca de seus votos.}{cli.en.te.lis.mo}{0}
\verb{clima}{}{}{}{}{}{Extensão de terra onde a temperatura é quase idêntica.}{cli.ma}{0}
\verb{clima}{}{}{}{}{s.m.}{Conjunto de fenômenos meteorológicos que caracterizam as condições atmosféricas de uma região, pela influência que exercem sobre a vida na Terra.}{cli.ma}{0}
\verb{clima}{}{Fig.}{}{}{}{Atmosfera moral; ambiente.}{cli.ma}{0}
\verb{climatérico}{}{}{}{}{adj.}{Relativo ao climatério.}{cli.ma.té.ri.co}{0}
\verb{climatério}{}{Med.}{}{}{s.m.}{Período que precede o término da vida reprodutiva da mulher, e, no homem, o declínio da atividade sexual, marcado, em ambos, por alterações endócrinas, somáticas e psíquicas.}{cli.ma.té.rio}{0}
\verb{climático}{}{}{}{}{adj.}{Relativo a clima.}{cli.má.ti.co}{0}
\verb{climatização}{}{}{"-ões}{}{s.f.}{Conjunto dos meios técnicos que permitem criar ou manter, por meio de aparelhos,  em recinto fechado, condições adequadas de temperatura, pressão, umidade, independentes da atmosfera exterior.}{cli.ma.ti.za.ção}{0}
\verb{climatização}{}{}{"-ões}{}{}{Preparo de um produto ou um aparelho para enfrentar determinadas condições climáticas.}{cli.ma.ti.za.ção}{0}
\verb{climatizar}{}{}{}{}{v.t.}{Proceder à climatização.}{cli.ma.ti.zar}{\verboinum{1}}
\verb{clímax}{cs}{}{}{}{s.m.}{Grau máximo; ponto culminante.}{clí.max}{0}
\verb{clímax}{cs}{}{}{}{}{Momento em que o prazer da excitação sexual atinge o máximo de intensidade; orgasmo.}{clí.max}{0}
\verb{clina}{}{}{}{}{}{Var. de \textit{crina}.}{cli.na}{0}
\verb{clínica}{}{}{}{}{s.f.}{A prática da medicina.}{clí.ni.ca}{0}
\verb{clínica}{}{}{}{}{}{Conjunto de pessoas que são tratadas por um médico.}{clí.ni.ca}{0}
\verb{clínica}{}{}{}{}{}{Estabelecimento hospitalar privado.}{clí.ni.ca}{0}
\verb{clinicar}{}{}{}{}{v.i.}{Exercer, praticar a medicina.}{cli.ni.car}{\verboinum{2}}
\verb{clínico}{}{}{}{}{adj.}{Relativo a clínica médica.}{clí.ni.co}{0}
\verb{clínico}{}{}{}{}{}{Diz"-se do tratamento que se efetua junto ao doente.}{clí.ni.co}{0}
\verb{clínico}{}{}{}{}{s.m.}{Médico que exerce a clínica.}{clí.ni.co}{0}
\verb{clip}{}{}{}{}{s.m.}{Videoclipe.}{\textit{clip}}{0}
\verb{clipagem}{}{}{"-ens}{}{s.f.}{Serviço profissional de recorte de matéria em jornais e revistas sobre determinado tópico, empresa, pessoa, etc.; \textit{clipping}.}{cli.pa.gem}{0}
\verb{clipagem}{}{}{"-ens}{}{}{Conjunto de recortes das principais notícias na mídia impressa ou eletrônica.}{cli.pa.gem}{0}
\verb{clipe}{}{}{}{}{s.m.}{Pequena peça de metal ou plástico usada para prender papéis.}{cli.pe}{0}
\verb{clipe}{}{}{}{}{}{Joia feita de uma placa de metal ou pedraria, usada como brinco ou broche.}{cli.pe}{0}
\verb{clipe}{}{}{}{}{}{Forma reduzida de \textit{videoclipe}.}{cli.pe}{0}
\verb{clipping}{}{}{}{}{s.m.}{Clipagem.}{\textit{clipping}}{0}
\verb{clique}{}{}{}{}{s.m.}{Ruído ou estalido curto e seco.}{cli.que}{0}
\verb{clique}{}{Informát.}{}{}{}{Ato de clicar.}{cli.que}{0}
\verb{clister}{é}{Med.}{}{}{s.m.}{Injeção de água ou medicamento no reto por meio de equipamento apropriado; lavagem intestinal.}{clis.ter}{0}
\verb{clitóris}{}{Anat.}{}{}{s.m.}{Protuberância erétil e muito sensível situada na parte superior da vulva.}{cli.tó.ris}{0}
\verb{clivagem}{}{}{"-ens}{}{s.f.}{Propriedade que têm certos cristais de se dividir em planos.}{cli.va.gem}{0}
\verb{cloaca}{}{}{}{}{s.f.}{Fossa ou cano para receber dejetos e sujeiras.}{clo.a.ca}{0}
\verb{cloaca}{}{}{}{}{}{Vaso sanitário; latrina.}{clo.a.ca}{0}
\verb{cloaca}{}{}{}{}{}{Depósito de imundícies.}{clo.a.ca}{0}
\verb{cloaca}{}{Zool.}{}{}{}{Cavidade por onde saem os produtos dos sistemas excretor, urinário e reprodutor, existente nos peixes, répteis, anfíbios e aves.}{clo.a.ca}{0}
\verb{clonagem}{}{}{"-ens}{}{s.f.}{Processo de obtenção de um clone.}{clo.na.gem}{0}
\verb{clonar}{}{}{}{}{v.t.}{Produzir por meio das técnicas de clonagem.}{clo.nar}{\verboinum{1}}
\verb{clone}{}{}{}{}{s.m.}{Ser vivo originado a partir de outro, com mesmo material genético e aparência.}{clo.ne}{0}
\verb{clorar}{}{}{}{}{v.t.}{Tratar água com cloro, geralmente para torná"-la estéril e potável.}{clo.rar}{\verboinum{1}}
\verb{cloreto}{ê}{Quím.}{}{}{s.m.}{Qualquer sal ou ânion derivado do ácido clorídrico.}{clo.re.to}{0}
\verb{clorídrico}{}{Quím.}{}{}{adj.}{Relativo ao ácido clorídrico.}{clo.rí.dri.co}{0}
\verb{cloro}{ó}{Quím.}{}{}{s.m.}{Elemento químico do grupo dos halogênios, gasoso, tóxico, de cheiro desagradável, utilizado como agente branqueador, oxidante e desinfetante, na limpeza da água e na  fabricação de produtos de limpeza, solventes, inseticidas, resinas etc. \elemento{17}{35.4527}{Cl}.}{clo.ro}{0}
\verb{clorofila}{}{Bot.}{}{}{s.f.}{Pigmento vegetal de cor verde, responsável pela fotossíntese.}{clo.ro.fi.la}{0}
\verb{clorofórmio}{}{Quím.}{}{}{s.m.}{Líquido incolor e muito volátil, de cheiro forte e usado como anestésico e solvente.}{clo.ro.fór.mio}{0}
\verb{cloroformizar}{}{}{}{}{v.t.}{Aplicar clorofórmio ou anestesiar utilizando clorofórmio.}{clo.ro.for.mi.zar}{\verboinum{1}}
\verb{clorose}{ó}{Med.}{}{}{s.f.}{Anemia peculiar às mulheres jovens, caracterizada pelo tom esverdeado da pele, perturbações menstruais e fraqueza.}{clo.ro.se}{0}
\verb{clorose}{ó}{Bot.}{}{}{}{Doença das plantas caracterizada pela perda da coloração verde.}{clo.ro.se}{0}
\verb{close}{ô}{}{}{}{s.m.}{Em fotografia ou cinegrafia, tipo de enquadramento em que o objeto fica muito próximo.}{clo.se}{0}
\verb{closet}{}{}{}{}{s.m.}{Compartimento de uma residência, geralmente pequeno e sem janelas, para guardar roupas, louças e utensílios.}{\textit{closet}}{0}
\verb{close"-up}{}{}{}{}{s.m.}{Close.}{\textit{close"-up}}{0}
\verb{clube}{}{}{}{}{s.m.}{Associação de pessoas com fins esportivos, recreativos, literários.}{clu.be}{0}
\verb{clube}{}{}{}{}{}{O local onde funciona essa associação.}{clu.be}{0}
\verb{Cm}{}{Quím.}{}{}{}{Símb. do \textit{cúrio}.}{Cm}{0}
\verb{cm}{}{}{}{}{}{Abrev. de \textit{centímetro}.}{cm}{0}
\verb{cnidário}{}{Zool.}{}{}{s.m.}{Espécime dos cnidários, filo de animais invertebrados aquáticos, em sua maioria marinhos, com tentáculos em volta da boca; são os pólipos, as medusas, os corais, as anêmonas"-do"-mar etc.  }{cni.dá.rio}{0}
\verb{Co}{}{Quím.}{}{}{}{Símb. do \textit{cobalto}.}{Co}{0}
%\verb{}{}{}{}{}{}{}{}{0}
\verb{coabitação}{}{}{"-ões}{}{s.f.}{Ato de coabitar.}{co.a.bi.ta.ção}{0}
\verb{coabitar}{}{}{}{}{v.t.}{Morar em comum; morar junto.}{co.a.bi.tar}{0}
\verb{coabitar}{}{}{}{}{v.i.}{Viver intimamente com alguém.}{co.a.bi.tar}{\verboinum{1}}
\verb{coação}{}{}{"-ões}{}{s.f.}{Ato de coagir; coação.}{co.a.ção}{0}
\verb{coação}{}{}{"-ões}{}{}{Estado de quem é coagido.}{co.a.ção}{0}
\verb{coação}{}{}{"-ões}{}{s.f.}{Ato ou efeito de coar.}{co.a.ção}{0}
\verb{coadjutor}{ô}{}{}{}{adj.}{Que coadjuva.}{co.ad.ju.tor}{0}
\verb{coadjutor}{ô}{}{}{}{s.m.}{Sacerdote ajudante de um pároco ou bispo.}{co.ad.ju.tor}{0}
\verb{coadjuvante}{}{}{}{}{adj.2g.}{Que coadjuva, que atua junto a outrem para um fim comum.}{co.ad.ju.van.te}{0}
\verb{coadjuvante}{}{}{}{}{}{Em cinema e teatro, diz"-se de ator que interpreta papel secundário.}{co.ad.ju.van.te}{0}
\verb{coadjuvar}{}{}{}{}{v.t.}{Ajudar, auxiliar.}{co.ad.ju.var}{\verboinum{1}}
\verb{coador}{ô}{}{}{}{adj.}{Que coa.}{co.a.dor}{0}
\verb{coador}{ô}{}{}{}{s.m.}{Utensílio em forma de recipiente com fundo furado ou feito de tela, que permite a passagem somente de líquidos ou de substâncias mais finas.}{co.a.dor}{0}
\verb{coador}{ô}{}{}{}{}{Saco próprio para coar café.}{co.a.dor}{0}
\verb{coadunar}{}{}{}{}{v.t.}{Reunir, incorporar, juntar formando um todo.}{co.a.du.nar}{0}
\verb{coadunar}{}{}{}{}{v.pron.}{Combinar"-se, conformar"-se.}{co.a.du.nar}{\verboinum{1}}
\verb{coagir}{}{}{}{}{v.t.}{Obrigar alguém a fazer alguma coisa, geralmente por meio da força ou de ameaça; forçar, constranger.}{co.a.gir}{\verboinum{22}}
\verb{coagulação}{}{}{"-ões}{}{s.f.}{Ato ou efeito de coagular.}{co.a.gu.la.ção}{0}
\verb{coagulação}{}{}{"-ões}{}{}{Passagem de um líquido ao estado sólido.}{co.a.gu.la.ção}{0}
\verb{coagulador}{ô}{}{}{}{adj.}{Que produz ou facilita a coagulação.}{co.a.gu.la.dor}{0}
\verb{coagulante}{}{}{}{}{adj.2g.}{Que coagula.}{co.a.gu.lan.te}{0}
\verb{coagular}{}{}{}{}{v.t.}{Causar a solidificação.}{co.a.gu.lar}{0}
\verb{coagular}{}{}{}{}{v.i.}{Solidificar"-se.}{co.a.gu.lar}{\verboinum{1}}
\verb{coágulo}{}{}{}{}{s.m.}{Parte coagulada de um líquido; coalho.}{co.á.gu.lo}{0}
\verb{coala}{}{Zool.}{}{}{s.m.}{Marsupial de orelhas grandes, pelo cinzento, sem rabo e que se alimenta de brotos de eucaliptos.}{co.a.la}{0}
\verb{coalhada}{}{}{}{}{s.f.}{Alimento feito com leite coalhado. (\textit{Quero uma coalhada para sobremesa.})}{co.a.lha.da}{0}
\verb{coalhado}{}{}{}{}{adj.}{Solidificado, coagulado. (\textit{O leite ficou coalhado.})}{co.a.lha.do}{0}
\verb{coalhado}{}{}{}{}{}{Cheio, apinhado. (\textit{O parque está coalhado de gente.})}{co.a.lha.do}{0}
\verb{coalhar}{}{}{}{}{v.t.}{Coagular, solidificar.}{co.a.lhar}{\verboinum{1}}
\verb{coalheira}{ê}{}{}{}{s.f.}{A quarta cavidade do estômago dos mamíferos ruminantes.}{co.a.lhei.ra}{0}
\verb{coalheira}{ê}{}{}{}{}{Líquido produzido nessa cavidade, utilizado nas queijarias para coalhar o leite; coalho, coagulador.}{co.a.lhei.ra}{0}
\verb{coalho}{}{}{}{}{s.m.}{Coágulo.}{co.a.lho}{0}
\verb{coalho}{}{}{}{}{}{Líquido coagulante utilizado na fabricação de queijo.}{co.a.lho}{0}
\verb{coalização}{}{}{"-ões}{}{s.f.}{Coalizão.}{co.a.li.za.ção}{0}
\verb{coalizão}{}{}{"-ões}{}{s.f.}{Acordo político.}{co.a.li.zão}{0}
\verb{coalizão}{}{}{"-ões}{}{}{Aliança entre nações.}{co.a.li.zão}{0}
\verb{coalizar"-se}{}{}{}{}{v.pron.}{Fazer acordo para um fim comum; aliar"-se, unir"-se.}{co.a.li.zar"-se}{\verboinum{1}}
\verb{coar}{}{}{}{}{v.t.}{Fazer passar pelo coador, geralmente separando a parte líquida da parte sólida de uma mistura.}{co.ar}{\verboinum{1}}
\verb{coarctar}{}{}{}{}{}{Var. de \textit{coartar}.}{co.arc.tar}{0}
\verb{coartar}{}{}{}{}{v.t.}{Obrigar, impor, coagir.}{co.ar.tar}{\verboinum{1}}
\verb{coativo}{}{}{}{}{adj.}{Que tem o direito ou o poder de obrigar ou compelir.}{co.a.ti.vo}{0}
\verb{coautor}{ô}{}{coautores ⟨ô⟩}{}{s.m.}{Quem produz uma obra ou um  trabalho, com outra ou outras pessoas.}{co.au.tor}{0}
\verb{coautor}{ô}{}{coautores ⟨ô⟩}{}{}{Quem é acusado de delito, com outra ou outras pessoas.}{co.au.tor}{0}
\verb{coautoria}{}{}{}{}{s.f.}{Estado, qualidade ou caráter de coautor.}{co.au.to.ri.a}{0}
\verb{coautoria}{}{}{}{}{}{Pluralidade de autores de uma obra, de um crime.}{co.au.to.ri.a}{0}
\verb{coaxar}{ch}{}{}{}{v.i.}{Soltar a voz (a rã ou o sapo).}{co.a.xar}{\verboinum{1}}
\verb{coaxial}{cs}{}{"-ais}{}{adj.2g.}{Que tem um eixo comum.}{co.a.xi.al}{0}
\verb{coaxo}{ch}{}{}{}{s.m.}{A voz da rã ou do sapo.}{co.a.xo}{0}
\verb{cobaia}{}{}{}{}{s.f.}{Pequeno mamífero roedor utilizado em experiências de laboratório; porquinho"-da"-índia.}{co.bai.a}{0}
\verb{cobaia}{}{Por ext.}{}{}{}{Qualquer animal ou pessoa usado experimentalmente em pesquisas e estudos de laboratório.}{co.bai.a}{0}
\verb{cobalto}{}{Quím.}{}{}{s.m.}{Elemento químico metálico, prateado, brilhante, duro e quebradiço, utilizado como fonte de radiação em radioterapia, em ligas com o ferro e o níquel, em aços duros,  etc. \elemento{27}{58.9332}{Co}.}{co.bal.to}{0}
\verb{cobalto}{}{}{}{}{}{A cor azul"-escura dessa substância.}{co.bal.to}{0}
\verb{cobarde}{}{}{}{}{}{Var. de \textit{covarde}.}{co.bar.de}{0}
\verb{cobardia}{}{}{}{}{}{Var. de \textit{covardia}.}{co.bar.di.a}{0}
\verb{coberta}{é}{}{}{}{s.f.}{Tudo o que serve para cobrir, agasalhar ou abrigar.}{co.ber.ta}{0}
\verb{coberta}{é}{}{}{}{}{Cada um dos pavimentos de um navio.}{co.ber.ta}{0}
\verb{coberto}{é}{}{}{}{adj.}{Protegido.}{co.ber.to}{0}
\verb{coberto}{é}{}{}{}{}{Vestido; envolto; oculto.}{co.ber.to}{0}
\verb{coberto}{é}{}{}{}{}{Cheio, repleto.}{co.ber.to}{0}
\verb{coberto}{é}{}{}{}{}{Garantido, afiançado.}{co.ber.to}{0}
\verb{coberto}{é}{}{}{}{s.m.}{Lugar coberto; alpendre.}{co.ber.to}{0}
\verb{cobertor}{ô}{}{}{}{s.m.}{Peça de lã ou de algodão, encorpada e felpuda, que serve de agasalho e que se usa na cama por cima dos lençóis.}{co.ber.tor}{0}
\verb{cobertura}{}{}{}{}{s.f.}{Tudo o que serve para cobrir.}{co.ber.tu.ra}{0}
\verb{cobertura}{}{}{}{}{}{Proteção.}{co.ber.tu.ra}{0}
\verb{cobertura}{}{}{}{}{}{Apartamento construído sobre a laje.}{co.ber.tu.ra}{0}
\verb{cobertura}{}{}{}{}{}{Pagamento, garantia.}{co.ber.tu.ra}{0}
\verb{cobertura}{}{}{}{}{}{Registro de um fato pela imprensa.}{co.ber.tu.ra}{0}
\verb{cobiça}{}{}{}{}{s.f.}{Desejo ardente de possuir ou conseguir alguma coisa.}{co.bi.ça}{0}
\verb{cobiça}{}{}{}{}{}{Desejo imoderado de bens, riquezas ou honras; ambição.}{co.bi.ça}{0}
\verb{cobiçar}{}{}{}{}{v.t.}{Desejar ardentemente.}{co.bi.çar}{0}
\verb{cobiçar}{}{}{}{}{}{Ter cobiça; ambicionar.}{co.bi.çar}{\verboinum{3}}
\verb{cobiçoso}{ô}{}{"-osos ⟨ó⟩}{"-osa ⟨ó⟩}{adj.}{Diz"-se de indivíduo que tem cobiça, que é muito desejoso, ávido.}{co.bi.ço.so}{0}
\verb{cobra}{ó}{Zool.}{}{}{s.f.}{Nome comum aos répteis ofídios; serpente.}{co.bra}{0}
\verb{cobra}{ó}{Fig.}{}{}{}{Pessoa ruim, de má índole, astuta ou falsa.}{co.bra}{0}
\verb{cobra}{ó}{Pop.}{}{}{s.2g.}{Pessoa que conhece muito determinado ofício, arte ou assunto.}{co.bra}{0}
\verb{cobra"-coral}{ó}{Zool.}{cobras"-coral \textit{ou} cobras"-corais ⟨ó⟩}{}{s.f.}{Nome comum a várias espécies de répteis ofídios, venenosos ou não, de coloração mista, em geral vermelha, preta, amarela e branca; coral.}{co.bra"-co.ral}{0}
\verb{cobrador}{ô}{}{}{}{adj.}{Diz"-se de indivíduo que cobra ou faz cobranças; recebedor.}{co.bra.dor}{0}
\verb{cobrança}{}{}{}{}{s.f.}{Ato ou efeito de cobrar.}{co.bran.ça}{0}
\verb{cobrança}{}{}{}{}{}{Arrecadação, coleta de quantias.}{co.bran.ça}{0}
\verb{cobrar}{}{}{}{}{v.t.}{Fazer com que algo seja pago ou cumprido.}{co.brar}{0}
\verb{cobrar}{}{}{}{}{}{Exigir.}{co.brar}{\verboinum{1}}
\verb{cobre}{ó}{Quím.}{}{}{s.m.}{Elemento químico metálico,  de cor avermelhada, dúctil, maleável, bom condutor de calor e eletricidade, utilizado na indústria  de equipamentos elétricos, na fabricação de ligas e sob a forma de compostos.\elemento{29}{63.546}{Cu}.}{co.bre}{0}
\verb{cobreiro}{ê}{Pop.}{}{}{s.m.}{Erupção da pele que o povo crê ser                                                                 proveniente do contato com bichos venenosos; cobrelo.}{co.brei.ro}{0}
\verb{cobrelo}{ê}{Pop.}{}{}{s.m.}{Cobreiro.}{co.bre.lo}{0}
\verb{cobres}{ó}{}{}{}{s.m.pl.}{Dinheiro miúdo, ou em moedas.}{co.bres}{0}
\verb{cobrição}{}{}{"-ões}{}{s.f.}{Ato ou efeito de cobrir; cobrimento.}{co.bri.ção}{0}
\verb{cobrição}{}{}{"-ões}{}{}{Cópula de animais quadrúpedes.}{co.bri.ção}{0}
\verb{cobrir}{}{}{}{}{v.t.}{Colocar algo sobre uma pessoa ou coisa.}{co.brir}{0}
\verb{cobrir}{}{}{}{}{}{Defender pessoa ou coisa, impedindo que o inimigo a ataque.}{co.brir}{0}
\verb{cobrir}{}{}{}{}{}{Andar determinada distância; percorrer, vencer.}{co.brir}{0}
\verb{cobrir}{}{}{}{}{}{Dar muita coisa a alguém; cumular, encher.}{co.brir}{0}
\verb{cobrir}{}{}{}{}{}{Ser suficiente para pagar alguma coisa.}{co.brir}{0}
\verb{cobrir}{}{}{}{}{}{Dar garantia de pagar os prejuízos causados a pessoa ou coisa por algum acidente.}{co.brir}{\verboinum{31}}
\verb{coca}{ó}{Bot.}{}{}{s.f.}{Arbusto originário da cordilheira dos Andes, de caule forte, e de cujas folhas se extrai a cocaína.}{co.ca}{0}
\verb{coca}{ó}{}{}{}{}{Forma reduzida de \textit{cocaína}.}{co.ca}{0}
\verb{coça}{ó}{}{}{}{s.f.}{Ato ou efeito de coçar.}{co.ça}{0}
\verb{coça}{ó}{}{}{}{}{Ato de bater muito em alguém; sova, surra.}{co.ça}{0}
\verb{cocada}{}{Cul.}{}{}{s.f.}{Doce de coco ralado e de calda de açúcar, em ponto de rapadura, apresentado em losangos, quadrados ou forma circular.}{co.ca.da}{0}
\verb{cocada}{}{}{}{}{}{Pancada com a cabeça.}{co.ca.da}{0}
\verb{cocaína}{}{}{}{}{s.f.}{Droga extraída da coca.}{co.ca.í.na}{0}
\verb{cocainomania}{}{}{}{}{s.f.}{O vício de tomar ou aspirar cocaína.}{co.ca.i.no.ma.ni.a}{0}
\verb{cocainômano}{}{}{}{}{s.m.}{Indivíduo que tem o vício de tomar ou aspirar cocaína.}{co.ca.i.nô.ma.no}{0}
\verb{cocar}{}{}{}{}{s.m.}{Enfeite de penas usado pelos índios, em torno da cabeça.}{co.car}{0}
\verb{coçar}{}{}{}{}{v.t.}{Esfregar ou roçar com as unhas ou com um objeto áspero uma parte do corpo.}{co.çar}{0}
\verb{coçar}{}{}{}{}{v.pron.}{Lutar com dificuldades; sofrer trabalhos.}{co.çar}{\verboinum{3}}
\verb{cocção}{}{}{"-ões}{}{s.f.}{Ato ou efeito de cozer; cozimento.}{coc.ção}{0}
\verb{cocção}{}{}{"-ões}{}{}{Digestão dos alimentos no estômago.}{coc.ção}{0}
\verb{cóccix}{s}{Anat.}{}{}{s.m.}{Pequeno osso que constitui a parte terminal da coluna vertebral.}{cóc.cix}{0}
%\verb{}{}{}{}{}{}{}{}{0}
\verb{cócegas}{}{}{}{}{s.f.pl.}{Sensação particular que provoca riso, irritação ou movimentos convulsivos, causada por toques ou fricções leves e repetidas na pele.}{có.ce.gas}{0}
\verb{coceira}{ê}{}{}{}{s.f.}{Irritação da pele; comichão.}{co.cei.ra}{0}
\verb{coche}{ô}{}{}{}{s.m.}{Carruagem antiga e luxuosa.}{co.che}{0}
\verb{cocheira}{ê}{}{}{}{s.f.}{Abrigo de carruagens, de cavalos; estrebaria.}{co.chei.ra}{0}
\verb{cocheiro}{ê}{}{}{}{s.m.}{Indivíduo que conduz veículo puxado por cavalos.}{co.chei.ro}{0}
\verb{cochichar}{}{}{}{}{v.i.}{Falar em voz baixa; murmurar.}{co.chi.char}{0}
\verb{cochichar}{}{}{}{}{}{Mexericar, intrigar.}{co.chi.char}{\verboinum{1}}
\verb{cochicho}{}{}{}{}{s.m.}{Ato ou efeito de cochichar; sussurro, murmúrio.}{co.chi.cho}{0}
\verb{cochilar}{}{}{}{}{v.i.}{Dormir de leve; passar pelo sono.}{co.chi.lar}{0}
\verb{cochilar}{}{Fig.}{}{}{}{Distrair"-se ou descuidar"-se}{co.chi.lar}{\verboinum{1}}
\verb{cochilo}{}{}{}{}{s.m.}{Ato ou efeito de cochilar; sono leve.}{co.chi.lo}{0}
\verb{cochilo}{}{Fig.}{}{}{}{Descuido, distração.}{co.chi.lo}{0}
\verb{cochinilha}{}{}{}{}{}{Var. de \textit{cochonilha}.}{co.chi.ni.lha}{0}
\verb{cocho}{ô}{}{}{}{s.m.}{Recipiente feito geralmente de um tronco de árvore cavado, em que se costuma dar de comer e beber aos animais.      }{co.cho}{0}
\verb{cocho}{ô}{}{}{}{}{Tabuleiro para transportar a cal amassada.}{co.cho}{0}
\verb{cochonilha}{}{Zool.}{}{}{s.f.}{Inseto minúsculo, muito nocivo às plantas cultivadas por se alimentar da sua seiva, e do qual se extrai um corante vermelho de grande importância econômica.}{co.cho.ni.lha}{0}
\verb{cochonilha}{}{Por ext.}{}{}{}{O corante vermelho que se obtém da cochonilha, muito usado no preparo de fármacos, para colorir alimentos, tingir tecidos etc.}{co.cho.ni.lha}{0}
\verb{cociente}{}{}{}{}{}{Var. de \textit{quociente}.}{co.ci.en.te}{0}
\verb{cóclea}{}{Anat.}{}{}{s.f.}{Estrutura de forma espiral situada no ouvido interno.}{có.clea}{0}
\verb{coco}{ô}{}{}{}{s.m.}{O fruto do coqueiro, muito utilizado na culinária brasileira.}{co.co}{0}
\verb{cocó}{}{}{}{}{s.m.}{Coque.}{co.có}{0}
\verb{cocô}{}{Pop.}{}{}{s.m.}{Excremento.}{co.cô}{0}
\verb{cocó}{}{Pop.}{}{}{s.f.}{A galinha.}{co.có}{0}
\verb{coco}{ó}{Biol.}{}{}{s.m.}{Bactéria de forma arredondada. }{co.co}{0}
\verb{coco}{ô}{Pop.}{}{}{}{A cabeça.}{co.co}{0}
\verb{cócoras}{}{}{}{}{s.f.pl.}{Usado na locução \textit{de cócoras}: agachado, sentado sobre os calcanhares.}{có.co.ras}{0}
\verb{cocoricar}{}{}{}{}{v.i.}{Cantar (o galo).}{co.co.ri.car}{\verboinum{2}}
\verb{cocoricó}{}{}{}{}{s.m.}{Onomatopeia do canto do galo.}{co.co.ri.có}{0}
\verb{cocorocó}{}{}{}{}{}{Var. de \textit{cocoricó}.}{co.co.ro.có}{0}
\verb{cocorocô}{}{}{}{}{}{Var. de \textit{cocoricó}.}{co.co.ro.cô}{0}
\verb{cocorote}{ó}{}{}{}{s.m.}{Pancada na cabeça com o nó dos dedos.}{co.co.ro.te}{0}
\verb{cocoruto}{}{}{}{}{}{Var. de \textit{cocuruto}.}{co.co.ru.to}{0}
\verb{cocota}{ó}{Pop.}{}{}{s.f.}{Menina pré"-adolescente, bonita e vaidosa.}{co.co.ta}{0}
\verb{cocre}{ó}{}{}{}{s.m.}{Cascudo, cocorote.}{co.cre}{0}
\verb{cocuruto}{}{}{}{}{s.m.}{O ponto mais alto de alguma coisa; vértice, ápice.}{co.cu.ru.to}{0}
\verb{cocuruto}{}{}{}{}{}{Alto de uma colina.}{co.cu.ru.to}{0}
\verb{cocuruto}{}{}{}{}{}{O alto da cabeça.}{co.cu.ru.to}{0}
\verb{coda}{ó}{}{}{}{s.f.}{Parte final de um movimento musical.}{co.da}{0}
\verb{côdea}{}{}{}{}{s.f.}{Parte exterior dura; casca, crosta.}{cô.dea}{0}
\verb{côdea}{}{}{}{}{}{Sujeira solidificada na roupa.}{cô.dea}{0}
\verb{côdea}{}{}{}{}{}{Crosta de pão endurecida pelo cozimento.}{cô.dea}{0}
\verb{codeína}{}{}{}{}{s.f.}{Produto anestésico e analgésico, presente no ópio e preperado a partir da morfina.}{co.de.í.na}{0}
\verb{códex}{cs}{}{}{}{s.m.}{Códice.}{có.dex}{0}
%\verb{}{}{}{}{}{}{}{}{0}
\verb{códice}{}{}{}{}{s.m.}{Volume manuscrito em pergaminho.}{có.di.ce}{0}
\verb{códice}{}{}{}{}{}{Obra antiga de autor clássico.}{có.di.ce}{0}
\verb{codicilo}{}{Jur.}{}{}{s.m.}{Escrito particular de última vontade, pelo qual alguém estabelece disposições sobre o seu enterro, dá esmolas e lega móveis, roupas ou joias de seu uso particular e não muito valiosas, e nomeia ou substitui testamenteiros.}{co.di.ci.lo}{0}
\verb{codificação}{}{}{"-ões}{}{s.f.}{Ato ou efeito de codificar.}{co.di.fi.ca.ção}{0}
\verb{codificação}{}{}{"-ões}{}{}{Reunião de leis em código.}{co.di.fi.ca.ção}{0}
\verb{codificação}{}{}{"-ões}{}{}{Ação de colocar em código.}{co.di.fi.ca.ção}{0}
\verb{codificado}{}{}{}{}{adj.}{Que se codificou.}{co.di.fi.ca.do}{0}
\verb{codificado}{}{}{}{}{}{Em forma de código.}{co.di.fi.ca.do}{0}
\verb{codificador}{ô}{}{}{}{adj.}{Que codifica.}{co.di.fi.ca.dor}{0}
\verb{codificar}{}{}{}{}{v.t.}{Reduzir a código.}{co.di.fi.car}{0}
\verb{codificar}{}{}{}{}{}{Reunir em código.}{co.di.fi.car}{0}
\verb{codificar}{}{}{}{}{}{Transformar em códice; compilar.}{co.di.fi.car}{\verboinum{2}}
\verb{código}{}{}{}{}{s.m.}{Conjunto sistematizado de leis ou normas.}{có.di.go}{0}
\verb{código}{}{}{}{}{}{Sistema de sinais que contém uma mensagem.}{có.di.go}{0}
\verb{codinome}{}{}{}{}{s.m.}{Designação que serve para ocultar a identidade de alguém ou para nomear de maneira secreta um plano de ação, uma organização etc.}{co.di.no.me}{0}
\verb{codorna}{ó}{Zool.}{}{}{s.f.}{Nome comum a algumas aves de campo, cuja carne e ovos são muito apreciados; cordoniz.}{co.dor.na}{0}
\verb{codorniz}{}{Zool.}{}{}{s.f.}{Codorna.}{co.dor.niz}{0}
\verb{coedição}{}{}{coedições}{}{s.f.}{Edição de uma mesma obra realizada por vários editores.}{co.e.di.ção}{0}
\verb{coeditar}{}{}{}{}{v.t.}{Editar uma obra em colaboração ocm outras pessoas.}{co.e.di.tar}{\verboinum{1}}
\verb{coeducação}{}{}{coeducações}{}{s.m.}{Educação em conjunto, de meninos e meninas.}{co.e.du.ca.ção}{0}
\verb{coeficiente}{}{Mat.}{}{}{s.m.}{Número pelo qual um outro é multiplicado.}{co.e.fi.ci.en.te}{0}
\verb{coeficiente}{}{}{}{}{}{Propriedade que possibilita a avaliação, em termos numéricos, de um corpo ou fenômeno; grau, nível.}{co.e.fi.ci.en.te}{0}
\verb{coelheira}{ê}{}{}{}{s.f.}{Local destinado à criação de coelhos.}{co.e.lhei.ra}{0}
\verb{coelheiro}{ê}{}{}{}{adj.}{Diz"-se de cão que caça coelhos.}{co.e.lhei.ro}{0}
\verb{coelheiro}{ê}{}{}{}{s.m.}{Caçador de coelhos.}{co.e.lhei.ro}{0}
\verb{coelho}{ê}{Zool.}{}{}{s.m.}{Nome comum a vários mamíferos, de orelhas grandes, da família dos leporídeos, nativos da Europa, que cavam tocas no solo, onde criam seus filhotes, e dos quais se aproveitam a pele e a carne.}{co.e.lho}{0}
\verb{coentro}{}{Bot.}{}{}{s.m.}{Planta hortense, nativa da Ásia, cujas folhas são largamente usadas como condimento.}{co.en.tro}{0}
\verb{coerção}{}{}{"-ões}{}{s.f.}{Ato ou efeito de coagir; coação.}{co.er.ção}{0}
\verb{coerção}{}{}{"-ões}{}{}{Repressão, coibição.}{co.er.ção}{0}
\verb{coercitivo}{}{}{}{}{adj.}{Coercivo.}{co.er.ci.ti.vo}{0}
\verb{coercível}{}{}{"-eis}{}{adj.2g.}{Que se pode coagir.}{co.er.cí.vel}{0}
\verb{coercível}{}{}{"-eis}{}{}{Que pode ser contido em menor espaço.}{co.er.cí.vel}{0}
\verb{coercivo}{}{}{}{}{adj.}{Que pode exercer coerção.}{co.er.ci.vo}{0}
\verb{coerência}{}{}{}{}{s.f.}{Qualidade de coerente.}{co.e.rên.cia}{0}
\verb{coerência}{}{}{}{}{}{Ligação ou harmonia entre situações, acontecimentos ou ideias; relação harmônica; nexo, lógica.}{co.e.rên.cia}{0}
\verb{coerente}{}{}{}{}{adj.2g.}{Em que há coesão, que liga, que adere reciprocamente.}{co.e.ren.te}{0}
\verb{coerente}{}{}{}{}{}{Que procede com lógica.}{co.e.ren.te}{0}
\verb{coesão}{}{}{"-ões}{}{s.f.}{Unidade lógica, coerência de um pensamento, de uma obra.}{co.e.são}{0}
\verb{coesão}{}{Fig.}{"-ões}{}{}{Harmonia, concordância, união.}{co.e.são}{0}
\verb{coeso}{ê}{}{}{}{adj.}{Unido por coesão.}{co.e.so}{0}
\verb{coeso}{ê}{Fig.}{}{}{}{Que apresenta harmonia; ajustado.}{co.e.so}{0}
\verb{coestaduano}{}{}{}{}{adj.}{Diz"-se daquele que é do mesmo estado que outro; conterrâneo.}{co.es.ta.du.a.no}{0}
\verb{coetâneo}{}{}{}{}{adj.}{Que é da mesma idade que outro.}{co.e.tâ.neo}{0}
\verb{coetâneo}{}{}{}{}{}{Que viveu na mesma época; contemporâneo.}{co.e.tâ.neo}{0}
\verb{coeterno}{é}{}{}{}{adj.}{Que existe com outro desde sempre.}{co.e.ter.no}{0}
\verb{coevo}{é}{}{}{}{adj.}{Coetâneo.}{co.e.vo}{0}
\verb{coexistência}{z}{}{}{}{s.f.}{Qualidade ou condição de coexistente; existência simultânea.}{co.e.xis.tên.cia}{0}
\verb{coexistente}{z}{}{}{}{adj.2g.}{Que coexiste, que existe junto.}{co.e.xis.ten.te}{0}
\verb{coexistir}{z}{}{}{}{v.i.}{Existir ao mesmo tempo.}{co.e.xis.tir}{0}
\verb{coexistir}{z}{}{}{}{}{Conviver.}{co.e.xis.tir}{\verboinum{18}}
\verb{cofiar}{}{}{}{}{v.t.}{Afagar, alisar cabelo, barba ou bigode, passando a mão.}{co.fi.ar}{\verboinum{6}}
\verb{cofo}{ô}{}{}{}{s.m.}{Cesto bojudo e de boca estreita, usado para carregar iscas e petrechos de pesca, e para recolher o pescado; samburá.}{co.fo}{0}
\verb{cofre}{ó}{}{}{}{s.m.}{Móvel em forma de caixa para guardar objetos.}{co.fre}{0}
\verb{cofre}{ó}{}{}{}{}{Móvel metálico, com revestimento que procura torná"-lo inviolável e fechaduras de segredo contra roubo, destinado a guardar objetos de valor.}{co.fre}{0}
\verb{cogitação}{}{}{"-ões}{}{s.f.}{Ato ou efeito de cogitar.}{co.gi.ta.ção}{0}
\verb{cogitação}{}{}{"-ões}{}{}{Pensamento profundo; meditação.}{co.gi.ta.ção}{0}
\verb{cogitar}{}{}{}{}{v.t.}{Pensar sobre; considerar.}{co.gi.tar}{0}
\verb{cogitar}{}{}{}{}{v.i.}{Meditar.}{co.gi.tar}{\verboinum{1}}
\verb{cognato}{}{Jur.}{}{}{adj.}{Diz"-se de parente consanguíneo.}{cog.na.to}{0}
\verb{cognato}{}{Gram.}{}{}{}{Diz"-se de palavra que vem de uma mesma raiz que outra(s).}{cog.na.to}{0}
\verb{cognição}{}{}{"-ões}{}{s.f.}{Ato ou efeito de conhecer.}{cog.ni.ção}{0}
\verb{cognição}{}{}{"-ões}{}{}{Processo ou faculdade de adquirir um conhecimento.}{cog.ni.ção}{0}
\verb{cognitivo}{}{}{}{}{adj.}{Relativo à cognição ou ao conhecimento.}{cog.ni.ti.vo}{0}
\verb{cognome}{}{}{}{}{s.m.}{Apelido, alcunha.}{cog.no.me}{0}
\verb{cognominar}{}{}{}{}{v.t.}{Denominar alguém por cognome; apelidar, alcunhar.}{cog.no.mi.nar}{\verboinum{1}}
\verb{cognoscitivo}{}{}{}{}{adj.}{Que tem o poder ou a capacidade de conhecer.}{cog.nos.ci.ti.vo}{0}
\verb{cognoscível}{}{}{"-eis}{}{adj.2g.}{Que se pode conhecer; conhecível.}{cog.nos.cí.vel}{0}
\verb{cogote}{ó}{Pop.}{}{}{s.m.}{Região occipital; nuca, cachaço.}{co.go.te}{0}
\verb{cogula}{}{}{}{}{s.f.}{Túnica larga, sem mangas, usada por certos religiosos, como os beneditinos.}{co.gu.la}{0}
\verb{cogulo}{}{}{}{}{s.m.}{Porção que, numa medida, excede o conteúdo até as bordas.}{co.gu.lo}{0}
\verb{cogumelo}{é}{Bot.}{}{}{s.m.}{Nome comum a inúmeros fungos, parasitas ou saprófagos, desprovidos de clorofila,  com muitas espécies comestíveis e várias venenosas ou alucinógenas.}{co.gu.me.lo}{0}
\verb{cogumelo}{é}{}{}{}{}{O corpo de frutificação dos fungos, em forma de guarda"-chuva.}{co.gu.me.lo}{0}
\verb{co"-herdar}{}{}{}{}{v.t.}{Herdar algo com outro(s).}{co"-her.dar}{\verboinum{1}}
\verb{co"-herdeiro}{ê}{}{}{}{s.m.}{Indivíduo que herda com outra pessoa.}{co"-her.dei.ro}{0}
\verb{coibição}{}{}{"-ões}{}{s.f.}{Ato ou efeito de coibir.}{co.i.bi.ção}{0}
\verb{coibir}{}{}{}{}{v.t.}{Fazer cessar; impedir que continue; reprimir.}{co.i.bir}{0}
\verb{coibir}{}{}{}{}{}{Impedir de fazer alguma coisa.}{co.i.bir}{\verboinum{27}}
\verb{coice}{}{}{}{}{s.m.}{Pancada para trás dada especialmente por quadrúpedes equinos com as patas traseiras.}{coi.ce}{0}
\verb{coice}{}{}{}{}{}{Recuo de arma de fogo ao disparar.}{coi.ce}{0}
\verb{coice}{}{Fig.}{}{}{}{Agressão moral; grosseria, brutalidade.}{coi.ce}{0}
\verb{coifa}{}{}{}{}{s.f.}{Campânula que fica sobre fogões e que conduz a fumaça à chaminé.}{coi.fa}{0}
\verb{coifa}{}{}{}{}{}{Rede ou pano para envolver os cabelos.}{coi.fa}{0}
\verb{coima}{}{}{}{}{s.f.}{Multa, castigo, pena.}{coi.ma}{0}
\verb{coimbrão}{}{}{"-ãos}{coimbrã}{adj.}{Relativo a Coimbra, cidade de Portugal.}{co.im.brão}{0}
\verb{coimbrão}{}{}{"-ãos}{coimbrã}{s.m.}{Indivíduo natural ou habitante dessa cidade.}{co.im.brão}{0}
\verb{coincidência}{}{}{}{}{s.f.}{Ato ou efeito de coincidir.}{co.in.ci.dên.cia}{0}
\verb{coincidência}{}{}{}{}{}{Simultaneidade não planejada de dois ou mais eventos.}{co.in.ci.dên.cia}{0}
\verb{coincidência}{}{}{}{}{}{Identidade de duas ou mais coisas.}{co.in.ci.dên.cia}{0}
\verb{coincidente}{}{}{}{}{adj.2g.}{Que coincide; diz"-se de coisas que ocupam o mesmo espaço ou eventos que ocorrem ao mesmo tempo.}{co.in.ci.den.te}{0}
\verb{coincidir}{}{}{}{}{v.t.}{Ser idêntico.}{co.in.ci.dir}{0}
\verb{coincidir}{}{}{}{}{}{Acontecer ao mesmo tempo.}{co.in.ci.dir}{0}
\verb{coincidir}{}{}{}{}{v.i.}{Ajustar"-se, combinar, concordar.}{co.in.ci.dir}{\verboinum{18}}
\verb{coió}{}{Zool.}{}{}{s.m.}{Peixe"-voador.}{coi.ó}{0}
\verb{coió}{}{Desus.}{}{}{}{Assobio galante dirigido a uma mulher.}{coi.ó}{0}
\verb{coió}{}{}{}{}{adj.}{Tolo, paspalhão.}{coi.ó}{0}
\verb{coió}{}{Pop.}{}{}{}{Covarde, medroso.}{coi.ó}{0}
\verb{coió}{}{Bras.}{}{}{}{Diz"-se de namorado ridículo.}{coi.ó}{0}
\verb{coiote}{ó}{Zool.}{}{}{s.m.}{Mamífero semelhante ao lobo, de orelhas compridas, encontrado na América do Norte.}{coi.o.te}{0}
\verb{coiraça}{}{}{}{}{}{Var. de \textit{couraça}.}{coi.ra.ça}{0}
\verb{coirama}{}{}{}{}{}{Var. de \textit{courama}.}{coi.ra.ma}{0}
\verb{coirmão}{}{}{"-ãos}{}{adj.}{Diz"-se de quem é sócio, filiado, membro.}{co.ir.mão}{0}
\verb{coiro}{ô}{}{}{}{}{Var. de \textit{couro}.}{coi.ro}{0}
\verb{coisa}{}{}{}{}{s.f.}{Aquilo que existe ou pode existir, seja concreto ou abstrato.}{coi.sa}{0}
\verb{coisa}{}{}{}{}{}{Objeto inanimado.}{coi.sa}{0}
\verb{coisa}{}{}{}{}{}{Assunto, tema, matéria.}{coi.sa}{0}
\verb{coisa}{}{Pop.}{}{}{}{Palavra utilizada no lugar de qualquer substantivo. (\textit{Escrever bem não é uma coisa difícil.})}{coi.sa}{0}
\verb{coisa}{}{}{}{}{}{Bens, objetos pessoais. (Usa"-se geralmente no plural nesta acepção. \textit{Esqueci minhas coisas no restaurante.})}{coi.sa}{0}
\verb{coisa}{}{}{}{}{}{Ocupações, atividades, negócios. (Usa"-se geralmente no plural nesta acepção. \textit{Amanhã vou cuidar das minhas coisas.})}{coi.sa}{0}
\verb{coisa"-feita}{ê}{Pop.}{coisas"-feitas}{}{s.f.}{Feitiçaria, bruxaria.}{coi.sa"-fei.ta}{0}
\verb{coisa"-ruim}{}{Bras.}{coisas"-ruins}{}{s.m.}{O diabo.}{coi.sa"-ru.im}{0}
\verb{coitado}{}{}{}{}{s.m.}{Indivíduo digno de pena; mísero, desgraçado.}{coi.ta.do}{0}
\verb{coitado}{}{}{}{}{interj.}{Expressão que denota pena, condolência, dó.}{coi.ta.do}{0}
\verb{coité}{}{Bot.}{}{}{s.f.}{Árvore baixa de caule tortuoso, madeira dura e forte e cujo fruto é usado como vasilha ou cuia; cuieira, cabaceira.}{coi.té}{0}
\verb{coiteiro}{ê}{}{}{}{s.m.}{Indivíduo que dá asilo a outrem, especialmente bandidos.}{coi.tei.ro}{0}
\verb{coito}{ô}{}{}{}{s.m.}{Ato sexual.}{coi.to}{0}
\verb{coito}{ô}{}{}{}{}{Abrigo de refugiados; couto.}{coi.to}{0}
\verb{coivara}{}{}{}{}{s.f.}{Restos de gravetos ou pequenos galhos não atingidos pela queimada nas roças.}{coi.va.ra}{0}
\verb{cola}{ó}{}{}{}{s.f.}{Substância própria para fazer aderir papel, madeira; goma.}{co.la}{0}
\verb{cola}{ó}{}{}{}{s.f.}{Cauda, rabo.}{co.la}{0}
\verb{cola}{ó}{}{}{}{}{Em provas ou exames, cópia ilegal feita de livros, cadernos ou da prova de outrem.}{co.la}{0}
\verb{cola}{ó}{}{}{}{}{Rastro, encalço.}{co.la}{0}
\verb{colaboração}{}{}{"-ões}{}{s.f.}{Trabalho feito em conjunto com outra pessoa; cooperação.}{co.la.bo.ra.ção}{0}
\verb{colaboração}{}{}{"-ões}{}{}{Auxílio, contribuição, esmola.}{co.la.bo.ra.ção}{0}
\verb{colaboração}{}{}{"-ões}{}{}{Texto com o qual se participa de obra literária, jornal, revista etc. sem figurar como autor ou fazer parte do corpo de redatores.}{co.la.bo.ra.ção}{0}
\verb{colaboracionista}{}{}{}{}{adj.2g.}{Diz"-se de pessoa ou regime que, em um país ocupado por estrangeiros, apoia ou colabora com as forças inimigas.}{co.la.bo.ra.ci.o.nis.ta}{0}
\verb{colaborador}{ô}{}{}{}{adj.}{Que colabora.}{co.la.bo.ra.dor}{0}
\verb{colaborador}{ô}{}{}{}{s.m.}{Indivíduo que colabora.}{co.la.bo.ra.dor}{0}
\verb{colaborador}{ô}{}{}{}{}{Coautor.}{co.la.bo.ra.dor}{0}
\verb{colaborar}{}{}{}{}{v.t.}{Prestar colaboração; trabalhar em comum.}{co.la.bo.rar}{0}
\verb{colaborar}{}{}{}{}{}{Auxiliar, contribuir, fazer doação, dar esmola.}{co.la.bo.rar}{0}
\verb{colaborar}{}{}{}{}{}{Escrever para determinada publicação em regime de colaboração.}{co.la.bo.rar}{\verboinum{1}}
\verb{colação}{}{}{"-ões}{}{s.f.}{Concessão de título, grau ou direito.}{co.la.ção}{0}
\verb{colação}{}{}{"-ões}{}{}{Nomeação para benefício eclesiástico.}{co.la.ção}{0}
\verb{colação}{}{}{"-ões}{}{}{Comparação, cotejo, confronto.}{co.la.ção}{0}
\verb{colaço}{}{}{}{}{adj.}{Diz"-se de irmão de leite.}{co.la.ço}{0}
\verb{colagem}{}{}{"-ens}{}{s.f.}{Ato ou efeito de colar.}{co.la.gem}{0}
\verb{colagem}{}{Art.}{"-ens}{}{}{Técnica de composição em que fragmentos de materiais ou texturas variadas são superpostos ou colocados lado a lado formando uma nova imagem.}{co.la.gem}{0}
\verb{colágeno}{}{Biol.}{}{}{s.m.}{Substância que forma o tecido conjuntivo.}{co.lá.ge.no}{0}
\verb{colante}{}{}{}{}{adj.2g.}{Que cola.}{co.lan.te}{0}
\verb{colante}{}{}{}{}{}{Diz"-se de roupa muito justa que fica colada ao corpo.}{co.lan.te}{0}
\verb{colapso}{}{Med.}{}{}{s.m.}{Falência do funcionamento de órgão ou organismo.}{co.lap.so}{0}
\verb{colapso}{}{Fig.}{}{}{}{Paralisação, crise.}{co.lap.so}{0}
\verb{colar}{}{}{}{}{s.m.}{Adorno que se usa em volta do pescoço.}{co.lar}{0}
\verb{colar}{}{}{}{}{v.t.}{Unir uma coisa a outra utilizando cola.}{co.lar}{0}
\verb{colar}{}{}{}{}{v.t.}{Receber grau ou título em cerimônia própria.}{co.lar}{\verboinum{1}}
\verb{colar}{}{}{}{}{}{Copiar informações ilegalmente em prova ou exame.}{co.lar}{\verboinum{1}}
\verb{colarinho}{}{}{}{}{s.m.}{Gola de pano que adorna o decote nas camisas.}{co.la.ri.nho}{0}
\verb{colarinho}{}{Bras.}{}{}{}{Camada de espuma em um copo de cerveja ou chope.}{co.la.ri.nho}{0}
\verb{colarinho"-branco}{}{Bras.}{colarinhos"-brancos}{}{s.m.}{Designação comum dos vários profissionais que utilizam terno e gravata como trajes obrigatórios e convencionais de trabalho.}{co.la.ri.nho"-bran.co}{0}
\verb{colateral}{}{}{"-ais}{}{adj.2g.}{Que está lado a lado; paralelo.}{co.la.te.ral}{0}
\verb{colcha}{ô}{}{}{}{s.f.}{Coberta de cama, geralmente utilizada por cima do lençol.}{col.cha}{0}
\verb{colchão}{}{}{"-ões}{}{s.m.}{Grande almofada cheia de material macio e flexível sobre a qual se dorme, e geralmente utilizada sobre o estrado da cama.}{col.chão}{0}
\verb{colcheia}{ê}{Mús.}{}{}{s.f.}{Figura de tempo da notação musical que vale a metade de uma semínima.}{col.chei.a}{0}
\verb{colchete}{ê}{}{}{}{s.m.}{Tipo de gancho de metal para prender uma parte da roupa em outra.}{col.che.te}{0}
\verb{colchete}{ê}{Mat.}{}{}{}{Símbolo gráfico em forma de parênteses quadrados ("[]") que indica associação de elementos nas expressões.}{col.che.te}{0}
\verb{colchoaria}{}{}{}{}{s.f.}{Lugar onde se fabricam ou vendem colchões.}{col.cho.a.ri.a}{0}
\verb{colchoeiro}{ê}{}{}{}{s.m.}{Indivíduo que fabrica, vende ou conserta colchões.}{col.cho.ei.ro}{0}
\verb{colchonete}{é}{}{}{}{s.m.}{Colchão portátil, geralmente fino, leve e dobrável.}{col.cho.ne.te}{0}
\verb{coldre}{ô}{}{}{}{s.m.}{Estojo de couro para guardar revólver ou pistola.}{col.dre}{0}
\verb{colear}{}{}{}{}{v.i.}{Mover o colo (diz"-se da serpente).}{co.le.ar}{0}
\verb{colear}{}{}{}{}{}{Mover"-se em ziguezagues como a serpente.}{co.le.ar}{\verboinum{4}}
\verb{coleção}{}{}{"-ões}{}{s.f.}{Conjunto de objetos relacionados entre si e de algum interesse específico.}{co.le.ção}{0}
\verb{colecionador}{ô}{}{}{}{s.m.}{Indivíduo que coleciona.}{co.le.ci.o.na.dor}{0}
\verb{colecionar}{}{}{}{}{v.t.}{Fazer coleção; reunir, juntar.}{co.le.ci.o.nar}{\verboinum{1}}
\verb{colega}{é}{}{}{}{s.2g.}{Indivíduo que exerce a mesma função ou que se encontra em mesma situação em instituição, escola ou profissão.}{co.le.ga}{0}
\verb{colegiado}{}{}{}{}{s.m.}{Conjunto de dirigentes cujos membros têm poderes iguais.}{co.le.gi.a.do}{0}
\verb{colegial}{}{}{"-ais}{}{adj.2g.}{Relativo a colégio.}{co.le.gi.al}{0}
\verb{colegial}{}{}{"-ais}{}{s.2g.}{Aluno de colégio.}{co.le.gi.al}{0}
\verb{colegial}{}{Bras.}{"-ais}{}{}{O conjunto de anos letivos que formam o ensino médio.}{co.le.gi.al}{0}
\verb{colégio}{}{}{}{}{s.m.}{Estabelecimento de ensino fundamental ou médio.}{co.lé.gio}{0}
\verb{colégio}{}{}{}{}{}{Conjunto de indivíduos reunidos para fins eleitorais.}{co.lé.gio}{0}
\verb{colégio}{}{}{}{}{}{Reunião ou associação de colegas.}{co.lé.gio}{0}
\verb{coleguismo}{}{}{}{}{s.m.}{Espírito de solidariedade entre colegas ou semelhantes.}{co.le.guis.mo}{0}
\verb{coleio}{ê}{}{}{}{s.m.}{Ato de colear; movimento em ziguezague.}{co.lei.o}{0}
\verb{coleira}{ê}{}{}{}{s.f.}{Tipo de colar que se coloca em volta do pescoço de animais para limitar"-lhes os movimentos.}{co.lei.ra}{0}
\verb{coleiro}{ê}{Zool.}{}{}{s.m.}{Ave encontrada na América do Sul que tem penas brancas no pescoço, semelhante a uma coleira.}{co.lei.ro}{0}
\verb{colendo}{}{}{}{}{adj.}{Respeitável.}{co.len.do}{0}
\verb{coleóptero}{}{}{}{}{adj.}{Relativo aos coleópteros.}{co.le.óp.te.ro}{0}
\verb{coleóptero}{}{Zool.}{}{}{s.m.}{Espécime do coleópteros, ordem de insetos com quatro asas conhecidos popularmente como besouros.}{co.le.óp.te.ro}{0}
\verb{cólera}{}{}{}{}{s.f.}{Impulso violento; ira, raiva.}{có.le.ra}{0}
\verb{cólera}{}{Med.}{}{}{s.2g.}{Redução de \textit{cólera"-morbo}.}{có.le.ra}{0}
\verb{cólera"-morbo}{ó}{Med.}{}{}{s.2g.}{Doença infecciosa aguda e contagiosa, causado por um bacilo, e caracterizada por diarreia, cãibras e prostração, sendo transmitida especialmente pela água.}{có.le.ra"-mor.bo}{0}
\verb{colérico}{}{}{}{}{adj.}{Atacado de cólera; enfurecido, raivoso, irado.}{co.lé.ri.co}{0}
\verb{colesterol}{ó}{Quím.}{}{}{s.m.}{Substância encontrada na maior parte dos tecidos e gorduras animais, e cujos ésteres são considerados responsáveis pela arteriosclerose.}{co.les.te.rol}{0}
\verb{coleta}{é}{}{}{}{s.f.}{Ato ou efeito de coletar; colheita.}{co.le.ta}{0}
\verb{coleta}{é}{}{}{}{}{Quantia a ser paga como imposto.}{co.le.ta}{0}
\verb{coleta}{é}{}{}{}{}{Cota a ser paga para custear despesa comum ou obra de caridade.}{co.le.ta}{0}
\verb{coletânea}{}{}{}{}{s.f.}{Conjunto de trechos ou fragmentos selecionados de várias obras.}{co.le.tâ.nea}{0}
\verb{coletar}{}{}{}{}{v.t.}{Colher, colecionar, reunir.}{co.le.tar}{0}
\verb{coletar}{}{}{}{}{}{Arrecadar, tributar.}{co.le.tar}{\verboinum{1}}
\verb{colete}{ê}{}{}{}{s.m.}{Peça de vestuário sem mangas nem gola e que se veste por cima da camisa.}{co.le.te}{0}
\verb{coletiva}{}{}{}{}{s.f.}{Redução de \textit{entrevista coletiva}, entrevista concedida por figura pública a vários jornalistas de diferentes órgãos de comunicação e geralmente agendada com antecedência.}{co.le.ti.va}{0}
\verb{coletividade}{}{}{}{}{s.f.}{Qualidade de coletivo.}{co.le.ti.vi.da.de}{0}
\verb{coletividade}{}{}{}{}{}{Agremiação, agrupamento.}{co.le.ti.vi.da.de}{0}
\verb{coletividade}{}{}{}{}{}{Sociedade, comunidade.}{co.le.ti.vi.da.de}{0}
\verb{coletivismo}{}{}{}{}{s.m.}{Sistema econômico e social em que a exploração dos meios de produção se dá de maneira igual por todos os membros da sociedade.}{co.le.ti.vis.mo}{0}
\verb{coletivo}{}{}{}{}{adj.}{Que pertence a ou abrange várias pessoas ou coisas.}{co.le.ti.vo}{0}
\verb{coletivo}{}{Gram.}{}{}{}{Diz"-se de substantivo que, na forma do singular, designa um conjunto de coisas, animais ou pessoas.}{co.le.ti.vo}{0}
\verb{coletivo}{}{Bras.}{}{}{s.m.}{Veículo de transporte coletivo; ônibus.}{co.le.ti.vo}{0}
\verb{coletivo}{}{Esport.}{}{}{}{Treino que envolve toda a equipe.}{co.le.ti.vo}{0}
\verb{coletor}{ô}{}{}{}{adj.}{Que colige, reúne.}{co.le.tor}{0}
\verb{coletor}{ô}{}{}{}{s.m.}{Funcionário público que recebe os tributos do Estado.}{co.le.tor}{0}
\verb{coletoria}{}{}{}{}{s.f.}{Repartição pública onde se recebem impostos.}{co.le.to.ri.a}{0}
\verb{colheita}{ê}{}{}{}{s.f.}{Ato ou efeito de colher produtos agrícolas.}{co.lhei.ta}{0}
\verb{colheita}{ê}{}{}{}{}{O conjunto desses produtos colhidos num determinado período; safra.}{co.lhei.ta}{0}
\verb{colheitadeira}{ê}{}{}{}{s.f.}{Máquina usada na colheita, especialmente de cereais, que ceifa, classifica e ensaca.}{co.lhei.ta.dei.ra}{0}
\verb{colher}{é}{}{}{}{s.f.}{Utensílio de cozinha e mesa, composto de um cabo e de uma parte cavada. (\textit{Faço doces mexendo sempre com uma colher de pau.})}{co.lher}{0}
\verb{colher}{ê}{}{}{}{v.t.}{Tirar flores, frutas, folhas, de galhos ou troncos; apanhar.}{co.lher}{0}
\verb{colher}{ê}{}{}{}{}{Coletar, recolher.}{co.lher}{\verboinum{12}}
\verb{colherada}{}{}{}{}{s.f.}{Porção que uma colher pode conter.}{co.lhe.ra.da}{0}
\verb{colibacilo}{}{Biol.}{}{}{s.m.}{Bactéria presente na água, no leite e em certos alimentos, que vive como parasita no intestino e pode ter ação patogênica.}{co.li.ba.ci.lo}{0}
\verb{colibri}{}{Zool.}{}{}{s.m.}{Nome comum às aves pequeninas, de voo veloz, bico longo e fino, dotado de língua comprida, usada para sugar o néctar das flores; beija"-flor.}{co.li.bri}{0}
\verb{cólica}{}{Med.}{}{}{s.f.}{Dor intensa e aguda na cavidade abdominal.}{có.li.ca}{0}
\verb{colidir}{}{}{}{}{v.t.}{Fazer ir ou ir de encontro; chocar.}{co.li.dir}{0}
\verb{colidir}{}{}{}{}{v.i.}{Ser oposto; contradizer"-se.}{co.li.dir}{\verboinum{18}}
\verb{coliforme}{ó}{Biol.}{}{}{s.m.}{Bacilo encontrado nas fezes de seres humanos e animais e cuja presença na água é um indicador de poluição fecal e de contaminação bacteriana potencial. }{co.li.for.me}{0}
\verb{coligação}{}{}{"-ões}{}{s.f.}{Ato ou efeito de coligar; união, ligação.}{co.li.ga.ção}{0}
\verb{coligação}{}{}{"-ões}{}{}{Associação ou aliança de pessoas, partidos, entidades para um fim comum.}{co.li.ga.ção}{0}
\verb{coligar}{}{}{}{}{v.t.}{Tornar ligado; unir, juntar.}{co.li.gar}{0}
\verb{coligar}{}{}{}{}{}{Aproximar, assemelhar, identificar.}{co.li.gar}{\verboinum{5}}
\verb{coligir}{}{}{}{}{v.t.}{Reunir em coleção; juntar, colecionar.}{co.li.gir}{0}
\verb{coligir}{}{}{}{}{}{Concluir, inferir, deduzir.}{co.li.gir}{\verboinum{22}}
\verb{colimar}{}{}{}{}{v.t.}{Ter em vista; visar, objetivar, pretender. }{co.li.mar}{\verboinum{1}}
\verb{colina}{}{}{}{}{s.f.}{Pequena elevação de terreno de altitude menor que um monte; morro, outeiro.}{co.li.na}{0}
\verb{colírio}{}{}{}{}{s.m.}{Medicamento líquido que se aplica sobre a conjuntiva ocular para tratamento de conjuntivite ou para alívio dos olhos.}{co.lí.rio}{0}
\verb{colírio}{}{Fig.}{}{}{}{Pessoa de bela aparência, muito agradável à vista.}{co.lí.rio}{0}
\verb{colisão}{}{}{"-ões}{}{s.f.}{Ato ou efeito de colidir.}{co.li.são}{0}
\verb{colisão}{}{}{"-ões}{}{}{Choque de dois corpos; trombada, batida.}{co.li.são}{0}
\verb{colisão}{}{}{"-ões}{}{}{Conflito entre duas hipóteses; discordância, divergência.}{co.li.são}{0}
\verb{coliseu}{}{}{}{}{s.m.}{O maior anfiteatro romano, onde se realizavam combates entre gladiadores, jogos públicos e se jogavam os primeiros cristãos aos leões.}{co.li.seu}{0}
\verb{coliseu}{}{}{}{}{}{Local para competições de atletismo; estádio. }{co.li.seu}{0}
\verb{colite}{}{Med.}{}{}{s.f.}{Inflamação do cólon.}{co.li.te}{0}
\verb{collant}{}{}{}{}{s.m.}{Roupa de malha elástica inteiriça que adere ao corpo, utilizada principalmente em ginástica ou em balé.}{\textit{collant}}{0}
\verb{colmeia}{é}{}{}{}{}{Var. de \textit{colmeia}.}{col.mei.a}{0}
\verb{colmeia}{é}{}{}{}{s.f.}{Grupo de abelhas; enxame.}{col.mei.a}{0}
\verb{colmeia}{é}{}{}{}{}{Conjunto de cortiços ou outro tipo de habitação de abelhas.}{col.mei.a}{0}
\verb{colmeia}{é}{Fig.}{}{}{}{Acumulação de pessoas ou coisas.}{col.mei.a}{0}
\verb{colmo}{ô}{}{}{}{s.m.}{Haste ou palha comprida extraída de diversas gramíneas, usada para atar feixes, cobrir cabanas etc. }{col.mo}{0}
\verb{colmo}{ô}{}{}{}{}{A cobertura de cabana feita com essas palhas.}{col.mo}{0}
\verb{colo}{ó}{}{}{}{s.m.}{Parte do corpo entre o pescoço e os ombros.}{co.lo}{0}
\verb{colo}{ó}{}{}{}{}{Espaço formado pelo abdômen e as coxas quando o corpo está sentado; regaço. (\textit{A mãe estava com as duas crianças no colo.})}{co.lo}{0}
\verb{colo}{ó}{Anat.}{}{}{}{Parte estreita próxima à extremidade de um órgão. (\textit{O médico lhe fez um exame do colo do útero.})}{co.lo}{0}
\verb{colocação}{}{}{"-ões}{}{s.f.}{Ato ou efeito de colocar.}{co.lo.ca.ção}{0}
\verb{colocação}{}{}{"-ões}{}{}{Serviço, emprego.}{co.lo.ca.ção}{0}
\verb{colocação}{}{}{"-ões}{}{}{Lugar, posição numa classificação.}{co.lo.ca.ção}{0}
\verb{colocar}{}{}{}{}{v.t.}{Pôr, instalar.}{co.lo.car}{0}
\verb{colocar}{}{}{}{}{}{Empregar, dar emprego.}{co.lo.car}{\verboinum{2}}
\verb{colofão}{}{}{"-ões}{}{s.m.}{Inscrição no final de um livro onde o tipógrafo indica a data e o lugar de impressão, dentre outras informações da obra.}{co.lo.fão}{0}
\verb{coloide}{}{}{}{}{adj.}{Que apresenta o aspecto ou a aparência de cola.}{co.loi.de}{0}
\verb{coloide}{}{Quím.}{}{}{}{Mistura heterogênea de partículas dispersas que não se precipitam nem se cristalizam.}{co.loi.de}{0}
\verb{colom}{}{}{}{}{s.m.}{Unidade monetária e moeda da Costa Rica e de El Salvador.}{co.lom}{0}
\verb{colombiano}{}{}{}{}{adj.}{Relativo à Colômbia.}{co.lom.bi.a.no}{0}
\verb{colombiano}{}{}{}{}{s.m.}{Indivíduo natural ou habitante desse país.}{co.lom.bi.a.no}{0}
\verb{colombina}{}{}{}{}{s.f.}{Personagem feminina da antiga comédia italiana, participante de um triângulo amoroso entre o Arlequim e o Pierrô. (Nesta acepção usa"-se maiúscula.)}{co.lom.bi.na}{0}
\verb{colombina}{}{}{}{}{}{Fantasia de carnaval inspirada nessa personagem.}{co.lom.bi.na}{0}
\verb{cólon}{}{Anat.}{}{}{s.m.}{Parte do intestino grosso que vai do ceco ao reto.}{có.lon}{0}
\verb{colônia}{}{}{}{}{s.f.}{Grupo de imigrantes que se estabelecem em terra estranha. (\textit{São muitas as colônias orientais na cidade de São Paulo.})}{co.lô.nia}{0}
\verb{colônia}{}{}{}{}{}{Grupo de pessoas que vivem juntas.}{co.lô.nia}{0}
\verb{colônia}{}{}{}{}{}{Território sem independência política; domínio. (\textit{Na África ainda há várias colônias de países europeus.})}{co.lô.nia}{0}
\verb{colônia}{}{}{}{}{s.f.}{Perfume leve; água"-de"-colônia.}{co.lô.nia}{0}
\verb{colonial}{}{}{"-ais}{}{adj.2g.}{Relativo a colônia, aos colonos ou ao período de colonização.}{co.lo.ni.al}{0}
\verb{colonialismo}{}{}{}{}{s.m.}{Prática, sistema e processo histórico de estabelecimento de colônias.}{co.lo.ni.a.lis.mo}{0}
\verb{colonialismo}{}{}{}{}{}{Interesse pelas colônias ou especialização em colônias ou colonos. }{co.lo.ni.a.lis.mo}{0}
\verb{colonização}{}{}{"-ões}{}{s.f.}{Ato ou efeito de colonizar.}{co.lo.ni.za.ção}{0}
\verb{colonizador}{ô}{}{}{}{adj.}{Que estabelece e explora colônias.}{co.lo.ni.za.dor}{0}
\verb{colonizar}{}{}{}{}{v.t.}{Transformar em colônia.}{co.lo.ni.zar}{0}
\verb{colonizar}{}{}{}{}{}{Migrar para alguma região e estabelecer"-se como colono.}{co.lo.ni.zar}{0}
\verb{colonizar}{}{}{}{}{}{Alastrar"-se; propagar"-se; invadir.}{co.lo.ni.zar}{\verboinum{1}}
\verb{colono}{}{}{}{}{s.m.}{Membro ou habitante de uma colônia.}{co.lo.no}{0}
\verb{colono}{}{}{}{}{}{Lavrador que trabalha em terra de outrem por um salário.}{co.lo.no}{0}
\verb{coloquial}{}{}{"-ais}{}{adj.2g.}{Relativo a colóquio.}{co.lo.qui.al}{0}
\verb{coloquial}{}{Gram.}{"-ais}{}{}{Diz"-se da variante linguística utilizada em situações informais ou cotidianas.}{co.lo.qui.al}{0}
\verb{coloquialismo}{}{}{}{}{s.m.}{Expressão própria da linguagem cotidiana.}{co.lo.qui.a.lis.mo}{0}
\verb{colóquio}{}{}{}{}{s.m.}{Conversa entre duas ou mais pessoas; diálogo.}{co.ló.quio}{0}
\verb{colóquio}{}{}{}{}{}{Seminário, simpósio, encontro.}{co.ló.quio}{0}
\verb{coloração}{}{}{"-ões}{}{v.t.}{Ato ou efeito de colorir.}{co.lo.ra.ção}{0}
\verb{coloração}{}{}{"-ões}{}{}{Efeito ou sensação produzida pelas cores.}{co.lo.ra.ção}{0}
\verb{colorante}{}{}{}{}{adj.2g.}{Diz"-se de substância que colore; corante.}{co.lo.ran.te}{0}
\verb{colorar}{}{}{}{}{}{Var. de \textit{colorir}.}{co.lo.rar}{0}
\verb{colorau}{}{}{}{}{s.m.}{Pó vermelho extraído do pimentão seco ou do urucum, usado como condimento e como corante.}{co.lo.rau}{0}
\verb{colorido}{}{}{}{}{adj.}{Que tem mais de uma cor; matizado.}{co.lo.ri.do}{0}
\verb{colorido}{}{Fig.}{}{}{}{Vivo, brilhante, expressivo.}{co.lo.ri.do}{0}
\verb{colorido}{}{}{}{}{s.m.}{Combinação de cores; coloração.}{co.lo.ri.do}{0}
\verb{colorir}{}{}{}{}{v.t.}{Dar ou adquirir cores; matizar.}{co.lo.rir}{0}
\verb{colorir}{}{}{}{}{}{Tornar brilhante; pintar com cores vivas. }{co.lo.rir}{0}
\verb{colorir}{}{}{}{}{}{Disfarçar, camuflar, dissimular.}{co.lo.rir}{0}
\verb{colorir}{}{}{}{}{v.pron.}{Tornar"-se rubro; corar.}{co.lo.rir}{\verboinum{34}\verboirregular{\emph{def.} colorimos, coloris}}
\verb{colossal}{}{}{"-ais}{}{adj.2g.}{Grande como um colosso; enorme, gigantesco.}{co.los.sal}{0}
\verb{colosso}{ô}{}{}{}{s.m.}{Estátua de proporções gigantescas.}{co.los.so}{0}
\verb{colosso}{ô}{}{}{}{}{Pessoa ou animal corpulento e forte.}{co.los.so}{0}
\verb{colosso}{ô}{}{}{}{}{Coisa muito vantajosa, excelente.}{co.los.so}{0}
\verb{colosso}{ô}{}{}{}{}{Aquilo que provoca espanto, admiração.}{co.los.so}{0}
\verb{colostro}{ô}{}{}{}{s.m.}{Líquido amarelado secretado pelas glândulas mamárias da mulher e das fêmeas dos animais, logo após o parto.}{co.los.tro}{0}
\verb{colpite}{}{Med.}{}{}{s.f.}{Inflamação das paredes da vagina; vaginite.}{col.pi.te}{0}
\verb{colubrídeo}{}{Zool.}{}{}{s.f.}{Espécime dos colubrídeos, família de répteis ofídios encontrados em todos os continentes, e que compreende a maior parte das serpentes, geralmente não peçonhentas. }{co.lu.brí.deo}{0}
\verb{colubrídeo}{}{Zool.}{}{}{adj.}{Relativo aos colubrídeos.}{co.lu.brí.deo}{0}
\verb{columbário}{}{}{}{}{s.m.}{Nos cemitérios, edifício com nichos onde se conservam as cinzas funerárias.}{co.lum.bá.rio}{0}
\verb{columbino}{}{}{}{}{adj.}{Relativo a pombo.}{co.lum.bi.no}{0}
\verb{columbino}{}{Fig.}{}{}{}{Inocente, cândido, puro.}{co.lum.bi.no}{0}
\verb{coluna}{}{}{}{}{s.f.}{Peça vertical construída para sustentar abóbadas ou para adornar espaços; pilar.}{co.lu.na}{0}
\verb{coluna}{}{}{}{}{}{Cada uma das subdivisões verticais de periódicos mantidas por comentaristas ou cronistas.}{co.lu.na}{0}
\verb{coluna}{}{Anat.}{}{}{}{Estrutura anatômica de sustentação; espinha dorsal; forma reduzida da expressão \textit{coluna vertebral}.}{co.lu.na}{0}
\verb{colunata}{}{}{}{}{s.f.}{Série de colunas simetricamente dispostas numa fachada ou ao redor de um edifício.}{co.lu.na.ta}{0}
\verb{colunista}{}{}{}{}{s.2g.}{Pessoa que escreve coluna em jornal ou revista; cronista, comentarista.}{co.lu.nis.ta}{0}
\verb{com}{}{}{}{}{prep.}{Indica relação de companhia, instrumento, modo, meio, ligação.}{com}{0}
\verb{coma}{}{Med.}{}{}{s.2g.}{Estado em que o paciente perde a consciência, a mobilidade e a sensibilidade, porém, com manutenção das funções vegetativas. }{co.ma}{0}
\verb{comadre}{}{}{}{}{s.f.}{Madrinha de uma pessoa em relação aos pais desta.}{co.ma.dre}{0}
\verb{comadre}{}{}{}{}{}{Mãe do afilhado em relação aos padrinhos deste.}{co.ma.dre}{0}
\verb{comadre}{}{Pop.}{}{}{}{Mulher mexeriqueira, fofoqueira.}{co.ma.dre}{0}
\verb{comadre}{}{}{}{}{}{Urinol de forma achatada para uso de pessoas que não podem se levantar da cama.}{co.ma.dre}{0}
\verb{comanda}{}{}{}{}{s.f.}{A anotação dos pedidos do cliente feita pelo garçom.}{co.man.da}{0}
\verb{comanda}{}{}{}{}{}{Ficha de controle de consumo em restaurantes, bares, lanchonetes.}{co.man.da}{0}
\verb{comandante}{}{}{}{"-a}{adj.2g.}{Que comanda, dá ordens aos subordinados.}{co.man.dan.te}{0}
\verb{comandante}{}{}{}{"-a}{s.m.}{Indivíduo que exerce o comando de  aeronave, espaçonave ou navio.}{co.man.dan.te}{0}
\verb{comandante}{}{}{}{"-a}{}{Título conferido aos oficiais da Marinha.}{co.man.dan.te}{0}
\verb{comandar}{}{}{}{}{v.t.}{Exercer o comando sobre um grupo de pessoas; dirigir, governar. (\textit{Ele comandou muito bem o time de basquete.})}{co.man.dar}{0}
\verb{comandar}{}{}{}{}{v.i.}{Exercer comando, governar. (\textit{A partir de hoje, ele vai comandar e nós vamos obedecer.})}{co.man.dar}{\verboinum{1}}
\verb{comandita}{}{Jur.}{}{}{s.f.}{Sociedade comercial em que há sócios que só participam com capital, sem participar da gerência dos negócios.}{co.man.di.ta}{0}
\verb{comanditário}{}{Jur.}{}{}{adj.}{Que participa como sócio capitalista de uma comandita.}{co.man.di.tá.rio}{0}
\verb{comanditário}{}{}{}{}{s.m.}{O sócio capitalista de uma comandita.}{co.man.di.tá.rio}{0}
\verb{comando}{}{}{}{}{s.m.}{O ato de comandar. (\textit{Ele caiu porque perdeu o comando do cavalo.})}{co.man.do}{0}
\verb{comando}{}{}{}{}{}{Função exercida por aquele que comanda. (\textit{Vamos passar o comando das operações para o próximo gerente.})}{co.man.do}{0}
\verb{comando}{}{}{}{}{}{Pessoa ou grupo que comanda; dirigente. (\textit{O comando da empresa fica em outro país.})}{co.man.do}{0}
\verb{comando}{}{}{}{}{}{Grupo militar treinado para missões especiais. (\textit{Já colocamos um comando na selva para encontrar os fugitivos.})}{co.man.do}{0}
\verb{comando}{}{}{}{}{}{Dispositivo que se usa para movimentar máquinas diversas. (\textit{Deixei cair o comando do aparelho de televisão e agora não há quem o conserte.})}{co.man.do}{0}
\verb{comando}{}{Informát.}{}{}{}{Instrução dada para a execução de algum processamento. (\textit{Os comandos dos programas são muito complicados para quem não os conhece.})}{co.man.do}{0}
\verb{comarca}{}{Jur.}{}{}{s.f.}{Região em que atuam um ou mais juízes de direito.}{co.mar.ca}{0}
\verb{comatoso}{ô}{}{"-osos ⟨ó⟩}{"-osa ⟨ó⟩}{adj.}{Relativo ao coma.}{co.ma.to.so}{0}
\verb{combalido}{}{}{}{}{adj.}{Que se combaliu, enfraqueceu. (\textit{Foi com o aspecto triste e combalido que ela o contemplou da cabeça aos pés. })}{com.ba.li.do}{0}
\verb{combalir}{}{}{}{}{v.t.}{Enfraquecer, estragar. (\textit{O problema da cachaça é que ela combale o corpo inteiro.})}{com.ba.lir}{\verboinum{18}\verboirregular{\emph{def.} combalimos, combalis}}
\verb{combate}{}{}{}{}{s.m.}{A ação de combater; luta.}{com.ba.te}{0}
\verb{combatente}{}{}{}{}{adj.2g.}{Que combate; que luta.}{com.ba.ten.te}{0}
\verb{combatente}{}{}{}{}{s.2g.}{Indivíduo que combate.}{com.ba.ten.te}{0}
\verb{combater}{ê}{}{}{}{v.t.}{Lutar contra um inimigo. }{com.ba.ter}{0}
\verb{combater}{ê}{Fig.}{}{}{}{Resistir a uma força natural que foge ao controle. (\textit{Ele combateu corajosamente a doença.})}{com.ba.ter}{\verboinum{12}}
\verb{combativo}{}{}{}{}{adj.}{Que não se recusa a lutar; militante.}{com.ba.ti.vo}{0}
\verb{combinação}{}{}{"-ões}{}{s.f.}{Ato ou efeito de combinar.}{com.bi.na.ção}{0}
\verb{combinação}{}{}{"-ões}{}{}{Ajuste, acordo, pacto.}{com.bi.na.ção}{0}
\verb{combinação}{}{}{"-ões}{}{}{Peça do vestuário feminino que se usa sob o vestido.}{com.bi.na.ção}{0}
\verb{combinado}{}{}{}{}{adj.}{Que se combinou; que foi agrupado, unido, fundido.}{com.bi.na.do}{0}
\verb{combinado}{}{}{}{}{}{Que foi ajustado, acordado, convencionado.}{com.bi.na.do}{0}
\verb{combinado}{}{}{}{}{s.m.}{O que se combinou; acordo, ajuste. }{com.bi.na.do}{0}
\verb{combinado}{}{Esport.}{}{}{}{Equipe de jogadores de futebol provenientes de vários clubes; seleção.}{com.bi.na.do}{0}
\verb{combinar}{}{}{}{}{v.t.}{Agrupar ordenadamente, ordenar, ajustar. (\textit{Eu vou combinar as peças do jogo, para ver se há faltas.})}{com.bi.nar}{0}
\verb{combinar}{}{}{}{}{}{Planejar com outrem a realização de alguma atividade. (\textit{Combinamos de ir ao cinema amanhã.})}{com.bi.nar}{0}
\verb{combinar}{}{}{}{}{}{Harmonizar. (\textit{Ele sabe combinar as cores de sua roupa.})}{com.bi.nar}{\verboinum{1}}
\verb{combinatório}{}{}{}{}{adj.}{Relativo a combinação ou combinações.}{com.bi.na.tó.rio}{0}
\verb{comboiar}{}{}{}{}{v.t.}{Escoltar um comboio.}{com.boi.ar}{0}
\verb{comboiar}{}{Por ext.}{}{}{}{Transportar, conduzir, levar.}{com.boi.ar}{\verboinum{1}}
\verb{comboio}{ô}{}{}{}{s.m.}{Grupo de carros que se dirigem para o mesmo lugar.}{com.boi.o}{0}
\verb{comboio}{ô}{}{}{}{}{Conjunto de vagões puxados por locomotiva(s); trem, composição.}{com.boi.o}{0}
\verb{comburente}{}{}{}{}{adj.2g.}{Que queima muito; que alimenta a combustão.}{com.bu.ren.te}{0}
\verb{comburente}{}{}{}{}{s.m.}{Aquilo que queima muito ou que alimenta a combustão.}{com.bu.ren.te}{0}
\verb{combustão}{}{}{"-ões}{}{s.f.}{Ato ou efeito de queimar, arder, combustar.}{com.bus.tão}{0}
\verb{combustão}{}{}{"-ões}{}{}{Estado de um corpo que se queima, produzindo calor e luz.}{com.bus.tão}{0}
\verb{combustível}{}{}{"-eis}{}{adj.2g.}{Que arde, queima ou tem a propriedade de se consumir por combustão. }{com.bus.tí.vel}{0}
\verb{combustível}{}{}{"-eis}{}{s.m.}{Qualquer substância usada para produzir energia a partir da combustão.}{com.bus.tí.vel}{0}
\verb{começar}{}{}{}{}{v.t.}{Dar começo, iniciar. (\textit{Ele começou a conferência depois do intervalo.})}{co.me.çar}{0}
\verb{começar}{}{}{}{}{v.i.}{Ter início. (\textit{A conferência começou depois do intervalo.})}{co.me.çar}{\verboinum{3}}
\verb{começo}{ê}{}{}{}{s.m.}{Ato ou efeito de começar; início, origem, princípio, primórdio.}{co.me.ço}{0}
\verb{começo}{ê}{}{}{}{}{O primeiro instante da existência ou da execução de uma coisa.}{co.me.ço}{0}
\verb{comédia}{}{}{}{}{s.f.}{Narrativa teatral ou cinematográfica em que o propósito é satirizar o comportamento humano. (\textit{O filme que veremos é uma comédia das melhores.})}{co.mé.dia}{0}
\verb{comédia}{}{Fig.}{}{}{}{Situação ou experiência constrangedora sofrida por alguém que provoca o riso dos demais. (\textit{Sua vida sempre foi uma comédia.})}{co.mé.dia}{0}
\verb{comediante}{}{}{}{}{s.2g.}{Ator de comédia.}{co.me.di.an.te}{0}
\verb{comediante}{}{Fig.}{}{}{}{Pessoa fingida, dissimulada, hipócrita.}{co.me.di.an.te}{0}
\verb{comedido}{}{}{}{}{adj.}{Que tem ou demonstra comedimento; moderado, contido, prudente, sóbrio.}{co.me.di.do}{0}
\verb{comedimento}{}{}{}{}{s.m.}{Ato ou efeito de comedir; contenção, temperança, moderação.}{co.me.di.men.to}{0}
\verb{comediógrafo}{}{}{}{}{s.m.}{Pessoa que escreve comédias.}{co.me.di.ó.gra.fo}{0}
\verb{comedir}{}{}{}{}{v.t.}{Usar de moderação; conter, moderar.}{co.me.dir}{\verboinum{20}\verboirregular{\emph{def.} comedes, comede, comedimos, comedis, comedem}}
\verb{comedor}{ô}{}{}{}{adj.}{Que come.}{co.me.dor}{0}
\verb{comedor}{ô}{}{}{}{}{Que come muito; comilão, glutão.}{co.me.dor}{0}
\verb{comedor}{ô}{}{}{}{s.m.}{Pessoa que come muito; comilão, glutão.}{co.me.dor}{0}
\verb{comedouro}{ô}{}{}{}{s.m.}{Lugar onde comem os animais silvestres.}{co.me.dou.ro}{0}
\verb{comedouro}{ô}{}{}{}{}{Lugar ou recipiente onde comem animais. }{co.me.dou.ro}{0}
%\verb{}{}{}{}{}{}{}{}{0}
\verb{comemoração}{}{}{"-ões}{}{s.f.}{Ato ou efeito de comemorar.}{co.me.mo.ra.ção}{0}
\verb{comemoração}{}{Por ext.}{"-ões}{}{}{Qualquer festa ou reunião em que se comemore algo.}{co.me.mo.ra.ção}{0}
\verb{comemorar}{}{}{}{}{v.t.}{Trazer à memória; recordar, lembrar, memorar.}{co.me.mo.rar}{0}
\verb{comemorar}{}{}{}{}{}{Celebrar uma data ou acontecimento com festa; festejar.}{co.me.mo.rar}{\verboinum{1}}
\verb{comemorativo}{}{}{}{}{adj.}{Que comemora, faz lembrar um acontecimento; memorativo. }{co.me.mo.ra.ti.vo}{0}
\verb{comenda}{}{}{}{}{s.f.}{Distinção ou condecoração honorífica.}{co.men.da}{0}
\verb{comenda}{}{}{}{}{}{Insígnia ou divisa dessa distinção.}{co.men.da}{0}
\verb{comendador}{ô}{}{}{}{s.m.}{Pessoa que recebeu uma comenda. }{co.men.da.dor}{0}
\verb{comenos}{}{}{}{}{s.m.}{Momento, oportunidade, instante, ocasião.}{co.me.nos}{0}
\verb{comensal}{}{}{"-ais}{}{s.2g.}{Cada uma das pessoas que comem juntas à mesa.}{co.men.sal}{0}
\verb{comensal}{}{}{"-ais}{}{}{Pessoa que tem o hábito de comer na casa alheia.}{co.men.sal}{0}
\verb{comensurar}{}{}{}{}{v.t.}{Medir duas ou mais grandezas com a mesma unidade.}{co.men.su.rar}{0}
\verb{comensurar}{}{}{}{}{}{Medir.}{co.men.su.rar}{\verboinum{1}}
\verb{comensurável}{}{}{"-eis}{}{adj.2g.}{Que se pode medir, comensurar, determinar; mensurável.}{co.men.su.rá.vel}{0}
\verb{comentar}{}{}{}{}{v.t.}{Explicar, analisar. (\textit{O professor comenta os exercícios antes de corrigi"-los.})}{co.men.tar}{0}
\verb{comentar}{}{}{}{}{}{Criticar. (\textit{Não é bom ficar comentando a vida dos outros.})}{co.men.tar}{\verboinum{1}}
\verb{comentário}{}{}{}{}{s.m.}{Nota ou observação que se faz para esclarecer ou complementar um fato. (\textit{A cada momento, ele faz um comentário da leitura.})}{co.men.tá.rio}{0}
\verb{comentário}{}{}{}{}{}{Crítica negativa. (\textit{Os comentários da vida alheia nem sempre são bem"-vindos.})}{co.men.tá.rio}{0}
\verb{comentarista}{}{}{}{}{s.2g.}{Pessoa que faz comentários; colunista.}{co.men.ta.ris.ta}{0}
\verb{comer}{ê}{}{}{}{v.t.}{Mastigar e engolir alimentos; alimentar"-se. (\textit{Ele comeu toda a carne que lhe demos.})}{co.mer}{0}
\verb{comer}{ê}{Por ext.}{}{}{}{Introduzir objetos no estômago, pela boca, eventualmente, mastigando"-os e engolindo"-os. (\textit{O nenê comeu as capas de todos os livros.})}{co.mer}{0}
\verb{comer}{ê}{Fig.}{}{}{}{Desgastar. (\textit{O atrito está fazendo as peças comerem umas às outras.})}{co.mer}{0}
\verb{comer}{ê}{Fig.}{}{}{}{Destruir. (\textit{A bebida comeu todo o fígado daquele pobre homem.})}{co.mer}{0}
\verb{comer}{ê}{}{}{}{v.i.}{Ingerir alimentos, alimentar"-se. (\textit{Não precisa lhe dar nada, ele já comeu hoje de manhã.})}{co.mer}{\verboinum{12}}
\verb{comercial}{}{}{"-ais}{}{adj.2g.}{Relativo a comércio.}{co.mer.ci.al}{0}
\verb{comercial}{}{}{"-ais}{}{s.m.}{Anúncio de emissora de rádio ou televisão.}{co.mer.ci.al}{0}
\verb{comercializar}{}{}{}{}{v.t.}{Introduzir um produto novo no mercado. (\textit{Ele quer comercializar fibras de folha de bananeira.})}{co.mer.ci.a.li.zar}{\verboinum{1}}
\verb{comerciante}{}{}{}{}{adj.2g.}{Que trabalha no comércio.}{co.mer.ci.an.te}{0}
\verb{comerciante}{}{}{}{}{s.2g.}{Pessoa que faz comércio por profissão; mercador, negociante.}{co.mer.ci.an.te}{0}
\verb{comerciar}{}{}{}{}{v.i.}{Exercer o comércio, negociar. (\textit{Ele comercia há muito tempo nas ruas do centro.})}{co.mer.ci.ar}{\verboinum{1}}
\verb{comerciário}{}{Bras.}{}{}{s.m.}{Pessoa que trabalha no comércio.}{co.mer.ci.á.rio}{0}
\verb{comércio}{}{}{}{}{s.m.}{Compra e venda, ou troca, de mercadorias ou de valores. (\textit{O comércio de joias é lucrativo e perigoso.})}{co.mér.cio}{0}
\verb{comércio}{}{}{}{}{}{Conjunto de todos os indivíduos, ou seus estabelecimentos, que praticam a compra e a venda, ou a troca de mercadorias e de valores. (\textit{O comércio da cidade vai estar aberto nos feriados.})}{co.mér.cio}{0}
\verb{comércio}{}{}{}{}{}{Atividade econômica exercida entre grupos socialmente diferenciados. (\textit{O comércio no Alto Xingu já é um fato secular.})}{co.mér.cio}{0}
\verb{comestível}{}{}{"-eis}{}{adj.2g.}{Que se pode comer, que é bom ou próprio para comer; comível.}{co.mes.tí.vel}{0}
\verb{cometa}{ê}{Astron.}{}{}{s.m.}{Corpo celeste, geralmente provido de uma cauda luminosa de milhões de quilômetros, que gira ao redor do Sol.}{co.me.ta}{0}
\verb{cometer}{ê}{}{}{}{}{Atribuir. (\textit{Você deve fazer apenas as atividades que lhe cometem.})}{co.me.ter}{\verboinum{12}}
\verb{cometer}{ê}{}{}{}{v.t.}{Executar ou praticar alguma ação prejudicial. (\textit{Eles cometeram a injustiça de condenar um inocente.})}{co.me.ter}{0}
\verb{cometida}{}{}{}{}{s.f.}{Ataque, investida; acometimento.}{co.me.ti.da}{0}
\verb{cometimento}{}{}{}{}{s.m.}{Ato ou efeito de cometer; execução. (\textit{Pelo cometimento do roubo, foram severamente castigados.})}{co.me.ti.men.to}{0}
\verb{cometimento}{}{}{}{}{}{Delegação, encargo. (\textit{Um dos mais nobres cometimentos dos chefes de nação é o abandono do poder, quando lhe termina o mandato.})}{co.me.ti.men.to}{0}
\verb{comezaina}{}{}{}{}{s.f.}{Refeição abundante.}{co.me.zai.na}{0}
\verb{comezaina}{}{}{}{}{}{Reunião festiva para comer e beber; patuscada.}{co.me.zai.na}{0}
\verb{comezinho}{}{}{}{}{adj.}{Que é bom de comer.}{co.me.zi.nho}{0}
\verb{comezinho}{}{Fig.}{}{}{}{Que é fácil de entender; evidente, simples, óbvio.}{co.me.zi.nho}{0}
\verb{comichão}{}{}{"-ões}{}{s.f.}{Coceira, prurido.}{co.mi.chão}{0}
\verb{comichão}{}{Fig.}{"-ões}{}{}{Desejo intenso de alguma coisa.}{co.mi.chão}{0}
\verb{comichar}{}{}{}{}{v.t.}{Causar comichão, coceira. (\textit{Por um tomate que eu comesse, já me comichava o corpo todo.})}{co.mi.char}{\verboinum{1}}
\verb{comicidade}{}{}{}{}{s.f.}{Qualidade ou caráter do que é cômico. }{co.mi.ci.da.de}{0}
\verb{comício}{}{}{}{}{s.m.}{Reunião pública, geralmente em local aberto, realizada para discutir ou apresentar questões que interessam a todos. (\textit{O comício do prefeito, para esclarecer a venda dos bens municipais, não teve nenhum acidente.})}{co.mí.cio}{0}
\verb{cômico}{}{}{}{}{adj.}{Que diz respeito à comédia. (\textit{Ele é um autor cômico muito conhecido.})}{cô.mi.co}{0}
\verb{cômico}{}{Por ext.}{}{}{}{Que causa o riso em outros indivíduos. (\textit{Aquela situação cômica parecia não acabar mais.})}{cô.mi.co}{0}
\verb{cômico}{}{}{}{}{s.m.}{O autor ou o ator de comédias ou de espetáculos especialmente elaborados para fazer rir. (\textit{Alguns cômicos ficam famosos por suas interpretações maliciosas de personagens públicas.})}{cô.mi.co}{0}
\verb{comida}{}{}{}{}{}{Ação de comer; refeição.}{co.mi.da}{0}
\verb{comida}{}{}{}{}{s.f.}{Tudo aquilo que se come ou que serve para comer; alimento.}{co.mi.da}{0}
\verb{comigo}{}{}{}{}{pron.}{Com a pessoa que fala ou na companhia da pessoa que fala. (\textit{Ele andou comigo boa parte do passeio.})}{co.mi.go}{0}
\verb{comigo"-ninguém"-pode}{ó}{Bot.}{}{}{s.m.}{Planta de folhas longas, verdes, com manchas brancas, largamente cultivada como ornamental, apesar de ser venenosa.}{co.mi.go"-nin.guém"-po.de}{0}
\verb{comilança}{}{Bras.}{}{}{s.f.}{Ato de comer muito e com avidez.}{co.mi.lan.ça}{0}
\verb{comilão}{}{}{"-ões}{"-ona}{adj.}{Que come muito; glutão.}{co.mi.lão}{0}
\verb{comilão}{}{}{"-ões}{"-ona}{s.m.}{Pessoa que come muito.}{co.mi.lão}{0}
\verb{cominação}{}{}{"-ões}{}{s.f.}{Ato ou efeito de cominar.}{co.mi.na.ção}{0}
\verb{cominar}{}{}{}{}{v.t.}{Ameaçar com punição no caso de falta ou infração; impor, prescrever.}{co.mi.nar}{\verboinum{1}}
\verb{cominativo}{}{}{}{}{adj.}{Cominatório.}{co.mi.na.ti.vo}{0}
\verb{cominatório}{}{}{}{}{adj.}{Que envolve cominação; cominativo.}{co.mi.na.tó.rio}{0}
\verb{cominho}{}{Bot.}{}{}{s.m.}{Planta cujas sementes aromáticas são muito usadas como condimento.}{co.mi.nho}{0}
\verb{comiseração}{}{}{"-ões}{}{s.f.}{Sentimento de compaixão, de piedade pelo sofrimento alheio; dó, pena, piedade, misericórdia.}{co.mi.se.ra.ção}{0}
\verb{comiserar}{}{}{}{}{v.t.}{Inspirar comiseração, piedade, dó, compaixão. }{co.mi.se.rar}{\verboinum{1}}
\verb{comissão}{}{}{"-ões}{}{s.f.}{Trabalho que se manda alguém fazer; missão, incumbência, encargo.}{co.mis.são}{0}
\verb{comissão}{}{}{"-ões}{}{}{Grupo de pessoas encarregadas dessa missão.}{co.mis.são}{0}
\verb{comissão}{}{}{"-ões}{}{}{O pagamento por esse trabalho.}{co.mis.são}{0}
\verb{comissariado}{}{}{}{}{s.m.}{Cargo de comissário, ou a repartição onde ele exerce suas funções.}{co.mis.sa.ri.a.do}{0}
\verb{comissário}{}{}{}{}{s.m.}{Pessoa que exerce comissão.}{co.mis.sá.rio}{0}
\verb{comissário}{}{}{}{}{}{Autoridade policial.}{co.mis.sá.rio}{0}
\verb{comissionado}{}{}{}{}{adj.}{Que recebeu uma comissão.}{co.mis.si.o.na.do}{0}
\verb{comissionar}{}{}{}{}{v.t.}{Encarregar de comissões, ou expedir como comissário.}{co.mis.si.o.nar}{\verboinum{1}}
\verb{comissura}{}{}{}{}{s.f.}{Ponto de união entre duas partes; junção, juntura.}{co.mis.su.ra}{0}
\verb{comissura}{}{Anat.}{}{}{}{Nome comum às junções de algumas partes do corpo, como as pálpebras, os lábios etc.}{co.mis.su.ra}{0}
\verb{comitê}{}{}{}{}{s.m.}{Grupo de pessoas que se reúne para examinar determinado assunto; comissão.}{co.mi.tê}{0}
\verb{comitente}{}{}{}{}{adj.2g.}{Que dá comissão ou encargo a alguém.}{co.mi.ten.te}{0}
\verb{comitente}{}{}{}{}{s.2g.}{Essa pessoa.}{co.mi.ten.te}{0}
\verb{comitiva}{}{}{}{}{s.f.}{Grupo de pessoas que acompanha alguém ou alguma coisa; cortejo, séquito.}{co.mi.ti.va}{0}
\verb{comível}{}{}{"-eis}{}{adj.2g.}{Comestível.}{co.mí.vel}{0}
\verb{commodity}{}{Econ.}{}{}{s.f.}{Qualquer produto primário produzido em larga escala, e geralmente destinado ao comércio internacional, como o café, o ferro, o petróleo etc.}{\textit{commodity}}{0}
\verb{como}{}{}{}{}{adv.}{De que modo; de que maneira. (\textit{Como foi que ele fez isso?})}{co.mo}{0}
\verb{como}{}{}{}{}{conj.}{Da mesma maneira que. (\textit{Ele ficou como eu: deitado.})}{co.mo}{0}
\verb{como}{}{}{}{}{}{Conforme, segundo; assim que, porque, visto que. (\textit{Como ele disse que queria, eu trouxe todos os livros.})}{co.mo}{0}
\verb{comoção}{}{}{"-ões}{}{s.f.}{Ato ou efeito de comover; abalo, perturbação.}{co.mo.ção}{0}
\verb{cômoda}{}{}{}{}{s.f.}{Móvel com gavetas de alto a baixo, usado para guardar roupas.}{cô.mo.da}{0}
\verb{comodidade}{}{}{}{}{s.f.}{Qualidade do que é cômodo.}{co.mo.di.da.de}{0}
\verb{comodidade}{}{}{}{}{}{Conforto.}{co.mo.di.da.de}{0}
\verb{comodismo}{}{}{}{}{s.m.}{Qualidade ou atitude de comodista; preguiça.}{co.mo.dis.mo}{0}
\verb{comodista}{}{}{}{}{adj.2g.}{Que é muito afeito a comodidades ou coloca a comodidade acima de tudo; egoísta, folgado.}{co.mo.dis.ta}{0}
\verb{comodista}{}{}{}{}{s.2g.}{Essa pessoa.}{co.mo.dis.ta}{0}
\verb{cômodo}{}{}{}{}{adj.}{Que tem conforto; confortável.}{cô.mo.do}{0}
\verb{cômodo}{}{}{}{}{}{Que oferece vantagem; lucrativo, favorável.}{cô.mo.do}{0}
\verb{cômodo}{}{}{}{}{s.m.}{Qualquer acomodação de uma habitação; aposento, compartimento.}{cô.mo.do}{0}
\verb{comodoro}{ó}{}{}{}{s.m.}{Comandante de esquadra holandesa.}{co.mo.do.ro}{0}
\verb{comodoro}{ó}{}{}{}{}{Oficial da marinha inglesa e norte"-americana, entre capitão"-de"-mar"-e"-guerra e contra"-almirante.}{co.mo.do.ro}{0}
\verb{comodoro}{ó}{}{}{}{}{Título honorífico em associações náuticas.}{co.mo.do.ro}{0}
\verb{comovedor}{ô}{}{}{}{adj.}{Que comove; comovente.}{co.mo.ve.dor}{0}
\verb{comovente}{}{}{}{}{adj.2g.}{Que comove, enternece; comovedor, emocionante.  }{co.mo.ven.te}{0}
\verb{comover}{ê}{}{}{}{v.t.}{Causar comoção, sentimento de pena ou comiseração em alguém; emocionar, sensibilizar, enternecer.}{co.mo.ver}{\verboinum{12}}
\verb{compactar}{}{}{}{}{v.t.}{Tornar compacto; reduzir, comprimir.}{com.pac.tar}{\verboinum{1}}
\verb{compacto}{}{}{}{}{adj.}{Que tem as partes bem unidas entre si; maciço, comprimido.}{com.pac.to}{0}
\verb{compacto}{}{}{}{}{}{Denso, espesso.}{com.pac.to}{0}
\verb{compadecer}{ê}{}{}{}{v.pron.}{Sentir"-se triste com o sofrimento de alguém, tendo o desejo de ajudar; condoer"-se, comiserar"-se.}{com.pa.de.cer"-se}{\verboinum{15}}
\verb{compadecido}{}{}{}{}{adj.}{Que sente compaixão; condoído.}{com.pa.de.ci.do}{0}
\verb{compadre}{}{}{}{comadre}{s.m.}{O padrinho de uma pessoa, em relação aos pais do afilhado.}{com.pa.dre}{0}
\verb{compadre}{}{}{}{}{}{O pai do afilhado em relação aos padrinhos.}{com.pa.dre}{0}
\verb{compadre}{}{Pop.}{}{}{}{Tratamento carinhoso dado aos amigos e companheiros, principalmente no interior do país.}{com.pa.dre}{0}
\verb{compadrio}{}{}{}{}{s.m.}{Relação entre compadres.}{com.pa.dri.o}{0}
\verb{compadrio}{}{Por ext.}{}{}{}{Proteção excessiva, geralmente injusta; mancomunação.}{com.pa.dri.o}{0}
\verb{compaixão}{ch}{}{"-ões}{}{s.f.}{Sentimento de piedade diante do sofrimento ou da infelicidade alheia; dó, comiseração, pena, piedade.}{com.pai.xão}{0}
\verb{companheirismo}{}{}{}{}{s.m.}{Convívio amistoso e afetuoso, próprio de companheiros; camaradagem, coleguismo.}{com.pa.nhei.ris.mo}{0}
\verb{companheiro}{ê}{}{}{}{adj.}{Que acompanha.}{com.pa.nhei.ro}{0}
\verb{companheiro}{ê}{}{}{}{s.m.}{Pessoa que acompanha; colega, camarada.}{com.pa.nhei.ro}{0}
\verb{companheiro}{ê}{}{}{}{}{Esposo ou amante.}{com.pa.nhei.ro}{0}
\verb{companhia}{}{}{}{}{s.f.}{Ato ou efeito de acompanhar; acompanhamento.}{com.pa.nhi.a}{0}
\verb{companhia}{}{}{}{}{}{A pessoa que acompanha.}{com.pa.nhi.a}{0}
\verb{companhia}{}{}{}{}{}{Grupo de pessoas que se juntam com um determinado fim.}{com.pa.nhi.a}{0}
\verb{companhia}{}{}{}{}{}{Grupo de soldados liderado por um capitão.}{com.pa.nhi.a}{0}
\verb{comparação}{}{}{"-ões}{}{s.f.}{Ato ou efeito de comparar; confronto, cotejo.  }{com.pa.ra.ção}{0}
\verb{comparar}{}{}{}{}{v.t.}{Estabelecer relações entre coisas e pessoas para descobrir semelhanças e diferenças; cotejar, confrontar.}{com.pa.rar}{\verboinum{1}}
\verb{comparativo}{}{}{}{}{adj.}{Que serve para comparar, ou que emprega comparação.}{com.pa.ra.ti.vo}{0}
\verb{comparativo}{}{Gram.}{}{}{s.m.}{O grau de um adjetivo que exprime comparação.}{com.pa.ra.ti.vo}{0}
\verb{comparecer}{ê}{}{}{}{v.t.}{Aparecer, apresentar"-se num determinado lugar ou evento.}{com.pa.re.cer}{\verboinum{15}}
\verb{comparecimento}{}{}{}{}{s.m.}{Ato ou efeito de comparecer; presença de alguém num lugar.}{com.pa.re.ci.men.to}{0}
\verb{comparsa}{}{}{}{}{s.m.}{Companheiro, parceiro, cúmplice, sócio.}{com.par.sa}{0}
\verb{comparsa}{}{}{}{}{}{Pessoa que tem um papel ou importância secundária num negócio, numa atividade etc.}{com.par.sa}{0}
\verb{comparsa}{}{}{}{}{}{Figurante.}{com.par.sa}{0}
\verb{compartilhar}{}{}{}{}{v.t.}{Ter ou tomar parte em alguma coisa; participar, partilhar, compartir, dividir.}{com.par.ti.lhar}{\verboinum{1}}
\verb{compartimento}{}{}{}{}{s.m.}{Cada uma das divisões de uma área, de uma casa, de um móvel, de um veículo etc.}{com.par.ti.men.to}{0}
\verb{compartir}{}{}{}{}{v.t.}{Compartilhar.}{com.par.tir}{0}
\verb{compartir}{}{}{}{}{}{Dividir em compartimentos.}{com.par.tir}{\verboinum{18}}
\verb{compassado}{}{}{}{}{adj.}{Que é pausado, regulado, medido, ritmado, moderado. }{com.pas.sa.do}{0}
\verb{compassar}{}{}{}{}{v.t.}{Medir com compasso.}{com.pas.sar}{0}
\verb{compassar}{}{}{}{}{}{Moderar, regular, equilibrar.}{com.pas.sar}{0}
\verb{compassar}{}{}{}{}{}{Dar ritmo; cadenciar.}{com.pas.sar}{\verboinum{1}}
\verb{compassivo}{}{}{}{}{adj.}{Que tem ou demonstra compaixão, compadecimento; condolente.}{com.pas.si.vo}{0}
\verb{compasso}{}{}{}{}{s.m.}{Instrumento utilizado para traçar circunferências ou marcar medidas.}{com.pas.so}{0}
\verb{compasso}{}{Mús.}{}{}{}{Agrupamento dos tempos; ritmo, cadência.}{com.pas.so}{0}
\verb{compasso}{}{Fig.}{}{}{}{Movimento cadenciado.}{com.pas.so}{0}
\verb{compatibilidade}{}{}{}{}{s.f.}{Qualidade, condição ou estado do que é compatível.}{com.pa.ti.bi.li.da.de}{0}
\verb{compatibilizar}{}{}{}{}{v.t.}{Tornar compatível; harmonizar; conciliar.  }{com.pa.ti.bi.li.zar}{\verboinum{1}}
\verb{compatível}{}{}{"-eis}{}{adj.2g.}{Que pode coexistir, conciliar"-se ou concordar com outro; conciliável, harmonizável.}{com.pa.tí.vel}{0}
\verb{compatrício}{}{}{}{}{adj. e s.m.  }{Compatriota.}{com.pa.trí.cio}{0}
\verb{compatriota}{ó}{}{}{}{adj.2g.}{Que é da mesma pátria; compatrício, conterrâneo.}{com.pa.tri.o.ta}{0}
\verb{compatriota}{ó}{}{}{}{s.2g.}{Essa pessoa.}{com.pa.tri.o.ta}{0}
\verb{compelir}{}{}{}{}{v.t.}{Fazer mover"-se à força; impelir, empurrar.}{com.pe.lir}{0}
\verb{compelir}{}{}{}{}{}{Forçar alguém a fazer alguma coisa; obrigar, coagir, constranger.}{com.pe.lir}{\verboinum{29}\verboirregular{compelimos, compelis}}
\verb{compendiar}{}{}{}{}{v.t.}{Reduzir a compêndio; resumir; sintetizar, abreviar.}{com.pen.di.ar}{\verboinum{1}}
\verb{compêndio}{}{}{}{}{s.m.}{Resumo, síntese.}{com.pên.dio}{0}
\verb{compêndio}{}{}{}{}{}{Livro de textos.}{com.pên.dio}{0}
\verb{compenetrado}{}{}{}{}{adj.}{Que se compenetrou; convicto.}{com.pe.ne.tra.do}{0}
\verb{compenetrar}{}{}{}{}{v.t.}{Convencer, persuadir.}{com.pe.ne.trar}{0}
\verb{compenetrar}{}{Fig.}{}{}{v.pron.}{Inteirar"-se.}{com.pe.ne.trar}{\verboinum{1}}
\verb{compensação}{}{}{"-ões}{}{s.f.}{Ato ou efeito de compensar.}{com.pen.sa.ção}{0}
\verb{compensação}{}{Por ext.}{"-ões}{}{}{Aquilo que compensa; recompensa, benefício, vantagem.}{com.pen.sa.ção}{0}
\verb{compensação}{}{Econ.}{"-ões}{}{}{Sistema em que os débitos e os créditos de vários estabelecimentos bancários são confrontados para liquidar os saldos.}{com.pen.sa.ção}{0}
\verb{compensado}{}{}{}{}{adj.}{Que se compensou.}{com.pen.sa.do}{0}
\verb{compensado}{}{}{}{}{s.m.}{Chapa de madeira composta de várias lâminas prensadas com as fibras cruzadas para evitar deformações.}{com.pen.sa.do}{0}
\verb{compensador}{ô}{}{}{}{adj.}{Que traz benefício, vantagem, recompensa.}{com.pen.sa.dor}{0}
\verb{compensador}{ô}{}{}{}{}{Que supre a ausência ou a deficiência de algo.}{com.pen.sa.dor}{0}
\verb{compensar}{}{}{}{}{v.t.}{Reparar (um mal ou uma perda) com algo benéfico ou vantajoso; indenizar.}{com.pen.sar}{0}
\verb{compensar}{}{}{}{}{}{Remunerar por um trabalho ou benefício; recompensar.}{com.pen.sar}{0}
\verb{compensar}{}{}{}{}{}{Suprir a ausência ou a deficiência de algo; contrabalançar, equilibrar.}{com.pen.sar}{0}
\verb{compensar}{}{}{}{}{v.i.}{Valer a pena, valer um sacrifício.}{com.pen.sar}{\verboinum{1}}
\verb{compensatório}{}{}{}{}{adj.}{Relativo a ou que representa compensação.}{com.pen.sa.tó.rio}{0}
\verb{competência}{}{}{}{}{s.f.}{Habilidade, conhecimento, experiência, aptidão.}{com.pe.tên.cia}{0}
\verb{competência}{}{}{}{}{}{Jurisdição, autoridade, alçada.}{com.pe.tên.cia}{0}
\verb{competência}{}{}{}{}{}{Qualidade de competente.}{com.pe.tên.cia}{0}
\verb{competente}{}{}{}{}{adj.2g.}{Que tem habilidade ou aptidão; hábil, idôneo.}{com.pe.ten.te}{0}
\verb{competente}{}{}{}{}{}{Que tem aptidão legal (para uma questão específica).}{com.pe.ten.te}{0}
\verb{competente}{}{}{}{}{}{Conveniente, devido, adequado.}{com.pe.ten.te}{0}
\verb{competição}{}{}{"-ões}{}{s.f.}{Ato ou efeito de competir.}{com.pe.ti.ção}{0}
\verb{competição}{}{}{"-ões}{}{}{Disputa, torneio.}{com.pe.ti.ção}{0}
\verb{competidor}{ô}{}{}{}{adj.}{Que toma parte em competição.}{com.pe.ti.dor}{0}
\verb{competidor}{ô}{}{}{}{}{Que tende a rivalizar"-se com outros.}{com.pe.ti.dor}{0}
\verb{competir}{}{}{}{}{v.t.}{Tomar parte em competição; concorrer, rivalizar.}{com.pe.tir}{0}
\verb{competir}{}{}{}{}{}{Ser da atribuição de; caber a; tocar a.}{com.pe.tir}{\verboinum{29}}
\verb{compilação}{}{}{"-ões}{}{s.f.}{Ato ou efeito de compilar.}{com.pi.la.ção}{0}
\verb{compilação}{}{}{"-ões}{}{}{Conjunto de textos ou documentos reunidos.}{com.pi.la.ção}{0}
\verb{compilador}{ô}{}{}{}{adj.}{Que compila.}{com.pi.la.dor}{0}
\verb{compilar}{}{}{}{}{v.t.}{Reunir em uma obra textos ou documentos diversos.}{com.pi.lar}{0}
\verb{compilar}{}{Informát.}{}{}{}{Converter (um programa escrito em linguagem de alto nível) para linguagem de máquina, tornando a sua execução muito mais rápida.}{com.pi.lar}{\verboinum{1}}
\verb{complacência}{}{}{}{}{s.f.}{Qualidade de complacente.}{com.pla.cên.cia}{0}
\verb{complacente}{}{}{}{}{adj.2g.}{Tolerante, condescendente, benevolente.}{com.pla.cen.te}{0}
\verb{compleição}{}{}{"-ões}{}{s.f.}{Constituição física.}{com.plei.ção}{0}
\verb{compleição}{}{}{"-ões}{}{}{Constituição psicológica; temperamento, feitio, índole.}{com.plei.ção}{0}
\verb{complementar}{}{}{}{}{adj.2g.}{Que complementa.}{com.ple.men.tar}{0}
\verb{complementar}{}{Geom.}{}{}{}{Diz"-se de ângulo que, somado com outro, produz um ângulo reto.}{com.ple.men.tar}{0}
\verb{complementar}{}{}{}{}{v.t.}{Dar complemento; completar, concluir.}{com.ple.men.tar}{\verboinum{1}}
\verb{complemento}{}{}{}{}{s.m.}{Elemento que se acrescenta a um todo para completá"-lo, aperfeiçoá"-lo, enfeitá"-lo.}{com.ple.men.to}{0}
\verb{completar}{}{}{}{}{v.t.}{Acrescentar (algo) para que fique cheio, concluído, perfeito.}{com.ple.tar}{0}
\verb{completar}{}{}{}{}{}{Atingir um número ou uma idade.}{com.ple.tar}{\verboinum{1}}
\verb{completo}{é}{}{}{}{adj.}{Em que não falta nada.}{com.ple.to}{0}
\verb{completo}{é}{}{}{}{}{Preenchido, cheio, lotado.}{com.ple.to}{0}
\verb{completo}{é}{}{}{}{}{Perfeito, inteiro.}{com.ple.to}{0}
\verb{completo}{é}{}{}{}{}{Que preenche os requisitos necessários.}{com.ple.to}{0}
\verb{complexidade}{cs}{}{}{}{s.f.}{Qualidade de complexo.}{com.ple.xi.da.de}{0}
\verb{complexo}{écs}{}{}{}{adj.}{Que se compõe de várias partes, características, ideias, diferentes entre si.}{com.ple.xo}{0}
\verb{complexo}{écs}{}{}{}{}{Que pode ser observado sob diversos aspectos.}{com.ple.xo}{0}
\verb{complexo}{écs}{}{}{}{}{Que não é simples; complicado.}{com.ple.xo}{0}
\verb{complexo}{écs}{}{}{}{s.m.}{Conjunto de elementos ligados por algo em comum e que formam um todo.}{com.ple.xo}{0}
\verb{complexo}{écs}{Psicol.}{}{}{}{Conjunto de imagens e desejos recalcados.}{com.ple.xo}{0}
\verb{complicação}{}{}{"-ões}{}{s.f.}{Ato ou efeito de complicar.}{com.pli.ca.ção}{0}
\verb{complicação}{}{}{"-ões}{}{}{Qualidade de complicado.}{com.pli.ca.ção}{0}
\verb{complicação}{}{}{"-ões}{}{}{Coisa ou situação complicada; dificuldade.}{com.pli.ca.ção}{0}
\verb{complicado}{}{}{}{}{adj.}{Que não é simples ou fácil de resolver; intrincado, confuso, complexo, difícil.}{com.pli.ca.do}{0}
\verb{complicador}{ô}{}{}{}{adj.}{Que complica, que torna complicado.}{com.pli.ca.dor}{0}
\verb{complicador}{ô}{}{}{}{s.m.}{Fator de complicação; coisa que complica.}{com.pli.ca.dor}{0}
\verb{complicar}{}{}{}{}{v.t.}{Tornar complexo, confuso; dificultar, embaraçar, confundir.}{com.pli.car}{0}
\verb{complicar}{}{}{}{}{}{Tornar mais grave (estado de saúde); agravar.}{com.pli.car}{\verboinum{2}}
\verb{complô}{}{}{}{}{s.m.}{Plano secreto contra uma pessoa ou uma organização; conspiração.}{com.plô}{0}
\verb{componente}{}{}{}{}{s.2g.}{Elemento que faz parte da composição de algo.}{com.po.nen.te}{0}
\verb{compor}{}{}{}{}{v.t.}{Formar um todo a partir de várias partes.}{com.por}{0}
\verb{compor}{}{}{}{}{}{Tomar parte na composição de.}{com.por}{0}
\verb{compor}{}{}{}{}{}{Criar, inventar, escrever, conceber, produzir.}{com.por}{0}
\verb{compor}{}{}{}{}{}{Adornar.}{com.por}{0}
\verb{compor}{}{}{}{}{}{Colocar em ordem; arranjar.}{com.por}{0}
\verb{compor}{}{}{}{}{v.pron.}{Ser composto de; constituir"-se de.}{com.por}{\verboinum{60}}
\verb{comporta}{ó}{}{}{}{s.f.}{Porta móvel operada mecanicamente para conter e controlar as águas de uma represa; dique, açude.}{com.por.ta}{0}
\verb{comportado}{}{}{}{}{adj.}{Que tem bom comportamento.}{com.por.ta.do}{0}
\verb{comportamento}{}{}{}{}{s.m.}{Conjunto de ações, procedimentos e hábitos observáveis de um indivíduo ou um grupo social.}{com.por.ta.men.to}{0}
\verb{comportamento}{}{}{}{}{}{Conduta, maneira de comportar"-se.}{com.por.ta.men.to}{0}
\verb{comportar}{}{}{}{}{v.t.}{Ser capaz de conter.}{com.por.tar}{0}
\verb{comportar}{}{}{}{}{}{Suportar, sofrer.}{com.por.tar}{0}
\verb{comportar}{}{}{}{}{v.pron.}{Proceder, agir, portar"-se.}{com.por.tar}{0}
\verb{comportar}{}{}{}{}{}{Portar"-se bem.}{com.por.tar}{\verboinum{1}}
\verb{composição}{}{}{"-ões}{}{s.f.}{Ato ou efeito de compor.}{com.po.si.ção}{0}
\verb{composição}{}{}{"-ões}{}{}{Conjunto dos elementos que compõem algo; constituição.}{com.po.si.ção}{0}
\verb{composição}{}{Quím.}{"-ões}{}{}{Proporção dos elementos que formam um composto.}{com.po.si.ção}{0}
\verb{composição}{}{}{"-ões}{}{}{Produção artística (literária, musical, plástica) ou científica.}{com.po.si.ção}{0}
\verb{composição}{}{Bras.}{"-ões}{}{}{O conjunto formado por todos os vagões de um trem.}{com.po.si.ção}{0}
\verb{composição}{}{}{"-ões}{}{}{Disposição, combinação, arranjo.}{com.po.si.ção}{0}
\verb{composição}{}{Gram.}{"-ões}{}{}{Processo de formação de palavras a partir de dois ou mais radicais.}{com.po.si.ção}{0}
\verb{compositor}{ô}{}{}{}{s.m.}{Indivíduo que compõe músicas.}{com.po.si.tor}{0}
\verb{composto}{ô}{}{"-s ⟨ó⟩}{"-a ⟨ó⟩}{adj.}{Que é formado por mais de um elemento.}{com.pos.to}{0}
\verb{composto}{ô}{Fig.}{"-s ⟨ó⟩}{"-a ⟨ó⟩}{}{Sério, grave, circunspecto.}{com.pos.to}{0}
\verb{composto}{ô}{Gram.}{"-s ⟨ó⟩}{"-a ⟨ó⟩}{}{Diz"-se de período formado por mais de uma oração, de constituinte (sujeito, objeto) formado por mais de um elemento ou de tempo verbal formado por mais de uma forma verbal.}{com.pos.to}{0}
\verb{composto}{ô}{}{"-s ⟨ó⟩}{"-a ⟨ó⟩}{s.m.}{A substância composta.}{com.pos.to}{0}
\verb{composto}{ô}{Gram.}{"-s ⟨ó⟩}{"-a ⟨ó⟩}{}{Vocábulo formado por processo de composição.}{com.pos.to}{0}
\verb{compostura}{}{}{}{}{s.f.}{Comedimento, sobriedade ou correção na maneira de se comportar; circunspecção.}{com.pos.tu.ra}{0}
\verb{compota}{ó}{}{}{}{s.f.}{Doce de frutas aos pedaços cozidas em calda de açúcar.}{com.po.ta}{0}
\verb{compoteira}{ê}{}{}{}{s.f.}{Recipiente com tampa próprio para guardar compotas.}{com.po.tei.ra}{0}
\verb{compra}{}{}{}{}{}{A mercadoria ou o conjunto de mercadorias compradas.}{com.pra}{0}
\verb{compra}{}{}{}{}{s.f.}{Ato ou efeito de comprar.}{com.pra}{0}
\verb{comprador}{ô}{}{}{}{adj.}{Diz"-se da parte que, em uma relação comercial, adquire um bem, pagando certa quantia à outra parte.}{com.pra.dor}{0}
\verb{comprar}{}{}{}{}{v.t.}{Adquirir a propriedade ou o uso (de algo) mediante pagamento.}{com.prar}{0}
\verb{comprar}{}{}{}{}{}{Subornar.}{com.prar}{0}
\verb{comprar}{}{}{}{}{}{Em jogos de cartas, pegar cartas do baralho ou da mesa.}{com.prar}{\verboinum{1}}
\verb{comprazer}{ê}{}{}{}{v.t.}{Fazer a vontade ou o gosto de.}{com.pra.zer}{0}
\verb{comprazer}{ê}{}{}{}{}{Condescender.}{com.pra.zer}{\verboinum{14}}
\verb{compreender}{ê}{}{}{}{v.t.}{Conter, abranger, incluir.}{com.pre.en.der}{0}
\verb{compreender}{ê}{}{}{}{}{Entender, perceber.}{com.pre.en.der}{0}
\verb{compreender}{ê}{}{}{}{}{Aceitar as razões ou os julgamentos de outrem.}{com.pre.en.der}{0}
\verb{compreender}{ê}{}{}{}{v.pron.}{Estar compreendido; encerrar"-se.}{com.pre.en.der}{\verboinum{12}}
\verb{compreensão}{}{}{"-ões}{}{s.f.}{Ato ou efeito de compreender.}{com.pre.en.são}{0}
\verb{compreensão}{}{}{"-ões}{}{}{Condescendência, indulgência.}{com.pre.en.são}{0}
\verb{compreensão}{}{}{"-ões}{}{}{Capacidade ou determinação de compreender.}{com.pre.en.são}{0}
\verb{compreensão}{}{}{"-ões}{}{}{Percepção.}{com.pre.en.são}{0}
\verb{compreensível}{}{}{"-eis}{}{adj.2g.}{Que se pode compreender.}{com.pre.en.sí.vel}{0}
\verb{compreensivo}{}{}{}{}{adj.}{Que demonstra compreensão.}{com.pre.en.si.vo}{0}
\verb{compressa}{é}{}{}{}{s.f.}{Pedaço de pano embebido em líquido que é aplicado sobre partes do corpo para aliviar dores, feridas ou febre.}{com.pres.sa}{0}
\verb{compressão}{}{}{"-ões}{}{s.f.}{Ato ou efeito de comprimir.}{com.pres.são}{0}
\verb{compressibilidade}{}{}{}{}{s.f.}{Qualidade de compressível.}{com.pres.si.bi.li.da.de}{0}
\verb{compressível}{}{}{"-eis}{}{adj.2g.}{Que se pode comprimir.}{com.pres.sí.vel}{0}
\verb{compressor}{ô}{}{}{}{adj.}{Que comprime.}{com.pres.sor}{0}
\verb{compressor}{ô}{}{}{}{s.m.}{Máquina pesada para comprimir o terreno, usada para construir ruas, estradas.}{com.pres.sor}{0}
\verb{compressor}{ô}{}{}{}{}{Máquina dotada de uma bomba de ar, para produzir ar sob pressão.}{com.pres.sor}{0}
\verb{comprido}{}{}{}{}{adj.}{Extenso, longo.}{com.pri.do}{0}
\verb{comprimento}{}{}{}{}{s.m.}{Distância medida em sentido longitudinal entre duas extremidades.}{com.pri.men.to}{0}
\verb{comprimento}{}{Pop.}{}{}{}{Duração.}{com.pri.men.to}{0}
\verb{comprimido}{}{}{}{}{adj.}{Que se comprimiu.}{com.pri.mi.do}{0}
\verb{comprimido}{}{}{}{}{s.m.}{Medicamento em forma de pastilha para ser ingerido por via oral.}{com.pri.mi.do}{0}
\verb{comprimir}{}{}{}{}{v.t.}{Exercer compressão, pressão; apertar.}{com.pri.mir}{0}
\verb{comprimir}{}{}{}{}{}{Reduzir o volume de.}{com.pri.mir}{0}
\verb{comprimir}{}{Fig.}{}{}{}{Oprimir, reprimir, confranger.}{com.pri.mir}{\verboinum{18}}
\verb{comprobatório}{}{}{}{}{adj.}{Que representa prova de algo.}{com.pro.ba.tó.rio}{0}
\verb{comprometedor}{ô}{}{}{}{adj.}{Que compromete ou pode comprometer.}{com.pro.me.te.dor}{0}
\verb{comprometer}{ê}{}{}{}{v.t.}{Dar como garantia; empenhar.}{com.pro.me.ter}{0}
\verb{comprometer}{ê}{}{}{}{}{Expor a risco ou perigo; arriscar.}{com.pro.me.ter}{0}
\verb{comprometer}{ê}{}{}{}{}{Colocar em situação desfavorável ou constrangedora.}{com.pro.me.ter}{0}
\verb{comprometer}{ê}{}{}{}{v.pron.}{Assumir compromisso.}{com.pro.me.ter}{\verboinum{12}}
\verb{comprometimento}{}{}{}{}{s.m.}{Ato ou efeito de comprometer.}{com.pro.me.ti.men.to}{0}
\verb{compromisso}{}{}{}{}{s.m.}{Obrigação assumida entre duas ou mais pessoas.}{com.pro.mis.so}{0}
\verb{compromisso}{}{}{}{}{}{Obrigação ou promessa solene.}{com.pro.mis.so}{0}
\verb{compromisso}{}{Pop.}{}{}{}{Obrigação de caráter social, geralmente com data e horário determinados.}{com.pro.mis.so}{0}
\verb{compromisso}{}{Pop.}{}{}{}{Dívida a ser paga em data estabelecida.}{com.pro.mis.so}{0}
\verb{comprovação}{}{}{"-ões}{}{s.f.}{Ato ou efeito de comprovar; prova.}{com.pro.va.ção}{0}
\verb{comprovante}{}{}{}{}{adj.2g.}{Que comprova.}{com.pro.van.te}{0}
\verb{comprovante}{}{Bras.}{}{}{s.m.}{Recibo que comprova a realização de uma transação.}{com.pro.van.te}{0}
\verb{comprovar}{}{}{}{}{v.t.}{Contribuir para provar.}{com.pro.var}{0}
\verb{comprovar}{}{}{}{}{}{Juntar novas provas.}{com.pro.var}{0}
\verb{comprovar}{}{}{}{}{}{Demonstrar.}{com.pro.var}{0}
\verb{comprovar}{}{}{}{}{}{Examinar para tomar (algo) como certo.}{com.pro.var}{\verboinum{1}}
\verb{comprovativo}{}{}{}{}{adj.}{Que representa prova; comprobatório.}{com.pro.va.ti.vo}{0}
\verb{compulsão}{}{}{"-ões}{}{s.f.}{Ato ou efeito de compelir.}{com.pul.são}{0}
\verb{compulsão}{}{Psicol.}{"-ões}{}{}{Impulso que leva à realização de um gesto físico ou psíquico de maneira repetitiva.}{com.pul.são}{0}
\verb{compulsar}{}{}{}{}{v.t.}{Examinar, manusear, folhear, consultar.}{com.pul.sar}{\verboinum{1}}
\verb{compulsivo}{}{}{}{}{adj.}{Relativo a compulsão; impulsivo.}{com.pul.si.vo}{0}
\verb{compulsória}{}{}{}{}{s.f.}{Mandado ou sentença de juiz a uma instância inferior.}{com.pul.só.ria}{0}
\verb{compulsória}{}{}{}{}{}{Aposentadoria imposta aos servidores públicos quando atingem certo limite de idade.}{com.pul.só.ria}{0}
\verb{compulsório}{}{}{}{}{adj.}{Que compele, que obriga.}{com.pul.só.rio}{0}
\verb{compunção}{}{}{"-ões}{}{s.f.}{Pesar, arrependimento, contrição.}{com.pun.ção}{0}
\verb{compungir}{}{}{}{}{v.t.}{Provocar compunção.}{com.pun.gir}{0}
\verb{compungir}{}{}{}{}{}{Magoar.}{com.pun.gir}{0}
\verb{compungir}{}{}{}{}{v.pron.}{Arrepender"-se.}{com.pun.gir}{\verboinum{22}}
\verb{computação}{}{}{"-ões}{}{s.f.}{Ato ou efeito de computar.}{com.pu.ta.ção}{0}
\verb{computação}{}{}{"-ões}{}{}{Cálculo, contagem.}{com.pu.ta.ção}{0}
\verb{computação}{}{Informát.}{"-ões}{}{}{Processamento de dados, feito em computadores, para sistematizar as informações e produzir os resultados determinados.}{com.pu.ta.ção}{0}
\verb{computador}{ô}{}{}{}{adj.}{Que computa; calculador.}{com.pu.ta.dor}{0}
\verb{computador}{ô}{Informát.}{}{}{s.m.}{Máquina eletrônica capaz de executar instruções, usada no tratamento dos dados com objetivos determinados.}{com.pu.ta.dor}{0}
\verb{computadorizar}{}{}{}{}{v.t.}{Empregar ou implantar computadores para realizar processamento de dados.}{com.pu.ta.do.ri.zar}{\verboinum{1}}
\verb{computar}{}{}{}{}{v.t.}{Fazer o cômputo de; calcular, contar, incluir.}{com.pu.tar}{0}
\verb{computar}{}{Informát.}{}{}{v.i.}{Executar um algoritmo produzindo um resultado determinado.}{com.pu.tar}{\verboinum{1}}
\verb{cômputo}{}{}{}{}{s.m.}{Contagem, averiguação, soma.}{côm.pu.to}{0}
\verb{comum}{}{}{"-uns}{}{adj.2g.}{Que pertence a todos os membros de um grupo determinado.}{co.mum}{0}
\verb{comum}{}{}{"-uns}{}{}{Trivial, vulgar, habitual, normal.}{co.mum}{0}
\verb{comum}{}{}{"-uns}{}{}{Feito em comunidade.}{co.mum}{0}
\verb{comum}{}{}{"-uns}{}{s.m.}{Qualidade de comum.}{co.mum}{0}
\verb{comum}{}{}{"-uns}{}{}{A maioria.}{co.mum}{0}
\verb{comum"-de"-dois}{}{Gram.}{comuns"-de"-dois}{}{adj.2g.}{Diz"-se do substantivo que tem só uma forma para o feminino e o masculino; comum"-de"-dois gêneros, uniforme.   (\textit{O imigrante, a imigrante.})}{co.mum"-de"-dois}{0}
\verb{comuna}{}{Hist.}{}{}{s.f.}{Na Idade Média, cidade emancipada e com autonomia em relação ao feudo.}{co.mu.na}{0}
\verb{comuna}{}{}{}{}{}{Município, municipalidade.}{co.mu.na}{0}
\verb{comuna}{}{Bras.}{}{}{}{Redução de \textit{comunista}.}{co.mu.na}{0}
\verb{comungante}{}{}{}{}{adj.2g.}{Que comunga, que recebe a comunhão.}{co.mun.gan.te}{0}
\verb{comungar}{}{}{}{}{v.t.}{Participar de crenças; estar de acordo; partilhar.}{co.mun.gar}{0}
\verb{comungar}{}{Relig.}{}{}{v.i.}{Receber a comunhão.}{co.mun.gar}{\verboinum{5}}
\verb{comunhão}{}{}{"-ões}{}{s.f.}{Compartilhamento de ideias.}{co.mu.nhão}{0}
\verb{comunhão}{}{}{"-ões}{}{}{Ação ou estado de fazer ou usufruir das coisas em comum.}{co.mu.nhão}{0}
\verb{comunhão}{}{Relig.}{"-ões}{}{}{No catolicismo, o Sacramento da Eucaristia.}{co.mu.nhão}{0}
\verb{comunicabilidade}{}{}{}{}{s.f.}{Qualidade de comunicável.}{co.mu.ni.ca.bi.li.da.de}{0}
\verb{comunicação}{}{}{"-ões}{}{s.f.}{Ato ou efeito de comunicar.}{co.mu.ni.ca.ção}{0}
\verb{comunicação}{}{}{"-ões}{}{}{Processo que envolve transmissão e recepção de mensagens através de meios ou recursos físicos.}{co.mu.ni.ca.ção}{0}
\verb{comunicação}{}{Por ext.}{"-ões}{}{}{A informação transmitida.}{co.mu.ni.ca.ção}{0}
\verb{comunicação}{}{}{"-ões}{}{}{Acesso entre dois lugares; passagem.}{co.mu.ni.ca.ção}{0}
\verb{comunicado}{}{}{}{}{adj.}{Que se comunicou.}{co.mu.ni.ca.do}{0}
\verb{comunicado}{}{}{}{}{s.m.}{Informação oficial; notícia.}{co.mu.ni.ca.do}{0}
\verb{comunicador}{ô}{}{}{}{adj.}{Que comunica.}{co.mu.ni.ca.dor}{0}
\verb{comunicador}{ô}{}{}{}{s.m.}{Profissional da área de comunicação.}{co.mu.ni.ca.dor}{0}
\verb{comunicar}{}{}{}{}{v.t.}{Tornar comum.}{co.mu.ni.car}{0}
\verb{comunicar}{}{}{}{}{}{Fazer saber; noticiar, informar.}{co.mu.ni.car}{0}
\verb{comunicar}{}{}{}{}{}{Ligar, unir.}{co.mu.ni.car}{0}
\verb{comunicar}{}{}{}{}{v.pron.}{Relacionar"-se.}{co.mu.ni.car}{0}
\verb{comunicar}{}{}{}{}{}{Corresponder"-se.}{co.mu.ni.car}{\verboinum{2}}
\verb{comunicativo}{}{}{}{}{adj.}{Relativo a comunicação.}{co.mu.ni.ca.ti.vo}{0}
\verb{comunicativo}{}{}{}{}{}{Que se comunica bastante ou com facilidade; expansivo, sociável.}{co.mu.ni.ca.ti.vo}{0}
\verb{comunicável}{}{}{"-eis}{}{adj.2g.}{Que pode ser comunicado.}{co.mu.ni.cá.vel}{0}
\verb{comunicável}{}{}{"-eis}{}{}{Expansivo, sociável, conversador.}{co.mu.ni.cá.vel}{0}
\verb{comunidade}{}{}{}{}{s.f.}{Conjunto de pessoas que vivem em comum ou que compartilham de determinada atividade ou situação.}{co.mu.ni.da.de}{0}
\verb{comunidade}{}{}{}{}{}{Qualidade de comum.}{co.mu.ni.da.de}{0}
\verb{comunismo}{}{}{}{}{s.m.}{Sistema socioeconômico baseado na propriedade coletiva dos meios de produção.}{co.mu.nis.mo}{0}
\verb{comunista}{}{}{}{}{adj.2g.}{Adepto do comunismo.}{co.mu.nis.ta}{0}
\verb{comunitário}{}{}{}{}{adj.}{Relativo a comunidade.}{co.mu.ni.tá.rio}{0}
\verb{comunitário}{}{}{}{}{}{De uso comum.}{co.mu.ni.tá.rio}{0}
\verb{comutação}{}{}{"-ões}{}{s.f.}{Ato ou efeito de comutar.}{co.mu.ta.ção}{0}
\verb{comutador}{ô}{}{}{}{adj.}{Que comuta.}{co.mu.ta.dor}{0}
\verb{comutador}{ô}{Fís.}{}{}{s.m.}{Dispositivo eletrônico que muda a direção de uma corrente elétrica.}{co.mu.ta.dor}{0}
\verb{comutar}{}{}{}{}{v.t.}{Trocar, permutar, substituir.}{co.mu.tar}{0}
\verb{comutar}{}{Jur.}{}{}{}{Mudar uma pena por outra menos severa.}{co.mu.tar}{\verboinum{1}}
\verb{comutativo}{}{}{}{}{adj.}{Que comuta, substitui.}{co.mu.ta.ti.vo}{0}
\verb{comutativo}{}{Mat.}{}{}{}{Diz"-se da propriedade de certas operações em que o resultado independe da ordem dos elementos que estão sendo operados.}{co.mu.ta.ti.vo}{0}
\verb{comutável}{}{}{"-eis}{}{adj.2g.}{Que se pode comutar.}{co.mu.tá.vel}{0}
\verb{concatenação}{}{}{"-ões}{}{s.f.}{Ato ou efeito de concatenar.}{con.ca.te.na.ção}{0}
\verb{concatenar}{}{}{}{}{v.t.}{Formar uma sequência lógica; encadear, ligar, relacionar, harmonizar.}{con.ca.te.nar}{\verboinum{1}}
\verb{concavidade}{}{}{}{}{s.f.}{Qualidade de côncavo.}{con.ca.vi.da.de}{0}
\verb{concavidade}{}{}{}{}{}{A parte côncava de um objeto.}{con.ca.vi.da.de}{0}
\verb{concavidade}{}{}{}{}{}{Depressão, cavidade.}{con.ca.vi.da.de}{0}
\verb{côncavo}{}{}{}{}{adj.}{Cuja superfície é mais baixa no centro do que nas bordas; escavado.}{côn.ca.vo}{0}
\verb{côncavo}{}{}{}{}{s.m.}{Cavidade, concavidade.}{côn.ca.vo}{0}
\verb{côncavo"-convexo}{écs}{}{côncavo"-convexos ⟨écs⟩}{}{adj.}{Que é côncavo de um lado e convexo de outro.}{côn.ca.vo"-con.ve.xo}{0}
\verb{conceber}{ê}{}{}{}{v.t.}{Gerar.}{con.ce.ber}{0}
\verb{conceber}{ê}{Fig.}{}{}{}{Idealizar, inventar.}{con.ce.ber}{0}
\verb{conceber}{ê}{}{}{}{}{Compreender, aceitar, entender.}{con.ce.ber}{\verboinum{12}}
\verb{conceder}{ê}{}{}{}{v.t.}{Dar permissão; consentir.}{con.ce.der}{0}
\verb{conceder}{ê}{}{}{}{}{Outorgar.}{con.ce.der}{0}
\verb{conceder}{ê}{}{}{}{}{Admitir por hipótese.}{con.ce.der}{0}
\verb{conceder}{ê}{}{}{}{}{Concordar, anuir, transigir.}{con.ce.der}{\verboinum{12}}
\verb{conceição}{}{}{"-ões}{}{s.f.}{Ato de conceber; concepção.}{con.cei.ção}{0}
\verb{conceição}{}{Relig.}{"-ões}{}{}{Dogma católico da concepção de um filho pela Virgem Maria mesmo sendo virgem.}{con.cei.ção}{0}
\verb{conceição}{}{Relig.}{"-ões}{}{}{Festa comemorativa desse episódio.}{con.cei.ção}{0}
\verb{conceito}{ê}{}{}{}{s.m.}{Ideia, opinião, juízo.}{con.cei.to}{0}
\verb{conceito}{ê}{}{}{}{}{Ideia expressa de maneira concisa; definição, conceituação.}{con.cei.to}{0}
\verb{conceito}{ê}{}{}{}{}{Reputação, fama.}{con.cei.to}{0}
\verb{conceituação}{}{}{"-ões}{}{s.f.}{Ato ou efeito de conceituar.}{con.cei.tu.a.ção}{0}
\verb{conceituação}{}{}{"-ões}{}{}{Ideia expressa de maneira concisa; definição.}{con.cei.tu.a.ção}{0}
\verb{conceituado}{}{}{}{}{adj.}{Que desfruta de um bom conceito.}{con.cei.tu.a.do}{0}
\verb{conceituar}{}{}{}{}{v.t.}{Criar ou enunciar um conceito.}{con.cei.tu.ar}{0}
\verb{conceituar}{}{}{}{}{}{Formar ou emitir conceito, opinião sobre.}{con.cei.tu.ar}{\verboinum{1}}
\verb{conceituoso}{ô}{}{"-osos ⟨ó⟩}{"-osa ⟨ó⟩}{adj.}{Em que há conceito; sentencioso.}{con.cei.tu.o.so}{0}
\verb{concentração}{}{}{"-ões}{}{s.f.}{Ato ou efeito de concentrar.}{con.cen.tra.ção}{0}
\verb{concentração}{}{}{"-ões}{}{}{Aglomeração, reunião.}{con.cen.tra.ção}{0}
\verb{concentração}{}{Quím.}{"-ões}{}{}{Proporção de uma substância em relação à quantidade total de mistura.}{con.cen.tra.ção}{0}
\verb{concentrado}{}{}{}{}{adj.}{Centralizado.}{con.cen.tra.do}{0}
\verb{concentrado}{}{}{}{}{}{Limitado, apertado.}{con.cen.tra.do}{0}
\verb{concentrado}{}{}{}{}{}{Absorto, ensimesmado.}{con.cen.tra.do}{0}
\verb{concentrado}{}{Quím.}{}{}{}{Diz"-se de mistura ou solução com alto teor do componente soluto.}{con.cen.tra.do}{0}
\verb{concentrado}{}{}{}{}{s.m.}{Alimento ou substância da qual se reduziu o volume, tornando mais alta a concentração das substâncias relevantes.}{con.cen.tra.do}{0}
\verb{concentrar}{}{}{}{}{v.t.}{Fazer convergir em um centro; centralizar.}{con.cen.trar}{0}
\verb{concentrar}{}{}{}{}{}{Reunir em um mesmo local.}{con.cen.trar}{0}
\verb{concentrar}{}{Quím.}{}{}{}{Tornar mais concentrado.}{con.cen.trar}{0}
\verb{concentrar}{}{}{}{}{v.pron.}{Dirigir as atenções para um assunto.}{con.cen.trar}{0}
\verb{concentrar}{}{Bras.}{}{}{}{Permanecer em concentração.}{con.cen.trar}{\verboinum{1}}
\verb{concêntrico}{}{}{}{}{adj.}{Diz"-se de círculos, figuras, objetos etc. que têm o centro localizado em um mesmo ponto.}{con.cên.tri.co}{0}
\verb{concepção}{}{}{"-ões}{}{s.f.}{Ato ou efeito de conceber.}{con.cep.ção}{0}
\verb{concepção}{}{}{"-ões}{}{}{Conceito, ideia (a respeito de algo).}{con.cep.ção}{0}
\verb{concernente}{}{}{}{}{adj.2g.}{Relativo a; que diz respeito a.}{con.cer.nen.te}{0}
\verb{concernir}{}{}{}{}{v.t.}{Estar relacionado com; dizer respeito a.}{con.cer.nir}{\verboinum{29}}
\verb{concertado}{}{}{}{}{adj.}{Combinado, ajustado.}{con.cer.ta.do}{0}
\verb{concertado}{}{}{}{}{}{Afinado, harmonioso, compassado.}{con.cer.ta.do}{0}
\verb{concertado}{}{}{}{}{}{Estudado.}{con.cer.ta.do}{0}
\verb{concertar}{}{}{}{}{v.t.}{Harmonizar, concordar, conciliar.}{con.cer.tar}{0}
\verb{concertar}{}{}{}{}{}{Combinar, pactuar, ajustar.}{con.cer.tar}{0}
\verb{concertar}{}{}{}{}{v.i.}{Soar harmoniosamente.}{con.cer.tar}{\verboinum{1}}
\verb{concertina}{}{Mús.}{}{}{s.f.}{Instrumento semelhante a um acordeão, com caixa hexagonal e dois teclados.}{con.cer.ti.na}{0}
\verb{concertista}{}{}{}{}{adj.2g.}{Que toma parte em concertos.}{con.cer.tis.ta}{0}
\verb{concerto}{ê}{}{}{}{s.m.}{Ato ou efeito de concertar.}{con.cer.to}{0}
\verb{concerto}{ê}{Mús.}{}{}{}{Composição musical para orquestra.}{con.cer.to}{0}
\verb{concerto}{ê}{}{}{}{}{Sessão em que é executada uma ou mais peças musicais.}{con.cer.to}{0}
\verb{concessão}{}{}{"-ões}{}{s.f.}{Ato ou efeito de conceder.}{con.ces.são}{0}
\verb{concessão}{}{}{"-ões}{}{}{Outorga, permissão.}{con.ces.são}{0}
\verb{concessão}{}{}{"-ões}{}{}{Privilégio obtido do Estado por uma empresa, teoricamente por tempo determinado, para explorar certa atividade econômica.}{con.ces.são}{0}
\verb{concessionária}{}{}{}{}{s.f.}{Empresa que recebeu uma concessão.}{con.ces.si.o.ná.ria}{0}
\verb{concessionária}{}{}{}{}{}{Estabelecimento que comercializa veículos, geralmente representante de uma montadora.}{con.ces.si.o.ná.ria}{0}
\verb{concessionário}{}{}{}{}{adj.}{Que obteve uma concessão.}{con.ces.si.o.ná.rio}{0}
\verb{concessivo}{}{}{}{}{adj.}{Relativo a concessão.}{con.ces.si.vo}{0}
\verb{concessivo}{}{Gram.}{}{}{}{Diz"-se de conjunção ou oração subordinada que exprime oposição a ideia ou fato da oração principal, não sendo, porém, suficiente para anular aquela ideia ou fato.}{con.ces.si.vo}{0}
\verb{concessor}{ô}{}{}{}{adj.}{Que concede.}{con.ces.sor}{0}
\verb{concha}{}{}{}{}{s.f.}{Invólucro calcário dos moluscos, de forma côncava de um lado e convexa de outro.}{con.cha}{0}
\verb{concha}{}{Por ext.}{}{}{}{Qualquer forma semelhante a esse invólucro.}{con.cha}{0}
\verb{concha}{}{}{}{}{}{Utensílio de cozinha próprio para servir sopas e caldos.}{con.cha}{0}
\verb{conchavar}{}{}{}{}{v.t.}{Combinar, ajustar.}{con.cha.var}{0}
\verb{conchavar}{}{}{}{}{}{Unir, ligar.}{con.cha.var}{0}
\verb{conchavar}{}{}{}{}{}{Fazer conchavo.}{con.cha.var}{\verboinum{1}}
\verb{conchavo}{}{}{}{}{s.m.}{Combinação com más intenções ou visando apenas benefício próprio; conluio, cambalacho.}{con.cha.vo}{0}
\verb{conchavo}{}{}{}{}{}{Ato ou efeito de conchavar.}{con.cha.vo}{0}
\verb{conchegar}{}{}{}{}{v.t.}{Aproximar, procurando conforto, abrigo, alento; chegar perto; achegar; aconchegar.  }{con.che.gar}{0}
\verb{conchegar}{}{}{}{}{}{Acomodar, ajeitar. (\textit{Conchegar o gorro na cabeça.})}{con.che.gar}{\verboinum{5}}
\verb{conchego}{ê}{}{}{}{s.m.}{Ato ou efeito de conchegar; aconchego.}{con.che.go}{0}
\verb{conchego}{ê}{}{}{}{}{Conforto, alento, abrigo.}{con.che.go}{0}
\verb{conchego}{ê}{}{}{}{}{Pessoa que protege; amparo. }{con.che.go}{0}
\verb{concidadão}{}{}{"-ãos}{}{s.m.}{Indivíduo que é natural do mesmo país ou cidade (que alguém); conterrâneo, compatriota.}{con.ci.da.dão}{0}
\verb{conciliábulo}{}{}{}{}{s.m.}{Reunião secreta de más intenções; conluio.}{con.ci.li.á.bu.lo}{0}
\verb{conciliábulo}{}{}{}{}{}{Conversação reservada.}{con.ci.li.á.bu.lo}{0}
\verb{conciliação}{}{}{"-ões}{}{s.f.}{Ato ou efeito de conciliar.}{con.ci.li.a.ção}{0}
\verb{conciliação}{}{}{"-ões}{}{}{Acordo, harmonização, concórdia.}{con.ci.li.a.ção}{0}
\verb{conciliador}{ô}{}{}{}{adj.}{Que tem a propriedade de conciliar; harmonizador, pacificador.}{con.ci.li.a.dor}{0}
\verb{conciliar}{}{}{}{}{adj.2g.}{Relativo a concílio.}{con.ci.li.ar}{0}
\verb{conciliar}{}{}{}{}{v.t.}{Colocar em acordo; harmonizar, congraçar.  }{con.ci.li.ar}{0}
\verb{conciliar}{}{}{}{}{v.pron.}{Entrar em comum acordo.}{con.ci.li.ar}{\verboinum{6}}
\verb{conciliatório}{}{}{}{}{adj.}{Que serve para conciliar.}{con.ci.li.a.tó.rio}{0}
\verb{conciliável}{}{}{"-eis}{}{adj.2g.}{Que se pode conciliar.}{con.ci.li.á.vel}{0}
\verb{concílio}{}{}{}{}{s.m.}{Reunião de altos membros da Igreja Católica para tratar de questões de doutrina, fé e disciplina.}{con.cí.lio}{0}
\verb{concisão}{}{}{"-ões}{}{s.f.}{Qualidade de conciso; brevidade.}{con.ci.são}{0}
\verb{conciso}{}{}{}{}{adj.}{Que tem curta duração; breve.}{con.ci.so}{0}
\verb{conciso}{}{}{}{}{}{Que expõe ideias com poucas palavras; sintético.}{con.ci.so}{0}
\verb{conciso}{}{}{}{}{}{Que se limita ao que é essencial; sucinto.}{con.ci.so}{0}
\verb{concitar}{}{}{}{}{v.t.}{Instigar, estimular, excitar, incitar.}{con.ci.tar}{\verboinum{1}}
\verb{conclamação}{}{}{"-ões}{}{s.f.}{Ato ou efeito de conclamar.}{con.cla.ma.ção}{0}
\verb{conclamar}{}{}{}{}{v.t.}{Gritar juntamente; bradar.}{con.cla.mar}{0}
\verb{conclamar}{}{}{}{}{}{Aclamar coletivamente, eleger.}{con.cla.mar}{0}
\verb{conclamar}{}{}{}{}{}{Chamar com veemência; convocar, reunir.}{con.cla.mar}{\verboinum{1}}
\verb{conclave}{}{}{}{}{s.m.}{Reunião de cardeais para eleger o papa.}{con.cla.ve}{0}
\verb{conclave}{}{Por ext.}{}{}{}{Qualquer reunião ou assembleia, especialmente para tratar de assuntos importantes.}{con.cla.ve}{0}
\verb{concludente}{}{}{}{}{adj.2g.}{Convincente, terminante.}{con.clu.den.te}{0}
\verb{concluir}{}{}{}{}{v.t.}{Levar a cabo; terminar, arrematar.}{con.clu.ir}{0}
\verb{concluir}{}{}{}{}{}{Deduzir, inferir.}{con.clu.ir}{0}
\verb{concluir}{}{}{}{}{v.i.}{Terminar de falar.}{con.clu.ir}{\verboinum{26}}
\verb{conclusão}{}{}{"-ões}{}{s.f.}{Ato ou efeito de concluir.}{con.clu.são}{0}
\verb{conclusão}{}{}{"-ões}{}{}{Desfecho, resultado.}{con.clu.são}{0}
\verb{conclusão}{}{}{"-ões}{}{}{Parte final de um texto escrito.}{con.clu.são}{0}
\verb{conclusivo}{}{}{}{}{adj.}{Que representa conclusão.}{con.clu.si.vo}{0}
\verb{concomitância}{}{}{}{}{s.f.}{Qualidade de concomitante.}{con.co.mi.tân.cia}{0}
\verb{concomitante}{}{}{}{}{adj.2g.}{Simultâneo, coexistente.}{con.co.mi.tan.te}{0}
\verb{concordância}{}{}{}{}{s.f.}{Ato ou efeito de concordar.}{con.cor.dân.cia}{0}
\verb{concordância}{}{}{}{}{}{Coerência, harmonia, consonância.}{con.cor.dân.cia}{0}
\verb{concordância}{}{Gram.}{}{}{}{Fenômeno gramatical em que, por flexão, o adjetivo copia as categorias de gênero e número dos substantivos aos quais se relaciona, e o verbo copia as categorias de número e pessoa do sujeito da oração.}{con.cor.dân.cia}{0}
\verb{concordar}{}{}{}{}{v.t.}{Estar ou passar a estar de acordo; convir.}{con.cor.dar}{0}
\verb{concordar}{}{}{}{}{}{Estar conforme; assentir, harmonizar"-se.}{con.cor.dar}{0}
\verb{concordar}{}{Gram.}{}{}{}{Copiar as categorias de gênero, número e pessoa; realizar concordância.}{con.cor.dar}{\verboinum{1}}
\verb{concordata}{}{}{}{}{s.f.}{Acordo ou benefício que dá ao falido a possibilidade de pagar dívidas em prazo negociado.}{con.cor.da.ta}{0}
\verb{concordata}{}{}{}{}{}{Convenção entre a Igreja e o Estado.}{con.cor.da.ta}{0}
\verb{concorde}{ó}{}{}{}{adj.2g.}{Que está de acordo.}{con.cor.de}{0}
\verb{concórdia}{}{}{}{}{s.f.}{Situação em que há acordo; harmonia, paz.}{con.cór.dia}{0}
\verb{concorrência}{}{}{}{}{s.f.}{Ato de concorrer.}{con.cor.rên.cia}{0}
\verb{concorrência}{}{}{}{}{}{Rivalidade, disputa, competição.}{con.cor.rên.cia}{0}
\verb{concorrência}{}{}{}{}{}{Conjunto dos concorrentes, outras empresas que atuam em um mesmo segmento de mercado.}{con.cor.rên.cia}{0}
\verb{concorrente}{}{}{}{}{adj.2g.}{Que concorre.}{con.cor.ren.te}{0}
\verb{concorrente}{}{}{}{}{s.2g.}{Empresa que atua em um mesmo segmento de mercado.}{con.cor.ren.te}{0}
\verb{concorrer}{ê}{}{}{}{v.t.}{Cooperar, contribuir.}{con.cor.rer}{0}
\verb{concorrer}{ê}{}{}{}{}{Dirigir"-se para um mesmo lugar; afluir.}{con.cor.rer}{0}
\verb{concorrer}{ê}{}{}{}{}{Tomar parte em competição.}{con.cor.rer}{0}
\verb{concorrer}{ê}{}{}{}{}{Participar de concurso.}{con.cor.rer}{\verboinum{12}}
\verb{concorrido}{}{}{}{}{adj.}{Muito disputado, muito frequentado.}{con.cor.ri.do}{0}
\verb{concreção}{}{}{"-ões}{}{s.f.}{Ato ou efeito de concretizar.}{con.cre.ção}{0}
\verb{concreção}{}{Med.}{"-ões}{}{}{Massa inorgânica endurecida, anormal, encontrada no interior de um tecido.}{con.cre.ção}{0}
\verb{concretismo}{}{Art.}{}{}{s.m.}{Movimento artístico que prega a necessidade de concretizar os conceitos intelectuais através das várias técnicas de composição.}{con.cre.tis.mo}{0}
\verb{concretização}{}{}{"-ões}{}{s.f.}{Ato ou efeito de concretizar.}{con.cre.ti.za.ção}{0}
\verb{concretizar}{}{}{}{}{v.t.}{Tornar concreto.}{con.cre.ti.zar}{\verboinum{1}}
\verb{concreto}{é}{}{}{}{adj.}{Que tem consistência.}{con.cre.to}{0}
\verb{concreto}{é}{}{}{}{}{Que tem existência no mundo material; sensível, material.}{con.cre.to}{0}
\verb{concreto}{é}{}{}{}{}{Solidificado, condensado, sólido.}{con.cre.to}{0}
\verb{concreto}{é}{}{}{}{}{Determinado, específico, particularizado.}{con.cre.to}{0}
\verb{concreto}{é}{Gram.}{}{}{}{Diz"-se de substantivo que designa objeto concreto.}{con.cre.to}{0}
\verb{concreto}{é}{}{}{}{s.m.}{Aquilo que é concreto.}{con.cre.to}{0}
\verb{concreto}{é}{}{}{}{}{Material de construção muito resistente, feito com cimento, areia, cascalho e água.}{con.cre.to}{0}
\verb{concubina}{}{}{}{}{s.f.}{Mulher que vive em união estável com um homem sem estar religiosa ou juridicamente casada.}{con.cu.bi.na}{0}
\verb{concubinato}{}{}{}{}{s.m.}{União livre e estável entre pessoas sem que estejam religiosa ou juridicamente casadas.}{con.cu.bi.na.to}{0}
\verb{conculcar}{}{}{}{}{v.t.}{Calcar os pés; espezinhar, esmagar.}{con.cul.car}{0}
\verb{conculcar}{}{Fig.}{}{}{}{Desdenhar, desprezar.}{con.cul.car}{\verboinum{2}}
\verb{concunhado}{}{}{}{}{s.m.}{O marido da cunhada.}{con.cu.nha.do}{0}
\verb{concupiscência}{}{}{}{}{s.f.}{Apetite sexual.}{con.cu.pis.cên.cia}{0}
\verb{concupiscência}{}{}{}{}{}{Desejo por prazeres e bens materiais.}{con.cu.pis.cên.cia}{0}
\verb{concupiscente}{}{}{}{}{adj.2g.}{Que tem ou expressa concupiscência.}{con.cu.pis.cen.te}{0}
\verb{concursado}{}{}{}{}{adj.}{Habilitado por concurso público.}{con.cur.sa.do}{0}
\verb{concursar}{}{}{}{}{v.t.}{Submeter a concurso.}{con.cur.sar}{\verboinum{1}}
\verb{concurso}{}{}{}{}{s.m.}{Provas públicas para admissão em certos cargos.}{con.cur.so}{0}
\verb{concurso}{}{}{}{}{}{Competição, certame.}{con.cur.so}{0}
\verb{concurso}{}{}{}{}{}{Ato ou efeito de concorrer; concorrência.}{con.cur.so}{0}
\verb{concussão}{}{}{"-ões}{}{s.f.}{Choque, pancada, abalo.}{con.cus.são}{0}
\verb{concussão}{}{}{"-ões}{}{}{Extorsão praticada por funcionário público.}{con.cus.são}{0}
\verb{concussionário}{}{}{}{}{adj.}{Que pratica concussão.}{con.cus.si.o.ná.rio}{0}
\verb{condado}{}{}{}{}{s.m.}{Dignidade de conde.}{con.da.do}{0}
\verb{condado}{}{}{}{}{}{Antiga jurisdição ou território de conde.}{con.da.do}{0}
\verb{condão}{}{}{"-ões}{}{s.m.}{Poder mágico.}{con.dão}{0}
\verb{condão}{}{}{"-ões}{}{}{Dom, faculdade, capacidade.}{con.dão}{0}
\verb{conde}{}{}{}{}{s.m.}{Indivíduo que tem o título de nobreza acima de visconde e abaixo de marquês.}{con.de}{0}
\verb{condecoração}{}{}{"-ões}{}{s.f.}{Ato ou efeito de condecorar.}{con.de.co.ra.ção}{0}
\verb{condecoração}{}{}{"-ões}{}{}{Distinção ou insígnia honorífica.}{con.de.co.ra.ção}{0}
\verb{condecorar}{}{}{}{}{v.t.}{Conferir título ou honraria a.}{con.de.co.rar}{\verboinum{1}}
\verb{condenação}{}{}{"-ões}{}{s.f.}{Ato ou efeito de condenar; sentença condenatória.}{con.de.na.ção}{0}
\verb{condenação}{}{}{"-ões}{}{}{A pena imposta por essa sentença.}{con.de.na.ção}{0}
\verb{condenação}{}{Fig.}{"-ões}{}{}{Censura, reprovação.}{con.de.na.ção}{0}
\verb{condenado}{}{Por ext.}{}{}{s.m.}{Criminoso que aguarda sentença.}{con.de.na.do}{0}
\verb{condenado}{}{}{}{}{}{Rejeitado, reprovado.}{con.de.na.do}{0}
\verb{condenado}{}{}{}{}{adj.}{Sentenciado como criminoso.}{con.de.na.do}{0}
\verb{condenar}{}{}{}{}{v.t.}{Proferir sentença condenatória contra; declarar culpado.}{con.de.nar}{0}
\verb{condenar}{}{}{}{}{}{Reprovar, censurar, rejeitar.}{con.de.nar}{0}
\verb{condenar}{}{}{}{}{}{Considerar caso perdido, por falta de condições de cura.}{con.de.nar}{\verboinum{1}}
\verb{condenatório}{}{}{}{}{adj.}{Que envolve condenação.}{con.de.na.tó.rio}{0}
\verb{condenável}{}{}{"-eis}{}{adj.2g.}{Que merece condenação.}{con.de.ná.vel}{0}
\verb{condenável}{}{}{"-eis}{}{}{Censurável, reprovável.}{con.de.ná.vel}{0}
\verb{condensação}{}{}{"-ões}{}{s.f.}{Ato ou efeito de condensar; agregação, reunião.}{con.den.sa.ção}{0}
\verb{condensação}{}{Fís.}{"-ões}{}{}{Passagem do estado de vapor ao estado líquido.}{con.den.sa.ção}{0}
\verb{condensação}{}{}{"-ões}{}{}{Síntese.}{con.den.sa.ção}{0}
\verb{condensador}{ô}{}{}{}{adj.}{Que condensa.}{con.den.sa.dor}{0}
\verb{condensador}{ô}{}{}{}{s.m.}{Dispositivo que armazena energia elétrica.}{con.den.sa.dor}{0}
\verb{condensador}{ô}{Quím.}{}{}{}{Dispositivo em que se realiza a condensação de um vapor.}{con.den.sa.dor}{0}
\verb{condensar}{}{}{}{}{v.t.}{Tornar denso ou mais denso; espessar, engrossar.}{con.den.sar}{0}
\verb{condensar}{}{}{}{}{}{Reduzir alguma coisa gasosa a líquido; liquefazer.}{con.den.sar}{0}
\verb{condensar}{}{}{}{}{}{Reduzir um texto ao mais importante; resumir, sintetizar.}{con.den.sar}{\verboinum{1}}
\verb{condescendência}{}{}{}{}{s.f.}{Ato de condescender; complacência, consentimento.}{con.des.cen.dên.cia}{0}
\verb{condescendente}{}{}{}{}{adj.2g.}{Que condescende; complacente, transigente.}{con.des.cen.den.te}{0}
\verb{condescender}{ê}{}{}{}{v.t.}{Ceder ou consentir em favor de alguém; transigir espontaneamente.}{con.des.cen.der}{\verboinum{12}}
\verb{condessa}{ê}{}{}{}{s.f.}{Na Idade Média, mulher que possuía um condado.}{con.des.sa}{0}
\verb{condessa}{ê}{}{}{}{}{Esposa do conde.}{con.des.sa}{0}
\verb{condição}{}{}{"-ões}{}{s.f.}{Fato que permite a realização de alguma coisa.}{con.di.ção}{0}
\verb{condição}{}{}{"-ões}{}{}{Qualidade que permite a realização de alguma coisa; característica.}{con.di.ção}{0}
\verb{condição}{}{}{"-ões}{}{}{Estado em que alguma coisa se encontra; situação.}{con.di.ção}{0}
\verb{condicente}{}{}{}{}{adj.2g.}{Condizente.}{con.di.cen.te}{0}
\verb{condicionado}{}{}{}{}{adj.}{Que é dependente de, ou imposto por condição.}{con.di.ci.o.na.do}{0}
\verb{condicionador}{ô}{}{}{}{adj.}{Que condiciona.}{con.di.ci.o.na.dor}{0}
\verb{condicionador}{ô}{}{}{}{s.m.}{Produto cosmético que torna os cabelos mais macios.}{con.di.ci.o.na.dor}{0}
\verb{condicional}{}{}{}{}{adj.2g.}{Que depende de, ou envolve condição.}{con.di.ci.o.nal}{0}
\verb{condicionar}{}{}{}{}{v.t.}{Pôr condições a; regular.}{con.di.ci.o.nar}{0}
\verb{condicionar}{}{}{}{}{}{Estabelecer como condição.}{con.di.ci.o.nar}{0}
\verb{condicionar}{}{}{}{}{v.pron.}{Habituar"-se a condições novas.}{con.di.ci.o.nar}{\verboinum{1}}
\verb{condigno}{}{}{}{}{adj.}{Que tem dignidade; digno.}{con.dig.no}{0}
\verb{condigno}{}{}{}{}{}{Adequado, merecido.}{con.dig.no}{0}
\verb{condiloma}{}{Med.}{}{}{s.m.}{Dermatose sexualmente transmissível, dolorosa, que surge no ânus, na vulva ou no pênis.}{con.di.lo.ma}{0}
\verb{condimentar}{}{}{}{}{v.t.}{Acrescentar condimento a uma preparação culinária; temperar.}{con.di.men.tar}{0}
\verb{condimentar}{}{Fig.}{}{}{}{Tornar picante, mordaz ou malicioso.}{con.di.men.tar}{\verboinum{1}}
\verb{condimentar}{}{}{}{}{adj.2g.}{Relativo a condimento.}{con.di.men.tar}{0}
\verb{condimento}{}{}{}{}{s.m.}{Substância que se mistura ao alimento para dar"-lhe mais sabor; tempero.}{con.di.men.to}{0}
\verb{condimento}{}{Pop.}{}{}{}{Malícia.}{con.di.men.to}{0}
\verb{condimentoso}{ô}{}{"-osos ⟨ó⟩}{"-osa ⟨ó⟩}{adj.}{Que condimenta.}{con.di.men.to.so}{0}
\verb{condimentoso}{ô}{}{"-osos ⟨ó⟩}{"-osa ⟨ó⟩}{}{Em que há condimentos em abundância; muito temperado.}{con.di.men.to.so}{0}
\verb{condiscípulo}{}{}{}{}{s.m.}{Companheiro de estudos em um estabelecimento de ensino, especialmente do mesmo ano e da mesma sala.}{con.dis.cí.pu.lo}{0}
\verb{condizente}{}{}{}{}{adj.2g.}{Que condiz; que está em harmonia, em proporção ou de acordo.}{con.di.zen.te}{0}
\verb{condizer}{ê}{}{}{}{v.t.}{Estar em harmonia ou ficar bem com algo; assentar, combinar.}{con.di.zer}{0}
\verb{condizer}{ê}{}{}{}{}{Estar de acordo com algo; concordar.}{con.di.zer}{\verboinum{12}}
\verb{condoer}{ê}{}{}{}{v.t.}{Sentir dó, pena, comiseração; compadecer, comover.}{con.do.er}{\verboinum{17}}
\verb{condolência}{}{}{}{}{s.f.}{Sentimento de pesar; compaixão, pena.}{con.do.lên.cia}{0}
\verb{condolências}{}{}{}{}{s.f.pl.}{Manifestação de pesar por infelicidade ou mal de outrem; pêsames.}{con.do.lên.cias}{0}
\verb{condolente}{}{}{}{}{adj.2g.}{Que tem ou revela condolência, pena, compaixão; compassivo.}{con.do.len.te}{0}
\verb{condomínio}{}{}{}{}{s.m.}{Domínio exercido com outrem; copropriedade.}{con.do.mí.nio}{0}
\verb{condomínio}{}{}{}{}{}{Administração de um prédio ou um conjunto de casas.}{con.do.mí.nio}{0}
\verb{condomínio}{}{}{}{}{}{Taxa mensal paga por cada condômino para as despesas com o condomínio.}{con.do.mí.nio}{0}
\verb{condômino}{}{}{}{}{s.m.}{Dono em parceria com outrem; coproprietário.}{con.dô.mi.no}{0}
\verb{condor}{ô}{Zool.}{}{}{s.m.}{Ave de grande porte, encontrada ao longo de toda a Cordilheira dos Andes, de plumagem negra, com espesso colar de plumas brancas e cabeça nua.}{con.dor}{0}
\verb{condoreirismo}{}{}{}{}{s.m.}{Escola brasileira de poesia da última fase romântica, de caráter social e político, que divulgava e defendia ideias igualitárias.}{con.do.rei.ris.mo}{0}
\verb{condoreiro}{ê}{}{}{}{adj.}{Relativo a condor.}{con.do.rei.ro}{0}
\verb{condoreiro}{ê}{}{}{}{}{Diz"-se do estilo poético elevado, característico da última fase do Romantismo brasileiro.}{con.do.rei.ro}{0}
\verb{condoreiro}{ê}{}{}{}{}{Diz"-se de poeta que tem esse estilo.}{con.do.rei.ro}{0}
\verb{condução}{}{}{"-ões}{}{s.f.}{Ato de levar ou trazer.}{con.du.ção}{0}
\verb{condução}{}{Pop.}{"-ões}{}{}{Meio de transporte; veículo, transporte.}{con.du.ção}{0}
\verb{conducente}{}{}{}{}{adj.2g.}{Que conduz ou tende para um fim.}{con.du.cen.te}{0}
\verb{conduíte}{}{}{}{}{s.m.}{Tubo de metal ou plástico, geralmente embutido na parede, por onde passam os fios elétricos.}{con.du.í.te}{0}
\verb{conduta}{}{}{}{}{s.f.}{Ato ou feito de conduzir.}{con.du.ta}{0}
\verb{conduta}{}{}{}{}{}{Modo de agir, de se portar, de viver; comportamento.}{con.du.ta}{0}
\verb{condutância}{}{Fís. e Quím.}{}{}{s.f.}{Propriedade que possui um condutor capaz de permitir a passagem da corrente elétrica.}{con.du.tân.cia}{0}
\verb{condutibilidade}{}{}{}{}{s.f.}{Propriedade que têm os corpos de serem condutores de calor, eletricidade, som etc.}{con.du.ti.bi.li.da.de}{0}
\verb{conduto}{}{}{}{}{s.m.}{Via por onde se escoa um fluido.}{con.du.to}{0}
\verb{conduto}{}{}{}{}{}{Canal.}{con.du.to}{0}
\verb{condutor}{ô}{}{}{}{adj.}{Que conduz.}{con.du.tor}{0}
\verb{condutor}{ô}{}{}{}{s.m.}{Corpo que conduz eletricidade, calor, som etc.}{con.du.tor}{0}
\verb{condutor}{ô}{}{}{}{}{Indivíduo que conduz; guia.}{con.du.tor}{0}
\verb{condutor}{ô}{}{}{}{}{Indivíduo que conduz um veículo; motorista.}{con.du.tor}{0}
\verb{conduzir}{}{}{}{}{v.t.}{Levar de um lugar para outro; transportar, carregar.}{con.du.zir}{0}
\verb{conduzir}{}{}{}{}{}{Estar no comando de uma pessoa ou coisa; dirigir, guiar.}{con.du.zir}{0}
\verb{conduzir}{}{}{}{}{v.pron.}{Portar"-se, comportar"-se, proceder.}{con.du.zir}{\verboinum{21}}
\verb{cone}{}{}{}{}{s.m.}{Objeto que tem uma base circular e termina em ponta.}{co.ne}{0}
\verb{conectar}{}{}{}{}{v.t.}{Estabelecer conexão entre; unir, ligar.}{co.nec.tar}{0}
\verb{conectar}{}{Informát.}{}{}{}{Estabelecer conexão entre dispositivos ou computadores com o objetivo de transferir dados; interligar.}{co.nec.tar}{\verboinum{1}}
\verb{conectivo}{}{}{}{}{adj.}{Que estabelece conexão; que une uma coisa a outra.}{co.nec.ti.vo}{0}
\verb{conector}{ô}{}{}{}{adj.}{Que conecta; que liga ou interliga.}{co.nec.tor}{0}
\verb{conector}{ô}{}{}{}{}{Diz"-se de peça ou dispositivo usado para conectar partes móveis em uma máquina.}{co.nec.tor}{0}
\verb{conector}{ô}{}{}{}{}{Diz"-se de componente passivo de um circuito elétrico destinado a conectar dois dispositivos.}{co.nec.tor}{0}
\verb{cônego}{}{}{}{}{s.m.}{Padre secular pertencente a um cabido, a uma colegiada ou a certas basílicas.}{cô.ne.go}{0}
\verb{conexão}{cs}{}{"-ões}{}{s.f.}{Ligação de uma coisa a outra; junção.}{co.ne.xão}{0}
\verb{conexão}{cs}{}{"-ões}{}{}{Peça que liga tubulações.}{co.ne.xão}{0}
\verb{conexo}{écs}{}{}{}{adj.}{Que tem, ou em que há conexão; relacionado, ligado.}{co.ne.xo}{0}
\verb{confabulação}{}{}{"-ões}{}{s.f.}{Ato ou efeito de confabular.}{con.fa.bu.la.ção}{0}
\verb{confabulação}{}{}{"-ões}{}{}{Ato ou efeito de tramar algo com outros.}{con.fa.bu.la.ção}{0}
\verb{confabular}{}{}{}{}{v.t.}{Conversar; trocar ideias.}{con.fa.bu.lar}{0}
\verb{confabular}{}{}{}{}{}{Maquinar, tramar.}{con.fa.bu.lar}{\verboinum{1}}
\verb{confecção}{}{}{"-ões}{}{s.f.}{Ato ou efeito de confeccionar; fabricação, preparação.}{con.fec.ção}{0}
\verb{confecção}{}{}{"-ões}{}{}{Pequena fábrica de roupas.}{con.fec.ção}{0}
\verb{confeccionar}{}{}{}{}{v.t.}{Preparar; dar acabamento a.}{con.fec.ci.o.nar}{0}
\verb{confeccionar}{}{}{}{}{}{Fabricar.}{con.fec.ci.o.nar}{\verboinum{1}}
\verb{confederação}{}{}{"-ões}{}{s.f.}{Associação de estados autônomos sob um governo central.}{con.fe.de.ra.ção}{0}
\verb{confederação}{}{}{"-ões}{}{}{Agrupamento de associações.}{con.fe.de.ra.ção}{0}
\verb{confederar}{}{}{}{}{v.t.}{Unir ou associar em confederação.}{con.fe.de.rar}{\verboinum{1}}
\verb{confeitar}{}{}{}{}{v.t.}{Recobrir um doce com açúcar ou glacê.}{con.fei.tar}{\verboinum{1}}
\verb{confeitaria}{}{}{}{}{s.f.}{Lugar onde se fabricam ou se vendem bolos, biscoitos, doces, salgadinhos etc.}{con.fei.ta.ri.a}{0}
\verb{confeiteira}{ê}{}{}{}{s.f.}{Prato para servir doces.}{con.fei.tei.ra}{0}
\verb{confeiteira}{ê}{}{}{}{}{Vasilha para guardar confeitos.}{con.fei.tei.ra}{0}
\verb{confeiteira}{ê}{}{}{}{}{Mulher que faz, vende ou confeita doces.}{con.fei.tei.ra}{0}
\verb{confeiteiro}{ê}{}{}{}{s.m.}{Indivíduo fabricante ou vendedor de confeitos, bolos  e outros doces.}{con.fei.tei.ro}{0}
\verb{confeito}{ê}{}{}{}{s.m.}{Pequena semente coberta de calda de açúcar e seca ao fogo.}{con.fei.to}{0}
\verb{confeito}{ê}{}{}{}{}{Bala.}{con.fei.to}{0}
\verb{confeito}{ê}{}{}{}{}{Pequenas pastilhas coloridas usadas para confeitar bolos.}{con.fei.to}{0}
\verb{conferência}{}{}{}{}{s.f.}{Ato ou efeito de conferir.}{con.fe.rên.cia}{0}
\verb{conferência}{}{}{}{}{}{Exposição de assunto para um grupo de ouvintes; palestra.}{con.fe.rên.cia}{0}
\verb{conferenciar}{}{}{}{}{v.t.}{Discutir, analisar algo em conversa.}{con.fe.ren.ci.ar}{0}
\verb{conferenciar}{}{}{}{}{}{Proferir uma conferência.}{con.fe.ren.ci.ar}{\verboinum{1}}
\verb{conferencista}{}{}{}{}{s.2g.}{Indivíduo que faz conferência.}{con.fe.ren.cis.ta}{0}
\verb{conferente}{}{}{}{}{adj.2g.}{Que confere.}{con.fe.ren.te}{0}
\verb{conferente}{}{}{}{}{s.2g.}{Indivíduo que faz conferência.}{con.fe.ren.te}{0}
\verb{conferente}{}{}{}{}{}{Auxiliar de revisão que lê os originais em voz alta para o revisor ou escuta a sua leitura, procurando discrepâncias. }{con.fe.ren.te}{0}
\verb{conferir}{}{}{}{}{v.t.}{Olhar algo outra vez para ver se não há erro; revisar.}{con.fe.rir}{0}
\verb{conferir}{}{}{}{}{}{Dar, conceder, outorgar.}{con.fe.rir}{0}
\verb{conferir}{}{}{}{}{}{Ser exatamente igual a outro.}{con.fe.rir}{\verboinum{18}}
\verb{confessar}{}{}{}{}{v.t.}{Contar algum segredo a alguém.}{con.fes.sar}{0}
\verb{confessar}{}{}{}{}{}{Falar seus pecados ao padre.}{con.fes.sar}{\verboinum{1}}
\verb{confessionário}{}{}{}{}{s.m.}{Local na igreja onde o padre ouve confissões.}{con.fes.si.o.ná.rio}{0}
\verb{confesso}{é}{}{}{}{adj.}{Que confessou suas culpas.}{con.fes.so}{0}
\verb{confessor}{ô}{}{}{}{s.m.}{Sacerdote que ouve a confissão de pecados.}{con.fes.sor}{0}
\verb{confessor}{ô}{}{}{}{}{Indivíduo que faz profissão de fé ou crença.}{con.fes.sor}{0}
\verb{confete}{é}{Pop.}{}{}{}{Elogio, adulação.}{con.fe.te}{0}
\verb{confete}{é}{}{}{}{s.m.}{Rodelinhas de papel colorido que os foliões, no carnaval, jogam uns nos outros.}{con.fe.te}{0}
\verb{confiado}{}{}{}{}{}{Atrevido, abusado.}{con.fi.a.do}{0}
\verb{confiado}{}{}{}{}{}{Que é digno de confiança; confiável.}{con.fi.a.do}{0}
\verb{confiado}{}{}{}{}{adj.}{Que confia, que acredita.}{con.fi.a.do}{0}
\verb{confiança}{}{Pop.}{}{}{}{Atrevimento.}{con.fi.an.ça}{0}
\verb{confiança}{}{}{}{}{}{Otimismo, esperança.}{con.fi.an.ça}{0}
\verb{confiança}{}{}{}{}{}{Crédito, fé.}{con.fi.an.ça}{0}
\verb{confiança}{}{}{}{}{s.f.}{Segurança íntima com que se realiza alguma coisa; sentimento de quem confia.}{con.fi.an.ça}{0}
\verb{confiante}{}{}{}{}{}{Que confia em si; seguro, firme, despreocupado.}{con.fi.an.te}{0}
\verb{confiante}{}{}{}{}{adj.2g.}{Que confia em outrem ou em algo; crédulo.}{con.fi.an.te}{0}
\verb{confiar}{}{}{}{}{}{Entregar pessoa ou coisa a alguém na certeza de que será bem cuidada.}{con.fi.ar}{\verboinum{6}}
\verb{confiar}{}{}{}{}{v.t.}{Ter confiança em pessoa ou coisa.}{con.fi.ar}{0}
\verb{confiável}{}{}{"-eis}{}{adj.2g.}{Em que se pode confiar; digno de confiança.}{con.fi.á.vel}{0}
\verb{confidência}{}{}{}{}{}{Confiança na discrição, sinceridade, honestidade de outrem.}{con.fi.dên.cia}{0}
\verb{confidência}{}{}{}{}{s.f.}{Informação ou revelação secreta.}{con.fi.dên.cia}{0}
\verb{confidencial}{}{}{}{}{adj.2g.}{Dito ou escrito em confidência; secreto.}{con.fi.den.ci.al}{0}
\verb{confidenciar}{}{}{}{}{v.t.}{Dizer em segredo, em confidência.}{con.fi.den.ci.ar}{\verboinum{6}}
\verb{confidente}{}{}{}{}{s.2g.}{Indivíduo a quem se confiam segredos.}{con.fi.den.te}{0}
\verb{configuração}{}{}{"-ões}{}{s.f.}{Forma exterior de um corpo; aspecto, feitio, figura.}{con.fi.gu.ra.ção}{0}
\verb{configurar}{}{}{}{}{v.t.}{Dar a forma ou a figura de; conformar.}{con.fi.gu.rar}{\verboinum{1}}
\verb{confim}{}{}{"-ins}{}{adj.2g.}{Que confina.}{con.fim}{0}
\verb{confinante}{}{}{}{}{adj.2g.}{Que confina; que tem os mesmos confins.}{con.fi.nan.te}{0}
\verb{confinar}{}{}{}{}{v.t.}{Limitar, demarcar, estar nos confins de.}{con.fi.nar}{0}
\verb{confinar}{}{}{}{}{}{Clausurar, encerrar.}{con.fi.nar}{\verboinum{1}}
\verb{confins}{}{}{}{}{s.m.pl.}{Limites de um território; fronteiras.}{con.fins}{0}
\verb{confirmação}{}{}{"-ões}{}{s.f.}{Ato ou efeito de confirmar; ratificação.}{con.fir.ma.ção}{0}
\verb{confirmação}{}{Relig.}{"-ões}{}{}{Sacramento da Igreja Católica, administrado pelo bispo, que assegura, fortalece, confirma a graça do batismo; crisma.}{con.fir.ma.ção}{0}
\verb{confirmação}{}{}{"-ões}{}{}{Parte do discurso no qual o orador desenrola as provas.}{con.fir.ma.ção}{0}
\verb{confirmar}{}{}{}{}{v.t.}{Declarar como verdadeiro.}{con.fir.mar}{0}
\verb{confirmar}{}{}{}{}{}{Comprovar, cumprir.}{con.fir.mar}{\verboinum{1}}
\verb{confiscar}{}{}{}{}{v.t.}{Apreender judicialmente; arrestar, embargar.}{con.fis.car}{\verboinum{2}}
\verb{confisco}{}{}{}{}{s.m.}{Apreensão fiscal ou judicial.}{con.fis.co}{0}
\verb{confissão}{}{}{"-ões}{}{s.f.}{Ato ou efeito de confessar.}{con.fis.são}{0}
\verb{confissão}{}{}{"-ões}{}{}{Declaração da própria fé; profissão.}{con.fis.são}{0}
\verb{confissão}{}{Relig.}{"-ões}{}{}{Declaração dos próprios pecados.}{con.fis.são}{0}
\verb{conflagração}{}{}{"-ões}{}{s.f.}{Grande incêndio.}{con.fla.gra.ção}{0}
\verb{conflagração}{}{}{"-ões}{}{}{Guerra.}{con.fla.gra.ção}{0}
\verb{conflagrar}{}{}{}{}{v.t.}{Incendiar vigorosa e extensamente.}{con.fla.grar}{0}
\verb{conflagrar}{}{Fig.}{}{}{}{Fazer ficar mais aceso, mais forte; excitar, estimular.}{con.fla.grar}{0}
\verb{conflagrar}{}{}{}{}{}{Pôr em confusão, agitação; convulsionar.}{con.fla.grar}{\verboinum{1}}
\verb{conflitar}{}{}{}{}{v.t.}{Estar em oposição; colidir.}{con.fli.tar}{\verboinum{1}}
\verb{conflito}{}{}{}{}{s.m.}{Falta de entendimento entre duas ou mais partes.}{con.fli.to}{0}
\verb{conflito}{}{}{}{}{}{Briga, guerra, luta.}{con.fli.to}{0}
\verb{conflituoso}{ô}{}{"-osos ⟨ó⟩}{"-osa ⟨ó⟩}{adj.}{Relativo a conflito.}{con.fli.tu.o.so}{0}
\verb{conflituoso}{ô}{}{"-osos ⟨ó⟩}{"-osa ⟨ó⟩}{}{Que tem caráter de conflito.}{con.fli.tu.o.so}{0}
\verb{confluência}{}{}{}{}{s.f.}{Convergência para um mesmo ponto.}{con.flu.ên.cia}{0}
\verb{confluência}{}{}{}{}{}{Ponto dessa convergência.}{con.flu.ên.cia}{0}
\verb{confluente}{}{}{}{}{adj.2g.}{Que conflui.}{con.flu.en.te}{0}
\verb{confluente}{}{}{}{}{}{Rio que vai se juntar a outro; afluente.}{con.flu.en.te}{0}
\verb{confluir}{}{}{}{}{v.t.}{Ir de pontos diferentes para um mesmo lugar.}{con.flu.ir}{\verboinum{26}}
\verb{conformação}{}{}{"-ões}{}{s.f.}{Maneira pela qual se juntam as partes de um corpo organizado; configuração.}{con.for.ma.ção}{0}
\verb{conformação}{}{}{"-ões}{}{}{Conformidade, resignação.}{con.for.ma.ção}{0}
\verb{conformado}{}{}{}{}{adj.}{Que se conforma; resignado.}{con.for.ma.do}{0}
\verb{conformar}{}{}{}{}{v.t.}{Dar forma a alguma coisa; configurar.}{con.for.mar}{0}
\verb{conformar}{}{}{}{}{v.pron.}{Aceitar algo contra a própria vontade; resignar"-se.}{con.for.mar}{\verboinum{1}}
\verb{conforme}{ó}{}{}{}{conj.}{De acordo com; como, consoante, segundo. (\textit{Conforme foi combinado, chegamos à reunião  às 2 horas.})}{con.for.me}{0}
\verb{conforme}{ó}{}{}{}{prep.}{De acordo com; como, consoante, segundo. (\textit{Chegamos à reunião às duas horas, conforme o combinado.})}{con.for.me}{0}
\verb{conforme}{ó}{}{}{}{adj.2g.}{Que tem a mesma forma; igual, idêntico. (\textit{São duas construções conformes.})}{con.for.me}{0}
\verb{conforme}{ó}{}{}{}{}{Que tem forma parecida; análogo, semelhante.}{con.for.me}{0}
\verb{conformidade}{}{}{}{}{s.f.}{Qualidade de estar de acordo com algo; concordância.}{con.for.mi.da.de}{0}
\verb{conformismo}{}{}{}{}{s.m.}{Atitude ou tendência de se aceitar uma situação incômoda ou desfavorável; resignação, passividade.}{con.for.mis.mo}{0}
\verb{confortador}{ô}{}{}{}{adj.}{Que conforta; que dá alento, força.}{con.for.ta.dor}{0}
\verb{confortar}{}{}{}{}{v.t.}{Dar forças a; fortificar.}{con.for.tar}{0}
\verb{confortar}{}{}{}{}{}{Dar ou proporcionar conforto a; tornar confortável.}{con.for.tar}{0}
\verb{confortar}{}{}{}{}{}{Consolar, animar.}{con.for.tar}{\verboinum{1}}
\verb{confortável}{}{}{"-eis}{}{adj.2g.}{Que propicia comodidade, bem"-estar.}{con.for.tá.vel}{0}
\verb{conforto}{ô}{}{}{}{s.m.}{Consolo, alívio.}{con.for.to}{0}
\verb{conforto}{ô}{}{}{}{}{Bem"-estar material; comodidade.}{con.for.to}{0}
\verb{confrade}{}{}{}{}{s.m.}{Membro de confraria.}{con.fra.de}{0}
\verb{confrade}{}{}{}{}{}{Colega, companheiro.}{con.fra.de}{0}
\verb{confranger}{ê}{}{}{}{v.t.}{Oprimir, afligir, angustiar.}{con.fran.ger}{\verboinum{16}}
\verb{confraria}{}{}{}{}{s.f.}{Associação com fins religiosos; irmandade.}{con.fra.ri.a}{0}
\verb{confraria}{}{}{}{}{}{Grupo ligado por profissão ou gostos comuns.}{con.fra.ri.a}{0}
\verb{confraternização}{}{}{"-ões}{}{s.f.}{Ato ou efeito de confraternizar; reunião de pessoas de um mesmo grupo ou profissão para comemorar algum fato.}{con.fra.ter.ni.za.ção}{0}
\verb{confraternizar}{}{}{}{}{v.t.}{Reunir"-se entre colegas para comemorar um fato, uma data.}{con.fra.ter.ni.zar}{0}
\verb{confraternizar}{}{}{}{}{}{Tratar ou viver como irmãos; unir.}{con.fra.ter.ni.zar}{0}
\verb{confraternizar}{}{}{}{}{}{Partilhar os mesmos pontos de vista, sentimentos, crenças.}{con.fra.ter.ni.zar}{\verboinum{1}}
\verb{confrontação}{}{}{"-ões}{}{s.f.}{Ato ou efeito de confrontar; confronto, comparação, cotejo.}{con.fron.ta.ção}{0}
\verb{confrontar}{}{}{}{}{v.t.}{Colocar frente a frente; afrontar, acarear.}{con.fron.tar}{0}
\verb{confrontar}{}{}{}{}{}{Fazer fronteira com; limitar"-se; confinar.}{con.fron.tar}{0}
\verb{confrontar}{}{}{}{}{}{Comparar, cotejar.}{con.fron.tar}{\verboinum{1}}
\verb{confronte}{}{}{}{}{adj.2g.}{Que está em frente; defronte.}{con.fron.te}{0}
\verb{confronto}{}{}{}{}{s.m.}{Ato ou efeito de confrontar; confrontação.}{con.fron.to}{0}
\verb{confronto}{}{}{}{}{}{Conferência, comparação, cotejo.}{con.fron.to}{0}
\verb{confronto}{}{}{}{}{}{Disputa, luta, enfrentamento.}{con.fron.to}{0}
\verb{confucionismo}{}{Hist.}{}{}{s.m.}{Doutrina ética e política elaborada pelo filósofo e teórico político chinês Confúcio, no século \textsc{v} a.C., e que foi seguida pelo império chinês do século \textsc{ii} a.C. até 1912.}{con.fu.ci.o.nis.mo}{0}
\verb{confundir}{}{}{}{}{v.t.}{Juntar de modo desordenado; misturar, baralhar.}{con.fun.dir}{0}
\verb{confundir}{}{}{}{}{}{Tomar uma coisa por outra; não distinguir.}{con.fun.dir}{0}
\verb{confundir}{}{}{}{}{v.pron.}{Equivocar"-se, perturbar"-se.}{con.fun.dir}{\verboinum{18}}
\verb{confusão}{}{}{"-ões}{}{s.f.}{Ato ou efeito de confundir.}{con.fu.são}{0}
\verb{confusão}{}{}{"-ões}{}{}{Atrapalhação, equívoco, engano.}{con.fu.são}{0}
\verb{confusão}{}{}{"-ões}{}{}{Falta de ordem; mistura, mixórdia.}{con.fu.são}{0}
\verb{confuso}{}{}{}{}{adj.}{Que se confundiu; embaraçado, perturbado.}{con.fu.so}{0}
\verb{confuso}{}{}{}{}{}{Desordenado, misturado, desconexo.}{con.fu.so}{0}
\verb{confutar}{}{}{}{}{v.t.}{Provar a falsidade; refutar, rebater, impugnar.}{con.fu.tar}{\verboinum{1}}
\verb{conga}{}{}{}{}{s.f.}{Dança popular de origem afro"-americana comum nos países da América Central, semelhante à rumba, executada por grupos formando cordões, especialmente no carnaval.}{con.ga}{0}
\verb{conga}{}{}{}{}{}{Calçado de lona e solado de borracha.}{con.ga}{0}
\verb{congada}{}{}{}{}{s.f.}{Tipo de bailado dramático que representa a coroação de um rei do Congo, muito popular em Pernambuco.}{con.ga.da}{0}
\verb{congado}{}{}{}{}{s.m.}{Congada.}{con.ga.do}{0}
\verb{congelação}{}{}{"-ões}{}{s.f.}{Congelamento.}{con.ge.la.ção}{0}
\verb{congelado}{}{}{}{}{adj.}{Que se congelou; frio, gélido.}{con.ge.la.do}{0}
\verb{congelado}{}{}{}{}{}{Diz"-se do alimento pronto e conservado em baixa temperatura.}{con.ge.la.do}{0}
\verb{congelador}{ô}{}{}{}{adj.}{Que congela.}{con.ge.la.dor}{0}
\verb{congelador}{ô}{}{}{}{s.m.}{Compartimento do refrigerador onde se faz gelo e se conservam alimentos para consumo posterior.}{con.ge.la.dor}{0}
\verb{congelamento}{}{}{}{}{s.m.}{Ato ou efeito de congelar; congelação.}{con.ge.la.men.to}{0}
\verb{congelamento}{}{}{}{}{}{Passagem de um corpo do estado líquido para o estado sólido pela ação do frio; solidificação.}{con.ge.la.men.to}{0}
\verb{congelamento}{}{}{}{}{}{Fixação de preços e salários em determinado patamar para proteger a economia em épocas de crise.}{con.ge.la.men.to}{0}
\verb{congelar}{}{}{}{}{v.t.}{Fazer um corpo passar do estado líquido para o estado sólido; solidificar.}{con.ge.lar}{0}
\verb{congelar}{}{}{}{}{}{Tornar frio; esfriar.}{con.ge.lar}{0}
\verb{congelar}{}{}{}{}{}{Fixar preços e salários por determinado tempo.}{con.ge.lar}{\verboinum{1}}
\verb{congeminar}{}{}{}{}{v.t.}{Multiplicar, redobrar, duplicar.}{con.ge.mi.nar}{0}
\verb{congeminar}{}{}{}{}{}{Pensar muito; cismar, meditar.}{con.ge.mi.nar}{\verboinum{1}}
\verb{congênere}{}{}{}{}{adj.2g.}{Do mesmo gênero; semelhante, idêntico.}{con.gê.ne.re}{0}
\verb{congênito}{}{}{}{}{adj.}{Que nasce com o indivíduo; de nascença; inato.}{con.gê.ni.to}{0}
\verb{congestão}{}{Med.}{"-ões}{}{s.f.}{Aumento do fluxo sanguíneo ou de outro líquido em determinada parte do corpo.}{con.ges.tão}{0}
\verb{congestionado}{}{}{}{}{adj.}{Que sofreu congestão.}{con.ges.ti.o.na.do}{0}
\verb{congestionado}{}{}{}{}{}{Rubro, alterado, afogueado.}{con.ges.ti.o.na.do}{0}
\verb{congestionado}{}{}{}{}{}{Diz"-se do trânsito paralisado, confuso.}{con.ges.ti.o.na.do}{0}
\verb{congestionamento}{}{}{}{}{s.m.}{Ato ou efeito de congestionar.}{con.ges.ti.o.na.men.to}{0}
\verb{congestionamento}{}{}{}{}{}{Acúmulo de veículos nas vias públicas paralisando o trânsito; engarrafamento.}{con.ges.ti.o.na.men.to}{0}
\verb{congestionar}{}{}{}{}{}{Paralisar o trânsito; engarrafar.}{con.ges.ti.o.nar}{\verboinum{1}}
\verb{congestionar}{}{}{}{}{v.t.}{Produzir congestão em.}{con.ges.ti.o.nar}{0}
\verb{conglobar}{}{}{}{}{v.t.}{Dar forma de globo; concentrar, resumir.}{con.glo.bar}{\verboinum{1}}
\verb{conglomeração}{}{}{"-ões}{}{s.f.}{Ato ou efeito de conglomerar, amontoar.}{con.glo.me.ra.ção}{0}
\verb{conglomerado}{}{}{}{}{adj.}{Que se conglomerou; reunido, agregado, coeso.}{con.glo.me.ra.do}{0}
\verb{conglomerar}{}{}{}{}{v.t.}{Reunir, aglomerar, amontoar.}{con.glo.me.rar}{\verboinum{1}}
\verb{congo}{}{}{}{}{s.m.}{Dança de origem africana apresentada nas festividades do Natal, de Nossa Senhora do Rosário e de São Benedito.}{con.go}{0}
\verb{congolês}{}{}{}{}{adj.}{Relativo ao Congo.}{con.go.lês}{0}
\verb{congolês}{}{}{}{}{s.m.}{Indivíduo natural ou habitante desse país.}{con.go.lês}{0}
\verb{congonha}{}{Bot.}{}{}{s.f.}{Nome de diversos arbustos de cujas folhas se faz chá; erva"-mate.}{con.go.nha}{0}
\verb{congraçar}{}{}{}{}{v.pron.}{Fazer amizade; tornar"-se amigo.}{con.gra.çar}{\verboinum{3}}
\verb{congraçar}{}{}{}{}{v.t.}{Harmonizar pessoas que estão em conflito; reconciliar.}{con.gra.çar}{0}
\verb{congratulação}{}{}{"-ões}{}{s.f.}{Ato ou efeito de congratular; felicitação.}{con.gra.tu.la.ção}{0}
\verb{congratular}{}{}{}{}{v.t.}{Dar os parabéns a alguém; felicitar, parabenizar.}{con.gra.tu.lar}{\verboinum{1}}
\verb{congratulatório}{}{}{}{}{adj.}{Que exprime congratulação.}{con.gra.tu.la.tó.rio}{0}
\verb{congregação}{}{}{"-ões}{}{s.f.}{Ato ou efeito de congregar; reunião, assembleia.}{con.gre.ga.ção}{0}
\verb{congregado}{}{}{}{}{adj.}{Que se congregou; reunido, agrupado.}{con.gre.ga.do}{0}
\verb{congregado}{}{}{}{}{}{Que pertence a uma congregação.}{con.gre.ga.do}{0}
\verb{congregar}{}{}{}{}{v.t.}{Formar um grupo de pessoas para um determinado fim; reunir, juntar.}{con.gre.gar}{\verboinum{5}}
\verb{congressional}{}{}{"-ais}{}{adj.2g.}{Relativo a congresso.}{con.gres.si.o.nal}{0}
\verb{congressista}{}{}{}{}{s.2g.}{Indivíduo que é membro ou participa de um congresso.}{con.gres.sis.ta}{0}
\verb{congresso}{é}{}{}{}{s.m.}{Reunião de indivíduos para discutir assuntos de interesse comum.}{con.gres.so}{0}
\verb{congresso}{é}{}{}{}{}{A Câmara Federal e o Senado reunidos.}{con.gres.so}{0}
\verb{côngrua}{}{}{}{}{s.f.}{Pensão que se concede ao pároco para seu sustento.}{côn.gru.a}{0}
\verb{congruência}{}{}{}{}{s.f.}{Conformidade de uma coisa com o fim a que se propõe; coerência, concordância.}{con.gru.ên.cia}{0}
\verb{congruente}{}{}{}{}{adj.2g.}{Em que há congruência; coerente, conforme.}{con.gru.en.te}{0}
\verb{congruente}{}{Geom.}{}{}{}{Que coincide linha por linha.}{con.gru.en.te}{0}
\verb{conhaque}{}{}{}{}{s.m.}{Aguardente de vinho branco originária da região de Cognac, na França.}{co.nha.que}{0}
\verb{conhecedor}{ô}{}{}{}{adj.}{Que conhece; entendido, perito.}{co.nhe.ce.dor}{0}
\verb{conhecer}{ê}{}{}{}{v.t.}{Ter o primeiro contato com pessoa ou coisa; ser apresentado a.}{co.nhe.cer}{0}
\verb{conhecer}{ê}{}{}{}{}{Ter informação ou experiência a respeito de pessoa ou coisa; saber, dominar.}{co.nhe.cer}{\verboinum{15}}
\verb{conhecido}{}{}{}{}{adj.}{De que se tem informação ou experiência.}{co.nhe.ci.do}{0}
\verb{conhecido}{}{}{}{}{}{Famoso, célebre, ilustre.}{co.nhe.ci.do}{0}
\verb{conhecido}{}{}{}{}{s.m.}{Indivíduo com quem se mantém relações.}{co.nhe.ci.do}{0}
\verb{conhecimento}{}{}{}{}{s.m.}{Ato ou efeito de conhecer.}{co.nhe.ci.men.to}{0}
\verb{conhecimento}{}{}{}{}{}{Informação, ideia, experiência.}{co.nhe.ci.men.to}{0}
\verb{cônico}{}{}{}{}{adj.}{Em forma de cone.}{cô.ni.co}{0}
\verb{conífera}{}{Bot.}{}{}{s.f.}{Espécie de árvore que apresenta frutos em forma de cone, como o pinheiro.}{co.ní.fe.ra}{0}
\verb{conivência}{}{}{}{}{s.f.}{Qualidade de conivente; cumplicidade, transigência.}{co.ni.vên.cia}{0}
\verb{conivente}{}{}{}{}{adj.2g.}{Que sabe das ações criminosas de alguém, mas não faz nada para impedi"-lo; cúmplice.}{co.ni.ven.te}{0}
\verb{conjetura}{}{}{}{}{s.f.}{Opinião sem fundamento certo; suposição, hipótese.}{con.je.tu.ra}{0}
\verb{conjetural}{}{}{"-ais}{}{adj.2g.}{Que se baseia em conjeturas; hipotético.}{con.je.tu.ral}{0}
\verb{conjeturar}{}{}{}{}{v.t.}{Afirmar por conjetura; presumir, supor.}{con.je.tu.rar}{0}
\verb{conjeturar}{}{}{}{}{}{Prever, antever, pressagiar.}{con.je.tu.rar}{\verboinum{1}}
\verb{conjugação}{}{}{"-ões}{}{s.f.}{Ato ou efeito de conjugar; união, junção.}{con.ju.ga.ção}{0}
\verb{conjugação}{}{Gram.}{"-ões}{}{}{Conjunto ordenado das flexões de um verbo em modo"-tempo e número"-pessoa.}{con.ju.ga.ção}{0}
\verb{conjugado}{}{}{}{}{adj.}{Que se conjugou; unido, ligado.}{con.ju.ga.do}{0}
\verb{conjugado}{}{}{}{}{s.m.}{Apartamento composto de sala e quarto reunidos em uma só peça, mais um banheiro e uma pequena cozinha; quitinete.}{con.ju.ga.do}{0}
\verb{conjugal}{}{}{"-ais}{}{adj.2g.}{Relativo a cônjuges ou a casamento.}{con.ju.gal}{0}
\verb{conjugar}{}{}{}{}{v.t.}{Juntar harmonicamente; reunir, combinar.}{con.ju.gar}{0}
\verb{conjugar}{}{}{}{}{}{Citar, dizendo ou escrevendo, de modo ordenado, as flexões de um verbo.}{con.ju.gar}{\verboinum{5}}
\verb{conjugável}{}{}{"-eis}{}{adj.2g.}{Que se pode conjugar.}{con.ju.gá.vel}{0}
\verb{cônjuge}{}{}{}{}{s.2g.}{Cada um dos esposos em relação ao outro.}{côn.ju.ge}{0}
\verb{conjunção}{}{}{"-ões}{}{s.f.}{Ato ou efeito de ligar, unir; união.}{con.jun.ção}{0}
\verb{conjunção}{}{Gram.}{"-ões}{}{}{Classe de palavra invariável que liga palavras ou orações, estabelecendo relações entre elas.}{con.jun.ção}{0}
\verb{conjuntiva}{}{Anat.}{}{}{s.f.}{Fina membrana que cobre a parte anterior do olho, ligando"-a à pálpebra.}{con.jun.ti.va}{0}
\verb{conjuntivite}{}{Med.}{}{}{s.f.}{Inflamação da conjuntiva, caracterizada por vermelhidão, coceira e secreção.}{con.jun.ti.vi.te}{0}
\verb{conjuntivo}{}{}{}{}{adj.}{Que serve para ligar, unir.}{con.jun.ti.vo}{0}
\verb{conjuntivo}{}{Gram.}{}{}{}{Que liga orações ou palavras.}{con.jun.ti.vo}{0}
\verb{conjuntivo}{}{Biol.}{}{}{}{Diz"-se do tecido que serve de sustentação a diversas estruturas do corpo e liga órgãos entre si.}{con.jun.ti.vo}{0}
\verb{conjunto}{}{}{}{}{adj.}{Que está unido a outro; combinado, conjugado.}{con.jun.to}{0}
\verb{conjunto}{}{}{}{}{s.m.}{Reunião de elementos ou partes de um todo.}{con.jun.to}{0}
\verb{conjunto}{}{}{}{}{}{Traje feminino composto de duas ou três peças combinadas.}{con.jun.to}{0}
\verb{conjuntura}{}{}{}{}{s.f.}{Conjunto de circunstâncias que determinam uma situação; quadro.}{con.jun.tu.ra}{0}
\verb{conjura}{}{}{}{}{s.f.}{Conjuração, conjuro.}{con.ju.ra}{0}
\verb{conjuração}{}{}{"-ões}{}{s.f.}{Ato ou efeito de conjurar; conspiração, sublevação, trama.}{con.ju.ra.ção}{0}
\verb{conjurado}{}{}{}{}{adj.}{Diz"-se do indivíduo que participa de uma conjuração; conspirador.}{con.ju.ra.do}{0}
\verb{conjurar}{}{}{}{}{v.t.}{Conclamar para uma conspiração; tramar, maquinar.}{con.ju.rar}{0}
\verb{conjurar}{}{}{}{}{}{Convocar, incitar, invocar.}{con.ju.rar}{0}
\verb{conjurar}{}{}{}{}{}{Esconjurar, exorcizar.}{con.ju.rar}{\verboinum{1}}
\verb{conjuro}{}{}{}{}{s.m.}{Imprecação dirigida a forças ocultas  para que obedeçam à vontade de alguém; esconjuro, exorcismo. }{con.ju.ro}{0}
\verb{conluiar}{}{}{}{}{v.t.}{Unir em conluio; tramar.}{con.lui.ar}{\verboinum{1}}
\verb{conluio}{}{}{}{}{s.m.}{Combinação entre pessoas para prejudicar outrem; maquinação, trama.}{con.lui.o}{0}
\verb{conosco}{ô}{}{}{}{pron.}{Em nossa companhia.}{co.nos.co}{0}
\verb{conotação}{}{}{"-ões}{}{s.f.}{Relação entre coisas que se comparam; implicação.}{co.no.ta.ção}{0}
\verb{conotação}{}{Gram.}{"-ões}{}{}{Sentido secundário de uma palavra que se soma à sua significação básica, graças a associações diversas em função do contexto.}{co.no.ta.ção}{0}
\verb{conotar}{}{}{}{}{v.t.}{Transmitir uma conotação; implicar.}{co.no.tar}{0}
\verb{conotar}{}{}{}{}{}{Sugerir outro sentido além do original de uma palavra.}{co.no.tar}{\verboinum{1}}
\verb{conotativo}{}{}{}{}{adj.}{Que exprime conotação.}{co.no.ta.ti.vo}{0}
\verb{conquanto}{}{}{}{}{conj.}{Relaciona ideias opositivas; ainda que, embora, posto que, se bem que. (\textit{A segunda reunião que fizemos, conquanto mais pacífica, não foi menos importante. })}{con.quan.to}{0}
\verb{conquista}{}{}{}{}{s.f.}{Ato ou efeito de conquistar.}{con.quis.ta}{0}
\verb{conquista}{}{}{}{}{}{Pessoa ou coisa que se conquistou.}{con.quis.ta}{0}
\verb{conquistador}{ô}{}{}{}{adj.}{Que conquista, triunfa; dominador.}{con.quis.ta.dor}{0}
\verb{conquistador}{ô}{}{}{}{s.m.}{Indivíduo que é dado a conquistas amorosas; namorador.}{con.quis.ta.dor}{0}
\verb{conquistar}{}{}{}{}{v.t.}{Conseguir algo por meio de luta ou de esforço; triunfar, vencer.}{con.quis.tar}{0}
\verb{conquistar}{}{}{}{}{}{Granjear amizade ou amor de alguém; cativar.}{con.quis.tar}{\verboinum{1}}
\verb{consagração}{}{}{"-ões}{}{s.f.}{Ato ou efeito de consagrar; glorificação.}{con.sa.gra.ção}{0}
\verb{consagrar}{}{Relig.}{}{}{v.t.}{Oferecer a Deus; sagrar.}{con.sa.grar}{0}
\verb{consagrar}{}{}{}{}{}{Dedicar, votar.}{con.sa.grar}{0}
\verb{consagrar}{}{}{}{}{}{Eleger, aclamar, promover.}{con.sa.grar}{\verboinum{1}}
\verb{consanguíneo}{}{}{}{}{adj.}{Que é do mesmo sangue; parente.}{con.san.guí.neo}{0}
\verb{consanguinidade}{}{}{}{}{s.f.}{Qualquer parentesco de sangue, seja por parte de pai ou por parte de mãe.}{con.san.gui.ni.da.de}{0}
\verb{consciência}{}{}{}{}{s.f.}{Capacidade de se perceber o que se passa à nossa volta; lucidez.}{cons.ci.ên.cia}{0}
\verb{consciência}{}{}{}{}{}{Faculdade racional de julgar os próprios atos; senso de responsabilidade.}{cons.ci.ên.cia}{0}
\verb{consciencioso}{ô}{}{"-osos ⟨ó⟩}{"-osa ⟨ó⟩}{adj.}{Que segue os ditames da consciência; escrupuloso, honesto, cuidadoso.}{cons.ci.en.ci.o.so}{0}
\verb{consciente}{}{}{}{}{adj.2g.}{Que percebe o que se passa à sua volta; lúcido.}{cons.ci.en.te}{0}
\verb{consciente}{}{}{}{}{}{Responsável, ciente, cônscio.}{cons.ci.en.te}{0}
\verb{conscientizar}{}{}{}{}{v.t.}{Tornar consciente; fazer sabedor.}{cons.ci.en.ti.zar}{\verboinum{1}}
\verb{cônscio}{}{}{}{}{adj.}{Que sabe bem o que deve fazer; ciente, consciente.}{côns.cio}{0}
\verb{conscrição}{}{}{"-ões}{}{s.f.}{Alistamento para o serviço militar.}{cons.cri.ção}{0}
\verb{conscrito}{}{}{}{}{adj.}{Que está alistado no serviço militar; recrutado.}{cons.cri.to}{0}
\verb{consecução}{}{}{"-ões}{}{s.f.}{Ato ou efeito de conseguir.}{con.se.cu.ção}{0}
\verb{consecutivo}{}{}{}{}{adj.}{Que se segue imediatamente a outro; imediato, sucessivo.}{con.se.cu.ti.vo}{0}
\verb{conseguinte}{}{}{}{}{adj.2g.}{Que se segue; consecutivo, sucessivo.}{con.se.guin.te}{0}
\verb{conseguinte}{}{}{}{}{}{Usado na expressão \textit{por conseguinte}: por isso; portanto, logo.}{con.se.guin.te}{0}
\verb{conseguir}{}{}{}{}{v.t.}{Obter como consequência ou resultado; alcançar, chegar a.}{con.se.guir}{\verboinum{24}}
\verb{conselheiro}{ê}{}{}{}{adj.}{Diz"-se daquele que aconselha.}{con.se.lhei.ro}{0}
\verb{conselheiro}{ê}{}{}{}{s.m.}{Membro de um conselho.}{con.se.lhei.ro}{0}
\verb{conselho}{ê}{}{}{}{s.m.}{Parecer que se dá a outrem para orientá"-lo; opinião, advertência.}{con.se.lho}{0}
\verb{conselho}{ê}{}{}{}{}{Conjunto de pessoas eleitas como corpo consultivo para dar pareceres sobre atividades públicas ou privadas.}{con.se.lho}{0}
\verb{consenso}{}{}{}{}{s.m.}{Concordância de opiniões ou ideias; acordo, unanimidade, aprovação.}{con.sen.so}{0}
\verb{consensual}{}{}{"-ais}{}{adj.2g.}{Relativo a ou que envolve consenso.}{con.sen.su.al}{0}
\verb{consentâneo}{}{}{}{}{adj.}{Conveniente, adequado, apropriado.}{con.sen.tâ.neo}{0}
\verb{consentimento}{}{}{}{}{s.m.}{Ato ou efeito de consentir; permissão, aprovação, concordância.}{con.sen.ti.men.to}{0}
\verb{consentir}{}{}{}{}{v.t.}{Não impedir; permitir, autorizar, concordar.}{con.sen.tir}{\verboinum{29}}
\verb{consequência}{}{}{}{}{s.f.}{Algo que acontece por causa de um conjunto de situações; resultado, efeito.}{con.se.quên.cia}{0}
\verb{consequência}{}{}{}{}{}{Conclusão, dedução, inferência.}{con.se.quên.cia}{0}
\verb{consequente}{}{}{}{}{adj.2g.}{Que se segue como resultado; natural, resultante.}{con.se.quen.te}{0}
\verb{consequente}{}{}{}{}{}{Lógico, coerente, racional.}{con.se.quen.te}{0}
\verb{consertar}{}{}{}{}{v.t.}{Reparar o que estava quebrado ou apresentava falhas; corrigir, arrumar, emendar.}{con.ser.tar}{\verboinum{1}}
\verb{conserto}{ê}{}{}{}{s.m.}{Ato ou efeito de consertar; reparo, remendo, arranjo.}{con.ser.to}{0}
\verb{conserva}{é}{Cul.}{}{}{s.f.}{Alimento conservado por algum processo que o preserve de alterações ou de deterioração.}{con.ser.va}{0}
\verb{conservação}{}{}{"-ões}{}{s.f.}{Ato ou efeito de conservar; manutenção, preservação.}{con.ser.va.ção}{0}
\verb{conservado}{}{}{}{}{adj.}{Que resiste à passagem do tempo; preservado.}{con.ser.va.do}{0}
\verb{conservador}{ô}{}{}{}{adj.}{Que conserva, preserva.}{con.ser.va.dor}{0}
\verb{conservador}{ô}{}{}{}{s.m.}{Indivíduo que defende ideias, valores e costumes ultrapassados e se opõe a inovações ou reformas.}{con.ser.va.dor}{0}
\verb{conservadorismo}{}{}{}{}{s.m.}{Sistema de crenças baseado no apego às tradições e no repúdio a inovações; conservantismo.}{con.ser.va.do.ris.mo}{0}
\verb{conservantismo}{}{}{}{}{s.m.}{Conservadorismo.}{con.ser.van.tis.mo}{0}
\verb{conservar}{}{}{}{}{v.t.}{Manter em bom estado, sem dano algum; preservar.}{con.ser.var}{0}
\verb{conservar}{}{}{}{}{}{Continuar a ter; sustentar, guardar.}{con.ser.var}{0}
\verb{conservar}{}{}{}{}{}{Resistir ao tempo; durar, permanecer.}{con.ser.var}{\verboinum{1}}
\verb{conservatório}{}{}{}{}{s.m.}{Escola dedicada ao ensino das belas"-artes, especialmente à música.}{con.ser.va.tó.rio}{0}
\verb{consideração}{}{}{"-ões}{}{s.f.}{Ato ou efeito de considerar; reflexão, observação.}{con.si.de.ra.ção}{0}
\verb{consideração}{}{}{"-ões}{}{}{Estima, apreço, respeito.}{con.si.de.ra.ção}{0}
\verb{considerado}{}{}{}{}{adj.}{Que se refletiu; ponderado, examinado.}{con.si.de.ra.do}{0}
\verb{considerado}{}{}{}{}{}{Estimado, prezado, respeitado.}{con.si.de.ra.do}{0}
\verb{considerando}{}{}{}{}{s.m.}{Cada uma das razões que fundamentam uma resolução ou uma sentença; argumentação, motivo.}{con.si.de.ran.do}{0}
\verb{considerar}{}{}{}{}{v.t.}{Examinar com cuidado; observar, refletir.}{con.si.de.rar}{0}
\verb{considerar}{}{}{}{}{}{Ter em alta conta; respeitar, estimar.}{con.si.de.rar}{\verboinum{1}}
\verb{considerável}{}{}{"-eis}{}{adj.2g.}{Que merece consideração; importante, respeitável.}{con.si.de.rá.vel}{0}
\verb{consignação}{}{}{"-ões}{}{s.f.}{Ato ou efeito de consignar.}{con.sig.na.ção}{0}
\verb{consignação}{}{}{"-ões}{}{}{Transação comercial que consiste na entrega da mercadoria ao comerciante, que só paga ao fornecedor se for efetuada a venda.}{con.sig.na.ção}{0}
\verb{consignar}{}{}{}{}{v.t.}{Afirmar, declarar. }{con.sig.nar}{0}
\verb{consignar}{}{}{}{}{}{Registrar, assinalar.}{con.sig.nar}{0}
\verb{consignar}{}{}{}{}{}{Enviar mercadorias a alguém para que as comercialize.}{con.sig.nar}{\verboinum{1}}
\verb{consignatário}{}{}{}{}{s.m.}{Pessoa a quem se enviam mercadorias para comercialização; destinatário.}{con.sig.na.tá.rio}{0}
\verb{consigo}{}{}{}{}{pron.}{Em sua companhia. (\textit{Quando for sair, me avise que vou consigo.})}{con.si.go}{0}
\verb{consigo}{}{}{}{}{}{De si para si. (\textit{Guarde consigo suas impressões mais íntimas.})}{con.si.go}{0}
\verb{consistência}{}{}{}{}{s.f.}{Qualidade de consistente.}{con.sis.tên.cia}{0}
\verb{consistência}{}{}{}{}{}{Firmeza, estabilidade.}{con.sis.tên.cia}{0}
\verb{consistente}{}{}{}{}{adj.2g.}{Que tem consistência; sólido, espesso, rijo.}{con.sis.ten.te}{0}
\verb{consistir}{}{}{}{}{v.t.}{Ser feito, constituído por.}{con.sis.tir}{0}
\verb{consistir}{}{}{}{}{}{Resumir"-se.}{con.sis.tir}{0}
\verb{consistir}{}{}{}{}{}{Ter base, fundar"-se.}{con.sis.tir}{\verboinum{18}}
\verb{consistório}{}{}{}{}{s.m.}{Conselho, assembleia solene.}{con.sis.tó.rio}{0}
\verb{consistório}{}{}{}{}{}{Assembleia de cardeais presidida pelo papa.}{con.sis.tó.rio}{0}
\verb{consoada}{}{}{}{}{s.f.}{Ceia da véspera do Natal ou do Ano"-Novo.}{con.so.a.da}{0}
\verb{consoante}{}{Gram.}{}{}{s.f.}{Cada um dos fonemas que se caracteriza na língua portuguesa por oferecer resistência à saída do ar no trato oral.}{con.so.an.te}{0}
\verb{consoante}{}{Gram.}{}{}{}{Cada uma das letras que representam os fonemas consoantes. (\textit{Não há palavras em português que se escrevam apenas com consoantes.})}{con.so.an.te}{0}
\verb{consoante}{}{}{}{}{conj.}{Conforme, de acordo com. (\textit{Consoante com o que vínhamos dizendo, iniciaremos uma nova etapa do trabalho.})}{con.so.an.te}{0}
\verb{consoante}{}{Desus.}{}{}{adj.2g.}{Diz"-se de palavras que apresentam sons semelhantes ou que rimam. (\textit{Todos os seus versos terminavam consoantes.})}{con.so.an.te}{0}
\verb{consociar}{}{}{}{}{v.t.}{Tornar sócio; associar.}{con.so.ci.ar}{0}
\verb{consociar}{}{}{}{}{}{Unir, conciliar.}{con.so.ci.ar}{\verboinum{6}}
\verb{consolação}{}{}{"-ões}{}{s.f.}{Ato ou efeito de consolar.}{con.so.la.ção}{0}
\verb{consolação}{}{}{"-ões}{}{}{Alívio, conforto.}{con.so.la.ção}{0}
\verb{consolação}{}{}{"-ões}{}{}{Consolo, compensação.}{con.so.la.ção}{0}
\verb{consolador}{ô}{}{}{}{adj.}{Que consola; confortador.}{con.so.la.dor}{0}
\verb{consolar}{}{}{}{}{v.t.}{Aliviar o sofrimento; confortar.}{con.so.lar}{\verboinum{1}}
\verb{console}{ó}{}{}{}{s.m.}{Parte do automóvel que fica entre os bancos dianteiros, usada para guardar objetos, como chaves, copos, telefone celular etc.  }{con.so.le}{0}
\verb{consolidação}{}{}{"-ões}{}{s.f.}{Ato ou efeito de consolidar.}{con.so.li.da.ção}{0}
\verb{consolidar}{}{}{}{}{v.t.}{Tornar sólido, consistente; reforçar, fortalecer.}{con.so.li.dar}{\verboinum{1}}
\verb{consolo}{ô}{}{}{}{s.m.}{O que consola; conforto, lenitivo, alívio.}{con.so.lo}{0}
\verb{consonância}{}{}{}{}{}{Conformidade, acordo, concordância.}{con.so.nân.cia}{0}
\verb{consonância}{}{}{}{}{s.f.}{Conjunto de sons que combinam; harmonia, rima.}{con.so.nân.cia}{0}
\verb{consonantal}{}{Gram.}{"-ais}{}{adj.2g.}{Relativo a consoante.}{con.so.nan.tal}{0}
\verb{consonantizar}{}{}{}{}{v.t.}{Transformar uma semivogal em consoante.}{con.so.nan.ti.zar}{\verboinum{1}}
\verb{consorciado}{}{}{}{}{adj.}{Participante de um consórcio.}{con.sor.ci.a.do}{0}
\verb{consorciar}{}{}{}{}{v.t.}{Unir em casamento; casar.}{con.sor.ci.ar}{0}
\verb{consorciar}{}{}{}{}{}{Ligar, associar, combinar.}{con.sor.ci.ar}{\verboinum{1}}
\verb{consórcio}{}{}{}{}{s.m.}{União conjugal; casamento.}{con.sór.cio}{0}
\verb{consórcio}{}{}{}{}{}{Associação de interesses ou de empresas.}{con.sór.cio}{0}
\verb{consórcio}{}{}{}{}{}{Associação de pessoas que assumem o compromisso de pagar prestações mensais para uma caixa comum, visando à compra  de um bem, sendo a entrega deste por sorteio ou lance.}{con.sór.cio}{0}
\verb{consorte}{ó}{}{}{}{s.2g.}{Cada um dos esposos em relação ao outro; cônjuge.}{con.sor.te}{0}
\verb{conspícuo}{}{}{}{}{adj.}{Notável pelo saber ou pela dignidade; ilustre, distinto, respeitável.}{cons.pí.cu.o}{0}
\verb{conspiração}{}{}{"-ões}{}{s.f.}{Ato ou efeito de conspirar; maquinação, trama, conluio.}{cons.pi.ra.ção}{0}
\verb{conspirador}{ô}{}{}{}{adj.}{Que conspira; maquinador, golpista.}{cons.pi.ra.dor}{0}
\verb{conspirar}{}{}{}{}{v.t.}{Planejar ações em segredo; tramar, conluiar, conjurar.}{cons.pi.rar}{\verboinum{1}}
\verb{conspurcar}{}{}{}{}{v.t.}{Macular, infamar, desonrar, corromper.}{cons.pur.car}{\verboinum{2}}
\verb{consta}{}{}{}{}{s.f.}{Notícia que passa por verdadeira; boato.}{cons.ta}{0}
\verb{constância}{}{}{}{}{s.f.}{Qualidade de constante; persistência, perseverança.}{cons.tân.cia}{0}
\verb{constante}{}{}{}{}{adj.2g.}{Que consta; mencionado, registrado.}{cons.tan.te}{0}
\verb{constante}{}{}{}{}{}{Que não muda; invariável, inalterável.}{cons.tan.te}{0}
\verb{constante}{}{}{}{}{}{Que não para; contínuo, incessante.}{cons.tan.te}{0}
\verb{constar}{}{}{}{}{v.t.}{Estar mencionado, escrito, incluído em lista, registro etc.}{cons.tar}{0}
\verb{constar}{}{}{}{}{}{Constituir"-se; consistir.}{cons.tar}{\verboinum{1}}
\verb{constatação}{}{}{"-ões}{}{s.f.}{Ato ou efeito de constatar; comprovação, confirmação.}{cons.ta.ta.ção}{0}
\verb{constatar}{}{}{}{}{v.t.}{Verificar a exatidão ou a veracidade; atestar, comprovar, certificar.}{cons.ta.tar}{\verboinum{1}}
\verb{constelação}{}{}{"-ões}{}{s.f.}{Grupo de estrelas fixas que, nas cartas celestes, se ligam por linhas imaginárias formando desenhos designados por nomes especiais.}{cons.te.la.ção}{0}
\verb{constelar}{}{}{}{}{v.t.}{Cobrir de estrelas, de constelações.}{cons.te.lar}{\verboinum{1}}
\verb{consternação}{}{}{"-ões}{}{s.f.}{Ato ou efeito de consternar; tristeza profunda; abatimento.}{cons.ter.na.ção}{0}
\verb{consternar}{}{}{}{}{v.t.}{Causar profunda tristeza; abalar, desalentar, desolar.}{cons.ter.nar}{\verboinum{1}}
\verb{constipação}{}{}{"-ões}{}{s.f.}{Dificuldade na evacuação das fezes; prisão de ventre.}{cons.ti.pa.ção}{0}
\verb{constipação}{}{}{"-ões}{}{}{Resfriado.}{cons.ti.pa.ção}{0}
\verb{constipar}{}{}{}{}{v.t.}{Causar prisão de ventre.}{cons.ti.par}{0}
\verb{constipar}{}{}{}{}{}{Causar resfriado.}{cons.ti.par}{\verboinum{1}}
\verb{constitucional}{}{}{"-ais}{}{adj.2g.}{Relativo à Constituição; legal, legítimo.}{cons.ti.tu.ci.o.nal}{0}
\verb{constitucionalismo}{}{}{}{}{s.m.}{Doutrina ou sistema político que segue o regime constitucional.}{cons.ti.tu.ci.o.na.lis.mo}{0}
\verb{constitucionalista}{}{}{}{}{adj.2g.}{Relativo ao constitucionalismo.}{cons.ti.tu.ci.o.na.lis.ta}{0}
\verb{constitucionalista}{}{}{}{}{s.2g.}{Adepto ou seguidor do constitucionalismo.}{cons.ti.tu.ci.o.na.lis.ta}{0}
\verb{constituição}{}{}{"-ões}{}{s.f.}{Ato ou efeito de constituir; composição, organização.}{cons.ti.tu.i.ção}{0}
\verb{constituição}{}{}{"-ões}{}{}{Conjunto das características físicas de um ser vivo; compleição.}{cons.ti.tu.i.ção}{0}
\verb{constituição}{}{Jur.}{"-ões}{}{}{Lei fundamental de um país que estabelece os direitos e os deveres de seus cidadãos.}{cons.ti.tu.i.ção}{0}
\verb{constituinte}{}{}{}{}{adj.2g.}{Que constitui; componente.}{cons.ti.tu.in.te}{0}
\verb{constituinte}{}{}{}{}{s.2g.}{Membro da Assembleia Legislativa encarregado de fazer ou reformar a Constituição de um país.}{cons.ti.tu.in.te}{0}
\verb{constituir}{}{}{}{}{v.t.}{Ser a parte principal; formar, compor.}{cons.ti.tu.ir}{0}
\verb{constituir}{}{}{}{}{v.pron.}{Representar, ser, tornar"-se.}{cons.ti.tu.ir}{\verboinum{26}}
\verb{constitutivo}{}{}{}{}{adj.}{Que entra na constituição; constituinte.}{cons.ti.tu.ti.vo}{0}
\verb{constranger}{ê}{}{}{}{v.t.}{Tirar a liberdade de; obrigar, subjugar, dominar.}{cons.tran.ger}{0}
\verb{constranger}{ê}{}{}{}{}{Embaraçar, incomodar, envergonhar.}{cons.tran.ger}{\verboinum{16}}
\verb{constrangido}{}{}{}{}{adj.}{Que foi forçado a fazer algo contra sua vontade; obrigado, subjugado.}{cons.tran.gi.do}{0}
\verb{constrangido}{}{}{}{}{}{Embaraçado, envergonhado, incomodado. (\textit{O amigo ficou constrangido ao ver o casal brigando.})}{cons.tran.gi.do}{0}
\verb{constrangimento}{}{}{}{}{s.m.}{Situação de embaraço; acanhamento.}{cons.tran.gi.men.to}{0}
\verb{constrangimento}{}{}{}{}{}{Coação, imposição.}{cons.tran.gi.men.to}{0}
\verb{constrição}{}{}{"-ões}{}{s.f.}{Ato ou efeito de constringir; aperto, estreitamento.}{cons.tri.ção}{0}
\verb{constringir}{}{}{}{}{v.t.}{Fazer pressão; apertar em volta; cingir, contrair.}{cons.trin.gir}{\verboinum{22}}
\verb{construção}{}{}{"-ões}{}{s.f.}{Ato ou efeito de construir; edificar.}{cons.tru.ção}{0}
\verb{construção}{}{}{"-ões}{}{}{Aquilo que foi construído; edifício, casa, prédio.}{cons.tru.ção}{0}
\verb{construir}{}{}{}{}{v.t.}{Montar a estrutura; edificar.}{cons.tru.ir}{0}
\verb{construir}{}{}{}{}{}{Organizar, compor, formar.}{cons.tru.ir}{\verboinum{26}}
\verb{construtivo}{}{}{}{}{adj.}{Que serve para construir.}{cons.tru.ti.vo}{0}
\verb{construtivo}{}{}{}{}{}{Diz"-se de crítica bem"-intencionada, positiva.}{cons.tru.ti.vo}{0}
\verb{construtor}{ô}{}{}{}{adj.}{Que constrói.}{cons.tru.tor}{0}
\verb{construtor}{ô}{}{}{}{s.m.}{Indivíduo que possui empresa de construção de imóveis.}{cons.tru.tor}{0}
\verb{consubstanciar}{}{}{}{}{v.t.}{Unir numa única substância.}{con.subs.tan.ci.ar}{0}
\verb{consubstanciar}{}{}{}{}{}{Materializar, consolidar, concretizar.}{con.subs.tan.ci.ar}{\verboinum{1}}
\verb{consuetudinário}{}{}{}{}{adj.}{Costumeiro, habitual.}{con.su.e.tu.di.ná.rio}{0}
\verb{cônsul}{}{}{cônsules}{consulesa}{s.m.}{Agente diplomático de uma nação, encarregado, em país estrangeiro, de proteger os seus concidadãos.}{côn.sul}{0}
\verb{consulado}{}{}{}{}{s.m.}{Repartição onde o cônsul exerce suas funções.}{con.su.la.do}{0}
\verb{consular}{}{}{}{}{adj.2g.}{Relativo a cônsul ou a consulado.}{con.su.lar}{0}
\verb{consulente}{}{}{}{}{adj.2g.}{Que consulta, que pede conselho.}{con.su.len.te}{0}
\verb{consulesa}{ê}{}{}{}{s.f.}{Feminino de \textit{cônsul}.}{con.su.le.sa}{0}
\verb{consulesa}{ê}{}{}{}{}{Esposa de cônsul.}{con.su.le.sa}{0}
\verb{consulta}{}{}{}{}{s.f.}{Ato ou efeito de consultar, de pedir ou dar opinião; parecer, conselho.}{con.sul.ta}{0}
\verb{consultar}{}{}{}{}{v.t.}{Buscar informações ou conselhos; sondar, inquirir.}{con.sul.tar}{\verboinum{1}}
\verb{consultivo}{}{}{}{}{adj.}{Relativo a consulta.}{con.sul.ti.vo}{0}
\verb{consultivo}{}{}{}{}{}{Diz"-se de órgão que emite parecer ou dá sugestões sem poder deliberar ou tomar decisões.}{con.sul.ti.vo}{0}
\verb{consultor}{ô}{}{}{}{s.m.}{Indivíduo que dá consultas ou emite pareceres de sua especialidade.}{con.sul.tor}{0}
\verb{consultoria}{}{}{}{}{s.f.}{Escritório ou agência especializada em dar consultas, pareceres etc.}{con.sul.to.ri.a}{0}
\verb{consultório}{}{}{}{}{s.m.}{Local onde médicos e dentistas dão consultas a seus clientes.}{con.sul.tó.rio}{0}
\verb{consumação}{}{}{"-ões}{}{s.f.}{Ato ou efeito de consumar; conclusão, término.}{con.su.ma.ção}{0}
\verb{consumação}{}{}{"-ões}{}{s.f.}{Despesa mínima de consumo de bebidas ou alimentos que o cliente é obrigado a pagar em bares, clubes e outras casas de diversão.}{con.su.ma.ção}{0}
\verb{consumado}{}{}{}{}{adj.}{Que chegou a um bom fim; completo, findo, perfeito.}{con.su.ma.do}{0}
\verb{consumar}{}{}{}{}{v.t.}{Ir até o fim; terminar, completar, acabar.}{con.su.mar}{\verboinum{1}}
\verb{consumição}{}{}{"-ões}{}{s.f.}{Ato ou efeito de consumir; destruição, aniquilamento.}{con.su.mi.ção}{0}
\verb{consumidor}{ô}{}{}{}{adj.}{Que consome; que adquire para uso próprio; freguês, comprador.}{con.su.mi.dor}{0}
\verb{consumir}{}{}{}{}{v.t.}{Gastar até o fim; fazer extinguir ou desaparecer; esgotar.}{con.su.mir}{0}
\verb{consumir}{}{}{}{}{}{Causar desgosto; mortificar, afligir.}{con.su.mir}{\verboinum{33}}
\verb{consumismo}{}{}{}{}{s.m.}{Consumo exagerado de bens ou produtos.}{con.su.mis.mo}{0}
\verb{consumista}{}{}{}{}{adj.2g.}{Relativo a consumismo.}{con.su.mis.ta}{0}
\verb{consumista}{}{}{}{}{}{Diz"-se daquele que tem o hábito de comprar exageradamente bens ou produtos.}{con.su.mis.ta}{0}
\verb{consumo}{}{}{}{}{s.m.}{Ato ou efeito de consumir; gasto, dispêndio. }{con.su.mo}{0}
\verb{consumo}{}{}{}{}{}{Utilização, pela população, dos bens produzidos pela indústria.}{con.su.mo}{0}
\verb{consunção}{}{Med.}{"-ões}{}{s.f.}{Definhamento lento e progressivo do organismo humano causado por doença.}{con.sun.ção}{0}
\verb{conta}{}{}{}{}{s.f.}{Cálculo envolvendo números; operação aritmética.}{con.ta}{0}
\verb{conta}{}{}{}{}{}{Registro de depósitos e retiradas em nome de pessoa ou empresa em um banco.}{con.ta}{0}
\verb{conta}{}{}{}{}{}{Soma de despesas a pagar em um estabelecimento comercial.}{con.ta}{0}
\verb{conta}{}{}{}{}{}{Consideração, atenção, importância.}{con.ta}{0}
\verb{conta}{}{}{}{}{}{Cada uma das bolinhas, com um buraco no centro, que se enfiam em colares, rosários etc.}{con.ta}{0}
\verb{contábil}{}{}{"-eis}{}{adj.2g.}{Relativo a contabilidade.}{con.tá.bil}{0}
\verb{contabilidade}{}{}{}{}{s.f.}{Ciência que estuda os métodos de cálculo das contas comerciais.}{con.ta.bi.li.da.de}{0}
\verb{contabilidade}{}{}{}{}{}{Seção de uma empresa encarregada do registro de entrada e saída do dinheiro.}{con.ta.bi.li.da.de}{0}
\verb{contabilista}{}{}{}{}{s.2g.}{Especialista em contabilidade; contador.}{con.ta.bi.lis.ta}{0}
\verb{contabilizar}{}{}{}{}{v.t.}{Lançar dados relativos a atividades comerciais de uma empresa em livros de registro; escriturar.}{con.ta.bi.li.zar}{\verboinum{1}}
\verb{conta"-corrente}{}{}{contas"-correntes}{}{s.f.}{Inscrição em instituição bancária que dá direito a utilizar os seus serviços.}{con.ta"-cor.ren.te}{0}
\verb{contactar}{}{}{}{}{}{Var. de \textit{contatar}.}{con.tac.tar}{0}
\verb{contacto}{}{}{}{}{}{Var. de \textit{contato}.}{con.tac.to}{0}
\verb{contado}{}{}{}{}{adj.}{Que foi computado; calculado, medido.}{con.ta.do}{0}
\verb{contado}{}{}{}{}{}{Narrado, relatado, referido.}{con.ta.do}{0}
\verb{contador}{ô}{}{}{}{adj.}{Que conta histórias; narrador.}{con.ta.dor}{0}
\verb{contador}{ô}{}{}{}{s.m.}{Indivíduo especializado em contabilidade; contabilista.}{con.ta.dor}{0}
\verb{contador}{ô}{}{}{}{}{Aparelho que mede o consumo de água, gás ou eletricidade; medidor.}{con.ta.dor}{0}
\verb{contadoria}{}{}{}{}{s.f.}{Repartição onde se conferem ou se verificam contas; tesouraria.}{con.ta.do.ri.a}{0}
\verb{contagem}{}{}{"-ens}{}{s.f.}{Ato ou efeito de contar; apuração, cômputo, soma.}{con.ta.gem}{0}
\verb{contagiante}{}{}{}{}{adj.2g.}{Que contagia, que espalha sentimentos como alegria, pessimismo etc. (\textit{A risada da minha mãe é contagiante.})}{con.ta.gi.an.te}{0}
\verb{contagiar}{}{}{}{}{v.t.}{Transmitir vírus ou doença por contato; contaminar.}{con.ta.gi.ar}{0}
\verb{contagiar}{}{}{}{}{}{Passar, comunicar sentimentos.}{con.ta.gi.ar}{\verboinum{1}}
\verb{contágio}{}{}{}{}{s.m.}{Contato que transmite doença de um indivíduo para outro.}{con.tá.gio}{0}
\verb{contágio}{}{}{}{}{}{Propagação, comunicação de ideias, costumes, sentimentos etc.}{con.tá.gio}{0}
\verb{contagioso}{ô}{}{"-osos ⟨ó⟩}{"-osa ⟨ó⟩}{adj.}{Que se transmite por contágio. (\textit{A rubéola e a catapora são doenças contagiosas.})}{con.ta.gi.o.so}{0}
\verb{conta"-gotas}{ô}{}{}{}{s.m.}{Instrumento com que se pingam as gotas de um líquido.}{con.ta"-go.tas}{0}
\verb{contaminação}{}{}{"-ões}{}{s.f.}{Ato ou efeito de contaminar; contágio, infecção.}{con.ta.mi.na.ção}{0}
\verb{contaminar}{}{}{}{}{v.t.}{Passar infecção ou doença; contagiar, infectar.}{con.ta.mi.nar}{\verboinum{1}}
\verb{contanto}{}{}{}{}{}{Usado na locução \textit{contanto que}: com a condição que; desde que.}{con.tan.to}{0}
\verb{contar}{}{}{}{}{v.t.}{Estabelecer ou verificar um número; calcular, computar.}{con.tar}{0}
\verb{contar}{}{}{}{}{}{Considerar como parte; incluir.}{con.tar}{0}
\verb{contar}{}{}{}{}{}{Ter, possuir, dispor de.}{con.tar}{0}
\verb{contar}{}{}{}{}{}{Confiar, esperar.}{con.tar}{0}
\verb{contar}{}{}{}{}{}{Levar em conta; considerar.}{con.tar}{0}
\verb{contar}{}{}{}{}{}{Narrar, relatar, referir.}{con.tar}{\verboinum{1}}
%\verb{}{}{}{}{}{}{}{}{0}
\verb{contatar}{}{}{}{}{v.t.}{Estabelecer contato; ligar, conectar.}{con.ta.tar}{\verboinum{1}}
\verb{contato}{}{}{}{}{s.m.}{Estado ou situação de um corpo tocando outro.}{con.ta.to}{0}
\verb{contato}{}{}{}{}{}{Comunicação entre pessoas; aproximação, ligação.}{con.ta.to}{0}
\verb{contêiner}{}{}{}{}{s.m.}{Grande caixa usada para acondicionar carga, facilitando seu transporte.}{con.têi.ner}{0}
\verb{contemplação}{}{}{"-ões}{}{s.f.}{Ato ou efeito de contemplar; atenção, consideração.}{con.tem.pla.ção}{0}
\verb{contemplado}{}{}{}{}{adj.}{Que se contemplou; que se deu atenção.}{con.tem.pla.do}{0}
\verb{contemplado}{}{}{}{}{}{Que foi premiado, agraciado. (\textit{Meu pai foi contemplado com um carro no sorteio.})}{con.tem.pla.do}{0}
\verb{contemplar}{}{}{}{}{v.t.}{Olhar demoradamente, com atenção.}{con.tem.plar}{0}
\verb{contemplar}{}{}{}{}{}{Conferir algo como prêmio.}{con.tem.plar}{\verboinum{1}}
\verb{contemplativo}{}{}{}{}{adj.}{Relativo a contemplação; meditativo. (\textit{No entardecer, meu avô sempre fitava o céu com um olhar contemplativo.})}{con.tem.pla.ti.vo}{0}
\verb{contemporâneo}{}{}{}{}{adj.}{Que pertence à mesma época.}{con.tem.po.râ.neo}{0}
\verb{contemporizar}{}{}{}{}{v.i.}{Chegar a um acordo; acomodar"-se às circunstâncias.}{con.tem.po.ri.zar}{0}
\verb{contemporizar}{}{}{}{}{v.t.}{Condescender, transigir.}{con.tem.po.ri.zar}{\verboinum{1}}
\verb{contenção}{}{}{"-ões}{}{s.f.}{Ato ou efeito de conter; moderação. (\textit{O governo teve que proceder à contenção das despesas públicas.})}{con.ten.ção}{0}
\verb{contencioso}{ô}{}{"-osos ⟨ó⟩}{"-osa ⟨ó⟩}{adj.}{Em que há contenda; litigioso.}{con.ten.ci.o.so}{0}
\verb{contencioso}{ô}{}{"-osos ⟨ó⟩}{"-osa ⟨ó⟩}{}{Duvidoso, incerto.}{con.ten.ci.o.so}{0}
\verb{contenda}{}{}{}{}{s.f.}{Falta de acordo; disputa, litígio, briga.}{con.ten.da}{0}
\verb{contender}{ê}{}{}{}{v.t.}{Entrar em disputa; altercar, discutir, lutar.}{con.ten.der}{\verboinum{12}}
\verb{contendor}{ô}{}{}{}{adj.}{Que contende; adversário, litigante.}{con.ten.dor}{0}
\verb{contensão}{}{}{"-ões}{}{s.f.}{Grande esforço intelectual; tensão.}{con.ten.são}{0}
\verb{contentamento}{}{}{}{}{s.m.}{Estado de quem está satisfeito; alegria, prazer.}{con.ten.ta.men.to}{0}
\verb{contentar}{}{}{}{}{v.t.}{Tornar contente, alegre; satisfazer.}{con.ten.tar}{\verboinum{1}}
\verb{contente}{}{}{}{}{adj.2g.}{Que está satisfeito; alegre, jubiloso.}{con.ten.te}{0}
\verb{contento}{}{}{}{}{}{Usado na locução \textit{a contento}: de acordo com o que se esperava; satisfatoriamente. (\textit{Esperamos ter terminado o trabalho a contento.})}{con.ten.to}{0}
\verb{conter}{ê}{}{}{}{v.t.}{Ter algo em si; encerrar, abranger.}{con.ter}{0}
\verb{conter}{ê}{}{}{}{}{Reprimir, refrear, limitar.}{con.ter}{\verboinum{39}}
\verb{conterrâneo}{}{}{}{}{adj.}{Que é da mesma pátria; compatriota.}{con.ter.râ.neo}{0}
\verb{contestação}{}{}{"-ões}{}{s.f.}{Ato ou efeito de contestar; objeção, questionamento.}{con.tes.ta.ção}{0}
\verb{contestar}{}{}{}{}{v.t.}{Colocar em discussão a validade ou a veracidade de algo; refutar, objetar.}{con.tes.tar}{\verboinum{1}}
\verb{conteste}{é}{}{}{}{adj.2g.}{Que concorda com outro em um depoimento.}{con.tes.te}{0}
\verb{conteúdo}{}{}{}{}{s.m.}{O que está contido em um recipiente.}{con.te.ú.do}{0}
\verb{conteúdo}{}{}{}{}{}{Tema, assunto de certo teor.}{con.te.ú.do}{0}
\verb{contexto}{ês}{}{}{}{s.m.}{Conjunto das partes que compõem um texto como um todo.}{con.tex.to}{0}
\verb{contexto}{ês}{}{}{}{}{Conjunto de circunstâncias que permitem compreender um fato.}{con.tex.to}{0}
\verb{contextura}{s}{}{}{}{s.f.}{Ligação entre as partes de um todo; trama, estrutura.}{con.tex.tu.ra}{0}
\verb{contido}{}{}{}{}{adj.}{Que está limitado no interior de algo; encerrado, compreendido.}{con.ti.do}{0}
\verb{contido}{}{}{}{}{}{Reprimido, refreado, coibido.}{con.ti.do}{0}
\verb{contigo}{}{}{}{}{pron.}{Em tua companhia. (\textit{Quando saíres, me avisa que vou contigo. })}{con.ti.go}{0}
\verb{contiguidade}{}{}{}{}{s.f.}{Estado de contíguo; vizinhança, proximidade, adjacência.}{con.ti.gui.da.de}{0}
\verb{contíguo}{}{}{}{}{adj.}{Que está próximo; adjacente, vizinho.}{con.tí.guo}{0}
\verb{continência}{}{}{}{}{s.f.}{Ato ou efeito de conter; moderação, comedimento.}{con.ti.nên.cia}{0}
\verb{continência}{}{}{}{}{}{Saudação militar. (\textit{Os soldados bateram continência ao capitão da tropa.})}{con.ti.nên.cia}{0}
\verb{continental}{}{}{"-ais}{}{adj.2g.}{Relativo a continente.}{con.ti.nen.tal}{0}
\verb{continental}{}{}{"-ais}{}{}{Que tem o tamanho de um continente.}{con.ti.nen.tal}{0}
\verb{continente}{}{}{}{}{adj.2g.}{Que contém algo.}{con.ti.nen.te}{0}
\verb{continente}{}{}{}{}{}{Sóbrio, casto, moderado.}{con.ti.nen.te}{0}
\verb{continente}{}{}{}{}{s.m.}{Cada uma das cinco grandes extensões de terra cercadas por oceanos.}{con.ti.nen.te}{0}
\verb{contingência}{}{}{}{}{s.f.}{Possibilidade de um evento ocorrer ou não; eventualidade, acaso.}{con.tin.gên.cia}{0}
\verb{contingente}{}{}{}{}{adj.2g.}{Que tem chance de acontecer; possível, eventual.}{con.tin.gen.te}{0}
\verb{contingente}{}{}{}{}{s.m.}{Grupo de militares; destacamento.}{con.tin.gen.te}{0}
\verb{continuação}{}{}{"-ões}{}{s.f.}{Ato ou efeito de continuar.}{con.ti.nu.a.ção}{0}
\verb{continuação}{}{}{"-ões}{}{}{Prolongamento no espaço ou no tempo; extensão, prosseguimento.}{con.ti.nu.a.ção}{0}
\verb{continuado}{}{}{}{}{adj.}{Que não tem interrupção; contínuo, repetido.}{con.ti.nu.a.do}{0}
\verb{continuador}{ô}{}{}{}{adj.}{Que dá continuidade a alguma coisa.}{con.ti.nu.a.dor}{0}
\verb{continuar}{}{}{}{}{v.t.}{Dar prosseguimento a; levar adiante.}{con.ti.nu.ar}{0}
\verb{continuar}{}{}{}{}{v.i.}{Seguir, prosseguir, perdurar.}{con.ti.nu.ar}{0}
\verb{continuar}{}{}{}{}{v.pred.}{Permanecer.}{con.ti.nu.ar}{\verboinum{1}}
\verb{continuidade}{}{}{}{}{s.f.}{Qualidade do que é contínuo.}{con.ti.nu.i.da.de}{0}
\verb{continuidade}{}{}{}{}{}{Em cinema ou televisão, coerência das imagens e do som em um roteiro.}{con.ti.nu.i.da.de}{0}
\verb{continuísmo}{}{}{}{}{s.m.}{Perpetuação de uma pessoa ou de um grupo político no poder.}{con.ti.nu.ís.mo}{0}
\verb{continuísta}{}{}{}{}{adj.2g.}{Relativo a continuísmo.}{con.ti.nu.ís.ta}{0}
\verb{continuísta}{}{}{}{}{s.2g.}{Profissional responsável pela coerência das imagens e do som em um roteiro.}{con.ti.nu.ís.ta}{0}
\verb{contínuo}{}{}{}{}{adj.}{Em que não há interrupções ou divisões.}{con.tí.nu.o}{0}
\verb{contínuo}{}{Mat.}{}{}{}{Diz"-se de uma grandeza cujas unidades não são distintas e que pode assumir qualquer valor, não necessariamente inteiro. }{con.tí.nu.o}{0}
\verb{contínuo}{}{}{}{}{s.m.}{Funcionário de escritório que faz serviços internos, entregas, transmite recados etc.; bói, \textit{office"-boy}.}{con.tí.nu.o}{0}
\verb{contista}{}{}{}{}{s.2g.}{Pessoa que escreve contos.}{con.tis.ta}{0}
\verb{conto}{}{}{}{}{s.m.}{Narrativa curta, falada ou escrita. (\textit{Dalton Trevisan é autor de vários contos.})}{con.to}{0}
\verb{conto}{}{}{}{}{}{Mentira, engodo, embuste.}{con.to}{0}
\verb{conto"-do"-vigário}{}{Pop.}{contos"-do"-vigário}{}{s.m.}{Golpe de vigaristas que consiste em ludibriar pessoas incautas, oferecendo"-lhes grandes vantagens aparentes.}{con.to"-do"-vi.gá.rio}{0}
\verb{conto"-do"-vigário}{}{Por ext.}{contos"-do"-vigário}{}{}{Qualquer embuste para tomar dinheiro aos incautos; tapeação, logro, vigarice.}{con.to"-do"-vi.gá.rio}{0}
\verb{contorção}{}{}{"-ões}{}{s.f.}{Ato de contorcer.}{con.tor.ção}{0}
\verb{contorção}{}{}{"-ões}{}{}{Movimento giratório pronunciado, de uma ou mais partes do corpo, em torno de si mesmo.}{con.tor.ção}{0}
\verb{contorcer}{ê}{}{}{}{v.t.}{Fazer movimentos giratórios pronunciados, de uma ou mais partes do corpo, em torno de si mesmo.}{con.tor.cer}{0}
\verb{contorcer}{ê}{}{}{}{}{Torcer, dobrar.}{con.tor.cer}{\verboinum{15}}
\verb{contorcionista}{}{}{}{}{s.2g.}{Pessoa que faz contorções acrobáticas.}{con.tor.ci.o.nis.ta}{0}
\verb{contornar}{}{}{}{}{v.t.}{Fazer o contorno de.}{con.tor.nar}{0}
\verb{contornar}{}{}{}{}{}{Andar em volta de.}{con.tor.nar}{0}
\verb{contornar}{}{Fig.}{}{}{}{Encontrar solução para um problema ou uma situação de difícil resolução. (\textit{Naquele momento, era importante contornarmos o clima de tensão.})}{con.tor.nar}{\verboinum{1}}
\verb{contorno}{ô}{}{}{}{s.m.}{Linha que limita exteriormente um corpo, uma superfície, uma figura. }{con.tor.no}{0}
\verb{contorno}{ô}{}{}{}{}{Volta, circunferência.}{con.tor.no}{0}
\verb{contra}{}{}{}{}{prep.}{Indica relação de oposição; em contradição com; em desfavor de.}{con.tra}{0}
\verb{contra}{}{}{}{}{adv.}{Em direção oposta de; em sentido contrário, contrariamente.}{con.tra}{0}
\verb{contra}{}{}{}{}{s.m.}{Objeção; negativa; obstáculo; oposição sistemática.}{con.tra}{0}
\verb{contra"-almirante}{}{}{contra"-almirantes}{}{}{}{con.tra"-al.mi.ran.te}{0}
\verb{contra"-almirante}{}{}{contra"-almirantes}{}{s.m.}{Oficial da Marinha que ocupa o posto acima de capitão"-de"-mar"-e"-guerra e abaixo de vice"-almirante. }{con.tra"-al.mi.ran.te}{0}
\verb{contra"-atacar}{}{}{}{}{v.t.}{Revidar um ataque a pessoa ou tropa.}{con.tra"-a.ta.car}{\verboinum{2}}
\verb{contra"-ataque}{}{}{contra"-ataques}{}{s.m.}{Ato de contra"-atacar.}{con.tra"-a.ta.que}{0}
\verb{contra"-ataque}{}{Esport.}{contra"-ataques}{}{}{Domínio da bola, de forma rápida e inesperada, sem dar tempo ao adversário de armar a defesa. }{con.tra"-a.ta.que}{0}
\verb{contrabaixista}{ch}{}{}{}{s.2g.}{Pessoa que toca contrabaixo.}{con.tra.bai.xis.ta}{0}
\verb{contrabaixo}{ch}{}{}{}{s.m.}{Instrumento musical de cordas, o maior e o mais grave da família do violino.}{con.tra.bai.xo}{0}
\verb{contrabalançar}{}{}{}{}{v.t.}{Equilibrar, igualar em peso.}{con.tra.ba.lan.çar}{0}
\verb{contrabalançar}{}{}{}{}{}{Compensar.}{con.tra.ba.lan.çar}{\verboinum{3}}
\verb{contrabandear}{}{}{}{}{v.t.}{Fazer contrabando.}{con.tra.ban.de.ar}{0}
\verb{contrabandear}{}{}{}{}{}{Introduzir clandestinamente no país mercadorias estrangeiras.}{con.tra.ban.de.ar}{\verboinum{4}}
\verb{contrabandista}{}{}{}{}{s.2g.}{Pessoa que faz contrabando.}{con.tra.ban.dis.ta}{0}
\verb{contrabando}{}{}{}{}{s.m.}{Introdução ilegal e sem pagamento de direitos de mercadorias estrangeiras no país.}{con.tra.ban.do}{0}
\verb{contração}{}{}{"-ões}{}{s.f.}{Ato de contrair; encolhimento, retração, diminuição.}{con.tra.ção}{0}
\verb{contração}{}{Med.}{"-ões}{}{}{Encurtamento ou redução, voluntário ou involuntário, do tamanho de um órgão ou de um músculo.  (\textit{A mulher percebe a chegada da hora do parto pelas muitas contrações que sente no útero.})}{con.tra.ção}{0}
\verb{contração}{}{Gram.}{"-ões}{}{}{Redução de duas ou mais sílabas, ou palavras, a uma só.}{con.tra.ção}{0}
\verb{contracapa}{}{}{}{}{s.f.}{Cada um dos lados internos da capa.}{con.tra.ca.pa}{0}
\verb{contracenar}{}{}{}{}{v.t.}{Representar, interpretar, atuar com outro(s) ator(es).}{con.tra.ce.nar}{\verboinum{1}}
\verb{contracepção}{}{}{"-ões}{}{s.f.}{Conjunto de métodos que se destinam a evitar a concepção de filhos.}{con.tra.cep.ção}{0}
\verb{contraceptivo}{}{Med.}{}{}{adj.}{Contrário à concepção; que evita a fecundação; anticoncepcional. }{con.tra.cep.ti.vo}{0}
\verb{contraceptivo}{}{}{}{}{s.m.}{Processo ou medicamento contra a concepção.}{con.tra.cep.ti.vo}{0}
\verb{contracheque}{é}{}{}{}{s.m.}{Documento fornecido pelo empregador a cada empregado, com a discriminação de seu salário bruto, deduções e eventuais acréscimos, e que o habilita a recebê"-lo; holerite.}{con.tra.che.que}{0}
\verb{contracorrente}{}{}{}{}{s.f.}{Corrente que flui contrariamente a outra.}{con.tra.cor.ren.te}{0}
\verb{contracosta}{}{}{}{}{s.f.}{A costa marítima oposta a outra, pertencente ao mesmo continente ou ilha.}{con.tra.cos.ta}{0}
\verb{contráctil}{}{}{"-eis}{}{adj.2g.}{Contrátil.}{con.trác.til}{0}
\verb{contracultura}{}{}{}{}{s.f.}{Forma engajada de cultura que questiona valores culturais vigentes, especialmente os massificados.}{con.tra.cul.tu.ra}{0}
\verb{contradança}{}{}{}{}{s.f.}{Dança em que os membros de cada casal se colocam frente a frente; quadrilha.}{con.tra.dan.ça}{0}
\verb{contradição}{}{}{"-ões}{}{s.f.}{Incoerência entre palavra e ação ou entre duas ou mais afirmações.}{con.tra.di.ção}{0}
\verb{contradição}{}{}{"-ões}{}{}{Ato ou efeito de contradizer.}{con.tra.di.ção}{0}
\verb{contradita}{}{Jur.}{}{}{s.f.}{Em um processo ou julgamento, alegação de uma das partes contra a outra; contestação.}{con.tra.di.ta}{0}
\verb{contraditar}{}{Jur.}{}{}{v.t.}{Apresentar contradita; contestar.}{con.tra.di.tar}{\verboinum{1}}
\verb{contraditório}{}{}{}{}{adj.}{Em que há contradição; incoerente.}{con.tra.di.tó.rio}{0}
\verb{contradizer}{ê}{}{}{}{v.t.}{Dizer algo contrário a; contestar.}{con.tra.di.zer}{\verboinum{41}}
\verb{contraente}{}{}{}{}{adj.2g.}{Que assume responsabilidade em um contrato.}{con.tra.en.te}{0}
\verb{contrafação}{}{}{"-ões}{}{s.f.}{Imitação fraudulenta; falsificação.}{con.tra.fa.ção}{0}
\verb{contrafazer}{ê}{}{}{}{v.t.}{Falsificar.}{con.tra.fa.zer}{\verboinum{42}}
\verb{contrafé}{}{Jur.}{}{}{s.f.}{Cópia autêntica de intimação judicial entregue à pessoa intimada.}{con.tra.fé}{0}
\verb{contrafeito}{ê}{}{}{}{adj.}{Falsificado.}{con.tra.fei.to}{0}
\verb{contrafeito}{ê}{Fig.}{}{}{}{Contra a vontade; contrariado, coagido.}{con.tra.fei.to}{0}
\verb{contrafilé}{}{}{}{}{s.m.}{Porção da carne do boi situada logo acima do filé.}{con.tra.fi.lé}{0}
\verb{contraforte}{ó}{}{}{}{s.m.}{Reforço de uma muralha.}{con.tra.for.te}{0}
\verb{contraforte}{ó}{Fig.}{}{}{}{Aquilo que protege.}{con.tra.for.te}{0}
\verb{contraforte}{ó}{}{}{}{}{Peça de couro que reforça o calcanhar do calçado.}{con.tra.for.te}{0}
\verb{contragolpe}{ó}{}{}{}{s.m.}{Golpe em sentido oposto a outro.}{con.tra.gol.pe}{0}
\verb{contragosto}{ô}{}{}{}{s.m.}{Oposição à vontade; constrangimento.}{con.tra.gos.to}{0}
\verb{contraído}{}{}{}{}{adj.}{Que se contraiu.}{con.tra.í.do}{0}
\verb{contraindicação}{}{}{contraindicações}{}{s.f.}{Ato de contraindicar.}{con.tra.in.di.ca.ção}{0}
\verb{contraindicação}{}{Med.}{contraindicações}{}{}{Qualquer condição ou sintoma que torna desaconselhável o uso de um medicamento ou a realização de uma cirurgia.}{con.tra.in.di.ca.ção}{0}
\verb{contraindicar}{}{}{}{}{v.t.}{Não aconselhar; dar indicação contrária.}{con.tra.in.di.car}{\verboinum{2}}
\verb{contrair}{}{}{}{}{v.t.}{Encolher, diminuir o volume; fazer a contração de. (\textit{O atleta contraía demais os músculos durante o exercício.})}{con.tra.ir}{0}
\verb{contrair}{}{}{}{}{}{Adquirir, passar a ter algo. (\textit{Contraiu o vírus da gripe no início do outono.})}{con.tra.ir}{0}
\verb{contrair}{}{}{}{}{}{Assumir um compromisso. (\textit{Contrair núpcias. Contrair uma dívida.})}{con.tra.ir}{\verboinum{19}}
\verb{contralto}{}{}{}{}{s.m.}{Voz feminina mais grave, entre soprano e tenor.}{con.tral.to}{0}
\verb{contralto}{}{}{}{}{}{Cantora que tem essa voz.}{con.tral.to}{0}
\verb{contramão}{}{}{"-ãos}{}{s.f.}{Direção oposta à determinada para tráfego de veículos.}{con.tra.mão}{0}
\verb{contramão}{}{}{"-ãos}{}{}{Fora do roteiro normal; de difícil acesso.}{con.tra.mão}{0}
\verb{contramarcha}{}{}{}{}{s.f.}{Marcha em sentido contrário.}{con.tra.mar.cha}{0}
\verb{contramestre}{é}{}{}{}{s.m.}{Subalterno imediato do mestre.}{con.tra.mes.tre}{0}
\verb{contraofensiva}{}{}{contraofensivas}{}{s.f.}{Ofensiva com que se procura anular o ataque do inimigo; contra"-ataque.}{con.tra.o.fen.si.va}{0}
\verb{contraordem}{ô\ldots{}ó}{}{contraordens}{}{s.f.}{Ordem que anula ou se opõe a outra anterior.}{con.tra.or.dem}{0}
\verb{contraparente}{}{}{}{}{s.2g.}{Parente remoto.}{con.tra.pa.ren.te}{0}
\verb{contraparente}{}{}{}{}{}{Parente por afinidade.}{con.tra.pa.ren.te}{0}
\verb{contraparte}{}{}{}{}{s.f.}{Parte correspondente situada em outro lugar.}{con.tra.par.te}{0}
\verb{contrapartida}{}{}{}{}{s.f.}{Compensação, correspondência.}{con.tra.par.ti.da}{0}
\verb{contrapelo}{ê}{}{}{}{s.m.}{Direção oposta à da maioria dos pelos.}{con.tra.pe.lo}{0}
\verb{contrapesar}{}{}{}{}{v.t.}{Equilibrar com contrapeso; contrabalançar, compensar.}{con.tra.pe.sar}{\verboinum{1}}
\verb{contrapeso}{ê}{}{}{}{s.m.}{Peso que serve para contrabalançar outro(s).}{con.tra.pe.so}{0}
\verb{contrapiso}{}{}{}{}{s.m.}{Espécie de revestimento feito de argamassa para nivelar pisos, sobre o qual é colocada a cobertura definitiva.}{con.tra.pi.so}{0}
\verb{contraponto}{}{Mús.}{}{}{s.m.}{Técnica de combinar duas ou mais melodias de forma harmoniosa.}{con.tra.pon.to}{0}
\verb{contrapor}{ô}{}{}{}{v.t.}{Pôr em oposição.}{con.tra.por}{0}
\verb{contrapor}{ô}{}{}{}{}{Pôr em paralelo, fazendo comparação.}{con.tra.por}{0}
\verb{contrapor}{ô}{}{}{}{v.pron.}{Opor"-se, divergir.}{con.tra.por}{\verboinum{60}}
\verb{contraposição}{}{}{"-ões}{}{s.f.}{Ato ou efeito de contrapor.}{con.tra.po.si.ção}{0}
\verb{contraproducente}{}{}{}{}{adj.2g.}{Diz"-se daquilo cujo efeito é oposto ao esperado ou ao desejável.}{con.tra.pro.du.cen.te}{0}
\verb{contrapropaganda}{}{}{}{}{s.f.}{Propaganda que veicula ideias contrárias às de outra.}{con.tra.pro.pa.gan.da}{0}
\verb{contrapropor}{}{}{}{}{v.t.}{Fazer contraproposta a.}{con.tra.pro.por}{\verboinum{60}}
\verb{contraproposta}{ó}{}{}{}{s.f.}{Proposta alternativa a uma já apresentada.}{con.tra.pro.pos.ta}{0}
\verb{contraprova}{ó}{}{}{}{s.f.}{Prova em contrário em relação a uma anterior.}{con.tra.pro.va}{0}
\verb{contrarregra}{é}{}{contrarregras ⟨é⟩}{}{s.2g.}{Funcionário encarregado de cuidar dos cenários e dos objetos de cena, marcar as entradas e as saídas dos atores etc.}{con.trar.re.gra}{0}
\verb{contrarrevolução}{}{}{contrarrevoluções}{}{s.f.}{Movimento político contrário a uma revolução anterior, procurando anular os seus efeitos.}{con.trar.re.vo.lu.ção}{0}
\verb{contrariado}{}{}{}{}{adj.}{Que se contrariou.}{con.tra.ri.a.do}{0}
\verb{contrariado}{}{}{}{}{}{Descontente, frustrado, aborrecido.}{con.tra.ri.a.do}{0}
\verb{contrariar}{}{}{}{}{v.t.}{Proceder em sentido contrário a.}{con.tra.ri.ar}{0}
\verb{contrariar}{}{}{}{}{}{Refutar, contestar.}{con.tra.ri.ar}{0}
\verb{contrariar}{}{}{}{}{}{Aborrecer, descontentar.}{con.tra.ri.ar}{\verboinum{1}}
\verb{contrariedade}{}{}{}{}{s.f.}{Qualidade de contrário.}{con.tra.ri.e.da.de}{0}
\verb{contrariedade}{}{}{}{}{}{Descontentamento, aborrecimento.}{con.tra.ri.e.da.de}{0}
\verb{contrariedade}{}{}{}{}{}{Dificuldade, contratempo, entrave.}{con.tra.ri.e.da.de}{0}
\verb{contrário}{}{}{}{}{adj.}{Oposto, inverso.}{con.trá.rio}{0}
\verb{contrário}{}{}{}{}{}{Desfavorável.}{con.trá.rio}{0}
\verb{contrário}{}{}{}{}{s.m.}{Aquilo que é oposto.}{con.trá.rio}{0}
\verb{contrário}{}{}{}{}{}{Adversário, rival ou inimigo.}{con.trá.rio}{0}
\verb{contrassenha}{}{}{contrassenhas}{}{s.f.}{Palavra ou palavras que são a resposta a uma senha.}{con.tras.se.nha}{0}
\verb{contrassenso}{}{}{contrassensos}{}{s.m.}{Fato contrário ao bom"-senso, à lógica; absurdo, disparate.}{con.tras.sen.so}{0}
\verb{contrastar}{}{}{}{}{v.t.}{Comparar, examinar, cotejar examinando as diferenças.}{con.tras.tar}{0}
\verb{contrastar}{}{}{}{}{}{Ser oposto a; opor"-se.}{con.tras.tar}{\verboinum{1}}
\verb{contraste}{}{}{}{}{s.m.}{Oposição bastante evidente entre pessoas, coisas, tonalidades de cor, ideias.}{con.tras.te}{0}
\verb{contratação}{}{}{"-ões}{}{s.f.}{Ato ou efeito de contratar.}{con.tra.ta.ção}{0}
\verb{contratador}{ô}{}{}{}{adj.}{Que contrata; contratante.}{con.tra.ta.dor}{0}
\verb{contratante}{}{}{}{}{adj.2g.}{Que contrata, que faz um contrato.}{con.tra.tan.te}{0}
\verb{contratante}{}{Jur.}{}{}{}{Que assume a condição de parte de um contrato; contraente.}{con.tra.tan.te}{0}
\verb{contratar}{}{}{}{}{v.t.}{Estabelecer algo formalmente por meio de contrato.}{con.tra.tar}{0}
\verb{contratar}{}{}{}{}{}{Adquirir bens, serviços ou mão de obra por meio de contrato.}{con.tra.tar}{\verboinum{1}}
\verb{contratempo}{}{}{}{}{s.m.}{Condição inesperada e desfavorável; obstáculo, empecilho, aborrecimento.}{con.tra.tem.po}{0}
\verb{contrátil}{}{}{"-eis}{}{adj.2g.}{Que pode contrair, encolher.}{con.trá.til}{0}
\verb{contrato}{}{}{}{}{s.m.}{Acordo formal e nos termos da lei que estabelece direitos e obrigações entre as partes.}{con.tra.to}{0}
\verb{contrato}{}{}{}{}{}{Acordo, trato, pacto.}{con.tra.to}{0}
\verb{contrato}{}{}{}{}{}{O documento oficial que contém os termos de um contrato.}{con.tra.to}{0}
\verb{contratorpedeiro}{ê}{}{}{}{s.m.}{Navio de guerra para combater ou neutralizar torpedos.}{con.tra.tor.pe.dei.ro}{0}
\verb{contratual}{}{}{"-ais}{}{adj.2g.}{Relativo a contrato.}{con.tra.tu.al}{0}
\verb{contratura}{}{}{}{}{s.f.}{Ato ou efeito de contrair, encolher; contração.}{con.tra.tu.ra}{0}
\verb{contravenção}{}{}{"-ões}{}{s.f.}{Transgressão de leis, regulamentos ou princípios.}{con.tra.ven.ção}{0}
\verb{contraveneno}{}{}{}{}{s.m.}{Substância que anula o efeito de um veneno; antídoto.}{con.tra.ve.ne.no}{0}
\verb{contraventor}{ô}{}{}{}{adj.}{Que comete contravenção; infrator, transgressor.}{con.tra.ven.tor}{0}
\verb{contravir}{}{}{}{}{v.t.}{Cometer contravenção; transgredir, infringir.}{con.tra.vir}{0}
\verb{contravir}{}{}{}{}{}{Revidar, retorquir, responder.}{con.tra.vir}{\verboinum{56}}
\verb{contribuição}{}{}{"-ões}{}{s.f.}{Ato ou efeito de contribuir.}{con.tri.bu.i.ção}{0}
\verb{contribuinte}{}{}{}{}{adj.2g.}{Que contribui; contribuidor.}{con.tri.bu.in.te}{0}
\verb{contribuinte}{}{Jur.}{}{}{s.2g.}{Indivíduo sujeito a tributação.}{con.tri.bu.in.te}{0}
\verb{contribuir}{}{}{}{}{v.t.}{Cooperar, colaborar, concorrer.}{con.tri.bu.ir}{0}
\verb{contribuir}{}{}{}{}{}{Dar a cota que lhe cabe por obrigação ou acordo.}{con.tri.bu.ir}{\verboinum{26}}
\verb{contributivo}{}{}{}{}{adj.}{Relativo a contribuição.}{con.tri.bu.ti.vo}{0}
\verb{contrição}{}{Relig.}{"-ões}{}{s.f.}{Arrependimento por haver pecado ou ofendido a Deus.}{con.tri.ção}{0}
\verb{contristar}{}{}{}{}{v.t.}{Entristecer, afligir, mortificar.}{con.tris.tar}{\verboinum{1}}
\verb{contrito}{}{}{}{}{adj.}{Em que há contrição; arrependido.}{con.tri.to}{0}
\verb{controlado}{}{}{}{}{adj.}{Sujeito a controle.}{con.tro.la.do}{0}
\verb{controlado}{}{}{}{}{}{Diz"-se de indivíduo que tem autocontrole; ponderado, comedido, moderado.}{con.tro.la.do}{0}
\verb{controlador}{ô}{}{}{}{adj.}{Que exerce controle.}{con.tro.la.dor}{0}
\verb{controlar}{}{}{}{}{v.t.}{Exercer domínio ou restrição sobre; regular.}{con.tro.lar}{0}
\verb{controlar}{}{}{}{}{}{Examinar, fiscalizar, verificar.}{con.tro.lar}{0}
\verb{controlar}{}{}{}{}{}{Exercer influência decisiva sobre a situação por meio de habilidade.}{con.tro.lar}{0}
\verb{controlar}{}{}{}{}{}{Moderar os impulsos; conter, dominar.}{con.tro.lar}{\verboinum{1}}
\verb{controle}{ô}{}{}{}{s.m.}{Ato ou efeito de controlar.}{con.tro.le}{0}
\verb{controle}{ô}{}{}{}{}{Dispositivo por meio do qual se controla uma máquina ou aparelho.}{con.tro.le}{0}
\verb{controvérsia}{}{}{}{}{s.f.}{Discussão, geralmente caracterizada por divergências, acerca de um assunto; polêmica.}{con.tro.vér.sia}{0}
\verb{controverso}{é}{}{}{}{adj.}{Em que há ou é caracterizado por controvérsia.}{con.tro.ver.so}{0}
\verb{controverso}{é}{}{}{}{}{Questionável, discutível.}{con.tro.ver.so}{0}
\verb{controverter}{ê}{}{}{}{v.t.}{Provocar controvérsia; discutir, questionar, polemizar.}{con.tro.ver.ter}{\verboinum{12}}
\verb{controvertido}{}{}{}{}{adj.}{Controverso.}{con.tro.ver.ti.do}{0}
\verb{contudo}{}{}{}{}{conj.}{Apesar de tudo; entretanto, porém, todavia.}{con.tu.do}{0}
\verb{contumácia}{}{}{}{}{s.f.}{Qualidade de contumaz; persistência.}{con.tu.má.cia}{0}
\verb{contumaz}{}{}{}{}{adj.2g.}{Persistente, obstinado, teimoso, obcecado.}{con.tu.maz}{0}
\verb{contundente}{}{}{}{}{adj.2g.}{Que contunde, que produz contusão.}{con.tun.den.te}{0}
\verb{contundente}{}{Fig.}{}{}{}{Decisivo, incisivo, categórico.}{con.tun.den.te}{0}
\verb{contundente}{}{Fig.}{}{}{}{Que magoa, que fere.}{con.tun.den.te}{0}
\verb{contundir}{}{}{}{}{v.t.}{Produzir contusão em.}{con.tun.dir}{\verboinum{18}}
\verb{conturbado}{}{}{}{}{adj.}{Agitado, perturbado.}{con.tur.ba.do}{0}
\verb{conturbar}{}{}{}{}{v.t.}{Agitar, perturbar, revoltar.}{con.tur.bar}{\verboinum{1}}
\verb{contusão}{}{}{"-ões}{}{s.f.}{Lesão produzida por choque de objeto rígido mas sem rompimento da pele ou fratura de ossos.}{con.tu.são}{0}
\verb{contuso}{}{}{}{}{adj.}{Que sofreu contusão; contundido.}{con.tu.so}{0}
\verb{conúbio}{}{}{}{}{s.m.}{Casamento, matrimônio, união.}{co.nú.bio}{0}
\verb{convalescença}{}{}{}{}{s.f.}{Ato ou efeito de convalescer.}{con.va.les.cen.ça}{0}
\verb{convalescença}{}{}{}{}{}{Estado intermediário entre o fim do quadro mais agudo de uma doença e o completo restabelecimento da saúde do paciente.}{con.va.les.cen.ça}{0}
\verb{convalescente}{}{}{}{}{adj.2g.}{Que convalesce, que se encontra em estado de convalescença.}{con.va.les.cen.te}{0}
\verb{convalescer}{ê}{}{}{}{v.t.}{Recuperar a saúde, passando, gradativamente, de doente a são.}{con.va.les.cer}{\verboinum{15}}
\verb{convenção}{}{}{"-ões}{}{s.f.}{Acordo entre partes interessadas.}{con.ven.ção}{0}
\verb{convenção}{}{}{"-ões}{}{}{Padronização de alcance internacional, regional ou restrito a um grupo sobre questões específicas.}{con.ven.ção}{0}
\verb{convenção}{}{}{"-ões}{}{}{Hábito social praticado costumeiramente.}{con.ven.ção}{0}
\verb{convenção}{}{}{"-ões}{}{}{Pacto entre grupos ou partidos políticos.}{con.ven.ção}{0}
\verb{convenção}{}{}{"-ões}{}{}{Assembleia de um partido ou grupo específico.}{con.ven.ção}{0}
\verb{convenção}{}{}{"-ões}{}{}{Assembleia convocada para criar ou modificar uma Constituição.}{con.ven.ção}{0}
\verb{convencer}{ê}{}{}{}{v.t.}{Persuadir alguém a aceitar uma ideia, tarefa, possibilidade.}{con.ven.cer}{\verboinum{15}}
\verb{convencido}{}{}{}{}{adj.}{Que se convenceu; persuadido.}{con.ven.ci.do}{0}
\verb{convencido}{}{Bras.}{}{}{}{Arrogante, presunçoso, enfatuado, pretensioso.}{con.ven.ci.do}{0}
\verb{convencimento}{}{}{}{}{s.m.}{Ato ou efeito de convencer.}{con.ven.ci.men.to}{0}
\verb{convencimento}{}{Bras.}{}{}{}{Qualidade de convencido; presunção, arrogância.}{con.ven.ci.men.to}{0}
\verb{convencional}{}{}{"-ais}{}{adj.2g.}{Relativo a convenção.}{con.ven.ci.o.nal}{0}
\verb{convencional}{}{}{"-ais}{}{}{Estabelecido como convenção.}{con.ven.ci.o.nal}{0}
\verb{convencional}{}{}{"-ais}{}{s.2g.}{Membro de uma convenção.}{con.ven.ci.o.nal}{0}
\verb{convencionalismo}{}{}{}{}{s.m.}{Qualidade de convencional.}{con.ven.ci.o.na.lis.mo}{0}
\verb{convencionalismo}{}{}{}{}{}{Apego a convenções.}{con.ven.ci.o.na.lis.mo}{0}
\verb{convencionalista}{}{}{}{}{adj.2g.}{Que tem caráter de convenção.}{con.ven.ci.o.na.lis.ta}{0}
\verb{convencionalista}{}{}{}{}{}{Baseado ou estabelecido em convenção.}{con.ven.ci.o.na.lis.ta}{0}
\verb{convencionalista}{}{}{}{}{}{Adepto do convencionalismo.}{con.ven.ci.o.na.lis.ta}{0}
\verb{convencionar}{}{}{}{}{v.t.}{Estabelecer em ou como convenção.}{con.ven.ci.o.nar}{\verboinum{1}}
\verb{conveniado}{}{}{}{}{adj.}{Que fez convênio.}{con.ve.ni.a.do}{0}
\verb{conveniar}{}{}{}{}{v.t.}{Fazer convênio, entrar em convênio.}{con.ve.ni.ar}{\verboinum{6}}
\verb{conveniência}{}{}{}{}{s.f.}{Qualidade de conveniente.}{con.ve.ni.ên.cia}{0}
\verb{conveniência}{}{}{}{}{}{Vantagem, interesse, utilidade.}{con.ve.ni.ên.cia}{0}
\verb{conveniência}{}{}{}{}{}{Normas, convenções. (Usa"-se no plural nesta acepção.)}{con.ve.ni.ên.cia}{0}
\verb{conveniente}{}{}{}{}{adj.2g.}{Que convém.}{con.ve.ni.en.te}{0}
\verb{conveniente}{}{}{}{}{}{Adequado, favorável, proveitoso, vantajoso.}{con.ve.ni.en.te}{0}
\verb{conveniente}{}{}{}{}{s.2g.}{Participante de convênio.}{con.ve.ni.en.te}{0}
\verb{convênio}{}{}{}{}{s.m.}{Acordo entre pessoas ou grupos; pacto.}{con.vê.nio}{0}
\verb{convênio}{}{}{}{}{}{Contrato entre órgão público e instituição privada para a prestação de serviços determinados.}{con.vê.nio}{0}
\verb{conventículo}{}{}{}{}{s.m.}{Reunião clandestina de conspiradores; conluio.}{con.ven.tí.cu.lo}{0}
\verb{conventilho}{}{Bras.}{}{}{s.m.}{Casa de prostituição.}{con.ven.ti.lho}{0}
\verb{convento}{}{}{}{}{s.m.}{Casa onde vive uma comunidade religiosa.}{con.ven.to}{0}
\verb{conventual}{}{}{"-ais}{}{adj.2g.}{Relativo a convento.}{con.ven.tu.al}{0}
\verb{conventual}{}{}{"-ais}{}{s.2g.}{Indivíduo que vive em um convento.}{con.ven.tu.al}{0}
\verb{convergência}{}{}{}{}{s.f.}{Ato ou efeito de convergir.}{con.ver.gên.cia}{0}
\verb{convergência}{}{Fís.}{}{}{}{Propriedade de raios luminosos ou feixe de partículas que se dirigem para um único ponto.}{con.ver.gên.cia}{0}
\verb{convergente}{}{}{}{}{adj.2g.}{Que converge.}{con.ver.gen.te}{0}
\verb{convergente}{}{}{}{}{}{Diz"-se de lente que concentra os raios luminosos em um único ponto.}{con.ver.gen.te}{0}
\verb{convergir}{}{}{}{}{v.t.}{Dirigir"-se para um mesmo ponto.}{con.ver.gir}{0}
\verb{convergir}{}{}{}{}{}{Tender para um mesmo objetivo.}{con.ver.gir}{0}
\verb{convergir}{}{}{}{}{}{Reunir, agrupar, agregar.}{con.ver.gir}{\verboinum{22}}
\verb{conversa}{é}{}{}{}{s.f.}{Troca de palavras; diálogo, colóquio.}{con.ver.sa}{0}
\verb{conversa}{é}{}{}{}{}{Diálogo formal buscando entendimento.}{con.ver.sa}{0}
\verb{conversa}{é}{Pop.}{}{}{}{Mentira, lábia.}{con.ver.sa}{0}
\verb{conversa}{é}{Relig.}{}{}{s.f.}{Mulher recolhida em convento sem professar.}{con.ver.sa}{0}
\verb{conversação}{}{}{"-ões}{}{s.f.}{Ato de conversar; conversa.}{con.ver.sa.ção}{0}
\verb{conversação}{}{}{"-ões}{}{}{Entendimentos. (Usa"-se geralmente no plural nesta acepção.)}{con.ver.sa.ção}{0}
\verb{conversador}{ô}{}{}{}{adj.}{Que gosta de conversar ou que conversa bastante.}{con.ver.sa.dor}{0}
\verb{conversa"-fiada}{é}{}{conversas"-fiadas ⟨é⟩}{}{s.2g.}{Indivíduo que não cumpre o que promete ou que gosta de contar vantagem; papo"-furado.}{con.ver.sa"-fi.a.da}{0}
\verb{conversão}{}{}{"-ões}{}{s.f.}{Ato ou efeito de converter.}{con.ver.são}{0}
\verb{conversar}{}{}{}{}{v.t.}{Falar com alguém; dialogar.}{con.ver.sar}{0}
\verb{conversar}{}{}{}{}{}{Discutir, palestrar, discorrer.}{con.ver.sar}{\verboinum{1}}
\verb{conversível}{}{}{"-eis}{}{adj.2g.}{Que se pode converter.}{con.ver.sí.vel}{0}
\verb{conversível}{}{}{"-eis}{}{}{Diz"-se de moeda ou título que se pode trocar por outros valores.}{con.ver.sí.vel}{0}
\verb{conversível}{}{}{"-eis}{}{}{Diz"-se de automóvel cuja capota é removível ou dobrável.}{con.ver.sí.vel}{0}
\verb{converso}{é}{}{}{}{adj.}{Que se converteu a uma religião; convertido.}{con.ver.so}{0}
\verb{converso}{é}{Relig.}{}{}{}{Diz"-se de indivíduo recolhido em convento sem professar.}{con.ver.so}{0}
\verb{conversor}{ô}{}{}{}{s.m.}{Aparelho eletrônico que faz a conversão de corrente elétrica ou sinal eletrônico entre diferentes modalidades.}{con.ver.sor}{0}
\verb{converter}{ê}{}{}{}{v.t.}{Transformar coisa, tipo, formato, padrão em outro; mudar.}{con.ver.ter}{0}
\verb{converter}{ê}{}{}{}{}{Fazer mudar de crença religiosa.}{con.ver.ter}{0}
\verb{converter}{ê}{}{}{}{}{Fazer mudar de ideologia, partido, modo de vida.}{con.ver.ter}{0}
\verb{converter}{ê}{}{}{}{}{Mudar de direção, entrando em outra via.}{con.ver.ter}{0}
\verb{converter}{ê}{Esport.}{}{}{}{No basquetebol, acertar um arremesso à cesta; encestar.}{con.ver.ter}{\verboinum{12}}
\verb{convés}{}{}{}{}{s.m.}{Espaço entre a proa e o mastro grande, acima dos porões.}{con.vés}{0}
\verb{convescote}{}{Bras.}{}{}{s.m.}{Piquenique.}{con.ves.co.te}{0}
\verb{convexidade}{cs}{}{}{}{s.f.}{Qualidade de convexo.}{con.ve.xi.da.de}{0}
\verb{convexidade}{cs}{}{}{}{}{Curvatura exterior de uma superfície.}{con.ve.xi.da.de}{0}
\verb{convexo}{écs}{}{}{}{adj.}{Cuja superfície é mais alta no centro do que nas bordas.}{con.ve.xo}{0}
\verb{convicção}{}{}{"-ões}{}{s.f.}{Certeza, opinião ou crença a respeito de algo.}{con.vic.ção}{0}
\verb{convicto}{}{}{}{}{adj.}{Que tem convicção; convencido.}{con.vic.to}{0}
\verb{convicto}{}{Jur.}{}{}{}{Diz"-se de réu cujo crime ou transgressão se provou.}{con.vic.to}{0}
\verb{convidado}{}{}{}{}{adj.}{Que recebeu convite.}{con.vi.da.do}{0}
\verb{convidado}{}{}{}{}{s.m.}{Indivíduo participante de festa ou cerimônia.}{con.vi.da.do}{0}
\verb{convidar}{}{}{}{}{v.t.}{Solicitar o comparecimento de.}{con.vi.dar}{0}
\verb{convidar}{}{}{}{}{}{Solicitar, requerer, instar.}{con.vi.dar}{0}
\verb{convidar}{}{}{}{}{}{Atrair, seduzir, incitar.}{con.vi.dar}{\verboinum{1}}
\verb{convidativo}{}{}{}{}{adj.}{Que convida; atraente, apetitoso.}{con.vi.da.ti.vo}{0}
\verb{convincente}{}{}{}{}{adj.2g.}{Que convence.}{con.vin.cen.te}{0}
\verb{convir}{}{}{}{}{v.t.}{Concordar, admitir.}{con.vir}{0}
\verb{convir}{}{}{}{}{}{Ser conveniente, proveitoso, interessante, útil.}{con.vir}{0}
\verb{convir}{}{}{}{}{}{Ficar bem; condizer.}{con.vir}{\verboinum{56}}
\verb{convite}{}{}{}{}{s.m.}{Ato de convidar.}{con.vi.te}{0}
\verb{convite}{}{}{}{}{}{Mensagem oral ou escrita na qual se convida alguém.}{con.vi.te}{0}
\verb{conviva}{}{}{}{}{s.2g.}{Participante de um jantar, festa, banquete.}{con.vi.va}{0}
\verb{conviva}{}{}{}{}{}{Convidado.}{con.vi.va}{0}
\verb{convivência}{}{}{}{}{s.f.}{Ato ou efeito de conviver.}{con.vi.vên.cia}{0}
\verb{convivência}{}{}{}{}{}{Trato diário, mútuo e íntimo entre pessoas; familiaridade, intimidade.}{con.vi.vên.cia}{0}
\verb{conviver}{ê}{}{}{}{v.t.}{Viver em comum, em relação de proximidade.}{con.vi.ver}{0}
\verb{conviver}{ê}{}{}{}{}{Relacionar"-se intimamente.}{con.vi.ver}{\verboinum{12}}
\verb{convívio}{}{}{}{}{s.m.}{Convivência, familiaridade, intimidade.}{con.ví.vio}{0}
\verb{convizinho}{}{}{}{}{adj.}{Que habita a vizinhança (de alguém).}{con.vi.zi.nho}{0}
\verb{convizinho}{}{Fig.}{}{}{}{Próximo, semelhante.}{con.vi.zi.nho}{0}
\verb{convocação}{}{}{"-ões}{}{s.f.}{Ato de convocar.}{con.vo.ca.ção}{0}
\verb{convocar}{}{}{}{}{v.t.}{Solicitar o comparecimento (de alguém) em caráter relativamente compulsório; mandar comparecer.}{con.vo.car}{0}
\verb{convocar}{}{}{}{}{}{Chamar para serviço militar.}{con.vo.car}{\verboinum{2}}
\verb{convosco}{ô}{}{}{}{pron.}{Contração da preposição \textit{com} e de \textit{vosco}, forma variante do pronome \textit{vós}; em vossa companhia; em relação a vós.}{con.vos.co}{0}
\verb{convulsão}{}{}{"-ões}{}{s.f.}{Forte agitação, tumulto, revolta.}{con.vul.são}{0}
\verb{convulsão}{}{Med.}{"-ões}{}{}{Forte contração muscular, involuntária, geralmente em grandes áreas, provocada por problemas no sistema nervoso central, e que pode levar à perda de consciência.}{con.vul.são}{0}
\verb{convulsionar}{}{}{}{}{v.t.}{Provocar convulsão, agitação, revolta.}{con.vul.si.o.nar}{0}
\verb{convulsionar}{}{Med.}{}{}{}{Entrar em convulsão.}{con.vul.si.o.nar}{\verboinum{1}}
\verb{convulsivo}{}{}{}{}{adj.}{Relativo a convulsão.}{con.vul.si.vo}{0}
\verb{convulsivo}{}{Fig.}{}{}{}{Diz"-se de indivíduo sujeito a fortes emoções.}{con.vul.si.vo}{0}
\verb{convulso}{}{}{}{}{adj.}{Em que há convulsão.}{con.vul.so}{0}
\verb{coonestar}{}{}{}{}{v.t.}{Dar aparência de honestidade.}{co.o.nes.tar}{\verboinum{1}}
\verb{cooperação}{}{}{"-ões}{}{s.f.}{Ato ou efeito de cooperar.}{co.o.pe.ra.ção}{0}
\verb{cooperado}{}{}{}{}{s.m.}{Indivíduo que faz parte de uma cooperativa.}{co.o.pe.ra.do}{0}
\verb{cooperador}{ô}{}{}{}{adj.}{Que coopera.}{co.o.pe.ra.dor}{0}
\verb{cooperar}{}{}{}{}{v.t.}{Trabalhar ou atuar em conjunto com outros, juntando esforços para um objetivo comum; colaborar.}{co.o.pe.rar}{\verboinum{1}}
\verb{cooperativa}{}{}{}{}{s.f.}{Sociedade juridicamente constituída para prestar serviços ou oferecer benefícios de diversas naturezas aos associados.}{co.o.pe.ra.ti.va}{0}
\verb{cooperativismo}{}{}{}{}{s.m.}{Sistema econômico baseado em cooperativas.}{co.o.pe.ra.ti.vis.mo}{0}
\verb{cooperativista}{}{}{}{}{adj.2g.}{Relativo a cooperativa.}{co.o.pe.ra.ti.vis.ta}{0}
\verb{cooperativista}{}{}{}{}{}{Adepto do cooperativismo.}{co.o.pe.ra.ti.vis.ta}{0}
\verb{cooperativo}{}{}{}{}{adj.}{Que coopera.}{co.o.pe.ra.ti.vo}{0}
\verb{cooperativo}{}{}{}{}{}{Que envolve o trabalho conjunto de vários indivíduos.}{co.o.pe.ra.ti.vo}{0}
\verb{cooptar}{}{}{}{}{v.t.}{Admitir em uma instituição, dispensando formalidades.}{co.op.tar}{0}
\verb{cooptar}{}{}{}{}{}{Trazer associados; aliciar, atrair.}{co.op.tar}{\verboinum{1}}
\verb{coordenação}{}{}{"-ões}{}{s.f.}{Ato ou efeito de coordenar.}{co.or.de.na.ção}{0}
\verb{coordenação}{}{}{"-ões}{}{}{Setor ou equipe de determinada instituição ou projeto responsável pelo gerenciamento, distribuição e controle das tarefas.}{co.or.de.na.ção}{0}
\verb{coordenação}{}{Gram.}{"-ões}{}{}{Processo gramatical em que unidades linguísticas funcionalmente equivalentes são ligadas em sequência.}{co.or.de.na.ção}{0}
\verb{coordenada}{}{}{}{}{s.f.}{Grandeza que serve para determinar a posição de um objeto no espaço ou em uma superfície.}{co.or.de.na.da}{0}
\verb{coordenado}{}{}{}{}{adj.}{Que se coordenou ou que é eficientemente coordenado.}{co.or.de.na.do}{0}
\verb{coordenador}{ô}{}{}{}{adj.}{Que coordena.}{co.or.de.na.dor}{0}
\verb{coordenar}{}{}{}{}{v.t.}{Organizar, dispor, arranjar.}{co.or.de.nar}{\verboinum{1}}
\verb{coordenativo}{}{}{}{}{adj.}{Relativo a coordenação.}{co.or.de.na.ti.vo}{0}
\verb{coordenativo}{}{Gram.}{}{}{}{Diz"-se de conjunção que estabelece relação de coordenação entre constituintes.}{co.or.de.na.ti.vo}{0}
\verb{coorte}{ó}{}{}{}{s.f.}{Na Roma antiga, tropa de infantaria.}{co.or.te}{0}
\verb{coorte}{ó}{Por ext.}{}{}{}{Conjunto de pessoas armadas; tropa.}{co.or.te}{0}
\verb{coorte}{ó}{Fig.}{}{}{}{Multidão.}{co.or.te}{0}
\verb{copa}{ó}{}{}{}{s.f.}{Aposento, geralmente anexo à cozinha, em que se fazem as refeições.}{co.pa}{0}
\verb{copa}{ó}{}{}{}{}{Parte superior de uma árvore, com seus ramos e folhas.}{co.pa}{0}
\verb{copa}{ó}{}{}{}{}{Parte superior do chapéu.}{co.pa}{0}
\verb{copa}{ó}{}{}{}{}{Conjunto dos utensílios de mesa; baixela.}{co.pa}{0}
\verb{copa}{ó}{Desus.}{}{}{}{Copo, taça.}{co.pa}{0}
\verb{copado}{}{}{}{}{adj.}{Diz"-se de árvore que apresenta copa abundante.}{co.pa.do}{0}
\verb{copaíba}{}{Bot.}{}{}{s.f.}{Árvore de flores brancas e boa madeira, de que se extrai óleo com propriedades medicinais.}{co.pa.í.ba}{0}
\verb{copal}{}{}{"-ais}{}{adj.2g.}{Diz"-se das resinas extraídas de certas árvores tropicais e usadas na preparação de verniz.}{co.pal}{0}
\verb{copar}{}{}{}{}{v.i.}{Formar copa (árvore).}{co.par}{\verboinum{1}}
\verb{coparticipante}{}{}{coparticipantes}{}{s.2g.}{Indivíduo que participa de algo com outra(s) pessoa(s); sócio.}{co.par.ti.ci.pan.te}{0}
\verb{coparticipar}{}{}{}{}{v.t.}{Fazer parte de algo com outra(s) pessoa(s).}{co.par.ti.ci.par}{\verboinum{1}}
\verb{copas}{ó}{}{}{}{s.f.pl.}{Um dos quatro naipes do baralho, representado por um coração vermelho.}{co.pas}{0}
\verb{copázio}{}{}{}{}{s.m.}{Copo grande.}{co.pá.zio}{0}
\verb{copeiro}{ê}{}{}{}{s.m.}{Indivíduo responsável pela copa de uma casa ou restaurante, e que serve à mesa.}{co.pei.ro}{0}
\verb{copernício}{}{Quím.}{}{}{s.m.}{Elemento químico artificial, superpesado e radioativo, também conhecido como eka-mercúrio, conhecido até 2009 por `'unúmbio'' (Uub). \elemento{112}{(277)}{Cn}.}{co.per.ní.cio}{0}
\verb{cópia}{}{}{}{}{s.f.}{Transcrição idêntica de um texto.}{có.pia}{0}
\verb{cópia}{}{}{}{}{}{Reprodução de uma obra pictográfica.}{có.pia}{0}
\verb{cópia}{}{}{}{}{}{Imitação, reprodução, falsificação, plágio.}{có.pia}{0}
\verb{copiador}{ô}{}{}{}{adj.}{Que copia; copista.}{co.pi.a.dor}{0}
\verb{copiadora}{ô}{}{}{}{s.f.}{Máquina que faz fotocópias de documentos impressos.}{co.pi.a.do.ra}{0}
\verb{copiar}{}{}{}{}{v.t.}{Fazer cópia de; transcrever, reproduzir.}{co.pi.ar}{0}
\verb{copiar}{}{}{}{}{}{Imitar, plagiar, falsificar.}{co.pi.ar}{\verboinum{6}}
\verb{copidescar}{}{}{}{}{v.t.}{Fazer o copidesque de.}{co.pi.des.car}{\verboinum{2}}
\verb{copidesque}{é}{}{}{}{s.m.}{Revisão de texto para impressão.}{co.pi.des.que}{0}
\verb{copidesque}{é}{}{}{}{}{Profissional que faz essa revisão.}{co.pi.des.que}{0}
\verb{copiloto}{ô}{}{copilotos ⟨ô⟩}{}{s.m.}{Piloto que auxilia ou substitui o piloto, o comandante da aeronave.}{co.pi.lo.to}{0}
\verb{copioso}{ô}{}{"-osos ⟨ó⟩}{"-osa ⟨ó⟩}{adj.}{Que é formado de um grande número de pessoas ou coisas; numeroso.}{co.pi.o.so}{0}
\verb{copioso}{ô}{}{"-osos ⟨ó⟩}{"-osa ⟨ó⟩}{}{Que aparece em grande quantidade; abundante, farto.}{co.pi.o.so}{0}
\verb{copista}{}{}{}{}{adj.2g.}{Que copia textos à mão.}{co.pis.ta}{0}
\verb{copla}{ó}{}{}{}{s.f.}{Pequena composição poética, geralmente em quadras, para ser cantada.}{co.pla}{0}
\verb{copo}{ó}{}{}{}{s.m.}{Recipiente cilíndrico próprio para beber.}{co.po}{0}
\verb{copo"-de"-leite}{ó}{Bot.}{copos"-de"-leite ⟨ó⟩}{}{s.m.}{Planta aquática e ornamental, de flores brancas e suavemente perfumadas.}{co.po"-de"-lei.te}{0}
\verb{coprodução}{}{}{coproduções}{}{s.f.}{Ação de coproduzir.}{co.pro.du.ção}{0}
\verb{coprodução}{}{}{coproduções}{}{}{Produção realizada em conjunto com outra(s) pessoa(s), entidade(s), empresa(s).}{co.pro.du.ção}{0}
%\verb{}{}{}{}{}{}{}{}{0}
\verb{coproduzir}{}{}{}{}{v.t.}{Produzir algo em conjunto com outra(s) pessoa(s).}{co.pro.du.zir}{\verboinum{21}}
\verb{copropriedade}{}{}{copropriedades}{}{s.f.}{Propriedade comum a várias pessoas; condomínio.}{co.pro.pri.e.da.de}{0}
\verb{coproprietário}{}{}{coproprietários}{}{s.m.}{Proprietário com outras pessoas; condômino.}{co.pro.pri.e.tá.rio}{0}
\verb{cópula}{}{}{}{}{s.f.}{União, ligação.}{có.pu.la}{0}
\verb{cópula}{}{}{}{}{}{O ato sexual; coito.}{có.pu.la}{0}
\verb{copular}{}{}{}{}{v.t.}{Juntar, unir.}{co.pu.lar}{0}
\verb{copular}{}{}{}{}{}{Manter relação sexual.}{co.pu.lar}{\verboinum{1}}
\verb{copulativo}{}{}{}{}{adj.}{Que une, que liga.}{co.pu.la.ti.vo}{0}
\verb{copulativo}{}{}{}{}{}{Relativo a ato sexual.}{co.pu.la.ti.vo}{0}
\verb{coque}{ó}{}{}{}{s.m.}{Leve pancada na cabeça com os nós dos dedos.}{co.que}{0}
\verb{coque}{ó}{}{}{}{}{Penteado em que os cabelos são enrolados e fixados por grampos.}{co.que}{0}
\verb{coque}{ó}{}{}{}{}{O resto do carvão mineral depois de ser elevado a uma temperatura muito alta.}{co.que}{0}
\verb{coqueiral}{}{}{"-ais}{}{s.m.}{Plantação de coqueiros.}{co.quei.ral}{0}
\verb{coqueiro}{ê}{}{}{}{s.m.}{Palmeira de tronco comprido que produz o coco, fruto comestível de largo emprego industrial.}{co.quei.ro}{0}
%\verb{}{}{}{}{}{}{}{}{0}
\verb{coqueluche}{}{}{}{}{s.f.}{Doença infecciosa que se manifesta por uma tosse violenta e contagiosa.}{co.que.lu.che}{0}
\verb{coqueluche}{}{Pop.}{}{}{}{Mania, moda.}{co.que.lu.che}{0}
\verb{coquete}{é}{}{}{}{adj.}{Diz"-se de mulher sedutora ou leviana.}{co.que.te}{0}
\verb{coquetel}{é}{}{"-éis}{}{s.m.}{Bebida feita com a mistura de outras.}{co.que.tel}{0}
\verb{coquetel}{é}{}{"-éis}{}{}{Reunião social com aperitivos finos.}{co.que.tel}{0}
\verb{coquetel}{é}{}{"-éis}{}{}{Combinação de remédios.}{co.que.tel}{0}
\verb{coqueteleira}{ê}{}{}{}{s.f.}{Recipiente alongado, com tampa, destinado a misturar os ingredientes de um coquetel.}{co.que.te.lei.ra}{0}
\verb{coquetismo}{}{}{}{}{s.m.}{Qualidade de coquete; garbosidade.}{co.que.tis.mo}{0}
\verb{cor}{ô}{}{}{}{s.f.}{Impressão que o olho recebe da luz que os objetos refletem.}{cor}{0}
\verb{cor}{ô}{}{}{}{}{A pigmentação da pele.}{cor}{0}
\verb{cor}{ó}{Desus.}{}{}{s.m.}{Coração.}{cor}{0}
\verb{cor}{ó}{}{}{}{}{Usado na expressão \textit{de cor}: de memória.}{cor}{0}
\verb{cor}{ó}{}{}{}{}{Usado na expressão \textit{de cor e salteado}: com grande facilidade.}{cor}{0}
\verb{coração}{}{}{"-ões}{}{s.m.}{Órgão vital que bombeia o sangue.}{co.ra.ção}{0}
\verb{coração}{}{}{"-ões}{}{}{Essa parte do corpo, considerada o lugar onde surgem os sentimentos.}{co.ra.ção}{0}
\verb{coração}{}{}{"-ões}{}{}{Centro, âmago.}{co.ra.ção}{0}
\verb{corado}{}{}{}{}{adj.}{Que tem cor.}{co.ra.do}{0}
\verb{corado}{}{}{}{}{}{Que tem as faces avermelhadas.}{co.ra.do}{0}
\verb{coradouro}{ô}{}{}{}{s.m.}{Ato de corar roupa.}{co.ra.dou.ro}{0}
\verb{coradouro}{ô}{}{}{}{}{Local ao ar livre, batido pelo sol, onde se cora a roupa.}{co.ra.dou.ro}{0}
\verb{coragem}{}{}{"-ens}{}{s.f.}{Esforço que leva a não recuar; firmeza, bravura.}{co.ra.gem}{0}
\verb{corajoso}{ô}{}{"-osos ⟨ó⟩}{"-osa ⟨ó⟩}{adj.}{Que tem coragem; destemido, intrépido, valente.}{co.ra.jo.so}{0}
\verb{coral}{}{Zool.}{"-ais}{}{s.m.}{Nome comum a diversos animais celenterados, geralmente de cor vermelha, que vivem nos mares quentes, a pouca profundidade e são responsáveis pela formação de recifes e atóis. }{co.ral}{0}
\verb{coral}{}{}{"-ais}{}{adj.2g.}{Relativo a coro, que forma coro, que canta em coro.}{co.ral}{0}
\verb{coral}{}{Fig.}{"-ais}{}{}{A cor vermelha desses animais. }{co.ral}{0}
\verb{coral}{}{}{"-ais}{}{s.m.}{Canto em coro.}{co.ral}{0}
\verb{coral}{}{Zool.}{"-ais}{}{}{Forma reduzida de \textit{cobra"-coral}.}{co.ral}{0}
\verb{coralino}{}{}{}{}{adj.}{Que tem a cor vermelho"-amarelada.}{co.ra.li.no}{0}
\verb{coramina}{}{}{}{}{s.f.}{Substância usada como estimulante cardíaco ou respiratório.}{co.ra.mi.na}{0}
\verb{corante}{}{}{}{}{adj.2g.}{Diz"-se de substância que dá cor.}{co.ran.te}{0}
\verb{Corão}{}{Relig.}{}{}{s.m.}{O livro sagrado dos muçulmanos; Alcorão.}{Co.rão}{0}
\verb{corar}{}{}{}{}{v.t.}{Dar cor; colorir.}{co.rar}{0}
\verb{corar}{}{}{}{}{v.i.}{Ruborizar"-se.}{co.rar}{\verboinum{1}}
\verb{corbelha}{é}{}{}{}{s.f.}{Cesta, em geral de vime ou madeira, com arranjos de flores ou frutas.}{cor.be.lha}{0}
\verb{corça}{ó}{}{}{}{s.f.}{A fêmea do corço.}{cor.ça}{0}
\verb{corcel}{é}{}{"-éis}{}{s.m.}{Cavalo muito veloz; cavalo corredor.}{cor.cel}{0}
\verb{corço}{ô}{Zool.}{}{}{s.m.}{Pequeno veado de galhas curtas.}{cor.ço}{0}
\verb{corcova}{ó}{}{}{}{s.f.}{Elevação natural nas costas de certos animais.}{cor.co.va}{0}
\verb{corcovado}{}{}{}{}{adj.}{Que tem corcova.}{cor.co.va.do}{0}
\verb{corcovar}{}{}{}{}{v.t.}{Arquear o corpo; curvar.}{cor.co.var}{0}
\verb{corcovar}{}{}{}{}{v.i.}{Formar corcova; tornar"-se corcunda.}{cor.co.var}{0}
\verb{corcovar}{}{}{}{}{}{Dar pinotes (diz"-se de cavalgadura).}{cor.co.var}{\verboinum{1}}
\verb{corcovear}{}{}{}{}{v.i.}{Corcovar.}{cor.co.ve.ar}{\verboinum{4}}
\verb{corcovo}{ô}{}{"-s ⟨ó⟩}{}{s.m.}{Salto que o cavalo dá, arqueando o dorso.}{cor.co.vo}{0}
\verb{corcunda}{}{}{}{}{s.f.}{Calombo nas costas.}{cor.cun.da}{0}
\verb{corcunda}{}{}{}{}{s.2g.}{Diz"-se de indivíduo com essa deformidade.}{cor.cun.da}{0}
\verb{corda}{ó}{}{}{}{s.f.}{Conjunto de muitos fios unidos e torcidos, próprios para amarrar.}{cor.da}{0}
\verb{corda}{ó}{}{}{}{}{Fio de tripa de animal ou de outro material, para uso em instrumentos musicais.}{cor.da}{0}
\verb{corda}{ó}{}{}{}{}{Mola que movimenta alguma coisa.}{cor.da}{0}
\verb{corda}{ó}{}{}{}{}{Corda vocal.}{cor.da}{0}
\verb{cordado}{}{Zool.}{}{}{s.m.}{Espécime dos cordados, filo animal dotado de uma notocorda, fendas branquiais na faringe e um cordão nervoso, ao qual pertencem todos os vertebrados.}{cor.da.do}{0}
\verb{cordado}{}{Zool.}{}{}{adj.}{Relativo aos cordados.}{cor.da.do}{0}
\verb{cordame}{}{}{}{}{s.m.}{Conjunto de cordas ou cabos.}{cor.da.me}{0}
\verb{cordão}{}{}{"-ões}{}{s.m.}{Corda fina.}{cor.dão}{0}
\verb{cordão}{}{}{"-ões}{}{}{Corrente que se usa ao pescoço.}{cor.dão}{0}
\verb{cordão}{}{}{"-ões}{}{}{Cadarço.}{cor.dão}{0}
\verb{cordato}{}{}{}{}{adj.}{Que se põe de acordo.}{cor.da.to}{0}
\verb{cordato}{}{}{}{}{}{Que tem bom"-senso; prudente, sensato.}{cor.da.to}{0}
\verb{cordeiro}{ê}{}{}{}{s.m.}{Filhote de ovelha.}{cor.dei.ro}{0}
\verb{cordel}{é}{}{"-éis}{}{s.m.}{Corda muito fina.}{cor.del}{0}
\verb{cordel}{é}{}{"-éis}{}{}{Folheto de literatura de cordel.}{cor.del}{0}
\verb{cor"-de"-rosa}{ô\ldots{}ó}{}{}{}{adj.2g.}{Que é vermelho"-claro; rosa, rosado, róseo. (\textit{A garota usava um vestido de fitas cor"-de"-rosa.})}{cor"-de"-ro.sa}{0}
\verb{cor"-de"-rosa}{ô\ldots{}ó}{}{}{}{s.m.}{Essa cor. (\textit{O cor"-de"-rosa predomina na decoração do seu apartamento.})}{cor"-de"-ro.sa}{0}
\verb{cor"-de"-rosa}{ô\ldots{}ó}{Fig.}{}{}{adj.2g.}{Feliz, sem problemas. (\textit{Vida cor"-de"-rosa.})}{cor"-de"-ro.sa}{0}
\verb{cordial}{}{}{"-ais}{}{adj.2g.}{Que é afetuoso, afável.}{cor.di.al}{0}
\verb{cordial}{}{}{"-ais}{}{}{Que é sincero, franco.}{cor.di.al}{0}
\verb{cordial}{}{}{"-ais}{}{s.m.}{Medicamento ou bebida que fortalece ou conforta.}{cor.di.al}{0}
\verb{cordilheira}{ê}{}{}{}{s.f.}{Cadeia de montanhas.}{cor.di.lhei.ra}{0}
\verb{cordoalha}{}{}{}{}{s.f.}{Conjunto de cordas; cordame.}{cor.do.a.lha}{0}
\verb{cordoaria}{}{}{}{}{s.f.}{Fábrica de cordas.}{cor.do.a.ri.a}{0}
\verb{cordovão}{}{}{}{}{s.m.}{Couro de cabra curtido e preparado especialmente para calçado.}{cor.do.vão}{0}
\verb{cordoveias}{}{}{}{}{s.f.pl.}{As veias jugulares e os tendões do pescoço.}{cor.do.vei.as}{0}
\verb{cordura}{}{}{}{}{s.f.}{Qualidade do que é cordato; sensatez, boas maneiras.}{cor.du.ra}{0}
\verb{coreano}{}{}{}{}{adj.}{Relativo à Coreia do Norte ou à Coreia do Sul.}{co.re.a.no}{0}
\verb{coreano}{}{}{}{}{s.m.}{Indivíduo natural ou habitante de qualquer um desses países.}{co.re.a.no}{0}
\verb{coreano}{}{}{}{}{}{A língua falada na Coreia do Norte e na Coreia do Sul.}{co.re.a.no}{0}
\verb{coreia}{é}{}{}{}{s.f.}{Na Grécia antiga, dança acompanhada de cantos.}{co.rei.a}{0}
\verb{coreia}{é}{}{}{}{}{Bailado, dança.}{co.rei.a}{0}
\verb{coreia}{é}{Med.}{}{}{}{Distúrbio encefálico caracterizado por movimentos musculares anormais e espontâneos, sugerindo uma dança.}{co.rei.a}{0}
\verb{coreografia}{}{}{}{}{s.f.}{Arte de compor movimentos de dança.}{co.re.o.gra.fi.a}{0}
\verb{coreografia}{}{}{}{}{}{O conjunto desses movimentos.}{co.re.o.gra.fi.a}{0}
\verb{coreográfico}{}{}{}{}{adj.}{Relativo a coreografia.}{co.re.o.grá.fi.co}{0}
\verb{coreógrafo}{}{}{}{}{s.m.}{Especialista em coreografia.}{co.re.ó.gra.fo}{0}
\verb{corresponsável}{}{}{corresponsáveis}{}{adj.2g.}{Que é responsável por algo com outra(s) pessoa(s).}{cor.res.pon.sá.vel}{0}
\verb{coreto}{ê}{}{}{}{s.m.}{Construção geralmente redonda, para a apresentação de bandas de música.}{co.re.to}{0}
\verb{corréu}{}{}{corréus}{corré}{s.m.}{Indivíduo que, com outro(s), responde a processo criminal.}{cor.réu}{0}
\verb{coriáceo}{}{}{}{}{adj.}{Que tem a consistência do couro.}{co.ri.á.ceo}{0}
\verb{coriáceo}{}{}{}{}{}{Que lembra couro.}{co.ri.á.ceo}{0}
\verb{corifeu}{}{}{}{}{s.m.}{Regente do coro, o antigo teatro grego.}{co.ri.feu}{0}
\verb{corifeu}{}{}{}{}{}{Chefe, líder.}{co.ri.feu}{0}
\verb{coriscar}{}{}{}{}{v.i.}{Brilhar como corisco, faiscar.}{co.ris.car}{0}
\verb{coriscar}{}{}{}{}{v.t.}{Lançar, dardejar.}{co.ris.car}{\verboinum{2}}
\verb{corisco}{}{}{}{}{s.m.}{Faísca elétrica da atmosfera, acompanhada ou não de trovão; raio.}{co.ris.co}{0}
\verb{corista}{}{}{}{}{adj.2g.}{Diz"-se de indivíduo que canta em coro.}{co.ris.ta}{0}
\verb{corista}{}{}{}{}{}{Diz"-se de indivíduo que participa do coro, cantado ou falado, de uma ópera, e que simultaneamente representa como personagem de segundo plano.}{co.ris.ta}{0}
\verb{coriza}{}{}{}{}{s.f.}{Saída de líquido do nariz, durante o resfriado.}{co.ri.za}{0}
\verb{corja}{ó}{}{}{}{s.f.}{Grupo de pessoas desonestas ou ladrões que obedecem a um chefe; súcia, quadrilha.}{cor.ja}{0}
\verb{cornada}{}{}{}{}{s.f.}{Golpe dado pelo animal, investido com os cornos; chifrada.}{cor.na.da}{0}
\verb{córnea}{}{}{}{}{s.f.}{Pele transparente e curva dos olhos.}{cór.nea}{0}
\verb{cornear}{}{}{}{}{v.t.}{Ferir com os chifres; chifrar.}{cor.ne.ar}{\verboinum{4}}
\verb{córneo}{}{}{}{}{adj.}{Relativo a corno.}{cór.neo}{0}
\verb{córneo}{}{}{}{}{}{Constituído principalmente por queratina.}{cór.neo}{0}
\verb{córneo}{}{}{}{}{}{Da natureza de ou semelhante a um corno.}{cór.neo}{0}
\verb{córneo}{}{}{}{}{}{Duro como o corno dos animais.}{cór.neo}{0}
\verb{córneo}{}{}{}{}{}{Feito com o corno dos animais.}{cór.neo}{0}
\verb{córner}{}{}{}{}{s.m.}{Cada um dos cantos de um campo de futebol.}{cór.ner}{0}
\verb{córner}{}{}{}{}{}{Falta cobrada desse ponto; escanteio.}{cór.ner}{0}
\verb{corneta}{ê}{}{}{}{s.f.}{Instrumento musical de sopro, formado por um tubo comprido que se alarga até a boca, por onde sai o som; trombeta.}{cor.ne.ta}{0}
\verb{corneteiro}{ê}{}{}{}{adj.}{Diz"-se de indivíduo que toca corneta.}{cor.ne.tei.ro}{0}
\verb{cornetim}{}{}{"-ins}{}{s.m.}{Corneta muito aguda.}{cor.ne.tim}{0}
\verb{cornetim}{}{}{"-ins}{}{}{Indivíduo que toca essa corneta.}{cor.ne.tim}{0}
\verb{cornífero}{}{}{}{}{adj.}{Que tem chifres.}{cor.ní.fe.ro}{0}
\verb{cornija}{}{}{}{}{s.f.}{Ornamento saliente na parte superior de uma parede, de uma porta, de um móvel etc. }{cor.ni.ja}{0}
\verb{corno}{ô}{}{}{}{s.m.}{Prolongamento de osso que alguns animais têm sobre a cabeça; chifre.}{cor.no}{0}
\verb{corno}{ô}{Pop.}{}{}{}{Homem a quem uma mulher trai com outro.}{cor.no}{0}
\verb{cornucópia}{}{Mit.}{}{}{s.f.}{Vaso em forma de chifre, com frutas e flores que dele extravasam, antigo símbolo da fertilidade, riqueza, abundância, e que hoje simboliza a agricultura e o comércio.}{cor.nu.có.pia}{0}
\verb{cornudo}{}{}{}{}{adj.}{Que tem corno; chifrudo.}{cor.nu.do}{0}
\verb{coro}{ô}{}{}{}{s.m.}{Conjunto de pessoas que cantam juntas.}{co.ro}{0}
\verb{coro}{ô}{}{}{}{}{Lugar, na igreja, onde ficam essas pessoas.}{co.ro}{0}
\verb{coroa}{}{}{}{}{s.f.}{Objeto que rodeia a cabeça, usado como sinal de poder e dignidade.}{co.ro.a}{0}
\verb{coroa}{}{}{}{}{}{Arranjo de flores em forma de coroa.}{co.ro.a}{0}
\verb{coroa}{}{}{}{}{}{Lado da moeda onde aparece o valor.}{co.ro.a}{0}
\verb{coroa}{}{}{}{}{}{Círculo luminoso em volta de um astro.}{co.ro.a}{0}
\verb{coroa}{}{Pop.}{}{}{s.2g.}{Indivíduo que já não é jovem.}{co.ro.a}{0}
\verb{coroação}{}{}{"-ões}{}{s.f.}{Rito em que se coroa alguém.}{co.ro.a.ção}{0}
\verb{coroação}{}{Fig.}{"-ões}{}{}{O desfecho grandioso de algo; remate perfeito.}{co.ro.a.ção}{0}
\verb{coroado}{}{}{}{}{adj.}{Que tem coroa.}{co.ro.a.do}{0}
\verb{coroado}{}{Fig.}{}{}{}{Premiado, laureado.}{co.ro.a.do}{0}
\verb{coroado}{}{Fig.}{}{}{}{Rematado, concluído.}{co.ro.a.do}{0}
\verb{coroado}{}{}{}{}{}{Indivíduo dos coroados, tribos indígenas de várias partes do Brasil, assim denominados por usarem a cabeça raspada à maneira de coroa.}{co.ro.a.do}{0}
\verb{coroamento}{}{}{}{}{s.m.}{Coroação.}{co.ro.a.men.to}{0}
\verb{coroamento}{}{}{}{}{}{Ornato ou remate que coroa um edifício.}{co.ro.a.men.to}{0}
\verb{coroar}{}{}{}{}{v.t.}{Pôr coroa em.}{co.ro.ar}{0}
\verb{coroar}{}{}{}{}{}{Dar acabamento.}{co.ro.ar}{0}
\verb{coroar}{}{}{}{}{}{Premiar.}{co.ro.ar}{0}
\verb{coroar}{}{}{}{}{}{Aclamar como rei.}{co.ro.ar}{\verboinum{7}}
\verb{coroca}{ó}{Pop.}{}{}{adj.2g.}{Que perdeu parte da capacidade mental por causa da idade; caduco.}{co.ro.ca}{0}
\verb{corografia}{}{}{}{}{s.f.}{Estudo ou descrição geográfica de um país, região, província ou município.}{co.ro.gra.fi.a}{0}
\verb{corográfico}{}{}{}{}{adj.}{Relativo à corografia.}{co.ro.grá.fi.co}{0}
\verb{coroinha}{}{}{}{}{s.m.}{Menino que ajuda o padre nos atos religiosos.}{co.ro.i.nha}{0}
\verb{corola}{ó}{}{}{}{s.f.}{Conjunto das pétalas de uma flor.}{co.ro.la}{0}
\verb{corolário}{}{}{}{}{s.m.}{Verdade que decorre de outra, que é sua consequência necessária ou continuação natural.}{co.ro.lá.rio}{0}
\verb{coronária}{}{Anat.}{}{}{s.f.}{Cada uma das artérias que irriga o coração.}{co.ro.ná.ria}{0}
\verb{coronário}{}{}{}{}{adj.}{Relativo a coroa ou coroação.}{co.ro.ná.rio}{0}
\verb{coronel}{é}{}{"-éis}{}{s.m.}{Oficial superior da Aeronáutica e do Exército.}{co.ro.nel}{0}
\verb{coronel}{é}{}{"-éis}{}{}{Chefe político no interior.}{co.ro.nel}{0}
\verb{coronel"-aviador}{é\ldots{}ô}{}{coronéis"-aviadores ⟨ô⟩}{}{s.m.}{Posto da hierarquia militar imediatamente abaixo do de brigadeiro e imediatamente acima do de tenente"-coronel. }{co.ro.nel"-a.vi.a.dor}{0}
\verb{coronel"-aviador}{é\ldots{}ô}{}{coronéis"-aviadores ⟨ô⟩}{}{}{Militar que ocupa esse posto.}{co.ro.nel"-a.vi.a.dor}{0}
\verb{coronha}{}{}{}{}{s.f.}{A parte das espingardas e de outras armas de fogo, geralmente de madeira, onde se encaixa o cano, e por onde são empunhadas.}{co.ro.nha}{0}
\verb{coronhada}{}{}{}{}{s.f.}{Golpe com coronha.}{co.ro.nha.da}{0}
\verb{corpaço}{}{}{}{}{s.m.}{Grande corpo.}{cor.pa.ço}{0}
\verb{corpanzil}{}{}{"-zis}{}{s.m.}{Corpaço.}{cor.pan.zil}{0}
\verb{corpete}{é}{}{}{}{s.m.}{Jaqueta justa e curta.}{cor.pe.te}{0}
\verb{corpete}{é}{}{}{}{}{Sutiã.}{cor.pe.te}{0}
\verb{corpinho}{}{}{}{}{s.m.}{Corpo pequeno.}{cor.pi.nho}{0}
\verb{corpinho}{}{}{}{}{}{Corpete.}{cor.pi.nho}{0}
\verb{corpo}{ô}{}{}{}{}{Porção de matéria que tem extensão e forma. (\textit{Aos poucos a massa amorfa ia tomando corpo nas mãos do escultor.})}{cor.po}{0}
\verb{corpo}{ô}{}{}{}{}{O cadáver do ser humano ou do animal. (\textit{O corpo ficou estendido na rua por várias  horas.})}{cor.po}{0}
\verb{corpo}{ô}{}{}{}{s.m.}{Parte física do homem ou do animal. (\textit{É preciso reconhecer que o homem não tem somente as necessidades do corpo, mas também as suas necessidades culturais. })}{cor.po}{0}
\verb{corpo}{ô}{}{}{}{}{A parte que dá sustento ao conjunto. (\textit{Não eram rachaduras superficiais, o acidente atingiu o corpo do edifício.})}{cor.po}{0}
\verb{corpo}{ô}{}{}{}{}{Grupo de pessoas que trabalham para a mesma instituição; corporação. (\textit{É preciso ter um sentimento de corpo entre os funcionários da empresa.})}{cor.po}{0}
\verb{corpo"-a"-corpo}{ô\ldots{}ô}{}{}{}{s.m.}{Luta corporal, de corpo contra corpo.}{cor.po"-a"-cor.po}{0}
\verb{corporação}{}{}{"-ões}{}{s.f.}{Organização em que os profissionais seguem regras especiais e trabalham em serviços de utilidade pública.}{cor.po.ra.ção}{0}
\verb{corporal}{}{}{}{}{adj.2g.}{Relativo ao corpo.}{cor.po.ral}{0}
\verb{corporativismo}{}{}{}{}{s.m.}{Defesa dos próprios interesses por parte de uma categoria profissional.}{cor.po.ra.ti.vis.mo}{0}
\verb{corporativista}{}{}{}{}{adj.2g.}{Relativo ao corporativismo.}{cor.po.ra.ti.vis.ta}{0}
\verb{corporativista}{}{}{}{}{}{Diz"-se de partidário do corporativismo.}{cor.po.ra.ti.vis.ta}{0}
\verb{corporativo}{}{}{}{}{adj.}{Relativo ou próprio de uma corporação.}{cor.po.ra.ti.vo}{0}
\verb{corpóreo}{}{}{}{}{adj.}{Relativo a corpo; corporal.}{cor.pó.re.o}{0}
\verb{corporificar}{}{}{}{}{v.t.}{Dar corpo, materialidade, transformar em algo concreto.}{cor.po.ri.fi.car}{0}
\verb{corporificar}{}{}{}{}{}{Reunir em um corpo elementos dispersos.}{cor.po.ri.fi.car}{\verboinum{2}}
\verb{corpulência}{}{}{}{}{s.f.}{Grandeza ou grossura do corpo.}{cor.pu.lên.cia}{0}
\verb{corpulento}{}{}{}{}{adj.}{Que tem corpo grande.}{cor.pu.len.to}{0}
\verb{corpulento}{}{}{}{}{}{Volumoso.}{cor.pu.len.to}{0}
\verb{corpúsculo}{}{}{}{}{s.m.}{Corpo muito pequeno, diminuto.}{cor.pús.cu.lo}{0}
\verb{correame}{}{}{}{}{s.m.}{Conjunto de correias.}{cor.re.a.me}{0}
\verb{correão}{}{}{"-ões}{}{s.m.}{Correia grande, usada para atar alguma coisa. }{cor.re.ão}{0}
\verb{correção}{}{}{ões}{}{s.f.}{Ato ou efeito de corrigir.}{cor.re.ção}{0}
\verb{correção}{}{}{ões}{}{}{Qualidade do que está correto.}{cor.re.ção}{0}
\verb{correção}{}{}{ões}{}{}{Castigo, corretivo.}{cor.re.ção}{0}
\verb{correcional}{}{}{"-ais}{}{adj.2g.}{Que se refere à correção.}{cor.re.ci.o.nal}{0}
\verb{corre"-corre}{ó\ldots{}ó}{Pop.}{corres"-corres \textit{ou} corre"-corres ⟨ó\ldots{}ó⟩}{}{s.m.}{Pressa para fazer algo; correria, lufa"-lufa.}{cor.re"-cor.re}{0}
\verb{corredeira}{ê}{}{}{}{s.f.}{Trecho do curso de um rio onde as águas correm velozmente, por inclinação do terreno.}{cor.re.dei.ra}{0}
\verb{corrediço}{}{}{}{}{adj.}{Que corre com facilidade; liso, corredio.}{cor.re.di.ço}{0}
\verb{corredio}{}{}{}{}{adj.}{Corrediço.}{cor.re.di.o}{0}
\verb{corredor}{ô}{}{}{}{adj.}{Que corre, muito ou em competições.}{cor.re.dor}{0}
\verb{corredor}{ô}{}{}{}{s.m.}{Passagem estreita e comprida no interior de um edifício.}{cor.re.dor}{0}
\verb{corredor}{ô}{}{}{}{}{Qualquer passagem estreita e longa.}{cor.re.dor}{0}
\verb{correeiro}{ê}{}{}{}{s.m.}{Pessoa que fabrica ou vende correias ou outros objetos de couro.  }{cor.re.ei.ro}{0}
\verb{corregedor}{ô}{}{}{}{s.m.}{Magistrado cuja função é supervisionar o andamento da Justiça.}{cor.re.ge.dor}{0}
\verb{corregedoria}{}{Jur.}{}{}{s.f.}{Cargo ou jurisdição de corregedor; corretoria. }{cor.re.ge.do.ri.a}{0}
\verb{córrego}{}{}{}{}{s.m.}{Ribeiro pequeno e estreito; riacho.}{cór.re.go}{0}
\verb{correia}{ê}{}{}{}{s.f.}{Tira, geralmente de couro, usada para atar, prender ou cingir.}{cor.rei.a}{0}
\verb{correição}{}{}{"-ões}{}{}{Marcha de formigas em fila.}{cor.rei.ção}{0}
\verb{correição}{}{}{"-ões}{}{}{Aparição, em determinada época, de numerosas formigas e outros insetos.  }{cor.rei.ção}{0}
\verb{correição}{}{Jur.}{"-ões}{}{}{Distrito sob a alçada de um juiz; comarca. }{cor.rei.ção}{0}
\verb{correição}{}{Jur.}{"-ões}{}{}{Ofício exercido pelo corregedor.}{cor.rei.ção}{0}
\verb{correição}{}{}{"-ões}{}{s.f.}{Ato ou efeito de corrigir; correção.}{cor.rei.ção}{0}
\verb{correio}{ê}{}{}{}{s.m.}{Instituição pública que se encarrega de levar e trazer a correspondência.}{cor.rei.o}{0}
\verb{correio}{ê}{Por ext.}{}{}{}{Prédio em que funciona essa instituição.}{cor.rei.o}{0}
\verb{correlação}{}{}{"-ões}{}{s.f.}{Relação mútua entre duas coisas; correspondência.}{cor.re.la.ção}{0}
\verb{correlacionar}{}{}{}{}{v.t.}{Estabelecer correlação entre.}{cor.re.la.ci.o.nar}{\verboinum{1}}
\verb{correlativo}{}{}{}{}{adj.}{Em que há correlação; correlato.}{cor.re.la.ti.vo}{0}
\verb{correlato}{}{}{}{}{adj.}{Correlativo.}{cor.re.la.to}{0}
\verb{correligionário}{}{}{}{}{adj.}{Que compartilha com outrem da mesma religião, partido ou doutrina.}{cor.re.li.gi.o.ná.rio}{0}
\verb{correligionário}{}{}{}{}{s.m.}{Essa pessoa.}{cor.re.li.gi.o.ná.rio}{0}
\verb{corrente}{}{}{}{}{s.f.}{Conjunto de argolas encadeadas. (\textit{A corrente que prendia a âncora ao navio começava a enferrujar.})}{cor.ren.te}{0}
\verb{corrente}{}{}{}{}{s.f.}{Movimento da água quando vai numa só direção. (\textit{As correntes marítimas muitas vezes desviam os navegadores de seu curso.})}{cor.ren.te}{0}
\verb{corrente}{}{}{}{}{}{Adorno que se utiliza em volta do pescoço, até o colo; colar. (\textit{Com o dinheiro em mãos, comprou a corrente  de prata que desejava.})}{cor.ren.te}{0}
\verb{corrente}{}{Por ext.}{}{}{}{Qualquer movimento de líquidos ou de gases que vá numa direção fixa. (\textit{Uma corrente de ar frio entrava pela janela.})}{cor.ren.te}{0}
\verb{corrente}{}{Fig.}{}{}{adj.2g.}{Que é conhecido de todos, que já se estendeu a todos. (\textit{O segredo já era uma notícia corrente. Calças com cintura baixa eram a moda corrente daquele tempo.})}{cor.ren.te}{0}
\verb{corrente}{}{}{}{}{}{Na passagem do tempo, refere"-se ao momento atual. (\textit{É na situação corrente que temos de pensar com maior cautela.})}{cor.ren.te}{0}
\verb{corrente}{}{Por ext.}{}{}{}{Qualquer peça que se use para prender um relógio ao pulso. (\textit{A corrente de couro do relógio havia se rompido.})}{cor.ren.te}{0}
\verb{corrente}{}{}{}{}{}{Diz"-se da água ou outro líquido que se renova naturalmente. (\textit{Era um aquário de água corrente.})}{cor.ren.te}{0}
\verb{correnteza}{ê}{}{}{}{s.f.}{Curso das águas de um rio ou de mar, forte e contínuo; corrente.}{cor.ren.te.za}{0}
\verb{correntio}{}{}{}{}{adj.}{Corrediço.}{cor.ren.ti.o}{0}
\verb{correntista}{}{Bras.}{}{}{s.2g.}{Pessoa que possui conta corrente em um banco.}{cor.ren.tis.ta}{0}
\verb{correr}{ê}{}{}{}{v.i.}{Deslocar"-se com grande velocidade. (\textit{Ele foi de carro, avião, a pé; correu até alcançá"-los.})}{cor.rer}{0}
\verb{correr}{ê}{}{}{}{}{Passar com facilidade; deslizar.}{cor.rer}{0}
\verb{correr}{ê}{}{}{}{v.t.}{Mover"-se por toda a extensão; percorrer, viajar.}{cor.rer}{\verboinum{12}}
\verb{correria}{}{}{}{}{s.f.}{Corrida desordenada; corre"-corre, tumulto.}{cor.re.ri.a}{0}
\verb{correria}{}{}{}{}{}{Grande pressa; urgência, azáfama.}{cor.re.ri.a}{0}
\verb{correspondência}{}{}{}{}{s.f.}{Ato de uma coisa ter relação com outra.}{cor.res.pon.dên.cia}{0}
\verb{correspondência}{}{}{}{}{}{Troca de cartas ou telegramas.}{cor.res.pon.dên.cia}{0}
\verb{correspondência}{}{}{}{}{}{Conjunto de cartas ou telegramas.}{cor.res.pon.dên.cia}{0}
\verb{correspondente}{}{}{}{}{adj.2g.}{Que corresponde.}{cor.res.pon.den.te}{0}
\verb{correspondente}{}{}{}{}{s.2g.}{Indivíduo que escreve para um jornal ou periódico, ou os representa em determinado lugar. }{cor.res.pon.den.te}{0}
\verb{corresponder}{ê}{}{}{}{v.t.}{Dar a resposta esperada a alguma coisa.}{cor.res.pon.der}{0}
\verb{corresponder}{ê}{}{}{}{}{Ter relação com alguma coisa; conformar"-se.}{cor.res.pon.der}{0}
\verb{corresponder}{ê}{}{}{}{}{Ser tanto quanto outra coisa.}{cor.res.pon.der}{0}
\verb{corresponder}{ê}{}{}{}{v.pron.}{Trocar carta com alguém.}{cor.res.pon.der}{\verboinum{12}}
\verb{corretagem}{}{}{}{}{s.f.}{Profissão, serviço ou comissão de corretor.}{cor.re.ta.gem}{0}
\verb{corretar}{}{Bras.}{}{}{v.i.}{Trabalhar como corretor.}{cor.re.tar}{\verboinum{1}}
\verb{corretivo}{}{}{}{}{adj.}{Que corrige ou serve para corrigir.}{cor.re.ti.vo}{0}
\verb{corretivo}{}{}{}{}{s.m.}{Castigo, punição aplicada a alguém que comete uma falta.}{cor.re.ti.vo}{0}
\verb{correto}{é}{}{}{}{adj.}{Certo, sem erros.}{cor.re.to}{0}
\verb{correto}{é}{}{}{}{}{Honesto, íntegro.}{cor.re.to}{0}
\verb{corretor}{}{}{}{}{s.m.}{Pessoa encarregada de corrigir alguma coisa.}{cor.re.tor}{0}
\verb{corretor}{}{}{}{}{s.m.}{Agenciador que faz a intermediação entre um comprador e um vendedor.}{cor.re.tor}{0}
\verb{corretor}{}{}{}{}{}{Revisor.}{cor.re.tor}{0}
\verb{corretora}{ô}{}{}{}{s.f.}{Instituição que negocia no comércio de títulos e valores mobiliários.}{cor.re.to.ra}{0}
\verb{corrida}{}{}{}{}{s.f.}{Competição de velocidade a pé ou em veículo.}{cor.ri.da}{0}
\verb{corrida}{}{}{}{}{}{Trajeto de táxi.}{cor.ri.da}{0}
\verb{corrido}{}{}{}{}{adj.}{Que passou; transcorrido.}{cor.ri.do}{0}
\verb{corrido}{}{}{}{}{}{Seguido, contínuo.}{cor.ri.do}{0}
\verb{corrigenda}{}{}{}{}{s.f.}{Errata.}{cor.ri.gen.da}{0}
\verb{corrigir}{}{}{}{}{v.t.}{Fazer a correção de alguma coisa; emendar, retificar.}{cor.ri.gir}{0}
\verb{corrigir}{}{}{}{}{}{Indicar os erros de alguma coisa.}{cor.ri.gir}{0}
\verb{corrigir}{}{}{}{}{}{Dar a alguém o castigo merecido; castigar, disciplinar.}{cor.ri.gir}{\verboinum{22}}
\verb{corrilho}{}{}{}{}{s.m.}{Reunião secreta de pessoas que tramam alguma coisa; conluio, conspiração, conciliábulo.}{cor.ri.lho}{0}
\verb{corrimão}{}{}{"-ãos, -ões}{}{s.m.}{Barra de apoio para a mão em escadas.}{cor.ri.mão}{0}
\verb{corrimento}{}{}{}{}{s.m.}{Secreção doentia de um órgão.}{cor.ri.men.to}{0}
\verb{corriola}{ó}{}{}{}{s.f.}{Motim de rua; arruaça, desordem, assuada.}{cor.ri.o.la}{0}
\verb{corriola}{ó}{}{}{}{}{Bando, quadrilha.}{cor.ri.o.la}{0}
\verb{corriqueiro}{ê}{}{}{}{adj.}{Que é comum, usual; habitual, trivial, banal, corrente.}{cor.ri.quei.ro}{0}
\verb{corrixo}{ch}{Bras.}{}{}{s.m.}{Chupim.}{cor.ri.xo}{0}
\verb{corroboração}{}{}{"-ões}{}{s.f.}{Ato ou efeito de corroborar.  }{cor.ro.bo.ra.ção}{0}
\verb{corroborar}{}{}{}{}{v.t.}{Dar força, robustez; tornar rijo, fortalecer.}{cor.ro.bo.rar}{0}
\verb{corroborar}{}{}{}{}{}{Comprovar, confirmar.}{cor.ro.bo.rar}{\verboinum{1}}
\verb{corroer}{ê}{}{}{}{v.t.}{Gastar aos poucos.}{cor.ro.er}{0}
\verb{corroer}{ê}{}{}{}{}{Destruir.}{cor.ro.er}{\verboinum{17}}
\verb{corromper}{ê}{}{}{}{v.t.}{Levar alguém a praticar uma ação desonesta; depravar, perverter.}{cor.rom.per}{\verboinum{12}}
\verb{corrosão}{}{}{"-ões}{}{s.f.}{Ato ou efeito de corroer.}{cor.ro.são}{0}
\verb{corrosivo}{}{}{}{}{adj.}{Que corrói.}{cor.ro.si.vo}{0}
\verb{corrosivo}{}{}{}{}{s.m.}{Aquilo que corrói.}{cor.ro.si.vo}{0}
\verb{corrução}{}{}{}{}{}{Var. de \textit{corrupção}.}{cor.ru.ção}{0}
\verb{corruíra}{}{Zool.}{}{}{s.f.}{Pássaro pequeno, canoro, de coloração parda, com listras pretas nas asas, no dorso e na cauda, que se alimenta de insetos e larvas; garrincha, cambaxirra.}{cor.ru.í.ra}{0}
\verb{corrupção}{}{}{"-ões}{}{s.f.}{Ato ou efeito de corromper; decomposição, putrefação.}{cor.rup.ção}{0}
\verb{corrupção}{}{}{"-ões}{}{}{Depravação, perversão.}{cor.rup.ção}{0}
\verb{corrupção}{}{}{"-ões}{}{}{Suborno.}{cor.rup.ção}{0}
\verb{corrupião}{}{Zool.}{"-ões}{}{s.m.}{Pássaro canoro, de cor preta, com o ventre alaranjado e uma mancha branca nas asas, notável pela capacidade de imitar sons; sofrê.}{cor.ru.pi.ão}{0}
\verb{corrupiar}{}{Bras.}{}{}{v.t.}{Girar muito; rodopiar.}{cor.ru.pi.ar}{\verboinum{6}}
\verb{corrupio}{}{}{}{}{s.m.}{Brincadeira em que um par rodopia de mãos dadas.}{cor.ru.pi.o}{0}
\verb{corrupio}{}{}{}{}{}{Ação de girar; giro, rodopio.}{cor.ru.pi.o}{0}
\verb{corrupio}{}{}{}{}{}{Cata"-vento.}{cor.ru.pi.o}{0}
\verb{corruptela}{é}{}{}{}{s.f.}{Modo errado de pronunciar ou de escrever uma palavra.}{cor.rup.te.la}{0}
\verb{corruptível}{}{}{"-eis}{}{adj.2g.}{Que é passível de se corromper; corrompível.}{cor.rup.tí.vel}{0}
\verb{corrupto}{}{}{}{}{adj.}{Que corrompe ou se deixa corromper.}{cor.rup.to}{0}
\verb{corruptor}{ô}{}{}{}{adj.}{Que corrompe.}{cor.rup.tor}{0}
\verb{corruptor}{ô}{}{}{}{s.m.}{Pessoa que corrompe ou que suborna.}{cor.rup.tor}{0}
\verb{corrutela}{é}{}{}{}{s.f.}{Corruptela.}{cor.ru.te.la}{0}
\verb{corrutela}{é}{}{}{}{}{Arraial, vilarejo.}{cor.ru.te.la}{0}
\verb{corrutível}{}{}{}{}{}{Var. de \textit{corruptível}.}{cor.ru.tí.vel}{0}
\verb{corruto}{}{}{}{}{}{Var. de \textit{corrupto}.}{cor.ru.to}{0}
\verb{corrutor}{ô}{}{}{}{}{Var. de \textit{corruptor}.}{cor.ru.tor}{0}
\verb{corsário}{}{}{}{}{adj.}{Relativo a corso.}{cor.sá.rio}{0}
\verb{corsário}{}{}{}{}{s.m.}{Navio que faz o corso.}{cor.sá.rio}{0}
\verb{corsário}{}{}{}{}{}{Comandante desse navio.}{cor.sá.rio}{0}
\verb{corsário}{}{}{}{}{}{Pirata.}{cor.sá.rio}{0}
\verb{córsico}{}{}{}{}{adj.}{Relativo à ilha de Córsega.}{cór.si.co}{0}
\verb{córsico}{}{}{}{}{s.m.}{Indivíduo natural ou habitante dessa ilha; corso.}{cór.si.co}{0}
\verb{corso}{ô}{}{}{}{s.m.}{Ataque de navios piratas; pirataria.}{cor.so}{0}
\verb{corso}{ô}{}{}{}{adj. e s.m.  }{Córsico.}{cor.so}{0}
\verb{corso}{ô}{}{}{}{}{Desfile de carros.}{cor.so}{0}
\verb{corso}{ô}{}{}{}{}{Cardume de sardinhas.}{cor.so}{0}
\verb{cortada}{}{}{}{}{s.f.}{Ato ou efeito de cortar.}{cor.ta.da}{0}
\verb{cortada}{}{Esport.}{}{}{}{Ato de bater na bola, com golpe seco e forte.}{cor.ta.da}{0}
\verb{cortado}{}{}{}{}{adj.}{Que se cortou, ou que foi separado de um todo.}{cor.ta.do}{0}
\verb{cortado}{}{}{}{}{}{Que foi interrompido, interceptado.}{cor.ta.do}{0}
\verb{cortador}{ô}{}{}{}{adj.}{Que corta.}{cor.ta.dor}{0}
\verb{cortador}{ô}{Esport.}{}{}{s.m.}{Jogador que dá cortadas na bola.}{cor.ta.dor}{0}
\verb{corta"-jaca}{ó}{Bras.}{corta"-jacas ⟨ó⟩}{}{s.m.}{Dança com sapateado, com coreografia original, muito em voga no final do séc. \textsc{xix}.}{cor.ta"-ja.ca}{0}
\verb{cortante}{}{}{}{}{adj.2g.}{Que corta; afiado.}{cor.tan.te}{0}
\verb{cortar}{}{}{}{}{v.t.}{Dividir alguma coisa em partes com um instrumento de corte.}{cor.tar}{0}
\verb{cortar}{}{}{}{}{}{Produzir ferimento em.}{cor.tar}{0}
\verb{cortar}{}{Esport.}{}{}{}{Bater na bola com golpe seco e forte.}{cor.tar}{0}
\verb{cortar}{}{}{}{}{}{Retirar algo de pessoa ou coisa.}{cor.tar}{\verboinum{1}}
\verb{corte}{ô}{}{}{}{s.f.}{Residência do soberano.}{cor.te}{0}
\verb{corte}{ó}{}{}{}{s.m.}{Ato ou efeito de cortar.}{cor.te}{0}
\verb{corte}{ô}{}{}{}{}{A nobreza que o rodeia.}{cor.te}{0}
\verb{corte}{ô}{}{}{}{}{Tribunal de justiça.}{cor.te}{0}
\verb{corte}{ô}{}{}{}{}{Atenção ou galanteio.}{cor.te}{0}
\verb{corte}{ó}{}{}{}{}{Golpe, incisão ou talho com instrumento cortante.}{cor.te}{0}
\verb{corte}{ó}{}{}{}{}{Fio ou gume de instrumento cortante.}{cor.te}{0}
\verb{corte}{ó}{}{}{}{}{Derrubada de árvores.}{cor.te}{0}
\verb{cortejar}{}{}{}{}{v.t.}{Cercar de atenções; galantear.}{cor.te.jar}{\verboinum{1}}
\verb{cortejo}{ê}{}{}{}{s.m.}{Grupo de pessoas que acompanham alguém numa caminhada; comitiva.}{cor.te.jo}{0}
\verb{cortejo}{ê}{}{}{}{}{Galanteio.}{cor.te.jo}{0}
\verb{cortês}{}{}{}{}{adj.2g.}{Que tem cortesia; que é delicado nas maneiras ou palavras; polido.}{cor.tês}{0}
\verb{cortês}{}{}{}{}{}{Gentil, cavalheiresco.}{cor.tês}{0}
\verb{cortesã}{}{}{}{}{s.f.}{Prostituta; meretriz.}{cor.te.sã}{0}
\verb{cortesão}{}{}{"-ãos \textit{ou} -ões}{}{adj.}{Que se refere a corte; palaciano, áulico.}{cor.te.são}{0}
\verb{cortesão}{}{}{"-ãos \textit{ou} -ões}{}{s.m.}{Homem da corte.}{cor.te.são}{0}
\verb{cortesia}{}{}{}{}{s.f.}{Procedimento de pessoa cortês. }{cor.te.si.a}{0}
\verb{cortesia}{}{}{}{}{}{Civilidade, gentileza.}{cor.te.si.a}{0}
\verb{cortesia}{}{}{}{}{}{Presente ou oferta de um estabelecimento comercial a seus clientes.}{cor.te.si.a}{0}
\verb{córtex}{cs}{Biol.}{}{}{s.m.}{Parte externa de órgão ou estrutura animal ou vegetal; córtice; cortiça.}{cór.tex}{0}
\verb{cortiça}{}{}{}{}{s.f.}{Camada macia e porosa de alguns troncos, usada no fabrico de rolhas, boias etc.}{cor.ti.ça}{0}
\verb{cortiça}{}{Biol.}{}{}{}{Córtex.}{cor.ti.ça}{0}
\verb{córtice}{}{Biol.}{}{}{s.m.}{Córtex.}{cór.ti.ce}{0}
\verb{corticeiro}{ê}{}{}{}{adj.}{Que se refere a cortiça.}{cor.ti.cei.ro}{0}
\verb{corticeiro}{ê}{}{}{}{s.m.}{Pessoa que trabalha nos sobreirais, no descortiçamento das árvores, ou  nas fábricas de cortiça.}{cor.ti.cei.ro}{0}
\verb{cortiço}{}{}{}{}{s.m.}{Colmeia.}{cor.ti.ço}{0}
\verb{cortiço}{}{}{}{}{}{Habitação coletiva das classes pobres.}{cor.ti.ço}{0}
\verb{cortina}{}{}{}{}{s.f.}{Pedaço de pano ou de outro material pendurado à frente de uma abertura ou em um vão.}{cor.ti.na}{0}
\verb{cortinado}{}{}{}{}{s.m.}{Armação de cortinas; cortina.}{cor.ti.na.do}{0}
\verb{cortisona}{}{Quím.}{}{}{s.f.}{Hormônio produzido pelas suprarrenais, e também obtido sinteticamente, empregado em medicina.}{cor.ti.so.na}{0}
\verb{coruja}{}{Zool.}{}{}{s.f.}{Nome comum a certas aves de rapina, de hábitos noturnos, com bico pequeno e curvo, olhos grandes, e que se alimentam principalmente de roedores.}{co.ru.ja}{0}
\verb{coruja}{}{}{}{}{adj.2g.}{Que exalta excessivamente as qualidades dos filhos.}{co.ru.ja}{0}
\verb{coruscar}{}{}{}{}{v.i.}{Reluzir, relampejar.}{co.rus.car}{\verboinum{2}}
\verb{corvejar}{}{}{}{}{v.i.}{Imitar a voz do corvo; crocitar.}{cor.ve.jar}{0}
\verb{corvejar}{}{Fig.}{}{}{}{Pensar muito em alguma coisa; remoer. }{cor.ve.jar}{\verboinum{1}}
\verb{corveta}{ê}{}{}{}{s.f.}{Navio de guerra veloz, menor que a fragata e armado com mísseis.}{cor.ve.ta}{0}
\verb{corvina}{}{Zool.}{}{}{s.f.}{Nome comum aos peixes teleósteos marinhos, que atingem 70 cm, e cuja carne é muito apreciada.}{cor.vi.na}{0}
\verb{corvo}{ô}{Zool.}{"-s ⟨ó⟩}{}{s.m.}{Nome comum a certas aves de plumagem preta, nativas da Europa, onívoras, e que emitem um grasnido áspero característico.}{cor.vo}{0}
\verb{corvo"-marinho}{ô}{Zool.}{corvos"-marinhos ⟨ó⟩}{}{s.m.}{Nome comum às aves pelicaniformes, aquáticas, de distribuição mundial; biguá, mergulhão.   }{cor.vo"-ma.ri.nho}{0}
\verb{cós}{}{}{}{}{s.m.}{Tira de pano que remata certas peças de vestuário, para reforçá"-las na cintura.}{cós}{0}
\verb{cós}{}{}{}{}{}{Parte do vestuário em que se ajusta essa tira de pano; cintura. }{cós}{0}
\verb{coscorão}{}{Cul.}{}{}{s.m.}{Tipo de filhó. }{cos.co.rão}{0}
\verb{coscorão}{}{Bras.}{}{}{}{Casca que se forma na cicatrização de uma ferida. }{cos.co.rão}{0}
\verb{cossecante}{}{Mat.}{cossecantes}{}{s.f.}{Secante do complemento de um ângulo ou de um arco.}{cos.se.can.te}{0}
\verb{cosedura}{}{}{}{}{s.f.}{Ato ou efeito de coser.}{co.se.du.ra}{0}
\verb{cosseno}{}{Mat.}{cossenos}{}{s.m.}{Seno do complemento de um ângulo ou arco.}{cos.se.no}{0}
\verb{coser}{ê}{}{}{}{v.t.}{Unir com pontos de agulha; costurar.}{co.ser}{\verboinum{12}}
\verb{cosmético}{}{}{}{}{adj.}{Que é próprio para os cuidados de beleza.}{cos.mé.ti.co}{0}
\verb{cosmético}{}{}{}{}{s.m.}{Qualquer produto de higiene ou beleza.}{cos.mé.ti.co}{0}
\verb{cósmico}{}{}{}{}{adj.}{Relativo ao cosmo.}{cós.mi.co}{0}
\verb{cosmo}{ó}{}{}{}{s.m.}{Conjunto dos astros do espaço; Universo.}{cos.mo}{0}
\verb{cosmogonia}{}{}{}{}{s.f.}{Qualquer teoria ou explicação a respeito da origem do Universo ou do Sistema Solar.}{cos.mo.go.ni.a}{0}
\verb{cosmografia}{}{Astron.}{}{}{s.f.}{Ramo da astronomia que trata da descrição do Universo; astronomia descritiva.}{cos.mo.gra.fi.a}{0}
\verb{cosmologia}{}{}{}{}{s.f.}{Doutrina ou narrativa sobre a origem do mundo ou do Universo.}{cos.mo.lo.gi.a}{0}
\verb{cosmológico}{}{}{}{}{adj.}{Relativo a cosmologia.}{cos.mo.ló.gi.co}{0}
\verb{cosmonauta}{}{}{}{}{s.2g.}{Indivíduo que navega em veículo espacial, fora da atmosfera terrestre; astronauta.}{cos.mo.nau.ta}{0}
\verb{cosmonáutica}{}{}{}{}{s.f.}{Ciência e técnica que trata do projeto, construção e operação de veículos espaciais; astronáutica.}{cos.mo.náu.ti.ca}{0}
\verb{cosmonave}{}{}{}{}{s.f.}{Nave espacial; astronave.}{cos.mo.na.ve}{0}
\verb{cosmopolita}{}{}{}{}{adj.2g.}{Relativo a ou próprio dos grandes centros urbanos.}{cos.mo.po.li.ta}{0}
\verb{cosmopolita}{}{}{}{}{}{Que é influenciado culturalmente por esses centros.}{cos.mo.po.li.ta}{0}
\verb{cosmopolita}{}{}{}{}{s.2g.}{Pessoa que viaja pelo mundo, vivendo sempre num país diferente e absorvendo os usos e costumes desse países.}{cos.mo.po.li.ta}{0}
\verb{cosmopolita}{}{}{}{}{}{Pessoa para quem o mundo é sua pátria; cidadão do mundo.}{cos.mo.po.li.ta}{0}
\verb{cosmopolitismo}{}{}{}{}{s.m.}{Qualidade ou modo de vida de cosmopolita.}{cos.mo.po.li.tis.mo}{0}
\verb{cosmos}{ó}{}{}{}{s.m.}{Cosmo.}{cos.mos}{0}
\verb{cossaco}{}{}{}{}{s.m.}{Soldado russo, outrora recrutado entre os povos das estepes da Rússia.}{cos.sa.co}{0}
\verb{costa}{ó}{}{}{}{s.f.}{Região próxima ao mar; litoral.}{cos.ta}{0}
\verb{costado}{}{}{}{}{s.m.}{Costas.}{cos.ta.do}{0}
\verb{costado}{}{}{}{}{}{Conjunto de pranchas que revestem exteriormente embarcações de grande porte.}{cos.ta.do}{0}
\verb{costa"-riquenho}{ó}{}{costa"-riquenhos ⟨ó⟩}{}{adj.}{Relativo a Costa Rica; costa"-riquense, costarriquenho.}{cos.ta"-ri.que.nho}{0}
\verb{costa"-riquenho}{ó}{}{costa"-riquenhos ⟨ó⟩}{}{s.m.}{Indivíduo natural ou habitante desse país.}{cos.ta"-ri.que.nho}{0}
\verb{costa"-riquense}{ó}{}{costa"-riquenses ⟨ó⟩}{}{adj.2g. e s.2g.}{Costa"-riquenho.}{cos.ta"-ri.quen.se}{0}
\verb{costarriquenho}{}{}{}{}{adj. e s.m.  }{Costa"-riquense.}{cos.tar.ri.que.nho}{0}
\verb{costas}{ó}{}{}{}{s.f.pl.}{A parte de trás do tronco humano; dorso.}{cos.tas}{0}
\verb{costear}{}{}{}{}{v.t.}{Navegar próximo à costa.}{cos.te.ar}{\verboinum{4}}
\verb{costeiro}{ê}{}{}{}{adj.}{Relativo a costa ou que navega junto à costa.}{cos.tei.ro}{0}
\verb{costela}{é}{Anat.}{}{}{s.f.}{Cada um dos doze pares de ossos curvos e alongados que formam a caixa torácica, protegendo os órgãos internos.}{cos.te.la}{0}
\verb{costeleta}{ê}{}{}{}{s.f.}{A costela de certos animais, separada do tronco e com carne aderente, com a qual se preparam algumas iguarias.}{cos.te.le.ta}{0}
\verb{costeleta}{ê}{Bras.}{}{}{}{Parte da barba e cabelo que se deixa crescer na lateral do rosto, perto da orelha.  }{cos.te.le.ta}{0}
\verb{costumado}{}{}{}{}{adj.}{Que se costumou; habituado, acostumado.}{cos.tu.ma.do}{0}
\verb{costumar}{}{}{}{}{v.t.}{Ter por costume, por hábito.}{cos.tu.mar}{0}
\verb{costumar}{}{}{}{}{v.i.}{Ser comum, habitual, costumeiro. (\textit{Costuma chover nessa época do ano.})}{cos.tu.mar}{\verboinum{1}}
\verb{costume}{}{}{}{}{s.m.}{Prática habitual; modo de proceder; uso.}{cos.tu.me}{0}
\verb{costume}{}{}{}{}{}{Conjunto de duas peças de vestuário.}{cos.tu.me}{0}
\verb{costumeiro}{ê}{}{}{}{adj.}{Que é comum; habitual, frequente.}{cos.tu.mei.ro}{0}
\verb{costura}{}{}{}{}{s.f.}{Ato ou efeito de costurar.}{cos.tu.ra}{0}
\verb{costura}{}{}{}{}{}{Ligação de peças de tecido, couro etc., costuradas uma a outra.}{cos.tu.ra}{0}
\verb{costurar}{}{}{}{}{v.t.}{Ligar peças de tecido, couro etc.; coser.}{cos.tu.rar}{\verboinum{1}}
\verb{costureira}{ê}{}{}{}{s.f.}{Mulher que costura profissionalmente.}{cos.tu.rei.ra}{0}
\verb{costureiro}{ê}{}{}{}{s.m.}{Profissional que se dedica à criação de roupas femininas.}{cos.tu.rei.ro}{0}
\verb{cota}{ó}{}{}{}{s.f.}{Quantia ou parte que cabe a cada um.}{co.ta}{0}
\verb{cotação}{}{}{"-ões}{}{s.f.}{Ato ou efeito de cotar; determinação de preços; avaliação.}{co.ta.ção}{0}
\verb{cotado}{}{}{}{}{adj.}{Que se cotou; avaliado.}{co.ta.do}{0}
\verb{cotado}{}{}{}{}{}{Conceituado, estimado.}{co.ta.do}{0}
\verb{cotangente}{}{Mat.}{cotangentes}{}{s.f.}{Inverso da tangente de um ângulo.}{co.tan.gen.te}{0}
\verb{cota"-parte}{ó}{}{cotas"-partes ⟨ó⟩}{}{s.f.}{Fração de uma dívida ou patrimônio que cada membro de um grupo deve pagar ou receber.}{co.ta"-par.te}{0}
\verb{cotar}{}{}{}{}{v.t.}{Fixar o preço; avaliar, taxar.}{co.tar}{\verboinum{1}}
\verb{cotejar}{}{}{}{}{v.t.}{Colocar lado a lado duas coisas para serem comparadas; confrontar.}{co.te.jar}{\verboinum{1}}
\verb{cotejo}{ê}{}{}{}{s.m.}{Ato ou efeito de cotejar; comparação, confronto.}{co.te.jo}{0}
\verb{cotidiano}{}{}{}{}{adj.}{De cada dia; diário.}{co.ti.di.a.no}{0}
\verb{cotidiano}{}{}{}{}{s.m.}{O que se faz todos os dias.}{co.ti.di.a.no}{0}
\verb{cotilédone}{}{Bot.}{}{}{s.m.}{Folha carregada de nutrientes que nasce da própria semente.}{co.ti.lé.do.ne}{0}
\verb{cotiledôneo}{}{}{}{}{adj.}{Relativo a cotilédone.}{co.ti.le.dô.neo}{0}
\verb{cotiledôneo}{}{Bot.}{}{}{}{Que possui cotilédones.}{co.ti.le.dô.neo}{0}
\verb{cotista}{}{}{}{}{s.2g.}{Indivíduo que possui cotas; quotista.}{co.tis.ta}{0}
\verb{cotização}{}{}{"-ões}{}{s.f.}{Ato ou efeito de cotizar.}{co.ti.za.ção}{0}
\verb{cotizar}{}{}{}{}{v.t.}{Distribuir por cotas.}{co.ti.zar}{\verboinum{1}}
\verb{coto}{ô}{}{}{}{s.m.}{Resto de algo comprido que se cortou ou que gastou; cotoco.}{co.to}{0}
\verb{cotó}{}{}{}{}{adj.}{Diz"-se de pessoa ou animal que teve um membro amputado ou mutilado.}{co.tó}{0}
\verb{cotoco}{ô}{}{}{}{s.m.}{Coto.}{co.to.co}{0}
\verb{cotonete}{é}{}{}{}{s.m.}{Palito com as pontas cobertas por chumaços de algodão, usado para fins higiênicos.}{co.to.ne.te}{0}
\verb{cotonicultor}{ô}{}{}{}{s.m.}{Agricultor que se dedica ao cultivo de algodão.}{co.to.ni.cul.tor}{0}
\verb{cotonicultura}{}{}{}{}{s.f.}{Cultivo de algodão.}{co.to.ni.cul.tu.ra}{0}
\verb{cotonifício}{}{}{}{}{s.m.}{Fábrica de tecidos de algodão.}{co.to.ni.fí.cio}{0}
\verb{cotovelada}{}{}{}{}{s.f.}{Pancada com o cotovelo.}{co.to.ve.la.da}{0}
\verb{cotovelar}{}{}{}{}{v.t.}{Tocar com o cotovelo; acotovelar.}{co.to.ve.lar}{\verboinum{1}}
\verb{cotovelo}{ê}{Anat.}{}{}{s.m.}{Articulação que une braço e antebraço.}{co.to.ve.lo}{0}
\verb{cotovelo}{ê}{}{}{}{}{Peça em ângulo reto que une dois canos.}{co.to.ve.lo}{0}
\verb{cotovia}{}{Zool.}{}{}{s.f.}{Ave de porte pequeno e canto mavioso.}{co.to.vi.a}{0}
\verb{coturno}{}{}{}{}{s.m.}{Bota militar, de cano alto fechado com cordões.}{co.tur.no}{0}
\verb{couce}{ô}{}{}{}{}{Var. de \textit{coice}.}{cou.ce}{0}
\verb{coudelaria}{}{}{}{}{s.f.}{Fazenda para criação e aperfeiçoamento de cavalos de corrida; haras.}{cou.de.la.ri.a}{0}
\verb{couraça}{}{}{}{}{s.f.}{Armadura de couro ou metal que cobre o peito e as costas.}{cou.ra.ça}{0}
\verb{couraça}{}{}{}{}{}{Conjunto de chapas de aço que reveste navios e carros de combate. }{cou.ra.ça}{0}
\verb{couraçado}{}{}{}{}{adj.}{Diz"-se do navio revestido de couraça; blindado, encouraçado.}{cou.ra.ça.do}{0}
\verb{courama}{}{}{}{}{s.f.}{Grande quantidade de couro ou de objetos de couro.}{cou.ra.ma}{0}
\verb{courama}{}{}{}{}{}{Veste de couro usada pelos vaqueiros.}{cou.ra.ma}{0}
\verb{couro}{ô}{}{}{}{s.m.}{Pele espessa e resistente de certos animais, como boi, porco etc.}{cou.ro}{0}
\verb{couro}{ô}{}{}{}{}{Pele curtida de animais usada para fabricação de calçados, bolsas etc.}{cou.ro}{0}
\verb{cousa}{ô}{}{}{}{}{Var. de \textit{coisa}.}{cou.sa}{0}
\verb{couteiro}{ê}{}{}{}{s.m.}{Indivíduo que dá asilo a bandidos.}{cou.tei.ro}{0}
\verb{couto}{ô}{}{}{}{s.m.}{Local de asilo; refúgio, abrigo.}{cou.to}{0}
\verb{couve}{ô}{Bot.}{}{}{s.f.}{Planta hortense comestível de caule fino e verde"-escuro, com folhas largas e grandes.}{cou.ve}{0}
%\verb{}{}{}{}{}{}{}{}{0}
\verb{couve"-flor}{ô\ldots{}ô}{}{couves"-flores ⟨ô⟩}{}{s.f.}{Planta hortalícia da família da couve, que produz um conjunto de flores comestíveis.}{cou.ve"-flor}{0}
\verb{cova}{ó}{}{}{}{s.f.}{Abertura na terra; cavidade, depressão, escavação.}{co.va}{0}
\verb{cova}{ó}{}{}{}{}{Buraco onde vivem certos animais; toca.}{co.va}{0}
\verb{côvado}{}{}{}{}{s.m.}{Antiga medida de comprimento equivalente a 66 cm.}{cô.va.do}{0}
\verb{covarde}{}{}{}{}{adj.2g.}{Que não tem coragem; medroso, pusilânime.}{co.var.de}{0}
\verb{covarde}{}{}{}{}{}{Traiçoeiro, desleal.}{co.var.de}{0}
\verb{covardia}{}{}{}{}{s.f.}{Qualidade de covarde; falta de coragem; medo.}{co.var.di.a}{0}
\verb{covardia}{}{}{}{}{}{Traição, deslealdade.}{co.var.di.a}{0}
\verb{coveiro}{ê}{}{}{}{s.m.}{Indivíduo que abre covas em cemitérios.}{co.vei.ro}{0}
\verb{covil}{}{}{"-is}{}{s.m.}{Toca habitada por animais ferozes.}{co.vil}{0}
\verb{covil}{}{}{"-is}{}{}{Esconderijo de ladrões e malfeitores.}{co.vil}{0}
\verb{covo}{ô}{}{}{}{s.m.}{Cesto comprido de vime ou bambu usado para pescar.}{co.vo}{0}
\verb{coxa}{ôch}{Anat.}{}{}{s.f.}{Parte superior da perna, compreendida entre o joelho e o quadril.}{co.xa}{0}
\verb{coxear}{ch}{}{}{}{v.i.}{Andar como coxo; mancar, capengar.}{co.xe.ar}{\verboinum{4}}
\verb{coxia}{ch}{}{}{}{s.f.}{Passagem estreita que circunda o palco teatral.}{co.xi.a}{0}
\verb{coxia}{ch}{}{}{}{}{Na estrebaria, espaço que cada cavalo ocupa.}{co.xi.a}{0}
\verb{coxilha}{ch}{}{}{}{s.f.}{Campina com elevações pequenas, utilizada para criação de gado.}{co.xi.lha}{0}
\verb{coxim}{ch}{}{"-ins}{}{s.m.}{Almofada grande usada como assento.}{co.xim}{0}
\verb{coxinha}{ch}{Cul.}{}{}{s.f.}{Salgadinho em forma de uma pequena coxa, recheado de carne de frango.}{co.xi.nha}{0}
\verb{coxo}{ôch}{}{}{}{adj.}{Que coxeia; manco, capenga.}{co.xo}{0}
\verb{cozedura}{}{}{}{}{s.f.}{Ato ou efeito de cozer; cozimento.}{co.ze.du.ra}{0}
\verb{cozedura}{}{}{}{}{}{Porção de coisas que se cozem de uma só vez.}{co.ze.du.ra}{0}
\verb{cozer}{ê}{}{}{}{v.t.}{Preparar alimentos sob a ação do fogo; cozinhar.}{co.zer}{\verboinum{12}}
\verb{cozido}{}{}{}{}{adj.}{Que se cozeu.}{co.zi.do}{0}
\verb{cozido}{}{Cul.}{}{}{s.m.}{Preparação culinária de carnes cozidas com verduras, legumes etc.}{co.zi.do}{0}
\verb{cozimento}{}{}{}{}{s.m.}{Ato ou efeito de cozer; cozedura, cocção.}{co.zi.men.to}{0}
\verb{cozinha}{}{}{}{}{s.f.}{Parte da casa em que se preparam os alimentos.}{co.zi.nha}{0}
\verb{cozinha}{}{}{}{}{}{Arte de cozinhar; culinária.}{co.zi.nha}{0}
\verb{cozinhar}{}{}{}{}{v.t.}{Preparar alimentos sob a ação do fogo; cozer.}{co.zi.nhar}{\verboinum{1}}
\verb{cozinheira}{ê}{}{}{}{s.f.}{Mulher que sabe cozinhar ou que cozinha profissionalmente.}{co.zi.nhei.ra}{0}
\verb{cozinheiro}{ê}{}{}{}{s.m.}{Indivíduo especializado em preparar refeições em restaurantes, hotéis etc.}{co.zi.nhei.ro}{0}
\verb{CPU}{}{Informát.}{}{}{s.m.}{Sigla em inglês de \textit{Unidade Central de Processamento}, que é a parte operacional do computador.}{CPU}{0}
\verb{Cr}{}{Quím.}{}{}{}{Símb. do \textit{crômio}.}{Cr}{0}
\verb{craca}{}{Zool.}{}{}{s.f.}{Espécie de crustáceo de carapaça branca em forma de tulipa, encontrado no litoral brasileiro.}{cra.ca}{0}
\verb{crachá}{}{}{}{}{s.m.}{Cartão que se prende na roupa para identificação pessoal.}{cra.chá}{0}
\verb{crack}{}{Quím.}{}{}{s.m.}{Droga extremamente tóxica, à base de cocaína, bicarbonato de sódio e outras substâncias, apresentada em forma de pedras para ser fumada numa espécie de cachimbo.}{\textit{crack}}{0}
\verb{craniano}{}{}{}{}{adj.}{Relativo a crânio.}{cra.ni.a.no}{0}
\verb{crânio}{}{}{}{}{s.m.}{Caixa óssea que protege o encéfalo e dá forma à cabeça.}{crâ.nio}{0}
\verb{crânio}{}{Fig.}{}{}{}{Indivíduo muito inteligente.}{crâ.nio}{0}
\verb{crápula}{}{}{}{}{s.f.}{Comportamento desregrado; devassidão, libertinagem.}{crá.pu.la}{0}
\verb{crápula}{}{}{}{}{s.m.}{Indivíduo canalha, desonesto, libertino.}{crá.pu.la}{0}
\verb{craque}{}{}{}{}{s.m.}{Indivíduo que entende ou sabe muito sobre alguma atividade.}{cra.que}{0}
\verb{craque}{}{}{}{}{}{Jogador de muita habilidade.}{cra.que}{0}
\verb{crase}{}{Gram.}{}{}{s.f.}{Contração ou fusão de duas vogais idênticas.}{cra.se}{0}
\verb{crase}{}{Por ext.}{}{}{}{Acento grave que indica essa fusão.}{cra.se}{0}
\verb{crasear}{}{}{}{}{v.t.}{Colocar o acento indicativo de crase em.}{cra.se.ar}{\verboinum{4}}
\verb{crasso}{}{}{}{}{adj.}{Que apresenta grande espessura; denso, grosso.}{cras.so}{0}
\verb{crasso}{}{}{}{}{}{Rudimentar, tosco, grosseiro.}{cras.so}{0}
\verb{cratera}{é}{}{}{}{s.f.}{Abertura no vulcão por onde sai a lava.}{cra.te.ra}{0}
\verb{cratera}{é}{}{}{}{}{Buraco muito grande; depressão profunda.}{cra.te.ra}{0}
\verb{cravar}{}{}{}{}{v.t.}{Fazer penetrar profundamente; fincar, engastar.}{cra.var}{\verboinum{1}}
\verb{craveira}{ê}{}{}{}{s.f.}{Medida para tomar a altura das pessoas.}{cra.vei.ra}{0}
\verb{craveira}{ê}{}{}{}{}{Buraco da fechadura onde entra o cravo.}{cra.vei.ra}{0}
\verb{craveiro}{ê}{Bot.}{}{}{s.m.}{Planta ornamental que dá flores vermelhas ou brancas.}{cra.vei.ro}{0}
\verb{craveiro"-da"-índia}{}{Bot.}{craveiros"-da"-índia}{}{s.m.}{Árvore alta e ornamental, cujo botão, o cravo"-da"-índia, é muito utilizado na indústria e como condimento.}{cra.vei.ro"-da"-ín.dia}{0}
\verb{cravejar}{}{}{}{}{v.t.}{Fixar com cravos.}{cra.ve.jar}{0}
\verb{cravejar}{}{}{}{}{}{Engastar pedras preciosas.}{cra.ve.jar}{\verboinum{1}}
\verb{cravelha}{ê}{}{}{}{s.f.}{Peça de certos instrumentos musicais utilizada para retesar"-lhes as cordas.}{cra.ve.lha}{0}
\verb{cravelho}{ê}{}{}{}{s.m.}{Peça tosca de madeira usada para fechar portas, janelas etc.; taramela.}{cra.ve.lho}{0}
\verb{cravista}{}{}{}{}{s.2g.}{Músico que toca cravo.}{cra.vis.ta}{0}
\verb{cravo}{}{}{}{}{s.m.}{Flor do craveiro, com pétalas crespas de cor vermelha ou branca e muito perfumadas.}{cra.vo}{0}
\verb{cravo}{}{}{}{}{s.m.}{Prego para ferradura.}{cra.vo}{0}
\verb{cravo}{}{}{}{}{s.m.}{Instrumento musical de cordas, com um ou dois teclados, predecessor do piano.}{cra.vo}{0}
\verb{cravo"-da"-índia}{}{}{cravos"-da"-índia}{}{s.m.}{Botão seco do craveiro"-da"-índia, utilizado como condimento e nas indústrias farmacêutica e de perfumes.}{cra.vo"-da"-ín.dia}{0}
\verb{cravo"-de"-defunto}{}{Bot.}{cravos"-de"-defunto}{}{s.m.}{Designação comum a várias plantas ornamentais.}{cra.vo"-de"-de.fun.to}{0}
\verb{cravo"-de"-defunto}{}{}{cravos"-de"-defunto}{}{}{A flor dessas plantas.}{cra.vo"-de"-de.fun.to}{0}
\verb{creche}{é}{}{}{}{s.f.}{Estabelecimento que cuida de crianças de pouca idade durante o dia, enquanto os pais trabalham.}{cre.che}{0}
\verb{credencial}{}{}{"-ais}{}{s.f.}{Documento que dá crédito a uma pessoa.}{cre.den.ci.al}{0}
\verb{credenciar}{}{}{}{}{v.t.}{Conferir credencial a; habilitar.}{cre.den.ci.ar}{\verboinum{1}}
\verb{crediário}{}{}{}{}{s.m.}{Sistema de vendas a crédito com pagamento a prestações.}{cre.di.á.rio}{0}
\verb{crediarista}{}{}{}{}{s.2g.}{Indivíduo que adquire bens de consumo pelo sistema de crediário.}{cre.di.a.ris.ta}{0}
\verb{credibilidade}{}{}{}{}{s.f.}{Qualidade do que é crível; confiabilidade.}{cre.di.bi.li.da.de}{0}
\verb{creditar}{}{}{}{}{v.t.}{Dar crédito a; garantir.}{cre.di.tar}{0}
\verb{creditar}{}{}{}{}{}{Lançar quantia em conta"-corrente; depositar.}{cre.di.tar}{\verboinum{1}}
\verb{creditício}{}{}{}{}{adj.}{Relativo a crédito. }{cre.di.tí.cio}{0}
\verb{crédito}{}{}{}{}{s.m.}{Convicção que se tem no outro; crença, fé.}{cré.di.to}{0}
\verb{crédito}{}{}{}{}{}{Confiança de que uma dívida será paga; credibilidade.}{cré.di.to}{0}
\verb{credo}{é}{}{}{}{s.m.}{Oração católica em que se faz a profissão de fé.}{cre.do}{0}
\verb{credo}{é}{}{}{}{interj.}{Expressão que denota espanto, aversão ou horror.}{cre.do}{0}
\verb{credor}{ô}{}{}{}{adj.}{Que é digno de confiança; merecedor.}{cre.dor}{0}
\verb{credor}{ô}{}{}{}{s.m.}{Indivíduo a quem se deve dinheiro.}{cre.dor}{0}
\verb{credulidade}{}{}{}{}{s.f.}{Qualidade de quem é crédulo; ingenuidade.}{cre.du.li.da.de}{0}
\verb{crédulo}{}{}{}{}{adj.}{Que acredita em tudo com muita facilidade; ingênuo, simplório.}{cré.du.lo}{0}
%\verb{}{}{}{}{}{}{}{}{0}
\verb{creiom}{}{}{}{}{s.m.}{Lápis de grafite próprio para desenho.}{crei.om}{0}
\verb{creiom}{}{}{}{}{}{Desenho feito com esse lápis.}{crei.om}{0}
\verb{cremação}{}{}{"-ões}{}{s.f.}{Ato ou efeito de cremar; incineração.}{cre.ma.ção}{0}
\verb{cremalheira}{ê}{}{}{}{s.f.}{Trilho dentado de linha férrea em trechos íngremes ou em serras.}{cre.ma.lhei.ra}{0}
\verb{cremalheira}{ê}{}{}{}{}{Peça dentada de certas engrenagens.}{cre.ma.lhei.ra}{0}
\verb{cremar}{}{}{}{}{v.t.}{Destruir cadáveres pelo fogo; queimar, incinerar.}{cre.mar}{\verboinum{1}}
\verb{crematório}{}{}{}{}{s.m.}{Local onde se incineram cadáveres.}{cre.ma.tó.rio}{0}
\verb{creme}{ê}{}{}{}{s.m.}{Camada espessa e gordurosa que se forma na superfície do leite; nata.}{cre.me}{0}
\verb{creme}{ê}{Cul.}{}{}{}{Doce feito com leite, ovos, açúcar, farinha ou maisena.}{cre.me}{0}
\verb{creme}{ê}{Farm.}{}{}{}{Preparado cosmético ou medicamentoso para se passar no rosto, nas mãos etc.}{cre.me}{0}
\verb{cremona}{}{}{}{}{s.f.}{Ferragem colocada em toda a extensão da porta ou da janela, composta por duas hastes engrenadas em uma espécie de cremalheira movida por maçaneta; carmona. }{cre.mo.na}{0}
\verb{cremoso}{ô}{}{"-osos ⟨ó⟩}{"-osa ⟨ó⟩}{adj.}{Que tem a consistência de creme.}{cre.mo.so}{0}
\verb{crença}{}{}{}{}{s.f.}{Ato ou efeito de crer.}{cren.ça}{0}
\verb{crença}{}{}{}{}{}{Fé, convicção.}{cren.ça}{0}
\verb{crendice}{}{}{}{}{s.f.}{Crença popular, sem fundamento religioso; superstição.}{cren.di.ce}{0}
\verb{crente}{}{}{}{}{adj.2g.}{Que crê, acredita.}{cren.te}{0}
\verb{crente}{}{}{}{}{s.2g.}{Indivíduo que acredita em Deus, que segue uma religião.}{cren.te}{0}
\verb{crente}{}{Relig.}{}{}{}{Protestante.}{cren.te}{0}
\verb{creolina}{}{Quím.}{}{}{s.f.}{Nome comercial de uma substância antisséptica e desodorante, extraída do óleo de alcatrão mineral ou creosoto.}{cre.o.li.na}{0}
\verb{creosoto}{ô}{Quím.}{}{}{s.m.}{Substância de cheiro forte extraída por destilação do alcatrão, usada em medicamentos e contra a cárie dos dentes.}{cre.o.so.to}{0}
\verb{crepe}{é}{}{}{}{s.m.}{Tecido fino e leve, geralmente transparente.}{cre.pe}{0}
\verb{crepe}{é}{}{}{}{}{Fita preta que se usa em sinal de luto.}{cre.pe}{0}
\verb{crepe}{é}{Cul.}{}{}{}{Espécie de panqueca de massa fina frita que envolve vários recheios, salgados ou doces.}{cre.pe}{0}
\verb{crepitar}{}{}{}{}{v.i.}{Estalar como lenha quando queima ou como sal lançado ao fogo.}{cre.pi.tar}{\verboinum{1}}
\verb{crepom}{}{}{}{}{adj.}{Diz"-se de um tipo de papel enrugado e fosco, usado na confecção de enfeites.}{cre.pom}{0}
\verb{crepuscular}{}{}{}{}{adj.2g.}{Relativo a crepúsculo.}{cre.pus.cu.lar}{0}
\verb{crepúsculo}{}{}{}{}{s.m.}{Luminosidade colorida que se forma no céu no momento do nascer e do pôr do sol.}{cre.pús.cu.lo}{0}
\verb{crer}{ê}{}{}{}{v.t.}{Tomar como um fato; ter por certo; acreditar.}{crer}{0}
\verb{crer}{ê}{}{}{}{}{Ter fé, ter crença em.}{crer}{0}
\verb{crer}{ê}{}{}{}{}{Dar crédito a; ter confiança em.}{crer}{\verboinum{13}}
\verb{crescendo}{}{Mús.}{}{}{s.m.}{Aumento gradativo da intensidade do som.}{cres.cen.do}{0}
\verb{crescendo}{}{}{}{}{}{Gradação.}{cres.cen.do}{0}
\verb{crescente}{}{}{}{}{adj.2g.}{Que cresce.}{cres.cen.te}{0}
\verb{crescente}{}{Gram.}{}{}{}{Diz"-se de ditongo em que a semivogal precede a vogal.}{cres.cen.te}{0}
\verb{crescente}{}{}{}{}{}{Diz"-se da fase da lua que segue a \textit{nova} e antecede a \textit{cheia}.}{cres.cen.te}{0}
\verb{crescente}{}{}{}{}{s.f.}{Enchente de rio ou mar.}{cres.cen.te}{0}
\verb{crescer}{ê}{}{}{}{v.i.}{Aumentar em tamanho, volume, quantidade, intensidade, duração.}{cres.cer}{0}
\verb{crescer}{ê}{}{}{}{}{Desenvolver"-se, amadurecer.}{cres.cer}{\verboinum{15}}
\verb{crescido}{}{}{}{}{adj.}{Que cresceu.}{cres.ci.do}{0}
\verb{crescimento}{}{}{}{}{s.m.}{Ato, efeito ou processo de crescer.}{cres.ci.men.to}{0}
\verb{crespo}{ê}{}{}{}{adj.}{De superfície áspera ou rugosa.}{cres.po}{0}
\verb{crespo}{ê}{}{}{}{}{Diz"-se de cabelo encaracolado, frisado.}{cres.po}{0}
\verb{crespo}{ê}{Fig.}{}{}{}{Perigoso, ameaçador, escabroso.}{cres.po}{0}
\verb{crestar}{}{}{}{}{v.t.}{Tostar, chamuscar.}{cres.tar}{\verboinum{1}}
\verb{cretáceo}{}{}{}{}{adj.}{Diz"-se do terceiro e último período da Era Mesozoica, no qual os dinossauros se extinguiram.}{cre.tá.ceo}{0}
\verb{cretinice}{}{}{}{}{s.f.}{Qualidade de cretino.}{cre.ti.ni.ce}{0}
\verb{cretinice}{}{}{}{}{}{Ato ou dito de cretino; estupidez, bobagem.}{cre.ti.ni.ce}{0}
\verb{cretinismo}{}{Med.}{}{}{s.m.}{Doença causada pela insuficiência da glândula tireoide, originada na vida fetal ou na infância, e que causa deficiência mental.}{cre.ti.nis.mo}{0}
\verb{cretinismo}{}{Fig.}{}{}{}{Imbecilidade, idiotice, estupidez.}{cre.ti.nis.mo}{0}
\verb{cretinizar}{}{}{}{}{v.t.}{Tornar cretino; imbecilizar.}{cre.ti.ni.zar}{\verboinum{1}}
\verb{cretino}{}{Med.}{}{}{adj.}{Que sofre de cretinismo.}{cre.ti.no}{0}
\verb{cretino}{}{Fig.}{}{}{}{Imbecil, idiota.}{cre.ti.no}{0}
\verb{cretone}{}{}{}{}{s.m.}{Tecido branco de linho e cânhamo geralmente usado para confeccionar lençóis e fronhas.}{cre.to.ne}{0}
\verb{cria}{}{}{}{}{s.f.}{Animal ou conjunto dos animais recém"-nascidos que ainda mamam.}{cri.a}{0}
\verb{criação}{}{}{"-ões}{}{s.f.}{Ato ou efeito de criar.}{cri.a.ção}{0}
\verb{criação}{}{}{"-ões}{}{}{Conjunto dos animais domésticos ou criados para alimentação do homem.}{cri.a.ção}{0}
\verb{criação}{}{}{"-ões}{}{}{A amamentação e educação de uma criança.}{cri.a.ção}{0}
\verb{criação}{}{}{"-ões}{}{}{Invenção, produção.}{cri.a.ção}{0}
\verb{criadagem}{}{}{}{}{s.f.}{Conjunto dos criados e criadas de uma residência.}{cri.a.da.gem}{0}
\verb{criado}{}{}{}{}{adj.}{Que se criou.}{cri.a.do}{0}
\verb{criado}{}{}{}{}{s.m.}{Empregado doméstico.}{cri.a.do}{0}
\verb{criado"-mudo}{}{}{criados"-mudos}{}{s.m.}{Pequeno móvel que se põe ao lado da cabeceira da cama em que se colocam abajur, livros ou outros objetos que podem ser necessários durante a noite; mesa"-de"-cabeceira.}{cri.a.do"-mu.do}{0}
\verb{criador}{ô}{}{}{}{adj.}{Que cria.}{cri.a.dor}{0}
\verb{criador}{ô}{}{}{}{}{Caracterizado pela inventividade; fecundo, imaginoso.}{cri.a.dor}{0}
\verb{criador}{ô}{Relig.}{}{}{s.m.}{Deus. (Usa"-se com maiúscula inicial nesta acepção.)}{cri.a.dor}{0}
\verb{criador}{ô}{Bras.}{}{}{}{Fazendeiro que cria gado.}{cri.a.dor}{0}
\verb{criança}{}{}{}{}{s.f.}{Ser humano jovem, em fase de criação.}{cri.an.ça}{0}
\verb{criança}{}{}{}{}{adj.2g.}{Imaturo, ingênuo, infantil.}{cri.an.ça}{0}
\verb{criançada}{}{}{}{}{s.f.}{Conjunto de crianças.}{cri.an.ça.da}{0}
\verb{criancice}{}{}{}{}{s.f.}{Ato ou dito de criança.}{cri.an.ci.ce}{0}
\verb{criançola}{ó}{}{}{}{s.m.}{Rapaz com maneiras ou mentalidade de criança.}{cri.an.ço.la}{0}
\verb{criar}{}{}{}{}{v.t.}{Dar existência a; gerar.}{cri.ar}{0}
\verb{criar}{}{}{}{}{}{Amamentar, educar e sustentar uma criança.}{cri.ar}{0}
\verb{criar}{}{}{}{}{}{Inventar, conceber.}{cri.ar}{0}
\verb{criar}{}{}{}{}{}{Fundar, instituir.}{cri.ar}{0}
\verb{criar}{}{}{}{}{}{Cultivar.}{cri.ar}{\verboinum{1}}
\verb{criatividade}{}{}{}{}{s.f.}{Capacidade de imaginar, conceber coisas ou ideias; inventividade.}{cri.a.ti.vi.da.de}{0}
\verb{criativo}{}{}{}{}{adj.}{Hábil em ter ideias novas; inventivo.}{cri.a.ti.vo}{0}
\verb{criativo}{}{}{}{}{}{Caracterizado pela novidade ou por fugir aos padrões.}{cri.a.ti.vo}{0}
\verb{criatura}{}{}{}{}{s.f.}{Qualquer ser criado; ser.}{cri.a.tu.ra}{0}
\verb{criatura}{}{}{}{}{}{Pessoa, indivíduo.}{cri.a.tu.ra}{0}
\verb{criciúma}{}{Bot.}{}{}{s.f.}{Certa planta trepadeira usada na fabricação de cestos.}{cri.ci.ú.ma}{0}
\verb{cricri}{}{Onomat.}{}{}{s.m.}{A voz dos grilos.}{cri.cri}{0}
\verb{cricri}{}{Pop.}{}{}{adj.2g.}{Maçante, chato, tedioso.}{cri.cri}{0}
\verb{cricri}{}{Pop.}{}{}{}{Diz"-se de quem fica criticando, achando defeito, reclamando.}{cri.cri}{0}
\verb{cricrilar}{}{}{}{}{v.i.}{Fazer cricri.}{cri.cri.lar}{\verboinum{1}}
\verb{crime}{}{}{}{}{s.m.}{Violação intencional ou não da lei penal.}{cri.me}{0}
\verb{crime}{}{}{}{}{}{Violação de lei moral, religiosa ou civil.}{cri.me}{0}
\verb{crime}{}{}{}{}{}{Ato bastante condenável.}{cri.me}{0}
\verb{criminal}{}{}{"-ais}{}{adj.2g.}{Relativo a crime.}{cri.mi.nal}{0}
\verb{criminalidade}{}{}{}{}{s.f.}{Qualidade de criminoso.}{cri.mi.na.li.da.de}{0}
\verb{criminalidade}{}{}{}{}{}{Índice de incidência de atos criminosos em determinada região.}{cri.mi.na.li.da.de}{0}
\verb{criminalista}{}{}{}{}{adj.2g.}{Especializado em direito penal.}{cri.mi.na.lis.ta}{0}
\verb{criminoso}{ô}{}{"-osos ⟨ó⟩}{"-osa ⟨ó⟩}{adj.}{Relativo a crime.}{cri.mi.no.so}{0}
\verb{criminoso}{ô}{}{"-osos ⟨ó⟩}{"-osa ⟨ó⟩}{}{Que pode ser enquadrado como crime; delituoso.}{cri.mi.no.so}{0}
\verb{criminoso}{ô}{}{"-osos ⟨ó⟩}{"-osa ⟨ó⟩}{s.m.}{Indivíduo que praticou crime.}{cri.mi.no.so}{0}
\verb{crina}{}{}{}{}{s.f.}{Pelos compridos do pescoço e da cauda do cavalo e animais semelhantes.}{cri.na}{0}
\verb{crina}{}{}{}{}{}{Tecido áspero usado para fricção.}{cri.na}{0}
\verb{crina}{}{Mús.}{}{}{}{A parte do arco que fricciona as cordas do instrumento.}{cri.na}{0}
\verb{crioulo}{ô}{}{}{}{adj.}{Diz"-se das pessoas de etnia europeia nascidas nas colônias.}{cri.ou.lo}{0}
\verb{crioulo}{ô}{}{}{}{}{Originário do próprio país onde vive.}{cri.ou.lo}{0}
\verb{crioulo}{ô}{Pejor.}{}{}{s.m.}{Indivíduo negro.}{cri.ou.lo}{0}
\verb{crioulo}{ô}{Gram.}{}{}{}{Língua simplificada formada em situação de contatos comerciais entre nativos e colonizadores.}{cri.ou.lo}{0}
\verb{crioulo}{ô}{Bras.}{}{}{}{Cigarro feito de fumo de corda e palha de milho.}{cri.ou.lo}{0}
\verb{cripta}{}{}{}{}{s.f.}{Galeria subterrânea de igreja, onde se enterravam os mortos.}{crip.ta}{0}
\verb{cripta}{}{}{}{}{}{Caverna, gruta, antro.}{crip.ta}{0}
%\verb{}{}{}{}{}{}{}{}{0}
\verb{criptógamo}{}{Bot.}{}{}{s.m.}{Espécime dos criptógamos, vegetais sem flores ou sementes e com órgãos reprodutores pouco aparentes, como os musgos e os cogumelos.}{crip.tó.ga.mo}{0}
\verb{criptografia}{}{}{}{}{s.f.}{Técnica de escrever de maneira codificada, usando abreviaturas ou outros sinais previamente convencionados.}{crip.to.gra.fi.a}{0}
\verb{criptônio}{}{Quím.}{}{}{s.m.}{Elemento químico da família dos gases nobres, incolor, usado em válvulas a gás, em \textit{laser} de ultravioleta e em lâmpadas especiais. \elemento{36}{83.8}{Kr}.}{crip.tô.nio}{0}
\verb{críquete}{}{}{}{}{s.m.}{Jogo inglês em que duas equipes de onze jogadores controlam uma bolinha usando um bastão achatado.}{crí.que.te}{0}
\verb{crisálida}{}{Zool.}{}{}{s.f.}{Estado intermediário entre a lagarta e a borboleta.}{cri.sá.li.da}{0}
\verb{crisálida}{}{Fig.}{}{}{}{Coisa em estado latente.}{cri.sá.li.da}{0}
\verb{crisântemo}{}{Bot.}{}{}{s.m.}{Certa planta de flores amarelas, alaranjadas ou rosadas, cultivada como ornamental.}{cri.sân.te.mo}{0}
\verb{crisântemo}{}{}{}{}{}{A flor dessa planta.}{cri.sân.te.mo}{0}
\verb{crise}{}{}{}{}{s.f.}{Situação de mudança em que se agravam todos os problemas existentes e em que se manifestam novos problemas. (\textit{A crise da adolescência é a das mais conhecidas.})}{cri.se}{0}
\verb{crisma}{}{Relig.}{}{}{s.f.}{No catolicismo, sacramento no qual se confirma ou muda o nome da pessoa.}{cris.ma}{0}
\verb{crisma}{}{Por ext.}{}{}{}{Mudança de nome.}{cris.ma}{0}
\verb{crisma}{}{}{}{}{s.m.}{Óleo sagrado usado em algumas cerimônias.}{cris.ma}{0}
\verb{crismar}{}{}{}{}{v.t.}{Conferir a crisma.}{cris.mar}{\verboinum{1}}
\verb{crisol}{ó}{}{"-óis}{}{s.m.}{Grande vaso para fundir ou calcinar minérios; cadinho.}{cri.sol}{0}
\verb{crisol}{ó}{Fig.}{"-óis}{}{}{Meio de purificação moral ou intelectual.}{cri.sol}{0}
\verb{crispar}{}{}{}{}{v.t.}{Encrespar, enrugar.}{cris.par}{0}
\verb{crispar}{}{}{}{}{}{Contrair.}{cris.par}{\verboinum{1}}
\verb{crista}{}{Zool.}{}{}{s.f.}{Estrutura carnosa sobre a cabeça de algumas aves e de alguns répteis.}{cris.ta}{0}
\verb{crista}{}{}{}{}{}{Penacho, topete.}{cris.ta}{0}
\verb{crista}{}{}{}{}{}{O ponto mais elevado (de montanha, onda).}{cris.ta}{0}
\verb{cristal}{}{}{"-ais}{}{s.m.}{Mineral sólido de estrutura poliédrica.}{cris.tal}{0}
\verb{cristal}{}{}{"-ais}{}{}{Tipo de vidro muito puro e límpido.}{cris.tal}{0}
\verb{cristal}{}{}{"-ais}{}{}{Objeto feito com esse vidro.}{cris.tal}{0}
\verb{cristaleira}{ê}{}{}{}{s.f.}{Armário envidraçado próprio para guardar cristais e louças, exibindo"-os.}{cris.ta.lei.ra}{0}
\verb{cristalino}{}{}{}{}{adj.}{Relativo a cristal.}{cris.ta.li.no}{0}
\verb{cristalino}{}{}{}{}{}{Feito de cristal.}{cris.ta.li.no}{0}
\verb{cristalino}{}{Fig.}{}{}{}{Límpido como cristal.}{cris.ta.li.no}{0}
\verb{cristalino}{}{Anat.}{}{}{s.m.}{Lente do olho, situada à frente do humor vítreo.}{cris.ta.li.no}{0}
\verb{cristalizar}{}{}{}{}{v.t.}{Transformar em cristal.}{cris.ta.li.zar}{0}
\verb{cristalizar}{}{}{}{}{}{Tornar estável.}{cris.ta.li.zar}{0}
\verb{cristalizar}{}{Fig.}{}{}{}{Permanecer em um mesmo lugar ou estado; estagnar.}{cris.ta.li.zar}{0}
\verb{cristalografia}{}{}{}{}{s.f.}{Ciência que estuda os cristais.}{cris.ta.lo.gra.fi.a}{0}
\verb{cristandade}{}{}{}{}{s.f.}{Conjunto dos povos ou dos países cristãos. }{cris.tan.da.de}{0}
\verb{cristandade}{}{}{}{}{}{Qualidade de ser cristão.}{cris.tan.da.de}{0}
\verb{cristão}{}{}{"-ãos}{cristã}{s.m.}{Seguidor do cristianismo.}{cris.tão}{0}
\verb{cristão}{}{}{"-ãos}{cristã}{adj.}{Relativo ao cristianismo.}{cris.tão}{0}
\verb{cristão}{}{Pop.}{"-ãos}{cristã}{}{Pessoa, ente.}{cris.tão}{0}
\verb{cristão"-novo}{ô}{}{cristãos"-novos ⟨ó⟩}{}{s.m.}{Judeu convertido ao cristianismo.}{cris.tão"-no.vo}{0}
\verb{cristianismo}{}{}{}{}{s.m.}{O conjunto das religiões baseadas nos ensinamentos e na vida de Jesus Cristo.}{cris.ti.a.nis.mo}{0}
\verb{cristianizar}{}{}{}{}{v.t.}{Tornar cristão, converter ao cristianismo.}{cris.ti.a.ni.zar}{\verboinum{1}}
\verb{cristo}{}{Relig.}{}{}{s.m.}{Imagem de Jesus Cristo crucificado; redentor.}{cris.to}{0}
\verb{cristo}{}{Fig.}{}{}{}{Pessoa que sofre em lugar de outrem; vítima de maus tratos e enganos.}{cris.to}{0}
\verb{critério}{}{}{}{}{s.m.}{Noção ou valor que serve como base ao julgar ou examinar algo.}{cri.té.rio}{0}
\verb{critério}{}{}{}{}{}{Juízo, equilíbrio, ponderação, discernimento.}{cri.té.rio}{0}
\verb{criterioso}{ô}{}{"-osos ⟨ó⟩}{"-osa ⟨ó⟩}{adj.}{Que revela critério; ajuizado.}{cri.te.ri.o.so}{0}
\verb{crítica}{}{}{}{}{s.f.}{Análise que se faz de obra, coisa ou pessoa, examinando suas características positivas e negativas.}{crí.ti.ca}{0}
\verb{crítica}{}{}{}{}{}{Exame minucioso e fundamentado.}{crí.ti.ca}{0}
\verb{crítica}{}{}{}{}{}{O conjunto dos críticos.}{crí.ti.ca}{0}
\verb{crítica}{}{}{}{}{}{Opinião desfavorável; censura, depreciação.}{crí.ti.ca}{0}
\verb{criticar}{}{}{}{}{v.t.}{Fazer a crítica de.}{cri.ti.car}{0}
\verb{criticar}{}{}{}{}{}{Emitir opinião desfavorável; censurar.}{cri.ti.car}{\verboinum{2}}
\verb{criticismo}{}{Filos.}{}{}{s.m.}{Doutrina formulada por Immanuel Kant (1724--1804), filósofo alemão, e que busca determinar os limites, o alcance e o valor da razão e do racionalismo crítico; kantismo.}{cri.ti.cis.mo}{0}
\verb{crítico}{}{}{}{}{adj.}{Relativo a uma crise ou a um momento de mudança que potencializa o surgimento ou o   recrudescimento de problemas graves. (\textit{As primeiras horas após o nascimento são período crítico na vida do bebê.})}{crí.ti.co}{0}
\verb{crítico}{}{}{}{}{}{Relativo a crítica ou a julgamento que se deve fazer. (\textit{O trabalho crítico do escritor é escolher as situações mais exemplares para a elaboração de sua narrativa.})}{crí.ti.co}{0}
\verb{crítico}{}{}{}{}{s.m.}{O indivíduo que exerce o trabalho de criticar, analisar ou julgar. (\textit{Ele trabalha como crítico de cinema no jornal.})}{crí.ti.co}{0}
\verb{crivar}{}{}{}{}{}{Fazer muitos furos em.}{cri.var}{0}
\verb{crivar}{}{}{}{}{v.t.}{Fazer passar através de crivo; peneirar.}{cri.var}{0}
\verb{crivar}{}{Fig.}{}{}{}{Cobrir de, encher de.}{cri.var}{\verboinum{1}}
\verb{crível}{}{}{"-eis}{}{adj.2g.}{Em que se pode crer; verossímil.}{crí.vel}{0}
\verb{crivo}{}{}{}{}{}{Ralo.}{cri.vo}{0}
\verb{crivo}{}{}{}{}{}{Coador.}{cri.vo}{0}
\verb{crivo}{}{}{}{}{s.m.}{Peneira.}{cri.vo}{0}
\verb{crivo}{}{}{}{}{}{Qualquer coisa que tem muitos furos ou buracos.}{cri.vo}{0}
\verb{croata}{}{}{}{}{adj.2g.}{Relativo a Croácia.}{cro.a.ta}{0}
\verb{croata}{}{}{}{}{s.2g.}{Indivíduo natural ou habitante desse país.}{cro.a.ta}{0}
\verb{crocante}{}{}{}{}{adj.2g.}{Que estala ao ser mordido ou mastigado.}{cro.can.te}{0}
\verb{croché}{}{}{}{}{}{Var. de \textit{crochê}.}{cro.ché}{0}
\verb{crochê}{}{}{}{}{s.m.}{Trabalho de renda feito com uma agulha apropriada, com gancho na ponta.}{cro.chê}{0}
\verb{crocitar}{}{}{}{}{v.i.}{Soltar crocitos; grasnar, corvejar.}{cro.ci.tar}{0}
\verb{crocitar}{}{}{}{}{}{Imitar essa voz.}{cro.ci.tar}{\verboinum{1}}
\verb{crocito}{}{}{}{}{s.m.}{A voz do corvo, do condor e de outras aves.}{cro.ci.to}{0}
\verb{crocodilo}{}{Zool.}{}{}{s.m.}{Réptil aquático de grande porte encontrado em regiões tropicais.}{cro.co.di.lo}{0}
\verb{cromado}{}{}{}{}{adj.}{Que tem cromo em sua composição.}{cro.ma.do}{0}
\verb{cromado}{}{}{}{}{}{Revestido de cromo.}{cro.ma.do}{0}
\verb{cromado}{}{}{}{}{s.m.}{O revestimento de cromo.}{cro.ma.do}{0}
\verb{cromar}{}{}{}{}{v.t.}{Revestir com camada de cromo.}{cro.mar}{\verboinum{1}}
\verb{cromático}{}{}{}{}{adj.}{Relativo a cores.}{cro.má.ti.co}{0}
\verb{cromático}{}{Mús.}{}{}{}{Diz"-se de escala ou frase musical formada de semitons.}{cro.má.ti.co}{0}
\verb{cromatismo}{}{}{}{}{s.m.}{Dispersão da luz.}{cro.ma.tis.mo}{0}
\verb{cromatismo}{}{}{}{}{}{Divisão das cores.}{cro.ma.tis.mo}{0}
\verb{crômico}{}{}{}{}{adj.}{Relativo a cromo.}{crô.mi.co}{0}
\verb{crômio}{}{Quím.}{}{}{s.m.}{Elemento químico metálico, prateado, brilhante, duro, utilizado como recobrimento protetor contra oxidação, em ligas metálicas, em diversos pigmentos etc.; cromo. \elemento{24}{51.9961}{Cr}.}{crô.mio}{0}
\verb{cromo}{}{Quím.}{}{}{s.m.}{Crômio.}{cro.mo}{0}
\verb{cromo}{}{}{}{}{s.m.}{Forma abreviada de \textit{cromolitografia}.}{cro.mo}{0}
\verb{cromo}{}{}{}{}{}{Figura ou desenho impresso em cores e recortado; figurinha.}{cro.mo}{0}
\verb{cromolitografia}{}{}{}{}{s.f.}{Imagem impressa em cores.}{cro.mo.li.to.gra.fi.a}{0}
\verb{cromolitografia}{}{}{}{}{}{Diapositivo; \textit{slide}.}{cro.mo.li.to.gra.fi.a}{0}
\verb{cromossomo}{}{Biol.}{}{}{s.m.}{Material genético contido em cada célula animal ou vegetal.}{cro.mos.so.mo}{0}
\verb{cromoterapia}{}{Med.}{}{}{s.f.}{Tratamento alternativo para cura de doenças, que utiliza luzes ou objetos de várias cores e intensidades. }{cro.mo.te.ra.pi.a}{0}
\verb{crônica}{}{}{}{}{s.f.}{Conto curto, geralmente com tema do cotidiano.}{crô.ni.ca}{0}
\verb{crônica}{}{}{}{}{}{Texto jornalístico sobre fato contemporâneo.}{crô.ni.ca}{0}
\verb{crônico}{}{}{}{}{adj.}{Diz"-se de doença de longa duração.}{crô.ni.co}{0}
\verb{cronista}{}{}{}{}{s.2g.}{Indivíduo que escreve crônicas.}{cro.nis.ta}{0}
\verb{cronograma}{}{}{}{}{s.m.}{Planejamento que estabelece as etapas e prazos de um trabalho.}{cro.no.gra.ma}{0}
\verb{cronologia}{}{}{}{}{s.f.}{Estudo das datas relevantes da história e das divisões do tempo.}{cro.no.lo.gi.a}{0}
\verb{cronologia}{}{}{}{}{}{Sucessão de fatos relevantes.}{cro.no.lo.gi.a}{0}
\verb{cronológico}{}{}{}{}{adj.}{Relativo a cronologia.}{cro.no.ló.gi.co}{0}
\verb{cronometrar}{}{}{}{}{v.t.}{Medir o tempo com cronômetro ou relógio.}{cro.no.me.trar}{\verboinum{1}}
\verb{cronômetro}{}{}{}{}{s.m.}{Relógio de alta precisão, muito utilizado para medir o tempo em competições esportivas.}{cro.nô.me.tro}{0}
\verb{croquete}{é}{Cul.}{}{}{s.m.}{Bolinho de carne moída recoberto de farinha e frito.}{cro.que.te}{0}
\verb{croqui}{}{}{}{}{s.m.}{Esboço bastante simplificado de um desenho ou de um mapa.}{cro.qui}{0}
\verb{crosta}{ô}{}{}{}{s.f.}{Camada dura e geralmente espessa que reveste um corpo.}{cros.ta}{0}
\verb{crótalo}{}{Mús.}{}{}{s.m.}{Instrumento de percussão semelhante à castanhola.}{cró.ta.lo}{0}
\verb{cru}{}{}{}{}{adj.}{Que não foi cozido.}{cru}{0}
\verb{cru}{}{Fig.}{}{}{}{Despreparado, não amadurecido, ingênuo.}{cru}{0}
\verb{cru}{}{Fig.}{}{}{}{Sem disfarce; duro, áspero.}{cru}{0}
\verb{crucial}{}{}{"-ais}{}{adj.2g.}{Decisivo, crítico.}{cru.ci.al}{0}
\verb{crucial}{}{}{"-ais}{}{}{Doloroso, difícil, duro.}{cru.ci.al}{0}
\verb{cruciante}{}{}{}{}{adj.2g.}{Que crucia; torturante, doloroso.}{cru.ci.an.te}{0}
\verb{cruciar}{}{}{}{}{v.t.}{Atormentar, torturar.}{cru.ci.ar}{\verboinum{6}}
\verb{crucífera}{}{Bot.}{}{}{s.f.}{Espécime das crucíferas, família de ervas que inclui repolho, couve, mostarda.}{cru.cí.fe.ra}{0}
\verb{crucificação}{}{}{"-ões}{}{s.f.}{Ato ou efeito de crucificar.}{cru.ci.fi.ca.ção}{0}
\verb{crucificado}{}{}{}{}{adj.}{Que se crucificou.}{cru.ci.fi.ca.do}{0}
\verb{crucificar}{}{}{}{}{v.t.}{Executar pregando na cruz.}{cru.ci.fi.car}{0}
\verb{crucificar}{}{Fig.}{}{}{}{Martirizar, torturar.}{cru.ci.fi.car}{\verboinum{2}}
\verb{crucifixão}{cs}{}{"-ões}{}{s.f.}{Crucificação.}{cru.ci.fi.xão}{0}
\verb{crucifixo}{cs}{}{}{}{s.m.}{Imagem de Jesus Cristo crucificado.}{cru.ci.fi.xo}{0}
\verb{cruciforme}{ó}{}{}{}{adj.2g.}{Em forma de cruz.}{cru.ci.for.me}{0}
\verb{cruel}{é}{}{"-éis}{}{adj.2g.}{Que gosta de causar mal; sanguinário.}{cru.el}{0}
\verb{cruel}{é}{}{"-éis}{}{}{Doloroso, pungente.}{cru.el}{0}
\verb{crueldade}{}{}{}{}{s.f.}{Ato ou dito cruel; atrocidade.}{cru.el.da.de}{0}
\verb{crueldade}{}{}{}{}{}{Qualidade de cruel.}{cru.el.da.de}{0}
\verb{cruento}{}{}{}{}{adj.}{Em que há derramamento de sangue; cruel, sangrento.}{cru.en.to}{0}
\verb{crueza}{ê}{}{}{}{s.f.}{Qualidade de cru.}{cru.e.za}{0}
\verb{crueza}{ê}{}{}{}{}{Crueldade.}{cru.e.za}{0}
\verb{crupe}{}{Med.}{}{}{s.m.}{Doença da infância, caracterizada pela obstrução aguda da laringe, e que causa sufocação.}{cru.pe}{0}
\verb{crupiê}{}{}{}{}{s.m.}{Funcionário dos cassinos que comanda o jogo, recolhe as apostas e paga os jogadores.}{cru.pi.ê}{0}
\verb{crusta}{}{}{}{}{s.f.}{Crosta.}{crus.ta}{0}
\verb{crusta}{}{}{}{}{}{Tártaro ou viscosidade marítima que endurece na superfície das conchas.}{crus.ta}{0}
\verb{crustáceo}{}{Zool.}{}{}{adj.}{Relativo ou pertencente à classe dos crustáceos, animais que vivem na água e têm uma casca que lhes cobre o corpo.}{crus.tá.ceo}{0}
\verb{cruz}{}{}{}{}{s.f.}{Antiga estrutura formada de duas tábuas, uma atravessada na outra, onde eram presos os condenados para ali morrerem.}{cruz}{0}
\verb{cruzada}{}{}{}{}{s.f.}{Na Idade Média, expedição militar dos cristãos para recuperar a Terra Santa.}{cru.za.da}{0}
\verb{cruzada}{}{}{}{}{}{Movimento de propaganda ou de defesa de alguma coisa de interesse público; campanha.}{cru.za.da}{0}
\verb{cruzado}{}{}{}{}{adj.}{Disposto em cruz.}{cru.za.do}{0}
\verb{cruzado}{}{}{}{}{}{Atravessado.}{cru.za.do}{0}
\verb{cruzado}{}{}{}{}{s.m.}{Soldado que lutava numa cruzada.}{cru.za.do}{0}
\verb{cruzador}{ô}{}{}{}{adj.}{Que cruza.}{cru.za.dor}{0}
\verb{cruzador}{ô}{}{}{}{s.m.}{Navio de guerra de grande porte.}{cru.za.dor}{0}
\verb{cruzamento}{}{}{}{}{s.m.}{Posição ou disposição em forma de cruz.}{cru.za.men.to}{0}
\verb{cruzamento}{}{}{}{}{}{Ponto onde se cruzam caminhos; encruzilhada.}{cru.za.men.to}{0}
\verb{cruzamento}{}{Biol.}{}{}{}{Acasalamento entre organismos distintos morfológica ou geneticamente.}{cru.za.men.to}{0}
\verb{cruzar}{}{}{}{}{v.t.}{Dispor em cruz.}{cru.zar}{0}
\verb{cruzar}{}{}{}{}{}{Atravessar.}{cru.zar}{0}
\verb{cruzar}{}{}{}{}{}{Acasalar.}{cru.zar}{0}
\verb{cruzar}{}{}{}{}{}{Encontrar com.}{cru.zar}{\verboinum{1}}
%\verb{}{}{}{}{}{}{}{}{0}
\verb{cruzeiro}{ê}{}{}{}{s.m.}{Cruz erguida em lugar público.}{cru.zei.ro}{0}
\verb{cruzeiro}{ê}{}{}{}{}{Viagem turística em navio.}{cru.zei.ro}{0}
\verb{cruzeta}{ê}{}{}{}{s.f.}{Pequena cruz.}{cru.ze.ta}{0}
\verb{cruzeta}{ê}{}{}{}{}{Régua em forma de \textsc{t}, usada pelos operários para fazer nivelamento.}{cru.ze.ta}{0}
\verb{Cs}{}{Quím.}{}{}{}{Símb. do \textit{césio}.}{Cs}{0}
\verb{csi}{}{}{}{}{s.m.}{Décima quarta letra do alfabeto grego.}{csi}{0}
\verb{ctenóforo}{}{Zool.}{}{}{s.m.}{Espécime dos ctenóforos, filo de animais marinhos, transparentes, de vida solitária, com o corpo em forma de balão, com algumas espécies providas de tentáculos, e que se diferenciam dos cnidários principalmente por não serem urticantes.}{cte.nó.fo.ro}{0}
\verb{Cu}{}{Quím.}{}{}{}{Símb. do \textit{cobre}.}{Cu}{0}
\verb{cu}{}{Pop.}{}{}{s.m.}{Ânus.}{cu}{0}
\verb{cuba}{}{}{}{}{s.f.}{Grande barril de madeira para guardar vinho.}{cu.ba}{0}
\verb{cuba}{}{}{}{}{}{Vasilha grande para usos diversos.}{cu.ba}{0}
\verb{cubagem}{}{}{}{}{s.f.}{Ato ou efeito de cubar.}{cu.ba.gem}{0}
\verb{cubagem}{}{}{}{}{}{Quantidade de unidades cúbicas que se podem conter num determinado espaço.}{cu.ba.gem}{0}
\verb{cubagem}{}{}{}{}{}{Cálculo da capacidade de um recipiente ou de um recinto.}{cu.ba.gem}{0}
\verb{cubano}{}{}{}{}{adj.}{Relativo a Cuba.}{cu.ba.no}{0}
\verb{cubano}{}{}{}{}{s.m.}{Indivíduo natural ou habitante desse país.}{cu.ba.no}{0}
\verb{cubar}{}{}{}{}{v.t.}{Elevar ao cubo.}{cu.bar}{0}
\verb{cubar}{}{}{}{}{}{Avaliar ou medir.}{cu.bar}{0}
\verb{cubar}{}{}{}{}{}{Fazer o cálculo da capacidade de um recipiente ou de um recinto.}{cu.bar}{\verboinum{1}}
\verb{cúbico}{}{}{}{}{adj.}{Relativo a cubo.}{cú.bi.co}{0}
\verb{cúbico}{}{}{}{}{}{Que tem forma de cubo.}{cú.bi.co}{0}
\verb{cúbico}{}{}{}{}{}{Que tem três dimensões.}{cú.bi.co}{0}
\verb{cubículo}{}{}{}{}{s.m.}{Pequeno quarto.}{cu.bí.cu.lo}{0}
\verb{cubículo}{}{}{}{}{}{Cela de convento.}{cu.bí.cu.lo}{0}
\verb{cubismo}{}{}{}{}{s.m.}{Estilo artístico caracterizado pela decomposição e geometrização das formas naturais e que tende a representar os objetos em sua totalidade, como se fossem contemplados simultaneamente por todos os lados.}{cu.bis.mo}{0}
\verb{cubista}{}{}{}{}{adj.2g.}{Relativo ao cubismo.}{cu.bis.ta}{0}
\verb{cubista}{}{}{}{}{}{Diz"-se de quem é adepto ou seguidor desse estilo artístico.}{cu.bis.ta}{0}
\verb{cúbito}{}{Anat.}{}{}{s.m.}{Osso que forma a parte interna do antebraço.}{cú.bi.to}{0}
\verb{cubo}{}{}{}{}{s.m.}{Sólido de seis faces quadradas de igual tamanho.}{cu.bo}{0}
\verb{cubo}{}{}{}{}{}{A terceira potência de um número.}{cu.bo}{0}
\verb{cuca}{}{}{}{}{s.f.}{Ser fantástico com que se assusta criança; bicho"-papão.}{cu.ca}{0}
\verb{cuca}{}{Pop.}{}{}{}{Cabeça.}{cu.ca}{0}
\verb{cuco}{}{Zool.}{}{}{s.m.}{Ave europeia cujo canto é formado por duas notas.}{cu.co}{0}
\verb{cuco}{}{}{}{}{}{Relógio de parede que, quando dá horas, imita o canto dessa ave.}{cu.co}{0}
\verb{cu"-de"-ferro}{é}{Pop.}{cus"-de"-ferro ⟨é⟩}{}{s.2g.}{Pessoa que se dedica excessivamente aos estudos, trabalhos, compromissos etc.; cê"-dê"-efe.}{cu"-de"-fer.ro}{0}
\verb{cueca}{é}{}{}{}{s.f.}{Peça íntima do vestuário masculino.}{cu.e.ca}{0}
\verb{cuecas}{é}{}{}{}{s.m.pl.}{Cueca.}{cu.e.cas}{0}
\verb{cueiro}{ê}{}{}{}{s.m.}{Pano leve e macio que envolve os bebês da cintura para baixo.}{cu.ei.ro}{0}
\verb{cuia}{}{}{}{}{s.f.}{Fruto redondo de casca muito dura.}{cui.a}{0}
\verb{cuia}{}{}{}{}{}{Vasilha feita desse fruto.}{cui.a}{0}
\verb{cuiabano}{}{}{}{}{adj.}{Relativo a Cuiabá, capital do Mato Grosso.}{cui.a.ba.no}{0}
\verb{cuiabano}{}{}{}{}{s.m.}{Indivíduo natural ou habitante dessa cidade.}{cui.a.ba.no}{0}
\verb{cuíca}{}{}{}{}{s.f.}{Instrumento musical formado de uma vara presa e uma pele esticada na boca de um pequeno barril.}{cu.í.ca}{0}
\verb{cuidado}{}{}{}{}{adj.}{Pensado, previsto, feito com esmero.}{cui.da.do}{0}
\verb{cuidado}{}{}{}{}{s.m.}{Atenção, desvelo.}{cui.da.do}{0}
\verb{cuidado}{}{}{}{}{}{Incumbência, responsabilidade.}{cui.da.do}{0}
\verb{cuidado}{}{}{}{}{interj.}{Usada para chamar a atenção, para advertir.}{cui.da.do}{0}
\verb{cuidadoso}{ô}{}{"-osos ⟨ó⟩}{"-osa ⟨ó⟩}{adj.}{Que tem ou denota cuidado.}{cui.da.do.so}{0}
\verb{cuidadoso}{ô}{}{"-osos ⟨ó⟩}{"-osa ⟨ó⟩}{}{Diligente, zeloso.}{cui.da.do.so}{0}
\verb{cuidar}{}{}{}{}{v.t.}{Fazer algo com atenção.}{cui.dar}{0}
\verb{cuidar}{}{}{}{}{}{Tomar conta de outro ou de si mesmo.}{cui.dar}{0}
\verb{cuidar}{}{}{}{}{v.pron.}{Prevenir"-se.}{cui.dar}{\verboinum{1}}
\verb{cuieira}{ê}{Bot.}{}{}{s.f.}{Árvore pequena cujo fruto é uma baga usada como vasilha, cuia ou instrumento musical; cabaceira, coité.}{cui.ei.ra}{0}
\verb{cujo}{}{}{}{}{pron.}{De que, de quem, do(a) qual. (\textit{Esse é o livro cuja ilustração foi premiada.})}{cu.jo}{0}
\verb{cujo}{}{Pop.}{}{}{s.m.}{Qualquer indivíduo, de quem não se sabe ou não se quer dizer o nome; sujeito, fulano, camarada, dito"-cujo.}{cu.jo}{0}
\verb{culatra}{}{}{}{}{s.f.}{Fundo do cano da arma de fogo.}{cu.la.tra}{0}
\verb{culinária}{}{}{}{}{s.f.}{A arte e a técnica de cozinhar.}{cu.li.ná.ria}{0}
\verb{culinária}{}{}{}{}{}{O conjunto de iguarias ou especialidades culinárias de um lugar.}{cu.li.ná.ria}{0}
\verb{culinário}{}{}{}{}{adj.}{Relativo a cozinha.}{cu.li.ná.rio}{0}
\verb{culminância}{}{}{}{}{s.f.}{Ato ou efeito de culminar; culminação, auge, apogeu. }{cul.mi.nân.cia}{0}
\verb{culminante}{}{}{}{}{adj.2g.}{Que culmina; que é o mais alto, elevado.}{cul.mi.nan.te}{0}
\verb{culminar}{}{}{}{}{v.t.}{Atingir o ponto mais alto, o cume; chegar ao máximo.}{cul.mi.nar}{\verboinum{1}}
\verb{culote}{ó}{}{}{}{s.m.}{Espécie de calça larga na parte superior e estreita a partir do joelho, própria para montaria.}{cu.lo.te}{0}
\verb{culote}{ó}{}{}{}{}{Excesso de gordura na face externa da coxa.}{cu.lo.te}{0}
\verb{culpa}{}{}{}{}{s.f.}{Responsabilidade por ação ou omissão que resultaram em danos ou em problemas a outrem.}{cul.pa}{0}
\verb{culpabilidade}{}{}{}{}{s.f.}{Estado ou atributo do que é culpável ou do que é culpado.  }{cul.pa.bi.li.da.de}{0}
\verb{culpado}{}{}{}{}{adj.}{Que tem culpa.}{cul.pa.do}{0}
\verb{culpado}{}{}{}{}{s.m.}{Pessoa que tem culpa.}{cul.pa.do}{0}
\verb{culpado}{}{}{}{}{}{Criminoso, delinquente, réu.}{cul.pa.do}{0}
\verb{culpar}{}{}{}{}{v.t.}{Acusar de culpa; incriminar.}{cul.par}{\verboinum{1}}
\verb{culpável}{}{}{"-eis}{}{adj.2g.}{Passível de ser culpado; a que se pode atribuir culpa.}{cul.pá.vel}{0}
\verb{culposo}{ô}{}{"-osos ⟨ó⟩}{"-osa ⟨ó⟩}{adj.}{Em que há culpa. }{cul.po.so}{0}
\verb{cultivador}{ô}{}{}{}{adj.}{Que cultiva; cultor.}{cul.ti.va.dor}{0}
\verb{cultivador}{ô}{}{}{}{s.m.}{Pessoa que cultiva; agricultor, lavrador.}{cul.ti.va.dor}{0}
\verb{cultivar}{}{}{}{}{v.t.}{Tratar a terra para que produza.}{cul.ti.var}{0}
\verb{cultivar}{}{}{}{}{}{Ocupar"-se com algo a fim de desenvolvê"-lo ou desenvolver"-se nele. (\textit{Devemos cultivar nossa amizade.})}{cul.ti.var}{\verboinum{1}}
\verb{cultivar}{}{}{}{}{s.m.}{Qualquer variedade de planta produzida por meio de técnicas de cultivo, normalmente não encontrada em estado silvestre. }{cul.ti.var}{0}
\verb{cultivo}{}{}{}{}{s.m.}{Ato ou efeito de cultivar; cultura.}{cul.ti.vo}{0}
\verb{culto}{}{}{}{}{s.m.}{Demonstração de admiração e respeito por pessoa ou coisa.}{cul.to}{0}
\verb{culto}{}{}{}{}{}{Ritual religioso.}{cul.to}{0}
\verb{culto}{}{}{}{}{adj.}{Que tem cultura; instruído.}{cul.to}{0}
\verb{cultor}{ô}{}{}{}{s.m.}{Indivíduo que cultiva; cultivador.}{cul.tor}{0}
\verb{cultor}{ô}{}{}{}{}{Indivíduo que se dedica a determinado estudo.}{cul.tor}{0}
\verb{cultual}{}{}{"-ais}{}{adj.2g.}{Relativo ao culto.}{cul.tu.al}{0}
\verb{cultuar}{}{}{}{}{v.t.}{Mostrar respeito e admiração por pessoa ou divindade; venerar, idolatrar. }{cul.tu.ar}{\verboinum{1}}
\verb{cultura}{}{}{}{}{s.f.}{Conjunto de ideias e procedimentos assumidos coletiva e espontaneamente para atingir o mesmo fim.}{cul.tu.ra}{0}
\verb{cultura}{}{}{}{}{}{Conjunto tradicional de conhecimentos, artefatos e ideias consagrado pela tradição histórica de um povo.}{cul.tu.ra}{0}
\verb{cultura}{}{}{}{}{s.f.}{Ato de preparar um terreno, com o efetivo plantio e a manutenção de uma espécie vegetal; roça, roçado.}{cul.tu.ra}{0}
\verb{cultura}{}{}{}{}{}{A espécie vegetal cultivada.}{cul.tu.ra}{0}
\verb{cultura}{}{Biol.}{}{}{}{Preparo de material orgânico para análises laboratoriais.}{cul.tu.ra}{0}
\verb{cultural}{}{}{"-ais}{}{adj.2g.}{Relativo a cultura.}{cul.tu.ral}{0}
\verb{cumbuca}{}{}{}{}{s.f.}{Vaso de abertura pequena feito da casca da cuia.}{cum.bu.ca}{0}
\verb{cumbuca}{}{}{}{}{}{Armadilha para macacos.}{cum.bu.ca}{0}
\verb{cume}{}{}{}{}{s.m.}{O ponto mais alto de um monte; pico, topo.}{cu.me}{0}
\verb{cumeada}{}{}{}{}{s.f.}{Cumes de montanhas em série.}{cu.me.a.da}{0}
\verb{cumeeira}{ê}{}{}{}{s.f.}{A parte mais alta de um telhado.}{cu.me.ei.ra}{0}
\verb{cúmplice}{}{}{}{}{s.2g.}{Pessoa que participa, que toma parte de um crime ou delito; coautor.}{cúm.pli.ce}{0}
\verb{cúmplice}{}{Por ext.}{}{}{}{Pessoa que colabora com outra em alguma coisa; parceiro.}{cúm.pli.ce}{0}
\verb{cumplicidade}{}{}{}{}{s.f.}{Ação, estado ou qualidade de cúmplice.  }{cum.pli.ci.da.de}{0}
\verb{cumprido}{}{}{}{}{adj.}{Que se cumpriu; realizado, feito, executado.}{cum.pri.do}{0}
\verb{cumprimentar}{}{}{}{}{v.t.}{Dirigir cumprimentos; saudar, felicitar.}{cum.pri.men.tar}{\verboinum{1}}
\verb{cumprimento}{}{}{}{}{s.m.}{Ato ou efeito de cumprir.}{cum.pri.men.to}{0}
\verb{cumprimento}{}{}{}{}{}{Ato ou efeito de cumprimentar; saudação.}{cum.pri.men.to}{0}
\verb{cumprir}{}{}{}{}{v.t.}{Realizar, executar, desempenhar.}{cum.prir}{0}
\verb{cumprir}{}{}{}{}{}{Submeter"-se, obedecer, observar. (\textit{Aqueles que não cumprem as leis de trânsito podem ter suas licenças cassadas.})}{cum.prir}{\verboinum{18}}
\verb{cumular}{}{}{}{}{v.t.}{Acumular.}{cu.mu.lar}{0}
\verb{cumular}{}{}{}{}{}{Dar alguma coisa em grande quantidade.}{cu.mu.lar}{\verboinum{1}}
\verb{cumulativo}{}{}{}{}{adj.}{Que cumula, acumula; que tem a propriedade de acumular; acumulativo.}{cu.mu.la.ti.vo}{0}
\verb{cúmulo}{}{}{}{}{s.m.}{O ponto mais alto; o mais alto grau; auge, ápice, máximo.}{cú.mu.lo}{0}
\verb{cuneiforme}{ó}{}{}{}{adj.2g.}{Que tem forma de cunha.}{cu.nei.for.me}{0}
\verb{cunha}{}{}{}{}{s.f.}{Peça de ferro ou de madeira, em forma de ângulo sólido, afiada em uma das extremidades, que se introduz em uma brecha para abrir pedras, rachar lenha, para servir de calço e firmar ou ajustar certas coisas. }{cu.nha}{0}
\verb{cunha}{}{Fig.}{}{}{}{Pessoa influente que se empenha em favor de outra; pistolão.}{cu.nha}{0}
\verb{cunhã}{}{Bras.}{}{}{s.f.}{Mulher.}{cu.nhã}{0}
\verb{cunhã}{}{}{}{}{}{Mulher jovem; moça.}{cu.nhã}{0}
\verb{cunhadio}{}{}{}{}{s.m.}{Grau de parentesco entre cunhados.}{cu.nha.di.o}{0}
\verb{cunhado}{}{}{}{}{s.m.}{Irmão de um dos cônjuges em relação ao outro. }{cu.nha.do}{0}
\verb{cunhado}{}{}{}{}{}{O marido, com relação ao irmão ou à irmã de sua esposa.}{cu.nha.do}{0}
\verb{cunhagem}{}{}{}{}{s.f.}{Ato ou efeito de cunhar (moedas).}{cu.nha.gem}{0}
\verb{cunhar}{}{}{}{}{v.t.}{Dar a forma de cunha; fender.}{cu.nhar}{0}
\verb{cunhar}{}{}{}{}{}{Imprimir o cunho, a marca. }{cu.nhar}{0}
\verb{cunhar}{}{}{}{}{}{Transformar o metal em moedas; amoedar. }{cu.nhar}{0}
\verb{cunhar}{}{}{}{}{}{Inventar, criar.}{cu.nhar}{0}
\verb{cunhar}{}{Fig.}{}{}{}{Tornar notável, saliente.}{cu.nhar}{\verboinum{1}}
\verb{cunhete}{ê}{}{}{}{s.m.}{Caixa de madeira usada para guardar ou transportar munição de guerra.  }{cu.nhe.te}{0}
\verb{cunho}{}{}{}{}{s.m.}{Placa de ferro que imprime marca em moedas, medalhas etc.}{cu.nho}{0}
\verb{cunho}{}{}{}{}{}{Traço característico; marca, selo, impressão.}{cu.nho}{0}
\verb{cunicultor}{ô}{}{}{}{s.m.}{Pessoa que tem criação de coelhos.}{cu.ni.cul.tor}{0}
\verb{cunicultura}{}{}{}{}{s.f.}{Criação de coelhos.}{cu.ni.cul.tu.ra}{0}
\verb{cupão}{}{Desus.}{"-ões}{}{s.m.}{Cupom.}{cu.pão}{0}
\verb{cupê}{}{}{}{}{s.m.}{Carro esporte ou carro de passeio, de duas portas. }{cu.pê}{0}
\verb{cupidez}{ê}{}{}{}{s.f.}{Ato ou atributo de cúpido.}{cu.pi.dez}{0}
\verb{cúpido}{}{}{}{}{adj.}{Que é ávido por dinheiro ou bens materiais; ambicioso, cobiçoso.}{cú.pi.do}{0}
\verb{cupido}{}{Mit.}{}{}{s.m.}{Na mitologia romana, deus do amor, provido de arco e flechas, as quais acertava no coração dos seres humanos.}{cu.pi.do}{0}
\verb{cupido}{}{}{}{}{}{O amor personificado.}{cu.pi.do}{0}
\verb{cupim}{}{Zool.}{"-ins}{}{s.m.}{Nome comum de alguns insetos sociais, que vivem em grandes comunidades e se alimentam de madeira, além de outras partes de plantas; térmita.}{cu.pim}{0}
\verb{cupincha}{}{Bras.}{}{}{s.2g.}{Pessoa pela qual se tem muita amizade; camarada, companheiro, cúmplice, comparsa, amigo.  }{cu.pin.cha}{0}
\verb{cupinzeiro}{ê}{}{}{}{s.m.}{Ninho de cupins.}{cu.pin.zei.ro}{0}
\verb{cupom}{}{}{"-ons}{}{s.m.}{Pedaço de cartão impresso que dá a seu portador certos direitos; tíquete.}{cu.pom}{0}
\verb{cúprico}{}{}{}{}{adj.}{Relativo a cobre; cúpreo.}{cú.pri.co}{0}
\verb{cupuaçu}{}{}{}{}{s.m.}{Árvore nativa da Amazônia, cujo fruto tem uma polpa doce e aromática, e é muito usado em doces e refrescos.}{cu.pu.a.çu}{0}
\verb{cupuaçu}{}{}{}{}{}{O fruto dessa árvore, ou a polpa já separada.}{cu.pu.a.çu}{0}
\verb{cúpula}{}{}{}{}{s.f.}{Grande concavidade na parte superior, interna, de certas edificações.}{cú.pu.la}{0}
\verb{cúpula}{}{Fig.}{}{}{}{As pessoas dirigentes de um partido, organização, empresa, instituição etc., responsáveis pelas decisões; direção, chefia.}{cú.pu.la}{0}
\verb{cúpula}{}{}{}{}{}{Parte do abajur que protege a lâmpada.}{cú.pu.la}{0}
\verb{cura}{}{}{}{}{s.f.}{Ato ou efeito de curar.}{cu.ra}{0}
\verb{cura}{}{}{}{}{}{Restabelecimento da saúde; melhora, recuperação.}{cu.ra}{0}
\verb{cura}{}{}{}{}{}{Ato ou efeito de secar ao sol ou no fumeiro (queijo, peixe, carne etc.).}{cu.ra}{0}
\verb{cura}{}{}{}{}{s.m.}{Pároco.}{cu.ra}{0}
\verb{curaçau}{}{}{}{}{s.m.}{Licor feito com aguardente de cana, casca de laranja amarga, canela e cravo.  }{cu.ra.çau}{0}
\verb{curador}{ô}{}{}{}{adj.}{Que cura.}{cu.ra.dor}{0}
\verb{curador}{ô}{}{}{}{s.m.}{Pessoa encarregada de administrar os bens ou interesses de outrem.}{cu.ra.dor}{0}
\verb{curador}{ô}{}{}{}{}{Curandeiro.}{cu.ra.dor}{0}
\verb{curador}{ô}{}{}{}{}{Curador de artes.}{cu.ra.dor}{0}
\verb{curadoria}{}{}{}{}{s.f.}{Cargo ou função de curador; curatela.}{cu.ra.do.ri.a}{0}
\verb{curandeirismo}{}{Bras.}{}{}{s.m.}{Atividade ou conjunto das práticas dos curandeiros.  }{cu.ran.dei.ris.mo}{0}
\verb{curandeiro}{ê}{}{}{}{s.m.}{Pessoa que tem o poder de curar, ou a quem é atribuído esse poder, por meio de feitiços, beberagens, rezas etc.; benzedeiro.}{cu.ran.dei.ro}{0}
\verb{curar}{}{}{}{}{v.t.}{Restituir a saúde.}{cu.rar}{0}
\verb{curar}{}{}{}{}{v.pron.}{Corrigir a si mesmo de defeito moral ou hábito prejudicial. (\textit{O rapaz curou"-se do vício da bebida.})}{cu.rar}{\verboinum{1}}
\verb{curare}{}{}{}{}{s.m.}{Veneno poderoso e letal, obtido a partir de certas plantas, usado por algumas tribos indígenas na ponta de suas flechas.}{cu.ra.re}{0}
\verb{curatela}{é}{}{}{}{s.f.}{Curadoria.}{cu.ra.te.la}{0}
\verb{curativo}{}{}{}{}{adj.}{Relativo a cura.}{cu.ra.ti.vo}{0}
\verb{curativo}{}{}{}{}{s.m.}{Tratamento, cura.}{cu.ra.ti.vo}{0}
\verb{curativo}{}{}{}{}{}{Ato ou efeito de curar.}{cu.ra.ti.vo}{0}
\verb{curativo}{}{}{}{}{}{Aplicação de remédios, ataduras ou aparelhos num ferimento.}{cu.ra.ti.vo}{0}
\verb{curativo}{}{Por ext.}{}{}{}{O material utilizado nessa aplicação.}{cu.ra.ti.vo}{0}
\verb{curato}{}{}{}{}{s.m.}{Cargo, função ou residência de um cura.}{cu.ra.to}{0}
\verb{curau}{}{Cul.}{}{}{s.m.}{Doce feito com milho verde, leite e açúcar.}{cu.rau}{0}
\verb{curável}{}{}{"-eis}{}{adj.2g.}{Que se pode curar.}{cu.rá.vel}{0}
\verb{cureta}{ê}{}{}{}{s.f.}{Instrumento cirúrgico para raspar, em forma de colher.  }{cu.re.ta}{0}
\verb{curetagem}{}{}{"-ens}{}{s.f.}{Ato ou efeito de curetar; raspagem.  }{cu.re.ta.gem}{0}
\verb{curetar}{}{Med.}{}{}{v.t.}{Fazer raspagem com cureta, para retirar alguma coisa das cavidades do corpo, especialmente do útero.}{cu.re.tar}{\verboinum{1}}
\verb{cúria}{}{}{}{}{s.f.}{Organismo governamental, administrativo e judiciário do Vaticano; a corte papal.}{cú.ria}{0}
\verb{cúria}{}{}{}{}{}{Conjunto de organismos e entidades eclesiásticas que cooperam com o bispo na direção da diocese.}{cú.ria}{0}
\verb{cúria}{}{}{}{}{}{Lugar onde se reunia o senado romano.}{cú.ria}{0}
\verb{curial}{}{}{"-ais}{}{adj.2g.}{Que se refere a cúria.}{cu.ri.al}{0}
\verb{curiango}{}{Zool.}{}{}{s.m.}{Certa ave comum do México a Argentina, de coloração parda com manchas pretas, que se alimenta de insetos e tem hábitos noturnos.}{cu.ri.an.go}{0}
\verb{curiboca}{ó}{Bras.}{}{}{s.2g.}{Caboclo.}{cu.ri.bo.ca}{0}
\verb{curimatã}{}{}{}{}{s.m.}{Nome comum a certos peixes largamente distribuídos pelos rios do Brasil, que se alimentam de lodo e se prestam a piscicultura; papa"-terra. }{cu.ri.ma.tã}{0}
\verb{curinga}{}{}{}{}{s.m.}{Carta de baralho capaz de substituir outras em determinadas situações.}{cu.rin.ga}{0}
\verb{cúrio}{}{Quím.}{}{}{s.m.}{Elemento químico radioativo, do grupo dos actinídeos, obtido artificialmente, usado como fonte de calor em baterias termonucleares. \elemento{96}{(247)}{Cm}.}{cú.rio}{0}
\verb{curió}{}{Zool.}{}{}{s.m.}{Ave largamente distribuída pelo Brasil, canora, cujo macho tem coloração preta, com abdômen vermelho; avinhado. }{cu.ri.ó}{0}
\verb{curiosa}{ó}{Pop.}{}{}{s.f.}{Mulher que faz partos sem ter habilitação legal; parteira.}{cu.ri.o.sa}{0}
\verb{curiosidade}{}{}{}{}{s.f.}{Qualidade ou característica de curioso.}{cu.ri.o.si.da.de}{0}
\verb{curiosidade}{}{}{}{}{}{Desejo de ver, saber ou conhecer alguma coisa. }{cu.ri.o.si.da.de}{0}
\verb{curiosidade}{}{}{}{}{}{Vontade de conhecer a vida e os segredos dos outros; indiscrição, bisbilhotice.}{cu.ri.o.si.da.de}{0}
\verb{curioso}{ô}{}{"-osos ⟨ó⟩}{"-osa ⟨ó⟩}{adj.}{Que tem curiosidade, que deseja saber ou ver alguma coisa.}{cu.ri.o.so}{0}
\verb{curioso}{ô}{}{"-osos ⟨ó⟩}{"-osa ⟨ó⟩}{}{Que desperta interesse; excêntrico, estranho, singular.}{cu.ri.o.so}{0}
\verb{curitibano}{}{}{}{}{adj.}{Relativo a Curitiba, capital do Paraná.}{cu.ri.ti.ba.no}{0}
\verb{curitibano}{}{}{}{}{s.m.}{Indivíduo natural ou habitante dessa cidade.}{cu.ri.ti.ba.no}{0}
\verb{curra}{}{Pop.}{}{}{s.f.}{Ato ou efeito de currar.}{cur.ra}{0}
\verb{curral}{}{}{"-ais}{}{s.m.}{Lugar onde se recolhe o gado.}{cur.ral}{0}
\verb{currar}{}{Pop.}{}{}{v.t.}{Praticar abuso sexual agindo em grupo.}{cur.rar}{\verboinum{1}}
\verb{currículo}{}{}{}{}{s.m.}{Conjunto das disciplinas que compõem um curso.}{cur.rí.cu.lo}{0}
\verb{currículo}{}{}{}{}{}{Documento com dados profissionais relevantes de uma pessoa.}{cur.rí.cu.lo}{0}
\verb{curriculum vitae}{}{}{curricula vitae}{}{s.m.}{Conjunto de dados concernentes ao estado civil, ao preparo profissional e às atividades anteriores de quem se candidata a um emprego, a um concurso etc.; currículo. }{\textit{curriculum vitae}}{0}
\verb{currupira}{}{}{}{}{s.m.}{Curupira.}{cur.ru.pi.ra}{0}
\verb{cursar}{}{}{}{}{v.t.}{Frequentar as aulas (de um curso ou disciplina).}{cur.sar}{\verboinum{1}}
\verb{cursinho}{}{}{}{}{s.m.}{Curso de preparação para o exame vestibular.}{cur.si.nho}{0}
\verb{cursivo}{}{}{}{}{adj.}{Diz"-se de letra ou caractere que se faz sem retirar o lápis ou caneta do papel.}{cur.si.vo}{0}
\verb{curso}{}{}{}{}{s.m.}{Ato de correr; corrida, movimento.}{cur.so}{0}
\verb{curso}{}{}{}{}{}{Caminho, direção.}{cur.so}{0}
\verb{curso}{}{}{}{}{}{Sucessão, seguimento, encaminhamento.}{cur.so}{0}
\verb{curso}{}{}{}{}{}{O caminho percorrido pelas águas de um rio ou córrego; leito.}{cur.so}{0}
\verb{curso}{}{}{}{}{}{Conjunto de disciplinas sobre um assunto determinado.}{cur.so}{0}
\verb{cursor}{ô}{Informát.}{}{}{s.m.}{Marca luminosa na tela de um terminal indicativa do ponto onde se aplicará a próxima ação do usuário.}{cur.sor}{0}
\verb{curta"-metragem}{}{}{curtas"-metragens}{}{s.m.}{Filme de curta duração, realizado para fins artísticos, educativos, comerciais etc.}{cur.ta"-me.tra.gem}{0}
\verb{curtição}{}{}{"-ões}{}{s.f.}{Ato ou efeito de curtir.}{cur.ti.ção}{0}
\verb{curtição}{}{Pop.}{"-ões}{}{}{Experiência prazerosa; barato.}{cur.ti.ção}{0}
\verb{curtido}{}{}{}{}{adj.}{Que se curtiu.}{cur.ti.do}{0}
\verb{curtir}{}{}{}{}{v.t.}{Preparar (couro) para impedir que apodreça.}{cur.tir}{0}
\verb{curtir}{}{}{}{}{}{Sofrer, suportar, padecer.}{cur.tir}{0}
\verb{curtir}{}{Pop.}{}{}{}{Desfrutar, deleitar, fruir.}{cur.tir}{\verboinum{18}}
\verb{curto}{}{}{}{}{adj.}{De pequena extensão ou duração.}{cur.to}{0}
\verb{curto}{}{}{}{}{}{Escasso, pouco.}{cur.to}{0}
\verb{curto}{}{Fig.}{}{}{}{De inteligência ou responsabilidade escassa.}{cur.to}{0}
\verb{curto"-circuito}{}{}{curtos"-circuitos}{}{s.m.}{Contato entre dois fios elétricos, que desliga a energia.}{cur.to"-cir.cui.to}{0}
\verb{curtume}{}{}{}{}{s.m.}{Ato ou efeito de curtir; curtimento.}{cur.tu.me}{0}
\verb{curtume}{}{}{}{}{}{Lugar onde se curte o couro.}{cur.tu.me}{0}
\verb{curumi}{}{}{}{}{s.m.}{Curumim.}{cu.ru.mi}{0}
\verb{curumim}{}{}{}{}{s.m.}{Menino índio.}{cu.ru.mim}{0}
\verb{curupira}{}{}{}{}{s.m.}{Ser mitológico que vive nas matas e tem os pés virados para trás.}{cu.ru.pi.ra}{0}
\verb{cururu}{}{Zool.}{}{}{s.m.}{Certo sapo de grande porte; sapo"-cururu.}{cu.ru.ru}{0}
\verb{curva}{}{}{}{}{s.f.}{Linha arqueada.}{cur.va}{0}
\verb{curva}{}{}{}{}{}{Trecho sinuoso de estrada.}{cur.va}{0}
\verb{curva}{}{}{}{}{}{Qualquer objeto ou figura de forma arqueada.}{cur.va}{0}
\verb{curvado}{}{}{}{}{adj.}{Que se curvou; curvo.}{cur.va.do}{0}
\verb{curvar}{}{}{}{}{v.t.}{Tornar curvo; arquear.}{cur.var}{0}
\verb{curvar}{}{}{}{}{v.pron.}{Prostrar"-se, ajoelhar"-se, geralmente em sinal de reverência ou respeito.}{cur.var}{\verboinum{1}}
\verb{curvatura}{}{}{}{}{s.f.}{A forma curva de um corpo.}{cur.va.tu.ra}{0}
\verb{curveta}{ê}{}{}{}{s.f.}{Pequena curva.}{cur.ve.ta}{0}
\verb{curveta}{ê}{}{}{}{}{Certo movimento do cavalo; pinote.}{cur.ve.ta}{0}
\verb{curvetear}{}{}{}{}{v.i.}{Fazer curvetas.}{cur.ve.te.ar}{\verboinum{4}}
\verb{curvilíneo}{}{}{}{}{adj.}{Que tem formas curvas.}{cur.vi.lí.neo}{0}
\verb{curvo}{}{}{}{}{adj.}{Que tem forma de arco; arqueado.}{cur.vo}{0}
\verb{curvo}{}{Geom.}{}{}{}{Que não forma ângulos mas também não é reto.}{cur.vo}{0}
\verb{curvo}{}{}{}{}{}{Torto, sinuoso.}{cur.vo}{0}
\verb{cuscuz}{}{Cul.}{}{}{s.m.}{Tipo de bolo salgado à base de farinha de milho ou arroz e com recheios diversos.}{cus.cuz}{0}
\verb{cusparada}{}{}{}{}{s.f.}{Porção de cuspe.}{cus.pa.ra.da}{0}
\verb{cusparada}{}{}{}{}{}{Ato de cuspir.}{cus.pa.ra.da}{0}
\verb{cuspe}{}{}{}{}{s.m.}{Líquido produzido pelas glândulas salivares; saliva.}{cus.pe}{0}
\verb{cuspideira}{ê}{}{}{}{s.f.}{Recipiente próprio para cuspir.}{cus.pi.dei.ra}{0}
\verb{cuspinhar}{}{}{}{}{v.i.}{Cuspir constantemente e em pequenas porções.}{cus.pi.nhar}{\verboinum{1}}
\verb{cuspir}{}{}{}{}{v.i.}{Lançar cuspe.}{cus.pir}{0}
\verb{cuspir}{}{Fig.}{}{}{v.t.}{Lançar fora (qualquer coisa); expelir, ejetar.}{cus.pir}{\verboinum{18}}
\verb{cuspo}{}{}{}{}{s.m.}{Cuspe.}{cus.po}{0}
\verb{custa}{}{}{}{}{s.f.}{Despesa, gasto.}{cus.ta}{0}
\verb{custar}{}{}{}{}{v.t.}{Ter o preço de.}{cus.tar}{0}
\verb{custar}{}{Fig.}{}{}{}{Demandar esforço ou ocasionar prejuízo.}{cus.tar}{0}
\verb{custar}{}{}{}{}{}{Tardar, demorar.}{cus.tar}{0}
\verb{custar}{}{}{}{}{v.i.}{Ser penoso, cansativo, difícil.}{cus.tar}{\verboinum{1}}
\verb{custas}{}{Jur.}{}{}{s.f.pl.}{Despesas geradas por processo judicial.}{cus.tas}{0}
\verb{custear}{}{}{}{}{v.t.}{Assumir as despesas de; financiar, bancar.}{cus.te.ar}{\verboinum{4}}
\verb{custeio}{ê}{}{}{}{s.m.}{Ato ou efeito de custear.}{cus.tei.o}{0}
\verb{custo}{}{}{}{}{s.m.}{Valor de alguma coisa; preço.}{cus.to}{0}
\verb{custo}{}{}{}{}{}{Esforço físico ou mental; dificuldade.}{cus.to}{0}
\verb{custo}{}{}{}{}{}{Demora.}{cus.to}{0}
\verb{custódia}{}{}{}{}{s.f.}{Guarda, proteção.}{cus.tó.dia}{0}
\verb{custodiar}{}{}{}{}{v.t.}{Guardar, proteger.}{cus.to.di.ar}{\verboinum{6}}
\verb{custódio}{}{}{}{}{adj.}{Que guarda, protege.}{cus.tó.dio}{0}
\verb{custoso}{ô}{}{"-osos ⟨ó⟩}{"-osa ⟨ó⟩}{adj.}{Que custa muito; caro.}{cus.to.so}{0}
\verb{custoso}{ô}{}{"-osos ⟨ó⟩}{"-osa ⟨ó⟩}{}{Árduo, trabalhoso, difícil.}{cus.to.so}{0}
\verb{custoso}{ô}{}{"-osos ⟨ó⟩}{"-osa ⟨ó⟩}{}{Que tarda muito; demorado.}{cus.to.so}{0}
\verb{cutâneo}{}{}{}{}{adj.}{Relativo à cutis, à pele.}{cu.tâ.neo}{0}
\verb{cutelaria}{}{}{}{}{s.f.}{Oficina, técnica ou ofício de cuteleiro.}{cu.te.la.ri.a}{0}
\verb{cuteleiro}{ê}{}{}{}{s.m.}{Fabricante ou vendedor de utensílios cortantes, como facas, lâminas, canivetes.}{cu.te.lei.ro}{0}
\verb{cutelo}{é}{}{}{}{s.m.}{Tipo de utensílio cortante, com lâmina convexa.}{cu.te.lo}{0}
\verb{cutia}{}{Zool.}{}{}{s.f.}{Mamífero roedor de pequeno porte, muito procurado como caça.}{cu.ti.a}{0}
\verb{cutícula}{}{}{}{}{s.f.}{Pequeno tecido de pele, especialmente o que se forma na base das unhas.}{cu.tí.cu.la}{0}
\verb{cutilada}{}{}{}{}{s.f.}{Golpe de cutelo ou de outro utensílio cortante.}{cu.ti.la.da}{0}
\verb{cútis}{}{}{}{}{s.f.}{Pele, epiderme.}{cú.tis}{0}
\verb{cutucada}{}{}{}{}{s.f.}{Ato de cutucar; cutucão.}{cu.tu.ca.da}{0}
\verb{cutucão}{}{}{"-ões}{}{s.m.}{Ato de cutucar; cutucada.}{cu.tu.cão}{0}
\verb{cutucar}{}{}{}{}{v.t.}{Tocar alguém de leve, geralmente para chamar a atenção.}{cu.tu.car}{0}
\verb{cutucar}{}{}{}{}{}{Ferir levemente com instrumento perfurante ou cortante.}{cu.tu.car}{\verboinum{2}}
\verb{czar}{}{}{}{czarina}{s.m.}{Antigo título do soberano da Rússia e de outros povos eslavos.}{czar}{0}
\verb{czarda}{}{}{}{}{s.f.}{Tipo de dança e música húngaras, de origem cigana; xarda. }{czar.da}{0}
\verb{czaréviche}{}{}{}{}{s.m.}{Título dado aos filhos do czar.}{cza.ré.vi.che}{0}
\verb{czarina}{}{}{}{}{s.f.}{Esposa do czar.}{cza.ri.na}{0}
\verb{czarismo}{}{}{}{}{s.m.}{Sistema político autocrático que vigorou na Rússia até a revolução de 1917.}{cza.ris.mo}{0}
\verb{czarista}{}{}{}{}{adj.2g.}{Relativo a czarismo.}{cza.ris.ta}{0}
\verb{czarista}{}{}{}{}{}{Diz"-se de indivíduo partidário do czarismo.}{cza.ris.ta}{0}
