\verb{p}{}{}{}{}{s.m.}{Décima sexta letra do alfabeto português. }{p}{0}
\verb{P}{}{Quím.}{}{}{}{Símb. do \textit{fósforo}.}{P}{0}
\verb{pá}{}{}{}{}{}{A parte mais larga e carnuda da perna das reses.}{pá}{0}
\verb{Pa}{}{Quím.}{}{}{}{Símb. do \textit{protactínio}.}{Pa}{0}
\verb{PA}{}{}{}{}{}{Sigla do estado do Pará.}{PA}{0}
\verb{pá}{}{}{}{}{s.f.}{Ferramenta com uma chapa larga e mais funda no meio, presa a um cabo comprido. (\textit{Com a pá, enchi a valeta de pedras.})}{pá}{0}
\verb{pabulagem}{}{}{"-ens}{}{s.f.}{Pedantismo, orgulho.}{pa.bu.la.gem}{0}
\verb{pabulagem}{}{}{"-ens}{}{}{Mentira, embuste.}{pa.bu.la.gem}{0}
\verb{paca}{}{Zool.}{}{}{s.f.}{Mamífero roedor que vive nas proximidades de lagos e rios, de pelagem parda com listras brancas, e cuja carne é apreciada.}{pa.ca}{0}
\verb{pacatez}{ê}{}{}{}{s.f.}{Qualidade ou condição de pacato.}{pa.ca.tez}{0}
\verb{pacato}{}{}{}{}{adj.}{Que aprecia a paz, o sossego; pacífico, tranquilo, paciente, sossegado.}{pa.ca.to}{0}
\verb{pacato}{}{}{}{}{}{Em que há paz e tranquilidade. (\textit{Cidade pacata.})}{pa.ca.to}{0}
\verb{pacato}{}{}{}{}{s.m.}{Pessoa pacífica, serena. }{pa.ca.to}{0}
\verb{pachola}{ó}{}{}{}{adj.2g.}{Que não gosta de trabalhar; preguiçoso, vadio.}{pa.cho.la}{0}
\verb{pachola}{ó}{}{}{}{}{Pedante, cheio de si.}{pa.cho.la}{0}
\verb{pachorra}{ô}{}{}{}{s.f.}{Falta de pressa ou de empenho; lentidão, vagar, fleuma; calma indolente.}{pa.chor.ra}{0}
\verb{pachorrento}{}{}{}{}{adj.}{Que demonstra pachorra.}{pa.chor.ren.to}{0}
\verb{pachorrento}{}{}{}{}{}{Que é feito com pachorra.}{pa.chor.ren.to}{0}
\verb{pachorrento}{}{}{}{}{}{Que tem pachorra.}{pa.chor.ren.to}{0}
\verb{paciência}{}{}{}{}{s.f.}{Qualidade do indivíduo que é capaz de persistir numa atividade penosa com resultados incertos. (\textit{Ele teve uma paciência infinita para suportar todos os tratamentos de sua doença.})}{pa.ci.ên.cia}{0}
\verb{paciência}{}{}{}{}{}{Jogo solitário e individual em que se concorre contra o tempo, o azar ou ambos. (\textit{Não suporto mais passar tantas madrugadas jogando paciência.})}{pa.ci.ên.cia}{0}
\verb{pacientar}{}{}{}{}{v.i.}{Ter paciência.}{pa.ci.en.tar}{\verboinum{1}}
\verb{paciente}{}{}{}{}{adj.2g.}{Que tem paciência. (\textit{Ele é um moço paciente com todos.})}{pa.ci.en.te}{0}
\verb{paciente}{}{Gram.}{}{}{}{Que é o objeto da ação indicada pelo verbo. (\textit{Independe dizermos "O rapaz abandonou o carro na rua" ou "O carro foi abandonado na rua pelo rapaz", pois nos dois casos termos "o carro" como o termo paciente da ação de "abandonar".})}{pa.ci.en.te}{0}
\verb{paciente}{}{}{}{}{s.2g.}{Indivíduo que está sob cuidados médicos. (\textit{Todos os pacientes são muito bem tratados naquele hospital.})}{pa.ci.en.te}{0}
\verb{pacificação}{}{}{"-ões}{}{s.f.}{Ato ou efeito de pacificar.}{pa.ci.fi.ca.ção}{0}
\verb{pacificador}{ô}{}{}{}{adj.}{Que pacifica; pacificante, apaziguador.  }{pa.ci.fi.ca.dor}{0}
\verb{pacificador}{ô}{}{}{}{s.m.}{Pessoa que pacifica.  }{pa.ci.fi.ca.dor}{0}
\verb{pacificar}{}{}{}{}{v.t.}{Restituir a paz; apaziguar, serenar, tranquilizar.}{pa.ci.fi.car}{\verboinum{2}}
\verb{pacífico}{}{}{}{}{adj.}{Que tem propensão à paz e às boas relações. (\textit{Ele é um indivíduo pacífico que se dá bem com todos.})}{pa.cí.fi.co}{0}
\verb{pacífico}{}{}{}{}{}{Que não pode ser questionado, indiscutível. (\textit{Chegamos a uma conclusão pacífica para todos.})}{pa.cí.fi.co}{0}
\verb{pacífico}{}{}{}{}{}{Referente ao Oceano Pacífico. (\textit{Na costa pacífica do México há cidades encantadoras.})}{pa.cí.fi.co}{0}
\verb{pacifismo}{}{}{}{}{s.m.}{Doutrina que defende a paz entre as nações e a ideia de que os conflitos internacionais não devem ser resolvidos pela guerra, mas por meio de negociações e arbitragens.}{pa.ci.fis.mo}{0}
\verb{pacifista}{}{}{}{}{adj.2g.}{Relativo ao pacifismo.}{pa.ci.fis.ta}{0}
\verb{pacifista}{}{}{}{}{s.2g.}{Pessoa partidária do pacifismo.}{pa.ci.fis.ta}{0}
\verb{paco}{}{Bras.}{}{}{s.m.}{Pacote de papéis sem importância recortados no tamanho de cédulas, cobertos por uma nota verdadeira, e usado pelos vigaristas ao aplicarem o conto"-do"-vigário.  }{pa.co}{0}
\verb{paço}{}{}{}{}{s.m.}{Palácio ou residência de reis, bispos, governadores etc.}{pa.ço}{0}
\verb{paço}{}{}{}{}{}{A corte; os cortesãos.}{pa.ço}{0}
\verb{pacoba}{ô}{}{}{}{}{Var. de \textit{pacova}.}{pa.co.ba}{0}
\verb{paçoca}{ó}{}{}{}{s.f.}{Doce de amendoim torrado e socado com açúcar e farinha.}{pa.ço.ca}{0}
\verb{paçoca}{ó}{}{}{}{}{Carne assada e desfiada, socada no pilão com farinha de mandioca ou de milho.}{pa.ço.ca}{0}
\verb{pacote}{ó}{}{}{}{s.m.}{Quaisquer objetos ou conjunto de objetos embrulhados ou amarrados. (\textit{Ele trouxe um pacote de telhas amarradas com barbante que facilmente se rompeu.})}{pa.co.te}{0}
\verb{pacote}{ó}{}{}{}{}{Conjunto de produtos ou de atividades que se adquirem ou se realizam de uma só vez. (\textit{Eles vendem programas que são pacotes estatísticos, capazes de fazer qualquer cálculo que se necessite.})}{pa.co.te}{0}
\verb{pacova}{ô}{Bras.}{}{}{s.f.}{Banana ou bananeira.}{pa.co.va}{0}
\verb{pacóvio}{}{}{}{}{adj.}{Que é idiota, imbecil, tolo.}{pa.có.vio}{0}
\verb{pacóvio}{}{}{}{}{s.m.}{Pessoa tola; parvo, imbecil, idiota.}{pa.có.vio}{0}
\verb{pacto}{}{}{}{}{s.m.}{Acordo, convenção que se institui entre duas ou mais pessoas, agremiações ou entidades políticas. (\textit{Todos fizeram um pacto de silêncio para que ninguém fosse condenado.})}{pac.to}{0}
\verb{pactuante}{}{}{}{}{adj.2g.}{Que pactua; pactário. }{pac.tu.an.te}{0}
\verb{pactuante}{}{}{}{}{s.2g.}{Pessoa que pactua; pactuário.}{pac.tu.an.te}{0}
\verb{pactuar}{}{}{}{}{v.t.}{Fazer pacto, acordo. (\textit{Uma vez que todos já pactuavam em favor do benefício coletivo, era preciso começar as atividades.})}{pac.tu.ar}{\verboinum{1}}
\verb{pactuário}{}{}{}{}{s.m.}{Pessoa que pactua, que faz ou tem pacto.}{pac.tu.á.rio}{0}
\verb{pacu}{}{Zool.}{}{}{s.m.}{Nome comum a certos peixes de água doce, de corpo ovalado e comprimido, e que se alimentam de frutos e insetos.}{pa.cu}{0}
\verb{pacuera}{é}{Bras.}{}{}{s.f.}{As vísceras de boi, porco ou carneiro.}{pa.cu.e.ra}{0}
\verb{padaria}{}{}{}{}{s.f.}{Lugar em que se faz ou se vende pão.}{pa.da.ri.a}{0}
\verb{padecente}{}{}{}{}{adj.2g.}{Que padece; padecedor.}{pa.de.cen.te}{0}
\verb{padecente}{}{}{}{}{s.2g.}{Pessoa que padece.}{pa.de.cen.te}{0}
\verb{padecer}{ê}{}{}{}{v.t.}{Sofrer física ou moralmente; aguentar, suportar. (\textit{Ele padece todas as injustiças de um julgamento precipitado.})}{pa.de.cer}{0}
\verb{padecer}{ê}{}{}{}{}{Sofrer devido a ação de um agente externo. (\textit{Sob a ação do sol, não há quem não padeça.})}{pa.de.cer}{\verboinum{15}}
\verb{padecimento}{}{}{}{}{s.m.}{Ato ou efeito de padecer; sofrimento.}{pa.de.ci.men.to}{0}
\verb{padeiro}{ê}{}{}{}{s.m.}{Profissional que faz ou vende pão.}{pa.dei.ro}{0}
\verb{padiola}{ó}{}{}{}{s.f.}{Cama portátil para transporte de emergência de doentes e feridos; maca.  }{pa.di.o.la}{0}
\verb{padioleiro}{ê}{}{}{}{s.m.}{Cada uma das pessoas que carregam uma padiola.}{pa.di.o.lei.ro}{0}
\verb{padioleiro}{ê}{}{}{}{}{Soldado que retira os feridos do campo de batalha.}{pa.di.o.lei.ro}{0}
\verb{padrão}{}{}{"-ões}{}{s.m.}{Objeto que serve de modelo a outro. }{pa.drão}{0}
\verb{padrão}{}{}{"-ões}{}{}{Conjunto de procedimentos que garantem a qualidade de algo.}{pa.drão}{0}
\verb{padrão}{}{}{"-ões}{}{}{Desenho decorativo estampado em tecido ou em outro material. }{pa.drão}{0}
\verb{padrasto}{}{}{madrasta}{}{s.m.}{O novo marido da mãe.}{pa.dras.to}{0}
\verb{padre}{}{}{}{madre}{s.m.}{Indivíduo que já recebeu ordenação sacerdotal; sacerdote; presbítero, reverendo.  }{pa.dre}{0}
\verb{padrear}{}{}{}{}{v.i.}{Reproduzir"-se, procriar (especialmente animais).}{pa.dre.ar}{\verboinum{4}}
\verb{padre"-cura}{}{}{padres"-curas}{}{s.m.}{Pároco.}{pa.dre"-cu.ra}{0}
\verb{padre"-nosso}{ó}{}{padre"-nossos \textit{ou} padres"-nossos ⟨ó⟩}{}{s.m.}{Oração religiosa; pai"-nosso.}{pa.dre"-nos.so}{0}
\verb{padrinho}{}{}{}{madrinha}{s.m.}{Homem que, na cerimônia de batismo ou crisma, assume o compromisso de exercer as atribuições dos pais na ausência deles.}{pa.dri.nho}{0}
\verb{padrinho}{}{}{}{}{}{Homem que, no ritual de casamento civil ou religioso, é escolhido por cada um dos noivos como testemunha.}{pa.dri.nho}{0}
\verb{padrinho}{}{}{}{}{}{Pessoa que representa o espírito de uma entidade, grupo ou corporação; protetor, patrono. }{pa.dri.nho}{0}
\verb{padroeiro}{ê}{}{}{}{adj.}{Que tem o direito de padroado.}{pa.dro.ei.ro}{0}
\verb{padroeiro}{ê}{}{}{}{}{Que defende, protege.}{pa.dro.ei.ro}{0}
\verb{padroeiro}{ê}{}{}{}{s.m.}{Santo protetor de uma cidade, de uma ordem religiosa, de uma corporação profissional etc. (\textit{São Cristóvão é o padroeiro dos motoristas.})}{pa.dro.ei.ro}{0}
\verb{padronização}{}{}{"-ões}{}{s.f.}{Ato ou efeito de padronizar.}{pa.dro.ni.za.ção}{0}
\verb{padronização}{}{}{"-ões}{}{}{Uniformização de comportamentos ou objetos, segundo modelos preestabelecidos.}{pa.dro.ni.za.ção}{0}
\verb{padronizado}{}{}{}{}{adj.}{Que se padronizou, que foi reduzido a um tipo apenas; estandartizado.  }{pa.dro.ni.za.do}{0}
\verb{padronizar}{}{}{}{}{v.t.}{Submeter a uma padronização; estabelecer um padrão; uniformizar. }{pa.dro.ni.zar}{\verboinum{1}}
\verb{paetê}{}{}{}{}{s.m.}{Bordado de lantejoulas.}{pa.e.tê}{0}
\verb{paga}{}{}{}{}{s.f.}{Pagamento.}{pa.ga}{0}
\verb{paga}{}{}{}{}{}{Retribuição, recompensa, remuneração. }{pa.ga}{0}
\verb{pagador}{ô}{}{}{}{adj.}{Que paga.}{pa.ga.dor}{0}
\verb{pagador}{ô}{}{}{}{s.m.}{Pessoa que paga, que faz os pagamentos. }{pa.ga.dor}{0}
\verb{pagadoria}{}{}{}{}{s.f.}{Lugar ou repartição pública onde se fazem pagamentos.}{pa.ga.do.ri.a}{0}
\verb{pagamento}{}{}{}{}{s.m.}{Ato ou efeito de pagar; paga.}{pa.ga.men.to}{0}
\verb{pagamento}{}{}{}{}{}{O que se recebe em troca do que se fez; retribuição.  }{pa.ga.men.to}{0}
\verb{pagamento}{}{}{}{}{}{Quantia que o trabalhador recebe ao fim de um período determinado; ordenado, salário.  }{pa.ga.men.to}{0}
\verb{paganismo}{}{}{}{}{s.m.}{O conjunto das pessoas que não foram batizadas; gentilidade.}{pa.ga.nis.mo}{0}
\verb{paganismo}{}{}{}{}{}{Religião em que se adoram muitos deuses; politeísmo.}{pa.ga.nis.mo}{0}
\verb{paganizar}{}{}{}{}{v.t.}{Tornar pagão.}{pa.ga.ni.zar}{0}
\verb{paganizar}{}{}{}{}{v.i.}{Pensar ou agir como pagão.}{pa.ga.ni.zar}{\verboinum{1}}
\verb{pagante}{}{}{}{}{adj.2g.}{Que paga.}{pa.gan.te}{0}
\verb{pagante}{}{}{}{}{s.2g.}{A pessoa que paga.}{pa.gan.te}{0}
\verb{pagão}{}{}{"-ãos}{pagã}{adj.}{Relativo ao paganismo.}{pa.gão}{0}
\verb{pagão}{}{}{"-ãos}{pagã}{}{Que não foi batizado ou não é cristão.}{pa.gão}{0}
\verb{pagão}{}{}{"-ãos}{pagã}{s.m.}{Pessoa que não foi batizada.}{pa.gão}{0}
\verb{pagão}{}{}{"-ãos}{pagã}{}{Pessoa que segue qualquer religião que não adota o batismo, ou que segue religião politeísta.}{pa.gão}{0}
\verb{pagar}{}{}{}{}{v.t.}{Dar remuneração em troca de um serviço feito; gratificar, recompensar.}{pa.gar}{0}
\verb{pagar}{}{}{}{}{}{Satisfazer o valor ou preço de algo. (\textit{Eu paguei pelo carro um preço abaixo da tabela.})}{pa.gar}{0}
\verb{pagar}{}{}{}{}{}{Sofrer um castigo por falta cometida. }{pa.gar}{\verboinum{5}}
\verb{pagável}{}{}{"-eis}{}{adj.2g.}{Que pode ou deve ser pago.  }{pa.gá.vel}{0}
\verb{pager}{}{}{}{}{s.m.}{Aparelho eletrônico portátil de comunicação que recebe sinais ou mensagens escritas; bipe.}{\textit{pager}}{0}
\verb{página}{}{}{}{}{s.f.}{Cada lado da folha de livro, revista etc.}{pá.gi.na}{0}
\verb{página}{}{}{}{}{}{Texto nele contido.}{pá.gi.na}{0}
\verb{paginação}{}{}{"-ões}{}{s.f.}{Ato ou efeito de paginar. }{pa.gi.na.ção}{0}
\verb{paginação}{}{}{"-ões}{}{}{Ordenação numérica das páginas de um livro.}{pa.gi.na.ção}{0}
\verb{paginador}{ô}{}{}{}{s.m.}{Pessoa encarregada de paginar uma publicação.}{pa.gi.na.dor}{0}
\verb{paginar}{}{}{}{}{v.t.}{Numerar páginas impressas em ordem crescente.}{pa.gi.nar}{0}
\verb{paginar}{}{}{}{}{}{Arranjar graficamente as páginas de um impresso.}{pa.gi.nar}{\verboinum{1}}
\verb{pago}{}{}{}{}{adj.}{Que se pagou.}{pa.go}{0}
\verb{pago}{}{}{}{}{}{Que recebeu paga; remunerado, recompensado.}{pa.go}{0}
\verb{pago}{}{}{}{}{}{Dado em pagamento.}{pa.go}{0}
\verb{pago}{}{}{}{}{}{Vingado, desforrado.}{pa.go}{0}
\verb{pagode}{ó}{}{}{}{s.m.}{Estilo de samba.}{pa.go.de}{0}
\verb{pagode}{ó}{}{}{}{}{Reunião informal em que se toca samba.}{pa.go.de}{0}
\verb{pagode}{ó}{}{}{}{}{Casa de oração de alguns povos asiáticos.}{pa.go.de}{0}
\verb{pagodear}{}{}{}{}{v.i.}{Farrear, pandegar, divertir"-se.}{pa.go.de.ar}{\verboinum{4}}
\verb{pagodeira}{ê}{}{}{}{s.f.}{Pagode.}{pa.go.dei.ra}{0}
\verb{pagodeiro}{ê}{Bras.}{}{}{s.m.}{Indivíduo que compõe, canta, toca ou frequenta pagodes.}{pa.go.dei.ro}{0}
\verb{pai}{}{}{}{mãe}{s.m.}{Homem que deu origem a outro ser; genitor.}{pai}{0}
\verb{pai}{}{}{}{}{}{Homem que inventou ou fundou alguma coisa.}{pai}{0}
\verb{pai"-de"-santo}{}{}{pais"-de"-santo}{}{s.m.}{Nos rituais afro"-brasileiros, nome dado ao chefe espiritual e administrador da casa; babalorixá.}{pai"-de"-san.to}{0}
\verb{pai"-de"-terreiro}{ê}{Bras.}{pais"-de"-terreiro}{mãe"-de"-terreiro}{s.m.}{Pai"-de"-santo.}{pai"-de"-ter.rei.ro}{0}
\verb{pai"-de"-todos}{ô}{}{pais"-de"-todos ⟨ô⟩}{}{s.m.}{O dedo médio da mão.}{pai"-de"-to.dos}{0}
\verb{pai"-dos"-burros}{}{Pop.}{pais"-dos"-burros}{}{s.m.}{Dicionário.}{pai"-dos"-bur.ros}{0}
\verb{paina}{}{}{}{}{s.f.}{Fibra sedosa que cobre várias sementes, especialmente as da paineira.}{pai.na}{0}
\verb{painço}{}{Bot.}{}{}{s.m.}{Tipo de capim comestível.}{pa.in.ço}{0}
\verb{painço}{}{}{}{}{}{A semente desse capim.}{pa.in.ço}{0}
\verb{paineira}{ê}{Bot.}{}{}{s.f.}{Árvore que chega a atingir 30 m, de flores róseas, cujo fruto fornece a paina, envolta nas sementes.}{pai.nei.ra}{0}
\verb{painel}{é}{}{"-éis}{}{s.m.}{Pintura feita sobre tela ou madeira; quadro.}{pai.nel}{0}
\verb{painel}{é}{}{"-éis}{}{}{Obra de arte que cobre uma parede ou parte dela.}{pai.nel}{0}
\verb{painel}{é}{}{"-éis}{}{}{Grande placa com anúncio em lugar público.}{pai.nel}{0}
\verb{painel}{é}{}{"-éis}{}{}{Chapa em que aparecem instrumentos de controle.}{pai.nel}{0}
\verb{pai"-nosso}{ó}{Relig.}{pais"-nossos ⟨ó⟩}{}{s.m.}{Oração religiosa; padre"-nosso.}{pai"-nos.so}{0}
\verb{paio}{}{}{}{}{s.m.}{Tipo de linguiça de carne de porco embutida em tripa de intestino grosso; linguiça de padre.}{pai.o}{0}
\verb{paiol}{ó}{}{"-óis}{}{s.m.}{Depósito de pólvora e de outras munições.}{pai.ol}{0}
\verb{paioleiro}{ê}{}{}{}{s.m.}{Guarda de paiol.}{pai.o.lei.ro}{0}
\verb{pairar}{}{}{}{}{v.t.}{Voar devagar.}{pai.rar}{0}
\verb{pairar}{}{}{}{}{}{Parar no ar.}{pai.rar}{0}
\verb{pairar}{}{}{}{}{}{Estar iminente; ameaçar.}{pai.rar}{\verboinum{1}}
\verb{país}{}{}{}{}{s.m.}{Território de uma nação.}{pa.ís}{0}
\verb{país}{}{}{}{}{}{O povo desse território.}{pa.ís}{0}
\verb{país}{}{}{}{}{}{Pátria, terra.}{pa.ís}{0}
\verb{paisagem}{}{}{"-ens}{}{s.f.}{Extensão de terreno que se avista de um determinado ponto.}{pai.sa.gem}{0}
\verb{paisagem}{}{}{"-ens}{}{}{Panorama.}{pai.sa.gem}{0}
\verb{paisagismo}{}{}{}{}{s.m.}{Desenho ou pintura de paisagens.}{pai.sa.gis.mo}{0}
\verb{paisagismo}{}{}{}{}{}{Planejamento arquitetônico de paisagem.}{pai.sa.gis.mo}{0}
\verb{paisagista}{}{}{}{}{adj.2g.}{Diz"-se de indivíduo que se dedica ao paisagismo.}{pai.sa.gis.ta}{0}
\verb{paisagística}{}{}{}{}{s.f.}{A arte do paisagista, de pintar e descrever paisagens.}{pai.sa.gís.ti.ca}{0}
\verb{paisagístico}{}{}{}{}{adj.}{Relativo a paisagem.}{pai.sa.gís.ti.co}{0}
\verb{paisagístico}{}{}{}{}{}{Relativo a paisagismo.}{pai.sa.gís.ti.co}{0}
\verb{paisana}{}{}{}{}{}{Usado na locução \textit{à paisana}: em traje civil, não militar.}{pai.sa.na}{0}
\verb{paisano}{}{}{}{}{adj.}{Que é compatriota, patrício.}{pai.sa.no}{0}
\verb{paisano}{}{}{}{}{}{Que não é militar.}{pai.sa.no}{0}
\verb{paixão}{ch}{}{"-ões}{}{s.f.}{Forte sentimento de amor por uma pessoa.}{pai.xão}{0}
\verb{paixão}{ch}{}{"-ões}{}{}{Dedicação muito grande por alguma coisa.}{pai.xão}{0}
\verb{paixão}{ch}{Relig.}{"-ões}{}{}{O sofrimento de Cristo.}{pai.xão}{0}
\verb{paixonite}{ch}{Pop.}{}{}{s.f.}{Paixão amorosa intensa, geralmente passageira.}{pai.xo.ni.te}{0}
\verb{pajé}{}{}{}{}{s.m.}{Nas sociedades indígenas brasileiras, indivíduo que tem o poder de comunicar"-se com os espíritos, de realizar curas e de prever o futuro por meio de rituais.}{pa.jé}{0}
\verb{pajear}{}{}{}{}{v.t.}{Tomar conta; vigiar, cuidar.}{pa.je.ar}{\verboinum{4}}
\verb{pajelança}{}{}{}{}{s.f.}{Série de rituais realizados pelo pajé com determinado fim, como para curar enfermos ou prever acontecimentos futuros.}{pa.je.lan.ça}{0}
\verb{pajem}{}{}{"-ens}{}{s.m.}{Na Idade Média, moço nobre que acompanha o rei, um príncipe ou outro fidalgo para se aperfeiçoar na carreira de armas a fim de se tornar cavaleiro.}{pa.jem}{0}
\verb{pajem}{}{}{"-ens}{}{}{Pessoa contratada para cuidar de criança; babá, ama"-seca.}{pa.jem}{0}
\verb{pala}{}{}{}{}{}{Peça quadrada com que o padre cobre o cálice.}{pa.la}{0}
\verb{pala}{}{}{}{}{}{Peça reforçada de certas roupas.}{pa.la}{0}
\verb{pala}{}{}{}{}{s.f.}{Aba de boné.}{pa.la}{0}
\verb{palacete}{ê}{}{}{}{s.m.}{Palácio pequeno.}{pa.la.ce.te}{0}
\verb{palacete}{ê}{}{}{}{}{Casa luxuosa, requintada.}{pa.la.ce.te}{0}
\verb{palaciano}{}{}{}{}{adj.}{Relativo a palácio ou corte; palatino.}{pa.la.ci.a.no}{0}
\verb{palaciano}{}{}{}{}{}{Próprio de quem vive na corte; cortesão.}{pa.la.ci.a.no}{0}
\verb{palaciano}{}{}{}{}{s.m.}{Cortesão; áulico.}{pa.la.ci.a.no}{0}
\verb{palácio}{}{}{}{}{s.m.}{Residência de governante e de autoridade de Igreja.}{pa.lá.cio}{0}
\verb{palácio}{}{}{}{}{}{Residência muito luxuosa.}{pa.lá.cio}{0}
\verb{paladar}{}{}{}{}{s.m.}{Sentido que distingue sabores.}{pa.la.dar}{0}
\verb{paladar}{}{}{}{}{}{Palato.}{pa.la.dar}{0}
\verb{paladino}{}{Desus.}{}{}{s.m.}{Cavaleiro andante da Idade Média.}{pa.la.di.no}{0}
\verb{paladino}{}{Fig.}{}{}{}{Pessoa que defende ardorosa e destemidamente uma causa.}{pa.la.di.no}{0}
\verb{paládio}{}{Quím.}{}{}{s.m.}{Elemento químico metálico, branco, brilhante, maleável, usado em joalheria, prótese dentária e em ligas com o ouro, a prata, o cobre etc. \elemento{46}{106.42}{Pd}.}{pa.lá.dio}{0}
\verb{palafita}{}{}{}{}{s.f.}{Armação de estacas que sustenta habitações construídas sobre a água.}{pa.la.fi.ta}{0}
\verb{palafita}{}{}{}{}{}{Essa habitação.}{pa.la.fi.ta}{0}
\verb{palanfrório}{}{}{}{}{s.m.}{Palavreado.}{pa.lan.fró.rio}{0}
\verb{palanque}{}{}{}{}{s.m.}{Estrado de madeira, com degraus, para espectadores de festas ao ar livre ou desfiles; tablado. }{pa.lan.que}{0}
\verb{palanquim}{}{}{}{}{s.m.}{Espécie de liteira usada no Oriente.}{pa.lan.quim}{0}
\verb{palatal}{}{}{"-ais}{}{adj.2g.}{Relativo ao palato.}{pa.la.tal}{0}
\verb{palatino}{}{}{}{}{adj.}{Palaciano.}{pa.la.ti.no}{0}
\verb{palatino}{}{}{}{}{adj.}{Relativo ao palato, à parte interna e superior da boca; palatal.}{pa.la.ti.no}{0}
\verb{palato}{}{Anat.}{}{}{s.m.}{A parte interna e superior da boca; o céu da boca.}{pa.la.to}{0}
\verb{palavra}{}{}{}{}{s.f.}{Unidade mínima da língua, com som e sentido.}{pa.la.vra}{0}
\verb{palavra}{}{}{}{}{}{A representação gráfica dessa unidade.}{pa.la.vra}{0}
\verb{palavra}{}{}{}{}{}{Manifestação escrita ou verbal.}{pa.la.vra}{0}
\verb{palavra}{}{}{}{}{}{Doutrina religiosa.}{pa.la.vra}{0}
\verb{palavrada}{}{}{}{}{s.f.}{Palavrão.}{pa.la.vra.da}{0}
\verb{palavrada}{}{}{}{}{}{Bravata, fanfarrice.}{pa.la.vra.da}{0}
\verb{palavrão}{}{}{"-ões}{}{s.m.}{Palavra grande ou difícil de pronunciar. }{pa.la.vrão}{0}
\verb{palavrão}{}{}{"-ões}{}{}{Palavra obscena, grosseira ou indecente.}{pa.la.vrão}{0}
%\verb{}{}{}{}{}{}{}{}{0}
\verb{palavreado}{}{}{}{}{s.m.}{Série de palavras sem nexo entre si, sem importância.}{pa.la.vre.a.do}{0}
\verb{palavreado}{}{}{}{}{}{Conversa ardilosa; lábia, palavrório.}{pa.la.vre.a.do}{0}
\verb{palavrear}{}{}{}{}{v.i.}{Falar muito e sem refletir; tagarelar.}{pa.la.vre.ar}{0}
\verb{palavrear}{}{}{}{}{v.t.}{Dirigir a palavra; falar.}{pa.la.vre.ar}{\verboinum{4}}
\verb{palavrório}{}{}{}{}{s.m.}{Palavreado.}{pa.la.vró.rio}{0}
\verb{palavroso}{ô}{}{"-osos ⟨ó⟩}{"-osa ⟨ó⟩}{adj.}{Que tem muitas palavras.}{pa.la.vro.so}{0}
\verb{palavroso}{ô}{}{"-osos ⟨ó⟩}{"-osa ⟨ó⟩}{}{Que é dito com muitas palavras, ou com mais palavras do que o necessário; prolixo, loquaz, verboso.}{pa.la.vro.so}{0}
\verb{palco}{}{}{}{}{s.m.}{Parte do teatro destinado às representações dos artistas; estrado, tablado.}{pal.co}{0}
\verb{palco}{}{Fig.}{}{}{}{Local onde acontece um fato. (\textit{O centro daquela cidade tornou"-se o palco de uma das maiores batalhas da guerra.})}{pal.co}{0}
\verb{paleoceno}{}{Geol.}{}{}{adj.}{Diz"-se da época geológica que se estendeu entre 65 e 60 milhões de anos atrás. }{pa.le.o.ce.no}{0}
\verb{paleoceno}{}{}{}{}{s.m.}{Essa época.}{pa.le.o.ce.no}{0}
\verb{paleografia}{}{}{}{}{s.f.}{Ciência que estuda as escritas antigas.}{pa.le.o.gra.fi.a}{0}
\verb{paleográfico}{}{}{}{}{adj.}{Relativo à paleografia.}{pa.le.o.grá.fi.co}{0}
\verb{paleógrafo}{}{}{}{}{s.m.}{Pessoa especialista em paleografia.}{pa.le.ó.gra.fo}{0}
\verb{paleolítico}{}{Geol.}{}{}{s.m.}{Primeiro e mais antigo Período da Idade da Pedra, também chamado período da Pedra Lascada.}{pa.le.o.lí.ti.co}{0}
\verb{paleolítico}{}{}{}{}{adj.}{Relativo a esse período.}{pa.le.o.lí.ti.co}{0}
\verb{paleologia}{}{}{}{}{s.f.}{Estudo de línguas antigas.}{pa.le.o.lo.gi.a}{0}
\verb{paleólogo}{}{}{}{}{s.m.}{Indivíduo especialista em paleologia.}{pa.le.ó.lo.go}{0}
\verb{paleontologia}{}{}{}{}{s.f.}{Ciência que estuda animais e vegetais fósseis. }{pa.le.on.to.lo.gi.a}{0}
\verb{paleontológico}{}{}{}{}{adj.}{Relativo à paleontologia.}{pa.le.on.to.ló.gi.co}{0}
\verb{paleontólogo}{}{}{}{}{s.m.}{Indivíduo especialista em paleontologia.}{pa.le.on.tó.lo.go}{0}
\verb{paleozoico}{ó}{Geol.}{}{}{s.m.}{Período caracterizado pela formação das rochas, e que se segue ao Proterozoico.}{pa.le.o.zoi.co}{0}
\verb{paleozoico}{ó}{}{}{}{adj.}{Relativo a esse período.}{pa.le.o.zoi.co}{0}
\verb{palerma}{é}{}{}{}{adj.2g.}{Que é tolo, imbecil.}{pa.ler.ma}{0}
\verb{palerma}{é}{}{}{}{s.2g.}{Pessoa tola; imbecil, parvo.}{pa.ler.ma}{0}
\verb{palermice}{}{}{}{}{s.f.}{Qualidade, ato ou dito de palerma.}{pa.ler.mi.ce}{0}
\verb{palestino}{}{}{}{}{adj.}{Relativo à Palestina.}{pa.les.ti.no}{0}
\verb{palestino}{}{}{}{}{s.m.}{Indivíduo natural ou habitante desse país.}{pa.les.ti.no}{0}
\verb{palestra}{é}{}{}{}{s.f.}{Conversa ligeira para passar o tempo; papo.}{pa.les.tra}{0}
\verb{palestra}{é}{}{}{}{}{Exposição para uma plateia sobre um determinado assunto, geralmente de cunho cultural.}{pa.les.tra}{0}
\verb{palestrador}{}{}{}{}{adj.}{Que palestra.}{pa.les.tra.dor}{0}
\verb{palestrador}{}{}{}{}{s.m.}{Pessoa que palestra ou é dada a palestrar; palestrante.}{pa.les.tra.dor}{0}
\verb{palestrante}{}{}{}{}{s.2g.}{Palestrador.}{pa.les.tran.te}{0}
\verb{palestrar}{}{}{}{}{v.i.}{Manter palestra; conversar.}{pa.les.trar}{0}
\verb{palestrar}{}{}{}{}{v.t.}{Conversar, falar, cavaquear.}{pa.les.trar}{\verboinum{1}}
\verb{paleta}{ê}{}{}{}{s.f.}{Pequena chapa, geralmente oval e de madeira, que tem um buraco num dos lados, onde se enfia o polegar, e sobre a qual os pintores colocam e misturam as tintas; palheta.}{pa.le.ta}{0}
\verb{paleta}{ê}{Pop.}{}{}{}{Escápula de animal ou, por extensão, das pessoas.}{pa.le.ta}{0}
\verb{paletó}{}{Bras.}{}{}{s.m.}{Casaco curto, com bolsos externos, que se veste por cima da camisa ou do colete.}{pa.le.tó}{0}
\verb{palha}{}{}{}{}{s.f.}{Folha ou haste seca do capim e de outras gramíneas.}{pa.lha}{0}
\verb{palha}{}{}{}{}{}{Folha que recobre a espiga de milho.}{pa.lha}{0}
\verb{palha}{}{}{}{}{}{Coisa de pouco valor; insignificância.}{pa.lha}{0}
\verb{palhaçada}{}{}{}{}{s.f.}{Ato, dito ou modos de palhaço.}{pa.lha.ça.da}{0}
\verb{palhaçada}{}{}{}{}{}{Grupo de palhaços.}{pa.lha.ça.da}{0}
\verb{palhaçada}{}{}{}{}{}{Cena ridícula, cômica ou divertida.}{pa.lha.ça.da}{0}
\verb{palhaço}{}{}{}{}{s.m.}{Artista de circo que usa maquiagem, veste roupas coloridas e bizarras, divertindo o público com brincadeiras e piadas.}{pa.lha.ço}{0}
\verb{palheiro}{ê}{}{}{}{s.m.}{Lugar onde se guarda a palha.}{pa.lhei.ro}{0}
\verb{palheta}{ê}{}{}{}{s.f.}{Lâmina ou chapa estreita e comprida, de material resistente, usada para diversos fins.}{pa.lhe.ta}{0}
\verb{palheta}{ê}{}{}{}{}{Paleta.}{pa.lhe.ta}{0}
\verb{palhinha}{}{}{}{}{s.f.}{Pedaço pequeno de palha.}{pa.lhi.nha}{0}
\verb{palhinha}{}{}{}{}{}{Tira fina de junco seco usada na confecção de assentos e encostos de cadeiras.}{pa.lhi.nha}{0}
\verb{palhinha}{}{}{}{}{s.m.}{Chapéu masculino de palha.}{pa.lhi.nha}{0}
\verb{palhoça}{ó}{}{}{}{s.f.}{Casa ou cabana coberta de palha.}{pa.lho.ça}{0}
\verb{paliar}{}{}{}{}{v.t.}{Dissimular com falsa aparência; disfarçar, tapear.}{pa.li.ar}{0}
\verb{paliar}{}{}{}{}{}{Tratar com paliativo; remediar.}{pa.li.ar}{\verboinum{1}}
\verb{paliativo}{}{}{}{}{adj.}{Diz"-se do medicamento ou tratamento que tem por finalidade abrandar ou acalmar, mas que não cura a enfermidade.}{pa.li.a.ti.vo}{0}
\verb{paliativo}{}{}{}{}{}{Diz"-se da medida ou ação que não serve para resolver uma situação, somente para adiar ou protelar o prazo.}{pa.li.a.ti.vo}{0}
\verb{paliçada}{}{}{}{}{s.f.}{Espécie de cerca feita com estacas fincadas na terra.}{pa.li.ça.da}{0}
\verb{palidez}{}{}{}{}{s.f.}{Qualidade ou estado de pálido.}{pa.li.dez}{0}
\verb{pálido}{}{}{}{}{adj.}{Que perdeu a cor natural; lívido, inexpressivo.}{pá.li.do}{0}
\verb{palimpsesto}{ê}{}{}{}{s.m.}{Papiro ou pergaminho cujo primeiro texto foi raspado, para dar lugar a um novo texto.}{pa.limp.ses.to}{0}
\verb{palíndromo}{}{}{}{}{adj.}{Que pode ser lido (palavra ou frase) igualmente da esquerda para a direita ou vice"-versa sem alteração na forma da palavra de origem, como \textit{ovo} ou \textit{Roma é amor}. }{pa.lín.dro.mo}{0}
\verb{palíndromo}{}{}{}{}{s.m.}{Frase ou verso palíndromo.}{pa.lín.dro.mo}{0}
\verb{pálio}{}{}{}{}{s.m.}{Cobertura portátil, sustentada por varas, sobre a qual caminha o sacerdote nas procissões e cortejos. }{pá.lio}{0}
\verb{palitar}{}{}{}{}{v.t.}{Limpar os dentes com palito.}{pa.li.tar}{\verboinum{1}}
\verb{paliteiro}{ê}{}{}{}{s.m.}{Utensílio para guardar palitos.}{pa.li.tei.ro}{0}
\verb{palito}{}{}{}{}{s.m.}{Vareta de madeira, fina e curta; pauzinho.}{pa.li.to}{0}
\verb{palma}{}{}{}{}{s.f.}{A face interna das mãos, entre o pulso e os dedos.}{pal.ma}{0}
\verb{palma}{}{Bot.}{}{}{}{A folha da palmeira.}{pal.ma}{0}
\verb{palmácea}{}{Bot.}{}{}{s.f.}{Espécime das palmáceas, família de plantas que reúne as palmeiras; palma.}{pal.má.cea}{0}
\verb{palmada}{}{}{}{}{s.f.}{Pancada dada com a mão aberta; tapa.}{pal.ma.da}{0}
%\verb{}{}{}{}{}{}{}{}{0}
\verb{palmado}{}{}{}{}{adj.}{Que tem forma semelhante a de uma mão com os dedos abertos.}{pal.ma.do}{0}
\verb{palmar}{}{}{}{}{adj.2g.}{Que se refere à palma de mão.}{pal.mar}{0}
\verb{palmar}{}{}{}{}{adj.2g.}{Que tem o comprimento de um palmo.}{pal.mar}{0}
\verb{palmar}{}{}{}{}{s.m.}{Palmeiral.}{pal.mar}{0}
\verb{palmar}{}{Fig.}{}{}{}{Que é evidente; óbvio, manifesto.}{pal.mar}{0}
\verb{palmas}{}{}{}{}{s.f.pl.}{Manifestação pública de apoio caracterizada pelo gesto de ambas as mãos de forma ritmada.}{pal.mas}{0}
\verb{palmatória}{}{}{}{}{s.f.}{Pequena peça de madeira, com orifícios em forma de cruz, que se usava para castigar as crianças nas escolas, golpeando a palma da mão com esse instrumento.}{pal.ma.tó.ria}{0}
\verb{palmear}{}{}{}{}{v.t.}{Aplaudir; bater palmas.}{pal.me.ar}{0}
\verb{palmear}{}{Bras.}{}{}{}{Percorrer um caminho palmo a palmo; palmilhar, trilhar.}{pal.me.ar}{\verboinum{4}}
\verb{palmeira}{ê}{Bot.}{}{}{s.f.}{Nome comum a várias árvores de tronco alto e reto, sem ramos e com folhas largas na parte superior, muitas das quais fornecem frutos comestíveis como cocos, tâmaras etc. ou matéria"-prima para produtos industriais como óleo ou fibras.}{pal.mei.ra}{0}
\verb{palmeiral}{}{}{"-ais}{}{s.m.}{Grande quantidade de palmeiras próximas entre si.}{pal.mei.ral}{0}
\verb{palmense}{}{}{}{}{adj.2g.}{Relativo a Palmas, capital do Tocantins.}{pal.men.se}{0}
\verb{palmense}{}{}{}{}{s.2g.}{Indivíduo natural ou habitante dessa cidade.}{pal.men.se}{0}
\verb{palmilha}{}{}{}{}{s.f.}{Revestimento interno da sola de um calçado.}{pal.mi.lha}{0}
\verb{palmilhar}{}{}{}{}{v.t.}{Colocar palmilhas nos calçados.}{pal.mi.lhar}{0}
\verb{palmilhar}{}{}{}{}{}{Percorrer a pé; caminhar, palmear.}{pal.mi.lhar}{0}
\verb{palmilhar}{}{}{}{}{v.i.}{Andar a pé.}{pal.mi.lhar}{\verboinum{1}}
\verb{palmípede}{}{Zool.}{}{}{s.m.}{Ave que tem os dedos dos pés ligados por membrana. }{pal.mí.pe.de}{0}
\verb{palmital}{}{}{"-ais}{}{s.m.}{Aglomerado de palmeiras cujo palmito é comestível.}{pal.mi.tal}{0}
\verb{palmito}{}{}{}{}{s.m.}{Parte macia e comestível do interior do caule de algumas palmeiras.}{pal.mi.to}{0}
\verb{palmo}{}{}{}{}{s.m.}{Medida tomada entre a ponta do dedo polegar e a do dedo mínimo da mão bem aberta.}{pal.mo}{0}
\verb{palpar}{}{}{}{}{v.t.}{Tocar com a mão; tatear, apalpar.}{pal.par}{\verboinum{1}}
\verb{palpável}{}{}{"-eis}{}{adj.2g.}{Que pode ser palpado, tocado, sentido, visto ou percebido; tangível.}{pal.pá.vel}{0}
\verb{pálpebra}{}{Anat.}{}{}{s.f.}{Cada uma das membranas móveis que recobrem e protegem os olhos.}{pál.pe.bra}{0}
\verb{palpebral}{}{}{"-ais}{}{adj.2g.}{Relativo ou pertencente à pálpebra. }{pal.pe.bral}{0}
\verb{palpitação}{}{}{"-ões}{}{s.f.}{Ato ou efeito de palpitar.}{pal.pi.ta.ção}{0}
\verb{palpitação}{}{}{"-ões}{}{}{Batimento cardíaco anormal, acelerado ou irregular.}{pal.pi.ta.ção}{0}
\verb{palpitante}{}{}{}{}{adj.2g.}{Que palpita.}{pal.pi.tan.te}{0}
\verb{palpitante}{}{}{}{}{}{Que desperta muito interesse ou curiosidade; vivo, emocionante.}{pal.pi.tan.te}{0}
\verb{palpitar}{}{}{}{}{v.i.}{Sentir palpitações; pulsar, latejar.}{pal.pi.tar}{0}
\verb{palpitar}{}{}{}{}{v.t.}{Dar palpite; intrometer"-se.}{pal.pi.tar}{\verboinum{1}}
\verb{palpite}{}{}{}{}{s.m.}{Opinião de intrometido ou de quem não entende do assunto.}{pal.pi.te}{0}
\verb{palpite}{}{}{}{}{}{Intuição, presságio, pressentimento.}{pal.pi.te}{0}
\verb{palpiteiro}{ê}{Bras.}{}{}{adj.}{Que gosta de dar palpites; enxerido, intrometido.}{pal.pi.tei.ro}{0}
\verb{palpiteiro}{ê}{}{}{}{s.m.}{Essa pessoa.}{pal.pi.tei.ro}{0}
\verb{palpo}{}{Zool.}{}{}{s.m.}{Cada um dos dois apêndices articulados e móveis da boca dos insetos.}{pal.po}{0}
\verb{palrador}{ô}{}{}{}{adj.}{Que palra; tagarela.}{pal.ra.dor}{0}
\verb{palrar}{}{}{}{}{v.i.}{Articular sons sem sentido; chalrar.}{pal.rar}{0}
\verb{palrar}{}{}{}{}{}{Falar demais; tagarelar.}{pal.rar}{0}
\verb{palrar}{}{}{}{}{v.t.}{Dizer, falar, proferir.}{pal.rar}{\verboinum{1}}
\verb{palrear}{}{}{}{}{v.t. e v.i.}{Palrar.}{pal.re.ar}{\verboinum{4}}
\verb{palude}{}{}{}{}{s.m.}{Pântano.}{pa.lu.de}{0}
\verb{paludismo}{}{Med.}{}{}{s.m.}{Malária.}{pa.lu.dis.mo}{0}
\verb{paludoso}{ô}{}{"-osos ⟨ó⟩}{"-osa ⟨ó⟩}{adj.}{Em que há paludes; alagadiço, palustre.}{pa.lu.do.so}{0}
\verb{palustre}{}{}{}{}{adj.2g.}{Paludoso.}{pa.lus.tre}{0}
\verb{palustre}{}{}{}{}{}{Que vive em pântanos.}{pa.lus.tre}{0}
\verb{pamonha}{}{Cul.}{}{}{s.f.}{Massa de milho verde ralado, açúcar e leite, cozida envolta na palha do próprio milho.}{pa.mo.nha}{0}
\verb{pampa}{}{Geogr.}{}{}{s.m.}{Tipo de vegetação campestre rasteira, na qual predominam gramíneas, característica das regiões meridionais da América do Sul, como Argentina, Brasil (\textsc{rs}) e Uruguai.}{pam.pa}{0}
\verb{pâmpano}{}{}{}{}{s.m.}{Ramo novo de videira.}{pâm.pa.no}{0}
\verb{pampeiro}{ê}{}{}{}{}{}{pam.pei.ro}{0}
\verb{pampeiro}{ê}{Bras.}{}{}{s.m.}{Vento forte que sopra dos pampas da Argentina, e que às vezes alcança o Rio Grande do Sul.}{pam.pei.ro}{0}
\verb{pampiano}{}{}{}{}{adj.}{Relativo à região dos pampas.}{pam.pi.a.no}{0}
\verb{pampiano}{}{}{}{}{s.m.}{Natural ou habitante dessa região. }{pam.pi.a.no}{0}
\verb{panaca}{}{}{}{}{s.2g.}{Pessoa muito ingênua, simplória; tolo.}{pa.na.ca}{0}
\verb{panaceia}{é}{}{}{}{s.f.}{Qualquer substância ou procedimento que se acredita ter o poder de curar todos os males.}{pa.na.cei.a}{0}
\verb{panado}{}{}{}{}{adj.}{Que foi passado em farinha de rosca ou de trigo antes de fritar; empanado.}{pa.na.do}{0}
\verb{panamá}{}{}{}{}{s.m.}{Chapéu de palha masculino, de copa e abas feitas de uma palha muito flexível.}{pa.na.má}{0}
\verb{panamenho}{}{}{}{}{adj.}{Relativo ao Panamá; panamense.}{pa.na.me.nho}{0}
\verb{panamenho}{}{}{}{}{s.m.}{Indivíduo natural ou habitante desse país.}{pa.na.me.nho}{0}
\verb{panamense}{}{}{}{}{adj.2g. e s.2g.}{Panamenho.}{pa.na.men.se}{0}
\verb{pan"-americanismo}{}{}{}{}{s.m.}{Doutrina que defende a solidariedade e a aliança política entre todos os países das Américas. }{pan"-a.me.ri.ca.nis.mo}{0}
\verb{pan"-americano}{}{}{pan"-americanos}{pan"-americana}{adj.}{Que se refere a todos os países das Américas.}{pan"-a.me.ri.ca.no}{0}
\verb{panarício}{}{Med.}{}{}{s.m.}{Inflamação aguda dos tecidos profundos da falange distal dos dedos; paroníquia.}{pa.na.rí.cio}{0}
\verb{panariz}{}{Med.}{}{}{s.m.}{Paroníquia.}{pa.na.riz}{0}
\verb{pança}{}{Pop.}{}{}{s.f.}{Barriga grande e protuberante.}{pan.ça}{0}
\verb{panca}{}{}{}{}{s.f.}{Pedaço de pau que serve de alavanca.}{pan.ca}{0}
\verb{panca}{}{Pop.}{}{}{}{Pose, pedantismo, afetação.}{pan.ca}{0}
\verb{pancada}{}{}{}{}{s.f.}{Choque de dois corpos; baque, golpe.}{pan.ca.da}{0}
\verb{pancada}{}{}{}{}{}{Som do pêndulo do relógio ou do sino.}{pan.ca.da}{0}
\verb{pancada}{}{}{}{}{adj.2g.}{Amalucado, doido.}{pan.ca.da}{0}
\verb{pancadaria}{}{}{}{}{s.f.}{Muitas pancadas, bordoadas, cacetadas, pauladas.}{pan.ca.da.ri.a}{0}
\verb{pancadaria}{}{}{}{}{}{Confusão ou tumulto em que há pancadas.}{pan.ca.da.ri.a}{0}
\verb{pâncreas}{}{Anat.}{}{}{s.m.}{Glândula exócrina e endócrina, localizada atrás do estômago, que atua na digestão e em determinados processos metabólicos.}{pân.cre.as}{0}
\verb{pancreático}{}{}{}{}{adj.}{Que se refere ao pâncreas. }{pan.cre.á.ti.co}{0}
\verb{pancreático}{}{}{}{}{}{Diz"-se do suco segregado pelo pâncreas.}{pan.cre.á.ti.co}{0}
\verb{pancreatite}{}{Med.}{}{}{s.f.}{Inflamação do pâncreas.}{pan.cre.a.ti.te}{0}
\verb{pançudo}{}{}{}{}{adj.}{Que tem a pança grande; barrigudo. }{pan.çu.do}{0}
\verb{panda}{}{Zool.}{}{}{s.m.}{Nome comum a certos mamíferos que vivem em florestas situadas a grande altitude, na Ásia, que se parecem com os ursos, com uma espécie que se alimenta quase apenas de bambu, e têm pelagem branca e preta. }{pan.da}{0}
\verb{pandarecos}{é}{}{}{}{s.m.pl.}{Pedaços, cacos, destroços, fragmentos.}{pan.da.re.cos}{0}
\verb{pândega}{}{}{}{}{s.f.}{Festa ou reunião alegre, com comida e bebida; brincadeira, farra, festança, folgança, folia. }{pân.de.ga}{0}
\verb{pândego}{}{}{}{}{adj.}{Que é dado a pândegas.}{pân.de.go}{0}
\verb{pândego}{}{}{}{}{}{Que é alegre e engraçado.}{pân.de.go}{0}
\verb{pândego}{}{}{}{}{s.m.}{Pessoa pândega.}{pân.de.go}{0}
\verb{pandeiro}{ê}{}{}{}{s.m.}{Instrumento de percussão que consiste num aro de madeira recoberto por uma pele esticada, guarnecido de guizos nas laterais.}{pan.dei.ro}{0}
\verb{pandemia}{}{Med.}{}{}{s.f.}{Doença epidêmica amplamente disseminada.}{pan.de.mi.a}{0}
\verb{pandêmico}{}{}{}{}{adj.}{Que tem o caráter de pandemia.  }{pan.dê.mi.co}{0}
\verb{pandemônio}{}{}{}{}{s.m.}{Grande confusão ou tumulto; balbúrdia, desordem. }{pan.de.mô.nio}{0}
\verb{pando}{}{}{}{}{adj.}{Inflado, inchado, enfunado.}{pan.do}{0}
\verb{pandorga}{ó}{}{}{}{s.f.}{Música desafinada e sem ritmo.}{pan.dor.ga}{0}
\verb{pandorga}{ó}{Bras.}{}{}{}{Papagaio de papel; pipa.}{pan.dor.ga}{0}
\verb{pandulho}{}{}{}{}{s.m.}{Lastro preso na tralha inferior das redes de pesca.}{pan.du.lho}{0}
\verb{pandulho}{}{Bras.}{}{}{}{Barriga grande; bandulho, pança.}{pan.du.lho}{0}
\verb{pane}{}{}{}{}{s.f.}{Parada súbita de motor de carro, de avião etc. por causa de defeito ou falta de combustível.}{pa.ne}{0}
\verb{panegírico}{}{}{}{}{adj.}{Que louva; laudatório, elogioso, encomiástico.}{pa.ne.gí.ri.co}{0}
\verb{panegírico}{}{}{}{}{s.m.}{Discurso em louvor de alguém ou de alguma coisa; elogio, encômio. }{pa.ne.gí.ri.co}{0}
\verb{panegirista}{}{}{}{}{s.2g.}{Pessoa que faz panegírico.}{pa.ne.gi.ris.ta}{0}
\verb{paneiro}{ê}{}{}{}{s.m.}{Cesto grande, geralmente de vime, com alças.}{pa.nei.ro}{0}
\verb{panejar}{}{}{}{}{v.t.}{Pôr panos ou roupas.}{pa.ne.jar}{0}
\verb{panejar}{}{}{}{}{v.i.}{Abanar, agitar"-se (bandeira, vela etc.).}{pa.ne.jar}{\verboinum{1}}
\verb{panela}{é}{}{}{}{s.f.}{Vasilha de barro ou de metal onde se cozinham os alimentos.}{pa.ne.la}{0}
\verb{panelada}{}{}{}{}{s.f.}{Panela cheia de algum alimento.}{pa.ne.la.da}{0}
\verb{panelada}{}{}{}{}{}{Tipo de cozido que se faz com mocotó, miúdos de boi, toucinho e verduras.}{pa.ne.la.da}{0}
\verb{panelinha}{}{}{}{}{s.f.}{Panela pequena.}{pa.ne.li.nha}{0}
\verb{panelinha}{}{Fig.}{}{}{}{Qualquer grupo de pessoas que age exclusivamente em benefício próprio, elogiando"-se uns aos outros e buscando obter vantagens nem sempre lícitas, como um grupo fechado de políticos, intelectuais etc.}{pa.ne.li.nha}{0}
\verb{panetone}{}{}{}{}{s.m.}{Bolo de massa fermentada, com passas e frutas cristalizadas, tradicionalmente servido durante as festas natalinas.  }{pa.ne.to.ne}{0}
\verb{panfletagem}{}{Bras.}{"-ens}{}{s.f.}{Ato ou efeito de panfletar.}{pan.fle.ta.gem}{0}
\verb{panfletário}{}{}{}{}{adj.}{Que se refere a panfleto.}{pan.fle.tá.rio}{0}
\verb{panfletário}{}{Fig.}{}{}{}{Que se expressa de forma violenta ou radical.}{pan.fle.tá.rio}{0}
\verb{panfletário}{}{}{}{}{s.m.}{Pessoa que escreve panfleto(s); panfletista.}{pan.fle.tá.rio}{0}
\verb{panfletista}{}{}{}{}{s.2g.}{Panfletário.}{pan.fle.tis.ta}{0}
\verb{panfleto}{ê}{}{}{}{s.m.}{Pequeno escrito satírico ou polêmico sobre um determinado assunto.}{pan.fle.to}{0}
\verb{pangaré}{}{}{}{}{s.m.}{Cavalo velho, imprestável ou de pouco valor.}{pan.ga.ré}{0}
\verb{pânico}{}{}{}{}{s.m.}{Medo muito grande e repentino que provoca reação descontrolada; pavor, terror.}{pâ.ni.co}{0}
\verb{panificação}{}{}{"-ões}{}{s.f.}{Ato ou efeito de panificar.}{pa.ni.fi.ca.ção}{0}
\verb{panificação}{}{}{"-ões}{}{}{Fabricação de pão.}{pa.ni.fi.ca.ção}{0}
\verb{panificação}{}{Bras.}{"-ões}{}{}{Padaria, panificadora.}{pa.ni.fi.ca.ção}{0}
\verb{panificador}{ô}{}{}{}{s.m.}{Indivíduo que panifica, que faz pão; padeiro.}{pa.ni.fi.ca.dor}{0}
\verb{panificadora}{ô}{}{}{}{s.f.}{Lugar em que se faz ou se vende pão; padaria.}{pa.ni.fi.ca.do.ra}{0}
\verb{panificar}{}{}{}{}{v.t.}{Transformar farinha em pão.}{pa.ni.fi.car}{\verboinum{2}}
\verb{pano}{}{}{}{}{s.m.}{Tecido feito de fios de algodão, linho, lã, seda; fazenda.}{pa.no}{0}
\verb{panorama}{}{}{}{}{s.m.}{Visão ampla, em todas as direções, sem obstáculos, de um ponto mais alto; paisagem, vista.}{pa.no.ra.ma}{0}
\verb{panorama}{}{Fig.}{}{}{}{Ampla exposição.}{pa.no.ra.ma}{0}
\verb{panorâmico}{}{}{}{}{adj.}{Relativo a panorama ou a paisagem.}{pa.no.râ.mi.co}{0}
\verb{panorâmico}{}{}{}{}{}{Diz"-se da janela que proporciona ampla vista.}{pa.no.râ.mi.co}{0}
\verb{panqueca}{é}{Cul.}{}{}{s.f.}{Massa fina feita basicamente com farinha e ovos, frita na frigideira e enrolada com recheio. (\textit{Gosto muito de panqueca com recheio doce.})}{pan.que.ca}{0}
\verb{pantagruélico}{}{}{}{}{adj.}{Relativo a Pantagruel, personagem comilão de Rabelais.}{pan.ta.gru.é.li.co}{0}
\verb{pantagruélico}{}{}{}{}{}{Abundante em comidas.}{pan.ta.gru.é.li.co}{0}
\verb{pantagruelismo}{}{}{}{}{s.m.}{Filosofia dos que cuidam exclusivamente dos gozos materiais.}{pan.ta.gru.e.lis.mo}{0}
\verb{pantalonas}{}{}{}{}{s.f.pl.}{Calças compridas de pernas e bocas largas.}{pan.ta.lo.nas}{0}
\verb{pantanal}{}{}{"-ais}{}{s.m.}{Pântano grande.  }{pan.ta.nal}{0}
\verb{pantanal}{}{Bras.}{"-ais}{}{}{Zona geofísica do Mato Grosso, na baixada do rio Paraguai, que abrange as terras baixas e as elevações e morros que por elas se espalham. (Inicial maiúscula nesta acepção.)}{pan.ta.nal}{0}
\verb{pantaneiro}{ê}{}{}{}{adj.}{Relativo ao Pantanal Mato"-grossense.}{pan.ta.nei.ro}{0}
\verb{pantaneiro}{ê}{}{}{}{s.m.}{Raça de bovinos e equinos típicos do Pantanal Mato"-grossense.}{pan.ta.nei.ro}{0}
\verb{pantaneiro}{ê}{}{}{}{}{Criador de gado; fazendeiro.}{pan.ta.nei.ro}{0}
\verb{pântano}{}{}{}{}{s.m.}{Terreno baixo e alagadiço; paul.}{pân.ta.no}{0}
\verb{pantanoso}{ô}{}{"-osos ⟨ó⟩}{"-osa ⟨ó⟩}{adj.}{Que tem pântanos; alagadiço.}{pan.ta.no.so}{0}
\verb{panteão}{}{}{"-ões}{}{s.m.}{Na Antiguidade, templo dedicado a todos os deuses.}{pan.te.ão}{0}
\verb{panteão}{}{}{"-ões}{}{}{Local consagrado aos restos mortais de pessoas ilustres.}{pan.te.ão}{0}
\verb{panteísmo}{}{}{}{}{s.m.}{Crença segundo a qual Deus é o conjunto de tudo o que existe, sendo uma única realidade integrada e universal.}{pan.te.ís.mo}{0}
\verb{panteísta}{}{}{}{}{adj.2g.}{Relativo ao panteísmo.}{pan.te.ís.ta}{0}
\verb{panteísta}{}{}{}{}{s.2g.}{Indivíduo adepto do panteísmo.}{pan.te.ís.ta}{0}
\verb{pantera}{é}{Zool.}{}{}{s.f.}{Animal mamífero felino de grande porte, de pelo preto ou malhado; leopardo.}{pan.te.ra}{0}
\verb{pantógrafo}{}{}{}{}{s.m.}{Instrumento de hastes articuladas com que se copiam mecanicamente figuras.}{pan.tó.gra.fo}{0}
\verb{pantomima}{}{}{}{}{s.f.}{Arte de exprimir ideias ou sentimentos por meio de gestos; mímica. }{pan.to.mi.ma}{0}
\verb{pantomima}{}{Fig.}{}{}{}{Mentira, embuste, logro.}{pan.to.mi.ma}{0}
\verb{pantomimeiro}{ê}{}{}{}{s.m.}{Indivíduo que faz pantomima; mímico.}{pan.to.mi.mei.ro}{0}
\verb{pantomímico}{}{}{}{}{adj.}{Relativo a pantomima.}{pan.to.mí.mi.co}{0}
\verb{pantufa}{}{}{}{}{s.f.}{Chinelo acolchoado.}{pan.tu.fa}{0}
\verb{pantufo}{}{}{}{}{s.m.}{Pantufa.}{pan.tu.fo}{0}
\verb{panturrilha}{}{Anat.}{}{}{s.f.}{Barriga da perna; sura.}{pan.tur.ri.lha}{0}
\verb{pão}{}{}{pães}{}{s.m.}{Alimento feito de farinha de trigo ou outros cereais, amassada e assada no forno.}{pão}{0}
\verb{pão}{}{Fig.}{pães}{}{}{O alimento, o sustento. (\textit{Sai todos os dias de madrugada para ganhar o pão.})}{pão}{0}
\verb{pão}{}{Relig.}{pães}{}{}{A hóstia consagrada.}{pão}{0}
\verb{pão"-de"-ló}{}{Cul.}{pães"-de"-ló}{}{s.m.}{Bolo de farinha, ovos e açúcar, de massa muito leve e fofa.}{pão"-de"-ló}{0}
\verb{pão"-durismo}{}{}{}{}{s.m.}{Avareza, sovinice, mesquinharia.}{pão"-du.ris.mo}{0}
\verb{pão"-duro}{}{}{}{}{adj.}{Que evita gastar dinheiro a todo custo; avarento, sovina, mesquinho.}{pão"-du.ro}{0}
\verb{papá}{}{Onomat.}{}{}{s.m.}{Na linguagem dos bebês e criancinhas, papai; tatá.}{pa.pá}{0}
\verb{papa}{}{}{}{}{s.m.}{Chefe supremo da Igreja Católica.}{pa.pa}{0}
\verb{papa}{}{}{}{}{s.f.}{Farinha cozida em água ou leite.}{pa.pa}{0}
\verb{papa}{}{Fig.}{}{}{}{Profissional que as pessoas reconhecem como o maior entendido sobre determinado assunto.}{pa.pa}{0}
%\verb{}{}{}{}{}{}{}{}{0}
\verb{papada}{}{}{}{}{s.f.}{Acúmulo de gordura abaixo do queixo.}{pa.pa.da}{0}
\verb{papa"-defuntos}{}{}{papa"-defuntos}{}{s.m.}{Indivíduo que agencia enterros.}{pa.pa"-de.fun.tos}{0}
\verb{papado}{}{}{}{}{s.m.}{Dignidade do papa.}{pa.pa.do}{0}
\verb{papa"-figo}{}{Zool.}{papa"-figos}{}{s.m.}{Pássaro de plumagem azul"-escura, semelhante ao melro.}{pa.pa"-fi.go}{0}
\verb{papa"-figo}{}{}{papa"-figos}{}{}{Papão.}{pa.pa"-fi.go}{0}
\verb{papa"-fina}{}{}{papas"-finas}{}{s.f.}{De gosto muito bom, excelente; saboroso.}{pa.pa"-fi.na}{0}
\verb{papagaiar}{}{}{}{}{v.i.}{Falar como papagaio; tagarelar.}{pa.pa.gai.ar}{\verboinum{1}}
\verb{papagaio}{}{}{}{}{s.m.}{Ave trepadora de plumagem verde que é capaz de imitar a voz humana; louro.}{pa.pa.gai.o}{0}
\verb{papagaio}{}{}{}{}{}{Pessoa que fala muito; tagarela.}{pa.pa.gai.o}{0}
\verb{papagaio}{}{}{}{}{}{Brinquedo feito de papel fino com uma armação de varetas que se solta ao vento preso por uma linha; pipa.}{pa.pa.gai.o}{0}
\verb{papaguear}{}{}{}{}{v.t.}{Papagaiar.}{pa.pa.gue.ar}{\verboinum{4}}
\verb{papai}{}{}{}{}{s.m.}{Tratamento familiar que os filhos dão ao pai.}{pa.pai}{0}
\verb{papaia}{}{}{}{}{s.m.}{Variedade de mamão, de tamanho pequeno e sabor muito doce.}{pa.pai.a}{0}
\verb{papa"-jantares}{}{}{}{}{s.2g.2n}{Indivíduo que tem por hábito jantar em casas alheias.}{pa.pa"-jan.ta.res}{0}
\verb{papal}{}{}{"-ais}{}{adj.2g.}{Relativo ou pertencente ao papa.}{pa.pal}{0}
\verb{papalvo}{}{}{}{}{adj.}{Que é simplório, tolo, pateta.}{pa.pal.vo}{0}
%\verb{}{}{}{}{}{}{}{}{0}
\verb{papão}{}{}{"-ões}{}{s.m.}{Bicho imaginário com que se mete medo em crianças; bicho"-papão; papa"-figo.}{pa.pão}{0}
\verb{papa"-ovo}{ô}{Zool.}{papa"-ovos ⟨ó⟩}{}{s.m.}{Espécie de cobra, de hábitos diurnos, que se alimenta de pequenos invertebrados.}{pa.pa"-o.vo}{0}
\verb{papar}{}{Pop.}{}{}{v.t.}{Ingerir alimento; comer.}{pa.par}{0}
\verb{papar}{}{Pop.}{}{}{}{Conseguir, ganhar.}{pa.par}{\verboinum{1}}
\verb{paparicar}{}{}{}{}{v.t.}{Mimar ou cuidar em excesso.}{pa.pa.ri.car}{0}
\verb{paparicar}{}{}{}{}{}{Comer pouco ou aos poucos; lambiscar.}{pa.pa.ri.car}{\verboinum{2}}
\verb{paparico}{}{}{}{}{s.m.}{Mimo ou cuidado excessivo.}{pa.pa.ri.co}{0}
\verb{paparico}{}{}{}{}{}{Gulodice, guloseima.}{pa.pa.ri.co}{0}
\verb{papa"-terra}{é}{Zool.}{papa"-terras ⟨é⟩}{}{s.m.}{Peixe de água doce que tem o hábito de remexer a terra do fundo das lagoas.}{pa.pa"-ter.ra}{0}
\verb{papável}{}{}{"-eis}{}{adj.2g.}{Que se pode papar.}{pa.pá.vel}{0}
\verb{papável}{}{}{"-eis}{}{}{Diz"-se do cardeal que tem possibilidades de ser eleito papa.}{pa.pá.vel}{0}
\verb{papear}{}{}{}{}{v.i.}{Conversar despreocupadamente; bater papo.}{pa.pe.ar}{\verboinum{4}}
\verb{papeira}{ê}{Med.}{}{}{s.f.}{Inchaço abaixo do queixo; bócio, papo.}{pa.pei.ra}{0}
\verb{papel}{é}{}{"-éis}{}{s.m.}{Folha fabricada com uma pasta de fibras de celulose, usada para embrulhar, forrar, escrever, imprimir etc.}{pa.pel}{0}
\verb{papel}{é}{}{"-éis}{}{}{Parte que cada ator desempenha em peças teatrais, filmes, novelas etc.}{pa.pel}{0}
\verb{papel}{é}{}{"-éis}{}{}{Função, desempenho, atribuição.}{pa.pel}{0}
\verb{papel}{é}{}{"-éis}{}{}{Documento escrito.}{pa.pel}{0}
\verb{papelada}{}{}{}{}{s.f.}{Grande porção de papéis, de documentos.}{pa.pe.la.da}{0}
%\verb{}{}{}{}{}{}{}{}{0}
\verb{papelão}{}{}{"-ões}{}{s.m.}{Papel grosso e rígido, usado como embalagem ou como capa de livros, pastas etc.}{pa.pe.lão}{0}
\verb{papelão}{}{Fig.}{"-ões}{}{}{Comportamento ridículo, vergonhoso. (\textit{Meu irmão aprontou um papelão na festa!})}{pa.pe.lão}{0}
\verb{papelaria}{}{}{}{}{s.f.}{Estabelecimento onde se vendem papel e artigos de escola e de escritório.}{pa.pe.la.ri.a}{0}
\verb{papel"-carbono}{é}{}{papéis"-carbono \textit{ou} papéis"-carbonos}{}{s.m.}{Folha fina de papel recoberta de tinta em uma das faces, usado para fazer cópias; carbono.}{pa.pel"-car.bo.no}{0}
\verb{papeleiro}{ê}{}{}{}{adj.}{Relativo a papel.}{pa.pe.lei.ro}{0}
\verb{papeleiro}{ê}{}{}{}{s.m.}{Indivíduo que trabalha em fábrica de papel ou é dono de papelaria.}{pa.pe.lei.ro}{0}
\verb{papeleta}{ê}{}{}{}{s.f.}{Folha pequena de papel usada para se fazerem observações ou anotações.}{pa.pe.le.ta}{0}
%\verb{}{}{}{}{}{}{}{}{0}
%\verb{}{}{}{}{}{}{}{}{0}
\verb{papel"-moeda}{é\ldots{}é}{}{papéis"-moedas \textit{ou} papéis"-moeda ⟨é⟩}{}{s.m.}{Dinheiro oficial de um país, emitido em papel; nota.}{pa.pel"-mo.e.da}{0}
\verb{papelório}{}{}{}{}{s.m.}{Grande quantidade de papéis sem importância; papelada.}{pa.pe.ló.rio}{0}
\verb{papelotes}{ó}{}{}{}{s.m.pl.}{Pedaços de papel em que são enroladas pequenas mechas de cabelo para frisá"-las.}{pa.pe.lo.tes}{0}
\verb{papelucho}{}{}{}{}{s.m.}{Papel sem importância, insignificante.}{pa.pe.lu.cho}{0}
\verb{papila}{}{Biol.}{}{}{s.f.}{Pequena saliência cônica semelhante a um mamilo.}{pa.pi.la}{0}
\verb{papiro}{}{Bot.}{}{}{s.m.}{Planta cujo caule é formado de folhas sobrepostas.}{pa.pi.ro}{0}
\verb{papiro}{}{}{}{}{}{Folha para escrever, feita de papiro.}{pa.pi.ro}{0}
\verb{papisa}{}{}{}{}{s.f.}{Mulher que teria exercido as funções de papa, conforme antigas lendas. }{pa.pi.sa}{0}
\verb{papo}{}{}{}{}{s.m.}{Saco ou bolsa que fica abaixo do pescoço das aves.}{pa.po}{0}
\verb{papo}{}{}{}{}{}{Inchaço no pescoço; bócio, papeira.}{pa.po}{0}
\verb{papo}{}{}{}{}{}{Conversa amigável e despreocupada.}{pa.po}{0}
\verb{papo}{}{}{}{}{}{Fanfarronice.}{pa.po}{0}
\verb{papo"-furado}{}{Pop.}{papos"-furados}{}{s.m.}{Indivíduo que não cumpre o que promete ou que gosta de contar vantagem; conversa"-fiada.}{pa.po"-fu.ra.do}{0}
\verb{papoula}{ô}{Bot.}{}{}{s.f.}{Planta ornamental cultivada por suas flores coloridas.}{pa.pou.la}{0}
\verb{páprica}{}{}{}{}{s.f.}{Tempero em pó, feito de pimentões vermelhos e secos.}{pá.pri.ca}{0}
\verb{papuásio}{}{}{}{}{adj.}{Relativo à Papua Nova Guiné (Oceania).}{pa.pu.á.sio}{0}
\verb{papuásio}{}{}{}{}{s.m.}{Indivíduo natural ou habitante desse país.}{pa.pu.á.sio}{0}
\verb{papudo}{}{}{}{}{adj.}{Que tem papo grande.}{pa.pu.do}{0}
\verb{papudo}{}{}{}{}{}{Metido a valentão; convencido, fanfarrão.}{pa.pu.do}{0}
\verb{paqueiro}{ê}{}{}{}{adj.}{Diz"-se de cão caçador de pacas.}{pa.quei.ro}{0}
\verb{paquera}{é}{}{}{}{s.f.}{Ato ou efeito de paquerar; tentativa de namoro.}{pa.que.ra}{0}
\verb{paquerador}{ô}{}{}{}{s.m.}{Indivíduo que paquera muito.}{pa.que.ra.dor}{0}
\verb{paquerar}{}{}{}{}{v.t.}{Tentar aproximação amorosa com alguém; demonstrar interesse amoroso.}{pa.que.rar}{\verboinum{1}}
\verb{paquete}{ê}{}{}{}{s.m.}{Grande navio, veloz e luxuoso, utilizado para transporte de passageiros.}{pa.que.te}{0}
\verb{paquiderme}{é}{Zool.}{}{}{adj.2g.}{Diz"-se do animal mamífero que tem a pele muito grossa.}{pa.qui.der.me}{0}
\verb{paquistanês}{}{}{}{}{adj.}{Relativo ao Paquistão.}{pa.quis.ta.nês}{0}
\verb{paquistanês}{}{}{}{}{s.m.}{Indivíduo natural ou habitante desse país.}{pa.quis.ta.nês}{0}
\verb{par}{}{}{}{}{adj.2g.}{Que pode ser dividido por dois.}{par}{0}
\verb{par}{}{}{}{}{s.m.}{Conjunto de duas coisas semelhantes ou iguais que se completam. (\textit{Precisei comprar um novo par de óculos.})}{par}{0}
\verb{par}{}{}{}{}{}{Conjunto de dois indivíduos ou dois animais de sexo diferente; casal.}{par}{0}
\verb{par}{}{}{}{}{}{Cada uma das pessoas que formam uma dupla na dança; parceiro.}{par}{0}
\verb{para}{}{}{}{}{prep.}{Estabelece relação de destino, direção, fim, objetivo, tendência.}{pa.ra}{0}
\verb{parabenizar}{}{}{}{}{v.t.}{Dar parabéns; felicitar, congratular.}{pa.ra.be.ni.zar}{\verboinum{1}}
\verb{parabéns}{}{}{}{}{s.m.pl.}{Congratulações, felicitações, cumprimentos.}{pa.ra.béns}{0}
\verb{parábola}{}{}{}{}{s.f.}{Narrativa alegórica que transmite algum preceito moral, por meio de comparação ou analogia.}{pa.rá.bo.la}{0}
\verb{parábola}{}{Geom.}{}{}{}{Curva plana cujos pontos têm distâncias iguais de um ponto fixo e de uma reta fixa.}{pa.rá.bo.la}{0}
\verb{parabólico}{}{}{}{}{adj.}{Relativo a ou em forma de parábola.}{pa.ra.bó.li.co}{0}
\verb{para"-brisa}{}{}{para"-brisas}{}{s.f.}{Vidro dianteiro do automóvel que serve para proteger o motorista do vento, poeira, chuva etc.}{pa.ra"-bri.sa}{0}
\verb{para"-choque}{ó}{}{para"-choques ⟨ó⟩}{}{s.m.}{Peça fixada na frente e na traseira do veículo para protegê"-lo contra estragos de uma batida.}{pa.ra"-cho.que}{0}
\verb{parada}{}{}{}{}{s.f.}{Ato ou efeito de parar; suspensão, interrupção.}{pa.ra.da}{0}
\verb{parada}{}{}{}{}{}{Local onde se para normalmente um veículo.}{pa.ra.da}{0}
\verb{parada}{}{}{}{}{}{Desfile militar.}{pa.ra.da}{0}
\verb{parada}{}{Pop.}{}{}{}{Empreendimento arriscado; aventura.}{pa.ra.da}{0}
\verb{paradeiro}{ê}{}{}{}{s.m.}{Lugar onde uma pessoa ou uma coisa se encontra ou vai parar.}{pa.ra.dei.ro}{0}
\verb{paradidático}{}{}{}{}{adj.}{Diz"-se do material escolar usado na complementação do ensino.}{pa.ra.di.dá.ti.co}{0}
\verb{paradigma}{}{}{}{}{s.m.}{Conjunto de formas que compõem um modelo; norma, padrão.}{pa.ra.dig.ma}{0}
\verb{paradisíaco}{}{}{}{}{adj.}{Relativo ao paraíso.}{pa.ra.di.sí.a.co}{0}
\verb{paradisíaco}{}{}{}{}{}{Que lembra o paraíso; encantador, aprazível.}{pa.ra.di.sí.a.co}{0}
\verb{parado}{}{}{}{}{adj.}{Que parou; sem movimento; fixo.}{pa.ra.do}{0}
\verb{parado}{}{Pop.}{}{}{}{Desempregado.}{pa.ra.do}{0}
\verb{parador}{ô}{}{}{}{adj.}{Diz"-se do veículo que para em todas as estações.}{pa.ra.dor}{0}
\verb{paradouro}{ô}{}{}{}{s.m.}{Paradeiro.}{pa.ra.dou.ro}{0}
\verb{paradoxal}{cs}{}{"-ais}{}{adj.2g.}{Em que há paradoxo.}{pa.ra.do.xal}{0}
\verb{paradoxo}{ócs}{}{}{}{s.m.}{Afirmação que parece contrária à opinião comum; contrassenso, absurdo, contradição.}{pa.ra.do.xo}{0}
\verb{paraense}{}{}{}{}{adj.2g.}{Relativo ao Pará.}{pa.ra.en.se}{0}
\verb{paraense}{}{}{}{}{s.2g.}{Indivíduo natural ou habitante desse estado.}{pa.ra.en.se}{0}
\verb{paraestatal}{}{}{"-ais}{}{adj.2g.}{Diz"-se de instituição autárquica em que o Estado intervém.}{pa.ra.es.ta.tal}{0}
\verb{parafernália}{}{}{}{}{s.f.}{Equipamento necessário a certas atividades.}{pa.ra.fer.ná.lia}{0}
\verb{parafernália}{}{}{}{}{}{Conjunto de objetos pessoais; pertences, tralha.}{pa.ra.fer.ná.lia}{0}
\verb{parafina}{}{Quím.}{}{}{s.f.}{Substância branca e sólida, proveniente da destilação do alcatrão do petróleo.}{pa.ra.fi.na}{0}
\verb{parafinar}{}{}{}{}{v.t.}{Misturar ou cobrir com parafina.}{pa.ra.fi.nar}{0}
\verb{parafinar}{}{}{}{}{}{Converter em parafina.}{pa.ra.fi.nar}{\verboinum{1}}
\verb{paráfrase}{}{}{}{}{s.f.}{Reprodução ou citação de um texto com palavras diferentes, mas com mesmo significado.}{pa.rá.fra.se}{0}
\verb{parafrasear}{}{}{}{}{v.t.}{Citar ou explicar um texto por meio de paráfrase.}{pa.ra.fra.se.ar}{\verboinum{4}}
\verb{parafusar}{}{}{}{}{v.t.}{Fixar ou prender com parafuso.}{pa.ra.fu.sar}{\verboinum{1}}
\verb{parafuso}{}{}{}{}{s.m.}{Tipo de prego com fenda na cabeça e com rosca, que se prende ou se solta girando para a direita ou para a esquerda.}{pa.ra.fu.so}{0}
\verb{paragem}{}{}{"-ens}{}{s.f.}{Ato ou efeito de parar; parada.}{pa.ra.gem}{0}
\verb{paragem}{}{}{"-ens}{}{}{Lugar onde se para; paradeiro, parada.}{pa.ra.gem}{0}
\verb{parágrafo}{}{}{}{}{s.m.}{Cada uma das seções de um texto que formam um sentido completo.}{pa.rá.gra.fo}{0}
\verb{parágrafo}{}{}{}{}{}{Disposição de um artigo de lei ou regulamento; alínea.}{pa.rá.gra.fo}{0}
\verb{paraguaio}{}{}{}{}{adj.}{Relativo ao Paraguai.}{pa.ra.guai.o}{0}
\verb{paraguaio}{}{}{}{}{s.m.}{Indivíduo natural ou habitante desse país.}{pa.ra.guai.o}{0}
\verb{paraibano}{}{}{}{}{adj.}{Relativo à Paraíba.}{pa.ra.i.ba.no}{0}
\verb{paraibano}{}{}{}{}{s.m.}{Indivíduo natural ou habitante desse estado.}{pa.ra.i.ba.no}{0}
\verb{paraíso}{}{}{}{}{s.m.}{Segundo a Bíblia, lugar onde Deus colocou Adão e Eva para viverem; éden, céu.}{pa.ra.í.so}{0}
\verb{paraíso}{}{Fig.}{}{}{}{Local aprazível, idílico.}{pa.ra.í.so}{0}
\verb{para"-lama}{}{}{para"-lamas}{}{s.m.}{Lâmina metálica que recobre as rodas de um veículo para protegê"-las da lama e de sujeira.}{pa.ra"-la.ma}{0}
\verb{paralela}{é}{Geom.}{}{}{s.f.}{Cada uma de duas ou mais retas que, localizadas em um mesmo plano, não têm ponto comum.}{pa.ra.le.la}{0}
\verb{paralelepípedo}{}{}{}{}{s.m.}{Corpo sólido de seis lados cujas faces são paralelogramos.}{pa.ra.le.le.pí.pe.do}{0}
\verb{paralelepípedo}{}{}{}{}{}{Pedra com esse formato usada no calçamento de ruas.}{pa.ra.le.le.pí.pe.do}{0}
\verb{paralelismo}{}{}{}{}{s.m.}{Posição de linhas ou superfícies paralelas.}{pa.ra.le.lis.mo}{0}
\verb{paralelismo}{}{}{}{}{}{Correspondências entre duas coisas ou situações.}{pa.ra.le.lis.mo}{0}
\verb{paralelo}{é}{}{}{}{adj.}{Diz"-se de linhas ou superfícies que ficam sempre à mesma distância.}{pa.ra.le.lo}{0}
\verb{paralelo}{é}{}{}{}{s.m.}{Termo de comparação; confronto, cotejo.}{pa.ra.le.lo}{0}
\verb{paralelo}{é}{Geogr.}{}{}{}{Cada um dos círculos da esfera terrestre perpendiculares aos meridianos e paralelos à linha do Equador. }{pa.ra.le.lo}{0}
\verb{paralelogramo}{}{Geom.}{}{}{s.m.}{Quadrilátero cujos lados opostos são iguais e paralelos.}{pa.ra.le.lo.gra.mo}{0}
\verb{paralisação}{}{}{"-ões}{}{s.f.}{Ato ou efeito de paralisar; interrupção, suspensão do movimento.}{pa.ra.li.sa.ção}{0}
\verb{paralisar}{}{}{}{}{v.t.}{Tornar inerte, sem movimento; imobilizar.}{pa.ra.li.sar}{\verboinum{1}}
\verb{paralisia}{}{Med.}{}{}{s.f.}{Perda ou diminuição dos movimentos musculares devido a problemas neurológicos.}{pa.ra.li.si.a}{0}
\verb{paralítico}{}{}{}{}{adj.}{Diz"-se de quem sofre de paralisia.}{pa.ra.lí.ti.co}{0}
\verb{paralogismo}{}{}{}{}{s.m.}{Argumento não conclusivo; falso raciocínio.}{pa.ra.lo.gis.mo}{0}
\verb{paramédico}{}{}{}{}{s.m.}{Indivíduo que atua em áreas auxiliares da medicina, sem ser médico.}{pa.ra.mé.di.co}{0}
\verb{paramentar}{}{}{}{}{v.t.}{Vestir com paramento.}{pa.ra.men.tar}{\verboinum{1}}
\verb{paramento}{}{}{}{}{s.m.}{Veste utilizada pelos clérigos em cerimônias religiosas.}{pa.ra.men.to}{0}
\verb{parâmetro}{}{Geom.}{}{}{s.m.}{Medida auxiliar para a representação analítica de curvas e superfícies.}{pa.râ.me.tro}{0}
\verb{parâmetro}{}{}{}{}{}{Norma, padrão, medida.}{pa.râ.me.tro}{0}
\verb{paramilitar}{}{}{}{}{adj.2g.}{Diz"-se de organizações particulares de cidadãos armados e exercitados que, contudo, não pertencem às forças militares.}{pa.ra.mi.li.tar}{0}
\verb{páramo}{}{}{}{}{s.m.}{Campo ou planície deserta.}{pá.ra.mo}{0}
\verb{páramo}{}{}{}{}{}{Abóbada celeste; firmamento, céu.}{pá.ra.mo}{0}
\verb{paraná}{}{}{}{}{s.m.}{Braço de um rio caudaloso, separado deste por uma ilha.}{pa.ra.ná}{0}
\verb{paraná}{}{}{}{}{}{Canal entre dois rios.}{pa.ra.ná}{0}
\verb{paranaense}{}{}{}{}{adj.2g.}{Relativo ao Paraná.}{pa.ra.na.en.se}{0}
\verb{paranaense}{}{}{}{}{s.2g.}{Indivíduo natural ou habitante desse estado.}{pa.ra.na.en.se}{0}
\verb{paraninfar}{}{}{}{}{v.t.}{Servir de paraninfo.}{pa.ra.nin.far}{\verboinum{1}}
\verb{paraninfo}{}{}{}{}{s.m.}{Pessoa escolhida por uma turma de formandos para ser homenageada na colação de grau; padrinho.}{pa.ra.nin.fo}{0}
\verb{paranoia}{ó}{Med.}{}{}{s.f.}{Conjunto de desordens psicóticas caracterizadas por mania de perseguição ou de grandeza, desconfiança, ciúme etc.}{pa.ra.noi.a}{0}
\verb{paranoico}{ó}{}{}{}{adj.}{Relativo a paranoia.}{pa.ra.noi.co}{0}
\verb{paranoico}{ó}{}{}{}{}{Diz"-se do indivíduo que sofre de paranoia.}{pa.ra.noi.co}{0}
\verb{paranormal}{}{}{"-ais}{}{adj.2g.}{Diz"-se do fenômeno ou experiência física não explicada cientificamente; sobrenatural.}{pa.ra.nor.mal}{0}
\verb{parapeito}{ê}{}{}{}{s.m.}{Parte de baixo de janelas, varandas, terraços etc. em que a pessoa pode se apoiar; peitoril. }{pa.ra.pei.to}{0}
\verb{parapente}{}{Esport.}{}{}{s.m.}{Aparelho esportivo na forma de um paraquedas retangular com o qual se salta de montanhas planando.}{pa.ra.pen.te}{0}
\verb{paraplegia}{}{Med.}{}{}{s.f.}{Paralisia dos membros inferiores, com alteração da sensibilidade e da movimentação.}{pa.ra.ple.gi.a}{0}
\verb{paraplégico}{}{}{}{}{adj.}{Relativo a paraplegia.}{pa.ra.plé.gi.co}{0}
\verb{paraplégico}{}{}{}{}{}{Diz"-se daquele que sofre de paraplegia.}{pa.ra.plé.gi.co}{0}
\verb{parapsicologia}{}{}{}{}{s.f.}{Teoria que busca explicar fenômenos transcendentais, tais como a telepatia, a premonição, a clarividência etc.}{pa.ra.psi.co.lo.gi.a}{0}
\verb{parapsicólogo}{}{}{}{}{s.m.}{Especialista em parapsicologia.}{pa.ra.psi.có.lo.go}{0}
\verb{paraquedas}{é}{}{}{}{s.m.}{Equipamento dobrável de \textit{nylon} em forma de guarda"-chuva ou em formato retangular, provido de cordas que sustentam pessoas ou carga reduzindo sua velocidade de queda.}{pa.ra.que.das}{0}
\verb{paraquedismo}{}{Esport.}{}{}{s.m.}{Conjunto de técnicas utilizadas no salto de paraquedas para fins militares ou esportivos.}{pa.ra.que.dis.mo}{0}
\verb{pára"-quedista}{}{}{}{}{s.2g.}{Indivíduo que pratica o salto de pára"-quedas como esporte.}{pá.ra"-que.dis.ta}{0}
\verb{pára"-quedista}{}{}{}{}{}{Militar que pertence ao corpo de pára"-quedismo.}{pá.ra"-que.dis.ta}{0}
\verb{parar}{}{}{}{}{v.t.}{Impedir o movimento; suspender, interromper.}{pa.rar}{0}
\verb{parar}{}{}{}{}{}{Não continuar; estacar.}{pa.rar}{0}
\verb{parar}{}{}{}{}{}{Deixar de fazer; largar.}{pa.rar}{\verboinum{1}}
\verb{pára"-raios}{}{}{}{}{s.m.}{Aparelho composto por uma haste metálica e um fio ligado à terra, colocado no ponto mais alto de um construção, destinado a atrair as descargas elétricas da atmosfera.}{pá.ra"-rai.os}{0}
\verb{parasita}{}{}{}{}{adj.2g.}{Diz"-se do ser vivo, animal ou vegetal, que se nutre do alimento de outrem.}{pa.ra.si.ta}{0}
\verb{parasita}{}{Fig.}{}{}{}{Diz"-se do indivíduo que vive à custa de outrem.}{pa.ra.si.ta}{0}
\verb{parasitar}{}{}{}{}{v.i.}{Viver como parasita, nutrindo"-se do alimento de outro ser vivo.}{pa.ra.si.tar}{\verboinum{1}}
\verb{parasitário}{}{}{}{}{adj.}{Relativo a parasita.}{pa.ra.si.tá.rio}{0}
\verb{parasitismo}{}{Biol.}{}{}{s.m.}{Interação entre duas espécies vegetais ou animais, na qual uma delas, o parasita, se nutre exclusivamente da outra, o hospedeiro, causando"-lhe danos que podem, inclusive, levá"-lo à morte.}{pa.ra.si.tis.mo}{0}
\verb{parasitismo}{}{Med.}{}{}{}{Infestação causada por parasitas.}{pa.ra.si.tis.mo}{0}
\verb{parasito}{}{}{}{}{s.m.}{Parasita.}{pa.ra.si.to}{0}
\verb{parasitologia}{}{}{}{}{s.f.}{Especialidade biomédica que estuda os parasitas.}{pa.ra.si.to.lo.gi.a}{0}
\verb{pára"-sol}{ó}{}{pára"-sóis}{}{s.m.}{Espécie de guarda"-chuva em formato grande, usado geralmente nas praias para proteção contra o sol; guarda"-sol.}{pá.ra"-sol}{0}
\verb{parati}{}{Zool.}{}{}{s.m.}{Espécie de peixe marinho com coloração prateada e carne saborosa, semelhante à tainha.}{pa.ra.ti}{0}
\verb{parati}{}{}{}{}{s.m.}{Aguardente de cana; cachaça.}{pa.ra.ti}{0}
\verb{pára"-vento}{}{}{pára"-ventos}{}{s.m.}{Espécie de biombo usado para proteger do vento; anteparo.}{pá.ra"-ven.to}{0}
\verb{parca}{}{}{}{}{s.f.}{Na mitologia grega, cada uma das três deusas que detinham o fio da vida humana.}{par.ca}{0}
\verb{parca}{}{Fig.}{}{}{}{A morte.}{par.ca}{0}
\verb{parceiro}{ê}{}{}{}{s.m.}{Companheiro para certas atividades, tais como danças, jogos, negócios etc.; sócio, par.}{par.cei.ro}{0}
\verb{parcel}{é}{}{"-éis}{}{s.m.}{Recife que aponta à superfície; baixio, escolho.}{par.cel}{0}
\verb{parcela}{é}{}{}{}{s.f.}{Pequena parte de algo; fragmento, fração.}{par.ce.la}{0}
\verb{parcela}{é}{Mat.}{}{}{}{Cada um dos números de uma operação de soma.}{par.ce.la}{0}
\verb{parcelamento}{}{}{}{}{s.m.}{Ato ou efeito de parcelar; divisão.}{par.ce.la.men.to}{0}
\verb{parcelar}{}{}{}{}{v.t.}{Dividir em parcelas. (\textit{Precisei parcelar minha dívida do cartão de crédito em dez vezes.})}{par.ce.lar}{\verboinum{1}}
\verb{parceria}{}{}{}{}{s.f.}{Reunião de indivíduos com interesses comuns; sociedade, companhia.}{par.ce.ri.a}{0}
\verb{parcial}{}{}{"-ais}{}{adj.2g.}{Que é parte de um todo ou se realiza em partes.}{par.ci.al}{0}
\verb{parcial}{}{}{"-ais}{}{}{Que é favorável a uma das partes em litígio; partidário, injusto.}{par.ci.al}{0}
\verb{parcialidade}{}{}{}{}{s.f.}{Qualidade de quem é parcial; partidarismo.}{par.ci.a.li.da.de}{0}
\verb{parcimônia}{}{}{}{}{s.f.}{Qualidade do que é parco; economia, sobriedade, frugalidade.}{par.ci.mô.nia}{0}
\verb{parcimonioso}{ô}{}{"-osos ⟨ó⟩}{"-osa ⟨ó⟩}{adj.}{Que faz economia; sóbrio, parco, frugal.}{par.ci.mo.ni.o.so}{0}
\verb{parco}{}{}{}{}{adj.}{Que faz economia; parcimonioso, sóbrio, comedido.}{par.co}{0}
\verb{pardacento}{}{}{}{}{adj.}{De cor próxima ao pardo.}{par.da.cen.to}{0}
\verb{pardal}{}{Zool.}{"-ais}{}{s.m.}{Pássaro de pequeno porte, de coloração parda, muito comum nas áreas urbanas.}{par.dal}{0}
\verb{pardavasco}{}{}{}{}{s.m.}{Mestiço de negro e índio.}{par.da.vas.co}{0}
\verb{pardieiro}{ê}{}{}{}{s.m.}{Edifício em ruínas, abandonado.}{par.di.ei.ro}{0}
\verb{pardo}{}{}{}{}{adj.}{De cor escura, entre o branco e o preto.}{par.do}{0}
\verb{pardo}{}{}{}{}{}{De cor fosca, entre o amarelo e o marrom escuro.}{par.do}{0}
\verb{pardo}{}{}{}{}{s.m.}{Indivíduo de cor escura; mulato.}{par.do}{0}
\verb{pardoca}{ó}{}{}{}{s.f.}{A fêmea do pardal.}{par.do.ca}{0}
\verb{parecença}{}{}{}{}{s.f.}{Qualidade do que é parecido; semelhança.}{pa.re.cen.ça}{0}
\verb{parecer}{ê}{}{}{}{v.pred.}{Dar a impressão de ser algo; aparentar.}{pa.re.cer}{0}
\verb{parecer}{ê}{}{}{}{v.pron.}{Ter o aspecto de; assemelhar"-se.}{pa.re.cer}{\verboinum{15}}
\verb{parecer}{ê}{}{}{}{s.m.}{Opinião dada por um especialista; julgamento, juízo.}{pa.re.cer}{0}
\verb{parecido}{}{}{}{}{adj.}{Que se parece, se assemelha; semelhante.}{pa.re.ci.do}{0}
\verb{paredão}{}{}{"-ões}{}{s.m.}{Parede alta e espessa; muralha.}{pa.re.dão}{0}
\verb{paredão}{}{}{"-ões}{}{}{Ribanceira alta de um rio.}{pa.re.dão}{0}
\verb{paredão}{}{Fig.}{"-ões}{}{}{Local onde se fuzilam prisioneiros.}{pa.re.dão}{0}
\verb{parede}{ê}{}{}{}{s.f.}{Cada uma das partes externas de um edifício e de suas divisões internas, feitas geralmente de alvenaria.}{pa.re.de}{0}
\verb{parede}{ê}{}{}{}{}{Cada uma das partes que dividem um espaço; divisória, tapume.}{pa.re.de}{0}
\verb{parede"-meia}{ê}{}{paredes"-meias}{}{s.f.}{Parede divisória comum a dois prédios contíguos.}{pa.re.de"-mei.a}{0}
\verb{paredista}{}{}{}{}{s.2g.}{Diz"-se do indivíduo que promove greves; grevista.}{pa.re.dis.ta}{0}
\verb{paredro}{ê}{}{}{}{s.m.}{Conselheiro que indica o caminho a seguir; mentor.}{pa.re.dro}{0}
\verb{parelha}{ê}{}{}{}{s.f.}{Par de animais que puxam um veículo; dupla.}{pa.re.lha}{0}
\verb{parelheiro}{ê}{}{}{}{adj.}{Diz"-se do cavalo que anda em parelha.}{pa.re.lhei.ro}{0}
\verb{parelho}{ê}{}{}{}{adj.}{Que possui a mesma medida; semelhante, igual.}{pa.re.lho}{0}
\verb{parênquima}{}{Biol.}{}{}{s.m.}{Tecido formado por células com função específica.}{pa.rên.qui.ma}{0}
\verb{parente}{}{}{}{}{s.2g.}{Indivíduo pertencente à mesma família.}{pa.ren.te}{0}
\verb{parentela}{é}{}{}{}{s.f.}{Conjunto de parentes.}{pa.ren.te.la}{0}
\verb{parentesco}{ê}{}{}{}{s.m.}{Relação existente entre membros de uma mesma família.}{pa.ren.tes.co}{0}
\verb{parêntese}{}{}{}{}{s.m.}{Frase intercalada em outra, com autonomia sintática e semântica.}{pa.rên.te.se}{0}
\verb{parêntese}{}{}{}{}{}{Cada um dos sinais gráficos -( )- utilizados para iniciar e encerrar uma frase intercalada em um texto escrito.}{pa.rên.te.se}{0}
\verb{parêntesis}{}{}{}{}{s.m.}{Parêntese.}{pa.rên.te.sis}{0}
\verb{páreo}{}{}{}{}{s.m.}{Competição de corrida de cavalos.}{pá.re.o}{0}
\verb{páreo}{}{Fig.}{}{}{}{Qualquer disputa.}{pá.re.o}{0}
\verb{parestatal}{}{}{"-ais}{}{adj.2g.}{Diz"-se de autarquia sujeita a intervenção do Estado.}{pa.res.ta.tal}{0}
\verb{parestesia}{}{Med.}{}{}{s.f.}{Distúrbio caracterizado por sensações táteis de formigamento e ardência, sem que haja fator externo que as justifique.}{pa.res.te.si.a}{0}
\verb{pária}{}{}{}{}{s.m.}{Na Índia, indivíduo da casta mais baixa.}{pá.ria}{0}
\verb{pária}{}{Fig.}{}{}{}{Indivíduo relegado pela sociedade.}{pá.ria}{0}
\verb{paridade}{}{}{}{}{s.f.}{Qualidade de par.}{pa.ri.da.de}{0}
\verb{paridade}{}{}{}{}{}{Igualdade de condições ou de valores.}{pa.ri.da.de}{0}
\verb{parietal}{}{}{"-ais}{}{adj.2g.}{Relativo a parede.}{pa.ri.e.tal}{0}
\verb{parietal}{}{Anat.}{"-ais}{}{s.m.}{Cada um dos ossos que formam a porção lateral do crânio.}{pa.ri.e.tal}{0}
\verb{parir}{}{}{}{}{v.t.}{Dar à luz.}{pa.rir}{\verboinum{35}}
\verb{parisiense}{}{}{}{}{adj.2g.}{Relativo a Paris, capital da França.}{pa.ri.si.en.se}{0}
\verb{parisiense}{}{}{}{}{s.2g.}{Indivíduo natural ou habitante dessa cidade.}{pa.ri.si.en.se}{0}
\verb{parlamentar}{}{}{}{}{adj.2g.}{Relativo a parlamento.}{par.la.men.tar}{0}
\verb{parlamentar}{}{}{}{}{v.t.}{Manter conversações com objetivo de negociar assuntos específicos; conferenciar.}{par.la.men.tar}{\verboinum{1}}
\verb{parlamentar}{}{}{}{}{s.2g.}{Indivíduo membro do parlamento.}{par.la.men.tar}{0}
\verb{parlamentarismo}{}{}{}{}{s.m.}{Sistema político em que um gabinete de ministros formado dentro do parlamento é o responsável pelo governo.}{par.la.men.ta.ris.mo}{0}
\verb{parlamentarista}{}{}{}{}{adj.2g.}{Relativo ao parlamentarismo.}{par.la.men.ta.ris.ta}{0}
\verb{parlamentarista}{}{}{}{}{s.2g.}{Indivíduo partidário do parlamentarismo.}{par.la.men.ta.ris.ta}{0}
\verb{parlamento}{}{}{}{}{s.m.}{O conjunto das câmaras e assembleias legislativas de um Estado.}{par.la.men.to}{0}
\verb{parlamento}{}{}{}{}{}{Congresso nacional.}{par.la.men.to}{0}
\verb{parlapatão}{}{}{"-ões}{}{adj.}{Impostor, mentiroso, fanfarrão.}{par.la.pa.tão}{0}
\verb{parlatório}{}{}{}{}{s.m.}{Em conventos e prisões, compartimento em que os internos conversam com visitantes.}{par.la.tó.rio}{0}
\verb{parlatório}{}{}{}{}{}{Tipo de balcão em um edifício onde uma personalidade se coloca para discursar.}{par.la.tó.rio}{0}
\verb{parmesão}{}{}{}{}{adj.}{Relativo a Parma, cidade da Itália.}{par.me.são}{0}
\verb{parmesão}{}{}{}{}{}{Diz"-se de um tipo de queijo de massa dura, sabor forte e razoavelmente salgado, muito usado para ralar.}{par.me.são}{0}
\verb{parmesão}{}{}{}{}{s.m.}{Indivíduo natural ou habitante de Parma.}{par.me.são}{0}
\verb{parmesão}{}{}{}{}{}{Redução de \textit{queijo parmesão}.}{par.me.são}{0}
\verb{Parnasianismo}{}{Liter.}{}{}{s.m.}{Escola poética em que se cultivava a perfeição da forma e a objetividade, em oposição ao lirismo característico do Romantismo.}{Par.na.si.a.nis.mo}{0}
\verb{parnasiano}{}{Liter.}{}{}{adj.}{Adepto do Parnasianismo.}{par.na.si.a.no}{0}
\verb{parnasiano}{}{}{}{}{}{Relativo ao parnaso, local simbólico em que viviam os poetas.}{par.na.si.a.no}{0}
\verb{paroara}{}{Bras.}{}{}{s.m.}{Certo pássaro de cor branca ou preta; cardeal.}{paroara}{0}
\verb{pároco}{}{}{}{}{s.m.}{Padre encarregado de uma paróquia; vigário.}{pá.ro.co}{0}
\verb{paródia}{}{Art.}{}{}{s.f.}{Obra, geralmente de caráter cômico, cujo objetivo ostensivo é imitar uma outra composição.}{pa.ró.dia}{0}
\verb{parodiar}{}{}{}{}{v.t.}{Fazer paródia de; imitar comicamente.}{pa.ro.di.ar}{\verboinum{6}}
\verb{parodista}{}{}{}{}{s.2g.}{Indivíduo que faz paródias.}{pa.ro.dis.ta}{0}
\verb{paronímia}{}{}{}{}{s.f.}{Qualidade de parônimo.}{pa.ro.ní.mia}{0}
\verb{parônimo}{}{Gram.}{}{}{adj.}{Diz"-se de palavras diferentes de som bastante semelhante, e portanto sujeitas a confusão.}{pa.rô.ni.mo}{0}
\verb{paroníquia}{}{Med.}{}{}{s.f.}{Inflamação na pele em torno de unha; panarício, panariz.}{pa.ro.ní.quia}{0}
\verb{paróquia}{}{}{}{}{s.f.}{Divisão territorial da Igreja, parte de uma diocese, e sob jurisdição de um pároco.}{pa.ró.quia}{0}
\verb{paroquial}{"-ais}{}{}{}{adj.2g.}{Relativo a paróquia ou pároco.}{pa.ro.qui.al}{0}
\verb{paroquiano}{}{}{}{}{adj.}{Que mora em uma paróquia.}{pa.ro.qui.a.no}{0}
\verb{parótida}{}{Anat.}{}{}{s.f.}{Cada uma de duas glândulas salivares, localizada adiante e abaixo de cada orelha.  }{pa.ró.ti.da}{0}
\verb{parótide}{}{Anat.}{}{}{s.f.}{Parótida.}{pa.ró.ti.de}{0}
\verb{parotidite}{}{Med.}{}{}{s.f.}{Inflamação das parótidas; caxumba.}{pa.ro.ti.di.te}{0}
\verb{paroxismo}{cs}{Med.}{}{}{s.m.}{Estágio de uma doença em que os sintomas têm maior intensidade.}{pa.ro.xis.mo}{0}
\verb{paroxismo}{cs}{Fig.}{}{}{}{Apogeu, auge, clímax.}{pa.ro.xis.mo}{0}
\verb{paroxítono}{cs}{Gram.}{}{}{adj.}{Diz"-se de vocábulo cujo acento primário se localiza na penúltima sílaba.}{pa.ro.xí.to.no}{0}
\verb{parque}{}{Bras.}{}{}{s.m.}{Local público geralmente com vegetação abundante.}{par.que}{0}
\verb{parque}{}{}{}{}{}{Bosque cercado em que existe caça.}{par.que}{0}
\verb{parque}{}{}{}{}{}{Porção de terra fechada e com vegetação anexa a uma propriedade.}{par.que}{0}
\verb{parquímetro}{}{}{}{}{s.m.}{Aparelho que, em locais públicos de estacionamento, mede o tempo de permanência dos veículos.}{par.quí.me.tro}{0}
\verb{parra}{}{}{}{}{s.f.}{Ramo de videira.}{par.ra}{0}
\verb{parreira}{ê}{Bot.}{}{}{s.f.}{Certas plantas trepadeiras, especialmente a videira.}{par.rei.ra}{0}
\verb{parreiral}{}{}{"-ais}{}{s.m.}{Aglomerado de parreiras.}{par.rei.ral}{0}
\verb{parricida}{}{}{}{}{adj.}{Que praticou parricídio.}{par.ri.ci.da}{0}
\verb{parricídio}{}{}{}{}{s.m.}{Homicídio praticado contra pai, mãe ou outros antepassados.}{par.ri.cí.dio}{0}
\verb{parrudo}{}{}{}{}{adj.}{Rasteiro como uma parra.}{par.ru.do}{0}
\verb{parrudo}{}{Fig.}{}{}{}{Baixo e musculoso.}{par.ru.do}{0}
\verb{parsec}{é}{Astron.}{}{}{s.m.}{Unidade astronômica de distância que corresponde a aproximadamente 3 anos"-luz. Símb.: pc. }{par.sec}{0}
\verb{parte}{}{}{}{}{s.f.}{Cada divisão de um todo.}{par.te}{0}
\verb{parte}{}{}{}{}{}{Lado, banda.}{par.te}{0}
\verb{parte}{}{}{}{}{}{Lugar, localidade, sítio.}{par.te}{0}
\verb{parte}{}{}{}{}{}{Tarefa, função, atribuição de cada um em um trabalho em equipe.}{par.te}{0}
\verb{parte}{}{Jur.}{}{}{}{Cada um dos lados que se encontram em oposição em um contrato ou em um processo judicial.}{par.te}{0}
\verb{parteira}{ê}{}{}{}{s.f.}{Mulher que ajuda parturientes no momento de parir.}{par.tei.ra}{0}
\verb{parteiro}{ê}{}{}{}{adj.}{Diz"-se de médico especializado em obstetrícia.}{par.tei.ro}{0}
\verb{partejar}{}{}{}{}{v.t.}{Prestar serviços como parteiro.}{par.te.jar}{0}
\verb{partejar}{}{}{}{}{v.i.}{Parir.}{par.te.jar}{\verboinum{1}}
\verb{partição}{}{}{"-ões}{}{s.f.}{Ato ou efeito de partir, dividir.}{par.ti.ção}{0}
\verb{participação}{}{}{"-ões}{}{s.f.}{Ato ou efeito de participar.}{par.ti.ci.pa.ção}{0}
\verb{participante}{}{}{}{}{adj.2g.}{Que toma parte em algo; que participa de.}{par.ti.ci.pan.te}{0}
\verb{participar}{}{}{}{}{v.t.}{Fazer saber; comunicar, informar, dar parte.}{par.ti.ci.par}{0}
\verb{participar}{}{}{}{}{}{Tomar parte; partilhar, compartilhar.}{par.ti.ci.par}{0}
\verb{participar}{}{}{}{}{}{Compartilhar de um sentimento ou pensamento.}{par.ti.ci.par}{\verboinum{1}}
\verb{particípio}{}{Gram.}{}{}{s.m.}{Forma nominal do verbo, com propriedades gramaticais de nome, formada, nos casos regulares, com os sufixos \textit{"-ado} (verbos da primeira conjugação) ou \textit{"-ido} (verbos da segunda e terceira conjugações).}{par.ti.cí.pio}{0}
\verb{partícula}{}{}{}{}{s.f.}{Parte pequena ou muito pequena.}{par.tí.cu.la}{0}
\verb{partícula}{}{}{}{}{}{Corpo de dimensões muito pequenas; corpúsculo.}{par.tí.cu.la}{0}
\verb{partícula}{}{Gram.}{}{}{}{Palavras invariáveis, em geral monossilábicas e átonas, de função gramatical.}{par.tí.cu.la}{0}
\verb{particular}{}{}{}{}{adj.2g.}{Relativo a um, a poucos ou a uma pequena parte.}{par.ti.cu.lar}{0}
\verb{particular}{}{}{}{}{}{De propriedade ou uso exclusivo; privativo, privado.}{par.ti.cu.lar}{0}
\verb{particular}{}{}{}{}{}{Peculiar, especial.}{par.ti.cu.lar}{0}
\verb{particular}{}{}{}{}{s.m.}{Um indivíduo qualquer.}{par.ti.cu.lar}{0}
\verb{particular}{}{}{}{}{}{Detalhes, pormenores. (Usa"-se geralmente no plural nesta acepção.)}{par.ti.cu.lar}{0}
\verb{particularidade}{}{}{}{}{s.f.}{Característica, singularidade, minúcia, pormenor.}{par.ti.cu.la.ri.da.de}{0}
\verb{particularidade}{}{}{}{}{}{Qualidade de particular.}{par.ti.cu.la.ri.da.de}{0}
\verb{particularizar}{}{}{}{}{v.t.}{Tornar específico; individualizar.}{par.ti.cu.la.ri.zar}{0}
\verb{particularizar}{}{}{}{}{}{Dar detalhes; pormenorizar, detalhar.}{par.ti.cu.la.ri.zar}{\verboinum{1}}
\verb{partida}{}{}{}{}{s.f.}{Ato ou efeito de partir; saída.}{par.ti.da}{0}
\verb{partida}{}{}{}{}{}{Competição, jogo.}{par.ti.da}{0}
\verb{partidário}{}{}{}{}{adj.}{Relativo a partido.}{par.ti.dá.rio}{0}
\verb{partidário}{}{}{}{}{}{Diz"-se de integrante de um partido.}{par.ti.dá.rio}{0}
\verb{partidário}{}{}{}{}{}{Diz"-se de seguidor ou simpatizante de ideia, crença etc.}{par.ti.dá.rio}{0}
\verb{partidarismo}{}{}{}{}{s.m.}{Fanatismo partidário.}{par.ti.da.ris.mo}{0}
\verb{partido}{}{}{}{}{s.m.}{Organização política com ideias comuns, que tem como objetivo chegar ao poder.}{par.ti.do}{0}
\verb{partilha}{}{}{}{}{s.f.}{Ato de dar a cada um a sua parte; divisão.}{par.ti.lha}{0}
\verb{partilhar}{}{}{}{}{v.t.}{Repartir com.}{par.ti.lhar}{0}
\verb{partilhar}{}{}{}{}{}{Participar de; compartilhar.}{par.ti.lhar}{\verboinum{1}}
\verb{partir}{}{}{}{}{v.t.}{Separar em pedaços; cortar, dividir.}{par.tir}{0}
\verb{partir}{}{}{}{}{}{Arrebentar, quebrar, romper.}{par.tir}{0}
\verb{partir}{}{}{}{}{}{Ir embora; sair.}{par.tir}{\verboinum{18}}
\verb{partitura}{}{}{}{}{s.f.}{Escrita das partes vocais ou instrumentais de uma obra musical.}{par.ti.tu.ra}{0}
\verb{parto}{}{}{}{}{s.m.}{Ato ou efeito de parir.}{par.to}{0}
\verb{parto}{}{}{}{}{}{Expulsão do feto e seus anexos do corpo da mãe.}{par.to}{0}
\verb{parto}{}{}{}{}{}{Trabalho, tarefa geralmente exaustiva e difícil.}{par.to}{0}
\verb{parturiente}{}{}{}{}{adj.}{Diz"-se de mulher que está prestes a ou acabou de parir.}{par.tu.ri.en.te}{0}
\verb{parvo}{}{}{}{}{adj.}{Diz"-se de indivíduo bobo, tolo, estúpido.}{par.vo}{0}
\verb{parvoíce}{}{}{}{}{}{Qualidade ou estado de parvo; burrice, tolice.}{par.vo.í.ce}{0}
\verb{pascal}{}{}{"-ais}{}{adj.2g.}{Relativo à Pascoa.}{pas.cal}{0}
\verb{Páscoa}{}{}{}{}{s.f.}{Festa cristã da ressurreição de Cristo.}{Pás.co.a}{0}
\verb{pascoal}{}{}{}{}{}{Var. de \textit{pascal}.}{pas.co.al}{0}
\verb{pasmaceira}{ê}{}{}{}{s.f.}{Estado ou condição caracterizada pela falta de interesse; apatia.}{pas.ma.cei.ra}{0}
\verb{pasmado}{}{}{}{}{adj.}{Que teve uma surpresa grande; espantado, surpreendido.}{pas.ma.do}{0}
\verb{pasmar}{}{}{}{}{v.t.}{Causar espanto; assombrar, espantar, surpreender.}{pas.mar}{\verboinum{1}}
\verb{pasmo}{}{}{}{}{s.m.}{Sentimento de espanto, de surpresa diante de algo que não se espera; admiração, assombro.}{pas.mo}{0}
\verb{pasmo}{}{}{}{}{adj.}{Tomado de pasmo; admirado, espantado, assombrado.}{pas.mo}{0}
\verb{pasmoso}{ô}{}{"-osos ⟨ó⟩}{"-osa ⟨ó⟩}{adj.}{Que causa pasmo; espantoso.}{pas.mo.so}{0}
\verb{paspalhão}{}{}{"-ões ⟨paspalhona⟩}{}{adj.}{Diz"-se de indivíduo tolo ou que é inútil.}{pas.pa.lhão}{0}
\verb{paspalhice}{}{}{}{}{s.f.}{Atitude de paspalhão.}{pas.pa.lhi.ce}{0}
\verb{paspalho}{}{}{}{}{adj.}{Paspalhão.}{pas.pa.lho}{0}
\verb{pasquim}{}{}{"-ins}{}{s.m.}{Texto satírico colocado em local público.}{pas.quim}{0}
\verb{pasquim}{}{}{"-ins}{}{}{Jornal ou folheto calunioso.}{pas.quim}{0}
\verb{passa}{}{}{}{}{s.f.}{Fruta seca ao sol, especialmente uva.}{pas.sa}{0}
\verb{passada}{}{}{}{}{s.f.}{Movimento dos pés para andar; passo.}{pas.sa.da}{0}
\verb{passada}{}{}{}{}{}{Visita rápida.}{pas.sa.da}{0}
\verb{passadeira}{ê}{}{}{}{s.f.}{Tapete estreito e longo usado em corredor.}{pas.sa.dei.ra}{0}
\verb{passadeira}{ê}{}{}{}{}{Empregada que passa roupa.}{pas.sa.dei.ra}{0}
\verb{passadeira}{ê}{}{}{}{}{Anel por onde passa a gravata.}{pas.sa.dei.ra}{0}
\verb{passadiço}{}{}{}{}{adj.}{Que passa logo; passageiro, transitório.}{pas.sa.di.ço}{0}
\verb{passadiço}{}{}{}{}{}{Corredor de comunicação.}{pas.sa.di.ço}{0}
\verb{passadiço}{}{}{}{}{}{Calçada.}{pas.sa.di.ço}{0}
\verb{passadio}{}{}{}{}{s.m.}{Alimentação diária.}{pas.sa.di.o}{0}
\verb{passadismo}{}{}{}{}{s.m.}{Devoção ao passado; saudosismo.}{pas.sa.dis.mo}{0}
\verb{passadista}{}{}{}{}{adj.2g.}{Relativo ao passado ou ao passadismo.}{pas.sa.dis.ta}{0}
\verb{passadista}{}{}{}{}{}{Diz"-se de quem venera o passado.}{pas.sa.dis.ta}{0}
\verb{passadista}{}{}{}{}{}{Diz"-se de quem é adepto do passadismo.}{pas.sa.dis.ta}{0}
\verb{passado}{}{}{}{}{adj.}{Que passou; decorrido.}{pas.sa.do}{0}
\verb{passado}{}{}{}{}{}{Que passou do tempo certo para se comer; estragado.}{pas.sa.do}{0}
\verb{passado}{}{}{}{}{s.m.}{O tempo que passou.}{pas.sa.do}{0}
\verb{passador}{ô}{}{}{}{s.m.}{Utensílio de cozinha para espremer batatas, massas.}{pas.sa.dor}{0}
\verb{passador}{ô}{}{}{}{}{Tira de roupa por onde passa o cinto.}{pas.sa.dor}{0}
\verb{passador}{ô}{}{}{}{}{Pregador de cabelo.}{pas.sa.dor}{0}
\verb{passador}{ô}{}{}{}{}{Coador.}{pas.sa.dor}{0}
\verb{passador}{ô}{}{}{}{}{Indivíduo que vende mercadoria falsificada ou roubada.}{pas.sa.dor}{0}
\verb{passageiro}{ê}{}{}{}{adj.}{Que não é permanente; efêmero.}{pas.sa.gei.ro}{0}
\verb{passageiro}{ê}{}{}{}{s.m.}{Indivíduo que é transportado num veículo.}{pas.sa.gei.ro}{0}
\verb{passagem}{}{}{"-ens}{}{s.f.}{Ato ou efeito de passar.}{pas.sa.gem}{0}
\verb{passagem}{}{}{"-ens}{}{}{Lugar por onde se passa.}{pas.sa.gem}{0}
\verb{passagem}{}{}{"-ens}{}{}{Parte que se cita de uma obra: trecho.}{pas.sa.gem}{0}
\verb{passagem}{}{}{"-ens}{}{}{Bilhete que permite viajar em veículo de transporte coletivo; passe.}{pas.sa.gem}{0}
\verb{passamanaria}{}{}{}{}{s.f.}{Trabalho com passamanes.}{pas.sa.ma.na.ri.a}{0}
\verb{passamanes}{}{}{}{}{s.m.pl.}{Bordado ou trannçado de seda usado em roupa, cortina etc.}{pas.sa.ma.nes}{0}
\verb{passamento}{}{}{}{}{s.m.}{Morte.}{pas.sa.men.to}{0}
\verb{passante}{}{}{}{}{adj.2g.}{Diz"-se de pessoa que passa por algum lugar; transeunte.}{pas.san.te}{0}
\verb{passante}{}{}{}{}{}{Que ultrapassa.}{pas.san.te}{0}
\verb{passaporte}{ó}{}{}{}{s.m.}{Documento oficial para viagem internacional.}{pas.sa.por.te}{0}
\verb{passar}{}{}{}{}{v.t.}{Ir além de algum lugar; transpor, ultrapassar.}{pas.sar}{0}
\verb{passar}{}{}{}{}{}{Apresentar em meio de comunicação; transmitir, exibir.}{pas.sar}{0}
\verb{passar}{}{}{}{}{}{Deixar como herança; legar.}{pas.sar}{0}
\verb{passar}{}{}{}{}{}{Mudar de um lugar a outro; deslocar.}{pas.sar}{0}
\verb{passar}{}{}{}{}{}{Espalhar sobre uma superfície.}{pas.sar}{0}
\verb{passar}{}{}{}{}{}{Ser aprovado em exame.}{pas.sar}{0}
\verb{passar}{}{}{}{}{v.i.}{Transcorrer no tempo; correr seu curso; decorrer.}{pas.sar}{0}
\verb{passar}{}{}{}{}{}{Sentir"-se física ou mentalmente; estar, ficar. (\textit{A menina passou mal após comer o sanduíche.})}{pas.sar}{\verboinum{1}}
\verb{passarada}{}{}{}{}{s.f.}{Porção de pássaros.}{pas.sa.ra.da}{0}
\verb{passarela}{é}{}{}{}{s.f.}{Ponte para passagem de pedestres sobre avenida.}{pas.sa.re.la}{0}
\verb{passarela}{é}{}{}{}{}{Plataforma para desfiles, geralmente de moda.}{pas.sa.re.la}{0}
\verb{passarinhar}{}{}{}{}{v.i.}{Caçar passarinhos.}{pas.sa.ri.nhar}{0}
\verb{passarinhar}{}{}{}{}{}{Ficar sem fazer nada; vadiar.}{pas.sa.ri.nhar}{\verboinum{1}}
\verb{passarinheiro}{ê}{}{}{}{adj.}{Diz"-se de quem caça, cria ou vende pássaros.}{pas.sa.ri.nhei.ro}{0}
\verb{passarinho}{}{}{}{}{s.m.}{Pássaro pequeno.}{pas.sa.ri.nho}{0}
\verb{pássaro}{}{Zool.}{}{}{s.m.}{Ave de tamanho pequeno ou médio, com três dedos anteriores e um posterior, que voa e canta.}{pás.sa.ro}{0}
\verb{passatempo}{}{}{}{}{s.m.}{Atividade que diverte, que distrai.}{pas.sa.tem.po}{0}
\verb{passável}{}{}{"-eis}{}{adj.2g.}{Que pode passar, ser aceito; aceitável, admissível.}{pas.sá.vel}{0}
\verb{passável}{}{}{"-eis}{}{}{Mais ou menos de acordo com o desejável; tolerável.}{pas.sá.vel}{0}
\verb{passe}{}{}{}{}{s.m.}{Licença para passar de um lugar a outro.}{pas.se}{0}
\verb{passe}{}{}{}{}{}{Suposta cura espiritual transmitida pelas mãos.}{pas.se}{0}
\verb{passe}{}{}{}{}{}{Documento que permite a entrada em um veículo.}{pas.se}{0}
\verb{passe}{}{}{}{}{}{Ato de passar a bola em jogo.}{pas.se}{0}
\verb{passeador}{ô}{}{}{}{adj.}{Que gosta de passear.}{pas.se.a.dor}{0}
\verb{passear}{}{}{}{}{v.i.}{Andar em algum lugar por lazer.}{pas.se.ar}{0}
\verb{passear}{}{}{}{}{}{Andar calmamente.}{pas.se.ar}{\verboinum{4}}
\verb{passeata}{}{}{}{}{s.f.}{Marcha coletiva para festejar ou protestar.}{pas.se.a.ta}{0}
\verb{passeio}{ê}{}{}{}{s.m.}{Caminhada ou saída para lazer.}{pas.sei.o}{0}
\verb{passeio}{ê}{}{}{}{}{Calçada.}{pas.sei.o}{0}
\verb{passeriforme}{ó}{Zool.}{}{}{s.m.}{Espécime dos passeriformes, ordem de aves de porte pequeno ou médio.}{pas.se.ri.for.me}{0}
\verb{passiflora}{ó}{Bot.}{}{}{s.f.}{Gênero de plantas das regiões tropicais, à qual pertencem os maracujás.}{pas.si.flo.ra}{0}
\verb{passional}{}{}{"-ais}{}{adj.2g.}{Próprio de ou regido por paixão.}{pas.si.o.nal}{0}
\verb{passista}{}{}{}{}{s.2g.}{Dançarino de frevo ou samba.}{pas.sis.ta}{0}
\verb{passiva}{}{Gram.}{}{}{s.f.}{Maneira de construir frases usando os verbos auxiliares ser ou estar; voz passiva.}{pas.si.va}{0}
\verb{passível}{}{}{"-eis}{}{adj.2g.}{Que é suscetível a experimentar sensações e emoções ou de ser objeto de certas ações.}{pas.sí.vel}{0}
\verb{passível}{}{}{"-eis}{}{}{Que está ou fica sujeito a penas ou sanções.}{pas.sí.vel}{0}
\verb{passividade}{}{}{}{}{s.f.}{Qualidade do que é passivo.}{pas.si.vi.da.de}{0}
\verb{passivo}{}{}{}{}{adj.}{Em que não existe; inerte.}{pas.si.vo}{0}
\verb{passivo}{}{}{}{}{s.m.}{Conjunto de dívidas e obrigações de pessoa ou empresa.}{pas.si.vo}{0}
\verb{passo}{}{}{}{}{s.m.}{Ato de avançar ou recuar cada pé ao andar.}{pas.so}{0}
\verb{passo}{}{}{}{}{}{Distância percorrida nesse movimento.}{pas.so}{0}
\verb{passo}{}{}{}{}{}{Movimento definido de dança.}{pas.so}{0}
\verb{passo}{}{}{}{}{}{Resolução, medida.}{pas.so}{0}
\verb{pasta}{}{}{}{}{s.f.}{Substância de substância entre mole e dura.}{pas.ta}{0}
\verb{pasta}{}{}{}{}{}{Capa de papel grosso para se guardar alguma coisa.}{pas.ta}{0}
\verb{pasta}{}{}{}{}{}{Maleta alta, comprida ou estreita para se carregar alguma coisa.}{pas.ta}{0}
\verb{pasta}{}{}{}{}{}{Posto de ministro.}{pas.ta}{0}
\verb{pastagem}{}{}{"-ens}{}{s.f.}{Lugar onde os animais pastam; pasto.}{pas.ta.gem}{0}
\verb{pastar}{}{}{}{}{v.i.}{Comer capim.}{pas.tar}{\verboinum{1}}
\verb{pastel}{é}{Cul.}{do s.m.: -éis}{}{s.m.}{Comida feita com massa de farinha de trigo, estendida com rolo e cortada em pequenas porções, que são dobradas sobre recheio, salgado ou doce, depois fritas.}{pas.tel}{0}
\verb{pastel}{é}{}{do s.m.: -éis}{}{}{Bastão feito com giz a que se adicionam pigmentos de várias cores.}{pas.tel}{0}
\verb{pastel}{é}{}{do s.m.: -éis}{}{adj.2g.}{Diz"-se de cores tênues e suaves que lembram os tons do pastel.}{pas.tel}{0}
\verb{pastelão}{}{Cul.}{"-ões}{}{s.m.}{Torta salgada de forno.}{pas.te.lão}{0}
\verb{pastelão}{}{}{"-ões}{}{}{Comédia cheia de trapalhadas.}{pas.te.lão}{0}
\verb{pastelaria}{}{}{}{}{s.f.}{Casa onde se fazem e vendem pastéis.}{pas.te.la.ri.a}{0}
\verb{pasteleiro}{ê}{}{}{}{s.m.}{Indivíduo que prepara ou vende pastéis.}{pas.te.lei.ro}{0}
\verb{pasteurização}{}{}{"-ões}{}{s.f.}{Ato ou efeito de pasteurizar; esterilização por aquecimento de temperatura, durante um tempo relativamente longo, sendo o líquido, em seguida, submetido a um resfriamento súbito.}{pas.teu.ri.za.ção}{0}
\verb{pasteurizado}{}{}{}{}{adj.}{Que foi submetido ao processo de pasteurização.}{pas.teu.ri.za.do}{0}
\verb{pasteurizador}{ô}{}{}{}{adj.}{Diz"-se de equipamento empregado em pasteurização.}{pas.teu.ri.za.dor}{0}
\verb{pasteurizar}{}{}{}{}{v.t.}{Esterilizar o leite e outros alimentos por aquecimento e resfriamento súbito.}{pas.teu.ri.zar}{\verboinum{1}}
\verb{pastiche}{}{}{}{}{s.m.}{Obra de arte que imita outra.}{pas.ti.che}{0}
\verb{pasticho}{}{}{}{}{s.m.}{Pastiche.}{pas.ti.cho}{0}
\verb{pastifício}{}{}{}{}{s.m.}{Indústria de massas alimentícias.}{pas.ti.fí.cio}{0}
\verb{pastilha}{}{}{}{}{s.f.}{Tipo de doce; bala.}{pas.ti.lha}{0}
\verb{pastilha}{}{}{}{}{}{Remédio em tablete.}{pas.ti.lha}{0}
\verb{pastilha}{}{}{}{}{}{Pequeno ladrilho para cobrir pisos ou paredes.}{pas.ti.lha}{0}
\verb{pasto}{}{}{}{}{s.m.}{Local para pastar; pastagem.}{pas.to}{0}
\verb{pasto}{}{}{}{}{}{Erva que cresce nesse local.}{pas.to}{0}
\verb{pastor}{ô}{}{}{}{s.m.}{Trabalhador que cuida de rebanho.}{pas.tor}{0}
\verb{pastor}{ô}{}{}{}{}{Indivíduo encarregado do culto em Igreja Protestante.}{pas.tor}{0}
\verb{pastoral}{}{}{"-ais}{}{adj.2g.}{Relativo a pastor espiritual, guia.}{pas.to.ral}{0}
\verb{pastoral}{}{}{"-ais}{}{}{Relativo a campo; pastoril, bucólico.}{pas.to.ral}{0}
\verb{pastoral}{}{Liter.}{"-ais}{}{s.f.}{Composição poética de temática amorosa e ambientação campestre e pastoril.}{pas.to.ral}{0}
\verb{pastoral}{}{}{"-ais}{}{}{Trabalho religioso dos pastores das igrejas cristãs.}{pas.to.ral}{0}
\verb{pastorar}{}{}{}{}{v.i.}{Guiar o rebanho ao pasto.}{pas.to.rar}{0}
\verb{pastorar}{}{}{}{}{}{Ser guia espiritual de.}{pas.to.rar}{\verboinum{1}}
\verb{pastorear}{}{}{}{}{}{Var. de \textit{pastorar}.}{pas.to.re.ar}{0}
\verb{pastoreio}{ê}{}{}{}{s.m.}{Atividade de pastor.}{pas.to.rei.o}{0}
\verb{pastoreio}{ê}{}{}{}{}{Local de pastagem.}{pas.to.rei.o}{0}
\verb{pastoril}{}{}{"-is}{}{adj.2g.}{Próprio de pastor.}{pas.to.ril}{0}
\verb{pastoso}{ô}{}{"-osos ⟨ó⟩}{"-osa ⟨ó⟩}{adj.}{Que se apresenta em consistência de pasta; nem líquido, nem sólido.}{pas.to.so}{0}
\verb{pata}{}{}{}{}{s.f.}{A fêmea do pato.}{pa.ta}{0}
\verb{pata}{}{}{}{}{}{Pé de animal.}{pa.ta}{0}
\verb{pataca}{}{}{}{}{s.f.}{Antiga moeda de prata.}{pa.ta.ca}{0}
\verb{patacão}{}{}{"-ões}{}{s.m.}{Nome dado a diversas moedas antigas do Brasil, de Portugal, Espanha e alguns países sul"-americanos.}{pa.ta.cão}{0}
\verb{pata"-choca}{ó}{Pop.}{patas"-chocas ⟨ó⟩}{}{s.m.}{Servente de sacristia.}{pa.ta"-cho.ca}{0}
\verb{pata"-choca}{ó}{}{patas"-chocas ⟨ó⟩}{}{}{Carro pesado.}{pa.ta"-cho.ca}{0}
\verb{pata"-choca}{ó}{}{patas"-chocas ⟨ó⟩}{}{s.f.}{Mulher de andar vagaroso.}{pa.ta"-cho.ca}{0}
\verb{patacoada}{}{}{}{}{s.f.}{Coisa que não se leva a sério; disparate.}{pa.ta.co.a.da}{0}
\verb{patacoada}{}{}{}{}{}{Brincadeira, gracejo.}{pa.ta.co.a.da}{0}
\verb{patacoada}{}{}{}{}{}{Mentira, lorota.}{pa.ta.co.a.da}{0}
\verb{patada}{}{}{}{}{s.f.}{Golpe com a pata.}{pa.ta.da}{0}
\verb{patada}{}{Pop.}{}{}{}{Grosseria.}{pa.ta.da}{0}
\verb{patamar}{}{}{}{}{s.m.}{Parte larga no topo ou entre dois lances de escada.}{pa.ta.mar}{0}
\verb{patativa}{}{Zool.}{}{}{s.f.}{Ave de coloração geral cinzenta, asas e caudas pretas, com um canto melodioso.}{pa.ta.ti.va}{0}
\verb{patavina}{}{Pop.}{}{}{s.f.}{Coisa nenhuma; nada.}{pa.ta.vi.na}{0}
\verb{patê}{}{}{}{}{s.m.}{Preparação culinária de consistência pastosa, bem temperada, que, em geral, se come fria.}{pa.tê}{0}
\verb{patear}{}{}{}{}{v.t.}{Reprovar, manifestar desagrado batendo com os pés no chão.}{pa.te.ar}{0}
\verb{patear}{}{}{}{}{v.i.}{Bater com as patas.}{pa.te.ar}{\verboinum{4}}
\verb{patela}{é}{Anat.}{}{}{s.f.}{Osso achatado e arredondado que fica na parte da frente do joelho; rótula.}{pa.te.la}{0}
\verb{patena}{}{}{}{}{s.f.}{Pires metálico que, na missa, cobre o cálice e guarda a hóstia.}{pa.te.na}{0}
\verb{pátena}{}{}{}{}{}{Var. de \textit{patena}.}{pá.te.na}{0}
\verb{patente}{}{}{}{}{s.f.}{Título oficial de uma concessão ou privilégio.}{pa.ten.te}{0}
\verb{patente}{}{}{}{}{}{Posto militar.}{pa.ten.te}{0}
\verb{patente}{}{}{}{}{adj.2g.}{Notado com facilidade; claro, evidente.}{pa.ten.te}{0}
\verb{patentear}{}{}{}{}{v.t.}{Fazer alguma coisa oculta ficar conhecida; evidenciar, exibir.}{pa.ten.te.ar}{0}
\verb{patentear}{}{}{}{}{}{Registrar os direitos de autor de uma invenção.}{pa.ten.te.ar}{\verboinum{4}}
\verb{paternal}{}{}{"-ais}{}{adj.2g.}{Relativo a pai.}{pa.ter.nal}{0}
\verb{paternal}{}{}{"-ais}{}{}{Benevolente, protetor.}{pa.ter.nal}{0}
\verb{paternalismo}{}{}{}{}{s.m.}{Regime fundado na autoridade paterna.}{pa.ter.na.lis.mo}{0}
\verb{paternalismo}{}{}{}{}{}{Prática protetora em relações de trabalho, política etc.}{pa.ter.na.lis.mo}{0}
\verb{paternalista}{}{}{}{}{adj.2g.}{Relativo a paternalismo.}{pa.ter.na.lis.ta}{0}
\verb{paternalista}{}{}{}{}{}{Diz"-se de partidário do paternalismo.}{pa.ter.na.lis.ta}{0}
\verb{paternidade}{}{}{}{}{s.f.}{Condição de pai.}{pa.ter.ni.da.de}{0}
\verb{paterno}{é}{}{}{}{adj.}{Relativo a pai.}{pa.ter.no}{0}
\verb{pateta}{é}{}{}{}{adj.2g.}{Diz"-se de quem é distraído.}{pa.te.ta}{0}
\verb{pateta}{é}{}{}{}{}{Idiota, tolo.}{pa.te.ta}{0}
\verb{patetice}{}{}{}{}{s.f.}{Qualidade de pateta; idiotice, palermice.}{pa.te.ti.ce}{0}
\verb{patético}{}{}{}{}{adj.}{Que comove muito; tocante.}{pa.té.ti.co}{0}
\verb{patibular}{}{}{}{}{adj.2g.}{Relativo a patíbulo.}{pa.ti.bu.lar}{0}
\verb{patibular}{}{}{}{}{}{Que tem o aspecto sombrio ou de criminoso.}{pa.ti.bu.lar}{0}
\verb{patíbulo}{}{}{}{}{s.m.}{Local para execução pública de condenados.}{pa.tí.bu.lo}{0}
\verb{patifaria}{}{}{}{}{s.f.}{Atitude de patife.}{pa.ti.fa.ri.a}{0}
\verb{patife}{}{}{}{}{adj.}{Diz"-se de indivíduo de comportamento desonesto; trapaceiro, canalha.}{pa.ti.fe}{0}
\verb{patim}{}{}{"-ins}{}{s.m.}{Calçado com roda ou lâmina para deslizar no solo ou no gelo.}{pa.tim}{0}
\verb{pátina}{}{}{}{}{s.f.}{Camada esverdeada que se forma no cobre ou no bronze, pela ação do tempo.}{pá.ti.na}{0}
\verb{pátina}{}{}{}{}{}{Técnica de pintura que se aplica a uma superfície, para diversos efeitos.}{pá.ti.na}{0}
\verb{patinação}{}{}{"-ões}{}{s.f.}{Ato ou efeito de patinar.}{pa.ti.na.ção}{0}
\verb{patinação}{}{}{"-ões}{}{}{Modalidade esportiva em que se usam patins.}{pa.ti.na.ção}{0}
\verb{patinação}{}{}{"-ões}{}{}{Local onde se patina.}{pa.ti.na.ção}{0}
\verb{patinador}{ô}{}{}{}{adj.}{Diz"-se de indivíduo que pratica a patinação.}{pa.ti.na.dor}{0}
\verb{patinar}{}{}{}{}{}{Deslizar sem controle; patinhar.}{pa.ti.nar}{\verboinum{1}}
\verb{patinar}{}{}{}{}{v.i.}{Deslocar"-se sobre patins.}{pa.ti.nar}{0}
\verb{patinete}{é}{}{}{}{s.m.}{Brinquedo feito de tábua sobre duas rodas com guidão.}{pa.ti.ne.te}{0}
\verb{patinhar}{}{}{}{}{v.i.}{Agitar"-se na água como os patos.}{pa.ti.nhar}{0}
\verb{patinhar}{}{}{}{}{}{Deslizar sem controle.}{pa.ti.nhar}{\verboinum{1}}
\verb{patinho}{}{}{}{}{s.m.}{Pato pequeno.}{pa.ti.nho}{0}
\verb{patinho}{}{}{}{}{}{Carne do alto das pernas traseiras do boi.}{pa.ti.nho}{0}
\verb{pátio}{}{}{}{}{s.m.}{Recinto cercado e descoberto anexo a uma construção.}{pá.tio}{0}
\verb{pato}{}{Zool.}{}{}{s.m.}{Ave aquática de pernas e pescoço curtos e dedos ligados por uma membrana, que facilita a natação.}{pa.to}{0}
\verb{pato}{}{Pop.}{}{}{}{Indivíduo tolo, otário.}{pa.to}{0}
\verb{patogenia}{}{Med.}{}{}{s.f.}{Ramo da patologia que trata da origem e evolução das doenças.}{pa.to.ge.ni.a}{0}
\verb{patogênico}{}{Med.}{}{}{adj.}{Relativo a patogenia ou a patogênese.}{pa.to.gê.ni.co}{0}
\verb{patogênico}{}{}{}{}{}{Que pode provocar doenças.}{pa.to.gê.ni.co}{0}
\verb{patola}{ó}{}{}{}{s.f.}{A pata dos caranguejos, siris etc.; tenaz, pinça.}{pa.to.la}{0}
\verb{patologia}{}{Med.}{}{}{s.f.}{Parte da medicina que estuda as causas das doenças e as alterações provocadas por elas no organismo.}{pa.to.lo.gi.a}{0}
\verb{patológico}{}{}{}{}{adj.}{Relativo a patologia.  }{pa.to.ló.gi.co}{0}
\verb{patologista}{}{Med.}{}{}{s.2g.}{Pessoa especialista em patologia.  }{pa.to.lo.gis.ta}{0}
\verb{patota}{ó}{}{}{}{s.f.}{Grupo de amigos; turma, bando.}{pa.to.ta}{0}
\verb{patranha}{}{}{}{}{s.f.}{Grande mentira; patranhada.}{pa.tra.nha}{0}
\verb{patranhada}{}{}{}{}{s.f.}{Conjunto de patranhas, de mentiras.}{pa.tra.nha.da}{0}
\verb{patranheiro}{ê}{}{}{}{adj.}{Que inventa patranhas, grandes mentiras.}{pa.tra.nhei.ro}{0}
\verb{patranheiro}{ê}{}{}{}{s.m.}{Pessoa dada a inventar patranhas.}{pa.tra.nhei.ro}{0}
\verb{patrão}{}{}{"-ões}{}{s.m.}{Chefe ou dono de um estabelecimento comercial ou industrial em relação a seus funcionários; empregador.}{pa.trão}{0}
\verb{pátria}{}{}{}{}{s.f.}{País onde se nasceu; terra natal.}{pá.tria}{0}
\verb{patriarca}{}{}{}{}{s.m.}{O chefe da família, entre os povos antigos. }{pa.tri.ar.ca}{0}
\verb{patriarca}{}{}{}{}{}{O homem mais velho de uma família, respeitado e obedecido pelos demais membros.}{pa.tri.ar.ca}{0}
\verb{patriarcado}{}{}{}{}{s.m.}{Dignidade ou jurisdição de patriarca.}{pa.tri.ar.ca.do}{0}
\verb{patriarcado}{}{}{}{}{}{Forma de organização social na qual o pai é a autoridade máxima.}{pa.tri.ar.ca.do}{0}
\verb{patriarcal}{}{}{"-ais}{}{adj.2g.}{Relativo a patriarca ou a patriarcado.}{pa.tri.ar.cal}{0}
\verb{patriciado}{}{}{}{}{s.m.}{Entre os antigos romanos, estado ou condição de patrício.}{pa.tri.ci.a.do}{0}
\verb{patriciado}{}{}{}{}{}{A classe dos nobres.}{pa.tri.ci.a.do}{0}
\verb{patrício}{}{}{}{}{s.m.}{Na antiga Roma, indivíduo que pertencia às classes nobres.}{pa.trí.cio}{0}
\verb{patrício}{}{}{}{}{}{Conterrâneo, compatriota.}{pa.trí.cio}{0}
\verb{patrimonial}{}{}{"-ais}{}{adj.2g.}{Relativo a patrimônio.}{pa.tri.mo.ni.al}{0}
\verb{patrimônio}{}{}{}{}{s.m.}{Conjunto dos bens de um indivíduo, uma família ou uma instituição.}{pa.tri.mô.nio}{0}
\verb{pátrio}{}{}{}{}{adj.}{Relativo a pátria.}{pá.trio}{0}
\verb{pátrio}{}{}{}{}{}{Relativo aos pais; paterno.}{pá.trio}{0}
\verb{patriota}{ó}{}{}{}{s.2g.}{Pessoa que ama e serve a sua pátria.}{pa.tri.o.ta}{0}
\verb{patriotada}{}{}{}{}{s.f.}{Alarde ridículo ou exaltado de patriotismo.}{pa.tri.o.ta.da}{0}
\verb{patrioteiro}{ê}{}{}{}{s.m.}{Pessoa dada a patriotadas.}{pa.tri.o.tei.ro}{0}
\verb{patriotice}{}{}{}{}{s.f.}{Qualidade de quem é patrioteiro.}{pa.tri.o.ti.ce}{0}
\verb{patriotice}{}{}{}{}{}{Patriotismo falso.}{pa.tri.o.ti.ce}{0}
\verb{patriótico}{}{}{}{}{adj.}{Que se refere a pátria, patriotismo ou patriota.}{pa.tri.ó.ti.co}{0}
\verb{patriotismo}{}{}{}{}{s.m.}{Qualidade de quem é patriota.}{pa.tri.o.tis.mo}{0}
\verb{patroa}{}{}{}{}{s.f.}{A mulher do patrão.}{pa.tro.a}{0}
\verb{patroa}{}{}{}{}{}{Dona"-de"-casa.}{pa.tro.a}{0}
\verb{patroa}{}{}{}{}{}{Dona de um estabelecimento comercial.}{pa.tro.a}{0}
\verb{patroa}{}{Pop.}{}{}{}{Esposa.}{pa.tro.a}{0}
\verb{patrocinador}{ô}{}{}{}{adj.}{Que patrocina.}{pa.tro.ci.na.dor}{0}
\verb{patrocinador}{ô}{}{}{}{s.m.}{Pessoa ou empresa que patrocina.}{pa.tro.ci.na.dor}{0}
\verb{patrocinar}{}{}{}{}{v.t.}{Custear as despesas de um indivíduo ou de uma atividade.}{pa.tro.ci.nar}{\verboinum{1}}
\verb{patrocínio}{}{}{}{}{s.m.}{Ato ou efeito de patrocinar; apoio financeiro; custeio.}{pa.tro.cí.nio}{0}
\verb{patrona}{}{}{}{}{s.f.}{Protetora, padroeira.}{pa.tro.na}{0}
\verb{patronal}{}{}{"-ais}{}{adj.2g.}{Que se refere a ou é próprio de patrão.}{pa.tro.nal}{0}
\verb{patronato}{}{}{}{}{s.m.}{Autoridade ou direito de patrão.}{pa.tro.na.to}{0}
\verb{patronato}{}{}{}{}{}{Patrocínio.}{pa.tro.na.to}{0}
\verb{patronesse}{}{}{}{}{s.f.}{Mulher de sociedade que se dedica a filantropia.}{\textit{patronesse}}{0}
\verb{patronímico}{}{}{}{}{adj.}{Relativo ao pai, especialmente no que se refere ao nome do pai ou a nomes de família.}{pa.tro.ní.mi.co}{0}
\verb{patronímico}{}{}{}{}{s.m.}{Sobrenome que deriva do nome do pai.}{pa.tro.ní.mi.co}{0}
\verb{patrono}{}{}{}{}{s.m.}{Indivíduo escolhido para defender uma causa; defensor, protetor.}{pa.tro.no}{0}
\verb{patrono}{}{}{}{}{}{Indivíduo escolhido por uma turma de formandos para receber homenagens na colação de grau.}{pa.tro.no}{0}
\verb{patrulha}{}{}{}{}{s.f.}{Pequeno destacamento de soldados em missão de ronda ou reconhecimento.}{pa.tru.lha}{0}
\verb{patrulhamento}{}{}{}{}{s.m.}{Ato ou efeito de patrulhar.}{pa.tru.lha.men.to}{0}
\verb{patrulhar}{}{}{}{}{v.t.}{Vigiar uma região para protegê"-la.}{pa.tru.lhar}{\verboinum{1}}
\verb{patrulheiro}{ê}{}{}{}{s.m.}{Pessoa que faz patrulha.}{pa.tru.lhei.ro}{0}
\verb{patuá}{}{}{}{}{s.m.}{Amuleto que se leva pendurado ao pescoço; bentinho.}{pa.tu.á}{0}
\verb{patudo}{}{}{}{}{adj.}{Que tem os pés ou as patas grandes.}{pa.tu.do}{0}
\verb{patuleia}{é}{}{}{}{s.f.}{Ralé, plebe, povo.}{pa.tu.lei.a}{0}
\verb{patuscada}{}{}{}{}{s.f.}{Reunião festiva na qual as pessoas comem e bebem; comezaina.}{pa.tus.ca.da}{0}
\verb{patuscar}{}{}{}{}{v.i.}{Fazer patuscadas; pandegar, farrear. }{pa.tus.car}{\verboinum{2}}
\verb{patusco}{}{}{}{}{adj.}{Que é dado a patuscadas.}{pa.tus.co}{0}
\verb{patusco}{}{}{}{}{s.m.}{Pessoa patusca.}{pa.tus.co}{0}
\verb{pau}{}{}{}{}{s.m.}{Qualquer pedaço de madeira; bastão, cacete.}{pau}{0}
\verb{pau}{}{Informát.}{}{}{}{Pane no computador.}{pau}{0}
\verb{pau}{}{Pop.}{}{}{}{Reprovação na escola; bomba.}{pau}{0}
\verb{pau"-a"-pique}{}{}{paus"-a"-pique}{}{s.m.}{Parede construída com uma trama de ripas ou varas fincadas no chão, que se cobre com barro.}{pau"-a"-pi.que}{0}
\verb{pau"-brasil}{}{Bot.}{}{}{s.m.}{Árvore de madeira avermelhada da qual se extrai uma tinta corante.}{pau"-bra.sil}{0}
\verb{pau"-d'água}{}{Pop.}{paus"-d'água}{}{s.m.}{Bêbado contumaz; ébrio.}{pau"-d'á.gua}{0}
\verb{pau"-d'arco}{}{Bot.}{paus"-d'arco}{}{s.m.}{Árvore ornamental de madeira resistente que fornece belas flores; ipê.}{pau"-d'ar.co}{0}
\verb{pau"-de"-arara}{}{}{paus"-de"-arara}{}{s.m.}{Instrumento de tortura em que se prende o indivíduo num pedaço de madeira, pendurando"-o de cabeça para baixo.}{pau"-de"-a.ra.ra}{0}
\verb{pau"-de"-arara}{}{}{paus"-de"-arara}{}{}{Caminhão que transporta retirantes nordestinos.}{pau"-de"-a.ra.ra}{0}
\verb{pau"-de"-arara}{}{Por ext.}{paus"-de"-arara}{}{}{Retirante que viaja nesses caminhões.}{pau"-de"-a.ra.ra}{0}
\verb{pau"-de"-sebo}{ê}{}{paus"-de"-sebo ⟨ê⟩}{}{s.m.}{Mastro untado com sebo, no topo do qual se colocam prêmios para aqueles que conseguirem alcançá"-los.}{pau"-de"-se.bo}{0}
\verb{pau"-ferro}{é}{Bot.}{paus"-ferros \textit{ou} paus"-ferro ⟨é⟩}{}{s.m.}{Árvore de tronco liso, madeira resistente e muito dura e flores amarelas e vistosas; jucá.  }{pau"-fer.ro}{0}
\verb{paul}{}{}{pauis ⟨ú⟩}{}{s.m.}{Pântano.}{pa.ul}{0}
\verb{paulada}{}{}{}{}{s.f.}{Golpe com pau; cacetada.}{pau.la.da}{0}
\verb{paulatino}{}{}{}{}{adj.}{Que é feito aos poucos.}{pau.la.ti.no}{0}
\verb{pauliceia}{é}{}{}{}{s.f.}{A cidade de São Paulo, capital do estado homônimo. }{pau.li.cei.a}{0}
\verb{paulificar}{}{}{}{}{v.t.}{Causar chateação, aborrecimento; amolar, importunar, maçar, cacetear. }{pau.li.fi.car}{\verboinum{2}}
\verb{paulista}{}{}{}{}{adj.2g.}{Relativo a São Paulo.}{pau.lis.ta}{0}
\verb{paulista}{}{}{}{}{s.2g.}{Indivíduo natural ou habitante desse estado.}{pau.lis.ta}{0}
\verb{paulistano}{}{}{}{}{adj.}{Relativo à cidade de São Paulo, capital do estado de São Paulo.}{pau.lis.ta.no}{0}
\verb{paulistano}{}{}{}{}{s.m.}{Indivíduo natural ou habitante dessa cidade.}{pau.lis.ta.no}{0}
\verb{pau"-mandado}{}{}{paus"-mandados}{}{s.m.}{Pessoa servil que faz qualquer coisa sem objeções nem resistência.}{pau"-man.da.do}{0}
\verb{pau"-marfim}{}{Bot.}{}{}{s.m.}{Árvore de madeira dura e clara, usada na fabricação de móveis, tacos, revestimentos internos etc.}{pau"-mar.fim}{0}
\verb{pau"-para"-toda"-obra}{ô\ldots{}ó}{Bot.}{}{}{s.m.}{Árvore nativa do Brasil, cuja madeira, de boa qualidade, é empregada para inúmeros fins.}{pau"-pa.ra"-to.da"-o.bra}{0}
\verb{pauperismo}{}{}{}{}{s.m.}{Estado de pobreza; miséria, penúria.}{pau.pe.ris.mo}{0}
\verb{paupérrimo}{}{}{}{}{adj.}{Superlativo absoluto sintético de \textit{pobre}; muito pobre.}{pau.pér.ri.mo}{0}
\verb{paus}{}{}{}{}{s.m.}{Um dos quatro naipes do baralho, representado por um trevo preto de três pontas.}{paus}{0}
\verb{pausa}{}{}{}{}{s.f.}{Interrupção temporária de uma ação.}{pau.sa}{0}
\verb{pausado}{}{}{}{}{adj.}{Que é feito com pausas; vagaroso, lento, paulatino.}{pau.sa.do}{0}
\verb{pausado}{}{}{}{}{adv.}{De modo pausado, lento; pausadamente, paulatinamente.}{pau.sa.do}{0}
\verb{pausar}{}{}{}{}{v.i.}{Fazer pausa.}{pau.sar}{0}
\verb{pausar}{}{}{}{}{v.t.}{Fazer algo de forma pausada, com lentidão, vagareza.}{pau.sar}{\verboinum{1}}
\verb{pauta}{}{}{}{}{s.f.}{Papel riscado com traços horizontais para se escrever alinhadamente.}{pau.ta}{0}
\verb{pauta}{}{}{}{}{}{Conjunto de cinco linhas paralelas onde se escrevem as notas musicais; pentagrama.}{pau.ta}{0}
\verb{pauta}{}{}{}{}{}{Relação de assuntos a serem tratados em uma reunião.}{pau.ta}{0}
\verb{pautado}{}{}{}{}{adj.}{Que está riscado com traços paralelos (papel).}{pau.ta.do}{0}
\verb{pautado}{}{}{}{}{}{Que foi relacionado; arrolado, inventariado.}{pau.ta.do}{0}
\verb{pautar}{}{}{}{}{v.t.}{Traçar linhas no papel à maneira de pauta.}{pau.tar}{0}
\verb{pautar}{}{}{}{}{}{Colocar na lista de assuntos de uma reunião.}{pau.tar}{0}
\verb{pautar}{}{}{}{}{}{Regular, orientar, regrar. (\textit{Ele sempre pautou sua conduta por princípios religiosos.})}{pau.tar}{\verboinum{1}}
\verb{pavana}{}{}{}{}{s.f.}{Dança renascentista, de origem italiana,  em compasso binário ou quaternário e andamento lento.}{pa.va.na}{0}
\verb{pavão}{}{Zool.}{"-ões}{}{s.m.}{Grande ave galiforme, de plumagem exuberante, principalmente na cauda, que se abre num leque colorido.}{pa.vão}{0}
\verb{pavê}{}{Cul.}{}{}{s.m.}{Doce feito com bolachas embebidas em licor ou suco entremeadas com camadas de chocolate, creme etc.  }{pa.vê}{0}
\verb{pávido}{}{}{}{}{adj.}{Que tem medo, pavor; assombrado, apavorado, medroso, assustado.}{pá.vi.do}{0}
\verb{pavilhão}{}{}{"-ões}{}{s.m.}{Parte de um conjunto de edifícios construída como anexo ao corpo principal.}{pa.vi.lhão}{0}
\verb{pavilhão}{}{}{"-ões}{}{}{Construção leve de madeira ou lona de caráter temporário; barraca, tenda.}{pa.vi.lhão}{0}
\verb{pavilhão}{}{}{"-ões}{}{}{Parte externa, mais larga e aberta, de certos tubos. }{pa.vi.lhão}{0}
\verb{pavilhão}{}{}{"-ões}{}{}{Estandarte, bandeira.}{pa.vi.lhão}{0}
\verb{pavimentação}{}{}{"-ões}{}{s.f.}{Ato ou efeito de pavimentar.}{pa.vi.men.ta.ção}{0}
\verb{pavimentar}{}{}{}{}{v.t.}{Fazer o pavimento.}{pa.vi.men.tar}{\verboinum{1}}
\verb{pavimento}{}{}{}{}{s.m.}{Cada um dos andares de um edifício.}{pa.vi.men.to}{0}
\verb{pavimento}{}{}{}{}{}{Revestimento do chão sobre o qual se pisa.}{pa.vi.men.to}{0}
\verb{pavio}{}{}{}{}{s.m.}{Torcida de fio próprio para se acender e queimar lentamente em vela ou lampião; mecha.}{pa.vi.o}{0}
\verb{pavonear}{}{}{}{}{v.t.}{Exibir com ostentação e vaidade.}{pa.vo.ne.ar}{\verboinum{4}}
\verb{pavor}{ô}{}{}{}{s.m.}{Medo muito grande; terror, pânico.}{pa.vor}{0}
\verb{pavoroso}{ô}{}{"-osos ⟨ó⟩}{"-osa ⟨ó⟩}{adj.}{Que inspira pavor; medonho, horroroso.}{pa.vo.ro.so}{0}
\verb{pavuna}{}{}{}{}{s.f.}{Vale escuro e íngreme.}{pa.vu.na}{0}
\verb{paxá}{ch}{}{}{}{s.m.}{Título dado aos governantes de províncias na Turquia, durante o Império Otomano.}{pa.xá}{0}
\verb{paxá}{ch}{Fig.}{}{}{}{Indivíduo indolente que leva uma vida faustosa.}{pa.xá}{0}
\verb{paz}{}{}{}{}{s.f.}{Ausência total de conflitos, perturbações, hostilidades.}{paz}{0}
\verb{paz}{}{}{}{}{}{Tranquilidade, sossego, calma, quietude.}{paz}{0}
\verb{Pb}{}{Quím.}{}{}{}{Símb. do \textit{chumbo}.}{Pb}{0}
\verb{PB}{}{}{}{}{}{Sigla do estado da Paraíba.}{PB}{0}
\verb{PC}{}{Informát.}{}{}{s.m.}{Sigla de \textit{personal computer} (computador pessoal), microcomputador com \textsc{cpu}, monitor, teclados e outros periféricos, geralmente para uso doméstico.}{PC}{0}
\verb{pc}{}{Astron.}{}{}{}{Símbolo de \textit{parsec}.}{pc}{0}
\verb{Pd}{}{Quím.}{}{}{}{Símb. do \textit{paládio}.}{Pd}{0}
\verb{PE}{}{}{}{}{}{Sigla do estado de Pernambuco.}{PE}{0}
\verb{pê}{}{}{}{}{s.m.}{Nome da letra \textit{p}.}{pê}{0}
\verb{pé}{}{}{}{}{s.m.}{Parte do membro inferior, abaixo do tornozelo, que toca o chão, servindo de apoio e como meio de locomoção.}{pé}{0}
\verb{pé}{}{}{}{}{}{Parte inferior de um objeto, que serve para sustentá"-lo.}{pé}{0}
\verb{pé}{}{}{}{}{}{Cada exemplar de uma planta.}{pé}{0}
\verb{pé}{}{}{}{}{}{Unidade de medida de comprimento equivalente a doze polegadas ou 30,48 cm.}{pé}{0}
\verb{peanha}{}{}{}{}{s.f.}{Pedestal pequeno para sustentar cruz, imagem, estandarte.}{pe.a.nha}{0}
\verb{peão}{}{}{"-ões}{}{s.m.}{Indivíduo que trabalha no campo.}{pe.ão}{0}
\verb{peão}{}{}{"-ões}{}{}{Amansador de cavalos.}{pe.ão}{0}
\verb{peão}{}{}{"-ões}{}{}{Peça do jogo de xadrez.}{pe.ão}{0}
\verb{pear}{}{}{}{}{v.t.}{Prender com cordas as pernas dos animais para impedi"-los de andar.}{pe.ar}{\verboinum{4}}
\verb{pebolim}{}{}{}{}{s.m.}{Jogo de futebol de mesa em que os bonecos dos jogadores ficam presos a hastes rotativas movimentadas com as mãos.}{pe.bo.lim}{0}
\verb{peça}{é}{}{}{}{s.f.}{Cada uma das partes de um mecanismo.}{pe.ça}{0}
\verb{peça}{é}{}{}{}{}{Cada um dos elementos de um conjunto.}{pe.ça}{0}
\verb{peça}{é}{}{}{}{}{Texto escrito para ser representado em teatro.}{pe.ça}{0}
\verb{peça}{é}{Jur.}{}{}{}{Documento de um processo.}{pe.ça}{0}
\verb{pecadilho}{}{}{}{}{s.m.}{Pecado leve, sem gravidade.}{pe.ca.di.lho}{0}
\verb{pecado}{}{}{}{}{s.m.}{Ato que contraria um preceito religioso; falta.}{pe.ca.do}{0}
\verb{pecador}{ô}{}{}{}{adj.}{Que peca. }{pe.ca.dor}{0}
\verb{pecador}{ô}{}{}{}{}{Que tem certos defeitos ou vícios.}{pe.ca.dor}{0}
\verb{pecador}{ô}{}{}{}{s.m.}{Pessoa que peca ou que confessa seus pecados; penitente.}{pe.ca.dor}{0}
\verb{pecaminoso}{ô}{}{"-osos ⟨ó⟩}{"-osa ⟨ó⟩}{adj.}{Em que há pecado.}{pe.ca.mi.no.so}{0}
\verb{pecar}{}{}{}{}{v.t.}{Cometer pecado; faltar, errar.}{pe.car}{\verboinum{2}}
\verb{pecha}{é}{}{}{}{s.f.}{Defeito moral; imperfeição, vício.}{pe.cha}{0}
\verb{pechincha}{}{}{}{}{s.f.}{Coisa comprada a preço muito baixo.}{pe.chin.cha}{0}
\verb{pechincha}{}{}{}{}{}{Lucro inesperado; vantagem.}{pe.chin.cha}{0}
\verb{pechinchar}{}{}{}{}{v.i.}{Procurar comprar por preço menor; regatear.}{pe.chin.char}{\verboinum{1}}
\verb{pechincheiro}{ê}{}{}{}{adj.}{Que pechincha, que procura pechinchas.}{pe.chin.chei.ro}{0}
\verb{pechisbeque}{é}{}{}{}{s.m.}{Lâmina fina de metal que imita ouro; ouro falso, ouropel.}{pe.chis.be.que}{0}
\verb{pecíolo}{}{Bot.}{}{}{s.m.}{Haste, geralmente cilíndrica, que sustenta o limbo das folhas.}{pe.cí.o.lo}{0}
\verb{peco}{ê}{}{}{}{adj.}{Que não se desenvolveu, que definhou; mirrado.}{pe.co}{0}
\verb{peçonha}{}{}{}{}{s.f.}{Substância venenosa secretada por alguns animais; veneno.}{pe.ço.nha}{0}
\verb{peçonha}{}{Fig.}{}{}{}{Malícia, intriga, maldade.}{pe.ço.nha}{0}
\verb{peçonhento}{}{}{}{}{adj.}{Que encerra peçonha; venenoso.}{pe.ço.nhen.to}{0}
\verb{peçonhento}{}{Fig.}{}{}{}{Que é pérfido; intrigante, malicioso.}{pe.ço.nhen.to}{0}
\verb{pecuária}{}{}{}{}{s.f.}{Técnica e indústria da criação e tratamento do gado.}{pe.cu.á.ria}{0}
\verb{pecuário}{}{}{}{}{adj.}{Relativo à pecuária.}{pe.cu.á.rio}{0}
\verb{pecuário}{}{}{}{}{s.m.}{Criador de gado; pecuarista.}{pe.cu.á.rio}{0}
\verb{pecuarista}{}{}{}{}{adj.2g.}{Relativo a pecuária.}{pe.cu.a.ris.ta}{0}
\verb{pecuarista}{}{}{}{}{s.2g.}{Indivíduo que se dedica à pecuária.}{pe.cu.a.ris.ta}{0}
\verb{peculatário}{}{}{}{}{adj.}{Que comete peculato.}{pe.cu.la.tá.rio}{0}
\verb{peculato}{}{}{}{}{s.m.}{Desvio de dinheiro praticado por funcionário público.}{pe.cu.la.to}{0}
\verb{peculiar}{}{}{}{}{adj.2g.}{Que é próprio de alguém; característico, privativo.}{pe.cu.li.ar}{0}
\verb{peculiaridade}{}{}{}{}{s.f.}{Qualidade peculiar; característica.}{pe.cu.li.a.ri.da.de}{0}
\verb{pecúlio}{}{}{}{}{s.m.}{Dinheiro que se junta e se guarda como reserva; economias.}{pe.cú.lio}{0}
\verb{pecúnia}{}{}{}{}{s.f.}{Meio de troca usado na compra de bens ou serviços; dinheiro.}{pe.cú.nia}{0}
\verb{pecuniário}{}{}{}{}{adj.}{Relativo a dinheiro.}{pe.cu.ni.á.rio}{0}
\verb{pedaço}{}{}{}{}{s.m.}{Quantidade separada de uma substância sólida ou de um todo; fração, fatia.}{pe.da.ço}{0}
\verb{pedaço}{}{}{}{}{}{Trecho, passagem. (\textit{Eu li somente um pedaço do texto, pois ele era extremamente longo.})}{pe.da.ço}{0}
\verb{pedágio}{}{}{}{}{s.m.}{Taxa que se cobra aos usuários de estradas de rodagem.}{pe.dá.gio}{0}
\verb{pedágio}{}{}{}{}{}{Posto fiscal onde se cobra essa taxa.}{pe.dá.gio}{0}
\verb{pedagogia}{}{}{}{}{s.f.}{Ciência que trata da educação e da instrução de jovens e crianças.}{pe.da.go.gi.a}{0}
\verb{pedagógico}{}{}{}{}{adj.}{Relativo ou conforme a pedagogia.}{pe.da.gó.gi.co}{0}
\verb{pedagogo}{ô}{}{}{}{s.m.}{Indivíduo que se dedica à pedagogia; educador, mestre.}{pe.da.go.go}{0}
\verb{pé"-d'água}{}{}{pés"-d'água}{}{s.m.}{Chuva forte e repentina; aguaceiro.}{pé"-d'á.gua}{0}
\verb{pedal}{}{}{"-ais}{}{s.m.}{Peça de comando de máquinas, veículos e instrumentos musicais, acionada com o pé.}{pe.dal}{0}
\verb{pedalada}{}{}{}{}{s.f.}{Cada impulso dado ao pedal.}{pe.da.la.da}{0}
\verb{pedalar}{}{}{}{}{v.t.}{Acional o pedal.}{pe.da.lar}{0}
\verb{pedalar}{}{}{}{}{v.i.}{Andar de bicicleta.}{pe.da.lar}{\verboinum{1}}
\verb{pedalinho}{}{}{}{}{s.m.}{Pequeno barco movido a pedais usado em lagoas ou represas e próprio para o lazer. }{pe.da.li.nho}{0}
\verb{pedante}{}{}{}{}{adj.2g.}{Que ostenta conhecimentos que não necessariamente possui; afetado, presunçoso.}{pe.dan.te}{0}
\verb{pedantismo}{}{}{}{}{s.m.}{Qualidade de pedante; presunção, afetação.}{pe.dan.tis.mo}{0}
\verb{pé"-de"-atleta}{é}{Med.}{pés"-de"-atleta ⟨é⟩}{}{s.m.}{Micose superficial crônica da pele dos pés, devido a fungos; frieira.}{pé"-de"-a.tle.ta}{0}
\verb{pé"-de"-boi}{}{Pop.}{pés"-de"-boi}{}{s.2g.}{Pessoa muito trabalhadora, esforçada, assídua, cumpridora de suas obrigações.}{pé"-de"-boi}{0}
\verb{pé"-de"-cabra}{}{}{pés"-de"-cabra}{}{s.m.}{Alavanca de ferro, com uma das extremidades fendida e que serve para arrancar pregos, abrir caixotes, arrombar portas.}{pé"-de"-ca.bra}{0}
\verb{pé"-de"-chinelo}{é}{Pop.}{pés"-de"-chinelo ⟨é⟩}{}{s.2g.}{Indivíduo pobre, sem posses ou poder.}{pé"-de"-chi.ne.lo}{0}
\verb{pé"-de"-galinha}{}{}{pés"-de"-galinha}{}{s.m.}{Ruga no canto externo dos olhos.}{pé"-de"-ga.li.nha}{0}
\verb{pé"-de"-meia}{ê}{}{pés"-de"-meia}{}{s.f.}{Dinheiro economizado e guardado para objetivos futuros; pecúlio.}{pé"-de"-mei.a}{0}
\verb{pé"-de"-moleque}{é}{Cul.}{pés"-de"-moleque ⟨é⟩}{}{s.m.}{Doce preparado com amendoim torrado e açúcar queimado ou rapadura, e cortado em tabletes depois de frio.}{pé"-de"-mo.le.que}{0}
\verb{pé"-de"-ouvido}{}{Pop.}{pés"-de"-ouvido}{}{s.m.}{Tapa forte dado no pescoço, abaixo do ouvido.}{pé"-de"-ou.vi.do}{0}
\verb{pé"-de"-pato}{}{}{pés"-de"-pato}{}{s.m.}{Calçado de borracha, em forma de pé de pato, que os nadadores e mergulhadores colocam nos pés para dar maior velocidade dentro da água.  }{pé"-de"-pa.to}{0}
\verb{pederasta}{}{}{}{}{s.m.}{Indivíduo que pratica a pederastia; homossexual.}{pe.de.ras.ta}{0}
\verb{pederastia}{}{}{}{}{s.f.}{Prática sexual entre homens; homossexualismo masculino.}{pe.de.ras.ti.a}{0}
\verb{pederneira}{ê}{Geol.}{}{}{s.f.}{Pedra muito dura, composta de calcedônia e opala, de cor vermelha ou negra; sílex.}{pe.der.nei.ra}{0}
\verb{pedestal}{}{}{"-ais}{}{s.m.}{Peça sobre a qual se coloca uma estátua, coluna etc.}{pe.des.tal}{0}
\verb{pedestre}{é}{}{}{}{s.2g.}{Pessoa que anda a pé ou está a pé.}{pe.des.tre}{0}
\verb{pedestrianismo}{}{Esport.}{}{}{s.m.}{Competição que consiste em grandes marchas a pé.}{pe.des.tri.a.nis.mo}{0}
\verb{pé"-de"-vento}{}{}{pés"-de"-vento}{}{s.m.}{Sopro forte e curto de vento.}{pé"-de"-ven.to}{0}
\verb{pediatra}{}{}{}{}{s.2g.}{Médico especialista em pediatria.}{pe.di.a.tra}{0}
\verb{pediatria}{}{Med.}{}{}{s.f.}{Ramo da medicina que trata das doenças infantis.}{pe.di.a.tri.a}{0}
\verb{pedicure}{}{}{}{}{s.2g.}{Especialista que trata dos pés, retirando calos, lixando e pintando unhas; pedicuro, calista.}{pe.di.cu.re}{0}
\verb{pedicuro}{}{}{}{}{s.m.}{Pedicure.}{pe.di.cu.ro}{0}
\verb{pedido}{}{}{}{}{adj.}{Que se pediu; solicitado.}{pe.di.do}{0}
\verb{pedido}{}{}{}{}{}{Ato ou efeito de pedir; solicitação, petição.}{pe.di.do}{0}
\verb{pedido}{}{}{}{}{}{Ordem de compra; encomenda.}{pe.di.do}{0}
\verb{pedigree}{}{}{}{}{s.m.}{Linhagem de um animal de raça, especialmente de cão ou cavalo.}{\textit{pedigree}}{0}
\verb{pedilúvio}{}{}{}{}{s.m.}{Banho terapêutico nos pés.}{pe.di.lú.vio}{0}
\verb{pedinchão}{}{}{"-ões}{}{adj.}{Que vive pedindo; pidão.}{pe.din.chão}{0}
\verb{pedinchar}{}{}{}{}{v.t.}{Pedir com insistência e de forma inorportuna.}{pe.din.char}{\verboinum{1}}
\verb{pedinte}{}{}{}{}{adj.2g.}{Diz"-se daquele que pede, mendiga; mendigo.}{pe.din.te}{0}
\verb{pedir}{}{}{}{}{v.t.}{Solicitar a alguém que se conceda algo; rogar, suplicar.}{pe.dir}{0}
\verb{pedir}{}{}{}{}{}{Pôr como preço; estipular o valor.}{pe.dir}{0}
\verb{pedir}{}{}{}{}{}{Rogar a Deus ou aos santos; implorar.}{pe.dir}{\verboinum{20}}
\verb{pé"-direito}{ê}{}{pés"-direitos}{}{s.m.}{Altura entre o piso e o teto de um cômodo ou pavimento.}{pé"-di.rei.to}{0}
\verb{peditório}{}{}{}{}{s.m.}{Ato de pedir para fins de caridade.}{pe.di.tó.rio}{0}
\verb{peditório}{}{}{}{}{}{Súplica insistente.}{pe.di.tó.rio}{0}
\verb{pedra}{é}{}{}{}{s.f.}{Matéria sólida, dura, constituída de vários minerais misturados.}{pe.dra}{0}
\verb{pedra}{é}{}{}{}{}{Peça retangular de grande tamanho sobre a qual se escreve; lousa, quadro"-negro.}{pe.dra}{0}
\verb{pedra}{é}{}{}{}{}{Peça de jogo de tabuleiro.}{pe.dra}{0}
\verb{pedrada}{}{}{}{}{s.f.}{Golpe dado com pedra. }{pe.dra.da}{0}
\verb{pedra"-de"-fogo}{é\ldots{}ô}{}{pedras"-de"-fogo ⟨é\ldots{}ô⟩}{}{s.f.}{Pedra muito dura que produz faísca quando ferida com fragmento de aço; pederneira.}{pe.dra"-de"-fo.go}{0}
\verb{pedra"-pomes}{é}{}{pedras"-pomes ⟨é⟩}{}{s.f.}{Pedra leve e porosa, própria para limpar e amaciar a pele.}{pe.dra"-po.mes}{0}
\verb{pedraria}{}{}{}{}{s.f.}{Grande quantidade de pedras preciosas; joias.}{pe.dra.ri.a}{0}
\verb{pedra"-sabão}{é}{}{pedras"-sabão \textit{ou} pedras"-sabões ⟨é⟩}{}{s.f.}{Pedra de pouca dureza, variedade do talco, muito usada para esculturas.}{pe.dra"-sa.bão}{0}
\verb{pedra"-ume}{é}{Geol.}{pedras"-umes ⟨é⟩}{}{s.f.}{Nome vulgar do composto de sulfato de alumínio e potássio, muito usada na fabricação de corantes, porcelana, na purificação de água etc.}{pe.dra"-u.me}{0}
\verb{pedregoso}{ô}{}{"-osos ⟨ó⟩}{"-osa ⟨ó⟩}{adj.}{Em que há muitas pedras. (\textit{O terreno que meu avô comprou é muito pedregoso.})}{pe.dre.go.so}{0}
\verb{pedregulho}{}{}{}{}{s.m.}{Pedra pequena geralmente retirada do fundo dos rios; seixo.}{pe.dre.gu.lho}{0}
\verb{pedreira}{ê}{}{}{}{s.f.}{Local ou rocha de onde se extrai pedra.}{pe.drei.ra}{0}
\verb{pedreiro}{ê}{}{}{}{s.m.}{Profissional que trabalha na construção de casas e edifícios.}{pe.drei.ro}{0}
\verb{pedreiro"-livre}{}{}{pedreiros"-livres}{}{s.m.}{Indivíduo filiado à maçonaria; maçom.}{pe.drei.ro"-li.vre}{0}
\verb{pedrento}{}{}{}{}{adj.}{Em que há muitas pedras; pedregoso.}{pe.dren.to}{0}
\verb{pedrento}{}{}{}{}{}{Que tem aspecto ou consistência de pedra.}{pe.dren.to}{0}
\verb{pedrento}{}{}{}{}{}{Diz"-se de céu com núvens espessas semelhantes a pedras.}{pe.dren.to}{0}
\verb{pedrês}{}{}{}{}{adj.2g.}{Diz"-se do que é salpicado de cores preta e branca.}{pe.drês}{0}
\verb{pedroso}{ô}{}{"-osos ⟨ó⟩}{"-osa ⟨ó⟩}{adj.}{Que tem aspecto ou consistência de pedra.}{pe.dro.so}{0}
\verb{pedunculado}{}{Biol.}{}{}{adj.}{Que apresenta pedúnculos.}{pe.dun.cu.la.do}{0}
\verb{peduncular}{}{}{}{}{adj.2g.}{Relativo a pedúnculo.}{pe.dun.cu.lar}{0}
\verb{pedúnculo}{}{Bot.}{}{}{s.m.}{Tipo de estrutura que sustenta inflorescências.}{pe.dún.cu.lo}{0}
\verb{pedúnculo}{}{Zool.}{}{}{}{Haste através da qual certos animais invertebrados se fixam ao substrato.}{pe.dún.cu.lo}{0}
\verb{pê"-efe}{é}{Pop.}{pê"-efes ⟨é⟩}{}{s.m.}{Prato pronto, servido em bares e restaurantes; prato feito.}{pê"-e.fe}{0}
\verb{pé"-frio}{}{}{pés"-frios}{}{s.2g.}{Pessoa sem sorte em jogos ou negócios, e que traz má sorte aos outros também.}{pé"-fri.o}{0}
\verb{pega}{é}{}{}{}{s.f.}{Ato de pegar.}{pe.ga}{0}
\verb{pega}{é}{}{}{}{}{Recrutamento forçado.}{pe.ga}{0}
\verb{pega}{é}{}{}{}{s.m.}{Conflito, discussão acalorada, briga.}{pe.ga}{0}
\verb{pegada}{}{}{}{}{s.f.}{Marca deixada pelo pé no solo.}{pe.ga.da}{0}
\verb{pegadiço}{}{}{}{}{adj.}{Que pega facilmente; pegajoso.}{pegadiço}{0}
\verb{pegadiço}{}{Fig.}{}{}{}{Contagioso.}{pegadiço}{0}
\verb{pegado}{}{}{}{}{adj.}{Que se pegou.}{pe.ga.do}{0}
\verb{pegado}{}{}{}{}{}{Colado, ligado, fixado.}{pe.ga.do}{0}
\verb{pegado}{}{}{}{}{}{Vizinho imediato; contíguo.}{pe.ga.do}{0}
\verb{pegador}{ô}{}{}{}{adj.}{Que pega.}{pe.ga.dor}{0}
\verb{pegajoso}{ô}{}{"-osos ⟨ó⟩}{"-osa ⟨ó⟩}{adj.}{Que adere facilmente; grudento, viscoso.}{pe.ga.jo.so}{0}
\verb{pega"-ladrão}{é}{}{pega"-ladrões ⟨é⟩}{}{s.m.}{Dispositivo mecânico ou elétrico de segurança, para evitar roubos ou furtos.}{pe.ga"-la.drão}{0}
\verb{pega"-ladrão}{é}{}{pega"-ladrões ⟨é⟩}{}{}{Dispositivo adaptado ao fecho de colares e outros adornos corporais para evitar roubo ou perda.}{pe.ga"-la.drão}{0}
\verb{pega"-pega}{é\ldots{}é}{}{pegas"-pegas \textit{ou} pega"-pegas ⟨é\ldots{}é⟩}{}{s.m.}{Jogo infantil em que um tem que pegar os outros participantes antes que cheguem a um local seguro; pique.}{pe.ga"-pe.ga}{0}
\verb{pega"-pega}{é\ldots{}é}{}{pegas"-pegas \textit{ou} pega"-pegas ⟨é\ldots{}é⟩}{}{}{Correria para pegar ladrões; confusão.}{pe.ga"-pe.ga}{0}
\verb{pegar}{}{}{}{}{v.t.}{Tomar nas mãos; segurar, agarrar.}{pe.gar}{0}
\verb{pegar}{}{}{}{}{}{Fazer aderir; grudar, colar.}{pe.gar}{0}
\verb{pegar}{}{}{}{}{}{Transmitir por contágio ou influência. (\textit{Peguei uma gripe forte no último inverno.})}{pe.gar}{0}
\verb{pegar}{}{}{}{}{}{Seguir caminho; dirigir"-se.}{pe.gar}{0}
\verb{pegar}{}{}{}{}{}{Tomar veículo; apanhar.}{pe.gar}{0}
\verb{pegar}{}{}{}{}{v.i.}{Criar raízes; vingar.}{pe.gar}{0}
\verb{pegar}{}{}{}{}{v.pron.}{Socorrer"-se, valer"-se.}{pe.gar}{\verboinum{5}}
\verb{pega"-rapaz}{é}{}{pega"-rapazes ⟨é⟩}{}{s.m.}{Mecha de cabelo recurvada que pende sobre a testa ou o rosto.}{pe.ga"-ra.paz}{0}
\verb{pego}{é}{}{}{}{s.m.}{A parte mais funda de lago ou rio.}{pe.go}{0}
\verb{pego}{é}{Fig.}{}{}{}{Abismo, pélago.}{pe.go}{0}
\verb{pego}{ê/ ou /é}{}{}{}{adj.}{Que se pegou.}{pe.go}{0}
\verb{pegureiro}{ê}{}{}{}{s.m.}{Pastor de gado.}{pe.gu.rei.ro}{0}
\verb{pegureiro}{ê}{}{}{}{}{Cão de caça.}{pe.gu.rei.ro}{0}
\verb{peia}{ê}{}{}{}{s.f.}{Corda ou correia que prende os pés dos animais para que não andem.}{pei.a}{0}
\verb{peia}{ê}{Fig.}{}{}{}{Estorvo, obstáculo, empecilho.}{pei.a}{0}
\verb{peidar}{}{Pop.}{}{}{v.i.}{Dar peido.}{pei.dar}{\verboinum{1}}
\verb{peido}{ê}{Pop.}{}{}{s.m.}{Porção de gás intestinal emitida pelo ânus; flatulência.}{pei.do}{0}
\verb{peita}{ê}{}{}{}{s.f.}{Suborno, corrupção.}{pei.ta}{0}
\verb{peitar}{}{}{}{}{v.t.}{Corromper com dádivas; subornar.}{pei.tar}{\verboinum{1}}
\verb{peitilho}{}{}{}{}{s.m.}{O que reveste o peito.}{pei.ti.lho}{0}
\verb{peitilho}{}{}{}{}{}{Peça do vestuário que fica sobre o peito.}{pei.ti.lho}{0}
\verb{peito}{ê}{}{}{}{s.m.}{Cavidade do tronco em que estão localizados os pulmões e coração.}{pei.to}{0}
\verb{peito}{ê}{}{}{}{}{A porção superior da parte frontal do tronco.}{pei.to}{0}
\verb{peito}{ê}{}{}{}{}{Seio.}{pei.to}{0}
\verb{peito}{ê}{}{}{}{}{Parte superior do pé.}{pei.to}{0}
\verb{peito}{ê}{Fig.}{}{}{}{Coragem, ânimo, valentia.}{pei.to}{0}
\verb{peitoral}{}{}{"-ais}{}{adj.2g.}{Relativo a peito.}{pei.to.ral}{0}
\verb{peitoral}{}{}{"-ais}{}{}{Diz"-se de medicamento ou substância que faz bem ao peito.}{pei.to.ral}{0}
\verb{peitoral}{}{Anat.}{"-ais}{}{}{Diz"-se da musculatura da parte anterior do tórax.}{pei.to.ral}{0}
\verb{peitoril}{}{}{"-is}{}{s.m.}{Parapeito.}{pei.to.ril}{0}
\verb{peitudo}{}{}{}{}{adj.}{De peito grande.}{pei.tu.do}{0}
\verb{peitudo}{}{Fig.}{}{}{}{Valente, corajoso.}{pei.tu.do}{0}
\verb{peixada}{ch}{Cul.}{}{}{s.f.}{Prato preparado à base de peixe.}{pei.xa.da}{0}
\verb{peixada}{ch}{}{}{}{}{Festa ou refeição em que predominam peixes.}{pei.xa.da}{0}
\verb{peixão}{ch}{}{"-ões}{}{s.m.}{Peixe grande.}{pei.xão}{0}
\verb{peixão}{ch}{Pop.}{"-ões}{}{}{Mulher corpulenta e vistosa.}{pei.xão}{0}
\verb{peixaria}{ch}{}{}{}{s.f.}{Estabelecimento em que se vende peixe.}{pei.xa.ri.a}{0}
\verb{peixe}{ch}{Zool.}{}{}{s.m.}{Grupo de animais vertebrados, geralmente com corpo recoberto por escamas, que vivem dentro da água.}{pei.xe}{0}
\verb{peixe"-boi}{ch}{Zool.}{peixes"-bois \textit{ou} peixes"-boi}{}{s.m.}{Grande mamífero aquático.}{pei.xe"-boi}{0}
\verb{peixe"-elétrico}{ch}{Zool.}{peixes"-elétricos}{}{s.m.}{Peixe que emite corrente elétrica através de um órgão específico.}{pei.xe"-e.lé.tri.co}{0}
\verb{peixe"-espada}{ch}{Zool.}{peixes"-espadas \textit{ou} peixes"-espada}{}{s.m.}{Peixe de corpo alongado com cauda pontiaguda semelhante a uma espada.}{pei.xe"-es.pa.da}{0}
\verb{peixeira}{ch}{}{}{}{s.f.}{Grande faca própria para cortar peixe.}{pei.xei.ra}{0}
\verb{peixeira}{ch}{}{}{}{}{Travessa própria para servir peixe.}{pei.xei.ra}{0}
\verb{peixeira}{ch}{}{}{}{}{Mulher que vende peixe.}{pei.xei.ra}{0}
\verb{peixeiro}{ch}{}{}{}{s.m.}{Homem que vende peixe.}{pei.xei.ro}{0}
\verb{peixes}{ch}{Astron.}{}{}{s.m.}{Décima segunda e última constelação zodiacal.}{pei.xes}{0}
\verb{peixes}{ch}{Astrol.}{}{}{}{O signo do zodíaco referente a essa constelação.}{pei.xes}{0}
\verb{peixe"-voador}{ch\ldots{}ô}{Zool.}{peixes"-voadores ⟨ch\ldots{}ô⟩}{}{s.m.}{Peixe marinho que dá pequenos saltos para fora da água quando se desloca em grande velocidade.}{pei.xe"-vo.a.dor}{0}
\verb{pejar}{}{}{}{}{v.t.}{Ocupar o espaço; encher.}{pe.jar}{0}
\verb{pejar}{}{}{}{}{}{Sobrecarregar.}{pe.jar}{0}
\verb{pejar}{}{}{}{}{}{Estorvar, embaraçar.}{pe.jar}{0}
\verb{pejar}{}{}{}{}{}{Causar vergonha.}{pe.jar}{0}
\verb{pejar}{}{}{}{}{}{Engravidar.}{pe.jar}{\verboinum{1}}
\verb{pejo}{ê}{}{}{}{s.m.}{Pudor, vergonha, acanhamento, timidez.}{pe.jo}{0}
\verb{pejorativo}{}{}{}{}{adj.}{Que exprime significado negativo, depreciativo.}{pe.jo.ra.ti.vo}{0}
\verb{pela}{ê}{}{}{}{prep.}{Forma feminina de \textit{pelo}.}{pe.la}{0}
\verb{pela}{}{}{}{}{s.f.}{Bola, geralmente de borracha, usada para brincar ou jogar.}{pe.la}{0}
\verb{pela}{}{Fig.}{}{}{}{Alvo de brincadeira maldosa; joguete.}{pe.la}{0}
\verb{pela}{}{}{}{}{s.f.}{Ato de pelar.}{pe.la}{0}
\verb{pelada}{}{}{}{}{s.f.}{Jogo de futebol em situação bastante informal e em campo geralmente improvisado.}{pe.la.da}{0}
\verb{pelado}{}{}{}{}{adj.}{Que se pelou.}{pe.la.do}{0}
\verb{pelado}{}{}{}{}{}{Que não tem pelo.}{pe.la.do}{0}
\verb{pelado}{}{}{}{}{}{Careca, calvo.}{pe.la.do}{0}
\verb{pelado}{}{Bras.}{}{}{adj.}{Despido, nu.}{pe.la.do}{0}
\verb{pelado}{}{}{}{}{}{Sem nenhum dinheiro; pobre.}{pe.la.do}{0}
\verb{pelagem}{}{}{"-ens}{}{s.f.}{A camada de pelo dos animais.}{pe.la.gem}{0}
\verb{pelágico}{}{}{}{}{adj.}{Relativo a pélago; oceânico.}{pe.lá.gi.co}{0}
\verb{pélago}{}{}{}{}{s.m.}{Abismo submarino.}{pé.la.go}{0}
\verb{pélago}{}{}{}{}{}{Altomar; oceano.}{pé.la.go}{0}
\verb{pélago}{}{Fig.}{}{}{}{Abismo.}{pé.la.go}{0}
\verb{pelame}{}{}{}{}{s.m.}{Pele dos animais com pelos.}{pe.la.me}{0}
\verb{pelanca}{}{}{}{}{s.f.}{Porção de pele e carne flácida, caída e que balança com os movimentos do corpo.}{pe.lan.ca}{0}
\verb{pelanca}{}{Bras.}{}{}{}{Em cortes de carne comestível, porção de tecidos menos apreciados, com pele, nervo ou gordura.}{pe.lan.ca}{0}
\verb{pelancudo}{}{}{}{}{adj.}{Em que há pelancas.}{pe.lan.cu.do}{0}
\verb{pelar}{}{}{}{}{v.t.}{Tirar os pelos de.}{pe.lar}{\verboinum{1}}
\verb{pelar}{}{}{}{}{v.t.}{Tirar pele ou casca de.}{pe.lar}{0}
\verb{pelar}{}{Pop.}{}{}{}{Tirar todos os pertences de; depenar, roubar.}{pe.lar}{\verboinum{1}}
\verb{pele}{é}{}{}{}{s.f.}{Tecido membranoso que reveste o corpo do homem e dos animais; epiderme.}{pe.le}{0}
\verb{pele}{é}{}{}{}{}{Couro.}{pe.le}{0}
\verb{pele}{é}{}{}{}{}{A casca de certos frutos e legumes.}{pe.le}{0}
\verb{pele}{é}{}{}{}{}{Pele de certos animais, geralmente suave e revestida de pelos macios, usada como peça de vestuário.}{pe.le}{0}
\verb{pelego}{ê}{Bras.}{}{}{s.m.}{Pele de carneiro com a lã.}{pe.le.go}{0}
\verb{pelego}{ê}{Bras.}{}{}{}{Operário membro de sindicato a serviço dos interesses patronais.}{pe.le.go}{0}
\verb{pelego}{ê}{Bras.}{}{}{}{Indivíduo subserviente e passivo; capacho.}{pe.le.go}{0}
\verb{peleja}{ê}{}{}{}{s.f.}{Ato de pelejar; luta.}{pe.le.ja}{0}
\verb{peleja}{ê}{}{}{}{}{Jogo, partida, disputa.}{pe.le.ja}{0}
\verb{pelejar}{}{}{}{}{v.t.}{Lutar, batalhar, combater.}{pe.le.jar}{0}
\verb{pelejar}{}{}{}{}{}{Discutir, debater (ideias, doutrinas).}{pe.le.jar}{\verboinum{1}}
\verb{pelerine}{}{}{}{}{s.f.}{Pequeno manto que recobre parte das costas e peito.}{pe.le.ri.ne}{0}
\verb{peleteria}{}{}{}{}{s.f.}{Loja em que se vendem peles.}{pe.le.te.ri.a}{0}
\verb{pelica}{}{}{}{}{s.f.}{Pele fina de animal preparada para confecçãos de luvas, sapatos, roupas etc.}{pe.li.ca}{0}
\verb{peliça}{}{}{}{}{s.f.}{Peça de vestuário, ou colcha, feita ou forrada de peles finas e macias.}{pe.li.ça}{0}
\verb{pelicaniforme}{ó}{Zool.}{}{}{s.m.}{Espécie dos pelicaniformes, ordem de aves aquáticas, de grande bico, pernas curtas e pés com quatro dedos palmados.}{pe.li.ca.ni.for.me}{0}
\verb{pelicaniforme}{ó}{}{}{}{adj.2g.}{Relativo a pelicano.}{pe.li.ca.ni.for.me}{0}
\verb{pelicano}{}{Zool.}{}{}{s.m.}{Gênero de grandes aves aquáticas, com bico muito grande e bolsa por baixo da mandíbula inferior, na qual são guardados peixes de que se alimentam.}{pe.li.ca.no}{0}
\verb{pelico}{}{}{}{}{s.m.}{Membrana que envolve o feto; âmnio.}{pe.li.co}{0}
\verb{película}{}{}{}{}{s.f.}{Membrana ou camada fina de pele.}{pe.lí.cu.la}{0}
\verb{película}{}{}{}{}{}{Faixa comprida, fina e transparente, em que se fixam as imagens de um filme.}{pe.lí.cu.la}{0}
\verb{pelintra}{}{}{}{}{s.2g.}{Indivíduo pretensioso.}{pe.lin.tra}{0}
\verb{pelintra}{}{}{}{}{adj.2g.}{Que é safado, descarado.}{pe.lin.tra}{0}
\verb{pelo}{ê}{}{}{}{prep.}{Contração da preposição \textit{per} com o artigo \textit{o}; através de, ao longo de, por causa de.}{pe.lo}{0}
\verb{pêlo}{}{}{}{}{s.m.}{Fio que cresce sobre a pele.}{pê.lo}{0}
\verb{pêlo}{}{}{}{}{}{Fios que recobrem frutas ou plantas; penugem.}{pê.lo}{0}
\verb{pelota}{ó}{}{}{}{s.f.}{Pequena bola.}{pe.lo.ta}{0}
\verb{pelota}{ó}{}{}{}{}{Bola de futebol.}{pe.lo.ta}{0}
\verb{pelotão}{}{}{"-ões}{}{s.m.}{Grupo de soldados comandados por um tenente.                      }{pe.lo.tão}{0}
\verb{pelourinho}{}{}{}{}{s.m.}{Coluna de pedra ou de madeira, em praça ou lugar público, junto da qual se expunham e castigavam criminosos.}{pe.lou.ri.nho}{0}
\verb{pelouro}{ô}{}{}{}{s.m.}{Barra de ferro ou de pedra, esférica, empregada antigamente em peças de artilharia.}{pe.lou.ro}{0}
\verb{pelúcia}{}{}{}{}{s.f.}{Tecido felpudo de um lado e liso de outro.}{pe.lú.cia}{0}
\verb{peludo}{}{}{}{}{adj.}{Que tem muitos pelos.}{pe.lu.do}{0}
\verb{pelugem}{}{}{"-ens}{}{s.f.}{Conjunto de pelos.}{pe.lu.gem}{0}
\verb{pelve}{é}{Anat.}{}{}{s.f.}{Cavidade no extremo inferior do tronco, formada pelos dois ossos do quadril; bacia.}{pel.ve}{0}
\verb{pélvis}{}{}{}{}{}{Var. de \textit{pelve}.}{pél.vis}{0}
\verb{pena}{}{}{}{}{s.f.}{Cada uma das hastes finas e compridas, cobertas de pelos, que recobrem o corpo das aves; pluma.}{pe.na}{0}
\verb{pena}{}{}{}{}{}{Peça de metal, de ponta fina, que se adapta a uma caneta para escrever.}{pe.na}{0}
\verb{pena}{}{}{}{}{}{Punição imposta pelo Estado a um culpado em julgamento; penalidade, castigo.}{pe.na}{0}
\verb{pena}{}{}{}{}{}{Compaixão, dó, pesar.}{pe.na}{0}
\verb{penacho}{}{}{}{}{s.m.}{Conjunto de penas que enfeitam a cabeça, o chapéu etc.}{pe.na.cho}{0}
\verb{penada}{}{}{}{}{s.f.}{Traço feito com caneta de pena.}{pe.na.da}{0}
\verb{penado}{}{}{}{}{adj.}{Que tem penas; emplumado.}{pe.na.do}{0}
\verb{penal}{}{}{"-ais}{}{adj.2g.}{Relativo a penas judiciais.}{pe.nal}{0}
\verb{penal}{}{}{"-ais}{}{}{Relativo à legislação que aplica penas judiciais.}{pe.nal}{0}
\verb{penalidade}{}{}{}{}{s.f.}{Sistema de penas ditadas pela lei.}{pe.na.li.da.de}{0}
\verb{penalidade}{}{}{}{}{}{Pena, castigo.}{pe.na.li.da.de}{0}
\verb{penalista}{}{}{}{}{s.2g.}{Indivíduo especializado em leis penais.}{pe.na.lis.ta}{0}
\verb{penalizar}{}{}{}{}{v.t.}{Fazer sentir dó.}{pe.na.li.zar}{0}
\verb{penalizar}{}{}{}{}{}{Aplicar pena, castigo.}{pe.na.li.zar}{0}
\verb{penalizar}{}{}{}{}{}{Causar prejuízo.}{pe.na.li.zar}{\verboinum{1}}
\verb{pênalti}{}{Esport.}{}{}{s.m.}{Em futebol, falta máxima cobrada com chute direto sem barreira.}{pê.nal.ti}{0}
\verb{penar}{}{}{}{}{s.m.}{Padecimento, sofrimento.}{pe.nar}{0}
\verb{penar}{}{}{}{}{v.i.}{Sofrer pena, dor, aflição; padecer.}{pe.nar}{\verboinum{1}}
\verb{penates}{}{}{}{}{s.m.pl.}{Deuses do lar, entre os romanos e etruscos.}{pe.na.tes}{0}
\verb{penates}{}{}{}{}{}{Casas paternas, lares, famílias.}{pe.na.tes}{0}
\verb{penca}{}{}{}{}{s.f.}{Conjunto de coisas presas ao mesmo suporte.}{pen.ca}{0}
\verb{pencenê}{}{}{}{}{}{Var. de \textit{pincenê}.}{pen.ce.nê}{0}
\verb{pendão}{}{}{"-ões}{}{s.m.}{Bandeira.}{pen.dão}{0}
\verb{pendão}{}{}{"-ões}{}{}{Conjunto de ramos, cheios de pequenas flores, que aparece no alto do pé de milho.}{pen.dão}{0}
\verb{pendão}{}{}{"-ões}{}{}{Emblema ou símbolo de um partido, de uma causa.}{pen.dão}{0}
\verb{pendência}{}{}{}{}{s.f.}{Questão ou causa não resolvida.}{pen.dên.cia}{0}
\verb{pendência}{}{}{}{}{}{Contenda, disputa, litígio.}{pen.dên.cia}{0}
\verb{pendente}{}{}{}{}{adj.2g.}{Que pende; pendurado, suspenso.}{pen.den.te}{0}
\verb{pendente}{}{}{}{}{}{Que ainda se deve resolver.}{pen.den.te}{0}
\verb{pender}{ê}{}{}{}{v.t.}{Estar dependurado.}{pen.der}{0}
\verb{pender}{ê}{}{}{}{}{Ter gosto ou preferência por alguma coisa; inclinar"-se, tender.}{pen.der}{\verboinum{12}}
\verb{pendor}{ô}{}{}{}{s.m.}{Superfície cuja altura diminui ou aumenta gradualmente; declive, rampa.}{pen.dor}{0}
\verb{pendor}{ô}{}{}{}{}{Capacidade natural para algo; propensão, inclinação, tendência.}{pen.dor}{0}
\verb{pendular}{}{}{}{}{adj.2g.}{Relativo a pêndulo.}{pen.du.lar}{0}
\verb{pendular}{}{}{}{}{}{Que oscila; oscilante.}{pen.du.lar}{0}
\verb{pêndulo}{}{}{}{}{s.m.}{Objeto que oscila preso à ponta de fio ou haste.}{pên.du.lo}{0}
\verb{pendura}{}{}{}{}{s.f.}{Ato ou efeito de pendurar; dependura.}{pen.du.ra}{0}
\verb{pendura}{}{}{}{}{}{Coisa pendurada, pendente.}{pen.du.ra}{0}
\verb{pendurado}{}{}{}{}{adj.}{Que se pendurou; pendente.}{pen.du.ra.do}{0}
\verb{pendurado}{}{}{}{}{}{Cheio de dívidas; endividado.}{pen.du.ra.do}{0}
\verb{pendurar}{}{}{}{}{v.t.}{Fazer algo ficar preso a certa altura, sem tocar o chão.}{pen.du.rar}{0}
\verb{pendurar}{}{Pop.}{}{}{}{Não pagar uma conta.}{pen.du.rar}{\verboinum{1}}
\verb{penduricalho}{}{}{}{}{s.m.}{Enfeite que fica dependurado; pingente.}{pen.du.ri.ca.lho}{0}
\verb{penedia}{}{}{}{}{s.f.}{Reunião de penedos.}{pe.ne.di.a}{0}
\verb{penedia}{}{}{}{}{}{Rocha, rochedo.}{pe.ne.di.a}{0}
\verb{penedo}{ê}{}{}{}{s.m.}{Grande pedra; rochedo.}{pe.ne.do}{0}
\verb{peneira}{ê}{}{}{}{s.f.}{Utensílio com furos para separar fragmentos maiores de menores.}{pe.nei.ra}{0}
\verb{peneirada}{}{}{}{}{}{Porção que se peneira de cada vez.}{pe.nei.ra.da}{0}
\verb{peneirada}{}{}{}{}{s.f.}{Peneiração.}{pe.nei.ra.da}{0}
\verb{peneirar}{}{}{}{}{v.t.}{Fazer alguma coisa passar pela peneira.}{pe.nei.rar}{0}
\verb{peneirar}{}{}{}{}{}{Selecionar.}{pe.nei.rar}{\verboinum{1}}
\verb{penetra}{é}{}{}{}{s.2g.}{Indivíduo que entra em algum lugar sem ser convidado.}{pe.ne.tra}{0}
\verb{penetração}{}{}{"-ões}{}{s.f.}{Ato ou efeito de penetrar; passagem para parte interna de.}{pe.ne.tra.ção}{0}
\verb{penetração}{}{}{"-ões}{}{}{Cópula sexual.}{pe.ne.tra.ção}{0}
\verb{penetração}{}{}{"-ões}{}{}{Capacidade de compreensão.}{pe.ne.tra.ção}{0}
\verb{penetrante}{}{}{}{}{adj.2g.}{Que penetra.}{pe.ne.tran.te}{0}
\verb{penetrante}{}{}{}{}{}{Muito forte ou intenso.}{pe.ne.tran.te}{0}
\verb{penetrante}{}{}{}{}{}{Arguto, perspicaz.}{pe.ne.tran.te}{0}
\verb{penetrar}{}{}{}{}{v.t.}{Avançar para algum lugar; entrar, introduzir"-se.}{pe.ne.trar}{\verboinum{1}}
\verb{penetrável}{}{}{"-eis}{}{adj.2g.}{Que pode ser penetrado.}{pe.ne.trá.vel}{0}
\verb{pênfigo}{}{Med.}{}{}{s.m.}{Dermatose que se manifesta pelo aparecimento de bolhas.}{pên.fi.go}{0}
\verb{penha}{}{}{}{}{s.f.}{Rochedo muito grande.}{pe.nha}{0}
\verb{penhascal}{}{}{"-ais}{}{s.m.}{Penhasqueira.}{pe.nhas.cal}{0}
\verb{penhasco}{}{}{}{}{s.m.}{Grande rochedo escarpado.}{pe.nhas.co}{0}
\verb{penhasqueira}{ê}{}{}{}{s.f.}{Série de penhascos; penhascal.}{pe.nhas.quei.ra}{0}
\verb{penhoar}{}{}{}{}{s.m.}{Vestido aberto na frente, geralmente usado sobre camisola.}{pe.nho.ar}{0}
\verb{penhor}{ô}{}{}{}{s.m.}{Objeto ou imóvel que se entrega para garantir o pagamento de uma dívida.}{pe.nhor}{0}
\verb{penhora}{ó}{}{}{}{s.f.}{Apreensão judicial dos bens do devedor como pagamento de dívida.}{pe.nho.ra}{0}
\verb{penhorado}{}{}{}{}{adj.}{Que foi tomado em penhor, ou apreendido em penhora.}{pe.nho.ra.do}{0}
\verb{penhorar}{}{}{}{}{v.t.}{Efetuar a penhora.}{pe.nho.rar}{0}
\verb{penhorar}{}{}{}{}{}{Dar como garantia de empréstimo.}{pe.nho.rar}{\verboinum{1}}
\verb{penicilina}{}{Farm.}{}{}{s.f.}{Substância extraída de vários tipos de fungos, usada como antibiótico.}{pe.ni.ci.li.na}{0}
\verb{penico}{}{}{}{}{s.m.}{Recipiente usado para urina e fezes; urinol.}{pe.ni.co}{0}
\verb{penífero}{}{}{}{}{adj.}{Que tem penas; penudo.}{pe.ní.fe.ro}{0}
\verb{peniforme}{}{}{}{}{adj.2g.}{Que apresenta forma de pena.}{pe.ni.for.me}{0}
\verb{península}{}{}{}{}{s.f.}{Porção de terra cercada de água por todos os lados, com exceção de um, que a liga ao continente.}{pe.nín.su.la}{0}
\verb{peninsular}{}{}{}{}{adj.2g.}{Relativo a península.}{pe.nin.su.lar}{0}
\verb{peninsular}{}{}{}{}{s.2g.}{Indivíduo natural ou habitante de península. }{pe.nin.su.lar}{0}
\verb{pênis}{}{Anat.}{}{}{s.m.}{Órgão genital masculino.}{pê.nis}{0}
\verb{penitência}{}{Relig.}{}{}{s.f.}{Pena imposta a um indivíduo pelo confessor para receber o perdão de seus pecados.}{pe.ni.tên.cia}{0}
\verb{penitencial}{}{}{"-ais}{}{adj.2g.}{Relativo a penitência; penitenciário. }{pe.ni.ten.ci.al}{0}
\verb{penitenciar}{}{}{}{}{v.t.}{Impor penitência.}{pe.ni.ten.ci.ar}{0}
\verb{penitenciar}{}{}{}{}{v.pron.}{Fazer penitência.}{pe.ni.ten.ci.ar}{\verboinum{1}}
\verb{penitenciária}{}{}{}{}{s.f.}{Estabelecimento em que ficam recolhidos os indivíduos que estão cumprindo pena após o julgamento da sentença; presídio, prisão.}{pe.ni.ten.ci.á.ria}{0}
\verb{penitenciário}{}{}{}{}{adj.}{Relativo a penitenciária.}{pe.ni.ten.ci.á.rio}{0}
\verb{penitenciário}{}{}{}{}{s.m.}{Indivíduo encarcerado em penitenciária; presidiário, preso.}{pe.ni.ten.ci.á.rio}{0}
\verb{penitente}{}{}{}{}{adj.2g.}{Diz"-se daquele que faz penitência ou confessa seus pecados.}{pe.ni.ten.te}{0}
\verb{penosa}{ó}{Pop.}{}{}{s.f.}{Galinha magra.}{pe.no.sa}{0}
\verb{penoso}{ô}{}{"-osos ⟨ó⟩}{"-osa ⟨ó⟩}{adj.}{Que causa pena, faz sofrer; doloroso. }{pe.no.so}{0}
\verb{penoso}{ô}{}{"-osos ⟨ó⟩}{"-osa ⟨ó⟩}{}{Que é difícil; cansativo, fatigante.}{pe.no.so}{0}
\verb{pensador}{ô}{}{}{}{adj.}{Diz"-se daquele que pensa ou faz reflexões; filósofo.}{pen.sa.dor}{0}
\verb{pensamento}{}{}{}{}{s.m.}{Ato ou efeito de pensar.}{pen.sa.men.to}{0}
\verb{pensamento}{}{}{}{}{}{Ideia, opinião, reflexão.}{pen.sa.men.to}{0}
\verb{pensamento}{}{}{}{}{}{Conceito, máxima.}{pen.sa.men.to}{0}
\verb{pensante}{}{}{}{}{adj.2g.}{Que pensa; que usa a razão.  }{pen.san.te}{0}
\verb{pensão}{}{}{"-ões}{}{s.f.}{Renda mensal que se recebe do Estado ou de um particular sem estar trabalhando.}{pen.são}{0}
\verb{pensão}{}{}{"-ões}{}{}{Pequeno hotel de caráter familiar e preços baixos.}{pen.são}{0}
\verb{pensar}{}{}{}{}{v.t.}{Formar ideias, conceitos; cogitar, refletir, meditar.}{pen.sar}{0}
\verb{pensar}{}{}{}{}{v.t.}{Aplicar curativos, pensos.}{pen.sar}{\verboinum{1}}
\verb{pensar}{}{}{}{}{}{Julgar, opinar, supor.}{pen.sar}{\verboinum{1}}
\verb{pensativo}{}{}{}{}{adj.}{Que está concentrado em seus pensamentos; meditativo.}{pen.sa.ti.vo}{0}
\verb{pênsil}{}{}{"-eis}{}{adj.2g.}{Que está suspenso; pendurado, dependurado.}{pên.sil}{0}
\verb{pensionar}{}{}{}{}{v.t.}{Conceder ou pagar pensão.}{pen.si.o.nar}{0}
\verb{pensionar}{}{}{}{}{}{Sobrecarregar com trabalhos ou tarefas.}{pen.si.o.nar}{\verboinum{1}}
\verb{pensionato}{}{}{}{}{s.m.}{Instituição em que os alunos estudam e residem; internato.}{pen.si.o.na.to}{0}
\verb{pensionato}{}{}{}{}{}{Casa que recebe pensionistas; pensão.}{pen.si.o.na.to}{0}
\verb{pensionista}{}{}{}{}{s.2g.}{Indivíduo que recebe pensão.}{pen.si.o.nis.ta}{0}
\verb{pensionista}{}{}{}{}{}{Indivíduo que mora em pensão ou pensionato.}{pen.si.o.nis.ta}{0}
\verb{penso}{}{}{}{}{adj.}{Que se pendeu; inclinado, pendido.}{pen.so}{0}
\verb{penso}{}{}{}{}{s.m.}{Curativo, emplastro.}{pen.so}{0}
\verb{pentacampeão}{}{}{"-ões}{}{adj.}{Diz"-se do indivíduo, equipe ou clube campeão pela quinta vez.}{pen.ta.cam.pe.ão}{0}
\verb{pentaedro}{é}{Geom.}{}{}{s.m.}{Poliedro que tem cinco faces.}{pen.ta.e.dro}{0}
\verb{pentagonal}{}{}{"-ais}{}{adj.2g.}{Relativo a pentágono.}{pen.ta.go.nal}{0}
\verb{pentagonal}{}{}{"-ais}{}{}{Que tem cinco lados ou cinco ângulos.}{pen.ta.go.nal}{0}
\verb{pentágono}{}{Geom.}{}{}{s.m.}{Polígono de cinco lados.  }{pen.tá.go.no}{0}
\verb{pentagrama}{}{Mús.}{}{}{s.m.}{Conjunto de cinco linhas paralelas, sobre as quais se escrevem as notas musicais.}{pen.ta.gra.ma}{0}
\verb{pentagrama}{}{}{}{}{}{Figura simbólica ou mágica de cinco letras ou sinais.}{pen.ta.gra.ma}{0}
\verb{pentassílabo}{}{}{}{}{adj.}{Diz"-se do verso ou palavra de cinco sílabas.}{pen.tas.sí.la.bo}{0}
\verb{pentateuco}{}{}{}{}{s.m.}{Nome coletivo dos cinco primeiros livros da Bíblia.}{pen.ta.teu.co}{0}
\verb{pentatlo}{}{Esport.}{}{}{s.m.}{Competição atlética atual em que cada concorrente participa de cinco modalidades desportivas: equitação, esgrima, tiro, natação e corrida.}{pen.ta.tlo}{0}
\verb{pente}{}{}{}{}{s.m.}{Utensílio cortado em forma de dentes usado para alisar, desembaraçar ou segurar os cabelos.}{pen.te}{0}
\verb{pente}{}{}{}{}{}{Peça das armas de fogo automáticas na qual se encaixam as balas.}{pen.te}{0}
\verb{penteadeira}{ê}{Bras.}{}{}{s.f.}{Pequena mesa com espelho, usada pelas mulheres para se pentearem ou se maquiarem.}{pen.te.a.dei.ra}{0}
\verb{penteado}{}{}{}{}{adj.}{Que se penteou.}{pen.te.a.do}{0}
\verb{penteado}{}{}{}{}{s.m.}{Forma de arrumar os cabelos.}{pen.te.a.do}{0}
\verb{pentear}{}{}{}{}{v.t.}{Alisar ou arrumar os cabelos com o pente.}{pen.te.ar}{\verboinum{4}}
\verb{pentecostal}{}{}{"-ais}{}{adj.2g.}{Relativo a Pentecostes, ou ao pentecostalismo.}{pen.te.cos.tal}{0}
\verb{Pentecostes}{ó}{Relig.}{}{}{s.m.}{Festa católica celebrada 50 dias após a Páscoa, quando se comemora a descida do Espírito Santo sobre os apóstolos. }{Pen.te.cos.tes}{0}
\verb{pente"-fino}{}{}{pentes"-finos}{}{s.m.}{Pente com dentes muito finos, usado para tirar piolhos da cabeça.}{pen.te"-fi.no}{0}
\verb{pente"-fino}{}{Fig.}{pentes"-finos}{}{}{Exame meticuloso; crivo.}{pen.te"-fi.no}{0}
\verb{pentelhar}{}{Chul.}{}{}{v.t.}{Chatear, azucrinar, amolar, aborrecer.}{pen.te.lhar}{\verboinum{1}}
\verb{pentelho}{ê}{Chul.}{}{}{s.m.}{Cada um dos pelos que cobrem a região pubiana.}{pen.te.lho}{0}
\verb{pentelho}{ê}{Bras.}{}{}{}{Pessoa muito chata, maçante.}{pen.te.lho}{0}
\verb{penugem}{}{}{"-ens}{}{s.f.}{Conjunto de pelos ou penas que nascem primeiro.}{pe.nu.gem}{0}
\verb{penugem}{}{}{"-ens}{}{}{Pelo macio e curto.}{pe.nu.gem}{0}
\verb{penugento}{}{}{}{}{adj.}{Que tem muita penugem.}{pe.nu.gen.to}{0}
\verb{penúltimo}{}{}{}{}{adj.}{Que vem logo antes do último.}{pe.núl.ti.mo}{0}
\verb{penumbra}{}{}{}{}{s.f.}{Ponto de transição entre a luz e a sombra; meia"-luz.}{pe.num.bra}{0}
\verb{penúria}{}{}{}{}{s.f.}{Falta do necessário para viver; pobreza extrema, miséria, indigência.}{pe.nú.ria}{0}
\verb{pepineiro}{ê}{Bot.}{}{}{s.m.}{Planta trepadeira, da família da aboboreira, que dá como fruto o pepino.}{pe.pi.nei.ro}{0}
\verb{pepino}{}{}{}{}{s.m.}{Fruto alongado, de casca verde e grossa, com polpa branca e cheia de sementes, próprio para salada ou conserva.}{pe.pi.no}{0}
\verb{pepita}{}{}{}{}{s.f.}{Grão ou fragmento de ouro, encontrado na natureza.}{pe.pi.ta}{0}
\verb{pepsina}{}{Bioquím.}{}{}{s.f.}{Enzima que faz parte do suco gástrico e é responsável pela digestão das proteínas.}{pep.si.na}{0}
\verb{péptico}{}{}{}{}{adj.}{Relativo a pepsina, ou à digestão.}{pép.ti.co}{0}
\verb{pequena}{}{}{}{}{s.f.}{Mulher jovem; moça, garota.}{pe.que.na}{0}
\verb{pequena}{}{Pop.}{}{}{}{Namorada.}{pe.que.na}{0}
\verb{pequenez}{ê}{}{}{}{s.f.}{Qualidade de pequeno.}{pe.que.nez}{0}
\verb{pequenez}{ê}{}{}{}{}{Infância, meninice, meninez.}{pe.que.nez}{0}
\verb{pequenez}{ê}{}{}{}{}{Pequena estatura.}{pe.que.nez}{0}
\verb{pequeneza}{ê}{}{}{}{s.f.}{Pequenez.}{pe.que.ne.za}{0}
\verb{pequenino}{}{}{}{}{adj.}{Muito pequeno.}{pe.que.ni.no}{0}
\verb{pequenino}{}{}{}{}{s.m.}{Criança pequena; menino.}{pe.que.ni.no}{0}
\verb{pequeno}{}{}{}{}{adj.}{De pouco tamanho, extensão ou volume.}{pe.que.no}{0}
\verb{pequeno}{}{}{}{}{}{De pouca idade.}{pe.que.no}{0}
\verb{pequerrucho}{}{}{}{}{adj.}{Diz"-se de criança muito pequena.}{pe.quer.ru.cho}{0}
\verb{pequinês}{}{}{}{}{adj.}{Relativo a Pequim, capital da China.}{pe.qui.nês}{0}
\verb{pequinês}{}{}{}{}{s.m.}{Indivíduo natural ou habitante de Pequim.}{pe.qui.nês}{0}
\verb{pequinês}{}{Zool.}{}{}{s.m.}{Raça de cão pequeno e de pelos longos, originária da China.}{pe.qui.nês}{0}
\verb{per}{ê}{}{}{}{prep.}{Por.}{per}{0}
\verb{pera}{ê}{Desus.}{}{}{prep.}{Para.}{pe.ra}{0}
\verb{pera}{}{}{}{}{s.f.}{O fruto comestível da pereira.}{pe.ra}{0}
\verb{pera}{}{}{}{}{}{Tipo de interruptor elétrico de forma semelhante à da pera e que fica pendente pelo fio.}{pe.ra}{0}
\verb{pera}{}{}{}{}{}{Porção de barba que se deixa crescer no queixo.}{pe.ra}{0}
\verb{peral}{}{}{"-ais}{}{s.m.}{Aglomerado de pereiras.}{pe.ral}{0}
\verb{peralta}{}{Bras.}{}{}{adj.2g.}{Diz"-se de menino travesso.}{pe.ral.ta}{0}
\verb{peralta}{}{}{}{}{s.2g.}{Pessoa que se veste de maneira afetada; janota.}{pe.ral.ta}{0}
\verb{peraltice}{}{}{}{}{s.f.}{Qualidade de peralta.}{pe.ral.ti.ce}{0}
\verb{perambeira}{ê}{Bras.}{}{}{s.f.}{Abismo, precipício, despenhadeiro.}{pe.ram.bei.ra}{0}
\verb{perambular}{}{}{}{}{v.t.}{Andar sem destino; vaguear.}{pe.ram.bu.lar}{\verboinum{1}}
\verb{perante}{}{}{}{}{prep.}{Na presença de; diante de; ante.}{pe.ran.te}{0}
\verb{pé"-rapado}{}{Pop.}{pés"-rapados}{}{s.2g.}{Indivíduo sem dinheiro, de condição humilde; pobretão.}{pé"-ra.pa.do}{0}
\verb{perau}{}{}{}{}{s.m.}{Terreno íngreme; precipício.}{pe.rau}{0}
\verb{percal}{}{}{}{}{s.m.}{Tipo de tecido de algodão, fino e macio.}{per.cal}{0}
\verb{percalço}{}{}{}{}{s.m.}{Dificuldade, incômodo, transtorno.}{per.cal.ço}{0}
\verb{percalina}{}{}{}{}{s.f.}{Tecido forte de algodão, muito usado em encadernação.}{per.ca.li.na}{0}
\verb{perceber}{ê}{}{}{}{v.t.}{Captar através dos sentidos.}{per.ce.ber}{0}
\verb{perceber}{ê}{}{}{}{}{Distinguir (visual ou auditivamente) de maneira clara.}{per.ce.ber}{0}
\verb{perceber}{ê}{}{}{}{}{Dar"-se conta da existência de; notar.}{per.ce.ber}{0}
\verb{perceber}{ê}{}{}{}{}{Entender, compreender.}{per.ce.ber}{0}
\verb{perceber}{ê}{}{}{}{}{Receber (ordenado, pagamento, rendimento).}{per.ce.ber}{\verboinum{12}}
\verb{percebimento}{}{}{}{}{s.m.}{Ato de perceber.}{per.ce.bi.men.to}{0}
\verb{percentagem}{}{}{}{}{}{Var. de \textit{porcentagem}.}{per.cen.ta.gem}{0}
\verb{percentual}{}{}{"-ais}{}{adj.2g.}{Relativo a porcentagem.}{per.cen.tu.al}{0}
\verb{percentual}{}{}{"-ais}{}{s.m.}{Porcentagem, taxa, alíquota.}{per.cen.tu.al}{0}
\verb{percepção}{}{}{"-ões}{}{s.f.}{Ato, efeito ou capacidade de perceber.}{per.cep.ção}{0}
\verb{perceptível}{}{}{"-eis}{}{adj.2g.}{Que se pode perceber.}{per.cep.tí.vel}{0}
\verb{perceptivo}{}{}{}{}{adj.}{Relativo a percepção.}{per.cep.ti.vo}{0}
\verb{perceptivo}{}{}{}{}{}{Que tem capacidade de perceber.}{per.cep.ti.vo}{0}
\verb{percevejo}{ê}{}{}{}{s.m.}{Prego fino, curto e com cabeça larga usado para fixar papéis em superfícies apropriadas.}{per.ce.ve.jo}{0}
\verb{percevejo}{ê}{Zool.}{}{}{}{Inseto parasita que se alimenta de sangue ou de plantas.}{per.ce.ve.jo}{0}
\verb{percorrer}{ê}{}{}{}{v.t.}{Passar, andar através de.}{per.cor.rer}{0}
\verb{percorrer}{ê}{}{}{}{}{Explorar, pesquisar.}{per.cor.rer}{\verboinum{12}}
\verb{percuciente}{}{}{}{}{adj.2g.}{Que percute.}{per.cu.ci.en.te}{0}
\verb{percuciente}{}{}{}{}{}{Penetrante, agudo.}{per.cu.ci.en.te}{0}
\verb{percurso}{}{}{}{}{s.m.}{Ato ou efeito de percorrer.}{per.cur.so}{0}
\verb{percurso}{}{}{}{}{}{O local percorrido ou a ser percorrido; trajeto, caminho.}{per.cur.so}{0}
\verb{percussão}{}{}{"-ões}{}{s.f.}{Ato ou efeito de percutir.}{per.cus.são}{0}
\verb{percussão}{}{}{"-ões}{}{}{Choque entre dois corpos.}{per.cus.são}{0}
\verb{percussão}{}{Mús.}{"-ões}{}{}{Conjunto dos instrumentos de percussão, que formam a seção rítmica de banda ou orquestra.}{per.cus.são}{0}
\verb{percussionista}{}{Mús.}{}{}{s.2g.}{Indivíduo que toca instrumentos de percussão.}{per.cus.si.o.nis.ta}{0}
\verb{percutir}{}{}{}{}{v.t.}{Bater, chocar.}{per.cu.tir}{0}
\verb{percutir}{}{}{}{}{}{Repercutir.}{per.cu.tir}{\verboinum{18}}
\verb{perda}{ê}{}{}{}{s.f.}{Ato ou efeito de perder.}{per.da}{0}
\verb{perda}{ê}{}{}{}{}{Sumiço, extravio.}{per.da}{0}
\verb{perda}{ê}{}{}{}{}{Destruição total ou parcial de bens materiais.}{per.da}{0}
\verb{perdão}{}{}{"-ões}{}{s.m.}{Remissão de falta; desculpa.}{per.dão}{0}
\verb{perde"-ganha}{é}{}{}{}{s.m.}{Jogo em que ganha aquele que faz menos pontos ou que primeiro perde.}{per.de"-ga.nha}{0}
\verb{perder}{ê}{}{}{}{v.t.}{Deixar de ter algo que se possuía; ficar privado.}{per.der}{0}
\verb{perder}{ê}{}{}{}{}{Deixar cair no esquecimento; esquecer.}{per.der}{0}
\verb{perder}{ê}{}{}{}{}{Deixar de obter vitória ou de ganhar algo; fracassar.}{per.der}{0}
\verb{perder}{ê}{}{}{}{v.pron.}{Cair na perdição; desgraçar"-se.}{per.der}{\verboinum{52}}
\verb{perdição}{}{}{"-ões}{}{s.f.}{Ato ou efeito de perder.}{per.di.ção}{0}
\verb{perdição}{}{}{"-ões}{}{}{Desgraça, desastre, estrago, ruína, danação.}{per.di.ção}{0}
\verb{perdição}{}{}{"-ões}{}{}{Desregramento, imoralidade.}{per.di.ção}{0}
\verb{perdido}{}{}{}{}{adj.}{Desaparecido, sumido, extraviado.}{per.di.do}{0}
\verb{perdido}{}{}{}{}{}{Pervertido, imoral, devasso.}{per.di.do}{0}
\verb{perdido}{}{}{}{}{}{Irrecuperável, inutilizado, destruído.}{per.di.do}{0}
\verb{perdido}{}{}{}{}{}{Longínquo, distante.}{per.di.do}{0}
\verb{perdigão}{}{Zool.}{"-ões}{}{s.m.}{Espécie semelhante e maior que a perdiz.}{per.di.gão}{0}
\verb{perdigão}{}{}{"-ões}{}{}{Macho de perdiz.}{per.di.gão}{0}
\verb{perdigoto}{ô}{}{}{}{s.m.}{Filhote de perdiz.}{per.di.go.to}{0}
\verb{perdigoto}{ô}{Bras.}{}{}{}{Gotícula de saliva lançada durante a fala.}{per.di.go.to}{0}
\verb{perdigueiro}{ê}{}{}{}{adj.}{Diz"-se de cão treinado para caçar perdizes.}{per.di.guei.ro}{0}
\verb{perdiz}{}{Zool.}{}{}{s.f.}{Ave de plumagem parda e bico forte encontrada em campos e cerrados.}{per.diz}{0}
\verb{perdoar}{}{}{}{}{v.t.}{Dar o perdão a; desculpar.}{per.do.ar}{\verboinum{7}}
\verb{perdoável}{}{}{"-eis}{}{adj.2g.}{Que pode ser perdoado.}{per.do.á.vel}{0}
\verb{perdulário}{}{}{}{}{adj.}{Gastador, esbanjador.}{per.du.lá.rio}{0}
\verb{perdurar}{}{}{}{}{v.i.}{Durar bastante tempo; persistir.}{per.du.rar}{\verboinum{1}}
\verb{pereba}{é}{Bras.}{}{}{s.f.}{Pequena lesão na pele.}{pe.re.ba}{0}
\verb{perebento}{}{Bras.}{}{}{adj.}{Que tem muitas perebas.}{pe.re.ben.to}{0}
\verb{perecedouro}{ô}{}{}{}{adj.}{Que não dura para sempre; mortal, efêmero.}{pe.re.ce.dou.ro}{0}
\verb{perecer}{ê}{}{}{}{v.i.}{Ter fim; acabar, findar, morrer.}{pe.re.cer}{\verboinum{15}}
\verb{perecimento}{}{}{}{}{s.m.}{Ato de perecer; extinção, esgotamento.}{pe.re.ci.men.to}{0}
\verb{perecível}{}{}{"-eis}{}{adj.2g.}{Que pode perecer.}{pe.re.cí.vel}{0}
\verb{perecível}{}{}{"-eis}{}{}{Diz"-se de alimento que se deteriora com facilidade.}{pe.re.cí.vel}{0}
\verb{peregrinação}{}{}{"-ões}{}{s.f.}{Ato de peregrinar.}{pe.re.gri.na.ção}{0}
\verb{peregrinação}{}{}{"-ões}{}{}{Viagem a lugares considerados santos; romaria.}{pe.re.gri.na.ção}{0}
\verb{peregrinar}{}{}{}{}{v.i.}{Caminhar através de uma região.}{pe.re.gri.nar}{0}
\verb{peregrinar}{}{}{}{}{}{Percorrer ou visitar em caráter de romaria localidades consideradas santas.}{pe.re.gri.nar}{\verboinum{1}}
\verb{peregrino}{}{}{}{}{adj.}{Que peregrina.}{pe.re.gri.no}{0}
\verb{peregrino}{}{}{}{}{}{Estrangeiro, estranho, forasteiro.}{pe.re.gri.no}{0}
\verb{peregrino}{}{}{}{}{}{Extraordinário, excepcional, raro.}{pe.re.gri.no}{0}
\verb{pereira}{ê}{Bot.}{}{}{s.f.}{Árvore de flores brancas, frutos comestíveis e madeira utilizada em trabalhos artesanais.}{pe.rei.ra}{0}
\verb{pereiral}{}{}{"-ais}{}{s.m.}{Aglomerado de pereiras.}{pe.rei.ral}{0}
\verb{peremptório}{}{}{}{}{adj.}{Decisivo, categórico, terminante.}{pe.remp.tó.rio}{0}
\verb{perene}{}{}{}{}{adj.}{Que dura por muito tempo.}{pe.re.ne}{0}
\verb{perene}{}{}{}{}{}{Permanente, perpétuo, eterno.}{pe.re.ne}{0}
\verb{perene}{}{}{}{}{}{Ininterrupto, contínuo.}{pe.re.ne}{0}
\verb{perenizar}{}{}{}{}{v.t.}{Tornar perene, duradouro.}{pe.re.ni.zar}{\verboinum{1}}
\verb{perequeté}{}{}{}{}{adj.2g.}{Diz"-se de indivíduo emperiquitado, faceiro, elegante.}{pe.re.que.té}{0}
\verb{perereca}{é}{Zool.}{}{}{s.f.}{Nome comum a vários anfíbios anuros de pele lisa, encontrados na água ou em lugares pantanosos, de larga distribuição geográfica, e cujos membros posteriores são mais longos que o dos sapos, adaptados para saltar; jia, rã. }{pe.re.re.ca}{0}
\verb{pererecar}{}{}{}{}{v.i.}{Andar de um lado para outro.}{pe.re.re.car}{\verboinum{2}}
\verb{perfazer}{ê}{}{}{}{v.t.}{Chegar a determinada quantidade; atingir, completar, inteirar.}{per.fa.zer}{0}
\verb{perfazer}{ê}{}{}{}{}{Fazer alguma coisa até o fim; acabar, concluir.}{per.fa.zer}{\verboinum{42}}
\verb{perfeccionismo}{}{}{}{}{s.m.}{Busca obstinada da perfeição.}{per.fec.ci.o.nis.mo}{0}
\verb{perfeccionista}{}{}{}{}{adj.2g.}{Diz"-se de pessoa que tem ou denota perfeccionismo.}{per.fec.ci.o.nis.ta}{0}
\verb{perfectível}{}{}{"-eis}{}{adj.2g.}{Que pode ser aperfeiçoado.}{per.fec.tí.vel}{0}
\verb{perfeição}{}{}{"-ões}{}{s.f.}{O conjunto de todas as qualidades; a ausência de quaisquer defeitos.}{per.fei.ção}{0}
\verb{perfeito}{ê}{}{}{}{adj.}{Que reúne todas as qualidades concebíveis.}{per.fei.to}{0}
\verb{perfeito}{ê}{Gram.}{}{}{}{Diz"-se de tempo verbal que indica uma ação acabada.}{per.fei.to}{0}
\verb{perfídia}{}{}{}{}{s.f.}{Ação ou qualidade de pérfido; deslealdade, traição.}{per.fí.dia}{0}
\verb{pérfido}{}{}{}{}{adj.}{Que é desleal, traidor.}{pér.fi.do}{0}
\verb{perfil}{}{}{"-is}{}{s.m.}{Contorno do rosto ou de objeto visto de lado.}{per.fil}{0}
\verb{perfil}{}{}{"-is}{}{}{Conjunto das qualidades principais de pessoa ou coisa.}{per.fil}{0}
\verb{perfilar}{}{}{}{}{v.t.}{Colocar em fila; alinhar.}{per.fi.lar}{0}
\verb{perfilar}{}{}{}{}{}{Endireitar o corpo; aprumar.}{per.fi.lar}{\verboinum{1}}
\verb{perfilhação}{}{}{"-ões}{}{s.f.}{Reconhecimento da paternidade de alguém.}{per.fi.lha.ção}{0}
\verb{perfilhar}{}{}{}{}{v.t.}{Assumir legalmente como filho.}{per.fi.lhar}{\verboinum{1}}
\verb{performance}{}{}{}{}{s.f.}{Atuação, desempenho especialmente em público.}{per.for.man.ce}{0}
\verb{perfumado}{}{}{}{}{adj.}{Que tem ou exala perfume; cheiroso.}{per.fu.ma.do}{0}
\verb{perfumar}{}{}{}{}{v.t.}{Aplicar perfume em.}{per.fu.mar}{\verboinum{1}}
\verb{perfumaria}{}{}{}{}{s.f.}{Lugar em que se fabrica ou se vende perfume.}{per.fu.ma.ri.a}{0}
\verb{perfume}{}{}{}{}{s.m.}{Cheiro agradável; fragrância.}{per.fu.me}{0}
\verb{perfume}{}{}{}{}{}{Produto cheiroso e geralmente líquido para perfumar.}{per.fu.me}{0}
\verb{perfumista}{}{}{}{}{s.2g.}{Indivíduo que fabrica ou vende perfumes.}{per.fu.mis.ta}{0}
\verb{perfunctório}{}{}{}{}{adj.}{Que feito por rotina, obrigação.}{per.func.tó.rio}{0}
\verb{perfunctório}{}{}{}{}{}{Superficial.}{per.func.tó.rio}{0}
\verb{perfuntório}{}{}{}{}{}{Var. de \textit{perfunctório}.}{per.fun.tó.rio}{0}
\verb{perfuração}{}{}{"-ões}{}{s.f.}{Ato ou efeito de perfurar.}{per.fu.ra.ção}{0}
\verb{perfuração}{}{}{"-ões}{}{}{Furo praticado por alguém ou por algo que fura.}{per.fu.ra.ção}{0}
\verb{perfurador}{ô}{}{}{}{adj.}{Que perfura ou é próprio para perfurar.}{per.fu.ra.dor}{0}
\verb{perfuradora}{ô}{}{}{}{s.f.}{Máquina para abrir perfurações em cartões, fichas etc.}{per.fu.ra.do.ra}{0}
\verb{perfurar}{}{}{}{}{v.t.}{Fazer furo em; penetrar, furar.}{per.fu.rar}{\verboinum{1}}
\verb{perfuratriz}{}{}{}{}{s.f.}{Máquina dotada de broca destinada a fazer perfurações no solo.}{per.fu.ra.triz}{0}
\verb{pergamináceo}{}{}{}{}{adj.}{Que apresenta aspecto semelhante ao do pergaminho.}{per.ga.mi.ná.ceo}{0}
\verb{pergamináceo}{}{}{}{}{}{Que é feito de pergaminho.}{per.ga.mi.ná.ceo}{0}
\verb{pergaminho}{}{}{}{}{s.m.}{Pele, geralmente de ovelha, preparada e usada como papel.}{per.ga.mi.nho}{0}
\verb{pergaminho}{}{}{}{}{}{Manuscrito nela feito.}{per.ga.mi.nho}{0}
\verb{pérgula}{}{}{}{}{s.f.}{Abrigo ou caramanchão, de madeira ou alvenaria, nos jardins ou pátios para servir de suporte a trepadeiras.}{pér.gu.la}{0}
\verb{pergunta}{}{}{}{}{s.f.}{Palavra ou frase que busca resposta, esclarecimento.}{per.gun.ta}{0}
\verb{perguntar}{}{}{}{}{v.t.}{Fazer pergunta; indagar, interrogar.}{per.gun.tar}{\verboinum{1}}
\verb{perianto}{}{Bot.}{}{}{s.m.}{Conjunto de verticilos protetores da flor, formado pelo cálice e pela corola.}{pe.ri.an.to}{0}
\verb{pericárdio}{}{Anat.}{}{}{s.m.}{Membrana serosa que envolve externamente o coração.}{pe.ri.cár.dio}{0}
\verb{pericarpo}{}{}{}{}{s.m.}{O fruto em si, com exclusão das sementes.}{pe.ri.car.po}{0}
\verb{perícia}{}{}{}{}{s.f.}{Agilidade com que se faz alguma coisa; habilidade, destreza.}{pe.rí.cia}{0}
\verb{perícia}{}{}{}{}{}{Vistoria especializada.}{pe.rí.cia}{0}
\verb{pericial}{}{}{"-ais}{}{adj.2g.}{Relativo à perícia.}{pe.ri.ci.al}{0}
\verb{pericial}{}{}{"-ais}{}{}{Que é feito ou apresentado pela perícia.}{pe.ri.ci.al}{0}
\verb{periclitante}{}{}{}{}{adj.2g.}{Que corre perigo.}{pe.ri.cli.tan.te}{0}
\verb{periclitar}{}{}{}{}{v.i.}{Correr perigo; estar em perigo.}{pe.ri.cli.tar}{0}
\verb{periculosidade}{}{}{}{}{s.f.}{Qualidade de perigoso.}{pe.ri.cu.lo.si.da.de}{0}
\verb{periculosidade}{}{}{}{}{}{Propensão para o crime.}{pe.ri.cu.lo.si.da.de}{0}
\verb{periecos}{}{}{}{}{s.m.pl.}{Habitantes da Terra que vivem no mesmo paralelo de latitude, mas em longitudes que diferem 12 horas entre si .}{pe.ri.e.cos}{0}
\verb{periélio}{}{Astron.}{}{}{s.m.}{Ponto em que um planeta, em sua translação, chega mais perto do Sol.}{pe.ri.é.lio}{0}
\verb{periferia}{}{}{}{}{s.f.}{Linha que delimita qualquer corpo ou superfície.}{pe.ri.fe.ri.a}{0}
\verb{periferia}{}{}{}{}{}{Subúrbio.}{pe.ri.fe.ri.a}{0}
\verb{periférico}{}{}{}{}{adj.}{Que se situa na periferia.}{pe.ri.fé.ri.co}{0}
\verb{periférico}{}{}{}{}{s.m.}{Equipamento que se liga ao computador.}{pe.ri.fé.ri.co}{0}
\verb{perífrase}{}{}{}{}{s.f.}{Frase que exprime aquilo que poderia ser expresso por menor número de palavras; rodeio de palavras; circunlóquio.}{pe.rí.fra.se}{0}
\verb{perifrasear}{}{}{}{}{v.t.}{Trocar uma palavra por um conjunto de palavras com o mesmo sentido.}{pe.ri.fra.se.ar}{\verboinum{4}}
\verb{perifrástico}{}{}{}{}{adj.}{Relativo ou expresso por perífrase.}{pe.ri.frás.ti.co}{0}
\verb{perigar}{}{}{}{}{v.i.}{Correr perigo.}{pe.ri.gar}{\verboinum{5}}
\verb{perigeu}{}{Astron.}{}{}{s.m.}{Ponto em que um astro que gravita em torno da Terra chega mais perto dela.}{pe.ri.geu}{0}
\verb{perigo}{}{}{}{}{s.m.}{Situação em que algo ruim pode acontecer.}{pe.ri.go}{0}
\verb{perigo}{}{}{}{}{}{O que a provoca.}{pe.ri.go}{0}
\verb{perigoso}{ô}{}{"-osos ⟨ó⟩}{"-osa ⟨ó⟩}{adj.}{Em que há perigo; arriscado.}{pe.ri.go.so}{0}
\verb{perigoso}{ô}{}{"-osos ⟨ó⟩}{"-osa ⟨ó⟩}{}{Que causa ou ameaça perigo.}{pe.ri.go.so}{0}
\verb{perímetro}{}{}{}{}{}{Linha que delimita uma área ou região.}{pe.rí.me.tro}{0}
\verb{perímetro}{}{Geom.}{}{}{s.m.}{Soma dos lados de um figura geométrica.}{pe.rí.me.tro}{0}
\verb{períneo}{}{Anat.}{}{}{s.m.}{Espaço entre o ânus e os órgãos sexuais.}{pe.rí.neo}{0}
\verb{periodicidade}{}{}{}{}{s.f.}{Qualidade do que é periódico.}{pe.ri.o.di.ci.da.de}{0}
\verb{periódico}{}{}{}{}{adj.}{Que se repete com intervalos regulares.}{pe.ri.ó.di.co}{0}
\verb{periódico}{}{}{}{}{s.m.}{Publicação que aparece em intervalos fixos ou regulares.}{pe.ri.ó.di.co}{0}
\verb{periodismo}{}{}{}{}{s.m.}{Estado daquilo que é sujeito a movimentos periódicos.}{pe.ri.o.dis.mo}{0}
\verb{periodismo}{}{}{}{}{}{Jornalismo.}{pe.ri.o.dis.mo}{0}
\verb{periodista}{}{}{}{}{s.2g.}{Jornalista de periódico.}{pe.ri.o.dis.ta}{0}
\verb{período}{}{}{}{}{s.m.}{Espaço de tempo; época, fase.}{pe.rí.o.do}{0}
\verb{periodontite}{}{Med.}{}{}{s.f.}{Inflamação da gengiva.}{pe.ri.o.don.ti.te}{0}
\verb{periósteo}{}{Anat.}{}{}{s.m.}{Membrana que reveste exteriormente os ossos.}{pe.ri.ós.teo}{0}
\verb{peripatético}{}{}{}{}{adj.}{Relativo ao filósofo grego Aristóteles ou a sua doutrina; aristotélico.}{pe.ri.pa.té.ti.co}{0}
\verb{peripécia}{}{}{}{}{s.f.}{Fato imprevisto; incidente, aventura.}{pe.ri.pé.cia}{0}
\verb{périplo}{}{}{}{}{s.m.}{Viagem de navegação em torno de um país ou de um continente.}{pé.ri.plo}{0}
\verb{periquito}{}{Zool.}{}{}{s.m.}{Pequeno pássaro da família dos papagaios, de coloração verde e parte das asas amareladas e azuis.}{pe.ri.qui.to}{0}
\verb{periscópio}{}{}{}{}{s.m.}{Aparelho óptico, empregado especialmente em submarinos, que permite observar um objeto por cima de obstáculos.}{pe.ris.có.pio}{0}
\verb{perissodáctilo}{}{Zool.}{}{}{s.m.}{Ordem de mamíferos de estômago simples e membros providos de um ou três dedos, como os cavalos, rinocerontes e antas.}{pe.ris.so.dác.ti.lo}{0}
\verb{peristalse}{}{Biol.}{}{}{s.f.}{Peristaltismo.}{pe.ris.tal.se}{0}
\verb{peristaltismo}{}{Biol.}{}{}{s.m.}{Conjunto de movimentos musculares involutários.}{pe.ris.tal.tis.mo}{0}
\verb{peristilo}{}{}{}{}{s.m.}{Galeria formada de colunas isoladas que circundam um edifício ou um pátio.}{pe.ris.ti.lo}{0}
\verb{perito}{}{}{}{}{adj.}{Especialista em determinado assunto.}{pe.ri.to}{0}
\verb{peritônio}{}{Anat.}{}{}{s.m.}{Membrana serosa que reveste as paredes do abdome e os órgãos que nele se encontram.}{pe.ri.tô.nio}{0}
\verb{peritonite}{}{Med.}{}{}{s.f.}{Inflamação do peritônio.}{pe.ri.to.ni.te}{0}
\verb{perjurar}{}{}{}{}{v.i.}{Quebrar juramento.}{per.ju.rar}{0}
\verb{perjurar}{}{}{}{}{}{Jurar falso.}{per.ju.rar}{\verboinum{1}}
\verb{perjúrio}{}{}{}{}{s.m.}{Ato ou efeito de perjurar; juramento falso.}{per.jú.rio}{0}
\verb{perjuro}{}{}{}{}{adj.}{Diz"-se daquele que falta ao juramento feito.}{per.ju.ro}{0}
\verb{perlongar}{}{}{}{}{v.t.}{Percorrer ao longo de; costear.}{per.lon.gar}{\verboinum{5}}
\verb{perlustrar}{}{}{}{}{v.t.}{Percorrer com a vista, examinando, observando.}{per.lus.trar}{\verboinum{1}}
\verb{permanecer}{ê}{}{}{}{v.pred.}{Continuar a ser, a existir; conservar"-se.}{per.ma.ne.cer}{0}
\verb{permanecer}{ê}{}{}{}{v.t.}{Continuar a estar; ficar, demorar"-se.}{per.ma.ne.cer}{\verboinum{15}}
\verb{permanência}{}{}{}{}{s.f.}{Ato ou efeito de permanecer.}{per.ma.nên.cia}{0}
\verb{permanência}{}{}{}{}{}{Estado de permanente; constância, firmeza, estabilidade.}{per.ma.nên.cia}{0}
\verb{permanente}{}{}{}{}{adj.2g.}{Que permanece; duradouro, constante, estável, contínuo.}{per.ma.nen.te}{0}
\verb{permanente}{}{}{}{}{s.m.}{Ondulação artificial do cabelo, de duração um tanto longa.}{per.ma.nen.te}{0}
\verb{permeabilidade}{}{}{}{}{s.f.}{Qualidade do que é permeável.}{per.me.a.bi.li.da.de}{0}
\verb{permeabilizar}{}{}{}{}{v.t.}{Tornar permeável.}{per.me.a.bi.li.zar}{\verboinum{1}}
\verb{permear}{}{}{}{}{v.t.}{Fazer passar pelo meio; atravessar, penetrar.}{per.me.ar}{0}
\verb{permear}{}{}{}{}{v.i.}{Estar de permeio; interpor"-se.}{per.me.ar}{\verboinum{4}}
\verb{permeável}{}{}{"-eis}{}{adj.2g.}{Que se pode atravessar, permear.}{per.me.á.vel}{0}
\verb{permeio}{ê}{}{}{}{loc. adv.}{(\textit{de permeio}) No meio de; de mistura.}{per.mei.o}{0}
\verb{permiano}{}{Geol.}{}{}{adj.}{Diz"-se do último período da Era Paleozoica, que se segue ao período carbonífero.}{per.mi.a.no}{0}
\verb{permissão}{}{}{"-ões}{}{s.f.}{Ato ou efeito de permitir; autorização, consentimento.}{per.mis.são}{0}
\verb{permissível}{}{}{"-eis}{}{adj.2g.}{Que se pode permitir; admissível, aceitável.}{per.mis.sí.vel}{0}
\verb{permissivo}{}{}{}{}{adj.}{Que permite; tolerante, indulgente.}{per.mis.si.vo}{0}
\verb{permitir}{}{}{}{}{v.t.}{Dar licença para; consentir, autorizar.}{per.mi.tir}{0}
\verb{permitir}{}{}{}{}{}{Tolerar, admitir.}{per.mi.tir}{\verboinum{18}}
\verb{permuta}{}{}{}{}{s.f.}{Ato ou efeito de permutar; troca recíproca; câmbio.}{per.mu.ta}{0}
\verb{permutação}{}{}{"-ões}{}{s.f.}{Permuta.}{per.mu.ta.ção}{0}
\verb{permutar}{}{}{}{}{v.t.}{Trocar uma coisa por outra; cambiar.}{per.mu.tar}{\verboinum{1}}
\verb{permutável}{}{}{"-eis}{}{adj.2g.}{Que se pode permutar, trocar; cambiável.}{per.mu.tá.vel}{0}
\verb{perna}{é}{Anat.}{}{}{s.f.}{Cada um dos membros inferiores do corpo, situado entre o joelho e o pé.}{per.na}{0}
\verb{perna}{é}{}{}{}{}{Cada um dos membros inferiores dos animais vertebrados terrestres; pata.}{per.na}{0}
\verb{perna}{é}{}{}{}{}{Peça que serve de apoio a um objeto.}{per.na}{0}
\verb{pernada}{}{}{}{}{s.f.}{Passo ou passada larga.}{per.na.da}{0}
\verb{pernada}{}{}{}{}{}{Caminhada longa e difícil.}{per.na.da}{0}
\verb{pernada}{}{}{}{}{}{Golpe dado com a perna; coice, pontapé.}{per.na.da}{0}
\verb{perna"-de"-pau}{}{}{pernas"-de"-pau}{}{adj.}{Diz"-se do indivíduo a quem falta uma perna; perneta.}{per.na"-de"-pau}{0}
\verb{perna"-de"-pau}{}{}{pernas"-de"-pau}{}{}{Diz"-se do jogador sem habilidade.}{per.na"-de"-pau}{0}
\verb{pernalta}{}{Zool.}{}{}{adj.2g.}{Diz"-se das aves que possuem pernas longas, como o tuiuti, a garça etc.}{per.nal.ta}{0}
\verb{pernalto}{}{}{}{}{adj.}{Que possui pernas altas, compridas.}{per.nal.to}{0}
\verb{pernambucano}{}{}{}{}{adj.}{Relativo a Pernambuco.}{per.nam.bu.ca.no}{0}
\verb{pernambucano}{}{}{}{}{s.m.}{Indivíduo natural ou habitante desse estado.}{per.nam.bu.ca.no}{0}
\verb{perneiras}{}{}{}{}{s.f.pl.}{Peças de couro usadas para proteger as pernas.}{per.nei.ras}{0}
\verb{perneta}{ê}{}{}{}{adj.2g.}{Diz"-se do indivíduo a quem falta uma perna; perna"-de"-pau.}{per.ne.ta}{0}
\verb{pernicioso}{ô}{}{"-osos ⟨ó⟩}{"-osa ⟨ó⟩}{adj.}{Que faz mal; nocivo, prejudicial.}{per.ni.ci.o.so}{0}
\verb{pernil}{}{}{"-is}{}{s.m.}{Coxa comestível de porco ou de outros animais.}{per.nil}{0}
\verb{pernilongo}{}{Zool.}{}{}{s.m.}{Pequeno inseto sugador de sangue, de corpo fino e pernas longas, que vive próximo da água.}{per.ni.lon.go}{0}
\verb{pernoitar}{}{}{}{}{v.i.}{Passar a noite em algum lugar.}{per.noi.tar}{\verboinum{1}}
\verb{pernoite}{}{}{}{}{s.m.}{Ato ou efeito de pernoitar, de passar a noite.}{per.noi.te}{0}
\verb{pernóstico}{}{}{}{}{adj.}{Que gosta de empregar termos inusuais ao falar ou escrever; presunçoso, afetado.}{per.nós.ti.co}{0}
\verb{peroba}{ó}{Bot.}{}{}{s.f.}{Árvore que possui madeira de boa qualidade, usada em construções.}{pe.ro.ba}{0}
\verb{pérola}{}{}{}{}{s.f.}{Pequeno globo duro e brilhante que se forma na concha de certas ostras.}{pé.ro.la}{0}
\verb{pérola}{}{}{}{}{adj.}{Da cor da pérola; nacarado.}{pé.ro.la}{0}
\verb{perolar}{}{}{}{}{v.t.}{Enfeitar com pérolas.}{pe.ro.lar}{\verboinum{1}}
\verb{perôneo}{}{}{}{}{}{Var. de \textit{perônio}.}{pe.rô.neo}{0}
\verb{perônio}{}{Anat.}{}{}{s.m.}{Osso da perna, situado ao lado da tíbia.}{pe.rô.nio}{0}
\verb{peroração}{}{}{"-ões}{}{s.f.}{Parte final de um discurso; conclusão, epílogo.}{pe.ro.ra.ção}{0}
\verb{perorar}{}{}{}{}{v.i.}{Concluir um discurso.}{pe.ro.rar}{\verboinum{1}}
\verb{peróxido}{cs}{Quím.}{}{}{s.m.}{Composto de uma substância simples e uma grande quantidade de oxigênio.}{pe.ró.xi.do}{0}
\verb{perpassar}{}{}{}{}{v.t.}{Passar perto ou junto de.}{per.pas.sar}{0}
\verb{perpassar}{}{}{}{}{}{Passar de leve; roçar.}{per.pas.sar}{\verboinum{1}}
\verb{perpendicular}{}{Geom.}{}{}{adj.2g.}{Diz"-se de qualquer elemento geométrico que forma ângulo reto com outro.}{per.pen.di.cu.lar}{0}
\verb{perpetrar}{}{}{}{}{v.t.}{Praticar, realizar, cometer ato condenável.}{per.pe.trar}{\verboinum{1}}
\verb{perpétua}{}{Bot.}{}{}{s.f.}{Pequeno arbusto de folhas simples e flores de cores variadas, propriedades medicinais e cultivada como ornamental.}{per.pé.tu.a}{0}
\verb{perpetuação}{}{}{"-ões}{}{s.f.}{Ato ou efeito de perpetuar.}{per.pe.tu.a.ção}{0}
\verb{perpetuar}{}{}{}{}{v.t.}{Tornar perpétuo; eternizar.}{per.pe.tu.ar}{\verboinum{1}}
\verb{perpétuo}{}{}{}{}{adj.}{Que dura para sempre; permanente, perene, eterno.}{per.pé.tu.o}{0}
\verb{perpétuo}{}{}{}{}{}{Que dura até o fim da vida; vitalício.}{per.pé.tu.o}{0}
\verb{perplexidade}{cs}{}{}{}{s.f.}{Qualidade de perplexo.}{per.ple.xi.da.de}{0}
\verb{perplexo}{écs}{}{}{}{adj.}{Admirado, espantado.}{per.ple.xo}{0}
\verb{perplexo}{écs}{}{}{}{}{Hesitante, indeciso.}{per.ple.xo}{0}
\verb{perquirir}{}{}{}{}{v.t.}{Investigar, pesquisar.}{per.qui.rir}{\verboinum{18}}
\verb{perrengue}{}{}{}{}{adj.2g.}{Que mal pode andar; debilitado, capenga.}{per.ren.gue}{0}
\verb{perrengue}{}{}{}{}{}{Fraco, covarde.}{per.ren.gue}{0}
\verb{perro}{ê}{}{}{}{s.m.}{Cão.}{per.ro}{0}
\verb{perro}{ê}{}{}{}{}{Homem vil.}{per.ro}{0}
\verb{perro}{ê}{}{}{}{adj.}{Emperrado.}{per.ro}{0}
\verb{perro}{ê}{Fig.}{}{}{}{Teimoso, obstinado.}{per.ro}{0}
\verb{persa}{é}{}{}{}{adj.2g.}{Relativo à Pérsia, localidade da Ásia correspondente ao atual Irã.}{per.sa}{0}
\verb{persa}{é}{}{}{}{s.2g.}{Indivíduo natural ou habitante da Pérsia.}{per.sa}{0}
\verb{perscrutação}{}{}{"-ões}{}{s.f.}{Ato ou efeito de perscrutar.}{pers.cru.ta.ção}{0}
\verb{perscrutar}{}{}{}{}{v.t.}{Examinar ou observar minuciosamente.}{pers.cru.tar}{\verboinum{1}}
\verb{persecução}{}{}{"-ões}{}{s.f.}{Esforço em busca de um objetivo.}{per.se.cu.ção}{0}
\verb{persecução}{}{}{"-ões}{}{}{Perseguição.}{per.se.cu.ção}{0}
\verb{persecutório}{}{}{}{}{adj.}{Relativo a persecução.}{per.se.cu.tó.rio}{0}
\verb{perseguição}{}{}{"-ões}{}{s.f.}{Ato ou efeito de perseguir.}{per.se.gui.ção}{0}
\verb{perseguir}{}{}{}{}{v.t.}{Ir no encalço de; seguir.}{per.se.guir}{0}
\verb{perseguir}{}{}{}{}{}{Importunar, incomodar.}{per.se.guir}{0}
\verb{perseguir}{}{}{}{}{}{Atormentar, castigar, punir.}{per.se.guir}{0}
\verb{perseguir}{}{}{}{}{}{Buscar conquistar.}{per.se.guir}{\verboinum{24}}
\verb{perseverança}{}{}{}{}{s.f.}{Qualidade de perseverante; firmeza, constância.}{per.se.ve.ran.ça}{0}
\verb{perseverante}{}{}{}{}{adj.2g.}{Que persevera.}{per.se.ve.ran.te}{0}
\verb{perseverar}{}{}{}{}{v.t.}{Continuar, manter"-se, persistir.}{per.se.ve.rar}{\verboinum{1}}
\verb{persiana}{}{}{}{}{s.f.}{Conjunto de pequenas tábuas móveis dispostas paralelamente e que se coloca nas janelas para servir de anteparo contra luz excessiva.}{per.si.a.na}{0}
\verb{pérsico}{}{}{}{}{adj.}{Relativo à Pérsia; persa.}{pér.si.co}{0}
\verb{persignar"-se}{}{}{}{}{v.pron.}{Benzer"-se fazendo o sinal da cruz com o polegar sobre a testa, a boca e o peito.}{per.sig.nar"-se}{\verboinum{1}}
\verb{persistência}{}{}{}{}{s.f.}{Qualidade de persistente.}{per.sis.tên.cia}{0}
\verb{persistente}{}{}{}{}{adj.2g.}{Que persiste; constante.}{per.sis.ten.te}{0}
\verb{persistir}{}{}{}{}{v.t.}{Ser constante; continuar, insistir.}{per.sis.tir}{0}
\verb{persistir}{}{}{}{}{v.pred.}{Continuar, manter"-se, conservar.}{per.sis.tir}{\verboinum{18}}
\verb{personagem}{}{}{}{}{s.2g.}{Cada uma das pessoas que fazem parte de uma história, narrativa ou acontecimento.}{per.so.na.gem}{0}
\verb{personagem}{}{}{}{}{}{Pessoa notável.}{per.so.na.gem}{0}
\verb{personalidade}{}{}{}{}{s.f.}{Conjunto das características da maneira de ser de uma pessoa que a distinguem das outras.}{per.so.na.li.da.de}{0}
\verb{personalidade}{}{}{}{}{}{Qualidade de pessoal.}{per.so.na.li.da.de}{0}
\verb{personalismo}{}{}{}{}{s.m.}{Qualidade de pessoal.}{per.so.na.lis.mo}{0}
\verb{personalista}{}{}{}{}{adj.2g.}{Subjetivo, pessoal.}{per.so.na.lis.ta}{0}
\verb{personalista}{}{}{}{}{}{Egocêntrico.}{per.so.na.lis.ta}{0}
\verb{personalização}{}{}{"-ões}{}{s.f.}{Ato ou efeito de personalizar.}{per.so.na.li.za.ção}{0}
\verb{personalizar}{}{}{}{}{v.t.}{Atribuir qualidades pessoais a.}{per.so.na.li.zar}{0}
\verb{personalizar}{}{}{}{}{}{Dar caráter pessoal.}{per.so.na.li.zar}{0}
\verb{personalizar}{}{}{}{}{}{Organizar de maneira pessoal, para o uso próprio ou de uma pessoa específica.}{per.so.na.li.zar}{\verboinum{1}}
\verb{personificação}{}{}{"-ões}{}{s.f.}{Ato ou efeito de personificar.}{per.so.ni.fi.ca.ção}{0}
\verb{personificação}{}{}{"-ões}{}{}{Pessoa que representa fisicamente uma ideia ou uma coisa abstrata.}{per.so.ni.fi.ca.ção}{0}
\verb{personificação}{}{Gram.}{"-ões}{}{}{Prosopopeia.}{per.so.ni.fi.ca.ção}{0}
\verb{personificar}{}{}{}{}{v.t.}{Considerar ou representar como pessoa.}{per.so.ni.fi.car}{\verboinum{2}}
\verb{perspectiva}{}{}{}{}{s.f.}{Técnica de representação de objetos tridimensionais sobre um plano.}{pers.pec.ti.va}{0}
\verb{perspectiva}{}{}{}{}{}{Representação ou desenho que utiliza essas técnicas.}{pers.pec.ti.va}{0}
\verb{perspectiva}{}{}{}{}{}{Ponto de vista.}{pers.pec.ti.va}{0}
\verb{perspetiva}{}{}{}{}{}{Var. de \textit{perspectiva}.}{pers.pe.ti.va}{0}
\verb{perspicácia}{}{}{}{}{s.f.}{Qualidade de perspicaz; sagacidade.}{pers.pi.cá.cia}{0}
\verb{perspicaz}{}{}{}{}{adj.2g.}{Que observa bem.}{pers.pi.caz}{0}
\verb{perspicaz}{}{}{}{}{}{Que tem agudeza de espírito; sagaz.}{pers.pi.caz}{0}
\verb{perspicaz}{}{}{}{}{}{Inteligente, esperto.}{pers.pi.caz}{0}
\verb{perspícuo}{}{}{}{}{adj.}{Nítido, claro, evidente.}{pers.pí.cu.o}{0}
\verb{perspirar}{}{}{}{}{v.i.}{Transpirar.}{pers.pi.rar}{0}
\verb{perspirar}{}{}{}{}{}{Perceber através de indícios; entrever.}{pers.pi.rar}{\verboinum{1}}
\verb{persuadir}{}{}{}{}{v.t.}{Levar a aceitar; convencer, induzir.}{per.su.a.dir}{\verboinum{18}}
\verb{persuasão}{}{}{"-ões}{}{s.f.}{Ato ou efeito de persuadir.}{per.su.a.são}{0}
\verb{persuasivo}{}{}{}{}{adj.}{Que tem capacidade de persuadir.}{per.su.a.si.vo}{0}
\verb{pertence}{}{}{}{}{s.m.}{Declaração constante em certos títulos em que se indica a pessoa a quem se transmite sua propriedade.}{per.ten.ce}{0}
\verb{pertence}{}{Bras.}{}{}{}{Ingredientes de uma feijoada.}{per.ten.ce}{0}
\verb{pertencente}{}{}{}{}{adj.2g.}{Que pertence a alguém ou algo.}{per.ten.cen.te}{0}
\verb{pertencer}{ê}{}{}{}{v.t.}{Ser de propriedade de.}{per.ten.cer}{0}
\verb{pertencer}{ê}{}{}{}{}{Ser parte de; estar contido em.}{per.ten.cer}{0}
\verb{pertencer}{ê}{}{}{}{}{Referir"-se a; dizer respeito a; concernir.}{per.ten.cer}{\verboinum{15}}
\verb{pertences}{}{}{}{}{s.m.pl.}{Objetos de uso pessoal.}{per.ten.ces}{0}
\verb{pertinácia}{}{}{}{}{s.f.}{Qualidade de pertinaz.}{per.ti.ná.cia}{0}
\verb{pertinaz}{}{}{}{}{adj.2g.}{Persistente, obstinado, perseverante.}{per.ti.naz}{0}
\verb{pertinência}{}{}{}{}{s.f.}{Qualidade de pertinente.}{per.ti.nên.cia}{0}
\verb{pertinente}{}{}{}{}{adj.2g.}{Que pertence; pertencente.}{per.ti.nen.te}{0}
\verb{pertinente}{}{}{}{}{}{Relativo, concernente, respeitante.}{per.ti.nen.te}{0}
\verb{pertinente}{}{}{}{}{}{Que vem a propósito; apropriado, adequado, oportuno.}{per.ti.nen.te}{0}
\verb{perto}{é}{}{}{}{adv.}{A pouca distância; próximo.}{per.to}{0}
\verb{perturbação}{}{}{"-ões}{}{s.f.}{Ato ou efeito de perturbar.}{per.tur.ba.ção}{0}
\verb{perturbação}{}{}{"-ões}{}{}{Transtorno, confusão, desordem, distúrbio.}{per.tur.ba.ção}{0}
\verb{perturbado}{}{}{}{}{adj.}{Que se perturbou.}{per.tur.ba.do}{0}
\verb{perturbado}{}{}{}{}{}{Desvairado, alucinado, louco.}{per.tur.ba.do}{0}
\verb{perturbar}{}{}{}{}{v.t.}{Causar confusão, desordem.}{per.tur.bar}{0}
\verb{perturbar}{}{}{}{}{}{Tirar a tranquilidade do espírito; desnortear.}{per.tur.bar}{\verboinum{1}}
\verb{peru}{}{Zool.}{}{}{s.m.}{Ave galiforme de grande porte e plumagem escura, muito apreciada em culinária.}{pe.ru}{0}
\verb{perua}{}{}{}{}{s.f.}{Fêmea do peru.}{pe.ru.a}{0}
\verb{perua}{}{}{}{}{}{Veículo para passageiros e pequena carga.}{pe.ru.a}{0}
\verb{perua}{}{Pop.}{}{}{}{Mulher que se veste de maneira afetada, ainda que com alguma elegância.}{pe.ru.a}{0}
\verb{peruano}{}{}{}{}{adj.}{Relativo a Peru.}{pe.ru.a.no}{0}
\verb{peruano}{}{}{}{}{s.m.}{Indivíduo natural ou habitante desse país.}{pe.ru.a.no}{0}
\verb{peruar}{}{}{}{}{v.i.}{Assistir a jogo dando palpites, incomodando os participantes.}{pe.ru.ar}{0}
\verb{peruar}{}{}{}{}{}{Paquerar, cortejar.}{pe.ru.ar}{\verboinum{1}}
\verb{peruca}{}{}{}{}{s.f.}{Cabeleira postiça.}{pe.ru.ca}{0}
\verb{perueiro}{ê}{Bras.}{}{}{s.m.}{Motorista de perua, especialmente utilizada em transporte de passageiros.}{pe.ru.ei.ro}{0}
\verb{peruviano}{}{}{}{}{adj. e s.m.  }{Peruano.}{pe.ru.vi.a.no}{0}
\verb{pervagar}{}{}{}{}{v.t.}{Percorrer em diversas direções; atravessar, cruzar.}{per.va.gar}{0}
\verb{pervagar}{}{}{}{}{v.i.}{Andar a esmo e sem destino; vagar.}{per.va.gar}{\verboinum{5}}
\verb{perversão}{}{}{"-ões}{}{s.f.}{Ato ou efeito de perverter.}{per.ver.são}{0}
\verb{perversão}{}{}{"-ões}{}{}{Condição de corrupto, de devasso.}{per.ver.são}{0}
\verb{perversidade}{}{}{}{}{s.f.}{Qualidade de perverso.}{per.ver.si.dade}{0}
\verb{perversidade}{}{}{}{}{}{Índole ou caráter ruim; maldade.}{per.ver.si.dade}{0}
\verb{perverso}{}{}{}{}{adj.}{Que é cruel, malvado.}{per.ver.so}{0}
\verb{perverter}{ê}{}{}{}{v.t.}{Tornar perverso; depravar, corromper.}{per.ver.ter}{\verboinum{12}}
\verb{pervertido}{}{}{}{}{adj.}{Que se perverteu; depravado, desmoralizado, corrupto.}{per.ver.ti.do}{0}
\verb{pesada}{}{}{}{}{s.f.}{Ato de pesar; pesagem.}{pe.sa.da}{0}
\verb{pesada}{}{}{}{}{}{Aquilo que se pesa de uma vez numa balança.}{pe.sa.da}{0}
\verb{pesadão}{}{}{"-ões}{"-ona}{adj.}{Que é muito pesado.}{pe.sa.dão}{0}
\verb{pesadão}{}{}{"-ões}{"-ona}{}{Que é vagaroso, lerdo.}{pe.sa.dão}{0}
\verb{pesadelo}{ê}{}{}{}{s.m.}{Sonho muito ruim.}{pe.sa.de.lo}{0}
\verb{pesado}{}{}{}{}{adj.}{Que pesa muito.}{pe.sa.do}{0}
\verb{pesado}{}{}{}{}{}{Que é profundo.}{pe.sa.do}{0}
\verb{pesado}{}{}{}{}{}{Que cansa muito o corpo ou a mente.}{pe.sa.do}{0}
\verb{pesado}{}{}{}{}{}{Em que há perigo de aborrecimento ou complicação.}{pe.sa.do}{0}
\verb{pesado}{}{Pop.}{}{}{}{Que não tem sorte; azarado.}{pe.sa.do}{0}
\verb{pesagem}{}{}{"-ens}{}{s.f.}{Ato ou efeito de pesar.}{pe.sa.gem}{0}
\verb{pêsames}{}{}{}{}{s.m.pl.}{Expressão de condolência pela morte de alguém ou por alguma desgraça.}{pê.sa.mes}{0}
\verb{pesar}{}{}{}{}{v.t.}{Pôr na balança para saber o peso de. (\textit{O açougueiro pesou a carne que comprei.})}{pe.sar}{0}
\verb{pesar}{}{}{}{}{}{Examinar atentamente; avaliar, considerar.  (\textit{É preciso pesar os prós e os contras antes de tomar uma decisão.})}{pe.sar}{0}
\verb{pesar}{}{}{}{}{}{Ter certo peso. (\textit{Meu filho pesa 25 quilos.})}{pe.sar}{\verboinum{1}}
\verb{pesar}{}{}{}{}{s.m.}{Sentimento, tristeza, desgosto. (\textit{Os jogadores receberam o pesar dos amigos })}{pe.sar}{0}
\verb{pesaroso}{ô}{}{"-osos ⟨ó⟩}{"-osa ⟨ó⟩}{adj.}{Que tem pesar, ou em que há pesar.}{pe.sa.ro.so}{0}
\verb{pesca}{é}{}{}{}{s.f.}{Captura especialmente de peixes.}{pes.ca}{0}
\verb{pesca}{é}{}{}{}{}{Aquilo que se pescou.}{pes.ca}{0}
\verb{pescada}{}{Zool.}{}{}{s.f.}{Peixe de escamas prateadas, com uma mancha escura nas costas perto da nadadeira, e que possui esqueleto ósseo.}{pes.ca.da}{0}
\verb{pescado}{}{}{}{}{s.m.}{O que se pesca para comer.}{pes.ca.do}{0}
\verb{pescador}{ô}{}{}{}{s.m.}{Indivíduo que pesca.}{pes.ca.dor}{0}
\verb{pescar}{}{}{}{}{v.t.}{Capturar animal que vive na água.}{pes.car}{0}
\verb{pescar}{}{Pop.}{}{}{}{Apanhar o sentido de alguma coisa; entender, sacar.}{pes.car}{\verboinum{2}}
\verb{pescaria}{}{}{}{}{s.f.}{Ato de pescar; pesca.}{pes.ca.ri.a}{0}
\verb{pescoção}{}{}{"-ões}{}{s.m.}{Tapa, pancada, especialmente no pescoço.}{pes.co.ção}{0}
\verb{pescoço}{ô}{Anat.}{}{}{s.m.}{Região do corpo entre o tronco e a cabeça.}{pes.co.ço}{0}
\verb{pescoço}{ô}{}{}{}{}{Gargalo de garrafa, pote etc.}{pes.co.ço}{0}
\verb{peseta}{ê}{}{}{}{s.f.}{Moeda espanhola.}{pe.se.ta}{0}
\verb{peso}{ê}{}{}{}{s.m.}{Resultado da força da gravidade sobre os corpos.}{pe.so}{0}
\verb{peso}{ê}{}{}{}{}{Sólido usado para avaliar a massa de um corpo na balança.}{pe.so}{0}
\verb{peso}{ê}{}{}{}{}{Carga, fardo.}{pe.so}{0}
\verb{peso}{ê}{}{}{}{}{Cada uma das categorias do boxe.}{pe.so}{0}
\verb{peso}{ê}{}{}{}{}{Moeda dos países sul"-americanos.}{pe.so}{0}
\verb{pespegar}{}{}{}{}{v.t.}{Aplicar com energia.}{pes.pe.gar}{\verboinum{5}}
\verb{pespontar}{}{}{}{}{v.t.}{Fazer pesponto em.}{pes.pon.tar}{\verboinum{1}}
\verb{pesponto}{}{}{}{}{s.m.}{Ponto de costura em que a agulha entra atrás do lugar em que saiu.}{pes.pon.to}{0}
\verb{pesponto}{}{}{}{}{}{Costura externa feita à máquina com pontos graúdos.}{pes.pon.to}{0}
\verb{pesqueiro}{ê}{}{}{}{adj.}{Relativo a ou próprio para pesca.}{pes.quei.ro}{0}
\verb{pesquisa}{}{}{}{}{s.f.}{Ato ou efeito de pesquisar.}{pes.qui.sa}{0}
\verb{pesquisa}{}{}{}{}{}{Investigação artística, científica, literária etc.}{pes.qui.sa}{0}
\verb{pesquisador}{ô}{}{}{}{s.m.}{Indivíduo que pesquisa.}{pes.qui.sa.dor}{0}
\verb{pesquisar}{}{}{}{}{v.t.}{Procurar com aplicação, com diligência.}{pes.qui.sar}{0}
\verb{pesquisar}{}{}{}{}{}{Tomar informações a respeito de. }{pes.qui.sar}{\verboinum{1}}
\verb{pessegada}{}{}{}{}{s.f.}{Doce de pêssego.}{pes.se.ga.da}{0}
\verb{pêssego}{}{Bot.}{}{}{s.m.}{Fruta de polpa amarelada e casca aveludada.}{pês.se.go}{0}
\verb{pessegueiro}{ê}{Bot.}{}{}{s.m.}{Pequena árvore, nativa da China, de flores roxas, que produz o pêssego.}{pes.se.guei.ro}{0}
\verb{pessimismo}{}{}{}{}{s.m.}{Disposição de quem sempre espera pelo pior.}{pes.si.mis.mo}{0}
\verb{pessimista}{}{}{}{}{adj.2g.}{Que tem pessimismo.}{pes.si.mis.ta}{0}
\verb{péssimo}{}{}{}{}{adj.}{Muito mau.}{pés.si.mo}{0}
\verb{pessoa}{}{}{}{}{s.f.}{Cada um dos seres humanos.}{pes.so.a}{0}
\verb{pessoa}{}{Gram.}{}{}{}{Categoria linguística ligada especialmente a verbos e pronomes, que mostra a relação dos participantes do ato de fala com os participantes do acontecimento narrado.}{pes.so.a}{0}
\verb{pessoal}{}{}{"-ais}{}{adj.2g.}{Próprio de pessoa.}{pes.so.al}{0}
\verb{pessoal}{}{}{"-ais}{}{}{Que interessa a uma pessoa só; particular, privado.}{pes.so.al}{0}
\verb{pessoal}{}{}{"-ais}{}{}{Diz"-se de pronome que representa pessoa gramatical.}{pes.so.al}{0}
\verb{pessoal}{}{}{"-ais}{}{s.m.}{Grupo determinado de pessoas.}{pes.so.al}{0}
\verb{pessoalizar}{}{}{}{}{v.t.}{Representar por meio de uma pessoa; personificar.}{pes.so.a.li.zar}{\verboinum{1}}
\verb{pessoense}{}{}{}{}{adj.2g.}{Relativo a João Pessoa, capital da Paraíba.}{pes.so.en.se}{0}
\verb{pessoense}{}{}{}{}{s.2g.}{Indivíduo natural ou habitante dessa cidade.}{pes.so.en.se}{0}
\verb{pestana}{}{}{}{}{s.f.}{Cada um dos pelos das pálpebras; cílio.}{pes.ta.na}{0}
\verb{pestanejar}{}{}{}{}{v.i.}{Fechar e abrir os olhos; piscar.}{pes.ta.ne.jar}{\verboinum{1}}
\verb{pestanudo}{}{}{}{}{adj.}{Que tem pestanas grandes.}{pes.ta.nu.do}{0}
\verb{peste}{é}{}{}{}{s.f.}{Doença contagiosa transmitida pela pulga do rato.}{pes.te}{0}
\verb{peste}{é}{}{}{}{}{Qualquer epidemia mortal.}{pes.te}{0}
\verb{peste}{é}{}{}{}{s.2g.}{Indivíduo mau.}{pes.te}{0}
\verb{pestear}{}{}{}{}{v.t.}{Infectar com peste.}{pes.te.ar}{\verboinum{4}}
\verb{pesticida}{}{}{}{}{s.m.}{Veneno usado contra as pragas da lavoura.}{pes.ti.ci.da}{0}
\verb{pestífero}{}{}{}{}{adj.}{Que causa peste.}{pes.tí.fe.ro}{0}
\verb{pestífero}{}{Fig.}{}{}{}{Nocivo, danoso.}{pes.tí.fe.ro}{0}
\verb{pestífero}{}{}{}{}{}{Diz"-se de quem foi contaminado com peste.}{pes.tí.fe.ro}{0}
\verb{pestilencial}{}{}{"-ais}{}{adj.2g.}{Pestilento.}{pes.ti.len.ci.al}{0}
\verb{pestilento}{}{}{}{}{adj.}{Próprio da peste.}{pes.ti.len.to}{0}
\verb{pestilento}{}{}{}{}{}{Que transmite peste.}{pes.ti.len.to}{0}
\verb{pestilento}{}{}{}{}{}{Infectado de peste.}{pes.ti.len.to}{0}
\verb{pestilento}{}{}{}{}{}{Que produz exalações nocivas à saúde; fétido, infecto. }{pes.ti.len.to}{0}
\verb{peta}{ê}{}{}{}{s.f.}{Mentira, fraude.}{pe.ta}{0}
\verb{peta}{ê}{Cul.}{}{}{}{Variedade de bolo de mandioca.}{pe.ta}{0}
\verb{pétala}{}{Bot.}{}{}{s.f.}{Cada peça colorida presa ao centro da flor.}{pé.ta.la}{0}
\verb{petardo}{}{}{}{}{s.m.}{Explosivo portátil para destruir obstáculos.}{pe.tar.do}{0}
\verb{petardo}{}{}{}{}{}{Chute violento contra o gol.}{pe.tar.do}{0}
\verb{peteca}{é}{}{}{}{s.f.}{Espécie de pequena bola achatada e leve, com penas espetadas em forma de penacho, própria para ser jogada para cima com a palma das mãos.}{pe.te.ca}{0}
\verb{peteleco}{é}{}{}{}{s.m.}{Pancada com a ponta do dedo da mão.}{pe.te.le.co}{0}
\verb{petição}{}{}{"-ões}{}{s.f.}{Pedido feito por escrito a uma autoridade.}{pe.ti.ção}{0}
\verb{petiscador}{ô}{}{}{}{adj.}{Diz"-se de indivíduo que dado a petiscar; lambiscador.}{pe.tis.ca.dor}{0}
\verb{petiscar}{}{}{}{}{v.i.}{Comer petisco.}{pe.tis.car}{0}
\verb{petiscar}{}{}{}{}{}{Comer um pouco, provando ou saboreando; provar.}{pe.tis.car}{\verboinum{2}}
\verb{petisco}{}{}{}{}{s.m.}{Quitute, geralmente aperitivo.}{pe.tis.co}{0}
\verb{petisqueira}{ê}{}{}{}{s.f.}{Petisco.}{pe.tis.quei.ra}{0}
\verb{petisqueira}{ê}{}{}{}{}{Louça para servir petisco.}{pe.tis.quei.ra}{0}
\verb{petisqueira}{ê}{}{}{}{}{Estabelecimento comercial que serve refeições; restaurante.}{pe.tis.quei.ra}{0}
\verb{petiz}{}{Pop.}{}{}{s.m.}{Menino, garoto.}{pe.tiz}{0}
\verb{petizada}{}{Pop.}{}{}{s.f.}{Conjunto de petizes; garotada.}{pe.ti.za.da}{0}
\verb{petrechar}{}{}{}{}{}{Var. de \textit{apetrechar}.}{pe.tre.char}{0}
\verb{petrecho}{ê}{}{}{}{}{Var. de \textit{apetrecho}.}{pe.tre.cho}{0}
\verb{pétreo}{}{}{}{}{adj.}{De pedra.}{pé.treo}{0}
\verb{pétreo}{}{Fig.}{}{}{}{Insensível, desumano.}{pé.treo}{0}
\verb{petrificação}{}{}{"-ões}{}{s.f.}{Transformação de algo em pedra.}{pe.tri.fi.ca.ção}{0}
\verb{petrificação}{}{Fig.}{"-ões}{}{}{Estabilização, inalteração.}{pe.tri.fi.ca.ção}{0}
\verb{petrificação}{}{Fig.}{"-ões}{}{}{Imobilidade que resulta de um susto ou de uma grande surpresa.}{pe.tri.fi.ca.ção}{0}
\verb{petrificar}{}{}{}{}{v.t.}{Transformar em pedra.}{pe.tri.fi.car}{0}
\verb{petrificar}{}{Fig.}{}{}{}{Tornar insensível.}{pe.tri.fi.car}{0}
\verb{petrificar}{}{Fig.}{}{}{}{Paralisar por susto.}{pe.tri.fi.car}{\verboinum{2}}
\verb{petrodólar}{}{}{}{}{s.m.}{Dólar proveniente do petróleo e aplicado no mercado financeiro internacional.}{pe.tro.dó.lar}{0}
\verb{petrografia}{}{}{}{}{s.f.}{Ramo da geologia que descreve e classifica as rochas.}{pe.tro.gra.fi.a}{0}
\verb{petroleiro}{ê}{}{}{}{adj.}{Que se refere a petróleo.}{pe.tro.lei.ro}{0}
\verb{petroleiro}{ê}{}{}{}{s.m.}{Trabalhador da indústria de petróleo.}{pe.tro.lei.ro}{0}
\verb{petroleiro}{ê}{}{}{}{}{Navio que transporta petróleo.}{pe.tro.lei.ro}{0}
\verb{petróleo}{}{}{}{}{s.m.}{Óleo combustível de cor escura ou amarelo"-esverdeada, extraído das cavidades de rochas encontradas no subsolo.}{pe.tró.leo}{0}
\verb{petrolífero}{}{}{}{}{adj.}{Que contém ou produz petróleo.}{pe.tro.lí.fe.ro}{0}
\verb{petrologia}{}{}{}{}{s.f.}{Ramo da geologia que estuda a formação das rochas; litologia.}{pe.tro.lo.gi.a}{0}
\verb{petroquímica}{}{}{}{}{s.f.}{Ciência, técnica e indústria de derivados de petróleo.}{pe.tro.quí.mi.ca}{0}
\verb{petulância}{}{}{}{}{s.f.}{Qualidade ou ato de petulante; atrevimento.}{pe.tu.lân.cia}{0}
\verb{petulante}{}{}{}{}{adj.2g.}{Que é atrevido, insolente.}{pe.tu.lan.te}{0}
\verb{petúnia}{}{Bot.}{}{}{s.f.}{Erva cultivada como ornamental pela beleza das grandes flores roxas, cuja corola é afunilada.}{pe.tú.nia}{0}
\verb{peúga}{}{}{}{}{s.f.}{Meia curta, feminina ou masculina.}{pe.ú.ga}{0}
\verb{pevide}{}{Bot.}{}{}{s.f.}{Semente de vários frutos carnosos.}{pe.vi.de}{0}
\verb{pevide}{}{Veter.}{}{}{}{Película mórbida na língua de algumas aves, que lhes impede beber, causando"-lhes a morte se não for retirada.}{pe.vi.de}{0}
\verb{pexotada}{ch}{}{}{}{}{Var. de \textit{pixotada}.}{pe.xo.ta.da}{0}
\verb{pexote}{chó}{}{}{}{}{Var. de \textit{pixote}.}{pe.xo.te}{0}
\verb{pez}{ê}{}{}{}{s.m.}{Substância resinosa excretada pelo pinheiro e outras árvores.}{pez}{0}
\verb{pez}{ê}{}{}{}{}{Piche, betume.}{pez}{0}
\verb{pezudo}{}{}{}{}{adj.}{Que tem pés grandes.}{pe.zu.do}{0}
\verb{pi}{}{}{}{}{s.m.}{Décima sexta letra do alfabeto grego, correspondente ao \textit{P} do latim e das línguas neolatinas.}{pi}{0}
\verb{pi}{}{Geom.}{}{}{}{Símbolo da relação entre a circunferência e o seu diâmetro.}{pi}{0}
\verb{pia}{}{}{}{}{s.f.}{Bacia presa à parede para lavar mãos, louças etc.}{pi.a}{0}
\verb{piá}{}{}{}{}{s.m.}{Menino.}{pi.á}{0}
\verb{piaba}{}{Zool.}{}{}{s.f.}{Peixe de rio com boca miúda e dentes fortes.}{pi.a.ba}{0}
\verb{piaçaba}{}{Bot.}{}{}{s.f.}{Palmeira que tem fibras na base das folhas.}{pi.a.ça.ba}{0}
\verb{piaçaba}{}{}{}{}{}{Vassoura fabricada com essa fibra.}{pi.a.ça.ba}{0}
\verb{piaçava}{}{}{}{}{}{Var. de \textit{piaçaba}.}{pi.a.ça.va}{0}
\verb{piada}{}{}{}{}{s.f.}{Voz característica de certas aves e animais; pio.}{pi.a.da}{0}
\verb{piada}{}{}{}{}{}{Historinha ou dito engraçado.}{pi.a.da}{0}
\verb{piadista}{}{}{}{}{adj.2g.}{Diz"-se de indivíduo que conta piada.}{pi.a.dis.ta}{0}
\verb{piado}{}{}{}{}{s.m.}{O pio de certas aves.}{pi.a.do}{0}
\verb{piaga}{}{}{}{}{s.m.}{Chefe espiritual dos indígenas; pajé.}{pi.a.ga}{0}
\verb{pia"-máter}{}{Anat.}{pias"-máteres}{}{s.f.}{A mais interna e vascularizada das três membranas que recobrem o cérebro e a medula espinhal.}{pi.a"-má.ter}{0}
\verb{pianista}{}{}{}{}{s.2g.}{Indivíduo que toca piano.}{pi.a.nis.ta}{0}
\verb{piano}{}{}{}{}{s.m.}{Instrumento de oitenta e oito teclas, com cordas percutidas por martelos.}{pi.a.no}{0}
\verb{piano}{}{}{}{}{adv.}{Executado suavemente.}{pi.a.no}{0}
\verb{pianola}{ó}{}{}{}{s.f.}{Tipo de piano mecânico.}{pi.a.no.la}{0}
\verb{pião}{}{}{piões}{}{s.m.}{Brinquedo geralmente de madeira, com ponta metálica, que gira impulsionado por um cordel, enrolado nele, ou por meio de uma mola.}{pi.ão}{0}
\verb{piar}{}{}{}{}{v.i.}{Dar pios.}{pi.ar}{\verboinum{1}}
\verb{piastra}{}{}{}{}{s.f.}{Nome adotado para unidade monetária fracionária por alguns países que tem a libra como moeda.}{pi.as.tra}{0}
\verb{piauiense}{}{}{}{}{adj.2g.}{Relativo a Piauí.}{pi.au.i.en.se}{0}
\verb{piauiense}{}{}{}{}{s.2g.}{Indivíduo natural ou habitante desse estado.}{pi.au.i.en.se}{0}
\verb{piava}{}{}{}{}{}{Var. de \textit{piaba}.}{pi.a.va}{0}
\verb{PIB}{}{}{}{}{s.m.}{Sigla de \textit{produto} \textit{interno} \textit{bruto}.}{PIB}{0}
\verb{pica}{}{}{}{}{s.f.}{Lança antiga.}{pi.ca}{0}
\verb{picada}{}{}{}{}{s.f.}{Ato ou efeito de picar.}{pi.ca.da}{0}
\verb{picada}{}{}{}{}{}{Ferimento feito por alguma coisa pontuda.}{pi.ca.da}{0}
\verb{picada}{}{}{}{}{}{Caminho estreito aberto no mato.}{pi.ca.da}{0}
\verb{picadeiro}{ê}{}{}{}{s.m.}{Espaço no meio do circo onde se apresentam os artistas.}{pi.ca.dei.ro}{0}
\verb{picadinho}{}{Cul.}{}{}{s.m.}{Prato feito com carne cortada em pedacinhos.}{pi.ca.di.nho}{0}
\verb{picado}{}{}{}{}{adj.}{Que se picou.}{pi.ca.do}{0}
\verb{picado}{}{}{}{}{}{Coberto de picadas.}{pi.ca.do}{0}
\verb{picado}{}{}{}{}{}{Ferido com picadas.}{pi.ca.do}{0}
\verb{picado}{}{}{}{}{}{Cortado em pedacinhos.}{pi.ca.do}{0}
\verb{picado}{}{}{}{}{}{Diz"-se do mar agitado.}{pi.ca.do}{0}
\verb{picadura}{}{}{}{}{s.f.}{Picada, ferida, mordedura.}{pi.ca.du.ra}{0}
\verb{picanha}{}{}{}{}{s.f.}{A parte posterior da região lombar da rês.}{pi.ca.nha}{0}
\verb{picanha}{}{}{}{}{}{A carne que constitui essa região.}{pi.ca.nha}{0}
\verb{picante}{}{}{}{}{adj.2g.}{Diz"-se de sabor que arde, queima; apimentado.}{pi.can.te}{0}
\verb{picante}{}{}{}{}{}{Malicioso, excitante.}{pi.can.te}{0}
\verb{picão}{}{Bot.}{"-ões}{}{s.m.}{Planta gramínea de propriedades diuréticas, que agarra na roupa.}{pi.cão}{0}
\verb{picão}{}{}{"-ões}{}{}{Picareta.}{pi.cão}{0}
\verb{pica"-pau}{}{Zool.}{pica"-paus}{}{s.m.}{Ave que martela troncos com o bico em busca de insetos.}{pi.ca"-pau}{0}
\verb{picape}{}{}{}{}{s.f.}{Pequeno caminhão, geralmente aberto, para transportar mercadorias.}{pi.ca.pe}{0}
\verb{picape}{}{}{}{}{}{Caminhonete.}{pi.ca.pe}{0}
\verb{picar}{}{}{}{}{v.t.}{Ferir com objeto pontudo.}{pi.car}{0}
\verb{picar}{}{}{}{}{}{Enfiar o ferrão em.}{pi.car}{0}
\verb{picar}{}{}{}{}{}{Cortar em pedacinhos.}{pi.car}{\verboinum{2}}
\verb{picardia}{}{}{}{}{s.f.}{Engano, logro.}{pi.car.di.a}{0}
\verb{picardia}{}{}{}{}{}{Desfeita, desconsideração.}{pi.car.di.a}{0}
\verb{picaresco}{ê}{}{}{}{adj.}{Burlesco, cômico, ridículo.}{pi.ca.res.co}{0}
\verb{picaresco}{ê}{}{}{}{}{Diz"-se do gênero literário no qual se descreve o comportamento dos pícaros.}{pi.ca.res.co}{0}
\verb{picareta}{ê}{}{}{}{s.f.}{Instrumento de ferro com duas pontas agudas, preso a um cabo, usado para escavar terra, arrancar pedras etc.}{pi.ca.re.ta}{0}
\verb{picareta}{ê}{}{}{}{}{Diz"-se do indivíduo aproveitador, dado a falcatruas.}{pi.ca.re.ta}{0}
\verb{picaretagem}{}{}{"-ens}{}{s.f.}{Ação ardilosa que visa enganar alguém; falcatrua.}{pi.ca.re.ta.gem}{0}
\verb{picaretar}{}{Pop.}{}{}{v.i.}{Praticar picaretagens.}{pi.ca.re.tar}{\verboinum{1}}
\verb{pícaro}{}{}{}{}{adj.}{Diz"-se daquele que é ardiloso, astuto, velhaco.}{pí.ca.ro}{0}
\verb{piçarra}{}{}{}{}{s.f.}{Terra misturada com areia e pedras.}{pi.çar.ra}{0}
\verb{pichação}{}{}{"-ões}{}{s.f.}{Ato ou efeito de pichar; pichamento.}{pi.cha.ção}{0}
\verb{pichador}{ô}{}{}{}{adj.}{Diz"-se daquele que picha.}{pi.cha.dor}{0}
\verb{pichamento}{}{}{}{}{s.m.}{Pichação.}{pi.cha.men.to}{0}
\verb{pichar}{}{}{}{}{v.t.}{Aplicar piche.}{pi.char}{0}
\verb{pichar}{}{}{}{}{}{Escrever ou desenhar em muros, paredes etc.}{pi.char}{0}
\verb{pichar}{}{Pop.}{}{}{}{Falar mal; criticar.}{pi.char}{\verboinum{1}}
\verb{piche}{}{}{}{}{s.m.}{Resina negra e pegajosa, produzida a partir da destilação do alcatrão ou da terebintina.}{pi.che}{0}
\verb{pichel}{é}{}{"-éis}{}{s.m.}{Vasilha com que se tira vinho das pipas ou dos tonéis.}{pi.chel}{0}
\verb{picles}{}{}{}{}{s.m.}{Legumes conservados em vinagre.}{pi.cles}{0}
\verb{pico}{}{}{}{}{s.m.}{O ponto mais alto de uma elevação; cume, topo, cimo.}{pi.co}{0}
\verb{picolé}{}{}{}{}{s.m.}{Sorvete, geralmente de frutas, provido de um pauzinho na base inferior, usado para segurá"-lo.}{pi.co.lé}{0}
\verb{picotador}{ô}{}{}{}{s.m.}{Instrumento usado para furar ou picotar bilhetes.}{pi.co.ta.dor}{0}
\verb{picotagem}{}{}{"-ens}{}{s.f.}{Ato ou efeito de picotar.}{pi.co.ta.gem}{0}
\verb{picotar}{}{}{}{}{v.t.}{Fazer picotes.}{pi.co.tar}{\verboinum{1}}
\verb{picote}{ó}{}{}{}{s.m.}{Recorte dentado em blocos de papel, selos, talões etc.}{pi.co.te}{0}
\verb{pictórico}{}{}{}{}{adj.}{Que se refere a pintura; pitoresco. }{pic.tó.ri.co}{0}
\verb{picuá}{}{}{}{}{s.m.}{Saco de duas bocas usado em cavalgadura para transportar mantimentos.}{pi.cu.á}{0}
\verb{picuinha}{}{}{}{}{s.f.}{O primeiro piado das aves.}{pi.cu.i.nha}{0}
\verb{picuinha}{}{}{}{}{}{Atitude feita para aborrecer ou contrariar alguém; pirraça, provocação.}{pi.cu.i.nha}{0}
\verb{pidão}{}{}{"-ões}{}{adj.}{Que pede muito, que vive a pedir coisas aos outros.}{pi.dão}{0}
\verb{pidão}{}{}{"-ões}{}{s.m.}{Essa pessoa.}{pi.dão}{0}
\verb{piedade}{}{}{}{}{s.f.}{Amor e respeito pelas coisas religiosas; devoção.}{pi.e.da.de}{0}
\verb{piedade}{}{}{}{}{}{Compaixão, dó, pena.}{pi.e.da.de}{0}
\verb{píer}{}{}{}{}{s.m.}{Construção que avança no mar, para atracação de navios; embarcadouro, cais.}{pí.er}{0}
\verb{piercing}{}{}{}{}{s.m.}{Enfeite de metal que se pendura em várias partes do corpo, através de perfurações.}{\textit{piercing}}{0}
\verb{pierrô}{}{}{}{}{s.m.}{Personagem de teatro, de caráter sentimental e ingênuo.}{pi.er.rô}{0}
\verb{pierrô}{}{}{}{}{}{Fantasia de carnaval que reproduz esse personagem.}{pi.er.rô}{0}
\verb{pifão}{}{Pop.}{"-ões}{}{s.m.}{Bebedeira, embriaguez.}{pi.fão}{0}
\verb{pifar}{}{}{}{}{v.i.}{Sofrer avaria; parar de funcionar; quebrar, falhar. (\textit{O motor do carro pifou e ficamos a pé.})}{pi.far}{\verboinum{1}}
\verb{pífaro}{}{Mús.}{}{}{s.m.}{Instrumento de sopro, desprovido de chaves, e de som mais agudo que a flauta.}{pí.fa.ro}{0}
\verb{pífio}{}{}{}{}{adj.}{De pouco ou nenhum valor; reles, ordinário, vil.}{pí.fio}{0}
\verb{pigarrear}{}{}{}{}{v.i.}{Tossir para livrar a garganta do pigarro.}{pi.gar.re.ar}{\verboinum{4}}
\verb{pigarro}{}{}{}{}{s.m.}{Perturbação na garganta causada por catarro ou outra mucosidade e que se procura expelir pela tosse.}{pi.gar.ro}{0}
\verb{pigmentação}{}{}{"-ões}{}{s.f.}{Formação de pigmentos em certos tecidos vegetais ou animais.}{pig.men.ta.ção}{0}
\verb{pigmentação}{}{}{"-ões}{}{}{Coloração que se obtém com pigmentos.}{pig.men.ta.ção}{0}
\verb{pigmentar}{}{}{}{}{adj.2g.}{Relativo a pigmento.}{pig.men.tar}{0}
\verb{pigmentar}{}{}{}{}{v.t.}{Colorir com pigmento.}{pig.men.tar}{\verboinum{1}}
\verb{pigmento}{}{}{}{}{s.m.}{Substância que determina a cor das células ou dos tecidos vegetais e animais.}{pig.men.to}{0}
\verb{pigmento}{}{}{}{}{}{Matéria corante que forma a base das tintas.}{pig.men.to}{0}
\verb{pigmeu}{}{}{}{pigmeia}{s.m.}{Indivíduo pertencente a uma etnia africana, cuja estatura é inferior a 1,50 metro.}{pig.meu}{0}
\verb{pigmeu}{}{}{}{pigmeia}{adj.}{Diz"-se do indivíduo de estatura muito pequena; anão.}{pig.meu}{0}
\verb{pijama}{}{}{}{}{s.m.}{Vestuário caseiro, próprio para dormir, composto de blusa e calças leves e folgadas.}{pi.ja.ma}{0}
\verb{pilantra}{}{}{}{}{adj.2g.}{Diz"-se do indivíduo mau"-caráter, malandro.}{pi.lan.tra}{0}
\verb{pilantragem}{}{}{}{}{s.f.}{Forma de agir do pilantra.}{pi.lan.tra.gem}{0}
\verb{pilão}{}{}{"-ões}{}{s.m.}{Recipiente feito de tronco de árvore onde se pila ou descasca arroz, milho, café etc, socando"-os com um pedaço de pau.}{pi.lão}{0}
\verb{pilar}{}{}{}{}{v.t.}{Moer ou socar no pilão.}{pi.lar}{\verboinum{1}}
\verb{pilar}{}{}{}{}{s.m.}{Coluna simples, sem ornamentos, que sustenta uma construção.}{pi.lar}{0}
\verb{pilastra}{}{}{}{}{s.f.}{Coluna de quatro faces que sustenta uma construção, encostada a uma parede.}{pi.las.tra}{0}
\verb{pileque}{é}{Pop.}{}{}{s.m.}{Estado de bêbado; bebedeira.}{pi.le.que}{0}
\verb{pilha}{}{}{}{}{s.f.}{Amontoado de objetos dispostos uns sobre os outros.}{pi.lha}{0}
\verb{pilha}{}{}{}{}{}{Aparelho que transforma energia química em energia elétrica.}{pi.lha}{0}
\verb{pilha}{}{Fig.}{}{}{}{Indivíduo irritado, agitado, elétrico.}{pi.lha}{0}
\verb{pilhagem}{}{}{"-ens}{}{s.f.}{Ato ou efeito de pilhar; roubo, saque.}{pi.lha.gem}{0}
\verb{pilhar}{}{}{}{}{v.t.}{Apossar"-se de objetos alheios; roubar, saquear.}{pi.lhar}{\verboinum{1}}
\verb{pilhéria}{}{}{}{}{s.f.}{Dito ou alusão engraçada; piada, chiste.}{pi.lhé.ria}{0}
\verb{pilheriar}{}{}{}{}{v.i.}{Dizer pilhérias, piadas; troçar.}{pi.lhe.ri.ar}{\verboinum{1}}
\verb{pilhérico}{}{}{}{}{adj.}{Que diz pilhérias; espirituoso, zombeteiro.}{pi.lhé.ri.co}{0}
\verb{pilórico}{}{}{}{}{adj.}{Relativo a piloro.}{pi.ló.ri.co}{0}
\verb{piloro}{ó}{Anat.}{}{}{s.m.}{Orifício que faz a comunicação entre o estômago e o duodeno.}{pi.lo.ro}{0}
\verb{pilosidade}{}{}{}{}{s.f.}{Cobertura da pele constituída de pelos finos.}{pi.lo.si.da.de}{0}
\verb{piloso}{ô}{}{"-osos ⟨ó⟩}{"-osa ⟨ó⟩}{adj.}{Que tem pelos; peludo.}{pi.lo.so}{0}
\verb{pilotagem}{}{}{"-ens}{}{s.f.}{Ato ou efeito de pilotar.}{pi.lo.ta.gem}{0}
\verb{pilotagem}{}{}{"-ens}{}{}{Ofício de piloto.}{pi.lo.ta.gem}{0}
\verb{pilotar}{}{}{}{}{v.t.}{Dirigir como piloto; conduzir, guiar.}{pi.lo.tar}{\verboinum{1}}
\verb{piloti}{}{}{}{}{s.m.}{Cada uma das pilastras que sustentam um edifício, deixando livre o pavimento térreo.}{pi.lo.ti}{0}
\verb{piloto}{ô}{}{}{}{s.m.}{Indivíduo habilitado a conduzir um veículo.}{pi.lo.to}{0}
\verb{piloto}{ô}{}{}{}{}{Bico de gás que se acende primeiro, propagando a chama para os demais bicos.}{pi.lo.to}{0}
\verb{pílula}{}{}{}{}{s.f.}{Medicamento em forma de bolinha para ser tomado por via oral.}{pí.lu.la}{0}
\verb{pimenta}{}{Bot.}{}{}{s.f.}{Nome comum a diversas plantas cujos frutos são picantes; pimenteira.}{pi.men.ta}{0}
\verb{pimenta}{}{}{}{}{}{O fruto dessa planta, usado como condimento.}{pi.men.ta}{0}
\verb{pimenta"-do"-reino}{é}{Bot.}{pimentas"-do"-reino}{}{s.f.}{Planta trepadeira de flores pequenas e frutos redondos e picantes.}{pi.men.ta"-do"-rei.no}{0}
\verb{pimenta"-do"-reino}{é}{}{pimentas"-do"-reino}{}{}{O fruto dessa planta, seco e moído, usado como condimento.}{pi.men.ta"-do"-rei.no}{0}
\verb{pimental}{}{}{"-ais}{}{s.m.}{Plantação extensa de pimentas.}{pi.men.tal}{0}
\verb{pimenta"-malagueta}{ê}{Bot.}{pimentas"-malaguetas \textit{ou} pimentas"-malagueta}{}{s.f.}{Arbusto pequeno, muito cultivado no Brasil, de folhas ovais, flores brancas e bagas vermelhas, bastante picantes, usadas como condimento.	}{pi.men.ta"-ma.la.gue.ta}{0}
\verb{pimenta"-malagueta}{ê}{}{pimentas"-malaguetas \textit{ou} pimentas"-malagueta}{}{}{O fruto dessa planta. }{pi.men.ta"-ma.la.gue.ta}{0}
\verb{pimentão}{}{}{"-ões}{}{s.m.}{Espécie de pimenta bem desenvolvida, oca por dentro e de gosto pouco picante, usada como alimento ou condimento.}{pi.men.tão}{0}
\verb{pimenteira}{ê}{Bot.}{}{}{s.f.}{Nome comum de várias plantas de fruto picante; pimenta.}{pi.men.tei.ra}{0}
\verb{pimenteiro}{ê}{}{}{}{s.m.}{Pequeno recipiente usado para levar pimentas à mesa.}{pi.men.tei.ro}{0}
\verb{pimpão}{}{}{"-ões}{}{adj.}{Que é vaidoso, cheio de si, fanfarrão.}{pim.pão}{0}
\verb{pimpolho}{ô}{}{}{}{s.m.}{Rebento da videira.}{pim.po.lho}{0}
\verb{pimpolho}{ô}{Fig.}{}{}{}{Criança bem desenvolvida, robusta.}{pim.po.lho}{0}
\verb{pina}{}{}{}{}{s.f.}{Cada uma das peças curvas da circunferência da roda de um veículo.}{pi.na}{0}
\verb{pinacoteca}{é}{}{}{}{s.f.}{Museu onde se expõem pinturas.}{pi.na.co.te.ca}{0}
\verb{pinacoteca}{é}{}{}{}{}{Coleção de quadros.}{pi.na.co.te.ca}{0}
\verb{pináculo}{}{}{}{}{s.m.}{O ponto mais elevado de um edifício ou monte; cume.}{pi.ná.cu.lo}{0}
\verb{pináculo}{}{Fig.}{}{}{}{Apogeu, auge.}{pi.ná.cu.lo}{0}
\verb{pinça}{}{}{}{}{s.f.}{Instrumento composto de duas hastes pequenas de metal que se apertam em volta de um objeto para segurá"-lo ou retirá"-lo de algo.}{pin.ça}{0}
\verb{pinça}{}{}{}{}{}{Extremidade em forma desse objeto que certos animais possuem usada para prenderem alimento ou se defenderem. (\textit{O caranguejo possui duas pinças nas patas dianteiras.})}{pin.ça}{0}
\verb{pinçar}{}{}{}{}{v.t.}{Prender ou segurar usando ou como se usasse uma pinça.}{pin.çar}{\verboinum{3}}
\verb{píncaro}{}{}{}{}{s.m.}{O ponto mais elevado de uma montanha; pico, cume.}{pín.ca.ro}{0}
\verb{pincel}{é}{}{"-éis}{}{s.m.}{Intrumento formado por um tufo de pelos fixados em um cabo, próprio para espalhar cola, tinta, creme etc sobre uma superfície.}{pin.cel}{0}
\verb{pincelada}{}{}{}{}{s.f.}{Traço feito com pincel. (\textit{Ela deu apenas algumas pinceladas no quadro para retocá"-lo.})}{pin.ce.la.da}{0}
\verb{pincelar}{}{}{}{}{v.t.}{Pintar com pincel.}{pin.ce.lar}{\verboinum{1}}
\verb{pincenê}{}{}{}{}{s.m.}{Óculos sem hastes, fixados sobre o nariz por uma mola.}{pin.ce.nê}{0}
\verb{pinchar}{}{}{}{}{}{Fazer saltar; impelir.}{pin.char}{\verboinum{1}}
\verb{pinchar}{}{}{}{}{v.t.}{Lançar com força; atirar, arremessar.}{pin.char}{0}
\verb{pincho}{}{}{}{}{s.m.}{Salto, pulo, cabriola.}{pin.cho}{0}
\verb{pindaíba}{}{Pop.}{}{}{}{Falta de dinheiro.}{pin.da.í.ba}{0}
\verb{pindaíba}{}{}{}{}{s.f.}{Corda feita de palha de coqueiro.}{pin.da.í.ba}{0}
\verb{pindoba}{ó}{Bot.}{}{}{s.f.}{Nome comum a diversas palmeiras de cujas sementes se extrai um óleo comestível.}{pin.do.ba}{0}
\verb{pineal}{}{}{"-ais}{}{adj.2g.}{Cuja forma se assemelha à da pinha.}{pi.ne.al}{0}
\verb{pineal}{}{Anat.}{"-ais}{}{}{Diz"-se de pequena glândula em formato oval, situada no cérebro.}{pi.ne.al}{0}
\verb{píneo}{}{}{}{}{adj.}{Relativo a pinheiro.}{pí.neo}{0}
\verb{pinga}{}{}{}{}{s.f.}{Bebida alcoólica feita da cana"-de"-açúcar; aguardente, cachaça.}{pin.ga}{0}
\verb{pingadeira}{ê}{}{}{}{s.f.}{Série de pingos.}{pin.ga.dei.ra}{0}
\verb{pingadeira}{ê}{}{}{}{}{Recipiente em que se recolhem os pingos da carne que está assando.}{pin.ga.dei.ra}{0}
\verb{pingado}{}{}{}{}{adj.}{Repleto de pingos.}{pin.ga.do}{0}
\verb{pingado}{}{}{}{}{s.m.}{Xícara de leite a que se acrescenta uma porção de café.}{pin.ga.do}{0}
\verb{pingar}{}{}{}{}{v.t.}{Fazer um líquido escorrer aos pingos; cair gota a gota; gotejar.}{pin.gar}{\verboinum{5}}
\verb{pingente}{}{}{}{}{s.m.}{Objeto, geralmente em forma de pingo, que se usa pendurado em uma corrente.}{pin.gen.te}{0}
\verb{pingente}{}{}{}{}{}{Indivíduo que viaja pendurado na porta de ônibus ou trem.}{pin.gen.te}{0}
\verb{pingo}{}{}{}{}{s.m.}{Menor porção de um líquido; gota.}{pin.go}{0}
\verb{pingo}{}{}{}{}{}{Porção muito pequena.}{pin.go}{0}
\verb{pingo"-d'água}{}{Geol.}{}{}{s.m.}{Nome dado ao topázio incolor em seixos rolados.}{pin.go"-d'á.gua}{0}
\verb{pinguço}{}{Pop.}{}{}{adj.}{Diz"-se daquele que se embriagou ou se embriaga com frequência; cachaceiro, bêbado.}{pin.gu.ço}{0}
\verb{pingue}{}{}{}{}{adj.2g.}{Que é gordo, farto, abundante.}{pin.gue}{0}
\verb{pinguela}{é}{}{}{}{s.f.}{Ponte rústica ou pedaço de tronco usado para se atravessar um curso de água.}{pin.gue.la}{0}
\verb{pingue"-pongue}{}{Esport.}{pingue"-pongues}{}{s.m.}{Jogo praticado sobre mesa, em que a bola, arremessada pela raquete de um dos jogadores, deve passar sobre a rede que divide a mesa e tocar somente uma vez a área do adversário antes de ser rebatida; tênis de mesa.}{pin.gue"-pon.gue}{0}
\verb{pinguim}{}{Zool.}{"-ins}{}{s.m.}{Ave de asas curtas, impróprias para o voo, que anda com postura reta e vive nas zonas geladas do hemisfério sul.}{pin.guim}{0}
\verb{pinha}{}{}{}{}{s.f.}{Conjunto das sementes do pinheiro, muito usada como enfeite natalino.}{pi.nha}{0}
\verb{pinha}{}{}{}{}{}{Fruto de casca rugosa e polpa doce e macia; fruta"-do"-conde.}{pi.nha}{0}
\verb{pinhal}{}{}{"-ais}{}{s.m.}{Bosque de pinheiros; pinheiral.}{pi.nhal}{0}
\verb{pinhão}{}{}{"-ões}{}{s.m.}{Semente do pinheiro"-do"-paraná, muito apreciada quando cozida.}{pi.nhão}{0}
\verb{pinheira}{ê}{Bot.}{}{}{s.f.}{Pequena árvore, nativa das regiões tropicas americanas, cujo fruto é a pinha.}{pi.nhei.ra}{0}
\verb{pinheiral}{ê}{}{"-ais}{}{s.m.}{Bosque de pinheiros; pinhal.}{pi.nhei.ral}{0}
\verb{pinheiro}{ê}{Bot.}{}{}{s.m.}{Nome comum a diversas árvores coníferas, dentre as quais se destacam o pinheiro"-do"-paraná e a araucária.}{pi.nhei.ro}{0}
\verb{pinheiro"-do"-paraná}{ê}{Bot.}{pinheiros"-do"-paraná}{}{s.m.}{Designação de cértas espécies de árvores, algumas com sementes comestíveis e que produzem madeira de qualidade; araucária.}{pi.nhei.ro"-do"-pa.ra.ná}{0}
\verb{pinho}{}{}{}{}{s.m.}{A madeira do pinheiro.}{pi.nho}{0}
\verb{pinho}{}{Fig.}{}{}{}{Viola ou violão.}{pi.nho}{0}
\verb{pinicão}{}{}{"-ões}{}{s.m.}{Ato de pinicar; beliscão.}{pi.ni.cão}{0}
\verb{pinicar}{}{}{}{}{v.t.}{Cutucar, beliscar, esporear. (\textit{Aquela blusa me pinicava.})}{pi.ni.car}{\verboinum{2}}
\verb{piniforme}{ó}{}{}{}{adj.2g.}{Que tem forma de pinha.}{pi.ni.for.me}{0}
\verb{pino}{}{}{}{}{s.m.}{Pequena peça que se introduz em um buraco de duas chapas para uni"-las ou articulá"-las.}{pi.no}{0}
\verb{pinoia}{ó}{}{}{}{s.f.}{Coisa sem valor, inútil.}{pi.noi.a}{0}
\verb{pinoia}{ó}{}{}{}{}{Mau negócio; engano, logro.}{pi.noi.a}{0}
\verb{pinote}{ó}{}{}{}{s.m.}{Salto que o cavalo dá quando escoicea.}{pi.no.te}{0}
\verb{pinote}{ó}{}{}{}{}{Fuga, escapada.}{pi.no.te}{0}
\verb{pinotear}{}{}{}{}{v.i.}{Dar pinotes.}{pi.no.te.ar}{\verboinum{4}}
\verb{pinta}{}{}{}{}{s.f.}{Pequena mancha, especialmente de forma redonda ou arredondada.}{pin.ta}{0}
\verb{pinta}{}{Pop.}{}{}{}{Aparência, aspecto, fisionomia.}{pin.ta}{0}
\verb{pinta"-brava}{}{Bras.}{pintas"-bravas}{}{s.2g.}{Pessoa suspeita.}{pin.ta"-bra.va}{0}
\verb{pintado}{}{}{}{}{adj.}{Que se pintou.}{pin.ta.do}{0}
\verb{pintado}{}{Zool.}{}{}{}{Peixe de água doce, cheio de pintas, de carne muito apreciada.}{pin.ta.do}{0}
\verb{pintainho}{}{}{}{}{s.m.}{Pinto ainda sem plumagem.}{pin.ta.i.nho}{0}
\verb{pintalgar}{}{}{}{}{v.t.}{Pintar de cores sortidas; matizar, sarapintar.}{pin.tal.gar}{\verboinum{5}}
\verb{pintar}{}{}{}{}{v.t.}{Cobrir de tinta, especialmente colorida.}{pin.tar}{0}
\verb{pintar}{}{}{}{}{}{Executar pela pintura.}{pin.tar}{0}
\verb{pintar}{}{}{}{}{}{Descrever minuciosamente; retratar.}{pin.tar}{0}
\verb{pintar}{}{Pop.}{}{}{v.i.}{Aparecer, comparecer, surgir.}{pin.tar}{0}
\verb{pintar}{}{}{}{}{v.pron.}{Maquiar"-se.}{pin.tar}{\verboinum{1}}
\verb{pintarroxo}{ôch}{Zool.}{}{}{s.m.}{Pássaro de plumagem parda e avermelhada e de canto suave.}{pin.tar.ro.xo}{0}
\verb{pintassilgo}{}{Zool.}{}{}{s.m.}{Pássaro de plumagem preta e amarela, semelhante ao canário.}{pin.tas.sil.go}{0}
\verb{pinto}{}{}{}{}{s.m.}{Filhote novo de galinha.}{pin.to}{0}
\verb{pinto}{}{Pop.}{}{}{}{Pênis.}{pin.to}{0}
\verb{pinto}{}{}{}{}{}{Antiga moeda portuguesa.}{pin.to}{0}
\verb{pintor}{ô}{}{}{}{s.m.}{Indivíduo que pinta profissionalmente paredes, ambientes, móveis etc.}{pin.tor}{0}
\verb{pintor}{ô}{Art.}{}{}{}{Indivíduo que domina as técnicas de pintura artística.}{pin.tor}{0}
\verb{pintura}{}{}{}{}{s.f.}{Ato, efeito ou técnica de pintar.}{pin.tu.ra}{0}
\verb{pintura}{}{Art.}{}{}{}{Obra realizada com técnicas de pintura.}{pin.tu.ra}{0}
\verb{pintura}{}{}{}{}{}{Descrição minuciosa.}{pin.tu.ra}{0}
\verb{pinturesco}{ê}{}{}{}{adj.}{Pitoresco.}{pin.tu.res.co}{0}
\verb{pio}{}{}{}{}{s.m.}{Voz aguda de muitas aves.}{pi.o}{0}
\verb{pio}{}{}{}{}{adj.}{Piedoso, caridoso, devoto.}{pi.o}{0}
\verb{piogênico}{}{Med.}{}{}{adj.}{Que gera pus.}{pi.o.gê.ni.co}{0}
\verb{piolhento}{}{}{}{}{adj.}{Que tem piolhos.}{pi.o.lhen.to}{0}
\verb{piolho}{ô}{Zool.}{}{}{s.m.}{Nome dado a diversos insetos parasitas sem asas.}{pi.o.lho}{0}
\verb{pioneiro}{ê}{}{}{}{}{Que antecipa tendências; precursor.}{pi.o.nei.ro}{0}
\verb{pioneiro}{ê}{}{}{}{adj.}{Diz"-se dos exploradores que percorrem ou povoam uma região desconhecida.}{pi.o.nei.ro}{0}
\verb{pior}{ó}{}{}{}{adj.2g.}{Comparativo de superioridade de \textit{mau}; mais mau.}{pi.or}{0}
\verb{pior}{ó}{}{}{}{adv.}{Comparativo de superioridade de \textit{mal}; mais mal.}{pi.or}{0}
\verb{piora}{ó}{}{}{}{s.f.}{Ato ou efeito de piorar.}{pi.o.ra}{0}
\verb{piorar}{}{}{}{}{v.t.}{Tornar pior.}{pi.o.rar}{\verboinum{1}}
\verb{piorreia}{é}{Med.}{}{}{s.f.}{Eliminação abundante de pus.}{pi.or.rei.a}{0}
\verb{pipa}{}{}{}{}{s.f.}{Vasilha bojuda para líquidos, especialmente vinhos.}{pi.pa}{0}
\verb{pipa}{}{Pop.}{}{}{}{Papagaio de papel.}{pi.pa}{0}
\verb{pipa}{}{Pop.}{}{}{}{Indivíduo baixo e gordo.}{pi.pa}{0}
\verb{piparote}{ó}{}{}{}{s.m.}{Pancada dada com a unha do dedo indicador ou médio, que se desprende rapidamente do polegar.}{pi.pa.ro.te}{0}
\verb{pipeta}{ê}{}{}{}{s.f.}{Tubo de vidro usado para medir ou dosar líquidos com precisão.}{pi.pe.ta}{0}
\verb{pipi}{}{Pop.}{}{}{s.m.}{Urina.}{pi.pi}{0}
\verb{pipi}{}{Pop.}{}{}{}{Órgão sexual masculino ou feminino.}{pi.pi}{0}
\verb{pipiar}{}{}{}{}{v.i.}{Piar.}{pi.pi.ar}{\verboinum{1}}
\verb{pipilar}{}{}{}{}{v.i.}{Pipiar.}{pi.pi.lar}{\verboinum{1}}
\verb{pipilo}{}{}{}{}{s.m.}{Ato de pipilar.}{pi.pi.lo}{0}
\verb{pipo}{}{}{}{}{s.m.}{Pequena pipa; barril.}{pi.po}{0}
\verb{pipoca}{ó}{}{}{}{s.f.}{Tipo de grão de milho que arrebenta com o calor.}{pi.po.ca}{0}
\verb{pipoca}{ó}{Pop.}{}{}{}{Pequenas erupções da pele.}{pi.po.ca}{0}
\verb{pipocar}{}{}{}{}{v.i.}{Dar estalidos secos, semelhantes ao ruído da pipoca arrebentando.}{pi.po.car}{0}
\verb{pipocar}{}{}{}{}{}{Ficar cheio de erupções ou pequenas pintas.}{pi.po.car}{\verboinum{2}}
\verb{pipoqueiro}{ê}{}{}{}{s.m.}{Vendedor de pipocas.}{pi.po.quei.ro}{0}
\verb{pique}{}{}{}{}{s.m.}{Jogo infantil em que alguém tem que pegar os outros.}{pi.que}{0}
\verb{pique}{}{}{}{}{}{Lugar determinado em que os participantes desse jogo não podem ser pegos.}{pi.que}{0}
\verb{pique}{}{}{}{}{}{Corrida.}{pi.que}{0}
\verb{pique}{}{}{}{}{}{Sabor picante.}{pi.que}{0}
\verb{pique}{}{}{}{}{}{Buraco ou fenda feito com instrumento cortante.}{pi.que}{0}
\verb{pique}{}{Pop.}{}{}{}{Grande disposição e energia; garra.}{pi.que}{0}
\verb{piquenique}{}{}{}{}{s.m.}{Passeio, geralmente a uma praia ou área rural, durante o qual se faz uma refeição.}{pi.que.ni.que}{0}
\verb{piquete}{ê}{}{}{}{s.m.}{Em operações de greve, grupo de pessoas que impedem o acesso ao local de trabalho.}{pi.que.te}{0}
\verb{piquete}{ê}{}{}{}{}{Conjunto de soldados prontos para operação militar.}{pi.que.te}{0}
\verb{piquete}{ê}{}{}{}{}{Estaca fincada no chão para demarcar território.}{pi.que.te}{0}
\verb{pira}{}{}{}{}{s.f.}{Fogueira que se fazia para queimar cadáveres.}{pi.ra}{0}
\verb{pira}{}{Por ext.}{}{}{}{Qualquer fogueira.}{pi.ra}{0}
\verb{pira}{}{}{}{}{}{Fogo de caráter simbólico.}{pi.ra}{0}
\verb{piracema}{}{}{}{}{s.f.}{Época em que grandes cardumes de peixe sobem em direção às nascentes para desovar.}{pi.ra.ce.ma}{0}
\verb{pirado}{}{}{}{}{adj.}{Que pirou; doido, maluco.}{pi.ra.do}{0}
\verb{pirambeira}{ê}{Bras.}{}{}{s.f.}{Abismo, precipício.}{pi.ram.bei.ra}{0}
\verb{piramidal}{}{}{"-ais}{}{adj.2g.}{Que tem forma de pirâmide.}{pi.ra.mi.dal}{0}
\verb{piramidal}{}{Fig.}{"-ais}{}{}{Estupendo, colossal, extraordinário.}{pi.ra.mi.dal}{0}
\verb{pirâmide}{}{Geom.}{}{}{s.f.}{Sólido cuja base é um polígono e cujas faces laterais são triângulos com um vértice comum a todas elas.}{pi.râ.mi.de}{0}
\verb{pirâmide}{}{}{}{}{}{Objeto, edificação ou monumento em forma de pirâmide.}{pi.râ.mi.de}{0}
\verb{pirâmide}{}{}{}{}{}{Monumento sepulcral dos antigos faraós egípcios.}{pi.râ.mi.de}{0}
\verb{piranha}{}{Zool.}{}{}{s.f.}{Peixe fluvial carnívoro e voraz.}{pi.ra.nha}{0}
\verb{piranha}{}{Pejor.}{}{}{}{Mulher devassa.}{pi.ra.nha}{0}
\verb{pirão}{}{Cul.}{"-ões}{}{s.m.}{Tipo de mingau de farinha cozida, geralmente servido junto com outras iguarias.}{pi.rão}{0}
\verb{piraquara}{}{Bras.}{}{}{s.2g.}{Habitante das margens do rio Paraíba do Sul.}{pi.ra.qua.ra}{0}
\verb{piraquara}{}{Bras.}{}{}{}{Caipira.}{pi.ra.qua.ra}{0}
\verb{pirar}{}{}{}{}{v.i.}{Enlouquecer.}{pi.rar}{0}
\verb{pirar}{}{}{}{}{v.pron.}{Safar"-se, fugir.}{pi.rar}{\verboinum{1}}
\verb{pirarucu}{}{Zool.}{}{}{s.m.}{Peixe fluvial de grande porte, conhecido como o maior peixe de água doce.}{pi.ra.ru.cu}{0}
\verb{pirata}{}{}{}{}{s.m.}{Ladrão que assalta navios.}{pi.ra.ta}{0}
\verb{pirata}{}{}{}{}{adj.2g.}{Relativo a pirata.}{pi.ra.ta}{0}
\verb{pirata}{}{}{}{}{}{Ilegal, fraudulento.}{pi.ra.ta}{0}
\verb{pirataria}{}{}{}{}{s.f.}{Ação ou atividade de pirata.}{pi.ra.ta.ri.a}{0}
\verb{piratear}{}{}{}{}{v.i.}{Exercer atividade de pirata.}{pi.ra.te.ar}{0}
\verb{piratear}{}{}{}{}{}{Copiar ilegalmente.}{pi.ra.te.ar}{\verboinum{4}}
\verb{pirenaico}{}{}{}{}{adj.}{Pireneu.}{pi.re.nai.co}{0}
\verb{pireneu}{}{}{}{pireneia}{adj.}{Relativo aos Pireneus, cordilheira localizada na região de fronteira entre Espanha e França; pirenaico.}{pi.re.neu}{0}
\verb{pires}{}{}{}{}{s.m.}{Prato pequeno sobre o qual se coloca a xícara.}{pi.res}{0}
\verb{pirético}{}{Med.}{}{}{adj.}{Em estado de febre; febril.}{pi.ré.ti.co}{0}
\verb{pirex}{cs}{}{}{}{s.m.}{Nome comercial de um tipo de vidro resistente a altas temperaturas.}{pi.rex}{0}
\verb{pirexia}{cs}{Med.}{}{}{s.f.}{Estado febril.}{pi.re.xi.a}{0}
\verb{pírico}{}{}{}{}{adj.}{Relativo ao fogo.}{pí.ri.co}{0}
\verb{piriforme}{ó}{}{}{}{adj.2g.}{Em forma de pera.}{pi.ri.for.me}{0}
\verb{pirilampo}{}{Zool.}{}{}{s.m.}{Inseto, da ordem dos besouros, que tem órgãos fosforescentes; vaga"-lume.}{pi.ri.lam.po}{0}
\verb{piripaque}{}{Pop.}{}{}{s.m.}{Síncope, desmaio.}{pi.ri.pa.que}{0}
\verb{piripaque}{}{Pop.}{}{}{}{Chilique, faniquito.}{pi.ri.pa.que}{0}
\verb{piriri}{}{Bot.}{}{}{s.m.}{Planta usada na fabricação de tubos de cachimbo e na extração de látex.}{pi.ri.ri}{0}
\verb{pirita}{}{Geol.}{}{}{s.f.}{Minério empregado na fabricação de ácido sulfúrico, muito parecido com ouro.}{pi.ri.ta}{0}
\verb{pirofobia}{}{}{}{}{s.f.}{Horror mórbido ao fogo.}{pi.ro.fo.bi.a}{0}
\verb{piroga}{ó}{}{}{}{s.f.}{Embarcação comprida e estreita feita de um tronco cavado.}{pi.ro.ga}{0}
\verb{pirogravura}{}{}{}{}{s.f.}{Gravação feita em madeira ou metal com uma ponta metálica incandescente.}{pi.ro.gra.vu.ra}{0}
\verb{piromancia}{}{}{}{}{s.f.}{Adivinhação feita por meio do fogo.}{pi.ro.man.ci.a}{0}
\verb{piromania}{}{Med.}{}{}{s.f.}{Distúrbio mental em que o indivíduo tem mania por atear fogo.}{pi.ro.ma.ni.a}{0}
\verb{piromaníaco}{}{}{}{}{adj.}{Relativo a piromania.}{pi.ro.ma.ní.a.co}{0}
\verb{piromaníaco}{}{}{}{}{}{Diz"-se do indivíduo que sofre de piromania.}{pi.ro.ma.ní.a.co}{0}
\verb{pirometria}{}{}{}{}{s.f.}{Medição de altas temperaturas.}{pi.ro.me.tri.a}{0}
\verb{pirômetro}{}{}{}{}{s.m.}{Instrumento com que se medem altas temperaturas.}{pi.rô.me.tro}{0}
\verb{pirose}{ó}{Med.}{}{}{s.f.}{Sensação de ardor ou calor, do estômago até a garganta; azia.}{pi.ro.se}{0}
\verb{pirosfera}{é}{Geol.}{}{}{s.f.}{A parte interna e incandescente do globo terrestre.}{pi.ros.fe.ra}{0}
\verb{pirotecnia}{}{}{}{}{s.m.}{Conjunto de conhecimentos necessários para a fabricação de fogos de artifício.}{pi.ro.tec.ni.a}{0}
\verb{pirotécnico}{}{}{}{}{adj.}{Relativo à pirotecnia.}{pi.ro.téc.ni.co}{0}
\verb{pirotécnico}{}{}{}{}{}{Diz"-se daquele que fabrica fogos de artifício.}{pi.ro.téc.ni.co}{0}
\verb{pirraça}{}{}{}{}{s.f.}{Atitude que se faz com a intenção de ofender ou contrariar alguém; teimosia, birra, desfeita.}{pir.ra.ça}{0}
\verb{pirraçar}{}{}{}{}{v.t.}{Fazer pirraça; contrariar.}{pir.ra.çar}{\verboinum{3}}
\verb{pirracento}{}{}{}{}{adj.}{Que gosta de fazer pirraça.}{pir.ra.cen.to}{0}
\verb{pirralhada}{}{}{}{}{s.f.}{Agrupamento de pirralhos; criançada.}{pir.ra.lha.da}{0}
\verb{pirralho}{}{}{}{}{s.m.}{Menino pequeno; criança, guri.}{pir.ra.lho}{0}
\verb{pirrônico}{}{}{}{}{adj.}{Diz"-se daquele que duvida de tudo; cético, teimoso.}{pir.rô.ni.co}{0}
\verb{pirueta}{ê}{}{}{}{s.f.}{Rodopio que se dá sobre um pé.}{pi.ru.e.ta}{0}
\verb{pirueta}{ê}{}{}{}{}{Volta que o cavalo dá sobre uma das patas.}{pi.ru.e.ta}{0}
\verb{piruetar}{}{}{}{}{v.i.}{Dar piruetas; saltar, rodopiar.}{pi.ru.e.tar}{\verboinum{1}}
\verb{pirulito}{}{}{}{}{s.m.}{Caramelo preso na ponta de um palito, que se come sugando.}{pi.ru.li.to}{0}
\verb{pirulito}{}{Pop.}{}{}{}{Pênis de menino.}{pi.ru.li.to}{0}
\verb{pisa}{}{}{}{}{s.f.}{Ato de pisar.}{pi.sa}{0}
\verb{pisa}{}{}{}{}{}{Ato de macerar as uvas com os pés.}{pi.sa}{0}
\verb{pisada}{}{}{}{}{s.f.}{Ato ou efeito de pisar.}{pi.sa.da}{0}
\verb{pisada}{}{}{}{}{}{Sinal que se deixa ao pisar; pegada, rastro.}{pi.sa.da}{0}
\verb{pisadela}{é}{}{}{}{s.f.}{Pisada rápida e leve.}{pi.sa.de.la}{0}
\verb{pisadura}{}{}{}{}{s.f.}{Sinal de pisada; pegada, rastro.}{pi.sa.du.ra}{0}
\verb{pisão}{}{}{"-ões}{}{s.m.}{Pisada forte e violenta.}{pi.são}{0}
\verb{pisar}{}{}{}{}{v.t.}{Pôr os pés sobre algo.}{pi.sar}{0}
\verb{pisar}{}{}{}{}{}{Humilhar, ofender, subjugar.}{pi.sar}{\verboinum{1}}
\verb{piscadela}{é}{}{}{}{s.f.}{Sinal que se faz, piscando um dos olhos.}{pis.ca.de.la}{0}
\verb{pisca"-pisca}{}{}{pisca"-piscas}{}{s.m.}{Sinal luminoso que acende e apaga seguidamente para indicar algo.}{pis.ca"-pis.ca}{0}
\verb{piscar}{}{}{}{}{v.i.}{Abrir e fechar os olhos rapidamente.}{pis.car}{0}
\verb{piscar}{}{}{}{}{}{Acender"-se e apagar"-se rapidamente.}{pis.car}{\verboinum{2}}
\verb{písceo}{}{}{}{}{adj.}{Relativo a peixe.}{pís.ceo}{0}
\verb{pisciano}{}{Astrol.}{}{}{s.m.}{Indivíduo que nasceu sob o signo de peixes.}{pis.ci.a.no}{0}
\verb{pisciano}{}{Astrol.}{}{}{adj.}{Relativo ou pertencente a esse signo.   }{pis.ci.a.no}{0}
\verb{piscicultor}{ô}{}{}{}{adj.}{Indivíduo que se dedica à piscicultura.}{pis.ci.cul.tor}{0}
\verb{piscicultura}{}{}{}{}{s.f.}{Criação de peixes.}{pis.ci.cul.tu.ra}{0}
\verb{pisciforme}{ó}{}{}{}{adj.2g.}{Que tem forma ou aspecto de peixe.}{pis.ci.for.me}{0}
\verb{piscina}{}{}{}{}{s.f.}{Tanque com instalações apropriadas para a prática de natação ou outros esportes aquáticos.}{pis.ci.na}{0}
\verb{piscoso}{ô}{}{"-osos ⟨"-ó⟩}{"-osa ⟨ó⟩}{adj.}{Em que se encontra grande quantidade de peixes.}{pis.co.so}{0}
\verb{piso}{}{}{}{}{s.m.}{Terreno onde se pisa.}{pi.so}{0}
\verb{piso}{}{}{}{}{}{Cada um dos andares de um prédio; pavimento.}{pi.so}{0}
\verb{pisotear}{}{}{}{}{v.t.}{Esmagar com os pés; calcar, espezinhar.}{pi.so.te.ar}{\verboinum{4}}
\verb{pisoteio}{ê}{}{}{}{s.m.}{Ato ou efeito de pisotear.}{pi.so.tei.o}{0}
\verb{pista}{}{}{}{}{s.f.}{Sinal que se deixa; vestígio, marca, indício.}{pis.ta}{0}
\verb{pista}{}{}{}{}{}{Caminho preparado para corrida ou tráfego de veículos e animais.}{pis.ta}{0}
\verb{pista}{}{}{}{}{}{Em um salão, parte reservada para dança.}{pis.ta}{0}
\verb{pistache}{}{}{}{}{s.m.}{Semente comestível de certo arbusto, usada também como condimento, e da qual se extrai uma substância aromática usada em preparados farmacêuticos.}{pis.ta.che}{0}
\verb{pistão}{}{}{"-ões}{}{s.m.}{Cilindro móvel de bombas, seringas e outros equipamentos; êmbolo.}{pis.tão}{0}
\verb{pistão}{}{Mús.}{"-ões}{}{}{Instrumento de sopro da família dos metais, em que há uma válvula, que, ao ser acionada, muda a altura das notas; trompete, pistom.}{pis.tão}{0}
\verb{pistilo}{}{Bot.}{}{}{s.m.}{Órgão sexual feminino da flor, composto por ovário, estilete e estigma e que é responsável pelo aparecimento do fruto; gineceu.}{pis.ti.lo}{0}
\verb{pistola}{ó}{}{}{}{s.f.}{Arma de fogo de cano curto, que se maneja com uma só mão.}{pis.to.la}{0}
\verb{pistola}{ó}{}{}{}{}{Aparelho de ar comprimido com que se esguicha tinta.}{pis.to.la}{0}
\verb{pistolão}{}{}{"-ões}{}{s.m.}{Indivíduo influente que recomenda alguém junto a outrem.}{pis.to.lão}{0}
\verb{pistoleiro}{ê}{}{}{}{s.m.}{Indivíduo que é pago para matar; matador profissional.}{pis.to.lei.ro}{0}
\verb{pistoleiro}{ê}{}{}{}{}{Indivíduo contratado para defender outro; capanga, jagunço.}{pis.to.lei.ro}{0}
\verb{pistom}{}{Mús.}{"-ons}{}{s.m.}{Pistão.}{pis.tom}{0}
\verb{pistonista}{}{}{}{}{s.2g.}{Músico que toca pistão.}{pis.to.nis.ta}{0}
\verb{pita}{}{}{}{}{s.f.}{Fibra que se extrai das folhas da piteira e com a qual se trançam barbantes.}{pi.ta}{0}
\verb{pitada}{}{}{}{}{s.f.}{Pequena porção de substância em pó que se pega com os dedos indicador e polegar.}{pi.ta.da}{0}
\verb{pitanga}{}{}{}{}{s.f.}{Fruto da pitangueira, de baga vermelha e pequena e de gosto um tanto doce e um tanto azedo.}{pi.tan.ga}{0}
\verb{pitangueira}{ê}{Bot.}{}{}{s.f.}{Planta arbustiva, de folhas aromáticas, flores brancas e pequenas e frutos vermelhos e angulosos.}{pi.tan.guei.ra}{0}
\verb{pitar}{}{}{}{}{v.i.}{Aspirar o fumo ou tabaco; fumar.}{pi.tar}{\verboinum{1}}
\verb{pitecantropo}{}{}{}{}{s.m.}{Ser hominídeo pertencente a um grupo de primatas extintos da espécie \textit{Homo erectus}.}{pi.te.can.tro.po}{0}
\verb{piteira}{ê}{Bot.}{}{}{s.f.}{Planta ornamental de cujas folhas se extrai uma fibra com a qual se faz barbantes.}{pi.tei.ra}{0}
\verb{piteira}{ê}{}{}{}{}{Tubo no qual o fumante adapta o cigarro.}{pi.tei.ra}{0}
\verb{pitéu}{}{}{}{}{s.m.}{Comida apetitosa; petisco, iguaria.}{pi.téu}{0}
\verb{pito}{}{}{}{}{s.m.}{Utensílio feito para fumar; cachimbo.}{pi.to}{0}
\verb{pito}{}{Pop.}{}{}{}{Repreensão, admoestação.}{pi.to}{0}
\verb{pitoco}{ô}{}{}{}{s.m.}{Pedaço de algum objeto quebrado.}{pi.to.co}{0}
\verb{pitoco}{ô}{}{}{}{}{Indivíduo de baixa estatura.}{pi.to.co}{0}
\verb{pitomba}{}{}{}{}{s.f.}{Fruto da pitombeira.}{pi.tom.ba}{0}
\verb{pitomba}{}{Pop.}{}{}{}{Tapa, supapo.}{pi.tom.ba}{0}
\verb{pitombeira}{ê}{Bot.}{}{}{s.f.}{Árvore com flores pequenas e brancas, cujo fruto é uma baga comestível carnosa.}{pi.tom.bei.ra}{0}
\verb{píton}{}{Zool.}{}{}{s.m.}{Designação comum a grandes serpentes.}{pí.ton}{0}
\verb{píton}{}{}{}{}{}{Na mitologia grega, serpente morta por Apolo.}{pí.ton}{0}
\verb{píton}{}{}{}{}{}{Mago.}{pí.ton}{0}
\verb{pitonisa}{}{}{}{}{s.f.}{Em Roma e na Grécia, mulher que fazia predições; profetisa.}{pi.to.ni.sa}{0}
\verb{pitoresco}{ê}{}{}{}{adj.}{Digno de ser pintado.}{pi.to.res.co}{0}
\verb{pitoresco}{ê}{}{}{}{}{Divertido, original.}{pi.to.res.co}{0}
\verb{pitoresco}{ê}{Fig.}{}{}{}{Imaginoso, conceituoso, colorido.}{pi.to.res.co}{0}
\verb{pit"-stop}{}{Esport.}{}{}{s.m.}{No automobilismo, parada para troca de pneus ou abastecimento.}{\textit{pit"-stop}}{0}
\verb{pitu}{}{Zool.}{}{}{s.m.}{Tipo de camarão grande de água doce.}{pi.tu}{0}
\verb{pituíta}{}{}{}{}{s.f.}{Mucosidade nasal e dos brônquios.}{pi.tu.í.ta}{0}
\verb{pituitária}{}{Anat.}{}{}{adj.}{Diz"-se da pequena glândula situada na base do cérebro; hipófise.}{pi.tu.i.tá.ria}{0}
\verb{pium}{}{Zool.}{}{}{s.m.}{Pequeno mosquito; borrachudo.}{pi.um}{0}
\verb{pivete}{é}{Pop.}{}{}{s.m.}{Menino esperto.}{pi.ve.te}{0}
\verb{pivete}{é}{Pop.}{}{}{}{Menino que pratica pequenos furtos.}{pi.ve.te}{0}
\verb{pivô}{}{}{}{}{s.m.}{Haste metálica que penetra na raiz de um dente artificial.}{pi.vô}{0}
\verb{pivô}{}{Fig.}{}{}{}{Suporte, base.}{pi.vô}{0}
\verb{pivô}{}{Esport.}{}{}{}{Em alguns esportes de equipe, jogador que articula as jogadas e fornece a bola para os atacantes.}{pi.vô}{0}
\verb{pixaim}{ch}{}{"-ins}{}{s.m.}{Cabelo crespo; carapinha. }{pi.xa.im}{0}
\verb{pixotada}{ch}{}{}{}{s.f.}{Ato de pixote; ação mal realizada.}{pi.xo.ta.da}{0}
\verb{pixote}{chó}{Pop.}{}{}{s.m.}{Menino novo; criança.}{pi.xo.te}{0}
\verb{pixote}{chó}{Pop.}{}{}{}{Indivíduo inexperiente, inábil;novato.}{pi.xo.te}{0}
\verb{pizicato}{}{Mús.}{}{}{adj.}{Diz"-se do modo de tocar um instrumento de cordas usando, ao invés do arco, um dedo, que puxa a corda.}{pi.zi.ca.to}{0}
\verb{pizza}{}{Cul.}{}{}{s.f.}{Prato de origem italiana à base de massa redonda e achatada sobre a qual se colocam camadas de mozarela, tomates e temperos diversos.}{\textit{pizza}}{0}
\verb{pizzaria}{}{}{}{}{s.f.}{Estabelecimento em que se preparam, vendem e, geralmente, servem pizzas.}{piz.za.ri.a}{0}
\verb{placa}{}{}{}{}{s.f.}{Lâmina de metal; chapa.}{pla.ca}{0}
\verb{placa}{}{}{}{}{}{Chapa de metal com número de licença, que os veículos automotores são obrigados a ostentar.}{pla.ca}{0}
\verb{placar}{}{}{}{}{s.m.}{Tabuleta ou sistema eletrônico em que se marcam os pontos em competições esportivas.}{pla.car}{0}
\verb{placar}{}{Por ext.}{}{}{}{O resultado da competição.}{pla.car}{0}
\verb{placar}{}{}{}{}{}{Condecoração, venera, insígnia.}{pla.car}{0}
\verb{placar}{}{}{}{}{v.t.}{Aplacar.}{pla.car}{\verboinum{2}}
\verb{placebo}{ê}{}{}{}{s.m.}{Substância inofensiva e inativa, administrada em lugar de um medicamento no decorrer de uma experimentação, para determinar a sua eficácia real, tendo em vista eliminar qualquer participação psicológica do doente.}{pla.ce.bo}{0}
\verb{placenta}{}{Biol.}{}{}{s.f.}{Tecido esponjoso que se forma no útero durante a gravidez para nutrir o feto.}{pla.cen.ta}{0}
%\verb{}{}{}{}{}{}{}{}{0}
\verb{placidez}{ê}{}{}{}{s.f.}{Qualidade de plácido.}{pla.ci.dez}{0}
\verb{plácido}{}{}{}{}{adj.}{Sereno, tranquilo.}{plá.ci.do}{0}
\verb{plaga}{}{}{}{}{s.f.}{País, região.}{pla.ga}{0}
\verb{plagiador}{ô}{}{}{}{adj.}{Que plagia; imitador.}{pla.gi.a.dor}{0}
\verb{plagiar}{}{}{}{}{v.t.}{Fazer plágio de.}{pla.gi.ar}{\verboinum{1}}
\verb{plagiário}{}{}{}{}{s.m.}{Indivíduo que plagia, plagiador.}{pla.gi.á.rio}{0}
\verb{plágio}{}{}{}{}{s.m.}{Apropriação de trabalho ou parte de trabalho alheio, apresentando como próprio.}{plá.gio}{0}
\verb{plaina}{}{}{}{}{s.f.}{Ferramenta para desbastar e alisar madeira, tornando"-a plana.}{plai.na}{0}
\verb{plana}{}{}{}{}{s.f.}{Categoria, classe.}{pla.na}{0}
\verb{planador}{ô}{}{}{}{adj.}{Que plana.}{pla.na.dor}{0}
\verb{planador}{ô}{}{}{}{s.m.}{Tipo de aeronave que não tem motor, sendo rebocado até as alturas por um outro avião.}{pla.na.dor}{0}
\verb{planalto}{}{Geogr.}{}{}{s.m.}{Grande extensão de terras planas e localizadas a uma certa altitude acima do nível do mar.}{pla.nal.to}{0}
\verb{planar}{}{}{}{}{v.i.}{Voar sem mexer asas ou sem ação de motor.}{pla.nar}{\verboinum{1}}
\verb{plancto}{}{Biol.}{}{}{s.m.}{Plâncton.}{planc.to}{0}
\verb{plâncton}{}{Biol.}{}{}{s.m.}{Massa de seres vivos que vivem na água, sem capacidade de locomoção, à mercê das correntezas.}{plânc.ton}{0}
\verb{planejamento}{}{}{}{}{s.m.}{Ato ou efeito de planejar.}{pla.ne.ja.men.to}{0}
\verb{planejamento}{}{}{}{}{}{Preparação de um trabalho, estabelecendo tarefas, prazos e métodos.}{pla.ne.ja.men.to}{0}
\verb{planejamento}{}{}{}{}{}{O produto dessa preparação.}{pla.ne.ja.men.to}{0}
\verb{planejar}{}{}{}{}{v.t.}{Fazer o esboço, o projeto de um trabalho ou obra.}{pla.ne.jar}{\verboinum{1}}
\verb{planeta}{ê}{Astron.}{}{}{s.m.}{Corpo celeste sem luz própria que gira em torno do Sol, ou em torno de uma outra estrela.}{pla.ne.ta}{0}
\verb{planetário}{}{}{}{}{adj.}{Relativo a planeta.}{pla.ne.tá.rio}{0}
\verb{planetário}{}{}{}{}{s.m.}{Tipo de anfiteatro em forma de cúpula em cujo teto se projetam programas educativos sobre astronomia.}{pla.ne.tá.rio}{0}
\verb{planetoide}{ó}{Astron.}{}{}{s.m.}{Corpo celeste menor que um planeta e que gira em torno de um sol; asteroide.}{pla.ne.toi.de}{0}
\verb{planeza}{ê}{}{}{}{s.f.}{Qualidade de plano.}{pla.ne.za}{0}
\verb{plangência}{}{}{}{}{s.f.}{Qualidade de plangente.}{plan.gên.cia}{0}
\verb{plangente}{}{}{}{}{adj.2g.}{Triste, lamentoso, choroso.}{plan.gen.te}{0}
\verb{planger}{ê}{}{}{}{v.i.}{Chorar.}{plan.ger}{0}
\verb{planger}{ê}{Fig.}{}{}{}{Soar tristemente.}{plan.ger}{\verboinum{16}}
\verb{planície}{}{Geogr.}{}{}{s.f.}{Grande extensão de terras planas.}{pla.ní.cie}{0}
\verb{planificar}{}{}{}{}{v.t.}{Fazer plano de; planejar.}{pla.ni.fi.car}{\verboinum{2}}
\verb{planilha}{}{}{}{}{s.f.}{Formulário onde se colocam informações determinadas, especialmente numéricas.}{pla.ni.lha}{0}
\verb{planisfério}{}{}{}{}{s.m.}{Representação bidimensional da esfera terrestre.}{pla.nis.fé.rio}{0}
\verb{plano}{}{}{}{}{adj.}{Que não apresenta diferenças de nível; liso.}{pla.no}{0}
\verb{plano}{}{}{}{}{}{Que não apresenta complexidade; simples, fácil.}{pla.no}{0}
\verb{plano}{}{}{}{}{s.m.}{Superfície plana, concreta ou abstrata.}{pla.no}{0}
\verb{plano}{}{}{}{}{}{Projeto, planejamento, delineamento.}{pla.no}{0}
\verb{planta}{}{}{}{}{s.f.}{Designação comum de qualquer vegetal.}{plan.ta}{0}
\verb{planta}{}{Anat.}{}{}{}{Parte do pé que fica em contato com o chão; sola.}{plan.ta}{0}
\verb{planta}{}{}{}{}{}{Representação bidimensional do plano horizontal de um ambiente, edificação, cidade, máquina de grande porte.}{plan.ta}{0}
\verb{plantação}{}{}{"-ões}{}{s.f.}{Ato ou efeito de plantar.}{plan.ta.ção}{0}
\verb{plantação}{}{}{"-ões}{}{}{Extensão de terra com vegetais plantados.}{plan.ta.ção}{0}
\verb{plantador}{ô}{}{}{}{adj.}{Que planta.}{plan.ta.dor}{0}
\verb{plantão}{}{}{"-ões}{}{s.m.}{Serviço atribuído a um militar de acordo com uma escala de serviços.}{plan.tão}{0}
\verb{plantão}{}{}{"-ões}{}{}{Horário de serviço atribuído a um profissional, especialmente em hospitais, farmácias, delegacias.}{plan.tão}{0}
\verb{plantão}{}{}{"-ões}{}{s.m.}{O profissional que está de serviço em um momento específico.}{plan.tão}{0}
\verb{plantar}{}{}{}{}{v.t.}{Semear, cultivar.}{plan.tar}{0}
\verb{plantar}{}{}{}{}{}{Fincar na terra.}{plan.tar}{0}
\verb{plantar}{}{}{}{}{v.pron.}{Manter"-se firme sem se mover; estacionar.}{plan.tar}{\verboinum{1}}
\verb{plantel}{é}{}{"-éis}{}{s.m.}{Conjunto de animais de raça selecionados.}{plan.tel}{0}
\verb{plantígrado}{}{Zool.}{}{}{adj.}{Diz"-se de mamífero que anda sobre a planta dos pés.}{plan.tí.gra.do}{0}
\verb{plantio}{}{}{}{}{s.m.}{Ato ou efeito de plantar.}{plan.ti.o}{0}
\verb{plantonista}{}{}{}{}{s.2g.}{Indivíduo responsável pelo plantão em um determinado momento.}{plan.to.nis.ta}{0}
\verb{planura}{}{}{}{}{s.f.}{Qualidade de plano.}{pla.nu.ra}{0}
\verb{planura}{}{}{}{}{}{Planície.}{pla.nu.ra}{0}
\verb{plaquê}{}{}{}{}{s.m.}{Lâmina de metal dourada que reveste objetos de metal sem valor.}{pla.quê}{0}
\verb{plaqueta}{ê}{}{}{}{s.f.}{Placa metálica pequena.}{pla.que.ta}{0}
\verb{plaqueta}{ê}{}{}{}{}{Peça de joalheria semelhante a uma medalha.}{pla.que.ta}{0}
\verb{plaqueta}{ê}{}{}{}{}{Livro de poucas páginas.}{pla.que.ta}{0}
\verb{plaqueta}{ê}{Anat.}{}{}{}{Corpúsculo do sangue relacionado à coagulação.}{pla.que.ta}{0}
\verb{plasma}{}{Anat.}{}{}{s.m.}{A parte líquida do sangue e da linfa.}{plas.ma}{0}
\verb{plasmar}{}{}{}{}{v.t.}{Dar forma a; modelar.}{plas.mar}{0}
\verb{plasmar}{}{Por ext.}{}{}{}{Criar, formar, fazer.}{plas.mar}{\verboinum{1}}
\verb{plástica}{}{}{}{}{s.f.}{Arte de plasmar, modelar.}{plás.ti.ca}{0}
\verb{plástica}{}{}{}{}{}{A forma de um corpo.}{plás.ti.ca}{0}
\verb{plástica}{}{}{}{}{}{Redução de \textit{cirurgia plástica}, em que se fazem correções na forma ou aparência do corpo com fins estéticos ou reparadores.}{plás.ti.ca}{0}
\verb{plasticidade}{}{}{}{}{s.f.}{Qualidade do que plástico.}{plas.ti.ci.da.de}{0}
\verb{plástico}{}{}{}{}{adj.}{Relativo a plástica.}{plás.ti.co}{0}
\verb{plástico}{}{}{}{}{}{Que adquire várias formas.}{plás.ti.co}{0}
\verb{plástico}{}{Art.}{}{}{}{Diz"-se das artes que trabalham com as formas dos materiais.}{plás.ti.co}{0}
\verb{plástico}{}{}{}{}{s.m.}{Produto sintético com diversos graus de resistência e que pode ser moldado em processos industriais.}{plás.ti.co}{0}
\verb{plastificação}{}{}{"-ões}{}{s.f.}{Ato ou efeito de plastificar.}{plas.ti.fi.ca.ção}{0}
\verb{plastificar}{}{}{}{}{v.t.}{Revestir papel, documentos etc com película de plástico transparente.}{plas.ti.fi.car}{\verboinum{2}}
\verb{plataforma}{ó}{}{}{}{s.f.}{Superfície plana de uma estação ferroviária, em que os passageiros aguardam o trem.}{pla.ta.for.ma}{0}
\verb{plataforma}{ó}{}{}{}{}{Programa para administração pública de determinado grupo.}{pla.ta.for.ma}{0}
\verb{plátano}{}{Bot.}{}{}{s.m.}{Árvore de flores pequenas encontrada no hemisfério norte, usada em arborização urbana e na fabricação de móveis.}{plá.ta.no}{0}
\verb{plateia}{é}{}{}{}{s.f.}{Espaço destinado aos espectadores em uma sala de espetáculos, teatro ou cinema.}{pla.tei.a}{0}
\verb{plateia}{é}{Fig.}{}{}{}{O conjunto dos espectadores; assistência, audiência.}{pla.tei.a}{0}
\verb{platibanda}{}{}{}{}{s.f.}{Grade ou muro que limita um terraço ou área aberta.}{pla.ti.ban.da}{0}
\verb{platina}{}{Quím.}{}{}{s.f.}{Elemento químico metálico, prateado,  dúctil e maleável, bom condutor de eletricidade, utilizado como calatisador, em joalheria, na fabricação de ligas resistentes à corrosão, em instrumentos cirúrgicos e científicos etc. \elemento{78}{195.078}{Pt}.}{pla.ti.na}{0}
\verb{platina}{}{}{}{}{}{Peça metálica usada em vários aparelhos conectados à corrente elétrica.}{pla.ti.na}{0}
\verb{platinado}{}{}{}{}{adj.}{Que contém platina.}{pla.ti.na.do}{0}
\verb{platinado}{}{}{}{}{}{Da cor da platina, cinza claro.}{pla.ti.na.do}{0}
\verb{platinado}{}{}{}{}{s.m.}{Dispositivo, em motor a gasolina, que distribui a corrente.}{pla.ti.na.do}{0}
\verb{platinagem}{}{}{"-ens}{}{s.f.}{Ato ou efeito de platinar; branqueamento.}{pla.ti.na.gem}{0}
\verb{platinar}{}{}{}{}{v.t.}{Recobrir de platina.}{pla.ti.nar}{0}
\verb{platinar}{}{}{}{}{}{Dar o tom ou o brilho da platina a; branquear.}{pla.ti.nar}{\verboinum{1}}
\verb{platino}{}{}{}{}{adj.}{Relativo à região do rio da Prata, América do Sul.}{pla.ti.no}{0}
\verb{platino}{}{}{}{}{s.m.}{Indivíduo natural ou habitante dessa região.}{pla.ti.no}{0}
\verb{platô}{}{}{}{}{s.m.}{Planalto.}{pla.tô}{0}
\verb{platônico}{}{}{}{}{adj.}{Relativo a Platão, filósofo grego, ou a sua teoria filosófica.}{pla.tô.ni.co}{0}
\verb{platônico}{}{Por ext.}{}{}{}{Que se mantém no plano ideal e não se realiza materialmente.}{pla.tô.ni.co}{0}
\verb{platônico}{}{}{}{}{}{Adepto do platonismo.}{pla.tô.ni.co}{0}
\verb{platonismo}{}{}{}{}{s.m.}{Doutrina de Platão, filósofo grego (429--347 a.C.), e de seus seguidores.}{pla.to.nis.mo}{0}
\verb{platonismo}{}{}{}{}{}{Qualidade de platônico.}{pla.to.nis.mo}{0}
\verb{plausível}{}{}{"-eis}{}{adj.2g.}{Razoável, aceitável, verossímil.}{plau.sí.vel}{0}
\verb{plausível}{}{}{"-eis}{}{}{Digno de aplauso.}{plau.sí.vel}{0}
\verb{playback}{}{}{}{}{s.m.}{Acompanhamento musical gravado usado na apresentação de um cantor.}{\textit{playback}}{0}
\verb{playboy}{}{}{}{}{s.m.}{Indivíduo que ostenta bens materiais, especialmente automóveis, geralmente ocioso e de comportamento imaturo.}{\textit{playboy}}{0}
\verb{playground}{}{}{}{}{s.m.}{Área de recreação infantil, geralmente equipada com balanço, gangorra e outros aparelhos.}{\textit{playground}}{0}
\verb{plebe}{é}{}{}{}{s.f.}{A camada mais baixa de uma sociedade segundo um critério socioeconômico.}{ple.be}{0}
\verb{plebeísmo}{}{}{}{}{s.m.}{Qualidade de plebeu.}{ple.be.ís.mo}{0}
\verb{plebeu}{}{}{}{plebeia}{adj.}{Relativo a plebe.}{ple.beu}{0}
\verb{plebeu}{}{}{}{plebeia}{s.m.}{Indivíduo da plebe.}{ple.beu}{0}
\verb{plebiscito}{}{}{}{}{s.m.}{Processo de consulta à população para aprovar ou não uma proposta.}{ple.bis.ci.to}{0}
\verb{plêiada}{}{}{}{}{}{Var. de \textit{plêiade}.}{plêi.a.da}{0}
\verb{plêiade}{}{}{}{}{s.f.}{Grupo de pessoas ilustres, especialmente poetas.}{plêi.a.de}{0}
\verb{pleistoceno}{}{Geol.}{}{}{adj.}{Diz"-se do período geológico anterior ao holoceno e posterior ao plioceno; época cenozoica.}{ple.is.to.ce.no}{0}
\verb{pleiteador}{ô}{}{}{}{adj.}{Que pleiteia; pleiteante.}{plei.te.a.dor}{0}
\verb{pleiteante}{}{}{}{}{adj.2g.}{Que pleiteia.}{plei.te.an.te}{0}
\verb{pleitear}{}{}{}{}{v.t.}{Requerer em juízo.}{plei.te.ar}{0}
\verb{pleitear}{}{}{}{}{}{Buscar conseguir algo por meio de argumentação ou negociação; discutir, disputar.}{plei.te.ar}{\verboinum{4}}
\verb{pleito}{ê}{}{}{}{s.m.}{Questão judicial; litígio.}{plei.to}{0}
\verb{pleito}{ê}{}{}{}{}{Discussão, disputa.}{plei.to}{0}
\verb{pleito}{ê}{}{}{}{}{Eleição.}{plei.to}{0}
\verb{plenário}{}{}{}{}{adj.}{Pleno, completo.}{ple.ná.rio}{0}
\verb{plenário}{}{}{}{}{s.m.}{Conjunto dos membros de um grupo reunidos em assembleia.}{ple.ná.rio}{0}
\verb{plenário}{}{}{}{}{}{Tribunal do juri.}{ple.ná.rio}{0}
\verb{plenilúnio}{}{}{}{}{s.m.}{Lua cheia.}{ple.ni.lú.nio}{0}
\verb{plenipotência}{}{}{}{}{s.f.}{Poder pleno.}{ple.ni.po.tên.cia}{0}
\verb{plenipotenciário}{}{}{}{}{adj.}{Diz"-se de agente do governo que tem plenos poderes.}{ple.ni.po.ten.ci.á.rio}{0}
\verb{plenitude}{}{}{}{}{s.f.}{Qualidade de pleno; completude, totalidade, integridade.}{ple.ni.tu.de}{0}
\verb{pleno}{}{}{}{}{adj.}{Cheio, repleto, inteiro, íntegro.}{ple.no}{0}
\verb{pleno}{}{}{}{}{}{Perfeito, cabal.}{ple.no}{0}
\verb{pleonasmo}{}{Gram.}{}{}{s.m.}{Redundância de termos ou ideias.}{ple.o.nas.mo}{0}
\verb{pleonástico}{}{}{}{}{adj.}{Em que há pleonasmo; redundante.}{ple.o.nás.ti.co}{0}
\verb{pletora}{ó}{Med.}{}{}{s.f.}{Abundância de sangue.}{ple.to.ra}{0}
\verb{pletora}{ó}{}{}{}{}{Energia, exuberância, vitalidade.}{ple.to.ra}{0}
\verb{pleura}{}{Anat.}{}{}{s.f.}{Membrana que envolve cada pulmão.}{pleu.ra}{0}
\verb{pleurisia}{}{Med.}{}{}{s.f.}{Inflamação da pleura; pleurite.}{pleu.ri.si.a}{0}
\verb{pleurite}{}{Med.}{}{}{s.f.}{Inflamação da pleura.}{pleu.ri.te}{0}
\verb{plexo}{écs}{Anat.}{}{}{s.m.}{Rede de nervos, vasos sanguíneos ou linfáticos.}{ple.xo}{0}
\verb{plinto}{}{Esport.}{}{}{s.m.}{Aparelho para saltos usado em modalidades de ginástica.}{plin.to}{0}
\verb{plioceno}{}{Geol.}{}{}{adj.}{Diz"-se de período geológico anterior ao pleistoceno e posterior ao mioceno.}{pli.o.ce.no}{0}
\verb{plissado}{}{}{}{}{adj.}{Diz"-se de tecido em que se fez plissê.}{plis.sa.do}{0}
\verb{plissar}{}{}{}{}{v.t.}{Fazer dobras permanentes, pregas; fazer plissê.}{plis.sar}{\verboinum{1}}
\verb{plissê}{}{}{}{}{s.m.}{Série de pregas, dobras permanentes, feitas à máquina em um tecido.}{plis.sê}{0}
\verb{plistoceno}{}{Geol.}{}{}{adj.}{Pleistoceno.}{plis.to.ce.no}{0}
\verb{plotagem}{}{Informát.}{"-ens}{}{s.f.}{Processo de impressão em papel próprio para trabalhos de grandes dimensões.}{plo.ta.gem}{0}
\verb{plotar}{}{}{}{}{v.t.}{Localizar em carta de navegação.}{plo.tar}{0}
\verb{plotar}{}{Informát.}{}{}{}{Imprimir por processo de plotagem.}{plo.tar}{\verboinum{1}}
\verb{plugar}{}{}{}{}{v.t.}{Ligar aparelho ou dispositivo em tomada, sistema ou rede de trabalho.}{plu.gar}{\verboinum{5}}
\verb{plugue}{}{}{}{}{s.m.}{Peça que se instala na extremidade de fio ou cabo elétrico para fazer conexão a uma tomada.}{plu.gue}{0}
\verb{pluma}{}{}{}{}{s.f.}{Pena de ave.}{plu.ma}{0}
\verb{plumagem}{}{}{"-ens}{}{s.f.}{Conjunto das penas de uma ave.}{plu.ma.gem}{0}
\verb{plúmbeo}{}{}{}{}{adj.}{Que tem cor de chumbo.}{plúm.beo}{0}
\verb{plural}{}{}{"-ais}{}{adj.2g.}{Que contém mais de um.}{plu.ral}{0}
\verb{plural}{}{Gram.}{"-ais}{}{s.m.}{Flexão de número de um nome ou de um verbo, que indica referência a mais de um.}{plu.ral}{0}
\verb{pluralidade}{}{}{}{}{s.f.}{Existência de grande quantidade de tipos ou de elementos; diversidade, multiplicidade.}{plu.ra.li.da.de}{0}
\verb{pluralidade}{}{}{}{}{}{Qualidade de plural.}{plu.ra.li.da.de}{0}
\verb{pluralismo}{}{}{}{}{s.m.}{Doutrina que defende a existência de muitas causas para cada processo histórico ou social.}{plu.ra.lis.mo}{0}
\verb{pluralização}{}{}{"-ões}{}{s.f.}{Ato ou efeito de pluralizar.}{plu.ra.li.za.ção}{0}
\verb{pluralizar}{}{Gram.}{}{}{v.t.}{Pôr no plural.}{plu.ra.li.zar}{0}
\verb{pluralizar}{}{}{}{}{}{Diversificar, aumentar, multiplicar.}{plu.ra.li.zar}{\verboinum{1}}
\verb{pluricelular}{}{Biol.}{}{}{adj.2g.}{Composto por muitas células.}{plu.ri.ce.lu.lar}{0}
\verb{pluripartidário}{}{}{}{}{adj.}{Relativo a ou composto de mais de um partido.}{plu.ri.par.ti.dá.rio}{0}
\verb{pluripartidarismo}{}{}{}{}{s.m.}{Sistema político em que há vários partidos.}{plu.ri.par.ti.da.ris.mo}{0}
\verb{plurissecular}{}{}{}{}{adj.2g.}{Que tem muitos séculos de existência.}{plu.ris.se.cu.lar}{0}
\verb{Plutão}{}{Astron.}{}{}{s.m.}{O nono planeta do sistema solar.}{Plu.tão}{0}
\verb{plutocracia}{}{}{}{}{s.f.}{Influência do dinheiro nas decisões políticas.}{plu.to.cra.ci.a}{0}
\verb{plutocrata}{}{}{}{}{s.2g.}{Indivíduo que exerce influência através do dinheiro.}{plu.to.cra.ta}{0}
\verb{plutônico}{}{Geol.}{}{}{adj.}{Diz"-se de rocha formada por cristalização de magma em grandes profundidades.}{plu.tô.ni.co}{0}
\verb{plutônio}{}{Quím.}{}{}{s.m.}{Elemento químico radioativo, do grupo dos actinídeos, semelhante ao urânio, usado na produção de energia atômica para usinas elétricas e bomba atômica. \elemento{94}{(244)}{Pu}.}{plu.tô.nio}{0}
\verb{pluvial}{}{}{"-ais}{}{adj.2g.}{Relativo a chuva.}{plu.vi.al}{0}
\verb{pluviômetro}{}{}{}{}{s.m.}{Aparelho para medir a quantidade de chuva que cai em determinado período.}{plu.vi.ô.me.tro}{0}
\verb{pluvioso}{ô}{Poét.}{"-osos ⟨ó⟩}{"-osa ⟨ó⟩}{adj.}{Chuvoso.}{plu.vi.o.so}{0}
\verb{Pm}{}{Quím.}{}{}{}{Símb. do \textit{promécio}. }{Pm}{0}
\verb{PNB}{}{}{}{}{s.m.}{Sigla de \textit{Produto Nacional Bruto}.}{PNB}{0}
\verb{pneu}{}{}{}{}{s.m.}{Aro de borracha preenchido por ar comprimido que reveste as rodas dos veículos; pneumático.}{pneu}{0}
\verb{pneu}{}{Pop.}{}{}{}{Excesso de gordura localizado na região lateral da cintura.}{pneu}{0}
\verb{pneumático}{}{}{}{}{adj.}{Relativo a ar.}{pneu.má.ti.co}{0}
\verb{pneumático}{}{}{}{}{}{Diz"-se de máquina acionada por ar comprimido.}{pneu.má.ti.co}{0}
\verb{pneumático}{}{}{}{}{s.m.}{Aro de borracha preenchido por ar comprimido que reveste as rodas dos veículos.}{pneu.má.ti.co}{0}
\verb{pneumonia}{}{Med.}{}{}{s.f.}{Doença caracterizada por inflamação dos pulmões e causada por vírus ou bactéria.}{pneu.mo.ni.a}{0}
\verb{Po}{}{Quím.}{}{}{}{Símb. do \textit{polônio}. }{Po}{0}
\verb{pó}{}{}{}{}{s.m.}{Partículas minúsculas de substâncias sólidas moídas ou desagregadas; poeira.}{pó}{0}
\verb{pobre}{ó}{}{}{}{adj.}{Que tem poucos recursos ou bens materiais.}{po.bre}{0}
\verb{pobre}{ó}{Fig.}{}{}{}{De pouca qualidade.}{po.bre}{0}
\verb{pobre}{ó}{}{}{}{}{Digno de pena, de compaixão.}{po.bre}{0}
\verb{pobre"-diabo}{ó}{}{pobres"-diabos ⟨ó⟩}{}{s.m.}{Indivíduo insignificante, sem importância, inofensivo.}{po.bre"-di.a.bo}{0}
\verb{pobretão}{}{Pejor.}{"-ões}{"-ona}{s.m.}{Indivíduo pobre.}{po.bre.tão}{0}
\verb{pobreza}{ê}{}{}{}{s.f.}{Qualidade de pobre.}{po.bre.za}{0}
\verb{pobreza}{ê}{}{}{}{}{O conjunto dos indivíduos pobres.}{po.bre.za}{0}
\verb{poça}{ó/ ou /ô}{}{}{}{s.f.}{Acúmulo de água no solo.}{po.ça}{0}
\verb{poção}{}{}{"-ões}{}{s.f.}{Bebida com propriedades medicinais.}{po.ção}{0}
\verb{pocilga}{}{}{}{}{s.f.}{Curral de porcos; chiqueiro.}{po.cil.ga}{0}
\verb{pocilga}{}{Fig.}{}{}{}{Lugar imundo ou muito bagunçado.}{po.cil.ga}{0}
\verb{poço}{ô}{}{Ös ⟨ó⟩}{}{s.m.}{Buraco aberto na terra para encontrar água.}{po.ço}{0}
\verb{poço}{ô}{}{Ös ⟨ó⟩}{}{}{Buraco aberto na terra para encontrar minério ou petróleo.}{po.ço}{0}
\verb{poço}{ô}{}{Ös ⟨ó⟩}{}{}{A parte mais funda de um lago ou rio.}{po.ço}{0}
\verb{poda}{ó}{}{}{}{s.f.}{Corte de ramos de vegetais.}{po.da}{0}
\verb{podador}{ô}{}{}{}{adj.}{Que poda.}{po.da.dor}{0}
\verb{podão}{}{}{"-ões}{}{s.m.}{Foice de cabo curto, muito afiada.}{po.dão}{0}
\verb{podar}{}{}{}{}{v.t.}{Cortar ramos de plantas; aparar.}{po.dar}{0}
\verb{podar}{}{Fig.}{}{}{}{Impor limites a; cercear.}{po.dar}{\verboinum{1}}
\verb{pó de arroz}{ô}{}{pós de arroz ⟨ô⟩}{}{s.m.}{Pó extremamente fino usado em maquiagem.}{pó de ar.roz}{0}
\verb{poder}{ê}{}{}{}{v.t.}{Ter a possibilidade de.}{po.der}{0}
\verb{poder}{ê}{}{}{}{}{Ter força física ou moral.}{po.der}{0}
\verb{poder}{ê}{}{}{}{}{Ter autorização para.}{po.der}{0}
\verb{poder}{ê}{}{}{}{}{Ter domínio sobre.}{po.der}{\verboinum{48}}
\verb{poder}{ê}{}{}{}{s.m.}{Domínio, autoridade.}{po.der}{0}
\verb{poder}{ê}{}{}{}{}{Vigor físico ou moral}{po.der}{0}
\verb{poderio}{}{}{}{}{s.m.}{Grande poder.}{po.de.ri.o}{0}
\verb{poderoso}{ô}{}{"-osos ⟨ó⟩}{"-osa ⟨ó⟩}{adj.}{Que tem muito poder.}{po.de.ro.so}{0}
\verb{poderoso}{ô}{}{"-osos ⟨ó⟩}{"-osa ⟨ó⟩}{}{Que faz efeito; eficaz.}{po.de.ro.so}{0}
\verb{pódio}{}{}{}{}{s.m.}{Plataforma onde sobem os jogadores ou os atletas que vencem.}{pó.dio}{0}
\verb{pódio}{}{Mús.}{}{}{}{Estrado para regente de orquestra.}{pó.dio}{0}
\verb{podre}{ô}{}{}{}{adj.2g.}{Que apresenta aspecto desagradável e mau cheiro; decomposto, deteriorado, estragado.}{po.dre}{0}
\verb{podridão}{}{}{"-ões}{}{s.f.}{Estado de podre; decomposição, deterioração.}{po.dri.dão}{0}
\verb{poedeira}{ê}{}{}{}{adj.}{Diz"-se de galinha que já põe, ou que põe muitos ovos.}{po.e.dei.ra}{0}
\verb{poeira}{ê}{}{}{}{s.f.}{Conjunto das partículas muito pequenas que ficam no ar e se depositam sobre as coisas; pó.}{po.ei.ra}{0}
\verb{poeirada}{}{}{}{}{s.f.}{Grande quantidade de pó ou de poeira suspensa no ar.}{po.ei.ra.da}{0}
\verb{poeirento}{}{}{}{}{adj.}{Cheio ou coberto de poeira.}{po.ei.ren.to}{0}
\verb{poejo}{ê}{Bot.}{}{}{s.m.}{Erva cultivada no Brasil como planta aromática, de folhas fortemente odoríferas quando esmagadas, e que cedem um óleo rico em mentol. }{po.e.jo}{0}
\verb{poema}{}{Liter.}{}{}{s.m.}{Obra de poesia em verso.}{po.e.ma}{0}
\verb{poema}{}{Liter.}{}{}{}{Composição poética com enredo e ação.}{po.e.ma}{0}
\verb{poemeto}{ê}{}{}{}{s.m.}{Pequeno poema.}{po.e.me.to}{0}
\verb{poente}{}{}{}{}{adj.2g.}{Que põe; que se põe.}{po.en.te}{0}
\verb{poente}{}{}{}{}{}{Diz"-do Sol quando se encaminha para o ocaso.}{po.en.te}{0}
\verb{poento}{}{}{}{}{adj.}{Que tem poeira; poeirento.}{po.en.to}{0}
\verb{poesia}{}{Liter.}{}{}{s.f.}{Arte de compor ou escrever versos.}{po.e.si.a}{0}
\verb{poesia}{}{Liter.}{}{}{}{Pequena composição poética.}{po.e.si.a}{0}
\verb{poesia}{}{}{}{}{}{Encanto, graça.}{po.e.si.a}{0}
\verb{poeta}{é}{}{}{}{s.m.}{Escritor que compõe poesia.}{po.e.ta}{0}
\verb{poeta}{é}{}{}{}{}{Indivíduo que é dado a devaneios; idealista.}{po.e.ta}{0}
\verb{poetar}{}{}{}{}{v.i.}{Fazer versos.}{po.e.tar}{\verboinum{1}}
\verb{poética}{}{Liter.}{}{}{s.f.}{A teoria da versificação.}{po.é.ti.ca}{0}
\verb{poética}{}{Liter.}{}{}{}{Estudo sobre a poesia e a estética.}{po.é.ti.ca}{0}
\verb{poético}{}{}{}{}{adj.}{Relativo à poesia.}{po.é.ti.co}{0}
\verb{poético}{}{}{}{}{}{Cheio de encantamento; inspirador.}{po.é.ti.co}{0}
\verb{poetisa}{}{}{}{}{s.f.}{Mulher que faz versos.}{po.e.ti.sa}{0}
\verb{poetizar}{}{}{}{}{v.t.}{Tornar poético.}{po.e.ti.zar}{0}
\verb{poetizar}{}{}{}{}{}{Fazer versos.}{po.e.ti.zar}{\verboinum{1}}
\verb{pois}{}{}{}{}{conj.}{Por conseguinte, portanto.}{pois}{0}
\verb{pois}{}{}{}{}{}{Então, nesse caso.}{pois}{0}
\verb{pois}{}{}{}{}{}{No entanto, porém.}{pois}{0}
\verb{pois}{}{}{}{}{}{Visto que.}{pois}{0}
\verb{polaco}{}{}{}{}{adj. e s.m.  }{Polonês.}{po.la.co}{0}
\verb{polaina}{}{}{}{}{s.f.}{Peça do vestuário que protege a parte inferior da perna e a superior do pé.}{po.lai.na}{0}
\verb{polar}{}{}{}{}{adj.2g.}{Relativo aos polos da Terra.}{po.lar}{0}
\verb{polarização}{}{}{"-ões}{}{s.f.}{Atração à volta de um ou mais polos, temas, opiniões etc.}{po.la.ri.za.ção}{0}
\verb{polarização}{}{}{"-ões}{}{}{Concentração em pontos opostos.}{po.la.ri.za.ção}{0}
\verb{polarizar}{}{}{}{}{v.t.}{Ser ponto de convergência.}{po.la.ri.zar}{0}
\verb{polarizar}{}{}{}{}{v.pron.}{Concentrar"-se em um ou mais pontos, ou em duas ou mais posições opostas.}{po.la.ri.zar}{\verboinum{1}}
\verb{polca}{ó}{}{}{}{s.f.}{Dança de andamento rápido, comum no século passado.}{pol.ca}{0}
\verb{polca}{ó}{}{}{}{}{A música que acompanha essa dança.}{pol.ca}{0}
\verb{poldro}{ô}{}{}{}{s.m.}{Cavalo novo; potro.}{pol.dro}{0}
\verb{polegada}{}{}{}{}{s.f.}{Medida que tem como referência a extensão do dedo polar.}{po.le.ga.da}{0}
\verb{polegada}{}{}{}{}{}{Medida de comprimento inglesa equivalente a 25,4 mm.}{po.le.ga.da}{0}
\verb{polegar}{}{}{}{}{s.m.}{O dedo mais curto e mais grosso da mão.}{po.le.gar}{0}
\verb{polegar}{}{}{}{}{}{O primeiro dedo do pé, o mais grosso.}{po.le.gar}{0}
\verb{poleiro}{ê}{}{}{}{s.m.}{Vara onde as aves pousam e dormem.}{po.lei.ro}{0}
\verb{polem}{}{}{}{}{}{Var. de \textit{pólen}.}{po.lem}{0}
\verb{polêmica}{}{}{}{}{s.f.}{Debate oral; questão, controvérsia.}{po.lê.mi.ca}{0}
\verb{polêmico}{}{}{}{}{adj.}{Próprio de polêmica; que desperta ou é capaz de despertar polêmica.}{po.lê.mi.co}{0}
\verb{polemista}{}{}{}{}{adj.2g.}{Que trava polêmicas; que participa ativamente de polêmicas.}{po.le.mis.ta}{0}
\verb{polemizar}{}{}{}{}{v.t.}{Travar polêmica.}{po.le.mi.zar}{\verboinum{1}}
\verb{pólen}{}{Bot.}{}{}{s.m.}{Pó muito fino que fecunda a flor.}{pó.len}{0}
\verb{polenta}{}{Cul.}{}{}{s.f.}{Comida feita de água, sal e fubá de milho, levada ao fogo para cozinhar e endurecer.}{po.len.ta}{0}
\verb{polia}{}{}{}{}{s.f.}{Roda para correia transmissora de movimento.}{po.li.a}{0}
\verb{poliandria}{}{}{}{}{s.f.}{Matrimônio da mulher com diversos homens.}{po.li.an.dri.a}{0}
\verb{poliandria}{}{}{}{}{}{Regime que se observa em sociedades matrilineares, no qual diversos homens, em geral irmãos ou filhos, participam da posse de uma mulher.}{po.li.an.dri.a}{0}
\verb{polichinelo}{é}{}{}{}{s.m.}{Boneco, títere; personagem cômico do teatro de marionetes.}{po.li.chi.ne.lo}{0}
\verb{polichinelo}{é}{}{}{}{}{Indivíduo sem caráter; palhaço, bobo.}{po.li.chi.ne.lo}{0}
\verb{polícia}{}{}{}{}{s.f.}{Corporação encarregada de manter a ordem pública.}{po.lí.cia}{0}
\verb{polícia}{}{}{}{}{}{Membro dessa corporação.}{po.lí.cia}{0}
\verb{policial}{}{}{"-ais}{}{adj.2g.}{Que se refere à polícia.}{po.li.ci.al}{0}
\verb{policial}{}{}{"-ais}{}{s.2g.}{Cada um dos membros da polícia.}{po.li.ci.al}{0}
\verb{policiamento}{}{}{}{}{s.m.}{Ato ou efeito de policiar.}{po.li.ci.a.men.to}{0}
\verb{policiar}{}{}{}{}{v.t.}{Vigiar ou fiscalizar por meio do trabalho da polícia.}{po.li.ci.ar}{0}
\verb{policiar}{}{}{}{}{}{Vigiar para evitar os erros.}{po.li.ci.ar}{\verboinum{1}}
\verb{policlínica}{}{}{}{}{s.f.}{Hospital onde se tratam doenças de todo os tipos, que conta com todas as especialidades.}{po.li.clí.ni.ca}{0}
\verb{policlínico}{}{}{}{}{s.m.}{Médico que trata das doenças em geral.}{po.li.clí.ni.co}{0}
\verb{policromia}{}{}{}{}{s.f.}{Estado de um corpo que apresenta várias cores.}{po.li.cro.mi.a}{0}
\verb{policromia}{}{}{}{}{}{Processo de impressão que utiliza mais de três cores.}{po.li.cro.mi.a}{0}
\verb{policromo}{}{}{}{}{adj.}{Que tem muitas cores; multicolor.}{po.li.cro.mo}{0}
\verb{policultura}{}{}{}{}{s.f.}{Cultura de diversos produtos agrícolas em uma mesma área.}{po.li.cul.tu.ra}{0}
\verb{polidez}{ê}{}{}{}{s.f.}{Qualidade de ser polido; civilidade, cortesia.}{po.li.dez}{0}
\verb{polido}{}{}{}{}{adj.}{Que se tornou lustroso por fricção.}{po.li.do}{0}
\verb{polido}{}{}{}{}{}{Que recebeu polimento, verniz ou similar; envernizado.}{po.li.do}{0}
\verb{polido}{}{}{}{}{}{Que recebeu fina educação; cortês.}{po.li.do}{0}
\verb{polidor}{ô}{}{}{}{adj.}{Que dá polimento.}{po.li.dor}{0}
\verb{poliedro}{é}{Geom.}{}{}{adj.}{Que tem muitas faces.}{po.li.e.dro}{0}
\verb{poliedro}{é}{Geom.}{}{}{s.m.}{Sólido de muitas faces planas.}{po.li.e.dro}{0}
\verb{poliéster}{}{Quím.}{}{}{s.m.}{Éster complexo, usado na fabricação de tintas, vernizes, resinas e fibra têxtil.}{po.li.és.ter}{0}
\verb{poliéster}{}{Por ext.}{}{}{}{Tecido feito com esse material.}{po.li.és.ter}{0}
\verb{polífago}{}{Zool.}{}{}{adj.}{Que se alimenta de animais e vegetais; onívoro.}{po.lí.fa.go}{0}
\verb{polifonia}{}{Mús.}{}{}{s.f.}{Combinação de várias melodias numa mesma composição; contraponto.}{po.li.fo.ni.a}{0}
\verb{poligamia}{}{}{}{}{s.f.}{Estado ou condição de pessoa que tem mais de um cônjuge simultaneamente. }{po.li.ga.mi.a}{0}
\verb{polígamo}{}{}{}{}{adj.}{Que tem mais de um cônjuge simultaneamente. }{po.lí.ga.mo}{0}
\verb{poliginia}{}{}{}{}{s.f.}{Estado ou condição de homem que se casa com muitas mulheres.  }{po.li.gi.ni.a}{0}
\verb{poliglota}{ó}{}{}{}{adj.2g.}{Que fala várias línguas.}{po.li.glo.ta}{0}
\verb{poligonal}{}{}{"-ais}{}{adj.2g.}{Relativo a polígono.}{po.li.go.nal}{0}
\verb{poligonal}{}{}{"-ais}{}{}{Que tem muitos ângulos.}{po.li.go.nal}{0}
\verb{polígono}{}{Geom.}{}{}{s.m.}{Figura plana limitada por vários ângulos e lados.}{po.lí.go.no}{0}
\verb{polígrafo}{}{}{}{}{s.m.}{Indivíduo que escreve sobre diversos assuntos.}{po.lí.gra.fo}{0}
\verb{polígrafo}{}{}{}{}{}{Instrumento médico que registra diversos fenômenos fisiológicos.}{po.lí.gra.fo}{0}
\verb{polimento}{}{}{}{}{s.m.}{Ato ou efeito de polir.}{po.li.men.to}{0}
\verb{polimerização}{}{Quím.}{"-ões}{}{s.f.}{Reação que provoca a combinação de um grande número de moléculas para formar uma molécula maior.}{po.li.me.ri.za.ção}{0}
\verb{polímero}{}{Quím.}{}{}{s.m.}{Composto cuja molécula é formada pela associação de diversas moléculas.}{po.lí.me.ro}{0}
\verb{polimorfo}{ó}{}{}{}{adj.}{Que tem muitas formas; multiforme.}{po.li.mor.fo}{0}
\verb{polinésio}{}{}{}{}{adj.}{Relativo à Polinésia.}{po.li.né.sio}{0}
\verb{polinésio}{}{}{}{}{s.m.}{Indivíduo natural ou habitante da Polinésia.}{po.li.né.sio}{0}
\verb{polinésio}{}{}{}{}{}{Língua dos polinésios.}{po.li.né.sio}{0}
\verb{polineurite}{}{Med.}{}{}{s.f.}{Inflamação de vários nervos simultaneamente.}{po.li.neu.ri.te}{0}
\verb{polinização}{}{Bot.}{"-ões}{}{s.f.}{Transporte de um grão de pólen da antera para o estigma, efetuado pelo vento, pela água, ou por algum animal polinizador, geralmente inseto.}{po.li.ni.za.ção}{0}
\verb{polinizar}{}{}{}{}{v.t.}{Fazer a polinização.}{po.li.ni.zar}{\verboinum{1}}
\verb{polinômio}{}{Mat.}{}{}{s.m.}{Expressão algébrica formada pela soma ou subtração de muitos termos. }{po.li.nô.mio}{0}
\verb{pólio}{}{Bras.}{}{}{s.f.}{Forma reduzida de \textit{poliomielite}.}{pó.lio}{0}
\verb{poliomielite}{}{Med.}{}{}{s.f.}{Doença infecciosa que ataca a substância cinza da medula espinhal, causando paralisia muscular.}{po.li.o.mi.e.li.te}{0}
\verb{pólipo}{}{Med.}{}{}{s.m.}{Excrescência carnosa que surge em algumas mucosas, como no útero e nas fossas nasais.}{pó.li.po}{0}
\verb{pólipo}{}{Zool.}{}{}{}{Forma fixa de certos invertebrados marinhos que se reproduzem por brotamento, como os corais.}{pó.li.po}{0}
\verb{polir}{}{}{}{}{v.t.}{Tornar lustroso por fricção.}{po.lir}{0}
\verb{polir}{}{}{}{}{}{Aplicar verniz em.}{po.lir}{0}
\verb{polir}{}{}{}{}{}{Adequar hábitos e costumes para a vida em sociedade.}{po.lir}{\verboinum{32}\verboirregular{\emph{def.} polimos, polis}}
\verb{polissílabo}{}{Gram.}{}{}{adj.}{Diz"-se do vocábulo que apresenta quatro ou mais sílabas.}{po.lis.sí.la.bo}{0}
\verb{polissíndeto}{}{Gram.}{}{}{s.m.}{Construção gramatical em que se repete muitas vezes a mesma conjunção.}{po.lis.sín.de.to}{0}
\verb{politécnica}{}{}{}{}{s.f.}{Escola que ensina diversas técnicas.  }{po.li.téc.ni.ca}{0}
\verb{politécnico}{}{}{}{}{adj.}{Relativo à instrução em muitas artes técnicas ou ciências aplicadas.}{po.li.téc.ni.co}{0}
\verb{politeísmo}{}{}{}{}{s.m.}{Religião em que há muitos deuses. }{po.li.te.ís.mo}{0}
\verb{politeísta}{}{}{}{}{adj.2g.}{Que é adepto de uma religião politeísta.}{po.li.te.ís.ta}{0}
\verb{politeísta}{}{}{}{}{}{Que se refere ao politeísmo.}{po.li.te.ís.ta}{0}
\verb{politeísta}{}{}{}{}{s.2g.}{Pessoa adepta de uma religião politeísta.}{po.li.te.ís.ta}{0}
\verb{política}{}{}{}{}{s.f.}{Arte ou ciência de governar um Estado e tratar dos negócios públicos.}{po.lí.ti.ca}{0}
\verb{política}{}{}{}{}{}{Habilidade no trato das relações humanas.}{po.lí.ti.ca}{0}
\verb{politicagem}{}{}{}{}{s.f.}{Política que se faz por interesses escusos e mesquinhos.}{po.li.ti.ca.gem}{0}
\verb{político}{}{}{}{}{adj.}{Relativo à política.}{po.lí.ti.co}{0}
\verb{político}{}{Fig.}{}{}{}{Cortês, polido.}{po.lí.ti.co}{0}
\verb{político}{}{}{}{}{s.m.}{Indivíduo que se ocupa de política; estadista.}{po.lí.ti.co}{0}
\verb{politiqueiro}{ê}{}{}{}{adj.}{Diz"-se do político que só trata dos interesses pessoais ou do partido, que faz politicagem.}{po.li.ti.quei.ro}{0}
\verb{politiquice}{}{Pejor.}{}{}{s.f.}{Ato de politiqueiro; politicagem.}{po.li.ti.qui.ce}{0}
\verb{politizar}{}{}{}{}{v.t.}{Tornar consciente dos deveres e direitos políticos.}{po.li.ti.zar}{\verboinum{1}}
\verb{polivalente}{}{}{}{}{adj.2g.}{Que executa várias tarefas; versátil.}{po.li.va.len.te}{0}
\verb{polivalente}{}{Quím.}{}{}{}{Diz"-se do elemento que possui diversas valências.}{po.li.va.len.te}{0}
\verb{polivalente}{}{Farm.}{}{}{}{Diz"-se do medicamento que é eficaz contra mais de um agente patogênico.}{po.li.va.len.te}{0}
\verb{polo}{}{}{}{}{s.m.}{Cada uma das extremidades do eixo imaginário da Terra.}{po.lo}{0}
\verb{polo}{}{}{}{}{}{Cada um dos pontos opostos de uma pilha ou imã.}{po.lo}{0}
\verb{polo}{}{Esport.}{}{}{}{Espécie de hóquei jogado a cavalo.}{po.lo}{0}
\verb{polonês}{}{}{}{}{adj.}{Relativo à Polônia.}{po.lo.nês}{0}
\verb{polonês}{}{}{}{}{s.m.}{Indivíduo natural ou habitante desse país.}{po.lo.nês}{0}
\verb{polonês}{}{}{}{}{}{A língua falada na Polônia.}{po.lo.nês}{0}
\verb{polônio}{}{}{}{}{adj. e s.m.  }{Polonês.}{po.lô.nio}{0}
\verb{polônio}{}{Quím.}{}{}{s.m.}{Elemento químico metálico, prateado, radioativo. \elemento{84}{(209)}{Po}.}{po.lô.nio}{0}
\verb{polpa}{ô}{}{}{}{s.f.}{Parte carnuda da fruta, coberta pela casca.}{pol.pa}{0}
\verb{polposo}{ô}{}{"-osos ⟨ó⟩}{"-osa ⟨ó⟩}{adj.}{Polpudo.}{pol.po.so}{0}
\verb{polpudo}{}{}{}{}{adj.}{Que tem muita polpa; carnudo, polposo.}{pol.pu.do}{0}
\verb{poltrão}{}{}{"-ões}{}{adj.}{Que não tem coragem; covarde, medroso.}{pol.trão}{0}
\verb{poltrona}{}{}{}{}{s.f.}{Cadeira grande, geralmente estofada e com braços.}{pol.tro.na}{0}
\verb{polução}{}{}{"-ões}{}{s.f.}{Ejaculação involuntária de esperma.}{po.lu.ção}{0}
\verb{poluente}{}{}{}{}{adj.2g.}{Que polui.}{po.lu.en.te}{0}
\verb{poluição}{}{}{"-ões}{}{s.f.}{Ato ou efeito de poluir.}{po.lu.i.ção}{0}
\verb{poluir}{}{}{}{}{v.t.}{Tornar prejudicial à saúde, através de sujeira, veneno etc.}{po.lu.ir}{\verboinum{26}}
\verb{polvilhar}{}{}{}{}{v.t.}{Cobrir com pó ou farinha.}{pol.vi.lhar}{\verboinum{1}}
\verb{polvilho}{}{}{}{}{s.m.}{Farinha muito fina obtida do amido de mandioca.}{pol.vi.lho}{0}
\verb{polvo}{ô}{Zool.}{}{}{s.m.}{Molusco provido de oito tentáculos com ventosas.}{pol.vo}{0}
\verb{pólvora}{}{}{}{}{s.f.}{Mistura inflamável e explosiva de enxofre, salitre e carvão.}{pól.vo.ra}{0}
\verb{polvorinho}{}{}{}{}{s.m.}{Recipiente para levar pólvora à caça.}{pol.vo.ri.nho}{0}
\verb{polvorosa}{ó}{}{}{}{s.f.}{Grande agitação; rebuliço, tumulto.}{pol.vo.ro.sa}{0}
\verb{pomada}{}{Farm.}{}{}{s.f.}{Preparado farmacêutico, de consistência pastosa ou cremosa, a que se junta substância medicinal.}{po.ma.da}{0}
\verb{pomar}{}{}{}{}{s.m.}{Terreno plantado com árvores frutíferas.}{po.mar}{0}
\verb{pomba}{}{Zool.}{}{}{s.f.}{Ave muito mansa, de bico revestido de cera na base, asas arredondadas e tarsos emplumados.}{pom.ba}{0}
\verb{pombal}{}{}{"-ais}{}{s.m.}{Local onde se criam pombos.}{pom.bal}{0}
\verb{pombalino}{}{}{}{}{adj.}{Relativo ao Marquês de Pombal ou à sua época, o século \textsc{xviii}.}{pom.ba.li.no}{0}
\verb{pombo}{}{}{}{}{s.m.}{O macho da pomba.}{pom.bo}{0}
\verb{pombo"-correio}{ê}{}{pombos"-correios}{}{s.m.}{Variedade de pombo, treinado para levar correspondência.}{pom.bo"-cor.rei.o}{0}
\verb{pomicultor}{ô}{}{}{}{s.m.}{Indivíduo que se dedica à pomicultura.}{po.mi.cul.tor}{0}
\verb{pomicultura}{}{}{}{}{s.f.}{Cultivo de árvores frutíferas.}{po.mi.cul.tu.ra}{0}
\verb{pomo}{}{}{}{}{s.m.}{Fruto carnudo e arredondado, como a maçã e a pera.}{po.mo}{0}
\verb{pomo"-de"-adão}{}{}{pomos"-de"-adão}{}{s.m.}{Parte saliente na frente do pescoço do homem; gogó.}{po.mo"-de"-a.dão}{0}
\verb{pompa}{}{}{}{}{s.f.}{Ostentação de luxo e riqueza; gala, aparato.}{pom.pa}{0}
\verb{pompear}{}{}{}{}{v.i.}{Ostentar pompa, luxo.}{pom.pe.ar}{\verboinum{4}}
\verb{pompom}{}{}{"-ons}{}{s.m.}{Pequena bola feita de fios curtos de lã ou seda, usada como enfeite em gorros ou para aplicação de pó"-de"-arroz na pele.}{pom.pom}{0}
\verb{pomposo}{ô}{}{"-osos ⟨ó⟩}{"-osa ⟨ó⟩}{adj.}{Em que há pompa; luxuoso, suntuoso.}{pom.po.so}{0}
\verb{pômulo}{}{Anat.}{}{}{s.m.}{O osso mais saliente do rosto; maçã do rosto.}{pô.mu.lo}{0}
\verb{poncã}{}{}{}{}{s.f.}{Variedade de tangerina grande, cuja casca se solta com facilidade.}{pon.cã}{0}
\verb{ponche}{}{}{}{}{s.m.}{Bebida feita geralmente com vinho, água mineral, frutas picadas, açúcar, suco de limão etc.}{pon.che}{0}
\verb{poncheira}{ê}{}{}{}{s.f.}{Recipiente onde se prepara e se serve o ponche.}{pon.chei.ra}{0}
\verb{poncho}{}{}{}{}{s.m.}{Agasalho de lã quadrado, sem mangas e com uma abertura no centro, que se enfia pela cabeça e serve de capa.}{pon.cho}{0}
\verb{ponderação}{}{}{"-ões}{}{s.f.}{Ato ou efeito de ponderar; reflexão, prudência, consideração.}{pon.de.ra.ção}{0}
\verb{ponderado}{}{}{}{}{adj.}{Que pondera, tem juízo; prudente, sereno, ajuizado.}{pon.de.ra.do}{0}
\verb{ponderar}{}{}{}{}{v.t.}{Refletir cuidadosamente; avaliar, considerar, pesar.}{pon.de.rar}{\verboinum{1}}
\verb{ponderável}{}{}{"-eis}{}{adj.2g.}{Que se pode ponderar; digno de ponderação, de reflexão.}{pon.de.rá.vel}{0}
\verb{ponderoso}{ô}{}{"-osos ⟨ó⟩}{"-osa ⟨ó⟩}{adj.}{Que tem importância; digno de atenção; relevante.}{pon.de.ro.so}{0}
\verb{pônei}{}{Zool.}{}{}{s.m.}{Cavalo pequeno e ágil, de uma raça que cresce pouco, e de pelos longos.}{pô.nei}{0}
\verb{ponta}{}{}{}{}{s.f.}{A extremidade de qualquer objeto; extremo.}{pon.ta}{0}
\verb{ponta}{}{}{}{}{}{A extremidade aguçada e perfurante de um objeto comprido e estreito.}{pon.ta}{0}
\verb{ponta}{}{}{}{}{}{A extremidade que forma um ângulo.}{pon.ta}{0}
\verb{ponta}{}{}{}{}{}{Pequena participação de ator ou atriz em filme, peça teatral etc.}{pon.ta}{0}
\verb{ponta}{}{Esport.}{}{}{}{No futebol, cada uma das laterais próximas do gol.}{pon.ta}{0}
\verb{ponta"-cabeça}{ê}{}{}{}{}{Usado na expressão \textit{de ponta"-cabeça}: de cabeça para baixo. (\textit{Meu filho gosta de ficar de ponta"-cabeça no sofá.})}{pon.ta"-ca.be.ça}{0}
\verb{pontada}{}{}{}{}{s.f.}{Dor aguda e repentina.}{pon.ta.da}{0}
\verb{ponta"-direita}{ê}{Esport.}{pontas"-direitas}{}{s.2g.}{No futebol, jogador que atua ofensivamente pela extremidade direita da linha dianteira; extrema"-direita.  }{pon.ta"-di.rei.ta}{0}
\verb{ponta"-esquerda}{ê}{Esport.}{pontas"-esquerdas ⟨ê⟩}{}{s.2g.}{No futebol, jogador que atua ofensivamente pela extremidade esquerda da linha dianteira; extrema"-esquerda.  }{pon.ta"-es.quer.da}{0}
\verb{pontal}{}{}{"-ais}{}{s.m.}{Ponta de terra que penetra um pouco no mar ou no rio, acima do nível da água.}{pon.tal}{0}
\verb{pontalete}{ê}{}{}{}{s.m.}{Peça de madeira que serve de escora ou apoio.}{pon.ta.le.te}{0}
\verb{pontão}{}{}{"-ões}{}{s.m.}{Pau com que se apoia alguma coisa para que não caia; escora, pontalete. }{pon.tão}{0}
\verb{pontapé}{}{}{}{}{s.m.}{Pancada dada com a ponta do pé; chute.}{pon.ta.pé}{0}
\verb{pontaria}{}{}{}{}{s.f.}{Ato de apontar uma arma na direção da linha de mira.}{pon.ta.ri.a}{0}
\verb{pontaria}{}{}{}{}{}{Habilidade de acertar um alvo.}{pon.ta.ri.a}{0}
\verb{ponte}{}{}{}{}{s.f.}{Construção feita para ligar dois pontos separados por um curso de água ou por uma depressão.}{pon.te}{0}
\verb{ponte}{}{Fig.}{}{}{}{Ligação, conexão, intermediação.}{pon.te}{0}
\verb{pontear}{}{}{}{}{v.t.}{Fazer pontos de costura; coser, alinhavar.}{pon.te.ar}{0}
\verb{pontear}{}{Mús.}{}{}{}{Dedilhar, tocar um instrumento de corda.}{pon.te.ar}{\verboinum{4}}
\verb{ponteira}{ê}{}{}{}{s.f.}{Ponta de metal que protege a extremidade de objetos alongados como guarda"-chuvas, tacos de bilhar etc.}{pon.tei.ra}{0}
\verb{ponteiro}{ê}{}{}{}{s.m.}{No mostrador de instrumentos, agulha de metal que indica a medição de algo.}{pon.tei.ro}{0}
\verb{pontiagudo}{}{}{}{}{adj.}{Que termina em ponta fina e aguda.}{pon.ti.a.gu.do}{0}
\verb{pontificado}{}{}{}{}{s.m.}{Dignidade de pontífice; papado.}{pon.ti.fi.ca.do}{0}
\verb{pontificado}{}{}{}{}{}{Tempo em que o pontífice exerce seu cargo.}{pon.ti.fi.ca.do}{0}
\verb{pontifical}{}{}{"-ais}{}{adj.2g.}{Relativo ao pontífice ou ao pontificado.}{pon.ti.fi.cal}{0}
\verb{pontifical}{}{}{"-ais}{}{s.m.}{Capa comprida, usada pelos bispos em ocasiões solenes.}{pon.ti.fi.cal}{0}
\verb{pontificar}{}{}{}{}{v.i.}{Celebrar missa com a capa pontifical.}{pon.ti.fi.car}{0}
\verb{pontificar}{}{}{}{}{}{Exprimir"-se ou ensinar com tom de autoridade.}{pon.ti.fi.car}{\verboinum{2}}
\verb{pontífice}{}{}{}{}{s.m.}{Dignitário religioso que tem jurisdição; bispo, arcebispo.}{pon.tí.fi.ce}{0}
\verb{pontifício}{}{}{}{}{adj.}{Relativo a pontífice; pontifical.}{pon.ti.fí.cio}{0}
\verb{pontilhão}{}{}{"-ões}{}{s.m.}{Pequena ponte.}{pon.ti.lhão}{0}
\verb{pontilhar}{}{}{}{}{v.t.}{Desenhar ou marcar com pequenos pontos.}{pon.ti.lhar}{\verboinum{1}}
\verb{pontinha}{}{}{}{}{s.f.}{Pequena ponta ou pequena quantidade.}{pon.ti.nha}{0}
\verb{ponto}{}{}{}{}{s.m.}{Sinal arredondado deixado por objeto pontudo; marca.}{pon.to}{0}
\verb{ponto}{}{}{}{}{}{Pequeno furo feito com agulha para que a linha passe por ele.}{pon.to}{0}
\verb{ponto}{}{}{}{}{}{Lugar determinado para alguma coisa.}{pon.to}{0}
\verb{ponto}{}{}{}{}{}{Registro, em livro ou máquina, da entrada e saída no trabalho.}{pon.to}{0}
\verb{ponto}{}{}{}{}{}{Cada uma das partes ou aspectos de um assunto, ciência etc.}{pon.to}{0}
\verb{ponto}{}{}{}{}{}{Estado a que se chega pelo resultado de uma ação, um pensamento, uma emoção etc. (\textit{Sua alegria chegou a tal ponto que ele ria e chorava ao mesmo tempo.})}{pon.to}{0}
\verb{ponto}{}{Gram.}{}{}{}{Sinal de pontuação que indica o encerramento de um período.}{pon.to}{0}
\verb{ponto"-e"-vírgula}{}{Gram.}{ponto"-e"-vírgulas \textit{ou} pontos"-e"-vírgulas}{}{s.m.}{Sinal de pontuação [;] que indica uma pausa mais acentuada que a da vírgula.}{pon.to"-e"-vír.gu.la}{0}
\verb{pontuação}{}{}{"-ões}{}{s.f.}{Ato ou efeito de pontuar.}{pon.tu.a.ção}{0}
\verb{pontuação}{}{Gram.}{"-ões}{}{}{Conjunto de sinais gráficos usados para dar maior clareza aos textos e para indicar mudanças de entonação.}{pon.tu.a.ção}{0}
\verb{pontual}{}{}{"-ais}{}{adj.2g.}{Que marca o tempo de modo preciso; exato.}{pon.tu.al}{0}
\verb{pontual}{}{}{"-ais}{}{}{Que cumpre regularmente seus deveres e obrigações.}{pon.tu.al}{0}
\verb{pontualidade}{}{}{}{}{s.f.}{Qualidade daquele que é pontual; exatidão.}{pon.tu.a.li.da.de}{0}
\verb{pontuar}{}{}{}{}{v.t.}{Marcar com sinais de pontuação.}{pon.tu.ar}{\verboinum{1}}
\verb{pontudo}{}{}{}{}{adj.}{Que tem ponta.}{pon.tu.do}{0}
\verb{pontudo}{}{}{}{}{}{Aguçado, pontiagudo.}{pon.tu.do}{0}
\verb{popa}{ô}{}{}{}{s.f.}{Parte traseira de uma embarcação, onde se localiza o leme.}{po.pa}{0}
\verb{popelina}{}{}{}{}{s.f.}{Tecido fino de algodão, de bom caimento, próprio para camisas, saias etc.}{po.pe.li.na}{0}
\verb{popeline}{}{}{}{}{}{Var. de \textit{popelina}.}{po.pe.li.ne}{0}
\verb{populaça}{}{}{}{}{s.f.}{Populacho.}{po.pu.la.ça}{0}
\verb{população}{}{}{"-ões}{}{s.f.}{Conjunto de habitantes de determinado lugar.}{po.pu.la.ção}{0}
\verb{populacho}{}{}{}{}{s.m.}{Conjunto das pessoas pobres de uma população; ralé, plebe.}{po.pu.la.cho}{0}
\verb{popular}{}{}{}{}{}{Que representa a vontade do povo.}{po.pu.lar}{0}
\verb{popular}{}{}{}{}{adj.2g.}{Relativo ao povo.}{po.pu.lar}{0}
\verb{popular}{}{}{}{}{}{Feito para as classes pobres; simples, barato.}{po.pu.lar}{0}
\verb{popular}{}{}{}{}{s.m.}{Indivíduo do povo; anônimo.}{po.pu.lar}{0}
\verb{popularidade}{}{}{}{}{s.f.}{Qualidade de popular; estima pública; fama.}{po.pu.la.ri.da.de}{0}
\verb{popularizar}{}{}{}{}{v.t.}{Tornar popular, conhecido; divulgar.}{po.pu.la.ri.zar}{\verboinum{1}}
\verb{populismo}{}{}{}{}{s.m.}{Simpatia pelo povo.}{po.pu.lis.mo}{0}
\verb{populismo}{}{}{}{}{}{Política que procura atender aos interesses das camadas mais pobres da população.}{po.pu.lis.mo}{0}
\verb{populoso}{ô}{}{"-osos ⟨ó⟩}{"-osa ⟨ó⟩}{adj.}{Que é densamente habitado; muito povoado.}{po.pu.lo.so}{0}
\verb{pôquer}{}{}{}{}{s.m.}{Jogo de cartas em que cada jogador dispõe de cinco cartas e pode ganhar cada rodada de apostas quando possui certa combinação de cartas, ou então, quando consegue, usando o blefe, que os demais jogadores assim o creiam.}{pô.quer}{0}
\verb{por}{ô}{}{}{}{prep.}{Estabelece relações de lugar, percurso, tempo, duração, agente, causa, preço, modo, meio, estado.}{por}{0}
\verb{pôr}{}{}{}{}{v.t.}{Deslocar algo de um lugar a outro; colocar, depositar.}{pôr}{0}
\verb{pôr}{}{}{}{}{}{Colocar em certa posição; postar, dispor.}{pôr}{0}
\verb{pôr}{}{}{}{}{}{Colocar roupa ou calçado; vestir, calçar.}{pôr}{0}
\verb{pôr}{}{}{}{}{}{Expelir, deitar, botar.}{pôr}{0}
\verb{pôr}{}{}{}{}{}{Fazer figurar em; incluir.}{pôr}{0}
\verb{pôr}{}{}{}{}{v.pron.}{Apresentar"-se, mostrar"-se.}{pôr}{0}
\verb{pôr}{}{}{}{}{}{Sumir, desaparecer no ocaso. (\textit{No inverno, o sol se põe mais cedo.})}{pôr}{\verboinum{60}}
\verb{porão}{}{}{"-ões}{}{s.m.}{Parte mais baixa do interior de um navio.}{po.rão}{0}
\verb{porão}{}{}{"-ões}{}{}{Parte de uma casa situada abaixo do primeiro pavimento.}{po.rão}{0}
\verb{poraquê}{}{Zool.}{}{}{s.m.}{Peixe de água doce, de corpo alongado, capaz de produzir uma forte descarga elétrica para se proteger ou atacar; peixe"-elétrico.}{po.ra.quê}{0}
\verb{porca}{ó}{}{}{}{s.f.}{A fêmea do porco.}{por.ca}{0}
\verb{porca}{ó}{}{}{}{}{Peça de metal com um furo no meio, escavada em espiral, onde se introduz um parafuso.}{por.ca}{0}
\verb{porcada}{}{}{}{}{s.f.}{Grande quantidade de porcos; vara.}{por.ca.da}{0}
\verb{porcalhão}{}{}{"-ões}{"-ona}{adj.}{Diz"-se do indivíduo que é muito sujo ou que trabalha mal, sem capricho.}{por.ca.lhão}{0}
\verb{porção}{}{}{"-ões}{}{s.f.}{Parte de um todo; parcela, pedaço.}{por.ção}{0}
\verb{porção}{}{}{"-ões}{}{}{Certa quantidade de algo.}{por.ção}{0}
\verb{porcaria}{}{}{}{}{s.f.}{Coisa muita suja ou malfeita.}{por.ca.ri.a}{0}
\verb{porcelana}{}{}{}{}{s.f.}{Produto de cerâmina fina, branca e dura.}{por.ce.la.na}{0}
\verb{porcentagem}{}{}{}{}{s.f.}{Parte proporcional calculada sobre 100 unidades.}{por.cen.ta.gem}{0}
\verb{porcentagem}{}{}{}{}{}{Taxa calculada sobre um capital de 100 unidades.}{por.cen.ta.gem}{0}
\verb{porcino}{}{}{}{}{adj.}{Relativo a porco; suíno.}{por.ci.no}{0}
\verb{porco}{ô}{Zool.}{"-s ⟨ó⟩}{"-a ⟨ó⟩}{s.m.}{Mamífero paquiderme, de corpo robusto, pernas curtas e focinho articulado, cuja carne é muito apreciada.}{por.co}{0}
\verb{porco}{ô}{Fig.}{"-s ⟨ó⟩}{"-a ⟨ó⟩}{adj.}{Muito sujo; imundo.}{por.co}{0}
\verb{porco"-do"-mato}{ô}{Zool.}{porcos"-do"-mato ⟨ó⟩}{}{s.m.}{Mamífero de pelagem branca e preta, com uma faixa branca no pescoço, em forma de colar; cateto, caititu.  }{por.co"-do"-ma.to}{0}
\verb{porco"-espinho}{ô}{Zool.}{porcos"-espinhos \textit{ou} porcos"-espinho ⟨ó⟩}{}{s.m.}{Mamífero roedor, que apresenta grandes pelos dorsais transformados em espinhos.}{por.co"-es.pi.nho}{0}
\verb{pôr do sol}{ó}{}{pores do sol ⟨ô\ldots{}ó⟩}{}{s.m.}{Momento em que o sol se põe no oeste; ocaso, crepúsculo.}{pôr do sol}{0}
\verb{porejar}{}{}{}{}{v.t.}{Fazer sair pelos poros; suar, transpirar.}{po.re.jar}{\verboinum{1}}
\verb{porém}{}{}{}{}{conj.}{Palavra que denota oposição ou restrição ao que foi expresso anteriormente; todavia, contudo, mas, entretanto.}{po.rém}{0}
\verb{porém}{}{}{}{}{s.m.}{Obstáculo, impedimento.}{po.rém}{0}
\verb{porfia}{}{}{}{}{s.f.}{Contenda de palavras; discussão, polêmica.}{por.fi.a}{0}
\verb{porfia}{}{}{}{}{}{Insistência, obstinação, teimosia.}{por.fi.a}{0}
\verb{porfiar}{}{}{}{}{v.i.}{Discutir com ardor; polemizar.}{por.fi.ar}{0}
\verb{porfiar}{}{}{}{}{v.t.}{Lutar por algo; disputar.}{por.fi.ar}{\verboinum{1}}
\verb{pormenor}{ó}{}{}{}{s.m.}{A menor parte de uma situação; particularidade, minúcia, detalhe.}{por.me.nor}{0}
\verb{pormenorizar}{}{}{}{}{v.t.}{Expor os pormenores de uma situação; detalhar, minuciar.}{por.me.no.ri.zar}{\verboinum{1}}
\verb{pornô}{}{}{}{}{adj.2g.}{Redução de \textit{pornográfico}.}{por.nô}{0}
\verb{pornô}{}{}{}{}{s.m.}{Filme ou peça que aborda o sexo de forma explícita e exagerada.}{por.nô}{0}
\verb{pornochanchada}{}{}{}{}{s.f.}{Subgênero de filme de baixa qualidade em que predominam o humorismo barato e recursos pornográficos.}{por.no.chan.cha.da}{0}
\verb{pornografia}{}{}{}{}{s.f.}{Maneira de tratar as obscenidades em filmes, livros, quadros etc.}{por.no.gra.fi.a}{0}
\verb{pornografia}{}{}{}{}{}{Devassidão, licenciosidade.}{por.no.gra.fi.a}{0}
\verb{pornográfico}{}{}{}{}{adj.}{Relativo a pornografia.}{por.no.grá.fi.co}{0}
\verb{pornográfico}{}{}{}{}{}{Em que há pornografia; libertino, indecente, devasso.}{por.no.grá.fi.co}{0}
\verb{poro}{ó}{}{}{}{s.m.}{Cada um dos pequenos orifícios da pele, de órgão ou parte vegetal.}{po.ro}{0}
\verb{porongo}{}{Bot.}{}{}{s.m.}{Planta trepadeira de cujos frutos se fazem cuias; cabaça.}{po.ron.go}{0}
\verb{pororoca}{ó}{}{}{}{s.f.}{Onda alta e muito violenta que se forma na foz e corre rio acima.}{po.ro.ro.ca}{0}
\verb{poroso}{ô}{}{"-osos ⟨ó⟩}{"-osa ⟨ó⟩}{adj.}{Que possui muitos poros; perfurado, arejado, permeável.}{po.ro.so}{0}
\verb{porquanto}{}{}{}{}{conj.}{Palavra que indica causa de uma ação anterior; porque; visto que; por isso que. (\textit{Eles tinham de chegar cedo, porquanto essa era a vontade de sua mãe.})}{por.quan.to}{0}
\verb{porquê}{}{}{}{}{s.m.}{Razão de um fato; motivo, explicação.}{por.quê}{0}
\verb{porque}{}{}{}{}{conj.}{Palavra que exprime justificativa, explicação à oração anterior.}{por.que}{0}
\verb{porqueira}{ê}{}{}{}{s.f.}{Local onde se guardam os porcos.}{por.quei.ra}{0}
\verb{porqueira}{ê}{}{}{}{}{Imundície, sujeira.}{por.quei.ra}{0}
\verb{porqueiro}{ê}{}{}{}{adj.}{Relativo a porco.}{por.quei.ro}{0}
\verb{porqueiro}{ê}{}{}{}{s.m.}{Guardador de porcos.}{por.quei.ro}{0}
\verb{porquinho"-da"-índia}{}{Zool.}{porquinhos"-da"-índia}{}{s.m.}{Mamífero roedor, de pelagem curta, muito usado em experimentos laboratoriais; preá, cobaia.}{por.qui.nho"-da"-ín.dia}{0}
\verb{porrada}{}{Pop.}{}{}{s.f.}{Pancada dada com porrete; cacetada, porretada.}{por.ra.da}{0}
\verb{porrada}{}{Pop.}{}{}{}{Grande quantidade de pessoas ou coisas.}{por.ra.da}{0}
\verb{porre}{ó}{Pop.}{}{}{s.m.}{Estado de bêbado; embriaguez, bebedeira.}{por.re}{0}
\verb{porre}{ó}{Pop.}{}{}{}{Tédio, aborrecimento.}{por.re}{0}
\verb{porretada}{}{}{}{}{s.f.}{Golpe dado com porrete; porrada.}{por.re.ta.da}{0}
\verb{porrete}{ê}{}{}{}{s.m.}{Pedaço de pau com uma das pontas arredondadas; cacete.}{por.re.te}{0}
\verb{porta}{ó}{}{}{}{s.f.}{Abertura em parede para entrar e sair.}{por.ta}{0}
\verb{porta}{ó}{}{}{}{}{Móvel de madeira, metal ou vidro que fecha essa abertura. }{por.ta}{0}
\verb{porta}{ó}{}{}{}{}{Peça que serve para abrir e fechar um móvel, um veículo etc.}{por.ta}{0}
\verb{porta"-aviões}{ó}{}{}{}{s.m.}{Navio de guerra dotado de pista para decolagem e aterrissagem de aviões.}{por.ta"-a.vi.ões}{0}
\verb{porta"-bagagem}{ó}{}{porta"-bagagens ⟨ó⟩}{}{s.m.}{Porta"-malas.}{por.ta"-ba.ga.gem}{0}
\verb{porta"-bandeira}{ó}{}{porta"-bandeiras ⟨ó⟩}{}{s.2g.}{Pessoa que carrega a bandeira em desfiles, paradas ou procissões.}{por.ta"-ban.dei.ra}{0}
\verb{porta"-bandeira}{ó}{}{porta"-bandeiras ⟨ó⟩}{}{s.f.}{Figura de destaque nas escolas de samba, que forma par com o mestre"-sala; porta"-estandarte.}{por.ta"-ban.dei.ra}{0}
\verb{porta"-chapéus}{ó}{}{}{}{s.m.}{Caixa própria para guardar ou transportar chapéus.}{por.ta"-cha.péus}{0}
\verb{porta"-chapéus}{ó}{}{}{}{}{Móvel para pendurar chapéus, bolsas, guarda"-chuvas etc; mancebo.}{por.ta"-cha.péus}{0}
\verb{porta"-chaves}{ó}{}{}{}{s.m.}{Pequeno objeto próprio para portar ou prender chaves; chaveiro.}{por.ta"-cha.ves}{0}
\verb{portada}{}{}{}{}{s.f.}{Porta grande com ornamentos; portal.}{por.ta.da}{0}
\verb{portada}{}{}{}{}{}{Página de rosto; frontispício.}{por.ta.da}{0}
\verb{portador}{ô}{}{}{}{adj.}{Diz"-se daquele que leva e traz algo consigo. }{por.ta.dor}{0}
\verb{portador}{ô}{Med.}{}{}{}{Diz"-se daquele que está infectado com vírus ou germes de doenças.}{por.ta.dor}{0}
\verb{porta"-estandarte}{ó}{}{porta"-estandartes ⟨ó⟩}{}{s.2g.}{Porta"-bandeira.}{por.ta"-es.tan.dar.te}{0}
\verb{porta"-joias}{ó\ldots{}ó}{}{}{}{s.m.}{Pequena caixa ou cofre onde se guardam joias.}{por.ta"-joi.as}{0}
\verb{portal}{}{}{"-ais}{}{s.m.}{Entrada principal de um grande edifício; pórtico.}{por.tal}{0}
\verb{portaló}{}{}{}{}{s.m.}{Abertura na amurada do navio, para dar entrada e saída a passageiros e a cargas leves.}{por.ta.ló}{0}
\verb{porta"-luvas}{ó}{}{}{}{s.m.}{Pequeno compartimento no painel dos veículos onde se guardam objetos miúdos.}{por.ta"-lu.vas}{0}
\verb{porta"-malas}{ó}{}{}{}{s.m.}{Compartimento nos automóveis para transportar bagagens.}{por.ta"-ma.las}{0}
\verb{porta"-níqueis}{ó}{}{}{}{s.m.}{Pequeno recipiente ou carteira para guardar ou carregar moedas.}{por.ta"-ní.queis}{0}
\verb{portanto}{}{}{}{}{conj.}{Introduz uma oração coordenada que contém a conclusão de um raciocínio ou a exposição de motivos anterior; logo, por conseguinte, consequentemente.}{por.tan.to}{0}
\verb{portão}{}{}{"-ões}{}{s.m.}{Porta de madeira ou ferro na entrada de garagem, quintal ou jardim.}{por.tão}{0}
\verb{portão}{}{}{"-ões}{}{}{Porta grande.}{por.tão}{0}
\verb{portar}{}{}{}{}{v.t.}{Trazer consigo.}{por.tar}{0}
\verb{portar}{}{}{}{}{v.pron.}{Ter determinado comportamento; comportar"-se, proceder.}{por.tar}{\verboinum{1}}
\verb{porta"-retratos}{ó}{}{}{}{s.m.}{Moldura com vidros para expor retratos.}{por.ta"-re.tra.tos}{0}
\verb{portaria}{}{}{}{}{s.f.}{Local, na entrada de edifícios, repartições públicas etc. onde fica o porteiro. }{por.ta.ri.a}{0}
\verb{portaria}{}{}{}{}{}{Documento oficial escrito por uma autoridade.}{por.ta.ri.a}{0}
\verb{porta"-seios}{ó}{}{}{}{s.m.}{Peça íntima do vestuário feminino usada para suster ou modelar os seios; sutiã.}{por.ta"-sei.os}{0}
\verb{portátil}{}{}{"-eis}{}{adj.2g.}{De fácil transporte.}{por.tá.til}{0}
\verb{porta"-toalhas}{ó}{}{}{}{s.m.}{Peça para pendurar toalhas em banheiros e lavatórios.}{por.ta"-to.a.lhas}{0}
\verb{porta"-voz}{ó\ldots{}ó}{}{porta"-vozes ⟨ó\ldots{}ó⟩}{}{s.2g.}{Indivíduo que fala publicamente por outro.}{por.ta"-voz}{0}
\verb{porte}{ó}{}{}{}{s.m.}{Ato de conduzir ou trazer.}{por.te}{0}
\verb{porte}{ó}{}{}{}{}{Postura, presença.}{por.te}{0}
\verb{porte}{ó}{}{}{}{}{Estatura, tamanho.}{por.te}{0}
\verb{porteira}{ê}{}{}{}{s.f.}{Portão de entrada em propriedades rurais.}{por.tei.ra}{0}
\verb{porteira}{ê}{}{}{}{}{Feminino de porteiro.}{por.tei.ra}{0}
\verb{porteiro}{ê}{}{}{}{s.m.}{Indivíduo encarregado de porta ou portaria.}{por.tei.ro}{0}
\verb{portenho}{}{}{}{}{adj.}{Relativo a Buenos Aires, capital da Argentina.}{por.te.nho}{0}
\verb{portenho}{}{}{}{}{}{Indivíduo natural ou habitante dessa capital.}{por.te.nho}{0}
\verb{portento}{}{}{}{}{s.m.}{Coisa maravilhosa; prodígio.}{por.ten.to}{0}
\verb{portento}{}{}{}{}{}{Indivíduo de especial talento.}{por.ten.to}{0}
\verb{portentoso}{ô}{}{"-osos ⟨ó⟩}{"-osa ⟨ó⟩}{adj.}{Em que há portento.}{por.ten.to.so}{0}
\verb{portentoso}{ô}{}{"-osos ⟨ó⟩}{"-osa ⟨ó⟩}{}{Extraordinário, maravilhoso, prodigioso.}{por.ten.to.so}{0}
\verb{portfólio}{}{}{}{}{s.m.}{Pasta flexível para guardar ou transportar documentos, fotos etc.}{port.fó.lio}{0}
\verb{portfólio}{}{}{}{}{}{Conjunto de trabalhos de artistas ou fotos para divulgação junto a futuros clientes.}{port.fó.lio}{0}
\verb{pórtico}{}{}{}{}{s.m.}{Porta principal de uma grande construção; portal.}{pór.ti.co}{0}
\verb{pórtico}{}{}{}{}{}{Espaço coberto, sustentado por colunas.}{pór.ti.co}{0}
\verb{portinhola}{ó}{}{}{}{s.f.}{Pequena porta.}{por.ti.nho.la}{0}
\verb{porto}{ô}{}{}{}{s.m.}{Lugar onde as embarcações param para se abrigar, abastecer ou para embarque e desembarque.}{por.to}{0}
\verb{porto"-alegrense}{ô}{}{porto"-alegrenses ⟨ô⟩}{}{adj.2g.}{Relativo a Porto Alegre, capital do Rio Grande do Sul.}{por.to"-a.le.gren.se}{0}
\verb{porto"-alegrense}{ô}{}{porto"-alegrenses ⟨ô⟩}{}{s.2g.}{Indivíduo natural ou habitante dessa cidade.}{por.to"-a.le.gren.se}{0}
\verb{porto"-riquenho}{ô}{}{porto"-riquenhos ⟨ô⟩}{}{adj.}{Relativo a Porto Rico, nas Antilhas.}{por.to"-ri.que.nho}{0}
\verb{porto"-riquenho}{ô}{}{porto"-riquenhos ⟨ô⟩}{}{}{Indivíduo natural ou habitante de Porto Rico.}{por.to"-ri.que.nho}{0}
\verb{porto"-riquense}{ô}{}{porto"-riquenses ⟨ô⟩}{}{adj.2g. e s.2g.}{Porto"-riquenho.}{por.to"-ri.quen.se}{0}
\verb{porto"-velhense}{ô}{}{porto"-velhenses ⟨ô⟩}{}{adj.2g.}{Relativo a Porto Velho, capital de Rondônia.}{por.to"-ve.lhen.se}{0}
\verb{porto"-velhense}{ô}{}{porto"-velhenses ⟨ô⟩}{}{s.2g.}{Indivíduo natural ou habitante dessa cidade.}{por.to"-ve.lhen.se}{0}
\verb{portuário}{}{}{}{}{adj.}{Relativo a porto.}{por.tu.á.rio}{0}
\verb{portuário}{}{}{}{}{}{Próximo ou ligado a um porto.}{por.tu.á.rio}{0}
\verb{portuário}{}{}{}{}{s.m.}{Indivíduo que trabalha em um porto.}{por.tu.á.rio}{0}
\verb{português}{}{}{}{}{adj.}{Relativo a Portugal.}{por.tu.guês}{0}
\verb{português}{}{}{}{}{s.m.}{Indivíduo natural ou habitante desse país.}{por.tu.guês}{0}
\verb{português}{}{}{}{}{}{Língua românica oficial de Portugal, Brasil, Angola, Cabo Verde, Guiné"-Bissau, Moçambique e São Tomé e Príncipe, é também falada em Goa, Macau e Timor Leste.}{por.tu.guês}{0}
\verb{português}{}{}{}{}{}{Disciplina escolar, cujo objetivo é o ensino e aprendizado da língua portuguesa.}{por.tu.guês}{0}
\verb{portuguesismo}{}{}{}{}{s.m.}{Peculiaridade exclusiva da língua portuguesa.}{por.tu.gue.sis.mo}{0}
\verb{portuguesismo}{}{}{}{}{}{Paixão por tudo o que é de Portugal.}{por.tu.gue.sis.mo}{0}
\verb{porventura}{}{}{}{}{adv.}{Talvez; acaso; por acaso.}{por.ven.tu.ra}{0}
\verb{porvir}{}{}{}{}{s.m.}{O que está por acontecer; futuro.}{por.vir}{0}
\verb{posar}{}{}{}{}{v.i.}{Servir de modelo para um fotógrafo, um pintor ou um escultor.}{po.sar}{\verboinum{1}}
\verb{pós"-datar}{}{}{}{}{v.t.}{Colocar, em documento, data posterior à real. }{pós"-da.tar}{\verboinum{1}}
\verb{pós"-diluviano}{}{}{pós"-diluvianos}{}{adj.}{Que ocorreu após o dilúvio descrito na Bíblia.}{pós"-di.lu.vi.a.no}{0}
\verb{pose}{ô}{}{}{}{s.f.}{Posição do corpo; postura.}{po.se}{0}
\verb{pose}{ô}{}{}{}{}{Atitude que se toma para impressionar.}{po.se}{0}
\verb{pós"-escrito}{}{}{pós"-escritos}{}{adj.}{Que foi escrito posteriormente ou no final.}{pós"-es.cri.to}{0}
\verb{posfácio}{}{}{}{}{s.m.}{Explicação ou advertência posta no fim de um livro.}{pos.fá.cio}{0}
\verb{pós"-graduação}{}{}{pós"-graduações}{}{s.f.}{Grau de ensino superior, para aqueles que já concluíram um curso de graduação, voltado para especialização e pesquisa.}{pós"-gra.du.a.ção}{0}
\verb{pós"-graduar}{}{}{}{}{v.t.}{Conferir certificado de conclusão de curso de pós"-graduação.}{pós"-gra.du.ar}{\verboinum{1}}
\verb{posição}{}{}{"-ões}{}{s.f.}{Lugar onde pessoa ou coisa se encontra.}{po.si.ção}{0}
\verb{posição}{}{}{"-ões}{}{}{Lugar que uma pessoa ocupa numa organização.}{po.si.ção}{0}
\verb{posição}{}{}{"-ões}{}{}{Maneira como pessoa ou coisa se apresenta.}{po.si.ção}{0}
\verb{posição}{}{}{"-ões}{}{}{Conjunto de opiniões sobre um assunto; ponto de vista.}{po.si.ção}{0}
\verb{posicionar}{}{}{}{}{v.t.}{Colocar em determinada posição.}{po.si.ci.o.nar}{0}
\verb{posicionar}{}{}{}{}{v.pron.}{Assumir uma opinião; tomar partido.}{po.si.ci.o.nar}{\verboinum{1}}
\verb{positivar}{}{}{}{}{v.t.}{Tornar positivo, concreto.}{po.si.ti.var}{0}
\verb{positivar}{}{}{}{}{v.pron.}{Tornar"-se evidente.}{po.si.ti.var}{\verboinum{1}}
\verb{positividade}{}{}{}{}{s.f.}{Qualidade ou estado do que é positivo.}{po.si.ti.vi.da.de}{0}
\verb{positividade}{}{}{}{}{}{Personalidade positiva; otimimo.}{po.si.ti.vi.da.de}{0}
\verb{positivismo}{}{Filos.}{}{}{s.m.}{Sistema que rejeita todas as noções \textit{a} \textit{priori} para só admitir os princípios tirados da observação e da experiência dos fenômenos. }{po.si.ti.vis.mo}{0}
\verb{positivismo}{}{}{}{}{}{Certeza, segurança.}{po.si.ti.vis.mo}{0}
\verb{positivista}{}{}{}{}{adj.2g.}{Relativo ao positivismo.}{po.si.ti.vis.ta}{0}
\verb{positivista}{}{}{}{}{s.2g.}{Indivíduo partidário do positivismo.}{po.si.ti.vis.ta}{0}
\verb{positivo}{}{}{}{}{adj.}{Que se baseia em algum fato indiscutível.}{po.si.ti.vo}{0}
\verb{positivo}{}{}{}{}{}{Que mostra concordância; afirmativo.}{po.si.ti.vo}{0}
\verb{positivo}{}{}{}{}{}{Que anima; animador, estimulante.}{po.si.ti.vo}{0}
\verb{positivo}{}{Mat.}{}{}{}{Maior que zero.}{po.si.ti.vo}{0}
\verb{pós"-meridiano}{}{}{pós"-meridianos}{}{adj.}{Que acontece após o meio"-dia ou é relativo a esse período.}{pós"-me.ri.di.a.no}{0}
\verb{posologia}{}{}{}{}{s.f.}{Indicação da dose adequada de um medicamento.}{po.so.lo.gi.a}{0}
\verb{pós"-operatório}{}{}{pós"-operatórios}{}{s.m.}{Período ou tratamento após uma cirurgia.}{pós"-o.pe.ra.tó.rio}{0}
\verb{posponto}{}{}{}{}{s.m.}{Pesponto.}{pos.pon.to}{0}
\verb{pospor}{}{}{}{}{v.t.}{Colocar alguma coisa depois de outra.}{pos.por}{\verboinum{60}}
\verb{posposto}{ô}{}{"-s ⟨ó⟩}{"-a ⟨ó⟩}{adj.}{Posto depois de algo.}{pos.pos.to}{0}
\verb{possante}{}{}{}{}{adj.2g.}{Que tem muita potência; forte.}{pos.san.te}{0}
\verb{posse}{ó}{}{}{}{s.f.}{Condição de se ter algo para uso próprio.}{pos.se}{0}
\verb{posse}{ó}{}{}{}{}{Ato de passar a ocupar um cargo.}{pos.se}{0}
\verb{posseiro}{ê}{}{}{}{s.m.}{Indivíduo que ocupa terra que pertence a outro.}{pos.sei.ro}{0}
\verb{posses}{ó}{}{}{}{s.f.pl.}{As coisas que se tem; haveres, bens.}{pos.ses}{0}
\verb{possessão}{}{}{"-ões}{}{s.f.}{Condição de possuir; posse.}{pos.ses.são}{0}
\verb{possessão}{}{}{"-ões}{}{}{Território governado por país estrangeiro; colônia, domínio.}{pos.ses.são}{0}
\verb{possessão}{}{}{"-ões}{}{}{Situação de quem está possesso.}{pos.ses.são}{0}
\verb{possessivo}{}{}{}{}{adj.}{Relativo a posse.}{pos.ses.si.vo}{0}
\verb{possessivo}{}{}{}{}{}{Diz"-se da pessoa que tende a ter o domínio ou a posse de tudo que o rodeia; egoísta.}{pos.ses.si.vo}{0}
\verb{possessivo}{}{}{}{}{}{Diz"-se da pessoa que tem ciúmes demais das pessoas que ama; ciumento.}{pos.ses.si.vo}{0}
\verb{possessivo}{}{Gram.}{}{}{}{Diz"-se do pronome que indica posse.}{pos.ses.si.vo}{0}
\verb{possesso}{é}{}{}{}{adj.}{Que se encontra possuído, endemoniado.}{pos.ses.so}{0}
\verb{possesso}{é}{}{}{}{}{Que está tomado de ira; furioso.}{pos.ses.so}{0}
\verb{possibilidade}{}{}{}{}{s.f.}{Condição do que é possível, do que pode acontecer.}{pos.si.bi.li.da.de}{0}
\verb{possibilitar}{}{}{}{}{v.t.}{Fazer alguma coisa ser possível a alguém.}{pos.si.bi.li.tar}{\verboinum{1}}
\verb{possível}{}{}{"-eis}{}{adj.2g.}{Que pode acontecer.}{pos.sí.vel}{0}
\verb{possível}{}{}{"-eis}{}{s.m.}{Tudo o que se pode, o que é permitido.}{pos.sí.vel}{0}
\verb{possuidor}{ô}{}{}{}{adj.}{Que possui algo; proprietário.}{pos.su.i.dor}{0}
\verb{possuir}{}{}{}{}{v.t.}{Ser dono de alguma coisa.}{pos.su.ir}{0}
\verb{possuir}{}{}{}{}{}{Ter em si; conter, encerrar.}{pos.su.ir}{\verboinum{26}}
\verb{posta}{ó}{}{}{}{s.f.}{Pedaço de peixe, carne etc.; fatia.}{pos.ta}{0}
\verb{postal}{}{}{"-ais}{}{adj.2g.}{Relativo a correio.}{pos.tal}{0}
\verb{postal}{}{}{"-ais}{}{s.m.}{Cartão com desenho ou fotografia que não tem envelope; cartão"-postal.}{pos.tal}{0}
\verb{postalista}{}{}{}{}{s.2g.}{Funcionário da repartição dos correios.}{pos.ta.lis.ta}{0}
\verb{postar}{}{}{}{}{v.t.}{Pôr no correio; expedir.}{pos.tar}{0}
\verb{postar}{}{}{}{}{v.pron.}{Ficar em pé parado.}{pos.tar}{\verboinum{1}}
\verb{posta"-restante}{ó}{}{postas"-restantes ⟨ó⟩}{}{s.f.}{Sistema de envio de correspondência em que esta não é levada até o endereço do destinatário, e sim fica depositada no correio até que haja reclamação sobre ela.}{pos.ta"-res.tan.te}{0}
\verb{posta"-restante}{ó}{}{postas"-restantes ⟨ó⟩}{}{}{Dependência do correio onde se guarda ou onde se reclama correspondência por posta"-restante.}{pos.ta"-res.tan.te}{0}
\verb{poste}{ó}{}{}{}{s.m.}{Coluna fincada no chão na qual são instalados os cabos elétricos e as lâmpadas de iluminação pública.}{pos.te}{0}
\verb{poste}{ó}{}{}{}{}{Cada uma das traves verticais do gol.}{pos.te}{0}
\verb{pôster}{}{}{}{}{s.m.}{Fotografia grande que se costuma pregar na parede.}{pôs.ter}{0}
\verb{postergar}{}{}{}{}{v.t.}{Colocar em segundo plano; desprezar, pospor.}{pos.ter.gar}{0}
\verb{postergar}{}{}{}{}{}{Deixar para depois; adiar.}{pos.ter.gar}{\verboinum{5}}
\verb{posteridade}{}{}{}{}{s.f.}{O tempo que ainda virá; futuro.}{pos.te.ri.da.de}{0}
\verb{posteridade}{}{}{}{}{}{Conjunto dos descendentes de um indivíduo; descendência.}{pos.te.ri.da.de}{0}
\verb{posteridade}{}{}{}{}{}{Glória futura; imortalidade.}{pos.te.ri.da.de}{0}
\verb{posterior}{ô}{}{}{}{adj.2g.}{Que se situa atrás ou na parte de trás.}{pos.te.ri.or}{0}
\verb{posterior}{ô}{}{}{}{}{Que acontece depois.}{pos.te.ri.or}{0}
\verb{posterior}{ô}{}{}{}{}{Seguinte, subsequente.}{pos.te.ri.or}{0}
\verb{póstero}{}{}{}{}{adj.}{Que está por acontecer; futuro.}{pós.te.ro}{0}
\verb{postiço}{}{}{}{}{adj.}{Que se pode pôr ou tirar.}{pos.ti.ço}{0}
\verb{postiço}{}{}{}{}{}{Que não é natural; falso.}{pos.ti.ço}{0}
\verb{postiço}{}{}{}{}{}{Que se acrescenta depois de obra pronta.}{pos.ti.ço}{0}
\verb{postigo}{}{}{}{}{s.m.}{Pequena janela.}{pos.ti.go}{0}
\verb{postigo}{}{}{}{}{}{Guichê.}{pos.ti.go}{0}
\verb{posto}{ô}{}{"-s ⟨ó⟩}{"-a ⟨ó⟩}{adj.}{Que foi colocado em determinado lugar.}{pos.to}{0}
\verb{posto}{ô}{}{"-s ⟨ó⟩}{"-a ⟨ó⟩}{}{Declarado, dito.}{pos.to}{0}
\verb{posto}{ô}{}{"-s ⟨ó⟩}{"-a ⟨ó⟩}{s.m.}{Lugar onde se fazem serviços de uma determinada área.}{pos.to}{0}
\verb{posto}{ô}{}{"-s ⟨ó⟩}{"-a ⟨ó⟩}{}{Posição numa carreira; cargo.}{pos.to}{0}
\verb{postulado}{}{}{}{}{s.m.}{Fato ou preceito reconhecido sem prévia demonstração.}{pos.tu.la.do}{0}
\verb{postular}{}{}{}{}{v.t.}{Pedir com instância; suplicar, rogar.}{pos.tu.lar}{0}
\verb{postular}{}{}{}{}{}{Requerer, documentando a alegação.}{pos.tu.lar}{\verboinum{1}}
\verb{póstumo}{}{}{}{}{adj.}{Que é posterior à morte.}{pós.tu.mo}{0}
\verb{postura}{}{}{}{}{s.f.}{Modo de manter o corpo; porte.}{pos.tu.ra}{0}
\verb{postura}{}{}{}{}{}{Cada uma das regras escritas de um município.}{pos.tu.ra}{0}
\verb{postura}{}{}{}{}{}{A deposição de ovos por animais.}{pos.tu.ra}{0}
\verb{posudo}{}{}{}{}{adj.}{Que faz muita pose; orgulhoso, vaidoso.}{po.su.do}{0}
\verb{potassa}{}{Quím.}{}{}{s.f.}{Nome comum dos diversos derivados do potássio.}{po.tas.sa}{0}
\verb{potássio}{}{Quím.}{}{}{s.m.}{Elemento químico radioativo, branco"-prateado, muito leve, mole e reativo, do grupo dos metais alcalinos; reage violentamente com a água; utilizado como fertilizante. \elemento{19}{39.0983}{K}.}{po.tás.sio}{0}
\verb{potável}{}{}{"-eis}{}{adj.2g.}{Que é saudável para o consumo.}{po.tá.vel}{0}
\verb{pote}{ó}{}{}{}{s.m.}{Vaso bojudo para conter líquido, mantimento etc.}{po.te}{0}
\verb{pote}{ó}{}{}{}{}{Recipiente pequeno de boca larga e com tampa.}{po.te}{0}
\verb{potência}{}{}{}{}{s.f.}{Capacidade de ação; força.}{po.tên.cia}{0}
\verb{potência}{}{}{}{}{}{Poder.}{po.tên.cia}{0}
\verb{potenciação}{}{Mat.}{"-ões}{}{s.f.}{Operação de elevar um número ou expressão a uma dada potência.}{po.ten.ci.a.ção}{0}
\verb{potencial}{}{}{"-ais}{}{adj.2g.}{Que só existe como possibilidade}{po.ten.ci.al}{0}
\verb{potencial}{}{}{"-ais}{}{s.m.}{Conjunto das capacidades de uma pessoa.}{po.ten.ci.al}{0}
\verb{potenciômetro}{}{}{}{}{s.m.}{Instrumento para medir diferenças de potencial elétrico.}{po.ten.ci.ô.me.tro}{0}
\verb{potentado}{}{}{}{}{s.m.}{Soberano de grande autoridade ou poder material}{po.ten.ta.do}{0}
\verb{potentado}{}{Por ext.}{}{}{}{Indivíduo muito influente ou poderoso.}{po.ten.ta.do}{0}
\verb{potente}{}{}{}{}{adj.2g.}{Que tem potência; poderoso.}{po.ten.te}{0}
\verb{potente}{}{}{}{}{}{Vigoroso, forte.}{po.ten.te}{0}
\verb{potestade}{}{}{}{}{s.f.}{Potência, poder.}{po.tes.ta.de}{0}
\verb{potestade}{}{Por ext.}{}{}{}{A divindade, o poder supremo.}{po.tes.ta.de}{0}
\verb{potiguar}{}{}{}{}{adj.2g.}{Relativo ao Rio Grande do Norte; norte"-rio"-grandense; rio"-grandense"-do"-norte.}{po.ti.guar}{0}
\verb{potiguar}{}{}{}{}{s.2g.}{Indivíduo natural ou habitante desse estado. }{po.ti.guar}{0}
\verb{potó}{}{Zool.}{}{}{s.m.}{Nome comum a alguns insetos que segregam um líquido cáustico.}{po.tó}{0}
\verb{potó}{}{}{}{}{s.m.}{Espada indiana, de dois gumes, usada em algumas festividades.}{po.tó}{0}
\verb{potoca}{ó}{Bras.}{}{}{s.f.}{Mentira, lorota.}{po.to.ca}{0}
\verb{potoqueiro}{ê}{}{}{}{adj.}{Que mente; mentiroso.}{po.to.quei.ro}{0}
\verb{potranca}{}{}{}{}{s.f.}{Potra de menos de dois anos.}{po.tran.ca}{0}
\verb{potranco}{}{}{}{}{s.m.}{Potro de menos de dois ou três anos.}{po.tran.co}{0}
\verb{potro}{ô}{}{}{}{s.m.}{Cavalo com menos de quatro anos.}{po.tro}{0}
\verb{pouca"-vergonha}{}{}{poucas"-vergonhas}{}{s.f.}{Ação vergonhosa, indecente.}{pou.ca"-ver.go.nha}{0}
\verb{pouca"-vergonha}{}{}{poucas"-vergonhas}{}{}{Descaramento, despudor.}{pou.ca"-ver.go.nha}{0}
\verb{pouco}{ô}{}{}{}{pron.}{Em pequena quantidade.}{pou.co}{0}
\verb{pouco}{ô}{}{}{}{adv.}{Não muito.}{pou.co}{0}
\verb{pouco}{ô}{}{}{}{s.m.}{Pequena quantidade de alguma coisa.}{pou.co}{0}
\verb{pouco"-caso}{ô}{}{poucos"-casos ⟨ô⟩}{}{s.m.}{Falta de atenção; desdém, desprezo.}{pou.co"-ca.so}{0}
\verb{poupa}{ô}{}{}{}{s.f.}{Conjunto de penas que ficam no alto da cabeça de alguma aves; crista, penacho.}{pou.pa}{0}
\verb{poupado}{}{}{}{}{adj.}{Que é dado a poupar, que não é gastador; econômico.}{pou.pa.do}{0}
\verb{poupança}{}{}{}{}{s.f.}{Despesa moderada; economia.}{pou.pan.ça}{0}
\verb{poupança}{}{Pop.}{}{}{}{Caderneta de poupança.}{pou.pan.ça}{0}
\verb{poupar}{}{}{}{}{v.t.}{Deixar de gastar; economizar.}{pou.par}{0}
\verb{poupar}{}{}{}{}{}{Tratar alguém com clemência.}{pou.par}{0}
\verb{poupar}{}{}{}{}{}{Não atingir, deixar intacto.}{pou.par}{0}
\verb{poupar}{}{}{}{}{}{Proteger de esforço, choque emocional etc.}{pou.par}{\verboinum{1}}
\verb{pousada}{}{}{}{}{s.f.}{Local em que se dorme uma noite.}{pou.sa.da}{0}
\verb{pousada}{}{}{}{}{}{Lugar que aceita hóspedes para dormir; hospedaria.}{pou.sa.da}{0}
\verb{pousar}{}{}{}{}{v.t.}{Terminar o voo descendo em algum lugar; aterrissar.}{pou.sar}{0}
\verb{pousar}{}{}{}{}{}{Colocar algo sobre uma superfície.}{pou.sar}{0}
\verb{pousar}{}{}{}{}{v.i.}{Hospedar"-se por breve tempo.}{pou.sar}{\verboinum{1}}
\verb{pouso}{ô}{}{}{}{s.m.}{Lugar onde uma ave descansa de voar.}{pou.so}{0}
\verb{pouso}{ô}{}{}{}{}{Aterrissagem.}{pou.so}{0}
\verb{pouso}{ô}{}{}{}{}{Pousada.}{pou.so}{0}
\verb{povaréu}{}{}{}{}{s.m.}{Grande quantidade de pessoas; multidão.}{po.va.réu}{0}
\verb{poviléu}{}{Pejor.}{}{}{s.m.}{A camada mais baixa da sociedade; ralé.}{po.vi.léu}{0}
\verb{povo}{ô}{}{}{}{s.m.}{Grupo de indivíduos que formam uma nação.}{po.vo}{0}
\verb{povo}{ô}{}{}{}{}{Multidão.}{po.vo}{0}
\verb{povo}{ô}{}{}{}{}{Classe social menos favorecida.}{po.vo}{0}
\verb{povoação}{}{}{"-ões}{}{s.f.}{Ato ou efeito de povoar; povoamento.}{po.vo.a.ção}{0}
\verb{povoação}{}{}{"-ões}{}{}{Habitantes de uma região, cidade, vila ou aldeia.}{po.vo.a.ção}{0}
\verb{povoação}{}{}{"-ões}{}{}{Lugar povoado.}{po.vo.a.ção}{0}
\verb{povoado}{}{}{}{}{adj.}{Que se povoou; habitado.}{po.vo.a.do}{0}
\verb{povoado}{}{}{}{}{s.m.}{Lugar que reúne poucas casas habitadas; vilarejo, aldeia.}{po.vo.a.do}{0}
\verb{povoador}{ô}{}{}{}{adj.}{Que povoa.}{po.vo.a.dor}{0}
\verb{povoador}{ô}{}{}{}{}{Imigrante, colonizador.}{po.vo.a.dor}{0}
\verb{povoamento}{}{}{}{}{s.m.}{Ato ou efeito de povoar; povoação.}{po.vo.a.men.to}{0}
\verb{povoar}{}{}{}{}{v.t.}{Tornar habitado.}{po.vo.ar}{\verboinum{7}}
\verb{Pr}{}{Quím.}{}{}{}{Símb. do \textit{praseodímio}. }{Pr}{0}
\verb{PR}{}{}{}{}{}{Sigla do estado do Paraná.}{PR}{0}
\verb{pra}{}{Pop.}{}{}{prep.}{Forma abreviada de \textit{para}.}{pra}{0}
\verb{praça}{}{}{}{}{s.f.}{Área urbanizada, com árvores, para descanso e lazer.}{pra.ça}{0}
\verb{praça}{}{}{}{}{}{O comércio local.}{pra.ça}{0}
\verb{praça}{}{}{}{}{}{Local público onde estacionam carros de aluguel.}{pra.ça}{0}
\verb{praça}{}{}{}{}{s.m.}{Militar que tem posto inferior ao de tenente.}{pra.ça}{0}
\verb{pracinha}{}{}{}{}{s.f.}{Praça pequena.}{pra.ci.nha}{0}
\verb{pracinha}{}{}{}{}{s.m.}{Soldado da Força Expedicionária Brasileira, que lutou na Segunda Guerra Mundial.}{pra.ci.nha}{0}
\verb{pracista}{}{}{}{}{s.2g.}{Vendedor de uma firma que trabalha em determinada praça.   }{pra.cis.ta}{0}
\verb{pradaria}{}{}{}{}{s.f.}{Grande extensão de terrenos planos.}{pra.da.ri.a}{0}
\verb{prado}{}{}{}{}{s.m.}{Campo próprio para pastagem.}{pra.do}{0}
\verb{prado}{}{}{}{}{}{Local próprio para corridas de cavalos.}{pra.do}{0}
\verb{praga}{}{}{}{}{s.f.}{Doença que se espalha rapidamente.}{pra.ga}{0}
\verb{praga}{}{}{}{}{}{Erva ou bicho que prejudica a plantação.}{pra.ga}{0}
\verb{praga}{}{}{}{}{}{Desgraça que se deseja para alguém; maldição.}{pra.ga}{0}
\verb{pragmática}{}{}{}{}{s.f.}{Conjunto de regras relativas à prática social, em oposição a palavras e fórmulas; praxe.}{prag.má.ti.ca}{0}
\verb{pragmático}{}{}{}{}{adj.}{Relativo à pragmática.}{prag.má.ti.co}{0}
\verb{pragmático}{}{}{}{}{}{Costumeiro, habitual, prático.}{prag.má.ti.co}{0}
\verb{pragmatismo}{}{}{}{}{s.m.}{Forma de considerar as coisas de um ponto de vista prático, não dogmático.}{prag.ma.tis.mo}{0}
\verb{praguejar}{}{}{}{}{v.t.}{Rogar pragas; amaldiçoar, imprecar.}{pra.gue.jar}{\verboinum{1}}
\verb{praia}{}{}{}{}{s.f.}{Faixa de terra, em decline suave, geralmente coberta de areia, mas também de terra ou cascalho fino, que confina com o mar.}{prai.a}{0}
\verb{praia}{}{}{}{}{}{Litoral.}{prai.a}{0}
\verb{praia}{}{Fig.}{}{}{}{Área de competência, de interesse; especialidade.}{prai.a}{0}
\verb{praiano}{}{}{}{}{adj.}{Relativo a praia; praieiro.}{prai.a.no}{0}
\verb{praiano}{}{}{}{}{s.m.}{Pessoa que mora na praia, ou no litoral; praieiro.}{prai.a.no}{0}
\verb{praieiro}{ê}{}{}{}{adj. e s.m.  }{Praiano.}{prai.ei.ro}{0}
\verb{prancha}{}{}{}{}{s.f.}{Tábua grande, grossa e larga.}{pran.cha}{0}
\verb{prancha}{}{}{}{}{}{Peça larga e plana feita de material leve e usada para esportes aquáticos, como o surfe.}{pran.cha}{0}
\verb{pranchada}{}{}{}{}{s.f.}{Golpe ou pancada com prancha.}{pran.cha.da}{0}
\verb{prancheta}{ê}{}{}{}{s.f.}{Mesa própria para desenhar.}{pran.che.ta}{0}
\verb{prancheta}{ê}{}{}{}{}{Pequena prancha usada como suporte para escrever.}{pran.che.ta}{0}
\verb{pranteado}{}{}{}{}{adj.}{Que é ou foi objeto de pranto; chorado.  }{pran.te.a.do}{0}
\verb{prantear}{}{}{}{}{v.i.}{Verter pranto; chorar.}{pran.te.ar}{\verboinum{4}}
\verb{pranto}{}{}{}{}{s.m.}{Ato de prantear; choro.}{pran.to}{0}
\verb{praseodímio}{}{Quím.}{}{}{s.m.}{Elemento químico metálico, prateado, mole, maleável, reativo, da família dos lantanídeos (terras"-raras), usado na fabricação de pedras de isqueiro e em cerâmica, para colorir vidros. \elemento{59}{140.90765}{Pr}.}{pra.se.o.dí.mio}{0}
\verb{prasiodímio}{}{}{}{}{}{Var. de \textit{praseodímio}.}{pra.si.o.dí.mio}{0}
\verb{prata}{}{Quím.}{}{}{s.f.}{Elemento químico metálico, branco, brilhante, dúctil, maleável, excelente condutor de calor e eletricidade, usado em ligas preciosas com o ouro, a platina e o cobre e sob a forma de compostos. \elemento{47}{107.8682}{Ag}.}{pra.ta}{0}
\verb{pratada}{}{}{}{}{s.f.}{Conteúdo de um prato cheio.}{pra.ta.da}{0}
\verb{prataria}{}{}{}{}{s.f.}{Conjunto de objetos de prata.}{pra.ta.ri.a}{0}
\verb{prataria}{}{}{}{}{}{Grande quantidade de pratos.}{pra.ta.ri.a}{0}
\verb{pratarraz}{}{}{}{}{s.m.}{Grande quantidade de comida.}{pra.tar.raz}{0}
\verb{prateado}{}{}{}{}{adj.}{Que tem a cor da prata.}{pra.te.a.do}{0}
\verb{prateado}{}{}{}{}{}{Diz"-se dessa cor.}{pra.te.a.do}{0}
\verb{pratear}{}{}{}{}{v.t.}{Cobrir com camada de prata.}{pra.te.ar}{0}
\verb{pratear}{}{}{}{}{}{Dar a cor ou o brilho da prata.}{pra.te.ar}{\verboinum{4}}
\verb{prateleira}{ê}{}{}{}{s.f.}{Tábua horizontal presa à parede, na qual se colocam os mais variados objetos.}{pra.te.lei.ra}{0}
\verb{prateleira}{ê}{}{}{}{}{Divisão horizontal de uma estante, um armário etc.}{pra.te.lei.ra}{0}
\verb{prática}{}{}{}{}{s.f.}{Ato ou efeito de praticar; exercício.}{prá.ti.ca}{0}
\verb{prática}{}{}{}{}{}{Habilidade que se adquire de tanto se repetir o que se faz.}{prá.ti.ca}{0}
\verb{praticante}{}{}{}{}{adj.2g.}{Que pratica algo.}{pra.ti.can.te}{0}
\verb{praticante}{}{}{}{}{}{Diz"-se de pessoa que está se exercitando numa atividade.}{pra.ti.can.te}{0}
\verb{praticante}{}{}{}{}{}{Que obedece a todos os preceitos de uma religião.}{pra.ti.can.te}{0}
\verb{praticar}{}{}{}{}{v.t.}{Realizar ato; executar, fazer.}{pra.ti.car}{0}
\verb{praticar}{}{}{}{}{}{Exercer uma profissão.}{pra.ti.car}{0}
\verb{praticar}{}{}{}{}{}{Ganhar experiência fazendo alguma coisa.}{pra.ti.car}{\verboinum{2}}
\verb{praticável}{}{}{"-eis}{}{adj.2g.}{Que se pode pôr em prática.}{pra.ti.cá.vel}{0}
\verb{praticável}{}{}{"-eis}{}{s.m.}{Suporte ou plataforma móvel usada em cenários.}{pra.ti.cá.vel}{0}
\verb{prático}{}{}{}{}{adj.}{Que se refere à prática.}{prá.ti.co}{0}
\verb{prático}{}{}{}{}{}{Que é fácil de usar; funcional.}{prá.ti.co}{0}
\verb{prático}{}{}{}{}{s.m.}{Piloto que conduz o navio em determinada área.}{prá.ti.co}{0}
\verb{prático}{}{}{}{}{}{Indivíduo que trabalha numa profissão sem ter o diploma.}{prá.ti.co}{0}
\verb{prato}{}{}{}{}{s.m.}{Peça de louça ou de outro material, geralmente de forma circular, de bordas muito baixas, em que se come.}{pra.to}{0}
\verb{prato}{}{}{}{}{}{O conteúdo posto nessa peça.}{pra.to}{0}
\verb{prato}{}{}{}{}{}{Cada uma das comidas preparadas para uma refeição.}{pra.to}{0}
\verb{prato}{}{}{}{}{}{Cada uma das conchas de uma balança.}{pra.to}{0}
\verb{prato"-feito}{ê}{}{pratos"-feitos}{}{s.m.}{Comida trivial e de baixo preço, e que já vem servida no prato.}{pra.to"-fei.to}{0}
\verb{praxe}{ch}{}{}{}{s.f.}{Aquilo que habitualmente se faz; costume, prática, rotina.}{pra.xe}{0}
\verb{prazenteiro}{ê}{}{}{}{adj.}{Simpático, adulador.}{pra.zen.tei.ro}{0}
\verb{prazenteiro}{ê}{}{}{}{}{Que tem ou manifesta satisfação; alegre, animado.}{pra.zen.tei.ro}{0}
\verb{prazer}{ê}{}{}{}{v.t.}{Causar satisfação, agradar; aprazer.}{pra.zer}{\verboinum{14}}
\verb{prazer}{ê}{}{}{}{s.m.}{Sentimento de alegria, de satisfação.}{pra.zer}{0}
\verb{prazer}{ê}{}{}{}{}{Aquilo que causa prazer.}{pra.zer}{0}
\verb{prazeroso}{ô}{}{"-osos ⟨ó⟩}{"-osa ⟨ó⟩}{adj.}{Com prazer, com satisfação, com boa vontade.}{pra.ze.ro.so}{0}
\verb{prazeroso}{ô}{}{"-osos ⟨ó⟩}{"-osa ⟨ó⟩}{}{Que causa prazer.}{pra.ze.ro.so}{0}
\verb{prazo}{}{}{}{}{s.m.}{Tempo determinado.}{pra.zo}{0}
\verb{prazo}{}{}{}{}{}{Espaço de tempo que se tem para realizar algo.}{pra.zo}{0}
\verb{pré}{}{}{}{}{s.m.}{O pagamento diário de um soldado.}{pré}{0}
\verb{preá}{}{Zool.}{}{}{s.2g.}{Pequeno roedor; cobaia.}{pre.á}{0}
\verb{preamar}{}{}{}{}{s.f.}{Nível máximo da maré.}{pre.a.mar}{0}
\verb{preâmbulo}{}{}{}{}{s.m.}{Relatório que antecede uma lei ou decreto.}{pre.âm.bu.lo}{0}
\verb{preâmbulo}{}{}{}{}{}{Palavreado vago que não vai diretamente ao fato.}{pre.âm.bu.lo}{0}
\verb{prear}{}{}{}{}{v.t.}{Tornar prisioneiro ou cativo; prender, aprisionar.}{pre.ar}{0}
\verb{prear}{}{}{}{}{v.i.}{Fazer presa.}{pre.ar}{\verboinum{4}}
\verb{prebenda}{}{}{}{}{s.f.}{Renda da igreja.}{pre.ben.da}{0}
\verb{prebenda}{}{Fig.}{}{}{}{Ocupação rendosa e de pouco trabalho.}{pre.ben.da}{0}
\verb{prebenda}{}{Fig.}{}{}{}{Tarefa ou ocupação trabalhosa, desagradável.}{pre.ben.da}{0}
\verb{precariedade}{}{}{}{}{s.f.}{Característica do que é ou está precário.}{pre.ca.ri.e.da.de}{0}
\verb{precário}{}{}{}{}{adj.}{Que tem pouca ou nenhuma estabilidade; incerto.}{pre.cá.rio}{0}
\verb{precário}{}{}{}{}{}{Com pouca resistência; frágil.}{pre.cá.rio}{0}
\verb{precário}{}{}{}{}{}{Que está em más condições; deficiente.}{pre.cá.rio}{0}
\verb{precatar}{}{}{}{}{v.t.}{Pôr de sobreaviso; prevenir, precaver.}{pre.ca.tar}{\verboinum{1}}
\verb{precatória}{}{Jur.}{}{}{s.f.}{Documento pelo qual um órgão judicial demanda a outro a prática de um ato judicial.}{pre.ca.tó.ria}{0}
\verb{precatório}{}{}{}{}{adj.}{Em que se pede algo.}{pre.ca.tó.rio}{0}
\verb{precatório}{}{}{}{}{s.m.}{Documento ou carta em que se solicita alguma coisa.}{pre.ca.tó.rio}{0}
\verb{precaução}{}{}{"-ões}{}{s.f.}{Cada uma das medidas tomadas para se evitar alguma coisa desagradável; cuidado, providência.}{pre.cau.ção}{0}
\verb{precaver}{ê}{}{}{}{v.t.}{Tomar medidas antecipadas para evitar algo ruim; acautelar, prevenir.}{pre.ca.ver}{\verboinum{12}}
\verb{precavido}{}{}{}{}{adj.}{Que é cauteloso, prevenido.}{pre.ca.vi.do}{0}
\verb{prece}{é}{}{}{}{s.f.}{Súplica dirigida a Deus, aos santos ou a uma divindade; oração, reza.}{pre.ce}{0}
\verb{precedência}{}{}{}{}{s.f.}{Situação do que vem antes, do que precede.}{pre.ce.dên.cia}{0}
\verb{precedência}{}{}{}{}{}{Preferência, primazia, prioridade.}{pre.ce.dên.cia}{0}
\verb{precedente}{}{}{}{}{adj.2g.}{Que precede; antecedente.}{pre.ce.den.te}{0}
\verb{precedente}{}{}{}{}{s.m.}{Acontecimento que serve de modelo para se julgar alguma coisa do presente.}{pre.ce.den.te}{0}
\verb{preceder}{ê}{}{}{}{v.t.}{Estar adiante de; anteceder.}{pre.ce.der}{0}
\verb{preceder}{ê}{}{}{}{}{Ocorrer ou existir antes de. }{pre.ce.der}{\verboinum{12}}
\verb{preceito}{ê}{}{}{}{s.m.}{Regra de procedimento; norma.}{pre.cei.to}{0}
\verb{preceituar}{}{}{}{}{v.t.}{Estabelecer como preceito; ordenar, determinar.}{pre.cei.tu.ar}{0}
\verb{preceituar}{}{}{}{}{}{Estabelecer regras, dar ordens ou instruções.}{pre.cei.tu.ar}{\verboinum{1}}
\verb{preceituário}{}{}{}{}{s.m.}{Conjunto de regras ou normas.}{pre.cei.tu.á.rio}{0}
\verb{preceptor}{ô}{}{}{}{s.m.}{Indivíduo que dá instruções, preceitos; educador, mestre, mentor.}{pre.cep.tor}{0}
\verb{preciosidade}{}{}{}{}{s.f.}{Qualidade de precioso; raridade.}{pre.ci.o.si.da.de}{0}
\verb{preciosismo}{}{}{}{}{s.m.}{Afetação, rebuscamento no modo de se expressar.}{pre.ci.o.sis.mo}{0}
\verb{precioso}{ô}{}{"-osos ⟨ó⟩}{"-osa ⟨ó⟩}{adj.}{Que é de grande valor; raro, importante, útil.}{pre.ci.o.so}{0}
\verb{precipício}{}{}{}{}{s.m.}{Abertura natural da terra muito profunda; despenhadeiro, abismo.}{pre.ci.pí.cio}{0}
\verb{precipitação}{}{}{"-ões}{}{s.f.}{Ato ou efeito de precipitar.}{pre.ci.pi.ta.ção}{0}
\verb{precipitação}{}{}{"-ões}{}{}{Queda, descida.}{pre.ci.pi.ta.ção}{0}
\verb{precipitação}{}{}{"-ões}{}{}{Pressa em se decidir algo; irreflexão.}{pre.ci.pi.ta.ção}{0}
\verb{precipitado}{}{}{}{}{adj.}{Que se precipitou; lançado do alto; caído.}{pre.ci.pi.ta.do}{0}
\verb{precipitado}{}{}{}{}{}{Irrefletido, imprudente.}{pre.ci.pi.ta.do}{0}
\verb{precipitar}{}{}{}{}{v.t.}{Lançar de cima para baixo; fazer cair.}{pre.ci.pi.tar}{0}
\verb{precipitar}{}{}{}{}{v.pron.}{Agir de forma apressada e sem pensar.}{pre.ci.pi.tar}{\verboinum{1}}
\verb{precípite}{}{}{}{}{adj.2g.}{Que corre o risco de se precipitar; apressado, rápido.}{pre.cí.pi.te}{0}
\verb{precípuo}{}{}{}{}{adj.}{Que é o mais importante; essencial, principal.}{pre.cí.pu.o}{0}
\verb{precisado}{}{}{}{}{adj.}{Que precisa muito; necessitado, carente.}{pre.ci.sa.do}{0}
\verb{precisão}{}{}{"-ões}{}{s.f.}{Falta do necessário; carência, necessidade.}{pre.ci.são}{0}
\verb{precisão}{}{}{"-ões}{}{}{Exatidão, pontualidade, perfeição.}{pre.ci.são}{0}
\verb{precisar}{}{}{}{}{v.t.}{Ter necessidade; carecer, necessitar.}{pre.ci.sar}{0}
\verb{precisar}{}{}{}{}{}{Determinar com exatidão.}{pre.ci.sar}{0}
\verb{precisar}{}{}{}{}{}{Detalhar, particularizar.}{pre.ci.sar}{\verboinum{1}}
\verb{preciso}{}{}{}{}{adj.}{Que se deve ter; indispensável, necessitado.}{pre.ci.so}{0}
\verb{preciso}{}{}{}{}{}{Bem definido; exato, rigoroso.}{pre.ci.so}{0}
\verb{precitado}{}{}{}{}{adj.}{Que foi citado anteriormente.}{pre.ci.ta.do}{0}
\verb{precito}{}{}{}{}{adj.}{Diz"-se daquele que foi condenado; réprobo, maldito.}{pre.ci.to}{0}
\verb{preclaro}{}{}{}{}{adj.}{Que se distingue por algo; notável, famoso, célebre.}{pre.cla.ro}{0}
\verb{preço}{ê}{}{}{}{s.m.}{Quantidade de dinheiro que se paga por algo; valor.}{pre.ço}{0}
\verb{preço}{ê}{}{}{}{}{Exigência, custo, consequência.}{pre.ço}{0}
\verb{precoce}{ó}{}{}{}{adj.2g.}{Que amadurece ou se desenvolve antes do tempo; prematuro, temporão.}{pre.co.ce}{0}
\verb{precocidade}{}{}{}{}{s.f.}{Qualidade do que é precoce, prematuro.}{pre.co.ci.da.de}{0}
\verb{pré"-colombiano}{}{}{}{}{adj.}{Que é anterior à chegada de Cristóvão Colombo ao continente americano.}{pré"-co.lom.bi.a.no}{0}
\verb{preconceber}{ê}{}{}{}{v.t.}{Conceber ou planejar com antecipação.}{pre.con.ce.ber}{\verboinum{12}}
\verb{preconcebido}{}{}{}{}{adj.}{Concebido ou planejado com antecipação ou precipitação.}{pre.con.ce.bi.do}{0}
\verb{preconceito}{ê}{}{}{}{s.m.}{Conceito formado com precipitação, sem fundamento.}{pre.con.cei.to}{0}
\verb{preconceito}{ê}{}{}{}{}{Opinião desfavorável e até hostil que se tem sobre algo ou alguém; intolerância.}{pre.con.cei.to}{0}
\verb{preconceituoso}{ô}{}{"-osos ⟨ó⟩}{"-osa ⟨ó⟩}{adj.}{Que revela preconceito; parcial, intolerante.}{pre.con.cei.tu.o.so}{0}
\verb{preconizar}{}{}{}{}{v.t.}{Apregoar com louvor; aconselhar, recomendar.}{pre.co.ni.zar}{\verboinum{1}}
\verb{precursor}{ô}{}{}{}{adj.}{Que antecede, que anuncia o que vem depois.}{pre.cur.sor}{0}
\verb{predador}{ô}{}{}{}{adj.}{Diz"-se do ser que destrói outro violentamente.}{pre.da.dor}{0}
\verb{pré"-datado}{}{}{pré"-datados}{}{adj.}{Diz"-se de cheque que foi datado para o futuro.}{pré"-da.ta.do}{0}
\verb{pré"-datar}{}{}{}{}{v.t.}{Colocar data futura.}{pré"-da.tar}{\verboinum{1}}
\verb{predatório}{}{}{}{}{adj.}{Relativo a predador.}{pre.da.tó.rio}{0}
\verb{predatório}{}{}{}{}{}{Que envolve destruição.}{pre.da.tó.rio}{0}
\verb{predecessor}{ô}{}{}{}{adj.}{Que precede no tempo; antecessor.}{pre.de.ces.sor}{0}
\verb{predestinação}{}{}{"-ões}{}{s.f.}{Ato ou efeito de predestinar.}{pre.des.ti.na.ção}{0}
\verb{predestinação}{}{}{"-ões}{}{}{Determinação antecipada do destino de alguém.}{pre.des.ti.na.ção}{0}
\verb{predestinado}{}{}{}{}{adj.}{Que está previamente destinado a alguma coisa.}{pre.des.ti.na.do}{0}
\verb{predestinar}{}{}{}{}{v.t.}{Destinar ou escolher com antecipação.}{pre.des.ti.nar}{\verboinum{1}}
\verb{predeterminado}{}{}{}{}{adj.}{Que foi determinado de antemão.}{pre.de.ter.mi.na.do}{0}
\verb{predeterminar}{}{}{}{}{v.t.}{Determinar, planejar com antecipação.}{pre.de.ter.mi.nar}{\verboinum{1}}
\verb{predial}{}{}{"-ais}{}{adj.2g.}{Relativo a prédio, edifício, casa etc.}{pre.di.al}{0}
\verb{prédica}{}{}{}{}{s.f.}{Discurso religioso; sermão, pregação.}{pré.di.ca}{0}
\verb{predicação}{}{}{"-ões}{}{s.f.}{Ato ou efeito de predicar; sermão, prédica.}{pre.di.ca.ção}{0}
\verb{predicação}{}{Gram.}{"-ões}{}{}{Em uma oração, relação semântica entre um verbo e um complemento.}{pre.di.ca.ção}{0}
\verb{predicado}{}{}{}{}{s.m.}{Qualidade inerente a um ser; característica, atributo.}{pre.di.ca.do}{0}
\verb{predicado}{}{Gram.}{}{}{}{Em uma oração, aquilo que se refere ao sujeito.}{pre.di.ca.do}{0}
\verb{predição}{}{}{"-ões}{}{s.f.}{Ato ou efeito de predizer; previsão, prognóstico.}{pre.di.ção}{0}
\verb{predicativo}{}{Gram.}{}{}{adj.}{Em uma oração, diz"-se de qualidade que se atribui ao sujeito ou ao objeto.}{pre.di.ca.ti.vo}{0}
\verb{predileção}{}{}{"-ões}{}{s.f.}{Preferência por algo ou alguém; escolha, inclinação.}{pre.di.le.ção}{0}
\verb{predileto}{é}{}{}{}{adj.}{Que é o mais querido; preferido, favorito.}{pre.di.le.to}{0}
\verb{prédio}{}{}{}{}{s.m.}{Construção de vários andares para fins residenciais, comerciais ou industriais; edifício.}{pré.dio}{0}
\verb{predisponente}{}{}{}{}{adj.2g.}{Que predispõe, que cria condições.}{pre.dis.po.nen.te}{0}
\verb{predispor}{}{}{}{}{v.t.}{Dispor antecipadamente.}{pre.dis.por}{0}
\verb{predispor}{}{}{}{}{}{Preparar de antemão.}{pre.dis.por}{\verboinum{60}}
\verb{predisposição}{}{}{"-ões}{}{s.f.}{Ato ou efeito de predispor; tendência natural; inclinação.}{pre.dis.po.si.ção}{0}
\verb{predisposto}{ô}{}{"-s ⟨ó⟩}{"-a ⟨ó⟩}{adj.}{Que se predispôs; propenso, inclinado.}{pre.dis.pos.to}{0}
\verb{predito}{}{}{}{}{adj.}{Que foi dito ou citado anteriormente.}{pre.di.to}{0}
\verb{predizer}{ê}{}{}{}{v.t.}{Dizer antecipadamente; prenunciar, prognosticar.}{pre.di.zer}{\verboinum{41}}
\verb{predominação}{}{}{"-ões}{}{s.f.}{Ato ou efeito de predominar; predominância, predomínio.}{pre.do.mi.na.ção}{0}
\verb{predominância}{}{}{}{}{s.f.}{Qualidade de predominante; predomínio.}{pre.do.mi.nân.cia}{0}
\verb{predominante}{}{}{}{}{adj.2g.}{Que predomina, prevalece; dominante.}{pre.do.mi.nan.te}{0}
\verb{predominar}{}{}{}{}{v.t.}{Ter mais domínio, mais influência; prevalecer, sobressair.}{pre.do.mi.nar}{\verboinum{1}}
\verb{predomínio}{}{}{}{}{s.m.}{Domínio sobre algo; supremacia, preponderância, ascendência.}{pre.do.mí.nio}{0}
\verb{pré"-eleitoral}{}{}{pré"-eleitorais}{}{adj.2g.}{Que antecede as eleições.}{pré"-e.lei.to.ral}{0}
\verb{preeminência}{}{}{}{}{s.f.}{Qualidade de preeminente; primazia, superioridade.}{pre.e.mi.nên.cia}{0}
\verb{preeminente}{}{}{}{}{adj.2g.}{Que ocupa uma posição mais elevada; superior, distinto, ilustre.}{pre.e.mi.nen.te}{0}
\verb{preencher}{ê}{}{}{}{v.t.}{Ocupar espaço, tempo, posição, função etc.}{pre.en.cher}{0}
\verb{preencher}{ê}{}{}{}{}{Encher completamente.}{pre.en.cher}{0}
\verb{preencher}{ê}{}{}{}{}{Cumprir plenamente as exigências.}{pre.en.cher}{\verboinum{12}}
\verb{preenchimento}{}{}{}{}{s.m.}{Ato ou efeito de preencher.}{pre.en.chi.men.to}{0}
\verb{preenchimento}{}{}{}{}{}{Aquilo que preenche.}{pre.en.chi.men.to}{0}
\verb{preensão}{}{}{"-ões}{}{s.f.}{Ato ou efeito de prender.}{pre.en.são}{0}
\verb{preênsil}{}{}{"-eis}{}{adj.2g.}{Que tem a faculdade de agarrar ou apanhar.}{pre.ên.sil}{0}
\verb{pré"-escola}{ó}{}{pré"-escolas ⟨ó⟩}{}{s.f.}{Educação formal dada a crianças antes que ingressem no ensino fundamental.}{pré"-es.co.la}{0}
\verb{pré"-escolar}{}{}{pré"-escolares}{}{adj.2g.}{Relativo à pré"-escola.}{pré"-es.co.lar}{0}
\verb{pré"-escolar}{}{}{pré"-escolares}{}{}{Que tem idade de frequentar a pré"-escola.}{pré"-es.co.lar}{0}
\verb{preestabelecer}{ê}{}{}{}{v.t.}{Estabelecer previamente; predeterminar, antecipar.}{pre.es.ta.be.le.cer}{\verboinum{15}}
\verb{pré"-estreia}{é}{}{pré"-estreias}{}{s.f.}{Apresentação de filme ou espetáculo antes da estreia no circuito comercial, feita somente para convidados.}{pré"-es.trei.a}{0}
\verb{preexistente}{z}{}{}{}{adj.2g.}{Que já existia antes de um momento determinado.}{pre.e.xis.ten.te}{0}
\verb{preexistir}{z}{}{}{}{v.i.}{Existir antes de um momento determinado.}{pre.e.xis.tir}{\verboinum{18}}
\verb{pré"-fabricado}{}{}{pré"-fabricados}{}{adj.}{Cujas peças são preparadas previamente para facilitar a montagem final.}{pré"-fa.bri.ca.do}{0}
\verb{prefaciar}{}{}{}{}{v.t.}{Escrever o prefácio de.}{pre.fa.ci.ar}{\verboinum{1}}
\verb{prefácio}{}{}{}{}{s.m.}{Texto que antecede uma obra, contendo explicações e comentários sobre ela.}{pre.fá.cio}{0}
\verb{prefeito}{ê}{}{}{}{s.m.}{Chefe do poder Executivo de um município.}{pre.fei.to}{0}
\verb{prefeitura}{}{}{}{}{s.f.}{Repartição pública onde funciona a administração municipal.}{pre.fei.tu.ra}{0}
\verb{prefeitura}{}{}{}{}{}{O cargo de prefeito.}{pre.fei.tu.ra}{0}
\verb{preferência}{}{}{}{}{s.f.}{Ato ou efeito de preferir.}{pre.fe.rên.cia}{0}
\verb{preferencial}{}{}{"-ais}{}{adj.2g.}{Que tem preferência, primazia.}{pre.fe.ren.ci.al}{0}
\verb{preferencial}{}{}{"-ais}{}{}{Diz"-se de via pública cujos veículos tem preferência de tráfego.}{pre.fe.ren.ci.al}{0}
\verb{preferente}{}{}{}{}{adj.2g.}{Que prefere.}{pre.fe.ren.te}{0}
\verb{preferir}{}{}{}{}{v.t.}{Dar primazia, prioridade a.}{pre.fe.rir}{0}
\verb{preferir}{}{}{}{}{}{Gostar mais de.}{pre.fe.rir}{0}
\verb{preferir}{}{}{}{}{}{Escolher, eleger.}{pre.fe.rir}{\verboinum{29}}
\verb{preferível}{}{}{"-eis}{}{adj.2g.}{Que é ou deve ser preferido.}{pre.fe.rí.vel}{0}
\verb{prefixação}{cs}{}{"-ões}{}{s.f.}{Ato ou efeito de prefixar.}{pre.fi.xa.ção}{0}
\verb{prefixação}{cs}{Gram.}{"-ões}{}{}{Processo de formação de palavras em que se adiciona um prefixo a uma palavra já existente.}{pre.fi.xa.ção}{0}
\verb{prefixar}{cs}{}{}{}{v.t.}{Fixar previamente; predeterminar.}{pre.fi.xar}{\verboinum{1}}
\verb{prefixo}{cs}{}{}{}{adj.}{Fixado previamente.}{pre.fi.xo}{0}
\verb{prefixo}{cs}{Gram.}{}{}{s.m.}{Elemento gramatical que se coloca antes do radical para formar uma nova palavra.}{pre.fi.xo}{0}
\verb{prega}{é}{}{}{}{s.f.}{Dobra permanente feita em um tecido.}{pre.ga}{0}
\verb{prega}{é}{}{}{}{}{Ruga, carquilha.}{pre.ga}{0}
\verb{pregação}{}{}{"-ões}{}{s.f.}{Ato ou efeito de pregar.}{pre.ga.ção}{0}
\verb{pregador}{ô}{}{}{}{s.m.}{Utensílio que serve para pregar, fixar.}{pre.ga.dor}{0}
\verb{pregador}{ô}{}{}{}{adj.}{Que faz pregação; orador, catequista.}{pre.ga.dor}{0}
\verb{pregão}{}{}{"-ões}{}{s.m.}{Ato ou efeito de apregoar.}{pre.gão}{0}
\verb{pregão}{}{}{"-ões}{}{}{Divulgação em voz alta de coisas a serem vendidas ou negociadas.}{pre.gão}{0}
\verb{pregão}{}{}{"-ões}{}{}{Local onde se negociam as ações na bolsa de valores.}{pre.gão}{0}
\verb{pregar}{}{}{}{}{v.t.}{Fixar, prender, unir, fincar.}{pre.gar}{\verboinum{5}}
\verb{pregar}{}{}{}{}{v.t.}{Preconizar, doutrinar, propagar, catequizar.}{pre.gar}{\verboinum{5}}
\verb{prego}{é}{}{}{}{s.m.}{Haste metálica com ponta afiada em uma extremidade e cabeça chata na outra, que é cravada no objeto a ser pregado.}{pre.go}{0}
\verb{pregoeiro}{ê}{}{}{}{adj.}{Que faz pregões.}{pre.go.ei.ro}{0}
\verb{pregoeiro}{ê}{}{}{}{s.m.}{Indivíduo que apregoa mercadorias em um leilão; leiloeiro.}{pre.go.ei.ro}{0}
\verb{pregresso}{é}{}{}{}{adj.}{Que ocorreu anteriormente.}{pre.gres.so}{0}
\verb{pregueado}{}{}{}{}{adj.}{Que tem pregas.}{pre.gue.a.do}{0}
\verb{pregueado}{}{}{}{}{s.m.}{Ornamento feito com pregas.}{pre.gue.a.do}{0}
\verb{preguear}{}{}{}{}{v.t.}{Fazer pregas em.}{pre.gue.ar}{\verboinum{4}}
\verb{preguiça}{}{}{}{}{s.f.}{Estado de moleza, desânimo.}{pre.gui.ça}{0}
\verb{preguiça}{}{Zool.}{}{}{}{Mamífero xenartro de pelagem densa, membros longos e patas com garras, que vive nas copas das árvores.}{pre.gui.ça}{0}
\verb{preguiçar}{}{}{}{}{v.i.}{Permanecer sem fazer nada, no ócio; descansar.}{pre.gui.çar}{\verboinum{3}}
\verb{preguiçosa}{ó}{}{}{}{s.f.}{Cadeira de encosto reclinável e apoio para as pernas.}{pre.gui.ço.sa}{0}
\verb{preguiçoso}{ô}{}{"-osos ⟨ó⟩}{"-osa ⟨ó⟩}{adj.}{Que tem preguiça; desanimado, indolente.}{pre.gui.ço.so}{0}
\verb{pregustar}{}{}{}{}{v.t.}{Provar, degustar.}{pre.gus.tar}{\verboinum{1}}
\verb{pré"-história}{}{}{}{}{s.f.}{Período da História anterior ao aparecimento da escrita.}{pré"-his.tó.ria}{0}
\verb{pré"-histórico}{}{}{pré"-históricos}{}{adj.}{Relativo à pré"-história.}{pré"-his.tó.ri.co}{0}
\verb{preito}{ê}{}{}{}{s.m.}{Manifestação de respeito; homenagem.}{prei.to}{0}
\verb{preito}{ê}{}{}{}{}{Ajuste, pacto.}{prei.to}{0}
\verb{prejudicar}{}{}{}{}{v.t.}{Causar dano ou prejuízo; danificar, afetar.}{pre.ju.di.car}{\verboinum{2}}
\verb{prejudicial}{}{}{"-ais}{}{adj.2g.}{Que prejudica; nocivo.}{pre.ju.di.ci.al}{0}
\verb{prejuízo}{}{}{}{}{s.m.}{Dano, perda, estrago.}{pre.ju.í.zo}{0}
\verb{prejulgar}{}{}{}{}{v.t.}{Julgar antecipadamente.}{pre.jul.gar}{\verboinum{5}}
\verb{prelado}{}{}{}{}{s.m.}{Título honorífico dado a alguns indivíduos eclesiásticos.}{pre.la.do}{0}
\verb{prelazia}{}{}{}{}{s.f.}{Cargo ou jurisdição de prelado.}{pre.la.zi.a}{0}
\verb{preleção}{}{}{"-ões}{}{s.f.}{Discurso ou conferência de caráter didático; lição.}{pre.le.ção}{0}
\verb{prelecionar}{}{}{}{}{v.t.}{Fazer preleções; lecionar, ensinar.}{pre.le.ci.o.nar}{\verboinum{1}}
\verb{prelibar}{}{}{}{}{v.t.}{Provar antes; degustar.}{pre.li.bar}{\verboinum{1}}
\verb{preliminar}{}{}{}{}{adj.2g.}{Que ocorre antes da etapa ou tarefa principal.}{pre.li.mi.nar}{0}
\verb{prélio}{}{}{}{}{s.m.}{Batalha, combate, luta.}{pré.lio}{0}
\verb{prelo}{é}{}{}{}{s.m.}{Máquina manual de impressão; prensa, impressora.}{pre.lo}{0}
\verb{preludiar}{}{}{}{}{v.t.}{Fazer prelúdio de; introduzir, iniciar.}{pre.lu.di.ar}{0}
\verb{preludiar}{}{}{}{}{}{Prenunciar.}{pre.lu.di.ar}{0}
\verb{preludiar}{}{}{}{}{}{Fazer prefácio de; prefaciar.}{pre.lu.di.ar}{\verboinum{1}}
\verb{prelúdio}{}{Mús.}{}{}{s.m.}{Introdução de peça musical.}{pre.lú.dio}{0}
\verb{prelúdio}{}{}{}{}{}{Aquilo que anuncia um acontecimento; prenúncio.}{pre.lú.dio}{0}
\verb{prelúdio}{}{}{}{}{}{Texto introdutório; prefácio.}{pre.lú.dio}{0}
\verb{prematuro}{}{}{}{}{adj.}{Que ocorre ou se manifesta antes do tempo previsto; precoce.}{pre.ma.tu.ro}{0}
\verb{premeditação}{}{}{"-ões}{}{s.f.}{Ato ou efeito de premeditar.}{pre.me.di.ta.ção}{0}
\verb{premeditar}{}{}{}{}{v.t.}{Planejar antecipadamente.}{pre.me.di.tar}{\verboinum{1}}
\verb{premência}{}{}{}{}{s.f.}{Qualidade de premente; urgência.}{pre.mên.cia}{0}
\verb{premente}{}{}{}{}{adj.2g.}{Que faz pressão, que aperta.}{pre.men.te}{0}
\verb{premente}{}{}{}{}{}{Urgente.}{pre.men.te}{0}
\verb{premer}{ê}{}{}{}{v.t.}{Fazer pressão em; espremer.}{pre.mer}{\verboinum{12}}
\verb{premiado}{}{}{}{}{adj.}{Que alcançou ou recebeu um prêmio.}{pre.mi.a.do}{0}
\verb{premiado}{}{}{}{}{}{Diz"-se de bilhete de loteria, número de rifa, que foi sorteado.}{pre.mi.a.do}{0}
\verb{premiar}{}{}{}{}{v.t.}{Dar prêmio ou galardão a.}{pre.mi.ar}{0}
\verb{premiar}{}{}{}{}{}{Pagar, recompensar, remunerar.}{pre.mi.ar}{\verboinum{1}}
\verb{premido}{}{}{}{}{adj.}{Pressionado, apertado, oprimido.}{pre.mi.do}{0}
\verb{prêmio}{}{}{}{}{s.m.}{Bem material ou moral recebido por um serviço prestado, por um trabalho executado, ou por méritos especiais; recompensa, galardão.}{prê.mio}{0}
\verb{premir}{}{}{}{}{v.t.}{Causar pressão; apertar, comprimir.}{pre.mir}{0}
\verb{premir}{}{}{}{}{}{Tornar estreito ou apertado; estreitar.}{pre.mir}{\verboinum{18}}
\verb{premissa}{}{}{}{}{s.f.}{Princípio que dá base a um raciocínio.}{pre.mis.sa}{0}
\verb{premissa}{}{Filos.}{}{}{}{Cada uma das proposições de um silogismo que levam a uma conclusão.}{pre.mis.sa}{0}
\verb{pré"-molar}{}{Anat.}{pré"-molares}{}{s.m.}{Dente localizado entre o canino e o molar.}{pré"-mo.lar}{0}
\verb{pré"-moldado}{}{}{pré"-moldados}{}{adj.}{Que foi previamente vazado em molde, para utilização posterior.}{pré"-mol.da.do}{0}
\verb{pré"-moldado}{}{}{pré"-moldados}{}{s.m.}{Bloco de concreto pré"-moldado.}{pré"-mol.da.do}{0}
\verb{premonição}{}{}{"-ões}{}{s.f.}{Pressentimento, palpite, intuição.}{pre.mo.ni.ção}{0}
\verb{premonição}{}{}{"-ões}{}{}{Acontecimento que deve ser tomado como aviso; presságio, advertência.}{pre.mo.ni.ção}{0}
\verb{premunir}{}{}{}{}{v.t.}{Evitar algo ou acautelar alguém contra.}{pre.mu.nir}{0}
\verb{premunir}{}{}{}{}{}{Prover com antecipação; munir, guarnecer.}{pre.mu.nir}{\verboinum{18}}
\verb{pré"-natal}{}{}{pré"-natais}{}{adj.2g.}{Que é anterior ao nascimento.}{pré"-na.tal}{0}
\verb{pré"-natal}{}{}{pré"-natais}{}{s.m.}{Acompanhamento médico durante a gravidez.}{pré"-na.tal}{0}
\verb{prenda}{}{}{}{}{s.f.}{Coisa que se oferece a alguém; brinde, presente.}{pren.da}{0}
\verb{prenda}{}{}{}{}{}{Habilidade que distingue uma pessoa; qualidade. (Mais usada no plural.)}{pren.da}{0}
\verb{prendado}{}{}{}{}{adj.}{Que tem muitas qualidades, habilidades.}{pren.da.do}{0}
\verb{prendar}{}{}{}{}{v.t.}{Ofertar prendas a; presentear.}{pren.dar}{0}
\verb{prendar}{}{}{}{}{}{Tornar hábil, dotar de capacidade.}{pren.dar}{\verboinum{1}}
\verb{prender}{ê}{}{}{}{v.t.}{Ligar firmemente uma coisa a outra; atar, fixar.}{pren.der}{0}
\verb{prender}{ê}{}{}{}{}{Privar da liberdade; encarcerar, aprisionar.}{pren.der}{0}
\verb{prender}{ê}{}{}{}{}{Fazer ficar mais tempo; reter.}{pren.der}{0}
\verb{prender}{ê}{}{}{}{}{Agarrar.}{pren.der}{\verboinum{12}}
\verb{prenhe}{}{}{}{}{adj.2g.}{Que está em período de gestação.}{pre.nhe}{0}
\verb{prenhe}{}{}{}{}{}{Que foi enchido ao máximo; cheio, repleto.}{pre.nhe}{0}
\verb{prenhez}{ê}{}{}{}{s.f.}{Estado de prenhe; gravidez.}{pre.nhez}{0}
\verb{prenome}{}{}{}{}{s.m.}{Nome de um indivíduo, que antecede o nome de família; nome de batismo.}{pre.no.me}{0}
\verb{prensa}{}{}{}{}{s.f.}{Máquina destinada a comprimir ou achatar algo.}{pren.sa}{0}
\verb{prensa}{}{}{}{}{}{Máquina de impressão.}{pren.sa}{0}
\verb{prensar}{}{}{}{}{v.t.}{Comprimir alguma coisa na prensa.}{pren.sar}{0}
\verb{prensar}{}{}{}{}{}{Comprimir muito; achatar, esmagar.}{pren.sar}{\verboinum{1}}
\verb{prenunciar}{}{}{}{}{v.t.}{Prever o que ainda não aconteceu; predizer, profetizar.}{pre.nun.ci.ar}{0}
\verb{prenunciar}{}{}{}{}{}{Ser indício de.}{pre.nun.ci.ar}{0}
\verb{prenunciar}{}{}{}{}{}{Ser precursor de algo; preceder.}{pre.nun.ci.ar}{\verboinum{1}}
\verb{prenúncio}{}{}{}{}{s.m.}{Indicação de alguma coisa futura; sinal.}{pre.nún.cio}{0}
\verb{pré"-nupcial}{}{}{pré"-nupciais}{}{adj.2g.}{Que ocorre antes do casamento.}{pré"-nup.ci.al}{0}
\verb{preocupação}{}{}{"-ões}{}{s.f.}{Estado de quem se preocupa; inquietude, apreensão, cuidado.}{pre.o.cu.pa.ção}{0}
\verb{preocupação}{}{}{"-ões}{}{}{Coisa com que a pessoa se preocupa.}{pre.o.cu.pa.ção}{0}
\verb{preocupar}{}{}{}{}{v.t.}{Prender a atenção de; absorver. }{pre.o.cu.par}{0}
\verb{preocupar}{}{}{}{}{}{Causar preocupação ou inquietação a; tornar inquieto, apreensivo.}{pre.o.cu.par}{\verboinum{1}}
\verb{pré"-operatório}{}{}{}{}{adj.}{Diz"-se de período que antecede uma cirurgia.}{pré"-o.pe.ra.tó.rio}{0}
\verb{pré"-operatório}{}{}{}{}{s.m.}{Conjunto de procedimentos ou exames realizados durante esse período.}{pré"-o.pe.ra.tó.rio}{0}
\verb{preparação}{}{}{"-ões}{}{s.f.}{Ato ou efeito de preparar.}{pre.pa.ra.ção}{0}
\verb{preparação}{}{}{"-ões}{}{}{Adequação, adaptação.}{pre.pa.ra.ção}{0}
\verb{preparação}{}{}{"-ões}{}{}{Planejamento, premeditação.}{pre.pa.ra.ção}{0}
\verb{preparação}{}{}{"-ões}{}{}{Arrumação, embelezamento.}{pre.pa.ra.ção}{0}
\verb{preparação}{}{}{"-ões}{}{}{Introdução delicada de notícia desagradável.}{pre.pa.ra.ção}{0}
\verb{preparação}{}{}{"-ões}{}{}{Elaboração de pratos culinários.}{pre.pa.ra.ção}{0}
\verb{preparado}{}{}{}{}{adj.}{Que está pronto para alguma coisa.}{pre.pa.ra.do}{0}
\verb{preparado}{}{}{}{}{}{Instruído, culto.}{pre.pa.ra.do}{0}
\verb{preparado}{}{}{}{}{s.m.}{Produto químico ou farmacêutico.}{pre.pa.ra.do}{0}
%\verb{}{}{}{}{}{}{}{}{0}
\verb{preparar}{}{}{}{}{v.t.}{Dispor com antecedência; predispor.}{pre.pa.rar}{0}
\verb{preparar}{}{}{}{}{}{Premeditar, planejar.}{pre.pa.rar}{0}
\verb{preparar}{}{}{}{}{}{Armar, maquinar.}{pre.pa.rar}{0}
\verb{preparar}{}{}{}{}{}{Aprontar, aparelhar.}{pre.pa.rar}{0}
\verb{preparar}{}{}{}{}{v.pron.}{Vestir"-se.}{pre.pa.rar}{\verboinum{1}}
\verb{preparativo}{}{}{}{}{adj.}{Preparatório.}{pre.pa.ra.tivo}{0}
\verb{preparativo}{}{}{}{}{s.m.pl.}{Diposições preliminares que dão início a algum empreendimento.}{pre.pa.ra.tivo}{0}
\verb{preparativo}{}{}{}{}{}{Ações práticas prévias que levam à concretização de um projeto.}{pre.pa.ra.tivo}{0}
\verb{preparatório}{}{}{}{}{adj.}{Que serve para preparar.}{pre.pa.ra.tó.rio}{0}
\verb{preparo}{}{}{}{}{s.m.}{Ato ou efeito de preparar; elaboração.}{pre.pa.ro}{0}
\verb{preparo}{}{}{}{}{}{Cultura; instrução.}{pre.pa.ro}{0}
\verb{preponderância}{}{}{}{}{s.f.}{Qualidade ou estado de preponderante.}{pre.pon.de.rân.cia}{0}
\verb{preponderância}{}{}{}{}{}{Predomínio, supremacia, hegemonia.}{pre.pon.de.rân.cia}{0}
\verb{preponderante}{}{}{}{}{adj.2g.}{Que prepondera; dominante, hegemônico.}{pre.pon.de.ran.te}{0}
\verb{preponderante}{}{}{}{}{}{Que tem mais peso, importância, influência ou força.}{pre.pon.de.ran.te}{0}
\verb{preponderante}{}{}{}{}{}{Superior em número ou quantidade.}{pre.pon.de.ran.te}{0}
\verb{preponderar}{}{}{}{}{v.t.}{Ser mais pesado que outra coisa.}{pre.pon.de.rar}{0}
\verb{preponderar}{}{}{}{}{}{Ter mais influência, maior importância; predominar.}{pre.pon.de.rar}{0}
\verb{preponderar}{}{}{}{}{}{Ser maior numericamente; ser maioria.}{pre.pon.de.rar}{\verboinum{1}}
\verb{prepor}{}{}{}{}{v.t.}{Pôr adiante ou antes.}{pre.por}{0}
\verb{prepor}{}{}{}{}{}{Anunciar previamente.}{pre.por}{\verboinum{60}}
\verb{preposição}{}{}{"-ões}{}{s.f.}{Ato ou efeito de prepor; anteposição.}{pre.po.si.ção}{0}
\verb{preposição}{}{Gram.}{"-ões}{}{}{Classe de palavra invariável que liga palavras ou orações, estabelecendo relações de subordinação entre elas.}{pre.po.si.ção}{0}
\verb{prepositivo}{}{}{}{}{adj.}{Que se põe adiante ou em primeiro lugar.}{pre.po.si.ti.vo}{0}
\verb{prepositivo}{}{Gram.}{}{}{}{Que diz respeito à preposição ou que é da natureza dela.}{pre.po.si.ti.vo}{0}
\verb{prepositivo}{}{Gram.}{}{}{}{Diz"-se de locução que tem função de preposição.}{pre.po.si.ti.vo}{0}
\verb{preposto}{ô}{}{"-s ⟨ó⟩}{"-a ⟨ó⟩}{adj.}{Que é posto antes.}{pre.pos.to}{0}
\verb{preposto}{ô}{}{"-s ⟨ó⟩}{"-a ⟨ó⟩}{}{Que é anunciado com antecedência.}{pre.pos.to}{0}
\verb{preposto}{ô}{}{"-s ⟨ó⟩}{"-a ⟨ó⟩}{}{Preferido, favorito.}{pre.pos.to}{0}
\verb{prepotência}{}{}{}{}{s.f.}{Característica do que é prepotente.}{pre.po.tên.cia}{0}
\verb{prepotência}{}{}{}{}{}{Poder mais alto.}{pre.po.tên.cia}{0}
\verb{prepotência}{}{}{}{}{}{Abuso do poder ou de autoridade.}{pre.po.tên.cia}{0}
\verb{prepotente}{}{}{}{}{adj.2g.}{Que revela prepotência, que abusa do poder.}{pre.po.ten.te}{0}
\verb{pré"-primário}{}{}{}{}{s.m.}{Curso que antecede o primário.}{pré"-pri.má.rio}{0}
\verb{prepúcio}{}{Anat.}{}{}{s.m.}{Prega que cobre a glande do pênis.}{pre.pú.cio}{0}
\verb{prequeté}{}{Pop.}{}{}{adj.2g.}{Diz"-se do indivíduo emperiquitado, faceiro, perequeté.}{pre.que.té}{0}
\verb{pré"-requisito}{}{}{pré"-requisitos}{}{s.m.}{Condição prévia e indispensável para a realização de algo.}{pré"-re.qui.si.to}{0}
\verb{prerrogativa}{}{}{}{}{s.f.}{Direito que tem uma pessoa pelo cargo que ocupa.}{prer.ro.ga.ti.va}{0}
\verb{presa}{ê}{}{}{}{s.f.}{Aquilo que se captura do inimigo.}{pre.sa}{0}
\verb{presa}{ê}{}{}{}{}{Caça do animal.}{pre.sa}{0}
\verb{presa}{ê}{}{}{}{}{Pessoa ou coisa subjugada.}{pre.sa}{0}
\verb{presa}{ê}{}{}{}{}{Dente canino ou comprido.}{pre.sa}{0}
\verb{presa}{ê}{}{}{}{}{Garra de ave.}{pre.sa}{0}
%\verb{}{}{}{}{}{}{}{}{0}
\verb{presbiteriano}{}{}{}{}{adj.}{Relativo ao presbiterianismo.}{pres.bi.te.ri.a.no}{0}
\verb{presbiteriano}{}{}{}{}{}{Adepto do presbiterianismo.}{pres.bi.te.ri.a.no}{0}
\verb{presbitério}{}{}{}{}{s.m.}{Igreja ou residência paroquial.}{pres.bi.té.rio}{0}
\verb{presbitério}{}{}{}{}{}{A capela principal de uma igreja.}{pres.bi.té.rio}{0}
\verb{presbitério}{}{}{}{}{}{Na Igreja Protestante, o conjunto dos presbíteros.}{pres.bi.té.rio}{0}
\verb{presbítero}{}{}{}{}{s.m.}{Sacerdote, padre.}{pres.bí.te.ro}{0}
\verb{presbítero}{}{}{}{}{}{Dirigente e chefe espiritual dos presbiterianos.}{pres.bí.te.ro}{0}
\verb{presciência}{}{}{}{}{s.f.}{Conhecimento do futuro.}{pres.ci.ên.cia}{0}
\verb{presciência}{}{}{}{}{}{Ciência inata, anterior ao estudo.}{pres.ci.ên.cia}{0}
\verb{presciente}{}{}{}{}{adj.2g.}{Que sabe com antecipação, que prevê o futuro.}{pres.ci.en.te}{0}
\verb{presciente}{}{Por ext.}{}{}{}{Prudente, acautelado.}{pres.ci.en.te}{0}
\verb{prescindir}{}{}{}{}{v.t.}{Renunciar a; dispensar.}{pres.cin.dir}{0}
\verb{prescindir}{}{}{}{}{}{Não levar em conta.}{pres.cin.dir}{\verboinum{18}}
\verb{prescrever}{ê}{}{}{}{v.t.}{Ordenar explícita e previamente.}{pres.cre.ver}{0}
\verb{prescrever}{ê}{}{}{}{}{Indicar como remédio.}{pres.cre.ver}{0}
\verb{prescrever}{ê}{}{}{}{}{Determinar, fixar.}{pres.cre.ver}{0}
\verb{prescrever}{ê}{}{}{}{v.i.}{Cair em desuso.}{pres.cre.ver}{0}
\verb{prescrever}{ê}{Jur.}{}{}{}{Extinguir"-se por haver decorrido o prazo legal.}{pres.cre.ver}{\verboinum{12}}
\verb{prescrição}{}{}{"-ões}{}{s.f.}{Ato ou efeito de prescrever.}{pres.cri.ção}{0}
\verb{prescrição}{}{}{"-ões}{}{}{Preceito, regra, indicação.}{pres.cri.ção}{0}
\verb{prescrição}{}{Jur.}{"-ões}{}{}{Extinção de um direito que não se exerceu no devido tempo.}{pres.cri.ção}{0}
\verb{prescrição}{}{}{"-ões}{}{}{Extinção de uma obrigação, por não se ter exigido o cumprimento dela.}{pres.cri.ção}{0}
\verb{prescrito}{}{}{}{}{adj.}{Explicitamente ordenado ou estabelecido.}{pres.cri.to}{0}
\verb{prescrito}{}{}{}{}{}{Que prescreveu, caducou, perdeu a validade por decurso de prazo.}{pres.cri.to}{0}
\verb{presença}{}{}{}{}{s.f.}{Fato de pessoa ou coisa estar em determinado lugar.}{pre.sen.ça}{0}
\verb{presença}{}{}{}{}{}{Ato de ir aonde foi chamado ou convidado; comparecimento.}{pre.sen.ça}{0}
\verb{presença}{}{}{}{}{}{Lugar à frente de uma pessoa.}{pre.sen.ça}{0}
\verb{presenciar}{}{}{}{}{v.t.}{Estar presente a; assistir.}{pre.sen.ci.ar}{\verboinum{1}}
\verb{presente}{}{}{}{}{s.m.}{Objeto ofertado; mimo.}{pre.sen.te}{0}
\verb{presente}{}{}{}{}{}{Dádiva, dom.}{pre.sen.te}{0}
\verb{presente}{}{}{}{}{}{O tempo atual.}{pre.sen.te}{0}
\verb{presente}{}{}{}{}{adj.2g.}{Que assiste pessoalmente.}{pre.sen.te}{0}
\verb{presente}{}{}{}{}{}{Atual, contemporâneo.}{pre.sen.te}{0}
\verb{presente}{}{}{}{}{}{Diz"-se de tempo verbal que indica que a ação se passa ou pode passar"-se no momento em que se fala. }{pre.sen.te}{0}
\verb{presentear}{}{}{}{}{v.t.}{Dar presente a.}{pre.sen.te.ar}{\verboinum{4}}
\verb{presepada}{}{}{}{}{s.f.}{Atitude de mau gosto; inconveniência, palhaçada.}{pre.se.pa.da}{0}
\verb{presepada}{}{}{}{}{}{Cena, escândalo.}{pre.se.pa.da}{0}
\verb{presepe}{é}{}{}{}{s.m.}{Presépio.}{pre.se.pe}{0}
\verb{presépio}{}{}{}{}{s.m.}{Pequena construção que representa a cena do nascimento de Jesus.}{pre.sé.pio}{0}
\verb{preservação}{}{}{"-ões}{}{s.f.}{Ato ou efeito de preservar; proteção, defesa.}{pre.ser.va.ção}{0}
\verb{preservação}{}{}{"-ões}{}{}{Conservação, manutenção.}{pre.ser.va.ção}{0}
\verb{preservar}{}{}{}{}{v.t.}{Pôr ao abrigo de algum mal, dano ou perigo; defender, resguadar.}{pre.ser.var}{0}
\verb{preservar}{}{}{}{}{}{Não destruir; conservar, salvar.}{pre.ser.var}{\verboinum{1}}
\verb{preservativo}{}{}{}{}{s.m.}{Dispositivo ou substância usada para evitar a concepção. }{pre.ser.va.ti.vo}{0}
\verb{preservativo}{}{}{}{}{}{Substância que se adiciona a um produto para conservá"-lo inalterado por muito tempo.}{pre.ser.va.ti.vo}{0}
\verb{preservativo}{}{}{}{}{adj.}{Que preserva.}{pre.ser.va.ti.vo}{0}
\verb{presidência}{}{}{}{}{s.f.}{Ato ou resultado de presidir; direção.}{pre.si.dên.cia}{0}
\verb{presidência}{}{}{}{}{}{Título ou cargo de presidente.}{pre.si.dên.cia}{0}
\verb{presidência}{}{}{}{}{}{Tempo de duração do exercício desse cargo.}{pre.si.dên.cia}{0}
\verb{presidencial}{}{}{"-ais}{}{adj.2g.}{Relativo à presidência ou a presidente.}{pre.si.den.ci.al}{0}
\verb{presidencial}{}{}{"-ais}{}{}{Que provém do presidente.}{pre.si.den.ci.al}{0}
\verb{presidencialismo}{}{}{}{}{s.m.}{Sistema político em que a chefia do governo cabe ao presidente da República.}{pre.si.den.ci.a.lis.mo}{0}
\verb{presidencialista}{}{}{}{}{adj.2g.}{Relativo ao presidencialismo.}{pre.si.den.ci.a.lis.ta}{0}
\verb{presidencialista}{}{}{}{}{}{Em que domina o presidencialismo.}{pre.si.den.ci.a.lis.ta}{0}
\verb{presidencialista}{}{}{}{}{}{Que é partidário do presidencialismo.}{pre.si.den.ci.a.lis.ta}{0}
\verb{presidenciável}{}{}{"-eis}{}{adj.2g.}{Que pode ser eleito presidente.}{pre.si.den.ci.á.vel}{0}
\verb{presidente}{}{}{}{}{adj.2g.}{Que preside.}{pre.si.den.te}{0}
\verb{presidente}{}{}{}{}{s.m.}{O chefe do executivo federal nos países republicanos.}{pre.si.den.te}{0}
\verb{presidiário}{}{}{}{}{adj.}{Relativo a presídio.}{pre.si.di.á.rio}{0}
\verb{presidiário}{}{}{}{}{s.m.}{Indivíduo que cumpre pena em presídio.}{pre.si.di.á.rio}{0}
\verb{presídio}{}{}{}{}{s.m.}{Casa de detenção; penitenciária.}{pre.sí.dio}{0}
\verb{presidir}{}{}{}{}{v.t.}{Dirigir como presidente; governar.}{pre.si.dir}{0}
\verb{presidir}{}{}{}{}{}{Guiar, orientar.}{pre.si.dir}{\verboinum{18}}
\verb{presilha}{}{}{}{}{s.f.}{Tira de pano ou outro material com uma casa para se abotoar.}{pre.si.lha}{0}
\verb{presilha}{}{}{}{}{}{Peça com fecho para prender o cabelo.}{pre.si.lha}{0}
\verb{preso}{ê}{}{}{}{adj.}{Que perdeu a liberdade; prisioneiro.}{pre.so}{0}
\verb{preso}{ê}{}{}{}{}{Que se fixou; atado, amarrado.}{pre.so}{0}
\verb{pressa}{é}{}{}{}{s.f.}{Falta de calma e paciência para fazer algo; precipitação.}{pres.sa}{0}
\verb{pressa}{é}{}{}{}{}{Necessidade de fazer ou obter algo com rapidez; precisão, urgência.}{pres.sa}{0}
\verb{pressagiar}{}{}{}{}{v.t.}{Anunciar por presságio ou agouro; profetizar.}{pres.sa.gi.ar}{0}
\verb{pressagiar}{}{}{}{}{}{Adivinhar, prever.}{pres.sa.gi.ar}{\verboinum{1}}
\verb{presságio}{}{}{}{}{s.m.}{Fato ou sinal que prenuncia o futuro; agouro.}{pres.sá.gio}{0}
\verb{pressago}{}{}{}{}{adj.}{Que anuncia ou prevê o futuro.}{pres.sa.go}{0}
\verb{pressão}{}{}{"-ões}{}{s.f.}{Força que age sobre uma superfície.}{pres.são}{0}
\verb{pressão}{}{}{"-ões}{}{}{Coação; constrangimento moral.}{pres.são}{0}
\verb{pressão}{}{}{"-ões}{}{}{Tensão do sangue nas artérias, veias etc.}{pres.são}{0}
\verb{pressentimento}{}{}{}{}{s.m.}{Ato de sentir antecipadamente, mais pela emoção do que pela razão, a ocorrência de um fato futuro; suspeita, desconfiança.}{pres.sen.ti.men.to}{0}
\verb{pressentir}{}{}{}{}{v.t.}{Sentir antecipadamente o que vai acontecer}{pres.sen.tir}{0}
\verb{pressentir}{}{}{}{}{}{Adivinhar por indícios; pressagiar, antever.}{pres.sen.tir}{\verboinum{29}}
\verb{pressionar}{}{}{}{}{v.t.}{Fazer pressão sobre alguma coisa; comprimir; apertar.}{pres.si.o.nar}{0}
\verb{pressionar}{}{}{}{}{}{Coagir.}{pres.si.o.nar}{\verboinum{1}}
\verb{pressupor}{ô}{}{}{}{v.t.}{Supor antecipadamente; conjecturar.}{pres.su.por}{0}
\verb{pressupor}{ô}{}{}{}{}{Dar a entender; presumir.}{pres.su.por}{\verboinum{60}}
\verb{pressuposição}{}{}{"-ões}{}{s.f.}{Ato ou efeito de pressupor; conjectura, suposição.}{pres.su.po.si.ção}{0}
\verb{pressuposto}{ô}{}{"-s ⟨ó⟩}{"-a ⟨ó⟩}{adj.}{Que se pressupõe; conjectura, suposição.}{pres.su.pos.to}{0}
\verb{pressuposto}{ô}{}{"-s ⟨ó⟩}{"-a ⟨ó⟩}{s.m.}{Aquilo que se busca alcançar; desígnio, tensão, projeto.}{pres.su.pos.to}{0}
\verb{pressuposto}{ô}{Jur.}{"-s ⟨ó⟩}{"-a ⟨ó⟩}{}{Circunstância ou fato em que se considera um antecedente necessário de outro.}{pres.su.pos.to}{0}
\verb{pressurização}{}{}{"-ões}{}{s.f.}{Ato ou efeito de pressurizar.}{pres.su.ri.za.ção}{0}
\verb{pressurizar}{}{}{}{}{v.t.}{Manter pressão normal em.}{pres.su.ri.zar}{\verboinum{1}}
\verb{pressuroso}{ô}{}{"-osos ⟨ó⟩}{"-osa ⟨ó⟩}{adj.}{Que tem ou age com pressa; apressado.}{pres.su.ro.so}{0}
\verb{pressuroso}{ô}{}{"-osos ⟨ó⟩}{"-osa ⟨ó⟩}{}{Que é impaciente, ansioso.}{pres.su.ro.so}{0}
\verb{pressuroso}{ô}{}{"-osos ⟨ó⟩}{"-osa ⟨ó⟩}{}{Que vive ou anda muito ocupado; sobrecarregado.}{pres.su.ro.so}{0}
\verb{prestação}{}{}{"-ões}{}{s.f.}{Ação de prestar algo; fornecimento.}{pres.ta.ção}{0}
\verb{prestação}{}{}{"-ões}{}{}{Cada um dos pagamentos em que se dividiu uma conta.}{pres.ta.ção}{0}
\verb{prestamista}{}{}{}{}{adj.2g.}{Que empresta dinheiro a juros.}{pres.ta.mis.ta}{0}
\verb{prestamista}{}{}{}{}{}{Que possui títulos da dívida pública.}{pres.ta.mis.ta}{0}
\verb{prestamista}{}{}{}{}{}{Que compra a prestações.}{pres.ta.mis.ta}{0}
\verb{prestar}{}{}{}{}{v.t.}{Atender alguém com algum tipo de serviço; dar, conceder.}{pres.tar}{0}
\verb{prestar}{}{}{}{}{}{Realizar por imposição legal.}{pres.tar}{0}
\verb{prestar}{}{}{}{}{}{Ser bom para alguma coisa; servir.}{pres.tar}{0}
\verb{prestar}{}{}{}{}{v.pron.}{Ter boa vontade.}{pres.tar}{\verboinum{1}}
\verb{prestativo}{}{}{}{}{adj.}{Que está pronto para servir; solícito.}{pres.ta.ti.vo}{0}
\verb{prestável}{}{}{"-eis}{}{adj.2g.}{Que pode servir para alguma coisa.}{pres.tá.vel}{0}
\verb{prestes}{é}{}{}{}{adj.2g.}{Que está a ponto de acontecer; próximo.}{pres.tes}{0}
\verb{prestes}{é}{}{}{}{}{Disposto; pronto.}{pres.tes}{0}
\verb{prestes}{é}{}{}{}{adv.}{Sem demora.}{pres.tes}{0}
\verb{presteza}{ê}{}{}{}{s.f.}{Característica do que é ligeiro para fazer algo; rapidez.}{pres.te.za}{0}
\verb{presteza}{ê}{}{}{}{}{Solicitude.}{pres.te.za}{0}
\verb{prestidigitação}{}{}{"-ões}{}{s.f.}{Ilusionismo, mágica.}{pres.ti.di.gi.ta.ção}{0}
\verb{prestidigitador}{ô}{}{}{}{s.m.}{Indivíduo que realiza a prestidigitação; ilusionista, mágico.}{pres.ti.di.gi.ta.dor}{0}
\verb{prestigiar}{}{}{}{}{v.t.}{Conferir prestígio a algo ou alguém.}{pres.ti.gi.ar}{0}
\verb{prestigiar}{}{}{}{}{}{Valorizar com sua presença, participação.}{pres.ti.gi.ar}{\verboinum{1}}
\verb{prestígio}{}{}{}{}{s.m.}{Valor positivo atribuído a algo ou alguém.}{pres.tí.gio}{0}
\verb{prestígio}{}{}{}{}{}{Reconhecimento das qualidades de algo ou alguém.}{pres.tí.gio}{0}
\verb{prestigioso}{ô}{}{"-osos ⟨ó⟩}{"-osa ⟨ó⟩}{adj.}{Que exerce grande influência.}{pres.ti.gi.o.so}{0}
\verb{prestigioso}{ô}{}{"-osos ⟨ó⟩}{"-osa ⟨ó⟩}{}{Que tem prestígio.}{pres.ti.gi.o.so}{0}
\verb{prestímano}{}{}{}{}{s.m.}{Indivíduo que tem muita habilidade e destreza com as mãos.}{pres.tí.ma.no}{0}
\verb{prestímano}{}{}{}{}{}{Ilusionista.}{pres.tí.ma.no}{0}
\verb{préstimo}{}{}{}{}{s.m.}{Qualidade do que é útil; serventia.}{prés.ti.mo}{0}
\verb{préstimo}{}{}{}{}{}{Ato de ajudar; socorro, auxílio.}{prés.ti.mo}{0}
\verb{prestimoso}{ô}{}{"-osos ⟨ó⟩}{"-osa ⟨ó⟩}{adj.}{Que tem préstimo, utilidade.}{pres.ti.mo.so}{0}
\verb{préstito}{}{}{}{}{s.m.}{Grupo de pessoas que caminham juntas, com determinada finalidade; procissão, cortejo.}{prés.ti.to}{0}
\verb{presumido}{}{}{}{}{adj.}{Conjecturado por suposição, hipótese, presunção.}{pre.su.mi.do}{0}
\verb{presumido}{}{}{}{}{}{Vaidoso, arrogante.}{pre.su.mi.do}{0}
\verb{presumir}{}{}{}{}{v.t.}{Tirar uma conclusão antecipada, baseada em indícios e suposições, e não em fatos comprovados; conjecturar, supor. }{pre.su.mir}{0}
\verb{presumir}{}{}{}{}{}{Desconfiar, suspeitar.}{pre.su.mir}{0}
\verb{presumir}{}{}{}{}{}{Vangloriar.}{pre.su.mir}{\verboinum{18}}
\verb{presumível}{}{}{"-eis}{}{adj.2g.}{Que se pode presumir, supor ou suspeitar.}{pre.su.mí.vel}{0}
\verb{presumível}{}{}{"-eis}{}{}{Provável.}{pre.su.mí.vel}{0}
\verb{presunção}{}{}{"-ões}{}{s.f.}{Suposição que se toma por verdadeira.}{pre.sun.ção}{0}
\verb{presunção}{}{}{"-ões}{}{}{Pretensão, vaidade.}{pre.sun.ção}{0}
\verb{presunçoso}{ô}{}{"-osos ⟨ó⟩}{"-osa ⟨ó⟩}{adj.}{Que se supõe melhor que os demais; vaidoso, pretensioso.}{pre.sun.ço.so}{0}
\verb{presuntivo}{}{}{}{}{adj.}{Que se pode presumir.}{pre.sun.ti.vo}{0}
\verb{presuntivo}{}{}{}{}{}{Que tem possibilidades de ser.}{pre.sun.ti.vo}{0}
\verb{presuntivo}{}{}{}{}{}{Baseado em presunção ou possibilidade.}{pre.sun.ti.vo}{0}
\verb{presuntivo}{}{}{}{}{}{Indicado de antemão, por proximidade de parentesco.}{pre.sun.ti.vo}{0}
\verb{presunto}{}{}{}{}{s.m.}{Pernil de porco em conserva.}{pre.sun.to}{0}
\verb{presunto}{}{Pop.}{}{}{}{Cadáver, defunto.}{pre.sun.to}{0}
\verb{pretejar}{}{}{}{}{v.i.}{Ficar preto; escurecer.}{pre.te.jar}{\verboinum{1}}
\verb{pretendente}{}{}{}{}{adj.2g.}{Que pretende; candidato, requerente, aspirante.}{pre.ten.den.te}{0}
\verb{pretender}{ê}{}{}{}{v.t.}{Tentar obter; requerer, solicitar.}{pre.ten.der}{0}
\verb{pretender}{ê}{}{}{}{}{Tencionar, projetar, planejar.}{pre.ten.der}{0}
\verb{pretender}{ê}{}{}{}{v.pron.}{Ter"-se em conta; julgar"-se.}{pre.ten.der}{\verboinum{12}}
\verb{pretensão}{}{}{"-ões}{}{s.f.}{Ato ou efeito de pretender.}{pre.ten.são}{0}
\verb{pretensão}{}{}{"-ões}{}{}{Desejo, exigência, ambição.}{pre.ten.são}{0}
\verb{pretensão}{}{}{"-ões}{}{}{Vaidade, presunção.}{pre.ten.são}{0}
\verb{pretensioso}{ô}{}{"-osos ⟨ó⟩}{"-osa ⟨ó⟩}{adj.}{Que tem pretensões; ambicioso, vaidoso, pedante.}{pre.ten.si.o.so}{0}
\verb{pretenso}{}{}{}{}{adj.}{Que se pretende; suposto, imaginado.}{pre.ten.so}{0}
\verb{preterir}{}{}{}{}{v.t.}{Deixar de lado; rejeitar, menosprezar.}{pre.te.rir}{\verboinum{29}}
\verb{pretérito}{}{}{}{}{adj.}{Que está situado no passado.}{pre.té.ri.to}{0}
\verb{pretérito}{}{Gram.}{}{}{}{Tempo verbal que indica ação ou estado anterior ao momento em que se fala.}{pre.té.ri.to}{0}
\verb{pretextar}{s}{}{}{}{v.t.}{Dar como pretexto; desculpar.}{pre.tex.tar}{\verboinum{1}}
\verb{pretexto}{ês}{}{}{}{s.m.}{Motivo que se dá para ocultar a verdadeira razão; desculpa, alegação.}{pre.tex.to}{0}
\verb{preto}{ê}{}{}{}{}{Indivíduo negro.}{pre.to}{0}
\verb{preto}{ê}{}{}{}{s.m.}{A cor preta.}{pre.to}{0}
\verb{preto}{ê}{}{}{}{adj.}{Da cor do ébano e do carvão; negro, escuro.}{pre.to}{0}
\verb{pretor}{ô}{}{}{}{s.m.}{Na antiga Roma, magistrado que administrava a justiça.}{pre.tor}{0}
\verb{pretor}{ô}{}{}{}{}{Magistrado de alçada inferior à do juiz de direito.}{pre.tor}{0}
\verb{pretoria}{}{}{}{}{s.f.}{Jurisdição ou repartição de pretor.}{pre.to.ri.a}{0}
\verb{pretório}{}{}{}{}{s.m.}{Tribunal de pretor ou de outros magistrados.}{pre.tó.rio}{0}
\verb{pretume}{}{}{}{}{s.m.}{Ausência total de luz; escuridão, negrume.}{pre.tu.me}{0}
\verb{prevalecer}{ê}{}{}{}{v.t.}{Levar vantagem; sobressair, predominar.}{pre.va.le.cer}{\verboinum{15}}
\verb{prevalência}{}{}{}{}{s.f.}{Qualidade daquilo que prevalece; predominância, superioridade.}{pre.va.lên.cia}{0}
\verb{prevaricação}{}{}{"-ões}{}{s.f.}{Ato ou efeito de prevaricar.}{pre.va.ri.ca.ção}{0}
\verb{prevaricação}{}{}{"-ões}{}{}{Falta de cumprimento do dever; improbidade.}{pre.va.ri.ca.ção}{0}
\verb{prevaricação}{}{}{"-ões}{}{}{Infidelidade conjugal; adultério.}{pre.va.ri.ca.ção}{0}
\verb{prevaricador}{ô}{}{}{}{adj.}{Que prevarica; que falta ao dever.}{pre.va.ri.ca.dor}{0}
\verb{prevaricar}{}{}{}{}{v.i.}{Faltar ao cumprimento do dever.}{pre.va.ri.car}{0}
\verb{prevaricar}{}{}{}{}{}{Cometer adultério.}{pre.va.ri.car}{\verboinum{2}}
\verb{prevenção}{}{}{"-ões}{}{s.f.}{Ato ou efeito de prevenir; precaução, cautela.}{pre.ven.ção}{0}
\verb{prevenção}{}{}{"-ões}{}{}{Opinião desfavorável; preconceito.}{pre.ven.ção}{0}
\verb{prevenido}{}{}{}{}{adj.}{Que se previne; cauteloso, precavido.}{pre.ve.ni.do}{0}
\verb{prevenir}{}{}{}{}{v.t.}{Tomar medidas antecipadamente; evitar.}{pre.ve.nir}{0}
\verb{prevenir}{}{}{}{}{}{Avisar, aconselhar com antecedência.}{pre.ve.nir}{\verboinum{30}}
\verb{preventivo}{}{}{}{}{adj.}{Próprio para prevenir, evitar, proteger.}{pre.ven.ti.vo}{0}
\verb{prever}{ê}{}{}{}{v.t.}{Ver com antecipação; antever, profetizar.}{pre.ver}{0}
\verb{prever}{ê}{}{}{}{}{Fazer conjecturas; pressupor.}{pre.ver}{\verboinum{46}}
\verb{pré"-vestibular}{}{}{pré"-vestibulares}{}{adj.2g.}{Que antecede o exame vestibular.}{pré"-ves.ti.bu.lar}{0}
\verb{pré"-vestibular}{}{}{pré"-vestibulares}{}{s.m.}{Curso preparatório para o vestibular.}{pré"-ves.ti.bu.lar}{0}
\verb{prévia}{}{}{}{}{s.f.}{Pesquisa antecipada junto aos eleitores para conhecer as suas tendências.}{pré.via}{0}
\verb{previdência}{}{}{}{}{s.f.}{Qualidade do que é previdente.}{pre.vi.dên.cia}{0}
\verb{previdência}{}{}{}{}{}{Faculdade de ver com antecipação; previsão, precaução.}{pre.vi.dên.cia}{0}
\verb{previdenciário}{}{}{}{}{adj.}{Relativo a previdência.}{pre.vi.den.ci.á.rio}{0}
\verb{previdenciário}{}{}{}{}{s.m.}{Funcionário do instituto de previdência social.}{pre.vi.den.ci.á.rio}{0}
\verb{previdente}{}{}{}{}{adj.2g.}{Que prevê; cauteloso, prudente.}{pre.vi.den.te}{0}
\verb{prévio}{}{}{}{}{adj.}{Que antecede o principal; antecipado, preliminar.}{pré.vio}{0}
\verb{previsão}{}{}{"-ões}{}{s.f.}{Ato ou efeito de prever; antevisão, conjectura.}{pre.vi.são}{0}
\verb{previsível}{}{}{"-eis}{}{adj.2g.}{Que se pode prever, esperar.}{pre.vi.sí.vel}{0}
\verb{previsto}{}{}{}{}{adj.}{Que se previu; esperado, pressuposto.}{pre.vis.to}{0}
\verb{prezado}{}{}{}{}{adj.}{Que se preza; estimado, querido, caro.}{pre.za.do}{0}
\verb{prezar}{}{}{}{}{v.t.}{Ter em alta consideração; estimar, respeitar.}{pre.zar}{\verboinum{1}}
\verb{prima}{}{}{}{}{s.f.}{Feminino de \textit{primo}.}{pri.ma}{0}
\verb{prima}{}{Mús.}{}{}{}{A primeira e mais fina corda de alguns instrumentos.}{pri.ma}{0}
\verb{primacial}{}{}{"-ais}{}{adj.2g.}{Que tem primazia; primordial.}{pri.ma.ci.al}{0}
\verb{primado}{}{}{}{}{s.m.}{Condição do que está em primeiro lugar; primazia, supremacia.}{pri.ma.do}{0}
\verb{prima"-dona}{}{}{prima"-donas}{}{s.f.}{Principal cantora de uma ópera.}{pri.ma"-do.na}{0}
\verb{primar}{}{}{}{}{}{Destacar, distinguir, sobressair.}{pri.mar}{\verboinum{1}}
\verb{primar}{}{}{}{}{v.t.}{Ser o primeiro; ter a primazia.}{pri.mar}{0}
\verb{primário}{}{}{}{}{adj.}{Que antecede; primeiro, primitivo, original.}{pri.má.rio}{0}
\verb{primário}{}{}{}{}{s.m.}{Curso do ensino fundamental que engloba as quatro primeiras séries.}{pri.má.rio}{0}
\verb{primário}{}{}{}{}{}{Simples, elementar, rudimentar.}{pri.má.rio}{0}
\verb{primata}{}{Zool.}{}{}{s.m.}{Ordem dos primatas que inclui o homem, o macaco, o lêmure etc., animais que se caracterizam por possuírem cinco dedos e duas mamas.}{pri.ma.ta}{0}
\verb{primavera}{é}{Bot.}{}{}{}{Trepadeira de ramos pendurados e flores de cor variada; buganvília.}{pri.ma.ve.ra}{0}
\verb{primavera}{é}{}{}{}{}{Ano de pessoa jovem. (\textit{Sua filha completou quinze primaveras.})}{pri.ma.ve.ra}{0}
\verb{primavera}{é}{}{}{}{s.f.}{Estação do ano que antecede ao verão e sucede ao inverno, e que, no hemisfério sul, compreende o período do final de setembro ao final de dezembro, com um clima que vai esquentando progressivamente.}{pri.ma.ve.ra}{0}
\verb{primavera}{é}{Bot.}{}{}{}{Planta pequena de jardim, de flores de cor brilhante e folhas largas e compridas; prímula.}{pri.ma.ve.ra}{0}
\verb{primaveril}{}{}{"-is}{}{adj.2g.}{Relativo à primavera.}{pri.ma.ve.ril}{0}
\verb{primaz}{}{}{}{}{s.m.}{Religioso que ocupa lugar superior ao dos bispos e arcebispos.}{pri.maz}{0}
\verb{primazia}{}{}{}{}{s.f.}{Dignidade ou cargo de primaz.}{pri.ma.zi.a}{0}
\verb{primazia}{}{}{}{}{}{O primeiro lugar; prioridade, excelência.}{pri.ma.zi.a}{0}
\verb{primeira}{ê}{}{}{}{s.f.}{A primeira marcha de velocidade de um carro.}{pri.mei.ra}{0}
\verb{primeira"-dama}{}{}{primeiras"-damas}{}{s.f.}{A esposa de um governante.}{pri.mei.ra"-da.ma}{0}
\verb{primeiro}{ê}{}{}{}{num.}{Que ocupa em uma sequência a posição do número um.}{pri.mei.ro}{0}
\verb{primeiro}{ê}{}{}{}{adj.}{Que precede outros em relação ao tempo ou ao lugar.}{pri.mei.ro}{0}
\verb{primeiro}{ê}{}{}{}{adv.}{Antes de tudo; primeiramente.}{pri.mei.ro}{0}
\verb{primeiro"-ministro}{}{}{primeiros"-ministros}{}{s.m.}{Político que governa uma nação no sistema parlamentarista.}{pri.mei.ro"-mi.nis.tro}{0}
\verb{primeiro"-sargento}{}{}{primeiros"-sargentos}{}{s.m.}{Patente das Forças Armadas, imediatamente acima da de segundo"-sargento.}{pri.mei.ro"-sar.gen.to}{0}
\verb{primeiro"-sargento}{}{}{primeiros"-sargentos}{}{}{Militar que ocupa esse posto.}{pri.mei.ro"-sar.gen.to}{0}
\verb{primeiro"-tenente}{}{}{primeiros"-tenentes}{}{s.m.}{Posto da hierarquia das Forças Armadas, imediatamente superior ao de segundo"-tenente e imediatamente inferior ao de capitão.}{pri.mei.ro"-te.nen.te}{0}
\verb{primeiro"-tenente}{}{}{primeiros"-tenentes}{}{}{Militar que ocupa esse posto.}{pri.mei.ro"-te.nen.te}{0}
\verb{primevo}{ê}{}{}{}{adj.}{Dos primeiros tempos; antigo, primitivo.}{pri.me.vo}{0}
\verb{primícias}{}{}{}{}{s.f.pl.}{Primeiros frutos da terra ou animais nascidos de um rebanho.}{pri.mí.cias}{0}
\verb{primícias}{}{}{}{}{}{Primeiras coisas de uma série; prelúdios.}{pri.mí.cias}{0}
\verb{primícias}{}{}{}{}{}{Primeiros sentimentos ou prazeres.}{pri.mí.cias}{0}
\verb{primípara}{}{}{}{}{s.f.}{Fêmea que pariu ou vai parir pela primeira vez.}{pri.mí.pa.ra}{0}
\verb{primitivismo}{}{}{}{}{s.m.}{Característica do que é primitivo, elementar.}{pri.mi.ti.vis.mo}{0}
\verb{primitivismo}{}{Art.}{}{}{}{Tendência artística que se inspira na ingenuidade e simplicidade dos povos primitivos.}{pri.mi.ti.vis.mo}{0}
\verb{primitivo}{}{}{}{}{adj.}{Dos primeiros tempos; inicial, primordial.}{pri.mi.ti.vo}{0}
\verb{primitivo}{}{}{}{}{}{Simples, tosco, rudimentar.}{pri.mi.ti.vo}{0}
\verb{primitivo}{}{}{}{}{}{Que não evoluiu; antiquado, arcaico.}{pri.mi.ti.vo}{0}
\verb{primo}{}{Mat.}{}{}{adj.}{Diz"-se do número que só é divisível por si mesmo e pela unidade.}{pri.mo}{0}
\verb{primo}{}{}{}{}{s.m.}{Filho de tio ou tia em relação aos sobrinhos destes.}{pri.mo}{0}
\verb{primogênito}{}{}{}{}{adj.}{Que nasceu primeiro, antes dos irmãos.}{pri.mo.gê.ni.to}{0}
\verb{primor}{ô}{}{}{}{s.m.}{Qualidade superior; excelência, apuro, perfeição.}{pri.mor}{0}
\verb{primordial}{}{}{"-ais}{}{adj.2g.}{Relativo a primórdio; primeiro, primitivo.}{pri.mor.di.al}{0}
\verb{primordial}{}{}{"-ais}{}{}{Essencial, principal, fundamental.}{pri.mor.di.al}{0}
\verb{primórdio}{}{}{}{}{s.m.}{Tempo inicial de algo; origem, princípio.}{pri.mór.dio}{0}
\verb{primoroso}{ô}{}{"-osos ⟨ó⟩}{"-osa ⟨ó⟩}{adj.}{Que é feito com primor; excelente, perfeito, caprichado.}{pri.mo.ro.so}{0}
\verb{prímula}{}{Bot.}{}{}{s.f.}{Planta pequena de flores de cor brilhante e folhas largas e compridas; primavera.}{prí.mu.la}{0}
\verb{princesa}{ê}{}{}{}{s.f.}{Filha de rei ou de rainha.}{prin.ce.sa}{0}
\verb{princesa}{ê}{}{}{}{}{Mulher que tem esse título de nobreza e governa um Estado.}{prin.ce.sa}{0}
\verb{principado}{}{}{}{}{s.m.}{Dignidade de príncipe.}{prin.ci.pa.do}{0}
\verb{principado}{}{}{}{}{}{Território ou Estado cujo soberano é um príncipe ou uma princesa.}{prin.ci.pa.do}{0}
\verb{principal}{}{}{"-ais}{}{adj.2g.}{Que é fundamental, essencial.}{prin.ci.pal}{0}
\verb{principal}{}{}{"-ais}{}{}{Que está em primeiro lugar.}{prin.ci.pal}{0}
\verb{príncipe}{}{}{}{}{s.m.}{Filho primogênito do rei.}{prín.ci.pe}{0}
\verb{príncipe}{}{}{}{}{}{Chefe de um principado.}{prín.ci.pe}{0}
\verb{principesco}{ê}{}{}{}{adj.}{Relativo a ou próprio de príncipe.}{prin.ci.pes.co}{0}
\verb{principesco}{ê}{}{}{}{}{Opulento, ostentoso.}{prin.ci.pes.co}{0}
\verb{principiante}{}{}{}{}{adj.2g.}{Que está ou se dá no início, que principia.}{prin.ci.pi.an.te}{0}
\verb{principiante}{}{}{}{}{}{Iniciante, inexperiente.}{prin.ci.pi.an.te}{0}
\verb{principiar}{}{}{}{}{v.t.}{Dar início a; começar.}{prin.ci.pi.ar}{\verboinum{1}}
\verb{princípio}{}{}{}{}{s.m.}{Ponto de partida de alguma coisa; começo, início.}{prin.cí.pio}{0}
\verb{princípio}{}{}{}{}{}{Regra, preceito, lei.}{prin.cí.pio}{0}
\verb{prior}{ô}{}{}{}{s.m.}{Superior de ordem religiosa; abade.}{pri.or}{0}
\verb{prioridade}{}{}{}{}{s.f.}{Condição do que está em primeiro lugar.}{pri.o.ri.da.de}{0}
\verb{prioridade}{}{}{}{}{}{Preferência, primazia.}{pri.o.ri.da.de}{0}
\verb{prioritário}{}{}{}{}{adj.}{Que tem prioridade, mais importante; preferencial.}{pri.o.ri.tá.rio}{0}
\verb{priorizar}{}{}{}{}{v.t.}{Dar prioridade a.}{pri.o.ri.zar}{\verboinum{1}}
\verb{prisão}{}{}{"-ões}{}{s.f.}{Ato de prender; aprisionamento, captura.}{pri.são}{0}
\verb{prisão}{}{}{"-ões}{}{}{Lugar em que fica o preso; cadeia.}{pri.são}{0}
\verb{prisco}{}{}{}{}{adj.}{Que pertence a tempos idos; antigo.}{pris.co}{0}
\verb{prisioneiro}{ê}{}{}{}{adj.}{Que perdeu a liberdade; preso, detento.}{pri.si.o.nei.ro}{0}
\verb{prisma}{}{Geom.}{}{}{s.m.}{Sólido formado por dois polígonos iguais na base e por paralelogramos nas laterais.}{pris.ma}{0}
\verb{prisma}{}{Fís.}{}{}{}{Sólido, triangular e transparente, capaz de decompor os raios de luz.}{pris.ma}{0}
\verb{prisma}{}{}{}{}{}{Ponto de vista.}{pris.ma}{0}
\verb{prismático}{}{}{}{}{adj.}{Referente a ou que tem feitio de prisma.}{pris.má.ti.co}{0}
\verb{prístino}{}{}{}{}{adj.}{Prisco.}{prís.ti.no}{0}
\verb{privação}{}{}{"-ões}{}{s.f.}{Supressão de um bem ou de uma faculdade normal.}{pri.va.ção}{0}
\verb{privacidade}{}{}{}{}{s.f.}{Vida particular; intimidade.}{pri.va.ci.da.de}{0}
\verb{privações}{}{}{}{}{s.f.pl.}{Necessidade, fome, miséria.}{pri.va.ções}{0}
\verb{privada}{}{}{}{}{s.f.}{Vaso sanitário.}{pri.va.da}{0}
\verb{privado}{}{}{}{}{adj.}{Que não é público ou que não tem caráter público; particular.}{pri.va.do}{0}
\verb{privado}{}{}{}{}{}{Que ficou sem alguma coisa de que precisava; desprovido.}{pri.va.do}{0}
%\verb{}{}{}{}{}{}{}{}{0}
\verb{privar}{}{}{}{}{v.t.}{Tirar algo de alguém ou de si mesmo.}{pri.var}{0}
\verb{privar}{}{}{}{}{}{Participar, gozar.}{pri.var}{0}
\verb{privar}{}{}{}{}{}{Proibir.}{pri.var}{\verboinum{1}}
\verb{privativo}{}{}{}{}{adj.}{Usado só por uma ou por algumas pessoas; de uso exclusivo, particular, próprio.}{pri.va.ti.vo}{0}
\verb{privatização}{}{}{"-ões}{}{s.f.}{Ato ou efeito de privatizar.}{pri.va.ti.za.ção}{0}
\verb{privatização}{}{}{"-ões}{}{}{Transferência do que é estatal para o domínio da iniciativa privada.}{pri.va.ti.za.ção}{0}
\verb{privatizar}{}{}{}{}{v.t.}{Realizar aquisição ou incorporação de empresa do setor público por empresa privada.}{pri.va.ti.zar}{\verboinum{1}}
\verb{privilegiado}{}{}{}{}{adj.}{Que goza de privilégio, de vantagem, de preferência, de prerrogativa etc.}{pri.vi.le.gi.a.do}{0}
\verb{privilegiar}{}{}{}{}{v.t.}{Conceder privilégio, direito especial, prerrogativa ou outro benefício a; beneficiar, favorecer.}{pri.vi.le.gi.ar}{\verboinum{1}}
\verb{privilégio}{}{}{}{}{s.m.}{Regalia para alguém ou um grupo, em detrimento da maioria.}{pri.vi.lé.gio}{0}
\verb{privilégio}{}{}{}{}{}{Oportunidade especial.}{pri.vi.lé.gio}{0}
\verb{privilégio}{}{}{}{}{}{Talento, dom.}{pri.vi.lé.gio}{0}
\verb{pró}{}{}{}{}{adv.}{A favor.}{pró}{0}
\verb{pró}{}{}{}{}{s.m.}{Vantagem, conveniência.}{pró}{0}
\verb{proa}{ô}{}{}{}{s.f.}{A parte dianteira de uma embarcação.}{pro.a}{0}
\verb{probabilidade}{}{}{}{}{s.f.}{Perspectiva favorável de algo vir a ocorrer; possibilidade, chance. }{pro.ba.bi.li.da.de}{0}
\verb{probabilidade}{}{}{}{}{}{Número provável correspondente a algo, calculado estatisticamente.}{pro.ba.bi.li.da.de}{0}
\verb{probante}{}{}{}{}{adj.2g.}{Que prova judicialmente.}{pro.ban.te}{0}
\verb{probatório}{}{}{}{}{adj.}{Que contém ou serve de prova.}{pro.ba.tó.rio}{0}
\verb{probidade}{}{}{}{}{s.f.}{Qualidade de probo; integridade de caráter; honradez.}{pro.bi.da.de}{0}
\verb{problema}{}{}{}{}{s.m.}{Algo de difícil solução ou explicação.}{pro.ble.ma}{0}
\verb{problema}{}{}{}{}{}{Situação difícil.}{pro.ble.ma}{0}
\verb{problema}{}{}{}{}{}{Questão matemática para ser solucionada.}{pro.ble.ma}{0}
\verb{problema}{}{}{}{}{}{Distúrbio orgânico ou físico.}{pro.ble.ma}{0}
\verb{problemática}{}{}{}{}{s.f.}{Conjunto de problemas da mesma natureza ou de um mesmo campo de atuação, ou concernentes a um mesmo objeto.}{pro.ble.má.ti.ca}{0}
\verb{problemático}{}{}{}{}{adj.}{Relativo a ou da natureza de um problema.}{pro.ble.má.ti.co}{0}
\verb{problemático}{}{}{}{}{}{Complicado, intrincado.}{pro.ble.má.ti.co}{0}
\verb{problemático}{}{}{}{}{}{Incerto, duvidoso.}{pro.ble.má.ti.co}{0}
\verb{problemático}{}{}{}{}{}{Que tem problemas psíquicos.}{pro.ble.má.ti.co}{0}
\verb{problemático}{}{}{}{}{}{Que faz de tudo um problema.}{pro.ble.má.ti.co}{0}
\verb{probo}{ô}{}{}{}{adj.}{Que é de caráter íntegro; honrado, honesto.}{pro.bo}{0}
\verb{probóscide}{}{Zool.}{}{}{s.f.}{A tromba do elefante.}{pro.bós.ci.de}{0}
\verb{probóscide}{}{Zool.}{}{}{}{Aparelho bucal dos mosquitos.}{pro.bós.ci.de}{0}
\verb{proboscídeo}{}{Zool.}{}{}{adj.}{Relativo aos proboscídeos.}{pro.bos.cí.deo}{0}
\verb{proboscídeo}{}{Zool.}{}{}{s.m.}{Espécime dos proboscídeos, mamíferos cujo focinho é prolongado em forma de tromba, como o elefante.}{pro.bos.cí.deo}{0}
\verb{procaz}{}{}{}{}{adj.2g.}{Que procede com petulância, atrevimento; insolente, descarado.}{pro.caz}{0}
\verb{procedência}{}{}{}{}{s.f.}{Ato ou efeito de proceder.}{pro.ce.dên.cia}{0}
\verb{procedência}{}{}{}{}{}{Lugar de onde se procede.}{pro.ce.dên.cia}{0}
\verb{procedência}{}{}{}{}{}{Proveniência, origem.}{pro.ce.dên.cia}{0}
\verb{procedência}{}{Jur.}{}{}{}{Justa causa; fundamento, razão.}{pro.ce.dên.cia}{0}
\verb{procedente}{}{}{}{}{adj.2g.}{Que tem fundamento, que se justifica.}{pro.ce.den.te}{0}
\verb{procedente}{}{}{}{}{}{Oriundo, originário.}{pro.ce.den.te}{0}
\verb{proceder}{ê}{}{}{}{v.t.}{Ser descendente; descender, provir.}{pro.ce.der}{0}
\verb{proceder}{ê}{}{}{}{}{Ter como ponto de partida um lugar; vir.}{pro.ce.der}{0}
\verb{proceder}{ê}{}{}{}{v.i.}{Ter fundamento.}{pro.ce.der}{0}
\verb{proceder}{ê}{}{}{}{}{Ter determinado comportamento; agir, comportar"-se.}{pro.ce.der}{0}
\verb{proceder}{ê}{}{}{}{}{Executar, realizar.}{pro.ce.der}{\verboinum{12}}
\verb{procedimento}{}{}{}{}{s.m.}{Maneira de fazer alguma coisa; método, processo.}{pro.ce.di.men.to}{0}
\verb{procedimento}{}{}{}{}{}{Maneira de agir; comportamento, conduta.}{pro.ce.di.men.to}{0}
\verb{procela}{é}{}{}{}{s.f.}{Forte tempestade no mar.}{pro.ce.la}{0}
\verb{procelária}{}{Zool.}{}{}{s.f.}{Ave marinha de grande porte, que chega ocasionalmente ao litoral.}{pro.ce.lá.ria}{0}
\verb{proceloso}{ô}{}{"-osos ⟨ó⟩}{"-osa ⟨ó⟩}{adj.}{Relativo a procela; tempestuoso.}{pro.ce.lo.so}{0}
\verb{proceloso}{ô}{}{"-osos ⟨ó⟩}{"-osa ⟨ó⟩}{}{Que traz procela.}{pro.ce.lo.so}{0}
\verb{prócer}{}{}{}{}{s.m.}{Indivíduo importante, influente; chefe, líder.}{pró.cer}{0}
\verb{processador}{ô}{}{}{}{adj.}{Que processa.}{pro.ces.sa.dor}{0}
\verb{processador}{ô}{}{}{}{}{Que realiza o processamento de dados em um computador.}{pro.ces.sa.dor}{0}
\verb{processamento}{}{}{}{}{s.m.}{Ato ou efeito de processar.}{pro.ces.sa.men.to}{0}
\verb{processar}{}{}{}{}{v.t.}{Mover ação judicial contra alguém ou algo.}{pro.ces.sar}{0}
\verb{processar}{}{}{}{}{}{Submeter a processamento de dados.}{pro.ces.sar}{\verboinum{1}}
\verb{processo}{é}{}{}{}{s.m.}{Ação continuada; andamento.}{pro.ces.so}{0}
\verb{processo}{é}{}{}{}{}{Método, procedimento.}{pro.ces.so}{0}
\verb{processo}{é}{}{}{}{}{Ação judicial.}{pro.ces.so}{0}
\verb{processual}{}{}{"-ais}{}{adj.2g.}{Relativo a processo judicial.}{pro.ces.su.al}{0}
\verb{procissão}{}{}{"-ões}{}{s.f.}{Cortejo de cunho religioso.}{pro.cis.são}{0}
\verb{proclama}{}{}{}{}{s.f.}{Proclamação de casamento na igreja.}{pro.cla.ma}{0}
\verb{proclama}{}{}{}{}{s.f.pl.}{Edital de casamento publicado.}{pro.cla.ma}{0}
\verb{proclamação}{}{}{"-ões}{}{s.f.}{Ato ou efeito de proclamar.}{pro.cla.ma.ção}{0}
\verb{proclamação}{}{}{"-ões}{}{}{Publicação solene.}{pro.cla.ma.ção}{0}
\verb{proclamação}{}{}{"-ões}{}{}{Manifesto.}{pro.cla.ma.ção}{0}
\verb{proclamar}{}{}{}{}{v.t.}{Declarar publicamente.}{pro.cla.mar}{0}
\verb{proclamar}{}{}{}{}{}{Atribuir título ou posto.}{pro.cla.mar}{0}
\verb{proclamar}{}{}{}{}{}{Afirmar categoricamente.}{pro.cla.mar}{0}
\verb{proclamar}{}{}{}{}{}{Promulgar uma lei.}{pro.cla.mar}{\verboinum{1}}
\verb{próclise}{}{Gram.}{}{}{s.f.}{Colocação do pronome átono antes do verbo.}{pró.cli.se}{0}
\verb{procrastinar}{}{}{}{}{v.t.}{Transferir para outro dia ou deixar para depois; adiar.}{pro.cras.ti.nar}{\verboinum{1}}
\verb{procriação}{}{}{"-ões}{}{s.f.}{Ato ou efeito de procriar; reprodução.}{pro.cri.a.ção}{0}
\verb{procriador}{ô}{}{}{}{s.m.}{Indivíduo que procria; reprodutor, genitor.}{pro.cri.a.dor}{0}
\verb{procriar}{}{}{}{}{v.t.}{Dar existência, nascimento, origem a.}{pro.cri.ar}{\verboinum{1}}
\verb{procura}{}{}{}{}{s.f.}{Ato ou efeito de procurar; busca.}{pro.cu.ra}{0}
\verb{procuração}{}{}{"-ões}{}{s.f.}{Poder que uma pessoa dá a outra para agir em seu nome; autorização. }{pro.cu.ra.ção}{0}
\verb{procuração}{}{}{"-ões}{}{}{Instrumento legal que confere essa autorização.}{pro.cu.ra.ção}{0}
\verb{procurador}{ô}{}{}{}{adj.}{Que procura algo.}{pro.cu.ra.dor}{0}
\verb{procurador}{ô}{}{}{}{s.m.}{Advogado do Estado.}{pro.cu.ra.dor}{0}
\verb{procurador}{ô}{}{}{}{}{Indivíduo que possui procuração para tratar dos negócios de outrem.}{pro.cu.ra.dor}{0}
\verb{procuradoria}{}{}{}{}{s.f.}{Cargo de procurador.}{pro.cu.ra.do.ri.a}{0}
\verb{procuradoria}{}{}{}{}{}{Repartição onde trabalha o procurador.}{pro.cu.ra.do.ri.a}{0}
\verb{procuradoria}{}{}{}{}{}{Honorário pago ao procurador.}{pro.cu.ra.do.ri.a}{0}
\verb{procurar}{}{}{}{}{v.t.}{Executar as ações necessárias para tentar encontrar algo.}{pro.cu.rar}{0}
\verb{procurar}{}{}{}{}{}{Fazer pesquisa para descobrir algo; investigar, pesquisar, buscar.}{pro.cu.rar}{0}
\verb{procurar}{}{}{}{}{}{Ir ao encontro de.}{pro.cu.rar}{\verboinum{1}}
\verb{prodigalidade}{}{}{}{}{s.f.}{Grande quantidade de algo; abundância, fartura.}{pro.di.ga.li.da.de}{0}
\verb{prodigalidade}{}{}{}{}{}{Esbanjamento, desperdício.}{pro.di.ga.li.da.de}{0}
\verb{prodigalidade}{}{}{}{}{}{Generosidade.}{pro.di.ga.li.da.de}{0}
\verb{prodigalizar}{}{}{}{}{v.t.}{Dar em grande quantidade.}{pro.di.ga.li.zar}{0}
\verb{prodigalizar}{}{}{}{}{}{Expor a perigos.}{pro.di.ga.li.zar}{0}
\verb{prodigalizar}{}{}{}{}{}{Dissipar, dilapidar.}{pro.di.ga.li.zar}{\verboinum{1}}
\verb{prodígio}{}{}{}{}{s.m.}{Fato extraordinário.}{pro.dí.gio}{0}
\verb{prodígio}{}{}{}{}{}{Indivíduo que apresenta um talento fora do comum.}{pro.dí.gio}{0}
\verb{prodigioso}{ô}{}{"-osos ⟨ó⟩}{"-osa ⟨ó⟩}{adj.}{Que é fora do comum; extraordinário.}{pro.di.gi.o.so}{0}
\verb{pródigo}{}{}{}{}{adj.}{Que gasta muito; esbanjador.}{pró.di.go}{0}
\verb{pródigo}{}{}{}{}{}{Que gosta de dar o que tem; generoso.}{pró.di.go}{0}
\verb{pródigo}{}{}{}{}{}{Fértil, abundante.}{pró.di.go}{0}
\verb{pródromo}{}{}{}{}{s.m.}{Espécie de prefácio.}{pró.dro.mo}{0}
\verb{pródromo}{}{Med.}{}{}{s.m.}{Sintoma inicial de uma doença.}{pró.dro.mo}{0}
\verb{produção}{}{}{"-ões}{}{s.f.}{Ato ou efeito de produzir; geração, criação.}{pro.du.ção}{0}
\verb{produção}{}{}{"-ões}{}{}{Obra de um artista, escritor etc.}{pro.du.ção}{0}
\verb{produção}{}{}{"-ões}{}{}{Volume do que foi produzido.}{pro.du.ção}{0}
\verb{produção}{}{}{"-ões}{}{}{Reunião dos recursos necessários para a realização de shows, filmes, peças teatrais etc.}{pro.du.ção}{0}
\verb{producente}{}{}{}{}{adj.2g.}{Que produz ou leva à produção.}{pro.du.cen.te}{0}
\verb{produtividade}{}{}{}{}{s.f.}{Qualidade do que é produtivo.}{pro.du.ti.vi.da.de}{0}
\verb{produtividade}{}{}{}{}{}{Fertilidade, capacidade de produzir.}{pro.du.ti.vi.da.de}{0}
\verb{produtivo}{}{}{}{}{adj.}{Que produz muito; fértil.}{pro.du.ti.vo}{0}
\verb{produtivo}{}{}{}{}{}{Que dá bons resultados; lucrativo, proveitoso.}{pro.du.ti.vo}{0}
\verb{produto}{}{}{}{}{s.m.}{Resultado de um trabalho ou de uma atividade.}{pro.du.to}{0}
\verb{produto}{}{Mat.}{}{}{}{Resultado da operação de multiplicação.}{pro.du.to}{0}
\verb{produtor}{ô}{}{}{}{adj.}{Que produz.}{pro.du.tor}{0}
\verb{produtor}{ô}{}{}{}{s.m.}{Indivíduo que financia ou supervisiona um filme, um show, uma montagem, um programa de televisão etc.}{pro.du.tor}{0}
\verb{produzir}{}{}{}{}{v.t.}{Ser fértil; gerar.}{pro.du.zir}{0}
\verb{produzir}{}{}{}{}{}{Criar.}{pro.du.zir}{0}
\verb{produzir}{}{}{}{}{}{Fabricar.}{pro.du.zir}{0}
\verb{produzir}{}{}{}{}{}{Financiar filmes, peças teatrais, shows etc.}{pro.du.zir}{0}
\verb{produzir}{}{}{}{}{v.pron.}{Embelezar"-se.}{pro.du.zir}{\verboinum{21}}
\verb{proeminência}{}{}{}{}{s.f.}{A parte mais saliente de algo.}{pro.e.mi.nên.cia}{0}
\verb{proeminência}{}{}{}{}{}{Superioridade, elevação.}{pro.e.mi.nên.cia}{0}
\verb{proeminente}{}{}{}{}{adj.2g.}{Que sobressai; saliente.}{pro.e.mi.nen.te}{0}
\verb{proeminente}{}{}{}{}{}{Superior.}{pro.e.mi.nen.te}{0}
\verb{proêmio}{}{}{}{}{s.m.}{Aquilo que abre ou inicia alguma coisa.}{pro.ê.mio}{0}
\verb{proeza}{ê}{}{}{}{s.f.}{Algo difícil de ser realizado; façanha, feito.}{pro.e.za}{0}
\verb{profanação}{}{}{"-ões}{}{s.f.}{Ato ou efeito de profanar; sacrilégio.}{pro.fa.na.ção}{0}
\verb{profanar}{}{}{}{}{v.t.}{Violar ou tratar com irreverência o que é sagrado.}{pro.fa.nar}{0}
\verb{profanar}{}{}{}{}{}{Desonrar.}{pro.fa.nar}{\verboinum{1}}
\verb{profano}{}{}{}{}{adj.}{Que não é sagrado.}{pro.fa.no}{0}
\verb{profano}{}{}{}{}{}{Que viola o sagrado.}{pro.fa.no}{0}
\verb{profano}{}{}{}{}{}{Que não tem conhecimento em algum assunto; leigo.}{pro.fa.no}{0}
\verb{profecia}{}{}{}{}{s.f.}{Predição do futuro feita por um profeta.}{pro.fe.ci.a}{0}
\verb{proferir}{}{}{}{}{v.t.}{Dizer em voz alta; pronunciar.}{pro.fe.rir}{\verboinum{29}}
\verb{professar}{}{}{}{}{v.t.}{Seguir uma doutrina.}{pro.fes.sar}{\verboinum{1}}
\verb{professo}{é}{}{}{}{adj.}{Que professa em uma ordem religiosa.}{pro.fes.so}{0}
\verb{professo}{é}{}{}{}{}{Que é perito, capaz.}{pro.fes.so}{0}
\verb{professor}{ô}{}{}{}{s.m.}{Indivíduo que ensina uma ciência ou arte; mestre.}{pro.fes.sor}{0}
\verb{professorado}{}{}{}{}{s.m.}{Cargo ou função de professor; magistério.}{pro.fes.so.ra.do}{0}
\verb{professorado}{}{}{}{}{}{A classe dos professores.}{pro.fes.so.ra.do}{0}
\verb{professoral}{}{}{"-ais}{}{adj.2g.}{Relativo a professor.}{pro.fes.so.ral}{0}
\verb{professoral}{}{}{"-ais}{}{}{Semelhante a ou que lembra um professor.}{pro.fes.so.ral}{0}
\verb{profeta}{é}{}{}{}{s.m.}{Indivíduo que prediz o futuro; adivinho, vidente.}{pro.fe.ta}{0}
\verb{profético}{}{}{}{}{adj.}{Relativo a profeta ou a profecia.}{pro.fé.ti.co}{0}
\verb{profetisa}{}{}{}{}{s.f.}{Feminino de \textit{profeta}.}{pro.fe.ti.sa}{0}
\verb{profetizar}{}{}{}{}{v.t.}{Predizer o futuro, por dom especial ou inspiração divina.}{pro.fe.ti.zar}{0}
\verb{profetizar}{}{}{}{}{}{Prever algo por dedução ou conjetura.}{pro.fe.ti.zar}{\verboinum{1}}
\verb{proficiente}{}{}{}{}{adj.2g.}{Que tem pleno conhecimento; competente, conhecedor.}{pro.fi.ci.en.te}{0}
\verb{profícuo}{}{}{}{}{adj.}{Que é útil, proveitoso, vantajoso.}{pro.fí.cu.o}{0}
\verb{profilático}{}{}{}{}{adj.}{Relativo a profilaxia.}{pro.fi.lá.ti.co}{0}
\verb{profilaxia}{cs}{}{}{}{s.f.}{Parte da medicina que cuida da preservação da saúde por meio de medidas preventivas.}{pro.fi.la.xi.a}{0}
\verb{profilaxia}{cs}{}{}{}{}{Utilização dessas medidas.}{pro.fi.la.xi.a}{0}
\verb{profilaxia}{cs}{}{}{}{}{Aquilo que preserva.}{pro.fi.la.xi.a}{0}
\verb{profissão}{}{}{"-ões}{}{s.f.}{Ato de professar.}{pro.fis.são}{0}
\verb{profissão}{}{}{"-ões}{}{}{Atividade especializada para a qual a pessoa se prepara. }{pro.fis.são}{0}
\verb{profissional}{}{}{"-ais}{}{adj.2g.}{Próprio de uma profissão.}{pro.fis.si.o.nal}{0}
\verb{profissional}{}{}{"-ais}{}{}{Que exerce uma atividade por profissão.}{pro.fis.si.o.nal}{0}
\verb{profissionalizar}{}{}{}{}{v.t.}{Ensinar alguém a ser profissional.}{pro.fis.si.o.na.li.zar}{0}
\verb{profissionalizar}{}{}{}{}{}{Aperfeiçoar, capacitar.}{pro.fis.si.o.na.li.zar}{0}
\verb{profissionalizar}{}{}{}{}{}{Dar caráter profissional a.}{pro.fis.si.o.na.li.zar}{\verboinum{1}}
\verb{profligar}{}{}{}{}{v.t.}{Destruir, arrasar.}{pro.fli.gar}{0}
\verb{profligar}{}{}{}{}{}{Atacar ou combater com palavras; reprovar, verberar.}{pro.fli.gar}{\verboinum{5}}
\verb{prófugo}{}{}{}{}{adj.}{Que anda a fugir; fugitivo.}{pró.fu.go}{0}
\verb{prófugo}{}{}{}{}{}{Desertor.}{pró.fu.go}{0}
\verb{profundeza}{ê}{}{}{}{s.f.}{Profundidade.}{pro.fun.de.za}{0}
\verb{profundidade}{}{}{}{}{s.f.}{Distância vertical da superfície ao fundo ou horizontal de fora para dentro; profundeza.}{pro.fun.di.da.de}{0}
\verb{profundidade}{}{}{}{}{}{O que vai além da aparência.}{pro.fun.di.da.de}{0}
\verb{profundidade}{}{}{}{}{}{Âmago, intimidade.}{pro.fun.di.da.de}{0}
\verb{profundo}{}{}{}{}{adj.}{Muito fundo.}{pro.fun.do}{0}
\verb{profundo}{}{}{}{}{}{Que vem do íntimo; entranhado.}{pro.fun.do}{0}
\verb{profundo}{}{}{}{}{}{Que sabe muito.}{pro.fun.do}{0}
\verb{profundo}{}{Fig.}{}{}{}{Muito importante, de grande alcance.}{pro.fun.do}{0}
\verb{profundo}{}{Fig.}{}{}{}{Que não é superficial, que penetra muito.}{pro.fun.do}{0}
\verb{profundo}{}{}{}{}{}{Marcante.}{pro.fun.do}{0}
\verb{profundo}{}{}{}{}{}{Desmedido, excessivo.}{pro.fun.do}{0}
\verb{profusão}{}{}{"-ões}{}{s.f.}{Grande quantidade de alguma coisa; fartura, abundância, exuberância.}{pro.fu.são}{0}
\verb{profuso}{}{}{}{}{adj.}{Que dá, gasta ou espalha com profusão; farto, abundante, exuberante, copioso, pródigo.}{pro.fu.so}{0}
\verb{progênie}{}{}{}{}{s.f.}{Origem, ascendência; procedência.}{pro.gê.nie}{0}
\verb{progênie}{}{}{}{}{}{Prole, geração, progenitura.}{pro.gê.nie}{0}
\verb{progenitor}{ô}{}{}{}{s.m.}{Pessoa da qual alguém descende; ancestral, ascendente, antepassado.}{pro.ge.ni.tor}{0}
\verb{progenitor}{ô}{Por ext.}{}{}{}{Pai.}{pro.ge.ni.tor}{0}
\verb{progenitura}{}{}{}{}{s.f.}{Progênie.}{pro.ge.ni.tu.ra}{0}
\verb{progesterona}{}{Bioquím.}{}{}{s.f.}{Hormônio sexual feminino responsável pela preparação do útero para a fixação e fertilização do óvulo. }{pro.ges.te.ro.na}{0}
\verb{prognata}{}{}{}{}{adj.2g.}{Prógnato.}{prog.na.ta}{0}
\verb{prógnata}{}{}{}{}{adj.2g.}{Prógnato.}{próg.na.ta}{0}
\verb{prognatismo}{}{Med.}{}{}{s.m.}{Projeção anormal do maxilar inferior para a frente. }{prog.na.tis.mo}{0}
\verb{prógnato}{}{}{}{}{adj.}{Que apresenta o maxilar inferior anormalmente projetado para a frente.}{próg.na.to}{0}
\verb{prognosticar}{}{}{}{}{v.t.}{Fazer prognóstico; predizer, conjeturar, antecipar, anunciar.}{prog.nos.ti.car}{\verboinum{2}}
\verb{prognóstico}{}{}{}{}{s.m.}{Conhecimento baseado em suposições; previsão.}{prog.nós.ti.co}{0}
\verb{prognóstico}{}{Med.}{}{}{}{Resultado provável sobre uma doença.}{prog.nós.ti.co}{0}
\verb{programa}{}{}{}{}{s.m.}{Lista em que se enumeram as partes de uma atividade; programação, plano.}{pro.gra.ma}{0}
\verb{programa}{}{}{}{}{}{Cada quadro de uma apresentação de rádio ou de televisão.}{pro.gra.ma}{0}
\verb{programa}{}{Informát.}{}{}{}{Conjunto de informações ou dados com que se alimenta um computador; \textit{software}.}{pro.gra.ma}{0}
\verb{programação}{}{}{"-ões}{}{s.f.}{Ato ou efeito de programar.}{pro.gra.ma.ção}{0}
\verb{programação}{}{Informát.}{"-ões}{}{}{Criação, desenvolvimento de um programa de computador.}{pro.gra.ma.ção}{0}
\verb{programador}{ô}{}{}{}{s.m.}{Indivíduo que programa, que faz programação.}{pro.gra.ma.dor}{0}
\verb{programador}{ô}{Informát.}{}{}{}{Indivíduo que desenvolve, escreve e testa programas de computador.}{pro.gra.ma.dor}{0}
\verb{programar}{}{}{}{}{v.t.}{Fazer o programa de alguma atividade; projetar, planejar.}{pro.gra.mar}{\verboinum{1}}
\verb{progredir}{}{}{}{}{v.i.}{Caminhar adiante; avançar, prosseguir.}{pro.gre.dir}{0}
\verb{progredir}{}{}{}{}{}{Tornar"-se melhor ou maior; desenvolver"-se, crescer, evoluir.}{pro.gre.dir}{\verboinum{30}}
\verb{progressão}{}{}{"-ões}{}{s.f.}{Desenvolvimento ou avanço contínuo; prosseguimento, sucessão, evolução.}{pro.gres.são}{0}
\verb{progressista}{}{}{}{}{adj.2g.}{Que é favorável ao progresso.}{pro.gres.sis.ta}{0}
\verb{progressivo}{}{}{}{}{adj.}{Que progride continuamente; gradual, gradativo.}{pro.gres.si.vo}{0}
\verb{progresso}{é}{}{}{}{s.m.}{Movimento para a frente; avanço, desenvolvimento, evolução.}{pro.gres.so}{0}
\verb{proibição}{}{}{"-ões}{}{s.f.}{Ato ou efeito de proibir.}{pro.i.bi.ção}{0}
\verb{proibido}{}{}{}{}{adj.}{Que não é permitido por lei; ilícito, ilegal.}{pro.i.bi.do}{0}
\verb{proibido}{}{}{}{}{}{Que não é permitido; interdito.}{pro.i.bi.do}{0}
\verb{proibidor}{ô}{}{}{}{adj.}{Que proíbe, interdita.}{pro.i.bi.dor}{0}
\verb{proibir}{}{}{}{}{v.t.}{Impedir a ação; não consentir; opor"-se a; obstar.}{pro.i.bir}{\verboinum{27}}
\verb{proibitivo}{}{}{}{}{adj.}{Que proíbe, que impede; proibitório.}{pro.i.bi.ti.vo}{0}
\verb{proibitivo}{}{}{}{}{}{Diz"-se de preço elevado, abusivo de alguma mercadoria ou serviço.}{pro.i.bi.ti.vo}{0}
\verb{projeção}{}{}{"-ões}{}{s.f.}{Ato ou efeito de projetar; lançamento, arremesso.}{pro.je.ção}{0}
\verb{projeção}{}{}{"-ões}{}{}{Exibição de um filme através de projetor.}{pro.je.ção}{0}
\verb{projeção}{}{}{"-ões}{}{}{Destaque, importância, prestígio.}{pro.je.ção}{0}
\verb{projetar}{}{}{}{}{v.t.}{Fazer um projeto ou uma planta; delinear.}{pro.je.tar}{0}
\verb{projetar}{}{}{}{}{}{Fazer a projeção de; exibir.}{pro.je.tar}{0}
\verb{projetar}{}{}{}{}{}{Atirar a distância; arremessar, lançar.}{pro.je.tar}{\verboinum{1}}
\verb{projetil}{}{}{"-is}{}{s.m.}{Projétil.}{pro.je.til}{0}
\verb{projétil}{}{}{"-eis}{}{s.m.}{Objeto atirado por uma arma.}{pro.jé.til}{0}
\verb{projetista}{}{}{}{}{s.2g.}{Profissional que faz projetos.}{pro.je.tis.ta}{0}
\verb{projeto}{é}{}{}{}{s.m.}{Plano para realizar algo no futuro; intento, ideia.}{pro.je.to}{0}
\verb{projeto}{é}{}{}{}{}{Plano, escrito e detalhado, de uma construção.}{pro.je.to}{0}
\verb{projeto}{é}{}{}{}{}{Redação provisória de uma lei.}{pro.je.to}{0}
\verb{projetor}{ô}{}{}{}{s.m.}{Qualquer aparelho destinado a lançar a distância ondas de luz, de som ou de calor.}{pro.je.tor}{0}
\verb{projetor}{ô}{}{}{}{}{Aparelho empregado para lançar feixes de luz a grandes distâncias ou iluminar exteriormente grandes edifícios; holofote.}{pro.je.tor}{0}
\verb{projetor}{ô}{}{}{}{}{Aparelho de projeção cinematográfica.}{pro.je.tor}{0}
\verb{prol}{ó}{}{}{}{s.m.}{Lucro, vantagem, proveito.}{prol}{0}
\verb{prol}{ó}{}{}{}{}{Usado na locução prepositiva \textit{em prol de}: em favor de, em proveito de; em defesa de.}{prol}{0}
\verb{prolação}{}{}{"-ões}{}{s.f.}{Ato ou efeito de pronunciar, de proferir; pronunciação, articulação. }{pro.la.ção}{0}
\verb{prolapso}{}{Med.}{}{}{s.m.}{Queda ou deslocamento de um órgão de sua posição normal.}{pro.lap.so}{0}
\verb{prolatar}{}{Bras.}{}{}{v.t.}{Pronunciar sentença; promulgar, proferir. }{pro.la.tar}{\verboinum{1}}
\verb{prole}{ó}{}{}{}{s.f.}{Descendência, geração.}{pro.le}{0}
\verb{prole}{ó}{}{}{}{}{Os filhos.}{pro.le}{0}
\verb{proletariado}{}{}{}{}{s.m.}{A classe dos proletários, dos trabalhadores assalariados.}{pro.le.ta.ri.a.do}{0}
\verb{proletário}{}{}{}{}{s.m.}{Indivíduo que vive de seu salário; operário.}{pro.le.tá.rio}{0}
\verb{proliferação}{}{}{"-ões}{}{s.f.}{Ato ou efeito de proliferar; multiplicação, reprodução.}{pro.li.fe.ra.ção}{0}
\verb{proliferar}{}{}{}{}{v.i.}{Crescer em número; multiplicar"-se, propagar"-se.}{pro.li.fe.rar}{\verboinum{1}}
\verb{prolífero}{}{}{}{}{adj.}{Prolífico.}{pro.lí.fe.ro}{0}
\verb{prolificar}{}{}{}{}{v.i.}{Gerar prole; reproduzir"-se.}{pro.li.fi.car}{\verboinum{2}}
\verb{prolífico}{}{}{}{}{adj.}{Que tem capacidade de gerar, reproduzir"-se; fecundo, fértil, prolífero.}{pro.lí.fi.co}{0}
\verb{prolixidade}{cs}{}{}{}{s.f.}{Qualidade do que é prolixo. }{pro.li.xi.da.de}{0}
\verb{prolixo}{cs}{}{}{}{adj.}{Que usa palavras em excesso para dizer algo.}{pro.li.xo}{0}
\verb{prolixo}{cs}{}{}{}{}{Muito extenso ou demorado.}{pro.li.xo}{0}
\verb{prólogo}{}{}{}{}{s.m.}{Primeira parte de obra literária, dramatúrgica, musical.}{pró.lo.go}{0}
\verb{prólogo}{}{Por ext.}{}{}{}{Prefácio.}{pró.lo.go}{0}
\verb{prolongamento}{}{}{}{}{s.m.}{Ato ou efeito de prolongar.}{pro.lon.ga.men.to}{0}
\verb{prolongar}{}{}{}{}{v.t.}{Aumentar a duração de; estender, alongar.}{pro.lon.gar}{0}
\verb{prolongar}{}{}{}{}{}{Tornar mais longo, extenso.}{pro.lon.gar}{0}
\verb{prolongar}{}{}{}{}{v.pron.}{Tratar longamente de um assunto; estender"-se, delongar"-se.}{pro.lon.gar}{0}
\verb{prolongar}{}{}{}{}{}{Estender"-se por grandes distâncias.}{pro.lon.gar}{\verboinum{5}}
\verb{prolóquio}{}{}{}{}{s.m.}{Regra, preceito, máxima.}{pro.ló.quio}{0}
%\verb{}{}{}{}{}{}{}{}{0}
\verb{promécio}{}{Quím.}{}{}{s.m.}{Elemento químico metálico, artificial, radioativo, da família dos lantanídeos (terras"-raras); usado na indústria de material fosforescente, pois pode ser incorporado a tintas. \elemento{61}{(145)}{Pm}.}{pro.mé.cio}{0}
\verb{promessa}{é}{}{}{}{s.f.}{Ato ou efeito de prometer; juramento, voto.}{pro.mes.sa}{0}
\verb{prometer}{ê}{}{}{}{v.t.}{Obrigar"-se a fazer ou dizer algo; comprometer"-se.}{pro.me.ter}{0}
\verb{prometer}{ê}{}{}{}{v.i.}{Dar esperanças ou probabilidades de progresso.}{pro.me.ter}{\verboinum{12}}
\verb{prometida}{}{}{}{}{s.f.}{Futura esposa; noiva.}{pro.me.ti.da}{0}
\verb{prometido}{}{}{}{}{adj.}{Que se prometeu; ajustado, apalavrado, reservado.}{pro.me.ti.do}{0}
\verb{prometido}{}{}{}{}{}{O futuro marido; noivo.}{pro.me.ti.do}{0}
\verb{promiscuidade}{}{}{}{}{s.f.}{Qualidade do que é promíscuo.}{pro.mis.cui.da.de}{0}
\verb{promiscuirse}{}{}{}{}{v.pron.}{Juntar"-se desordenadamente; misturar"-se, baralhar"-se.}{pro.mis.cu.ir"-se}{\verboinum{26}}
\verb{promíscuo}{}{}{}{}{adj.}{Agrupado sem ordem; misturado, confuso.}{pro.mís.cu.o}{0}
\verb{promissão}{}{}{"-ões}{}{s.f.}{Ato ou efeito de prometer; promessa.}{pro.mis.são}{0}
\verb{promissor}{ô}{}{}{}{adj.}{Que promete coisas boas; auspicioso.}{pro.mis.sor}{0}
\verb{promissória}{}{}{}{}{s.f.}{Documento que contém a promessa do pagamento de uma dívida.}{pro.mis.só.ria}{0}
\verb{promissório}{}{}{}{}{adj.}{Em que há promessa; promissivo.}{pro.mis.só.rio}{0}
\verb{promitente}{}{}{}{}{adj.2g.}{Que promete; promissor.}{pro.mi.ten.te}{0}
\verb{promoção}{}{}{"-ões}{}{s.f.}{Ato ou efeito de promover.}{pro.mo.ção}{0}
\verb{promoção}{}{}{"-ões}{}{}{Oferta, pelo comércio, de produtos com preços mais baixos.}{pro.mo.ção}{0}
\verb{promocional}{}{}{"-ais}{}{adj.2g.}{Relativo a promoção. }{pro.mo.ci.o.nal}{0}
\verb{promontório}{}{Geol.}{}{}{s.m.}{Ponta de terra, formada por rochas escarpadas, que avança mar adentro.}{pro.mon.tó.rio}{0}
\verb{promotor}{ô}{}{}{}{s.m.}{Indivíduo que promove, faz propaganda, anuncia.}{pro.mo.tor}{0}
\verb{promotor}{ô}{Jur.}{}{}{}{Funcionário público que promove o andamento de causas e atos da justiça. }{pro.mo.tor}{0}
\verb{promotoria}{}{}{}{}{s.f.}{Cargo, função ou gabinete de promotor.}{pro.mo.to.ri.a}{0}
\verb{promover}{ê}{}{}{}{v.t.}{Trabalhar a fim de realizar alguma coisa.}{pro.mo.ver}{0}
\verb{promover}{ê}{}{}{}{}{Fazer uma pessoa elevar de cargo, de posto, de emprego.}{pro.mo.ver}{0}
\verb{promover}{ê}{}{}{}{}{Fazer a promoção, tentar vender alguma coisa.}{pro.mo.ver}{\verboinum{12}}
\verb{promulgação}{}{}{"-ões}{}{s.f.}{Ato ou efeito de promulgar.}{pro.mul.ga.ção}{0}
\verb{promulgar}{}{}{}{}{v.t.}{Ordenar a publicação de lei ou similar.}{pro.mul.gar}{0}
\verb{promulgar}{}{}{}{}{}{Tornar público; publicar oficialmente.}{pro.mul.gar}{\verboinum{5}}
\verb{pronome}{}{Gram.}{}{}{s.m.}{Classe de palavra usada para acompanhar ou substituir o substantivo.}{pro.no.me}{0}
\verb{pronominal}{}{}{"-ais}{}{adj.2g.}{Relativo ao pronome.}{pro.no.mi.nal}{0}
\verb{prontidão}{}{}{"-ões}{}{s.f.}{Qualidade do que é ou está pronto.}{pron.ti.dão}{0}
\verb{prontidão}{}{}{"-ões}{}{}{Presteza, solicitude, diligência.}{pron.ti.dão}{0}
\verb{prontidão}{}{}{"-ões}{}{}{Situação de alerta em que as tropas militares são mantidas nos quartéis para entrar em ação se necessário.}{pron.ti.dão}{0}
\verb{prontificar}{}{}{}{}{v.t.}{Tornar pronto; aprontar.}{pron.ti.fi.car}{0}
\verb{prontificar}{}{}{}{}{}{Pôr à disposição; oferecer.}{pron.ti.fi.car}{0}
\verb{prontificar}{}{}{}{}{v.pron.}{Mostrar"-se disposto a; disponibilizar"-se, oferecer"-se.}{pron.ti.fi.car}{\verboinum{2}}
\verb{pronto}{}{}{}{}{adj.}{Terminado, acabado.}{pron.to}{0}
\verb{pronto}{}{}{}{}{}{Preparado, disposto.}{pron.to}{0}
\verb{pronto}{}{}{}{}{}{Que se faz ou resolve em pouco tempo; rápido.}{pron.to}{0}
\verb{pronto}{}{}{}{}{adv.}{Prontamente.}{pron.to}{0}
\verb{pronto"-socorro}{}{}{prontos"-socorros ⟨ó⟩}{}{s.m.}{Hospital ou seção de hospital para atender sem demora a casos de urgência.}{pron.to"-so.cor.ro}{0}
\verb{prontuário}{}{}{}{}{s.m.}{Ficha com dados diversos a respeito de uma pessoa.}{pron.tu.á.rio}{0}
\verb{pronúncia}{}{}{}{}{s.f.}{Ato ou efeito de pronunciar.}{pro.nún.cia}{0}
\verb{pronúncia}{}{}{}{}{}{Modo particular de alguém pronunciar as palavras; sotaque.}{pro.nún.cia}{0}
\verb{pronunciamento}{}{}{}{}{s.m.}{Ato ou efeito de pronunciar.}{pro.nun.ci.a.men.to}{0}
\verb{pronunciar}{}{}{}{}{v.t.}{Articular palavras; proferir, manifestar verbalmente.}{pro.nun.ci.ar}{0}
\verb{pronunciar}{}{}{}{}{v.pron.}{Emitir opinião.}{pro.nun.ci.ar}{0}
\verb{pronunciar}{}{}{}{}{}{Rebelar"-se.}{pro.nun.ci.ar}{\verboinum{1}}
\verb{propaganda}{}{}{}{}{s.f.}{Exaltação ostensiva ou dissimulada das qualidades positivas de uma ideologia, religião, mercadoria ou pessoa.}{pro.pa.gan.da}{0}
\verb{propaganda}{}{}{}{}{}{Peça publicitária; anúncio, reclame.}{pro.pa.gan.da}{0}
\verb{propagandista}{}{}{}{}{adj.2g.}{Que faz propaganda, que apregoa.}{pro.pa.gan.dis.ta}{0}
\verb{propagar}{}{}{}{}{v.t.}{Multiplicar, reproduzir.}{pro.pa.gar}{0}
\verb{propagar}{}{}{}{}{}{Espalhar, difundir.}{pro.pa.gar}{0}
\verb{propagar}{}{}{}{}{v.pron.}{Alastrar"-se por contágio.}{pro.pa.gar}{0}
\verb{propagar}{}{}{}{}{}{Tornar"-se geral; generalizar"-se.}{pro.pa.gar}{\verboinum{5}}
\verb{propalar}{}{}{}{}{v.t.}{Divulgar, publicar.}{pro.pa.lar}{0}
\verb{propalar}{}{}{}{}{v.pron.}{Propagar"-se.}{pro.pa.lar}{\verboinum{1}}
\verb{proparoxítono}{cs}{Gram.}{}{}{adj.}{Diz"-se do vocábulo cuja acentuação tônica cai na antepenúltima sílaba; esdrúxulo.}{pro.pa.ro.xí.to.no}{0}
\verb{propedêutica}{}{}{}{}{s.f.}{Conjunto de ensinamentos introdutórios de uma disciplina; introdução.}{pro.pe.dêu.ti.ca}{0}
\verb{propedêutico}{}{}{}{}{adj.}{Que serve de introdução; preliminar.}{pro.pe.dêu.ti.co}{0}
\verb{propelir}{}{}{}{}{v.t.}{Empurrar para diante; impelir, arremessar.}{pro.pe.lir}{\verboinum{29}}
\verb{propender}{ê}{}{}{}{v.t.}{Ter pendor; pender, inclinar"-se.}{pro.pen.der}{\verboinum{12}}
\verb{propensão}{}{}{"-ões}{}{s.f.}{Ato ou efeito de propender; inclinação, tendência.}{pro.pen.são}{0}
\verb{propenso}{}{}{}{}{adj.}{Que demonstra vontade em fazer algo; inclinado, disposto.}{pro.pen.so}{0}
\verb{propiciar}{}{}{}{}{v.t.}{Tornar possível, apropriado; proporcionar.}{pro.pi.ci.ar}{\verboinum{1}}
\verb{propício}{}{}{}{}{adj.}{Que proporciona; favorável, bom, apropriado.}{pro.pí.cio}{0}
\verb{propina}{}{}{}{}{s.f.}{Gratificação extra; gorjeta.}{pro.pi.na}{0}
\verb{propinar}{}{}{}{}{v.t.}{Oferecer bebida; ministrar.}{pro.pi.nar}{\verboinum{1}}
\verb{propínquo}{}{}{}{}{adj.}{Que está próximo, vizinho.}{pro.pín.quo}{0}
\verb{própole}{}{}{}{}{s.f.}{Própolis.}{pró.po.le}{0}
\verb{própolis}{}{}{}{}{s.2g.}{Matéria resinosa avermelhada segregada pelas abelhas e usada por elas para vedar fendas em suas colmeias.}{pró.po.lis}{0}
\verb{proponente}{}{}{}{}{adj.2g.}{Que propõe algo.}{pro.po.nen.te}{0}
\verb{propor}{}{}{}{}{v.t.}{Apresentar ou submeter à discussão; sugerir, expor.}{pro.por}{0}
\verb{propor}{}{}{}{}{v.pron.}{Dispor"-se, oferecer"-se, tencionar.}{pro.por}{\verboinum{60}}
\verb{proporção}{}{}{"-ões}{}{s.f.}{Correspondência ou relação entre as partes de um todo; harmonia, equilíbrio.}{pro.por.ção}{0}
\verb{proporção}{}{}{"-ões}{}{}{Relação entre medidas e tamanhos.}{pro.por.ção}{0}
\verb{proporcionado}{}{}{}{}{adj.}{Em que há proporção; harmônico, simétrico.}{pro.por.ci.o.na.do}{0}
\verb{proporcional}{}{}{"-ais}{}{adj.2g.}{Que varia conforme a mudança de algo.}{pro.por.ci.o.nal}{0}
\verb{proporcionar}{}{}{}{}{v.t.}{Dar oportunidade; possibilitar, propiciar.}{pro.por.ci.o.nar}{\verboinum{1}}
\verb{proposição}{}{}{"-ões}{}{s.f.}{Ato ou efeito de propor; proposta, sugestão.}{pro.po.si.ção}{0}
\verb{proposição}{}{Gram.}{"-ões}{}{}{Asserção, afirmativa, enunciado.}{pro.po.si.ção}{0}
\verb{propositado}{}{}{}{}{adj.}{Em que há propósito; intencional, proposital.}{pro.po.si.ta.do}{0}
\verb{proposital}{}{}{"-ais}{}{adj.2g.}{Feito com propósito; intencional, voluntário.}{pro.po.si.tal}{0}
\verb{propósito}{}{}{}{}{s.m.}{Intenção de se fazer algo; intento, resolução, determinação.}{pro.pó.si.to}{0}
\verb{proposta}{ó}{}{}{}{s.f.}{Aquilo que se propõe; sugestão, oferta.}{pro.pos.ta}{0}
\verb{proposto}{ô}{}{"-s ⟨ó⟩}{"-a ⟨ó⟩}{adj.}{Que se propôs.}{pro.pos.to}{0}
\verb{propriedade}{}{}{}{}{s.f.}{Qualidade do que é apropriado; adequação.}{pro.pri.e.da.de}{0}
\verb{propriedade}{}{}{}{}{}{Característica, peculiaridade, particularidade.}{pro.pri.e.da.de}{0}
\verb{propriedade}{}{}{}{}{}{Direito de usar, gozar e dispor de um bem.}{pro.pri.e.da.de}{0}
\verb{propriedade}{}{}{}{}{}{Imóvel pertencente a alguém.}{pro.pri.e.da.de}{0}
\verb{proprietário}{}{}{}{}{adj.}{Que tem a posse de algo.}{pro.pri.e.tá.rio}{0}
\verb{próprio}{}{}{}{}{adj.}{Que pertence a um indivíduo; pertencente.}{pró.prio}{0}
\verb{próprio}{}{}{}{}{}{Relativo ao sujeito da oração; inerente, peculiar, característico.}{pró.prio}{0}
\verb{próprio}{}{}{}{}{}{Conveniente, adequado, apropriado. (\textit{Aquele não era o momento próprio para risadas.})}{pró.prio}{0}
\verb{próprio}{}{}{}{}{}{Em pessoa; mesmo. (\textit{Ele próprio lava suas roupas.})}{pró.prio}{0}
\verb{propugnador}{ô}{}{}{}{adj.}{Que propugna ou luta por algo; batalhador.}{pro.pug.na.dor}{0}
\verb{propugnar}{}{}{}{}{v.t.}{Lutar por algo; batalhar.}{pro.pug.nar}{\verboinum{1}}
\verb{propulsão}{}{}{"-ões}{}{s.f.}{Ato ou efeito de propelir ou propulsar; impulso, empurrão.}{pro.pul.são}{0}
\verb{propulsar}{}{}{}{}{v.t.}{Impelir para fora ou para longe; repelir.}{pro.pul.sar}{\verboinum{1}}
\verb{propulsionar}{}{}{}{}{v.t.}{Impulsionar para a frente; estimular.}{pro.pul.si.o.nar}{\verboinum{1}}
\verb{propulsor}{ô}{}{}{}{adj.}{Que propulsa; motor, impulsor.}{pro.pul.sor}{0}
\verb{prorrogação}{}{}{"-ões}{}{s.f.}{Ato ou efeito de prorrogar; dilatação do tempo.}{pror.ro.ga.ção}{0}
\verb{prorrogar}{}{}{}{}{v.t.}{Adiar o tempo; prolongar, adiar.}{pror.ro.gar}{\verboinum{5}}
\verb{prorrogável}{}{}{"-eis}{}{adj.2g.}{Que se pode prorrogar; adiável.}{pror.ro.gá.vel}{0}
\verb{prorromper}{ê}{}{}{}{v.i.}{Manifestar"-se com ímpeto; irromper.}{pror.rom.per}{\verboinum{12}}
\verb{prosa}{ó}{}{}{}{s.f.}{Maneira natural da linguagem escrita e falada, sem versificação.}{pro.sa}{0}
\verb{prosa}{ó}{}{}{}{}{Conversa informal, despreocupada.}{pro.sa}{0}
\verb{prosa}{ó}{}{}{}{adj.}{Diz"-se daquele que fala muito; conversador.}{pro.sa}{0}
\verb{prosador}{ô}{}{}{}{s.m.}{Escritor que faz obras em prosa.}{pro.sa.dor}{0}
\verb{prosaico}{}{}{}{}{adj.}{Relativo a prosa.}{pro.sai.co}{0}
\verb{prosaico}{}{}{}{}{}{Que não tem poesia; trivial, corriqueiro, vulgar.}{pro.sai.co}{0}
\verb{prosápia}{}{}{}{}{s.f.}{Linha de descendência; estirpe, linhagem.}{pro.sá.pia}{0}
\verb{prosápia}{}{}{}{}{}{Orgulho, vaidade, fanfarrice.}{pro.sá.pia}{0}
\verb{prosar}{}{}{}{}{v.i.}{Escrever em prosa.}{pro.sar}{\verboinum{1}}
\verb{proscênio}{}{}{}{}{s.m.}{A parte da frente do palco, junto à ribalta.}{pros.cê.nio}{0}
\verb{proscrever}{ê}{}{}{}{v.t.}{Decretar o exílio; banir, deportar, degredar.}{pros.cre.ver}{\verboinum{12}}
\verb{proscrição}{}{}{"-ões}{}{s.f.}{Ato ou efeito de proscrever; banimento, deportação, degredo.}{pros.cri.ção}{0}
\verb{proscrito}{}{}{}{}{adj.}{Que se proscreveu; exilado, deportado, degredado.}{pros.cri.to}{0}
\verb{proseador}{ô}{}{}{}{adj.}{Que fala muito; conversador.}{pro.se.a.dor}{0}
\verb{prosear}{}{}{}{}{v.i.}{Falar ou conversar muito.}{pro.se.ar}{\verboinum{4}}
\verb{proselitismo}{}{}{}{}{s.m.}{Atividade para formar prosélitos; catequese, apostolado.}{pro.se.li.tis.mo}{0}
\verb{prosélito}{}{}{}{}{s.m.}{Indivíduo que se converteu a uma religião; adepto, partidário.}{pro.sé.li.to}{0}
\verb{prosódia}{}{Gram.}{}{}{s.f.}{Pronúncia padrão das palavras, especialmente quanto à localização da sílaba tônica; ortoépia.}{pro.só.dia}{0}
\verb{prosódia}{}{}{}{}{}{Conjunto de características que acompanham os sons, como acentuação, duração e entonação.}{pro.só.dia}{0}
\verb{prosódico}{}{}{}{}{adj.}{Relativo a prosódia.}{pro.só.di.co}{0}
\verb{prosopopeia}{é}{Gram.}{}{}{s.f.}{Figura de linguagem que consiste em dar sentimentos humanos e palavras a seres inanimados, a animais etc.; personificação.}{pro.so.po.pei.a}{0}
\verb{prospecção}{}{}{"-ões}{}{s.f.}{Pesquisa de terreno por meio da qual se procuram jazidas minerais.}{pros.pec.ção}{0}
\verb{prospecto}{é}{}{}{}{s.m.}{Folheto promocional por meio do qual se faz propaganda de algo.}{pros.pec.to}{0}
\verb{prosperar}{}{}{}{}{v.t.}{Tornar próspero; desenvolver, progredir, enriquecer.}{pros.pe.rar}{\verboinum{1}}
\verb{prosperidade}{}{}{}{}{s.f.}{Qualidade do que é próspero, bem"-sucedido.}{pros.pe.ri.da.de}{0}
\verb{próspero}{}{}{}{}{adj.}{Que progride, desenvolve.}{prós.pe.ro}{0}
\verb{próspero}{}{}{}{}{}{Bem"-sucedido, afortunado, feliz.}{prós.pe.ro}{0}
\verb{prospeto}{}{}{}{}{}{Var. de \textit{prospecto}.}{pros.pe.to}{0}
\verb{prosseguimento}{}{}{}{}{s.m.}{Ato ou efeito de prosseguir.}{pros.se.gui.men.to}{0}
\verb{prosseguir}{}{}{}{}{v.i.}{Ir adiante; continuar.}{pros.se.guir}{\verboinum{24}}
\verb{próstata}{}{Anat.}{}{}{s.f.}{Glândula sexual masculina que produz o líquido espermático.}{prós.ta.ta}{0}
\verb{prosternar}{}{}{}{}{v.t.}{Lançar por terra; derrubar.}{pros.ter.nar}{0}
\verb{prosternar}{}{}{}{}{v.pron.}{Prostrar"-se, curvar"-se em sinal de respeito ou adoração.}{pros.ter.nar}{\verboinum{1}}
\verb{prostíbulo}{}{}{}{}{s.m.}{Casa de prostituição; bordel.}{pros.tí.bu.lo}{0}
\verb{prostituição}{}{}{"-ões}{}{s.f.}{Ato ou efeito de prostituir.}{pros.ti.tu.i.ção}{0}
\verb{prostituição}{}{}{"-ões}{}{}{A atividade de praticar relações sexuais comercialmente.}{pros.ti.tu.i.ção}{0}
\verb{prostituir}{}{}{}{}{v.t.}{Entregar à prática sexual mediante dinheiro.}{pros.ti.tu.ir}{0}
\verb{prostituir}{}{Fig.}{}{}{}{Corromper, degradar.}{pros.ti.tu.ir}{\verboinum{26}}
\verb{prostituta}{}{}{}{}{s.f.}{Mulher que pratica sexo comercialmente.}{pros.ti.tu.ta}{0}
\verb{prostituto}{}{}{}{}{s.m.}{Homem que pratica sexo comercialmente.}{pros.ti.tu.to}{0}
\verb{prostituto}{}{Fig.}{}{}{adj.}{Diz"-se de quem, por dinheiro, coloca"-se a serviço de entidades, ideologias ou atividades indignas ou contrárias aos seus próprios ideais.}{pros.ti.tu.to}{0}
\verb{prostração}{}{}{"-ões}{}{s.f.}{Ato ou efeito de prostrar.}{pros.tra.ção}{0}
\verb{prostração}{}{Fig.}{"-ões}{}{}{Submissão, sujeição.}{pros.tra.ção}{0}
\verb{prostração}{}{}{"-ões}{}{}{Abatimento, cansaço, fraqueza.}{pros.tra.ção}{0}
\verb{prostrar}{}{}{}{}{v.t.}{Lançar por terra; derrubar.}{pros.trar}{0}
\verb{prostrar}{}{Fig.}{}{}{}{Abater, extenuar, enfraquecer, humilhar.}{pros.trar}{0}
\verb{prostrar}{}{}{}{}{v.pron.}{Lançar"-se ao chão, curvar"-se em atitude de respeito.}{pros.trar}{\verboinum{1}}
\verb{protactínio}{}{Quím.}{}{}{s.m.}{Elemento químico metálico brilhante, radioativo, do grupo dos actinídeos, de obtenção muito difícil e sem aplicações diretas. \elemento{91}{231.03587}{Pa}.}{pro.tac.tí.nio}{0}
\verb{protagonista}{}{}{}{}{s.2g.}{Personagem principal de uma peça, história, filme, episódio.}{pro.ta.go.nis.ta}{0}
\verb{protagonizar}{}{Bras.}{}{}{v.t.}{Ser o personagem principal de.}{pro.ta.go.ni.zar}{\verboinum{1}}
\verb{proteção}{}{}{"-ões}{}{s.f.}{Ato ou efeito de proteger.}{pro.te.ção}{0}
\verb{proteção}{}{}{"-ões}{}{}{Aquilo que protege.}{pro.te.ção}{0}
\verb{protecionismo}{}{}{}{}{s.m.}{Política comercial ou fiscal para favorecer um grupo econômico, geralmente a indústria nacional em detrimento da estrangeira.}{pro.te.ci.o.nis.mo}{0}
\verb{proteger}{ê}{}{}{}{}{Socorrer, auxiliar.}{pro.te.ger}{0}
\verb{proteger}{ê}{}{}{}{v.t.}{Defender, abrigar.}{pro.te.ger}{0}
\verb{proteger}{ê}{}{}{}{}{Favorecer, beneficiar.}{pro.te.ger}{\verboinum{16}}
\verb{protegido}{}{}{}{}{adj.}{Que se protege ou que se protegeu.}{pro.te.gi.do}{0}
\verb{protegido}{}{}{}{}{}{Diz"-se de indivíduo que recebe proteção especial, benefício, favor.}{pro.te.gi.do}{0}
\verb{proteico}{é}{Quím.}{}{}{adj.}{Relativo a proteína.}{pro.tei.co}{0}
\verb{proteína}{}{Quím.}{}{}{s.f.}{Substância orgânica formada por aminoácidos, sendo o principal componente das células dos seres vivos.}{pro.te.í.na}{0}
\verb{protelação}{}{}{"-ões}{}{s.f.}{Ato ou efeito de protelar.}{pro.te.la.ção}{0}
\verb{protelar}{}{}{}{}{v.t.}{Adiar, retardar, prorrogar.}{pro.te.lar}{\verboinum{1}}
\verb{protervo}{}{}{}{}{adj.}{Insolente, descarado, petulante.}{pro.ter.vo}{0}
\verb{prótese}{}{Med.}{}{}{s.f.}{Elemento implantado no corpo para substituir um órgão ou tecido comprometido, perdido ou ausente.}{pró.te.se}{0}
\verb{prótese}{}{Gram.}{}{}{}{Inserção de um elemento fonético no início de um vocábulo sem alteração no significado ou função gramatical.}{pró.te.se}{0}
\verb{protestante}{}{}{}{}{adj.2g.}{Que protesta.}{pro.tes.tan.te}{0}
\verb{protestante}{}{Relig.}{}{}{}{Relativo ao protestantismo.}{pro.tes.tan.te}{0}
\verb{protestante}{}{Relig.}{}{}{}{Diz"-se de indivíduo seguidor do protestantismo.}{pro.tes.tan.te}{0}
\verb{protestantismo}{}{Relig.}{}{}{s.m.}{Conjunto de doutrinas religiosas surgidas na Reforma Protestante do século \textsc{xvi}.}{pro.tes.tan.tis.mo}{0}
\verb{protestar}{}{}{}{}{v.t.}{Manifestar reprovação; reclamar.}{pro.tes.tar}{0}
\verb{protestar}{}{}{}{}{}{Declarar formalmente.}{pro.tes.tar}{0}
\verb{protestar}{}{}{}{}{}{Prometer.}{pro.tes.tar}{0}
\verb{protestar}{}{Jur.}{}{}{}{Declarar o não pagamento de um título na data estabelecida.}{pro.tes.tar}{\verboinum{1}}
\verb{protesto}{é}{}{}{}{s.m.}{Ato ou efeito de protestar.}{pro.tes.to}{0}
\verb{protético}{}{}{}{}{adj.}{Relativo a prótese.}{pro.té.ti.co}{0}
\verb{protético}{}{}{}{}{s.m.}{Especialista em prótese dentária.}{pro.té.ti.co}{0}
\verb{protetor}{ô}{}{}{}{s.m.}{Indivíduo ou coisa que protege.}{pro.te.tor}{0}
\verb{protetorado}{}{}{}{}{s.m.}{Estado que se encontra subordinado a outro para questões políticas e que deste recebe proteção para suas instituições.}{pro.te.to.ra.do}{0}
\verb{protista}{}{Biol.}{}{}{adj.2g.}{Relativo ao reino Protista, que engloba os seres vivos unicelulares.}{pro.tis.ta}{0}
\verb{protista}{}{Biol.}{}{}{s.m.}{Designação comum a qualquer organismo pertencente a esse reino.}{pro.tis.ta}{0}
\verb{protocolar}{}{}{}{}{adj.2g.}{Relativo a protocolo.}{pro.to.co.lar}{0}
\verb{protocolo}{ó}{}{}{}{s.m.}{Registro de atos oficiais.}{pro.to.co.lo}{0}
\verb{protocolo}{ó}{}{}{}{}{Registro da correspondência oficial de uma instituição.}{pro.to.co.lo}{0}
\verb{protocolo}{ó}{}{}{}{}{Recibo em que constam informações sobre recebimento de correspondência e o seu registro no livro de protocolo.}{pro.to.co.lo}{0}
\verb{protocolo}{ó}{}{}{}{}{Conjunto de normas que regulam atos públicos; cerimonial.}{pro.to.co.lo}{0}
\verb{protocolo}{ó}{}{}{}{}{Norma rígida de procedimento; formalidade, etiqueta.}{pro.to.co.lo}{0}
\verb{protocolo}{ó}{}{}{}{}{Acordo entre duas nações.}{pro.to.co.lo}{0}
\verb{protocolo}{ó}{Informát.}{}{}{}{Conjunto de padrões que regulam a transmissão de dados entre máquinas ou dispositivos.}{pro.to.co.lo}{0}
\verb{protofonia}{}{Mús.}{}{}{s.f.}{Primeira parte de uma composição; abertura.}{pro.to.fo.ni.a}{0}
\verb{protomártir}{}{}{}{}{s.m.}{O primeiro entre os mártires de um grupo ou doutrina.}{pro.to.már.tir}{0}
\verb{próton}{}{Fís.}{}{}{s.m.}{Partícula com carga positiva e que é parte do núcleo do átomo.}{pró.ton}{0}
\verb{protoplasma}{}{Biol.}{}{}{s.m.}{O conteúdo de uma célula, formado pelo núcleo e o citoplasma.}{pro.to.plas.ma}{0}
\verb{protoplasmático}{}{}{}{}{adj.}{Relativo a protoplasma.}{pro.to.plas.má.ti.co}{0}
\verb{protoplásmico}{}{}{}{}{adj.}{Protoplasmático.}{pro.to.plás.mi.co}{0}
\verb{protótipo}{}{}{}{}{s.m.}{Modelo principal, do qual derivam todos os outros.}{pro.tó.ti.po}{0}
\verb{protótipo}{}{}{}{}{}{O exemplar perfeito de uma categoria ou estereótipo.}{pro.tó.ti.po}{0}
\verb{protótipo}{}{}{}{}{}{Produto industrial unitário para ser usado em testes.}{pro.tó.ti.po}{0}
\verb{protozoário}{}{Zool.}{}{}{adj.}{Relativo aos protozoários, animais formados por uma única célula.}{pro.to.zo.á.rio}{0}
\verb{protozoário}{}{Zool.}{}{}{s.m.}{Espécime dos protozoários.}{pro.to.zo.á.rio}{0}
\verb{protrair}{}{}{}{}{v.t.}{Tirar para fora.}{pro.tra.ir}{0}
\verb{protrair}{}{}{}{}{}{Adiar, prolongar.}{pro.tra.ir}{\verboinum{19}}
\verb{protuberância}{}{}{}{}{s.f.}{Parte mais alta ou que se destaca em uma superfície; saliência, proeminência.}{pro.tu.be.rân.cia}{0}
\verb{protuberante}{}{}{}{}{adj.2g.}{Saliente, proeminente.}{pro.tu.be.ran.te}{0}
\verb{prova}{ó}{}{}{}{s.f.}{Aquilo que demonstra uma afirmação; evidência.}{pro.va}{0}
\verb{prova}{ó}{}{}{}{}{Documento ou testemunho que contém essa demonstração.}{pro.va}{0}
\verb{prova}{ó}{}{}{}{}{Teste com a finalidade de avaliar conhecimentos; exame.}{pro.va}{0}
\verb{prova}{ó}{}{}{}{}{Competição esportiva.}{pro.va}{0}
\verb{prova}{ó}{}{}{}{}{Experimento científico; experiência.}{pro.va}{0}
\verb{provação}{}{}{"-ões}{}{s.f.}{Ato ou efeito de provar.}{pro.va.ção}{0}
\verb{provação}{}{}{"-ões}{}{}{Situação desfavorável ou que gera sofrimento; infortúnio, dificuldade.}{pro.va.ção}{0}
\verb{provador}{ô}{}{}{}{adj.}{Que prova ou serve para provar.}{pro.va.dor}{0}
\verb{provador}{ô}{}{}{}{}{Indivíduo encarregado de provar produtos ou degustar alimentos.}{pro.va.dor}{0}
\verb{provador}{ô}{Bras.}{}{}{}{Em lojas de roupas, cabine onde os fregueses podem vestir os produtos antes de comprá"-los.}{pro.va.dor}{0}
\verb{provar}{}{}{}{}{v.t.}{Demonstrar a verdade, a validade ou a autenticidade de.}{pro.var}{0}
\verb{provar}{}{}{}{}{}{Testemunhar.}{pro.var}{0}
\verb{provar}{}{}{}{}{}{Tornar evidente; mostrar.}{pro.var}{0}
\verb{provar}{}{}{}{}{}{Degustar, experimentar.}{pro.var}{0}
\verb{provar}{}{}{}{}{}{Submeter a provação.}{pro.var}{\verboinum{1}}
\verb{provável}{}{}{"-eis}{}{adj.2g.}{Que se pode provar.}{pro.vá.vel}{0}
\verb{provável}{}{}{"-eis}{}{}{Que tem chances favoráveis de acontecer.}{pro.vá.vel}{0}
\verb{provável}{}{}{"-eis}{}{}{Que tende a ser verdadeiro; verossímil.}{pro.vá.vel}{0}
\verb{provecto}{é}{}{}{}{adj.}{Experiente, versado, abalizado, conhecedor.}{pro.vec.to}{0}
\verb{provecto}{é}{Fig.}{}{}{}{De idade avançada; velho.}{pro.vec.to}{0}
\verb{provedor}{ô}{}{}{}{adj.}{Que provê.}{pro.ve.dor}{0}
\verb{provedor}{ô}{}{}{}{}{Chefe, diretor ou líder de certas instituições.}{pro.ve.dor}{0}
\verb{provedor}{ô}{Informát.}{}{}{s.m.}{Empresa que coloca à disposição dos usuários de computadores acesso à internet por meio de linhas telefônicas, cabo etc., cobrando ou não por esse serviço.  }{pro.ve.dor}{0}
\verb{provedoria}{}{}{}{}{s.f.}{Cargo ou jurisdição de provedor.}{pro.ve.do.ri.a}{0}
\verb{provedoria}{}{}{}{}{}{Gabinete de um provedor.}{pro.ve.do.ri.a}{0}
\verb{proveito}{ê}{}{}{}{s.m.}{Utilidade, serventia.}{pro.vei.to}{0}
\verb{proveito}{ê}{}{}{}{}{Vantagem, ganho, benefício, lucro.}{pro.vei.to}{0}
\verb{proveitoso}{ô}{}{"-osos ⟨ó⟩}{"-osa ⟨"-ó⟩}{adj.}{Que tem ou traz proveito.}{pro.vei.to.so}{0}
\verb{provençal}{}{}{"-ais}{}{adj.2g.}{Relativo a Provença, região sul da França.}{pro.ven.çal}{0}
\verb{provençal}{}{}{"-ais}{}{s.2g.}{Indivíduo natural ou habitante dessa região.}{pro.ven.çal}{0}
\verb{provençal}{}{}{"-ais}{}{}{Língua falada na antiga Provença.}{pro.ven.çal}{0}
\verb{proveniência}{}{}{}{}{s.f.}{Procedência, origem.}{pro.ve.ni.ên.cia}{0}
\verb{proveniente}{}{}{}{}{adj.2g.}{Oriundo, procedente.}{pro.ve.ni.en.te}{0}
\verb{provento}{}{}{}{}{s.m.}{Renda, ganho, lucro.}{pro.ven.to}{0}
\verb{prover}{ê}{}{}{}{v.t.}{Abastecer, fornecer, munir.}{pro.ver}{0}
\verb{prover}{ê}{}{}{}{}{Tomar providência a respeito de; providenciar.}{pro.ver}{0}
\verb{prover}{ê}{}{}{}{}{Nomear para cargo ou função.}{pro.ver}{\verboinum{47}}
\verb{proverbial}{}{}{"-ais}{}{adj.2g.}{Relativo a provérbio.}{pro.ver.bi.al}{0}
\verb{proverbial}{}{}{"-ais}{}{}{Conhecido, famoso.}{pro.ver.bi.al}{0}
\verb{provérbio}{}{}{}{}{s.m.}{Frase curta que sintetiza um conceito moral ou uma regra social; máxima, ditado.}{pro.vér.bio}{0}
\verb{proveta}{ê}{}{}{}{s.f.}{Vaso de vidro em forma de tubo cilíndrico, fechado em uma das extremidades, em que se fazem experiências em laboratório, dosagens, misturas etc.; tubo de ensaio.}{pro.ve.ta}{0}
\verb{providência}{}{}{}{}{s.f.}{Ação concreta para realização de algo.}{pro.vi.dên.cia}{0}
\verb{providência}{}{}{}{}{}{Ação divina.}{pro.vi.dên.cia}{0}
\verb{providência}{}{}{}{}{}{O próprio Deus.}{pro.vi.dên.cia}{0}
\verb{providencial}{}{}{"-ais}{}{adj.2g.}{Relativo a providência.}{pro.vi.den.ci.al}{0}
\verb{providencial}{}{}{"-ais}{}{}{Que parte da providência divina; inevitável.}{pro.vi.den.ci.al}{0}
\verb{providencial}{}{}{"-ais}{}{}{Que vem a calhar; oportuno, conveniente.}{pro.vi.den.ci.al}{0}
\verb{providenciar}{}{}{}{}{v.t.}{Dispor providentemente, com medidas adequadas; prover.}{pro.vi.den.ci.ar}{\verboinum{1}}
\verb{providente}{}{}{}{}{adj.2g.}{Relativo a providência; providencial.}{pro.vi.den.te}{0}
\verb{providente}{}{}{}{}{}{Que provê.}{pro.vi.den.te}{0}
\verb{providente}{}{}{}{}{}{Prudente, cauteloso, cuidadoso.}{pro.vi.den.te}{0}
\verb{provido}{}{}{}{}{adj.}{Que tem abundância do que é necessário.}{pro.vi.do}{0}
\verb{provido}{}{}{}{}{}{Que foi nomeado ou designado para cargo ou função pública.}{pro.vi.do}{0}
\verb{próvido}{}{}{}{}{adj.}{Providente.}{pró.vi.do}{0}
\verb{provimento}{}{}{}{}{s.m.}{Ato ou efeito de prover; abastecimento, sortimento.}{pro.vi.men.to}{0}
\verb{província}{}{}{}{}{s.f.}{Subdivisão de um país ou império.}{pro.vín.cia}{0}
\verb{província}{}{}{}{}{}{Região que fica mais afastada do governo central.}{pro.vín.cia}{0}
\verb{provincial}{}{}{"-ais}{}{adj.2g.}{Relativo a província.}{pro.vin.ci.al}{0}
\verb{provincial}{}{}{"-ais}{}{s.m.}{O superior de certo número de casas religiosas de uma determinada ordem que formam uma província.}{pro.vin.ci.al}{0}
\verb{provincianismo}{}{}{}{}{s.m.}{Maneira de ser ou costume próprio de uma província.}{pro.vin.ci.a.nis.mo}{0}
\verb{provincianismo}{}{}{}{}{}{Palavra ou locução, acento ou pronúncia característica de uma província.}{pro.vin.ci.a.nis.mo}{0}
\verb{provincianismo}{}{Por ext.}{}{}{}{Atraso, mau gosto.}{pro.vin.ci.a.nis.mo}{0}
\verb{provinciano}{}{}{}{}{adj.}{Relativo a província.}{pro.vin.ci.a.no}{0}
\verb{provinciano}{}{}{}{}{}{Que vive na província.}{pro.vin.ci.a.no}{0}
\verb{provindo}{}{}{}{}{adj.}{Que veio; procedente, derivado, oriundo. }{pro.vin.do}{0}
\verb{provir}{}{}{}{}{v.t.}{Ter origem; derivar, proceder.}{pro.vir}{0}
\verb{provir}{}{}{}{}{}{Proceder por geração; descender.}{pro.vir}{0}
\verb{provir}{}{}{}{}{}{Vir como consequência ou resultado; resultar, advir.}{pro.vir}{\verboinum{56}}
\verb{provisão}{}{}{"-ões}{}{s.f.}{Ato ou efeito de prover; abastecimento, fornecimento.}{pro.vi.são}{0}
\verb{provisão}{}{}{"-ões}{}{}{Reserva em dinheiro ou em valores.}{pro.vi.são}{0}
\verb{provisão}{}{}{"-ões}{}{}{Estoque.}{pro.vi.são}{0}
\verb{provisional}{}{}{"-ais}{}{adj.2g.}{Relativo a provisão.}{pro.vi.si.o.nal}{0}
\verb{provisional}{}{}{"-ais}{}{}{Que não é permanente; provisório.}{pro.vi.si.o.nal}{0}
\verb{provisório}{}{}{}{}{adj.}{Que não tem caráter de permanente; temporário.}{pro.vi.só.rio}{0}
\verb{provocação}{}{}{"-ões}{}{s.f.}{Ato ou efeito de provocar; ocasionamento, desencadeamento.}{pro.vo.ca.ção}{0}
\verb{provocação}{}{}{"-ões}{}{}{Afronta, ofensa.}{pro.vo.ca.ção}{0}
\verb{provocação}{}{}{"-ões}{}{}{Desafio.}{pro.vo.ca.ção}{0}
\verb{provocador}{ô}{}{}{}{adj.}{Que provoca; provocante.}{pro.vo.ca.dor}{0}
\verb{provocante}{}{}{}{}{adj.2g.}{Que provoca; provocador, causador.}{pro.vo.can.te}{0}
\verb{provocante}{}{}{}{}{}{Tentador, atraente.}{pro.vo.can.te}{0}
\verb{provocar}{}{}{}{}{v.t.}{Forçar alguém a responder a um desafio; desafiar.}{pro.vo.car}{0}
\verb{provocar}{}{}{}{}{}{Impelir alguém a algo; estimular, instigar.}{pro.vo.car}{0}
\verb{provocar}{}{}{}{}{}{Ser a causa de; causar, ocasionar.}{pro.vo.car}{0}
\verb{provocar}{}{}{}{}{}{Despertar desejo sexual.}{pro.vo.car}{\verboinum{2}}
\verb{proxeneta}{cs\ldots{}ê}{}{}{}{s.2g.}{Indivíduo que cobra para servir de intermediário em casos amorosos.}{pro.xe.ne.ta}{0}
\verb{proxeneta}{cs\ldots{}ê}{}{}{}{}{Indivíduo que explora a prostituição de outrem.}{pro.xe.ne.ta}{0}
\verb{proximidade}{s}{}{}{}{s.f.}{Condição do que é ou está próximo.}{pro.xi.mi.da.de}{0}
\verb{próximo}{s}{}{}{}{adj.}{Que fica perto; vizinho.}{pró.xi.mo}{0}
\verb{próximo}{s}{}{}{}{}{Que vai acontecer dentro de pouco tempo; iminente.}{pró.xi.mo}{0}
\verb{próximo}{s}{}{}{}{}{Que vem logo depois de outro; seguinte.}{pró.xi.mo}{0}
\verb{próximo}{s}{}{}{}{s.m.}{Cada ser humano; semelhante.}{pró.xi.mo}{0}
\verb{próximo}{s}{}{}{}{adv.}{Perto.}{pró.xi.mo}{0}
\verb{prudência}{}{}{}{}{s.f.}{Qualidade de ser prudente.}{pru.dên.cia}{0}
\verb{prudência}{}{}{}{}{}{Atitude de quem pensa no que pode acontecer; cautela, cuidado.}{pru.dên.cia}{0}
\verb{prudente}{}{}{}{}{adj.2g.}{Que tem prudência, que não procura o perigo; cauteloso, sensato.}{pru.den.te}{0}
\verb{prudente}{}{}{}{}{}{Que costuma precaver"-se; preparar"-se antecipadamente; precavido, previdente.}{pru.den.te}{0}
\verb{prumada}{}{}{}{}{s.f.}{Direção vertical determinada pela linha de prumo.}{pru.ma.da}{0}
\verb{prumada}{}{}{}{}{}{Profundidade em certo ponto de rio, mar etc.}{pru.ma.da}{0}
\verb{prumo}{}{}{}{}{s.m.}{Instrumento constituído de um corpo pesado preso a um fio usado para determinar a linha vertical.}{pru.mo}{0}
\verb{prumo}{}{}{}{}{}{Juízo, prudência.}{pru.mo}{0}
\verb{prurido}{}{}{}{}{s.m.}{Sensação incômoda na pele ou mucosa que leva a coçar; coceira.}{pru.ri.do}{0}
\verb{prurido}{}{Fig.}{}{}{}{Estado de hesitação ou dor na consciência; pudor, preocupação.}{pru.ri.do}{0}
\verb{prurido}{}{Fig.}{}{}{}{Desejo ardente; impaciência.}{pru.ri.do}{0}
\verb{pruriginoso}{ô}{}{"-osos ⟨ó⟩}{"-osa ⟨ó⟩}{adj.}{Relativo a prurido.}{pru.ri.gi.no.so}{0}
\verb{pruriginoso}{ô}{}{"-osos ⟨ó⟩}{"-osa ⟨ó⟩}{}{Que coça, que causa coceira ou comichão.}{pru.ri.gi.no.so}{0}
\verb{prussiano}{}{}{}{}{adj.}{Relativo à Prússia, antigo estado da confederação da Alemanha do Norte.}{prus.si.a.no}{0}
\verb{prussiano}{}{}{}{}{s.m.}{Indivíduo natural ou habitante da Prússia.}{prus.si.a.no}{0}
\verb{pseudofruto}{}{Bot.}{}{}{s.m.}{Falso fruto, parecido com uma baga, resultado do desenvolvimento de alguma outra parte da planta que não o ovário, que se torna comestível em muitos casos, como o do caju, em que o fruto verdadeiro é a castanha; fruto complexo.    }{pseu.do.fru.to}{0}
\verb{pseudônimo}{}{}{}{}{s.m.}{Nome falso adotado por um autor.}{pseu.dô.ni.mo}{0}
\verb{pseudópode}{}{Biol.}{}{}{s.m.}{Extensão fluida de citoplasma de seres unicelulares, utilizada para a locomoção e a alimentação.}{pseu.dó.po.de}{0}
\verb{psi}{}{}{}{}{s.m.}{Vigésima terceira letra do alfabeto grego, correspondente ao grupo \textit{PS} do latim e das línguas neolatinas.}{psi}{0}
\verb{psicanálise}{}{}{}{}{s.f.}{Tratamento que consiste em interpretar os processos psicológicos conscientes e inconscientes.}{psi.ca.ná.li.se}{0}
\verb{psicanalista}{}{}{}{}{s.2g.}{Indivíduo especialista em psicanálise.}{psi.ca.na.lis.ta}{0}
\verb{psicodélico}{}{}{}{}{adj.}{Que produz efeitos alucinógenos.}{psi.co.dé.li.co}{0}
\verb{psicodélico}{}{}{}{}{}{Relativo a esses efeitos.}{psi.co.dé.li.co}{0}
\verb{psicodrama}{}{}{}{}{s.m.}{Psicoterapia de grupo baseada na dramatização de situações de forte carga emocional.}{psi.co.dra.ma}{0}
\verb{psicografar}{}{}{}{}{v.t.}{Escrever o que é ditado por um espírito.}{psi.co.gra.far}{\verboinum{1}}
\verb{psicografia}{}{}{}{}{s.f.}{Escrita dos espíritos pela mão de um médium.}{psi.co.gra.fi.a}{0}
\verb{psicógrafo}{}{}{}{}{s.m.}{Médium que psicografa.}{psi.có.gra.fo}{0}
\verb{psicologia}{}{}{}{}{s.f.}{A ciência dos fenômenos psíquicos e do comportamento. }{psi.co.lo.gi.a}{0}
\verb{psicologia}{}{}{}{}{}{Modo de pensar e agir.}{psi.co.lo.gi.a}{0}
\verb{psicológico}{}{}{}{}{adj.}{Relativo à psicologia.}{psi.co.ló.gi.co}{0}
\verb{psicológico}{}{}{}{}{}{Que se refere à mente; psíquico.}{psi.co.ló.gi.co}{0}
\verb{psicólogo}{}{}{}{}{s.m.}{Indivíduo especialista em psicologia.}{psi.có.lo.go}{0}
\verb{psicopata}{}{}{}{}{adj.2g.}{Diz"-se do indivíduo que apresenta distúrbios mentais.}{psi.co.pa.ta}{0}
\verb{psicopatia}{}{}{}{}{s.f.}{Doença mental que gera comportamento antissocial.}{psi.co.pa.ti.a}{0}
\verb{psicopatologia}{}{Med.}{}{}{s.f.}{Estudo das doenças mentais no tocante a sua descrição, classificação, mecanismos de produção e evolução.}{psi.co.pa.to.lo.gi.a}{0}
\verb{psicose}{ó}{}{}{}{s.f.}{Distúrbio mental grave marcado pelo conflito com a realidade.}{psi.co.se}{0}
\verb{psicossomático}{}{}{}{}{adj.}{Que pertence ao mesmo tempo ao orgânico e ao psíquico.}{psi.cos.so.má.ti.co}{0}
\verb{psicotécnica}{}{}{}{}{s.f.}{Conjunto de métodos psicológicos aplicados a problemas sociológicos.}{psi.co.téc.ni.ca}{0}
\verb{psicotécnica}{}{}{}{}{}{Estudo do uso de conhecimentos psicológicos no domínio prático; orientação profissional etc.}{psi.co.téc.ni.ca}{0}
\verb{psicotécnico}{}{}{}{}{adj.}{Diz"-se de exame baseado na psicotécnica.}{psi.co.téc.ni.co}{0}
\verb{psicoterapeuta}{}{}{}{}{s.2g.}{Especialista em psicoterapia.}{psi.co.te.ra.peu.ta}{0}
\verb{psicoterapia}{}{}{}{}{s.f.}{Aplicação de técnicas de tratamento de doenças e problemas psíquicos.}{psi.co.te.ra.pi.a}{0}
\verb{psicótico}{}{}{}{}{adj.}{Relativo a psicose.}{psi.có.ti.co}{0}
\verb{psicótico}{}{}{}{}{}{Que sofre de psicose.}{psi.có.ti.co}{0}
\verb{psicotrópico}{}{Farm.}{}{}{adj.}{Diz"-se de droga que atua quimicamente sobre o psiquismo, o comportamento, a percepção etc.}{psi.co.tró.pi.co}{0}
\verb{psique}{}{}{}{}{s.f.}{A mente.}{psi.que}{0}
\verb{psique}{}{}{}{}{}{Estrutura mental ou psicológica de um indivíduo.}{psi.que}{0}
\verb{psiquiatra}{}{}{}{}{adj.2g.}{Especialista em psiquiatria.}{psi.qui.a.tra}{0}
\verb{psiquiatria}{}{Med.}{}{}{s.f.}{Ramo da medicina que se dedica ao estudo e tratamento das doenças mentais.}{psi.qui.a.tri.a}{0}
\verb{psiquiátrico}{}{}{}{}{adj.}{Relativo à psiquiatria.}{psi.qui.á.tri.co}{0}
\verb{psíquico}{}{}{}{}{adj.}{Relativo a psique.}{psí.qui.co}{0}
\verb{psíquico}{}{}{}{}{}{Relativo à esfera mental ou ao comportamental do indivíduo; psicológico, mental.}{psí.qui.co}{0}
\verb{psiquismo}{}{}{}{}{s.m.}{Conjunto das características psíquicas de um indivíduo.}{psi.quis.mo}{0}
\verb{psitacídeo}{}{Zool.}{}{}{s.m.}{Espécime dos psitacídeos, família de aves de distribuição mundial, caracterizadas pelo bico recurvado, de comprimento igual à altura, pelos dedos livres, e cujos representantes mais comuns são as araras, os papagaios e os periquitos.}{psi.ta.cí.deo}{0}
\verb{psitacismo}{}{}{}{}{s.m.}{Distúrbio de linguagem que consiste na repetição mecânica de palavras e frases sem saber o sentido delas.}{psi.ta.cis.mo}{0}
\verb{psiu}{}{}{}{}{interj.}{Palavra usada para chamar a atenção de alguém ou para pedir silêncio.}{psiu}{0}
\verb{psoríase}{}{Med.}{}{}{s.f.}{Doença de pele caracterizada pelo aparecimento de escamas brancacentas, cuja causa é desconhecida.}{pso.rí.a.se}{0}
\verb{Pt}{}{Quím.}{}{}{}{Símb. da \textit{platina}.}{Pt}{0}
\verb{pterodáctilo}{}{Paleo.}{}{}{s.m.}{Nome comum aos répteis voadores que viveram nos períodos Jurássico e Cretáceo, com bicos longos, dotados de dentes, há muito extintos, e que são conhecidos apenas por registros fósseis.}{pte.ro.dác.ti.lo}{0}
\verb{Pu}{}{Quím.}{}{}{}{Símb. do \textit{plutônio}.}{Pu}{0}
\verb{pua}{}{}{}{}{s.f.}{Ponta afiada; espinho, bico.}{pu.a}{0}
\verb{pua}{}{}{}{}{}{Utensílio para furar madeira, girando manualmente; broca, verruma.}{pu.a}{0}
\verb{pua}{}{}{}{}{}{Espora que se coloca nos galos de briga.}{pu.a}{0}
\verb{pua}{}{}{}{}{}{Espaço entre os dentes do pente do tear.}{pu.a}{0}
\verb{puberdade}{}{}{}{}{s.f.}{Fase da juventude em que se atinge a maturidade sexual.}{pu.ber.da.de}{0}
\verb{púbere}{}{}{}{}{adj.2g.}{Que está na fase da puberdade; sexualmente maduro, apto a procriar.}{pú.be.re}{0}
\verb{pubiano}{}{}{}{}{adj.}{Relativo ao púbis.}{pu.bi.a.no}{0}
\verb{púbico}{}{}{}{}{adj.}{Pubiano.}{pú.bi.co}{0}
\verb{púbis}{}{Anat.}{}{}{s.m.}{Parte anterior do osso ilíaco.}{pú.bis}{0}
\verb{púbis}{}{}{}{}{}{Região de forma triangular, na parte baixa do abdome, recoberta de pelos nos indivíduos adultos.}{pú.bis}{0}
\verb{publicação}{}{}{"-ões}{}{s.f.}{Ato ou efeito de publicar.}{pu.bli.ca.ção}{0}
\verb{publicação}{}{}{"-ões}{}{}{Qualquer obra publicada, periódica ou não.}{pu.bli.ca.ção}{0}
\verb{pública"-forma}{ó}{Jur.}{públicas"-formas ⟨ó⟩}{}{s.f.}{Cópia fiel de um documento, feita e reconhecida por tabelião, e que substitui o original na maioria dos casos.}{pú.bli.ca"-for.ma}{0}
\verb{publicar}{}{}{}{}{v.t.}{Tornar público; divulgar, propagar.}{pu.bli.car}{0}
\verb{publicar}{}{}{}{}{}{Preparar e imprimir uma obra; editar.}{pu.bli.car}{\verboinum{2}}
\verb{publicidade}{}{}{}{}{s.f.}{Qualidade de público.}{pu.bli.ci.da.de}{0}
\verb{publicidade}{}{}{}{}{}{Divulgação, propaganda.}{pu.bli.ci.da.de}{0}
\verb{publicitário}{}{}{}{}{adj.}{Relativo a publicidade.}{pu.bli.ci.tá.rio}{0}
\verb{publicitário}{}{}{}{}{s.m.}{Profissional que lida com publicidade.}{pu.bli.ci.tá.rio}{0}
\verb{público}{}{}{}{}{adj.}{De usufruto comum.}{pú.bli.co}{0}
\verb{público}{}{}{}{}{}{Pertencente ao Estado.}{pú.bli.co}{0}
\verb{público}{}{}{}{}{}{Conhecido pelas pessoas em geral.}{pú.bli.co}{0}
\verb{público}{}{}{}{}{s.m.}{O povo.}{pú.bli.co}{0}
\verb{público}{}{}{}{}{}{O conjunto dos espectadores; audiência.}{pú.bli.co}{0}
\verb{puçá}{}{}{}{}{s.f.}{Rede presa a um aro para capturar pequenos animais aquáticos.}{pu.çá}{0}
\verb{púcaro}{}{}{}{}{s.m.}{Pequeno vaso de barro com asa.}{pú.ca.ro}{0}
\verb{púcaro}{}{}{}{}{}{Caneca.}{pú.ca.ro}{0}
\verb{pudendo}{}{}{}{}{adj.}{Que tem pudor; pudico.}{pu.den.do}{0}
\verb{pudente}{}{}{}{}{adj.2g.}{Que tem pudor; pudico.}{pu.den.te}{0}
\verb{pudera}{é}{}{}{}{interj.}{Não era para menos; claro; óbvio.}{pu.de.ra}{0}
\verb{pudibundo}{}{}{}{}{adj.}{Que tem pudor; que se envergonha; pudico.}{pu.di.bun.do}{0}
\verb{pudicícia}{}{}{}{}{s.f.}{Qualidade de pudico; pudor.}{pu.di.cí.cia}{0}
\verb{pudicícia}{}{}{}{}{}{Castidade.}{pu.di.cí.cia}{0}
\verb{pudico}{}{}{}{}{adj.}{Que tem pudor; casto, recatado.}{pu.di.co}{0}
\verb{pudico}{}{}{}{}{}{Que tem timidez; acanhado, envergonhado, tímido.}{pu.di.co}{0}
\verb{pudim}{}{Cul.}{}{}{s.m.}{Prato doce ou salgado de consistência macia feito à base de leite e ovos, cozido ou assado.}{pu.dim}{0}
\verb{pudor}{ô}{}{}{}{s.m.}{Sentimento de vergonha em relação aos atos que possam ferir a moral, a honra, a honestidade.}{pu.dor}{0}
\verb{puericultura}{}{}{}{}{s.f.}{Conjunto de conhecimentos e técnicas para garantir o bom desenvolvimento físico, intelectual e moral das crianças.}{pu.e.ri.cul.tu.ra}{0}
\verb{pueril}{}{}{"-is}{}{adj.2g.}{Relativo a criança; infantil.}{pu.e.ril}{0}
\verb{pueril}{}{}{"-is}{}{}{Que denota imaturidade; infantil, tolo.}{pu.e.ril}{0}
\verb{puerilidade}{}{}{}{}{s.f.}{Qualidade ou condição de pueril; infância.}{pu.e.ri.li.da.de}{0}
\verb{puerilidade}{}{}{}{}{}{Ato pueril; infantilidade, imaturidade.}{pu.e.ri.li.da.de}{0}
\verb{puérpera}{}{Desus.}{}{}{s.f.}{Mulher que pariu há pouco tempo.}{pu.ér.pe.ra}{0}
\verb{puerperal}{}{}{"-ais}{}{adj.2g.}{Relativo a puérpera.}{pu.er.pe.ral}{0}
\verb{puerperal}{}{}{"-ais}{}{}{Relativo a parto.}{pu.er.pe.ral}{0}
\verb{puerpério}{}{Med.}{}{}{s.m.}{Período entre o parto e a recuperação total do estado geral da mulher.}{pu.er.pé.rio}{0}
\verb{pufe}{}{}{}{}{s.m.}{Assento baixo estofado, geralmente de forma circular.}{pu.fe}{0}
\verb{pufe}{}{}{}{}{}{Tipo de toucado volumoso.}{pu.fe}{0}
\verb{pufe}{}{}{}{}{}{Armação para dar volume à saia.}{pu.fe}{0}
\verb{pugilato}{}{}{}{}{s.m.}{Modalidade de luta com os punhos.}{pu.gi.la.to}{0}
\verb{pugilato}{}{Fig.}{}{}{}{Discussão acalorada.}{pu.gi.la.to}{0}
\verb{pugilismo}{}{Esport.}{}{}{s.m.}{Esporte de luta com os punhos, com regras específicas; boxe.}{pu.gi.lis.mo}{0}
\verb{pugilista}{}{}{}{}{s.2g.}{Praticante de pugilismo.}{pu.gi.lis.ta}{0}
\verb{pugilo}{}{}{}{}{s.m.}{Quantidade de qualquer coisa que se pode pegar com os dedos polegar, indicador e médio; pitada.}{pu.gi.lo}{0}
\verb{pugna}{}{}{}{}{s.f.}{Ato de pugnar; luta.}{pug.na}{0}
\verb{pugna}{}{}{}{}{}{Ato de discutir; debate, discussão.}{pug.na}{0}
\verb{pugna}{}{}{}{}{}{Ato de empenhar"-se; esforço, peleja.}{pug.na}{0}
\verb{pugnar}{}{}{}{}{v.t.}{Lutar, pelejar, combater.}{pug.nar}{\verboinum{1}}
\verb{pugnaz}{}{}{}{}{adj.2g.}{Valente, lutador.}{pug.naz}{0}
\verb{puir}{}{}{}{}{v.t.}{Desgastar pelo uso, pela fricção.}{pu.ir}{\verboinum{26}\verboirregular{\emph{def.} puem}}
\verb{pujança}{}{}{}{}{s.f.}{Vigor, robustez, força.}{pu.jan.ça}{0}
\verb{pujante}{}{}{}{}{adj.2g.}{Que tem pujança; forte, robusto, poderoso.}{pu.jan.te}{0}
\verb{pular}{}{}{}{}{v.i.}{Elevar"-se do chão dando impulso com as pernas; saltar.}{pu.lar}{0}
\verb{pular}{}{}{}{}{}{Palpitar, pulsar.}{pu.lar}{\verboinum{1}}
\verb{pulcro}{}{}{}{}{adj.}{Belo, formoso.}{pul.cro}{0}
\verb{pule}{}{Bras.}{}{}{s.f.}{Bilhete que registra uma aposta no turfe.}{pu.le}{0}
\verb{pulga}{}{Zool.}{}{}{s.f.}{Inseto hematófago, parasita de homens e animais.}{pul.ga}{0}
\verb{pulgão}{}{Zool.}{"-ões}{}{s.m.}{Inseto sem asas parasita das plantas.}{pul.gão}{0}
\verb{pulgueiro}{ê}{}{}{}{s.m.}{Lugar repleto de pulgas.}{pul.guei.ro}{0}
\verb{pulguento}{}{}{}{}{adj.}{Que tem muitas pulgas.}{pul.guen.to}{0}
\verb{pulha}{}{}{}{}{adj.2g.}{Relaxado, desmazelado.}{pu.lha}{0}
\verb{pulha}{}{}{}{}{}{Desprezível, miserável, indecente.}{pu.lha}{0}
\verb{pulha}{}{}{}{}{s.f.}{Gracejo, escarninho.}{pu.lha}{0}
\verb{pulha}{}{}{}{}{s.m.}{Indivíduo sem caráter, sem brio, desprezível; patife, calhorda.}{pu.lha}{0}
\verb{pulmão}{}{Anat.}{"-ões}{}{s.m.}{Cada um dos dois órgãos situados no tórax, nos quais se realiza a respiração.}{pul.mão}{0}
\verb{pulmonar}{}{}{}{}{adj.2g.}{Relativo aos pulmões.}{pul.mo.nar}{0}
\verb{pulo}{}{}{}{}{s.m.}{Ato ou efeito de pular; salto.}{pu.lo}{0}
\verb{pulo}{}{}{}{}{}{Ida rápida a um lugar. (\textit{Eu vou dar um pulo ao supermercado.})}{pu.lo}{0}
\verb{pulôver}{}{}{}{}{s.m.}{Agasalho de malha sem mangas que se veste sobre a camisa.  }{pu.lô.ver}{0}
\verb{púlpito}{}{}{}{}{s.m.}{Tribuna alta em uma das laterais da igreja de onde o padre faz o sermão aos fiéis.}{púl.pi.to}{0}
\verb{pulsação}{}{}{"-ões}{}{s.f.}{Ato ou efeito de pulsar; palpitação, latejamento.}{pul.sa.ção}{0}
\verb{pulsação}{}{Biol.}{"-ões}{}{}{Batimento ritmado no coração ou nas artérias.}{pul.sa.ção}{0}
\verb{pulsar}{}{}{}{}{v.i.}{Contrair"-se e dilatar"-se repetidamente fazendo o sangue correr em ondas;  bater.}{pul.sar}{0}
\verb{pulsar}{}{}{}{}{}{Latejar, palpitar.}{pul.sar}{\verboinum{1}}
\verb{pulseira}{ê}{}{}{}{s.f.}{Joia ou bijuteria que enfeita o pulso; bracelete.}{pul.sei.ra}{0}
\verb{pulso}{}{Biol.}{}{}{s.m.}{Batimento de uma artéria superficial; pulsação.}{pul.so}{0}
\verb{pulso}{}{Anat.}{}{}{}{Parte do antebraço que se junta à mão; punho.}{pul.so}{0}
\verb{pulso}{}{Fig.}{}{}{}{Firmeza, energia.}{pul.so}{0}
\verb{pululante}{}{}{}{}{adj.2g.}{Que pulula; fervilhante.}{pu.lu.lan.te}{0}
\verb{pulular}{}{}{}{}{v.i.}{Brotar e multiplicar"-se rapidamente e em grande número; irromper, surgir.}{pu.lu.lar}{0}
\verb{pulular}{}{}{}{}{}{Abundar, fervilhar, formigar.}{pu.lu.lar}{\verboinum{1}}
\verb{pulverização}{}{}{"-ões}{}{s.f.}{Ato ou efeito de pulverizar; borrifação.}{pul.ve.ri.za.ção}{0}
\verb{pulverizador}{ô}{}{}{}{adj.}{Diz"-se do instrumento que espalha um líquido, reduzido a minúsculas gotículas; borrifador.}{pul.ve.ri.za.dor}{0}
\verb{pulverizar}{}{Fig.}{}{}{}{Reduzir a pó; aniquilar, destruir.}{pul.ve.ri.zar}{\verboinum{1}}
\verb{pulverizar}{}{}{}{}{v.t.}{Borrifar líquido em minúsculas gotas; nebulizar.}{pul.ve.ri.zar}{0}
\verb{pulverulento}{}{}{}{}{adj.}{Coberto de pó; poeirento.}{pul.ve.ru.len.to}{0}
\verb{pum}{}{Pop.}{}{}{s.m.}{Saída de gases do intestino; peido.}{pum}{0}
\verb{pum}{}{}{}{}{interj.}{Expressão que indica estrondo ou detonação.}{pum}{0}
\verb{puma}{}{Zool.}{}{}{s.m.}{Grande felídeo americano de pelo amarelado; suçuarana.}{pu.ma}{0}
\verb{punção}{}{}{"-ões}{}{s.f.}{Ato ou efeito de pungir ou puncionar.}{pun.ção}{0}
\verb{puncionar}{}{}{}{}{v.t.}{Cortar ou abrir com bisturi.}{pun.ci.o.nar}{\verboinum{1}}
\verb{pundonor}{ô}{}{}{}{s.m.}{Sentimento de dignidade; amor"-próprio.}{pun.do.nor}{0}
\verb{punga}{}{}{}{}{adj.}{Diz"-se do cavalo que geralmente chega na última colocação no turfe.}{pun.ga}{0}
\verb{punga}{}{}{}{}{s.f.}{Furto praticado com rapidez e destreza.}{pun.ga}{0}
\verb{pungente}{}{}{}{}{adj.2g.}{Que causa muita dor; lancinante.}{pun.gen.te}{0}
\verb{pungir}{}{}{}{}{v.t.}{Picar, ferir com objeto pontudo; espicaçar.}{pun.gir}{\verboinum{34}}
\verb{punguear}{}{}{}{}{v.t.}{Furtar objetos das pessoas, geralmente em grandes aglomerações; bater carteira.}{pun.gue.ar}{\verboinum{4}}
\verb{punguista}{}{}{}{}{adj.2g.}{Diz"-se daquele que pungueia; batedor de carteira; trombadinha.}{pun.guis.ta}{0}
\verb{punhado}{}{}{}{}{s.m.}{Porção que cabe numa mão fechada.}{pu.nha.do}{0}
\verb{punhado}{}{}{}{}{}{Pequena quantidade.}{pu.nha.do}{0}
\verb{punhal}{}{}{"-ais}{}{s.m.}{Arma branca, constituída de cabo e lâmina aguda e penetrante, usada para cortar.}{pu.nhal}{0}
\verb{punhalada}{}{}{}{}{s.f.}{Golpe dado com punhal.}{pu.nha.la.da}{0}
\verb{punho}{}{}{}{}{s.m.}{Parte do antebraço que se junta à mão; pulso.}{pu.nho}{0}
\verb{punho}{}{}{}{}{}{Mão fechada.}{pu.nho}{0}
\verb{punho}{}{}{}{}{}{Tira de pano que fica na extremidade da manga longa de uma camisa e que envolve o pulso.}{pu.nho}{0}
\verb{punho}{}{}{}{}{}{Parte da espada em que se segura; cabo.}{pu.nho}{0}
\verb{punibilidade}{}{}{}{}{s.f.}{Característica do que pode ser punido.}{pu.ni.bi.li.da.de}{0}
\verb{punição}{}{}{"-ões}{}{s.f.}{Ato ou efeito de punir; castigo, pena.}{pu.ni.ção}{0}
\verb{púnico}{}{}{}{}{adj.}{Relativo a Cartago, antiga cidade do norte da África, ou a seus habitantes.}{pú.ni.co}{0}
\verb{punir}{}{}{}{}{v.t.}{Infligir pena a; castigar.}{pu.nir}{\verboinum{18}}
\verb{punitivo}{}{}{}{}{adj.}{Relativo a punição.}{pu.ni.ti.vo}{0}
\verb{punitivo}{}{}{}{}{}{Que pune ou é próprio para punir.}{pu.ni.ti.vo}{0}
\verb{punk}{}{}{}{}{s.m.}{Movimento surgido na Inglaterra na década de 1960 de caráter contestador, reunindo jovens que exibem vários sinais exteriores como cortes de cabelos, roupas etc. provocando a ordem social vigente e escarnecendo dela. }{\textit{punk}}{0}
\verb{punk}{}{}{}{}{}{Indivíduo seguidor desse movimento.}{\textit{punk}}{0}
\verb{punk}{}{}{}{}{adj.2g.}{Relativo a esse movimento.}{\textit{punk}}{0}
\verb{pupila}{}{Anat.}{}{}{s.f.}{Abertura situada no centro da íris que, por ser retrátil, permite regular a quantidade de luminosidade que penetra no olho.}{pu.pi.la}{0}
\verb{pupilo}{}{}{}{}{s.m.}{Menor de idade sob a tutela ou os cuidados de alguém.}{pu.pi.lo}{0}
\verb{pupunha}{}{}{}{}{s.f.}{Fruto comestível da pupunheira.}{pu.pu.nha}{0}
\verb{pupunheira}{ê}{Bot.}{}{}{s.f.}{Palmeira alta e espinhosa da qual se extrai um palmito muito apreciado.}{pu.pu.nhei.ra}{0}
\verb{purê}{}{Cul.}{}{}{s.m.}{Prato feito com batatas ou legumes espremidos ou passados em peneira.}{pu.rê}{0}
\verb{pureza}{ê}{}{}{}{s.f.}{Qualidade do que é puro, do que não tem mistura.}{pu.re.za}{0}
\verb{pureza}{ê}{}{}{}{}{Inocência, castidade.}{pu.re.za}{0}
\verb{purga}{}{}{}{}{s.f.}{Purgante, laxante.}{pur.ga}{0}
\verb{purgação}{}{}{"-ões}{}{s.f.}{Ato ou efeito de purgar.}{pur.ga.ção}{0}
\verb{purgante}{}{Farm.}{}{}{adj.2g.}{Diz"-se do medicamento que faz purgar, evacuar; laxante.}{pur.gan.te}{0}
\verb{purgar}{}{}{}{}{v.t.}{Livrar de impurezas; purificar, depurar.}{pur.gar}{0}
\verb{purgar}{}{}{}{}{}{Expiar os pecados; redimir.}{pur.gar}{\verboinum{5}}
\verb{purgativo}{}{}{}{}{adj.}{Que purga, limpa, purifica.}{pur.ga.ti.vo}{0}
\verb{purgatório}{}{Relig.}{}{}{s.m.}{Local para onde  as almas que cometeram pecados leves vão a fim de se purificarem antes de ir ao céu.}{pur.ga.tó.rio}{0}
\verb{purgatório}{}{}{}{}{}{Sofrimento, expiação.}{pur.ga.tó.rio}{0}
\verb{purificação}{}{}{"-ões}{}{s.f.}{Ato ou efeito de purificar; purgação.}{pu.ri.fi.ca.ção}{0}
\verb{purificador}{ô}{}{}{}{adj.}{Que purifica, limpa.}{pu.ri.fi.ca.dor}{0}
\verb{purificar}{}{}{}{}{v.t.}{Tornar puro; depurar.}{pu.ri.fi.car}{0}
\verb{purificar}{}{}{}{}{}{Livrar de pecados; expiar.}{pu.ri.fi.car}{\verboinum{2}}
\verb{purismo}{}{}{}{}{s.m.}{Zelo excessivo com a pureza e tradição da língua.}{pu.ris.mo}{0}
\verb{purista}{}{}{}{}{adj.2g.}{Relativo ao purismo.}{pu.ris.ta}{0}
\verb{purista}{}{}{}{}{s.2g.}{Indivíduo que se preocupa demasiadamente com a pureza da língua.}{pu.ris.ta}{0}
\verb{puritanismo}{}{Relig.}{}{}{s.m.}{Ramo do Protestantismo que se desligou da Igreja da Inglaterra nos séculos \textsc{xvi} e \textsc{xvii} e cujos adeptos desejavam praticar um Cristianismo puro e próximo à letra das Escrituras.}{pu.ri.ta.nis.mo}{0}
\verb{puritano}{}{}{}{}{adj.}{Relativo ao puritanismo.}{pu.ri.ta.no}{0}
\verb{puritano}{}{}{}{}{}{Adepto dessa religião.}{pu.ri.ta.no}{0}
\verb{puritano}{}{}{}{}{}{Diz"-se daquele que é rígido nos costumes; moralista.}{pu.ri.ta.no}{0}
\verb{puro}{}{}{}{}{adj.}{Sem mistura com outra substância. (\textit{O ar puro da montanha é excelente para a cura de doenças pulmonares.})}{pu.ro}{0}
\verb{puro}{}{}{}{}{}{Não contaminado; imaculado, limpo.}{pu.ro}{0}
\verb{puro}{}{}{}{}{}{Casto, virgem, inocente.}{pu.ro}{0}
\verb{puro"-sangue}{}{}{puros"-sangues}{}{adj.2g.}{Diz"-se do cavalo que não descende de cruzamento com outras raças, que pertence a uma raça pura.}{pu.ro"-san.gue}{0}
\verb{púrpura}{}{}{}{}{s.f.}{Tinta corante extraída da cochonilha.}{púr.pu.ra}{0}
\verb{púrpura}{}{}{}{}{}{Cor vermelho"-escura.}{púr.pu.ra}{0}
\verb{purpúreo}{}{}{}{}{adj.}{Da cor da púrpura; purpurino.}{pur.pú.re.o}{0}
\verb{purpurina}{}{}{}{}{s.f.}{Pó metálico brilhante usado em maquiagem ou enfeite de roupas.}{pur.pu.ri.na}{0}
\verb{purpurino}{}{}{}{}{adj.}{Purpúreo.}{pur.pu.ri.no}{0}
\verb{purulência}{}{}{}{}{s.f.}{Quantidade de pus; supuração.}{pu.ru.lên.cia}{0}
\verb{purulento}{}{}{}{}{adj.}{Cheio de pus; infeccionado.}{pu.ru.len.to}{0}
\verb{pururuca}{}{Cul.}{}{}{s.f.}{Pele do porco preparada como alimento, bem tostada.}{pu.ru.ru.ca}{0}
\verb{pus}{}{Med.}{}{}{s.m.}{Líquido que se forma em feridas infeccionadas, composto de glóbulos brancos e bactérias, vivos e mortos, e células dos tecidos afetados.}{pus}{0}
\verb{pusilânime}{}{}{}{}{adj.2g.}{Fraco, medroso, covarde.}{pu.si.lâ.ni.me}{0}
\verb{pusilanimidade}{}{}{}{}{s.f.}{Qualidade de pusilânime.}{pu.si.la.ni.mi.da.de}{0}
\verb{pústula}{}{Med.}{}{}{s.f.}{Erupção cutânea com pus.}{pús.tu.la}{0}
\verb{pustulento}{}{}{}{}{adj.}{Cheio de pústulas.}{pus.tu.len.to}{0}
\verb{putativo}{}{}{}{}{adj.}{Suposto, considerado como verdadeiro.}{pu.ta.ti.vo}{0}
\verb{putrefação}{}{}{"-ões}{}{s.f.}{Ato ou efeito de putrefazer; processo de decomposição de matéria orgânica.}{pu.tre.fa.ção}{0}
\verb{putrefacto}{}{}{}{}{}{Var. de \textit{putrefato}.}{pu.tre.fac.to}{0}
\verb{putrefato}{}{}{}{}{adj.}{Que apodreceu; podre, decomposto, apodrecido.}{pu.tre.fa.to}{0}
\verb{putrefazer}{}{}{}{}{v.t.}{Decompor, deteriorar.}{pu.tre.fa.zer}{0}
\verb{putrefazer}{}{Fig.}{}{}{}{Corromper, deteriorar moralmente.}{pu.tre.fa.zer}{0}
\verb{putrefazerse}{}{}{}{}{v.pron.}{Apodrecer, decompor"-se.}{pu.tre.fa.zer}{\verboinum{42}}
\verb{putrescível}{}{}{"-eis}{}{adj.2g.}{Que pode apodrecer.}{pu.tres.cí.vel}{0}
\verb{pútrido}{}{}{}{}{adj.}{Que se decompôs; apodrecido, podre.}{pú.tri.do}{0}
\verb{pútrido}{}{}{}{}{}{Fétido, pestilento, fedido.}{pú.tri.do}{0}
\verb{puxa}{ch}{}{}{}{interj.}{Expressão que indica espanto, surpresa, frustração ou raiva.}{pu.xa}{0}
\verb{puxada}{ch}{}{}{}{s.f.}{Ato ou efeito de puxar; puxão.}{pu.xa.da}{0}
\verb{puxado}{ch}{}{}{}{adj.}{Que se puxou; esticado.}{pu.xa.do}{0}
\verb{puxado}{ch}{}{}{}{}{Diz"-se de trabalho cansativo, intenso ou muito longo.}{pu.xa.do}{0}
\verb{puxado}{ch}{}{}{}{}{Diz"-se de preço elevado ou acima das possibilidades.}{pu.xa.do}{0}
\verb{puxador}{ch\ldots{}ô}{}{}{}{s.m.}{Peça de qualquer material pela qual se abrem portas de armários ou gavetas.}{pu.xa.dor}{0}
\verb{puxador}{ch\ldots{}ô}{}{}{}{}{Indivíduo que lidera o canto ou a reza quando feitos em grupo.}{pu.xa.dor}{0}
\verb{puxão}{ch}{}{"-ões}{}{s.m.}{Ato ou efeito de puxar, especialmente de maneira brusca e forte.}{pu.xão}{0}
\verb{puxa"-puxa}{ch\ldots{}ch}{Bras.}{puxas"-puxas \textit{ou} puxa"-puxas}{}{adj.}{De consistência grudenta ou elástica.}{pu.xa"-pu.xa}{0}
\verb{puxar}{ch}{}{}{}{v.t.}{Fazer mover na direção de si mesmo.}{pu.xar}{0}
\verb{puxar}{ch}{}{}{}{}{Deslocar"-se arrastando algo atrás de si; tracionar.}{pu.xar}{0}
\verb{puxar}{ch}{}{}{}{}{Tracionar, esticar, estirar.}{pu.xar}{0}
\verb{puxar}{ch}{}{}{}{}{Tirar com esforço ou movimento brusco; arrancar.}{pu.xar}{0}
\verb{puxar}{ch}{}{}{}{}{Herdar características de.}{pu.xar}{0}
\verb{puxar}{ch}{}{}{}{}{Provocar conversa, discussão; instigar, estimular.}{pu.xar}{0}
\verb{puxar}{ch}{}{}{}{}{Liderar canto, reza, fila.}{pu.xar}{0}
\verb{puxar}{ch}{}{}{}{}{Demandar esforço, despesa; exigir, consumir.}{pu.xar}{\verboinum{1}}
\verb{puxa"-saco}{ch}{Bras.}{puxa"-sacos}{}{adj.}{Diz"-se de indivíduo bajulador.}{pu.xa"-sa.co}{0}
\verb{puxo}{ch}{}{}{}{s.m.}{Dor no ânus que acompanha evacuação difícil.}{pu.xo}{0}
