\verb{z}{}{}{}{}{s.m.}{Vigésima sexta e última letra do alfabeto português.}{z}{0}
\verb{zabelê}{}{Zool.}{}{}{s.2g.}{Nome comum a várias aves de penas coloridas, conhecidas pelo canto triste que entoam quando anoitece; jaó.}{za.be.lê}{0}
\verb{zabumba}{}{Mús.}{}{}{s.2g.}{Tambor grande, de sonoridade grave; bumbo.}{za.bum.ba}{0}
\verb{zabumba}{}{}{}{}{}{Que toca zabumba.}{za.bum.ba}{0}
\verb{zaburro}{}{}{}{}{}{Diz"-se de variedade de milho de grão avermelhado de certas regiões de Portugal.}{za.bur.ro}{0}
\verb{zaburro}{}{}{}{}{adj.}{Diz"-se de uma variedade de milho indiano}{za.bur.ro}{0}
\verb{zaga}{}{Bot.}{}{}{s.f.}{Espécie de palmeira com que se fazem azagaias. }{za.ga}{0}
\verb{zaga}{}{Esport.}{}{}{}{Posição recuada e defensiva ocupada pelos zagueiros de um time de futebol.}{za.ga}{0}
\verb{zaga}{}{Esport.}{}{}{}{Os dois beques.}{za.ga}{0}
\verb{zagaia}{}{}{}{}{s.f.}{Lança curta de arremesso; azagaia.}{za.gai.a}{0}
\verb{zagal}{}{}{"-ais}{zagala}{s.m.}{Pastor, pegureiro.}{za.gal}{0}
\verb{zagueiro}{ê}{Esport.}{}{}{s.m.}{No futebol, jogador de defesa que ocupa a zaga; beque.}{za.guei.ro}{0}
\verb{zaino}{}{}{}{}{adj.}{Diz"-se de cavalo que tem o pelo castanho"-escuro.}{zai.no}{0}
\verb{zairense}{}{}{}{}{adj.2g.}{Relativo ou pertencente ao antigo Zaire, atual República Democrática do Congo.}{zai.ren.se}{0}
\verb{zairense}{}{}{}{}{s.2g.}{Natural ou habitante desse país. }{zai.ren.se}{0}
\verb{zambi}{}{}{}{}{}{Var. de \textit{zumbi}.}{zam.bi}{0}
\verb{zambiano}{}{}{}{}{adj.}{Relativo à Zâmbia.}{zam.bi.a.no}{0}
\verb{zambiano}{}{}{}{}{s.m.}{Indivíduo natural ou habitante desse país. }{zam.bi.a.no}{0}
\verb{zambo}{}{}{}{}{adj.}{Diz"-se de indivíduo que tem pés ou pernas tortos; cambaio.}{zam.bo}{0}
\verb{zambo}{}{Bras.}{}{}{}{Diz"-se de mestiço que descende de negro e indígena.}{zam.bo}{0}
\verb{zambro}{}{}{}{}{adj.}{Cambaio, zambo.}{zam.bro}{0}
\verb{zanga}{}{}{}{}{s.f.}{Aborrecimento, irritação, ira.}{zan.ga}{0}
\verb{zangado}{}{}{}{}{adj.}{Que se zangou; aborrecido, encolerizado, irritado.}{zan.ga.do}{0}
\verb{zangão}{}{}{"-ões}{}{s.m.}{Macho de diversas espécies de abelhas sociais, de tamanho maior que o das abelhas operárias, desprovido de ferrão, que não fabrica mel, e cujo papel na colmeia se restringe à reprodução.}{zan.gão}{0}
\verb{zângão}{}{}{}{}{}{Var. de \textit{zangão}.}{zân.gão}{0}
\verb{zangar}{}{}{}{}{v.t.}{Aborrecer, irar, molestar, afligir.}{zan.gar}{\verboinum{5}}
\verb{zanzar}{}{Bras.}{}{}{v.i.}{Andar a esmo, sem destino; vaguear, perambular.}{zan.zar}{\verboinum{1}}
\verb{zapear}{}{}{}{}{v.i.}{Passar ao acaso de uma emissora de televisão a outra, especialmente por meio de um controle remoto.}{za.pe.ar}{\verboinum{4}}
\verb{zarabatana}{}{}{}{}{s.f.}{Tubo comprido com o qual se atiram, soprando com força, pequenas setas, pedrinhas, grãos, entre outros.}{za.ra.ba.ta.na}{0}
\verb{zaragata}{}{Pop.}{}{}{s.f.}{Confusão, tumulto, algazarra, desordem.}{za.ra.ga.ta}{0}
\verb{zarcão}{}{}{"-ões}{}{s.m.}{Tinta anticorrosiva à base de óxido de chumbo.}{zar.cão}{0}
\verb{zarcão}{}{}{"-ões}{}{adj.2g.}{Que tem a cor do zarcão, laranja bem vivo.}{zar.cão}{0}
\verb{zarolho}{ô}{}{}{}{adj.}{Diz"-se daquele que não tem um olho, ou é cego de um olho; caolho.}{za.ro.lho}{0}
\verb{zarpar}{}{}{}{}{v.i.}{Levantar âncora.}{zar.par}{0}
\verb{zarpar}{}{Por ext.}{}{}{}{Partir, fugir.}{zar.par}{\verboinum{1}}
\verb{zarzuela}{é}{Art.}{}{}{s.f.}{Ópera"-cômica espanhola, com canções e peças instrumentais entremeadas por diálogos.}{zar.zu.e.la}{0}
\verb{zás}{}{}{}{}{interj.}{Expressão que representa ou imita pancada rápida, ou ação rápida e decidida; zás"-trás.  }{zás}{0}
\verb{zás"-trás}{}{}{}{}{interj.}{Zás.}{zás"-trás}{0}
\verb{zê}{}{}{}{}{s.m.}{Nome da letra \textit{z}.}{zê}{0}
\verb{zebra}{ê}{Zool.}{}{}{s.f.}{Nome comum a certos equídeos africanos, herbívoros, de pelos brancos com listras pretas e crina curta.}{ze.bra}{0}
\verb{zebra}{ê}{Bras.}{}{}{}{Indivíduo estúpido, burro.}{ze.bra}{0}
\verb{zebra}{ê}{}{}{}{}{Resultado inesperado, contrário às expectativas; azarão.}{ze.bra}{0}
\verb{zebrado}{}{}{}{}{adj.}{Que apresenta listras como as da zebra.}{ze.bra.do}{0}
\verb{zebrar}{}{}{}{}{v.t.}{Listrar, dando aparência de pelo de zebra.}{ze.brar}{\verboinum{1}}
\verb{zebrino}{}{}{}{}{adj.}{Relativo a ou próprio da zebra.}{ze.bri.no}{0}
\verb{zebroide}{ó}{}{}{}{adj.2g.}{Semelhante a zebra.}{ze.broi.de}{0}
\verb{zebroide}{ó}{}{}{}{s.2g.}{Híbrido de cavalo e fêmea de zebra.}{ze.broi.de}{0}
\verb{zebruno}{}{}{}{}{adj.}{Diz"-se de cavalo baio.}{ze.bru.no}{0}
\verb{zebu}{}{Zool.}{}{}{s.m.}{Tipo de boi, originário da Índia, que tem chifres pequenos e uma corcova grande. }{ze.bu}{0}
\verb{zebueiro}{ê}{Bras.}{}{}{s.m.}{Criador ou negociante de gado zebu; zebuzeiro.}{ze.bu.ei.ro}{0}
\verb{zebuzeiro}{ê}{}{}{}{}{Var. de \textit{zebueiro}.}{ze.bu.zei.ro}{0}
\verb{zefir}{}{}{}{}{s.m.}{Certo tecido fino e transparente de algodão.}{ze.fir}{0}
\verb{zéfiro}{}{}{}{}{s.m.}{Vento que sopra do ocidente.}{zé.fi.ro}{0}
\verb{zéfiro}{}{Por ext.}{}{}{}{Vento suave e agradável; brisa, aragem.}{zé.fi.ro}{0}
\verb{zéfiro}{}{Mit.}{}{}{}{Entre os antigos, personificação mitológica do vento do ocidente.}{zé.fi.ro}{0}
\verb{zelador}{ô}{}{}{}{adj.}{Que zela.}{ze.la.dor}{0}
\verb{zelador}{ô}{}{}{}{s.m.}{Indivíduo encarregado de cuidar de um prédio de apartamentos ou de escritórios.}{ze.la.dor}{0}
\verb{zelar}{}{}{}{}{v.t.}{Ter zelo por pessoa ou coisa; cuidar de.}{ze.lar}{0}
\verb{zelar}{}{}{}{}{}{Vigiar, defender com cuidado e interesse.}{ze.lar}{0}
\verb{zelar}{}{}{}{}{}{Velar, interessar"-se por.}{ze.lar}{\verboinum{1}}
\verb{zelo}{ê}{}{}{}{s.m.}{Dedicação, cuidado com o que se faz.}{ze.lo}{0}
\verb{zelo}{ê}{}{}{}{}{Desvelo, empenho na execução de uma tarefa.}{ze.lo}{0}
\verb{zeloso}{ô}{}{"-osos ⟨ó⟩}{"-osa ⟨ó⟩}{adj.}{Que tem zelo; cuidadoso, dedicado, diligente.}{ze.lo.so}{0}
\verb{zelote}{ó}{Desus.}{}{}{adj.}{Que finge ter zelos.}{ze.lo.te}{0}
\verb{zelote}{ó}{}{}{}{s.m.}{Membro de seita e partido político judaico, da época de Cristo, que se opunha à dominação romana.}{ze.lo.te}{0}
\verb{zen}{}{Relig.}{}{}{s.m.}{Zen"-budismo.}{zen}{0}
\verb{zen}{}{}{}{}{adj.}{Relativo ao zen ou próprio dele; zen"-budista.}{zen}{0}
\verb{zen"-budismo}{}{Relig.}{}{}{s.m.}{Escola do budismo surgida na China do século \textsc{vi}, disseminada posteriormente no Japão, caracterizada pela busca de autoconhecimento por meio da prática da meditação; zen.}{zen"-bu.dis.mo}{0}
\verb{zen"-budista}{}{}{zen"-budistas}{}{adj.2g.}{Relativo ao zen"-budismo ou próprio dele.}{zen"-bu.dis.ta}{0}
\verb{zen"-budista}{}{}{zen"-budistas}{}{s.2g.}{Adepto do zen"-budismo.}{zen"-bu.dis.ta}{0}
\verb{zé"-ninguém}{}{}{zés"-ninguéns \textit{ou} zés"-ninguém}{}{s.m.}{Indivíduo sem importância; joão"-ninguém.}{zé"-nin.guém}{0}
\verb{zênite}{}{Astron.}{}{}{s.m.}{Ponto da esfera celeste situado na vertical do observador, exatamente acima da sua cabeça.}{zê.ni.te}{0}
\verb{zênite}{}{Fig.}{}{}{}{Auge, ápice, apogeu.}{zê.ni.te}{0}
\verb{zepelim}{}{}{"-ins}{}{s.m.}{Grande balão dirigível, de forma alongada, construído pelos alemães em 1900 e usado até fins da década de 1930 para carregar passageiros.}{ze.pe.lim}{0}
\verb{zé"-pereira}{ê}{}{zé"-pereiras}{}{s.m.}{Certo ritmo carnavalesco executado no zabumba.}{zé"-pe.rei.ra}{0}
\verb{zé"-pereira}{ê}{}{zé"-pereiras}{}{}{Grupo de foliões que tocam esse ritmo. }{zé"-pe.rei.ra}{0}
\verb{zé"-povinho}{}{Pop.}{}{}{s.m.}{Homem simples, do povo, comum.}{zé"-po.vi.nho}{0}
\verb{zé"-povinho}{}{Pop.}{}{}{}{A camada mais baixa da sociedade; ralé, gentalha.}{zé"-po.vi.nho}{0}
\verb{zé"-povo}{ô}{}{zé"-povos ⟨ó⟩}{}{s.m.}{Zé"-povinho.}{zé"-po.vo}{0}
\verb{zé"-pregos}{é}{Zool.}{}{}{s.m.}{Macho da tartaruga"-da"-amazônia.}{zé"-pre.gos}{0}
\verb{zerar}{}{}{}{}{v.t.}{Reduzir a zero.}{ze.rar}{0}
\verb{zerar}{}{}{}{}{}{Dar ou tirar nota zero.}{ze.rar}{0}
\verb{zerar}{}{}{}{}{}{Saldar, liquidar.}{ze.rar}{0}
\verb{zerar}{}{Pop.}{}{}{v.i.}{Ficar sem dinheiro.}{ze.rar}{\verboinum{1}}
\verb{zero}{é}{}{}{}{num.}{Algarismo sem valor absoluto, mas que colocado à direita de outro algarismo lhe aumenta dez vezes o valor.}{ze.ro}{0}
\verb{zero}{é}{}{}{}{}{Nenhuma quantidade; nenhum, nada.}{ze.ro}{0}
\verb{zero}{é}{}{}{}{s.m.}{Algarismo que representa a ausência de quantidade.}{ze.ro}{0}
\verb{zero"-quilômetro}{é}{}{}{}{adj.}{Diz"-se de veículo novo, que não foi rodado ainda.}{ze.ro"-qui.lô.me.tro}{0}
\verb{zero"-quilômetro}{é}{Por ext.}{}{}{}{Diz"-se de qualquer coisa, especialmente máquina ou aparelho, ainda não usada; novo.}{ze.ro"-qui.lô.me.tro}{0}
\verb{zero"-quilômetro}{é}{Bras.}{}{}{s.m.}{Automóvel novo.}{ze.ro"-qui.lô.me.tro}{0}
\verb{zeugma}{ê}{Gram.}{}{}{s.2g.}{Espécie de elipse que consiste na supressão de um termo do período que pode ser subentendido em outra parte, posterior ou anterior àquela. }{zeug.ma}{0}
\verb{zibelina}{}{Zool.}{}{}{s.f.}{Variedade de marta encontrada nas regiões frias do norte da Europa e da Ásia, cuja pele, marrom"-escura, é muito valorizada.}{zi.be.li.na}{0}
\verb{zibelina}{}{Por ext.}{}{}{}{A pele da zibelina.}{zi.be.li.na}{0}
\verb{zigoma}{ô}{Anat.}{}{}{s.m.}{Osso saliente da maçã do rosto; osso zigomático.}{zi.go.ma}{0}
\verb{zigoto}{ô}{Biol.}{}{}{s.m.}{Célula reprodutora resultante da união dos gametas masculino e feminino.}{zi.go.to}{0}
\verb{ziguezague}{}{}{}{}{s.m.}{Linha sinuosa ou quebrada, que forma ângulos salientes e reentrantes.}{zi.gue.za.gue}{0}
\verb{ziguezaguear}{}{}{}{}{v.i.}{Fazer ziguezagues.}{zi.gue.za.gue.ar}{0}
\verb{ziguezaguear}{}{}{}{}{}{Andar em ziguezague.}{zi.gue.za.gue.ar}{\verboinum{4}}
\verb{zimbabuano}{}{}{}{}{adj.}{Relativo ao Zimbábue.}{zim.ba.bu.a.no}{0}
\verb{zimbabuano}{}{}{}{}{s.m.}{Indivíduo natural ou habitante desse país. }{zim.ba.bu.a.no}{0}
\verb{zimbório}{}{Arquit.}{}{}{s.m.}{A parte que arremata a cúpula de um edifício; domo.}{zim.bó.rio}{0}
\verb{zimbro}{}{Bot.}{}{}{s.m.}{Planta nativa do hemisfério norte cujos frutos são usados na preparação do gim e da genebra e na aromatização de defumados e conservas; junípero.}{zim.bro}{0}
\verb{zinabre}{}{}{}{}{}{Var. de \textit{azinhavre}.}{zi.na.bre}{0}
\verb{zincar}{}{}{}{}{v.t.}{Revestir de zinco.}{zin.car}{\verboinum{2}}
\verb{zinco}{}{Quím.}{}{}{s.m.}{Elemento químico metálico, branco azulado, bom condutor de calor e de eletricidade, usado como protetor de outros metais contra a corrosão atmosférica, empregado na fabricação de ligas, entre outros. \elemento{30}{65.39}{Zn}.}{zin.co}{0}
\verb{zincografar}{}{}{}{}{v.t.}{Grafar ou imprimir pelo processo de zincografia.}{zin.co.gra.far}{\verboinum{1}}
\verb{zincografia}{}{}{}{}{s.f.}{Técnica de impressão litográfica que usa uma placa de zinco como matriz.}{zin.co.gra.fi.a}{0}
\verb{zincogravura}{}{Art.}{}{}{s.f.}{Processo ou técnica de gravura em zinco.}{zin.co.gra.vu.ra}{0}
\verb{zincogravura}{}{}{}{}{}{Estampa ou ilustração obtida com a técnica de zincogravura.}{zin.co.gra.vu.ra}{0}
\verb{zíngaro}{}{}{}{}{s.m.}{Músico cigano.}{zín.ga.ro}{0}
\verb{zinha}{}{Pop.}{}{}{s.f.}{Mulher inexpressiva, ou a quem não se dá importância.}{zi.nha}{0}
\verb{zinho}{}{Pop.}{}{}{s.m.}{Indivíduo inexpressivo, ou a quem não se dá importância; sujeito.}{zi.nho}{0}
\verb{zínia}{}{Bot.}{}{}{s.f.}{Nome comum a diversos tipos de ervas e arbustos ornamentais nativos da América, de que há variedades com flores de todas as cores.}{zí.nia}{0}
\verb{zinir}{}{}{}{}{}{Var. de \textit{zunir}.}{zi.nir}{0}
\verb{zinzilular}{}{}{}{}{v.i.}{Cantar, soltar a voz (andorinha e outras aves).}{zin.zi.lu.lar}{\verboinum{1}}
\verb{zinzilular}{}{}{}{}{s.m.}{O canto ou a voz de aves que zinzilulam, como a andorinha.}{zin.zi.lu.lar}{0}
\verb{zipar}{}{Informát.}{}{}{v.t.}{Compactar ou diminuir o tamanho de um arquivo para armazenamento ou transmissão de dados.}{zi.par}{\verboinum{1}}
\verb{zíper}{}{}{}{}{s.m.}{Fecho corrediço formado por duas tiras com dentes que se encaixam e desencaixam, fechando e abrindo a peça; usado em bolsas, roupas, artefatos de couro, entre outros; fecho ecler.}{zí.per}{0}
\verb{zircônio}{}{Quím.}{}{}{s.m.}{Elemento químico metálico, prateado, usado em ligas anticorrosivas, em filamentos de lâmpadas, na fabricação de barcos, em ímãs, em reatores nucleares, entre outros. \elemento{40}{91.224}{Zr}.}{zir.cô.nio}{0}
\verb{ziziar}{}{}{}{}{v.i.}{Emitir som estridente (a cigarra ou o gafanhoto); sibilar, fretenir.}{zi.zi.ar}{\verboinum{1}}
\verb{ziziar}{}{}{}{}{s.m.}{O canto de insetos como a cigarra.}{zi.zi.ar}{0}
\verb{Zn}{}{Quím.}{}{}{}{Símb. do \textit{zinco}.}{Zn}{0}
\verb{zoada}{}{}{}{}{s.f.}{Barulho forte e confuso; zumbido, algazarra, gritaria, zoeira.}{zo.a.da}{0}
\verb{zoar}{}{}{}{}{v.i.}{Fazer zoada; zumbir, soar fortemente.}{zo.ar}{\verboinum{7}}
\verb{zodiacal}{}{}{"-ais}{}{adj.2g.}{Relativo ou pertencente ao zodíaco.}{zo.di.a.cal}{0}
\verb{zodíaco}{}{Astron.}{}{}{s.m.}{Região do espaço celeste dividida em doze partes iguais chamadas signos, na qual se situa o movimento aparente do Sol, da Lua e dos planetas do Sistema Solar. }{zo.dí.a.co}{0}
\verb{zodíaco}{}{Astrol.}{}{}{}{A representação figurada dessa região, com seus doze signos.}{zo.dí.a.co}{0}
\verb{zoeira}{ê}{Pop.}{}{}{s.f.}{Zoada.}{zo.ei.ra}{0}
\verb{zoeira}{ê}{}{}{}{}{Vozerio, confusão, barulho.}{zo.ei.ra}{0}
\verb{zoilo}{ô}{}{}{}{s.m.}{Crítico mordaz e invejoso.}{zoi.lo}{0}
\verb{zombador}{ô}{}{}{}{adj.}{Que é dado a zombarias; zombeteiro.}{zom.ba.dor}{0}
\verb{zombador}{ô}{}{}{}{s.m.}{Esse indivíduo. }{zom.ba.dor}{0}
\verb{zombar}{}{}{}{}{v.t.}{Falar com ironia ou maldade de coisa ou pessoa com o propósito de ofender ou expor ao ridículo; escarnecer, mofar, mangar, troçar.}{zom.bar}{\verboinum{1}}
\verb{zombaria}{}{}{}{}{s.f.}{Dito ou ação com o objetivo de ridicularizar alguém ou algo, demonstrando descrédito, ironia, descrença, escárnio; troça, caçoada, mofa, pilhéria.}{zom.ba.ri.a}{0}
\verb{zombetear}{}{}{}{}{v.t.}{Zombar.}{zom.be.te.ar}{\verboinum{4}}
\verb{zombeteiro}{ê}{}{}{}{adj.}{Que zomba; zombador.}{zom.be.tei.ro}{0}
\verb{zona}{}{}{}{}{s.f.}{Faixa, parte, local.}{zo.na}{0}
\verb{zona}{}{}{}{}{}{Região caracterizada por um tipo de vegetação, relevo, temperatura ou atividade econômica.}{zo.na}{0}
\verb{zona}{}{}{}{}{}{Parte de uma cidade em que funcionam certas atividades ou que possui alguma característica especial.}{zo.na}{0}
\verb{zona}{}{}{}{}{}{Região onde se acha estabelecido o meretrício.}{zo.na}{0}
\verb{zoneamento}{}{}{}{}{s.m.}{Ato ou efeito de zonear.}{zo.ne.a.men.to}{0}
\verb{zoneamento}{}{}{}{}{}{Divisão de uma área em zonas ou setores. }{zo.ne.a.men.to}{0}
\verb{zonear}{}{}{}{}{v.t.}{Dividir uma área em zonas ou setores.}{zo.ne.ar}{0}
\verb{zonear}{}{Pop.}{}{}{v.i.}{Fazer bagunça.}{zo.ne.ar}{\verboinum{4}}
\verb{zonzar}{}{}{}{}{v.i.}{Ficar zonzo.}{zon.zar}{\verboinum{1}}
\verb{zonzeira}{ê}{Bras.}{}{}{s.f.}{Estado ou condição de quem está tonto, zonzo; tonteira, tontura, vertigem.}{zon.zei.ra}{0}
\verb{zonzeira}{ê}{Por ext.}{}{}{}{Desmaio, fraqueza, atordoamento.}{zon.zei.ra}{0}
\verb{zonzo}{}{Bras.}{}{}{adj.}{Que sente tudo rodar a sua volta; tonto, atordoado.}{zon.zo}{0}
\verb{zoo}{ô}{}{}{}{s.m.}{Forma reduzida de \textit{zoológico}.}{zo.o}{0}
\verb{zoofagia}{}{}{}{}{s.f.}{Qualidade de zoófago; característica de certos animais de comer a sua presa antes que ela esteja morta.}{zo.o.fa.gi.a}{0}
\verb{zoófago}{}{}{}{}{adj.}{Diz"-se de animal que se alimenta de outro; que pratica a zoofagia.  }{zo.ó.fa.go}{0}
\verb{zoófago}{}{}{}{}{s.m.}{Esse animal.}{zo.ó.fa.go}{0}
\verb{zoofilia}{}{}{}{}{s.f.}{Amizade ou amor pelos animais.}{zo.o.fi.li.a}{0}
\verb{zoofilia}{}{Bot.}{}{}{}{Polinização efetuada por animais, geralmente insetos.}{zo.o.fi.li.a}{0}
\verb{zoofilia}{}{}{}{}{}{Afetividade por animais.}{zo.o.fi.li.a}{0}
%\verb{zoófilo}{}{}{}{}{}{0}{zo.ó.fi.lo}{0}
\verb{zoofobia}{}{Med.}{}{}{s.f.}{Horror, medo mórbido de qualquer animal.}{zo.o.fo.bi.a}{0}
\verb{zoófobo}{}{}{}{}{adj.}{Relativo a zoofobia.}{zo.ó.fo.bo}{0}
\verb{zoófobo}{}{}{}{}{s.m.}{Indivíduo que sofre de zoofobia. }{zo.ó.fo.bo}{0}
\verb{zoogeografia}{}{}{}{}{s.f.}{Ramo da biogeografia que estuda a distribuição geográfica das espécies animais atuais e fósseis. }{zo.o.ge.o.gra.fi.a}{0}
\verb{zoogeográfico}{}{}{}{}{adj.}{Relativo ou pertencente à zoogeografia.}{zo.o.ge.o.grá.fi.co}{0}
\verb{zoólatra}{}{}{}{}{adj.}{Relativo a zoolatria.}{zo.ó.la.tra}{0}
\verb{zoólatra}{}{}{}{}{}{Que pratica a zoolatria.}{zo.ó.la.tra}{0}
\verb{zoolatria}{}{Relig.}{}{}{s.f.}{Culto a ou adoração de animais, tidos como manifestações ou encarnações de divindades.}{zo.o.la.tri.a}{0}
\verb{zoólite}{}{Paleo.}{}{}{s.m.}{Animal ou parte de animal fossilizado.}{zo.ó.li.te}{0}
\verb{zoologia}{}{}{}{}{s.f.}{Ramo da biologia que tem como objeto de estudo os animais.}{zo.o.lo.gi.a}{0}
\verb{zoológico}{}{}{}{}{adj.}{Relativo a zoologia.}{zo.o.ló.gi.co}{0}
\verb{zoológico}{}{}{}{}{s.m.}{Local destinado à exposição permanente de várias espécies de animais; zoo, jardim zoológico.}{zo.o.ló.gi.co}{0}
\verb{zoologista}{}{}{}{}{s.2g.}{Indivíduo especializado ou versado em zoologia; zoólogo.}{zo.o.lo.gis.ta}{0}
\verb{zoólogo}{}{}{}{}{s.m.}{Zoologista.}{zo.ó.lo.go}{0}
\verb{zoom}{}{Por ext.}{}{}{}{Ampliação de imagem ou texto na tela do computador ou da televisão.}{\textit{zoom}}{0}
\verb{zoom}{}{}{}{}{s.m.}{Efeito de aproximação de imagem obtido por um conjunto de lentes cujo alcance focal pode ser ajustado; zum.}{\textit{zoom}}{0}
\verb{zoomorfismo}{}{}{}{}{s.m.}{Culto religioso cujas divindades são representadas sob forma animal.}{zo.o.mor.fis.mo}{0}
\verb{zoomorfismo}{}{}{}{}{}{Crença de que o homem pode se transformar em um animal.}{zo.o.mor.fis.mo}{0}
\verb{zoonose}{ó}{Med.}{}{}{s.f.}{Doença que se manifesta em animais e que pode ser transmitida ao homem. }{zo.o.no.se}{0}
\verb{zoonose}{ó}{}{}{}{}{Doença transmitida aos seres humanos pelos animais, e vice"-versa.}{zo.o.no.se}{0}
\verb{zooplâncton}{}{Biol.}{}{}{s.m.}{Plâncton animal.}{zo.o.plânc.ton}{0}
\verb{zootecnia}{}{}{}{}{}{Ciência do aperfeiçoamento do manejo de animais economicamente úteis.}{zo.o.tec.ni.a}{0}
\verb{zootecnia}{}{}{}{}{s.f.}{Ciência da criação e da domesticação de animais.}{zo.o.tec.ni.a}{0}
\verb{zootécnico}{}{}{}{}{adj.}{Relativo a ou próprio da zootecnia.}{zo.o.téc.ni.co}{0}
\verb{zootécnico}{}{}{}{}{s.m.}{Especialista em zootecnia.}{zo.o.téc.ni.co}{0}
\verb{zorra}{ô}{}{}{}{s.f.}{Espécie de veículo baixo e resistente, de quatro rodas, usado para o transporte de carga pesada.}{zor.ra}{0}
\verb{zorra}{ô}{Fig.}{}{}{}{Pessoa ou coisa vagarosa, lenta, lerda.}{zor.ra}{0}
\verb{zorra}{ô}{}{}{}{}{Certa rede de arrasto usada na pesca de caranguejo.}{zor.ra}{0}
\verb{zorra}{ô}{}{}{}{}{Raposa velha, astuta.}{zor.ra}{0}
\verb{zorra}{ô}{Fig.}{}{}{}{Indivíduo astuto, malicioso; espertalhão.}{zor.ra}{0}
\verb{zorra}{ô}{Pop.}{}{}{}{Grande bagunça; desordem, balbúrdia, confusão, tumulto, zona.}{zor.ra}{0}
\verb{zorrilho}{}{Zool.}{}{}{s.m.}{Mamífero carnívoro, assemelhado à jaritataca, encontrado nas regiões campestres do sul da América do Sul; jaguaré. }{zor.ri.lho}{0}
\verb{Zr}{}{Quím.}{}{}{}{Símb. do \textit{zircônio}.}{Zr}{0}
\verb{zuarte}{}{}{}{}{s.m.}{Certo tecido rústico de algodão.}{zu.ar.te}{0}
\verb{zuído}{}{}{}{}{s.m.}{Zumbido.}{zu.í.do}{0}
\verb{zuir}{}{}{}{}{v.i.}{Zumbir.}{zu.ir}{\verboinum{26}}
\verb{zum}{}{}{}{}{s.m.}{Forma aportuguesada de \textit{zoom}.  }{zum}{0}
\verb{zumbaia}{}{Pop.}{}{}{s.f.}{Cortesia, mesura ou cumprimento exagerado, que demonstra afetação; rapapé, salamaleque.}{zum.bai.a}{0}
\verb{zumbaiar}{}{}{}{}{v.t.}{Fazer zumbaia a; cumprimentar, lisonjear, saudar.}{zum.bai.ar}{0}
\verb{zumbaiar}{}{}{}{}{}{Bajular, adular, cortejar.}{zum.bai.ar}{\verboinum{1}}
\verb{zumbi}{}{}{}{}{s.m.}{Na crença afro"-brasileira, fantasma que vagueia à noite; cazumbi.}{zum.bi}{0}
\verb{zumbi}{}{Fig.}{}{}{}{Indivíduo de hábitos noturnos, que só sai à noite.}{zum.bi}{0}
\verb{zumbi}{}{Hist.}{}{}{}{Chefe do quilombo dos Palmares.}{zum.bi}{0}
\verb{zumbido}{}{}{}{}{s.m.}{Ato ou efeito de zumbir.}{zum.bi.do}{0}
\verb{zumbido}{}{}{}{}{}{Ruído produzido por insetos; zunzum.	}{zum.bi.do}{0}
\verb{zumbido}{}{}{}{}{}{Impressão de zumbido nos ouvidos, produzida por causa patológica ou psicológica, ou decorrente de estampido ou explosão.}{zum.bi.do}{0}
\verb{zumbir}{}{}{}{}{v.i.}{Produzir ruído ao esvoaçar (inseto); zunir.}{zum.bir}{0}
\verb{zumbir}{}{Fig.}{}{}{}{Perceber (o ouvido) ruído semelhante a zumbido.}{zum.bir}{0}
\verb{zumbir}{}{Fig.}{}{}{v.t.}{Dizer em voz baixa; sussurrar.}{zum.bir}{\verboinum{18}}
\verb{zunideira}{ê}{}{}{}{s.f.}{Pedra usada em ourivesaria sobre a qual se alisa o ouro.}{zu.ni.dei.ra}{0}
\verb{zunideira}{ê}{}{}{}{}{Som agudo intenso e prolongado.}{zu.ni.dei.ra}{0}
\verb{zunido}{}{}{}{}{s.m.}{Ato ou efeito de zunir; zumbido.}{zu.ni.do}{0}
\verb{zunir}{}{}{}{}{v.i.}{Produzir ruído agudo e sibilante ao deslocar"-se (como o vento).}{zu.nir}{0}
\verb{zunir}{}{}{}{}{}{Zumbir.}{zu.nir}{\verboinum{18}}
\verb{zunzum}{}{Onomat.}{"-uns}{}{s.m.}{Zumbido.}{zun.zum}{0}
\verb{zunzum}{}{}{"-uns}{}{}{Boato, rumor, falatório, zunzunzum.}{zun.zum}{0}
\verb{zunzunar}{}{}{}{}{v.i.}{Fazer zunzum; zumbir. }{zun.zu.nar}{\verboinum{1}}
\verb{zunzunzum}{}{}{}{}{s.m.}{Zunzum.}{zun.zun.zum}{0}
\verb{zura}{}{}{}{}{adj.2g.}{Sovina, avaro.}{zu.ra}{0}
\verb{zura}{}{}{}{}{s.2g.}{Esse indivíduo.}{zu.ra}{0}
\verb{zureta}{ê}{Pop.}{}{}{adj.}{Amalucado, adoidado, baratinado.}{zu.re.ta}{0}
\verb{zureta}{ê}{}{}{}{}{Atordoado, confuso, transtornado.}{zu.re.ta}{0}
\verb{zurrador}{ô}{}{}{}{adj.}{Que zurra.}{zur.ra.dor}{0}
\verb{zurrapa}{}{}{}{}{adj.}{Que é de má qualidade; ruim, ordinário.}{zur.ra.pa}{0}
\verb{zurrapa}{}{}{}{}{s.f.}{Vinho de má qualidade ou estragado.}{zur.ra.pa}{0}
\verb{zurrar}{}{}{}{}{v.i.}{Emitir (burro, jumento ou outro equídeo) zurro.}{zur.rar}{0}
\verb{zurrar}{}{Pop.}{}{}{}{Trabalhar muito, com afinco.}{zur.rar}{0}
\verb{zurrar}{}{Pop.}{}{}{v.t.}{Dizer (tolices, asneiras).}{zur.rar}{\verboinum{1}}
\verb{zurro}{}{}{}{}{s.m.}{Som produzido por burro, jumento e outros equídeos; ornejo.}{zur.ro}{0}
\verb{zurzir}{}{}{}{}{v.t.}{Espancar, chibatar, açoitar.}{zur.zir}{0}
\verb{zurzir}{}{}{}{}{}{Castigar, fazer mal a, punir.}{zur.zir}{0}
\verb{zurzir}{}{Fig.}{}{}{}{Criticar severamente; repreender com aspereza; magoar.}{zur.zir}{\verboinum{18}}
