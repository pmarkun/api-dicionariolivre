\verb{r}{}{}{}{}{s.m.}{Décima oitava letra do alfabeto português.}{r}{0}
\verb{Ra}{}{Quím.}{}{}{}{Símb. do \textit{rádio}.}{Ra}{0}
\verb{rã}{}{Zool.}{}{}{s.f.}{Nome comum a vários anfíbios anuros de pele lisa, encontrados na água ou em lugares pantanosos, de larga distribuição geográfica, e cujos membros posteriores são mais longos que os dos sapos, adaptados para saltar.}{rã}{0}
\verb{rabada}{}{}{}{}{s.f.}{Cauda de qualquer animal; rabadela, rabadilha.}{ra.ba.da}{0}
\verb{rabada}{}{Cul.}{}{}{}{Prato preparado com o rabo do boi.}{ra.ba.da}{0}
\verb{rabadela}{é}{}{}{}{s.f.}{Rabadilha.}{ra.ba.de.la}{0}
\verb{rabadilha}{}{}{}{}{s.f.}{Parte posterior das aves e mamíferos; uropígio, rabadela.}{ra.ba.di.lha}{0}
\verb{rabanada}{}{}{}{}{s.f.}{Golpe com o rabo ou a cauda.}{ra.ba.na.da}{0}
\verb{rabanada}{}{Cul.}{}{}{}{Fatia de pão amanhecido, mergulhada em açúcar e leite, passada em ovos batidos antes da fritura, e que se serve polvilhada com açúcar e canela.}{ra.ba.na.da}{0}
\verb{rabanete}{ê}{Bot.}{}{}{s.m.}{Planta herbácea cuja raiz, curta e carnosa, é comestível.}{ra.ba.ne.te}{0}
\verb{rabanete}{ê}{}{}{}{}{A raiz dessa planta, de sabor picante, que se come crua em saladas.}{ra.ba.ne.te}{0}
\verb{rábano}{}{Bot.}{}{}{s.m.}{Nome comum a várias plantas hortenses, cujas raízes e folhas, em geral, são comestíveis.}{rá.ba.no}{0}
\verb{rábano}{}{}{}{}{}{Rabanete.}{rá.ba.no}{0}
\verb{rabear}{}{}{}{}{v.i.}{Mexer o rabo ou a cauda.}{ra.be.ar}{0}
\verb{rabear}{}{Bras.}{}{}{}{Derrapar (um automóvel) nas rodas de trás.}{ra.be.ar}{\verboinum{4}}
\verb{rabeca}{é}{}{}{}{s.f.}{Antiga designação do violino.}{ra.be.ca}{0}
\verb{rabeca}{é}{}{}{}{}{Tipo de violino, com quatro cordas e de som fanhoso.}{ra.be.ca}{0}
\verb{rabecão}{}{Pop.}{"-ões}{}{s.m.}{Contrabaixo.}{ra.be.cão}{0}
\verb{rabecão}{}{Pop.}{"-ões}{}{}{Veículo que transporta cadáveres.}{ra.be.cão}{0}
\verb{rabeira}{ê}{}{}{}{s.f.}{Pegada, rastro.}{ra.bei.ra}{0}
\verb{rabeira}{ê}{}{}{}{}{Cauda de vestido.}{ra.bei.ra}{0}
\verb{rabeira}{ê}{Bras.}{}{}{}{Parte traseira de um veículo.}{ra.bei.ra}{0}
\verb{rabi}{}{}{}{}{s.m.}{Rabino.}{ra.bi}{0}
\verb{rabiça}{}{}{}{}{s.f.}{Braço do arado, destinado ao manejo desse utensílio pelo lavrador.}{ra.bi.ça}{0}
\verb{rabicho}{}{}{}{}{s.m.}{Trança de cabelo pendente da parte de trás da cabeça.}{ra.bi.cho}{0}
\verb{rabicho}{}{Pop.}{}{}{}{Relacionamento amoroso; namoro, namorico.}{ra.bi.cho}{0}
\verb{rábico}{}{}{}{}{adj.}{Relativo à raiva (doença), à hidrofobia.}{rá.bi.co}{0}
\verb{rabicó}{}{Bras.}{}{}{adj.2g.}{Animal sem rabo ou com o rabo curto; suro, suru.}{ra.bi.có}{0}
\verb{rabicó}{}{Bras.}{}{}{s.m.}{Pipa sem cauda.}{ra.bi.có}{0}
\verb{rábido}{}{}{}{}{adj.}{Encolerizado, raivoso, furioso.}{rá.bi.do}{0}
\verb{rabínico}{}{}{}{}{adj.}{Relativo ou pertencente aos rabinos.}{ra.bí.ni.co}{0}
\verb{rabino}{}{}{}{}{s.m.}{Líder religioso de uma comunidade judaica; rabi.}{ra.bi.no}{0}
\verb{rabiscar}{}{}{}{}{v.t.}{Fazer rabiscos, garatujas.}{ra.bis.car}{0}
\verb{rabiscar}{}{}{}{}{}{Escrever mal e apressadamente, de forma quase ilegível; escrevinhar, garatujar.}{ra.bis.car}{\verboinum{2}}
\verb{rabisco}{}{}{}{}{s.m.}{Risco torto e mal traçado; garatuja, garrancho.}{ra.bis.co}{0}
\verb{rabo}{}{}{}{}{s.m.}{Prolongamento da coluna vertebral ou da parte posterior do corpo de vários animais; cauda.}{ra.bo}{0}
\verb{rabo}{}{Chul.}{}{}{}{As nádegas, o traseiro.}{ra.bo}{0}
\verb{rabo"-de"-arraia}{}{Bras.}{rabos"-de"-arraia}{}{s.m.}{Golpe de capoeira no qual o lutador gira o corpo sobre a cabeça e tenta atingir com os calcanhares a cabeça do adversário.}{ra.bo"-de"-ar.rai.a}{0}
\verb{rabo"-de"-cavalo}{}{}{rabos"-de"-cavalo}{}{s.m.}{Tipo de penteado em que se amarram, na parte de trás da cabeça, os cabelos, que pendem como a cauda de um cavalo.}{ra.bo"-de"-ca.va.lo}{0}
\verb{rabo"-de"-galo}{}{}{rabos"-de"-galo}{}{s.m.}{Aperitivo de aguardente com vermute.}{ra.bo"-de"-ga.lo}{0}
\verb{rabo"-de"-palha}{}{Bras.}{rabos"-de"-palha}{}{s.m.}{Mácula na reputação, na honra ou qualquer defeito moral que possa ser motivo de censura.}{ra.bo"-de"-pa.lha}{0}
\verb{rabo"-de"-saia}{}{Pop.}{rabos"-de"-saia}{}{s.m.}{Mulher.}{ra.bo"-de"-sai.a}{0}
\verb{rabo"-de"-tatu}{}{}{rabos"-de"-tatu}{}{s.m.}{Pequeno chicote de couro.}{ra.bo"-de"-ta.tu}{0}
\verb{rabona}{}{}{}{}{s.f.}{Fraque de abas curtas.}{ra.bo.na}{0}
\verb{rabudo}{}{}{}{}{adj.}{Que tem a cauda ou o rabo longo.}{ra.bu.do}{0}
\verb{rabugem}{}{}{"-ens}{}{s.f.}{Tipo de sarna que ataca os cães.}{ra.bu.gem}{0}
\verb{rabugem}{}{Fig.}{"-ens}{}{}{Rabugice.}{ra.bu.gem}{0}
\verb{rabugento}{}{}{}{}{adj.}{Que tem rabugem (tipo de sarna).}{ra.bu.gen.to}{0}
\verb{rabugento}{}{Fig.}{}{}{}{Que tem mau humor, que se queixa por qualquer coisa; mal"-humorado, ranheta, ranzinza.}{ra.bu.gen.to}{0}
\verb{rabugice}{}{}{}{}{s.f.}{Qualidade ou modos de rabugento; mau humor, rabugem, impertinência.  }{ra.bu.gi.ce}{0}
\verb{rabujar}{}{}{}{}{v.i.}{Ter rabugice, comportar"-se como rabugento; resmungar,ralhar.}{ra.bu.jar}{\verboinum{1}}
\verb{rábula}{}{Pejor.}{}{}{s.m.}{Advogado incompetente, de pouca cultura ou que se vale de ardis em suas causas.}{rá.bu.la}{0}
\verb{rábula}{}{Bras.}{}{}{}{Pessoa que advoga sem ter diploma.}{rá.bu.la}{0}
\verb{raça}{}{}{}{}{s.f.}{Variedade de uma espécie animal cujos traços se conservam através das gerações. (\textit{Na competição, participavam cães de todas as raças.})}{ra.ça}{0}
\verb{raça}{}{Bras.}{}{}{}{Vontade, determinação, empenho, coragem.}{ra.ça}{0}
\verb{ração}{}{}{"-ões}{}{s.f.}{Porção de alimento calculada para o consumo diário ou para cada refeição de uma pessoa. }{ra.ção}{0}
\verb{ração}{}{}{"-ões}{}{}{Porção de alimento que se dá a um animal para suprir suas necessidades nutricionais diárias.}{ra.ção}{0}
\verb{racemo}{}{Bot.}{}{}{s.m.}{Tipo de inflorescência, em forma de cachos.}{ra.ce.mo}{0}
\verb{racemo}{}{}{}{}{}{Cacho de uvas.}{ra.ce.mo}{0}
\verb{racha}{}{}{}{}{s.f.}{Fenda, abertura de alguma coisa que se rachou; rachadura, greta.}{ra.cha}{0}
\verb{racha}{}{Pop.}{}{}{}{Divisão, partilha que se faz entre duas ou mais pessoas.}{ra.cha}{0}
\verb{racha}{}{Pop.}{}{}{}{Corrida de carro proibida por lei, em geral entre garotos ou amadores, e que se realiza em ruas, avenidas ou pista improvisada. (\textit{Os garotos que participavam de um racha na  avenida foram presos em flagrante.})}{ra.cha}{0}
\verb{racha}{}{Bras.}{}{}{s.m.}{Separação, rompimento, divergência, dissensão. (\textit{Houve um racha no partido dos trabalhadores.})}{ra.cha}{0}
\verb{rachadura}{}{}{}{}{s.f.}{Ato ou efeito de rachar; racha, fenda, greta, abertura.}{ra.cha.du.ra}{0}
\verb{rachar}{}{}{}{}{v.t.}{Partir alguma coisa no sentido do comprimento; trincar, fender, gretar.}{ra.char}{0}
\verb{rachar}{}{Bras.}{}{}{}{Dividir algo pela metade.}{ra.char}{0}
\verb{rachar}{}{}{}{}{}{Partir em lascas; lascar.}{ra.char}{\verboinum{1}}
\verb{racial}{}{}{"-ais}{}{adj.2g.}{Relativo a raça. }{ra.ci.al}{0}
\verb{racimo}{}{}{}{}{s.m.}{Racemo.}{ra.ci.mo}{0}
\verb{raciocinar}{}{}{}{}{v.i.}{Fazer uso da razão para conhecer, julgar, calcular ou entender fatos, coisas e relações; fazer um raciocínio; pensar, refletir.}{ra.ci.o.ci.nar}{0}
\verb{raciocinar}{}{}{}{}{v.t.}{Considerar, ponderar, avaliar.}{ra.ci.o.ci.nar}{\verboinum{1}}
\verb{raciocínio}{}{}{}{}{s.m.}{Ato ou efeito de raciocinar; reflexão, ponderação.}{ra.ci.o.cí.nio}{0}
\verb{raciocínio}{}{}{}{}{}{Encadeamento lógico de argumentos.}{ra.ci.o.cí.nio}{0}
\verb{racional}{}{}{"-ais}{}{adj.2g.}{Que raciocina, faz uso da razão; que tem a faculdade de raciocinar.}{ra.ci.o.nal}{0}
\verb{racional}{}{}{"-ais}{}{}{Conforme à razão; razoável.}{ra.ci.o.nal}{0}
\verb{racional}{}{Mat.}{"-ais}{}{}{Diz"-se do número que pode ser expresso pelo quociente de dois números inteiros.}{ra.ci.o.nal}{0}
\verb{racional}{}{}{"-ais}{}{s.m.}{O ser capaz de raciocínio.}{ra.ci.o.nal}{0}
\verb{racionalismo}{}{}{}{}{s.m.}{Doutrina para a qual a razão independe da experiência.}{ra.ci.o.na.lis.mo}{0}
\verb{racionalização}{}{}{"-ões}{}{s.f.}{Ato ou efeito de racionalizar.}{ra.ci.o.na.li.za.ção}{0}
\verb{racionalizar}{}{}{}{}{v.t.}{Tornar racional.}{ra.ci.o.na.li.zar}{0}
\verb{racionalizar}{}{}{}{}{}{Tornar um trabalho, sistema, operação etc. mais eficiente por meio de métodos científicos.}{ra.ci.o.na.li.zar}{\verboinum{1}}
\verb{racionamento}{}{}{}{}{s.m.}{Ato ou efeito de racionar.}{ra.ci.o.na.men.to}{0}
\verb{racionamento}{}{}{}{}{}{Limitação da distribuição ou venda de certos bens ou produtos escassos a fim de garantir a sua distribuição justa entre a população.}{ra.ci.o.na.men.to}{0}
\verb{racionar}{}{}{}{}{v.t.}{Distribuir em rações, em quantidades determinadas.}{ra.ci.o.nar}{0}
\verb{racionar}{}{}{}{}{}{Utilizar ou consumir com parcimônia, moderadamente; poupar, economizar.}{ra.ci.o.nar}{0}
\verb{racionar}{}{}{}{}{}{Limitar, restringir um serviço ou a venda de um produto.}{ra.ci.o.nar}{\verboinum{1}}
\verb{racismo}{}{}{}{}{s.m.}{Doutrina que considera certa raça superior à outra.}{ra.cis.mo}{0}
\verb{racismo}{}{}{}{}{}{Qualidade ou atitude de pessoa partidária dessa doutrina.}{ra.cis.mo}{0}
\verb{racista}{}{}{}{}{adj.2g.}{Relativo a racismo.}{ra.cis.ta}{0}
\verb{racista}{}{}{}{}{s.2g.}{Adepto ou simpatizante do racismo.}{ra.cis.ta}{0}
\verb{radar}{}{}{}{}{s.m.}{Técnica que permite localizar objetos a grande distância por meio de ondas eletromagnéticas.}{ra.dar}{0}
\verb{radar}{}{}{}{}{}{O equipamento usado para essa finalidade.}{ra.dar}{0}
\verb{radiação}{}{}{"-ões}{}{s.f.}{Ato ou efeito de radiar.}{ra.di.a.ção}{0}
\verb{radiação}{}{Fís.}{"-ões}{}{}{A emissão de energia por meio de ondas ou partículas. }{ra.di.a.ção}{0}
\verb{radiador}{ô}{}{}{}{s.m.}{Aparelho utilizado para refrigerar a água em circulação num motor de explosão.}{ra.di.a.dor}{0}
\verb{radial}{}{}{"-ais}{}{adj.2g.}{Relativo a raio.}{ra.di.al}{0}
\verb{radial}{}{}{"-ais}{}{}{Que emite raios.}{ra.di.al}{0}
\verb{radial}{}{}{"-ais}{}{s.f.}{Rua de uma cidade que liga o centro à periferia.}{ra.di.al}{0}
\verb{radialista}{}{}{}{}{s.2g.}{Pessoa que trabalha com programas de rádio e televisão.}{ra.di.a.lis.ta}{0}
\verb{radiano}{}{Mat.}{}{}{s.m.}{Unidade de medida de um ângulo plano. Símb.: rad.}{ra.di.a.no}{0}
\verb{radiante}{}{}{}{}{adj.2g.}{Que emite raios.}{ra.di.an.te}{0}
\verb{radiante}{}{}{}{}{}{Fulgurante, brilhante.}{ra.di.an.te}{0}
\verb{radiante}{}{}{}{}{}{Transbordante, cheio.}{ra.di.an.te}{0}
\verb{radiar}{}{}{}{}{v.t.}{Espalhar alguma forma de energia.}{ra.di.ar}{0}
\verb{radiar}{}{}{}{}{}{Espalhar algum sentimento.}{ra.di.ar}{\verboinum{1}}
\verb{radiatividade}{}{}{}{}{s.f.}{Radiação eletromagnética.}{ra.di.a.ti.vi.da.de}{0}
\verb{radiativo}{}{}{}{}{}{Var. de \textit{radioativo}.}{ra.di.a.ti.vo}{0}
\verb{radical}{}{}{"-ais}{}{adj.2g.}{Em que se faz uma mudança muito grande, sem que nada deixe de ser alterado; completo, total.}{ra.di.cal}{0}
\verb{radical}{}{}{"-ais}{}{}{Que deseja fazer reformas muito grandes na vida política, econômica e social de uma país.}{ra.di.cal}{0}
\verb{radical}{}{Gram.}{"-ais}{}{s.m.}{Parte da estrutura de uma palavra que contém seu significado básico e recebe os sufixos flexionais.}{ra.di.cal}{0}
\verb{radicalismo}{}{}{}{}{s.m.}{Sistema político que visa a mudanças sociais extremadas e imediatas.}{ra.di.ca.lis.mo}{0}
\verb{radicalizar}{}{}{}{}{v.t.}{Tornar radical; adotar postura, ponto de vista extremado em relação a algo.}{ra.di.ca.li.zar}{\verboinum{1}}
\verb{radicando}{}{Mat.}{}{}{s.m.}{Número ou expressão algébrica sob o radical.}{ra.di.can.do}{0}
\verb{radicar}{}{}{}{}{v.t.}{Infundir ou estabelecer de maneira profunda; enraizar, arraigar.}{ra.di.car}{0}
\verb{radicar}{}{}{}{}{v.pron.}{Fixar residência.}{ra.di.car}{\verboinum{2}}
\verb{radiciação}{}{Mat.}{"-ões}{}{s.f.}{Operação que permite obter a raiz de um número ou expressão.}{ra.di.ci.a.ção}{0}
\verb{radícula}{}{}{}{}{s.f.}{Raiz pequena; embrião de raiz.}{ra.dí.cu.la}{0}
\verb{radieletricidade}{}{Fís.}{}{}{s.f.}{Radioeletricidade.}{ra.di.e.le.tri.ci.da.de}{0}
\verb{rádio}{}{Quím.}{}{}{s.m.}{Elemento químico metálico, brilhante, radioativo, do grupo dos alcalino"-terrosos, utilizado na fabricação de tintas luminescentes e pela medicina, no tratamento do câncer e em radiografias. \elemento{88}{(226)}{Ra}.}{rá.dio}{0}
\verb{rádio}{}{}{}{}{s.m.}{Aparelho receptor ou emissor de telegrafia ou telefonia sem fios.}{rá.dio}{0}
\verb{rádio}{}{}{}{}{}{Estação emissora de ondas sonoras ou radiofônicas; radiorreceptor.}{rá.dio}{0}
\verb{rádio}{}{}{}{}{s.f.}{Emissora e transmissora de programas ou notícias pela radiofonia; radioemissora; radiodifusora.}{rá.dio}{0}
\verb{rádio}{}{Anat.}{}{}{s.m.}{Um dos ossos do antebraço.}{rá.dio}{0}
\verb{radioamador}{ô}{}{}{}{s.m.}{Indivíduo que opera sem finalidade lucrativa em estação particular de rádio.}{ra.di.o.a.ma.dor}{0}
\verb{radioamadorismo}{}{}{}{}{s.m.}{Atividade de radioamador.}{ra.di.o.a.ma.do.ris.mo}{0}
\verb{radioatividade}{}{}{}{}{s.f.}{Radiação eletromagnética.}{ra.di.o.a.ti.vi.da.de}{0}
\verb{radioativo}{}{}{}{}{adj.}{Que solta partícula de dentro de seus átomos, produzindo energia.}{ra.di.o.a.ti.vo}{0}
\verb{radioator}{ô}{}{}{}{s.m.}{Ator de programas de rádio.}{ra.di.o.a.tor}{0}
\verb{radiocomunicação}{}{}{"-ões}{}{s.f.}{Transmissão de sons ou imagens por meio de ondas eletromagnéticas, usando"-se aparelhos especiais.}{ra.di.o.co.mu.ni.ca.ção}{0}
\verb{radiodifusão}{}{}{"-ões}{}{s.f.}{Emissão e transmissão de programas por meio de radiofonia.}{ra.di.o.di.fu.são}{0}
\verb{radiodifusor}{ô}{}{}{}{s.m.}{Pessoa física ou jurídica que faz difundir um programa de rádio para um número ilimitado de estações receptoras.}{ra.di.o.di.fu.sor}{0}
\verb{radiodifusor}{ô}{}{}{}{}{Aparelho de radiodifusão.}{ra.di.o.di.fu.sor}{0}
\verb{radiodifusor}{ô}{}{}{}{adj.}{Que faz radiodifusão.}{ra.di.o.di.fu.sor}{0}
\verb{radiodifusora}{ô}{}{}{}{s.f.}{Emissora de radiodifusão.}{ra.di.o.di.fu.so.ra}{0}
\verb{radioeletricidade}{}{Fís.}{}{}{s.f.}{Parte da física que trata do estudo e das aplicações das ondas hertzianas, como  telegrafia ou telefonia sem fio, rádio, televisão etc.  }{ra.di.o.e.le.tri.ci.da.de}{0}
\verb{radioemissora}{ô}{}{}{}{s.f.}{Empresa onde se produzem e de onde se transmitem programas de rádio; emissora, estação de rádio, rádio.}{ra.di.o.e.mis.so.ra}{0}
\verb{radiofonia}{}{}{}{}{s.f.}{Sistema de transmissão sonora por ondas eletromagnéticas.}{ra.di.o.fo.ni.a}{0}
\verb{radiofônico}{}{}{}{}{adj.}{Relativo a radiofonia; radiotelefônico.}{ra.di.o.fô.ni.co}{0}
\verb{radiofoto}{ó}{}{}{}{s.f.}{Imagem convertida em linhas de pontos claros e escuros e desta forma transmitida por meio de ondas de rádio para ser reproduzida a distância por aparelhos receptores.}{ra.di.o.fo.to}{0}
\verb{radiofotografia}{}{}{}{}{s.f.}{Radiofoto.}{ra.di.o.fo.to.gra.fi.a}{0}
\verb{radiografar}{}{}{}{}{v.t.}{Tirar radiografia de parte do corpo.}{ra.di.o.gra.far}{\verboinum{1}}
\verb{radiografia}{}{}{}{}{s.f.}{Fotografia de alguma parte interna do corpo; chapa.}{ra.di.o.gra.fi.a}{0}
\verb{radiográfico}{}{}{}{}{adj.}{Relativo a radiografia.}{ra.di.o.grá.fi.co}{0}
\verb{radiograma}{}{}{}{}{s.m.}{Mensagem enviada por meio de ondas de rádio.}{ra.di.o.gra.ma}{0}
\verb{radiola}{ó}{}{}{}{s.f.}{Aparelho com rádio e vitrola; radiovitrola.}{ra.di.o.la}{0}
\verb{radiologia}{}{}{}{}{s.f.}{Estudo científico dos raios luminosos e raios \textsc{x}.}{ra.di.o.lo.gi.a}{0}
\verb{radiologia}{}{}{}{}{}{Uso de raios \textsc{x} no diagnóstico e tratamento de doenças.}{ra.di.o.lo.gi.a}{0}
\verb{radiológico}{}{}{}{}{adj.}{Relativo a radiologia.}{ra.di.o.ló.gi.co}{0}
\verb{radiologista}{}{}{}{}{s.2g.}{Especialista em radiologia.}{ra.di.o.lo.gis.ta}{0}
\verb{radionovela}{é}{}{}{}{s.f.}{Novela veiculada no rádio.}{ra.di.o.no.ve.la}{0}
\verb{radiopatrulha}{}{}{}{}{s.f.}{Carro de polícia ligado a uma central de rádio.}{ra.di.o.pa.tru.lha}{0}
\verb{radiopatrulha}{}{}{}{}{}{Ronda feita com esse carro.}{ra.di.o.pa.tru.lha}{0}
\verb{radiorreceptor}{ô}{}{}{}{s.m.}{Aparelho que capta ondas de rádio.}{ra.di.or.re.cep.tor}{0}
\verb{radioscopia}{}{}{}{}{s.f.}{Exame de órgão em tela fluorescente por meio de raios \textsc{x}.}{ra.di.os.co.pi.a}{0}
\verb{radioscópico}{}{}{}{}{adj.}{Relativo a radioscopia.}{ra.di.os.có.pi.co}{0}
\verb{radioso}{ô}{}{"-osos ⟨ó⟩}{"-osa ⟨ó⟩}{adj.}{Que lança raios de luz; resplandescente.}{ra.di.o.so}{0}
\verb{radioso}{ô}{Fig.}{"-osos ⟨ó⟩}{"-osa ⟨ó⟩}{}{Muito alegre; contente, radiante.}{ra.di.o.so}{0}
\verb{radiotáxi}{cs}{}{}{}{s.m.}{Táxi ligado a uma central de rádio.}{ra.di.o.tá.xi}{0}
\verb{radioteatro}{}{}{}{}{s.m.}{Peça teatral veiculada pelo rádio.}{ra.di.o.te.a.tro}{0}
\verb{radiotécnica}{}{}{}{}{s.f.}{Técnica das ciências radioelétricas.}{ra.di.o.téc.ni.ca}{0}
\verb{radiotécnica}{}{}{}{}{}{Aplicação industrial de radiações.}{ra.di.o.téc.ni.ca}{0}
\verb{radiotelefonia}{}{}{}{}{s.f.}{Telefonia por meio de ondas eletromagnéticas.}{ra.di.o.te.le.fo.ni.a}{0}
\verb{radiotelefônico}{}{}{}{}{adj.}{Relativo à radiotelefonia; radiofônico.}{ra.di.o.te.le.fô.ni.co}{0}
\verb{radiotelegrafia}{}{}{}{}{s.f.}{Telegrafia por meio de ondas eletromagnéticas.}{ra.di.o.te.le.gra.fi.a}{0}
\verb{radiotelegráfico}{}{}{}{}{adj.}{Relativo a radiotelegrafia.}{ra.di.o.te.le.grá.fi.co}{0}
\verb{radiotelegrafista}{}{}{}{}{s.2g.}{Operador de radiotelegrafia.}{ra.di.o.te.le.gra.fis.ta}{0}
\verb{radioterapêutico}{}{}{}{}{adj.}{Que se refere ou pertence à radioterapia; radioterápico.}{ra.di.o.te.ra.pêu.ti.co}{0}
\verb{radioterapia}{}{}{}{}{s.f.}{Emprego terapêutico de raios ionizantes no tratamento de certas doenças, especialmente do câncer.}{ra.di.o.te.ra.pi.a}{0}
\verb{radiotransmissão}{}{}{"-ões}{}{s.f.}{Transmissão sonora por meio da radiofonia.}{ra.di.o.trans.mis.são}{0}
\verb{radiotransmissor}{ô}{}{}{}{adj.}{Diz"-se de aparelho radioelétrico destinado à emissão de ondas hertzianas.}{ra.di.o.trans.mis.sor}{0}
\verb{radiouvinte}{}{}{}{}{s.2g.}{Ouvinte de emissão radiofônica.}{ra.di.ou.vin.te}{0}
\verb{radiovitrola}{ó}{}{}{}{s.f.}{Aparelho em que se conjugam o rádio e a vitrola; radiola.}{ra.di.o.vi.tro.la}{0}
\verb{radônio}{}{Quím.}{}{}{s.m.}{Elemento químico da família dos gases nobres, radioativo, utilizado em radioterapia. \elemento{86}{(222)}{Rn}.}{ra.dô.nio}{0}
\verb{rafeiro}{ê}{}{}{}{adj.}{Diz"-se de cão que guarda gado.}{ra.fei.ro}{0}
\verb{ráfia}{}{Bot.}{}{}{s.f.}{Gênero de palmeiras africanas e americanas, cujas palmas dão ótimas fibras.}{rá.fia}{0}
\verb{ráfia}{}{}{}{}{}{O fio industrializado dessa fibra.}{rá.fia}{0}
\verb{ráfia}{}{}{}{}{}{Fio sintético semelhante à ráfia, usado em bordados, crochê etc.}{rá.fia}{0}
\verb{rafting}{}{Esport.}{}{}{s.m.}{Modalidade esportiva que consiste em descer rios encachoeirados em botes de borracha infláveis.}{\textit{rafting}}{0}
\verb{ragu}{}{Cul.}{}{}{s.m.}{Ensopado de carne com legumes e muito molho.}{ra.gu}{0}
\verb{raia}{}{Zool.}{}{}{s.f.}{Peixe de corpo achatado e nadadeiras em forma de asas.}{rai.a}{0}
\verb{raia}{}{}{}{}{}{Brinquedo feito de uma armação de varetas, coberta com papel fino, que se prende a um fio e é solto ao vento; papagaio, pipa.}{rai.a}{0}
\verb{raia}{}{}{}{}{}{Linha feita com alguma coisa pontuda; estria, listra, risca.}{rai.a}{0}
\verb{raia}{}{}{}{}{}{Faixa de pista de corridas de cavalo.}{rai.a}{0}
\verb{raia}{}{}{}{}{}{Cada uma das faixas que dividem uma piscina nas competições de natação.}{rai.a}{0}
\verb{raiado}{}{}{}{}{adj.}{Cheio de raias ou riscas; rajado}{rai.a.do}{0}
\verb{raiar}{}{}{}{}{v.i.}{Emitir raios de luz.}{rai.ar}{0}
\verb{raiar}{}{}{}{}{}{Despontar, surgir.}{rai.ar}{0}
\verb{raiar}{}{}{}{}{v.t.}{Cobrir de riscas ou raias.}{rai.ar}{\verboinum{1}}
\verb{rainha}{}{}{}{}{s.f.}{Mulher que governa ou representa um reino.}{ra.i.nha}{0}
\verb{rainha}{}{}{}{}{}{Esposa de rei.}{ra.i.nha}{0}
\verb{rainha}{}{}{}{}{}{Peça mais importante, depois do rei, no jogo de xadrez.}{ra.i.nha}{0}
\verb{rainha}{}{}{}{}{}{Abelha que põe ovos, permitindo a formação e garantindo a sobrevivência de uma colmeia.}{ra.i.nha}{0}
\verb{raio}{}{}{}{}{s.m.}{Cada um dos traços de luz que saem de um foco luminoso e seguem em linha reta.}{rai.o}{0}
\verb{raio}{}{}{}{}{}{Descarga elétrica que é produzida em uma nuvem e que atinge o chão; faísca.}{rai.o}{0}
\verb{raio}{}{}{}{}{}{Reta que vai do centro até a borda de uma circunferência.}{rai.o}{0}
\verb{raio}{}{}{}{}{}{Cada uma das varas que vão do eixo ao aro de uma roda.}{rai.o}{0}
\verb{raiom}{}{}{"-ons}{}{s.m.}{Fibra sintética macia.}{rai.om}{0}
\verb{raiom}{}{}{"-ons}{}{}{Tecido dessa fibra.}{rai.om}{0}
\verb{raiva}{}{Veter.}{}{}{s.f.}{Doença infecciosa causada por vírus e transmitida pela mordida de animais infectados como o cão, lobo, gato, e que acomete o sistema nervoso central, provocando parada respiratória e convulsões; hidrofobia. }{rai.va}{0}
\verb{raiva}{}{}{}{}{}{Sentimento de irritação, agressividade, rancor ou frustração, motivados por aborrecimento, injustiça ou rejeição sofridas etc.  }{rai.va}{0}
\verb{raivoso}{ô}{}{"-osos ⟨ó⟩}{"-osa ⟨ó⟩}{adj.}{Doente de raiva; hidrófobo.}{rai.vo.so}{0}
\verb{raivoso}{ô}{}{"-osos ⟨ó⟩}{"-osa ⟨ó⟩}{}{Cheio de raiva; furioso, irado.}{rai.vo.so}{0}
\verb{raiz}{}{Bot.}{}{}{s.f.}{Órgão da planta geralmente fixo ao solo, de onde ela tira nutrientes.}{ra.iz}{0}
\verb{raiz}{}{Gram.}{}{}{}{Parte da palavra que guarda seu sentido e origem.}{ra.iz}{0}
\verb{raiz}{}{}{}{}{}{Parte do dente que se prende ao osso.}{ra.iz}{0}
\verb{raiz}{}{}{}{}{}{Base, origem.}{ra.iz}{0}
\verb{raiz}{}{Mat.}{}{}{}{Valor da incógnita que resolve uma equação.}{ra.iz}{0}
\verb{raiz}{}{Mat.}{}{}{}{Número que elevado ao índice do radical reproduz o radicando.}{ra.iz}{0}
\verb{raizama}{}{}{}{}{}{Var. de \textit{raizame}.}{ra.i.za.ma}{0}
\verb{raizame}{}{}{}{}{s.m.}{Conjunto das raízes de uma planta.}{ra.i.za.me}{0}
\verb{rajá}{}{}{}{}{s.m.}{Príncipe indiano subordinado a um suserano.}{ra.já}{0}
\verb{rajada}{}{}{}{}{s.f.}{Vento forte e de curta duração; lufada.}{ra.ja.da}{0}
\verb{rajada}{}{}{}{}{}{Conjunto dos tiros disparados por uma arma automática a cada vez que se aciona o gatilho.}{ra.ja.da}{0}
\verb{rajado}{}{}{}{}{adj.}{Cheio de raias ou riscas; listrado, raiado.}{ra.ja.do}{0}
\verb{rajar}{}{}{}{}{v.t.}{Fazer raias ou riscas em alguma coisa; estriar, listrar, raiar.                                        }{ra.jar}{\verboinum{1}}
\verb{ralador}{ô}{}{}{}{s.m.}{Utensílio metálico com furos farpados, próprio para ralar.}{ra.la.dor}{0}
\verb{ralar}{}{}{}{}{v.t.}{Triturar no ralador.}{ra.lar}{0}
\verb{ralar}{}{}{}{}{}{Ferir arranhando a pele.}{ra.lar}{\verboinum{1}}
\verb{ralé}{}{}{}{}{s.f.}{A camada mais baixa da sociedade; escória.}{ra.lé}{0}
\verb{ralear}{}{}{}{}{v.t.}{Tornar ralo, menos denso.}{ra.le.ar}{\verboinum{4}}
\verb{ralhar}{}{}{}{}{v.t.}{Repreender severamente.}{ra.lhar}{\verboinum{1}}
\verb{ralho}{}{}{}{}{s.m.}{Ato ou efeito de ralhar; repreensão dirigida autoritariamente a alguém a quem se pretende corrigir ou censurar.}{ra.lho}{0}
\verb{ralho}{}{}{}{}{}{Troca violenta de palavras ou acusações; discussão, altercação.}{ra.lho}{0}
\verb{rali}{}{}{}{}{s.m.}{Corrida de carros ou motos, dentro de uma velocidade média preestabelecida.}{ra.li}{0}
\verb{ralo}{}{}{}{}{s.m.}{Ralador.}{ra.lo}{0}
\verb{ralo}{}{}{}{}{}{Fundo de peneira; crivo.}{ra.lo}{0}
\verb{ralo}{}{}{}{}{}{Chapa com furos para deixar passar água, colocada na boca de um encanamento.}{ra.lo}{0}
\verb{ralo}{}{}{}{}{adj.}{Que é pouco denso, pouco espesso.}{ra.lo}{0}
\verb{ralo}{}{}{}{}{}{Escasso, raro.}{ra.lo}{0}
\verb{RAM}{}{Informát.}{}{}{}{Sigla inglesa de \textit{Random Access Memory}; dispositivo de memória temporária que permite leitura e gravação de dados e programas.}{RAM}{0}
\verb{rama}{}{}{}{}{s.f.}{Conjunto dos ramos de uma planta; ramada, ramagem.}{ra.ma}{0}
\verb{ramada}{}{}{}{}{s.f.}{Rama.}{ra.ma.da}{0}
\verb{ramagem}{}{}{"-ens}{}{s.f.}{Conjunto de galhos e ramos.}{ra.ma.gem}{0}
\verb{ramal}{}{}{"-ais}{}{s.m.}{Caminho que sai de uma estrada principal.}{ra.mal}{0}
\verb{ramal}{}{}{"-ais}{}{}{Cada uma das linhas internas de uma rede telefônica.}{ra.mal}{0}
\verb{ramalhar}{}{}{}{}{v.t.}{Agitar ou sacudir os ramos de.}{ra.ma.lhar}{0}
\verb{ramalhar}{}{}{}{}{}{Sussurrar com vento; fazer ruído, agitando"-se os ramos das árvores.}{ra.ma.lhar}{\verboinum{1}}
\verb{ramalhete}{ê}{}{}{}{s.m.}{Ramo de flores; buquê.}{ra.ma.lhe.te}{0}
\verb{ramalho}{}{}{}{}{s.m.}{Grande ramo cortado de árvore.}{ra.ma.lho}{0}
\verb{ramalhudo}{}{}{}{}{adj.}{Que tem muita rama.}{ra.ma.lhu.do}{0}
\verb{ramalhudo}{}{}{}{}{}{Dividido em muitos ramos ou galhos.}{ra.ma.lhu.do}{0}
\verb{ramalhudo}{}{Fig.}{}{}{}{Que tem muitas palavras, muitas frases, mas poucas ideias.}{ra.ma.lhu.do}{0}
\verb{ramalhudo}{}{}{}{}{}{Diz"-se dos olhos que têm grandes pestanas.}{ra.ma.lhu.do}{0}
\verb{ramaria}{}{}{}{}{s.f.}{O conjunto dos ramos de uma planta.}{ra.ma.ri.a}{0}
\verb{rameira}{ê}{}{}{}{s.f.}{Prostituta.}{ra.mei.ra}{0}
\verb{ramela}{é}{}{}{}{}{Var. de \textit{remela}.}{ra.me.la}{0}
\verb{ramerrão}{}{}{"-ões}{}{s.m.}{Repetição monótona, enfadonha.}{ra.mer.rão}{0}
\verb{ramerrão}{}{Por ext.}{"-ões}{}{}{Uso continuado e costumeiro; rotina.}{ra.mer.rão}{0}
\verb{rami}{}{Bot.}{}{}{s.m.}{Planta de folhas grandes e pequenas flores verdes.}{ra.mi}{0}
\verb{rami}{}{}{}{}{}{Fibra dessa planta, usada na indústria de tecidos.}{ra.mi}{0}
\verb{rami}{}{}{}{}{}{Tecido feito com essa fibra.}{ra.mi}{0}
\verb{ramificação}{}{}{"-ões}{}{s.f.}{Ato ou efeito de ramificar.}{ra.mi.fi.ca.ção}{0}
\verb{ramificação}{}{}{"-ões}{}{}{Cada um dos ramos que partem do caule.}{ra.mi.fi.ca.ção}{0}
\verb{ramificação}{}{}{"-ões}{}{}{O conjunto desses ramos.}{ra.mi.fi.ca.ção}{0}
\verb{ramificação}{}{Fig.}{"-ões}{}{}{Propagação, difusão.}{ra.mi.fi.ca.ção}{0}
\verb{ramificar}{}{}{}{}{v.t.}{Dividir em ramos.}{ra.mi.fi.car}{0}
\verb{ramificar}{}{}{}{}{}{Subdividir.}{ra.mi.fi.car}{\verboinum{2}}
\verb{ramilhete}{ê}{}{}{}{}{Var. de \textit{ramalhete}.}{ra.mi.lhe.te}{0}
\verb{ramo}{}{}{}{}{s.m.}{Cada uma das partes que brotam do tronco ou dos galhos de uma planta.}{ra.mo}{0}
\verb{ramo}{}{}{}{}{}{Cada família que se forma a partir do mesmo tronco.}{ra.mo}{0}
\verb{ramo}{}{}{}{}{}{Cada uma das partes de uma ciência, que se diferenciam pelas áreas de atividade.}{ra.mo}{0}
\verb{ramoso}{ô}{}{"-osos ⟨ó⟩}{"-osa ⟨ó⟩}{adj.}{Cheio de ramos.}{ra.mo.so}{0}
\verb{rampa}{}{}{}{}{s.f.}{Plano inclinado.}{ram.pa}{0}
\verb{rampa}{}{}{}{}{}{Ladeira.}{ram.pa}{0}
\verb{ranário}{}{}{}{}{s.m.}{Lugar onde se criam rãs, para fins culinários ou científicos.}{ra.ná.rio}{0}
\verb{rançar}{}{}{}{}{v.i.}{Tornar"-se rançoso; adquirir gosto acre e cheiro desagradável.}{ran.çar}{\verboinum{3}}
\verb{rancharia}{}{}{}{}{s.f.}{Grupo de ranchos ou choupanas.}{ran.cha.ri.a}{0}
\verb{rancheira}{ê}{}{}{}{s.f.}{Mulher proprietária ou moradora de um rancho.}{ran.chei.ra}{0}
\verb{rancheira}{ê}{}{}{}{}{Dança e música popular, de origem argentina.}{ran.chei.ra}{0}
\verb{rancheiro}{ê}{}{}{}{s.m.}{Dono ou morador de um rancho.}{ran.chei.ro}{0}
\verb{rancho}{}{}{}{}{s.m.}{Refeição preparada para muitas pessoas.}{ran.cho}{0}
\verb{rancho}{}{}{}{}{}{Conjunto de alimentos que se compram para um certo período de tempo.}{ran.cho}{0}
\verb{rancho}{}{}{}{}{}{Habitação pequena e pobre; cabana, choupana.}{ran.cho}{0}
\verb{rancho}{}{}{}{}{}{Propriedade rural onde se cria gado; fazenda.}{ran.cho}{0}
\verb{ranço}{}{}{}{}{s.m.}{Mudança que um alimento gorduroso sofre ao ficar exposto ao contato com o ar por muito tempo.}{ran.ço}{0}
\verb{rancor}{ô}{}{}{}{s.m.}{Sentimento de ódio de uma pessoa que não esquece a ofensa que sofreu de outra pessoa.}{ran.cor}{0}
\verb{rancoroso}{ô}{}{"-osos ⟨ó⟩}{"-osa ⟨ó⟩}{adj.}{Cheio de rancor.}{ran.co.ro.so}{0}
\verb{rançoso}{ô}{}{"-osos ⟨ó⟩}{"-osa ⟨ó⟩}{adj.}{Que tem ranço, que apresenta gosto acre e cheiro desagradável.}{ran.ço.so}{0}
\verb{ranger}{ê}{}{}{}{v.i.}{Produzir rangido; chiar, rinchar.}{ran.ger}{\verboinum{12}}
\verb{rangido}{}{}{}{}{s.m.}{Ato ou efeito de ranger.}{ran.gi.do}{0}
\verb{rangido}{}{}{}{}{}{Som produzido pelo atrito de uma coisa que se move sobre outra.}{ran.gi.do}{0}
\verb{rangífer}{}{Zool.}{}{}{s.m.}{Gênero de mamíferos ruminantes do hemisfério boreal, usados como animais de tiro; rena.}{ran.gí.fer}{0}
\verb{rango}{}{Pop.}{}{}{s.m.}{Alimento servido numa refeição; comida, refeição.}{ran.go}{0}
\verb{ranheta}{ê}{Pop.}{}{}{adj.2g.}{Que vive resmungando e reclamando; mal"-humorado, rabugento.}{ra.nhe.ta}{0}
\verb{ranho}{}{}{}{}{s.m.}{Muco  que se acumula nas fossas nasais e escorre das narinas.}{ra.nho}{0}
\verb{ranhura}{}{}{}{}{s.f.}{Sulco longo na espessura da madeira.}{ra.nhu.ra}{0}
\verb{ranhura}{}{}{}{}{}{Entalhe alongado em qualquer superfície.}{ra.nhu.ra}{0}
\verb{rani}{}{}{}{}{s.f.}{Mulher de rajá; rainha ou princesa na Índia.}{ra.ni}{0}
\verb{ranicultor}{ô}{}{}{}{s.m.}{Criador de rãs.}{ra.ni.cul.tor}{0}
\verb{ranicultura}{}{}{}{}{s.f.}{Atividade de criação de rãs.}{ra.ni.cul.tu.ra}{0}
\verb{ranídeo}{}{Zool.}{}{}{s.m.}{Espécime dos ranídeos, família de anfíbios que compreende mais de 500 espécies de rãs.}{ra.ní.deo}{0}
\verb{ranúnculo}{}{Bot.}{}{}{s.m.}{Planta ornamental com flores amarelas, brancas ou vermelhas.}{ra.nún.cu.lo}{0}
\verb{ranzinza}{}{}{}{}{adj.2g.}{Rabugento, mal"-humorado, implicante, teimoso.}{ran.zin.za}{0}
\verb{ranzinza}{}{}{}{}{}{Aborrecido, zangado.}{ran.zin.za}{0}
\verb{ranzinzar}{}{Bras.}{}{}{v.i.}{Tornar"-se ranzinza.}{ran.zin.zar}{\verboinum{1}}
\verb{rap}{}{Mús.}{}{}{s.m.}{Gênero de música popular surgido nas comunidades negras estadunidenses e hoje muito comum na periferia de grandes cidades brasileiras, caracterizado pela poesia narrativa subordinada ao ritmo e forte apelo crítico em relação a questões sociais e problemas vividos pelos miseráveis e excluídos da sociedade.}{\textit{rap}}{0}
\verb{rapa}{}{}{}{}{s.m.}{Certo jogo de dados.}{ra.pa}{0}
\verb{rapa}{}{Pop.}{}{}{}{Glutão, comilão.}{ra.pa}{0}
\verb{rapa}{}{Pop.}{}{}{}{Veículo da prefeitura que fiscaliza vendedores ambulantes nas vias públicas.}{ra.pa}{0}
\verb{rapa}{}{}{}{}{s.f.}{Comida que se acumula e se fixa no fundo da panela; raspa.}{ra.pa}{0}
\verb{rapace}{}{}{}{}{adj.2g.}{Que tem tendência ou hábito de roubar; rapinante.}{ra.pa.ce}{0}
\verb{rapace}{}{Zool.}{}{}{}{Relativo aos rapaces, denominação de uma antiga categoria taxionômica que incluía gaviões e corujas.}{ra.pa.ce}{0}
\verb{rapacidade}{}{}{}{}{s.f.}{Qualidade de rapace; hábito de ou tendência para roubar.}{ra.pa.ci.da.de}{0}
\verb{rapadura}{}{}{}{}{s.f.}{Tijolo de açúcar mascavo aglomerado.}{ra.pa.du.ra}{0}
\verb{rapadura}{}{}{}{}{}{Ato ou efeito de rapar.}{ra.pa.du.ra}{0}
\verb{rapagão}{}{}{"-ões}{}{s.m.}{Rapaz corpulento.}{ra.pa.gão}{0}
\verb{rapapé}{}{}{}{}{s.m.}{Ato de arrastar o pé quando se cumprimenta alguém.}{ra.pa.pé}{0}
\verb{rapapé}{}{}{}{}{}{Cumprimento com gesto exagerado.}{ra.pa.pé}{0}
\verb{rapapé}{}{}{}{}{}{Bajulação, adulação.}{ra.pa.pé}{0}
\verb{rapar}{}{}{}{}{v.t.}{Ralar, desgastar, raspar.}{ra.par}{0}
\verb{rapar}{}{}{}{}{}{Cortar rente pelos ou cabelo.}{ra.par}{0}
\verb{rapar}{}{}{}{}{}{Furtar, roubar.}{ra.par}{\verboinum{1}}
\verb{rapariga}{}{}{}{}{s.f.}{Mulher adolescente.}{ra.pa.ri.ga}{0}
\verb{rapariga}{}{Pop.}{}{}{}{Prostituta.}{ra.pa.ri.ga}{0}
\verb{rapaz}{}{}{}{}{s.m.}{Homem jovem ou adolescente.}{ra.paz}{0}
\verb{rapaziada}{}{}{}{}{s.f.}{Grupo de rapazes.}{ra.pa.zi.a.da}{0}
\verb{rapazola}{ó}{}{}{}{s.m.}{Rapaz adolescente ou muito jovem.}{ra.pa.zo.la}{0}
\verb{rapazola}{ó}{}{}{}{}{Homem com jeito de rapaz.}{ra.pa.zo.la}{0}
\verb{rapazote}{ó}{}{}{}{s.m.}{Homem no começo da adolescência.}{ra.pa.zo.te}{0}
\verb{rapé}{}{}{}{}{s.m.}{Pó feito de tabaco torrado e moído, e eventualmente outras substâncias aromáticas, feito para cheirar e provocar espirros.}{ra.pé}{0}
\verb{rapel}{é}{Esport.}{}{}{s.m.}{Conjunto de técnicas para vencer obstáculos naturais ou artificiais como penhascos e paredões, sendo utilizado em diversas atividades como escaladas, arvorismo, estudos espeleológicos, resgate em montanhas etc. }{ra.pel}{0}
\verb{rapidez}{ê}{}{}{}{s.f.}{Qualidade de rápido; ligeireza.}{ra.pi.dez}{0}
\verb{rapidez}{ê}{}{}{}{}{Brevidade, transitoriedade.}{ra.pi.dez}{0}
\verb{rápido}{}{}{}{}{adj.}{Que se move com velocidade; veloz.}{rá.pi.do}{0}
\verb{rápido}{}{}{}{}{}{Que dura pouco tempo; breve, transitório.}{rá.pi.do}{0}
\verb{rápido}{}{}{}{}{}{Que se dá ou se faz em pouco tempo; ligeiro, instantâneo.}{rá.pi.do}{0}
\verb{rápido}{}{}{}{}{adv.}{Rapidamente, com rapidez.}{rá.pi.do}{0}
\verb{rápido}{}{}{}{}{s.m.}{Indivíduo que entrega mensagens; mensageiro.}{rá.pi.do}{0}
\verb{rapina}{}{}{}{}{s.f.}{Ato ou efeito de rapinar; roubo violento ou astucioso.}{ra.pi.na}{0}
\verb{rapinagem}{}{}{"-ens}{}{s.f.}{Qualidade de rapinante.}{ra.pi.na.gem}{0}
\verb{rapinagem}{}{}{"-ens}{}{}{Hábito de rapinar.}{ra.pi.na.gem}{0}
\verb{rapinagem}{}{}{"-ens}{}{}{Conjunto de roubos.}{ra.pi.na.gem}{0}
\verb{rapinante}{}{}{}{}{adj.2g.}{Que rapina.}{ra.pi.nan.te}{0}
\verb{rapinar}{}{}{}{}{v.t.}{Que rouba de maneira violenta ou astuciosa.}{ra.pi.nar}{\verboinum{1}}
\verb{raposa}{ô}{Zool.}{}{}{s.f.}{Mamífero canídeo de focinho alongado, orelhas pontudas e cauda longa.}{ra.po.sa}{0}
\verb{raposa}{ô}{Bras.}{}{}{}{Gambá.}{ra.po.sa}{0}
\verb{raposa}{ô}{Pop.}{}{}{}{Pessoa esperta, astuta, sagaz.}{ra.po.sa}{0}
\verb{raposice}{}{}{}{}{s.f.}{Astúcia semelhante à desse animal.}{ra.po.si.ce}{0}
\verb{raposino}{}{}{}{}{adj.}{Relativo a raposa.}{ra.po.si.no}{0}
\verb{raposino}{}{Fig.}{}{}{}{Esperto, sagaz, astuto.}{ra.po.si.no}{0}
\verb{raposo}{ô}{Fig.}{}{}{s.m.}{Indivíduo esperto, astuto.}{ra.po.so}{0}
\verb{rapsódia}{}{}{}{}{s.f.}{Na Grécia Antiga, trecho de poema épico recitado por um rapsodo.}{rap.só.dia}{0}
\verb{rapsódia}{}{}{}{}{}{Fragmentos de poema.}{rap.só.dia}{0}
\verb{rapsódia}{}{Mús.}{}{}{}{Peça musical improvisada composta livremente a partir de canções populares tradicionais.}{rap.só.dia}{0}
\verb{rapsódico}{s}{}{}{}{adj.}{Relativo a rapsódia.}{rap.só.di.co}{0}
\verb{rapsodo}{ô}{}{}{}{s.m.}{Na Grécia Antiga, cantor ambulante que recitava trechos de poemas épicos.}{rap.so.do}{0}
\verb{rapsodo}{ô}{Fig.}{}{}{}{Poeta.}{rap.so.do}{0}
\verb{raptar}{}{}{}{}{v.t.}{Cometer rapto; sequestrar.}{rap.tar}{\verboinum{1}}
\verb{rapto}{}{}{}{}{s.m.}{Ato ou efeito de roubar uma pessoa mediante violência ou sedução.}{rap.to}{0}
\verb{rapto}{}{}{}{}{}{Roubo, furto.}{rap.to}{0}
\verb{rapto}{}{Fig.}{}{}{}{Êxtase, arrebatamento.}{rap.to}{0}
\verb{raptor}{ô}{}{}{}{adj.}{Que rapta.}{rap.tor}{0}
\verb{raque}{}{Anat.}{}{}{s.f.}{Coluna vertebral.}{ra.que}{0}
\verb{raqueta}{ê}{Esport.}{}{}{s.f.}{Raquete.}{ra.que.ta}{0}
\verb{raquetada}{}{}{}{}{s.f.}{Golpe dado com uma raquete.}{ra.que.ta.da}{0}
\verb{raquete}{é}{Esport.}{}{}{s.f.}{Instrumento plano, de tela ou de madeira, usado para rebater a bola nos jogos esportivos como tênis e frescobol, dentre outros.}{ra.que.te}{0}
\verb{raquiano}{}{Anat.}{}{}{adj.}{Relativo à raque; raquidiano.}{ra.qui.a.no}{0}
\verb{raquidiano}{}{}{}{}{adj.}{Relativo à raque.}{ra.qui.di.a.no}{0}
\verb{raquítico}{}{}{}{}{adj.}{Que tem raquitismo.}{ra.quí.ti.co}{0}
\verb{raquítico}{}{Pop.}{}{}{}{Pouco desenvolvido; acanhado, franzino.}{ra.quí.ti.co}{0}
\verb{raquitismo}{}{Med.}{}{}{s.m.}{Doença caracterizada pela mineralização insuficiente dos ossos devido à carência de vitamina \textsc{d}.}{ra.qui.tis.mo}{0}
\verb{rarear}{}{}{}{}{v.t.}{Tornar raro.}{ra.re.ar}{0}
\verb{rarear}{}{}{}{}{}{Tornar"-se menos denso; rarefazer"-se.}{ra.re.ar}{\verboinum{4}}
\verb{rarefação}{}{}{"-ões}{}{s.f.}{Ato ou efeito de rarefazer.}{ra.re.fa.ção}{0}
\verb{rarefazer}{ê}{}{}{}{v.t.}{Tornar menos denso ou espesso.}{ra.re.fa.zer}{\verboinum{12}}
\verb{rarefeito}{ê}{}{}{}{adj.}{Que se rarefez; pouco denso.}{ra.re.fei.to}{0}
\verb{raridade}{}{}{}{}{s.f.}{Qualidade de raro.}{ra.ri.da.de}{0}
\verb{raridade}{}{}{}{}{}{Objeto ou evento raro, pouco comum ou pouco frequente.}{ra.ri.da.de}{0}
\verb{raro}{}{}{}{}{adj.}{Que existe em pouca quantidade.}{ra.ro}{0}
\verb{raro}{}{}{}{}{}{Pouco frequente; incomum.}{ra.ro}{0}
\verb{raro}{}{}{}{}{}{Pouco denso ou espesso; ralo.}{ra.ro}{0}
\verb{raro}{}{}{}{}{adv.}{Raramente; poucas vezes.}{ra.ro}{0}
\verb{rasa}{}{}{}{}{s.f.}{Antiga medida de capacidade muito usada para cereais, farináceos, sólidos.}{ra.sa}{0}
\verb{rasa}{}{}{}{}{}{O preço mais baixo.}{ra.sa}{0}
\verb{rasante}{}{}{}{}{adj.2g.}{De pouca altitude, próximo ao solo.}{ra.san.te}{0}
\verb{rasar}{}{}{}{}{v.t.}{Medir com rasa.}{ra.sar}{0}
\verb{rasar}{}{}{}{}{}{Tornar raso.}{ra.sar}{0}
\verb{rasar}{}{}{}{}{}{Igualar, nivelar.}{ra.sar}{0}
\verb{rasar}{}{}{}{}{}{Tocar levemente; roçar.}{ra.sar}{\verboinum{1}}
\verb{rascante}{}{}{}{}{adj.2g.}{Diz"-se de alimento ou bebida que deixa travo na garganta; adstringente.}{ras.can.te}{0}
\verb{rascante}{}{}{}{}{}{De som áspero.}{ras.can.te}{0}
\verb{rascar}{}{}{}{}{v.t.}{Raspar, desbastar, rapar.}{ras.car}{0}
\verb{rascar}{}{}{}{}{}{Lascar.}{ras.car}{0}
\verb{rascar}{}{}{}{}{}{Perturbar, incomodar.}{ras.car}{0}
\verb{rascar}{}{}{}{}{}{Arranhar.}{ras.car}{0}
\verb{rascar}{}{}{}{}{v.i.}{Deixar travo na garganta.}{ras.car}{\verboinum{2}}
\verb{rascunhar}{}{}{}{}{v.t.}{Fazer rascunho de.}{ras.cu.nhar}{\verboinum{1}}
\verb{rascunho}{}{}{}{}{s.m.}{Esboço ou primeira versão de qualquer escrito.}{ras.cu.nho}{0}
\verb{rasgado}{}{}{}{}{adj.}{Que se rasgou; que apresenta rasgão.}{ras.ga.do}{0}
\verb{rasgado}{}{Pop.}{}{}{}{Desembaraçado, franco.}{ras.ga.do}{0}
\verb{rasgão}{}{}{"-ões}{}{s.m.}{Ruptura ou abertura em tecido, papel, plástico etc; rasgo, fenda.}{ras.gão}{0}
\verb{rasgar}{}{}{}{}{v.t.}{Abrir um rasgo ou ruptura; fazer em pedaços.}{ras.gar}{0}
\verb{rasgar}{}{}{}{}{}{Abrir sulco ou fenda; cavar.}{ras.gar}{\verboinum{5}}
\verb{rasgo}{}{}{}{}{s.m.}{Ato ou efeito de rasgar; fenda, rasgão, abertura.}{ras.go}{0}
\verb{rasgo}{}{Fig.}{}{}{}{Ímpeto, arroubo, impulso.}{ras.go}{0}
\verb{raso}{}{}{}{}{adj.}{Que tem pouca profundidade; pouco abaixo do nível da superfície.}{ra.so}{0}
\verb{raso}{}{}{}{}{}{Pouco acima do nível; rente, rasteiro.}{ra.so}{0}
\verb{rasoura}{ô}{}{}{}{s.f.}{Pau roliço com que se tira o excesso nas medidas dos terrenos secos.}{ra.sou.ra}{0}
\verb{rasoura}{ô}{}{}{}{}{Instrumento de aço com que se tiram as rebarbas e asperezas de madeira a ser entalhada.}{ra.sou.ra}{0}
\verb{rasourar}{}{}{}{}{v.t.}{Nivelar com a rasoura; debastar.}{ra.sou.rar}{\verboinum{1}}
\verb{raspa}{}{}{}{}{s.f.}{Pequena lasca que se retira de uma superfície que se raspa; apara.}{ras.pa}{0}
\verb{raspadeira}{ê}{}{}{}{s.f.}{Instrumento que se usa para raspar ou limpar superfícies.}{ras.pa.dei.ra}{0}
\verb{raspagem}{}{}{"-ens}{}{s.f.}{Ato ou efeito de raspar.}{ras.pa.gem}{0}
\verb{raspagem}{}{Med.}{"-ens}{}{}{Operação que consiste em raspar com uma cureta o interior de uma cavidade natural; curetagem.}{ras.pa.gem}{0}
\verb{raspança}{}{}{}{}{s.f.}{Raspagem.}{ras.pan.ça}{0}
\verb{raspão}{}{}{"-ões}{}{s.m.}{Ferimento leve causado por atrito; arranhão, escoriação.}{ras.pão}{0}
\verb{raspar}{}{}{}{}{v.t.}{Retirar resíduo de uma superfície com o instrumento adequado.}{ras.par}{0}
\verb{raspar}{}{}{}{}{}{Tocar de leve; ferir de raspão; resvalar.}{ras.par}{0}
\verb{raspar}{}{}{}{}{}{Cortar totalmente os pelos ou cabelo; rapar, tosar.}{ras.par}{0}
\verb{raspar}{}{}{}{}{}{Limpar esfregando ou desbastando.}{ras.par}{\verboinum{1}}
\verb{rastear}{}{}{}{}{v.t.}{Seguir o rasto; rastrear.}{ras.te.ar}{\verboinum{4}}
\verb{rasteira}{ê}{}{}{}{s.f.}{Golpe dado com o pé ou a perna nas pernas de outrem com intenção de derrubá"-lo; cambapé, pernada.}{ras.tei.ra}{0}
\verb{rasteiro}{ê}{}{}{}{adj.}{Que se eleva um pouco acima do nível de uma superfície; rente. (\textit{Naquele terreno o mato crescia rasteiro e ralo. })}{ras.tei.ro}{0}
\verb{rastejador}{ô}{}{}{}{adj.}{Que anda rente ao chão se arrastando.}{ras.te.ja.dor}{0}
\verb{rastejante}{}{}{}{}{adj.2g.}{Que rasteja; rastejador.}{ras.te.jan.te}{0}
\verb{rastejante}{}{Bot.}{}{}{}{Diz"-se da planta cujo caule se desenvolve sobre o solo em vez de crescer verticalmente.}{ras.te.jan.te}{0}
\verb{rastejar}{}{}{}{}{v.i.}{Andar rente ao chão, encostando a barriga. (\textit{O jacaré é um animal que rasteja.})}{ras.te.jar}{0}
\verb{rastejar}{}{}{}{}{v.t.}{Seguir o rastro de; rastrear.}{ras.te.jar}{\verboinum{1}}
\verb{rastelar}{}{}{}{}{v.t.}{Limpar com rastelo.}{ras.te.lar}{\verboinum{1}}
\verb{rastelo}{é}{}{}{}{s.m.}{Ferramenta de cabo longo com travessa dentada usada para juntar folhas secas e para preparar a terra para o plantio; ancinho.}{ras.te.lo}{0}
\verb{rastilho}{}{}{}{}{s.m.}{Fio embebido de substância inflamável ou sulco cheio de pólvora usados para detonar uma bomba.}{ras.ti.lho}{0}
\verb{rasto}{}{}{}{}{s.m.}{Rastro.}{ras.to}{0}
\verb{rastreamento}{}{}{}{}{s.m.}{Ato ou efeito de rastrear; rastreio.}{ras.tre.a.men.to}{0}
\verb{rastreamento}{}{Astron.}{}{}{}{Acompanhamento da trajetória de um satélite ou míssil, por meio de radar, rádio ou fotografia.}{ras.tre.a.men.to}{0}
\verb{rastrear}{}{}{}{}{v.t.}{Seguir rastro ou pista; rastear.}{ras.tre.ar}{0}
\verb{rastrear}{}{}{}{}{}{Limpar a terra com rastrilho.}{ras.tre.ar}{\verboinum{4}}
\verb{rastreio}{ê}{}{}{}{s.m.}{Rastreamento.}{ras.trei.o}{0}
\verb{rastrilho}{}{}{}{}{s.m.}{Ancinho com pontas que espicaçam e limpam a terra.}{ras.tri.lho}{0}
\verb{rastro}{}{}{}{}{s.m.}{Sinal que se deixa no solo ao caminhar; pegada, rasto.}{ras.tro}{0}
\verb{rasura}{}{}{}{}{s.f.}{Ato ou efeito de rasurar; mancha ou emenda no escrito.}{ra.su.ra}{0}
\verb{rasurar}{}{}{}{}{v.t.}{Apagar ou riscar letras de um texto para o alterar ou emendar.}{ra.su.rar}{\verboinum{1}}
\verb{rata}{}{Zool.}{}{}{s.f.}{A fêmea do rato; ratazana.}{ra.ta}{0}
\verb{rata}{}{}{}{}{s.f.}{Ato ou dito impróprio que expõe seu autor ao ridículo; gafe, fiasco.}{ra.ta}{0}
\verb{rataplã}{}{}{}{}{s.m.}{Onomatopeia que reproduz o barulho do tambor.}{ra.ta.plã}{0}
\verb{rataria}{}{}{}{}{s.f.}{Grande quantidade de ratos.}{ra.ta.ri.a}{0}
\verb{ratazana}{}{}{}{}{s.f.}{Qualquer rato ou rata grande.}{ra.ta.za.na}{0}
\verb{ratear}{}{}{}{}{v.t.}{Dividir ou distribuir proporcionalmente.}{ra.te.ar}{\verboinum{4}}
\verb{ratear}{}{}{}{}{v.i.}{Funcionar mal; falhar.}{ra.te.ar}{\verboinum{4}}
\verb{rateio}{ê}{}{}{}{s.m.}{Ato ou efeito de ratear; divisão proporcional.}{ra.tei.o}{0}
\verb{rateio}{ê}{}{}{}{s.m.}{Falha.}{ra.tei.o}{0}
\verb{rateiro}{ê}{}{}{}{adj.}{Diz"-se de cão ou gato que é bom caçador de ratos.}{ra.tei.ro}{0}
\verb{raticida}{}{}{}{}{s.m.}{Substância própria para matar ratos.}{ra.ti.ci.da}{0}
\verb{ratificação}{}{}{"-ões}{}{s.f.}{Ato ou efeito de ratificar; confirmação, validação.}{ra.ti.fi.ca.ção}{0}
\verb{ratificar}{}{}{}{}{v.t.}{Tornar válido; confirmar, autenticar.}{ra.ti.fi.car}{\verboinum{2}}
\verb{ratinhar}{}{}{}{}{v.t.}{Insistir por um preço mais baixo; pechinchar, regatear.}{ra.ti.nhar}{\verboinum{1}}
\verb{rato}{}{Zool.}{}{}{s.m.}{Nome dado a várias espécies de mamíferos roedores de pequeno tamanho.}{ra.to}{0}
\verb{rato}{}{Fig.}{}{}{}{Ladrão, larápio.}{ra.to}{0}
\verb{ratoeira}{ê}{}{}{}{s.f.}{Armadilha para apanhar ratos.}{ra.to.ei.ra}{0}
\verb{ravina}{}{}{}{}{s.f.}{Depressão no solo produzida por enxurradas.}{ra.vi.na}{0}
\verb{ravióli}{}{Cul.}{}{}{s.m.}{Pequeno pastel feito com massa de macarrão recheado de carne moída ou queijo e servido com molho de tomates.}{ra.vi.ó.li}{0}
\verb{razão}{}{}{"-ões}{}{s.f.}{Faculdade de raciocinar, julgar, refletir.}{ra.zão}{0}
\verb{razão}{}{}{"-ões}{}{}{Opinião que corresponde à verdade; tino, bom"-senso.}{ra.zão}{0}
\verb{razão}{}{}{"-ões}{}{}{Causa, motivo, argumento.}{ra.zão}{0}
\verb{razia}{}{}{}{}{s.f.}{Incursão rápida em território inimigo.}{ra.zi.a}{0}
\verb{razia}{}{}{}{}{}{Destruição, devastação, depredação.}{ra.zi.a}{0}
\verb{razoável}{}{}{"-eis}{}{adj.2g.}{De acordo com a razão; racional.}{ra.zo.á.vel}{0}
\verb{razoável}{}{}{"-eis}{}{}{Aceitável, suficiente.}{ra.zo.á.vel}{0}
\verb{Rb}{}{Quím.}{}{}{}{Símb. do \textit{rubídio}.}{Rb}{0}
\verb{Re}{}{Quím.}{}{}{}{Símb. do \textit{rênio}.}{Re}{0}
\verb{ré}{}{Mús.}{}{}{s.m.}{A segunda nota musical na escala de \textit{dó}.}{ré}{0}
\verb{ré}{}{}{}{}{s.f.}{A parte de trás, retaguarda.}{ré}{0}
\verb{ré}{}{}{}{}{}{Marcha do carro que o faz andar para trás.}{ré}{0}
\verb{ré}{}{}{}{}{s.f.}{Fem. de \textit{réu}.}{ré}{0}
\verb{reabastecer}{ê}{}{}{}{v.t.}{Tornar a abastecer.}{re.a.bas.te.cer}{\verboinum{15}}
\verb{reabastecimento}{}{}{}{}{s.m.}{Ato ou efeito de reabastecer.}{re.a.bas.te.ci.men.to}{0}
\verb{reaberto}{é}{}{}{}{adj.}{Que foi novamente aberto.}{re.a.ber.to}{0}
\verb{reabertura}{}{}{}{}{s.f.}{Ato ou efeito de reabrir; reinício.}{re.a.ber.tu.ra}{0}
\verb{reabilitação}{}{}{"-ões}{}{s.f.}{Ato ou efeito de reabilitar; recapacitação, recuperação.}{re.a.bi.li.ta.ção}{0}
\verb{reabilitar}{}{}{}{}{v.t.}{Restituir os direitos; recuperar, regenerar.}{re.a.bi.li.tar}{\verboinum{1}}
\verb{reabitar}{}{}{}{}{v.t.}{Voltar a habitar.}{re.a.bi.tar}{\verboinum{1}}
\verb{reabrir}{}{}{}{}{v.t.}{Tornar a abrir.}{re.a.brir}{\verboinum{18}}
\verb{reabsorção}{}{}{"-ões}{}{s.f.}{Ato ou efeito de reabsorver.}{re.ab.sor.ção}{0}
\verb{reabsorver}{ê}{}{}{}{v.t.}{Tornar a absorver.}{re.ab.sor.ver}{\verboinum{12}}
\verb{reação}{}{}{"-ões}{}{s.f.}{Ato ou efeito de reagir; oposição, resistência.}{re.a.ção}{0}
\verb{reação}{}{}{"-ões}{}{}{Atitude que se tem em resposta a um fato.}{re.a.ção}{0}
\verb{reação}{}{Quím.}{"-ões}{}{}{Processo pelo qual substâncias agem sobre as outras, produzindo novas substâncias.}{re.a.ção}{0}
\verb{reacender}{ê}{}{}{}{v.t.}{Acender novamente; reativar.}{re.a.cen.der}{\verboinum{12}}
\verb{reacionário}{}{}{}{}{adj.}{Que se opõe ao progresso e à liberdade; retrógrado.}{re.a.ci.o.ná.rio}{0}
\verb{reacionário}{}{}{}{}{}{Diz"-se do indivíduo, partido ou movimento contrário a quaisquer mudanças no campo das atividades humanas.}{re.a.ci.o.ná.rio}{0}
\verb{readaptação}{}{}{"-ões}{}{s.f.}{Ato ou efeito de readaptar; reajuste.}{re.a.dap.ta.ção}{0}
\verb{readaptar}{}{}{}{}{v.t.}{Adaptar a uma nova situação; reajustar.}{re.a.dap.tar}{\verboinum{1}}
\verb{readmissão}{}{}{"-ões}{}{s.f.}{Ato ou efeito de readmitir.}{re.ad.mis.são}{0}
\verb{readmitir}{}{}{}{}{v.t.}{Tornar a admitir; reconsiderar.}{re.ad.mi.tir}{\verboinum{18}}
\verb{readquirir}{}{}{}{}{v.t.}{Tornar a adquirir; recobrar. (\textit{Após o susto, sua face readquiriu a cor.})}{re.ad.qui.rir}{\verboinum{18}}
\verb{reafirmar}{}{}{}{}{v.t.}{Tornar a afirmar; confirmar.}{re.a.fir.mar}{\verboinum{1}}
\verb{reagente}{}{}{}{}{adj.2g.}{Que reage, que tem reação; que participa em reações.}{re.a.gen.te}{0}
\verb{reagente}{}{Quím.}{}{}{s.m.}{Substância que provoca reação química e serve nas análises para reconhecer as substâncias simples ou compostas que entram na composição do material estudado; reativo.}{re.a.gen.te}{0}
\verb{reagir}{}{}{}{}{v.t.}{Opor uma ação a outra; resistir, lutar.}{re.a.gir}{\verboinum{22}}
\verb{reagrupar}{}{}{}{}{v.t.}{Tornar a reunir em grupo.}{re.a.gru.par}{\verboinum{1}}
\verb{reajustamento}{}{}{}{}{s.m.}{Ato ou efeito de reajustar; reajuste.}{re.a.jus.ta.men.to}{0}
\verb{reajustar}{}{}{}{}{v.t.}{Tornar a ajustar; readaptar.}{re.a.jus.tar}{0}
\verb{reajustar}{}{Bras.}{}{}{}{Adequar o vencimento ao custo de vida. (\textit{O governo irá reajustar o salário mínimo em maio.})}{re.a.jus.tar}{\verboinum{1}}
\verb{reajuste}{}{}{}{}{s.m.}{Ato ou efeito de reajustar; reajustamento.}{re.a.jus.te}{0}
\verb{real}{}{}{"-ais}{}{adj.2g.}{Que existiu ou existe de fato; verdadeiro. (\textit{O sonho que tive essa noite pareceu tão real!})}{re.al}{0}
\verb{real}{}{}{"-ais}{}{adj.2g.}{Relativo ao rei; régio.}{re.al}{0}
\verb{real}{}{}{"-ais}{}{s.m.}{Unidade monetária do Brasil, em vigor desde julho de 1994.}{re.al}{0}
\verb{realçar}{}{}{}{}{v.t.}{Fazer sobressair; salientar, destacar.}{re.al.çar}{\verboinum{3}}
\verb{realce}{}{}{}{}{s.m.}{Ato ou efeito de realçar; destaque, relevo.}{re.al.ce}{0}
\verb{realejo}{ê}{Mús.}{}{}{s.m.}{Instrumento musical semelhante a um órgão portátil cujo teclado é acionado por manivela.}{re.a.le.jo}{0}
\verb{realeza}{ê}{}{}{}{s.f.}{Dignidade de rei ou rainha.}{re.a.le.za}{0}
\verb{realeza}{ê}{}{}{}{}{Magnificência, grandeza.}{re.a.le.za}{0}
\verb{realidade}{}{}{}{}{s.f.}{Qualidade do que é real, do que existe efetivamente.}{re.a.li.da.de}{0}
\verb{realidade}{}{}{}{}{}{O conjunto das coisas e fatos reais.}{re.a.li.da.de}{0}
\verb{realimentação}{}{}{"-ões}{}{s.f.}{Ato de retornar automaticamente a informação processada ao ponto inicial; retroalimentação.}{re.a.li.men.ta.ção}{0}
\verb{realismo}{}{}{}{}{s.m.}{Qualidade ou estado do que é real.}{re.a.lis.mo}{0}
\verb{realismo}{}{}{}{}{}{Atitude de quem encara a realidade e a avalia com justeza; pragmatismo.}{re.a.lis.mo}{0}
\verb{realismo}{}{Liter.}{}{}{}{Escola artístico"-literária do final do século \textsc{xix} que se opunha ao Romantismo e almejava a apresentação da realidade sem idealizações. (Usa"-se inicial maiúscula nesta acepção.)}{re.a.lis.mo}{0}
\verb{realista}{}{}{}{}{adj.2g.}{Relativo ao realismo.}{re.a.lis.ta}{0}
\verb{realista}{}{}{}{}{}{Que tem espírito prático.}{re.a.lis.ta}{0}
\verb{realista}{}{}{}{}{s.2g.}{Artista partidário do Realismo.}{re.a.lis.ta}{0}
\verb{realização}{}{}{"-ões}{}{s.f.}{Ato ou efeito de realizar, de pôr em prática.}{re.a.li.za.ção}{0}
\verb{realizador}{ô}{}{}{}{adj.}{Que realiza, põe em prática; empreendedor.}{re.a.li.za.dor}{0}
\verb{realizar}{}{}{}{}{v.t.}{Tornar real; efetivar, executar.}{re.a.li.zar}{0}
\verb{realizar}{}{}{}{}{v.pron.}{Suceder, acontecer, efetuar"-se.}{re.a.li.zar}{0}
\verb{realizar}{}{}{}{}{}{Atingir os objetivos almejados.}{re.a.li.zar}{\verboinum{1}}
\verb{realizável}{}{}{"-eis}{}{adj.2g.}{Que pode ser realizado; executável.}{re.a.li.zá.vel}{0}
\verb{reanimação}{}{}{"-ões}{}{s.f.}{Ato ou efeito de reanimar.}{re.a.ni.ma.ção}{0}
\verb{reanimador}{ô}{}{}{}{adj.}{Que reanima; estimulante.}{re.a.ni.ma.dor}{0}
\verb{reanimar}{}{}{}{}{v.t.}{Dar ânimo novo; revigorar.}{re.a.ni.mar}{0}
\verb{reanimar}{}{}{}{}{}{Restituir à vida; fazer reviver.}{re.a.ni.mar}{\verboinum{1}}
\verb{reaparecer}{ê}{}{}{}{v.t.}{Tornar a aparecer; ressurgir.}{re.a.pa.re.cer}{\verboinum{15}}
\verb{reaparecimento}{}{}{}{}{s.m.}{Reaparição.}{re.a.pa.re.ci.men.to}{0}
\verb{reaparição}{}{}{"-ões}{}{s.f.}{Ato ou efeito de reaparecer; reaparecimento, ressurgimento.}{re.a.pa.ri.ção}{0}
\verb{reapresentar}{}{}{}{}{v.t.}{Tornar a apresentar.}{re.a.pre.sen.tar}{\verboinum{1}}
\verb{reaproveitar}{}{}{}{}{v.t.}{Voltar a aproveitar.}{re.a.pro.vei.tar}{\verboinum{1}}
\verb{reaproximar}{s}{}{}{}{v.t.}{Tornar a aproximar; restabelecer relações; reconciliar.}{re.a.pro.xi.mar}{\verboinum{1}}
\verb{reaquisição}{}{}{"-ões}{}{s.f.}{Ato ou efeito de readquirir; retomada, recuperação.}{re.a.qui.si.ção}{0}
\verb{reascender}{ê}{}{}{}{v.t.}{Tornar a ascender; fazer subir novamente.}{re.as.cen.der}{\verboinum{12}}
\verb{reassumir}{}{}{}{}{v.t.}{Tornar a assumir; readquirir.}{re.as.su.mir}{\verboinum{18}}
\verb{reassunção}{}{}{"-ões}{}{s.f.}{Ato ou efeito de reassumir.}{re.as.sun.ção}{0}
\verb{reatar}{}{}{}{}{v.t.}{Atar novamente; restabelecer.}{re.a.tar}{\verboinum{1}}
\verb{reativar}{}{}{}{}{v.t.}{Voltar a ativar.}{re.a.ti.var}{\verboinum{1}}
\verb{reatividade}{}{Quím.}{}{}{s.f.}{Capacidade de participar em reações químicas; propriedade do que é reativo ou reagente.}{re.a.ti.vi.da.de}{0}
\verb{reativo}{}{}{}{}{adj.}{Que reage; próprio para reagir. }{re.a.ti.vo}{0}
\verb{reativo}{}{Quím.}{}{}{s.m.}{Substância que participa em reação; reagente. }{re.a.ti.vo}{0}
\verb{reator}{ô}{}{}{}{adj.}{Que reage.}{re.a.tor}{0}
\verb{reator}{ô}{}{}{}{s.m.}{Motor propulsor que se utiliza do ar ambiente como fonte de energia térmica de combustão para transformá"-la em energia cinética por expansão.}{re.a.tor}{0}
\verb{reavaliar}{}{}{}{}{v.t.}{Fazer nova avaliação; reconsiderar.}{re.a.va.li.ar}{\verboinum{1}}
\verb{reaver}{ê}{}{}{}{v.t.}{Tornar a haver; recuperar, recobrar. (\textit{Os policiais conseguiram reaver todas as joias roubadas do museu.})}{re.a.ver}{\verboinum{40}}
\verb{reavivar}{}{}{}{}{v.t.}{Dar novo alento; estimular, reacender.}{re.a.vi.var}{0}
\verb{reavivar}{}{}{}{}{}{Tornar bem lembrado; recordar.}{re.a.vi.var}{\verboinum{1}}
\verb{rebaixamento}{ch}{}{}{}{s.m.}{Ato ou efeito de rebaixar.}{re.bai.xa.men.to}{0}
\verb{rebaixamento}{ch}{}{}{}{}{Diminuição da altura; abaixamento. (\textit{A prefeitura providenciou o rebaixamento da guia daquela calçada.})}{re.bai.xa.men.to}{0}
\verb{rebaixar}{ch}{}{}{}{v.t.}{Tornar mais baixo.}{re.bai.xar}{0}
\verb{rebaixar}{ch}{}{}{}{}{Humilhar, vexar, aviltar.}{re.bai.xar}{0}
\verb{rebaixar}{ch}{}{}{}{}{Punir dando uma posição mais baixa.}{re.bai.xar}{\verboinum{1}}
\verb{rebanho}{}{}{}{}{s.m.}{Conjunto de animais da mesma espécie criados pelo homem para corte ou para extração de pelo.}{re.ba.nho}{0}
\verb{rebanho}{}{Fig.}{}{}{}{Conjunto dos paroquianos sob a orientação de um pastor, vigário etc.}{re.ba.nho}{0}
\verb{rebarba}{}{}{}{}{s.f.}{Saliência cheia de ângulos resultante da fundição de materiais; aresta.}{re.bar.ba}{0}
\verb{rebarbativo}{}{}{}{}{adj.}{Que causa estranheza por destoar do comum; desagradável, irritante.}{re.bar.ba.ti.vo}{0}
\verb{rebate}{}{}{}{}{s.m.}{Ato ou efeito de rebater; ataque.}{re.ba.te}{0}
\verb{rebate}{}{}{}{}{}{Ato de chamar ou avisar em caso de perigo; alarme.}{re.ba.te}{0}
\verb{rebatedor}{ô}{}{}{}{adj.}{Que rebate, rechaça.}{re.ba.te.dor}{0}
\verb{rebatedor}{ô}{Esport.}{}{}{}{Jogador que rebate a bola no beisebol. }{re.ba.te.dor}{0}
\verb{rebater}{ê}{}{}{}{v.t.}{Tornar a bater.}{re.ba.ter}{0}
\verb{rebater}{ê}{}{}{}{}{Aparar um golpe; rechaçar, repelir.}{re.ba.ter}{0}
\verb{rebater}{ê}{}{}{}{}{Refutar, contestar, desmentir.}{re.ba.ter}{\verboinum{12}}
\verb{rebelar}{}{}{}{}{v.t.}{Incitar à rebeldia; revoltar, insurgir.}{re.be.lar}{\verboinum{1}}
\verb{rebelde}{é}{}{}{}{adj.2g.}{Que se rebela contra a autoridade constituída; revoltoso, insurreto.}{re.bel.de}{0}
\verb{rebelde}{é}{}{}{}{}{Que não acata ordem ou disciplina; teimoso, indisciplinado.}{re.bel.de}{0}
\verb{rebelde}{é}{}{}{}{}{Diz"-se de doença ou sintoma que custa a ceder, difícil de curar.}{re.bel.de}{0}
\verb{rebeldia}{}{}{}{}{s.f.}{Estado ou qualidade de rebelde; resistência, teimosia.}{re.bel.di.a}{0}
\verb{rebelião}{}{}{"-ões}{}{s.f.}{Ato ou efeito de rebelar"-se; revolta, insurreição.}{re.be.li.ão}{0}
\verb{rebenque}{}{}{}{}{s.m.}{Pequeno chicote de couro, usado para tocar a montaria.}{re.ben.que}{0}
\verb{rebentação}{}{}{"-ões}{}{s.f.}{Ato ou efeito de rebentar; arrebentação.}{re.ben.ta.ção}{0}
\verb{rebentação}{}{}{"-ões}{}{}{Ato ou efeito de quebrar"-se as ondas do mar contra as rochas ou as embarcações. (\textit{Não se deve mergulhar no mar perto da rebentação.})}{re.ben.ta.ção}{0}
\verb{rebentar}{}{}{}{}{v.t.}{Estourar; explodir; arrebentar.}{re.ben.tar}{\verboinum{1}}
\verb{rebento}{}{}{}{}{s.m.}{Broto vegetal; botão, gema.}{re.ben.to}{0}
\verb{rebento}{}{Fig.}{}{}{}{Filho, descendência.}{re.ben.to}{0}
\verb{rebitar}{}{}{}{}{v.t.}{Unir com rebites.}{re.bi.tar}{0}
\verb{rebitar}{}{}{}{}{}{Virar a ponta de algo para cima; arrebitar.}{re.bi.tar}{\verboinum{1}}
\verb{rebite}{}{}{}{}{s.m.}{Parafuso ou prego de duas cabeças que liga chapas de metal. }{re.bi.te}{0}
\verb{rebite}{}{}{}{}{}{Volta que se dá na ponta do prego para que não saia da madeira.}{re.bi.te}{0}
\verb{reboar}{}{}{}{}{v.i.}{Ressoar com força; ecoar, retumbar.}{re.bo.ar}{\verboinum{7}}
\verb{rebobinar}{}{}{}{}{v.t.}{Enrolar novamente alguma coisa em uma bobina ou carretel.}{re.bo.bi.nar}{0}
\verb{rebobinar}{}{}{}{}{}{Enrolar novamente filme ou fita na bobina para voltar ao começo. }{re.bo.bi.nar}{\verboinum{1}}
\verb{rebocador}{ô}{}{}{}{adj.}{Que reboca.}{re.bo.ca.dor}{0}
\verb{rebocador}{ô}{}{}{}{s.m.}{Embarcação, em geral pequena e potente, destinada a puxar ou empurrar outras embarcações. }{re.bo.ca.dor}{0}
\verb{rebocador}{ô}{}{}{}{adj.}{Que cobre superfície com reboco.}{re.bo.ca.dor}{0}
\verb{rebocar}{}{}{}{}{v.t.}{Puxar embarcação ou veículo, com corda, cabo, corrente etc., para levá"-lo a outro lugar ou auxiliá"-lo em manobras.}{re.bo.car}{0}
\verb{rebocar}{}{}{}{}{v.t.}{Revestir, cobrir superfície com reboco.}{re.bo.car}{\verboinum{2}}
\verb{reboco}{ô}{}{}{}{s.m.}{Massa de areia e cal usada para revestir paredes.}{re.bo.co}{0}
\verb{rebojo}{ô}{}{}{}{s.m.}{Repercussão, desvio ou redemoinho de vento ou de água.}{re.bo.jo}{0}
\verb{rebojo}{ô}{}{}{}{}{Sorvedouro que se forma nos rios.}{re.bo.jo}{0}
\verb{rebolado}{}{}{}{}{s.m.}{Movimento sinuoso feito com os quadris; requebro, ginga.}{re.bo.la.do}{0}
\verb{rebolar}{}{}{}{}{v.t.}{Fazer mover sobre si.}{re.bo.lar}{0}
\verb{rebolar}{}{}{}{}{}{Mover os quadris; gingar, requebrar, saracotear.}{re.bo.lar}{\verboinum{1}}
\verb{reboliço}{}{}{}{}{adj.}{Que tem forma de rebolo.}{re.bo.li.ço}{0}
\verb{rebolo}{ô}{}{}{}{s.m.}{Roda de pedra fixada em um eixo giratório na qual se afiam ferramentas de corte.}{re.bo.lo}{0}
\verb{reboo}{}{}{}{}{s.m.}{Ato ou efeito de reboar; estrondo.}{re.bo.o}{0}
\verb{reboque}{ó}{}{}{}{s.m.}{Corda que prende um veículo a outro que o puxa.}{re.bo.que}{0}
\verb{reboque}{ó}{}{}{}{}{Veículo sem motor que se move quando rebocado.}{re.bo.que}{0}
\verb{reboque}{ó}{}{}{}{}{Veículo com guindaste, próprio para ligar"-se a outro para rebocá"-lo.}{re.bo.que}{0}
\verb{rebordo}{ô}{}{}{}{s.m.}{Borda virada ou voltada para fora.}{re.bor.do}{0}
\verb{rebordosa}{ó}{}{}{}{s.f.}{Repreensão severa; censura, admoestação.}{re.bor.do.sa}{0}
\verb{rebordosa}{ó}{}{}{}{}{Situação difícil, desagradável; alvoroço, confusão.}{re.bor.do.sa}{0}
\verb{rebordosa}{ó}{}{}{}{}{Doença grave.}{re.bor.do.sa}{0}
\verb{rebotalho}{}{}{}{}{s.m.}{Coisa que não presta mais; restos inúteis; refugo.}{re.bo.ta.lho}{0}
\verb{rebote}{ó}{Esport.}{}{}{s.m.}{No futebol, retorno da bola após ser rebatida.}{re.bo.te}{0}
\verb{rebote}{ó}{Esport.}{}{}{}{No basquete, bola que bate na tabela e é disputada por atacantes e defensores.}{re.bo.te}{0}
\verb{rebrilhante}{}{}{}{}{adj.2g.}{Que rebrilha; resplandecente, esplendoroso.}{re.bri.lhan.te}{0}
\verb{rebrilhar}{}{}{}{}{v.i.}{Brilhar de novo com mais intensidade; resplandecer.}{re.bri.lhar}{\verboinum{1}}
\verb{rebu}{}{Pop.}{}{}{s.m.}{Rebuliço, desordem, gritaria.}{re.bu}{0}
\verb{rebuçado}{}{}{}{}{s.m.}{Bala de chupar feita com calda de açúcar endurecida à qual se acrescentam corantes e essências de vários sabores.}{re.bu.ça.do}{0}
\verb{rebuçar}{}{}{}{}{v.t.}{Cobrir com rebuço.}{re.bu.çar}{0}
\verb{rebuçar}{}{Fig.}{}{}{}{Disfarçar, dissimular.}{re.bu.çar}{\verboinum{3}}
\verb{rebuço}{}{}{}{}{s.m.}{Parte da capa que cobre o rosto; gola, lapela.}{re.bu.ço}{0}
\verb{rebuço}{}{Fig.}{}{}{}{Disfarce, dissimulação.}{re.bu.ço}{0}
\verb{rebuliço}{}{}{}{}{s.m.}{Grande agitação; desordem, confusão.}{re.bu.li.ço}{0}
\verb{rebuscado}{}{}{}{}{adj.}{Que se tornou a buscar.}{re.bus.ca.do}{0}
\verb{rebuscado}{}{}{}{}{}{Diz"-se do estilo pretensioso, empolado.}{re.bus.ca.do}{0}
\verb{rebuscar}{}{}{}{}{v.t.}{Tornar a buscar; procurar insistentemente.}{re.bus.car}{0}
\verb{rebuscar}{}{}{}{}{}{Abusar de um estilo requintado; exagerar.}{re.bus.car}{\verboinum{2}}
\verb{recado}{}{}{}{}{s.m.}{Mensagem, oral ou escrita, que uma pessoa recebe para passar a outra; aviso. (\textit{Uma amiga pediu para eu transmitir um recado a meu irmão.})}{re.ca.do}{0}
\verb{recaída}{}{}{}{}{s.f.}{Ato ou efeito de recair.}{re.ca.í.da}{0}
\verb{recaída}{}{Med.}{}{}{}{Reaparecimento dos sintomas de uma doença, antes de sua cura completa.}{re.ca.í.da}{0}
\verb{recair}{}{}{}{}{v.i.}{Tornar a cair.}{re.ca.ir}{0}
\verb{recair}{}{}{}{}{}{Ser novamente atacado por uma doença; ter uma recaída.}{re.ca.ir}{0}
\verb{recair}{}{}{}{}{}{Reincidir em culpa ou erro.}{re.ca.ir}{\verboinum{19}}
\verb{recalcado}{}{}{}{}{adj.}{Calcado muitas vezes; repisado.}{re.cal.ca.do}{0}
\verb{recalcado}{}{}{}{}{}{Que sofre de recalques; reprimido, contido.}{re.cal.ca.do}{0}
\verb{recalcar}{}{}{}{}{v.t.}{Calcar muitas vezes; repisar.}{re.cal.car}{0}
\verb{recalcar}{}{}{}{}{}{Conter, reprimir, refrear.}{re.cal.car}{\verboinum{2}}
\verb{recalcitrante}{}{}{}{}{adj.2g.}{Que recalcitra; desobediente, teimoso.}{re.cal.ci.tran.te}{0}
\verb{recalcitrar}{}{}{}{}{v.i.}{Teimar em desobedecer; obstinar"-se, resistir.}{re.cal.ci.trar}{\verboinum{1}}
\verb{recalque}{}{}{}{}{s.m.}{Ato ou efeito de recalcar.}{re.cal.que}{0}
\verb{recalque}{}{Psicol.}{}{}{}{Processo psíquico que reprime desejos e impulsos, passando"-os do campo do consciente para o inconsciente e causando, muitas vezes, graves distúrbios.}{re.cal.que}{0}
\verb{recamar}{}{}{}{}{v.t.}{Cobrir com recamo, adorno; revestir, enfeitar.}{re.ca.mar}{\verboinum{1}}
\verb{recambiar}{}{}{}{}{v.t.}{Fazer voltar ao lugar de origem; devolver.}{re.cam.bi.ar}{\verboinum{1}}
\verb{recamo}{}{}{}{}{s.m.}{Aquilo com que se enfeita; adorno, ornato.}{re.ca.mo}{0}
\verb{recanto}{}{}{}{}{s.m.}{Lugar escondido ou retirado; retiro.}{re.can.to}{0}
\verb{recapar}{}{}{}{}{v.t.}{Recauchutar.}{re.ca.par}{\verboinum{1}}
\verb{recapeamento}{}{}{}{}{s.m.}{Ato ou efeito de recapear; asfaltamento.}{re.ca.pe.a.men.to}{0}
\verb{recapear}{}{}{}{}{v.t.}{Revestir via pública com nova camada de asfalto.}{re.ca.pe.ar}{\verboinum{4}}
\verb{recapitulação}{}{}{"-ões}{}{s.f.}{Ato ou efeito de recapitular; resumo, repetição.}{re.ca.pi.tu.la.ção}{0}
\verb{recapitular}{}{}{}{}{v.t.}{Repetir os pontos principais de um assunto de forma resumida. (\textit{Na véspera da prova, o professor recapitulou a matéria.})}{re.ca.pi.tu.lar}{\verboinum{1}}
\verb{recapturar}{}{}{}{}{v.t.}{Tornar a capturar.}{re.cap.tu.rar}{\verboinum{1}}
\verb{recarga}{}{}{}{}{s.f.}{Ato ou efeito de recarregar. (\textit{Na compra daquele celular, o cliente ganha uma recarga de créditos no mesmo mês.})}{re.car.ga}{0}
\verb{recarga}{}{}{}{}{}{Segunda investida de um ataque; réplica.}{re.car.ga}{0}
\verb{recarregar}{}{}{}{}{v.t.}{Tornar a carregar. (\textit{Meu pai precisou recarregar a bateria do celular.})}{re.car.re.gar}{\verboinum{5}}
\verb{recatado}{}{}{}{}{adj.}{Que tem recato; modesto, comedido.}{re.ca.ta.do}{0}
\verb{recatar}{}{}{}{}{v.t.}{Guardar com recato; resguardar.}{re.ca.tar}{\verboinum{1}}
\verb{recato}{}{}{}{}{s.m.}{Resguardo, pudor, modéstia.}{re.ca.to}{0}
\verb{recauchutado}{}{}{}{}{adj.}{Diz"-se do pneu que recebeu nova camada de borracha.}{re.cau.chu.ta.do}{0}
\verb{recauchutagem}{}{}{"-ens}{}{s.f.}{Ato ou efeito de recauchutar, recapar.}{re.cau.chu.ta.gem}{0}
\verb{recauchutar}{}{}{}{}{v.t.}{Aplicar nova camada de borracha ao pneu para recuperá"-lo; recapar.}{re.cau.chu.tar}{\verboinum{1}}
\verb{recear}{}{}{}{}{v.t.}{Sentir receio; temer.}{re.ce.ar}{\verboinum{4}}
\verb{recebedor}{ô}{}{}{}{adj.}{Que recebe alguma coisa; receptor.}{re.ce.be.dor}{0}
\verb{recebedor}{ô}{}{}{}{}{Indivíduo encarregado de receber e arrecadar impostos.}{re.ce.be.dor}{0}
\verb{recebedoria}{}{}{}{}{s.f.}{Repartição onde se recebem impostos.}{re.ce.be.do.ri.a}{0}
\verb{recebedoria}{}{}{}{}{}{Cargo de recebedor.}{re.ce.be.do.ri.a}{0}
\verb{receber}{ê}{}{}{}{v.t.}{Passar a ter alguma coisa que veio de outra pessoa.}{re.ce.ber}{0}
\verb{receber}{ê}{}{}{}{}{Deixar uma pessoa entrar em casa; acolher, hospedar.}{re.ce.ber}{0}
\verb{receber}{ê}{}{}{}{}{Passar a ter pagamento.}{re.ce.ber}{0}
\verb{receber}{ê}{}{}{}{}{Ser atingido por alguma coisa; levar.}{re.ce.ber}{\verboinum{12}}
\verb{recebimento}{}{}{}{}{s.m.}{Ato ou efeito de receber.}{re.ce.bi.men.to}{0}
\verb{receio}{ê}{}{}{}{s.m.}{Apreensão diante de incerteza; medo.}{re.cei.o}{0}
\verb{receita}{ê}{}{}{}{s.f.}{Dinheiro que uma organização recebe.}{re.cei.ta}{0}
\verb{receita}{ê}{}{}{}{}{Texto em que se explica a maneira de se preparar comida ou remédio.}{re.cei.ta}{0}
\verb{receita}{ê}{}{}{}{}{Texto com o nome dos remédios que o médico recomenda para os doentes.}{re.cei.ta}{0}
\verb{receita}{ê}{}{}{}{}{Folha de papel em que aparece qualquer um desses textos.}{re.cei.ta}{0}
\verb{receitar}{}{}{}{}{v.t.}{Indicar o remédio que o doente deve tomar.}{re.cei.tar}{\verboinum{1}}
\verb{receituário}{}{}{}{}{s.m.}{Formulário que o médico usa para receitar.}{re.cei.tu.á.rio}{0}
\verb{recém"-casado}{}{}{recém"-casados}{}{adj.}{Que se casou há pouco tempo.}{re.cém"-ca.sa.do}{0}
\verb{recém"-chegado}{}{}{recém"-chegados}{}{adj.}{Que chegou há pouco tempo.}{re.cém"-che.ga.do}{0}
\verb{recém"-nascido}{}{}{recém"-nascidos}{}{adj.}{Que nasceu há pouco tempo.}{re.cém"-nas.ci.do}{0}
%\verb{}{}{}{}{}{}{}{}{0}
%\verb{}{}{}{}{}{}{}{}{0}
\verb{recendente}{}{}{}{}{adj.2g.}{Que recende, exala; aromático.}{re.cen.den.te}{0}
\verb{recender}{ê}{}{}{}{v.t.}{Exalar cheiro forte.}{re.cen.der}{\verboinum{12}}
\verb{recensão}{}{}{"-ões}{}{s.f.}{Censo de pessoas ou de animais; recenseamento.}{re.cen.são}{0}
\verb{recensão}{}{}{"-ões}{}{}{Qualquer lista, rol, catálogo.}{re.cen.são}{0}
\verb{recensão}{}{}{"-ões}{}{}{Resenha.}{re.cen.são}{0}
\verb{recenseador}{ô}{}{}{}{adj.}{Que executa o recenseamento.}{re.cen.se.a.dor}{0}
\verb{recenseamento}{}{}{}{}{s.m.}{Conjunto de dados estatísticos de uma população.}{re.cen.se.a.men.to}{0}
\verb{recensear}{}{}{}{}{v.t.}{Fazer o recenseamento.}{re.cen.se.ar}{\verboinum{4}}
\verb{recente}{}{}{}{}{adj.2g.}{Ocorrido há pouco.}{re.cen.te}{0}
\verb{recente}{}{}{}{}{}{Novo.}{re.cen.te}{0}
\verb{receoso}{ô}{}{"-osos ⟨ó⟩}{"-osa ⟨ó⟩}{adj.}{Que tem receio de alguma coisa; apreensivo, temeroso.}{re.ce.o.so}{0}
\verb{recepção}{}{}{"-ões}{}{s.f.}{Ato de receber; acolhida.}{re.cep.ção}{0}
\verb{recepção}{}{}{"-ões}{}{}{Lugar em uma empresa onde as pessoas são recebidas.}{re.cep.ção}{0}
\verb{recepção}{}{}{"-ões}{}{}{Reunião feita para receber convidados importantes.}{re.cep.ção}{0}
\verb{recepcionar}{}{}{}{}{v.i.}{Promover reunião social.}{re.cep.ci.o.nar}{0}
\verb{recepcionar}{}{}{}{}{}{Receber viajante em estação, aeroporto etc., com certa cerimônia.}{re.cep.ci.o.nar}{\verboinum{1}}
\verb{recepcionista}{}{}{}{}{s.2g.}{Indivíduo encarregado de receber quem chega a um lugar, geralmente a uma empresa.}{re.cep.ci.o.nis.ta}{0}
\verb{receptação}{}{}{"-ões}{}{s.f.}{Ato ou efeito de receptar.}{re.cep.ta.ção}{0}
\verb{receptação}{}{}{"-ões}{}{}{O ato de comprar, receber ou ocultar, em proveito próprio ou alheio, produto de crime.}{re.cep.ta.ção}{0}
\verb{receptáculo}{}{}{}{}{s.m.}{Objeto feito para guardar ou proteger alguma coisa.}{re.cep.tá.cu.lo}{0}
\verb{receptador}{ô}{}{}{}{s.m.}{Que adquire, guarda ou oculta produtos de crime; receptor.}{re.cep.ta.dor}{0}
\verb{receptar}{}{}{}{}{v.t.}{Comprar mercadoria roubada.}{re.cep.tar}{\verboinum{1}}
\verb{receptividade}{}{}{}{}{s.f.}{Faculdade de receber impressões.}{re.cep.ti.vi.da.de}{0}
\verb{receptividade}{}{}{}{}{}{Aceitação, acolhida.}{re.cep.ti.vi.da.de}{0}
\verb{receptivo}{}{}{}{}{adj.}{Aberto a estímulos ou impressões.}{re.cep.ti.vo}{0}
\verb{receptivo}{}{}{}{}{}{Afável.}{re.cep.ti.vo}{0}
\verb{receptor}{ô}{}{}{}{s.m.}{Indivíduo a quem se comunica alguma coisa.}{re.cep.tor}{0}
\verb{receptor}{ô}{}{}{}{}{Aparelho que recebe sinais e os transforma em sons ou imagens.}{re.cep.tor}{0}
\verb{recessão}{}{}{"-ões}{}{s.f.}{Situação em que a produção e as vendas diminuem e muitas pessoas ficam sem emprego.}{re.ces.são}{0}
\verb{recesso}{é}{}{}{}{s.m.}{Lugar retirado.}{re.ces.so}{0}
\verb{recesso}{é}{}{}{}{}{Suspensão de atividade.}{re.ces.so}{0}
\verb{rechaçar}{}{}{}{}{v.t.}{Fazer o inimigo ir embora; fazer retroceder; repelir.}{re.cha.çar}{0}
\verb{rechaçar}{}{}{}{}{}{Deixar de aceitar alguma coisa; recusar.}{re.cha.çar}{\verboinum{3}}
\verb{rechear}{}{}{}{}{v.t.}{Colocar algum alimento dentro de outro alimento.}{re.che.ar}{0}
\verb{rechear}{}{}{}{}{}{Encher bem alguma coisa com outra coisa.}{re.che.ar}{\verboinum{4}}
\verb{recheio}{ê}{}{}{}{s.m.}{O que enche; conteúdo.}{re.chei.o}{0}
\verb{recheio}{ê}{}{}{}{}{Alimento posto dentro de outro.}{re.chei.o}{0}
\verb{rechonchudo}{}{}{}{}{adj.}{Muito gordo; gorducho.}{re.chon.chu.do}{0}
\verb{recibo}{}{}{}{}{s.m.}{Documento em que se declara ter recebido alguma coisa; comprovante.}{re.ci.bo}{0}
\verb{reciclagem}{}{}{"-ens}{}{s.f.}{Reaproveitamento de materiais.}{re.ci.cla.gem}{0}
\verb{reciclagem}{}{}{"-ens}{}{}{Atualização de conhecimentos.}{re.ci.cla.gem}{0}
\verb{reciclar}{}{}{}{}{v.t.}{Fabricar um produto novo com materiais usados.}{re.ci.clar}{0}
\verb{reciclar}{}{}{}{}{}{Levar alguém a ter conhecimentos mais novos sobre determinado assunto; atualizar.}{re.ci.clar}{\verboinum{1}}
\verb{recidiva}{}{}{}{}{s.f.}{Reaparecimento de uma doença ou de um sintoma, após período de cura mais ou menos longo; recorrência.}{re.ci.di.va}{0}
\verb{recidivo}{}{}{}{}{adj.}{Que torna a aparecer ou manifestar"-se; reincidente.}{re.ci.di.vo}{0}
\verb{recife}{}{}{}{}{s.m.}{Rocha que fica perto da costa, pouco acima ou abaixo da superfície do mar.}{re.ci.fe}{0}
\verb{recifense}{}{}{}{}{adj.2g.}{Relativo a Recife, capital de Pernambuco.}{re.ci.fen.se}{0}
\verb{recifense}{}{}{}{}{s.2g.}{Indivíduo natural ou habitante dessa cidade.}{re.ci.fen.se}{0}
\verb{recinto}{}{}{}{}{s.m.}{Espaço delimitado, geralmente fechado.}{re.cin.to}{0}
\verb{recipiente}{}{}{}{}{s.m.}{Objeto feito para guardar alguma coisa.}{re.ci.pi.en.te}{0}
\verb{reciprocidade}{}{}{}{}{s.f.}{Qualidade ou caráter de recíproco; correspondência mútua.}{re.ci.pro.ci.da.de}{0}
\verb{recíproco}{}{}{}{}{adj.}{Que implica troca; mútuo.}{re.cí.pro.co}{0}
\verb{récita}{}{}{}{}{s.f.}{Representação teatral.}{ré.ci.ta}{0}
\verb{récita}{}{}{}{}{}{Recital.}{ré.ci.ta}{0}
\verb{recitado}{}{}{}{}{adj.}{Que se recitou; que se leu alto ou declamou de cor.}{re.ci.ta.do}{0}
%\verb{}{}{}{}{}{}{}{}{0}
\verb{recital}{}{}{"-ais}{}{s.m.}{Apresentação musical ou de declamação.}{re.ci.tal}{0}
\verb{recitar}{}{}{}{}{v.t.}{Ler em voz alta e clara.}{re.ci.tar}{0}
\verb{recitar}{}{}{}{}{}{Pronunciar declamando.}{re.ci.tar}{\verboinum{1}}
\verb{recitativo}{}{}{}{}{s.m.}{Texto ou canto lírico declamado.}{re.ci.ta.ti.vo}{0}
\verb{recitativo}{}{}{}{}{}{Trecho de ópera com esse canto.}{re.ci.ta.ti.vo}{0}
\verb{reclamação}{}{}{"-ões}{}{s.f.}{Ato ou efeito de reclamar; queixa, protesto.}{re.cla.ma.ção}{0}
%\verb{}{}{}{}{}{}{}{}{0}
\verb{reclamar}{}{}{}{}{v.t.}{Mostrar descontentamento contra pessoa ou coisa; protestar, queixar.}{re.cla.mar}{0}
\verb{reclamar}{}{}{}{}{}{Pedir alguma coisa a que se tem direito; exigir, reivindicar.}{re.cla.mar}{\verboinum{1}}
\verb{reclame}{}{}{}{}{s.m.}{Propaganda de um produto; anúncio.}{re.cla.me}{0}
\verb{reclamo}{}{}{}{}{}{Var. de \textit{reclame}.}{re.cla.mo}{0}
\verb{reclassificação}{}{}{"-ões}{}{s.f.}{Nova classificação.}{re.clas.si.fi.ca.ção}{0}
\verb{reclassificar}{}{}{}{}{v.t.}{Classificar de novo.}{re.clas.si.fi.car}{\verboinum{2}}
\verb{reclinar}{}{}{}{}{v.t.}{Deitar alguma coisa para trás.}{re.cli.nar}{0}
\verb{reclinar}{}{}{}{}{}{Deitar alguma coisa para algum lado; curvar, inclinar.}{re.cli.nar}{\verboinum{1}}
\verb{reclusão}{}{}{"-ões}{}{s.f.}{Estado de preso; cativeiro, prisão, cárcere.}{re.clu.são}{0}
\verb{reclusão}{}{}{"-ões}{}{}{Afastamento voluntário do convívio social.}{re.clu.são}{0}
\verb{recluso}{}{}{}{}{adj.}{Diz"-se de quem foi metido em cela, em clausura; enclausurado.}{re.clu.so}{0}
\verb{recluso}{}{}{}{}{}{Diz"-se de quem foi recolhido a convento.}{re.clu.so}{0}
\verb{recluso}{}{}{}{}{}{Diz"-se de quem é afastado do convívio do mundo.}{re.clu.so}{0}
\verb{recobramento}{}{}{}{}{s.m.}{Ato ou efeito de recobrar; recuperação, reaquisição.}{re.co.bra.men.to}{0}
\verb{recobramento}{}{}{}{}{}{Refazimento do ânimo, da disposição.}{re.co.bra.men.to}{0}
\verb{recobrar}{}{}{}{}{v.t.}{Voltar a ter alguma coisa perdida; recuperar, retomar.}{re.co.brar}{0}
\verb{recobrar}{}{}{}{}{v.i.}{Ficar livre de alguma coisa ruim; livrar"-se, recuperar"-se.}{re.co.brar}{\verboinum{1}}
\verb{recobrir}{}{}{}{}{v.t.}{Cobrir de novo pessoa ou coisa.}{re.co.brir}{0}
\verb{recobrir}{}{}{}{}{}{Cobrir alguma coisa por completo.}{re.co.brir}{\verboinum{18}}
\verb{recobro}{ô}{}{}{}{s.m.}{Ato ou efeito de recobrar; recobramento, recuperação.}{re.co.bro}{0}
\verb{recolher}{ê}{}{}{}{v.t.}{Levar pessoa, animal ou coisa para um abrigo.}{re.co.lher}{0}
\verb{recolher}{ê}{}{}{}{}{Retirar alguma coisa de uso.}{re.co.lher}{0}
\verb{recolher}{ê}{}{}{}{}{Reunir coisas de vários lugares.}{re.co.lher}{0}
\verb{recolher}{ê}{}{}{}{v.pron.}{Ir para o quarto; deitar"-se.}{re.co.lher}{\verboinum{12}}
\verb{recolhido}{}{}{}{}{adj.}{Que se recolheu.}{re.co.lhi.do}{0}
\verb{recolhido}{}{}{}{}{}{Afastado do convívio social; retraído.}{re.co.lhi.do}{0}
\verb{recolhimento}{}{}{}{}{s.m.}{Ato ou efeito de recolher.}{re.co.lhi.men.to}{0}
\verb{recolhimento}{}{}{}{}{}{Vida recatada.}{re.co.lhi.men.to}{0}
\verb{recolhimento}{}{}{}{}{}{Lugar onde se recolhe alguém ou algo.}{re.co.lhi.men.to}{0}
\verb{recolhimento}{}{}{}{}{}{Retiro.}{re.co.lhi.men.to}{0}
\verb{recolocar}{}{}{}{}{v.t.}{Colocar alguma coisa de novo; repor.}{re.co.lo.car}{\verboinum{2}}
\verb{recomeçar}{}{}{}{}{v.t.}{Começar alguma coisa de novo.}{re.co.me.çar}{\verboinum{3}}
\verb{recomeço}{ê}{}{}{}{s.m.}{Novo começo.}{re.co.me.ço}{0}
\verb{recomendação}{}{}{"-ões}{}{s.f.}{Ato de recomendar; conselho.}{re.co.men.da.ção}{0}
\verb{recomendado}{}{}{}{}{adj.}{Que é objeto de recomendação ou empenho.}{re.co.men.da.do}{0}
\verb{recomendado}{}{}{}{}{s.m.}{Indivíduo recomendado ou protegido.}{re.co.men.da.do}{0}
\verb{recomendar}{}{}{}{}{v.t.}{Indicar alguma coisa que seria boa para uma pessoa; aconselhar.}{re.co.men.dar}{0}
\verb{recomendar}{}{}{}{}{}{Indicar que pessoa ou organização merece respeito e consideração.}{re.co.men.dar}{0}
\verb{recomendar}{}{}{}{}{}{Pedir a uma pessoa uma atenção especial para outra pessoa. }{re.co.men.dar}{0}
\verb{recomendar}{}{}{}{}{v.pron.}{Pedir a proteção de alguém.}{re.co.men.dar}{\verboinum{1}}
\verb{recomendável}{}{}{"-eis}{}{adj.2g.}{Que se pode ou deve recomendar; apreciável.}{re.co.men.dá.vel}{0}
\verb{recomendável}{}{}{"-eis}{}{}{Digno de respeito, de consideração, de apreço.}{re.co.men.dá.vel}{0}
\verb{recompensa}{}{}{}{}{s.f.}{Favor, presente com que se mostra reconhecimento por um obséquio, por uma boa ação; retribuição, prêmio.}{re.com.pen.sa}{0}
\verb{recompensa}{}{}{}{}{}{Compensação por dano.}{re.com.pen.sa}{0}
\verb{recompensador}{ô}{}{}{}{adj.}{Que recompensa; retribuidor.}{re.com.pen.sa.dor}{0}
\verb{recompensar}{}{}{}{}{v.t.}{Presentear uma pessoa com alguma coisa para demonstrar o contentamento ou a gratidão por algo que ela fez; premiar.}{re.com.pen.sar}{\verboinum{1}}
\verb{recompor}{ô}{}{}{}{v.t.}{Fazer com que alguma coisa tenha de novo uma apresentação adequada a uma situação; arrumar de novo.}{re.com.por}{\verboinum{60}}
\verb{recomposição}{}{}{"-ões}{}{s.f.}{Restituição de formato anterior; reconstituição.}{re.com.po.si.ção}{0}
\verb{recomposição}{}{}{"-ões}{}{}{Nova organização; reorganização.}{re.com.po.si.ção}{0}
\verb{recôncavo}{}{}{}{}{s.m.}{Cavidade funda.}{re.côn.ca.vo}{0}
\verb{recôncavo}{}{}{}{}{}{Cavidade entre rochedos; gruta.}{re.côn.ca.vo}{0}
\verb{recôncavo}{}{}{}{}{}{Pequena baía.}{re.côn.ca.vo}{0}
\verb{reconcentrar}{}{}{}{}{v.t.}{Fazer convergir para um centro comum.}{re.con.cen.trar}{0}
\verb{reconcentrar}{}{}{}{}{}{Reunir num ponto.}{re.con.cen.trar}{\verboinum{1}}
\verb{reconciliação}{}{}{"-ões}{}{s.f.}{Ato ou efeito de reconciliar.}{re.con.ci.li.a.ção}{0}
\verb{reconciliação}{}{}{"-ões}{}{}{Restabelecimento das relações ou acordo entre pessoas que se tinham desavindo.}{re.con.ci.li.a.ção}{0}
\verb{reconciliador}{ô}{}{}{}{adj.}{Que reconcilia; pacificador.}{re.con.ci.li.a.dor}{0}
\verb{reconciliar}{}{}{}{}{v.t.}{Estabelecer a paz entre.}{re.con.ci.li.ar}{\verboinum{1}}
\verb{reconciliatório}{}{}{}{}{adj.}{Que proporciona reconciliação.}{re.con.ci.li.a.tó.rio}{0}
\verb{recondicionar}{}{}{}{}{v.t.}{Restituir à primitiva condição; restaurar.}{re.con.di.ci.o.nar}{\verboinum{1}}
\verb{recôndito}{}{}{}{}{adj.}{Diz"-se de lugar escondido, oculto.}{re.côn.di.to}{0}
\verb{recôndito}{}{}{}{}{s.m.}{Íntimo, âmago.}{re.côn.di.to}{0}
\verb{recondução}{}{Jur.}{"-ões}{}{s.f.}{Ato ou efeito de reconduzir; nomeação para novo período de exercício de uma função.}{re.con.du.ção}{0}
\verb{recondução}{}{}{"-ões}{}{}{Prorrogação de contrato, aluguel, arrendamento, sem alteração das condições.}{re.con.du.ção}{0}
\verb{recondução}{}{Jur.}{"-ões}{}{}{Restituição ao local de origem; devolução, retorno.}{re.con.du.ção}{0}
\verb{reconduzir}{}{}{}{}{v.t.}{Fazer voltar ao ponto de partida.}{re.con.du.zir}{\verboinum{21}}
%\verb{}{}{}{}{}{}{}{}{0}
\verb{reconfortar}{}{}{}{}{v.t.}{Dar ânimo a alguém numa situação difícil.}{re.con.for.tar}{\verboinum{1}}
\verb{reconhecer}{ê}{}{}{}{v.t.}{Perceber que já conhecia determinada pessoa ou coisa; identificar.}{re.co.nhe.cer}{0}
\verb{reconhecer}{ê}{}{}{}{}{Aceitar que alguma coisa é verdade; admitir; confessar.}{re.co.nhe.cer}{\verboinum{15}}
\verb{reconhecido}{}{}{}{}{adj.}{Que mostra reconhecimento; grato.}{re.co.nhe.ci.do}{0}
\verb{reconhecido}{}{}{}{}{}{Confessado, declarado.}{re.co.nhe.ci.do}{0}
\verb{reconhecido}{}{}{}{}{}{Admitido como verdadeiro; aceito.}{re.co.nhe.ci.do}{0}
\verb{reconhecimento}{}{}{}{}{s.m.}{Percepção da familiaridade de.}{re.co.nhe.ci.men.to}{0}
\verb{reconhecimento}{}{}{}{}{}{Exame, verificação.}{re.co.nhe.ci.men.to}{0}
\verb{reconhecimento}{}{}{}{}{}{Confirmação, legitimidade.}{re.co.nhe.ci.men.to}{0}
\verb{reconhecimento}{}{}{}{}{}{Gratidão.}{re.co.nhe.ci.men.to}{0}
\verb{reconquista}{}{}{}{}{s.f.}{Ato ou efeito de reconquistar.}{re.con.quis.ta}{0}
\verb{reconquista}{}{}{}{}{}{Objeto ou lugar reconquistado.}{re.con.quis.ta}{0}
\verb{reconquistar}{}{}{}{}{v.t.}{Conseguir de novo alguma coisa que tinha perdido, lutando por ela.}{re.con.quis.tar}{0}
\verb{reconquistar}{}{}{}{}{}{Voltar a ter a situação que tinha com outra pessoa.}{re.con.quis.tar}{\verboinum{1}}
\verb{reconsiderar}{}{}{}{}{v.t.}{Considerar alguma coisa novamente.}{re.con.si.de.rar}{\verboinum{1}}
\verb{reconstituição}{}{}{"-ões}{}{s.f.}{Ato ou efeito de reconstituir; recomposição.}{re.cons.ti.tu.i.ção}{0}
\verb{reconstituinte}{}{}{}{}{adj.2g.}{Que reconstitui.}{re.cons.ti.tu.in.te}{0}
\verb{reconstituinte}{}{}{}{}{s.m.}{Substância medicamentosa usada para restabelecer as forças de pessoa fraca.}{re.cons.ti.tu.in.te}{0}
\verb{reconstituir}{}{}{}{}{v.t.}{Formar ou constituir de novo.}{re.cons.ti.tu.ir}{0}
\verb{reconstituir}{}{}{}{}{}{Relembrar um fato em detalhes.}{re.cons.ti.tu.ir}{\verboinum{26}}
\verb{reconstrução}{}{}{"-ões}{}{s.f.}{Reedificação do que foi total ou parcialmente arruinado.}{re.cons.tru.ção}{0}
\verb{reconstrução}{}{}{"-ões}{}{}{O que se reconstruiu ou reformou.}{re.cons.tru.ção}{0}
\verb{reconstruir}{}{}{}{}{v.t.}{Construir de novo o que foi destruído; reconstituir.}{re.cons.tru.ir}{\verboinum{26}}
\verb{recontagem}{}{}{"-ens}{}{s.f.}{Repetição de contagem.}{re.con.ta.gem}{0}
\verb{recontar}{}{}{}{}{v.t.}{Computar novamente.}{re.con.tar}{0}
\verb{recontar}{}{}{}{}{}{Calcular novamente.}{re.con.tar}{\verboinum{1}}
\verb{recontratar}{}{}{}{}{v.t.}{Tornar a contratar; readmitir.}{re.con.tra.tar}{\verboinum{1}}
\verb{recontro}{}{}{}{}{s.m.}{Combate breve.}{re.con.tro}{0}
\verb{recordação}{}{}{"-ões}{}{s.f.}{Ato ou efeito de recordar.}{re.cor.da.ção}{0}
\verb{recordação}{}{}{"-ões}{}{}{Lembrança.}{re.cor.da.ção}{0}
\verb{recordação}{}{}{"-ões}{}{}{Objeto que evoca uma lembrança.}{re.cor.da.ção}{0}
\verb{recordar}{}{}{}{}{v.t.}{Trazer à memória; lembrar.}{re.cor.dar}{\verboinum{1}}
\verb{recordativo}{}{}{}{}{adj.}{Que faz recordar.}{re.cor.da.ti.vo}{0}
\verb{recorde}{ó}{}{}{}{s.m.}{O melhor resultado em uma competição.}{re.cor.de}{0}
\verb{recorde}{ó}{}{}{}{}{Que consegue ser melhor que os resultados obtidos por outra pessoa.}{re.cor.de}{0}
\verb{recordista}{}{}{}{}{adj.2g.}{Que bate um recorde.}{re.cor.dis.ta}{0}
\verb{reco"-reco}{é-é}{}{reco"-recos ⟨é-é⟩}{}{s.m.}{Instrumento musical, feito de um pedaço de bambu ou madeira com cortes sobre os quais se passa um bastão pequeno. }{re.co"-re.co}{0}
\verb{recorrente}{}{}{}{}{adj.2g.}{Que ressurge ou volta a ocorrer.}{re.cor.ren.te}{0}
\verb{recorrer}{ê}{}{}{}{v.t.}{Dirigir"-se a alguém pedindo ajuda; socorrer"-se, valer"-se.}{re.cor.rer}{\verboinum{12}}
\verb{recortar}{}{}{}{}{v.t.}{Cortar seguindo um contorno.}{re.cor.tar}{0}
\verb{recortar}{}{}{}{}{}{Separar, cortando.}{re.cor.tar}{\verboinum{1}}
\verb{recorte}{ó}{}{}{}{s.m.}{Ato ou efeito de recortar.}{re.cor.te}{0}
\verb{recorte}{ó}{}{}{}{}{Desenho que se obtém recortando.}{re.cor.te}{0}
\verb{recorte}{ó}{}{}{}{}{Artigo, notícia, recortados de jornal ou revista.}{re.cor.te}{0}
\verb{recostar}{}{}{}{}{v.t.}{Apoiar obliquamente em algo.}{re.cos.tar}{0}
\verb{recostar}{}{}{}{}{v.pron.}{Repousar.}{re.cos.tar}{\verboinum{1}}
\verb{recosto}{ô}{}{}{}{s.m.}{Apoio para as costas em cadeiras, sofás etc.}{re.cos.to}{0}
\verb{recozer}{ê}{}{}{}{v.t.}{Cozer novamente.}{re.co.zer}{0}
\verb{recozer}{ê}{}{}{}{}{Cozer por bastante tempo.}{re.co.zer}{0}
\verb{recozer}{ê}{}{}{}{}{Conservar (cerâmica) em forno até que esfrie.}{re.co.zer}{\verboinum{12}}
\verb{recreação}{}{}{"-ões}{}{s.f.}{Ato ou efeito de recrear; divertimento.}{re.cre.a.ção}{0}
\verb{recrear}{}{}{}{}{v.t.}{Alegrar, divertir, brincar.}{re.cre.ar}{0}
\verb{recrear}{}{}{}{}{}{Aliviar de trabalho pesado, difícil ou demorado.}{re.cre.ar}{\verboinum{4}}
\verb{recreativo}{}{}{}{}{adj.}{Em que há prazer ou divertimento.}{re.cre.a.ti.vo}{0}
\verb{recreio}{ê}{}{}{}{s.m.}{Divertimento, brincadeira.}{re.crei.o}{0}
\verb{recreio}{ê}{}{}{}{}{Intervalo de tempo, existente no meio de um dia letivo, para que as crianças brinquem.}{re.crei.o}{0}
\verb{recreio}{ê}{}{}{}{}{Lugar apropriado para recrear.}{re.crei.o}{0}
\verb{recrescer}{ê}{}{}{}{v.i.}{Aumentar consideravelmente de intensidade.}{re.cres.cer}{0}
\verb{recrescer}{ê}{}{}{}{}{Tornar a crescer.}{re.cres.cer}{0}
\verb{recrescer}{ê}{}{}{}{}{Acontecer, sobrevir.}{re.cres.cer}{\verboinum{15}}
\verb{recrescimento}{}{}{}{}{s.m.}{Ato ou efeito de recrescer.}{re.cres.ci.men.to}{0}
\verb{recriação}{}{}{"-ões}{}{s.f.}{Ato ou efeito de recriar.}{re.cri.a.ção}{0}
\verb{recriar}{}{}{}{}{v.t.}{Criar novamente.}{re.cri.ar}{\verboinum{6}}
\verb{recriminação}{}{}{"-ões}{}{s.f.}{Ato ou efeito de recriminar.}{re.cri.mi.na.ção}{0}
\verb{recriminador}{ô}{}{}{}{adj.}{Que recrimina.}{re.cri.mi.na.dor}{0}
\verb{recriminar}{}{}{}{}{v.t.}{Censurar, criticar.}{re.cri.mi.nar}{0}
\verb{recriminar}{}{Jur.}{}{}{}{Defender"-se de uma acusação com uma acusação contra o acusador.}{re.cri.mi.nar}{\verboinum{1}}
\verb{recriminatório}{}{}{}{}{adj.}{Relativo a recriminação; que envolve recriminação.}{re.cri.mi.na.tó.rio}{0}
\verb{recrudescente}{}{}{}{}{adj.2g.}{Que se torna mais intenso.}{re.cru.des.cen.te}{0}
\verb{recrudescer}{ê}{}{}{}{v.i.}{Tornar"-se mais intenso; aumentar.}{re.cru.des.cer}{\verboinum{15}}
\verb{recruta}{}{Por ext.}{}{}{}{Indivíduo inexperiente.}{re.cru.ta}{0}
\verb{recruta}{}{}{}{}{s.m.}{Indivíduo recentemente admitido em um quartel para prestar serviço militar e que ainda não completou o período de instrução.}{re.cru.ta}{0}
\verb{recruta}{}{}{}{}{s.f.}{O conjunto de soldados em período de instrução.}{re.cru.ta}{0}
\verb{recrutador}{ô}{}{}{}{adj.}{Que recruta.}{re.cru.ta.dor}{0}
\verb{recrutamento}{}{}{}{}{s.m.}{Ato ou efeito de recrutar.}{re.cru.ta.men.to}{0}
\verb{recrutar}{}{}{}{}{v.t.}{Convocar pessoas.}{re.cru.tar}{0}
\verb{recrutar}{}{}{}{}{}{Convocar para o serviço militar.}{re.cru.tar}{0}
\verb{recrutar}{}{}{}{}{}{Atrair pessoas para que tomem parte em um movimento ou organização.}{re.cru.tar}{\verboinum{1}}
\verb{récua}{}{}{}{}{s.f.}{Tropa de animais de carga.}{ré.cu.a}{0}
\verb{récua}{}{Fig.}{}{}{}{Grupo de indivíduos desprezíveis; corja.}{ré.cu.a}{0}
\verb{recuar}{}{}{}{}{v.t.}{Movimentar"-se para trás; retroceder.}{re.cu.ar}{0}
\verb{recuar}{}{}{}{}{}{Desistir de algo; renunciar.}{re.cu.ar}{\verboinum{1}}
\verb{recuo}{}{}{}{}{s.m.}{Ato ou efeito de recuar.}{re.cu.o}{0}
\verb{recuo}{}{}{}{}{}{Espaço a mais que se forma no meio de um alinhamento.}{re.cu.o}{0}
\verb{recuperação}{}{}{"-ões}{}{s.f.}{Ato ou efeito de recuperar.}{re.cu.pe.ra.ção}{0}
\verb{recuperação}{}{}{"-ões}{}{}{Período de estudos ou prova a que um aluno reprovado é convocado para que tenha uma segunda chance de aprovação.}{re.cu.pe.ra.ção}{0}
\verb{recuperar}{}{}{}{}{v.t.}{Passar a ter novamente um bem, uma habilidade, a saúde, uma oportunidade.}{re.cu.pe.rar}{0}
\verb{recuperar}{}{}{}{}{}{Reintegrar na sociedade; reabilitar.}{re.cu.pe.rar}{0}
\verb{recuperar}{}{}{}{}{v.pron.}{Recobrar a saúde ou o ânimo.}{re.cu.pe.rar}{\verboinum{1}}
\verb{recuperável}{}{}{"-eis}{}{adj.2g.}{Que pode ser recuperado.}{re.cu.pe.rá.vel}{0}
\verb{recurso}{}{}{}{}{s.m.}{Ato ou efeito de recorrer.}{re.cur.so}{0}
\verb{recurso}{}{}{}{}{}{Meio empregado para superar situação desfavorável.}{re.cur.so}{0}
\verb{recurso}{}{Jur.}{}{}{}{Meio de que a parte vencida em um processo dispõe para provocar a revisão do julgamento.}{re.cur.so}{0}
\verb{recurso}{}{}{}{}{s.m.pl.}{Dinheiro, bens materiais, riquezas.}{re.cur.so}{0}
\verb{recurvado}{}{}{}{}{adj.}{Que se recurvou; torcido.}{re.cur.va.do}{0}
\verb{recurvar}{}{}{}{}{v.t.}{Tornar curvo; dobrar, curvar.}{re.cur.var}{\verboinum{1}}
\verb{recurvo}{}{}{}{}{adj.}{Que se recurvou; recurvado, torcido.}{re.cur.vo}{0}
\verb{recusa}{}{}{}{}{s.f.}{Ato ou efeito de recusar; resposta negativa.}{re.cu.sa}{0}
\verb{recusar}{}{}{}{}{v.t.}{Não aceitar uma oferta; declinar de.}{re.cu.sar}{0}
\verb{recusar}{}{}{}{}{}{Não atender a um pedido.}{re.cu.sar}{0}
\verb{recusar}{}{}{}{}{v.pron.}{Negar"-se, opor"-se.}{re.cu.sar}{\verboinum{1}}
\verb{recusável}{}{}{"-eis}{}{adj.2g.}{Que pode ser recusado.}{re.cu.sá.vel}{0}
\verb{redação}{}{}{"-ões}{}{s.f.}{Ato ou efeito de redigir.}{re.da.ção}{0}
\verb{redação}{}{}{"-ões}{}{}{Exercício de produção de texto com clareza e organização de ideias.}{re.da.ção}{0}
\verb{redação}{}{}{"-ões}{}{}{Conjunto dos redatores de um jornal.}{re.da.ção}{0}
\verb{redarguir}{}{}{}{}{v.t.}{Responder com argumentos.}{re.dar.guir}{0}
\verb{redarguir}{}{}{}{}{}{Apresentar acusação em resposta a uma outra acusação; recriminar.}{re.dar.guir}{\verboinum{28}}
\verb{redator}{ô}{}{}{}{s.m.}{Indivíduo que escreve profissionalmente textos não literários, como textos de jornais, revistas, publicidade.}{re.da.tor}{0}
\verb{rede}{ê}{}{}{}{s.f.}{Tecido aberto formado por um entrelaçado de fios ou arame que compõem losangos ou quadrados.}{re.de}{0}
\verb{rede}{ê}{}{}{}{}{Artefato utilizado para capturar peixes, aves ou outros animais.}{re.de}{0}
\verb{rede}{ê}{Bras.}{}{}{}{Peça de tecido que fica suspensa pelas extremidades e é usada para dormir ou descansar.}{re.de}{0}
\verb{rede}{ê}{}{}{}{}{Conjunto de tubos, canais, estradas, ferrovias que se entrecruzam.}{re.de}{0}
\verb{rede}{ê}{Informát.}{}{}{}{Sistema constituído pela conexão de dois ou mais computadores para que se comuniquem entre si e compartilhem informações e periféricos.}{re.de}{0}
\verb{rédea}{}{}{}{}{s.f.}{Correia presa ao freio de montaria para controlar o animal.}{ré.dea}{0}
\verb{rédea}{}{Fig.}{}{}{}{Comando, controle, restrição de liberdade.}{ré.dea}{0}
\verb{redemoinhar}{}{}{}{}{v.i.}{Mover"-se de maneira circular; girar, rodar; remoinhar.}{re.de.mo.i.nhar}{\verboinum{1}}
\verb{redemoinho}{}{}{}{}{s.m.}{Turbilhão de água ou vento em movimento de rotação; remoinho.}{re.de.mo.i.nho}{0}
\verb{redenção}{}{}{"-ões}{}{s.f.}{Ato ou efeito de remir.}{re.den.ção}{0}
\verb{redenção}{}{Relig.}{"-ões}{}{}{Auxílio, salvação.}{re.den.ção}{0}
\verb{redentor}{ô}{}{}{}{s.m.}{Indivíduo que redime.}{re.den.tor}{0}
\verb{redentor}{ô}{Relig.}{}{}{}{Jesus Cristo. (Usa"-se com inicial maiúscula nesta acepção.)}{re.den.tor}{0}
\verb{redescobrir}{}{}{}{}{v.t.}{Descobrir novamente; reencontrar.}{re.des.co.brir}{\verboinum{31}}
\verb{redescontar}{}{}{}{}{v.t.}{Fazer o redesconto de.}{re.des.con.tar}{\verboinum{1}}
\verb{redesconto}{}{Econ.}{}{}{s.m.}{Desconto que um banco faz em outra instituição financeira de um título anteriormente descontado de um cliente.}{re.des.con.to}{0}
\verb{redigir}{}{}{}{}{v.t.}{Expressar ideias por escrito e de maneira organizada.}{re.di.gir}{\verboinum{22}}
\verb{redil}{}{}{"-is}{}{s.m.}{Curral para gado.}{re.dil}{0}
\verb{redil}{}{}{"-is}{}{}{Rebanho de ovinos.}{re.dil}{0}
\verb{redil}{}{Relig.}{"-is}{}{}{Rebanho de fiéis.}{re.dil}{0}
\verb{redimensionar}{}{}{}{}{v.t.}{Mudar as dimensões de.}{re.di.men.si.o.nar}{\verboinum{1}}
\verb{redimir}{}{}{}{}{v.t.}{Remir.}{re.di.mir}{\verboinum{18}}
\verb{redingote}{ó}{}{}{}{s.m.}{Tipo de casaco comprido, ajustado à cintura e abotoado à frente.}{re.din.go.te}{0}
\verb{redistribuir}{}{}{}{}{v.t.}{Refazer ou reformular a distribuição de algo.}{re.dis.tri.bu.ir}{\verboinum{26}}
\verb{rédito}{}{}{}{}{s.m.}{Ato ou efeito de voltar.}{ré.di.to}{0}
\verb{rédito}{}{}{}{}{}{Rendimento que se tira de algo; lucro, juro.}{ré.di.to}{0}
\verb{redivivo}{}{}{}{}{adj.}{Que voltou à vida; ressuscitado.}{re.di.vi.vo}{0}
\verb{redivivo}{}{}{}{}{}{Rejuvenescido, renovado.}{re.di.vi.vo}{0}
\verb{redizer}{ê}{}{}{}{v.t.}{Dizer novamente.}{re.di.zer}{0}
\verb{redizer}{ê}{}{}{}{}{Dizer contínua e repetidamente.}{re.di.zer}{0}
\verb{redizer}{ê}{}{}{}{}{Contar um fato; narrar.}{re.di.zer}{\verboinum{41}}
\verb{redobrar}{}{}{}{}{v.t.}{Tornar a dobrar.}{re.do.brar}{0}
\verb{redobrar}{}{}{}{}{}{Tornar a realizar; repetir.}{re.do.brar}{0}
\verb{redobrar}{}{}{}{}{}{Aumentar consideravelmente em número, tamanho ou intensidade.}{re.do.brar}{0}
\verb{redobrar}{}{}{}{}{}{Soar (o sino).}{re.do.brar}{\verboinum{1}}
\verb{redobro}{ô}{}{}{}{s.m.}{Ato ou efeito de redobrar.}{re.do.bro}{0}
\verb{redobro}{ô}{Gram.}{}{}{}{Processo morfológico pelo qual se repetem fonemas de uma palavra para obter efeitos gramaticais ou expressivos; reduplicação.}{re.do.bro}{0}
\verb{redoma}{}{}{}{}{s.f.}{Cobertura arredondada de vidro ou de outro material usada para proteger alguma coisa.}{re.do.ma}{0}
\verb{redondeza}{ê}{}{}{}{s.f.}{Qualidade de redondo.}{re.don.de.za}{0}
\verb{redondeza}{ê}{}{}{}{}{Localidades próximas; arredores, cercanias. [usa"-se geralmente no plural nesta acepção]}{re.don.de.za}{0}
\verb{redondilha}{}{Liter.}{}{}{s.f.}{Verso de cinco ou sete sílabas.}{re.don.di.lha}{0}
\verb{redondo}{}{}{}{}{adj.}{Que tem forma de círculo, esfera ou cilindro.}{re.don.do}{0}
\verb{redondo}{}{}{}{}{}{Que tem forma curvada; arredondado.}{re.don.do}{0}
\verb{redondo}{}{Mat.}{}{}{}{Diz"-se de número inteiro ou número cujo último algarismo é zero.}{re.don.do}{0}
\verb{redor}{ó}{}{}{}{s.m.}{Contorno, circuito.}{re.dor}{0}
\verb{redor}{ó}{}{}{}{}{Volta, roda.}{re.dor}{0}
\verb{redor}{ó}{}{}{}{}{Arredores, arrabalde.}{re.dor}{0}
\verb{redução}{}{}{"-ões}{}{s.f.}{Ato ou efeito de reduzir.}{re.du.ção}{0}
\verb{redundância}{}{}{}{}{s.f.}{Qualidade de redundante.}{re.dun.dân.cia}{0}
\verb{redundância}{}{}{}{}{}{Palavra ou informação em excesso, supérflua; pleonasmo.}{re.dun.dân.cia}{0}
\verb{redundante}{}{}{}{}{adj.2g.}{Que redunda, que existe em excesso.}{re.dun.dan.te}{0}
\verb{redundante}{}{}{}{}{}{Diz"-se de palavra ou informação que não acrescenta nada de novo.}{re.dun.dan.te}{0}
\verb{redundar}{}{}{}{}{v.i.}{Transbordar, sobrar, existir em excesso.}{re.dun.dar}{0}
\verb{redundar}{}{}{}{}{v.t.}{Resultar, acontecer, nascer.}{re.dun.dar}{0}
\verb{redundar}{}{}{}{}{}{Converter"-se, reverter.}{re.dun.dar}{\verboinum{1}}
\verb{reduplicação}{}{}{"-ões}{}{s.f.}{Ato ou efeito de reduplicar.}{re.du.pli.ca.ção}{0}
\verb{reduplicação}{}{Gram.}{"-ões}{}{}{Processo morfológico pelo qual se repetem fonemas de uma palavra para obter efeitos gramaticais ou expressivos.}{re.du.pli.ca.ção}{0}
\verb{reduplicar}{}{}{}{}{v.t.}{Duplicar novamente; redobrar, repetir.}{re.du.pli.car}{0}
\verb{reduplicar}{}{}{}{}{}{Aumentar em quantidade, tamanho ou intensidade.}{re.du.pli.car}{\verboinum{2}}
\verb{redutível}{}{}{"-eis}{}{adj.2g.}{Que pode ser reduzido.}{re.du.tí.vel}{0}
\verb{reduto}{}{}{}{}{s.m.}{Lugar fechado que serve de abrigo ou refúgio.}{re.du.to}{0}
\verb{reduto}{}{}{}{}{}{Lugar onde se reúnem pessoas de ideologias ou comportamentos comuns.}{re.du.to}{0}
\verb{redutor}{ô}{}{}{}{adj.}{Que reduz ou tem a propriedade de reduzir.}{re.du.tor}{0}
\verb{reduzida}{}{Gram.}{}{}{s.f.}{Redução de \textit{oração reduzida}, oração cujo verbo está no infinitivo, particípio ou gerúndio e que se encontra subordinada a uma oração principal.}{re.du.zi.da}{0}
\verb{reduzida}{}{Bras.}{}{}{}{Ato ou efeito de reduzir a marcha de um veículo.}{re.du.zi.da}{0}
\verb{reduzida}{}{Bras.}{}{}{}{Marcha de grande poder de tração existente somente em veículos utilitários.}{re.du.zi.da}{0}
\verb{reduzir}{}{}{}{}{v.t.}{Tornar menor; diminuir, restringir.}{re.du.zir}{0}
\verb{reduzir}{}{}{}{}{}{Subjugar, submeter.}{re.du.zir}{0}
\verb{reduzir}{}{}{}{}{}{Transformar, converter.}{re.du.zir}{0}
\verb{reduzir}{}{Quím.}{}{}{}{Desagregar uma substância, separando os elementos constituintes.}{re.du.zir}{0}
\verb{reduzir}{}{}{}{}{v.pron.}{Limitar"-se, resumir"-se.}{re.du.zir}{\verboinum{21}}
\verb{reedição}{}{}{"-ões}{}{s.f.}{Ato ou efeito de reeditar.}{re.e.di.ção}{0}
\verb{reedição}{}{}{"-ões}{}{}{Nova edição de uma obra que contém alterações em relação à edição anterior.}{re.e.di.ção}{0}
\verb{reedificação}{}{}{"-ões}{}{s.f.}{Ato ou efeito de reedificar.}{re.e.di.fi.ca.ção}{0}
\verb{reedificar}{}{}{}{}{v.t.}{Edificar novamente; reconstruir.}{re.e.di.fi.car}{0}
\verb{reedificar}{}{}{}{}{}{Reformar, restaurar.}{re.e.di.fi.car}{\verboinum{2}}
\verb{reeditar}{}{}{}{}{v.t.}{Editar novamente; reproduzir.}{re.e.di.tar}{\verboinum{1}}
\verb{reeducação}{}{}{"-ões}{}{s.f.}{Ato ou efeito de reeducar.}{re.e.du.ca.ção}{0}
\verb{reeducar}{}{}{}{}{v.t.}{Educar novamente ou completar a educação de.}{re.e.du.car}{0}
\verb{reeducar}{}{}{}{}{}{Reabilitar através de educação.}{re.e.du.car}{\verboinum{2}}
\verb{reeleger}{ê}{}{}{}{v.t.}{Eleger novamente.}{re.e.le.ger}{\verboinum{16}}
\verb{reeleição}{}{}{"-ões}{}{s.f.}{Ato ou efeito de reeleger.}{re.e.lei.ção}{0}
\verb{reeleito}{ê}{}{}{}{adj.}{Que se reelegeu.}{re.e.lei.to}{0}
\verb{reembolsar}{}{}{}{}{v.t.}{Embolsar novamente.}{re.em.bol.sar}{0}
\verb{reembolsar}{}{}{}{}{}{Restituir, devolver, indenizar, compensar.}{re.em.bol.sar}{0}
\verb{reembolsar}{}{}{}{}{v.pron.}{Recuperar a posse do que anteriormente se emprestou.}{re.em.bol.sar}{\verboinum{1}}
\verb{reembolso}{ô}{}{}{}{s.m.}{Ato ou efeito de reembolsar.}{re.em.bol.so}{0}
\verb{reempossar}{}{}{}{}{v.t.}{Empossar novamente.}{re.em.pos.sar}{\verboinum{1}}
\verb{reencarnação}{}{}{"-ões}{}{s.f.}{Ato ou efeito de reencarnar.}{re.en.car.na.ção}{0}
\verb{reencarnação}{}{Relig.}{"-ões}{}{}{Crença na possibilidade de, após a morte, voltar a viver em outro corpo ou sob a forma de outra espécie.}{re.en.car.na.ção}{0}
\verb{reencarnação}{}{Fig.}{"-ões}{}{}{O novo portador de uma ideia ou doutrina desaparecida ou abandonada.}{re.en.car.na.ção}{0}
\verb{reencarnar}{}{}{}{}{v.i.}{Tornar a ter um corpo físico.}{re.en.car.nar}{\verboinum{1}}
\verb{reencher}{ê}{}{}{}{v.t.}{Encher novamente.}{re.en.cher}{\verboinum{12}}
\verb{reencontrar}{}{}{}{}{v.t.}{Encontrar novamente.}{re.en.con.trar}{\verboinum{1}}
\verb{reencontro}{}{}{}{}{s.m.}{Ato ou efeito de reencontrar.}{re.en.con.tro}{0}
\verb{reentrância}{}{}{}{}{s.f.}{Ondulação ou sinuosidade para dentro; concavidade.}{re.en.trân.cia}{0}
\verb{reentrante}{}{}{}{}{adj.2g.}{Que forma ondulação ou sinuosidade para dentro.}{re.en.tran.te}{0}
\verb{reentrar}{}{}{}{}{v.i.}{Tornar a entrar; recolher"-se.}{re.en.trar}{\verboinum{1}}
\verb{reenviar}{}{}{}{}{v.t.}{Enviar novamente.}{re.en.vi.ar}{0}
\verb{reenviar}{}{}{}{}{}{Devolver.}{re.en.vi.ar}{\verboinum{6}}
\verb{reerguer}{ê}{}{}{}{v.t.}{Erguer novamente.}{re.er.guer}{\verboinum{53}}
\verb{reescrever}{ê}{}{}{}{v.t.}{Escrever novamente.}{re.es.cre.ver}{0}
\verb{reescrever}{ê}{}{}{}{}{Reformular um texto para que fique mais claro ou adequado.}{re.es.cre.ver}{\verboinum{12}}
\verb{reestruturação}{}{}{"-ões}{}{s.f.}{Ato ou efeito de reestruturar.}{re.es.tru.tu.ra.ção}{0}
\verb{reestruturar}{}{}{}{}{v.t.}{Reconstruir ou reformular a estrutura de.}{re.es.tru.tu.rar}{\verboinum{1}}
\verb{reexaminar}{z}{}{}{}{v.t.}{Tornar a examinar.}{re.e.xa.mi.nar}{\verboinum{1}}
\verb{reexportar}{s}{}{}{}{v.t.}{Exportar de novo.}{re.ex.por.tar}{\verboinum{1}}
\verb{refazer}{ê}{}{}{}{v.t.}{Fazer novamente.}{re.fa.zer}{0}
\verb{refazer}{ê}{}{}{}{}{Reformar, restaurar, consertar, corrigir, reorganizar, reconstruir.}{re.fa.zer}{0}
\verb{refazer}{ê}{}{}{}{}{Restaurar as próprias forças; reanimar"-se.}{re.fa.zer}{\verboinum{42}}
\verb{refeição}{}{}{"-ões}{}{s.f.}{Ato de ingerir uma quantidade razoável de alimentos, geralmente em horas determinadas do dia.}{re.fei.ção}{0}
\verb{refeito}{ê}{}{}{}{adj.}{Que se refez.}{re.fei.to}{0}
\verb{refeito}{ê}{}{}{}{}{Reparado, corrigido, restaurado, restabelecido.}{re.fei.to}{0}
\verb{refeito}{ê}{}{}{}{}{Que foi nutrido; saudável, robusto.}{re.fei.to}{0}
\verb{refeitório}{}{Bras.}{}{}{s.m.}{Lugar onde se fazem as refeições.}{re.fei.tó.rio}{0}
\verb{refém}{}{}{"-éns}{}{s.2g.}{Indivíduo que fica, contra a vontade, em poder de outra pessoa como garantia de que algum pedido será atendido.}{re.fém}{0}
\verb{referência}{}{}{}{}{s.f.}{Ato ou efeito de referir.}{re.fe.rên.cia}{0}
\verb{referência}{}{}{}{}{}{Aquilo que se refere.}{re.fe.rên.cia}{0}
\verb{referência}{}{Bras.}{}{}{}{Informações que pessoas conhecidas ou sabidamente idôneas prestam a respeito de alguém, em certas situações. [usa"-se geralmente no plural nesta acepção]}{re.fe.rên.cia}{0}
\verb{referencial}{}{}{"-ais}{}{adj.2g.}{Que constitui referência.}{re.fe.ren.ci.al}{0}
\verb{referencial}{}{Fís.}{"-ais}{}{s.m.}{Ponto ideal a partir do qual se observam os fenômenos e se fazem as medições.}{re.fe.ren.ci.al}{0}
\verb{referendar}{}{}{}{}{v.t.}{Assinar (um documento) assumindo responsabilidade.}{re.fe.ren.dar}{0}
\verb{referendar}{}{}{}{}{}{Aceitar a aprovação dada por outrem.}{re.fe.ren.dar}{\verboinum{1}}
\verb{referendo}{}{}{}{}{s.m.}{Direito que têm os cidadãos de se pronunciarem diretamente sobre assuntos de interesse nacional.}{re.fe.ren.do}{0}
\verb{referente}{}{}{}{}{adj.2g.}{Que diz respeito a algo; relativo, respeitante.}{re.fe.ren.te}{0}
\verb{referente}{}{Gram.}{}{}{s.m.}{Elemento concreto ou abstrato no mundo extralinguístico ao qual se refere determinado elemento linguístico.}{re.fe.ren.te}{0}
\verb{referir}{}{}{}{}{v.t.}{Fazer menção a; aludir.}{re.fe.rir}{0}
\verb{referir}{}{}{}{}{}{Citar, alegar.}{re.fe.rir}{0}
\verb{referir}{}{}{}{}{}{Relatar, expor, contar, narrar.}{re.fe.rir}{0}
\verb{referir}{}{}{}{}{v.pron.}{Estar relacionado com.}{re.fe.rir}{\verboinum{29}}
\verb{referver}{ê}{}{}{}{v.t.}{Tornar a ferver ou ferver bastante.}{re.fer.ver}{0}
\verb{referver}{ê}{}{}{}{v.i.}{Agitar"-se, vibrar.}{re.fer.ver}{0}
\verb{referver}{ê}{}{}{}{}{Fermentar.}{re.fer.ver}{\verboinum{12}}
\verb{refestelar"-se}{}{}{}{}{v.pron.}{Fazer algo agradável, prazeroso; deleitar"-se.}{re.fes.te.lar"-se}{0}
\verb{refestelar"-se}{}{}{}{}{}{Acomodar"-se confortavelmente; recostar"-se.}{re.fes.te.lar"-se}{\verboinum{1}}
\verb{refil}{}{}{"-is}{}{s.m.}{Produto que contém apenas a carga de um utensílio, tornando desnecessária a troca do utensílio todo.}{re.fil}{0}
\verb{refinação}{}{}{"-ões}{}{s.f.}{Ato ou efeito de refinar.}{re.fi.na.ção}{0}
\verb{refinação}{}{}{"-ões}{}{}{Lugar onde se refina; refinaria.}{re.fi.na.ção}{0}
\verb{refinado}{}{}{}{}{adj.}{Que se refinou.}{re.fi.na.do}{0}
\verb{refinado}{}{}{}{}{}{Requintado, delicado.}{re.fi.na.do}{0}
\verb{refinador}{ô}{}{}{}{adj.}{Que refina.}{re.fi.na.dor}{0}
\verb{refinamento}{}{}{}{}{s.m.}{Ato ou efeito de refinar.}{re.fi.na.men.to}{0}
\verb{refinamento}{}{}{}{}{}{Requinte, apuro, delicadeza.}{re.fi.na.men.to}{0}
\verb{refinar}{}{}{}{}{v.t.}{Tornar mais fino ou mais puro.}{re.fi.nar}{0}
\verb{refinar}{}{}{}{}{}{Tornar mais requintado; aprimorar.}{re.fi.nar}{\verboinum{1}}
\verb{refinaria}{}{}{}{}{s.f.}{Lugar onde se refinam produtos.}{re.fi.na.ri.a}{0}
\verb{refle}{é}{}{}{}{s.m.}{Tipo de espingarda curta.}{re.fle}{0}
\verb{refletido}{}{Fís.}{}{}{adj.}{Que sofreu reflexão.}{re.fle.ti.do}{0}
\verb{refletido}{}{}{}{}{}{Prudente, ponderado, sensato.}{re.fle.ti.do}{0}
\verb{refletir}{}{}{}{}{v.t.}{Fazer som ou luz seguir na direção de alguma coisa.}{re.fle.tir}{0}
\verb{refletir}{}{}{}{}{}{Reproduzir a imagem de pessoa ou coisa.}{re.fle.tir}{0}
\verb{refletir}{}{}{}{}{}{Ficar com o pensamento concentrado em alguma coisa; meditar.}{re.fle.tir}{\verboinum{18}}
\verb{refletor}{ô}{}{}{}{adj.}{Que reflete.}{re.fle.tor}{0}
\verb{refletor}{ô}{}{}{}{s.m.}{Aparelho que reflete luz.}{re.fle.tor}{0}
\verb{reflexão}{cs}{}{"-ões}{}{s.f.}{Ato de refletir a luz ou o som.}{re.fle.xão}{0}
\verb{reflexão}{cs}{}{"-ões}{}{}{Ato de refletir sobre um assunto; meditação.}{re.fle.xão}{0}
\verb{reflexionar}{cs}{}{}{}{v.t.}{Refletir ou considerar sobre alguma coisa; ponderar, pesar, meditar.}{re.fle.xi.o.nar}{0}
\verb{reflexionar}{cs}{}{}{}{}{Fazer ponderação ou objeção.}{re.fle.xi.o.nar}{\verboinum{1}}
\verb{reflexivo}{cs}{}{}{}{adj.}{Que reflete.}{re.fle.xi.vo}{0}
\verb{reflexivo}{cs}{}{}{}{}{Meditativo.}{re.fle.xi.vo}{0}
\verb{reflexivo}{cs}{Gram.}{}{}{}{Diz"-se de verbo cujo sujeito e objeto se referem ao mesmo ser.}{re.fle.xi.vo}{0}
\verb{reflexivo}{cs}{Gram.}{}{}{}{Diz"-se do pronome que complementa esse verbo.}{re.fle.xi.vo}{0}
\verb{reflexo}{écs}{}{}{}{adj.}{Que se volta para si mesmo.}{re.fle.xo}{0}
\verb{reflexo}{écs}{}{}{}{s.m.}{Luz refletida.}{re.fle.xo}{0}
\verb{reflexo}{écs}{}{}{}{}{Imagem refletida.}{re.fle.xo}{0}
\verb{reflexo}{écs}{}{}{}{}{Reação rápida de alguém a um acontecimento súbito.}{re.fle.xo}{0}
\verb{reflorescente}{}{}{}{}{adj.2g.}{Que refloresce; que torna a florescer.}{re.flo.res.cen.te}{0}
\verb{reflorescer}{ê}{}{}{}{v.i.}{Florescer de novo.}{re.flo.res.cer}{0}
\verb{reflorescer}{ê}{Fig.}{}{}{}{Rejuvenescer.}{re.flo.res.cer}{\verboinum{15}}
\verb{reflorestamento}{}{}{}{}{s.m.}{Ato ou efeito de reflorestar.}{re.flo.res.ta.men.to}{0}
\verb{reflorestamento}{}{}{}{}{}{Plantação de árvores em lugar onde foi derrubada floresta virgem.}{re.flo.res.ta.men.to}{0}
\verb{reflorestar}{}{}{}{}{v.t.}{Plantar árvores em floresta devastada.}{re.flo.res.tar}{\verboinum{1}}
\verb{reflorir}{}{}{}{}{v.t.}{Reflorescer.}{re.flo.rir}{\verboinum{18}}
\verb{refluir}{}{}{}{}{v.i.}{Correr para trás; retroceder.}{re.flu.ir}{\verboinum{18}}
\verb{refluxo}{cs}{}{}{}{s.m.}{Ato ou efeito de refluir.}{re.flu.xo}{0}
\verb{refluxo}{cs}{}{}{}{}{Movimento da maré vazante.}{re.flu.xo}{0}
\verb{refluxo}{cs}{}{}{}{}{Movimento contrário e sucessivo a outro.}{re.flu.xo}{0}
\verb{refocilar}{}{}{}{}{v.t.}{Dar novas forças a; reforçar, restaurar.}{re.fo.ci.lar}{0}
\verb{refocilar}{}{}{}{}{}{Dar descanso a; recrear.}{re.fo.ci.lar}{0}
\verb{refocilar}{}{}{}{}{v.pron.}{Distrair"-se do trabalho ou do estudo.}{re.fo.ci.lar}{0}
\verb{refocilar}{}{}{}{}{}{Deitar"-se ou sentar"-se confortavelmente para descansar; refestelar"-se.}{re.fo.ci.lar}{\verboinum{1}}
\verb{refogado}{}{}{}{}{adj.}{Que se refogou; repassado em mistura feita de gordura e temperos diversos.}{re.fo.ga.do}{0}
\verb{refogado}{}{Cul.}{}{}{s.m.}{Alimento cozido com essa mistura de temperos.}{re.fo.ga.do}{0}
\verb{refogar}{}{Cul.}{}{}{v.t.}{Fazer o tempero ferver em gordura.}{re.fo.gar}{\verboinum{5}}
\verb{reforçado}{}{}{}{}{adj.}{Que adquiriu forças; fortalecido, revigorado.}{re.for.ça.do}{0}
\verb{reforçado}{}{}{}{}{}{Robusto, vigoroso.}{re.for.ça.do}{0}
\verb{reforçado}{}{}{}{}{}{Que recebeu reforço.}{re.for.ça.do}{0}
\verb{reforçar}{}{}{}{}{v.t.}{Tornar mais forte, resistente, intenso ou numeroso.}{re.for.çar}{\verboinum{3}}
\verb{reforço}{ô}{}{}{}{s.m.}{Coisa que serve para reforçar.}{re.for.ço}{0}
\verb{reforma}{ó}{}{}{}{s.f.}{Ato ou efeito de reformar; modificação.}{re.for.ma}{0}
\verb{reforma}{ó}{}{}{}{}{Aposentadoria de militar.}{re.for.ma}{0}
\verb{reforma}{ó}{}{}{}{}{Conserto, restauração.}{re.for.ma}{0}
\verb{reformado}{}{}{}{}{adj.}{Que sofreu reforma.}{re.for.ma.do}{0}
\verb{reformado}{}{}{}{}{}{Emendado, melhorado.}{re.for.ma.do}{0}
\verb{reformado}{}{}{}{}{}{Diz"-se de militar que obteve reforma.}{re.for.ma.do}{0}
\verb{reformador}{ô}{}{}{}{adj.}{Que reforma, conserta, reconstrói.}{re.for.ma.dor}{0}
\verb{reformador}{ô}{}{}{}{s.m.}{Promotor da Reforma Protestante.}{re.for.ma.dor}{0}
\verb{reformar}{}{}{}{}{v.t.}{Mudar para melhor a forma de alguma coisa; restaurar.}{re.for.mar}{0}
\verb{reformar}{}{}{}{}{}{Fazer alguém voltar a ter bom comportamento; corrigir, recuperar, regenerar.}{re.for.mar}{0}
\verb{reformar}{}{}{}{}{}{Afastar um militar de atividade por invalidez permanente, idade avançada ou tempo de serviço, garantindo"-lhe o pagamento do soldo e outros benefícios.}{re.for.mar}{\verboinum{1}}
\verb{reformatório}{}{}{}{}{s.m.}{Estabelecimento oficial para regenerar menores delinquentes.}{re.for.ma.tó.rio}{0}
\verb{reformista}{}{}{}{}{s.2g.}{Adepto de modificação política ou religiosa.}{re.for.mis.ta}{0}
\verb{reformular}{}{}{}{}{v.t.}{Formular de novo; reelaborar.}{re.for.mu.lar}{\verboinum{1}}
\verb{refração}{}{}{"-ões}{}{s.f.}{Mudança de direção de onda de luz, som etc., ao mudar de meio.}{re.fra.ção}{0}
\verb{refrangente}{}{}{}{}{adj.2g.}{Que refrange ou causa refração.}{re.fran.gen.te}{0}
\verb{refranger}{ê}{}{}{}{v.t.}{Refratar.}{re.fran.ger}{\verboinum{16}}
\verb{refrão}{}{}{"-ões}{}{s.m.}{Conjunto de versos repetidos de uma poesia; estribilho.}{re.frão}{0}
\verb{refratar}{}{}{}{}{v.t.}{Produzir refração em; refranger, quebrar.}{re.fra.tar}{0}
\verb{refratar}{}{}{}{}{v.pron.}{Desviar"-se de sua primitiva direção.}{re.fra.tar}{0}
\verb{refratar}{}{}{}{}{}{Sofrer refração; refletir"-se.}{re.fra.tar}{\verboinum{1}}
\verb{refratário}{}{}{}{}{adj.}{Capaz de enfrentar o frio e o calor sem se danificar.}{re.fra.tá.rio}{0}
\verb{refratário}{}{}{}{}{}{Que se recusa a obedecer; rebelde.}{re.fra.tá.rio}{0}
\verb{refrear}{}{}{}{}{v.t.}{Fazer com que pessoa ou coisa fique dentro de certos limites; conter, controlar, dominar.}{re.fre.ar}{\verboinum{4}}
\verb{refrega}{é}{}{}{}{s.f.}{Combate entre forças ou indivíduos inimigos entre si; luta, confronto.}{re.fre.ga}{0}
\verb{refrescante}{}{}{}{}{adj.2g.}{Que refresca; refrigerante.}{re.fres.can.te}{0}
\verb{refrescar}{}{}{}{}{v.t.}{Tornar menos quente.}{re.fres.car}{0}
\verb{refrescar}{}{}{}{}{}{Aliviar, suavizar.}{re.fres.car}{\verboinum{2}}
\verb{refresco}{ê}{}{}{}{s.m.}{Suco de frutas diluído, adoçado e gelado.}{re.fres.co}{0}
\verb{refresco}{ê}{Pop.}{}{}{}{Alívio, consolo.}{re.fres.co}{0}
\verb{refrigeração}{}{}{"-ões}{}{s.f.}{Ato ou efeito de refrigerar.}{re.fri.ge.ra.ção}{0}
\verb{refrigeração}{}{}{"-ões}{}{}{Estabelecimento, empresa especializada em refrigeração.}{re.fri.ge.ra.ção}{0}
\verb{refrigeração}{}{}{"-ões}{}{}{Abaixamento artificial da temperatura.}{re.fri.ge.ra.ção}{0}
\verb{refrigerador}{ô}{}{}{}{adj.}{Que refrigera.}{re.fri.ge.ra.dor}{0}
\verb{refrigerador}{ô}{}{}{}{s.m.}{Aparelho em forma de caixa, com porta, próprio para resfriar e conservar os alimentos; geladeira.}{re.fri.ge.ra.dor}{0}
\verb{refrigerante}{}{}{}{}{adj.2g.}{Que é capaz de refrigerar; refrescante.}{re.fri.ge.ran.te}{0}
\verb{refrigerante}{}{}{}{}{s.m.}{Bebida não alcoólica, produzida industrialmente.}{re.fri.ge.ran.te}{0}
\verb{refrigerar}{}{}{}{}{v.t.}{Tornar frio ou fresco; refrescar.}{re.fri.ge.rar}{\verboinum{1}}
\verb{refrigério}{}{}{}{}{s.m.}{Ato ou efeito de refrigerar.}{re.fri.gé.rio}{0}
\verb{refrigério}{}{}{}{}{}{Sensação agradável produzida pela frescura.}{re.fri.gé.rio}{0}
\verb{refrigério}{}{Fig.}{}{}{}{Consolo, alívio de qualquer natureza; conforto moral.}{re.fri.gé.rio}{0}
\verb{refugar}{}{}{}{}{v.t.}{Não aceitar alguma coisa; recusar, rejeitar.}{re.fu.gar}{\verboinum{5}}
\verb{refugiar"-se}{}{}{}{}{v.pron.}{Procurar refúgio em algum lugar; abrigar"-se.}{re.fu.gi.ar"-se}{\verboinum{1}}
\verb{refúgio}{}{}{}{}{s.m.}{Lugar em que pessoa ou animal busca proteção; abrigo.}{re.fú.gio}{0}
\verb{refugir}{}{}{}{}{}{Tornar"-se isento, desobrigado; furtar"-se, eximir"-se, esquivar"-se.}{re.fu.gir}{0}
\verb{refugir}{}{}{}{}{}{Fugir ou desviar"-se de; evitar.}{re.fu.gir}{0}
\verb{refugir}{}{}{}{}{v.t.}{Tornar a fugir.}{re.fu.gir}{0}
\verb{refugir}{}{}{}{}{}{Movimentar"-se em sentido contrário àquele em que antes seguia; refluir, retroceder.}{re.fu.gir}{\verboinum{22}}
\verb{refugo}{}{}{}{}{s.m.}{Coisa que não presta mais e se joga fora; rebotalho.}{re.fu.go}{0}
\verb{refulgente}{}{}{}{}{adj.2g.}{Muito brilhante.}{re.ful.gen.te}{0}
\verb{refulgir}{}{}{}{}{v.i.}{Brilhar muito.}{re.ful.gir}{\verboinum{22}}
\verb{refundir}{}{}{}{}{v.t.}{Fundir, derreter novamente.}{re.fun.dir}{\verboinum{18}}
\verb{refutação}{}{}{"-ões}{}{s.f.}{Ação de refutar; contestação, réplica.}{re.fu.ta.ção}{0}
\verb{refutador}{ô}{}{}{}{adj.}{Que refuta, contesta.}{re.fu.ta.dor}{0}
\verb{refutar}{}{}{}{}{v.t.}{Argumentar contra; contestar.}{re.fu.tar}{\verboinum{1}}
\verb{rega}{é}{}{}{}{s.f.}{Ato ou efeito de regar; regadio.}{re.ga}{0}
\verb{rega}{é}{}{}{}{}{Chuva.}{re.ga}{0}
\verb{rega"-bofe}{é-ó}{Pop.}{rega"-bofes ⟨é-ó⟩}{}{s.m.}{Festa com muita comida e bebida.}{re.ga"-bo.fe}{0}
\verb{regaço}{}{}{}{}{s.m.}{Espaço entre o abdômen e as coxas de alguém sentado; colo.}{re.ga.ço}{0}
\verb{regador}{ô}{}{}{}{adj.}{Que rega; molhador.}{re.ga.dor}{0}
\verb{regador}{ô}{}{}{}{s.m.}{Recipiente com bico, próprio para regar planta.}{re.ga.dor}{0}
\verb{regalado}{}{}{}{}{adj.}{Que é tratado com regalo.}{re.ga.la.do}{0}
\verb{regalado}{}{}{}{}{}{Que agrada ou deleita.}{re.ga.la.do}{0}
\verb{regalado}{}{}{}{}{}{Farto, abundante.}{re.ga.la.do}{0}
\verb{regalar}{}{}{}{}{v.t.}{Causar prazer a alguém com alguma coisa.}{re.ga.lar}{\verboinum{1}}
\verb{regalia}{}{}{}{}{s.f.}{Privilégio, vantagem.}{re.ga.li.a}{0}
\verb{regalo}{}{}{}{}{s.m.}{Coisa que se dá a uma pessoa para agradá"-la; presente.}{re.ga.lo}{0}
\verb{regar}{}{}{}{}{v.t.}{Banhar, molhar.}{re.gar}{0}
\verb{regar}{}{Pop.}{}{}{}{Servir muita bebida em festa ou refeição.}{re.gar}{\verboinum{5}}
\verb{regata}{}{}{}{}{s.f.}{Corrida de embarcações.}{re.ga.ta}{0}
\verb{regata}{}{}{}{}{}{Camisa sem manga.}{re.ga.ta}{0}
\verb{regatear}{}{}{}{}{v.i.}{Discutir com o vendedor para abaixar o preço de uma mercadoria; pechinchar.}{re.ga.te.ar}{\verboinum{4}}
\verb{regateira}{ê}{}{}{}{s.f.}{Mulher que regateia.}{re.ga.tei.ra}{0}
\verb{regateira}{ê}{}{}{}{}{Mulher que compra mantimentos para os revender pelas ruas; vendedora ambulante.}{re.ga.tei.ra}{0}
\verb{regateira}{ê}{}{}{}{}{Mulher que usa expressões grosseiras.}{re.ga.tei.ra}{0}
\verb{regateira}{ê}{Pop.}{}{}{}{Mulher assanhada.}{re.ga.tei.ra}{0}
\verb{regato}{}{}{}{}{s.m.}{Rio pequeno; córrego, riacho.}{re.ga.to}{0}
\verb{regelar}{}{}{}{}{v.t.}{Congelar, gelar.}{re.ge.lar}{\verboinum{1}}
\verb{regência}{}{}{}{}{s.f.}{Governo interino na falta de soberano.}{re.gên.cia}{0}
\verb{regência}{}{Gram.}{}{}{}{Relação entre termos de uma oração ou entre orações de um período.}{re.gên.cia}{0}
\verb{regência}{}{}{}{}{}{Condução de execução musical feita por um maestro.}{re.gên.cia}{0}
\verb{regencial}{}{}{"-ais}{}{adj.2g.}{Relativo à regência.}{re.gen.ci.al}{0}
\verb{regeneração}{}{}{"-ões}{}{s.f.}{Nova geração ou formação.}{re.ge.ne.ra.ção}{0}
\verb{regeneração}{}{}{"-ões}{}{}{Reconstituição do que estava parcialmente destruído.}{re.ge.ne.ra.ção}{0}
\verb{regeneração}{}{}{"-ões}{}{}{Recuperação moral; reabilitação.}{re.ge.ne.ra.ção}{0}
%\verb{}{}{}{}{}{}{}{}{0}
\verb{regenerar}{}{}{}{}{v.t.}{Fazer com que um ser vivo ou parte dele volte ao estado em que estava antes; recuperar.}{re.ge.ne.rar}{0}
\verb{regenerar}{}{}{}{}{}{Fazer alguém voltar a ter bom comportamento; corrigir, emendar, recuperar.}{re.ge.ne.rar}{\verboinum{1}}
\verb{regenerativo}{}{}{}{}{adj.}{Diz"-se do que pode regenerar.}{re.ge.ne.ra.ti.vo}{0}
\verb{regente}{}{}{}{}{adj.2g.}{Que rege.}{re.gen.te}{0}
\verb{regente}{}{}{}{}{s.2g.}{Indivíduo que governa em lugar do rei ou do imperador durante algum tempo.}{re.gen.te}{0}
\verb{regente}{}{}{}{}{}{Indivíduo que dirige um grupo de cantores ou músicos; maestro.}{re.gen.te}{0}
\verb{reger}{ê}{}{}{}{v.t.}{Ser o chefe que dirige um país ou parte dele; administrar, governar.}{re.ger}{0}
\verb{reger}{ê}{}{}{}{}{Dirigir um grupo de pessoas em determinada atividade.}{re.ger}{\verboinum{16}}
\verb{região}{}{}{"-ões}{}{s.f.}{Grande extensão de terras.}{re.gi.ão}{0}
\verb{região}{}{}{"-ões}{}{}{Cada uma das partes de um país que apresentam características próprias.}{re.gi.ão}{0}
\verb{região}{}{}{"-ões}{}{}{Cada uma das partes em que se divide o corpo humano.}{re.gi.ão}{0}
\verb{regicida}{}{}{}{}{adj.2g.}{Diz"-se de assassino de rei ou rainha.}{re.gi.ci.da}{0}
\verb{regicídio}{}{}{}{}{s.m.}{Assassinato de rei ou rainha.}{re.gi.cí.dio}{0}
\verb{regime}{}{}{}{}{s.m.}{Forma de governar um país.}{re.gi.me}{0}
\verb{regime}{}{}{}{}{}{Modo de se alimentar para manter ou recuperar a saúde ou forma física; dieta.}{re.gi.me}{0}
\verb{regímem}{}{}{}{}{}{Var. de \textit{regime}.}{re.gí.mem}{0}
\verb{regimento}{}{}{}{}{s.m.}{Conjunto de regras que dirigem uma organização.}{re.gi.men.to}{0}
\verb{regimento}{}{}{}{}{}{Grupo de soldados comandados por um coronel.}{re.gi.men.to}{0}
\verb{régio}{}{}{}{}{adj.}{Que pertence ou diz respeito ao rei; que emana do rei.}{ré.gio}{0}
\verb{régio}{}{Fig.}{}{}{}{Que tem suntuosidade; magnífico.}{ré.gio}{0}
\verb{regional}{}{}{"-ais}{}{adj.2g.}{Que pertence ou é próprio de uma região.}{re.gi.o.nal}{0}
\verb{regional}{}{}{"-ais}{}{s.m.}{Conjunto musical que executa composições próprias de uma região, usando instrumentos típicos locais.}{re.gi.o.nal}{0}
\verb{regionalismo}{}{}{}{}{s.m.}{Palavra ou locução própria de uma região.}{re.gi.o.na.lis.mo}{0}
\verb{regionalismo}{}{}{}{}{}{Caráter da literatura baseada na cultura de uma região.}{re.gi.o.na.lis.mo}{0}
\verb{regionalismo}{}{}{}{}{}{Aquilo que diz respeito a uma só região.}{re.gi.o.na.lis.mo}{0}
\verb{regionalista}{}{}{}{}{adj.2g.}{Que se refere particularmente a uma região.}{re.gi.o.na.lis.ta}{0}
\verb{regionalista}{}{}{}{}{}{Que segue ou cultiva o regionalismo.}{re.gi.o.na.lis.ta}{0}
\verb{regirar}{}{}{}{}{v.t.}{Fazer mover em roda; fazer girar.}{re.gi.rar}{\verboinum{1}}
\verb{registar}{}{}{}{}{}{Var. de \textit{registrar}.}{re.gis.tar}{0}
\verb{registo}{}{}{}{}{}{Var. de \textit{registro}.}{re.gis.to}{0}
\verb{registradora}{ô}{}{}{}{s.f.}{Máquina que registra o produto das vendas, nas casas comerciais.}{re.gis.tra.do.ra}{0}
\verb{registrar}{}{}{}{}{v.t.}{Anotar em livro especial; inscrever.}{re.gis.trar}{0}
\verb{registrar}{}{}{}{}{}{Gravar na memória; anotar, historiar.}{re.gis.trar}{0}
\verb{registrar}{}{}{}{}{}{Marcar o valor a ser pago. (\textit{O supermercado registrava todas as ofertas  nos caixas.})}{re.gis.trar}{0}
\verb{registrar}{}{}{}{}{}{Enviar correspondência com seguro pelo correio.}{re.gis.trar}{0}
\verb{registrar}{}{}{}{}{}{Mencionar; referir; assinalar.}{re.gis.trar}{\verboinum{1}}
\verb{registro}{}{}{}{}{s.m.}{Ato ou efeito de registrar.}{re.gis.tro}{0}
\verb{registro}{}{}{}{}{}{Livro em que são inscritos determinados fatos ou documentos.}{re.gis.tro}{0}
\verb{registro}{}{}{}{}{}{Certidão de nascimento. (\textit{O governo pretende que todos os bebês tenham registro no cartório, logo ao nascer.})}{re.gis.tro}{0}
\verb{registro}{}{}{}{}{}{Anotação, marca. (\textit{Fizemos registros durante a pesquisa.})}{re.gis.tro}{0}
\verb{registro}{}{}{}{}{}{Seguro do correio.}{re.gis.tro}{0}
\verb{registro}{}{}{}{}{}{Torneira ou válvula para parar ou regular o fluxo de um líquido através de um cano.}{re.gis.tro}{0}
\verb{registro}{}{}{}{}{}{Aparelho que mostra o que se consumiu; relógio. (\textit{O encanador já instalou o registro de água.})}{re.gis.tro}{0}
\verb{rego}{ê}{}{}{}{s.m.}{Sulco para escoar água; valeta.}{re.go}{0}
\verb{regougar}{}{}{}{}{v.i.}{Emitir regougos.}{re.gou.gar}{0}
\verb{regougar}{}{Fig.}{}{}{}{Falar com pouca clareza; resmungar.}{re.gou.gar}{\verboinum{5}}
\verb{regougo}{ô}{}{}{}{s.m.}{Voz da raposa, ou qualquer outra voz ou som que a imite.}{re.gou.go}{0}
\verb{regougo}{ô}{}{}{}{}{Ronco, roncadura.}{re.gou.go}{0}
\verb{regozijar}{}{}{}{}{v.t.}{Alegrar muito.}{re.go.zi.jar}{\verboinum{1}}
\verb{regozijo}{}{}{}{}{s.m.}{Alegria ou satisfação intensa.}{re.go.zi.jo}{0}
\verb{regra}{é}{}{}{}{s.f.}{O que regula, rege, governa; norma.}{re.gra}{0}
\verb{regra}{é}{}{}{}{}{Ordem, método.}{re.gra}{0}
\verb{regrado}{}{}{}{}{adj.}{Riscado com régua.}{re.gra.do}{0}
\verb{regrado}{}{}{}{}{}{Bem"-comportado, moderado, sensato.}{re.gra.do}{0}
\verb{regrar}{}{}{}{}{v.t.}{Submeter a certas regras; regular.}{re.grar}{0}
\verb{regrar}{}{}{}{}{}{Comedir, moderar.}{re.grar}{\verboinum{1}}
\verb{regredir}{}{}{}{}{v.t.}{Ir para trás; retroceder.}{re.gre.dir}{\verboinum{18}}
\verb{regressão}{}{}{"-ões}{}{s.f.}{Ato ou efeito de regressar; retorno, regresso.}{re.gres.são}{0}
\verb{regressão}{}{}{"-ões}{}{}{Ato ou efeito de regredir; retrocesso.}{re.gres.são}{0}
\verb{regressar}{}{}{}{}{v.t.}{Voltar ao ponto de partida; retornar.}{re.gres.sar}{\verboinum{1}}
\verb{regressivo}{}{}{}{}{adj.}{Que retrocede; regride.}{re.gres.si.vo}{0}
\verb{regresso}{é}{}{}{}{s.m.}{Ato ou efeito de regressar; retorno.}{re.gres.so}{0}
\verb{régua}{}{}{}{}{s.f.}{Instrumento plano para traçar linhas, especialmente retas.}{ré.gua}{0}
\verb{régua}{}{}{}{}{}{Esse instrumento graduado para medir.}{ré.gua}{0}
\verb{regulado}{}{}{}{}{adj.}{Que funciona ou ocorre com irregularidade.}{re.gu.la.do}{0}
\verb{regulador}{ô}{}{}{}{adj.}{Que regula ou serve para regular.}{re.gu.la.dor}{0}
\verb{regulagem}{}{}{"-ens}{}{s.f.}{Ato ou efeito de regular, de ajustar máquinas, motores etc.}{re.gu.la.gem}{0}
\verb{regulamentação}{}{}{"-ões}{}{s.f.}{Ato ou efeito de regulamentar, de impor regulamento.}{re.gu.la.men.ta.ção}{0}
\verb{regulamentação}{}{}{"-ões}{}{}{Conjunto das medidas legais ou regulamentares que regem um assunto, uma instituição, um instituto.}{re.gu.la.men.ta.ção}{0}
\verb{regulamentar}{}{}{}{}{adj.2g.}{Relativo a regulamento; regimental.}{re.gu.la.men.tar}{0}
\verb{regulamentar}{}{}{}{}{v.t.}{Submeter algo a regulamento.}{re.gu.la.men.tar}{\verboinum{1}}
\verb{regulamento}{}{}{}{}{s.m.}{Conjunto de regras; estatuto.}{re.gu.la.men.to}{0}
\verb{regular}{}{}{}{}{adj.2g.}{Que segue as regras.}{re.gu.lar}{0}
\verb{regular}{}{}{}{}{}{Nem bom nem mau; aceitável, mediano.}{re.gu.lar}{0}
\verb{regular}{}{}{}{}{v.t.}{Dar regras para alguma atividade.}{re.gu.lar}{0}
\verb{regular}{}{}{}{}{}{Fazer uma máquina ter bom funcionamento; acertar, ajustar.}{re.gu.lar}{0}
\verb{regular}{}{}{}{}{v.pron.}{Seguir alguma regra; guiar"-se, orientar"-se.}{re.gu.lar}{\verboinum{1}}
\verb{regularidade}{}{}{}{}{s.f.}{Qualidade do que é regular.}{re.gu.la.ri.da.de}{0}
\verb{regularizar}{}{}{}{}{v.t.}{Tornar regular; normalizar.}{re.gu.la.ri.zar}{0}
\verb{regularizar}{}{}{}{}{}{Tornar razoável; conveniente.}{re.gu.la.ri.zar}{0}
\verb{regularizar}{}{}{}{}{}{Pôr em ordem; pôr em dia; corrigir.}{re.gu.la.ri.zar}{\verboinum{1}}
\verb{régulo}{}{}{}{}{s.m.}{Rei ainda criança.}{ré.gu.lo}{0}
\verb{régulo}{}{}{}{}{}{Chefe tirânico.}{ré.gu.lo}{0}
\verb{regurgitação}{}{}{"-ões}{}{s.f.}{Ato ou efeito de regurgitar; transbordamento; refluxo.}{re.gur.gi.ta.ção}{0}
\verb{regurgitar}{}{}{}{}{v.t.}{Estar repleto; estar muito cheio; transbordar.}{re.gur.gi.tar}{0}
\verb{regurgitar}{}{}{}{}{}{Expelir o excedente, especialmente o conteúdo gástrico.}{re.gur.gi.tar}{\verboinum{1}}
\verb{rei}{ê}{}{}{}{s.m.}{Autoridade suprema de uma monarquia; soberano.}{rei}{0}
\verb{rei}{ê}{}{}{}{}{Certa carta do baralho.}{rei}{0}
\verb{rei}{ê}{}{}{}{}{Peça principal do xadrez.}{rei}{0}
\verb{reide}{ê}{}{}{}{s.m.}{Excursão a pé, a cavalo, de bicicleta, automóvel etc., por longas distâncias.}{rei.de}{0}
\verb{reimpressão}{}{}{"-ões}{}{s.f.}{Ato ou efeito de reimprimir.}{re.im.pres.são}{0}
\verb{reimprimir}{}{}{}{}{v.t.}{Imprimir de novo, fazer nova impressão de uma obra.}{re.im.pri.mir}{\verboinum{18}}
\verb{reinação}{}{Bras.}{"-ões}{}{s.f.}{Travessura, traquinagem, arte. (\textit{Crianças fazem reinações.})}{rei.na.ção}{0}
\verb{reinado}{}{}{}{}{s.m.}{Tempo de governo de um soberano (rei, imperador, príncipe etc.). }{rei.na.do}{0}
\verb{reinado}{}{Fig.}{}{}{}{Espaço de tempo no qual predomina alguma coisa (moda, comportamento, estilo etc.).}{rei.na.do}{0}
\verb{reinador}{ô}{Bras.}{}{}{adj.}{Que reina, que é dado a reinações, travessuras.}{rei.na.dor}{0}
\verb{reinador}{ô}{}{}{}{s.m.}{Pessoa, geralmente criança, dada a reinações, travessuras.}{rei.na.dor}{0}
\verb{reinante}{}{}{}{}{adj.2g.}{Que reina.}{rei.nan.te}{0}
\verb{reinante}{}{}{}{}{s.2g.}{Pessoa que reina (rei ou rainha).}{rei.nan.te}{0}
\verb{reinar}{}{}{}{}{v.i.}{Governar um Estado como rei ou soberano.}{rei.nar}{0}
\verb{reinar}{}{}{}{}{}{Fazer travessuras; brincar.}{rei.nar}{0}
\verb{reinar}{}{}{}{}{v.t.}{Exercer influência; dominar.}{rei.nar}{\verboinum{1}}
\verb{reincidência}{}{}{}{}{s.f.}{Ato ou efeito de reincidir.}{re.in.ci.dên.cia}{0}
\verb{reincidente}{}{}{}{}{adj.2g.}{Que reincide; recidivo.}{re.in.ci.den.te}{0}
\verb{reincidente}{}{}{}{}{s.2g.}{Pessoa que comete de novo um erro, crime etc.}{re.in.ci.den.te}{0}
\verb{reincidir}{}{}{}{}{v.t.}{Incidir de novo, voltar a fazer uma mesma coisa; recair, repetir.}{re.in.ci.dir}{0}
\verb{reincidir}{}{Jur.}{}{}{}{Cometer de novo um crime ou delito.}{re.in.ci.dir}{\verboinum{18}}
\verb{reincorporação}{}{}{"-ões}{}{s.f.}{Ato ou efeito de reincorporar.}{re.in.cor.po.ra.ção}{0}
\verb{reincorporar}{}{}{}{}{v.t.}{Incorporar de novo.}{re.in.cor.po.rar}{\verboinum{1}}
\verb{reingressar}{}{}{}{}{v.t.}{Ingressar de novo.}{re.in.gres.sar}{\verboinum{1}}
\verb{reiniciar}{}{}{}{}{v.t.}{Iniciar de novo; recomeçar.}{re.i.ni.ci.ar}{0}
\verb{reiniciar}{}{Informát.}{}{}{}{Fazer com que um computador em operação retorne ao processo de iniciação, executando novamente as rotinas de teste, carregamento do sistema operacional etc.}{re.i.ni.ci.ar}{\verboinum{1}}
\verb{reinício}{}{}{}{}{s.m.}{Ato ou efeito de reiniciar; recomeço. }{re.i.ní.cio}{0}
\verb{reino}{é}{}{}{}{s.m.}{Estado governado por um rei.}{rei.no}{0}
\verb{reino}{é}{Biol.}{}{}{}{Cada uma das três grandes divisões em que se agrupam os seres da natureza. (\textit{Os seres da natureza se dividem nos reinos animal, vegetal e mineral.})}{rei.no}{0}
\verb{reinol}{ó}{}{"-óis}{}{adj.2g.}{Relativo a ou natural do reino.}{rei.nol}{0}
\verb{reinol}{ó}{}{"-óis}{}{s.2g.}{Pessoa que é natural de um reino.}{rei.nol}{0}
\verb{reinscrever}{ê}{}{}{}{v.t.}{Inscrever de novo.}{re.ins.cre.ver}{\verboinum{12}}
\verb{reinstalar}{}{}{}{}{v.t.}{Instalar de novo.}{re.ins.ta.lar}{\verboinum{1}}
\verb{reintegração}{}{}{"-ões}{}{s.f.}{Ato ou efeito de reintegrar.}{re.in.te.gra.ção}{0}
\verb{reintegrar}{}{}{}{}{v.t.}{Restabelecer alguém na posse de alguma coisa, como bens ou emprego.}{re.in.te.grar}{0}
\verb{reintegrar}{}{}{}{}{}{Repor no mesmo lugar; reconduzir.}{re.in.te.grar}{\verboinum{1}}
\verb{reintroduzir}{}{}{}{}{v.t.}{Introduzir de novo.}{re.in.tro.du.zir}{\verboinum{21}}
\verb{reinvestir}{}{}{}{}{v.t.}{Investir de novo.}{re.in.ves.tir}{\verboinum{29}}
\verb{réis}{}{}{}{}{s.m.pl.}{Plural de \textit{real}, antiga moeda do Brasil.}{réis}{0}
\verb{reisado}{}{Bras.}{}{}{s.m.}{Dança dramática popular com que se festeja o dia de Reis.}{rei.sa.do}{0}
\verb{reiterar}{}{}{}{}{v.t.}{Dizer de novo; repetir, insistir, iterar.}{re.i.te.rar}{\verboinum{1}}
\verb{reiterativo}{}{}{}{}{adj.}{Que reitera ou serve para reiterar.}{re.i.te.ra.ti.vo}{0}
\verb{reitor}{ô}{}{}{}{s.m.}{Pessoa que rege, governa ou administra.}{rei.tor}{0}
\verb{reitor}{ô}{}{}{}{}{Pessoa encarregada de dirigir uma universidade.}{rei.tor}{0}
\verb{reitorado}{}{}{}{}{s.m.}{Cargo ocupado por reitor; reitoria.}{rei.to.ra.do}{0}
\verb{reitorado}{}{}{}{}{}{Tempo que dura esse cargo.}{rei.to.ra.do}{0}
\verb{reitoria}{}{}{}{}{s.f.}{Cargo ocupado por reitor.}{rei.to.ri.a}{0}
\verb{reitoria}{}{}{}{}{}{Prédio ou gabinete do reitor.}{rei.to.ri.a}{0}
\verb{reiúno}{}{}{}{}{adj.}{Que é fornecido pelo Estado, para uso dos soldados.}{rei.ú.no}{0}
\verb{reiúno}{}{}{}{}{}{De má qualidade; ruim, ordinário, inferior.}{rei.ú.no}{0}
\verb{reiúno}{}{}{}{}{s.m.}{Animal que pertence ao Estado, ou que não tem dono.}{rei.ú.no}{0}
\verb{reivindicação}{}{}{"-ões}{}{s.f.}{Ato ou efeito de reivindicar.}{re.i.vin.di.ca.ção}{0}
\verb{reivindicador}{ô}{}{}{}{adj.}{Que reivindica; reivindicante.   }{rei.vin.di.ca.dor}{0}
\verb{reivindicador}{ô}{}{}{}{s.m.}{Pessoa que reivindica.}{rei.vin.di.ca.dor}{0}
\verb{reivindicante}{}{}{}{}{adj.2g.}{Reivindicador.}{rei.vin.di.can.te}{0}
\verb{reivindicar}{}{}{}{}{v.t.}{Reclamar para si o que é de seu direito; exigir.}{rei.vin.di.car}{\verboinum{2}}
\verb{rejeição}{}{}{"-ões}{}{s.f.}{Ato ou efeito de rejeitar; recusa, repúdio.}{re.jei.ção}{0}
\verb{rejeitar}{}{}{}{}{v.t.}{Lançar fora; repelir. (\textit{O organismo rejeitou o medicamento.})}{re.jei.tar}{0}
\verb{rejeitar}{}{}{}{}{}{Não aceitar; recusar, repudiar. (\textit{A diretoria rejeitou ontem nossa proposta.})}{re.jei.tar}{\verboinum{1}}
\verb{rejeito}{ê}{}{}{}{s.m.}{Resíduos do processamento do urânio.}{re.jei.to}{0}
\verb{rejubilar}{}{}{}{}{v.t.}{Encher de júbilo, de grande alegria.}{re.ju.bi.lar}{\verboinum{1}}
\verb{rejuntar}{}{}{}{}{v.t.}{Tapar as juntas de tijolo, azulejo etc., para melhor vedação.}{re.jun.tar}{\verboinum{1}}
\verb{rejuvenescer}{ê}{}{}{}{v.t.}{Tornar jovem, aparentar juventude; remoçar.}{re.ju.ve.nes.cer}{0}
\verb{rejuvenescer}{ê}{}{}{}{v.i.}{Parecer jovem sem o ser.}{re.ju.ve.nes.cer}{\verboinum{15}}
\verb{rejuvenescimento}{}{}{}{}{s.m.}{Ato ou efeito de rejuvenescer.}{re.ju.ve.nes.ci.men.to}{0}
\verb{relação}{}{}{"-ões}{}{s.f.}{Vínculo que existe entre duas coisas ou pessoas; correspondência, conexão, ligação.}{re.la.ção}{0}
\verb{relação}{}{}{"-ões}{}{}{Lista de pessoas ou coisas; rol, listagem.}{re.la.ção}{0}
\verb{relação}{}{}{"-ões}{}{}{Convivência entre pessoas; relacionamento.}{re.la.ção}{0}
\verb{relacionamento}{}{Bras.}{}{}{}{Amizade, ligação, intimidade.}{re.la.ci.o.na.men.to}{0}
\verb{relacionamento}{}{}{}{}{s.m.}{Ato ou efeito de relacionar.}{re.la.ci.o.na.men.to}{0}
\verb{relacionar}{}{}{}{}{v.t.}{Estabelecer uma ligação entre fatos diversos.}{re.la.ci.o.nar}{0}
\verb{relacionar}{}{}{}{}{}{Fazer lista de pessoas ou coisas; arrolar.}{re.la.ci.o.nar}{0}
\verb{relacionar}{}{}{}{}{}{Relatar, referir, narrar.}{re.la.ci.o.nar}{0}
\verb{relacionar}{}{}{}{}{v.pron.}{Fazer amizades; travar conhecimentos; conviver.}{re.la.ci.o.nar}{\verboinum{1}}
%\verb{}{}{}{}{}{}{}{}{0}
%\verb{}{}{}{}{}{}{}{}{0}
\verb{relâmpago}{}{}{}{}{s.m.}{Clarão rápido e intenso causado por descarga elétrica atmosférica.}{re.lâm.pa.go}{0}
\verb{relâmpago}{}{}{}{}{adj.2g.}{Rápido, instantâneo como um relâmpago. (\textit{Os policiais fizeram uma operação relâmpago nos pedágios.})}{re.lâm.pa.go}{0}
\verb{relampaguear}{}{}{}{}{v.i.}{Produzir"-se relâmpago(s); relampejar.}{re.lam.pa.gue.ar}{\verboinum{4}}
\verb{relampear}{}{}{}{}{v.i.}{Relampejar; relampaguear.}{re.lam.pe.ar}{\verboinum{4}}
\verb{relampejar}{}{}{}{}{v.i.}{Relampear; relampaguear.}{re.lam.pe.jar}{\verboinum{1}}
\verb{relance}{}{}{}{}{s.m.}{Ato ou efeito de relancear.}{re.lan.ce}{0}
\verb{relancear}{}{}{}{}{v.t.}{Dirigir os olhos rapidamente para alguém ou alguma coisa. }{re.lan.ce.ar}{0}
\verb{relancear}{}{}{}{}{}{Olhar de relance.}{re.lan.ce.ar}{\verboinum{4}}
\verb{relapso}{}{}{}{}{adj.}{Que reincide no erro, que comete de novo uma falta.}{re.lap.so}{0}
\verb{relapso}{}{}{}{}{}{Que é descuidado no cumprimento de seus deveres ou obrigações; desatento, negligente, relaxado, displicente.}{re.lap.so}{0}
\verb{relapso}{}{}{}{}{s.m.}{Essa pessoa.}{re.lap.so}{0}
\verb{relatar}{}{}{}{}{v.t.}{Fazer um relato; contar, narrar, expor.}{re.la.tar}{\verboinum{1}}
\verb{relatividade}{}{}{}{}{s.f.}{Qualidade do que é relativo.}{re.la.ti.vi.da.de}{0}
\verb{relativismo}{}{Filos.}{}{}{s.m.}{Teoria que afirma a relatividade do conhecimento humano e a impossibilidade de se conhecer o absoluto e a verdade.}{re.la.ti.vis.mo}{0}
\verb{relativo}{}{}{}{}{adj.}{Que exprime relação; referente, concernente.}{re.la.ti.vo}{0}
\verb{relativo}{}{}{}{}{}{Que não é absoluto; condicional, dependente.}{re.la.ti.vo}{0}
\verb{relativo}{}{Gram.}{}{}{}{Diz"-se do pronome que relaciona uma oração a uma palavra ou sentido antecedente.}{re.la.ti.vo}{0}
\verb{relato}{}{}{}{}{s.m.}{Ato ou efeito de relatar; narração, exposição.}{re.la.to}{0}
\verb{relator}{ô}{}{}{}{adj.}{Que relata.}{re.la.tor}{0}
\verb{relator}{ô}{}{}{}{s.m.}{Pessoa encarregada de escrever um relatório.}{re.la.tor}{0}
\verb{relatório}{}{}{}{}{s.m.}{Exposição detalhada, verbal ou escrita, de fatos, acontecimentos, atividades etc.}{re.la.tó.rio}{0}
\verb{relaxação}{ch}{}{"-ões}{}{s.f.}{Ato ou efeito de relaxar; relaxamento.}{re.la.xa.ção}{0}
\verb{relaxado}{ch}{}{}{}{adj.}{Não contraído; distendido, descansado.}{re.la.xa.do}{0}
\verb{relaxado}{ch}{Fig.}{}{}{}{Descuidado, desleixado, desmazelado.}{re.la.xa.do}{0}
\verb{relaxador}{ch\ldots{}ô}{}{}{}{adj.}{Que relaxa; relaxante.}{re.la.xa.dor}{0}
\verb{relaxamento}{ch}{}{}{}{s.m.}{Ato ou efeito de relaxar; relaxação.}{re.la.xa.men.to}{0}
\verb{relaxamento}{ch}{}{}{}{}{Diminuição da tensão; descontração, distensão.}{re.la.xa.men.to}{0}
\verb{relaxamento}{ch}{Fig.}{}{}{}{Negligência, desmazelo, desleixo.}{re.la.xa.men.to}{0}
\verb{relaxante}{ch}{}{}{}{adj.2g.}{Que relaxa, que faz diminuir a tensão física ou mental; repousante.}{re.la.xan.te}{0}
\verb{relaxante}{ch}{Farm.}{}{}{}{Diz"-se do medicamento que faz relaxar os músculos.}{re.la.xan.te}{0}
\verb{relaxar}{ch}{}{}{}{v.t.}{Diminuir a força; afrouxar, desapertar.}{re.la.xar}{0}
\verb{relaxar}{ch}{}{}{}{}{Descansar, descontrair, repousar.}{re.la.xar}{0}
\verb{relaxar}{ch}{}{}{}{}{Negligenciar, desmazelar, descuidar.}{re.la.xar}{\verboinum{1}}
\verb{relé}{}{}{}{}{s.m.}{Dispositivo destinado a abrir ou fechar contatos elétricos, a fim de estabelecer ou interromper circuitos.}{re.lé}{0}
\verb{relegar}{}{}{}{}{v.t.}{Afastar de um lugar para outro; banir, expatriar.}{re.le.gar}{0}
\verb{relegar}{}{}{}{}{}{Afastar com desdém; desprezar.}{re.le.gar}{\verboinum{5}}
\verb{relembrar}{}{}{}{}{v.t.}{Tornar a lembrar; recordar.}{re.lem.brar}{\verboinum{1}}
\verb{relento}{}{}{}{}{s.m.}{Umidade da noite; sereno.}{re.len.to}{0}
\verb{reler}{ê}{}{}{}{v.t.}{Ler de novo ou repetidamente.}{re.ler}{0}
\verb{reler}{ê}{}{}{}{}{Rever o que se escreveu.}{re.ler}{\verboinum{12}}
\verb{reles}{é}{}{}{}{adj.2g.}{Sem importância, insignificante, mero, simples.}{re.les}{0}
\verb{relevância}{}{}{}{}{s.f.}{Qualidade de relevante.}{re.le.vân.cia}{0}
\verb{relevância}{}{}{}{}{}{Parte protuberante em superfície lisa; saliência.}{re.le.vân.cia}{0}
\verb{relevância}{}{}{}{}{}{Aquilo que se destaca em escala comparativa ou de valores; importância, relevo.}{re.le.vân.cia}{0}
\verb{relevante}{}{}{}{}{adj.2g.}{Que é importante ou essencial.}{re.le.van.te}{0}
\verb{relevar}{}{}{}{}{v.t.}{Mostrar a importância de pessoa ou coisa; destacar, salientar.}{re.le.var}{0}
\verb{relevar}{}{}{}{}{}{Deixar de considerar o erro de alguém; desculpar, perdoar.}{re.le.var}{\verboinum{1}}
\verb{relevo}{ê}{}{}{}{s.m.}{Parte mais alta de uma superfície; saliência.}{re.le.vo}{0}
\verb{relevo}{ê}{}{}{}{}{Grande importância; destaque.}{re.le.vo}{0}
\verb{relevo}{ê}{}{}{}{}{Conjunto de montanhas, vales, planícies etc.}{re.le.vo}{0}
\verb{relha}{ê}{}{}{}{s.f.}{Peça de arado que perfura o solo.}{re.lha}{0}
\verb{relha}{ê}{}{}{}{}{Peça que reforça as rodas de carro de boi.}{re.lha}{0}
\verb{relhada}{}{}{}{}{s.f.}{Açoite ou pancada com o relho; chibatada, chicotada.}{re.lha.da}{0}
\verb{relho}{ê}{}{}{}{s.m.}{Chicote de couro torcido usado para fustigar animais; chibata.}{re.lho}{0}
\verb{relicário}{}{}{}{}{s.m.}{Local para guardar relíquia.}{re.li.cá.rio}{0}
\verb{religar}{}{}{}{}{v.t.}{Ligar novamente ou com mais segurança.}{re.li.gar}{0}
\verb{religar}{}{}{}{}{}{Atar, ligar bem.}{re.li.gar}{\verboinum{5}}
\verb{religião}{}{}{"-ões}{}{s.f.}{Crença e culto à divindade.}{re.li.gi.ão}{0}
\verb{religião}{}{}{"-ões}{}{}{Conjunto de dogmas que geralmente envolvem tal culto.}{re.li.gi.ão}{0}
\verb{religiosa}{ó}{}{}{}{s.f.}{Mulher que fez votos monásticos; freira.}{re.li.gi.o.sa}{0}
\verb{religiosidade}{}{}{}{}{s.f.}{Qualidade do que é religioso.}{re.li.gi.o.si.da.de}{0}
\verb{religiosidade}{}{}{}{}{}{Tendência para os sentimentos religiosos, para as coisas sagradas.}{re.li.gi.o.si.da.de}{0}
\verb{religiosidade}{}{}{}{}{}{Conjunto de escrúpulos religiosos ou valores éticos que apresentam certo teor religioso.}{re.li.gi.o.si.da.de}{0}
\verb{religioso}{ô}{}{"-osos ⟨ó⟩}{"-osa ⟨ó⟩}{adj.}{Que segue ou professa uma religião.}{re.li.gi.o.so}{0}
\verb{religioso}{ô}{}{"-osos ⟨ó⟩}{"-osa ⟨ó⟩}{}{Próprio de religião.}{re.li.gi.o.so}{0}
\verb{relinchar}{}{}{}{}{v.i.}{Soltar relinchos.}{re.lin.char}{\verboinum{1}}
\verb{relincho}{}{}{}{}{s.m.}{Som produzido pelo cavalo.}{re.lin.cho}{0}
\verb{relíquia}{}{}{}{}{s.f.}{Coisa preciosa, rara e geralmente antiga.}{re.lí.quia}{0}
\verb{relíquia}{}{}{}{}{}{O que resta do corpo ou dos objetos dos santos.}{re.lí.quia}{0}
\verb{relógio}{}{}{}{}{s.m.}{Aparelho para marcar as horas.}{re.ló.gio}{0}
\verb{relógio}{}{}{}{}{}{Medidor de água, luz etc.}{re.ló.gio}{0}
\verb{relógio}{}{}{}{}{}{Cronômetro.}{re.ló.gio}{0}
\verb{relógio}{}{}{}{}{}{Taxímetro.}{re.ló.gio}{0}
\verb{relojoaria}{}{}{}{}{s.f.}{Casa que fabrica, repara ou vende relógios.}{re.lo.jo.a.ri.a}{0}
\verb{relojoeiro}{ê}{}{}{}{s.m.}{Indivíduo que fabrica, repara ou vende relógios.}{re.lo.jo.ei.ro}{0}
\verb{relutância}{}{}{}{}{s.f.}{Qualidade de quem é relutante; obstinação, oposição, resistência, teimosia.}{re.lu.tân.cia}{0}
\verb{relutante}{}{}{}{}{adj.2g.}{Que reluta; obstinado, resistente, teimoso.}{re.lu.tan.te}{0}
\verb{relutar}{}{}{}{}{v.t.}{Mostrar pouca vontade de fazer alguma coisa; resistir.}{re.lu.tar}{\verboinum{1}}
\verb{reluzente}{}{}{}{}{adj.2g.}{Que reluz; brilhante, resplandescente.}{re.lu.zen.te}{0}
\verb{reluzir}{}{}{}{}{v.i.}{Brilhar muito; resplandecer.}{re.lu.zir}{\verboinum{21}}
\verb{relva}{é}{}{}{}{s.f.}{Camada de vegetação rasteira, que cobre o chão; grama.}{rel.va}{0}
\verb{relvado}{}{}{}{}{s.m.}{Terreno coberto de relva; gramado.}{rel.va.do}{0}
\verb{remada}{}{}{}{}{s.f.}{Ato ou efeito de remar.}{re.ma.da}{0}
\verb{remada}{}{}{}{}{}{Pancada com o remo.}{re.ma.da}{0}
%\verb{}{}{}{}{}{}{}{}{0}
\verb{remanchar}{}{}{}{}{v.i.}{Tardar em fazer alguma coisa; demorar"-se.}{re.man.char}{\verboinum{1}}
\verb{remanejar}{}{}{}{}{v.t.}{Tornar a manejar.}{re.ma.ne.jar}{0}
\verb{remanejar}{}{}{}{}{}{Recompor, refazer.}{re.ma.ne.jar}{0}
\verb{remanejar}{}{}{}{}{}{Redistribuir, transferir.}{re.ma.ne.jar}{\verboinum{1}}
\verb{remanescente}{}{}{}{}{adj.2g.}{Que remanesce, que sobeja, que resta.}{re.ma.nes.cen.te}{0}
\verb{remanescer}{ê}{}{}{}{v.i.}{Subsistir como sobrevivente, sobra, resto etc.; sobrar, restar.}{re.ma.nes.cer}{\verboinum{15}}
\verb{remanso}{}{}{}{}{s.m.}{Trecho de um rio em que as águas parecem estar paradas.}{re.man.so}{0}
\verb{remanso}{}{}{}{}{}{Lugar calmo e sossegado.}{re.man.so}{0}
\verb{remansoso}{ô}{}{"-osos ⟨ó⟩}{"-osa ⟨ó⟩}{adj.}{Que revela quietação, tranquilidade.}{re.man.so.so}{0}
\verb{remansoso}{ô}{}{"-osos ⟨ó⟩}{"-osa ⟨ó⟩}{}{Que demonstra lentidão; pachorrento, vagaroso.}{re.man.so.so}{0}
\verb{remar}{}{}{}{}{v.t.}{Fazer uma embarcação ir numa direção empurrando a água com remo para outra direção.}{re.mar}{\verboinum{1}}
\verb{remarcação}{}{}{"-ões}{}{s.f.}{Ato ou efeito de remarcar, de marcar novamente.}{re.mar.ca.ção}{0}
\verb{remarcação}{}{Bras.}{"-ões}{}{}{Alteração de preços em mercadorias.}{re.mar.ca.ção}{0}
\verb{remarcar}{}{}{}{}{v.t.}{Marcar algo novamente.}{re.mar.car}{0}
\verb{remarcar}{}{}{}{}{}{Fixar novo preço.}{re.mar.car}{\verboinum{2}}
\verb{rematado}{}{}{}{}{adj.}{Que se rematou; concluído, terminado.}{re.ma.ta.do}{0}
\verb{rematado}{}{}{}{}{}{Que chegou ao máximo de uma qualidade má.}{re.ma.ta.do}{0}
\verb{rematar}{}{}{}{}{v.t.}{Fazer acabamento em; arrematar.}{re.ma.tar}{\verboinum{1}}
\verb{remate}{}{}{}{}{s.m.}{Ato de rematar; acabamento, arremate.}{re.ma.te}{0}
\verb{remedar}{}{}{}{}{v.i.}{Imitar uma pessoa ou a maneira dela de fazer alguma coisa; arremedar.}{re.me.dar}{\verboinum{1}}
\verb{remediado}{}{}{}{}{adj.}{Que tem o suficiente para sobreviver.}{re.me.di.a.do}{0}
\verb{remediar}{}{}{}{}{v.t.}{Dar remédio.}{re.me.di.ar}{0}
\verb{remediar}{}{}{}{}{}{Corrigir, reparar.}{re.me.di.ar}{0}
\verb{remediar}{}{}{}{}{}{Prover do mais básico; arranjar.}{re.me.di.ar}{\verboinum{6}}
\verb{remédio}{}{}{}{}{s.m.}{Preparado próprio para combater e curar as doenças e aliviar dores.}{re.mé.dio}{0}
\verb{remédio}{}{}{}{}{}{Maneira de resolver uma dificuldade; recurso.}{re.mé.dio}{0}
\verb{remedo}{ê}{}{}{}{s.m.}{Ato ou efeito de arremedar; remedo.}{re.me.do}{0}
\verb{remedo}{ê}{}{}{}{}{Imitação grotesca, ridícula; paródia.}{re.me.do}{0}
\verb{remeiro}{ê}{}{}{}{adj.}{Que rema; remador.}{re.mei.ro}{0}
\verb{remela}{é}{}{}{}{s.f.}{Secreção nos olhos.}{re.me.la}{0}
\verb{remelento}{}{}{}{}{adj.}{Que tem remela.}{re.me.len.to}{0}
\verb{remelexo}{êch}{}{}{}{s.m.}{Movimento do corpo e dos quadris; bamboleio, rebolado.}{re.me.le.xo}{0}
\verb{rememorar}{}{}{}{}{v.t.}{Trazer novamente imagens à memória; relembrar.}{re.me.mo.rar}{0}
\verb{rememorar}{}{}{}{}{}{Produzir, por similaridade, ideias ou imagens na mente; lembrar.}{re.me.mo.rar}{\verboinum{1}}
\verb{rememorativo}{}{}{}{}{adj.}{Que rememora ou faz rememorar.}{re.me.mo.ra.ti.vo}{0}
\verb{remendão}{}{}{"-ões}{}{adj.}{Que faz remendos.}{re.men.dão}{0}
\verb{remendão}{}{}{"-ões}{}{}{Diz"-se de artesão de pouca habilidade.}{re.men.dão}{0}
\verb{remendão}{}{}{"-ões}{}{s.m.}{Sapateiro.}{re.men.dão}{0}
\verb{remendar}{}{}{}{}{v.t.}{Pôr remendo; consertar.}{re.men.dar}{0}
\verb{remendar}{}{Fig.}{}{}{}{Desfazer um erro; retificar, corrigir.}{re.men.dar}{\verboinum{1}}
\verb{remendo}{}{}{}{}{s.m.}{Pedaço de material com o qual se conserta ou oculta um rasgo, uma fissura, uma fenda, um furo.}{re.men.do}{0}
\verb{remessa}{é}{}{}{}{}{O objeto remetido.}{re.mes.sa}{0}
\verb{remessa}{é}{}{}{}{s.f.}{Ato ou efeito de remeter.}{re.mes.sa}{0}
\verb{remetente}{}{}{}{}{adj.2g.}{Que remete.}{re.me.ten.te}{0}
\verb{remeter}{ê}{}{}{}{v.t.}{Fazer ir (a algum lugar); enviar, mandar.}{re.me.ter}{0}
\verb{remeter}{ê}{}{}{}{}{Submeter, expor.}{re.me.ter}{\verboinum{12}}
\verb{remexer}{ch}{}{}{}{v.t.}{Mexer novamente.}{re.me.xer}{0}
\verb{remexer}{ch}{}{}{}{}{Mexer continuamente.}{re.me.xer}{0}
\verb{remexer}{ch}{}{}{}{}{Sacudir, agitar.}{re.me.xer}{0}
\verb{remexer}{ch}{}{}{}{}{Revolver, revirar.}{re.me.xer}{0}
\verb{remexer}{ch}{Bras.}{}{}{}{Movimentar o corpo ou os quadris.}{re.me.xer}{\verboinum{12}}
\verb{remição}{}{}{"-ões}{}{s.f.}{Ato ou efeito de remir; perdão, libertação.}{re.mi.ção}{0}
\verb{remido}{}{}{}{}{adj.}{Que se remiu; perdoado, libertado.}{re.mi.do}{0}
\verb{reminiscência}{}{}{}{}{s.f.}{Imagem do passado; lembrança.}{re.mi.nis.cên.cia}{0}
\verb{reminiscência}{}{}{}{}{}{Resto de algo extinto.}{re.mi.nis.cên.cia}{0}
\verb{remir}{}{}{}{}{v.t.}{Conseguir novamente.}{re.mir}{0}
\verb{remir}{}{}{}{}{}{Livrar (de prisão ou culpa); libertar, perdoar.}{re.mir}{0}
\verb{remir}{}{}{}{}{}{Compensar, indenizar, reparar.}{re.mir}{0}
\verb{remir}{}{}{}{}{v.pron.}{Recuperar"-se, reabilitar"-se.}{re.mir}{\verboinum{35}\verboirregular{\emph{def.} remimos, remis}}
\verb{remirar"-se}{}{}{}{}{v.pron.}{Observar"-se com atenção.}{re.mi.rar"-se}{\verboinum{1}}
\verb{remissão}{}{}{"-ões}{}{s.f.}{Ato ou efeito de remeter.}{re.mis.são}{0}
\verb{remissão}{}{}{"-ões}{}{s.f.}{Ato ou efeito de remir.}{re.mis.são}{0}
\verb{remissível}{}{}{"-eis}{}{adj.2g.}{Que pode ser remetido.}{re.mis.sí.vel}{0}
\verb{remissível}{}{}{"-eis}{}{adj.2g.}{Que pode ser remido.}{re.mis.sí.vel}{0}
\verb{remissivo}{}{}{}{}{adj.}{Que remete para ou se refere a outro lugar ou pessoa.}{re.mis.si.vo}{0}
\verb{remisso}{}{}{}{}{adj.}{Negligente, descuidado, indolente.}{re.mis.so}{0}
\verb{remitir}{}{}{}{}{v.t.}{Perdoar, indultar.}{re.mi.tir}{\verboinum{18}}
\verb{remo}{}{}{}{}{s.m.}{Haste longa de madeira com extremidade larga, plana e achatada, usada para impulsionar embarcações.}{re.mo}{0}
\verb{remo}{}{Esport.}{}{}{}{Modalidade esportiva de barcos movidos com remos.}{re.mo}{0}
\verb{remoçante}{}{}{}{}{adj.2g.}{Que torna moço, rejuvenece; remoçador.}{re.mo.çan.te}{0}
\verb{remoção}{}{}{"-ões}{}{s.f.}{Ato ou efeito de remover.}{re.mo.ção}{0}
\verb{remoçar}{}{}{}{}{v.t.}{Tornar moço; rejuvenecer.}{re.mo.çar}{\verboinum{3}}
\verb{remodelar}{}{}{}{}{v.t.}{Modelar novamente, fazendo modificações.}{re.mo.de.lar}{\verboinum{1}}
\verb{remoer}{ê}{}{}{}{v.t.}{Tornar a moer.}{re.mo.er}{0}
\verb{remoer}{ê}{}{}{}{}{Remastigar, ruminar.}{re.mo.er}{0}
\verb{remoer}{ê}{Fig.}{}{}{}{Pensar contínua e insistentemente em; ruminar.}{re.mo.er}{0}
\verb{remoer}{ê}{}{}{}{}{Molestar, importunar.}{re.mo.er}{0}
\verb{remoer}{ê}{}{}{}{v.pron.}{Afligir"-se, angustiar"-se, encolerizar"-se.}{re.mo.er}{\verboinum{17}}
\verb{remoinhar}{}{}{}{}{v.i.}{Redemoinhar.}{re.mo.i.nhar}{\verboinum{1}}
\verb{remoinho}{}{}{}{}{s.m.}{Ato ou efeito de remoinhar.}{re.mo.i.nho}{0}
\verb{remoinho}{}{}{}{}{}{Movimento em círculo dos ventos ou das águas; redemoinho.}{re.mo.i.nho}{0}
\verb{remoinho}{}{}{}{}{}{Porções espiralados de cabelo próximas à raiz.}{re.mo.i.nho}{0}
\verb{remonta}{}{}{}{}{s.f.}{Cavalos novos para suprir tropas de cavalaria.}{re.mon.ta}{0}
\verb{remonta}{}{Pop.}{}{}{}{Reforma, reparo, conserto.}{re.mon.ta}{0}
\verb{remontar}{}{}{}{}{v.t.}{Levantar, erguer bastante.}{re.mon.tar}{0}
\verb{remontar}{}{}{}{}{}{Remendar, consertar.}{re.mon.tar}{0}
\verb{remontar}{}{}{}{}{}{Substituir, geralmente, animais da tropa.}{re.mon.tar}{0}
\verb{remontar}{}{}{}{}{}{Recuar no tempo; estar muito atrás, no passado.}{re.mon.tar}{\verboinum{1}}
\verb{remoque}{ó}{}{}{}{s.m.}{Frase maliciosa ou picante.}{re.mo.que}{0}
\verb{remorder}{ê}{}{}{}{v.t.}{Morder novamente.}{re.mor.der}{0}
\verb{remorder}{ê}{}{}{}{}{Morder continuamente.}{re.mor.der}{0}
\verb{remorder}{ê}{}{}{}{}{Tratar continuamente um assunto; insistir, repisar.}{re.mor.der}{0}
\verb{remorder}{ê}{}{}{}{v.pron.}{Enraivecer"-se, irar"-se.}{re.mor.der}{\verboinum{12}}
\verb{remorso}{ó}{}{}{}{s.m.}{Estado atormentado de consciência causado por sentimento de culpa ou impressão de ter feito um mal.}{re.mor.so}{0}
\verb{remoto}{ó}{}{}{}{adj.}{Distante, no tempo ou no espaço; antigo, longínquo.}{re.mo.to}{0}
\verb{removedor}{ô}{Bras.}{}{}{s.m.}{Produto próprio para remover manchas ou tinta de diversos tipos de superfície.}{re.mo.ve.dor}{0}
\verb{remover}{ê}{}{}{}{v.t.}{Mover novamente.}{re.mo.ver}{0}
\verb{remover}{ê}{}{}{}{}{Mudar de lugar; transferir.}{re.mo.ver}{0}
\verb{remover}{ê}{}{}{}{}{Afastar de si.}{re.mo.ver}{0}
\verb{remover}{ê}{}{}{}{}{Tirar, arrancar.}{re.mo.ver}{\verboinum{12}}
\verb{removível}{}{}{"-eis}{}{adj.2g.}{Que pode ser removido.}{re.mo.ví.vel}{0}
\verb{remuneração}{}{}{"-ões}{}{s.f.}{Ato ou efeito de remunerar.}{re.mu.ne.ra.ção}{0}
\verb{remuneração}{}{}{"-ões}{}{}{Recompensa, salário, gratificação, prêmio.}{re.mu.ne.ra.ção}{0}
\verb{remunerador}{ô}{}{}{}{adj.}{Que remunera, recompensa.}{re.mu.ne.ra.dor}{0}
\verb{remunerar}{}{}{}{}{v.t.}{Dar recompensa, prêmio, gratificação.}{re.mu.ne.rar}{0}
\verb{remunerar}{}{}{}{}{}{Pagar salário, honorários, rendas.}{re.mu.ne.rar}{\verboinum{1}}
\verb{remunerativo}{}{}{}{}{adj.}{Relativo a remuneração.}{re.mu.ne.ra.ti.vo}{0}
\verb{remuneratório}{}{}{}{}{adj.}{Relativo a remuneração.}{re.mu.ne.ra.tó.rio}{0}
\verb{rena}{}{Zool.}{}{}{s.f.}{Mamífero dotado de galhadas e usado como animal tração na neve; rangífer.}{re.na}{0}
\verb{renal}{}{}{"-ais}{}{adj.2g.}{Relativo aos rins.}{re.nal}{0}
\verb{renascença}{}{}{}{}{s.f.}{Ato ou efeito de renascer.}{re.nas.cen.ça}{0}
\verb{renascença}{}{}{}{}{}{Nova vida.}{re.nas.cen.ça}{0}
\verb{renascença}{}{Art.}{}{}{}{Renascimento. (Usa"-se com inicial maiúscula nesta acepção.)}{re.nas.cen.ça}{0}
\verb{renascentista}{}{Art.}{}{}{adj.2g.}{Relativo a Renascença ou Renascimento.}{re.nas.cen.tis.ta}{0}
\verb{renascer}{ê}{}{}{}{v.t.}{Nascer novamente.}{re.nas.cer}{0}
\verb{renascer}{ê}{}{}{}{}{Adquirir nova vitalidade; revigorar"-se, renovar"-se, recuperar"-se, reabilitar"-se.}{re.nas.cer}{0}
\verb{renascer}{ê}{}{}{}{}{Rejuvenescer, remoçar.}{re.nas.cer}{0}
\verb{renascer}{ê}{}{}{}{}{Ressurgir, reaparecer.}{re.nas.cer}{\verboinum{15}}
\verb{renascimento}{}{}{}{}{s.m.}{Ato ou efeito de renascer.}{re.nas.ci.men.to}{0}
\verb{renascimento}{}{}{}{}{}{Nova vida.}{re.nas.ci.men.to}{0}
\verb{renascimento}{}{Art.}{}{}{}{Movimento artístico e intelectual surgido no século \textsc{xv} que trouxe de volta valores e modelos da Antiguidade Clássica greco"-romana. (Usa"-se com inicial maiúscula nesta acepção.)}{re.nas.ci.men.to}{0}
\verb{renda}{}{}{}{}{s.f.}{Lucro, rendimento, receita.}{ren.da}{0}
\verb{renda}{}{}{}{}{s.f.}{Tecido fino com malhas abertas que formam desenhos, usado para dar acabamento a toalhas e peças de vestuário.}{ren.da}{0}
\verb{rendado}{}{}{}{}{adj.}{Ornado com renda ou que tem aspecto de renda.}{ren.da.do}{0}
\verb{rendar}{}{}{}{}{v.t.}{Ornar com renda.}{ren.dar}{\verboinum{1}}
\verb{rendar}{}{}{}{}{v.t.}{Arrendar.}{ren.dar}{0}
\verb{rendar}{}{}{}{}{}{Pagar rendimentos; render.}{ren.dar}{\verboinum{1}}
\verb{rendaria}{}{}{}{}{}{Grande quantidade de rendas.}{ren.da.ri.a}{0}
\verb{rendaria}{}{}{}{}{s.f.}{Técnica de tecer rendas.}{ren.da.ri.a}{0}
\verb{rendaria}{}{}{}{}{}{Estabelecimento que fabrica ou comercializa rendas.}{ren.da.ri.a}{0}
\verb{rendeira}{ê}{}{}{}{s.f.}{Mulher que faz rendas.}{ren.dei.ra}{0}
\verb{rendeira}{ê}{Zool.}{}{}{}{Ave de asas e cauda negras e pés alaranjados, encontrada em grande parte da América do Sul; uirapuru.}{ren.dei.ra}{0}
\verb{rendeiro}{ê}{}{}{}{s.m.}{Indivíduo que toma propriedade ou coisa em arrendamento; arrendatário.}{ren.dei.ro}{0}
\verb{rendeiro}{ê}{}{}{}{s.m.}{Indivíduo que fabrica ou comercializa rendas.}{ren.dei.ro}{0}
\verb{render}{ê}{}{}{}{v.t.}{Fazer capitular; sujeitar.}{ren.der}{0}
\verb{render}{ê}{}{}{}{}{Ocupar o lugar de; substituir.}{ren.der}{0}
\verb{render}{ê}{}{}{}{}{Prestar homenagem, culto.}{ren.der}{0}
\verb{render}{ê}{}{}{}{}{Gerar lucro.}{ren.der}{0}
\verb{render}{ê}{}{}{}{}{Produzir.}{ren.der}{0}
\verb{render}{ê}{}{}{}{v.pron.}{Capitular; entregar"-se.}{ren.der}{\verboinum{12}}
\verb{rendição}{}{}{"-ões}{}{s.f.}{Ato ou efeito de render.}{ren.di.ção}{0}
\verb{rendido}{}{}{}{}{adj.}{Que se rendeu.}{ren.di.do}{0}
\verb{rendido}{}{}{}{}{}{Sem vontade própria, dominado, vencido.}{ren.di.do}{0}
\verb{rendido}{}{}{}{}{}{Que sofre de hérnia intestinal ou quebradura; quebrado.}{ren.di.do}{0}
\verb{rendilha}{}{}{}{}{s.f.}{Renda pequena e delicada, cheia de babados.}{ren.di.lha}{0}
\verb{rendilhado}{}{}{}{}{adj.}{Enfeitado com rendilhas.}{ren.di.lha.do}{0}
\verb{rendilhado}{}{Fig.}{}{}{}{Delicadamente trabalhado e adornado.}{ren.di.lha.do}{0}
\verb{rendilhar}{}{}{}{}{v.t.}{Enfeitar com rendilhas.}{ren.di.lhar}{\verboinum{1}}
\verb{rendimento}{}{}{}{}{s.m.}{Ato ou efeito de render.}{ren.di.men.to}{0}
\verb{rendimento}{}{}{}{}{}{Produto, lucro, renda.}{ren.di.men.to}{0}
\verb{rendoso}{ô}{}{"-osos ⟨ó⟩}{"-osa ⟨ó⟩}{adj.}{Que rende lucros; vantajoso, lucrativo.}{ren.do.so}{0}
\verb{renegado}{}{}{}{}{adj.}{Que renega suas convicções políticas ou religiosas.}{re.ne.ga.do}{0}
\verb{renegado}{}{}{}{}{s.m.}{Indivíduo amaldiçoado, execrado, cruel.}{re.ne.ga.do}{0}
\verb{renegar}{}{}{}{}{v.t.}{Renunciar a sua crença religiosa ou suas convicções políticas; abjurar. }{re.ne.gar}{0}
\verb{renegar}{}{}{}{}{}{Repudiar, rejeitar, desprezar.}{re.ne.gar}{\verboinum{5}}
\verb{renegociar}{}{}{}{}{v.t.}{Tornar a negociar; revisar um negócio.}{re.ne.go.ci.ar}{\verboinum{1}}
\verb{renhido}{}{}{}{}{adj.}{Disputado com ardor; encarniçado, porfiado.}{re.nhi.do}{0}
\verb{renhir}{}{}{}{}{v.t.}{Entrar em disputa; contender, combater, lutar.}{re.nhir}{\verboinum{35}\verboirregular{\emph{def.} renhimos, renhis}}
\verb{rênio}{}{Quím.}{}{}{s.m.}{Elemento químico metálico, brilhante, usado como aditivo em certas ligas de tungstênio e molibdênio. \elemento{75}{186.207}{Re}.}{rê.nio}{0}
\verb{renitente}{}{}{}{}{adj.2g.}{Que renite; obstinado, teimoso.}{re.ni.ten.te}{0}
\verb{renitir}{}{}{}{}{v.t.}{Persistir em uma ação; obstinar, teimar.}{re.ni.tir}{\verboinum{18}}
\verb{renomado}{}{}{}{}{adj.}{Que tem renome; famoso, célebre.}{re.no.ma.do}{0}
\verb{renome}{}{}{}{}{s.m.}{Celebridade, fama, reputação.}{re.no.me}{0}
\verb{renovação}{}{}{"-ões}{}{s.f.}{Ato ou efeito de renovar, recomeçar.}{re.no.va.ção}{0}
\verb{renovador}{ô}{}{}{}{adj.}{Que renova; restaurador, reformador.}{re.no.va.dor}{0}
\verb{renovar}{}{}{}{}{v.t.}{Trocar coisas velhas ou inutilizadas por coisas novas; substituir. (\textit{Ele renovou sua carteira de habilitação.})}{re.no.var}{0}
\verb{renovar}{}{}{}{}{}{Tornar a fazer; reestruturar, reordenar.}{re.no.var}{\verboinum{1}}
\verb{renovo}{ô}{}{"-s ⟨ó⟩}{}{s.m.}{Broto, rebento.}{re.no.vo}{0}
\verb{renque}{}{}{}{}{s.m.}{Disposição de objetos ou pessoas em uma linha; fileira, alinhamento.}{ren.que}{0}
\verb{rentabilidade}{}{}{}{}{s.f.}{Qualidade ou estado do que produz renda; vantagem.}{ren.ta.bi.li.da.de}{0}
\verb{rentável}{}{}{"-eis}{}{adj.2g.}{Que produz lucro; rendoso, vantajoso.}{ren.tá.vel}{0}
\verb{rente}{}{}{}{}{adj.2g.}{Muito próximo ou junto; contíguo.}{ren.te}{0}
\verb{renúncia}{}{}{}{}{s.f.}{Ato ou efeito de renunciar; recusa, abdicação, desistência.}{re.nún.cia}{0}
\verb{renunciador}{ô}{}{}{}{adj.}{Que renuncia, abdica; renunciante.}{re.nun.ci.a.dor}{0}
\verb{renunciante}{}{}{}{}{adj.2g.}{Que renuncia, abdica; renunciador.}{re.nun.ci.an.te}{0}
\verb{renunciar}{}{}{}{}{v.t.}{Não querer; rejeitar, abandonar, desistir, recusar.}{re.nun.ci.ar}{\verboinum{1}}
\verb{reordenar}{}{}{}{}{v.t.}{Colocar novamente em ordem.}{re.or.de.nar}{\verboinum{1}}
\verb{reorganização}{}{}{"-ões}{}{s.f.}{Ato ou efeito de reorganizar; reestruturação.}{re.or.ga.ni.za.ção}{0}
\verb{reorganizador}{ô}{}{}{}{adj.}{Que reorganiza; reestruturador.}{re.or.ga.ni.za.dor}{0}
\verb{reorganizar}{}{}{}{}{v.t.}{Organizar novamente; reestruturar.}{re.or.ga.ni.zar}{\verboinum{1}}
\verb{reóstato}{}{}{}{}{}{Var. de \textit{reostato}.}{re.ós.ta.to}{0}
\verb{reostato}{}{Fís.}{}{}{s.m.}{Aparelho utilizado como resistor variável, que permite limitar a corrente elétrica em um circuito.}{re.os.ta.to}{0}
\verb{reparação}{}{}{"-ões}{}{s.f.}{Ato ou efeito de reparar; reparo.}{re.pa.ra.ção}{0}
\verb{reparação}{}{}{"-ões}{}{}{Retratação, desagravo.}{re.pa.ra.ção}{0}
\verb{reparadeira}{ê}{}{}{}{s.f.}{Mulher curiosa que repara em tudo.}{re.pa.ra.dei.ra}{0}
\verb{reparador}{ô}{}{}{}{adj.}{Que repara, conserta; restaurador.}{re.pa.ra.dor}{0}
\verb{reparador}{ô}{}{}{}{}{Que repara, nota; observador.}{re.pa.ra.dor}{0}
\verb{reparar}{}{}{}{}{v.t.}{Colocar em bom estado de funcionamento; consertar, restaurar.}{re.pa.rar}{0}
\verb{reparar}{}{}{}{}{}{Retratar, ressarcir, indenizar.}{re.pa.rar}{0}
\verb{reparar}{}{}{}{}{}{Olhar atentamente; observar, notar.}{re.pa.rar}{\verboinum{1}}
\verb{reparo}{}{}{}{}{s.m.}{Ato ou efeito de reparar; reparação.}{re.pa.ro}{0}
\verb{reparo}{}{}{}{}{}{Conserto, reforma.}{re.pa.ro}{0}
\verb{reparo}{}{}{}{}{}{Olhar atento, observação, análise.}{re.pa.ro}{0}
\verb{repartição}{}{}{"-ões}{}{s.f.}{Ato ou efeito de repartir; divisão.}{re.par.ti.ção}{0}
\verb{repartição}{}{}{"-ões}{}{}{Seção, divisão ou departamento que se destina a atender serviços públicos.}{re.par.ti.ção}{0}
\verb{repartimento}{}{}{}{}{s.m.}{Ato ou efeito de repartir; repartição.}{re.par.ti.men.to}{0}
\verb{repartimento}{}{}{}{}{}{Compartimento de dormir; dormitório. }{re.par.ti.men.to}{0}
\verb{repartir}{}{}{}{}{v.t.}{Separar em partes; dividir.}{re.par.tir}{0}
\verb{repartir}{}{}{}{}{}{Compartilhar, distribuir.}{re.par.tir}{\verboinum{18}}
\verb{repassar}{}{}{}{}{v.t.}{Passar novamente.}{re.pas.sar}{0}
\verb{repassar}{}{}{}{}{}{Passar para frente o que se recebeu; transferir. (\textit{O governo do estado repassou a verba para as prefeituras.})}{re.pas.sar}{0}
\verb{repassar}{}{}{}{}{}{Examinar ou estudar novamente. (\textit{O aluno resolveu repassar a matéria antes da prova.})}{re.pas.sar}{\verboinum{1}}
\verb{repasse}{}{}{}{}{s.m.}{Ato ou efeito de repassar.}{re.pas.se}{0}
\verb{repasto}{}{}{}{}{s.m.}{Refeição farta e festiva; banquete.}{re.pas.to}{0}
\verb{repatriar}{}{}{}{}{v.t.}{Fazer retornar à pátria.}{re.pa.tri.ar}{\verboinum{1}}
\verb{repelão}{}{}{"-ões}{}{s.m.}{Encontro violento; choque, empurrão.}{re.pe.lão}{0}
\verb{repelente}{}{}{}{}{adj.2g.}{Que repele, afasta; nojento, asqueroso.}{re.pe.len.te}{0}
\verb{repelente}{}{Quím.}{}{}{s.m.}{Substância que espanta insetos.}{re.pe.len.te}{0}
\verb{repelir}{}{}{}{}{v.t.}{Impelir para longe; rechaçar, afastar.}{re.pe.lir}{0}
\verb{repelir}{}{}{}{}{}{Rejeitar, recusar.}{re.pe.lir}{0}
\verb{repelir}{}{}{}{}{}{Ter nojo, repugnância.}{re.pe.lir}{\verboinum{29}}
\verb{repenicado}{}{}{}{}{adj.}{Que repenicou, que emitiu sons agudos.}{re.pe.ni.ca.do}{0}
\verb{repenicar}{}{}{}{}{v.t.}{Produzir sons agudos; repicar.}{re.pe.ni.car}{\verboinum{2}}
\verb{repensar}{}{}{}{}{v.t.}{Pensar novamente; reconsiderar.}{re.pen.sar}{\verboinum{1}}
\verb{repente}{}{}{}{}{s.m.}{Ato ou dito repentino; ímpeto, impulso.}{re.pen.te}{0}
\verb{repente}{}{}{}{}{}{Canto popular com versos improvisados.}{re.pen.te}{0}
\verb{repentino}{}{}{}{}{adj.}{Que ocorre de imprevisto; súbito, inesperado.}{re.pen.ti.no}{0}
\verb{repentista}{}{}{}{}{s.2g.}{Cantor que improvisa versos.}{re.pen.tis.ta}{0}
\verb{repentista}{}{}{}{}{}{Indivíduo que age por impulso.}{re.pen.tis.ta}{0}
\verb{repercussão}{}{}{"-ões}{}{s.f.}{Ato ou efeito de repercutir.}{re.per.cus.são}{0}
\verb{repercutir}{}{}{}{}{v.t.}{Reproduzir sons e luz; refletir.}{re.per.cu.tir}{0}
\verb{repercutir}{}{}{}{}{v.i.}{Causar influência e impressão generalizada. (\textit{O assunto repercutiu nos bastidores da política.})}{re.per.cu.tir}{\verboinum{18}}
\verb{repertório}{}{}{}{}{s.m.}{Conjunto de peças teatrais ou musicais, noticiários, piadas etc; coleção, compilação.}{re.per.tó.rio}{0}
\verb{repesar}{}{}{}{}{v.t.}{Tornar a pesar.}{re.pe.sar}{\verboinum{1}}
\verb{repetência}{}{}{}{}{s.f.}{Ato ou efeito de repetir; repetição.}{re.pe.tên.cia}{0}
\verb{repetência}{}{}{}{}{}{Condição de repetente.}{re.pe.tên.cia}{0}
\verb{repetente}{}{}{}{}{adj.2g.}{Diz"-se do aluno que é reprovado e precisa repetir o ano escolar.}{re.pe.ten.te}{0}
\verb{repetição}{}{}{"-ões}{}{s.f.}{Ato ou efeito de repetir; reiteração.}{re.pe.ti.ção}{0}
\verb{repetidor}{ô}{}{}{}{adj.}{Que repete, reitera.}{re.pe.ti.dor}{0}
\verb{repetidor}{ô}{}{}{}{}{Diz"-se do professor que repete as lições aos alunos. }{re.pe.ti.dor}{0}
\verb{repetir}{}{}{}{}{v.t.}{Tornar a fazer ou dizer alguma coisa; repisar.}{re.pe.tir}{0}
\verb{repetir}{}{}{}{}{}{Cursar pela segunda vez.}{re.pe.tir}{0}
\verb{repetir}{}{}{}{}{v.pron.}{Acontecer, suceder de novo.}{re.pe.tir}{\verboinum{29}}
\verb{repetitivo}{}{}{}{}{adj.}{Que repete muitas vezes; enfadonho, chato.}{re.pe.ti.ti.vo}{0}
\verb{repicar}{}{}{}{}{v.t.}{Tornar a picar.}{re.pi.car}{0}
\verb{repicar}{}{}{}{}{}{Tocar o sino de modo festivo ou tocar repetidas vezes a campainha; repenicar.}{re.pi.car}{\verboinum{2}}
\verb{repimpar}{}{}{}{}{v.t.}{Encher a barriga; abarrotar, fartar.}{re.pim.par}{\verboinum{1}}
\verb{repintar}{}{}{}{}{v.t.}{Tornar a pintar.}{re.pin.tar}{\verboinum{1}}
\verb{repique}{}{}{}{}{s.m.}{Ato ou efeito de repicar.}{re.pi.que}{0}
\verb{repique}{}{}{}{}{}{Toque festivo de sinos; badalada.}{re.pi.que}{0}
\verb{repiquete}{ê}{}{}{}{s.m.}{Repique de sinos com intervalo pequeno entre as badaladas.}{re.pi.que.te}{0}
\verb{repiquete}{ê}{}{}{}{}{Terreno íngreme; ladeira.}{re.pi.que.te}{0}
\verb{repiquete}{ê}{}{}{}{}{Recaída de doença.}{re.pi.que.te}{0}
\verb{repisar}{}{}{}{}{v.t.}{Tornar a pisar.}{re.pi.sar}{0}
\verb{repisar}{}{}{}{}{}{Repetir, reiterar.}{re.pi.sar}{0}
\verb{repisar}{}{}{}{}{}{Falar com insistência.}{re.pi.sar}{\verboinum{1}}
\verb{replantar}{}{}{}{}{v.t.}{Tornar a plantar.}{re.plan.tar}{\verboinum{1}}
\verb{replay}{}{}{}{}{s.m.}{Repetição de imagens ou cenas gravadas em televisão; reprise.}{\textit{replay}}{0}
\verb{repleno}{}{}{}{}{adj.}{Repleto.}{re.ple.no}{0}
\verb{repleto}{é}{}{}{}{adj.}{Completamente cheio; abarrotado.}{re.ple.to}{0}
\verb{réplica}{}{}{}{}{s.f.}{Ato ou efeito de replicar; contestação.}{ré.pli.ca}{0}
\verb{réplica}{}{}{}{}{}{Reprodução de um original; cópia, imitação.}{ré.pli.ca}{0}
\verb{replicar}{}{}{}{}{v.t.}{Responder com argumentos contrários; contestar, refutar, redarguir.}{re.pli.car}{0}
\verb{replicar}{}{}{}{}{}{Fazer réplica; imitar, copiar, reproduzir.}{re.pli.car}{\verboinum{2}}
\verb{repolho}{ô}{Bot.}{}{}{s.m.}{Planta hortense, variedade de couve com folhas curvadas umas sobre as outras, formando uma espécie de globo.}{re.po.lho}{0}
\verb{repolhudo}{}{}{}{}{adj.}{Cuja forma ou aspecto se assemelham ao repolho.}{re.po.lhu.do}{0}
\verb{repolhudo}{}{Fig.}{}{}{}{Gordo, rechonchudo.}{re.po.lhu.do}{0}
\verb{repontar}{}{}{}{}{v.i.}{Começar a aparecer novamente.}{re.pon.tar}{0}
\verb{repontar}{}{}{}{}{}{Amanhecer, raiar.}{re.pon.tar}{0}
\verb{repontar}{}{}{}{}{}{Responder com aspereza; replicar.}{re.pon.tar}{\verboinum{1}}
\verb{repor}{}{}{}{}{v.t.}{Tornar a pôr; recolocar.}{re.por}{0}
\verb{repor}{}{}{}{}{}{Substituir, compensar.}{re.por}{0}
\verb{repor}{}{}{}{}{}{Devolver, restituir.}{re.por}{\verboinum{60}}
\verb{reportagem}{}{}{"-ens}{}{s.f.}{Ato ou efeito de reportar, de fornecer notícias em jornal, revista, televisão, rádio ou cinema.}{re.por.ta.gem}{0}
\verb{reportagem}{}{}{"-ens}{}{}{Noticiário sobre algum assunto.}{re.por.ta.gem}{0}
\verb{reportagem}{}{}{"-ens}{}{}{Equipe de repórteres.}{re.por.ta.gem}{0}
\verb{reportar}{}{}{}{}{v.t.}{Voltar para trás; volver, recuar.}{re.por.tar}{0}
\verb{reportar}{}{}{}{}{}{Referir, atribuir.}{re.por.tar}{0}
\verb{reportar}{}{}{}{}{v.pron.}{Referir"-se, dirigir"-se, aludir. (\textit{A secretária se reportava diretamente ao diretor.})}{re.por.tar}{\verboinum{1}}
\verb{repórter}{}{}{}{}{s.2g.}{Jornalista que coleta informações para fazer reportagens.}{re.pór.ter}{0}
\verb{reposição}{}{}{"-ões}{}{s.f.}{Ato ou efeito de repor.}{re.po.si.ção}{0}
\verb{repositório}{}{}{}{}{s.m.}{Local onde se guarda alguma coisa; depósito.}{re.po.si.tó.rio}{0}
\verb{repositório}{}{}{}{}{}{Coleção, coletânea, repertório.}{re.po.si.tó.rio}{0}
\verb{reposteiro}{ê}{}{}{}{s.m.}{Cortina ou peça de estofo que pende das portas internas de uma casa.}{re.pos.tei.ro}{0}
\verb{repousar}{}{}{}{}{v.t.}{Pôr em sossego; descansar.}{re.pou.sar}{0}
\verb{repousar}{}{}{}{}{}{Fitar, observar atentamente.}{re.pou.sar}{0}
\verb{repousar}{}{}{}{}{v.i.}{Estar em repouso; dormir.}{re.pou.sar}{\verboinum{1}}
\verb{repouso}{ô}{}{}{}{s.m.}{Ato ou efeito de repousar; descanso, quietude.}{re.pou.so}{0}
\verb{repovoar}{}{}{}{}{v.t.}{Povoar novamente.}{re.po.vo.ar}{\verboinum{7}}
\verb{repreender}{ê}{}{}{}{v.t.}{Advertir com energia; censurar, admoestar.}{re.pre.en.der}{\verboinum{12}}
\verb{repreensão}{}{}{"-ões}{}{s.f.}{Ato ou efeito de repreender; censura, reprimenda.}{re.pre.en.são}{0}
\verb{repreensível}{}{}{"-eis}{}{adj.2g.}{Que merece repreensão; reprovável, censurável.}{re.pre.en.sí.vel}{0}
\verb{repregar}{}{}{}{}{v.t.}{Tornar a pregar.}{re.pre.gar}{\verboinum{5}}
\verb{represa}{ê}{}{}{}{s.f.}{Ato ou efeito de represar, conter.}{re.pre.sa}{0}
\verb{represa}{ê}{}{}{}{}{Construção que impede a passagem das águas de um rio para fins industriais ou agrícolas; barragem, açude.}{re.pre.sa}{0}
\verb{represália}{}{}{}{}{s.f.}{Ato de se vingar de um ação praticada por alguém; desforra, retaliação.}{re.pre.sá.lia}{0}
\verb{represar}{}{}{}{}{v.t.}{Impedir a passagem das águas de um rio.}{re.pre.sar}{0}
\verb{represar}{}{}{}{}{}{Conter, reprimir, refrear.}{re.pre.sar}{\verboinum{1}}
\verb{representação}{}{}{"-ões}{}{s.f.}{Ato ou efeito de representar.}{re.pre.sen.ta.ção}{0}
\verb{representação}{}{}{"-ões}{}{}{Espetáculo teatral; encenação.}{re.pre.sen.ta.ção}{0}
\verb{representação}{}{}{"-ões}{}{}{Conjunto de representantes; delegação.}{re.pre.sen.ta.ção}{0}
\verb{representante}{}{}{}{}{adj.2g.}{Diz"-se do indivíduo que representa outro ou uma instituição ou um governo.}{re.pre.sen.tan.te}{0}
\verb{representar}{}{}{}{}{v.t.}{Ser a imagem ou um símbolo de algo; simbolizar.}{re.pre.sen.tar}{0}
\verb{representar}{}{}{}{}{}{Apresentar peça teatral; encenar.}{re.pre.sen.tar}{0}
\verb{representar}{}{}{}{}{}{Desempenhar o papel de uma personagem em teatro, cinema ou televisão.}{re.pre.sen.tar}{0}
\verb{representar}{}{}{}{}{}{Estar no lugar de alguém; substituir.}{re.pre.sen.tar}{0}
\verb{representar}{}{}{}{}{}{Ter importância para alguém; significar.}{re.pre.sen.tar}{\verboinum{1}}
\verb{representativo}{}{}{}{}{adj.}{Que representa.}{re.pre.sen.ta.ti.vo}{0}
\verb{representativo}{}{}{}{}{}{Significativo, importante, relevante.}{re.pre.sen.ta.ti.vo}{0}
\verb{repressão}{}{}{"-ões}{}{s.f.}{Ato ou efeito de reprimir; coerção, coibição.}{re.pres.são}{0}
\verb{repressivo}{}{}{}{}{adj.}{Que reprime ou é próprio para reprimir.}{re.pres.si.vo}{0}
\verb{repressor}{ô}{}{}{}{adj.}{Que reprime, coage.}{re.pres.sor}{0}
\verb{reprimenda}{}{}{}{}{s.f.}{Ato ou efeito de reprimir; censura, repreensão, admoestação.}{re.pri.men.da}{0}
\verb{reprimir}{}{}{}{}{v.t.}{Conter a ação ou o movimento de; refrear, coibir.}{re.pri.mir}{\verboinum{18}}
\verb{reprisar}{}{}{}{}{v.t.}{Voltar a apresentar peça teatral, filme etc.}{re.pri.sar}{\verboinum{1}}
\verb{reprise}{}{}{}{}{s.m.}{Reapresentação de espetáculo, filme etc.}{re.pri.se}{0}
\verb{réprobo}{}{}{}{}{adj.}{Que foi banido; amaldiçoado, condenado, renegado.}{ré.pro.bo}{0}
\verb{reprochar}{}{}{}{}{v.t.}{Lançar censura; repreender, exprobar.}{re.pro.char}{\verboinum{1}}
\verb{reproche}{ó}{}{}{}{s.m.}{Censura, repreensão, admoestação.}{re.pro.che}{0}
\verb{reprodução}{}{}{"-ões}{}{s.f.}{Ato ou efeito de reproduzir.}{re.pro.du.ção}{0}
\verb{reprodução}{}{}{"-ões}{}{}{Cópia, imitação.}{re.pro.du.ção}{0}
\verb{reprodutivo}{}{}{}{}{adj.}{Que produz ou que se reproduz.}{re.pro.du.ti.vo}{0}
\verb{reprodutor}{ô}{}{}{}{adj.}{Que reproduz.}{re.pro.du.tor}{0}
\verb{reprodutor}{ô}{}{}{}{}{Diz"-se do animal que se destina à reprodução. (\textit{Esse cavalo é um bom reprodutor.})}{re.pro.du.tor}{0}
\verb{reproduzir}{}{}{}{}{v.t.}{Tornar a produzir.}{re.pro.du.zir}{0}
\verb{reproduzir}{}{}{}{}{}{Fazer cópia fiel de um original.}{re.pro.du.zir}{0}
\verb{reproduzir}{}{}{}{}{v.pron.}{Multiplicar"-se, perpetuar"-se pela geração.}{re.pro.du.zir}{\verboinum{21}}
\verb{reprovação}{}{}{"-ões}{}{s.f.}{Ato ou efeito de reprovar; rejeição, censura.}{re.pro.va.ção}{0}
\verb{reprovador}{ô}{}{}{}{adj.}{Que reprova; reprovativo.}{re.pro.va.dor}{0}
\verb{reprovar}{}{}{}{}{v.t.}{Não aprovar; rejeitar, desaprovar.}{re.pro.var}{0}
\verb{reprovar}{}{}{}{}{}{Em exame, julgar o candidato ou aluno inabilitado.}{re.pro.var}{0}
\verb{reprovar}{}{}{}{}{}{Votar contra.}{re.pro.var}{\verboinum{1}}
\verb{reprovativo}{}{}{}{}{adj.}{Que exprime reprovação, que recrimina.}{re.pro.va.ti.vo}{0}
\verb{reprovável}{}{}{"-eis}{}{adj.2g.}{Que merece reprovação; censurável.}{re.pro.vá.vel}{0}
\verb{reptar}{}{}{}{}{v.t.}{Manter oposição a.}{rep.tar}{0}
\verb{reptar}{}{}{}{}{}{Lançar desafio; provocar.}{rep.tar}{\verboinum{1}}
\verb{reptil}{}{}{}{}{}{Var. de \textit{réptil}.}{rep.til}{0}
\verb{réptil}{}{Zool.}{"-eis}{}{s.m.}{Animal vertebrado que se arrasta pelo chão, caracterizado por patas curtas ou nulas, põe e ovos e tem o corpo coberto de escamas ou placas. }{rép.til}{0}
\verb{repto}{é}{}{}{}{s.m.}{Ato ou efeito de reptar, de opor"-se.}{rep.to}{0}
\verb{repto}{é}{}{}{}{}{Desafio, provocação.}{rep.to}{0}
\verb{república}{}{}{}{}{s.f.}{Forma de governo em que o poder é exercido por pessoas eleitas pelo povo.}{re.pú.bli.ca}{0}
\verb{república}{}{}{}{}{}{País que é governado dessa forma.}{re.pú.bli.ca}{0}
\verb{república}{}{}{}{}{}{Casa de estudantes.}{re.pú.bli.ca}{0}
\verb{republicano}{}{}{}{}{adj.}{Que pertence ou diz respeito à república.}{re.pu.bli.ca.no}{0}
\verb{republicano}{}{}{}{}{}{Diz"-se de partidário da república.}{re.pu.bli.ca.no}{0}
\verb{republicar}{}{}{}{}{v.t.}{Tornar a publicar; reeditar.}{re.pu.bli.car}{\verboinum{2}}
\verb{repudiar}{}{}{}{}{v.t.}{Não aceitar; rejeitar, desprezar, recusar.}{re.pu.di.ar}{\verboinum{6}}
\verb{repúdio}{}{}{}{}{s.m.}{Ato ou efeito de repudiar; rejeição, reprovação.}{re.pú.dio}{0}
\verb{repugnância}{}{}{}{}{s.f.}{Sentimento de aversão, de repulsa; asco, nojo.}{re.pug.nân.cia}{0}
\verb{repugnância}{}{}{}{}{}{Senso moral; escrúpulo.}{re.pug.nân.cia}{0}
\verb{repugnante}{}{}{}{}{adj.2g.}{Que provoca mal"-estar, repulsa, asco.}{re.pug.nan.te}{0}
\verb{repugnante}{}{Fig.}{}{}{}{Que provoca indignação moral.}{re.pug.nan.te}{0}
\verb{repugnar}{}{}{}{}{v.t.}{Não aceitar; recusar.}{re.pug.nar}{0}
\verb{repugnar}{}{}{}{}{}{Causar repugnância, asco; inspirar antipatia ou aversão.}{re.pug.nar}{\verboinum{1}}
\verb{repulsa}{}{}{}{}{s.f.}{Ato ou efeito de repulsar ou repelir; rejeitar.}{re.pul.sa}{0}
\verb{repulsa}{}{}{}{}{}{Sentimento de repugnância, de aversão.}{re.pul.sa}{0}
\verb{repulsão}{}{}{"-ões}{}{s.f.}{Repulsa.}{re.pul.são}{0}
\verb{repulsivo}{}{}{}{}{adj.}{Que causa repulsa; repugnante, asqueroso, repelente.}{re.pul.si.vo}{0}
\verb{reputação}{}{}{"-ões}{}{s.f.}{Opinião que se tem sobre o modo de agir de uma pessoa; imagem, conceito.}{re.pu.ta.ção}{0}
\verb{reputar}{}{}{}{}{v.t.}{Ter em conta; julgar, considerar.}{re.pu.tar}{\verboinum{1}}
\verb{repuxar}{ch}{}{}{}{v.t.}{Puxar de novo com violência.}{re.pu.xar}{0}
\verb{repuxar}{ch}{}{}{}{}{Puxar para trás.}{re.pu.xar}{0}
\verb{repuxar}{ch}{}{}{}{}{Formar jato.}{re.pu.xar}{\verboinum{1}}
\verb{repuxo}{ch}{}{}{}{s.m.}{Ato ou efeito de repuxar.}{re.pu.xo}{0}
\verb{repuxo}{ch}{}{}{}{}{Tubo por onde a água se eleva saindo em jato.}{re.pu.xo}{0}
\verb{requebrado}{}{}{}{}{adj.}{Que tem requebros nos gestos ou na voz; amoroso, lânguido.}{re.que.bra.do}{0}
\verb{requebrado}{}{}{}{}{s.m.}{Requebro de corpo; bamboleio, rebolado.}{re.que.bra.do}{0}
\verb{requebrar}{}{}{}{}{v.t.}{Mexer o corpo; gingar, rebolar.}{re.que.brar}{\verboinum{1}}
\verb{requebro}{é}{}{}{}{s.m.}{Ato ou efeito de requebrar; balanço, meneio.}{re.que.bro}{0}
\verb{requeijão}{}{}{"-ões}{}{s.m.}{Queijo fresco de consistência pastosa, cuja massa é feita da mistura de coalhada com leite fresco, que é levada ao fogo para cozinhar.}{re.quei.jão}{0}
\verb{requeimar}{}{}{}{}{v.t.}{Queimar muito; tostar.}{re.quei.mar}{\verboinum{1}}
\verb{requentar}{}{}{}{}{v.t.}{Esquentar novamente.}{re.quen.tar}{\verboinum{1}}
\verb{requerente}{}{}{}{}{adj.2g.}{Diz"-se daquele que solicita, que reivindica.}{re.que.ren.te}{0}
\verb{requerer}{ê}{}{}{}{v.t.}{Solicitar por escrito.}{re.que.rer}{0}
\verb{requerer}{ê}{}{}{}{}{Ter falta de alguma coisa que deve ser feita; demandar, exigir, necessitar.}{re.que.rer}{\verboinum{50}}
\verb{requerimento}{}{}{}{}{s.m.}{Ato ou efeito de requerer.}{re.que.ri.men.to}{0}
\verb{requerimento}{}{}{}{}{}{Documento que contém uma reivindicação, um pedido.}{re.que.ri.men.to}{0}
\verb{requestar}{}{}{}{}{v.t.}{Pedir insistentemente.}{re.ques.tar}{0}
\verb{requestar}{}{}{}{}{}{Cortejar, galantear.}{re.ques.tar}{\verboinum{1}}
\verb{requestar}{}{}{}{}{v.t.}{Fazer um pedido, uma solicitação.}{re.ques.tar}{0}
\verb{requestar}{}{}{}{}{}{Pedir com insistência; suplicar.}{re.ques.tar}{0}
\verb{requestar}{}{}{}{}{}{Pretender, namorar.}{re.ques.tar}{\verboinum{1}}
\verb{réquiem}{}{}{"-ens}{}{s.m.}{Prece feita pela Igreja para os mortos.}{ré.qui.em}{0}
\verb{requintado}{}{}{}{}{adj.}{Que tem ou denota apuro e elegância; aprimorado, apurado, refinado.}{re.quin.ta.do}{0}
\verb{requintar}{}{}{}{}{v.t.}{Levar ao mais alto grau de qualidade; elevar, sublimar, aprimorar.}{re.quin.tar}{\verboinum{1}}
\verb{requinte}{}{}{}{}{s.m.}{Ato ou efeito de requintar.}{re.quin.te}{0}
\verb{requinte}{}{}{}{}{}{Excesso de aperfeiçoamento, apuro extremo.}{re.quin.te}{0}
\verb{requisição}{}{}{"-ões}{}{s.f.}{Ato ou efeito de requisitar; pedido, exigência legal.}{re.qui.si.ção}{0}
\verb{requisitar}{}{}{}{}{v.t.}{Pedir ou exigir legalmente; requerer, exigir.}{re.qui.si.tar}{\verboinum{1}}
\verb{requisito}{}{}{}{}{s.m.}{Condição para se alcançar determinado fim.}{re.qui.si.to}{0}
\verb{requisitório}{}{Jur.}{}{}{s.m.}{Exposição de motivos feita pelo representante do Ministério Público para justificar a acusação judicial contra alguém.}{re.qui.si.tó.rio}{0}
\verb{rés}{}{}{}{}{adj.2g.}{Raso, rente.}{rés}{0}
\verb{rés}{}{}{}{}{adv.}{Pela raiz; rente.}{rés}{0}
\verb{rês}{}{}{}{}{s.f.}{Qualquer animal quadrúpede que se abate para a alimentação do homem.  }{rês}{0}
\verb{rescaldo}{}{}{}{}{s.m.}{Calor refletido de um incêndio ou fornalha.}{res.cal.do}{0}
\verb{rescaldo}{}{}{}{}{}{Ato de jogar água nas cinzas de um incêndio.}{res.cal.do}{0}
\verb{rescindir}{}{}{}{}{v.t.}{Pôr fim; anular, quebrar.}{res.cin.dir}{\verboinum{18}}
\verb{rescisão}{}{}{"-ões}{}{s.f.}{Anulação de contrato.}{res.ci.são}{0}
\verb{rescisório}{}{}{}{}{adj.}{Que tem por objeto a recisão.}{res.ci.só.rio}{0}
\verb{rescisório}{}{}{}{}{}{Que anula, invalida contrato.}{res.ci.só.rio}{0}
\verb{rescrito}{}{}{}{}{s.m.}{Decisão papal em assuntos teológicos.}{res.cri.to}{0}
\verb{rescrito}{}{Jur.}{}{}{}{Resposta escrita.}{res.cri.to}{0}
\verb{rés"-do"-chão}{}{}{}{}{s.m.}{Pavimento de uma casa que fica ao nível do solo; andar térreo.}{rés"-do"-chão}{0}
\verb{resedá}{}{Bot.}{}{}{s.f.}{Planta de origem africana de flor amarela e perfume intenso.}{re.se.dá}{0}
\verb{resedá}{}{}{}{}{}{A flor dessa planta.}{re.se.dá}{0}
\verb{resedá}{}{}{}{}{}{O perfume que se prepara com essa flor.}{re.se.dá}{0}
\verb{resenha}{}{}{}{}{s.f.}{Texto curto que analisa uma obra.}{re.se.nha}{0}
\verb{resenhar}{}{}{}{}{v.t.}{Fazer resenha de; enumerar, sintetizar.}{re.se.nhar}{\verboinum{1}}
\verb{reserva}{é}{}{}{}{s.f.}{Ato ou efeito de reservar.}{re.ser.va}{0}
\verb{reserva}{é}{}{}{}{}{Lugar garantido com antecedência.}{re.ser.va}{0}
\verb{reserva}{é}{}{}{}{}{Aquilo que se reserva ou guarda para circunstâncias imprevistas.}{re.ser.va}{0}
\verb{reserva}{é}{}{}{}{}{Parque florestal administrado pelo Estado e que se destina a assegurar a conservação das espécies animais e vegetais; reserva natural.}{re.ser.va}{0}
\verb{reserva}{é}{}{}{}{}{Desconfiança natural por pessoa ou coisa desconhecida.}{re.ser.va}{0}
\verb{reserva}{é}{}{}{}{s.2g.}{No futebol ou em outros esportes, atleta que substitui o efetivo em caso de necessidade.}{re.ser.va}{0}
\verb{reservado}{}{}{}{}{adj.}{Que tem reserva; em que há reserva.}{re.ser.va.do}{0}
\verb{reservado}{}{}{}{}{}{Que fala pouco; discreto, retraído.}{re.ser.va.do}{0}
\verb{reservado}{}{}{}{}{s.m.}{Lugar em bares e restaurantes destinado a um pequeno grupo de pessoas que não desejam servir"-se do salão comum.}{re.ser.va.do}{0}
\verb{reservar}{}{}{}{}{v.t.}{Fazer reserva de; pôr de parte; guardar, poupar, conservar.}{re.ser.var}{0}
\verb{reservar}{}{}{}{}{}{Garantir lugar para alguém com antecedência.}{re.ser.var}{0}
\verb{reservar}{}{}{}{}{}{Guardar para si; fazer segredo de; ocultar.}{re.ser.var}{\verboinum{1}}
\verb{reservatório}{}{}{}{}{adj.}{Próprio para armazenar, guardar, conservar.}{re.ser.va.tó.rio}{0}
\verb{reservatório}{}{}{}{}{s.m.}{Depósito de água, caixa de água.}{re.ser.va.tó.rio}{0}
\verb{reservista}{}{}{}{}{s.2g.}{Indivíduo que passou para reserva militar.}{re.ser.vis.ta}{0}
\verb{resfolegar}{}{}{}{}{v.i.}{Recuperar o ar, o fôlego; respirar com dificuldade.}{res.fo.le.gar}{\verboinum{5}}
\verb{resfolgar}{}{}{}{}{}{Var. de \textit{resfolegar}.}{res.fol.gar}{0}
\verb{resfriado}{}{}{}{}{s.m.}{Estado gripal, doença em que o nariz fica congestionado e escorrendo. }{res.fri.a.do}{0}
\verb{resfriado}{}{}{}{}{adj.}{Que está com essa doença.}{res.fri.a.do}{0}
\verb{resfriado}{}{}{}{}{}{Que sofreu resfriamento, que perdeu calor.}{res.fri.a.do}{0}
\verb{resfriamento}{}{}{}{}{s.m.}{Ato ou efeito de resfriar.}{res.fri.a.men.to}{0}
\verb{resfriamento}{}{}{}{}{}{Condição de quem ficou resfriado.}{res.fri.a.men.to}{0}
\verb{resfriamento}{}{}{}{}{}{Diminuição de calor.}{res.fri.a.men.to}{0}
\verb{resfriar}{}{}{}{}{v.t.}{Fazer perder o calor.}{res.fri.ar}{0}
\verb{resfriar}{}{}{}{}{}{Diminuir o ardor, a atividade de; desanimar, desalentar.}{res.fri.ar}{0}
\verb{resfriar}{}{}{}{}{v.i.}{Apanhar um resfriado; gripar.}{res.fri.ar}{\verboinum{6}}
\verb{resgatar}{}{}{}{}{v.t.}{Libertar a preço de dinheiro ou concessões.}{res.ga.tar}{0}
\verb{resgatar}{}{}{}{}{}{Libertar, livrar.}{res.ga.tar}{0}
\verb{resgatar}{}{}{}{}{}{Remir pecado, culpa; apagar, expiar.}{res.ga.tar}{\verboinum{1}}
\verb{resgate}{}{}{}{}{s.m.}{Ato ou efeito de resgatar, de libertar mediante o pagamento de quantia determinada. }{res.ga.te}{0}
\verb{resgate}{}{}{}{}{}{A quantia paga por essa libertação.}{res.ga.te}{0}
\verb{resgate}{}{}{}{}{}{Ato de libertar, de livrar.}{res.ga.te}{0}
\verb{resguardar}{}{}{}{}{}{Abrigar, pôr a salvo.}{res.guar.dar}{\verboinum{1}}
\verb{resguardar}{}{}{}{}{}{Defender, proteger.}{res.guar.dar}{0}
\verb{resguardar}{}{}{}{}{v.t.}{Guardar com cuidado.}{res.guar.dar}{0}
\verb{resguardo}{}{}{}{}{s.m.}{Ato ou efeito de resguardar.}{res.guar.do}{0}
\verb{resguardo}{}{}{}{}{}{Dieta e outros cuidados de saúde da mãe no período pós"-parto.}{res.guar.do}{0}
\verb{resguardo}{}{}{}{}{}{Precaução, cuidado.}{res.guar.do}{0}
\verb{residência}{}{}{}{}{}{Período em que o médico recém"-formado faz treinamento em hospitais para poder exercer plenamente suas funções.  }{re.si.dên.cia}{0}
\verb{residência}{}{}{}{}{s.f.}{Lugar em que se mora; casa, moradia, domicílio.}{re.si.dên.cia}{0}
\verb{residencial}{}{}{"-ais}{}{adj.2g.}{Relativo a ou em que há residência.}{re.si.den.ci.al}{0}
\verb{residente}{}{}{}{}{adj.2g.}{Que reside, que mora.}{re.si.den.te}{0}
\verb{residente}{}{}{}{}{s.m.}{Médico recém"-formado que faz treinamento em hospitais a fim de poder exercer plenamente suas funções. }{re.si.den.te}{0}
\verb{residir}{}{}{}{}{v.t.}{Morar, habitar.}{re.si.dir}{0}
\verb{residir}{}{}{}{}{}{Estar; achar"-se.}{re.si.dir}{0}
\verb{residir}{}{}{}{}{}{Consistir em.}{re.si.dir}{\verboinum{18}}
\verb{residual}{}{}{"-ais}{}{adj.2g.}{Referente a ou próprio de resíduo.}{re.si.du.al}{0}
\verb{resíduo}{}{}{}{}{s.m.}{Aquilo que resta; restante, sobra, resto.}{re.sí.du.o}{0}
\verb{resignação}{}{}{"-ões}{}{s.f.}{Ato ou efeito de resignar.  }{re.sig.na.ção}{0}
\verb{resignação}{}{}{"-ões}{}{}{Aceitação da dor física ou  moral; paciência.}{re.sig.na.ção}{0}
\verb{resignado}{}{}{}{}{adj.}{Que aceita as coisas ruins sem reclamar.}{re.sig.na.do}{0}
\verb{resignar}{}{}{}{}{v.pron.}{Aceitar os problemas sem reclamar; conformar"-se.}{re.sig.nar}{0}
\verb{resignar}{}{}{}{}{v.t.}{Demitir"-se voluntariamente; abandonar.}{re.sig.nar}{\verboinum{1}}
\verb{resignatário}{}{}{}{}{adj.}{Que resigna um cargo ou uma dignidade. }{re.sig.na.tá.rio}{0}
\verb{resilir}{}{}{}{}{v.t.}{Romper um contrato; anular, rescindir.}{re.si.lir}{0}
\verb{resilir}{}{}{}{}{}{Voltar para trás, retornar.}{re.si.lir}{\verboinum{18}}
\verb{resina}{}{}{}{}{s.f.}{Substância viscosa, odorífera, combustível, produzida por certas plantas.  }{re.si.na}{0}
\verb{resinoso}{ô}{}{"-osos ⟨ó⟩}{"-osa ⟨ó⟩}{adj.}{Cheio, coberto de resina.}{re.si.no.so}{0}
\verb{resinoso}{ô}{}{"-osos ⟨ó⟩}{"-osa ⟨ó⟩}{}{Que produz ou tem as propriedades da resina.}{re.si.no.so}{0}
\verb{resistência}{}{}{}{}{}{Causa que se opõe; oposição.}{re.sis.tên.cia}{0}
\verb{resistência}{}{}{}{}{s.f.}{Ato ou efeito de resistir.}{re.sis.tên.cia}{0}
\verb{resistência}{}{}{}{}{}{Reação defensiva do organismo.}{re.sis.tên.cia}{0}
\verb{resistência}{}{}{}{}{}{Capacidade de suportar peso, desgaste.}{re.sis.tên.cia}{0}
\verb{resistente}{}{}{}{}{adj.2g.}{Que resiste, que apresenta resistência.}{re.sis.ten.te}{0}
\verb{resistente}{}{}{}{}{}{Duro, sólido, firme.}{re.sis.ten.te}{0}
\verb{resistir}{}{}{}{}{v.t.}{Lutar para não ser vencido; opor"-se.}{re.sis.tir}{0}
\verb{resistir}{}{}{}{}{}{Não ceder; aguentar, suportar.}{re.sis.tir}{0}
\verb{resistir}{}{}{}{}{}{Recusar"-se; negar"-se a.}{re.sis.tir}{\verboinum{18}}
\verb{resistor}{ô}{}{}{}{s.m.}{Peça que, num circuito elétrico, apresenta  resistência.}{re.sis.tor}{0}
\verb{resma}{ê}{}{}{}{s.f.}{Pacote com 500 folhas de papel.}{res.ma}{0}
\verb{resmungão}{}{}{"-ões}{resmungona}{adj.}{Que resmunga muito.}{res.mun.gão}{0}
\verb{resmungar}{}{}{}{}{v.t.}{Falar em voz baixa, de forma incompreensível, reclamando de alguma coisa.}{res.mun.gar}{\verboinum{5}}
\verb{resmungo}{}{}{}{}{s.m.}{Ato ou efeito de resmungar; reclamação feita em voz baixa e entre dentes.}{res.mun.go}{0}
\verb{resolução}{}{}{"-ões}{}{s.f.}{Ato ou efeito de resolver.}{re.so.lu.ção}{0}
\verb{resolução}{}{}{"-ões}{}{}{Decisão, deliberação.}{re.so.lu.ção}{0}
\verb{resolução}{}{}{"-ões}{}{}{Capacidade de tornar bem visível e nítida a imagem produzida por câmara de fotografia, cinema ou \textsc{tv}; qualidade da imagem.}{re.so.lu.ção}{0}
\verb{resoluto}{}{}{}{}{adj.}{Decidido, disposto, firme na decisão.}{re.so.lu.to}{0}
\verb{resolúvel}{}{}{"-eis}{}{adj.2g.}{Que é possível resolver.}{re.so.lú.vel}{0}
\verb{resolver}{ê}{}{}{}{v.t.}{Encontrar solução; solucionar, esclarecer, explicar.}{re.sol.ver}{0}
\verb{resolver}{ê}{}{}{}{}{Decidir, deliberar.}{re.sol.ver}{\verboinum{12}}
\verb{resolvido}{}{}{}{}{adj.}{Combinado, acertado.}{re.sol.vi.do}{0}
\verb{resolvido}{}{}{}{}{}{Decidido, determinado.}{re.sol.vi.do}{0}
\verb{respaldar}{}{}{}{}{v.t.}{Tornar plano ou liso.}{res.pal.dar}{0}
\verb{respaldar}{}{}{}{}{}{Dar respaldo, proteção; amparar.}{res.pal.dar}{\verboinum{1}}
\verb{respaldo}{}{}{}{}{s.m.}{Ato ou efeito de respaldar.}{res.pal.do}{0}
\verb{respaldo}{}{}{}{}{}{O encosto da cadeira; espaldar.}{res.pal.do}{0}
\verb{respaldo}{}{Fig.}{}{}{}{Apoio, proteção.}{res.pal.do}{0}
\verb{respectivo}{}{}{}{}{adj.}{Relativo a cada um em particular.}{res.pec.ti.vo}{0}
\verb{respectivo}{}{}{}{}{}{Correspondente, próprio, seu, devido.}{res.pec.ti.vo}{0}
\verb{respeitabilidade}{}{}{}{}{s.f.}{Qualidade de respeitável.}{res.pei.ta.bi.li.da.de}{0}
\verb{respeitador}{ô}{}{}{}{adj.}{Que respeita.}{res.pei.ta.dor}{0}
\verb{respeitante}{}{}{}{}{adj.2g.}{Que diz respeito; referente, relativo.}{res.pei.tan.te}{0}
\verb{respeitar}{}{}{}{}{v.t.}{Tratar com respeito; considerar. (\textit{Faz parte da educação respeitar as pessoas mais velhas. })}{res.pei.tar}{0}
\verb{respeitar}{}{}{}{}{}{Obedecer, seguir, cumprir. (\textit{Há várias campanhas no país para que as pessoas respeitem as leis de trânsito.})}{res.pei.tar}{\verboinum{1}}
\verb{respeitável}{}{}{"-eis}{}{adj.2g.}{Que merece respeito.}{res.pei.tá.vel}{0}
\verb{respeitável}{}{}{"-eis}{}{}{Importante. (\textit{O acusado é um respeitável morador da cidade.})}{res.pei.tá.vel}{0}
\verb{respeito}{ê}{}{}{}{s.m.}{Ato ou efeito de respeitar.}{res.pei.to}{0}
\verb{respeito}{ê}{}{}{}{}{Sentimento que faz alguém tratar uma pessoa ou coisa com grande atenção; consideração, reverência, apreço. }{res.pei.to}{0}
\verb{respeito}{ê}{}{}{}{}{Obediência, acatamento.}{res.pei.to}{0}
\verb{respeito}{ê}{}{}{}{}{Medo, receio.}{res.pei.to}{0}
\verb{respeitoso}{ô}{}{"-osos ⟨ó⟩}{"-osa ⟨ó⟩}{adj.}{Que mostra respeito; cortês.}{res.pei.to.so}{0}
\verb{respetivo}{}{}{}{}{}{Var. de \textit{respectivo}.}{res.pe.ti.vo}{0}
\verb{respigar}{}{}{}{}{v.t.}{Recolher as espigas que ficaram depois da colheita.}{res.pi.gar}{0}
\verb{respigar}{}{Fig.}{}{}{}{Catar, apanhar aqui e ali; compilar.}{res.pi.gar}{\verboinum{5}}
\verb{respingar}{}{}{}{}{v.t.}{Molhar alguma coisa lançando pingos, borrifos ou salpicos; salpicar.}{res.pin.gar}{\verboinum{5}}
\verb{respingo}{}{}{}{}{s.m.}{Ato ou efeito de respingar; pingo, salpico.}{res.pin.go}{0}
\verb{respiração}{}{}{"-ões}{}{s.f.}{Ato ou efeito de respirar.}{res.pi.ra.ção}{0}
\verb{respiração}{}{}{"-ões}{}{}{Função pela qual os seres vivos absorvem oxigênio e expelem gás carbônico.}{res.pi.ra.ção}{0}
\verb{respiração}{}{}{"-ões}{}{}{Hálito, fôlego, bafo.}{res.pi.ra.ção}{0}
\verb{respirador}{ô}{}{}{}{adj.}{Que respira ou serve para respirar.}{res.pi.ra.dor}{0}
\verb{respirador}{ô}{}{}{}{s.m.}{Aparelho que serve para ajudar a respiração.}{res.pi.ra.dor}{0}
\verb{respiradouro}{ô}{}{}{}{s.m.}{Respiro.}{res.pi.ra.dou.ro}{0}
\verb{respirar}{}{}{}{}{v.t.}{Encher e esvaziar os pulmões de ar.}{res.pi.rar}{0}
\verb{respirar}{}{}{}{}{}{Descansar depois de tarefa árdua ou de preocupação.}{res.pi.rar}{\verboinum{1}}
\verb{respiratório}{}{}{}{}{adj.}{Relativo à respiração.}{res.pi.ra.tó.rio}{0}
\verb{respiratório}{}{}{}{}{}{Que facilita a respiração.}{res.pi.ra.tó.rio}{0}
\verb{respirável}{}{}{"-eis}{}{adj.2g.}{Que se pode respirar.}{res.pi.rá.vel}{0}
\verb{respiro}{}{}{}{}{s.m.}{Respiração.}{res.pi.ro}{0}
\verb{respiro}{}{}{}{}{}{Abertura nos fornos, aparelhos de aquecimento, dentre outros, para dar passagem ao ar, à fumaça, a gases; respiradouro.}{res.pi.ro}{0}
\verb{respiro}{}{Fig.}{}{}{}{Folga, descanso.}{res.pi.ro}{0}
\verb{resplandecência}{}{}{}{}{s.f.}{Ato ou efeito de resplandecer; resplendor.}{res.plan.de.cên.cia}{0}
\verb{resplandecência}{}{}{}{}{}{Grande brilho formado pela reflexão da luz.}{res.plan.de.cên.cia}{0}
\verb{resplandecente}{}{}{}{}{adj.2g.}{Que resplandece; brilhante.}{res.plan.de.cen.te}{0}
\verb{resplandecer}{ê}{}{}{}{v.i.}{Brilhar intensamente; reluzir.}{res.plan.de.cer}{0}
\verb{resplandecer}{ê}{}{}{}{}{Sobressair; mostrar"-se notável.}{res.plan.de.cer}{\verboinum{15}}
\verb{resplendente}{}{}{}{}{adj.2g.}{Que resplende; resplendecente.}{res.plen.den.te}{0}
\verb{resplender}{ê}{}{}{}{v.t.}{Resplandecer.}{res.plen.der}{\verboinum{12}}
\verb{resplendor}{ô}{}{}{}{s.m.}{Brilho intenso; resplandecência.}{res.plen.dor}{0}
\verb{resplendor}{ô}{Fig.}{}{}{}{Fama, glória.}{res.plen.dor}{0}
\verb{resplendoroso}{ô}{}{"-osos ⟨ó⟩}{"-osa ⟨ó⟩}{adj.}{Que tem resplendor; resplandecente.}{res.plen.do.ro.so}{0}
\verb{respondão}{}{}{"-ões}{}{adj.}{Que responde mal, que dá respostas mal"-educadas, grosseiras.}{res.pon.dão}{0}
\verb{responder}{ê}{}{}{}{v.t.}{Dar resposta. (\textit{Respondi ontem à carta da minha amiga.})}{res.pon.der}{0}
\verb{responder}{ê}{}{}{}{}{Estar de acordo, em harmonia; corresponder.}{res.pon.der}{0}
\verb{responder}{ê}{}{}{}{}{Apresentar reação; reagir. (\textit{Apesar da gravidade da doença, o paciente respondia à medicação.})}{res.pon.der}{0}
\verb{responder}{ê}{}{}{}{}{Ser responsável por algo; responsabilizar"-se. (\textit{A secretária respondia por todos os serviços do departamento.})}{res.pon.der}{0}
\verb{responder}{ê}{}{}{}{}{Repetir a voz, o som. (\textit{O eco responde a nossa voz na caverna.})}{res.pon.der}{\verboinum{12}}
\verb{responsabilidade}{}{}{}{}{s.f.}{Qualidade de responsável.}{res.pon.sa.bi.li.da.de}{0}
\verb{responsabilidade}{}{}{}{}{}{Obrigação de responder por certos atos ou fatos.}{res.pon.sa.bi.li.da.de}{0}
\verb{responsabilizar}{}{}{}{}{v.t.}{Atribuir responsabilidade a.}{res.pon.sa.bi.li.zar}{\verboinum{1}}
\verb{responsável}{}{}{"-eis}{}{adj.2g.}{Que tem ou assumiu responsabilidade.}{res.pon.sá.vel}{0}
\verb{responsável}{}{}{"-eis}{}{}{Que tem de responder pelos seus atos.}{res.pon.sá.vel}{0}
\verb{responsável}{}{}{"-eis}{}{}{Que tem culpa; culpado, causador.}{res.pon.sá.vel}{0}
\verb{responsável}{}{}{"-eis}{}{s.m.}{Pessoa que é chamada a responder pelos atos de outra, menor de idade ou incapacitada.}{res.pon.sá.vel}{0}
\verb{responso}{}{Relig.}{}{}{s.m.}{Versículos rezados ou cantados com o coro na liturgia da missa.}{res.pon.so}{0}
\verb{responso}{}{Relig.}{}{}{}{Oração dirigida a Santo Antônio para se achar alguma coisa que se perdeu.}{res.pon.so}{0}
\verb{responsório}{}{Relig.}{}{}{s.m.}{Conjunto, coleção de responsos.}{res.pon.só.rio}{0}
\verb{resposta}{ó}{}{}{}{s.f.}{Ato ou efeito de responder; palavra ou gesto com que se responde a uma pergunta.}{res.pos.ta}{0}
\verb{resposta}{ó}{}{}{}{}{Réplica, contestação.}{res.pos.ta}{0}
\verb{resposta}{ó}{}{}{}{}{Reação a um estímulo.}{res.pos.ta}{0}
\verb{resquício}{}{}{}{}{s.m.}{Aquilo que restou; vestígio, resíduo.}{res.quí.cio}{0}
\verb{ressabiado}{}{}{}{}{adj.}{Que se ressabiou; desconfiado, assustadiço.}{res.sa.bi.a.do}{0}
\verb{ressabiar}{}{}{}{}{v.i.}{Ofender, desgostar, melindrar.}{res.sa.bi.ar}{\verboinum{6}}
\verb{ressaca}{}{}{}{}{s.f.}{Movimento forte de fluxo e refluxo das ondas do mar, batendo nas rochas ou chegando à praia.}{res.sa.ca}{0}
\verb{ressaca}{}{}{}{}{}{Mal"-estar causado por noite passada em claro ou por ingestão de bebida alcoólica.}{res.sa.ca}{0}
\verb{ressaibo}{}{}{}{}{s.m.}{Sabor ruim, desagradável; ranço.}{res.sai.bo}{0}
\verb{ressaibo}{}{Fig.}{}{}{}{Ressentimento, decepção, desagrado.}{res.sai.bo}{0}
\verb{ressaibo}{}{}{}{}{}{Vestígio, indício, sinal.}{res.sai.bo}{0}
\verb{ressaltar}{}{}{}{}{v.t.}{Tornar saliente; realçar, relevar.}{res.sal.tar}{0}
\verb{ressaltar}{}{}{}{}{v.i.}{Distinguir"-se, destacar"-se.}{res.sal.tar}{\verboinum{1}}
\verb{ressalva}{}{}{}{}{s.f.}{Observação feita para emendar ou corrigir um texto.}{res.sal.va}{0}
\verb{ressalva}{}{}{}{}{}{Exceção, restrição, reserva.}{res.sal.va}{0}
\verb{ressalvar}{}{}{}{}{v.t.}{Fazer emenda em; corrigir.}{res.sal.var}{0}
\verb{ressalvar}{}{}{}{}{}{Livrar de responsabilidade ou culpa; eximir.}{res.sal.var}{\verboinum{1}}
\verb{ressarcimento}{}{}{}{}{s.m.}{Ato ou efeito de ressarcir; indenização, reparação.}{res.sar.ci.men.to}{0}
\verb{ressarcir}{}{}{}{}{v.t.}{Pagar a despesa ou o prejuízo de alguém; indenizar, compensar.}{res.sar.cir}{\verboinum{35}}
\verb{resseção}{}{}{}{}{}{Var. de \textit{ressecção}.}{res.se.ção}{0}
\verb{ressecar}{}{}{}{}{v.t.}{Tornar seco ou muito seco.}{res.se.car}{\verboinum{2}}
\verb{ressecção}{}{Med.}{"-ões}{}{s.f.}{Ato de extirpar parte de um órgão.}{res.sec.ção}{0}
\verb{ressegurar}{}{}{}{}{v.t.}{Pôr novamente em seguro.}{res.se.gu.rar}{0}
\verb{ressegurar}{}{}{}{}{}{Fazer resseguro.}{res.se.gu.rar}{\verboinum{1}}
\verb{resseguro}{}{}{}{}{s.m.}{Operação realizada por uma seguradora ao contrair o seguro efetuado por outra companhia, aliviando parte do risco de um seguro importante.}{res.se.gu.ro}{0}
\verb{ressentido}{}{}{}{}{adj.}{Que se ressentiu; magoado, melindrado.}{res.sen.ti.do}{0}
\verb{ressentimento}{}{}{}{}{s.m.}{Sentimento de mágoa por uma ofensa ou por um mal que se recebeu; rancor.}{res.sen.ti.men.to}{0}
\verb{ressentir}{}{}{}{}{v.t.}{Magoar profundamente; ofender, melindrar.}{res.sen.tir}{0}
\verb{ressentir}{}{}{}{}{v.pron.}{Sentir os efeitos de.}{res.sen.tir}{\verboinum{29}}
\verb{ressequido}{}{}{}{}{adj.}{Que está muito seco, sem umidade; mirrado. (\textit{Devido à falta de chuvas, o solo ficou ressequido.})}{res.se.qui.do}{0}
\verb{ressequir}{}{}{}{}{v.t.}{Fazer perder toda a umidade; ressecar.}{res.se.quir}{\verboinum{35}\verboirregular{\emph{def.} ressequimos, ressequis}}
\verb{ressoante}{}{}{}{}{adj.2g.}{Que ressoa; retumbante.}{res.so.an.te}{0}
\verb{ressoar}{}{}{}{}{v.t.}{Soar novamente.}{res.so.ar}{0}
\verb{ressoar}{}{}{}{}{}{Repercutir, ecoar.}{res.so.ar}{\verboinum{7}}
\verb{ressonância}{}{}{}{}{s.f.}{Repercussão do som; eco.}{res.so.nân.cia}{0}
\verb{ressonância}{}{}{}{}{}{Aumento da intensidade de um som.}{res.so.nân.cia}{0}
\verb{ressonância}{}{Fig.}{}{}{}{Reação despertada nas pessoas por um fato; repercussão.}{res.so.nân.cia}{0}
\verb{ressonante}{}{}{}{}{adj.2g.}{Que ressoa; retumbante, ressoante.}{res.so.nan.te}{0}
\verb{ressonar}{}{}{}{}{v.t.}{Fazer soar; ressoar.}{res.so.nar}{0}
\verb{ressonar}{}{}{}{}{}{Respirar ruidosamente enquanto dorme; roncar.}{res.so.nar}{\verboinum{1}}
\verb{ressuar}{}{}{}{}{v.t.}{Transpirar, suar em demasia.}{res.su.ar}{\verboinum{1}}
\verb{ressudar}{}{}{}{}{v.i.}{Expelir suor pelos poros em demasia; gotejar.}{res.su.dar}{\verboinum{1}}
\verb{ressumar}{}{}{}{}{v.t.}{Deixar líquido passar gota a gota; destilar, gotejar, ressumbrar.}{res.su.mar}{\verboinum{1}}
\verb{ressumbrar}{}{}{}{}{v.t.}{Ressumar.}{res.sum.brar}{\verboinum{1}}
\verb{ressupino}{}{}{}{}{adj.}{Que está deitado de costas.}{res.su.pi.no}{0}
\verb{ressurgir}{}{}{}{}{v.i.}{Surgir de novo; reaparecer.}{res.sur.gir}{0}
\verb{ressurgir}{}{}{}{}{}{Voltar à vida; ressuscitar.}{res.sur.gir}{\verboinum{22}}
\verb{ressurreição}{}{}{"-ões}{}{s.f.}{Ato ou efeito de ressurgir ou ressuscitar.}{res.sur.rei.ção}{0}
\verb{ressurreição}{}{Relig.}{"-ões}{}{}{Festa em que a Igreja Católica celebra a ressurreição de Jesus Cristo; Páscoa.}{res.sur.rei.ção}{0}
\verb{ressuscitar}{}{}{}{}{v.i.}{Voltar à vida; ressurgir.}{res.sus.ci.tar}{\verboinum{1}}
\verb{restabelecer}{ê}{}{}{}{v.t.}{Fazer alguma coisa retornar ao seu estado normal; restaurar, reparar, recuperar.}{res.ta.be.le.cer}{0}
\verb{restabelecer}{ê}{}{}{}{v.pron.}{Readquirir a saúde; curar"-se, recuperar"-se.}{res.ta.be.le.cer}{\verboinum{15}}
\verb{restabelecimento}{}{}{}{}{s.m.}{Ato ou efeito de restabelecer; restauração, recuperação.}{res.ta.be.le.ci.men.to}{0}
\verb{restante}{}{}{}{}{adj.2g.}{Que resta, que sobra; resto.}{res.tan.te}{0}
\verb{restar}{}{}{}{}{v.i.}{Existir após o uso ou o gasto; ficar, sobrar.}{res.tar}{0}
\verb{restar}{}{}{}{}{}{Ficar de sobra; sobrar, sobejar.}{res.tar}{0}
\verb{restar}{}{}{}{}{v.t.}{Continuar como remanescente; subsistir.}{res.tar}{\verboinum{1}}
\verb{restauração}{}{}{"-ões}{}{s.f.}{Ato ou efeito de restaurar; recuperação, renovação.}{res.tau.ra.ção}{0}
\verb{restaurador}{ô}{}{}{}{adj.}{Que restaura, recupera.}{res.tau.ra.dor}{0}
\verb{restaurante}{}{}{}{}{s.m.}{Estabelecimento onde se preparam e servem refeições.}{res.tau.ran.te}{0}
\verb{restaurar}{}{}{}{}{v.t.}{Colocar em bom estado; recuperar, reformar, reparar.}{res.tau.rar}{0}
\verb{restaurar}{}{}{}{}{}{Dar novo ânimo; restabelecer, revigorar.}{res.tau.rar}{\verboinum{1}}
\verb{réstia}{}{}{}{}{}{Corda feita de hastes entrelaçadas.}{rés.tia}{0}
\verb{réstia}{}{}{}{}{s.f.}{Feixe de luz que passa por uma abertura estreita.}{rés.tia}{0}
\verb{restinga}{}{Geol.}{}{}{s.f.}{Faixa de areia ou pedra que se prolonga mar adentro.}{res.tin.ga}{0}
\verb{restinga}{}{}{}{}{}{Faixa de mato à beira de um rio ou lagoa.}{res.tin.ga}{0}
\verb{restituição}{}{}{"-ões}{}{s.f.}{Ato ou efeito de restituir; devolução, indenização, restauração.}{res.ti.tu.i.ção}{0}
\verb{restituir}{}{}{}{}{v.t.}{Fazer voltar ao seu dono algo que lhe pertence; devolver, ressarcir.}{res.ti.tu.ir}{0}
\verb{restituir}{}{}{}{}{}{Fazer voltar ao estado normal; restabelecer. (\textit{O medicamento restituiu"-lhe a força.})}{res.ti.tu.ir}{\verboinum{26}}
\verb{resto}{é}{}{}{}{s.m.}{Aquilo que fica, que sobra; resíduo, resquício.}{resto}{0}
\verb{resto}{é}{}{}{}{}{Diferença, saldo.}{resto}{0}
\verb{resto}{é}{Mat.}{}{}{}{Numa divisão, quantidade final que não pode ser dividida em números inteiros por ser menor que o divisor.}{resto}{0}
\verb{restolho}{ô}{}{}{}{s.m.}{Parte do caule das plantas gramíneas que fica enraizada depois da ceifa.}{res.to.lho}{0}
\verb{restrição}{}{}{"-ões}{}{s.f.}{Ato ou efeito de restringir; limitação.}{res.tri.ção}{0}
\verb{restringente}{}{Farm.}{}{}{adj.2g.}{Diz"-se de medicamento que fortalece e une as partes relaxadas.}{res.trin.gen.te}{0}
\verb{restringir}{}{}{}{}{v.t.}{Tornar mais estreito ou limitado; delimitar.}{res.trin.gir}{0}
\verb{restringir}{}{}{}{}{v.pron.}{Refrear"-se, ater"-se.}{res.trin.gir}{\verboinum{22}}
\verb{restritivo}{}{}{}{}{adj.}{Que restringe, limita.}{res.tri.ti.vo}{0}
\verb{restrito}{}{}{}{}{adj.}{Que se restringiu; limitado, reduzido.}{res.tri.to}{0}
\verb{restrito}{}{}{}{}{}{Específico, estrito.}{res.tri.to}{0}
\verb{resultado}{}{}{}{}{s.m.}{Ato ou efeito de resultar; consequência, efeito.}{re.sul.ta.do}{0}
\verb{resultado}{}{Mat.}{}{}{}{Produto de uma operação.}{re.sul.ta.do}{0}
\verb{resultante}{}{}{}{}{adj.2g.}{Que resulta; consequente.}{re.sul.tan.te}{0}
\verb{resultar}{}{}{}{}{v.t.}{Ser o efeito, a consequência, a conclusão lógica.}{re.sul.tar}{0}
\verb{resultar}{}{}{}{}{}{Originar, provir, proceder.}{re.sul.tar}{\verboinum{1}}
\verb{resumido}{}{}{}{}{adj.}{Que se resumiu; abreviado, curto, sintético.}{re.su.mi.do}{0}
\verb{resumir}{}{}{}{}{v.t.}{Fazer o resumo; abreviar, sintetizar.}{re.su.mir}{0}
\verb{resumir}{}{}{}{}{}{Restringir, limitar, reduzir.}{re.su.mir}{\verboinum{18}}
\verb{resumo}{}{}{}{}{s.m.}{Conjunto das principais ideias de um tema ou assunto; síntese, sumário.}{re.su.mo}{0}
\verb{resvaladiço}{}{}{}{}{adj.}{Que tem inclinação acentuada; íngreme, escorregadio, deslizante.}{res.va.la.di.ço}{0}
\verb{resvaladio}{}{}{}{}{adj.}{Resvaladiço.}{res.va.la.di.o}{0}
\verb{resvalar}{}{}{}{}{v.t.}{Passar levemente por uma superfície; deslizar.}{res.va.lar}{\verboinum{1}}
\verb{reta}{é}{}{}{}{s.f.}{Linha que segue a mesma direção; traço direito.}{re.ta}{0}
\verb{retábulo}{}{}{}{}{s.m.}{Painel ou quadro que enfeita um altar.}{re.tá.bu.lo}{0}
\verb{retaguarda}{}{}{}{}{s.f.}{Grupo de soldados que formam a última parte do corpo de um exército em marcha.}{re.ta.guar.da}{0}
\verb{retaguarda}{}{}{}{}{}{A parte de trás; traseira.}{re.ta.guar.da}{0}
\verb{retal}{}{Anat.}{"-ais}{}{adj.2g.}{Relativo ou pertencente ao reto.}{re.tal}{0}
\verb{retalhar}{}{}{}{}{v.t.}{Cortar alguma coisa em pedaços ou retalhos.}{re.ta.lhar}{0}
\verb{retalhar}{}{}{}{}{}{Abrir cortes sobre uma superfície.}{re.ta.lhar}{\verboinum{1}}
\verb{retalheiro}{ê}{}{}{}{s.m.}{Retalhista.}{re.ta.lhei.ro}{0}
\verb{retalhista}{}{}{}{}{s.2g.}{Indivíduo que vende a retalho; varejista.}{re.ta.lhis.ta}{0}
\verb{retalho}{}{}{}{}{s.m.}{Pedaço de coisa retalhada, principalmente tecido.}{re.ta.lho}{0}
\verb{retaliação}{}{}{"-ões}{}{s.f.}{Ato ou efeito de retaliar; represália, vingança.}{re.ta.li.a.ção}{0}
\verb{retaliar}{}{}{}{}{v.t.}{Revidar da mesma forma uma ofensa recebida; vingar, desagravar.}{re.ta.li.ar}{\verboinum{6}}
\verb{retangular}{}{}{}{}{adj.2g.}{Que tem a forma de um retângulo.}{re.tan.gu.lar}{0}
\verb{retângulo}{}{Geom.}{}{}{adj.}{Diz"-se do triângulo que tem um ângulo reto.}{re.tân.gu.lo}{0}
\verb{retângulo}{}{Geom.}{}{}{s.m.}{Quadrilátero cujos ângulos são retos.}{re.tân.gu.lo}{0}
\verb{retardado}{}{}{}{}{adj.}{Que se retardou; atrasado, demorado.}{re.tar.da.do}{0}
\verb{retardado}{}{}{}{}{}{Diz"-se do indivíduo com desenvolvimento mental abaixo do índice normal para sua idade.}{re.tar.da.do}{0}
\verb{retardador}{ô}{}{}{}{adj.}{Que retarda, demora.}{re.tar.da.dor}{0}
\verb{retardamento}{}{}{}{}{s.m.}{Ato ou efeito de retardar; demora, atraso.}{re.tar.da.men.to}{0}
\verb{retardar}{}{}{}{}{v.t.}{Fazer chegar mais tarde; atrasar.}{re.tar.dar}{0}
\verb{retardar}{}{}{}{}{}{Adiar, protelar.}{re.tar.dar}{\verboinum{1}}
\verb{retardatário}{}{}{}{}{adj.}{Que chega tarde; atrasado.}{re.tar.da.tá.rio}{0}
\verb{retelhar}{}{}{}{}{v.t.}{Cobrir uma construção com novo telhado.}{re.te.lhar}{\verboinum{1}}
\verb{retemperar}{}{}{}{}{v.t.}{Dar nova têmpera; revigorar, fortificar.}{re.tem.pe.rar}{0}
\verb{retemperar}{}{}{}{}{}{Colocar novo tempero, condimento.}{re.tem.pe.rar}{\verboinum{1}}
\verb{retenção}{}{}{"-ões}{}{s.f.}{Ato ou efeito de reter; demora, permanência, detenção.}{re.ten.ção}{0}
\verb{retentiva}{}{}{}{}{s.f.}{Faculdade de conservar na memória as impressões registradas.}{re.ten.ti.va}{0}
\verb{retentor}{ô}{}{}{}{adj.}{Que retém, que impede de sair.}{re.ten.tor}{0}
\verb{reter}{ê}{}{}{}{v.t.}{Guardar, conservar consigo.}{re.ter}{0}
\verb{reter}{ê}{}{}{}{}{Impedir de sair; deter, conter, reprimir.}{re.ter}{0}
\verb{reter}{ê}{}{}{}{}{Fazer parar; não deixar avançar.}{re.ter}{\verboinum{39}}
\verb{retesar}{}{}{}{}{v.t.}{Tornar tenso, rígido; esticar, estirar.}{re.te.sar}{\verboinum{1}}
\verb{reticência}{}{}{}{}{s.f.}{Omissão voluntária do que se devia ou podia dizer.}{re.ti.cên.cia}{0}
\verb{reticências}{}{Gram.}{}{}{s.f.pl.}{Sinal de pontuação, marcado por três pontos sucessivos, que indica interrupção do que se estava dizendo.}{re.ti.cên.cias}{0}
\verb{reticente}{}{}{}{}{adj.2g.}{Que age com reticência; vacilante, hesitante.}{re.ti.cen.te}{0}
\verb{retícula}{}{}{}{}{s.f.}{Pequena rede formada por um grande número de retas finíssimas usada em aparelhos ópticos ou na reprodução de imagens.}{re.tí.cu.la}{0}
\verb{retícula}{}{}{}{}{}{Os pontos, linhas e traços que formam essa rede.}{re.tí.cu.la}{0}
\verb{reticulado}{}{}{}{}{adj.}{Que tem forma de rede; reticular.}{re.ti.cu.la.do}{0}
\verb{reticular}{}{}{}{}{adj.2g.}{Reticulado.}{re.ti.cu.lar}{0}
\verb{retículo}{}{Biol.}{}{}{s.m.}{Estrutura fina formada por células ou por algumas estruturas no interior das fibras de tecido conjuntivo das células.}{re.tí.cu.lo}{0}
\verb{retidão}{}{}{"-ões}{}{s.f.}{Qualidade de reto.}{re.ti.dão}{0}
\verb{retidão}{}{}{"-ões}{}{}{Comportamento correto; integridade, honestidade.}{re.ti.dão}{0}
\verb{retífica}{}{}{}{}{s.f.}{Oficina onde é feita a retificação de motores de automóveis.}{re.tí.fi.ca}{0}
\verb{retificação}{}{}{"-ões}{}{s.f.}{Ato ou efeito de retificar.}{re.ti.fi.ca.ção}{0}
\verb{retificação}{}{}{"-ões}{}{}{Correção, emenda, alinhamento.}{re.ti.fi.ca.ção}{0}
\verb{retificação}{}{}{"-ões}{}{}{Recondicionamento de peças, motores etc.}{re.ti.fi.ca.ção}{0}
\verb{retificar}{}{}{}{}{v.t.}{Tornar reto; alinhar.}{re.ti.fi.car}{0}
\verb{retificar}{}{}{}{}{}{Retratar, corrigir, emendar.}{re.ti.fi.car}{0}
\verb{retificar}{}{}{}{}{}{Recondicionar peças, motores etc.}{re.ti.fi.car}{\verboinum{2}}
\verb{retilíneo}{}{}{}{}{adj.}{Que está em linha reta.}{re.ti.lí.neo}{0}
\verb{retilíneo}{}{Fig.}{}{}{}{Honesto, justo, correto.}{re.ti.lí.neo}{0}
\verb{retina}{}{Anat.}{}{}{s.f.}{Membrana interna do olho capaz de captar os sinais luminosos.}{re.ti.na}{0}
\verb{retinir}{}{}{}{}{v.i.}{Tinir por muito tempo.}{re.ti.nir}{0}
\verb{retinir}{}{}{}{}{}{Ressoar, ecoar.}{re.ti.nir}{\verboinum{34}}
\verb{retinto}{}{}{}{}{adj.}{Que é muito carregado na cor.}{re.tin.to}{0}
\verb{retirada}{}{}{}{}{s.f.}{Ato ou efeito de retirar.}{re.ti.ra.da}{0}
\verb{retirada}{}{}{}{}{}{Movimento de parte de uma população para fugir das calamidades como seca, enchente etc; emigração.}{re.ti.ra.da}{0}
\verb{retirada}{}{}{}{}{}{Movimento de recuo das tropas de um exército fugindo do inimigo.}{re.ti.ra.da}{0}
\verb{retirada}{}{}{}{}{}{Saque de dinheiro em banco.}{re.ti.ra.da}{0}
\verb{retirado}{}{}{}{}{adj.}{Que se retirou.}{re.ti.ra.do}{0}
\verb{retirado}{}{}{}{}{}{Pouco frequentado; afastado, distante.}{re.ti.ra.do}{0}
\verb{retirante}{}{}{}{}{s.2g.}{Sertanejo que, fugindo da seca, parte do lugar onde mora em direção a outras regiões.}{re.ti.ran.te}{0}
\verb{retirar}{}{}{}{}{v.t.}{Tirar de onde estava; afastar, recolher.}{re.ti.rar}{0}
\verb{retirar}{}{}{}{}{}{Retratar, desdizer.}{re.ti.rar}{0}
\verb{retirar}{}{}{}{}{v.i.}{Ir"-se embora; afastar"-se.}{re.ti.rar}{\verboinum{1}}
\verb{retiro}{}{}{}{}{s.m.}{Lugar deserto e solitário.}{re.ti.ro}{0}
\verb{retiro}{}{}{}{}{}{Afastamento temporário do convívio social para meditação e refazimento.}{re.ti.ro}{0}
\verb{reto}{é}{}{}{}{adj.}{Que vai sempre na mesma direção, sem curvas; direito.}{re.to}{0}
\verb{reto}{é}{}{}{}{}{Íntegro, honesto, justo.}{re.to}{0}
\verb{reto}{é}{Geom.}{}{}{}{Diz"-se do ângulo formado por linhas perpendiculares entre si.}{re.to}{0}
\verb{reto}{é}{Gram.}{}{}{}{Diz"-se do pronome pessoal usado como sujeito da oração.}{re.to}{0}
\verb{reto}{é}{Anat.}{}{}{s.m.}{Parte final do intestino grosso.}{re.to}{0}
\verb{retocar}{}{}{}{}{v.t.}{Corrigir falhas ou defeitos de uma obra já feita; aperfeiçoar.}{re.to.car}{\verboinum{2}}
\verb{retomar}{}{}{}{}{v.t.}{Tomar novamente.}{re.to.mar}{0}
\verb{retomar}{}{}{}{}{}{Reiniciar, recomeçar. (\textit{A prefeitura retomou as obras do corredor de ônibus.})}{re.to.mar}{0}
\verb{retomar}{}{}{}{}{}{Readquirir, recuperar, reaver.}{re.to.mar}{\verboinum{1}}
\verb{retoque}{ó}{}{}{}{s.m.}{Correção feita em obra já terminada para reparar pequenos defeitos; aperfeiçoamento.}{re.to.que}{0}
\verb{retorcer}{ê}{}{}{}{v.t.}{Tornar a torcer ou torcer repetidas vezes.}{re.tor.cer}{0}
\verb{retorcer}{ê}{}{}{}{v.pron.}{Contorcer"-se.}{re.tor.cer}{\verboinum{15}}
\verb{retorcido}{}{}{}{}{adj.}{Torcido novamente ou muitas vezes.}{re.tor.ci.do}{0}
\verb{retorcido}{}{}{}{}{}{Rebuscado, exagerado, arrevesado.}{re.tor.ci.do}{0}
\verb{retórica}{}{}{}{}{s.f.}{Arte de bem falar; eloquência, oratória.}{re.tó.ri.ca}{0}
\verb{retórica}{}{}{}{}{}{Emprego de procedimentos enfáticos e exagerados na linguagem na tentativa de persuadir ou por exibição.}{re.tó.ri.ca}{0}
\verb{retórico}{}{}{}{}{adj.}{Relativo à retórica.}{re.tó.ri.co}{0}
\verb{retórico}{}{}{}{}{}{De estilo pomposo, rebuscado, exagerado.}{re.tó.ri.co}{0}
\verb{retórico}{}{}{}{}{s.m.}{Mestre em retórica.}{re.tó.ri.co}{0}
\verb{retornar}{}{}{}{}{v.t.}{Voltar ao ponto de partida; regressar.}{re.tor.nar}{0}
\verb{retornar}{}{}{}{}{}{Fazer voltar; restituir, devolver.}{re.tor.nar}{\verboinum{1}}
\verb{retorno}{ô}{}{}{}{s.m.}{Ato ou efeito de retornar.}{re.tor.no}{0}
\verb{retorno}{ô}{}{}{}{}{Volta, regresso.}{re.tor.no}{0}
\verb{retorno}{ô}{}{}{}{}{Devolução, restituição.}{re.tor.no}{0}
\verb{retorquir}{}{}{}{}{v.t.}{Responder retrucando; replicar, contestar.}{re.tor.quir}{\verboinum{34}\verboirregular{\emph{def.} retorquimos, retorquis}}
\verb{retorta}{ó}{}{}{}{s.f.}{Recipiente de vidro, de gargalo voltado para baixo, usado nos procedimentos químicos em laboratório.}{re.tor.ta}{0}
\verb{retração}{}{}{"-ões}{}{s.f.}{Ato ou efeito de retrair; retraimento, contração.}{re.tra.ção}{0}
\verb{retráctil}{}{}{}{}{}{Var. de \textit{retrátil}.}{re.trác.til}{0}
\verb{retraído}{}{}{}{}{adj.}{Que se retraiu; encolhido, contraído.}{re.tra.í.do}{0}
\verb{retraído}{}{}{}{}{}{Tímido, acanhado, reservado.}{re.tra.í.do}{0}
\verb{retraimento}{}{}{}{}{s.m.}{Ato ou efeito de retrair; encolhimento, contração.}{re.tra.i.men.to}{0}
\verb{retraimento}{}{}{}{}{}{Acanhamento, timidez.}{re.tra.i.men.to}{0}
\verb{retrair}{}{}{}{}{v.t.}{Puxar para dentro; encolher, contrair.}{re.tra.ir}{0}
\verb{retrair}{}{}{}{}{v.pron.}{Voltar"-se para dentro de si mesmo; acanhar"-se, isolar"-se.}{re.tra.ir}{\verboinum{19}}
\verb{retranca}{}{}{}{}{s.f.}{Correia que passa sob a cauda das bestas.}{re.tran.ca}{0}
\verb{retranca}{}{}{}{}{}{Limitação de gastos; arrocho, economia.}{re.tran.ca}{0}
\verb{retranca}{}{Esport.}{}{}{}{No futebol, jogo defensivo.}{re.tran.ca}{0}
\verb{retransmissor}{ô}{}{}{}{adj.}{Diz"-se de aparelho que retransmite os sinais recebidos.}{re.trans.mis.sor}{0}
\verb{retransmissora}{ô}{}{}{}{s.f.}{Estação que capta e retransmite ondas radioelétricas.}{re.trans.mis.so.ra}{0}
\verb{retransmitir}{}{}{}{}{v.t.}{Transmitir novamente.}{re.trans.mi.tir}{\verboinum{18}}
\verb{retrasado}{}{}{}{}{adj.}{Imediatamente anterior ao passado; penúltimo. (\textit{No ano retrasado, estivemos viajando por toda a Europa.})}{re.tra.sa.do}{0}
\verb{retratação}{}{}{"-ões}{}{s.f.}{Ato ou efeito de retratar"-se; confissão de erro.}{re.tra.ta.ção}{0}
\verb{retratador}{ô}{}{}{}{adj.}{Diz"-se daquele que se retrata, que confessa seus erros.}{re.tra.ta.dor}{0}
\verb{retratador}{ô}{}{}{}{adj.}{Diz"-se daquele que faz retratos; retratista.}{re.tra.ta.dor}{0}
\verb{retratar"-se}{}{}{}{}{v.pron.}{Retirar o que dissera; confessar seus erros; retificar"-se.}{re.tra.tar"-se}{\verboinum{1}}
\verb{retratar}{}{}{}{}{v.t.}{Fazer retratos, pintando ou fotografando.   }{re.tra.tar}{\verboinum{1}}
\verb{retrátil}{}{}{"-eis}{}{adj.2g.}{Que é capaz de se retrair, encolher.}{re.trá.til}{0}
\verb{retratista}{}{}{}{}{s.2g.}{Indivíduo que faz retratos; pintor, fotógrafo.}{re.tra.tis.ta}{0}
\verb{retrato}{}{}{}{}{s.m.}{Representação fiel de uma pessoa por meio de pintura ou fotografia.}{re.tra.to}{0}
\verb{retrato}{}{Fig.}{}{}{}{Pessoa muito semelhante a outra. (\textit{Essa menina é o retrato da avó.})}{re.tra.to}{0}
\verb{retreta}{ê}{}{}{}{s.f.}{Apresentação de banda musical em praça pública.}{re.tre.ta}{0}
\verb{retrete}{ê}{}{}{}{s.m.}{Vaso sanitário; latrina, privada.}{re.tre.te}{0}
\verb{retribuição}{}{}{"-ões}{}{s.f.}{Ato ou efeito de retribuir; reconhecimento, compensação.}{re.tri.bu.i.ção}{0}
\verb{retribuir}{}{}{}{}{v.t.}{Dar algo em compensação; gratificar, recompensar, corresponder a.}{re.tri.bu.ir}{\verboinum{26}}
\verb{retriz}{}{}{}{}{s.f.}{Cada uma das penas da cauda das aves que servem para orientar o voo.}{re.triz}{0}
\verb{retroagir}{}{}{}{}{v.i.}{Ter efeito sobre o passado.}{re.tro.a.gir}{\verboinum{22}}
\verb{retroalimentação}{}{}{"-ões}{}{s.f.}{Ato de retornar automaticamente a informação processada ao ponto inicial; realimentação.}{re.tro.a.li.men.ta.ção}{0}
\verb{retroatividade}{}{}{}{}{s.f.}{Qualidade ou caráter do que é retroativo.}{re.tro.a.ti.vi.da.de}{0}
\verb{retroativo}{}{}{}{}{adj.}{Relativo ou pertencente ao passado.}{re.tro.a.ti.vo}{0}
\verb{retroativo}{}{}{}{}{}{Que retroage; que tem efeito sobre fatos passados.}{re.tro.a.ti.vo}{0}
\verb{retroceder}{ê}{}{}{}{v.i.}{Voltar para trás; recuar.}{re.tro.ce.der}{0}
\verb{retroceder}{ê}{}{}{}{}{Decair, declinar, regredir.}{re.tro.ce.der}{\verboinum{12}}
\verb{retrocessivo}{}{}{}{}{adj.}{Que produz retrocessão.}{re.tro.ces.si.vo}{0}
\verb{retrocesso}{é}{}{}{}{s.m.}{Ato ou efeito de retroceder; volta ao passado; recuo, regressão.}{re.tro.ces.so}{0}
\verb{retrogradar}{}{}{}{}{v.i.}{Voltar para trás; retroceder.}{re.tro.gra.dar}{\verboinum{1}}
\verb{retrógrado}{}{}{}{}{adj.}{Que retrograda; que volta para trás, recua.}{re.tró.gra.do}{0}
\verb{retrógrado}{}{}{}{}{}{Que é contrário a mudanças; reacionário, antiquado.}{re.tró.gra.do}{0}
\verb{retroprojetor}{ô}{}{}{}{s.m.}{Aparelho óptico que permite projetar sobre uma tela ou parede, de modo ampliado, textos, gráficos, mapas, gravuras etc. impressos em transparência .}{re.tro.pro.je.tor}{0}
\verb{retrós}{}{}{}{}{s.m.pl.}{Fios de algodão ou seda retorcidos.}{re.trós}{0}
\verb{retrospeção}{}{}{}{}{}{Var. de \textit{retrospecção}.}{re.tros.pe.ção}{0}
\verb{retrospecção}{}{}{"-ões}{}{s.f.}{Observação e análise de fatos acontecidos no passado; retrospecto.}{re.tros.pec.ção}{0}
\verb{retrospectiva}{}{}{}{}{s.f.}{Exposição com perspectiva histórica em homenagem às obras de um artista.}{re.tros.pec.ti.va}{0}
\verb{retrospectiva}{}{}{}{}{}{Retrospecção, retrospecto.}{re.tros.pec.ti.va}{0}
\verb{retrospectivo}{}{}{}{}{adj.}{Que se volta para o passado.}{re.tros.pec.ti.vo}{0}
\verb{retrospecto}{é}{}{}{}{s.m.}{Observação e análise de fatos passados; retrospecção.}{re.tros.pec.to}{0}
\verb{retrospetivo}{}{}{}{}{}{Var. de \textit{retrospectivo}.}{re.tros.pe.ti.vo}{0}
\verb{retrospeto}{é}{}{}{}{}{Var. de \textit{retrospecto}.}{re.tros.pe.to}{0}
\verb{retrotrair}{}{}{}{}{v.t.}{Fazer deslocar"-se para trás; fazer retroceder, recuar.}{re.tro.tra.ir}{\verboinum{19}}
\verb{retrovisor}{ô}{}{}{}{s.m.}{Pequeno espelho colocado nos veículos de tal forma que permita ao motorista ter visão do que se passa atrás.}{re.tro.vi.sor}{0}
\verb{retrucar}{}{}{}{}{v.t.}{Responder imediatamente; objetar, contestar, revidar.}{re.tru.car}{\verboinum{2}}
\verb{retumbante}{}{}{}{}{adj.2g.}{Que retumba, estrondeia; ressonante.}{re.tum.ban.te}{0}
\verb{retumbar}{}{}{}{}{v.i.}{Soar com estrondo; ressoar, ecoar, ribombar.}{re.tum.bar}{\verboinum{1}}
\verb{returno}{}{Esport.}{}{}{s.m.}{Nos campeonatos esportivos, segundo período ou turno em que se disputam as mesmas provas entre os mesmos concorrentes. }{re.tur.no}{0}
\verb{réu}{}{Jur.}{}{ré}{s.m.}{Indivíduo acusado de ato criminoso e que está sendo julgado.}{réu}{0}
\verb{reumático}{}{}{}{}{adj.}{Relativo ao reumatismo.}{reu.má.ti.co}{0}
\verb{reumático}{}{}{}{}{}{Diz"-se do indivíduo que sofre de reumatismo.}{reu.má.ti.co}{0}
\verb{reumatismo}{}{Med.}{}{}{s.m.}{Nome dado a várias afecções caracterizadas por inflamações nos tendões, nos músculos e nas articulações.}{reu.ma.tis.mo}{0}
\verb{reumatologia}{}{Med.}{}{}{s.f.}{Estudo e tratamento de doenças reumáticas.}{reu.ma.to.lo.gi.a}{0}
\verb{reunião}{}{}{"-ões}{}{s.f.}{Ato ou efeito de reunir; agrupamento.}{re.u.ni.ão}{0}
\verb{reunião}{}{}{"-ões}{}{}{Encontro de pessoas para tratar de determinado assunto.}{re.u.ni.ão}{0}
\verb{reunificar}{}{}{}{}{v.t.}{Tornar a unificar.}{re.u.ni.fi.car}{\verboinum{2}}
\verb{reunir}{}{}{}{}{v.t.}{Colocar pessoas, animais ou coisas no mesmo lugar; agrupar, juntar.}{re.u.nir}{0}
\verb{reunir}{}{}{}{}{v.pron.}{Ir encontrar alguém; juntar"-se.}{re.u.nir}{\verboinum{18}}
\verb{reurbanizar}{}{}{}{}{v.t.}{Urbanizar novamente, reorganizando o espaço.}{re.ur.ba.ni.zar}{\verboinum{1}}
\verb{reutilizar}{}{}{}{}{v.t.}{Tornar a utilizar. (\textit{Não se deve reutilizar frascos de remédio.})}{re.u.ti.li.zar}{0}
\verb{reutilizar}{}{}{}{}{}{Dar novo uso; reciclar. (\textit{O escritório deverá ser reutilizado pelo advogado.})}{re.u.ti.li.zar}{\verboinum{1}}
\verb{revalidar}{}{}{}{}{v.t.}{Dar nova validade; reconfirmar.}{re.va.li.dar}{\verboinum{1}}
\verb{revalorizar}{}{}{}{}{v.t.}{Tornar a valorizar.}{re.va.lo.ri.zar}{\verboinum{1}}
\verb{revanche}{}{Esport.}{}{}{s.f.}{Nas disputas esportivas, oportunidade que o perdedor tem de enfrentar novamente quem o venceu.}{re.van.che}{0}
\verb{revanche}{}{}{}{}{}{Desforra, vingança.}{re.van.che}{0}
\verb{revanchismo}{}{}{}{}{s.m.}{Tendência para a revanche ou desforra, especialmente nos meios políticos.}{re.van.chis.mo}{0}
\verb{réveillon}{}{}{}{}{s.m.}{Festa de Ano"-Novo, no dia 31 de dezembro.}{\textit{réveillon}}{0}
\verb{revel}{é}{}{"-éis}{}{adj.2g.}{Que se rebela; insubordinado, rebelde.}{re.vel}{0}
\verb{revel}{é}{Jur.}{"-éis}{}{}{Diz"-se do réu que não se apresenta em juízo.}{re.vel}{0}
\verb{revelação}{}{}{"-ões}{}{s.f.}{Ato ou efeito de revelar.}{re.ve.la.ção}{0}
\verb{revelação}{}{}{"-ões}{}{}{Declaração ou divulgação de um segredo.}{re.ve.la.ção}{0}
\verb{revelação}{}{}{"-ões}{}{}{Operação feita para transformar uma imagem fotográfica latente em imagem visível e estável. }{re.ve.la.ção}{0}
\verb{revelação}{}{}{"-ões}{}{}{Inspiração, conhecimento súbito.}{re.ve.la.ção}{0}
\verb{revelação}{}{}{"-ões}{}{}{Pessoa que se destaca por algum atributo, qualidade ou vocação. }{re.ve.la.ção}{0}
\verb{revelar}{}{}{}{}{v.t.}{Dar a conhecer; manifestar, divulgar.}{re.ve.lar}{0}
\verb{revelar}{}{}{}{}{}{Declarar, denunciar, delatar.}{re.ve.lar}{0}
\verb{revelar}{}{}{}{}{}{Fazer aparecer a imagem em chapa fotográfica.}{re.ve.lar}{\verboinum{1}}
\verb{revelia}{}{Jur.}{}{}{s.f.}{Falta de comparecimento do réu em juízo.}{re.ve.li.a}{0}
\verb{revenda}{}{}{}{}{s.f.}{Ato ou efeito de revender.}{re.ven.da}{0}
\verb{revender}{ê}{}{}{}{v.t.}{Vender o que se tinha comprado.}{re.ven.der}{\verboinum{12}}
\verb{rever}{ê}{}{}{}{v.t.}{Tornar a ver.}{re.ver}{0}
\verb{rever}{ê}{}{}{}{}{Reanalisar, reexaminar, revisar.}{re.ver}{\verboinum{46}}
\verb{reverberação}{}{}{"-ões}{}{s.f.}{Ato ou efeito de reverberar; reflexão.}{re.ver.be.ra.ção}{0}
\verb{reverberar}{}{}{}{}{v.t.}{Refletir luz, calor.}{re.ver.be.rar}{0}
\verb{reverberar}{}{}{}{}{}{Repercutir ondas sonoras.}{re.ver.be.rar}{\verboinum{1}}
\verb{revérbero}{}{}{}{}{s.m.}{Ato ou efeito de reverberar; reflexão, brilho, esplendor.}{re.vér.be.ro}{0}
\verb{reverdecer}{ê}{}{}{}{v.t.}{Tornar verde; cobrir de verdura.}{re.ver.de.cer}{0}
\verb{reverdecer}{ê}{}{}{}{v.i.}{Rejuvenescer, revitalizar"-se.}{re.ver.de.cer}{\verboinum{15}}
\verb{reverência}{}{}{}{}{s.f.}{Respeito profundo; consideração, deferência.}{re.ve.rên.cia}{0}
\verb{reverência}{}{}{}{}{}{Tratamento que se dispensa aos padres.}{re.ve.rên.cia}{0}
\verb{reverenciar}{}{}{}{}{v.t.}{Tratar com reverência; respeitar, venerar.}{re.ve.ren.ci.ar}{\verboinum{6}}
\verb{reverendíssima}{}{}{}{}{s.f.}{Tratamento dispensado aos eclesiásticos.}{re.ve.ren.dís.si.ma}{0}
\verb{reverendíssimo}{}{}{}{}{adj.}{Superlativo absoluto sintético de \textit{reverendo}.}{re.ve.ren.dís.si.mo}{0}
\verb{reverendíssimo}{}{}{}{}{}{Título que se dá aos eclesiásticos.}{re.ve.ren.dís.si.mo}{0}
\verb{reverendo}{}{}{}{}{adj.}{Que merece ser reverenciado; venerável.}{re.ve.ren.do}{0}
\verb{reverendo}{}{}{}{}{s.m.}{Padre, pastor.}{re.ve.ren.do}{0}
\verb{reverente}{}{}{}{}{adj.2g.}{Que reverencia; que manifesta reverência, respeito.}{re.ve.ren.te}{0}
\verb{reversão}{}{}{"-ões}{}{s.f.}{Ato ou efeito de reverter.}{re.ver.são}{0}
\verb{reversível}{}{}{"-eis}{}{adj.2g.}{Que se pode reverter.}{re.ver.sí.vel}{0}
\verb{reversivo}{}{}{}{}{adj.}{Sujeito a reversão.}{re.ver.si.vo}{0}
\verb{reverso}{é}{}{}{}{s.m.}{O lado oposto ao principal; verso.}{re.ver.so}{0}
\verb{reverso}{é}{}{}{}{adj.}{Que está em posição, sentido ou lado oposto ao principal.}{re.ver.so}{0}
\verb{reverter}{ê}{}{}{}{v.t.}{Tornar ao ponto de partida.}{re.ver.ter}{0}
\verb{reverter}{ê}{}{}{}{}{Passar novamente à posse (de alguém).}{re.ver.ter}{0}
\verb{reverter}{ê}{}{}{}{}{Transformar"-se, converter"-se, redundar.}{re.ver.ter}{\verboinum{12}}
\verb{revertério}{}{}{}{}{s.m.}{Reviravolta em uma situação, que passa de favorável para desfavorável.}{re.ver.té.rio}{0}
\verb{revés}{}{}{}{}{s.m.}{O lado oposto ao normal; o avesso.}{re.vés}{0}
\verb{revés}{}{}{}{}{}{Infortúnio, fatalidade.}{re.vés}{0}
\verb{revesso}{é}{}{}{}{adj.}{Voltado ao contrário; revertido, torto.}{re.ves.so}{0}
\verb{revestimento}{}{}{}{}{s.m.}{Ato ou efeito de revestir.}{re.ves.ti.men.to}{0}
\verb{revestimento}{}{}{}{}{}{O material que reveste; forro, cobertura.}{re.ves.ti.men.to}{0}
\verb{revestir}{}{}{}{}{v.t.}{Vestir novamente.}{re.ves.tir}{0}
\verb{revestir}{}{}{}{}{}{Estender por sobre; cobrir.}{re.ves.tir}{0}
\verb{revestir}{}{}{}{}{}{Encher"-se de; munir"-se, armar"-se.}{re.ves.tir}{\verboinum{18}}
\verb{revezamento}{}{}{}{}{s.m.}{Ato ou efeito de revezar.}{re.ve.za.men.to}{0}
\verb{revezar}{}{}{}{}{v.t.}{Substituir alternadamente.}{re.ve.zar}{\verboinum{1}}
\verb{revidar}{}{}{}{}{v.t.}{Responder uma ofensa ou agressão com outra maior; vingar.}{re.vi.dar}{\verboinum{1}}
\verb{revide}{}{}{}{}{s.m.}{Ato ou efeito de revidar.}{re.vi.de}{0}
\verb{revigorar}{}{}{}{}{v.t.}{Dar novo vigor; robustecer, fortalecer.}{re.vi.go.rar}{\verboinum{1}}
\verb{revindita}{}{}{}{}{s.f.}{Vingança em resposta a outra vingança.}{re.vin.di.ta}{0}
\verb{revirar}{}{}{}{}{v.t.}{Virar novamente.}{re.vi.rar}{0}
\verb{revirar}{}{}{}{}{}{Virar pelo avesso.}{re.vi.rar}{0}
\verb{revirar}{}{}{}{}{}{Remexer, vasculhar, revolver.}{re.vi.rar}{0}
\verb{reviravolta}{}{}{}{}{s.f.}{Ato ou efeito de desfazer uma volta.}{re.vi.ra.vol.ta}{0}
\verb{reviravolta}{}{}{}{}{}{Volta sobre si mesmo; giro.}{re.vi.ra.vol.ta}{0}
\verb{reviravolta}{}{}{}{}{}{Mudança repentina e abrangente.}{re.vi.ra.vol.ta}{0}
\verb{revisão}{}{}{"-ões}{}{s.f.}{Ato ou efeito de revisar.}{re.vi.são}{0}
\verb{revisão}{}{}{"-ões}{}{}{Exame minucioso de um texto em busca de coisas que possam necessitar de alteração.}{re.vi.são}{0}
\verb{revisar}{}{}{}{}{v.t.}{Visar novamente.}{re.vi.sar}{0}
\verb{revisar}{}{}{}{}{}{Fazer leitura minuciosa de um texto.}{re.vi.sar}{\verboinum{1}}
\verb{revisionismo}{}{}{}{}{s.m.}{Tendência para se propor a revisão de uma doutrina ou da constituição de um país.}{re.vi.si.o.nis.mo}{0}
\verb{revisionista}{}{}{}{}{adj.2g.}{Adepto do revisionismo.}{re.vi.si.o.nis.ta}{0}
\verb{revisor}{ô}{}{}{}{s.m.}{Indivíduo que revisa textos ou provas tipográficas.}{re.vi.sor}{0}
\verb{revista}{}{}{}{}{s.f.}{Ato ou efeito de revistar.}{re.vis.ta}{0}
\verb{revista}{}{}{}{}{s.f.}{Publicação periódica sobre assunto específico.}{re.vis.ta}{0}
\verb{revistar}{}{}{}{}{v.t.}{Examinar de maneira minuciosa, geralmente em busca de alguma coisa.}{re.vis.tar}{\verboinum{1}}
\verb{revisto}{}{}{}{}{adj.}{Que foi submetido a revisão.}{re.vis.to}{0}
\verb{revitalizar}{}{}{}{}{v.t.}{Tornar a vitalizar; revigorar.}{re.vi.ta.li.zar}{\verboinum{1}}
\verb{reviver}{ê}{}{}{}{v.t.}{Tornar a viver; renascer.}{re.vi.ver}{0}
\verb{reviver}{ê}{}{}{}{}{Ganhar novo vigor; revigorar"-se.}{re.vi.ver}{0}
\verb{reviver}{ê}{}{}{}{}{Trazer (fato ou época) à memória; recordar.}{re.vi.ver}{\verboinum{12}}
\verb{revivescer}{ê}{}{}{}{v.t.}{Reviver.}{re.vi.ves.cer}{\verboinum{15}}
\verb{revivificar}{}{}{}{}{v.t.}{Vivificar novamente; fazer reviver.}{re.vi.vi.fi.car}{\verboinum{2}}
\verb{revoada}{}{}{}{}{s.f.}{Ato de revoar.}{re.vo.a.da}{0}
\verb{revoada}{}{}{}{}{}{Bando de aves voando em conjunto.}{re.vo.a.da}{0}
\verb{revoar}{}{}{}{}{v.i.}{Tornar a voar.}{re.vo.ar}{0}
\verb{revoar}{}{}{}{}{}{Esvoaçar, pairar, voejar.}{re.vo.ar}{\verboinum{7}}
\verb{revocar}{}{}{}{}{v.t.}{Chamar de volta.}{re.vo.car}{0}
\verb{revocar}{}{}{}{}{}{Tornar a chamar.}{re.vo.car}{\verboinum{2}}
\verb{revogação}{}{}{"-ões}{}{s.f.}{Ato ou efeito de revogar; anulação.}{re.vo.ga.ção}{0}
\verb{revogar}{}{}{}{}{v.t.}{Tornar sem efeito; anular, cancelar.}{re.vo.gar}{\verboinum{5}}
\verb{revogável}{}{}{"-eis}{}{adj.2g.}{Que se pode revogar, anular.}{re.vo.gá.vel}{0}
\verb{revolta}{ó}{}{}{}{s.f.}{Ato ou efeito de revoltar.}{re.vol.ta}{0}
\verb{revolta}{ó}{}{}{}{}{Rebelião, motim, levante.}{re.vol.ta}{0}
\verb{revolta}{ó}{}{}{}{}{Indignação, repulsa.}{re.vol.ta}{0}
\verb{revoltado}{}{}{}{}{adj.}{Revoltoso, rebelde, indignado, enfurecido.}{re.vol.ta.do}{0}
\verb{revoltante}{}{}{}{}{adj.2g.}{Que causa revolta, indignação.}{re.vol.tan.te}{0}
\verb{revoltar}{}{}{}{}{v.t.}{Provocar indisposição ou insurreição contra um superior ou autoridade.}{re.vol.tar}{0}
\verb{revoltar}{}{}{}{}{}{Causar indignação, repulsa.}{re.vol.tar}{0}
\verb{revoltar}{}{}{}{}{v.pron.}{Insurgir"-se, amotinar"-se.}{re.vol.tar}{0}
\verb{revoltear}{}{}{}{}{v.t.}{Voltear muito ou continuamente.}{re.vol.te.ar}{\verboinum{4}}
\verb{revolto}{ô}{}{}{}{adj.}{Muito revolvido, remexido.}{re.vol.to}{0}
\verb{revolto}{ô}{}{}{}{}{Muito agitado (o mar).}{re.vol.to}{0}
\verb{revoltoso}{ô}{}{"-osos ⟨ó⟩}{"-osa ⟨ó⟩}{adj.}{Que se revoltou; rebelde, insurrecto.}{re.vol.to.so}{0}
\verb{revolução}{}{}{"-ões}{}{s.f.}{Ato ou efeito de revolver.}{re.vo.lu.ção}{0}
\verb{revolução}{}{Astron.}{"-ões}{}{}{Cada movimento completo de um corpo celeste em sua própria órbita.}{re.vo.lu.ção}{0}
\verb{revolução}{}{}{"-ões}{}{}{Grande transformação dentro de um sistema, especialmente de maneira repentina.}{re.vo.lu.ção}{0}
\verb{revolução}{}{}{"-ões}{}{}{Movimento revoltoso de um grande número de pessoas contra um poder estabelecido.}{re.vo.lu.ção}{0}
\verb{revolucionar}{}{}{}{}{v.t.}{Provocar grandes tranformações.}{re.vo.lu.ci.o.nar}{0}
\verb{revolucionar}{}{}{}{}{}{Fazer a revolução.}{re.vo.lu.ci.o.nar}{0}
\verb{revolucionar}{}{}{}{}{}{Revolver, mexer.}{re.vo.lu.ci.o.nar}{\verboinum{1}}
\verb{revolucionário}{}{}{}{}{adj.}{Relativo a revolução.}{re.vo.lu.ci.o.ná.rio}{0}
\verb{revolucionário}{}{}{}{}{}{Caracterizado por grandes transformações; inovador, ousado, original.}{re.vo.lu.ci.o.ná.rio}{0}
\verb{revolucionário}{}{}{}{}{}{Que participa de uma revolução.}{re.vo.lu.ci.o.ná.rio}{0}
\verb{revolutear}{}{}{}{}{v.i.}{Agitar"-se em várias direções; revolver"-se.}{re.vo.lu.te.ar}{0}
\verb{revolutear}{}{}{}{}{}{Esvoaçar, voejar, adejar.}{re.vo.lu.te.ar}{\verboinum{4}}
\verb{revolver}{ê}{}{}{}{v.t.}{Revirar, retorcer.}{re.vol.ver}{0}
\verb{revolver}{ê}{}{}{}{}{Movimentar, mover, volver, agitar, revirar.}{re.vol.ver}{0}
\verb{revolver}{ê}{}{}{}{}{Examinar minuciosamente.}{re.vol.ver}{\verboinum{12}}
\verb{revólver}{}{}{}{}{s.m.}{Arma de fogo manual em que os cartuchos são acomodados em um tambor.}{re.vól.ver}{0}
\verb{revôo}{}{}{}{}{s.m.}{Ato ou efeito de revoar.}{re.vô.o}{0}
\verb{revulsão}{}{Med.}{"-ões}{}{s.f.}{Irritação causada por medicamento utilizado para tratar uma inflamação em outra parte do corpo.}{re.vul.são}{0}
\verb{revulsivo}{}{}{}{}{adj.}{Relativo a revulsão.}{re.vul.si.vo}{0}
\verb{revulsivo}{}{}{}{}{}{Que provoca revulsão.}{re.vul.si.vo}{0}
\verb{reza}{é}{}{}{}{s.f.}{Ato ou efeito de rezar; oração, prece.}{re.za}{0}
\verb{reza}{é}{}{}{}{}{Súplica feita a uma divindade.}{re.za}{0}
\verb{reza}{é}{Bras.}{}{}{}{Palavras que são proferidas com a finalidade de afastar um mal ou obter algo.}{re.za}{0}
\verb{rezador}{ô}{}{}{}{adj.}{Que reza bastante; devoto.}{re.za.dor}{0}
\verb{rezador}{ô}{Bras.}{}{}{s.m.}{Indivíduo que profere palavras para afastar um mal ou obter algo; benzedor, curandeiro.}{re.za.dor}{0}
\verb{rezar}{}{}{}{}{v.t.}{Fazer preces; orar.}{re.zar}{0}
\verb{rezar}{}{}{}{}{}{Narrar, contar.}{re.zar}{\verboinum{1}}
\verb{rezingar}{}{}{}{}{v.t.}{Falar baixo e com mau humor; resmungar.}{re.zin.gar}{\verboinum{5}}
\verb{Rh}{}{Quím.}{}{}{}{Símb. do \textit{ródio}.}{Rh}{0}
\verb{riacho}{}{}{}{}{s.m.}{Rio pequeno.}{ri.a.cho}{0}
\verb{riba}{}{}{}{}{s.f.}{Margem alta de rio; ribanceira.}{ri.ba}{0}
\verb{riba}{}{Pop.}{}{}{}{A parte mais elevada de algo; cima.}{ri.ba}{0}
\verb{ribalta}{}{}{}{}{s.f.}{Conjunto de lâmpadas situadas no proscênio para iluminar os primeiros planos do palco.}{ri.bal.ta}{0}
\verb{ribalta}{}{}{}{}{}{O proscênio.}{ri.bal.ta}{0}
\verb{ribalta}{}{Por ext.}{}{}{}{O palco, a cena, o teatro.}{ri.bal.ta}{0}
\verb{ribanceira}{ê}{}{}{}{s.f.}{Rochedo alto à margem de um rio.}{ri.ban.cei.ra}{0}
\verb{ribeira}{ê}{}{}{}{s.f.}{Local ou região banhada por rio.}{ri.bei.ra}{0}
\verb{ribeira}{ê}{}{}{}{}{Curso d'água relativamente grande, mas menor que um rio.}{ri.bei.ra}{0}
\verb{ribeirão}{}{Bras.}{"-ões}{}{s.m.}{Curso d'água pouco menor que um rio.}{ri.bei.rão}{0}
\verb{ribeirão}{}{Bras.}{"-ões}{}{}{Terreno apropriado para mineração.}{ri.bei.rão}{0}
\verb{ribeirinho}{}{}{}{}{adj.}{Que se encontra ou vive próximo a cursos d'água.}{ri.bei.ri.nho}{0}
\verb{ribeirinho}{}{}{}{}{s.m.}{Indivíduo que leva recados.}{ri.bei.ri.nho}{0}
\verb{ribeirinho}{}{}{}{}{}{Indivíduo que tranporta areia e entulho em animais de carga.}{ri.bei.ri.nho}{0}
\verb{ribeiro}{ê}{}{}{}{s.m.}{Curso d'água estreito; regato.}{ri.bei.ro}{0}
\verb{ribombar}{}{}{}{}{v.i.}{Fazer grande ruído; retumbar, estrondar.}{ri.bom.bar}{\verboinum{1}}
\verb{ribombo}{}{}{}{}{s.m.}{Ato ou efeito de ribombar; estrondo.}{ri.bom.bo}{0}
\verb{ricaço}{}{Pop.}{}{}{adj.}{Diz"-se de indivíduo muito rico; milionário.}{ri.ca.ço}{0}
\verb{rícino}{}{Bot.}{}{}{s.m.}{Arbusto que dá cápsulas verdes das quais se extrai óleo; mamona.}{rí.ci.no}{0}
\verb{rico}{}{}{}{}{adj.}{Que possui muitos bens e riquezas.}{ri.co}{0}
\verb{rico}{}{}{}{}{}{Que produz fartamente; fértil, abundante.}{ri.co}{0}
\verb{rico}{}{}{}{}{}{Opulento, farto, pomposo.}{ri.co}{0}
\verb{rico}{}{}{}{}{}{Apetitoso, bom.}{ri.co}{0}
\verb{rico}{}{}{}{}{}{Belo, lindo, bonito.}{ri.co}{0}
\verb{ricochetar}{}{}{}{}{v.i.}{Ricochetear.}{ri.co.che.tar}{\verboinum{1}}
\verb{ricochete}{ê}{}{}{}{s.m.}{Salto dado por um projétil ou um corpo ao chocar"-se com o chão ou com qualquer superfície firme e rígida.}{ri.co.che.te}{0}
\verb{ricochete}{ê}{Fig.}{}{}{}{Retrocesso, volta.}{ri.co.che.te}{0}
\verb{ricochetear}{}{}{}{}{v.i.}{Fazer ricochete.}{ri.co.che.te.ar}{0}
\verb{ricochetear}{}{Fig.}{}{}{v.t.}{Atingir indiretamente.}{ri.co.che.te.ar}{\verboinum{4}}
\verb{ricota}{ó}{}{}{}{s.f.}{Queijo macio, branco e de sabor bastante suave, preparado com o soro do leite fervido e coalhado.}{ri.co.ta}{0}
\verb{ricto}{}{}{}{}{s.m.}{Abertura da boca.}{ric.to}{0}
\verb{ricto}{}{}{}{}{}{Contração dos músculos faciais que produz expressão forçada.}{ric.to}{0}
\verb{ríctus}{}{}{}{}{s.m.}{Ricto.}{ríc.tus}{0}
\verb{ridicularia}{}{}{}{}{s.f.}{Ato ou dito ridículo.}{ri.di.cu.la.ri.a}{0}
\verb{ridicularia}{}{}{}{}{}{Objeto ou atitude sem nenhuma importância; insignificância.}{ri.di.cu.la.ri.a}{0}
\verb{ridicularizar}{}{}{}{}{v.t.}{Zombar, escarnecer.}{ri.di.cu.la.ri.zar}{\verboinum{1}}
\verb{ridiculizar}{}{}{}{}{v.t.}{Ridicularizar.}{ri.di.cu.li.zar}{\verboinum{1}}
\verb{ridículo}{}{}{}{}{adj.}{Que provoca riso.}{ri.dí.cu.lo}{0}
\verb{ridículo}{}{}{}{}{}{Irrisório, insignificante.}{ri.dí.cu.lo}{0}
\verb{ridículo}{}{Bras.}{}{}{}{Muito apegado ao dinheiro; avarento.}{ri.dí.cu.lo}{0}
\verb{rifa}{}{}{}{}{s.f.}{Sorteio de objeto ou prêmio, geralmente de algum valor, por meio da venda de bilhetes numerados.}{ri.fa}{0}
\verb{rifão}{}{}{"-ões}{}{s.m.}{Provérbio.}{ri.fão}{0}
\verb{rifar}{}{}{}{}{v.t.}{Sortear por meio de rifa.}{ri.far}{\verboinum{1}}
\verb{rifle}{}{}{}{}{s.m.}{Arma de cano longo; espingarda.}{ri.fle}{0}
\verb{rigidez}{ê}{}{}{}{s.f.}{Qualidade de rígido.}{ri.gi.dez}{0}
\verb{rigidez}{ê}{}{}{}{}{Rigor, severidade, austeridade.}{ri.gi.dez}{0}
\verb{rigidez}{ê}{Fig.}{}{}{}{Rudeza, aspereza.}{ri.gi.dez}{0}
\verb{rígido}{}{}{}{}{adj.}{Retesado, teso, tenso, hirto.}{rí.gi.do}{0}
\verb{rígido}{}{}{}{}{}{Não flexível; rijo, resistente.}{rí.gi.do}{0}
\verb{rígido}{}{Fig.}{}{}{}{Rigoroso, austero, severo.}{rí.gi.do}{0}
\verb{rigor}{ô}{}{}{}{s.m.}{Rigidez, dureza, resistência.}{ri.gor}{0}
\verb{rigor}{ô}{}{}{}{}{Severidade, inflexibilidade, insensibilidade.}{ri.gor}{0}
\verb{rigor}{ô}{}{}{}{}{Ausência de falha; exatidão, precisão.}{ri.gor}{0}
\verb{rigor}{ô}{}{}{}{}{Vigor, força, vitalidade.}{ri.gor}{0}
\verb{rigoroso}{ô}{}{"-osos ⟨ó⟩}{"-osa ⟨ó⟩}{adj.}{Que tem rigor.}{ri.go.ro.so}{0}
\verb{rijo}{}{}{}{}{adj.}{Rígido, resistente, duro.}{ri.jo}{0}
\verb{rijo}{}{}{}{}{}{Vigoroso, forte, robusto, enérgico.}{ri.jo}{0}
\verb{rijo}{}{Fig.}{}{}{}{Inflexível, rígido, severo.}{ri.jo}{0}
\verb{rijo}{}{Fig.}{}{}{}{Intenso, áspero, forte.}{ri.jo}{0}
\verb{rilhar}{}{}{}{}{v.t.}{Roer.}{ri.lhar}{0}
\verb{rilhar}{}{}{}{}{}{Ranger (os dentes).}{ri.lhar}{\verboinum{1}}
\verb{rim}{}{Anat.}{rins}{}{s.m.}{Cada um dos dois órgãos responsáveis pela filtragem do sangue e produção da urina.}{rim}{0}
\verb{rima}{}{Liter.}{}{}{s.f.}{Ocorrência de sons iguais ou semelhantes no final dos versos ou em partes de um mesmo verso.}{ri.ma}{0}
\verb{rimado}{}{}{}{}{adj.}{Que tem rimas.}{ri.ma.do}{0}
\verb{rimar}{}{}{}{}{v.t.}{Formar rimas.}{ri.mar}{0}
\verb{rimar}{}{}{}{}{}{Escrever versos, rimados ou não.}{ri.mar}{0}
\verb{rimar}{}{Fig.}{}{}{}{Colocar em harmonia.}{ri.mar}{\verboinum{1}}
\verb{rimário}{}{}{}{}{s.m.}{Coleção de rimas.}{ri.má.rio}{0}
\verb{rímel}{}{}{"-eis}{}{s.m.}{Produto cosmético para acentuar os cílios.}{rí.mel}{0}
\verb{rinçagem}{}{}{"-ens}{}{s.f.}{Banho dado nos cabelos, logo após a lavagem, com produto próprio para tratar ou dar brilho.}{rin.ça.gem}{0}
\verb{rincão}{}{}{"-ões}{}{s.m.}{Local afastado.}{rin.cão}{0}
\verb{rincão}{}{}{"-ões}{}{}{Local naturalmente abrigado por rios ou florestas.}{rin.cão}{0}
\verb{rinchar}{}{}{}{}{v.i.}{Dar rinchos; relinchar.}{rin.char}{\verboinum{1}}
\verb{rincho}{}{}{}{}{s.m.}{A voz do cavalo; relincho.}{rin.cho}{0}
\verb{rincho}{}{}{}{}{}{Ruído agudo, cortante, áspero.}{rin.cho}{0}
\verb{ringir}{}{}{}{}{v.t.}{Ranger (os dentes).}{rin.gir}{0}
\verb{ringir}{}{Por ext.}{}{}{}{Produzir ruído agudo, áspero.}{rin.gir}{\verboinum{22}}
\verb{ringue}{}{Esport.}{}{}{s.m.}{Espaço próprio para prática de modalidades esportivas de luta, geralmente delimitado por cordas.}{rin.gue}{0}
\verb{rinha}{}{Bras.}{}{}{s.f.}{Briga de galos.}{ri.nha}{0}
\verb{rinite}{}{Med.}{}{}{s.f.}{Inflamação da mucosa nasal, caracterizada por edema e aumento da secreção de mucosa.}{ri.ni.te}{0}
\verb{rinoceronte}{}{Zool.}{}{}{s.m.}{Mamífero de grande porte com cabeça grande, um ou dois cornos e patas com três dedos, encontrado na África e em regiões da Ásia.}{ri.no.ce.ron.te}{0}
\verb{rinque}{}{Esport.}{}{}{s.m.}{Local próprio para patinação.}{rin.que}{0}
\verb{rio}{}{}{}{}{s.m.}{Curso d'água natural que corre em direção a regiões relativamente mais baixas, desaguando em outro, em lago ou no mar.}{ri.o}{0}
\verb{rio}{}{Fig.}{}{}{}{Grande quantidade de qualquer coisa.}{ri.o}{0}
\verb{rio"-branquense}{}{}{rio"-branquenses}{}{adj.2g.}{Relativo a Rio Branco, capital do Acre.}{ri.o"-bran.quen.se}{0}
\verb{rio"-branquense}{}{}{rio"-branquenses}{}{s.2g.}{Indivíduo natural ou habitante dessa cidade.}{ri.o"-bran.quen.se}{0}
\verb{rio"-grandense"-do"-norte}{ó}{}{rio"-grandenses"-do"-norte ⟨ó⟩}{}{adj.2g.}{Relativo ao Rio Grande do Norte; norte"-rio"-grandense; potiguar.}{ri.o"-gran.den.se"-do"-nor.te}{0}
\verb{rio"-grandense"-do"-norte}{ó}{}{rio"-grandenses"-do"-norte ⟨ó⟩}{}{s.2g.}{Indivíduo natural ou habitante desse estado.}{ri.o"-gran.den.se"-do"-nor.te}{0}
\verb{rio"-grandense"-do"-sul}{}{}{rio"-grandenses"-do"-sul}{}{adj.2g.}{Relativo ao Rio Grande do Sul; sul"-rio"-grandense.}{ri.o"-gran.den.se"-do"-sul}{0}
\verb{rio"-grandense"-do"-sul}{}{}{rio"-grandenses"-do"-sul}{}{s.2g.}{Indivíduo natural ou habitante desse estado.}{ri.o"-gran.den.se"-do"-sul}{0}
\verb{ripa}{}{}{}{}{s.f.}{Pedaço de madeira longo e estreito.}{ri.pa}{0}
\verb{ripa}{}{Bras.}{}{}{}{Aguardente de cana.}{ri.pa}{0}
\verb{ripa}{}{}{}{}{s.f.}{Ato ou efeito de ripar.}{ri.pa}{0}
\verb{ripada}{}{}{}{}{s.f.}{Golpe dado com uma ripa; bordoada.}{ri.pa.da}{0}
\verb{ripada}{}{Pop.}{}{}{}{Censura severa ou violenta.}{ri.pa.da}{0}
\verb{ripar}{}{}{}{}{v.t.}{Pregar ou colocar ripas.}{ri.par}{0}
\verb{ripar}{}{}{}{}{}{Separar a barganha do linho.}{ri.par}{0}
\verb{ripar}{}{}{}{}{}{Raspar (a terra).}{ri.par}{\verboinum{1}}
\verb{ripostar}{}{Esport.}{}{}{v.t.}{Na esgrima, rebater um golpe adversário.}{ri.pos.tar}{0}
\verb{ripostar}{}{Fig.}{}{}{}{Replicar argumentando; retrucar.}{ri.pos.tar}{\verboinum{1}}
\verb{riqueza}{ê}{}{}{}{s.f.}{Qualidade de rico.}{ri.que.za}{0}
\verb{riqueza}{ê}{}{}{}{}{Grande quantidade de dinheiro e bens materiais de valor.}{ri.que.za}{0}
\verb{riqueza}{ê}{Fig.}{}{}{}{Abundância de qualquer coisa.}{ri.que.za}{0}
\verb{riqueza}{ê}{Fig.}{}{}{}{Beleza, suntuosidade, imponência.}{ri.que.za}{0}
\verb{rir}{}{}{}{}{v.i.}{Adquirir uma expressão facial característica e emitir voz em decorrência de uma impressão alegre.}{rir}{\verboinum{57}}
\verb{risada}{}{}{}{}{s.f.}{Ato de rir.}{ri.sa.da}{0}
\verb{risada}{}{}{}{}{}{Série de risos simultâneos de muitas pessoas.}{ri.sa.da}{0}
\verb{risca}{}{}{}{}{s.f.}{Ato ou efeito de riscar.}{ris.ca}{0}
\verb{risca}{}{}{}{}{}{Traço feito com lápis, caneta ou objeto pontiagudo; risco.}{ris.ca}{0}
\verb{riscado}{}{}{}{}{adj.}{Que apresenta riscos.}{ris.ca.do}{0}
\verb{riscado}{}{Pop.}{}{}{}{Cortado, eliminado.}{ris.ca.do}{0}
\verb{riscar}{}{}{}{}{v.t.}{Fazer traços.}{ris.car}{0}
\verb{riscar}{}{}{}{}{}{Fazer desenho ou esboço.}{ris.car}{0}
\verb{riscar}{}{}{}{}{}{Eliminar, suprimir, excluir.}{ris.car}{\verboinum{2}}
\verb{risco}{}{}{}{}{s.m.}{Probabilidade de que algo desfavorável aconteça.}{ris.co}{0}
\verb{risco}{}{}{}{}{s.m.}{Traço feito com lápis, caneta ou objeto pontiagudo; risca. }{ris.co}{0}
\verb{risível}{}{}{"-eis}{}{adj.2g.}{Que provoca riso; cômico, ridículo.}{ri.sí.vel}{0}
\verb{riso}{}{}{}{}{s.m.}{Ato de rir; risada.}{ri.so}{0}
\verb{riso}{}{}{}{}{}{Expressão facial característica que demonstra alegria, satisfação.}{ri.so}{0}
\verb{riso}{}{}{}{}{}{Demonstração de desprezo, escárnio; caçoada, zombaria.}{ri.so}{0}
\verb{risonho}{}{}{}{}{adj.}{Diz"-se de indivíduo que está sempre contente, alegre, satisfeito ou que ri bastante.}{ri.so.nho}{0}
\verb{risonho}{}{}{}{}{}{Que causa satisfação, alegria, prazer.}{ri.so.nho}{0}
\verb{risota}{ó}{}{}{}{s.f.}{Risada curta.}{ri.so.ta}{0}
\verb{risota}{ó}{}{}{}{}{Riso ou expressão de desprezo, escárnio.}{ri.so.ta}{0}
\verb{risoto}{ô}{Cul.}{}{}{s.m.}{Prato preparado com arroz e outras iguarias, como queijo, legumes ou carnes.}{ri.so.to}{0}
\verb{rispidez}{ê}{}{}{}{s.f.}{Qualidade de ríspido.}{ris.pi.dez}{0}
\verb{ríspido}{}{}{}{}{adj.}{Que não é macio; áspero.}{rís.pi.do}{0}
\verb{ríspido}{}{}{}{}{}{Rude, grosseiro, severo, rígido.}{rís.pi.do}{0}
\verb{rissole}{ó}{Cul.}{}{}{s.m.}{Bolinho de massa cozida recheada de carne ou outras iguarias que se passam em ovo e farinha antes de fritar.}{ris.so.le}{0}
\verb{riste}{}{}{}{}{s.m.}{Peça em que os cavaleiros apoiam a lança ao carregá"-la na horizontal para atacar.}{ris.te}{0}
\verb{riste}{}{}{}{}{loc. adv.}{\textit{(em riste)} De maneira erguida.}{ris.te}{0}
\verb{ritmar}{}{}{}{}{v.t.}{Dar ritmo a; cadenciar.}{rit.mar}{\verboinum{1}}
\verb{rítmico}{}{}{}{}{adj.}{Relativo a ritmo.}{rít.mi.co}{0}
\verb{ritmista}{}{Mús.}{}{}{s.2g.}{Percussionista ou especialista em instrumentos de percussão.}{rit.mis.ta}{0}
\verb{ritmista}{}{Mús.}{}{}{}{Indivíduo responsável pela marcação do ritmo nos ensaios e apresentações de uma escola de samba.}{rit.mis.ta}{0}
\verb{ritmo}{}{}{}{}{s.m.}{Movimento regular manifestado pela repetição de um padrão sonoro.}{rit.mo}{0}
\verb{ritmo}{}{Por ext.}{}{}{}{Movimento regular que marca o desenvolvimento de qualquer processo.}{rit.mo}{0}
\verb{ritmo}{}{Art.}{}{}{}{Padrão rítmico que define um gênero de música ou de poesia.}{rit.mo}{0}
\verb{rito}{}{}{}{}{s.m.}{Cerimônia ou conjunto de cerimônias que se praticam regularmente no âmbito de uma religião ou de uma prática social.}{ri.to}{0}
\verb{ritual}{}{}{"-ais}{}{s.m.}{Livro que contém os ritos de uma religião.}{ri.tu.al}{0}
\verb{ritual}{}{Por ext.}{"-ais}{}{}{Culto religioso; cerimônia.}{ri.tu.al}{0}
\verb{ritual}{}{Por ext.}{"-ais}{}{}{Conjunto de regras e práticas socialmente estabelecidas; cerimonial.}{ri.tu.al}{0}
\verb{ritualismo}{}{}{}{}{s.m.}{Conjunto de ritos de uma religião ou sociedade.}{ri.tu.a.lis.mo}{0}
\verb{ritualismo}{}{}{}{}{}{Apego excessivo a cerimônias ou formalidades, sem haver necessariamente conhecimento de seu significado ou relação legítima com ele.}{ri.tu.a.lis.mo}{0}
\verb{ritualista}{}{}{}{}{adj.2g.}{Relativo a ritual ou a ritualismo.}{ri.tu.a.lis.ta}{0}
\verb{rival}{}{}{"-ais}{}{adj.2g.}{Diz"-se daquele com quem se disputa alguma coisa.}{ri.val}{0}
\verb{rivalidade}{}{}{}{}{s.f.}{Qualidade ou condição de rival.}{ri.va.li.da.de}{0}
\verb{rivalidade}{}{}{}{}{}{Situação em que há disputa entre dois ou mais grupos ou pessoas.}{ri.va.li.da.de}{0}
\verb{rivalizar}{}{}{}{}{v.t.}{Disputar, concorrer, competir.}{ri.va.li.zar}{\verboinum{1}}
\verb{rixa}{ch}{}{}{}{s.f.}{Hostilidade, disputa, briga, desacordo.}{ri.xa}{0}
\verb{rixento}{ch}{}{}{}{adj.}{Diz"-se de indivíduo que costuma provocar rixas ou envolver"-se nelas; briguento.}{ri.xen.to}{0}
\verb{rizicultor}{ô}{}{}{}{adj.}{Que cultiva arroz.}{ri.zi.cul.tor}{0}
\verb{rizicultura}{}{}{}{}{s.f.}{Cultura de arroz.}{ri.zi.cul.tu.ra}{0}
\verb{rizófago}{}{}{}{}{adj.}{Que se alimenta de raízes.}{ri.zó.fa.go}{0}
\verb{rizoma}{}{Bot.}{}{}{s.m.}{Caule subterrâneo encontrado em algumas espécies de plantas.}{ri.zo.ma}{0}
\verb{rizotônico}{}{Gram.}{}{}{adj.}{Diz"-se dos vocábulos cujo acento tônico cai em sílaba que pertence ao radical.}{ri.zo.tô.ni.co}{0}
\verb{RJ}{}{}{}{}{}{Sigla do estado do Rio de Janeiro.}{RJ}{0}
\verb{RN}{}{}{}{}{}{Sigla do estado do Rio Grande do Norte.}{RN}{0}
\verb{Rn}{}{Quím.}{}{}{}{Símb. do \textit{radônio}. }{Rn}{0}
\verb{RNA}{}{Bioquím.}{}{}{}{Sigla, em inglês, de \textit{ácido ribonucleico}.}{RNA}{0}
\verb{rô}{}{}{}{}{s.m.}{Décima sétima letra do alfabeto grego.}{rô}{0}
\verb{RO}{}{}{}{}{}{Sigla do estado de Rondônia.}{RO}{0}
\verb{roaz}{}{}{}{}{adj.2g.}{Que rói; roedor.}{ro.az}{0}
\verb{robalo}{}{Zool.}{}{}{s.m.}{Peixe de cor cinzenta e flancos e abdômen brancos, que vive em água doce ou salobra e cuja carne é muito apreciada em culinária.}{ro.ba.lo}{0}
\verb{robe}{ó}{}{}{}{s.m.}{Peça de vestuário larga, comprida e aberta na frente, geralmente usada em casa.}{ro.be}{0}
\verb{roble}{ó}{Bot.}{}{}{s.m.}{Grande árvore cultivada como ornamental e por sua casca e madeira.}{ro.ble}{0}
\verb{robledo}{ê}{}{}{}{s.m.}{Aglomerado de robles.}{ro.ble.do}{0}
\verb{robô}{}{}{}{}{s.m.}{Máquina de aparência humana.}{ro.bô}{0}
\verb{robô}{}{}{}{}{}{Máquina automática dotada de sensores e mecanismos de movimento precisos usada para desempenhar tarefas repetitivas em linhas de montagens ou missões específicas em locais contaminados ou de alto risco.}{ro.bô}{0}
\verb{roborar}{}{}{}{}{v.t.}{Tornar mais forte; reanimar, fortificar.}{ro.bo.rar}{0}
\verb{roborar}{}{}{}{}{}{Confirmar, ratificar, corroborar.}{ro.bo.rar}{\verboinum{1}}
\verb{robótica}{}{}{}{}{s.f.}{Ciência que se ocupa do projeto e construção de robôs.}{ro.bó.ti.ca}{0}
\verb{robotizar}{}{}{}{}{v.t.}{Implantar robôs (em uma indústria ou linha de produção).}{ro.bo.ti.zar}{0}
\verb{robotizar}{}{}{}{}{}{Dar (a pessoas) um comportamento automático, semelhante ao de um robô, tirando a liberdade, a iniciativa ou a capacidade de raciocinar e tomar decisões.}{ro.bo.ti.zar}{\verboinum{1}}
\verb{robustecer}{ê}{}{}{}{v.t.}{Tornar robusto; fortalecer.}{ro.bus.te.cer}{\verboinum{15}}
\verb{robustez}{ê}{}{}{}{s.f.}{Qualidade de robusto; vitalidade, solidez, disposição.}{ro.bus.tez}{0}
\verb{robusteza}{ê}{}{}{}{s.f.}{Robustez.}{ro.bus.te.za}{0}
\verb{robusto}{}{}{}{}{adj.}{Fisicamente forte; potente, vigoroso.}{ro.bus.to}{0}
\verb{robusto}{}{}{}{}{}{Saudável, sadio, vigoroso.}{ro.bus.to}{0}
\verb{robusto}{}{Fig.}{}{}{}{Que não cede; resistente, inabalável, firme.}{ro.bus.to}{0}
\verb{robusto}{}{Fig.}{}{}{}{Influente, poderoso, dominador.}{ro.bus.to}{0}
\verb{roca}{ó}{}{}{}{s.f.}{Rochedo, penhasco, penedo.}{ro.ca}{0}
\verb{roca}{ó}{}{}{}{s.f.}{Bastão próprio para enrolar o fio a ser fiado.}{ro.ca}{0}
\verb{roca}{ó}{Por ext.}{}{}{}{O fio enrolado nesse bastão.}{ro.ca}{0}
\verb{roça}{ó}{}{}{}{s.f.}{Terreno de cultivo; plantação.}{ro.ça}{0}
\verb{roça}{ó}{}{}{}{}{Zona rural.}{ro.ça}{0}
\verb{roça}{ó}{}{}{}{}{Ato ou efeito de roçar.}{ro.ça}{0}
\verb{roçado}{}{}{}{}{adj.}{Que se roçou; derrubado, cortado.}{ro.ça.do}{0}
\verb{roçado}{}{}{}{}{}{Desgastado pelo uso ou pela ação do tempo; gasto, danificado.}{ro.ça.do}{0}
\verb{roçado}{}{}{}{}{}{Pequena lavoura; roça.}{ro.ça.do}{0}
\verb{roçado}{}{}{}{}{}{Região sem vegetação ou de vegetação baixa no meio da mata; clareira.}{ro.ça.do}{0}
\verb{roçadura}{}{}{}{}{s.f.}{Ato ou efeito de roçar.}{ro.ça.du.ra}{0}
\verb{roçadura}{}{}{}{}{}{Atrito entre superfícies.}{ro.ça.du.ra}{0}
\verb{rocambole}{ó}{Cul.}{}{}{s.m.}{Tipo de bolo feito de massa de farinha e recheio, doce ou salgado, enrolada sobre si mesma.}{ro.cam.bo.le}{0}
\verb{rocambolesco}{ê}{}{}{}{adj.}{Cheio de imprevistos; confuso, acidentado.}{ro.cam.bo.les.co}{0}
\verb{roçar}{}{}{}{}{v.t.}{Derrubar vegetação; cortar.}{ro.çar}{0}
\verb{roçar}{}{}{}{}{}{Fazer atrito; friccionar.}{ro.çar}{0}
\verb{roçar}{}{}{}{}{}{Gastar pelo atrito; desgastar, corroer.}{ro.çar}{\verboinum{3}}
\verb{roceiro}{ê}{Bras.}{}{}{adj.}{Que mora na roça; caipira, interiorano.}{ro.cei.ro}{0}
\verb{roceiro}{ê}{}{}{}{s.m.}{Pequeno lavrador.}{ro.cei.ro}{0}
\verb{rocha}{ó}{}{}{}{s.f.}{Massa de pedra.}{ro.cha}{0}
\verb{rochedo}{ê}{}{}{}{s.m.}{Massa de pedra; grande rocha.}{ro.che.do}{0}
\verb{rochedo}{ê}{}{}{}{}{Grande porção de pedra escarpada; penhasco.}{ro.che.do}{0}
\verb{rochedo}{ê}{Fig.}{}{}{}{Aquilo que é firme, sólido, resistente.}{ro.che.do}{0}
\verb{rochoso}{ô}{}{"-osos ⟨ó⟩}{"-osa ⟨ó⟩}{adj.}{Relativo a rocha.}{ro.cho.so}{0}
\verb{rochoso}{ô}{}{"-osos ⟨ó⟩}{"-osa ⟨ó⟩}{}{Formado de rocha.}{ro.cho.so}{0}
\verb{rociar}{}{}{}{}{v.t.}{Cobrir de rocio; orvalhar.}{ro.ci.ar}{0}
\verb{rociar}{}{}{}{}{}{Borrifar, aspergir, umedecer.}{ro.ci.ar}{\verboinum{6}}
\verb{rocim}{}{}{"-ins}{}{s.m.}{Cavalo pequeno, magro e sem vigor.}{ro.cim}{0}
\verb{rocio}{}{}{}{}{s.m.}{Orvalho.}{ro.ci.o}{0}
\verb{rock}{}{Mús.}{}{}{s.m.}{Redução de \textit{rock"-and"-roll}, gênero de música popular derivado do \textit{blues}, de compasso quaternário e executado com guitarra e contrabaixo elétricos e bateria.}{\textit{rock}}{0}
\verb{rococó}{}{Art.}{}{}{s.m.}{Estilo decorativo surgido na França no século \textsc{xviii}, caracterizado pelo excesso de elementos ornamentais como curvas, flores, folhagens.}{ro.co.có}{0}
\verb{roda}{ó}{}{}{}{s.f.}{Peça de forma circular que gira em torno de um eixo.}{ro.da}{0}
\verb{roda}{ó}{}{}{}{}{Qualquer objeto circular ou que gira em torno de um eixo.}{ro.da}{0}
\verb{roda}{ó}{}{}{}{}{Grupo determinado de pessoas; círculo de amigos.}{ro.da}{0}
\verb{rodada}{}{}{}{}{s.f.}{Cada giro completo de uma roda; volta.}{ro.da.da}{0}
\verb{rodada}{}{Fig.}{}{}{}{Cada ciclo completo de qualquer processo que se desenvolve de maneira cíclica e repetitiva.}{ro.da.da}{0}
\verb{rodado}{}{}{}{}{adj.}{Que tem roda.}{ro.da.do}{0}
\verb{rodado}{}{}{}{}{}{Que já decorreu.}{ro.da.do}{0}
\verb{rodado}{}{}{}{}{}{Em forma de roda.}{ro.da.do}{0}
\verb{rodado}{}{}{}{}{}{Diz"-se de máquinas, especialmente veículos, que já foram submetidos a uso e desgaste longo ou intenso.}{ro.da.do}{0}
\verb{rodagem}{}{}{"-ens}{}{s.f.}{Ato ou efeito de rodar.}{ro.da.gem}{0}
\verb{rodagem}{}{}{"-ens}{}{}{Conjunto das rodas de uma máquina.}{ro.da.gem}{0}
\verb{roda"-gigante}{}{}{rodas"-gigantes}{}{s.f.}{Aparelho de parques de diversão constituído de uma grande roda em posição vertical que sustenta assentos nos quais vão as pessoas.}{ro.da"-gi.gan.te}{0}
\verb{rodamoinho}{}{}{}{}{s.m.}{Remoinho.}{ro.da.mo.i.nho}{0}
\verb{rodapé}{}{}{}{}{s.m.}{Peça longa de madeira ou outro material que se coloca rente ao chão nas paredes, para dar acabamento.}{ro.da.pé}{0}
\verb{rodapé}{}{}{}{}{}{A parte inferior de uma página impressa, onde são eventualmente colocadas anotações sobre o texto da página.}{ro.da.pé}{0}
\verb{rodar}{}{}{}{}{v.t.}{Dar movimento circular; girar.}{ro.dar}{0}
\verb{rodar}{}{}{}{}{}{Fazer a impressão de um grande número de cópias.}{ro.dar}{0}
\verb{rodar}{}{}{}{}{v.i.}{Girar, andar em volta.}{ro.dar}{0}
\verb{rodar}{}{}{}{}{}{Percorrer (determinada distância).}{ro.dar}{\verboinum{1}}
\verb{roda"-viva}{}{}{rodas"-vivas}{}{s.f.}{Movimento contínuo; corrupio, atividade.}{ro.da"-vi.va}{0}
\verb{roda"-viva}{}{}{rodas"-vivas}{}{}{Confusão, barafunda.}{ro.da"-vi.va}{0}
\verb{rodear}{}{}{}{}{v.t.}{Andar em volta de.}{ro.de.ar}{0}
\verb{rodear}{}{}{}{}{}{Fazer círculo em volta de; cercar.}{ro.de.ar}{\verboinum{4}}
\verb{rodeio}{ê}{}{}{}{s.m.}{Ato ou efeito de rodear.}{ro.dei.o}{0}
\verb{rodeio}{ê}{}{}{}{}{Discurso alongado para não abordar diretamente uma questão; circunlóquio.}{ro.dei.o}{0}
\verb{rodeio}{ê}{Bras.}{}{}{}{Ato de reunir gado para marcar ou ferrar.}{ro.dei.o}{0}
\verb{rodeio}{ê}{}{}{}{}{Prática ou competição de montar cavalo ou boi não domesticado.}{ro.dei.o}{0}
\verb{rodela}{é}{}{}{}{s.f.}{Pequena roda ou objeto de forma circular.}{ro.de.la}{0}
\verb{rodela}{é}{}{}{}{}{Fatia redonda.}{ro.de.la}{0}
\verb{rodilha}{}{}{}{}{s.f.}{Rosca de tecido que se coloca sobre a cabeça para levar cargas.}{ro.di.lha}{0}
\verb{rodilha}{}{}{}{}{}{Coisa que se arruma de maneira rápida e despreocupada.}{ro.di.lha}{0}
\verb{rodilho}{}{}{}{}{s.m.}{Rodilha.}{ro.di.lho}{0}
\verb{ródio}{}{Quím.}{}{}{s.m.}{Elemento químico metálico, branco, duro e brilhante, bastante dúctil e maleável, utilizado em ligas com a platina, em contatos elétricos, em dispositivos que devem suportar altas temperaturas e como catalisador. \elemento{45}{102.9055}{Rh}.}{ró.dio}{0}
\verb{rodízio}{}{}{}{}{s.m.}{Revezamento de funções, posições ou tarefas.}{ro.dí.zio}{0}
\verb{rodízio}{}{}{}{}{}{Roda metálica que se coloca nos pés de móveis pesados para facilitar o seu deslocamento.}{ro.dí.zio}{0}
\verb{rodízio}{}{}{}{}{}{Sistema de certos restaurantes em que os clientes, pagando um preço fixo, são servidos, à vontade, de vários pratos semelhantes.}{ro.dí.zio}{0}
\verb{rodízio}{}{}{}{}{}{Sistema de restrição da circulação de automóveis em regiões da cidade de acordo com o número final das placas de licença.}{ro.dí.zio}{0}
\verb{rodo}{ô}{}{}{}{s.m.}{Utensílio constituído de cabo longo e peça de madeira com uma faixa de borracha, usado para puxar água do chão.}{ro.do}{0}
\verb{rodo}{ô}{}{}{}{}{Tipo de enxada para juntar cereais e sal sobre o chão.}{ro.do}{0}
\verb{rododendro}{}{Bot.}{}{}{s.m.}{Arbusto com flores brancas ou vermelhas, cultivado como ornamental; azaléa.}{ro.do.den.dro}{0}
\verb{rodologia}{}{}{}{}{s.f.}{Ramo da botânica que estuda as rosas.}{ro.do.lo.gi.a}{0}
\verb{rodopiar}{}{}{}{}{v.i.}{Girar muito; corrupiar, remoinhar.}{ro.do.pi.ar}{\verboinum{6}}
\verb{rodopio}{}{}{}{}{s.m.}{Ato ou efeito de rodopiar.}{ro.do.pi.o}{0}
\verb{rodopio}{}{}{}{}{}{Rotação do corpo sobre uma só perna.}{ro.do.pi.o}{0}
\verb{rodovia}{}{}{}{}{s.f.}{Via para tráfego de veículos sobre pneus; estrada de rodagem; autovia, estrada.}{ro.do.vi.a}{0}
\verb{rodoviária}{}{Bras.}{}{}{s.f.}{Redução de \textit{estação rodoviária}, local onde ônibus de viagem fazem paradas para embarque e desembarque de passageiros e cargas.}{ro.do.vi.á.ria}{0}
\verb{rodoviário}{}{}{}{}{adj.}{Relativo a rodovia.}{ro.do.vi.á.rio}{0}
\verb{roedor}{ô}{}{}{}{adj.}{Que tem hábito de roer.}{ro.e.dor}{0}
\verb{roedor}{ô}{Zool.}{}{}{}{Relativo aos roedores, mamíferos herbívoros de pequeno porte que têm um par de dentes incisivos que crescem constantemente.}{ro.e.dor}{0}
\verb{roedor}{ô}{}{}{}{s.m.}{Espécime dos roedores.}{ro.e.dor}{0}
\verb{roedura}{}{}{}{}{s.f.}{Ato ou efeito de roer.}{ro.e.du.ra}{0}
\verb{roedura}{}{}{}{}{}{Ferimento na pele causado pelo atrito.}{ro.e.du.ra}{0}
\verb{roedura}{}{Fig.}{}{}{}{Tristeza, melancolia, pesar.}{ro.e.du.ra}{0}
\verb{roentgenfotografia}{}{}{}{}{s.f.}{Fotografia por meio de raio \textsc{x}; abreugrafia.}{ro.ent.gen.fo.to.gra.fi.a}{0}
\verb{roentgênio}{}{Quím.}{}{}{s.m.}{Elemento químico artificial, chamado anteriormente de \emph{unununium} (Unn), considerado como metal de transição e provavelmente metálico e sólido. \elemento{111}{(272)}{Rg}.}{roent.gê.nio}{0}
\verb{roer}{ê}{}{}{}{v.t.}{Cortar, triturar ou desgastar com os dentes.}{ro.er}{0}
\verb{roer}{ê}{}{}{}{}{Destruir vagarosa e constantemente; corroer.}{ro.er}{0}
\verb{roer}{ê}{}{}{}{}{Produzir uma ferida pelo atrito constante; machucar, ulcerar.}{ro.er}{0}
\verb{roer}{ê}{Fig.}{}{}{}{Causar sofrimento; atormentar, incomodar.}{ro.er}{\verboinum{17}}
\verb{rogar}{}{}{}{}{v.t.}{Pedir com submissão; implorar, suplicar.}{ro.gar}{0}
\verb{rogar}{}{}{}{}{}{Pedir com empenho; insistir, exortar.}{ro.gar}{\verboinum{5}}
\verb{rogativa}{}{}{}{}{s.f.}{Ato ou efeito de rogar.}{ro.ga.ti.va}{0}
\verb{rogatória}{}{}{}{}{s.f.}{Rogativa.}{ro.ga.tó.ria}{0}
\verb{rogo}{ô}{}{}{}{s.m.}{Ato ou efeito de rogar; rogativa.}{ro.go}{0}
\verb{rogo}{ô}{Relig.}{}{}{}{Súplica feita a uma divindade; prece, oração.}{ro.go}{0}
\verb{rojão}{}{}{"-ões}{}{}{Artefato pirotécnico feito com pólvora que estoura no alto com grande ruído.}{ro.jão}{0}
\verb{rojão}{}{}{"-ões}{}{s.m.}{Ato ou efeito de rojar; rojo.}{ro.jão}{0}
\verb{rojar}{}{}{}{}{v.t.}{Lançar, arremessar.}{rojar}{0}
\verb{rojar}{}{}{}{}{v.pron.}{Atirar"-se ao chão em sinal de respeito; prosternar"-se.}{rojar}{\verboinum{1}}
\verb{rojo}{ô}{}{}{}{s.m.}{Ato ou efeito de rojar.}{ro.jo}{0}
\verb{rol}{ó}{}{róis}{}{s.m.}{Relação de pessoas ou coisas; lista, listagem.}{rol}{0}
\verb{rola}{ô}{Zool.}{}{}{s.m.}{Certa ave de pequeno porte; pomba"-rola.}{ro.la}{0}
\verb{rola}{ô}{Chul.}{}{}{}{O pênis.}{ro.la}{0}
\verb{rolagem}{}{}{"-ens}{}{s.f.}{Ato ou efeito de rolar.}{ro.la.gem}{0}
\verb{rolamento}{}{}{}{}{s.m.}{Ato ou efeito de rolar.}{ro.la.men.to}{0}
\verb{rolamento}{}{}{}{}{}{Peça constituída de bolinhas de metal e graxa abundante, usada para diminuir o atrito em mecanismos que rodam.}{ro.la.men.to}{0}
\verb{rolar}{}{}{}{}{v.t.}{Fazer girar; rodar.}{ro.lar}{0}
\verb{rolar}{}{}{}{}{}{Fluir, escorrer (líquidos).}{ro.lar}{0}
\verb{rolar}{}{}{}{}{}{Deslocar"-se sobre rodas; rodar.}{ro.lar}{0}
\verb{rolar}{}{Bras.}{}{}{}{Adiar o pagamento (de um débito), negociando nova data de vencimento.}{ro.lar}{\verboinum{1}}
\verb{roldana}{}{}{}{}{s.f.}{Tipo de roda com um sulco sobre a superfície para guiar o movimento de cabos, correntes, cordas.}{rol.da.na}{0}
\verb{roldão}{}{}{"-ões}{}{s.m.}{Confusão, bagunça.}{rol.dão}{0}
\verb{roleta}{ê}{}{}{}{s.f.}{Jogo de azar em que há um disco numerado giratório e uma bola que indica o número premiado quando para sobre ele.}{ro.le.ta}{0}
\verb{roleta"-russa}{ê}{}{roletas"-russas ⟨ê⟩}{}{s.f.}{Jogo que consiste em deixar uma só bala no tambor de um revólver e apertar o gatilho apontando para o crânio próprio ou alheio, sem saber a posição exata da bala.}{ro.le.ta"-rus.sa}{0}
\verb{rolete}{ê}{}{}{}{s.m.}{Rolo pequeno.}{ro.le.te}{0}
\verb{rolete}{ê}{Bras.}{}{}{}{Fatia de cana para chupar.}{ro.le.te}{0}
\verb{rolha}{ô}{}{}{}{s.f.}{Peça cilíndrica de material flexível usada para tampar frascos e garrafas.}{ro.lha}{0}
\verb{rolha}{ô}{Fig.}{}{}{}{Silêncio imposto; censura.}{ro.lha}{0}
\verb{roliço}{}{}{}{}{adj.}{Que tem forma de rolo; arredondado, cilíndrico.}{ro.li.ço}{0}
\verb{roliço}{}{}{}{}{}{Diz"-se de indivíduo com formas corporais arredondadas, geralmente por causa de gordura.}{ro.li.ço}{0}
\verb{rolimã}{}{Bras.}{}{}{s.m.}{Mecanismo para reduzir o atrito; rolamento.}{ro.li.mã}{0}
\verb{rolimã}{}{}{}{}{}{Carrinho de madeira com rolamentos nas extremidades que anda rente ao chão.}{ro.li.mã}{0}
\verb{rolinha}{}{Bras.}{}{}{s.f.}{Certa ave de pequeno porte; rola.}{ro.li.nha}{0}
\verb{rolo}{ô}{}{}{}{s.m.}{Cilindro comprido e maciço.}{ro.lo}{0}
\verb{rolo}{ô}{}{}{}{}{Máquina para aplainar o solo por compressão, dotada de grandes cilindros.}{ro.lo}{0}
\verb{rolo}{ô}{}{}{}{}{Embrulho que se faz enrolando os objetos ou o invólucro.}{ro.lo}{0}
\verb{rolo}{ô}{}{}{}{}{Cilindro com eixo e cabo, revestido de material absorvente, usado para pintar superfícies planas.}{ro.lo}{0}
\verb{rolo}{ô}{Pop.}{}{}{}{Confusão, desordem, bagunça.}{ro.lo}{0}
\verb{rolo}{ô}{Bras.}{}{}{}{Caso amoroso superficial e passageiro.}{ro.lo}{0}
\verb{ROM}{}{Informát.}{}{}{s.f.}{Sigla inglesa de \textit{Read Only Memory}, dispositivo de memória permanente que armazena informações básicas sobre o funcionamento da máquina.}{ROM}{0}
\verb{romã}{}{}{}{}{s.f.}{Fruto com polpa comestível, muitas sementes e películas no interior e sabor agridoce.}{ro.mã}{0}
\verb{romance}{}{}{}{}{adj.2g.}{Relativo a Roma, seus habitantes, o Império Romano ou os países que o sucederam.}{ro.man.ce}{0}
\verb{romance}{}{Liter.}{}{}{s.m.}{Gênero literário em prosa, mais extenso e complexo que a novela ou o conto.}{ro.man.ce}{0}
\verb{romance}{}{Por ext.}{}{}{}{Fato ou narrativa marcada por fantasia ou exagero.}{ro.man.ce}{0}
\verb{romance}{}{Bras.}{}{}{}{Caso amoroso; namoro.}{ro.man.ce}{0}
\verb{romancear}{}{}{}{}{v.t.}{Dar forma de romance.}{ro.man.ce.ar}{0}
\verb{romancear}{}{}{}{}{v.i.}{Contar fatos inverossímeis ou com exagero.}{ro.man.ce.ar}{\verboinum{4}}
\verb{romanceiro}{ê}{}{}{}{adj.}{Relativo a romance.}{ro.man.cei.ro}{0}
\verb{romanceiro}{ê}{Liter.}{}{}{s.m.}{Coleção de poemas ou canções.}{ro.man.cei.ro}{0}
\verb{romancista}{}{}{}{}{adj.2g.}{Que escreve obras de ficção, especialmente romances.}{ro.man.cis.ta}{0}
\verb{romanesco}{ê}{}{}{}{adj.}{Que tem caráter de romance.}{ro.ma.nes.co}{0}
\verb{romanesco}{ê}{}{}{}{}{Apaixonado, romântico.}{ro.ma.nes.co}{0}
\verb{romanesco}{ê}{}{}{}{}{Sonhador, fantasioso, quimérico.}{ro.ma.nes.co}{0}
\verb{românico}{}{}{}{}{adj.}{Diz"-se das línguas que se formaram do latim, entre elas o português, o castelhano, o italiano, o romeno, o francês.}{ro.mâ.ni.co}{0}
\verb{românico}{}{}{}{}{}{Escrito em qualquer uma das línguas românicas.}{ro.mâ.ni.co}{0}
\verb{românico}{}{}{}{}{}{Relativo a Roma.}{ro.mâ.ni.co}{0}
\verb{românico}{}{Art.}{}{}{}{Diz"-se do período da história da arte, nos séculos \textsc{xi} e \textsc{xii}, em que predomina a arquitetura religiosa, o uso de arcos e abóbadas e a decoração rica das fachadas.}{ro.mâ.ni.co}{0}
\verb{romano}{}{}{}{}{adj.}{Relativo a Roma, capital da Itália.}{ro.ma.no}{0}
\verb{romano}{}{}{}{}{}{Diz"-se da língua falada na Roma antiga.}{ro.ma.no}{0}
\verb{romano}{}{}{}{}{s.m.}{Indivíduo natural ou habitante de Roma.}{ro.ma.no}{0}
\verb{romântico}{}{}{}{}{adj.}{Relativo a romance ou ao romantismo.}{ro.mân.ti.co}{0}
\verb{romântico}{}{}{}{}{}{Sonhador, fantasioso, sentimental.}{ro.mân.ti.co}{0}
\verb{romantismo}{}{Art.}{}{}{s.m.}{Movimento artístico e intelectual em que houve o predomínio do sentimento, da imaginação e do subjetivismo sobre a razão, relacionado historicamente à ascensão da burguesia como classe politicamente influente.}{ro.man.tis.mo}{0}
\verb{romantismo}{}{}{}{}{}{Qualidade de romântico.}{ro.man.tis.mo}{0}
\verb{romantizar}{}{}{}{}{v.t.}{Dar características de romance.}{ro.man.ti.zar}{\verboinum{1}}
\verb{romaria}{}{}{}{}{s.f.}{Peregrinação ou viagem de caráter religioso.}{ro.ma.ri.a}{0}
\verb{romãzeira}{ê}{Bot.}{}{}{s.f.}{Árvore que dá a romã.}{ro.mã.zei.ra}{0}
\verb{rômbico}{}{}{}{}{adj.}{Que tem formato de losango.}{rôm.bi.co}{0}
\verb{rombo}{}{}{}{}{s.m.}{Grande buraco.}{rom.bo}{0}
\verb{rombo}{}{}{}{}{s.m.}{Losango.}{rom.bo}{0}
\verb{rombo}{}{}{}{}{}{Ato de abrir à força; arrombamento.}{rom.bo}{0}
\verb{rombo}{}{Fig.}{}{}{}{Prejuízo, desfalque, déficit.}{rom.bo}{0}
\verb{rombo}{}{}{}{}{adj.}{Que tem pontas arredondadas.}{rom.bo}{0}
\verb{rombo}{}{}{}{}{}{Estúpido, obtuso, rombudo, imbecil.}{rom.bo}{0}
\verb{romboide}{}{Geom.}{}{}{s.m.}{Paralelogramo.}{rom.boi.de}{0}
\verb{romboide}{}{}{}{}{adj.2g.}{Que tem forma de paralelogramo.}{rom.boi.de}{0}
\verb{rombudo}{}{}{}{}{adj.}{Mal afiado.}{rom.bu.do}{0}
\verb{rombudo}{}{Fig.}{}{}{}{Estúpido, imbecil, rombo.}{rom.bu.do}{0}
\verb{rombudo}{}{Fig.}{}{}{}{Mal"-humorado; carrancudo.}{rom.bu.do}{0}
\verb{romeiro}{ê}{}{}{}{s.m.}{Participante de romaria.}{ro.mei.ro}{0}
\verb{romeno}{}{}{}{}{adj.}{Relativo à Romênia.}{ro.me.no}{0}
\verb{romeno}{}{}{}{}{s.m.}{Indivíduo natural ou habitante desse país.}{ro.me.no}{0}
\verb{romeno}{}{}{}{}{}{Língua românica oriental, oficial da Romênia, falada também em parte da Macedônia.}{ro.me.no}{0}
\verb{romeu"-e"-julieta}{ê}{}{romeus"-e"-julietas ⟨ê⟩}{}{s.m.}{Sobremesa feita de queijo com goiabada.}{ro.meu"-e"-ju.li.e.ta}{0}
\verb{rompante}{}{}{}{}{s.m.}{Impulso, ímpeto.}{rom.pan.te}{0}
\verb{rompante}{}{}{}{}{adj.2g.}{Impetuoso, arrebatado.}{rom.pan.te}{0}
\verb{rompante}{}{}{}{}{}{Arrogante, presunçoso, orgulhoso.}{rom.pan.te}{0}
\verb{romper}{ê}{}{}{}{v.t.}{Separar com violência; quebrar, arrebentar, partir.}{rom.per}{0}
\verb{romper}{ê}{}{}{}{}{Rasgar, despedaçar.}{rom.per}{0}
\verb{romper}{ê}{}{}{}{}{Atravessar, furar, entrar.}{rom.per}{0}
\verb{romper}{ê}{}{}{}{}{Cortar relações com alguém.}{rom.per}{0}
\verb{romper}{ê}{}{}{}{v.i.}{Aparecer de repente; surgir.}{rom.per}{\verboinum{12}}
\verb{rompimento}{}{}{}{}{s.m.}{Ato ou efeito de romper.}{rom.pi.men.to}{0}
\verb{ronca}{}{}{}{}{s.f.}{Ato ou efeito de roncar.}{ron.ca}{0}
\verb{ronca}{}{}{}{}{}{Intimidação, bravata.}{ron.ca}{0}
\verb{roncadura}{}{}{}{}{s.f.}{Ato ou efeito de roncar.}{ron.ca.du.ra}{0}
\verb{roncar}{}{}{}{}{v.i.}{Respirar ruidosamente durante o sono.}{roncar}{0}
\verb{roncar}{}{}{}{}{}{Dar roncos (os animais).}{roncar}{0}
\verb{roncar}{}{}{}{}{}{Produzir ruído grave.}{roncar}{0}
\verb{roncar}{}{}{}{}{}{Jactanciar"-se, gabar"-se.}{roncar}{\verboinum{2}}
\verb{ronceiro}{ê}{}{}{}{adj.}{Lento, vagaroso.}{ron.cei.ro}{0}
\verb{ronceiro}{ê}{}{}{}{}{Preguiçoso, indolente, molenga.}{ron.cei.ro}{0}
\verb{ronco}{}{}{}{}{s.m.}{Ruído emitido pela respiração de algumas pessoas enquanto dormem.}{ron.co}{0}
\verb{ronco}{}{}{}{}{}{Voz de certos animais, como tigre, urso, pantera, avestruz, javali.}{ron.co}{0}
\verb{ronco}{}{}{}{}{}{Barulho forte e grave.}{ron.co}{0}
\verb{ronco}{}{}{}{}{}{Bravata, fanfarronada, gabolice.}{ron.co}{0}
\verb{ronda}{}{}{}{}{s.f.}{Ato ou efeito de rondar.}{ron.da}{0}
\verb{ronda}{}{}{}{}{}{Grupo de militares ou funcionários civis que percorre as ruas zelando pela ordem e pela segurança.}{ron.da}{0}
\verb{ronda}{}{Por ext.}{}{}{}{Qualquer tipo de vigilância ou de inspeção.}{ron.da}{0}
\verb{ronda}{}{}{}{}{}{Dança de roda em que as pessoas dão as mãos e giram.}{ron.da}{0}
\verb{ronda}{}{}{}{}{}{Certo jogo de azar jogado com um baralho.}{ron.da}{0}
\verb{rondar}{}{}{}{}{v.t.}{Percorrer vigiando ou inspecionando.}{ron.dar}{0}
\verb{rondar}{}{}{}{}{}{Andar à volta de; rodear, circundar.}{ron.dar}{\verboinum{1}}
\verb{rondó}{}{Liter.}{}{}{s.m.}{Poema de forma fixa que às vezes apresenta estribilho.}{ron.dó}{0}
\verb{rondó}{}{Mús.}{}{}{}{Forma musical em que a seção principal retorna entre as seções subsidiárias.}{ron.dó}{0}
\verb{rondoniano}{}{}{}{}{adj.}{Relativo a Rondônia.}{ron.do.ni.a.no}{0}
\verb{rondoniano}{}{}{}{}{s.m.}{Indivíduo natural ou habitante desse estado.}{ron.do.ni.a.no}{0}
\verb{rondoniense}{}{}{}{}{adj.2g. e s.2g.}{Rondoniano.}{ron.do.ni.en.se}{0}
\verb{ronha}{}{}{}{}{s.f.}{Doença de pele que ataca alguns animais, como cavalos e ovelhas.}{ro.nha}{0}
\verb{ronha}{}{}{}{}{}{Habilidade de prejudicar os outros; malícia, ardil.}{ro.nha}{0}
\verb{ronqueira}{ê}{}{}{}{s.f.}{Doença respiratória equina.}{ron.quei.ra}{0}
\verb{ronqueira}{ê}{}{}{}{}{Ruído emitido por quem está com respiração difícil; ronco.}{ron.quei.ra}{0}
\verb{ronrom}{}{Onomat.}{"-ons}{}{s.m.}{Ruído respiratório que o gato faz quando descansa e quando é agradado.}{ron.rom}{0}
\verb{ronronar}{}{}{}{}{v.i.}{Fazer ronrom.}{ron.ro.nar}{\verboinum{1}}
\verb{roque}{ó}{}{}{}{s.m.}{Jogada do xadrez em que torre e rei são movimentados simultaneamente.}{ro.que}{0}
\verb{roque}{ó}{}{}{}{s.m.}{Forma aportuguesada de \textit{rock}.}{ro.que}{0}
\verb{roqueira}{ê}{}{}{}{s.f.}{Tipo de canhão que era usado pelos holandeses na região nordeste do Brasil no século \textsc{xvi}.}{ro.quei.ra}{0}
\verb{roqueiro}{ê}{}{}{}{adj.}{Relativo a roca (bastão para enrolar fio).}{ro.quei.ro}{0}
\verb{roqueiro}{ê}{}{}{}{s.m.}{Fabricante ou vendedor de rocas.}{ro.quei.ro}{0}
\verb{roqueiro}{ê}{}{}{}{}{Indivíduo que fia na roca.}{ro.quei.ro}{0}
\verb{roqueiro}{ê}{}{}{}{adj.}{Relativo a roca (rochedo).}{ro.quei.ro}{0}
\verb{roqueiro}{ê}{}{}{}{adj.}{Diz"-se de indivíduo que aprecia, executa ou compõe \textit{rock"-and"-roll}.}{ro.quei.ro}{0}
\verb{roquete}{ê}{}{}{}{s.m.}{Espécie de vestimenta estreita e com mangas, usada por religiosos. }{ro.que.te}{0}
\verb{ror}{ó}{Pop.}{}{}{s.m.}{Grande quantidade de pessoas ou de coisas.}{ror}{0}
\verb{roraimense}{}{}{}{}{adj.2g.}{Relativo a Roraima.}{ro.rai.men.se}{0}
\verb{roraimense}{}{}{}{}{s.2g.}{Indivíduo natural ou habitante desse estado.}{ro.rai.men.se}{0}
\verb{rorejar}{}{}{}{}{v.t.}{Banhar gota a gota.  }{ro.re.jar}{0}
\verb{rorejar}{}{}{}{}{v.i.}{Brotar, gotejar.}{ro.re.jar}{\verboinum{1}}
\verb{rosa}{ó}{}{}{}{s.f.}{A flor da roseira.}{ro.sa}{0}
\verb{rosácea}{}{Bot.}{}{}{s.f.}{Espécime das rosáceas, ordem que inclui árvores e arbustos com folhas simples e flores solitárias, cultivadas pelos frutos e como ornamentais.}{ro.sá.cea}{0}
\verb{rosácea}{}{}{}{}{}{Estrutura semelhante à da rosa desabrochada.}{ro.sá.cea}{0}
\verb{rosácea}{}{Art.}{}{}{}{Roseta. }{ro.sá.cea}{0}
\verb{rosáceo}{}{}{}{}{adj.}{Relativo a rosáceas ou a rosas. }{ro.sá.ceo}{0}
\verb{rosa"-cruz}{ó}{}{}{}{s.f.}{Sétimo grau ou quarta ordem da maçonaria.  }{ro.sa"-cruz}{0}
\verb{rosa"-cruz}{ó}{}{}{}{s.m.}{Indivíduo maçom com grau de rosa"-cruz. }{ro.sa"-cruz}{0}
\verb{rosado}{}{}{}{}{adj.}{Cuja cor tende a rosa. }{ro.sa.do}{0}
\verb{rosa"-dos"-ventos}{ó}{}{rosas"-dos"-ventos ⟨ó⟩}{}{s.f.}{Representação circular tradicional dos pontos cardeais (norte, sul, leste e oeste) e das direções intermediárias. }{ro.sa"-dos"-ven.tos}{0}
\verb{rosal}{}{}{"-ais}{}{s.m.}{Aglomerado de rosas; roseiral. }{ro.sal}{0}
\verb{rosário}{}{}{}{}{s.m.}{Conjunto de 165 contas dispostas em um fio representando, cada uma delas, uma oração.  }{ro.sá.rio}{0}
\verb{rosário}{}{Fig.}{}{}{}{Grande quantidade; sucessão. }{ro.sá.rio}{0}
\verb{rosbife}{}{Cul.}{}{}{s.m.}{Assado ou fritura de carne bovina, bem tostado por fora e malpassado por dentro, servido em fatias. }{ros.bi.fe}{0}
\verb{rosca}{ô}{}{}{}{s.f.}{Espiral na superfície interior ou exterior de um cilindro, como no parafuso, geralmente usada para fixação.}{ros.ca}{0}
\verb{rosca}{ô}{}{}{}{}{Pão em forma de argola. }{ros.ca}{0}
\verb{roscar}{}{}{}{}{v.t.}{Fixar com rosca ou parafuso; aparafusar.   }{ros.car}{0}
\verb{roscar}{}{}{}{}{}{Fazer roscas em.}{ros.car}{\verboinum{2}}
\verb{roseira}{ê}{Bot.}{}{}{s.f.}{Designação comum a vários arbustos e trepadeiras cultivados como ornamentais.  }{ro.sei.ra}{0}
\verb{roseiral}{}{}{"-ais}{}{s.m.}{Plantio de roseiras.  }{ro.sei.ral}{0}
\verb{róseo}{}{}{}{}{adj.}{Relativo a rosa.}{ró.seo}{0}
\verb{róseo}{}{}{}{}{}{Que tem cheiro semelhante ao das rosas.   }{ró.seo}{0}
\verb{róseo}{}{}{}{}{}{Cuja cor tende à das rosas; rosado.}{ró.seo}{0}
\verb{roseta}{ê}{}{}{}{s.f.}{Pequena rosa. }{ro.se.ta}{0}
\verb{roseta}{ê}{}{}{}{}{Roda dentada da espora.}{ro.se.ta}{0}
\verb{roseta}{ê}{}{}{}{}{Denominação de vários objetos circulares que são semelhantes a uma rosa.}{ro.se.ta}{0}
\verb{roseta}{ê}{}{}{}{}{Figura decorativa semelhante a uma rosa, usada em várias épocas como elemento de composição nas artes plásticas.}{ro.se.ta}{0}
\verb{rosicler}{é}{}{}{}{adj.2g.}{De cor rosa clara.   }{ro.si.cler}{0}
\verb{rosilho}{}{}{}{}{adj.}{Diz"-se de animal que tem pelo avermelhado entremeado de fios brancos. }{ro.si.lho}{0}
\verb{rosmaninho}{}{Bot.}{}{}{s.m.}{Alecrim.}{ros.ma.ni.nho}{0}
\verb{rosnar}{}{}{}{}{v.i.}{Emitir ruído com caráter de raiva e ameaça, mostrando os dentes.}{ros.nar}{\verboinum{1}}
\verb{rosquear}{}{}{}{}{v.t.}{Roscar.}{ros.que.ar}{\verboinum{4}}
\verb{rossio}{}{}{}{}{s.m.}{Praça ampla.  }{ros.si.o}{0}
\verb{rosto}{ô}{}{}{}{s.m.}{Parte anterior da cabeça; face, cara. }{ros.to}{0}
\verb{rosto}{ô}{}{}{}{}{Expressão da face; semblante, fisionomia. }{ros.to}{0}
\verb{rostro}{ó}{}{}{}{s.m.}{O bico das aves. }{ros.tro}{0}
\verb{rota}{ó}{}{}{}{s.f.}{Caminho que se segue para ir a algum lugar; trajeto, percurso. }{ro.ta}{0}
\verb{rotação}{}{}{"-ões}{}{s.f.}{Ato ou efeito de rotar; movimento giratório em torno de um eixo; giro. }{ro.ta.ção}{0}
\verb{rotariano}{}{}{}{}{adj.}{Relativo ao membro do Rotary Club, associação que tem por fim prestar serviços à comunidade e estabelecer laços de compreensão entre os povos.  }{ro.ta.ri.a.no}{0}
\verb{rotativa}{}{}{}{}{s.f.}{Máquina impressora usada para grandes tiragens de jornais. }{ro.ta.ti.va}{0}
\verb{rotatividade}{}{}{}{}{s.f.}{Qualidade do que é rotativo. }{ro.ta.ti.vi.da.de}{0}
\verb{rotatividade}{}{}{}{}{}{Rodízio.}{ro.ta.ti.vi.da.de}{0}
\verb{rotativo}{}{}{}{}{adj.}{Que faz girar, que imprime rotação a.}{ro.ta.ti.vo}{0}
\verb{rotativo}{}{}{}{}{}{Que se transfere em rodízio ou em revezamento. }{ro.ta.ti.vo}{0}
\verb{rotatória}{}{}{}{}{s.f.}{Entroncamento de rodovias em forma circular; trevo.  }{ro.ta.tó.ria}{0}
\verb{rotatório}{}{}{}{}{adj.}{Relativo à rotação.}{ro.ta.tó.rio}{0}
\verb{rotatório}{}{}{}{}{}{Que tem movimento de rotação; giratório.  }{ro.ta.tó.rio}{0}
\verb{roteirista}{}{}{}{}{adj.2g.}{Diz"-se de indivíduo que escreve roteiros para filmes e ações cênicas em geral. }{ro.tei.ris.ta}{0}
\verb{roteiro}{ê}{}{}{}{s.m.}{Itinerário de viagem.}{ro.tei.ro}{0}
\verb{roteiro}{ê}{}{}{}{}{Plano que deve ser seguido para se fazer alguma coisa. }{ro.tei.ro}{0}
\verb{rotina}{}{}{}{}{s.f.}{Sequência de atos ou procedimentos que se obseva pela força do hábito. }{ro.ti.na}{0}
\verb{rotineiro}{ê}{}{}{}{adj.}{Relativo à rotina. }{ro.ti.nei.ro}{0}
\verb{rotineiro}{ê}{}{}{}{}{Que segue a rotina. }{ro.ti.nei.ro}{0}
\verb{rotisseria}{}{}{}{}{s.f.}{Loja onde se vendem queijos, frios, carnes e outras viandas.  }{ro.tis.se.ri.a}{0}
\verb{roto}{ô}{}{}{}{adj.}{Que se rompeu; rasgado. }{ro.to}{0}
\verb{rotor}{ô}{}{}{}{s.m.}{Parte giratória de certas máquinas e motores, especialmente dos elétricos. }{ro.tor}{0}
\verb{rótula}{}{Anat.}{}{}{s.f.}{Osso achatado e arredondado que fica na parte da frente do joelho.}{ró.tu.la}{0}
\verb{rotular}{}{}{}{}{v.t.}{Colocar rótulo em alguma coisa; etiquetar.}{ro.tu.lar}{0}
\verb{rotular}{}{}{}{}{}{Dar determinado nome a uma pessoa por pensar que ela tem determinada qualidade ou característica; classificar, tachar.}{ro.tu.lar}{\verboinum{1}}
\verb{rótulo}{}{}{}{}{s.m.}{Papel impresso que se cola na embalagem de uma mercadoria para dar informações sobre ela.}{ró.tu.lo}{0}
\verb{rotunda}{}{}{}{}{s.f.}{Construção circular que termina em cúpula.}{ro.tun.da}{0}
\verb{rotunda}{}{}{}{}{}{Praça circular onde desembocam várias ruas ou avenidas.}{ro.tun.da}{0}
\verb{rotunda}{}{}{}{}{}{Pano de fundo.}{ro.tun.da}{0}
\verb{rotundo}{}{}{}{}{adj.}{Cujo formato é igual ou semelhante ao de uma esfera.}{ro.tun.do}{0}
\verb{rotundo}{}{Fig.}{}{}{}{Que soluciona; decide; que encerra uma questão, uma pendência; decisivo, categórico.}{ro.tun.do}{0}
\verb{rotura}{}{}{}{}{s.f.}{Ruptura.}{ro.tu.ra}{0}
\verb{roubalheira}{ê}{}{}{}{s.f.}{Roubo escandaloso, especialmente de dinheiro público.}{rou.ba.lhei.ra}{0}
\verb{roubar}{}{}{}{}{v.t.}{Apropriar"-se indevidamente.}{rou.bar}{0}
\verb{roubar}{}{}{}{}{}{Furtar sob violência ou ameaça.}{rou.bar}{0}
\verb{roubar}{}{}{}{}{}{Beneficiar competidor indevidamente.}{rou.bar}{\verboinum{1}}
\verb{roubo}{ô}{}{}{}{s.m.}{Apropriação ou favorecimento indevido.}{rou.bo}{0}
\verb{roubo}{ô}{}{}{}{}{Furto sob ameaça ou violência.}{rou.bo}{0}
\verb{roubo}{ô}{}{}{}{}{Produto dessa apropriação.}{rou.bo}{0}
\verb{rouco}{ô}{}{}{}{adj.}{Que apresenta rouquidão, que fala baixo, com uma voz grossa e áspera.}{rou.co}{0}
\verb{roufenho}{}{}{}{}{adj.}{Que parece falar pelo nariz; fanhoso.}{rou.fe.nho}{0}
\verb{round}{}{Esport.}{}{}{s.m.}{Cada um dos tempos em se divide uma luta de boxe; assalto.}{\textit{round}}{0}
\verb{roupa}{ô}{}{}{}{s.f.}{Conjunto de peças de um vestuário.}{rou.pa}{0}
\verb{roupa}{ô}{}{}{}{}{Cada peça desse conjunto.}{rou.pa}{0}
\verb{roupagem}{}{}{"-ens}{}{s.f.}{Conjunto de roupas; vestimenta.}{rou.pa.gem}{0}
\verb{roupagem}{}{Fig.}{"-ens}{}{}{Aspecto exterior; aparência.}{rou.pa.gem}{0}
\verb{roupão}{}{}{"-ões}{}{s.m.}{Peça caseira de vestuário, longa ou curta e aberta na frente, com mangas compridas ou curtas e um cinto do mesmo tecido.}{rou.pão}{0}
\verb{rouparia}{}{}{}{}{s.f.}{Grande quantidade de roupas.}{rou.pa.ri.a}{0}
\verb{rouparia}{}{}{}{}{}{Local para se guardar roupas.}{rou.pa.ri.a}{0}
\verb{roupa"-velha}{é}{Cul.}{roupas"-velhas ⟨é⟩}{}{s.f.}{Guisado preparado com sobras de carne, peixe etc.}{rou.pa"-ve.lha}{0}
\verb{roupa"-velha}{é}{Cul.}{roupas"-velhas ⟨é⟩}{}{}{Iguaria feita da mistura de feijão, farinha de mandioca e charque desfiado.}{rou.pa"-ve.lha}{0}
\verb{roupeiro}{ê}{}{}{}{s.m.}{Encarregado da rouparia de hospitais, internatos etc.}{rou.pei.ro}{0}
\verb{roupeiro}{ê}{}{}{}{}{Móvel em que se guardam roupas.}{rou.pei.ro}{0}
\verb{roupeiro}{ê}{}{}{}{}{Indivíduo que faz ou cuida de roupa.}{rou.pei.ro}{0}
\verb{roupeta}{ê}{}{}{}{s.f.}{Indumentária dos sacerdotes; batina.}{rou.pe.ta}{0}
\verb{rouquejar}{}{}{}{}{v.t.}{Falar com voz rouca; ter rouquidão.}{rou.que.jar}{0}
\verb{rouquejar}{}{}{}{}{v.i.}{Emitir sons roucos, especialmente pessoa ou animal.}{rou.que.jar}{\verboinum{1}}
\verb{rouquidão}{}{}{"-ões}{}{s.f.}{Alteração da voz, que a torna grossa e áspera e dificulta a fala.}{rou.qui.dão}{0}
\verb{rouxinol}{ch\ldots{}ó}{Zool.}{"-óis}{}{s.m.}{Pássaro europeu de canto melodioso.}{rou.xi.nol}{0}
\verb{roxear}{ch}{}{}{}{v.t.}{Tornar roxo ou semelhante ao roxo.}{ro.xe.ar}{\verboinum{4}}
\verb{roxo}{ôch}{}{}{}{s.m.}{Cor que resulta da mistura do vermelho e do azul.}{ro.xo}{0}
\verb{roxo}{ôch}{}{}{}{adj.}{Que tem essa cor.}{ro.xo}{0}
\verb{royalty}{}{}{}{}{s.m.}{Parte do lucro ou comissão paga ao detentor de um direito qualquer.}{\textit{royalty}}{0}
\verb{RR}{}{}{}{}{}{Sigla do estado de Roraima.}{RR}{0}
\verb{RS}{}{}{}{}{}{Sigla do estado do Rio Grande do Sul.}{RS}{0}
\verb{Ru}{}{Quím.}{}{}{}{Símb. do \textit{rutênio}.}{Ru}{0}
\verb{rua}{}{}{}{}{s.f.}{Caminho público por onde os veículos e as pessoas circulam em uma cidade.}{ru.a}{0}
\verb{ruandense}{}{}{}{}{adj.2g.}{Relativo a Ruanda (África).}{ru.an.den.se}{0}
\verb{ruandense}{}{}{}{}{s.2g.}{Indivíduo natural ou habitante desse país.}{ru.an.den.se}{0}
\verb{ruandês}{}{}{}{}{adj.}{Relativo a Ruanda, república da África Central.}{ru.an.dês}{0}
\verb{ruandês}{}{}{}{}{s.m.}{Indivíduo natural ou habitante desse país.}{ru.an.dês}{0}
\verb{rubéola}{}{Med.}{}{}{s.f.}{Doença que provoca o aparecimento de bolhas avermelhadas na pele, parecida com o sarampo.}{ru.bé.o.la}{0}
\verb{rubi}{}{}{}{}{s.m.}{Pedra preciosa de cor vermelha.}{ru.bi}{0}
\verb{rubiácea}{}{Bot.}{}{}{s.f.}{Espécime das Rubiáceas, família de plantas que inclui o cafeeiro.}{ru.bi.á.cea}{0}
\verb{rubicundo}{}{}{}{}{adj.}{Bastante avermelhado; rubro.}{ru.bi.cun.do}{0}
\verb{rubicundo}{}{}{}{}{}{Diz"-se de pessoa muito corada.}{ru.bi.cun.do}{0}
\verb{rubídio}{}{Quím.}{}{}{s.m.}{Elemento químico radioativo, prateado, mole, do grupo dos metais alcalinos, decompõe violentamente a água e se inflama espontaneamente exposto ao ar; utilizado em vidros especiais, em cerâmica e em válvulas eletrônicas. \elemento{37}{85.4678}{Rb}.}{ru.bí.dio}{0}
\verb{rúbido}{}{}{}{}{adj.}{Bastante avermelhado; rubro.}{rú.bi.do}{0}
\verb{rubiginoso}{ô}{}{"-osos ⟨ó⟩}{"-osa ⟨ó⟩}{adj.}{Revestido por ferrugem; enferrujado.}{ru.bi.gi.no.so}{0}
\verb{rubiginoso}{ô}{}{"-osos ⟨ó⟩}{"-osa ⟨ó⟩}{}{Cuja cor é vermelho"-alaranjada, semelhante à cor da ferrugem.}{ru.bi.gi.no.so}{0}
\verb{rubim}{}{}{}{}{}{Var. de \textit{rubi}.}{ru.bim}{0}
\verb{rublo}{}{}{}{}{s.m.}{Unidade monetária e moeda da ex"-União das Repúblicas Socialistas Soviéticas.}{ru.blo}{0}
\verb{rubor}{ô}{}{}{}{s.m.}{A cor vermelha e suas variações.}{ru.bor}{0}
\verb{rubor}{ô}{}{}{}{}{Vermelhidão no rosto causada por vergonha, febre etc.}{ru.bor}{0}
\verb{ruborizar}{}{}{}{}{v.t.}{Tornar rubro; avermelhar, enrubescer.}{ru.bo.ri.zar}{0}
\verb{ruborizar}{}{}{}{}{}{Fazer corar devido a constrangimento, indignação, timidez etc.; envergonhar.}{ru.bo.ri.zar}{\verboinum{1}}
\verb{rubrica}{}{}{}{}{s.f.}{Assinatura abreviada.}{ru.bri.ca}{0}
\verb{rubricar}{}{}{}{}{v.t.}{Colocar assinatura abreviada em.}{ru.bri.car}{\verboinum{2}}
\verb{rubro}{}{}{}{}{adj.}{Que tem cor vermelha intensa.}{ru.bro}{0}
\verb{rubro}{}{}{}{}{}{Diz"-se dessa cor.}{ru.bro}{0}
\verb{ruçar}{}{}{}{}{v.t.}{Tornar ruço ou pardacento.}{ru.çar}{0}
\verb{ruçar}{}{}{}{}{}{Tornar grisalho.}{ru.çar}{\verboinum{3}}
\verb{ruço}{}{}{}{}{adj.}{Desbotado pelo uso; surrado.}{ru.ço}{0}
\verb{ruço}{}{}{}{}{}{Entremeado de fios brancos; grisalho.}{ru.ço}{0}
\verb{ruço}{}{}{}{}{}{Diz"-se de quem tem cabelo louro ou castanho muito claro.}{ru.ço}{0}
\verb{rúcula}{}{}{}{}{s.f.}{Verdura de gosto meio amargo.}{rú.cu.la}{0}
\verb{rude}{}{}{}{}{adj.2g.}{Que não tem estudos; inculto, rústico.}{ru.de}{0}
\verb{rudez}{ê}{}{}{}{adj.}{Qualidade do que é rude.}{ru.dez}{0}
\verb{rudez}{ê}{}{}{}{}{Falta de conhecimento, de cultura; ignorância.}{ru.dez}{0}
\verb{rudez}{ê}{}{}{}{}{Falta de educação; grosseria, rispidez.}{ru.dez}{0}
\verb{rudeza}{ê}{}{}{}{s.f.}{Rudez.}{ru.de.za}{0}
\verb{rudimentar}{}{}{}{}{adj.2g.}{Relativo aos primeiros elementos, noções ou princípios de algo; elementar, básico, fundamental.}{ru.di.men.tar}{0}
\verb{rudimento}{}{}{}{}{s.m.}{Estrutura inicial; origem, primórdio.}{ru.di.men.to}{0}
\verb{rudimento}{}{}{}{}{}{O que se apresenta em estado primitivo.}{ru.di.men.to}{0}
\verb{rudimento}{}{}{}{}{}{Conjunto das noções básicas de qualquer ciência ou arte.}{ru.di.men.to}{0}
\verb{rueiro}{ê}{}{}{}{adj.}{Diz"-se de indivíduo que gosta de andar pelas ruas.}{ru.ei.ro}{0}
\verb{ruela}{é}{}{}{}{s.f.}{Rua pequena ou estreita; travessia, viela.}{ru.e.la}{0}
\verb{rufar}{}{}{}{}{v.t.}{Produzir rufos em viola, em tambor.}{ru.far}{0}
\verb{rufar}{}{}{}{}{}{Produzir som grosso, cadenciado e trêmulo.}{ru.far}{\verboinum{1}}
\verb{rufião}{}{}{"-ães \textit{ou} -ões}{rufiona}{s.m.}{Indivíduo que vive às custas de prostitutas.}{ru.fi.ão}{0}
\verb{rufião}{}{}{"-ães \textit{ou} -ões}{rufiona}{}{Indivíduo que vive provocando confusão e se envolvendo em brigas.}{ru.fi.ão}{0}
\verb{ruflar}{}{}{}{}{v.i.}{Mover"-se, produzindo som semelhante ao de ave que esvoaça.}{ru.flar}{0}
\verb{ruflar}{}{}{}{}{v.t.}{Agitar asas para alcançar voo.}{ru.flar}{\verboinum{1}}
\verb{rufo}{}{}{}{}{s.m.}{Toque de tambor com batidas rápidas e sucessivas.}{ru.fo}{0}
\verb{ruga}{}{}{}{}{s.f.}{Prega ou dobra na pele.}{ru.ga}{0}
\verb{rúgbi}{}{Esport.}{}{}{s.m.}{Jogo em que duas equipes se enfrentam, usando as mãos e os pés, na tentativa de levar a bola oval até a linha de fundo adversária ou fazê"-la passar por cima da barra horizontal, com um chute.}{rúg.bi}{0}
\verb{ruge"-ruge}{}{Onomat.}{ruges"-ruges \textit{ou} ruge"-ruges}{}{s.m.}{Barulho que a saia faz quando arrasta no chão.}{ru.ge"-ru.ge}{0}
\verb{ruge"-ruge}{}{}{ruges"-ruges \textit{ou} ruge"-ruges}{}{}{Som semelhante a esse.}{ru.ge"-ru.ge}{0}
\verb{rugido}{}{}{}{}{s.m.}{O urro dos leões, tigres e outros felinos.}{ru.gi.do}{0}
\verb{rugido}{}{}{}{}{}{Som semelhante a esse urro.}{ru.gi.do}{0}
\verb{rugir}{}{}{}{}{v.i.}{Emitir rugidos; urrar, bramir.}{ru.gir}{\verboinum{22}}
\verb{rugoso}{ô}{}{"-osos ⟨ó⟩}{"-osa ⟨ó⟩}{adj.}{Que tem rugas ou elevações.}{ru.go.so}{0}
\verb{ruído}{}{}{}{}{s.m.}{Qualquer som causado pela queda de um corpo ou pelo choque entre corpos; estrondo, barulho.}{ru.í.do}{0}
\verb{ruído}{}{}{}{}{}{Rumor contínuo e prolongado; bulício. (\textit{O defeito dessa televisão é o ruído que ela faz quando ligada.})}{ru.í.do}{0}
\verb{ruído}{}{Fig.}{}{}{}{Boato.}{ru.í.do}{0}
\verb{ruidoso}{ô}{}{"-osos ⟨ó⟩}{"-osa ⟨ó⟩}{adj.}{Que produz rumor; barulhento.}{rui.do.so}{0}
\verb{ruidoso}{ô}{}{"-osos ⟨ó⟩}{"-osa ⟨ó⟩}{}{Em que há ruído.}{rui.do.so}{0}
\verb{ruim}{}{}{"-ins}{}{adj.2g.}{Que não faz bem, que é prejudicial, mau, nocivo.}{ru.im}{0}
\verb{ruim}{}{}{"-ins}{}{}{De má qualidade.}{ru.im}{0}
\verb{ruína}{}{}{}{}{s.f.}{Decadência moral ou material.}{ru.í.na}{0}
\verb{ruína}{}{}{}{}{}{Restos de construções desmoronados; destroços, escombros.}{ru.í.na}{0}
\verb{ruindade}{}{}{}{}{s.f.}{Qualidade do que é ruim.}{ru.in.da.de}{0}
\verb{ruindade}{}{}{}{}{}{Ação má, perversa; maldade.}{ru.in.da.de}{0}
\verb{ruinoso}{ô}{}{"-osos ⟨ó⟩}{"-osa ⟨ó⟩}{adj.}{Que está em ruínas.}{rui.no.so}{0}
\verb{ruinoso}{ô}{}{"-osos ⟨ó⟩}{"-osa ⟨ó⟩}{}{Que está prestes a desmoronar.}{rui.no.so}{0}
\verb{ruinoso}{ô}{}{"-osos ⟨ó⟩}{"-osa ⟨ó⟩}{}{Prejudicial, mau, nocivo.}{rui.no.so}{0}
\verb{ruir}{}{}{}{}{v.i.}{Cair com ímpeto e rapidamente; desmoronar"-se.}{ru.ir}{0}
\verb{ruir}{}{Fig.}{}{}{}{Desfazer"-se, frustrar"-se.}{ru.ir}{\verboinum{26}\verboirregular{\emph{def.} ruímos, ruís}}
\verb{ruivacento}{}{}{}{}{adj.}{Um tanto ruivo; avermelhado.}{rui.va.cen.to}{0}
\verb{ruivo}{}{}{}{}{adj.}{Da cor entre o amarelo e o vermelho.}{rui.vo}{0}
\verb{ruivo}{}{}{}{}{}{Indivíduo que tem o cabelo dessa cor.}{rui.vo}{0}
\verb{rum}{}{}{}{}{s.m.}{Aguardente obtida pela fermentação alcoólica e destilação do caldo ou melaço da cana"-de"-açúcar.}{rum}{0}
\verb{ruma}{}{}{}{}{s.f.}{Quantidade de coisas sobrepostas.}{ru.ma}{0}
\verb{rumar}{}{}{}{}{v.t.}{Pôr rumo a embarcações; dirigir"-se.}{ru.mar}{0}
\verb{rumar}{}{}{}{}{}{Pôr"-se em direção a; ir.}{ru.mar}{\verboinum{1}}
\verb{rumba}{}{}{}{}{s.f.}{Dança popular afro"-cubana, em compasso binário e ritmo complexo.}{rum.ba}{0}
\verb{ruminação}{}{}{"-ões}{}{s.f.}{Ato ou efeito de ruminar.}{ru.mi.na.ção}{0}
\verb{ruminante}{}{Zool.}{}{}{s.m.}{Mamífero, como o veado, a girafa e os bovídeos, que possui um estômago complexo, com três ou quatro câmaras, adaptado à ruminação.}{ru.mi.nan.te}{0}
\verb{ruminar}{}{}{}{}{v.t.}{Entre os ruminantes, regurgitar e novamente remastigar o alimento.}{ru.mi.nar}{0}
\verb{ruminar}{}{Fig.}{}{}{}{Pensar muito em; meditar, refletir.}{ru.mi.nar}{\verboinum{1}}
\verb{rumo}{}{}{}{}{s.m.}{Percurso, orientação a seguir para ir de um lugar para outro; caminho, itinerário, rota.}{ru.mo}{0}
\verb{rumor}{ô}{}{}{}{s.m.}{Murmúrio ou ruído de coisas que mudam de lugar.}{ru.mor}{0}
\verb{rumor}{ô}{}{}{}{}{Som confuso; barulho, burburinho.}{ru.mor}{0}
\verb{rumor}{ô}{}{}{}{}{Notícia que se espalha rapidamente; boato. }{ru.mor}{0}
\verb{rumorejante}{}{}{}{}{adj.2g.}{Que rumoreja; sussurrante.}{ru.mo.re.jan.te}{0}
\verb{rumorejar}{}{}{}{}{v.t.}{Produzir rumor; sussurrar brandamente.}{ru.mo.re.jar}{\verboinum{1}}
\verb{rumorejo}{ê}{}{}{}{s.m.}{Ato ou efeito de rumorejar; murmúrio, sussurro.}{ru.mo.re.jo}{0}
\verb{rumorejo}{ê}{}{}{}{}{Clamor ou ressonância de múltiplas vozes; vozaria.}{ru.mo.re.jo}{0}
\verb{rumoroso}{ô}{}{"-osos ⟨ó⟩}{"-osa ⟨ó⟩}{adj.}{Que produz rumor; ruidoso, barulhento.}{ru.mo.ro.so}{0}
\verb{rumoroso}{ô}{}{"-osos ⟨ó⟩}{"-osa ⟨ó⟩}{}{Que provoca muito falatório.}{ru.mo.ro.so}{0}
\verb{rupestre}{é}{}{}{}{adj.2g.}{Gravado ou traçado na rocha.}{ru.pes.tre}{0}
\verb{rupestre}{é}{}{}{}{}{Construído em rochedo.}{ru.pes.tre}{0}
\verb{rupestre}{é}{}{}{}{}{Que vive nas pedras.}{ru.pes.tre}{0}
\verb{rupia}{}{}{}{}{s.f.}{Unidade do sistema monetário da Índia, Paquistão, Nepal, Indonésia, Butão, Mascate, Omã, Sri Lanka, República das Maldivas e Maurício.}{ru.pi.a}{0}
\verb{rupia}{}{Med.}{}{}{s.f.}{Inflamação da pele caracterizada por afecção ulcerosa da perna, acompanhada de congestão local.}{ru.pi.a}{0}
\verb{rupícola}{}{}{}{}{adj.2g.}{Que vive nas rochas e cavernas; rupestre.}{ru.pí.co.la}{0}
\verb{ruptura}{}{}{}{}{s.f.}{Ato ou efeito de romper.}{rup.tu.ra}{0}
\verb{ruptura}{}{}{}{}{}{Rompimento de relações sociais ou de compromisso. }{rup.tu.ra}{0}
\verb{ruptura}{}{}{}{}{}{Suspensão, corte, interrupção, quebra. }{rup.tu.ra}{0}
\verb{rural}{}{}{"-ais}{}{adj.2g.}{Pertencente ou relativo ao, ou próprio do campo ou da vida no campo; campestre.}{ru.ral}{0}
\verb{rural}{}{}{"-ais}{}{}{Agrícola, campesino, camponês, rústico.}{ru.ral}{0}
\verb{ruralismo}{}{}{}{}{s.m.}{Representação de cenas da vida rural em obras de arte.}{ru.ra.lis.mo}{0}
\verb{ruralismo}{}{}{}{}{}{Modo de vida que preconiza a vida no campo e tudo que lhe diz respeito.}{ru.ra.lis.mo}{0}
\verb{ruralismo}{}{}{}{}{}{Predomínio da vida e população do campo sobre a vida e população da cidade.}{ru.ra.lis.mo}{0}
\verb{ruralista}{}{}{}{}{adj.2g.}{Relativo ao ruralismo.}{ru.ra.lis.ta}{0}
\verb{ruralista}{}{}{}{}{}{Que defende os interesses do campo.}{ru.ra.lis.ta}{0}
\verb{ruralista}{}{}{}{}{}{Diz"-se do artista que, nos seus trabalhos, tem preferência por cenas do campo.}{ru.ra.lis.ta}{0}
\verb{rurícola}{}{}{}{}{adj.2g.}{Que vive no campo.}{ru.rí.co.la}{0}
\verb{rurícola}{}{}{}{}{}{Que cultiva o campo.}{ru.rí.co.la}{0}
\verb{rusga}{}{}{}{}{s.f.}{Pequeno desentendimento entre pessoas; briga, questão, confusão.}{rus.ga}{0}
\verb{rusgar}{}{}{}{}{v.t.}{Brigar.}{rus.gar}{\verboinum{5}}
\verb{rusguento}{}{}{}{}{adj.}{Que vive envolvido em rusgas; briguento, barulhento.}{rus.guen.to}{0}
\verb{rush}{}{}{}{}{s.m.}{Tráfego intenso numa mesma direção. (\textit{Nos finais de semana, é comum o }rush\textit{ nas estradas.})}{\textit{rush}}{0}
\verb{russo}{}{}{}{}{adj.}{Relativo à Rússia.}{rus.so}{0}
\verb{russo}{}{}{}{}{s.m.}{Indivíduo natural ou habitante desse país.}{rus.so}{0}
\verb{rusticidade}{}{}{}{}{s.f.}{Qualidade do que é rústico.}{rus.ti.ci.da.de}{0}
\verb{rusticidade}{}{}{}{}{}{Grosseria; falta de delicadeza.}{rus.ti.ci.da.de}{0}
\verb{rústico}{}{}{}{}{adj.}{Pertencente ou relativo ao, ou próprio do campo; rural, campestre.}{rús.ti.co}{0}
\verb{rústico}{}{}{}{}{}{Grosseiro, rude.}{rús.ti.co}{0}
\verb{rustir}{}{Pop.}{}{}{v.t.}{Enganar, iludir, ludibriar; lesar na partilha do roubo.}{rus.tir}{0}
\verb{rustir}{}{}{}{}{}{Tornar oculto; encobrir, esconder, enrustir.}{rus.tir}{\verboinum{18}}
\verb{rutácea}{}{Bot.}{}{}{s.f.}{Espécime das rutáceas, família cujas árvores e arbustos do gênero \textit{Citrus} são muito cultivados pelos seus frutos, como a laranjeira, o limoeiro, a tangerineira, a limeira etc.}{ru.tá.cea}{0}
\verb{rutênio}{}{Quím.}{}{}{s.m.}{Elemento químico metálico, branco"-prateado, duro,  pouco dúctil e maleável, utilizado como elemento endurecedor em ligas com o ósmio, o paládio e a platina. \elemento{44}{101.07}{Ru}.}{ru.tê.nio}{0}
\verb{rutherfórdio}{}{Quím.}{}{}{s.m.}{Elemento químico sintético, radioativo, que apresenta propriedades químicas semelhantes ao elemento háfnio. \elemento{104}{(261)}{Rf}.}{ru.ther.fór.dio}{0}
%\verb{}{}{}{}{}{}{}{}{0}
\verb{rutilante}{}{}{}{}{adj.2g.}{Que rutila; muito brilhante, resplandecente, rútilo.}{ru.ti.lan.te}{0}
\verb{rutilar}{}{}{}{}{v.t.}{Tornar rútilo; brilhar intensamente, resplandecer.}{ru.ti.lar}{\verboinum{1}}
\verb{rútilo}{}{}{}{}{adj.}{Rutilante.}{rú.ti.lo}{0}
