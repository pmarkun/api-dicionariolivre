\verb{e}{ê/ ou /é}{}{}{}{s.m.}{Quinta letra e segunda vogal do alfabeto português.}{e}{0}
\verb{e}{ê}{}{}{}{conj.}{Une palavras ou orações.}{e}{0}
\verb{e}{ê}{}{}{}{}{Expressa valor adversativo; mas, porém.}{e}{0}
\verb{e}{ê}{Gram.}{}{}{}{No começo do período, na fala, pode valer como partícula interrogativa.}{e}{0}
\verb{e}{ê}{}{}{}{}{Usada para reforçar uma resposta.}{e}{0}
\verb{E}{}{Mat.}{}{}{}{No sistema hexadecimal, representa o décimo quinto algarismo, equivalente ao número decimal 14.}{E}{0}
\verb{E}{}{Mús.}{}{}{}{A nota ou acorde referente ao \textit{mi}, ou à terceira nota da escala de \textit{dó}.}{E}{0}
%\verb{}{}{}{}{}{}{}{}{0}
\verb{ébano}{}{Bot.}{}{}{s.m.}{Árvore de madeira escura e resistente}{é.ba.no}{0}
\verb{ébano}{}{}{}{}{}{Madeira dessa árvore.}{é.ba.no}{0}
\verb{ebonite}{}{}{}{}{s.f.}{Borracha vulcanizada, usada na indústria elétrica.}{e.bo.ni.te}{0}
\verb{ébrio}{}{}{}{}{adj.}{Diz"-se de indivíduo que costumeiramente ingere bebida alcoólica demais; bêbado, cachaceiro.}{é.brio}{0}
\verb{ebulição}{}{}{"-ões}{}{s.f.}{Ação de ferver; fervura.}{e.bu.li.ção}{0}
\verb{ebulição}{}{}{"-ões}{}{}{Passagem do estado líquido ao gasoso.}{e.bu.li.ção}{0}
\verb{ebulição}{}{Fig.}{"-ões}{}{}{Estado de nervosismo; agitação, exaltação.}{e.bu.li.ção}{0}
\verb{ebulir}{}{}{}{}{v.i.}{Entrar em ebulição.}{e.bu.lir}{\verboinum{34}}
\verb{ebúrneo}{}{}{}{}{adj.}{Relativo a ou feito de marfim.}{e.búr.neo}{0}
\verb{ebúrneo}{}{}{}{}{}{Relativo ou pertencente à Costa do Marfim.}{e.búr.neo}{0}
\verb{ebúrneo}{}{}{}{}{s.m.}{Indivíduo natural ou habitante desse país.}{e.búr.neo}{0}
\verb{eclampsia}{}{Med.}{}{}{s.f.}{Convulsão grave no final da gravidez.}{e.clamp.si.a}{0}
\verb{eclâmpsia}{}{}{}{}{}{Var. de \textit{eclampsia}.}{e.clâmp.si.a}{0}
\verb{ecler}{é}{}{}{}{s.m.}{Fecho de correr usado em roupas, artefatos de couro etc., constituído de dois cadarços com dentes metálicos, que se encaixam por ação de um cursor; fecho ecler, zíper.}{e.cler}{0}
\verb{eclesiástico}{}{}{}{}{adj.}{Relativo à Igreja ou aos seus sacerdotes.}{e.cle.si.ás.ti.co}{0}
\verb{eclesiástico}{}{}{}{}{s.m.}{Membro do corpo social da Igreja; sacerdote, clérigo, padre.}{e.cle.si.ás.ti.co}{0}
\verb{eclético}{}{}{}{}{adj.}{Que mistura um pouco de cada estilo, método, doutrina etc.}{e.clé.ti.co}{0}
\verb{ecletismo}{}{}{}{}{s.m.}{Método filosófico que conjuga teses originárias de doutrinas diferentes.}{e.cle.tis.mo}{0}
\verb{eclipsar}{}{}{}{}{v.t.}{Impedir que um astro seja visto, total ou parcialmente, pela passagem de um outro astro à sua frente.}{e.clip.sar}{0}
\verb{eclipsar}{}{Fig.}{}{}{}{Fazer diminuir a importância de outra pessoa ou coisa; sobrepujar.}{e.clip.sar}{\verboinum{1}}
\verb{eclipse}{}{Astron.}{}{}{s.m.}{Encobrimento de um astro pela passagem de outro astro à sua frente.}{e.clip.se}{0}
\verb{eclíptico}{}{}{}{}{adj.}{Relativo a eclipse.}{e.clíp.ti.co}{0}
\verb{eclodir}{}{}{}{}{v.i.}{Ter um começo repentino.}{e.clo.dir}{\verboinum{34}}
\verb{écloga}{}{Liter.}{}{}{s.f.}{Poema de temática campestre em que dialogam dois pastores.}{é.clo.ga}{0}
\verb{eclosão}{}{}{"-ões}{}{s.f.}{Ação de vir à luz, de aparecer, de desenvolver"-se; desabrochamento.}{e.clo.são}{0}
\verb{eclusa}{}{}{}{}{s.f.}{Em rios com grande desnível, dique que viabiliza a navegação.}{e.clu.sa}{0}
\verb{eco}{é}{}{}{}{s.m.}{Fenômeno físico que gera repetição de sons.}{e.co}{0}
\verb{eco}{é}{}{}{}{}{Repercussão de um fato.}{e.co}{0}
\verb{ecoar}{}{}{}{}{v.i.}{Produzir eco.}{e.co.ar}{0}
\verb{ecoar}{}{}{}{}{v.t.}{Repetir, reproduzir.}{e.co.ar}{\verboinum{7}}
\verb{ecocardiograma}{}{Med.}{}{}{s.m.}{Ultrassonografia do coração.}{e.co.car.di.o.gra.ma}{0}
\verb{ecologia}{}{Biol.}{}{}{s.f.}{Ciência que estuda as relações entre os seres vivos e o seu ambiente natural.}{e.co.lo.gi.a}{0}
\verb{ecológico}{}{}{}{}{adj.}{Relativo à ecologia.}{e.co.ló.gi.co}{0}
\verb{ecologista}{}{}{}{}{s.2g.}{Indivíduo que se dedica ao estudo e à defesa do meio ambiente e dos seres que nele vivem.}{e.co.lo.gis.ta}{0}
\verb{econometria}{}{}{}{}{s.f.}{Método de análise de dados estatísticos que mede as grandezas econômicas.}{e.co.no.me.tri.a}{0}
\verb{economia}{}{}{}{}{s.f.}{Estudo da produção, distribuição, acumulação e consumo de bens.}{e.co.no.mi.a}{0}
\verb{economia}{}{}{}{}{}{Moderação ou redução de gastos.}{e.co.no.mi.a}{0}
\verb{economiário}{}{}{}{}{adj.}{Relativo à Caixa Econômica Federal.}{e.co.no.mi.á.rio}{0}
\verb{economiário}{}{}{}{}{s.m.}{Funcionário dessa instituição.}{e.co.no.mi.á.rio}{0}
\verb{economias}{}{}{}{}{s.f.pl.}{Bens acumulados por efeito de economia.}{e.co.no.mi.as}{0}
\verb{econômico}{}{}{}{}{adj.}{Que diz respeito à economia.}{e.co.nô.mi.co}{0}
\verb{econômico}{}{}{}{}{}{Que gasta pouco.}{e.co.nô.mi.co}{0}
\verb{economista}{}{}{}{}{s.2g.}{Especialista em questões econômicas.}{e.co.no.mis.ta}{0}
\verb{economista}{}{}{}{}{}{Bacharel em ciências econômicas.}{e.co.no.mis.ta}{0}
\verb{economizar}{}{}{}{}{v.t.}{Deixar de gastar alguma coisa; poupar.}{e.co.no.mi.zar}{0}
\verb{economizar}{}{}{}{}{}{Gastar alguma coisa na menor quantidade possível.}{e.co.no.mi.zar}{\verboinum{1}}
\verb{ecônomo}{}{}{}{}{s.m.}{Administrador ou governante de uma residência, ou de uma instituição privada ou pública; mordomo.}{e.cô.no.mo}{0}
\verb{ecônomo}{}{}{}{}{}{Eclesiástico que administra os bens de uma abadia, um benefício etc.}{e.cô.no.mo}{0}
\verb{ecônomo}{}{}{}{}{}{Encarregado da despensa; despenseiro.}{e.cô.no.mo}{0}
\verb{ecosfera}{é}{}{}{}{s.f.}{Conjunto de regiões do universo onde podem existir organismos vivos.}{e.cos.fe.ra}{0}
\verb{ecossistema}{}{}{}{}{s.m.}{Conjunto do relacionamento entre o meio ambiente, as plantas e os animais que nele vivem.}{e.cos.sis.te.ma}{0}
\verb{ecoturismo}{}{}{}{}{s.m.}{Turismo que respeita e preserva o equilíbrio do meio, promovendo a educação ambiental; turismo ecológico.}{e.co.tu.ris.mo}{0}
\verb{écran}{}{}{}{}{s.m.}{Quadro branco sobre o qual se projetam imagens fixas ou animadas, no cinema ou na televisão.}{é.cran}{0}
\verb{écran}{}{}{}{}{}{Tela de cinema.}{é.cran}{0}
\verb{ectoplasma}{}{Biol.}{}{}{s.m.}{A camada mais externa do citoplasma de uma célula.}{ec.to.plas.ma}{0}
\verb{ectoplasma}{}{}{}{}{}{A substância visível, de origem psíquica, que se supõe emanar de certos médiuns. }{ec.to.plas.ma}{0}
\verb{ecumênico}{}{}{}{}{adj.}{Relativo ao ecumenismo.}{e.cu.mê.ni.co}{0}
\verb{ecumênico}{}{}{}{}{}{Que comporta pessoas e grupos de diferentes credos ou ideologias.}{e.cu.mê.ni.co}{0}
\verb{ecumênico}{}{}{}{}{}{Relativo a toda a Terra habitada; universal.}{e.cu.mê.ni.co}{0}
\verb{ecumenismo}{}{Relig.}{}{}{s.m.}{Movimento de unificação das igrejas cristãs.}{e.cu.me.nis.mo}{0}
\verb{ecúmeno}{}{}{}{}{s.m.}{A área habitada ou habitável da Terra.}{e.cú.me.no}{0}
\verb{ecúmeno}{}{}{}{}{}{O universal, o geral.}{e.cú.me.no}{0}
\verb{eczema}{}{Med.}{}{}{s.m.}{Tipo de inflamação da pele caracterizada por coceira e formação de vesículas e crostas.}{ec.ze.ma}{0}
\verb{edaz}{}{}{}{}{adj.2g.}{Que devora; voraz.}{e.daz}{0}
\verb{edaz}{}{}{}{}{}{Glutão.}{e.daz}{0}
\verb{edema}{}{Med.}{}{}{s.m.}{Inchaço causado pelo acúmulo anormal de líquido nos tecidos do organismo.}{e.de.ma}{0}
\verb{éden}{}{Relig.}{}{}{s.m.}{De acordo com a Bíblia, o paraíso terrestre.}{é.den}{0}
\verb{edição}{}{}{"-ões}{}{s.f.}{Ato ou efeito de editar.}{e.di.ção}{0}
\verb{edição}{}{}{"-ões}{}{}{Conjunto dos exemplares de obra impressa tirados das mesmas matrizes de impressão.}{e.di.ção}{0}
\verb{edícula}{}{}{}{}{s.f.}{Pequena casa.}{e.dí.cu.la}{0}
\verb{edificação}{}{}{"-ões}{}{s.f.}{Ato ou efeito de edificar.}{e.di.fi.ca.ção}{0}
\verb{edificação}{}{}{"-ões}{}{}{Edifício, casa, prédio.}{e.di.fi.ca.ção}{0}
\verb{edificação}{}{}{"-ões}{}{}{Aperfeiçoamento moral ou religioso.}{e.di.fi.ca.ção}{0}
\verb{edificante}{}{}{}{}{adj.2g.}{Instrutivo, moralizador, esclarecedor.}{e.di.fi.can.te}{0}
\verb{edificar}{}{}{}{}{v.t.}{Construir.}{e.di.fi.car}{0}
\verb{edificar}{}{}{}{}{}{Instituir, fundar, criar.}{e.di.fi.car}{0}
\verb{edificar}{}{}{}{}{}{Conduzir à virtude; instruir.}{e.di.fi.car}{\verboinum{2}}
\verb{edifício}{}{}{}{}{s.m.}{Construção, prédio, casa, edificação.}{e.di.fí.cio}{0}
\verb{edil}{}{}{}{}{s.m.}{Vereador; conselheiro municipal.}{e.dil}{0}
\verb{edilidade}{}{}{}{}{s.f.}{O cargo de edil.}{e.di.li.da.de}{0}
\verb{edital}{}{}{"-ais}{}{s.m.}{Escrito oficial com aviso ou determinação para ser publicado.}{e.di.tal}{0}
\verb{editar}{}{}{}{}{v.t.}{Preparar texto para publicação.}{e.di.tar}{0}
\verb{editar}{}{}{}{}{}{Publicar livro, periódico.}{e.di.tar}{0}
\verb{editar}{}{}{}{}{}{Montar filme de cinema ou de vídeo fazendo cortes e colocando as cenas em ordem.}{e.di.tar}{\verboinum{1}}
\verb{edito}{}{}{}{}{s.m.}{Ordem, mandato, decreto.}{e.di.to}{0}
\verb{édito}{}{Jur.}{}{}{s.m.}{Ordem judicial que se publica em anúncio ou edital.}{é.di.to}{0}
\verb{editor}{ô}{}{}{}{adj.}{Que edita.}{e.di.tor}{0}
\verb{editor}{ô}{}{}{}{s.m.}{Responsável por uma publicação.}{e.di.tor}{0}
\verb{editor}{ô}{}{}{}{}{Responsável pela editoração em uma publicação.}{e.di.tor}{0}
\verb{editora}{ô}{}{}{}{s.f.}{Estabelecimento que edita.}{e.di.to.ra}{0}
\verb{editoração}{}{}{"-ões}{}{s.f.}{Preparação de originais para uma publicação, fazendo acertos no conteúdo e na forma do material a ser impresso.}{e.di.to.ra.ção}{0}
\verb{editorar}{}{}{}{}{v.t.}{Preparar o material para ser impresso e publicado.}{e.di.to.rar}{0}
\verb{editorar}{}{Desus.}{}{}{}{Editar.}{e.di.to.rar}{\verboinum{1}}
\verb{editoria}{}{}{}{}{s.f.}{Cada uma das seções de um órgão de imprensa.}{e.di.to.ri.a}{0}
\verb{editorial}{}{}{"-ais}{}{adj.2g.}{Relativo a editor ou a editora.}{e.di.to.ri.al}{0}
\verb{editorial}{}{}{"-ais}{}{}{Artigo publicado em periódico que apresenta ostensivamente a opinião do órgão de imprensa, geralmente escrito por jornalistas ou editores dos altos níveis hierárquicos da empresa.}{e.di.to.ri.al}{0}
\verb{editorialista}{}{}{}{}{s.2g.}{Indivíduo que escreve editoriais em um órgão de imprensa.}{e.di.to.ri.a.lis.ta}{0}
\verb{edredão}{}{}{"-ões \textit{ou} -ãos}{}{s.m.}{Edredom.}{e.dre.dão}{0}
\verb{edredom}{}{}{"-ons}{}{s.m.}{Coberta acolchoada bastante quente, recheada de algodão, paina ou material sintético.}{e.dre.dom}{0}
\verb{educação}{}{}{"-ões}{}{s.f.}{Ato ou efeito de educar.}{e.du.ca.ção}{0}
\verb{educação}{}{}{"-ões}{}{}{Desenvolvimento das capacidades intelectuais, morais e físicas do ser humano; ensino.}{e.du.ca.ção}{0}
\verb{educação}{}{}{"-ões}{}{}{Bons modos; polidez, delicadeza, civilidade, cortesia.}{e.du.ca.ção}{0}
\verb{educacional}{}{}{"-ais}{}{adj.2g.}{Relativo a educação.}{e.du.ca.ci.o.nal}{0}
\verb{educado}{}{}{}{}{adj.}{Que recebeu educação; instruído.}{e.du.ca.do}{0}
\verb{educado}{}{}{}{}{}{Polido, delicado, bem"-educado, cortês.}{e.du.ca.do}{0}
\verb{educador}{ô}{}{}{}{s.m.}{Profissional da educação.}{e.du.ca.dor}{0}
\verb{educandário}{}{}{}{}{s.m.}{Estabelecimento de educação.}{e.du.can.dá.rio}{0}
\verb{educando}{}{}{}{}{s.m.}{Indivíduo que está sendo educado; aluno.}{e.du.can.do}{0}
\verb{educar}{}{}{}{}{v.t.}{Desenvolver as capacidades intelectuais, morais e físicas; transmitir conhecimentos; instruir.}{e.du.car}{\verboinum{2}}
\verb{educativo}{}{}{}{}{adj.}{Que educa; instrutivo.}{e.du.ca.ti.vo}{0}
\verb{edulcorante}{}{}{}{}{adj.2g.}{Que adoça; adoçante.}{e.dul.co.ran.te}{0}
\verb{edulcorar}{}{}{}{}{v.t.}{Tornar doce; adoçar.}{e.dul.co.rar}{\verboinum{1}}
\verb{edule}{}{}{}{}{adj.2g.}{Próprio para comer; comestível.}{e.du.le}{0}
\verb{efe}{é}{}{}{}{s.m.}{Nome da letra \textit{f}.}{e.fe}{0}
\verb{efeito}{ê}{}{}{}{s.m.}{Resultado de uma causa. (\textit{Os desmoronamentos foram efeito da tempestade.})}{e.fei.to}{0}
\verb{efeito}{ê}{}{}{}{}{Execução, realização, efetivação. (\textit{Precisamos levar a efeito os nossos planos.})}{e.fei.to}{0}
\verb{efeito}{ê}{}{}{}{}{Impressão, sensação. (\textit{A perspectiva produz o efeito de profundidade.})}{e.fei.to}{0}
\verb{efeméride}{}{Astron.}{}{}{s.f.}{Tábua astronômica que indica a posição relativa dos astros em intervalos regulares.}{e.fe.mé.ri.de}{0}
\verb{efeméride}{}{}{}{}{}{Registro dos fatos diários.}{e.fe.mé.ri.de}{0}
\verb{efeméride}{}{Bras.}{}{}{}{Notícia diária.}{e.fe.mé.ri.de}{0}
\verb{efeméride}{}{}{}{}{}{Livro em que se registram fatos ocorridos no mesmo dia, mas em anos diferentes.}{e.fe.mé.ri.de}{0}
\verb{efêmero}{}{}{}{}{adj.}{Transitório, passageiro, temporário.}{e.fê.me.ro}{0}
\verb{efeminado}{}{}{}{}{adj.}{Diz"-se de homem que tem aparência ou gestos femininos.}{e.fe.mi.na.do}{0}
\verb{efeminar}{}{}{}{}{v.t.}{Dar aparência ou gestos femininos.}{e.fe.mi.nar}{\verboinum{1}}
\verb{efervescência}{}{}{}{}{s.f.}{Formação de bolhas de gás dentro de um líquido.}{e.fer.ves.cên.cia}{0}
\verb{efervescência}{}{Fig.}{}{}{}{Agitação do espírito; exaltação, comoção, inquietação.}{e.fer.ves.cên.cia}{0}
\verb{efervescente}{}{}{}{}{adj.2g.}{Que apresenta efervescência.}{e.fer.ves.cen.te}{0}
\verb{efervescer}{ê}{}{}{}{v.i.}{Entrar em efervescência.}{e.fer.ves.cer}{\verboinum{15}}
\verb{efetivar}{}{}{}{}{v.t.}{Tornar efetivo; realizar.}{e.fe.ti.var}{\verboinum{1}}
\verb{efetivo}{}{}{}{}{adj.}{Real, verdadeiro, positivo.}{e.fe.ti.vo}{0}
\verb{efetivo}{}{}{}{}{}{Estável, permanente, fixo.}{e.fe.ti.vo}{0}
\verb{efetivo}{}{}{}{}{s.m.}{O número de militares que compõem uma tropa.}{e.fe.ti.vo}{0}
\verb{efetuar}{}{}{}{}{v.t.}{Levar a efeito; executar, realizar.}{e.fe.tu.ar}{\verboinum{1}}
\verb{eficácia}{}{}{}{}{s.f.}{Qualidade de produzir bons resultados.}{e.fi.cá.cia}{0}
\verb{eficaz}{}{}{}{}{adj.2g.}{Que produz o resultado desejado; bom.}{e.fi.caz}{0}
\verb{eficiência}{}{}{}{}{s.f.}{Ação, força, virtude de produzir um efeito; eficácia.}{e.fi.ci.ên.cia}{0}
\verb{eficiente}{}{}{}{}{adj.2g.}{Que realiza bem suas funções.}{e.fi.ci.en.te}{0}
\verb{eficiente}{}{}{}{}{}{Que traz bons resultados.}{e.fi.ci.en.te}{0}
\verb{efígie}{}{}{}{}{s.f.}{Imagem, figura, retrato de pessoa ou personagem.}{e.fí.gie}{0}
\verb{eflorescência}{}{Bot.}{}{}{s.f.}{Formação e surgimento da flor.}{e.flo.res.cên.cia}{0}
\verb{eflorescer}{ê}{}{}{}{v.i.}{Começar a florescer.}{e.flo.res.cer}{0}
\verb{eflorescer}{ê}{}{}{}{}{Apresentar eflorescência.}{e.flo.res.cer}{\verboinum{15}}
\verb{efluência}{}{}{}{}{s.f.}{Eflúvio.}{e.flu.ên.cia}{0}
\verb{efluente}{}{}{}{}{adj.2g.}{Que emana de certos corpos invisivelmente.}{e.flu.en.te}{0}
\verb{efluir}{}{}{}{}{v.i.}{Irradiar de um ponto; emanar, correr.}{e.flu.ir}{\verboinum{26}}
\verb{eflúvio}{}{}{}{}{s.m.}{Emanação de um fluido; exalação, efluência.}{e.flú.vio}{0}
\verb{eflúvio}{}{Fig.}{}{}{}{Aroma, perfume.}{e.flú.vio}{0}
\verb{efundir}{}{}{}{}{v.t.}{Lançar em dispersão; derramar, entortar.}{e.fun.dir}{0}
\verb{efundir}{}{}{}{}{}{Tornar público; transmitir, propagar.}{e.fun.dir}{\verboinum{18}}
\verb{efusão}{}{}{"-ões}{}{s.f.}{Derramamento de líquido ou expansão de gás.}{e.fu.são}{0}
\verb{efusão}{}{Fig.}{"-ões}{}{}{Entusiasmo.}{e.fu.são}{0}
\verb{efusivo}{}{}{}{}{adj.}{Em que há efusão.}{e.fu.si.vo}{0}
\verb{efusivo}{}{}{}{}{}{Comunicativo, caloroso, expansivo.}{e.fu.si.vo}{0}
\verb{égide}{}{}{}{}{s.f.}{Escudo, proteção, defesa.}{é.gi.de}{0}
\verb{egípcio}{}{}{}{}{adj.}{Relativo ao Egito.}{e.gíp.cio}{0}
\verb{egípcio}{}{}{}{}{s.m.}{Indivíduo natural ou habitante desse país.}{e.gíp.cio}{0}
\verb{egiptologia}{}{Arqueol.}{}{}{s.f.}{Ciência que estuda o antigo Egito.}{e.gip.to.lo.gi.a}{0}
\verb{egiptólogo}{}{}{}{}{s.m.}{Especialista em egiptologia.}{e.gip.tó.lo.go}{0}
\verb{égloga}{}{}{}{}{}{Var. de \textit{écloga}.}{é.glo.ga}{0}
\verb{ego}{é}{}{}{}{s.m.}{O núcleo da personalidade de uma pessoa.}{e.go}{0}
\verb{egocêntrico}{}{}{}{}{adj.}{Diz"-se de quem se vê como centro de tudo.}{e.go.cên.tri.co}{0}
\verb{egocentrismo}{}{}{}{}{s.m.}{Sentimento de pensar em si mesmo em primeiro lugar; egoísmo.}{e.go.cen.tris.mo}{0}
\verb{egoísmo}{}{}{}{}{s.m.}{Apego exclusivo aos próprios interesses.}{e.go.ís.mo}{0}
\verb{egoísta}{}{}{}{}{adj.2g.}{Que se preocupa somente consigo mesmo.}{e.go.ís.ta}{0}
\verb{egrégio}{}{}{}{}{adj.}{Extremamente distinto; insigne, muito importante.}{e.gré.gio}{0}
\verb{egrégio}{}{}{}{}{}{Digno de admiração; notável.}{e.gré.gio}{0}
\verb{egresso}{é}{}{}{}{adj.}{Que saiu ou se afastou.}{e.gres.so}{0}
\verb{egresso}{é}{}{}{}{}{Que deixou de fazer parte de uma comunidade.}{e.gres.so}{0}
\verb{egresso}{é}{}{}{}{s.m.}{Indivíduo que deixou convento, hospital etc.}{e.gres.so}{0}
\verb{égua}{}{}{}{}{s.f.}{Fêmea do cavalo.}{é.gua}{0}
\verb{eh}{ê}{}{}{}{interj.}{Expressão usada para chamar a atenção.}{eh}{0}
\verb{ei}{}{}{}{}{interj.}{Expressão que denota chamamento.}{ei}{0}
\verb{eia}{ê}{}{}{}{interj.}{Expressão que denota ânimo, excitação.}{ei.a}{0}
\verb{eia}{ê}{}{}{}{}{Expressão que denota surpresa, espanto.}{ei.a}{0}
\verb{einstêinio}{}{}{}{}{}{Var. de \textit{einstênio}.}{eins.têi.nio}{0}
\verb{einstênio}{}{Quím.}{}{}{s.m.}{Elemento químico radioativo, do grupo dos actinídeos, obtido artificialmente. \elemento{99}{(252)}{Es}.}{eins.tê.nio}{0}
\verb{eira}{ê}{}{}{}{s.f.}{Terreno lajeado onde se malham, secam, trilham e limpam cereais.}{ei.ra}{0}
\verb{eis}{}{}{}{}{adv.}{Aqui está; veja.}{eis}{0}
\verb{eita}{ê}{}{}{}{interj.}{Expressão que denota alegria, espanto.}{ei.ta}{0}
\verb{eito}{ê}{Bras.}{}{}{s.m.}{Limpeza de uma plantação, feita com enxada, por turmas.}{ei.to}{0}
\verb{eito}{ê}{}{}{}{}{Roça onde trabalhavam escravos.}{ei.to}{0}
\verb{eiva}{ê}{}{}{}{s.f.}{Rachadura em louça ou vidro.}{ei.va}{0}
\verb{eiva}{ê}{}{}{}{}{Falha em metais.}{ei.va}{0}
\verb{eiva}{ê}{}{}{}{}{Começo de podridão numa fruta.}{ei.va}{0}
\verb{eiva}{ê}{}{}{}{}{Defeito físico ou moral.}{ei.va}{0}
\verb{eivar}{}{}{}{}{v.t.}{Produzir mancha.}{ei.var}{0}
\verb{eivar}{}{}{}{}{}{Contaminar, infectar física ou moralmente.}{ei.var}{0}
\verb{eivar}{}{}{}{}{v.i.}{Estar a terra com eiva.}{ei.var}{0}
\verb{eivar}{}{}{}{}{v.pron.}{Principiar a apodrecer.}{ei.var}{0}
\verb{eivar}{}{}{}{}{}{Rachar"-se.}{ei.var}{0}
\verb{eivar}{}{}{}{}{}{Enfraquecer"-se, debilitar"-se.}{ei.var}{\verboinum{1}}
\verb{eixo}{ch}{}{}{}{s.m.}{Reta no centro de um corpo.}{ei.xo}{0}
\verb{eixo}{ch}{}{}{}{}{Peça em torno da qual gira um corpo.}{ei.xo}{0}
\verb{eixo}{ch}{}{}{}{}{Ponto ou ideia central.}{ei.xo}{0}
\verb{ejaculação}{}{}{"-ões}{}{s.f.}{Ato ou efeito de ejacular; emissão de esperma pela uretra no momento do orgasmo.}{e.ja.cu.la.ção}{0}
\verb{ejaculação}{}{}{"-ões}{}{}{Forte derramamento de líquido; jato.}{e.ja.cu.la.ção}{0}
\verb{ejacular}{}{}{}{}{v.i.}{Emitir esperma.}{e.ja.cu.lar}{0}
\verb{ejacular}{}{}{}{}{v.t.}{Derramar líquido fartamente.}{e.ja.cu.lar}{\verboinum{1}}
\verb{ejaculatório}{}{}{}{}{adj.}{Que propicia a ejaculação.}{e.ja.cu.la.tó.rio}{0}
\verb{ejeção}{}{}{"-ões}{}{s.f.}{Ato ou efeito de ejetar; expulsão.}{e.je.ção}{0}
\verb{ejetar}{}{}{}{}{v.t.}{Fazer a ejeção; expulsar.}{e.je.tar}{\verboinum{1}}
\verb{ejetor}{ô}{}{}{}{adj.}{Que ejeta.}{e.je.tor}{0}
\verb{ejetor}{ô}{}{}{}{}{Mecanismo ou dispositivo que ejeta.}{e.je.tor}{0}
\verb{ela}{é}{}{}{}{pron.}{Pronome pessoal feminino da terceira pessoa do singular.}{e.la}{0}
\verb{elaboração}{}{}{"-ões}{}{s.f.}{Ato ou efeito de elaborar; produção, feitura.}{e.la.bo.ra.ção}{0}
\verb{elaboração}{}{}{"-ões}{}{}{Preparação cuidadosa.}{e.la.bo.ra.ção}{0}
\verb{elaborado}{}{}{}{}{adj.}{Que sofreu elaboração; produzido com muito cuidado.}{e.la.bo.ra.do}{0}
\verb{elaborar}{}{}{}{}{v.t.}{Produzir alguma coisa; preparar.}{e.la.bo.rar}{\verboinum{1}}
\verb{elasticidade}{}{Fís.}{}{}{s.f.}{Propriedade de um corpo sofrer deformação, quando submetido à tração, e retornar parcial ou totalmente à forma original.}{e.las.ti.ci.da.de}{0}
\verb{elasticidade}{}{Por ext.}{}{}{}{Flexibilidade; agilidade física.}{e.las.ti.ci.da.de}{0}
\verb{elástico}{}{}{}{}{adj.}{Que pode esticar e encolher.}{e.lás.ti.co}{0}
\verb{elástico}{}{}{}{}{s.m.}{Tira de borracha que serve para prender objetos.}{e.lás.ti.co}{0}
\verb{eldorado}{}{}{}{}{s.m.}{Cidade ou país fictício que exploradores afirmavam existir na América do Sul.}{el.do.ra.do}{0}
\verb{eldorado}{}{}{}{}{}{Local cheio de riquezas e oportunidades.}{el.do.ra.do}{0}
\verb{ele}{é}{}{}{}{s.m.}{Nome da letra \textit{l}.}{e.le}{0}
\verb{ele}{ê}{}{}{}{pron.}{Pronome pessoal da terceira pessoal do singular.}{e.le}{0}
\verb{elefante}{}{Zool.}{}{}{s.m.}{Mamífero de grande tamanho, de pele muito grossa, que tem uma tromba no lugar de lábio superior e dois dentes que avançam para fora da boca.}{e.le.fan.te}{0}
\verb{elefantíase}{}{Med.}{}{}{s.f.}{Doença inflamatória que obstrui os vasos linfáticos, causada por filária e transmitida por mosquitos; filariose.  }{e.le.fan.tí.a.se}{0}
\verb{elefantino}{}{}{}{}{adj.}{Relativo ou semelhante a elefante.}{e.le.fan.ti.no}{0}
\verb{elegância}{}{}{}{}{s.f.}{Graça e distinção no porte e nos modos.}{e.le.gân.cia}{0}
\verb{elegante}{}{}{}{}{adj.2g.}{Em que se nota encanto e bom gosto.}{e.le.gan.te}{0}
\verb{eleger}{ê}{}{}{}{v.t.}{Escolher por processo de votação.}{e.le.ger}{0}
\verb{eleger}{ê}{}{}{}{}{Preferir entre duas ou mais opções; escolher.}{e.le.ger}{\verboinum{16}}
\verb{elegia}{}{}{}{}{s.f.}{Poema lírico, geralmente em tom triste.}{e.le.gi.a}{0}
\verb{elegíaco}{}{}{}{}{adj.}{Relativo a elegia.}{e.le.gí.a.co}{0}
\verb{elegíaco}{}{}{}{}{}{Triste, lamentoso.}{e.le.gí.a.co}{0}
\verb{elegíaco}{}{}{}{}{}{Chorão.}{e.le.gí.a.co}{0}
\verb{elegibilidade}{}{}{}{}{s.f.}{Qualidade de elegível.}{e.le.gi.bi.li.da.de}{0}
\verb{elegível}{}{}{"-eis}{}{adj.2g.}{Que pode ser eleito.}{e.le.gí.vel}{0}
\verb{eleição}{}{}{"-ões}{}{s.f.}{Ato ou efeito de eleger.}{e.lei.ção}{0}
\verb{eleição}{}{}{"-ões}{}{}{Escolha, através de voto, de pessoa para ocupar cargo.}{e.lei.ção}{0}
\verb{eleito}{ê}{}{}{}{adj.}{Que foi escolhido.}{e.lei.to}{0}
\verb{eleito}{ê}{}{}{}{}{Preferido.}{e.lei.to}{0}
\verb{eleito}{ê}{Relig.}{}{}{s.m.}{Indivíduo escolhido pela vontade divina.}{e.lei.to}{0}
\verb{eleitor}{ô}{}{}{}{adj.}{Que tem o direito de eleger ou escolher.}{e.lei.tor}{0}
\verb{eleitorado}{}{}{}{}{s.m.}{O conjunto dos eleitores.}{e.lei.to.ra.do}{0}
\verb{eleitoral}{}{}{"-ais}{}{adj.2g.}{Relativo a eleição.}{e.lei.to.ral}{0}
\verb{eleitoreiro}{ê}{Bras.}{}{}{adj.}{Diz"-se de atos ou discursos com finalidade apenas de obter votos em detrimento dos interesses comuns.}{e.lei.to.rei.ro}{0}
\verb{elementar}{}{}{}{}{adj.2g.}{Relativo a elemento.}{e.le.men.tar}{0}
\verb{elementar}{}{}{}{}{}{De natureza simples; rudimentar.}{e.le.men.tar}{0}
\verb{elementar}{}{}{}{}{}{Fácil, simples.}{e.le.men.tar}{0}
\verb{elemento}{}{}{}{}{}{Cada um dos componentes do universo físico segundo a ciência antiga: água, fogo, terra e ar.}{e.le.men.to}{0}
\verb{elemento}{}{}{}{}{}{Informação, meio, recurso. (\textit{Não tenho elementos para supor outra coisa.})}{e.le.men.to}{0}
\verb{elemento}{}{}{}{}{s.m.}{Cada parte de um todo.}{e.le.men.to}{0}
\verb{elementos}{}{}{}{}{s.m.pl.}{Noções fundamentais; rudimentos. (\textit{Elementos de geometria.})}{e.le.men.tos}{0}
\verb{elenco}{}{}{}{}{s.m.}{O conjunto dos atores de um espetáculo.}{e.len.co}{0}
\verb{elenco}{}{}{}{}{}{Lista, enumeração, rol.}{e.len.co}{0}
\verb{elepê}{}{}{}{}{s.m.}{Abreviação do termo inglês \textit{long"-play} (\textsc{lp}), disco de 10 ou 12 polegadas de diâmetro, com gravação fonográfica feita em microssulcos no material plástico, geralmente vinil.}{e.le.pê}{0}
\verb{eletivo}{}{}{}{}{adj.}{Relativo a eleição.}{e.le.ti.vo}{0}
\verb{eletivo}{}{}{}{}{}{Que é objeto de escolha, preferência.}{e.le.ti.vo}{0}
\verb{eletracústica}{}{Fís.}{}{}{s.f.}{Ramo da física que estuda as transformações entre energia elétrica e sonora.}{e.le.tra.cús.ti.ca}{0}
\verb{eletricidade}{}{Fís.}{}{}{s.f.}{Fenômeno natural que envolve cargas elétricas estacionárias ou em movimento.}{e.le.tri.ci.da.de}{0}
\verb{eletricista}{}{}{}{}{s.2g.}{Profissional que faz manutenção e reparos em aparelhos e redes elétricas.}{e.le.tri.cis.ta}{0}
\verb{elétrico}{}{}{}{}{adj.}{Relativo a eletricidade.}{e.lé.tri.co}{0}
\verb{elétrico}{}{}{}{}{}{Que utiliza eletricidade como fonte de energia.}{e.lé.tri.co}{0}
\verb{elétrico}{}{Fig.}{}{}{}{Rápido, agitado, nervoso.}{e.lé.tri.co}{0}
\verb{eletrificação}{}{}{"-ões}{}{s.f.}{Ato ou efeito de eletrificar.}{e.le.tri.fi.ca.ção}{0}
\verb{eletrificar}{}{}{}{}{v.t.}{Colocar corrente elétrica. (\textit{O dono mandou eletrificar todas as cercas.})}{e.le.tri.fi.car}{0}
\verb{eletrificar}{}{}{}{}{}{Prover de instalações elétricas. (\textit{A prefeitura vai eletrificar toda a área rural.})}{e.le.tri.fi.car}{\verboinum{1}}
\verb{eletrizante}{}{}{}{}{adj.2g.}{Que eletriza.}{e.le.tri.zan.te}{0}
\verb{eletrizar}{}{}{}{}{v.t.}{Carregar com carga elétrica; eletrificar.}{e.le.tri.zar}{0}
\verb{eletrizar}{}{Fig.}{}{}{}{Maravilhar, encantar, arrebatar.}{e.le.tri.zar}{0}
\verb{eletrizar}{}{}{}{}{v.pron.}{Entusiasmar"-se; excitar"-se.}{e.le.tri.zar}{\verboinum{1}}
\verb{eletroacústica}{}{}{}{}{}{Var. de \textit{eletracústica}.}{e.le.tro.a.cús.ti.ca}{0}
\verb{eletroacústico}{}{}{}{}{adj.}{Relativo à eletroacústica.}{e.le.tro.a.cús.ti.co}{0}
\verb{eletroacústico}{}{Mús.}{}{}{}{Diz"-se de música produzida ou manipulada por dispositivos eletrônicos.}{e.le.tro.a.cús.ti.co}{0}
\verb{eletrocardiografia}{}{Med.}{}{}{s.f.}{Análise de um eletrocardiograma a fim de diagnosticar distúrbios cardíacos.}{e.le.tro.car.di.o.gra.fi.a}{0}
\verb{eletrocardiógrafo}{}{Med.}{}{}{s.m.}{Aparelho que capta e registra graficamente as correntes geradas no músculo cardíaco.}{e.le.tro.car.di.ó.gra.fo}{0}
\verb{eletrocardiograma}{}{Med.}{}{}{s.m.}{Registro gráfico das correntes originadas no músculo cardíaco.}{e.le.tro.car.di.o.gra.ma}{0}
\verb{eletrocoagulação}{}{Med.}{"-ões}{}{s.f.}{Método de coagulação que utiliza corrente elétrica de alta frequência.}{e.le.tro.co.a.gu.la.ção}{0}
\verb{eletrocução}{}{}{"-ões}{}{s.f.}{Ato ou efeito de eletrocutar.}{e.le.tro.cu.ção}{0}
\verb{eletrocução}{}{}{"-ões}{}{}{Morte causada por descarga elétrica.}{e.le.tro.cu.ção}{0}
\verb{eletrocução}{}{}{"-ões}{}{}{Meio de execução de pena de morte.}{e.le.tro.cu.ção}{0}
\verb{eletrocussão}{}{}{}{}{}{Var. de \textit{eletrocução}.}{e.le.tro.cus.são}{0}
\verb{eletrocutar}{}{}{}{}{v.t.}{Matar por descarga elétrica.}{e.le.tro.cu.tar}{0}
\verb{eletrocutar}{}{}{}{}{}{Executar pena de morte em cadeira elétrica.}{e.le.tro.cu.tar}{\verboinum{1}}
\verb{eletrodinâmica}{}{Fís.}{}{}{s.f.}{Ramo da física que estuda as cargas elétricas e os campos eletromagnéticos.}{e.le.tro.di.nâ.mi.ca}{0}
\verb{eletrodinâmico}{}{}{}{}{adj.}{Relativo à eletrodinâmica.}{e.le.tro.di.nâ.mi.co}{0}
\verb{eletrodo}{ô}{Fís.}{}{}{s.m.}{Condutor metálico que introduz energia elétrica em um sistema.}{e.le.tro.do}{0}
\verb{elétrodo}{}{}{}{}{s.m.}{Eletrodo.}{e.lé.tro.do}{0}
\verb{eletrodoméstico}{}{}{}{}{s.m.}{Qualquer aparelho elétrico de uso residencial, como televisor, espremedor de frutas, lavadora de roupas.}{e.le.tro.do.més.ti.co}{0}
\verb{eletroeletrônico}{}{}{}{}{s.m.}{Aparelho eletrônico doméstico, como televisor, videocassete, aparelho de som.}{e.le.tro.e.le.trô.ni.co}{0}
\verb{eletroeletrônico}{}{}{}{}{adj.}{Referente aos eletroeletrônicos. }{e.le.tro.e.le.trô.ni.co}{0}
\verb{eletroencefalografia}{}{Med.}{}{}{s.f.}{Análise de um eletroencefalograma a fim de diagnosticar problemas cerebrais.}{e.le.tro.en.ce.fa.lo.gra.fi.a}{0}
\verb{eletroencefalógrafo}{}{Med.}{}{}{s.m.}{Aparelho que capta e registra graficamente as variações do potencial elétrico do cérebro.}{e.le.tro.en.ce.fa.ló.gra.fo}{0}
\verb{eletroencefalograma}{}{Med.}{}{}{s.m.}{Registro gráfico das variações de potencial elétrico do cérebro.}{e.le.tro.en.ce.fa.lo.gra.ma}{0}
\verb{eletroímã}{}{}{}{}{s.m.}{Ímã acionado por corrente elétrica.}{e.le.tro.í.mã}{0}
\verb{eletrola}{ó}{}{}{}{s.f.}{Aparelho que reproduz sons gravados através de microssulcos em discos de material plástico.}{e.le.tro.la}{0}
\verb{eletrolisar}{}{}{}{}{v.t.}{Fazer eletrólise.}{e.le.tro.li.sar}{\verboinum{1}}
\verb{eletrólise}{}{Quím.}{}{}{s.f.}{Técnica de utilizar corrente elétrica para produzir reações químicas de diversas naturezas, como oxidações, reduções e decomposições.}{e.le.tró.li.se}{0}
\verb{eletrolítico}{}{}{}{}{adj.}{Relativo a eletrólise.}{e.le.tro.lí.ti.co}{0}
\verb{eletrólito}{}{Fís.}{}{}{s.m.}{Substância líquida ou sólida que conduz energia elétrica por meio de íons.}{e.le.tró.li.to}{0}
\verb{eletromagnético}{}{}{}{}{adj.}{Relativo a eletromagnetismo.}{e.le.tro.mag.né.ti.co}{0}
\verb{eletromagnetismo}{}{Fís.}{}{}{s.m.}{Ramo da física que estuda as propriedades dos campos elétricos.}{e.le.tro.mag.ne.tis.mo}{0}
\verb{eletromagneto}{é}{Fís.}{}{}{s.m.}{Eletroímã.}{e.le.tro.mag.ne.to}{0}
\verb{eletromecânico}{}{}{}{}{adj.}{Diz"-se de dispositivos mecânicos acionados por comandos elétricos.}{e.le.tro.me.câ.ni.co}{0}
\verb{eletrômetro}{}{Fís.}{}{}{s.m.}{Instrumento para medir corrente, potencial e carga elétrica.}{e.le.trô.me.tro}{0}
\verb{elétron}{}{Fís.}{}{}{s.m.}{A menor partícula constituinte do átomo, com carga elétrica negativa.}{e.lé.tron}{0}
\verb{eletrônica}{}{Fís.}{}{}{s.f.}{Ramo da física que estuda os circuitos elétricos que utilizam semicondutores.}{e.le.trô.ni.ca}{0}
\verb{eletrônica}{}{}{}{}{}{Estudo e técnica de projetar e construir circuitos elétricos que utilizem semicondutores.}{e.le.trô.ni.ca}{0}
\verb{eletrônico}{}{}{}{}{adj.}{Relativo à eletrônica.}{e.le.trô.ni.co}{0}
\verb{eletroquímica}{}{Fís. e Quím.}{}{}{s.f.}{Ramo da físico"-química que estuda os íons e as propriedades elétricas das soluções.}{e.le.tro.quí.mi.ca}{0}
\verb{eletroquímico}{}{}{}{}{adj.}{Relativo à eletroquímica.}{e.le.tro.quí.mi.co}{0}
\verb{eletroscópio}{}{Fís.}{}{}{s.m.}{Instrumento que permite a observação indireta de cargas elétricas através do movimento de peças metálicas.}{e.le.tros.có.pio}{0}
\verb{eletrostática}{}{Fís.}{}{}{s.f.}{Ramo da física que estuda as cargas elétricas estacionárias.}{e.le.tros.tá.ti.ca}{0}
\verb{eletrostático}{}{}{}{}{adj.}{Relativo à eletrostática.}{e.le.tros.tá.ti.co}{0}
\verb{eletrotecnia}{}{}{}{}{s.f.}{Conjunto das técnicas de aplicação da eletricidade.}{e.le.tro.tec.ni.a}{0}
\verb{eletroterapia}{}{Med.}{}{}{s.f.}{Terapia que utiliza eletricidade como recurso de tratamento de certas doenças.}{e.le.tro.te.ra.pi.a}{0}
\verb{elevação}{}{}{"-ões}{}{s.f.}{Ato ou efeito de elevar.}{e.le.va.ção}{0}
\verb{elevação}{}{}{"-ões}{}{}{Aumento, acréscimo, alta.}{e.le.va.ção}{0}
\verb{elevação}{}{}{"-ões}{}{}{Ponto elevado; altura.}{e.le.va.ção}{0}
\verb{elevado}{}{}{}{}{adj.}{Que se elevou ou que tem elevação.}{e.le.va.do}{0}
\verb{elevado}{}{}{}{}{}{Nobre, superior, transcendente, grande, alto.}{e.le.va.do}{0}
\verb{elevado}{}{}{}{}{s.m.}{Via urbana em nível superior ao do solo.}{e.le.va.do}{0}
\verb{elevador}{ô}{}{}{}{adj.}{Que eleva.}{e.le.va.dor}{0}
\verb{elevador}{ô}{}{}{}{s.m.}{Máquina com cabina ou plataforma que se desloca na vertical para levar pessoas ou cargas aos diversos pavimentos de um edifício.}{e.le.va.dor}{0}
\verb{elevar}{}{}{}{}{v.t.}{Colocar em posição superior; levantar, erguer.}{e.le.var}{0}
\verb{elevar}{}{}{}{}{}{Aumentar em número ou valor.}{e.le.var}{0}
\verb{elevar}{}{}{}{}{}{Aumentar o volume ou tom (da voz).}{e.le.var}{\verboinum{1}}
\verb{elevatória}{}{}{}{}{s.f.}{Redução de \textit{estação elevatória}, sistema que serve para levar a água para reservatórios localizados em um nível superior àquele em que ela se encontra.}{e.le.va.tó.ria}{0}
\verb{elevatório}{}{}{}{}{adj.}{Relativo a elevação.}{e.le.va.tó.rio}{0}
\verb{elevatório}{}{}{}{}{}{Que serve para elevar.}{e.le.va.tó.rio}{0}
\verb{elidir}{}{}{}{}{v.t.}{Eliminar, suprimir.}{e.li.dir}{\verboinum{18}}
\verb{eliminação}{}{}{"-ões}{}{s.f.}{Ato ou efeito de eliminar.}{e.li.mi.na.ção}{0}
\verb{eliminar}{}{}{}{}{v.t.}{Suprimir, excluir, tirar.}{e.li.mi.nar}{0}
\verb{eliminar}{}{}{}{}{}{Fazer sair do organismo.}{e.li.mi.nar}{0}
\verb{eliminar}{}{}{}{}{}{Matar, extinguir.}{e.li.mi.nar}{\verboinum{1}}
\verb{eliminatória}{}{}{}{}{s.f.}{Exame ou competição cujo resultado provoca a eliminação do concorrente.}{e.li.mi.na.tó.ria}{0}
\verb{eliminatório}{}{}{}{}{adj.}{Que tem por efeito eliminar; seletivo.}{e.li.mi.na.tó.rio}{0}
\verb{elipse}{}{Geom.}{}{}{s.f.}{Lugar geométrico dos pontos de um plano cuja soma das distâncias aos focos desse plano é constante.}{e.lip.se}{0}
\verb{elipse}{}{Gram.}{}{}{}{Supressão de uma ou mais palavras sem prejudicar o sentido do enunciado.}{e.lip.se}{0}
\verb{elíptico}{}{}{}{}{adj.}{Relativo a elipse.}{e.líp.ti.co}{0}
\verb{elíptico}{}{}{}{}{}{Que tem forma de elipse.}{e.líp.ti.co}{0}
\verb{elisão}{}{}{"-ões}{}{s.f.}{Ato ou efeito de elidir; supressão, eliminação.}{e.li.são}{0}
\verb{elisão}{}{Gram.}{"-ões}{}{}{Processo fonológico de supressão de uma vogal átona, quando há encontro entre duas vogais diferentes.}{e.li.são}{0}
\verb{elite}{}{}{}{}{s.f.}{A parte de uma sociedade considerada superior segundo critérios específicos, como cultural, intelectual e, mais frequentemente, econômico.}{e.li.te}{0}
\verb{elite}{}{}{}{}{}{Minoria privilegiada e dominante da sociedade.}{e.li.te}{0}
\verb{elítico}{}{}{}{}{}{Var. de \textit{elíptico}.}{e.lí.ti.co}{0}
\verb{elitismo}{}{}{}{}{s.m.}{Política, ideologia ou prática social em que há o favorecimento de uma elite.}{e.li.tis.mo}{0}
\verb{elitista}{}{}{}{}{adj.2g.}{Relativo a elitismo.}{e.li.tis.ta}{0}
\verb{elitista}{}{}{}{}{}{Adepto do elitismo.}{e.li.tis.ta}{0}
\verb{élitro}{}{Zool.}{}{}{s.m.}{Asa anterior sem nervuras dos besouros.}{é.li.tro}{0}
\verb{elixir}{ch}{}{}{}{s.m.}{Bebida aromática ou medicamento dissolvido em álcool.}{e.li.xir}{0}
\verb{elmo}{é}{}{}{}{s.m.}{Peça da armadura medieval que protege a cabeça; capacete.}{el.mo}{0}
\verb{elo}{é}{}{}{}{s.m.}{Cada argola de uma corrente.}{e.lo}{0}
\verb{elo}{é}{Fig.}{}{}{}{Ligação, união.}{e.lo}{0}
\verb{elocução}{}{}{"-ões}{}{s.f.}{Modo de expressar"-se, tanto em língua oral como em língua escrita; estilo.}{e.lo.cu.ção}{0}
\verb{elogiar}{}{}{}{}{v.t.}{Fazer elogios; louvar.}{e.lo.gi.ar}{\verboinum{6}}
\verb{elogio}{}{}{}{}{s.m.}{Julgamento favorável a algo ou alguém, destacando suas qualidades positivas.}{e.lo.gi.o}{0}
\verb{elogio}{}{}{}{}{}{O discurso, oral ou escrito, em que se expressa esse julgamento favorável.}{e.lo.gi.o}{0}
\verb{elogioso}{ô}{}{"-osos ⟨ó⟩}{"-osa ⟨ó⟩}{adj.}{Em que há elogio.}{e.lo.gi.o.so}{0}
\verb{eloquência}{}{}{}{}{s.f.}{Capacidade de exprimir"-se com facilidade e desenvoltura.}{e.lo.quên.cia}{0}
\verb{eloquência}{}{}{}{}{}{Capacidade de persuadir através da palavra.}{e.lo.quên.cia}{0}
\verb{eloquente}{}{}{}{}{adj.2g.}{Que tem eloquência.}{e.lo.quen.te}{0}
\verb{elucidação}{}{}{"-ões}{}{s.f.}{Ato de elucidar.}{e.lu.ci.da.ção}{0}
\verb{elucidar}{}{}{}{}{v.t.}{Explicar, esclarecer.}{e.lu.ci.dar}{\verboinum{1}}
\verb{elucidário}{}{}{}{}{s.m.}{Livro que esclarece coisas ininteligíveis.}{e.lu.ci.dá.rio}{0}
\verb{elucidativo}{}{}{}{}{adj.}{Que elucida, esclarece.}{e.lu.ci.da.ti.vo}{0}
\verb{elucubração}{}{}{"-ões}{}{s.f.}{Reflexão profunda; meditação.}{e.lu.cu.bra.ção}{0}
\verb{elucubração}{}{Desus.}{"-ões}{}{}{Trabalho prolongado feito à noite, sob luz artificial.}{e.lu.cu.bra.ção}{0}
\verb{elucubrar}{}{}{}{}{v.t.}{Refletir demoradamente sobre algo.}{e.lu.cu.brar}{0}
\verb{elucubrar}{}{Desus.}{}{}{}{Trabalhar à noite sob luz artificial.}{e.lu.cu.brar}{\verboinum{1}}
\verb{em}{}{}{}{}{prep.}{Indica o lugar onde, meio, modo, sucessão, tempo, causa, estado, fim, divisão etc.}{em}{0}
\verb{ema}{}{Zool.}{}{}{s.f.}{Ave de grande porte, sem cauda, de cabeça e pescoço cinzentos e corpo branco; nhandu.}{e.ma}{0}
\verb{emaçar}{}{}{}{}{v.t.}{Formar maços.}{e.ma.çar}{0}
\verb{emaçar}{}{}{}{}{}{Embrulhar em papel.}{e.ma.çar}{\verboinum{3}}
\verb{emaciação}{}{Med.}{"-ões}{}{s.f.}{Emagrecimento.}{e.ma.ci.a.ção}{0}
\verb{emaciado}{}{}{}{}{adj.}{Emagrecido, extenuado.}{e.ma.ci.a.do}{0}
\verb{emaciar}{}{}{}{}{v.t.}{Emagrecer, extenuar.}{e.ma.ci.ar}{\verboinum{6}}
\verb{emagrecer}{ê}{}{}{}{v.t.}{Tornar magro; definhar.}{e.ma.gre.cer}{\verboinum{15}}
\verb{emagrecimento}{}{}{}{}{s.m.}{Ato ou efeito de emagrecer.}{e.ma.gre.ci.men.to}{0}
\verb{e"-mail}{}{}{}{}{s.m.}{Abrev. do inglês \textit{electronic mail}, correio eletrônico; recurso do computador que permite o envio de mensagens eletrônicas por meio de redes de computadores.}{\textit{e"-mail}}{0}
\verb{e"-mail}{}{}{}{}{}{Cada uma dessas mensagens.}{\textit{e"-mail}}{0}
\verb{e"-mail}{}{}{}{}{}{O endereço, na \textit{internet}, para onde uma mensagem eletrônica é enviada; endereço eletrônico.}{\textit{e"-mail}}{0}
\verb{emalar}{}{}{}{}{v.t.}{Colocar em mala.}{e.ma.lar}{\verboinum{1}}
\verb{emanação}{}{}{"-ões}{}{s.f.}{Ato ou efeito de emanar.}{e.ma.na.ção}{0}
\verb{emanação}{}{}{"-ões}{}{}{Origem, procedência.}{e.ma.na.ção}{0}
\verb{emanar}{}{}{}{}{v.t.}{Proceder, sair, ter origem em.}{e.ma.nar}{0}
\verb{emanar}{}{}{}{}{}{Exalar"-se, soltar"-se.}{e.ma.nar}{\verboinum{1}}
\verb{emancipação}{}{}{"-ões}{}{s.f.}{Ato ou efeito de emancipar.}{e.man.ci.pa.ção}{0}
\verb{emancipar}{}{}{}{}{v.t.}{Tornar independente.}{e.man.ci.par}{\verboinum{1}}
\verb{emaranhado}{}{}{}{}{adj.}{Que se apresenta misturado, confuso, embaraçado.}{e.ma.ra.nha.do}{0}
\verb{emaranhar}{}{}{}{}{v.t.}{Misturar de modo confuso; embaraçar, enroscar.}{e.ma.ra.nhar}{\verboinum{1}}
\verb{emasculação}{}{}{"-ões}{}{s.f.}{Ato ou efeito de emascular; castração.}{e.mas.cu.la.ção}{0}
\verb{emascular}{}{}{}{}{v.t.}{Tirar a virilidade, a força; castrar.}{e.mas.cu.lar}{\verboinum{1}}
\verb{emassar}{}{}{}{}{v.t.}{Cobrir ou revestir com massa.}{e.mas.sar}{\verboinum{1}}
\verb{embaçar}{}{}{}{}{v.t.}{Tornar baço, opaco; fazer perder o brilho; empanar.}{em.ba.çar}{\verboinum{3}}
\verb{embaciar}{}{}{}{}{v.t.}{Embaçar.}{em.ba.ci.ar}{\verboinum{1}}
\verb{embainhar}{}{}{}{}{v.t.}{Colocar espada ou punhal na bainha.}{em.ba.i.nhar}{0}
\verb{embainhar}{}{}{}{}{}{Fazer bainha em roupas.}{em.ba.i.nhar}{\verboinum{1}}
\verb{embair}{}{}{}{}{v.t.}{Induzir propositadamente ao erro; enganar, iludir.}{em.ba.ir}{\verboinum{35}\verboirregular{\emph{def.}}}
\verb{embaixada}{ch}{}{}{}{s.f.}{Cargo ou função do embaixador.}{em.bai.xa.da}{0}
\verb{embaixada}{ch}{}{}{}{}{Local onde reside ou trabalha o embaixador.}{em.bai.xa.da}{0}
\verb{embaixada}{ch}{Esport.}{}{}{}{Exercício do jogador de futebol que consiste em não deixar a bola tocar no chão, chutando"-a repetidamente com um dos pés ou equilibrando"-a com a cabeça e outras partes do corpo.}{em.bai.xa.da}{0}
\verb{embaixador}{ch\ldots{}ô}{}{}{}{s.m.}{O mais alto representante de um país junto a outro.}{em.bai.xa.dor}{0}
\verb{embaixadora}{ch\ldots{}ô}{}{}{}{s.f.}{Mulher que ocupa o cargo de embaixador.}{em.bai.xa.do.ra}{0}
\verb{embaixatriz}{ch}{}{}{}{s.f.}{Esposa do embaixador.}{em.bai.xa.triz}{0}
\verb{embaixo}{ch}{}{}{}{adv.}{Situado em lugar inferior; sob.}{em.bai.xo}{0}
\verb{embalado}{}{}{}{}{adj.}{Balançado em berço ou rede; acalentado.}{em.ba.la.do}{0}
\verb{embalado}{}{}{}{}{adj.}{Que foi colocado em embalagem; acondicionado, empacotado.}{em.ba.la.do}{0}
\verb{embalado}{}{Bras.}{}{}{adv.}{Às pressas; rapidamente.}{em.ba.la.do}{0}
\verb{embalado}{}{}{}{}{}{Diz"-se da arma carregada com bala.}{em.ba.la.do}{0}
\verb{embalador}{ô}{}{}{}{adj.}{Que embala ou balança; acalentador.}{em.ba.la.dor}{0}
\verb{embalador}{ô}{}{}{}{adj.}{Que faz pacotes, embrulhos; empacotador.}{em.ba.la.dor}{0}
\verb{embalagem}{}{}{"-ens}{}{s.f.}{Ato ou efeito de embalar; acondicionamento.}{em.ba.la.gem}{0}
\verb{embalagem}{}{}{"-ens}{}{}{O envoltório utilizado para acondicionar uma mercadoria. (\textit{A prefeitura daquela cidade instituiu a coleta seletiva de lixo, que visa principalmente a reciclar as embalagens de plástico, vidro e papelão.})}{em.ba.la.gem}{0}
\verb{embalar}{}{}{}{}{v.t.}{Balançar a criança para acalmá"-la ou fazê"-la dormir; acalentar.}{em.ba.lar}{0}
\verb{embalar}{}{}{}{}{v.t.}{Acondicionar em pacotes; empacotar, embrulhar.}{em.ba.lar}{0}
\verb{embalar}{}{}{}{}{}{Dar impulso; acelerar.}{em.ba.lar}{\verboinum{1}}
\verb{embalar}{}{}{}{}{}{Carregar uma arma com balas.}{em.ba.lar}{\verboinum{1}}
\verb{embalde}{}{}{}{}{adv.}{Em vão; inutilmente, debalde.}{em.bal.de}{0}
\verb{embalo}{}{}{}{}{s.m.}{Ato ou efeito de embalar; balanço.}{em.ba.lo}{0}
\verb{embalo}{}{}{}{}{}{Impulso, ímpeto, precipitação.}{em.ba.lo}{0}
\verb{embalsamamento}{}{}{}{}{s.m.}{Ato ou efeito de embalsamar.}{em.bal.sa.ma.men.to}{0}
\verb{embalsamar}{}{}{}{}{v.t.}{Tratar um cadáver com substâncias preservativas que evitem sua decomposição.}{em.bal.sa.mar}{0}
\verb{embalsamar}{}{}{}{}{}{Impregnar o ambiente com bálsamo e outros perfumes; aromatizar.}{em.bal.sa.mar}{\verboinum{1}}
\verb{embananado}{}{}{}{}{adj.}{Que se embananou; metido em confusão; atrapalhado.}{em.ba.na.na.do}{0}
\verb{embananar}{}{}{}{}{v.t.}{Meter em confusão; complicar, atrapalhar.}{em.ba.na.nar}{\verboinum{1}}
\verb{embandeirar}{}{}{}{}{v.t.}{Enfeitar com bandeiras. (\textit{As crianças embandeiraram a rua para comemorar a vitória da seleção brasileira de futebol.})}{em.ban.dei.rar}{\verboinum{1}}
\verb{embaraçado}{}{}{}{}{adj.}{Que apresenta dificuldades; sem jeito, atrapalhado, confuso.}{em.ba.ra.ça.do}{0}
\verb{embaraçar}{}{}{}{}{v.t.}{Causar embaraço; dificultar, atrapalhar.}{em.ba.ra.çar}{0}
\verb{embaraçar}{}{}{}{}{}{Emaranhar, enredar, enroscar.}{em.ba.ra.çar}{\verboinum{3}}
\verb{embaraço}{}{}{}{}{s.m.}{Aquilo que dificulta ou impede; complicação, atrapalhação.}{em.ba.ra.ço}{0}
\verb{embaraçoso}{ô}{}{"-osos ⟨ó⟩}{"-osa ⟨ó⟩}{adj.}{Que causa embaraço; constrangedor, perturbador.}{em.ba.ra.ço.so}{0}
\verb{embarafustar}{}{}{}{}{v.t.}{Entrar num recinto de forma impetuosa e desordenada.}{em.ba.ra.fus.tar}{\verboinum{1}}
\verb{embaralhamento}{}{}{}{}{s.m.}{Ato ou efeito de embaralhar.}{em.ba.ra.lha.men.to}{0}
\verb{embaralhamento}{}{}{}{}{}{Confusão, mistura, desordem.}{em.ba.ra.lha.men.to}{0}
\verb{embaralhar}{}{}{}{}{v.t.}{Misturar as cartas do baralho.}{em.ba.ra.lhar}{0}
\verb{embaralhar}{}{}{}{}{}{Tirar da ordem; misturar, confundir. }{em.ba.ra.lhar}{\verboinum{1}}
\verb{embarcação}{}{}{"-ões}{}{s.f.}{Qualquer veículo destinado a navegar; barco, navio.}{em.bar.ca.ção}{0}
\verb{embarcadiço}{}{}{}{}{s.m.}{Diz"-se do indivíduo que costuma estar embarcado; marinheiro.}{em.bar.ca.di.ço}{0}
\verb{embarcadoiro}{ô}{}{}{}{}{Var. de \textit{embarcadouro}.}{em.bar.ca.doi.ro}{0}
\verb{embarcadouro}{ô}{}{}{}{s.m.}{Local de embarque e desembarque de passageiros e de carga transportados por navio; cais, porto.}{em.bar.ca.dou.ro}{0}
\verb{embarcar}{}{}{}{}{v.t.}{Colocar a bordo de uma embarcação. (\textit{Eu próprio embarquei a caixa no navio.})}{em.bar.car}{0}
\verb{embarcar}{}{}{}{}{}{Entrar em um veículo. (\textit{Os passageiros embarcaram naquele avião.})}{em.bar.car}{\verboinum{2}}
\verb{embargador}{ô}{}{}{}{adj.}{Que embarga; que coloca embargos.}{em.bar.ga.dor}{0}
\verb{embargar}{}{}{}{}{v.t.}{Pôr embargo; não permitir; impedir. (\textit{A prefeitura embargou a obra por medida de segurança.})}{em.bar.gar}{\verboinum{5}}
\verb{embargo}{}{}{}{}{s.m.}{Aquilo que impede; embaraço, obstáculo, dificuldade.}{em.bar.go}{0}
\verb{embarque}{}{}{}{}{s.m.}{Ato ou efeito de embarcar.}{em.bar.que}{0}
\verb{embarque}{}{}{}{}{}{Lugar onde se embarca. (\textit{Os passageiros se dirigiram ao portão de embarque.})}{em.bar.que}{0}
\verb{embarrigar}{}{Pop.}{}{}{v.i.}{Criar barriga; engordar.}{em.bar.ri.gar}{0}
\verb{embarrigar}{}{Pop.}{}{}{}{Ficar grávida; engravidar.}{em.bar.ri.gar}{\verboinum{5}}
\verb{embasamento}{}{}{}{}{s.m.}{Aquilo que serve de base a uma construção.}{em.ba.sa.men.to}{0}
\verb{embasamento}{}{}{}{}{}{Fundamento, motivo, razão.}{em.ba.sa.men.to}{0}
\verb{embasar}{}{}{}{}{v.t.}{Fixar as bases; fundamentar, alicerçar.}{em.ba.sar}{\verboinum{1}}
\verb{embasbacado}{}{}{}{}{adj.}{Tomado de espanto; estupefato, boquiaberto.}{em.bas.ba.ca.do}{0}
\verb{embasbacar}{}{}{}{}{v.t.}{Causar surpresa ou admiração; pasmar.}{em.bas.ba.car}{0}
\verb{embasbacar}{}{}{}{}{v.i.}{Ficar boquiaberto; espantar"-se.}{em.bas.ba.car}{\verboinum{2}}
\verb{embate}{}{}{}{}{s.m.}{Choque violento; oposição, resistência.}{em.ba.te}{0}
\verb{embater}{ê}{}{}{}{v.t.}{Ir de encontro; chocar.}{em.ba.ter}{\verboinum{12}}
\verb{embatocar}{}{}{}{}{v.t.}{Vedar com batoque.}{em.ba.to.car}{\verboinum{2}}
\verb{embatucar}{}{}{}{}{v.t.}{Ficar sem resposta; calar"-se.}{em.ba.tu.car}{\verboinum{2}}
\verb{embaular}{}{}{}{}{v.t.}{Colocar em baú; guardar.}{em.ba.u.lar}{\verboinum{1}}
\verb{embebedamento}{}{}{}{}{s.m.}{Ato ou efeito de embebedar; embriaguez.}{em.be.be.da.men.to}{0}
\verb{embebedar}{}{}{}{}{v.t.}{Tornar bêbado; embriagar.}{em.be.be.dar}{\verboinum{1}}
\verb{embeber}{ê}{}{}{}{v.t.}{Sorver pelos poros; atrair a umidade; absorver.}{em.be.ber}{0}
\verb{embeber}{ê}{}{}{}{}{Fazer penetrar por um líquido; molhar;  infiltrar.}{em.be.ber}{0}
\verb{embeber}{ê}{}{}{}{}{Ensopar, encharcar.}{em.be.ber}{\verboinum{12}}
\verb{embeiçar}{}{}{}{}{v.t.}{Cativar pelo amor ou pela sedução; encantar.}{em.bei.çar}{\verboinum{3}}
\verb{embelecar}{}{}{}{}{v.t.}{Enganar com ardil; iludir, engodar.}{em.be.le.car}{\verboinum{2}}
\verb{embeleco}{ê}{}{}{}{s.m.}{Ato ou efeito de embelecar; ardil, engodo.}{em.be.le.co}{0}
\verb{embelezador}{ô}{}{}{}{adj.}{Que embeleza, enfeita.}{em.be.le.za.dor}{0}
\verb{embelezamento}{}{}{}{}{s.m.}{Ato ou efeito de embelezar, enfeitar.}{em.be.le.za.men.to}{0}
\verb{embelezar}{}{}{}{}{v.t.}{Tornar belo, formoso; enfeitar.}{em.be.le.zar}{\verboinum{1}}
\verb{embestar}{}{}{}{}{v.t.}{Tornar estúpido; bestificar.}{em.bes.tar}{0}
\verb{embestar}{}{}{}{}{v.i.}{Obstinar"-se; teimar.}{em.bes.tar}{\verboinum{1}}
\verb{embevecer}{ê}{}{}{}{v.t.}{Provocar admiração profunda; enlevar, extasiar.}{em.be.ve.cer}{\verboinum{15}}
\verb{embevecimento}{}{}{}{}{s.m.}{Ato ou efeito de embevecer; enlevo, arrebatamento.}{em.be.ve.ci.men.to}{0}
\verb{embezerrar}{}{}{}{}{v.i.}{Revelar descontentamento; amuar, emburrar.}{em.be.zer.rar}{\verboinum{1}}
\verb{embicar}{}{}{}{}{v.t.}{Dar aspecto ou forma de bico.}{em.bi.car}{0}
\verb{embicar}{}{}{}{}{}{Dar de bico, de frente; embater.}{em.bi.car}{0}
\verb{embicar}{}{}{}{}{v.i.}{Implicar, embirrar.}{em.bi.car}{\verboinum{2}}
\verb{embigo}{}{Pop.}{}{}{s.m.}{Umbigo.}{em.bi.go}{0}
\verb{embiocar"-se}{}{}{}{}{v.pron.}{Cobrir"-se com véu; ocultar"-se, esconder"-se.}{em.bi.o.car"-se}{\verboinum{2}}
\verb{embira}{}{}{}{}{s.f.}{Fibra fornecida por algumas árvores, que serve para amarrar e fazer cordas e estopas.}{em.bi.ra}{0}
\verb{embirração}{}{}{"-ões}{}{s.f.}{Ato ou efeito de embirrar; teima, obstinação.}{em.bir.ra.ção}{0}
\verb{embirrante}{}{}{}{}{adj.2g.}{Que embirra; birrento, obstinado, teimoso.}{em.bir.ran.te}{0}
\verb{embirrar}{}{}{}{}{v.t.}{Insistir por birra; teimar.}{em.bir.rar}{0}
\verb{embirrar}{}{}{}{}{}{Ter aversão; antipatizar, implicar.}{em.bir.rar}{\verboinum{1}}
\verb{emblema}{ê}{}{}{}{s.m.}{Sinal distintivo; divisa, insígnia.}{em.ble.ma}{0}
\verb{emblemático}{}{}{}{}{adj.}{Relativo a emblema.}{em.ble.má.ti.co}{0}
\verb{emblemático}{}{}{}{}{}{Que tem caráter simbólico; representativo.}{em.ble.má.ti.co}{0}
\verb{emboaba}{}{Hist.}{}{}{s.2g.}{Na época da colonização, alcunha dada pelos descendentes dos bandeirantes paulistas aos forasteiros portugueses que chegavam à região das minas atraídos pelo ouro.}{em.bo.a.ba}{0}
\verb{embocadura}{}{}{}{}{s.f.}{Lugar em que o rio deságua; foz.}{em.bo.ca.du.ra}{0}
\verb{embocadura}{}{Mús.}{}{}{}{Parte do instrumento de sopro em que o músico coloca a boca.}{em.bo.ca.du.ra}{0}
\verb{embocar}{}{}{}{}{v.t.}{Chegar à boca.}{em.bo.car}{0}
\verb{embocar}{}{}{}{}{}{Fazer entrar por um lugar; encaçapar.}{em.bo.car}{\verboinum{2}}
\verb{emboçar}{}{}{}{}{v.t.}{Aplicar emboço; rebocar. (\textit{O pedreiro emboçou a parede.})}{em.bo.çar}{\verboinum{3}}
\verb{emboço}{ô}{}{}{}{s.m.}{Argamassa de cal ou cimento e areia; reboco.}{em.bo.ço}{0}
\verb{embolada}{}{}{}{}{s.f.}{Poesia cantada ou declamada do Nordeste, de andamento rápido, que ocorre nas estrofes de cocos ou desafios.}{em.bo.la.da}{0}
\verb{embolar}{}{}{}{}{v.t.}{Transformar num bolo; enrolar, emaranhar.}{em.bo.lar}{\verboinum{1}}
\verb{embolar}{}{}{}{}{v.i.}{Engalfinhar"-se com o adversário, rolando por terra.}{em.bo.lar}{\verboinum{1}}
\verb{embolia}{}{Med.}{}{}{s.f.}{Obstrução de um vaso sanguíneo por um coágulo ou uma partícula trazida pelo sangue.}{em.bo.li.a}{0}
\verb{êmbolo}{}{}{}{}{s.m.}{Cilindro ou disco que faz movimento de vaivém deslizando dentro de um tubo de pistão, bomba, seringa etc.}{êm.bo.lo}{0}
\verb{embolorar}{}{}{}{}{v.i.}{Criar bolor; mofar. (\textit{O pão de forma embolorou com o calor.})}{em.bo.lo.rar}{\verboinum{1}}
\verb{embolsar}{}{}{}{}{v.t.}{Colocar no bolso ou na bolsa.}{em.bol.sar}{0}
\verb{embolsar}{}{}{}{}{}{Pagar, indenizar.}{em.bol.sar}{\verboinum{1}}
\verb{embolso}{ô}{}{}{}{s.m.}{Ato ou efeito de embolsar.}{em.bol.so}{0}
\verb{embolso}{ô}{}{}{}{}{Pagamento, indenização.}{em.bol.so}{0}
\verb{embonecar}{}{}{}{}{v.t.}{Enfeitar com exagero.}{em.bo.ne.car}{0}
\verb{embonecar}{}{}{}{}{v.i.}{Criar boneca ou espiga. (\textit{O milho embonecou há uma semana.})}{em.bo.ne.car}{\verboinum{2}}
\verb{embora}{ó}{}{}{}{adv.}{Para outro lugar. (\textit{Quando acabou o espetáculo, fomos todos embora.})}{em.bo.ra}{0}
\verb{embora}{ó}{}{}{}{conj.}{Ainda que; apesar de que; se bem que. (\textit{Embora fosse tarde, precisei ligar para ela.})}{em.bo.ra}{0}
\verb{emborcar}{}{}{}{}{v.t.}{Colocar de boca para baixo; entornar.}{em.bor.car}{0}
\verb{emborcar}{}{}{}{}{}{Derramar na boca; beber com avidez.}{em.bor.car}{\verboinum{2}}
\verb{embornal}{}{}{"-ais}{}{s.m.}{Pequeno saco para carregar comida e objetos; farnel, bornal.}{em.bor.nal}{0}
\verb{emborrachar}{}{}{}{}{v.t.}{Tornar tecido ou outro material semelhante à borracha.}{em.bor.ra.char}{0}
\verb{emborrachar}{}{}{}{}{}{Tornar bêbado; embriagar, embebedar.}{em.bor.ra.char}{\verboinum{1}}
\verb{emboscada}{}{}{}{}{s.f.}{Ação de se esconder à espera de alguém para atacá"-lo; tocaia, cilada.}{em.bos.ca.da}{0}
\verb{emboscar}{}{}{}{}{v.t.}{Pôr de emboscada; esconder para atacar.}{em.bos.car}{\verboinum{2}}
\verb{embotamento}{}{}{}{}{s.m.}{Ato ou efeito de embotar; enfraquecimento.}{em.bo.ta.men.to}{0}
\verb{embotar}{}{}{}{}{v.t.}{Tornar sem gume ou sem fio.}{em.bo.tar}{0}
\verb{embotar}{}{}{}{}{}{Tirar o vigor; enfraquecer.}{em.bo.tar}{\verboinum{1}}
\verb{embrabecer}{ê}{}{}{}{}{Var. de \textit{embravecer}.}{em.bra.be.cer}{0}
\verb{embranquecer}{ê}{}{}{}{v.t.}{Tornar branco; alvejar, branquear.}{em.bran.que.cer}{\verboinum{15}}
\verb{embranquecimento}{}{}{}{}{s.m.}{Ato ou efeito de embranquecer; alvejamento.}{em.bran.que.ci.men.to}{0}
\verb{embravecer}{ê}{}{}{}{v.t.}{Tornar bravo, feroz; enfurecer, irritar.}{em.bra.ve.cer}{\verboinum{15}}
\verb{embreagem}{}{}{"-ens}{}{s.f.}{Mecanismo do automóvel, situado entre o motor e a caixa de marchas, o qual permite ligar e desligar a força do motor ao eixo de transmissão por meio de discos de fricção. }{em.bre.a.gem}{0}
\verb{embrear}{}{}{}{}{v.t.}{Acionar a embreagem do automóvel.}{em.bre.ar}{\verboinum{4}}
\verb{embrechado}{}{}{}{}{s.m.}{Incrustação de conchas e pedras em parede ou muro.}{em.bre.cha.do}{0}
\verb{embrenhar}{}{}{}{}{v.t.}{Esconder no mato, nas brenhas.}{em.bre.nhar}{\verboinum{1}}
\verb{embriagado}{}{}{}{}{adj.}{Que se embriagou; bêbado, ébrio.}{em.bri.a.ga.do}{0}
\verb{embriagador}{ô}{}{}{}{adj.}{Que embriaga; inebriante, perturbador.}{em.bri.a.ga.dor}{0}
\verb{embriagar}{}{}{}{}{v.t.}{Tornar ébrio, bêbado; embebedar, alcoolizar.}{em.bri.a.gar}{0}
\verb{embriagar}{}{}{}{}{}{Extasiar, inebriar, encantar.}{em.bri.a.gar}{\verboinum{5}}
\verb{embriaguez}{ê}{}{}{}{s.f.}{Estado de quem se embriagou; bebedeira.}{em.bri.a.guez}{0}
\verb{embriaguez}{ê}{}{}{}{}{Êxtase, encantamento, enlevo.}{em.bri.a.guez}{0}
\verb{embrião}{}{}{"-ões}{}{s.m.}{Ser vivo nas primeiras fases de desenvolvimento.}{em.bri.ão}{0}
\verb{embrião}{}{}{"-ões}{}{}{O feto até o terceiro mês de vida intrauterina.}{em.bri.ão}{0}
\verb{embrião}{}{Fig.}{"-ões}{}{}{Origem, princípio.}{em.bri.ão}{0}
\verb{embriologia}{}{Biol.}{}{}{s.f.}{Ramo da biologia e da medicina que estuda a formação e o desenvolvimento do embrião.}{em.bri.o.lo.gi.a}{0}
\verb{embrionário}{}{}{}{}{adj.}{Que está em formação, no estado de embrião.}{em.bri.o.ná.rio}{0}
\verb{embrocação}{}{}{"-ões}{}{s.f.}{Aplicação de medicamento líquido em parte doente do corpo.}{em.bro.ca.ção}{0}
\verb{embromação}{}{}{"-ões}{}{s.f.}{Ato ou efeito de embromar; enganação.}{em.bro.ma.ção}{0}
\verb{embromador}{ô}{}{}{}{adj.}{Que embroma; enganador.}{em.bro.ma.dor}{0}
\verb{embromar}{}{Pop.}{}{}{v.t.}{Abusar da confiança de outrem; enganar, ludibriar.}{em.bro.mar}{0}
\verb{embromar}{}{Pop.}{}{}{}{Retardar a conclusão de algo; enrolar. }{em.bro.mar}{\verboinum{1}}
\verb{embrulhada}{}{}{}{}{s.f.}{Falta de organização; confusão, trapalhada.}{em.bru.lha.da}{0}
\verb{embrulhão}{}{}{"-ões}{}{adj.}{Diz"-se do indivíduo que é muito atrapalhado ou se envolve em confusões; enrolador.}{em.bru.lhão}{0}
\verb{embrulhar}{}{}{}{}{v.t.}{Colocar em um invólucro; empacotar, embalar.}{em.bru.lhar}{0}
\verb{embrulhar}{}{Fig.}{}{}{}{Enganar, tapear, ludibriar.}{em.bru.lhar}{\verboinum{1}}
\verb{embrulho}{}{}{}{}{s.m.}{Aquilo que se embrulhou; pacote, volume.}{em.bru.lho}{0}
\verb{embrulho}{}{Fig.}{}{}{}{Confusão, trapalhada, embrulhada.}{em.bru.lho}{0}
\verb{embruscar}{}{}{}{}{v.t.}{Tornar brusco, sombrio; escurecer.}{em.brus.car}{\verboinum{2}}
\verb{embrutecer}{ê}{}{}{}{v.t.}{Tornar bruto, estúpido.}{em.bru.te.cer}{\verboinum{15}}
\verb{embrutecimento}{}{}{}{}{s.m.}{Ato ou efeito de embrutecer, enrudecer. }{em.bru.te.ci.men.to}{0}
\verb{embuá}{}{Zool.}{}{}{s.m.}{Nome comum a pequenos animais invertebrados, de corpo alongado e com dezenas de pernas, semelhante à lacraia e à centopeia; piolho"-de"-cobra.}{em.bu.á}{0}
\verb{embuçar}{}{}{}{}{v.t.}{Cobrir o rosto com o embuço; disfarçar.}{em.bu.çar}{\verboinum{3}}
\verb{embuchar}{}{Pop.}{}{}{v.t.}{Encher o bucho, o estômago; fartar, saciar.}{em.bu.char}{0}
\verb{embuchar}{}{Pop.}{}{}{}{Ficar grávida; engravidar.}{em.bu.char}{\verboinum{1}}
\verb{embuço}{}{}{}{}{s.m.}{Parte da capa com que se cobre o rosto.}{em.bu.ço}{0}
\verb{emburrar}{}{}{}{}{v.i.}{Firmar"-se em uma posição; empacar.}{em.bur.rar}{0}
\verb{emburrar}{}{}{}{}{}{Tornar burro, estúpido; embrutecer. }{em.bur.rar}{\verboinum{1}}
\verb{embuste}{}{}{}{}{s.m.}{Mentira ardilosa; logro, artifício.}{em.bus.te}{0}
\verb{embusteiro}{ê}{}{}{}{adj.}{Que usa de embuste; mentiroso, impostor.}{em.bus.tei.ro}{0}
\verb{embutido}{}{}{}{}{adj.}{Colocado à força; introduzido.}{em.bu.ti.do}{0}
\verb{embutido}{}{}{}{}{}{Diz"-se do móvel ou eletrodoméstico adaptado a um vão apropriado.}{em.bu.ti.do}{0}
\verb{embutido}{}{Cul.}{}{}{s.m.}{Alimento preparado com tripa recheada de algum tipo de carne picada e temperada como salsicha, linguiça, paio etc.}{em.bu.ti.do}{0}
\verb{embutir}{}{}{}{}{v.t.}{Colocar à força; introduzir.}{em.bu.tir}{0}
\verb{embutir}{}{}{}{}{}{Ajustar em vão apropriado.}{em.bu.tir}{\verboinum{18}}
\verb{eme}{}{}{}{}{s.m.}{Nome da letra \textit{m}.}{e.me}{0}
\verb{emenda}{}{}{}{}{s.f.}{Ato de emendar, reparar.}{e.men.da}{0}
\verb{emenda}{}{}{}{}{}{Conserto, reparo, remendo.}{e.men.da}{0}
\verb{emendar}{}{}{}{}{v.t.}{Corrigir defeito; consertar, reparar.}{e.men.dar}{0}
\verb{emendar}{}{}{}{}{}{Unir partes para formar um todo.}{e.men.dar}{\verboinum{1}}
\verb{ementa}{}{}{}{}{s.f.}{Registro escrito; apontamento, nota.}{e.men.ta}{0}
\verb{ementa}{}{}{}{}{}{Sumário, resumo, síntese.}{e.men.ta}{0}
\verb{ementário}{}{}{}{}{s.m.}{Livro ou caderno de apontamentos; agenda.}{e.men.tá.rio}{0}
\verb{emergência}{}{}{}{}{s.f.}{Ato ou efeito de emergir.}{e.mer.gên.cia}{0}
\verb{emergência}{}{}{}{}{}{Situação crítica que deve ser resolvida logo; imprevisto.}{e.mer.gên.cia}{0}
\verb{emergencial}{}{}{"-ais}{}{adj.2g.}{Que tem condição de emergência; conjuntural, circunstancial.}{e.mer.gen.ci.al}{0}
\verb{emergente}{}{}{}{}{adj.2g.}{Que emerge, surge.}{e.mer.gen.te}{0}
\verb{emergente}{}{Bras.}{}{}{}{Diz"-se do indivíduo que está em ascensão econômica e social.}{e.mer.gen.te}{0}
\verb{emergir}{}{}{}{}{v.i.}{Vir à tona; elevar"-se; surgir.}{e.mer.gir}{\verboinum{22}\verboirregular{\emph{def.} emerges, emerge, emergimos, emergis, emergem}}
\verb{emérito}{}{}{}{}{adj.}{Grande especialista em uma arte ou ciência; eminente, notável.}{e.mé.ri.to}{0}
\verb{emérito}{}{}{}{}{}{Diz"-se do título concedido a profissionais eminentes quando se aposentam.}{e.mé.ri.to}{0}
\verb{emersão}{}{}{"-ões}{}{s.f.}{Ato ou efeito de emergir, de vir à tona.}{e.mer.são}{0}
\verb{emerso}{é}{}{}{}{adj.}{Que emergiu, apareceu.}{e.mer.so}{0}
\verb{emético}{}{}{}{}{adj.}{Diz"-se de medicamento que provoca vômito.}{e.mé.ti.co}{0}
\verb{emigração}{}{}{"-ões}{}{s.f.}{Ato ou efeito de emigrar.}{e.mi.gra.ção}{0}
\verb{emigração}{}{}{"-ões}{}{}{Movimentação de uma região para outra dentro de um mesmo país.}{e.mi.gra.ção}{0}
\verb{emigrado}{}{}{}{}{adj.}{Que emigrou; que saiu de sua pátria para morar em outro país.}{e.mi.gra.do}{0}
\verb{emigrante}{}{}{}{}{adj.2g.}{Que emigra; que costuma ir de uma região a outra dentro de um mesmo país.}{e.mi.gran.te}{0}
\verb{emigrar}{}{}{}{}{v.i.}{Sair de seu país para se estabelecer em outro.}{e.mi.grar}{0}
\verb{emigrar}{}{}{}{}{}{Mudar de uma região para outra dentro de um mesmo país.}{e.mi.grar}{\verboinum{1}}
\verb{eminência}{}{}{}{}{s.f.}{Lugar alto; elevação, cume.}{e.mi.nên.cia}{0}
\verb{eminência}{}{}{}{}{}{Tratamento dado a cardeal.}{e.mi.nên.cia}{0}
\verb{eminente}{}{}{}{}{adj.2g.}{Que está acima dos outros; elevado, alto.}{e.mi.nen.te}{0}
\verb{eminente}{}{}{}{}{}{Superior, notável, ilustre.}{e.mi.nen.te}{0}
\verb{emir}{}{}{}{}{s.m.}{Título dado aos descendentes de Maomé ou aos soberanos muçulmanos.}{e.mir}{0}
\verb{emirado}{}{}{}{}{s.m.}{Estado ou região governada por um emir.}{e.mi.ra.do}{0}
\verb{emirado}{}{}{}{}{}{Dignidade de emir.}{e.mi.ra.do}{0}
\verb{emissão}{}{}{"-ões}{}{s.f.}{Ato ou efeito de emitir, de colocar em circulação.}{e.mis.são}{0}
\verb{emissário}{}{}{}{}{adj.}{Que é enviado em missão; mensageiro.}{e.mis.sá.rio}{0}
\verb{emissário}{}{}{}{}{}{Canal integrante de uma rede de escoamento de água ou esgoto. (\textit{As cidades litorâneas têm em sua rede de esgoto muitos emissários submarinos.})}{e.mis.sá.rio}{0}
\verb{emissor}{ô}{}{}{}{adj.}{Que emite, envia; emitente.}{e.mis.sor}{0}
\verb{emissor}{ô}{}{}{}{s.m.}{Indivíduo que transmite uma mensagem ao receptor; fonte.}{e.mis.sor}{0}
\verb{emissora}{ô}{}{}{}{s.f.}{Estação que transmite sinais de rádio ou televisão.}{e.mis.so.ra}{0}
\verb{emitente}{}{}{}{}{adj.2g.}{Que emite ou envia; emissor.}{e.mi.ten.te}{0}
\verb{emitir}{}{}{}{}{v.t.}{Lançar de si; soltar, expelir.}{e.mi.tir}{0}
\verb{emitir}{}{}{}{}{}{Enviar, remeter, expedir.}{e.mi.tir}{0}
\verb{emitir}{}{}{}{}{}{Manifestar, exprimir, enunciar.}{e.mi.tir}{0}
\verb{emitir}{}{}{}{}{}{Colocar em circulação.}{e.mi.tir}{\verboinum{18}}
\verb{emoção}{}{}{"-ões}{}{s.f.}{Agitação moral; abalo afetivo; comoção.}{e.mo.ção}{0}
\verb{emocional}{}{}{"-ais}{}{adj.2g.}{Relativo a emoção.}{e.mo.ci.o.nal}{0}
\verb{emocional}{}{}{"-ais}{}{}{Que provoca emoção; que desperta sentimentos intensos.}{e.mo.ci.o.nal}{0}
\verb{emocionante}{}{}{}{}{adj.2g.}{Que causa emoção; comovente, impressionante.}{e.mo.ci.o.nan.te}{0}
\verb{emocionar}{}{}{}{}{v.t.}{Causar emoção; impressionar, comover. (\textit{A cena do acidente emocionou muitos telespectadores.})}{e.mo.ci.o.nar}{\verboinum{1}}
\verb{emoldurar}{}{}{}{}{v.t.}{Colocar em moldura; encaixilhar.}{e.mol.du.rar}{\verboinum{1}}
\verb{emoliente}{}{}{}{}{adj.2g.}{Diz"-se da substância que abranda uma inflamação.}{e.mo.li.en.te}{0}
\verb{emolumento}{}{}{}{}{s.m.}{Retribuição por serviço prestado; gratificação, recompensa.}{e.mo.lu.men.to}{0}
\verb{emotividade}{}{}{}{}{s.f.}{Estado de emotivo; tendência a se emocionar.}{e.mo.ti.vi.da.de}{0}
\verb{emotivo}{}{}{}{}{adj.}{Que se emociona com facilidade; sensível.}{e.mo.ti.vo}{0}
\verb{empacador}{ô}{}{}{}{adj.}{Diz"-se do animal que costuma empacar.}{em.pa.ca.dor}{0}
\verb{empacar}{}{}{}{}{v.i.}{Ficar parado; estacar, emperrar.}{em.pa.car}{\verboinum{2}}
\verb{empachado}{}{}{}{}{adj.}{Extremamente cheio; repleto, empanturrado.}{em.pa.cha.do}{0}
\verb{empachamento}{}{}{}{}{s.m.}{Ato ou efeito de empachar; empacho, obstáculo, estorvo.}{em.pa.cha.men.to}{0}
\verb{empachar}{}{}{}{}{v.t.}{Encher muito o estômago de alimentos; empanturrar.}{em.pa.char}{0}
\verb{empachar}{}{}{}{}{}{Obstruir, impedir, embaraçar.}{em.pa.char}{\verboinum{1}}
\verb{empacho}{}{}{}{}{s.m.}{Ato ou efeito de empachar; empachamento, obstáculo, estorvo.}{em.pa.cho}{0}
\verb{empacho}{}{}{}{}{}{Desconforto devido ao excesso de comida no estômago.}{em.pa.cho}{0}
\verb{empaçocar}{}{}{}{}{v.t.}{Juntar desordenadamente; misturar, emaranhar, embolar.}{em.pa.ço.car}{\verboinum{2}}
\verb{empacotadeira}{ê}{}{}{}{s.f.}{Empacotadora.}{em.pa.co.ta.dei.ra}{0}
\verb{empacotador}{ô}{}{}{}{adj.}{Que empacota; embalador.}{em.pa.co.ta.dor}{0}
\verb{empacotadora}{ô}{}{}{}{s.f.}{Máquina agrícola que faz pacotes de feno, palha etc.; empacotadeira.}{em.pa.co.ta.do.ra}{0}
\verb{empacotamento}{}{}{}{}{s.m.}{Ato ou efeito de empacotar, embalar.}{em.pa.co.ta.men.to}{0}
\verb{empacotar}{}{}{}{}{v.t.}{Colocar em pacotes; embalar, acondicionar.}{em.pa.co.tar}{\verboinum{1}}
\verb{empada}{}{Cul.}{}{}{s.f.}{Salgadinho de massa com recheio de galinha, queijo, camarão, palmito etc., e que se assa em forminhas ao forno.}{em.pa.da}{0}
\verb{empadão}{}{Cul.}{"-ões}{}{s.m.}{Empada assada em forma grande e servida em pedaços.}{em.pa.dão}{0}
\verb{empáfia}{}{}{}{}{s.f.}{Orgulho vão; soberba, altivez, arrogância.}{em.pá.fia}{0}
\verb{empalação}{}{Hist.}{"-ões}{}{s.f.}{Antigo suplício que consistia em espetar o condenado com uma estaca pelo ânus, deixando"-o assim até morrer.}{em.pa.la.ção}{0}
\verb{empalar}{}{}{}{}{v.t.}{Submeter ao suplício da empalação; enfiar, penetrar.}{em.pa.lar}{\verboinum{1}}
\verb{empalhação}{}{}{"-ões}{}{s.f.}{Ato ou efeito de empalhar.}{em.pa.lha.ção}{0}
\verb{empalhação}{}{}{"-ões}{}{}{Malha de vime ou palha usada em forração.}{em.pa.lha.ção}{0}
\verb{empalhador}{ô}{}{}{}{adj.}{Diz"-se do indivíduo que empalha, que faz trama usando palhinha.}{em.pa.lha.dor}{0}
\verb{empalhar}{}{}{}{}{v.t.}{Forrar, revestir com palha.}{em.pa.lhar}{0}
\verb{empalhar}{}{}{}{}{}{Encher de palha a pele de animal morto para conservação.}{em.pa.lhar}{\verboinum{1}}
\verb{empalidecer}{ê}{}{}{}{v.t.}{Perder a cor natural; ficar pálido.}{em.pa.li.de.cer}{\verboinum{15}}
\verb{empalmação}{}{}{"-ões}{}{s.f.}{Ato ou efeito de empalmar; roubo, furto.}{em.pal.ma.ção}{0}
\verb{empalmar}{}{}{}{}{v.t.}{Esconder na palma da mão.}{em.pal.mar}{0}
\verb{empalmar}{}{}{}{}{}{Roubar, furtar, surrupiar. }{em.pal.mar}{\verboinum{1}}
\verb{empanamento}{}{}{}{}{s.m.}{Ato ou efeito de empanar; deslustre.}{em.pa.na.men.to}{0}
\verb{empanar}{}{}{}{}{v.t.}{Cobrir com panos.}{em.pa.nar}{0}
\verb{empanar}{}{}{}{}{}{Fazer perder o brilho; embaçar, deslustrar.}{em.pa.nar}{\verboinum{1}}
\verb{empanar}{}{}{}{}{v.t.}{Passar carne, peixe etc. no ovo batido e em farinha de trigo ou de rosca, para depois fritar.}{em.pa.nar}{\verboinum{1}}
\verb{empanturrar}{}{}{}{}{v.t.}{Encher de comida; empanzinar.}{em.pan.tur.rar}{\verboinum{1}}
\verb{empanzinamento}{}{}{}{}{s.m.}{Ato ou efeito de empanzinar, empanturrar.}{em.pan.zi.na.men.to}{0}
\verb{empanzinar}{}{}{}{}{v.t.}{Comer em excesso; empanturrar.}{em.pan.zi.nar}{\verboinum{1}}
\verb{empapar}{}{}{}{}{v.t.}{Tornar mole como papa; encharcar, embeber.}{em.pa.par}{\verboinum{1}}
\verb{empapelar}{}{}{}{}{v.t.}{Envolver, embrulhar em papel.}{em.pa.pe.lar}{0}
\verb{empapelar}{}{}{}{}{}{Revestir de papel.}{em.pa.pe.lar}{\verboinum{1}}
\verb{empapuçado}{}{}{}{}{adj.}{Que se empapuçou; inchado, papudo.}{em.pa.pu.ça.do}{0}
\verb{empapuçado}{}{Pop.}{}{}{}{Enfastiado, entediado.}{em.pa.pu.ça.do}{0}
\verb{empapuçamento}{}{}{}{}{s.m.}{Ato ou efeito de empapuçar; inchação.}{em.pa.pu.ça.men.to}{0}
\verb{empapuçar}{}{}{}{}{v.t.}{Encher de papos ou pregas.}{em.pa.pu.çar}{0}
\verb{empapuçar}{}{Pop.}{}{}{v.pron.}{Enfastiar"-se. }{em.pa.pu.çar}{0}
\verb{empapuçar}{}{}{}{}{}{Tornar"-se inchado, papudo.}{em.pa.pu.çar}{\verboinum{3}}
\verb{emparceirar}{}{}{}{}{v.t.}{Juntar como parceiro; unir, emparelhar.}{em.par.cei.rar}{\verboinum{1}}
\verb{emparedar}{}{}{}{}{v.t.}{Encerrar entre paredes; enclausurar.}{em.pa.re.dar}{\verboinum{1}}
\verb{emparelhar}{}{}{}{}{v.t.}{Pôr lado a lado, de par em par.}{em.pa.re.lhar}{0}
\verb{emparelhar}{}{}{}{}{}{Tornar semelhante; equiparar, igualar.}{em.pa.re.lhar}{\verboinum{1}}
\verb{empastamento}{}{}{}{}{s.m.}{Ato ou efeito de empastar.}{em.pas.ta.men.to}{0}
\verb{empastamento}{}{}{}{}{}{Camada espessa de tintas que se usa nas pinturas a óleo, para realçar seu brilho.}{em.pas.ta.men.to}{0}
\verb{empastar}{}{}{}{}{v.t.}{Aplicar tinta em grande quantidade.}{em.pas.tar}{0}
\verb{empastar}{}{}{}{}{}{Prender as pastas ao livro para encapá"-lo.}{em.pas.tar}{\verboinum{1}}
\verb{empastelamento}{}{}{}{}{s.m.}{Ato ou efeito de empastelar.}{em.pas.te.la.men.to}{0}
\verb{empastelar}{}{}{}{}{v.t.}{Estragar o texto com erros tipográficos.}{em.pas.te.lar}{\verboinum{1}}
\verb{empatar}{}{}{}{}{v.t.}{Dificultar o prosseguimento; estorvar, embaraçar.}{em.pa.tar}{0}
\verb{empatar}{}{}{}{}{}{Deixar parado dinheiro sem render.}{em.pa.tar}{0}
\verb{empatar}{}{}{}{}{}{Igualar na contagem de votos ou pontos.}{em.pa.tar}{\verboinum{1}}
\verb{empate}{}{}{}{}{s.m.}{Término de jogo ou de votação sem que haja um vencedor.}{em.pa.te}{0}
\verb{empatia}{}{}{}{}{s.f.}{Capacidade de identificação de uma pessoa com o que a outra está sentindo ou pensando.}{em.pa.ti.a}{0}
\verb{empavonar}{}{}{}{}{v.t.}{Tornar vaidoso e inchado como um pavão; pavonear.}{em.pa.vo.nar}{\verboinum{1}}
\verb{empecer}{ê}{}{}{}{v.t.}{Causar impedimentos; estorvar, dificultar.}{em.pe.cer}{\verboinum{15}}
\verb{empecilho}{}{}{}{}{s.m.}{Aquilo que dificulta; estorvo, obstáculo.}{em.pe.ci.lho}{0}
\verb{empeçonhar}{}{}{}{}{v.t.}{Pôr peçonha; envenenar, perverter.}{em.pe.ço.nhar}{\verboinum{1}}
\verb{empedernido}{}{}{}{}{adj.}{Que se transformou em pedra; petrificado.}{em.pe.der.ni.do}{0}
\verb{empedernido}{}{Fig.}{}{}{}{Duro, inflexível, insensível. (\textit{Após tantos anos de amargura e solidão, ele tinha o coração empedernido.})}{em.pe.der.ni.do}{0}
\verb{empedernir}{}{}{}{}{v.t.}{Transformar em pedra; petrificar.}{em.pe.der.nir}{0}
\verb{empedernir}{}{}{}{}{}{Tornar duro, insensível, desumano.}{em.pe.der.nir}{\verboinum{35}\verboirregular{\emph{def.} empedernimos, empedernis}}
\verb{empedrado}{}{}{}{}{adj.}{Calçado, revestido com pedras.}{em.pe.dra.do}{0}
\verb{empedrado}{}{}{}{}{}{Que tem a consistência de pedra; endurecido.}{em.pe.dra.do}{0}
\verb{empedrar}{}{}{}{}{v.t.}{Calçar ou revestir com pedra.}{em.pe.drar}{0}
\verb{empedrar}{}{}{}{}{}{Tornar duro como pedra; endurecer, empelotar.}{em.pe.drar}{\verboinum{1}}
\verb{empelicado}{}{}{}{}{adj.}{Diz"-se de criança que nasce envolta no âmnio materno.}{em.pe.li.ca.do}{0}
\verb{empelotar}{}{}{}{}{v.t.}{Reduzir a pelotas. (\textit{A mulher deixou empelotar o manjar.})}{em.pe.lo.tar}{\verboinum{1}}
\verb{empena}{}{}{}{}{s.f.}{Parede lateral ou cabeceira de um prédio.}{em.pe.na}{0}
\verb{empena}{}{}{}{}{}{Cada um dos lados de um frontão.}{em.pe.na}{0}
\verb{empenar}{}{}{}{}{v.i.}{Fazer alguma coisa ficar torta.}{em.pe.nar}{0}
\verb{empenar}{}{}{}{}{}{Criar penas; emplumar"-se.}{em.pe.nar}{\verboinum{1}}
\verb{empenhar}{}{}{}{}{v.t.}{Entregar alguma coisa como garantia; penhorar.}{em.pe.nhar}{0}
\verb{empenhar}{}{}{}{}{}{Levar alguém a trabalhar por alguma coisa.}{em.pe.nhar}{0}
\verb{empenhar}{}{}{}{}{}{Usar de algum poder para alguma finalidade; aplicar, empregar.}{em.pe.nhar}{\verboinum{1}}
\verb{empenho}{}{}{}{}{s.m.}{Ato de entregar alguma coisa como garantia; penhora.}{em.pe.nho}{0}
\verb{empenho}{}{}{}{}{}{Vontade de trabalhar por alguma coisa; interesse.}{em.pe.nho}{0}
\verb{empeno}{}{}{}{}{s.m.}{Ato ou efeito de empenar.}{em.pe.no}{0}
\verb{empeno}{}{}{}{}{}{Deformação em peça de madeira por ação de calor e umidade.}{em.pe.no}{0}
\verb{empepinar}{}{Pop.}{}{}{v.i.}{Ficar difícil; dificultar.}{em.pe.pi.nar}{0}
\verb{empepinar}{}{}{}{}{}{No futebol, embolar o meio de campo.}{em.pe.pi.nar}{\verboinum{1}}
\verb{emperiquitar}{}{}{}{}{v.t.}{Enfeitar ou trajar alguém com apuro exagerado; embonecar.}{em.pe.ri.qui.tar}{\verboinum{1}}
\verb{empernar}{}{}{}{}{v.i.}{Cruzar as pernas.}{em.per.nar}{\verboinum{1}}
\verb{emperramento}{}{}{}{}{s.m.}{Ato ou efeito de emperrar.}{em.per.ra.men.to}{0}
\verb{emperramento}{}{}{}{}{}{Dificuldade de movimento.}{em.per.ra.men.to}{0}
\verb{emperramento}{}{Fig.}{}{}{}{Grande teimosia; birra.}{em.per.ra.men.to}{0}
\verb{emperrar}{}{}{}{}{v.t.}{Fazer alguma coisa perder a facilidade de se mover.}{em.per.rar}{\verboinum{1}}
\verb{emperro}{ê}{}{}{}{s.m.}{Emperramento.}{em.per.ro}{0}
\verb{empertigado}{}{}{}{}{adj.}{Com as partes do corpo esticadas; aprumado.}{em.per.ti.ga.do}{0}
\verb{empertigar}{}{}{}{}{v.t.}{Tornar rígido, teso.}{em.per.ti.gar}{0}
\verb{empertigar}{}{}{}{}{v.pron.}{Portar"-se com altivez.}{em.per.ti.gar}{\verboinum{5}}
\verb{empestar}{}{}{}{}{v.t.}{Infectar com peste.}{em.pes.tar}{0}
\verb{empestar}{}{}{}{}{}{Contaminar.}{em.pes.tar}{0}
\verb{empestar}{}{}{}{}{}{Tornar fedorento.}{em.pes.tar}{\verboinum{1}}
\verb{empestear}{}{}{}{}{}{Var. de \textit{empestar}.}{em.pes.te.ar}{0}
\verb{empetecar}{}{}{}{}{v.t.}{Enfeitar alguém demais; emperiquitar.}{em.pe.te.car}{\verboinum{2}}
\verb{empilhadeira}{ê}{}{}{}{s.f.}{Máquina móvel própria para executar empilhamento e arrumação de certos produtos ou carga.}{em.pi.lha.dei.ra}{0}
\verb{empilhamento}{}{}{}{}{s.m.}{Arrumação em pilhas.}{em.pi.lha.men.to}{0}
\verb{empilhar}{}{}{}{}{v.t.}{Colocar coisas umas sobre as outras; amontoar.}{em.pi.lhar}{\verboinum{1}}
\verb{empinado}{}{}{}{}{adj.}{Em posição reta, ereta; levantado, erguido.}{em.pi.na.do}{0}
\verb{empinar}{}{}{}{}{v.t.}{Pôr na vertical.}{em.pi.nar}{0}
\verb{empinar}{}{}{}{}{}{Fazer subir; levantar.}{em.pi.nar}{0}
\verb{empinar}{}{}{}{}{v.pron.}{Levantar"-se sobre as patas traseiras (animal).}{em.pi.nar}{\verboinum{1}}
\verb{empipocar}{}{}{}{}{v.t.}{Formar pústulas ou bolhas no corpo.}{em.pi.po.car}{\verboinum{2}}
\verb{empíreo}{}{}{}{}{adj.}{Pertencente ou relativo ao céu; celeste. }{em.pí.re.o}{0}
\verb{empíreo}{}{Mit.}{}{}{s.m.}{Morada dos deuses; céu.}{em.pí.re.o}{0}
\verb{empírico}{}{}{}{}{adj.}{Que é baseado apenas na experiência.}{em.pí.ri.co}{0}
\verb{empirismo}{}{}{}{}{s.m.}{Doutrina de que todo conhecimento se origina da experiência.}{em.pi.ris.mo}{0}
\verb{empirismo}{}{}{}{}{}{Atitude de quem só usa conhecimentos práticos.}{em.pi.ris.mo}{0}
\verb{emplacar}{}{}{}{}{v.t.}{Colocar placa ou chapa em alguma coisa.}{em.pla.car}{0}
\verb{emplacar}{}{}{}{}{v.i.}{Ter um bom resultado.}{em.pla.car}{\verboinum{2}}
\verb{emplastar}{}{}{}{}{}{Var. de \textit{emplastrar}.}{em.plas.tar}{0}
\verb{emplasto}{}{}{}{}{}{Var. de \textit{emplastro}.}{em.plas.to}{0}
\verb{emplastrar}{}{}{}{}{v.t.}{Colocar emplastro.}{em.plas.trar}{0}
\verb{emplastrar}{}{}{}{}{}{Dispor em camadas.}{em.plas.trar}{0}
\verb{emplastrar}{}{}{}{}{}{Revestir como se cobrisse com emplastro.}{em.plas.trar}{\verboinum{1}}
\verb{emplastro}{}{}{}{}{s.m.}{Remédio que, amolecendo com o calor, adere à pele.}{em.plas.tro}{0}
\verb{emplastro}{}{}{}{}{}{Curativo com esse remédio.}{em.plas.tro}{0}
\verb{emplumar}{}{}{}{}{v.t.}{Cobrir com pluma ou pena.}{em.plu.mar}{\verboinum{1}}
\verb{empoar}{}{}{}{}{v.t.}{Cobrir de pó; polvilhar.}{em.po.ar}{0}
\verb{empoar}{}{}{}{}{}{Cobrir de poeira; empoeirar.}{em.po.ar}{\verboinum{7}}
\verb{empobrecer}{ê}{}{}{}{v.t.}{Tornar pobre; desprover de recursos.}{em.po.bre.cer}{\verboinum{15}}
\verb{empobrecimento}{}{}{}{}{s.m.}{Ato ou efeito de empobrecer; perda de riquezas.}{em.po.bre.ci.men.to}{0}
\verb{empoçar}{}{}{}{}{v.t.}{Colocar dentro de um poço ou uma poça.}{em.po.çar}{0}
\verb{empoçar}{}{}{}{}{v.i.}{Formar poças.}{em.po.çar}{\verboinum{3}}
\verb{empoeirar}{}{}{}{}{v.t.}{Cobrir pessoa ou coisa de poeira.}{em.po.ei.rar}{\verboinum{1}}
\verb{empola}{ô}{}{}{}{s.f.}{Bolha cheia de serosidade, entre a derme e a epiderme.}{em.po.la}{0}
\verb{empola}{ô}{}{}{}{}{Bolha de água fervendo; ampola.}{em.po.la}{0}
\verb{empolado}{}{}{}{}{adj.}{Que está coberto de empolas.}{em.po.la.do}{0}
\verb{empolado}{}{Fig.}{}{}{}{Estilo ou modo de falar pomposo, bombástico.}{em.po.la.do}{0}
\verb{empolar}{}{}{}{}{v.t.}{Cobrir de empolas ou bolhas.}{em.po.lar}{0}
\verb{empolar}{}{Fig.}{}{}{}{Tornar pomposo, exagerado.}{em.po.lar}{\verboinum{1}}
\verb{empoleirar}{}{}{}{}{v.t.}{Colocar no poleiro.}{em.po.lei.rar}{\verboinum{1}}
\verb{empolgação}{}{}{"-ões}{}{s.f.}{Ato ou efeito de empolgar.}{em.pol.ga.ção}{0}
\verb{empolgação}{}{}{"-ões}{}{}{Animação, entusiasmo.}{em.pol.ga.ção}{0}
\verb{empolgante}{}{}{}{}{adj.2g.}{Que empolga, que cativa a atenção; entusiasmante.}{em.pol.gan.te}{0}
\verb{empolgar}{}{}{}{}{v.t.}{Prender a atenção.}{em.pol.gar}{0}
\verb{empolgar}{}{}{}{}{}{Animar, entusiasmar.}{em.pol.gar}{\verboinum{5}}
\verb{empombar}{}{Pop.}{}{}{v.t.}{Ficar nervoso, irritado com alguém ou consigo mesmo.}{em.pom.bar}{\verboinum{1}}
\verb{emporcalhar}{}{}{}{}{v.t.}{Tornar sujo como porco; imundar.}{em.por.ca.lhar}{\verboinum{1}}
\verb{empório}{}{}{}{}{s.m.}{Estabelecimento comercial; armazém, loja.}{em.pó.rio}{0}
\verb{empós}{}{}{}{}{prep.}{Após, depois; atrás.}{em.pós}{0}
\verb{empossar}{}{}{}{}{v.t.}{Dar posse a.}{em.pos.sar}{0}
\verb{empossar}{}{}{}{}{v.pron.}{Tomar posse; apoderar"-se; apossar"-se.}{em.pos.sar}{\verboinum{1}}
\verb{empostar}{}{}{}{}{}{Var. de \textit{impostar}.}{em.pos.tar}{0}
\verb{emprazar}{}{}{}{}{v.t.}{Convocar ou convidar, definindo prazo ou data.}{em.pra.zar}{0}
\verb{emprazar}{}{}{}{}{v.pron.}{Acertar local e data de encontro.}{em.pra.zar}{\verboinum{1}}
\verb{empreendedor}{ô}{}{}{}{adj.}{Diz"-se daquele que empreende; ativo, arrojado.}{em.pre.en.de.dor}{0}
\verb{empreender}{ê}{}{}{}{v.t.}{Decidir realizar tarefa difícil e trabalhosa; tentar.}{em.pre.en.der}{\verboinum{12}}
\verb{empreendimento}{}{}{}{}{s.m.}{Ato de quem assume tarefa ou responsabilidade.}{em.pre.en.di.men.to}{0}
\verb{empreendimento}{}{}{}{}{}{Empresa, projeto, realização.}{em.pre.en.di.men.to}{0}
\verb{empreendimento}{}{}{}{}{}{Firma montada para explorar um negócio.}{em.pre.en.di.men.to}{0}
\verb{empregada}{}{Bras.}{}{}{s.f.}{Mulher encarregada dos serviços domésticos; criada.}{em.pre.ga.da}{0}
\verb{empregado}{}{}{}{}{adj.}{Que se empregou; utilizado.}{em.pre.ga.do}{0}
\verb{empregado}{}{}{}{}{s.m.}{Pessoa que trabalha e recebe pagamento; funcionário.}{em.pre.ga.do}{0}
\verb{empregador}{ô}{}{}{}{adj.}{Que emprega, que contrata para um serviço.}{em.pre.ga.dor}{0}
\verb{empregador}{ô}{}{}{}{s.m.}{Essa pessoa; patrão, empresário.}{em.pre.ga.dor}{0}
\verb{empregar}{}{}{}{}{v.t.}{Fazer uso, aplicação}{em.pre.gar}{0}
\verb{empregar}{}{}{}{}{}{Dar emprego.}{em.pre.gar}{0}
\verb{empregar}{}{}{}{}{}{Conseguir emprego em empresa pública ou privada.}{em.pre.gar}{0}
\verb{empregar}{}{}{}{}{v.pron.}{Dedicar"-se com afinco em alguma atividade.}{em.pre.gar}{\verboinum{5}}
\verb{emprego}{ê}{}{}{}{s.m.}{Ato eu efeito de empregar; uso, ocupação.}{em.pre.go}{0}
\verb{emprego}{ê}{}{}{}{}{Trabalho permanente, regular e remunerado em empresa pública ou privada.}{em.pre.go}{0}
\verb{emprego}{ê}{}{}{}{}{Local em que se exerce esse trabalho.}{em.pre.go}{0}
\verb{empreguismo}{}{Bras.}{}{}{s.m.}{Tendência a conceder empregos públicos por razões políticas.}{em.pre.guis.mo}{0}
\verb{empreitada}{}{}{}{}{s.f.}{Obra ou trabalho por conta de outrem, mediante pagamento previamente combinado.}{em.prei.ta.da}{0}
\verb{empreitada}{}{Por ext.}{}{}{}{Realização de um projeto; empreendimento, trabalho.}{em.prei.ta.da}{0}
\verb{empreitar}{}{}{}{}{v.t.}{Ajustar um trabalho por empreitada.}{em.prei.tar}{\verboinum{1}}
\verb{empreiteira}{ê}{Bras.}{}{}{s.f.}{Empresa que ajusta obras por empreitada.}{em.prei.tei.ra}{0}
\verb{empreiteiro}{ê}{}{}{}{adj.}{Que faz obra ou serviço de empreitada.}{em.prei.tei.ro}{0}
\verb{empreiteiro}{ê}{}{}{}{s.m.}{Essa pessoa.}{em.prei.tei.ro}{0}
\verb{emprenhar}{}{}{}{}{v.t.}{Tornar prenha (mulher ou fêmea); engravidar.}{em.pre.nhar}{\verboinum{1}}
\verb{empresa}{ê}{}{}{}{s.f.}{Realização de tarefa de grande porte. (\textit{As viagens das bandeiras eram empresas notáveis.})}{em.pre.sa}{0}
\verb{empresa}{ê}{}{}{}{}{Sociedade organizada para a exploração de uma indústria ou comércio; estabelecimento, casa. (\textit{Empresa industrial. Empresa mercantil. Empresa de transportes.})}{em.pre.sa}{0}
\verb{empresa}{ê}{}{}{}{}{Sociedade devidamente legalizada, constituída para a exploração de um fim específico.}{em.pre.sa}{0}
\verb{empresar}{}{}{}{}{v.t.}{Financiar a produção, ou trabalhar como empresário de um espetáculo, evento etc.}{em.pre.sar}{\verboinum{1}}
\verb{empresar}{}{}{}{}{v.t.}{Represar. }{em.pre.sar}{\verboinum{1}}
\verb{empresariado}{}{}{}{}{s.m.}{Classe dos empresários.}{em.pre.sa.ri.a.do}{0}
\verb{empresarial}{}{}{"-ais}{}{adj.2g.}{Relativo a empresa ou a empresário.}{em.pre.sa.ri.al}{0}
\verb{empresário}{}{}{}{}{s.m.}{Pessoa que dirige ou tem uma empresa; homem de negócios.}{em.pre.sá.rio}{0}
\verb{empresário}{}{Por ext.}{}{}{}{Pessoa encarregada dos interesses e da vida profissional de um artista, de um atleta etc.}{em.pre.sá.rio}{0}
\verb{emprestar}{}{}{}{}{v.t.}{Ceder ou confiar alguma coisa a alguém por algum tempo, com promessa de devolução.}{em.pres.tar}{0}
\verb{emprestar}{}{}{}{}{}{Conceder empréstimo de dinheiro a juros.}{em.pres.tar}{\verboinum{1}}
\verb{empréstimo}{}{}{}{}{s.m.}{Ato ou efeito de emprestar.}{em.prés.ti.mo}{0}
\verb{empréstimo}{}{}{}{}{}{A coisa emprestada.}{em.prés.ti.mo}{0}
\verb{emproado}{}{}{}{}{adj.}{Enfatuado, orgulhoso, pretensioso, vaidoso, presunçoso.}{em.pro.a.do}{0}
\verb{emproar}{}{}{}{}{v.t.}{Tornar emproado; orgulhar.}{em.pro.ar}{0}
\verb{emproar}{}{}{}{}{}{Aproar.}{em.pro.ar}{\verboinum{7}}
\verb{empubescer}{ê}{}{}{}{v.i.}{Tornar"-se púbere, alcançar a puberdade.}{em.pu.bes.cer}{0}
\verb{empubescer}{ê}{}{}{}{}{Criar pelos.}{em.pu.bes.cer}{\verboinum{15}}
\verb{empulhação}{}{Pop.}{"-ões}{}{s.f.}{Ato ou efeito de empulhar; tapeação, embuste, mentira.}{em.pu.lha.ção}{0}
\verb{empulhar}{}{}{}{}{v.t.}{Enganar, lograr, iludir, tapear, ludibriar.}{em.pu.lhar}{\verboinum{1}}
\verb{empunhadura}{}{}{}{}{s.f.}{Parte pela qual se seguram ferramentas, armas, utensílios etc.; punho.}{em.pu.nha.du.ra}{0}
\verb{empunhar}{}{}{}{}{v.t.}{Segurar, pegar pela empunhadura ou pelo punho.}{em.pu.nhar}{\verboinum{1}}
\verb{empurra"-empurra}{}{Bras.}{empurras"-empurras \textit{ou} empurra"-empurras}{}{s.m.}{Confusão de pessoas que se esbarram e se acotovelam umas nas outras em locais muito cheios.}{em.pur.ra"-em.pur.ra}{0}
\verb{empurrão}{}{}{"-ões}{}{s.m.}{Ato ou efeito de empurrar; repelão, encontrão, empuxão.}{em.pur.rão}{0}
\verb{empurrãozinho}{}{Bras.}{empurrõezinhos}{}{s.m.}{Pequeno empurrão.}{em.pur.rão.zi.nho}{0}
\verb{empurrãozinho}{}{Fig.}{empurrõezinhos}{}{}{Estímulo; ajuda, auxílio.}{em.pur.rão.zi.nho}{0}
\verb{empurrar}{}{}{}{}{v.t.}{Impelir com força ou violência; empuxar.}{em.pur.rar}{0}
\verb{empurrar}{}{}{}{}{}{Dar encontrão.}{em.pur.rar}{0}
\verb{empurrar}{}{}{}{}{}{Obrigar, forçar alguém a aceitar alguma coisa; impingir, impor.}{em.pur.rar}{\verboinum{1}}
\verb{empuxão}{ch}{}{"-ões}{}{s.m.}{Ato ou efeito de empuxar; puxão, empurrão, repelão.}{em.pu.xão}{0}
\verb{empuxar}{ch}{}{}{}{v.t.}{Impelir com força ou violência; empurrar.}{em.pu.xar}{\verboinum{1}}
\verb{empuxo}{ch}{}{}{}{s.m.}{Ato ou efeito de empuxar.}{em.pu.xo}{0}
\verb{emu}{}{Zool.}{}{}{s.m.}{Ave grande, com até 1,5 m de altura, de asas pequenas, incapaz de voar, encontrada na Austrália.}{e.mu}{0}
\verb{emudecer}{ê}{}{}{}{v.t.}{Tornar mudo, calar.}{e.mu.de.cer}{\verboinum{15}}
\verb{emulação}{}{}{"-ões}{}{s.f.}{Sentimento que leva alguém a igualar ou suplantar outrem em habilidade, virtude, merecimento etc.}{e.mu.la.ção}{0}
\verb{emulação}{}{}{"-ões}{}{}{Competição, disputa, rivalidade, concorrência.}{e.mu.la.ção}{0}
\verb{emulação}{}{Informát.}{"-ões}{}{}{Capacidade de um programa ou de um dispositivo de simular as funções de outro.}{e.mu.la.ção}{0}
\verb{emular}{}{}{}{}{v.t.}{Competir, rivalizar com alguém; concorrer.}{e.mu.lar}{0}
\verb{emular}{}{Informát.}{}{}{}{Simular (um programa ou dispositivo) as funções de outro programa ou dispositivo.}{e.mu.lar}{\verboinum{1}}
\verb{êmulo}{}{}{}{}{s.m.}{Pessoa com quem se compete; rival, adversário, competidor.}{ê.mu.lo}{0}
\verb{emulsão}{}{}{"-ões}{}{s.f.}{Preparado farmacêutico cuja base é uma substância gordurosa em suspensão, de consistência leitosa, a qual se acrescenta um medicamento.}{e.mul.são}{0}
\verb{emulsificar}{}{}{}{}{v.t.}{Emulsionar.}{e.mul.si.fi.car}{\verboinum{2}}
\verb{emulsionar}{}{}{}{}{v.t.}{Fazer emulsão; emulsificar.}{e.mul.si.o.nar}{\verboinum{1}}
\verb{emulsivo}{}{}{}{}{adj.}{De que se pode extrair óleo por meio de pressão.}{e.mul.si.vo}{0}
\verb{emurchecer}{ê}{}{}{}{v.t.}{Tornar murcho; perder o viço, o frescor.}{e.mur.che.cer}{\verboinum{15}}
\verb{enaltecer}{ê}{}{}{}{v.t.}{Tornar alto, elevado.}{e.nal.te.cer}{\verboinum{15}}
\verb{enaltecer}{ê}{Fig.}{}{}{}{Louvar, engrandecer, exaltar.}{e.nal.te.cer}{0}
\verb{enamorado}{}{}{}{}{adj.}{Que se enamorou; apaixonado.}{e.na.mo.ra.do}{0}
\verb{enamorado}{}{}{}{}{s.m.}{Essa pessoa.}{e.na.mo.ra.do}{0}
\verb{enamorar}{}{}{}{}{v.t.}{Tornar apaixonado; encantar.}{e.na.mo.rar}{\verboinum{1}}
\verb{encabar}{}{}{}{}{v.t.}{Pôr cabo num instrumento, numa ferramenta.}{en.ca.bar}{\verboinum{1}}
\verb{encabeçar}{}{}{}{}{v.t.}{Dirigir, chefiar; ser o cabeça de.}{en.ca.be.çar}{0}
\verb{encabeçar}{}{}{}{}{}{Fazer o título de um escrito.}{en.ca.be.çar}{\verboinum{3}}
\verb{encabrestar}{}{}{}{}{v.t.}{Pôr cabresto numa cavalgadura.}{en.ca.bres.tar}{0}
\verb{encabrestar}{}{Fig.}{}{}{}{Submeter, dominar, subjugar.}{en.ca.bres.tar}{\verboinum{1}}
\verb{encabulação}{}{Bras.}{"-ões}{}{s.f.}{Ato ou efeito de encabular; vergonha, acanhamento, constrangimento.}{en.ca.bu.la.ção}{0}
\verb{encabulado}{}{}{}{}{adj.}{Acanhado, envergonhado, constrangido, vexado.}{en.ca.bu.la.do}{0}
\verb{encabular}{}{}{}{}{v.t.}{Causar ou sentir vergonha ou encabulação; vexar, acanhar, embaraçar.}{en.ca.bu.lar}{\verboinum{1}}
\verb{encachoeirado}{}{Bras.}{}{}{adj.}{Semelhante a ou que tem cachoeira.}{en.ca.cho.ei.ra.do}{0}
\verb{encachoeiramento}{}{Bras.}{}{}{s.m.}{Formação de cachoeira.}{en.ca.cho.ei.ra.men.to}{0}
\verb{encachoeirar}{}{}{}{}{v.t.}{Transformar em cachoeira.}{en.ca.cho.ei.rar}{\verboinum{1}}
\verb{encadeado}{}{}{}{}{adj.}{Preso com cadeia; sujeito.}{en.ca.de.a.do}{0}
\verb{encadeado}{}{}{}{}{}{Disposto em série.}{en.ca.de.a.do}{0}
\verb{encadeamento}{}{}{}{}{s.m.}{União, conexão.}{en.ca.de.a.men.to}{0}
\verb{encadeamento}{}{}{}{}{}{Sucessão de ideias correlacionadas.}{en.ca.de.a.men.to}{0}
\verb{encadear}{}{}{}{}{v.t.}{Prender com cadeia.}{en.ca.de.ar}{0}
\verb{encadear}{}{}{}{}{}{Coordenar ou dispor ideias, frases.}{en.ca.de.ar}{\verboinum{4}}
\verb{encadernação}{}{}{"-ões}{}{s.f.}{Ato ou efeito de encadernar.}{en.ca.der.na.ção}{0}
\verb{encadernação}{}{}{"-ões}{}{}{A capa e a costura ou colagem de um livro encadernado.}{en.ca.der.na.ção}{0}
\verb{encadernador}{ô}{}{}{}{s.m.}{Profissional que faz encadernação.}{en.ca.der.na.dor}{0}
\verb{encadernar}{}{}{}{}{v.t.}{Juntar folhas formando cadernos, unir esses cadernos e posteriormente colocar uma capa.}{en.ca.der.nar}{\verboinum{1}}
\verb{encafifado}{}{Bras.}{}{}{adj.}{Envergonhado, encabulado, cheio de vergonha.}{en.ca.fi.fa.do}{0}
\verb{encafifado}{}{}{}{}{}{Que pensa muito em alguma coisa que não compreende;  encucado, intrigado.}{en.ca.fi.fa.do}{0}
\verb{encafifar}{}{}{}{}{v.t.}{Encabular, envergonhar.}{en.ca.fi.far}{0}
\verb{encafifar}{}{}{}{}{}{Pensar muito em algo que não compreende; encucar, intrigar.}{en.ca.fi.far}{\verboinum{1}}
\verb{encafuar}{}{}{}{}{v.t.}{Colocar em cafua.}{en.ca.fu.ar}{0}
\verb{encafuar}{}{}{}{}{}{Ocultar, esconder.}{en.ca.fu.ar}{\verboinum{1}}
\verb{encaiporar}{}{}{}{}{v.t.}{Tornar caipora, infeliz, azarado.}{en.cai.po.rar}{\verboinum{1}}
\verb{encaixar}{ch}{}{}{}{v.t.}{Colocar em caixa.}{en.cai.xar}{0}
\verb{encaixar}{ch}{}{}{}{}{Ligar uma peça a outra, cujas formas facilitem essa ligação.}{en.cai.xar}{0}
\verb{encaixar}{ch}{Fig.}{}{}{}{Inserir algo ou alguém em um grupo já existente.}{en.cai.xar}{0}
\verb{encaixar}{ch}{}{}{}{v.pron.}{Introduzir"-se, intrometer"-se.}{en.cai.xar}{\verboinum{1}}
\verb{encaixe}{ch}{}{}{}{s.m.}{Ato ou efeito de encaixar.}{en.cai.xe}{0}
\verb{encaixe}{ch}{}{}{}{}{Cavidade ou saliência feita para adaptar uma peça a outra.}{en.cai.xe}{0}
\verb{encaixe}{ch}{}{}{}{}{União, conexão, juntura.}{en.cai.xe}{0}
\verb{encaixilhar}{ch}{}{}{}{v.t.}{Colocar em caixilho.}{en.cai.xi.lhar}{\verboinum{1}}
\verb{encaixotador}{ch\ldots{}ô}{}{}{}{s.m.}{Funcionário que encaixota mercadorias.}{en.cai.xo.ta.dor}{0}
\verb{encaixotar}{ch}{}{}{}{v.t.}{Colocar em caixotes ou caixas.}{en.cai.xo.tar}{\verboinum{1}}
\verb{encalacrar}{}{Pop.}{}{}{v.t.}{Colocar em situação difícil.}{en.ca.la.crar}{\verboinum{1}}
\verb{encalçar}{}{}{}{}{v.t.}{Seguir de perto; ir no encalço.}{en.cal.çar}{\verboinum{3}}
\verb{encalço}{}{}{}{}{s.m.}{Ato de encalçar.}{en.cal.ço}{0}
\verb{encalço}{}{}{}{}{}{Pista, rasto, vestígio.}{en.cal.ço}{0}
%\verb{}{}{}{}{}{}{}{}{0}
\verb{encalhar}{}{}{}{}{v.t.}{Fazer (embarcação) apoiar no fundo do mar ou rio.}{en.ca.lhar}{0}
\verb{encalhar}{}{Bras.}{}{}{v.i.}{Não ter venda; permanecer no estoque.}{en.ca.lhar}{\verboinum{1}}
\verb{encalhe}{}{}{}{}{s.m.}{Ato ou efeito de encalhar; obstrução.}{en.ca.lhe}{0}
\verb{encalhe}{}{Bras.}{}{}{}{Mercadorias que não foram vendidas.}{en.ca.lhe}{0}
\verb{encalistrar}{}{Bras.}{}{}{v.t.}{Envergonhar, encabular.}{en.ca.lis.trar}{\verboinum{1}}
\verb{encalmar}{}{}{}{}{v.t.}{Tornar calmoso; aquecer.}{en.cal.mar}{0}
\verb{encalmar}{}{}{}{}{}{Zangar, esquentar, aborrecer.}{en.cal.mar}{0}
\verb{encalmar}{}{}{}{}{v.i.}{Tranquilizar"-se, acalmar"-se.}{en.cal.mar}{\verboinum{1}}
\verb{encalombar}{}{Bras.}{}{}{v.i.}{Criar calombo.}{en.ca.lom.bar}{\verboinum{1}}
\verb{encalorado}{}{}{}{}{adj.}{Cheio de calor.}{en.ca.lo.ra.do}{0}
\verb{encalvecer}{ê}{}{}{}{v.i.}{Tornar"-se calvo.}{en.cal.ve.cer}{\verboinum{15}}
\verb{encaminhar}{}{}{}{}{v.t.}{Mostrar o caminho; conduzir, guiar, orientar.}{en.ca.mi.nhar}{0}
\verb{encaminhar}{}{}{}{}{}{Pôr no bom caminho; aconselhar bem.}{en.ca.mi.nhar}{0}
\verb{encaminhar}{}{}{}{}{}{Fazer seguir pelos meios cabíveis.}{en.ca.mi.nhar}{0}
\verb{encaminhar}{}{}{}{}{v.pron.}{Tender para uma situação conclusiva.}{en.ca.mi.nhar}{\verboinum{1}}
\verb{encampador}{ô}{}{}{}{adj.}{Que encampa.}{en.cam.pa.dor}{0}
\verb{encampar}{}{}{}{}{v.t.}{Tomar posse de, geralmente empresa, mediante acordo de indenização.}{en.cam.par}{\verboinum{1}}
\verb{encanado}{}{}{}{}{adj.}{Canalizado.}{en.ca.na.do}{0}
\verb{encanado}{}{Pop.}{}{}{}{Que foi em cana; preso.}{en.ca.na.do}{0}
\verb{encanador}{ô}{}{}{}{s.m.}{Profissional que instala e conserta canos de água e gás, pias, tanques e aparelhos sanitários.}{en.ca.na.dor}{0}
\verb{encanamento}{}{}{}{}{s.m.}{Ato ou efeito de encanar.}{en.ca.na.men.to}{0}
\verb{encanamento}{}{}{}{}{}{Conjunto de canos.}{en.ca.na.men.to}{0}
\verb{encanar}{}{}{}{}{v.t.}{Colocar em cano ou canal.}{en.ca.nar}{0}
\verb{encanar}{}{}{}{}{}{Instalar rede de canos.}{en.ca.nar}{0}
\verb{encanar}{}{}{}{}{}{Imobilizar osso fraturado com canas ou talas.}{en.ca.nar}{0}
\verb{encanar}{}{Pop.}{}{}{}{Colocar em cana; prender.}{en.ca.nar}{\verboinum{1}}
\verb{encandear}{}{}{}{}{v.t.}{Atrair peixe ou caça ofuscando"-o com o candeio.}{en.can.de.ar}{0}
\verb{encandear}{}{Fig.}{}{}{}{Deslumbrar, estontear, cegar.}{en.can.de.ar}{\verboinum{4}}
\verb{encanecer}{ê}{}{}{}{v.t.}{Tornar brancos (os cabelos) gradativamente.}{en.ca.ne.cer}{\verboinum{15}}
\verb{encangalhar}{}{}{}{}{v.t.}{Colocar cangalha em animal de carga.}{en.can.ga.lhar}{\verboinum{1}}
\verb{encangar}{}{}{}{}{v.t.}{Pôr canga em.}{en.can.gar}{\verboinum{5}}
\verb{encantado}{}{}{}{}{adj.}{Que sofreu ou foi criado por encantamento, feitiço, sortilégio.}{en.can.ta.do}{0}
\verb{encantado}{}{}{}{}{}{Arrebatado, fascinado, seduzido.}{en.can.ta.do}{0}
\verb{encantado}{}{}{}{}{}{Muito satisfeito.}{en.can.ta.do}{0}
\verb{encantador}{ô}{}{}{}{adj.}{Que seduz, atrai.}{en.can.ta.dor}{0}
\verb{encantador}{ô}{}{}{}{}{Que causa grande satisfação; deleitoso.}{en.can.ta.dor}{0}
\verb{encantamento}{}{}{}{}{s.m.}{Ato ou efeito de encantar.}{en.can.ta.men.to}{0}
\verb{encantar}{}{}{}{}{v.t.}{Lançar feitiço, magia.}{en.can.tar}{0}
\verb{encantar}{}{}{}{}{}{Seduzir, arrebatar, cativar.}{en.can.tar}{0}
\verb{encantar}{}{}{}{}{}{Causar prazer; deleitar, deliciar.}{en.can.tar}{\verboinum{1}}
\verb{encanto}{}{}{}{}{s.m.}{Coisa que encanta, seduz, delicia.}{en.can.to}{0}
\verb{encanto}{}{}{}{}{}{Situação de quem ou o que foi submetido a magia, feitiço.}{en.can.to}{0}
\verb{encantoar}{}{}{}{}{v.t.}{Pôr em um canto.}{en.can.to.ar}{0}
\verb{encantoar}{}{}{}{}{}{Afastar do convívio social.}{en.can.to.ar}{\verboinum{7}}
\verb{encanzinar}{}{}{}{}{v.t.}{Fazer zangar; enfurecer.}{en.can.zi.nar}{\verboinum{1}}
\verb{encapar}{}{}{}{}{v.t.}{Revestir com capa.}{en.ca.par}{\verboinum{1}}
\verb{encapelar}{}{}{}{}{v.t.}{Tornar agitado (especialmente o mar); encrespar.}{en.ca.pe.lar}{\verboinum{1}}
\verb{encapetado}{}{Bras.}{}{}{adj.}{Endiabrado, travesso.}{en.ca.pe.ta.do}{0}
\verb{encapetar"-se}{}{}{}{}{v.pron.}{Fazer travessuras; tornar"-se travesso.}{en.ca.pe.tar"-se}{\verboinum{1}}
%\verb{}{}{}{}{}{}{}{}{0}
\verb{encapotar}{}{}{}{}{v.t.}{Cobrir com capa ou capote.}{en.ca.po.tar}{0}
\verb{encapotar}{}{Fig.}{}{}{}{Esconder, disfarçar.}{en.ca.po.tar}{\verboinum{1}}
\verb{encapuzar}{}{}{}{}{v.t.}{Cobrir com capuz.}{en.ca.pu.zar}{\verboinum{1}}
\verb{encaracolado}{}{}{}{}{adj.}{Enrolado de forma semelhante a caracol ou espiral.}{en.ca.ra.co.la.do}{0}
\verb{encaracolar}{}{}{}{}{v.t.}{Enrolar em forma de caracol ou espiral.}{en.ca.ra.co.lar}{\verboinum{1}}
\verb{encaramujar"-se}{}{}{}{}{v.pron.}{Encolher"-se como o caramujo.}{en.ca.ra.mu.jar"-se}{0}
\verb{encaramujar"-se}{}{}{}{}{}{Entristecer"-se, deprimir"-se.}{en.ca.ra.mu.jar"-se}{\verboinum{1}}
\verb{encarangado}{}{}{}{}{adj.}{Entrevado, acarangado.}{en.ca.ran.ga.do}{0}
\verb{encarangado}{}{Bras.}{}{}{}{Franzino, frágil, pouco desenvolvido.}{en.ca.ran.ga.do}{0}
\verb{encarangar}{}{}{}{}{v.t.}{Prejudicar os movimentos; paralisar.}{en.ca.ran.gar}{0}
\verb{encarangar}{}{Bras.}{}{}{}{Encolher, enrugar.}{en.ca.ran.gar}{0}
\verb{encarangar}{}{}{}{}{v.pron.}{Tornar"-se adoentado.}{en.ca.ran.gar}{\verboinum{5}}
\verb{encarapinhado}{}{}{}{}{adj.}{Diz"-se do cabelo crespo e lanoso.}{en.ca.ra.pi.nha.do}{0}
\verb{encarapinhar}{}{}{}{}{v.t.}{Fazer carapinha; encrespar.}{en.ca.ra.pi.nhar}{\verboinum{1}}
\verb{encarapitar}{}{}{}{}{v.t.}{Colocar no alto.}{en.ca.ra.pi.tar}{\verboinum{1}}
\verb{encarapuçar}{}{}{}{}{v.t.}{Pôr carapuça em.}{en.ca.ra.pu.çar}{\verboinum{3}}
\verb{encarar}{}{}{}{}{v.t.}{Olhar de frente e fixamente; fitar.}{en.ca.rar}{0}
\verb{encarar}{}{}{}{}{}{Enfrentar.}{en.ca.rar}{\verboinum{1}}
\verb{encarcerar}{}{}{}{}{v.t.}{Colocar no cárcere.}{en.car.ce.rar}{0}
\verb{encarcerar}{}{}{}{}{}{Afastar do convívio social.}{en.car.ce.rar}{\verboinum{1}}
\verb{encardido}{}{}{}{}{adj.}{Muito sujo; imundo.}{en.car.di.do}{0}
\verb{encardido}{}{}{}{}{}{Que adquiriu cor acinzentada ou amarelada.}{en.car.di.do}{0}
\verb{encardido}{}{}{}{}{}{Diz"-se da pele sem brilho ou aspecto saudável devido a doença, velhice ou falta de cuidado.}{en.car.di.do}{0}
\verb{encardir}{}{}{}{}{v.t.}{Sujar ou deixar que se suje.}{en.car.dir}{\verboinum{18}}
\verb{encarecer}{ê}{}{}{}{v.t.}{Tornar caro.}{en.ca.re.cer}{0}
\verb{encarecer}{ê}{}{}{}{}{Elogiar, exaltar.}{en.ca.re.cer}{\verboinum{15}}
\verb{encarecimento}{}{}{}{}{s.m.}{Ato ou efeito de encarecer.}{en.ca.re.ci.men.to}{0}
\verb{encargo}{}{}{}{}{s.m.}{Responsabilidade, obrigação.}{en.car.go}{0}
\verb{encargo}{}{}{}{}{}{Sentimento de culpa; remorso.}{en.car.go}{0}
\verb{encarnação}{}{}{"-ões}{}{s.f.}{Ato ou efeito de encarnar.}{en.car.na.ção}{0}
\verb{encarnação}{}{Relig.}{"-ões}{}{}{Cada uma das existências físicas de um espírito, segundo algumas crenças religiosas.}{en.car.na.ção}{0}
\verb{encarnação}{}{Relig.}{"-ões}{}{}{Materialização dos seres divinos.}{en.car.na.ção}{0}
\verb{encarnado}{}{}{}{}{adj.}{Que encarnou.}{en.car.na.do}{0}
\verb{encarnado}{}{}{}{}{}{Da cor vermelha da carne.}{en.car.na.do}{0}
\verb{encarnar}{}{}{}{}{v.t.}{Dar a cor vermelha da carne.}{en.car.nar}{0}
\verb{encarnar}{}{}{}{}{}{Penetrar (um espírito) em um corpo.}{en.car.nar}{0}
\verb{encarnar}{}{}{}{}{}{Representar (um papel) nas artes cênicas.}{en.car.nar}{0}
\verb{encarnar}{}{}{}{}{}{Ser a personificação de.}{en.car.nar}{0}
\verb{encarnar}{}{Relig.}{}{}{v.i.}{Fazer"-se homem (entidades divinas).}{en.car.nar}{\verboinum{1}}
\verb{encarne}{}{}{}{}{s.m.}{Ato ou efeito de encarnar.}{en.car.ne}{0}
\verb{encarne}{}{}{}{}{}{Carne dada aos cães de caça para que se acostumem ao cheiro dela.}{en.car.ne}{0}
\verb{encarniçado}{}{}{}{}{adj.}{Que se alimenta de carniça.}{en.car.ni.ça.do}{0}
\verb{encarniçado}{}{}{}{}{}{Sanguinário, feroz, cruel.}{en.car.ni.ça.do}{0}
\verb{encarniçamento}{}{}{}{}{s.m.}{Ato de encarniçar"-se sobre a presa.}{en.car.ni.ça.men.to}{0}
\verb{encarniçamento}{}{}{}{}{}{Teimosia, obstinação.}{en.car.ni.ça.men.to}{0}
\verb{encarniçamento}{}{}{}{}{}{Ódio, rancor.}{en.car.ni.ça.men.to}{0}
\verb{encarniçar}{}{}{}{}{v.t.}{Dar carne aos cães de caça para torná"-los ferozes.}{en.car.ni.çar}{0}
\verb{encarniçar}{}{}{}{}{}{Incitar, açular.}{en.car.ni.çar}{\verboinum{3}}
\verb{encaroçar}{}{Bras.}{}{}{v.i.}{Criar caroço.}{en.ca.ro.çar}{\verboinum{3}}
\verb{encarquilhar}{}{}{}{}{v.t.}{Encher de carquilhas ou rugas.}{en.car.qui.lhar}{\verboinum{1}}
\verb{encarregado}{}{}{}{}{adj.}{Que tem a responsabilidade por alguma coisa, geralmente um cargo.}{en.car.re.ga.do}{0}
\verb{encarregado}{}{}{}{}{s.m.}{Indivíduo responsável por vigiar os funcionários de uma seção ou construção.}{en.car.re.ga.do}{0}
\verb{encarregar}{}{}{}{}{v.t.}{Dar cargo, missão, ocupação.}{en.car.re.gar}{0}
\verb{encarregar}{}{}{}{}{}{Encomendar, recomendar.}{en.car.re.gar}{\verboinum{5}}
\verb{encarreirar}{}{}{}{}{v.t.}{Abrir caminho; encaminhar, orientar.}{en.car.rei.rar}{\verboinum{1}}
\verb{encarrilar}{}{}{}{}{v.t.}{Encarrilhar.}{en.car.ri.lar}{\verboinum{1}}
\verb{encarrilhar}{}{}{}{}{}{Pôr no bom caminho; encaminhar.}{en.car.ri.lhar}{\verboinum{1}}
\verb{encarrilhar}{}{}{}{}{v.t.}{Pôr nos trilhos ou calhas.}{en.car.ri.lhar}{0}
\verb{encartar}{}{}{}{}{}{Fazer um encarte de.}{en.car.tar}{\verboinum{1}}
\verb{encartar}{}{}{}{}{v.t.}{Conceder licença ou diploma para o exercício de certo ofício.}{en.car.tar}{0}
\verb{encarte}{}{}{}{}{s.m.}{Ato de encartar em um emprego.}{en.car.te}{0}
\verb{encarte}{}{}{}{}{}{Folhas especiais intercaladas no caderno de uma publicação.}{en.car.te}{0}
\verb{encartuchar}{}{}{}{}{v.t.}{Colocar em cartucho.}{en.car.tu.char}{\verboinum{1}}
\verb{encarvoar}{}{}{}{}{v.t.}{Sujar com carvão.}{en.car.vo.ar}{\verboinum{7}}
\verb{encasacado}{}{}{}{}{adj.}{Vestido com casaco.}{en.ca.sa.ca.do}{0}
\verb{encasacar}{}{}{}{}{v.t.}{Vestir casaco ou casaca.}{en.ca.sa.car}{0}
\verb{encasacar}{}{}{}{}{v.pron.}{Pôr traje de cerimônia.}{en.ca.sa.car}{\verboinum{2}}
\verb{encasquetar}{}{}{}{}{v.t.}{Cobrir com casquete.}{en.cas.que.tar}{0}
\verb{encasquetar}{}{}{}{}{}{Introduzir ideia na cabeça; ter ideia fixa.}{en.cas.que.tar}{\verboinum{1}}
\verb{encastelar}{}{}{}{}{v.t.}{Construir ou fortificar como castelo.}{en.cas.te.lar}{0}
\verb{encastelar}{}{}{}{}{}{Amontoar, empilhar.}{en.cas.te.lar}{\verboinum{1}}
\verb{encastoar}{}{}{}{}{v.t.}{Pôr castão.}{en.cas.to.ar}{0}
\verb{encastoar}{}{}{}{}{}{Engastar, embutir.}{en.cas.to.ar}{\verboinum{7}}
\verb{encatarrar"-se}{}{}{}{}{v.pron.}{Criar catarro.}{en.ca.tar.rar"-se}{\verboinum{1}}
\verb{encatarroar"-se}{}{}{}{}{v.pron.}{Encatarrar"-se.}{en.ca.tar.ro.ar"-se}{\verboinum{7}}
\verb{encavacar}{}{}{}{}{v.i.}{Dar o cavaco.}{en.ca.va.car}{0}
\verb{encavacar}{}{}{}{}{}{Envergonhar"-se.}{en.ca.va.car}{\verboinum{2}}
\verb{encavalar}{}{}{}{}{v.t.}{Montar a cavalo; cavalgar.}{en.ca.va.lar}{0}
\verb{encavalar}{}{}{}{}{}{Sobrepor, amontoar.}{en.ca.va.lar}{\verboinum{1}}
\verb{encefálico}{}{}{}{}{adj.}{Relativo ao encéfalo.}{en.ce.fá.li.co}{0}
\verb{encefalite}{}{Med.}{}{}{s.f.}{Inflamação no encéfalo.}{en.ce.fa.li.te}{0}
\verb{encéfalo}{}{Anat.}{}{}{s.m.}{Parte do sistema nervoso que fica na cabeça e que compreende o cérebro, o tronco cerebral e o bulbo.}{en.cé.fa.lo}{0}
\verb{encefalograma}{}{Med.}{}{}{s.m.}{Radiografia do conteúdo do crânio.}{en.ce.fa.lo.gra.ma}{0}
\verb{enceguecer}{ê}{}{}{}{v.i.}{Tornar"-se cego.}{en.ce.gue.cer}{\verboinum{15}}
\verb{enceleirar}{}{}{}{}{v.t.}{Guardar em celeiro.}{en.ce.lei.rar}{0}
\verb{enceleirar}{}{}{}{}{}{Armazenar, juntar.}{en.ce.lei.rar}{\verboinum{1}}
\verb{encenação}{}{}{"-ões}{}{s.f.}{Montagem e execução de cena teatral ou cinematográfica.}{en.ce.na.ção}{0}
\verb{encenação}{}{Pop.}{"-ões}{}{}{Fingimento.}{en.ce.na.ção}{0}
\verb{encenador}{ô}{}{}{}{s.m.}{Indivíduo que põe em cena uma peça; diretor teatral.}{en.ce.na.dor}{0}
\verb{encenar}{}{}{}{}{v.t.}{Apresentar peça teatral.}{en.ce.nar}{0}
\verb{encenar}{}{Fig.}{}{}{}{Fingir, simular.}{en.ce.nar}{\verboinum{1}}
\verb{enceradeira}{ê}{}{}{}{s.f.}{Aparelho elétrico para encerar o chão.}{en.ce.ra.dei.ra}{0}
\verb{encerado}{}{}{}{}{adj.}{Que está coberto de cera.}{en.ce.ra.do}{0}
\verb{encerado}{}{}{}{}{}{Que foi lustrado com cera.}{en.ce.ra.do}{0}
\verb{encerado}{}{}{}{}{s.m.}{Pano revestido de cera, verniz, óleo etc., para se tornar impermeável; oleado.}{en.ce.ra.do}{0}
\verb{encerar}{}{}{}{}{v.t.}{Passar cera em assoalho ou móveis.}{en.ce.rar}{\verboinum{1}}
\verb{encerramento}{}{}{}{}{s.m.}{Ato ou efeito de encerrar; conclusão, término.}{en.cer.ra.men.to}{0}
\verb{encerrar}{}{}{}{}{v.t.}{Colocar uma pessoa ou coisa dentro de um lugar fechado.}{en.cer.rar}{0}
\verb{encerrar}{}{}{}{}{}{Fazer uma atividade chegar ao fim; concluir, findar, terminar.}{en.cer.rar}{0}
\verb{encerrar}{}{}{}{}{}{Ter dentro de si; conter.}{en.cer.rar}{\verboinum{1}}
\verb{encestar}{}{}{}{}{v.t.}{Guardar alguma coisa em um cesto.}{en.ces.tar}{0}
\verb{encestar}{}{Esport.}{}{}{v.i.}{Em basquete, fazer a bola entrar na cesta.}{en.ces.tar}{\verboinum{1}}
\verb{encetar}{}{}{}{}{v.t.}{Dar início a alguma ação; começar, iniciar.}{en.ce.tar}{\verboinum{1}}
\verb{enchapelar}{}{}{}{}{v.t.}{Pôr chapéu.}{en.cha.pe.lar}{\verboinum{1}}
\verb{encharcadiço}{}{}{}{}{adj.}{Que está sujeito a encharcar.}{en.char.ca.di.ço}{0}
\verb{encharcar}{}{}{}{}{v.t.}{Transformar um lugar em um charco; alagar.}{en.char.car}{0}
\verb{encharcar}{}{}{}{}{}{Deixar alguma coisa muito molhada; ensopar.}{en.char.car}{\verboinum{2}}
\verb{encheção}{}{}{"-ões}{}{s.f.}{Ato ou efeito de encher; amolação, aborrecimento, chateação.}{en.che.ção}{0}
\verb{enchente}{}{}{}{}{s.f.}{Acúmulo de águas causado por maré, chuva forte etc.}{en.chen.te}{0}
\verb{encher}{ê}{}{}{}{v.t.}{Ocupar determinado espaço ou determinado volume; tornar cheio ou pleno.}{en.cher}{0}
\verb{encher}{ê}{Pop.}{}{}{}{Aborrecer, amolar, chatear.}{en.cher}{\verboinum{12}}
\verb{enchido}{}{}{}{}{adj.}{Que se encheu; cheio, repleto.}{en.chi.do}{0}
\verb{enchido}{}{Lus.}{}{}{s.m.}{Peça de salsicharia; embutido.}{en.chi.do}{0}
\verb{enchimento}{}{}{}{}{s.m.}{Ato ou efeito de encher.}{en.chi.men.to}{0}
\verb{enchimento}{}{}{}{}{}{Coisa com que se enche; chumaço, recheio.}{en.chi.men.to}{0}
\verb{enchiqueirar}{}{Bras.}{}{}{v.t.}{Meter no chiqueiro.}{en.chi.quei.rar}{\verboinum{1}}
\verb{enchova}{ô}{}{}{}{}{Var. de \textit{anchova}.}{en.cho.va}{0}
\verb{enchumaçar}{}{}{}{}{v.t.}{Pôr chumaço.}{en.chu.ma.çar}{0}
\verb{enchumaçar}{}{}{}{}{}{Estofar, acolchoar.}{en.chu.ma.çar}{\verboinum{3}}
\verb{encíclica}{}{}{}{}{s.f.}{Carta circular do papa abordando algum tema da doutrina católica.}{en.cí.cli.ca}{0}
\verb{enciclopédia}{}{}{}{}{s.f.}{Livro de consulta, sobre todas as áreas do conhecimento.}{en.ci.clo.pé.dia}{0}
\verb{enciclopédico}{}{}{}{}{adj.}{Relativo a enciclopédia.}{en.ci.clo.pé.di.co}{0}
\verb{enciclopédico}{}{}{}{}{}{Que abrange todo o saber.}{en.ci.clo.pé.di.co}{0}
\verb{enciclopedista}{}{}{}{}{adj.2g.}{Diz"-se de indivíduo que atua como autor ou colaborador de enciclopédia.}{en.ci.clo.pe.dis.ta}{0}
\verb{encilhar}{}{}{}{}{v.t.}{Preparar um animal para ser montado; arrear.}{en.ci.lhar}{\verboinum{1}}
\verb{encimar}{}{}{}{}{v.t.}{Colocar alguma coisa no alto de outra.}{en.ci.mar}{\verboinum{1}}
\verb{enciumar}{}{}{}{}{v.t.}{Encher alguém de ciúme.}{en.ci.u.mar}{\verboinum{8}}
\verb{enclaustrar}{}{}{}{}{v.t.}{Meter no convento; enclausurar.}{en.claus.trar}{0}
\verb{enclaustrar}{}{}{}{}{}{Prender, encarcerar, encerrar.}{en.claus.trar}{\verboinum{1}}
\verb{enclausurar}{}{}{}{}{v.t.}{Fazer alguém viver fechado em um lugar; isolar.}{en.clau.su.rar}{\verboinum{1}}
\verb{enclave}{}{}{}{}{}{Var. de \textit{encrave}.}{en.cla.ve}{0}
\verb{enclavinhar}{}{}{}{}{v.t.}{Travar ou cruzar mãos, dedos, pernas etc., fortemente; apertar.}{en.cla.vi.nhar}{\verboinum{1}}
\verb{ênclise}{}{Gram.}{}{}{s.f.}{Colocação do pronome átono depois do verbo.}{ên.cli.se}{0}
\verb{enclítico}{}{}{}{}{adj.}{Relativo a ou em que ocorre ênclise.}{en.clí.ti.co}{0}
\verb{encoberto}{é}{}{}{}{adj.}{Que está oculto, escondido.}{en.co.ber.to}{0}
\verb{encoberto}{é}{}{}{}{}{Que está incógnito, disfarçado.}{en.co.ber.to}{0}
\verb{encoberto}{é}{}{}{}{}{Coberto de nuvens.}{en.co.ber.to}{0}
\verb{encobrir}{}{}{}{}{v.t.}{Impedir pessoa ou coisa de ser vista; esconder, tapar.}{en.co.brir}{0}
\verb{encobrir}{}{}{}{}{}{Deixar alguma coisa em segredo; esconder, ocultar.}{en.co.brir}{\verboinum{31}}
\verb{encoiraçado}{}{}{}{}{}{Var. de \textit{encouraçado}.}{en.coi.ra.ça.do}{0}
\verb{encoiraçar}{}{}{}{}{}{Var. de \textit{encouraçar}. }{en.coi.ra.çar}{0}
\verb{encolerizar}{}{}{}{}{v.t.}{Fazer alguém ficar cheio de raiva; enraivecer.}{en.co.le.ri.zar}{\verboinum{1}}
\verb{encolha}{ô}{}{}{}{s.f.}{Encolhimento.}{en.co.lha}{0}
\verb{encolher}{ê}{}{}{}{v.t.}{Diminuir o tamanho de alguma coisa; contrair, encurtar.}{en.co.lher}{0}
\verb{encolher}{ê}{}{}{}{}{Puxar parte do corpo para ocupar menos espaço; contrair, retrair.}{en.co.lher}{0}
\verb{encolher}{ê}{}{}{}{v.pron.}{Mostrar"-se acanhado e medroso.}{en.co.lher}{\verboinum{12}}
\verb{encolhimento}{}{}{}{}{s.m.}{Ato ou efeito de encolher.}{en.co.lhi.men.to}{0}
\verb{encolhimento}{}{}{}{}{}{Retraimento, timidez.}{en.co.lhi.men.to}{0}
\verb{encolhimento}{}{}{}{}{}{Falta de energia; submissão.}{en.co.lhi.men.to}{0}
\verb{encomenda}{}{}{}{}{}{Aquilo que se encomenda; incumbência, encargo.}{en.co.men.da}{0}
\verb{encomenda}{}{}{}{}{s.f.}{Ato ou efeito de encomendar; encomendação.}{en.co.men.da}{0}
\verb{encomenda}{}{Bras.}{}{}{}{Feitiço, mandinga, bruxaria.}{en.co.men.da}{0}
\verb{encomendação}{}{}{"-ões}{}{s.f.}{Ato ou efeito de encomendar; incumbência, encargo.}{en.co.men.da.ção}{0}
\verb{encomendação}{}{}{"-ões}{}{}{Recomendação, conselho, advertência.}{en.co.men.da.ção}{0}
\verb{encomendação}{}{Relig.}{"-ões}{}{}{Oração por um defunto, recitada pelo sacerdote antes do enterro; encomendação do corpo.}{en.co.men.da.ção}{0}
\verb{encomendar}{}{}{}{}{v.t.}{Mandar fazer alguma coisa; incumbir, encarregar, comissionar.}{en.co.men.dar}{0}
\verb{encomendar}{}{}{}{}{v.pron.}{Entregar"-se, confiar"-se à proteção de alguém.}{en.co.men.dar}{\verboinum{1}}
\verb{encomiar}{}{}{}{}{v.t.}{Dirigir encômios a alguém; elogiar, louvar.}{en.co.mi.ar}{\verboinum{1}}
\verb{encomiasta}{}{}{}{}{s.2g.}{Pessoa encarregada de fazer o elogio de alguém ou de alguma coisa; elogiador, panegirista. }{en.co.mi.as.ta}{0}
\verb{encomiástico}{}{}{}{}{adj.}{Em que há encômios; laudatório, elogioso.}{en.co.mi.ás.ti.co}{0}
\verb{encômio}{}{}{}{}{s.m.}{Fala ou discurso com o propósito de elogiar alguém ou alguma coisa; louvor, elogio, apologia, panegírico, gabo, aplauso.}{en.cô.mio}{0}
\verb{encompridar}{}{}{}{}{v.t.}{Tornar mais comprido; espichar.}{en.com.pri.dar}{0}
\verb{encompridar}{}{}{}{}{}{Fazer durar conversa, discurso etc., geralmente mais do que o necessário ou o aceitável.  (\textit{Encompridou a conversa, fazendo"-nos bocejar.})}{en.com.pri.dar}{\verboinum{1}}
\verb{encontradiço}{}{}{}{}{adj.}{Que se encontra facilmente ou com frequência.}{en.con.tra.di.ço}{0}
\verb{encontrão}{}{}{"-ões}{}{s.m.}{Empurrão, choque, embate.}{en.con.trão}{0}
\verb{encontrar}{}{}{}{}{v.t.}{Ir de encontro a; topar com; deparar.}{en.con.trar}{0}
\verb{encontrar}{}{}{}{}{}{Achar, atinar com. (\textit{Encontramos a solução dos nossos problemas!})}{en.con.trar}{0}
\verb{encontrar}{}{}{}{}{v.pron.}{Chocar"-se; colidir.}{en.con.trar}{0}
\verb{encontrar}{}{}{}{}{}{Unir"-se, ir ter com.}{en.con.trar}{\verboinum{1}}
\verb{encontro}{}{}{}{}{s.m.}{Ato de encontrar.}{en.con.tro}{0}
\verb{encontro}{}{}{}{}{}{Choque, topada, empurrão.}{en.con.tro}{0}
\verb{encontro}{}{}{}{}{}{Reunião de pessoas ou coisas.}{en.con.tro}{0}
\verb{encorajar}{}{}{}{}{v.t.}{Dar coragem a; estimular.}{en.co.ra.jar}{\verboinum{1}}
\verb{encordoamento}{}{}{}{}{s.m.}{Ato ou efeito de encordoar.}{en.cor.do.a.men.to}{0}
\verb{encordoamento}{}{}{}{}{}{Conjunto de cordas colocadas num instrumento musical, numa raquete de tênis etc.}{en.cor.do.a.men.to}{0}
\verb{encordoar}{}{}{}{}{v.t.}{Colocar cordas num instrumento musical, numa raquete de tênis etc.}{en.cor.do.ar}{\verboinum{7}}
\verb{encorpado}{}{}{}{}{adj.}{Que tem muito corpo; corpulento, forte, avolumado.}{en.cor.pa.do}{0}
\verb{encorpado}{}{}{}{}{}{Que tem consistência, espessura; denso, substancioso. (\textit{A calda do doce ficou encorpada.})}{en.cor.pa.do}{0}
\verb{encorpar}{}{}{}{}{v.t.}{Dar mais corpo; engrossar, aumentar, ampliar.}{en.cor.par}{0}
\verb{encorpar}{}{}{}{}{v.i.}{Tomar corpo; crescer, desenvolver"-se fisicamente.}{en.cor.par}{\verboinum{1}}
\verb{encortinar}{}{}{}{}{v.t.}{Pôr cortina.}{en.cor.ti.nar}{\verboinum{1}}
\verb{encorujar"-se}{}{}{}{}{v.pron.}{Ficar retraído, esconder"-se, fugir ao convívio; isolar"-se, ensimesmar"-se, esquivar"-se.}{en.co.ru.jar"-se}{0}
\verb{encorujar"-se}{}{}{}{}{}{Ficar triste, jururu.}{en.co.ru.jar"-se}{\verboinum{1}}
\verb{encoscorar}{}{}{}{}{v.t.}{Encher de coscoros; encrespar.}{en.cos.co.rar}{\verboinum{1}}
\verb{encosta}{ó}{}{}{}{s.f.}{Declive, inclinação acentuada de um terreno, montanha, colina etc.; ladeira, vertente.}{en.cos.ta}{0}
\verb{encostado}{}{}{}{}{adj.}{Que se encostou; apoiado, arrimado.}{en.cos.ta.do}{0}
\verb{encostado}{}{Fig.}{}{}{}{Que vive à custa de outra pessoa.}{en.cos.ta.do}{0}
\verb{encostado}{}{Fig.}{}{}{}{Que não se casou; solteirão, encalhado.}{en.cos.ta.do}{0}
\verb{encostado}{}{Fig.}{}{}{}{Que não gosta de trabalhar ou se esforçar; preguiçoso, vagabundo.}{en.cos.ta.do}{0}
\verb{encostar}{}{}{}{}{v.t.}{Colocar alguma coisa perto de outra, fazendo a primeira tocar na segunda.}{en.cos.tar}{0}
\verb{encostar}{}{}{}{}{}{Deixar veículo parado em algum lugar. (\textit{O rapaz encostou o carro para conversar com a moça.})}{en.cos.tar}{0}
\verb{encostar}{}{}{}{}{}{Deixar de usar alguma coisa; aposentar. (\textit{Depois de tanto tempo, consegui encostar o velho tênis.})}{en.cos.tar}{0}
\verb{encostar}{}{}{}{}{}{Fechar porta ou janela sem passar a chave. (\textit{Quando sair, por favor, encoste a porta.})}{en.cos.tar}{\verboinum{1}}
\verb{encosto}{ô}{}{}{}{s.m.}{Coisa a que alguém ou algo se encosta. }{en.cos.to}{0}
\verb{encosto}{ô}{Fig.}{}{}{}{Proteção, arrimo, amparo.}{en.cos.to}{0}
\verb{encouraçado}{}{}{}{}{adj.}{Couraçado.}{en.cou.ra.ça.do}{0}
\verb{encouraçado}{}{}{}{}{s.m.}{Grande navio de guerra, armado de canhões, e fortemente protegido por couraças.}{en.cou.ra.ça.do}{0}
\verb{encouraçar}{}{}{}{}{v.t.}{Prover, revestir de couraças; couraçar, blindar.}{en.cou.ra.çar}{\verboinum{3}}
\verb{encovado}{}{}{}{}{adj.}{Que se encovou, que foi colocado em cova ou buraco.}{en.co.va.do}{0}
\verb{encovado}{}{}{}{}{}{Diz"-se dos olhos que parecem estar afundados nas órbitas.}{en.co.va.do}{0}
\verb{encovar}{}{}{}{}{v.t.}{Pôr em covas; enterrar.}{en.co.var}{0}
\verb{encovar}{}{}{}{}{}{Esconder, ocultar.}{en.co.var}{\verboinum{1}}
\verb{encravado}{}{}{}{}{adj.}{Que se encravou; cravado, fixo, preso.}{en.cra.va.do}{0}
\verb{encravado}{}{}{}{}{}{Diz"-se de unha ou pelo que cresceu penetrando na carne ou na pele.}{en.cra.va.do}{0}
\verb{encravar}{}{}{}{}{v.t.}{Fixar, pregar cravo, prego; fincar, cravar.}{en.cra.var}{\verboinum{1}}
\verb{encrave}{}{}{}{}{s.m.}{Território situado dentro de outro; enclave.}{en.cra.ve}{0}
\verb{encrenca}{}{}{}{}{s.f.}{Dificuldade séria e perigosa; complicação, enrascada.}{en.cren.ca}{0}
\verb{encrenca}{}{}{}{}{}{Desentendimento com outras pessoas.}{en.cren.ca}{0}
\verb{encrencar}{}{}{}{}{v.t.}{Tornar uma situação difícil ou complicada; complicar.}{en.cren.car}{0}
\verb{encrencar}{}{}{}{}{}{Criar encrenca com alguém; atritar, implicar.}{en.cren.car}{0}
\verb{encrencar}{}{}{}{}{}{Fazer alguma coisa deixar de funcionar; enguiçar.}{en.cren.car}{\verboinum{2}}
\verb{encrenqueiro}{ê}{}{}{}{adj.}{Que arma encrenca, cria confusão. }{en.cren.quei.ro}{0}
\verb{encrespar}{}{}{}{}{v.t.}{Tornar crespo; enrugar, frisar, anelar, franzir.}{en.cres.par}{0}
\verb{encrespar}{}{}{}{}{v.pron.}{Irritar"-se, zangar"-se.}{en.cres.par}{\verboinum{1}}
\verb{encrostar}{}{}{}{}{v.i.}{Criar crosta.}{en.cros.tar}{\verboinum{1}}
\verb{encruamento}{}{}{}{}{s.m.}{Ato ou efeito de encruar.}{en.cru.a.men.to}{0}
\verb{encruamento}{}{}{}{}{}{Má digestão.}{en.cru.a.men.to}{0}
\verb{encruar}{}{}{}{}{v.t.}{Fazer a comida ficar crua, retirando"-a antes do cozimento.}{en.cru.ar}{0}
\verb{encruar}{}{Fig.}{}{}{}{Tornar duro, cruel, empedernido, insensível.}{en.cru.ar}{\verboinum{1}}
\verb{encruzilhada}{}{}{}{}{s.f.}{Lugar onde se cruzam dois ou mais caminhos.}{en.cru.zi.lha.da}{0}
\verb{encubar}{}{}{}{}{v.t.}{Pôr em cuba; envasilhar.}{en.cu.bar}{\verboinum{1}}
\verb{encucar}{}{Bras.}{}{}{v.i.}{Meter ideia ou pensamento na cuca, na cabeça; ter ideia fixa; cismar, encasquetar.}{en.cu.car}{\verboinum{2}}
\verb{encurralar}{}{}{}{}{v.t.}{Colocar pessoa ou animal em lugar sem saída; cercar.}{en.cur.ra.lar}{\verboinum{1}}
\verb{encurtar}{}{}{}{}{v.t.}{Tornar curto ou mais curto; reduzir, diminuir.}{en.cur.tar}{\verboinum{1}}
\verb{encurvar}{}{}{}{}{v.t.}{Tornar curvo; curvar, dobrar, arquear.}{en.cur.var}{0}
\verb{encurvar}{}{Fig.}{}{}{v.pron.}{Humilhar"-se, rebaixar"-se, submeter"-se.}{en.cur.var}{\verboinum{1}}
\verb{endecha}{ê}{}{}{}{s.f.}{Poema ou canção fúnebre, melancólica.}{en.de.cha}{0}
\verb{endefluxar"-se}{cs}{}{}{}{v.pron.}{Apanhar, contrair defluxo; ficar constipado.}{en.de.flu.xar"-se}{\verboinum{1}}
\verb{endemia}{}{Med.}{}{}{s.f.}{Doença que ocorre habitualmente em certas regiões.}{en.de.mi.a}{0}
\verb{endêmico}{}{}{}{}{adj.}{Relativo ou pertencente a endemia.}{en.dê.mi.co}{0}
\verb{endemoniado}{}{}{}{}{adj.}{Que tem ou está com o demônio no corpo; possesso, endemoninhado, endiabrado.}{en.de.mo.ni.a.do}{0}
\verb{endemoniado}{}{Fig.}{}{}{}{Que é travesso, traquinas, inquieto; arteiro, custoso.}{en.de.mo.ni.a.do}{0}
\verb{endemoniar}{}{}{}{}{v.t.}{Pôr o demônio no corpo; endemoninhar, endiabrar.}{en.de.mo.ni.ar}{0}
\verb{endemoniar}{}{}{}{}{}{Enfurecer, enraivecer, encolerizar.}{en.de.mo.ni.ar}{\verboinum{1}}
\verb{endemoninhado}{}{}{}{}{adj.}{Endemoniado.}{en.de.mo.ni.nha.do}{0}
\verb{endemoninhar}{}{}{}{}{v.t.}{Endemoniar.}{en.de.mo.ni.nhar}{\verboinum{1}}
\verb{endereçar}{}{}{}{}{v.t.}{Pôr endereço; sobrescritar.}{en.de.re.çar}{0}
\verb{endereçar}{}{}{}{}{}{Enviar, encaminhar.}{en.de.re.çar}{\verboinum{3}}
\verb{endereço}{ê}{}{}{}{s.m.}{Indicação completa do lugar em que uma pessoa ou um imóvel se encontra.}{en.de.re.ço}{0}
\verb{endeusado}{}{}{}{}{adj.}{Que foi divinizado; deificado.}{en.deu.sa.do}{0}
\verb{endeusado}{}{Fig.}{}{}{}{Que é muito admirado, elogiado; bajulado, incensado, adulado.}{en.deu.sa.do}{0}
\verb{endeusar}{}{}{}{}{v.t.}{Atribuir qualidades divinas; divinizar, deificar.}{en.deu.sar}{0}
\verb{endeusar}{}{Fig.}{}{}{}{Admirar, elogiar muito alguém; incensar, adular, bajular.}{en.deu.sar}{\verboinum{1}}
\verb{endiabrado}{}{}{}{}{adj.}{Endemoniado.}{en.di.a.bra.do}{0}
\verb{endiabrar}{}{}{}{}{v.t.}{Endemoniar.}{en.di.a.brar}{\verboinum{1}}
\verb{endinheirado}{}{}{}{}{adj.}{Que está cheio de dinheiro; rico, abastado.}{en.di.nhei.ra.do}{0}
\verb{endinheirar}{}{}{}{}{v.t.}{Encher de dinheiro; enriquecer.}{en.di.nhei.rar}{\verboinum{1}}
\verb{endireitar}{}{}{}{}{v.t.}{Pôr direito o que estava torto, inclinado, dobrado etc.}{en.di.rei.tar}{0}
\verb{endireitar}{}{}{}{}{}{Corrigir, retificar.}{en.di.rei.tar}{\verboinum{1}}
\verb{endividar}{}{}{}{}{v.t.}{Fazer com que alguém contraia dívida; tornar devedor.}{en.di.vi.dar}{\verboinum{1}}
\verb{endocárdio}{}{Anat.}{}{}{s.m.}{Membrana que reveste o interior do coração.}{en.do.cár.dio}{0}
\verb{endocardite}{}{Med.}{}{}{s.f.}{Inflamação do endocárdio.}{en.do.car.di.te}{0}
\verb{endocarpo}{}{Bot.}{}{}{s.m.}{Camada ou membrana mais interna do pericarpo dos frutos, que se acha agarrada à semente.}{en.do.car.po}{0}
\verb{endócrino}{}{}{}{}{adj.}{Relativo a glândula.}{en.dó.cri.no}{0}
\verb{endocrinologia}{}{Med.}{}{}{s.f.}{Parte da medicina que se dedica ao estudo das glândulas.}{en.do.cri.no.lo.gi.a}{0}
\verb{endocrinologista}{}{Med.}{}{}{s.2g.}{Médico especializado em endocrinologia.}{en.do.cri.no.lo.gis.ta}{0}
\verb{endodontia}{}{Med.}{}{}{s.f.}{Parte da odontologia que trata das doenças da polpa e da raiz dos dentes.}{en.do.don.ti.a}{0}
\verb{endoenças}{}{Relig.}{}{}{s.f.pl.}{Solenidades religiosas da quinta"-feira santa.}{en.do.en.ças}{0}
\verb{endogamia}{}{}{}{}{s.f.}{Costume ou obrigação entre os membros de uma classe ou tribo de se casarem com membros do seu próprio grupo.}{en.do.ga.mi.a}{0}
\verb{endógamo}{}{}{}{}{adj.}{Relativo a endogamia.}{en.dó.ga.mo}{0}
\verb{endógamo}{}{}{}{}{s.m.}{Pessoa que pratica a endogamia.}{en.dó.ga.mo}{0}
\verb{endógeno}{}{Biol.}{}{}{adj.}{Que se forma no interior do organismo.}{en.dó.ge.no}{0}
\verb{endógeno}{}{}{}{}{}{Que tem origem por fatores internos.}{en.dó.ge.no}{0}
\verb{endoidar}{}{}{}{}{v.t.}{Endoidecer.}{en.doi.dar}{\verboinum{1}}
\verb{endoidecer}{ê}{}{}{}{v.t.}{Tornar doido; enlouquecer, endoidar.}{en.doi.de.cer}{\verboinum{15}}
\verb{endolinfa}{}{Bioquím.}{}{}{s.f.}{Líquido aquoso encontrado no interior do labirinto membranoso da orelha interna.}{en.do.lin.fa}{0}
\verb{endomorfina}{}{Bioquím.}{}{}{s.f.}{Endorfina.}{en.do.mor.fi.na}{0}
\verb{endoparasita}{}{Biol.}{}{}{adj.2g.}{Diz"-se do organismo que vive dentro de outro.}{en.do.pa.ra.si.ta}{0}
\verb{endoparasita}{}{}{}{}{s.m.}{Esse organismo.}{en.do.pa.ra.si.ta}{0}
\verb{endoparasito}{}{Biol.}{}{}{s.m.}{Endoparasita.}{en.do.pa.ra.si.to}{0}
\verb{endorfina}{}{Bioquím.}{}{}{s.f.}{Substância capaz de diminuir ou eliminar a dor, que se encontra em estado natural no cérebro; endomorfina.}{en.dor.fi.na}{0}
\verb{endoscopia}{}{Med.}{}{}{s.f.}{Exame visual de uma cavidade do corpo, feito por meio do endoscópio.}{en.dos.co.pi.a}{0}
\verb{endoscópio}{}{Med.}{}{}{s.m.}{Instrumento usado para examinar certas cavidades do corpo.}{en.dos.có.pio}{0}
\verb{endosmose}{ó}{Fís.}{}{}{s.f.}{Corrente, fluxo de fora para dentro, entre duas soluções de densidade diferente, separadas por uma membrana.}{en.dos.mo.se}{0}
\verb{endosperma}{é}{Bot.}{}{}{s.m.}{Parte nutritiva encontrada nas sementes das angiospermas, que fornece alimento ao embrião.}{en.dos.per.ma}{0}
\verb{endossado}{}{}{}{}{adj.}{Que tem endosso.}{en.dos.sa.do}{0}
\verb{endossado}{}{}{}{}{s.m.}{Endossatário.}{en.dos.sa.do}{0}
\verb{endossante}{}{}{}{}{adj.2g.}{Que endossa; endossador.}{en.dos.san.te}{0}
\verb{endossar}{}{}{}{}{v.t.}{Pôr endosso em letra de câmbio, título, ordem etc.}{en.dos.sar}{0}
\verb{endossar}{}{}{}{}{}{Transferir para outro a responsabilidade de alguma coisa.}{en.dos.sar}{\verboinum{1}}
\verb{endossatário}{}{Jur.}{}{}{s.m.}{Pessoa a quem se endossou um título ou uma letra de câmbio; endossado.}{en.dos.sa.tá.rio}{0}
\verb{endosso}{ô}{}{}{}{s.m.}{Ato ou efeito de endossar.}{en.dos.so}{0}
\verb{endovenoso}{ô}{}{"-osos ⟨ó⟩}{"-osa ⟨ó⟩}{adj.}{Que se aplica na veia; intravenoso.}{en.do.ve.no.so}{0}
\verb{endurecer}{ê}{}{}{}{v.t.}{Fazer alguma coisa ficar dura, rija; enrijecer, enrijar.}{en.du.re.cer}{0}
\verb{endurecer}{ê}{Fig.}{}{}{}{Tornar insensível, indiferente; embrutecer.}{en.du.re.cer}{\verboinum{15}}
\verb{endurecimento}{}{}{}{}{s.m.}{Ato ou efeito de endurecer; enrijecimento.}{en.du.re.ci.men.to}{0}
\verb{endurecimento}{}{}{}{}{}{Calo, tumor duro.}{en.du.re.ci.men.to}{0}
\verb{enduro}{}{Esport.}{}{}{s.m.}{Competição de resistência para motociclistas, em terreno acidentado.}{en.du.ro}{0}
\verb{ENE}{}{}{}{}{}{Abrev. de \textit{és"-nordeste}. }{e.n.e.}{0}
\verb{ene}{}{}{}{}{s.m.}{Nome da letra \textit{n}.}{e.ne}{0}
\verb{eneagonal}{}{}{"-ais}{}{adj.2g.}{Relativo a eneágono.}{e.ne.a.go.nal}{0}
\verb{eneagonal}{}{}{"-ais}{}{}{Que tem nove lados.}{e.ne.a.go.nal}{0}
\verb{eneágono}{}{Geom.}{}{}{s.m.}{Polígono de nove lados e nove ângulos.}{e.ne.á.go.no}{0}
\verb{eneassílabo}{}{Gram.}{}{}{adj.}{Que tem nove sílabas.}{e.ne.as.sí.la.bo}{0}
\verb{enegésimo}{}{Mat.}{}{}{num.}{Enésimo.}{e.ne.gé.si.mo}{0}
\verb{enegrecer}{ê}{}{}{}{v.t.}{Tornar negro; escurecer.}{e.ne.gre.cer}{0}
\verb{enegrecer}{ê}{Fig.}{}{}{}{Difamar, caluniar, denegrir.}{e.ne.gre.cer}{\verboinum{15}}
\verb{êneo}{}{}{}{}{adj.}{Relativo a ou feito de bronze; brônzeo.}{ê.neo}{0}
\verb{energética}{}{}{}{}{s.f.}{Ramo da ciência que estuda os assuntos relativos a energia.}{e.ner.gé.ti.ca}{0}
\verb{energético}{}{}{}{}{adj.}{Que se refere a energia.}{e.ner.gé.ti.co}{0}
\verb{energético}{}{Por ext.}{}{}{}{Diz"-se do alimento com muitas calorias.}{e.ner.gé.ti.co}{0}
\verb{energia}{}{}{}{}{s.f.}{Capacidade de realizar trabalho; força, potência. }{e.ner.gi.a}{0}
\verb{energia}{}{}{}{}{}{Rigor, firmeza, determinação.}{e.ner.gi.a}{0}
\verb{energia}{}{}{}{}{}{Força física ou moral; vigor. }{e.ner.gi.a}{0}
\verb{enérgico}{}{}{}{}{adj.}{Que tem ou revela energia física; vigoroso.}{e.nér.gi.co}{0}
\verb{energizar}{}{}{}{}{v.t.}{Dar energia.}{e.ner.gi.zar}{0}
\verb{energizar}{}{}{}{}{}{Fazer corrente elétrica correr num circuito.  }{e.ner.gi.zar}{\verboinum{1}}
\verb{energúmeno}{}{}{}{}{s.m.}{Ignorante, boçal, idiota.}{e.ner.gú.me.no}{0}
\verb{energúmeno}{}{Desus.}{}{}{}{Possuído pelo demônio; possesso.}{e.ner.gú.me.no}{0}
\verb{enervante}{}{}{}{}{adj.2g.}{Que enerva; irritante, exasperante, aborrecedor.}{e.ner.van.te}{0}
\verb{enervar}{}{}{}{}{v.t.}{Deixar alguém nervoso, causar nervosismo, irritação; impacientar, irritar.}{e.ner.var}{\verboinum{1}}
\verb{enésimo}{}{Mat.}{}{}{num.}{Que ocupa a posição do número \textit{n}; enegésimo.}{e.né.si.mo}{0}
\verb{enésimo}{}{Por ext.}{}{}{adj.}{Que corresponde a um número muito grande para ser contado; incontável.}{e.né.si.mo}{0}
\verb{enevoar}{}{}{}{}{v.t.}{Cobrir de neve ou nevoeiro; nublar.}{e.ne.vo.ar}{0}
\verb{enevoar}{}{Por ext.}{}{}{}{Tornar obscuro, sombrio.}{e.ne.vo.ar}{\verboinum{7}}
\verb{enfadar}{}{}{}{}{v.t.}{Causar aborrecimento; entediar, enfarar.}{en.fa.dar}{0}
\verb{enfadar}{}{}{}{}{}{Molestar, incomodar, cansar.}{en.fa.dar}{\verboinum{1}}
\verb{enfado}{}{}{}{}{s.m.}{Sensação de quem está enfadado; tédio, aborrecimento. }{en.fa.do}{0}
\verb{enfadonho}{ô}{}{}{}{adj.}{Que enfada, chateia, aborrece; cansativo, monótono, aborrecido, maçante, fastidioso. }{en.fa.do.nho}{0}
\verb{enfaixar}{ch}{}{}{}{v.t.}{Envolver ou atar com faixas.}{en.fai.xar}{\verboinum{1}}
\verb{enfaramento}{}{}{}{}{s.m.}{Ato ou efeito de enfarar; tédio, fastio, enfado. }{en.fa.ra.men.to}{0}
\verb{enfarar}{}{}{}{}{v.t.}{Causar enfaro; enfastiar, entediar, aborrecer, enfadar.}{en.fa.rar}{0}
\verb{enfarar}{}{}{}{}{v.pron.}{Entediar"-se, aborrecer"-se, enfastiar"-se.}{en.fa.rar}{\verboinum{1}}
\verb{enfardadeira}{ê}{}{}{}{s.f.}{Máquina agrícola usada para juntar palha ou feno em pequenos fardos ou feixes. }{en.far.da.dei.ra}{0}
\verb{enfardar}{}{}{}{}{v.t.}{Fazer ou juntar fardo; empacotar, enfeixar, entrouxar.}{en.far.dar}{\verboinum{1}}
\verb{enfarinhar}{}{}{}{}{v.t.}{Recobrir com farinha ou pó; empoar.}{en.fa.ri.nhar}{\verboinum{1}}
\verb{enfaro}{}{}{}{}{s.m.}{Qualidade do que é tedioso, aborrecido; enfado.}{en.fa.ro}{0}
\verb{enfaro}{}{}{}{}{}{Ato ou efeito de enfarar; asco, enjoo, fastio.}{en.fa.ro}{0}
\verb{enfarpelar}{}{}{}{}{v.t.}{Vestir com roupa nova.}{en.far.pe.lar}{\verboinum{1}}
\verb{enfarruscar}{}{}{}{}{v.t.}{Manchar de carvão ou fuligem; enegrecer, encarvoar.}{en.far.rus.car}{0}
\verb{enfarruscar}{}{Bras.}{}{}{v.i.}{Ficar zangado; amuar"-se.}{en.far.rus.car}{\verboinum{2}}
\verb{enfartar}{}{Med.}{}{}{v.t.}{Causar enfarte a; infartar.}{en.far.tar}{\verboinum{1}}
\verb{enfarte}{}{Med.}{}{}{s.m.}{Necrose de um tecido por obstrução de uma artéria que o irriga; infarto.}{en.far.te}{0}
\verb{enfarto}{}{}{}{}{s.m.}{Enfarte.}{en.far.to}{0}
\verb{ênfase}{}{}{}{}{s.f.}{Entonação afetada ou marcante que serve para ressaltar uma fala, um discurso  etc.}{ên.fa.se}{0}
\verb{ênfase}{}{Por ext.}{}{}{}{Realce, relevo, destaque.}{ên.fa.se}{0}
\verb{enfastiar}{}{}{}{}{v.t.}{Causar fastio, aborrecimento; entediar, enfadar.}{en.fas.ti.ar}{\verboinum{1}}
\verb{enfático}{}{}{}{}{adj.}{Que tem ou revela ênfase.}{en.fá.ti.co}{0}
\verb{enfatiotar}{}{}{}{}{v.t.}{Vestir com apuro, com roupa nova. (\textit{Enfatiotou o filho para a festa.})}{en.fa.ti.o.tar}{\verboinum{1}}
\verb{enfatuar}{}{}{}{}{v.t.}{Tornar fátuo, arrogante, orgulhoso; envaidecer.}{en.fa.tu.ar}{\verboinum{1}}
\verb{enfear}{}{}{}{}{v.t.}{Tornar feio; afear.}{en.fe.ar}{\verboinum{4}}
\verb{enfeitar}{}{}{}{}{v.t.}{Pôr enfeites; adornar, ataviar, ornamentar, embelezar.}{en.fei.tar}{0}
\verb{enfeitar}{}{}{}{}{}{Conferir boa aparência a alguma coisa ou alguém.}{en.fei.tar}{\verboinum{1}}
\verb{enfeite}{}{}{}{}{s.m.}{Ornamento, adorno; algo que enfeita.}{en.fei.te}{0}
\verb{enfeitiçar}{}{}{}{}{v.t.}{Jogar feitiço em alguém; embruxar, encantar.}{en.fei.ti.çar}{0}
\verb{enfeitiçar}{}{Por ext.}{}{}{}{Fascinar, cativar, seduzir.}{en.fei.ti.çar}{\verboinum{3}}
\verb{enfeixar}{ch}{}{}{}{v.t.}{Prender, amarrar em feixe.}{en.fei.xar}{\verboinum{1}}
\verb{enfermagem}{}{}{"-ens}{}{s.f.}{A função de tratar de pessoas enfermas.}{en.fer.ma.gem}{0}
\verb{enfermagem}{}{}{"-ens}{}{}{O conjunto de serviços de enfermaria.}{en.fer.ma.gem}{0}
\verb{enfermar}{}{}{}{}{v.t.}{Fazer ficar doente; tornar doente.}{en.fer.mar}{\verboinum{1}}
\verb{enfermaria}{}{}{}{}{s.f.}{Local destinado ao tratamento de pessoas doentes.}{en.fer.ma.ri.a}{0}
\verb{enfermeiro}{ê}{}{}{}{s.m.}{Profissional que cuida de doentes.}{en.fer.mei.ro}{0}
\verb{enfermiço}{}{}{}{}{adj.}{Que está sempre doente.}{en.fer.mi.ço}{0}
\verb{enfermidade}{}{}{}{}{s.f.}{Doença.}{en.fer.mi.da.de}{0}
\verb{enfermo}{ê}{}{}{}{adj.}{Que sofre de enfermidade; doente.}{en.fer.mo}{0}
\verb{enferrujar}{}{}{}{}{v.t.}{Fazer aparecer em um objeto de ferro uma camada escura que ataca o metal; oxidar.}{en.fer.ru.jar}{0}
\verb{enferrujar}{}{Pop.}{}{}{v.i.}{Perder a mobilidade.}{en.fer.ru.jar}{\verboinum{1}}
\verb{enfestado}{}{}{}{}{adj.}{Diz"-se de pano dobrado ao meio, no sentido de sua largura, e enrolado na peça.}{en.fes.ta.do}{0}
\verb{enfestar}{}{}{}{}{v.t.}{Dobrar pano pelo meio em sua largura.}{en.fes.tar}{\verboinum{1}}
\verb{enfezado}{}{}{}{}{adj.}{Que não se desenvolveu de modo suficiente; raquítico, pequeno.}{en.fe.za.do}{0}
\verb{enfezado}{}{Fig.}{}{}{}{Irritado, aborrecido.}{en.fe.za.do}{0}
\verb{enfezado}{}{}{}{}{}{Rabugento.}{en.fe.za.do}{0}
\verb{enfezamento}{}{}{}{}{s.m.}{Ato ou efeito de enfezar; aborrecimento, irritação.}{en.fe.za.men.to}{0}
\verb{enfezamento}{}{}{}{}{}{Falta de desenvolvimento, raquitismo.}{en.fe.za.men.to}{0}
\verb{enfezar}{}{}{}{}{v.t.}{Prejudicar o crescimento de pessoa, animal ou coisa.}{en.fe.zar}{0}
\verb{enfezar}{}{}{}{}{}{Provocar raiva em alguém; irritar.}{en.fe.zar}{\verboinum{1}}
\verb{enfiada}{}{}{}{}{s.f.}{Fileira, série, porção de coisas que se dispõem em linha.}{en.fi.a.da}{0}
\verb{enfiar}{}{}{}{}{v.t.}{Meter, introduzir um fio num orifício.}{en.fi.ar}{0}
\verb{enfiar}{}{}{}{}{}{Meter em fio ou linha.}{en.fi.ar}{0}
\verb{enfiar}{}{}{}{}{}{Introduzir, meter.}{en.fi.ar}{0}
\verb{enfiar}{}{}{}{}{}{Vestir, calçar.}{en.fi.ar}{\verboinum{1}}
\verb{enfileiramento}{}{}{}{}{s.m.}{Ato ou efeito de enfileirar.}{en.fi.lei.ra.men.to}{0}
\verb{enfileiramento}{}{}{}{}{}{Organização em fila; alinhamento.}{en.fi.lei.ra.men.to}{0}
\verb{enfileirar}{}{}{}{}{v.t.}{Colocar pessoas ou coisas uma atrás da outra; alinhar.}{en.fi.lei.rar}{\verboinum{1}}
\verb{enfim}{}{}{}{}{adv.}{Por fim; finalmente.}{en.fim}{0}
\verb{enfisema}{}{Med.}{}{}{s.m.}{Presença de ar nas fendas do tecido de um órgão.}{en.fi.se.ma}{0}
\verb{enfiteuse}{}{Jur.}{}{}{s.m.}{Direito real em contrato perpétuo, pelo qual o proprietário de um imóvel cede a outrem o domínio útil da propriedade, mediante o pagamento de uma pensão anual chamada foro.}{en.fi.teu.se}{0}
\verb{enfivelar}{}{}{}{}{v.t.}{Pôr fivelas; guarnecer com fivelas.}{en.fi.ve.lar}{\verboinum{1}}
\verb{enflorar}{}{}{}{}{v.t.}{Encher de flores.}{en.flo.rar}{0}
\verb{enflorar}{}{Por ext.}{}{}{}{Tornar alegre; encher de alegria.}{en.flo.rar}{\verboinum{1}}
\verb{enfocar}{}{}{}{}{v.t.}{Fazer com que a imagem de pessoa ou coisa apareça com nitidez em filme ou tela; focalizar.}{en.fo.car}{0}
\verb{enfocar}{}{}{}{}{}{Dar certo enfoque em um tema.}{en.fo.car}{\verboinum{2}}
\verb{enfolhar}{}{}{}{}{v.i.}{Revestir de folha.}{en.fo.lhar}{\verboinum{1}}
\verb{enfoque}{ó}{}{}{}{s.m.}{Maneira de enfocar ou focalizar um assunto, uma questão.}{en.fo.que}{0}
\verb{enforcado}{}{}{}{}{adj.}{Diz"-se de indivíduo que foi morto por enforcamento.}{en.for.ca.do}{0}
\verb{enforcado}{}{}{}{}{}{Diz"-se de parreira enleada a árvores.}{en.for.ca.do}{0}
\verb{enforcado}{}{}{}{}{}{Que se acha em apuros financeiros; apertado.}{en.for.ca.do}{0}
\verb{enforcamento}{}{}{}{}{s.m.}{Ato ou efeito de enforcar; morte na forca.}{en.for.ca.men.to}{0}
\verb{enforcar}{}{}{}{}{v.t.}{Suspender alguém pelo pescoço até que morra.}{en.for.car}{0}
\verb{enforcar}{}{Pop.}{}{}{}{Deixar de trabalhar ou de ir à aula em dia útil que fica entre dois feriados.}{en.for.car}{0}
\verb{enforcar}{}{Pop.}{}{}{v.pron.}{Casar"-se.}{en.for.car}{\verboinum{2}}
\verb{enformar}{}{}{}{}{v.t.}{Colocar alguma coisa dentro de uma forma.}{en.for.mar}{\verboinum{1}}
\verb{enfornar}{}{}{}{}{v.t.}{Pôr no forno.}{en.for.nar}{\verboinum{1}}
\verb{enfraquecer}{ê}{}{}{}{v.t.}{Tornar fraco; debilitar.}{en.fra.que.cer}{\verboinum{15}}
\verb{enfraquecimento}{}{}{}{}{s.m.}{Ato ou efeito de enfraquecer; perda de intensidade ou força.}{en.fra.que.ci.men.to}{0}
\verb{enfrascar}{}{}{}{}{v.t.}{Pôr em frasco; engarrafar.}{en.fras.car}{\verboinum{2}}
\verb{enfrear}{}{}{}{}{v.t.}{Colocar freio em animal.}{en.fre.ar}{0}
\verb{enfrear}{}{}{}{}{}{Brecar, travar.}{en.fre.ar}{0}
\verb{enfrear}{}{Fig.}{}{}{}{Deter, reprimir sentimentos.}{en.fre.ar}{\verboinum{4}}
\verb{enfrentamento}{}{}{}{}{s.m.}{Ato ou efeito de enfrentar; oposição, polêmica.}{en.fren.ta.men.to}{0}
\verb{enfrentamento}{}{}{}{}{}{Choque entre dois grupos adversários; briga, luta.}{en.fren.ta.men.to}{0}
\verb{enfrentar}{}{}{}{}{v.t.}{Colocar"-se frente a frente com pessoa ou coisa, pronto para tomar a atitude exigida pela situação; encarar, afrontar, combater.}{en.fren.tar}{\verboinum{1}}
\verb{enfronhar}{}{}{}{}{v.t.}{Pôr peça de roupa dentro da fronha para conservá"-la limpa.}{en.fro.nhar}{0}
\verb{enfronhar}{}{}{}{}{v.pron.}{Tomar conhecimento de algum assunto; instruir"-se.}{en.fro.nhar}{\verboinum{1}}
\verb{enfumaçar}{}{}{}{}{v.t.}{Encher, recobrir de fumaça.}{en.fu.ma.çar}{\verboinum{3}}
\verb{enfunar}{}{}{}{}{v.t.}{Encher vela ou pano, tornando abaulado.}{en.fu.nar}{0}
\verb{enfunar}{}{Fig.}{}{}{}{Tornar orgulhoso; envaidecer.}{en.fu.nar}{\verboinum{1}}
\verb{enfunilar}{}{}{}{}{v.t.}{Dar a forma de funil; afunilar.}{en.fu.ni.lar}{0}
\verb{enfunilar}{}{}{}{}{}{Encher ou vazar por funil.}{en.fu.ni.lar}{\verboinum{1}}
\verb{enfurecer}{ê}{}{}{}{v.t.}{Encher de fúria; irar, zangar.}{en.fu.re.cer}{\verboinum{15}}
\verb{enfurecido}{}{}{}{}{adj.}{Que se enfureceu; raivoso, furioso, irado.}{en.fu.re.ci.do}{0}
\verb{enfurnar}{}{}{}{}{v.t.}{Colocar alguma coisa em lugar escondido.}{en.fur.nar}{0}
\verb{enfurnar}{}{}{}{}{v.pron.}{Ficar isolado das pessoas.}{en.fur.nar}{\verboinum{1}}
\verb{engabelar}{}{}{}{}{}{Var. de \textit{engambelar}.}{en.ga.be.lar}{0}
%\verb{}{}{}{}{}{}{}{}{0}
\verb{engaiolar}{}{}{}{}{v.t.}{Prender numa gaiola.}{en.gai.o.lar}{0}
\verb{engaiolar}{}{Pop.}{}{}{}{Pôr na cadeia; prender.}{en.gai.o.lar}{\verboinum{1}}
\verb{engajado}{}{}{}{}{adj.}{Diz"-se de indivíduo que é contratado para certos serviços.}{en.ga.ja.do}{0}
\verb{engajado}{}{}{}{}{}{Diz"-se de indivíduo que se engajou no serviço militar.}{en.ga.ja.do}{0}
\verb{engajado}{}{}{}{}{}{Diz"-se de indivíduo que se engajou, se filiou a uma linha política, filosófica etc.}{en.ga.ja.do}{0}
\verb{engajamento}{}{}{}{}{s.m.}{Ato ou efeito de engajar; alistamento.}{en.ga.ja.men.to}{0}
\verb{engajamento}{}{}{}{}{}{Contrato para certos serviços.}{en.ga.ja.men.to}{0}
\verb{engajar}{}{}{}{}{v.t.}{Comprometer por contrato.}{en.ga.jar}{0}
\verb{engajar}{}{}{}{}{}{Alistar no serviço militar.}{en.ga.jar}{0}
\verb{engajar}{}{}{}{}{v.pron.}{Empenhar"-se por causa, ideal etc.}{en.ga.jar}{\verboinum{1}}
\verb{engalanar}{}{}{}{}{v.t.}{Ornar com gala; adornar, enfeitar.}{en.ga.la.nar}{\verboinum{1}}
\verb{engalfinhar"-se}{}{}{}{}{v.pron.}{Agarrar"-se a outra pessoa numa briga. (\textit{Os meninos engalfinharam"-se por causa das figurinhas do álbum.})}{en.gal.fi.nhar"-se}{\verboinum{1}}
\verb{engambelar}{}{}{}{}{v.t.}{Enganar alguém com uma conversa mentirosa; embrulhar, enrolar.}{en.gam.be.lar}{\verboinum{1}}
\verb{enganado}{}{}{}{}{adj.}{Que incorreu em engano, erro.}{en.ga.na.do}{0}
\verb{enganador}{ô}{}{}{}{adj.}{Que engana.}{en.ga.na.dor}{0}
\verb{enganar}{}{}{}{}{v.t.}{Induzir em erro.}{en.ga.nar}{0}
\verb{enganar}{}{}{}{}{}{Burlar, mentir, lograr, iludir.}{en.ga.nar}{0}
\verb{enganar}{}{}{}{}{}{Esconder, dissimular, disfarçar.}{en.ga.nar}{0}
\verb{enganar}{}{}{}{}{}{Trair.}{en.ga.nar}{\verboinum{1}}
\verb{enganchar}{}{}{}{}{v.t.}{Prender em gancho.}{en.gan.char}{\verboinum{1}}
\verb{engano}{}{}{}{}{s.m.}{Ato ou efeito de enganar; ilusão, erro, equívoco, falha.}{en.ga.no}{0}
\verb{enganoso}{ô}{}{"-osos ⟨ó⟩}{"-osa ⟨ó⟩}{adj.}{Que induz ao erro, que engana; ilusório, artificioso, enganador.}{en.ga.no.so}{0}
\verb{engarrafamento}{}{}{}{}{s.m.}{Ato ou efeito de engarrafar.}{en.gar.ra.fa.men.to}{0}
\verb{engarrafamento}{}{}{}{}{}{Acúmulo de veículos em determinado ponto da via pública; congestionamento.}{en.gar.ra.fa.men.to}{0}
\verb{engarrafar}{}{}{}{}{v.t.}{Colocar em garrafa.}{en.gar.ra.far}{0}
\verb{engarrafar}{}{}{}{}{}{Formar engarrafamento de trânsito.}{en.gar.ra.far}{\verboinum{1}}
\verb{engasgar}{}{}{}{}{v.t.}{Produzir engasgo, obstrução na garganta.}{en.gas.gar}{0}
\verb{engasgar}{}{}{}{}{}{Impedir ou perder temporariamente a fala.}{en.gas.gar}{\verboinum{5}}
\verb{engasgo}{}{}{}{}{s.m.}{Ato de engasgar.}{en.gas.go}{0}
\verb{engastar}{}{}{}{}{v.t.}{Embutir pedra preciosa em metal.}{en.gas.tar}{\verboinum{1}}
\verb{engaste}{}{}{}{}{s.m.}{Ato ou efeito de engastar.}{en.gas.te}{0}
\verb{engatar}{}{}{}{}{v.t.}{Prender com engate.}{en.ga.tar}{0}
\verb{engatar}{}{}{}{}{}{Conectar o sistema de transmissão de um veículo automotor; engrenar marcha.}{en.ga.tar}{\verboinum{1}}
\verb{engate}{}{}{}{}{s.m.}{Ato ou efeito de engatar.}{en.ga.te}{0}
\verb{engate}{}{}{}{}{}{Peça ou conjunto de peças para ligar vagões, carros, reboques entre si.}{en.ga.te}{0}
\verb{engatilhar}{}{}{}{}{v.t.}{Armar o gatilho, deixando a arma de fogo pronta para atirar.}{en.ga.ti.lhar}{0}
\verb{engatilhar}{}{Fig.}{}{}{}{Deixar pronto; preparar.}{en.ga.ti.lhar}{\verboinum{1}}
\verb{engatinhar}{}{}{}{}{v.i.}{Andar apoiado sobre as mãos e os joelhos.}{en.ga.ti.nhar}{\verboinum{1}}
\verb{engavetamento}{}{}{}{}{s.m.}{Ato ou efeito de engavetar.}{en.ga.ve.ta.men.to}{0}
\verb{engavetamento}{}{}{}{}{}{Acidente em que ocorre colisão em série entre vários automóveis ou vagões.}{en.ga.ve.ta.men.to}{0}
\verb{engavetar}{}{}{}{}{v.t.}{Pôr em gaveta.}{en.ga.ve.tar}{0}
\verb{engavetar}{}{Bras.}{}{}{}{Impedir ou retardar o andamento de um processo.}{en.ga.ve.tar}{0}
\verb{engavetar}{}{Bras.}{}{}{v.i.}{Bater veículos ou vagões um atrás do outro.}{en.ga.ve.tar}{\verboinum{1}}
\verb{engazopar}{}{}{}{}{v.t.}{Enganar, ludibriar, mentir.}{en.ga.zo.par}{\verboinum{1}}
\verb{engelhar}{}{}{}{}{v.t.}{Fazer gelhas, pregas, rugas; enrugar.}{en.ge.lhar}{\verboinum{1}}
\verb{engendrar}{}{}{}{}{v.t.}{Dar origem; gerar, formar, produzir.}{en.gen.drar}{0}
\verb{engendrar}{}{}{}{}{}{Imaginar, inventar.}{en.gen.drar}{\verboinum{1}}
\verb{engenhar}{}{}{}{}{v.t.}{Criar na imaginação; inventar, engendrar.}{en.ge.nhar}{0}
\verb{engenhar}{}{}{}{}{}{Arquitetar, conspirar, tramar.}{en.ge.nhar}{\verboinum{1}}
\verb{engenharia}{}{}{}{}{s.f.}{Conjunto de técnicas de aplicação de conhecimentos científicos à criação de dispositivos, aparelhos, estruturas e processos.}{en.ge.nha.ri.a}{0}
\verb{engenheiro}{ê}{}{}{}{s.m.}{Indivíduo diplomado em engenharia ou que trabalha profissionalmente com ela.}{en.ge.nhei.ro}{0}
\verb{engenheiro}{ê}{Bras.}{}{}{}{Proprietário de engenho.}{en.ge.nhei.ro}{0}
\verb{engenho}{ê}{Bras.}{}{}{s.m.}{Moenda de cana"-de"-açúcar.}{en.ge.nho}{0}
\verb{engenho}{ê}{Bras.}{}{}{}{Propriedade agrícola onde se produz açúcar.}{en.ge.nho}{0}
\verb{engenho}{ê}{}{}{}{}{Capacidade de criar; talento, habilidade, criatividade.}{en.ge.nho}{0}
\verb{engenho}{ê}{}{}{}{}{Qualquer máquina.}{en.ge.nho}{0}
\verb{engenhoca}{ó}{}{}{}{s.f.}{Máquina ou dispositivo improvisado ou em condições precárias.}{en.ge.nho.ca}{0}
\verb{engenhoca}{ó}{}{}{}{}{Artimanha.}{en.ge.nho.ca}{0}
\verb{engenhoso}{ô}{}{"-osos ⟨ó⟩}{"-osa}{adj.}{Que tem ou demonstra engenho, criatividade, funcionalidade.}{en.ge.nho.so}{0}
\verb{engessar}{}{}{}{}{v.t.}{Cobrir, revestir ou proteger com gesso.}{en.ges.sar}{0}
\verb{engessar}{}{}{}{}{}{Branquear com gesso.}{en.ges.sar}{\verboinum{1}}
\verb{englobar}{}{}{}{}{v.t.}{Reunir ou incluir em um todo; juntar.}{en.glo.bar}{\verboinum{1}}
\verb{engodar}{}{}{}{}{v.t.}{Atrair com engodo; enganar.}{en.go.dar}{\verboinum{1}}
\verb{engodo}{ô}{}{}{}{s.m.}{Isca, chamariz, artifício para atrair.}{en.go.do}{0}
\verb{engolfar}{}{}{}{}{v.t.}{Dirigir para um golfo.}{en.gol.far}{0}
\verb{engolfar}{}{}{}{}{}{Encaminhar para alto"-mar.}{en.gol.far}{\verboinum{1}}
\verb{engolir}{}{}{}{}{v.t.}{Fazer passar alimento ou líquido da boca para o estômago; deglutir.}{en.go.lir}{0}
\verb{engolir}{}{}{}{}{}{Tolerar contra a vontade.}{en.go.lir}{0}
\verb{engolir}{}{}{}{}{}{Deixar de dizer; reprimir.}{en.go.lir}{\verboinum{31}}
\verb{engomadeira}{ê}{}{}{}{s.f.}{Mulher que engoma ou passa roupas.}{en.go.ma.dei.ra}{0}
\verb{engomar}{}{}{}{}{v.t.}{Pôr goma e passar roupa a ferro.}{en.go.mar}{0}
\verb{engomar}{}{}{}{}{}{Avolumar, engrossar.}{en.go.mar}{\verboinum{1}}
\verb{engonçar}{}{}{}{}{v.t.}{Colocar engonços.}{en.gon.çar}{\verboinum{3}}
\verb{engonço}{}{}{}{}{s.m.}{Tipo de dobradiça.}{en.gon.ço}{0}
\verb{engorda}{ó}{}{}{}{s.f.}{Ato ou efeito de engordar animais; ceva.}{en.gor.da}{0}
\verb{engordar}{}{}{}{}{v.t.}{Tornar gordo.}{en.gor.dar}{\verboinum{1}}
\verb{engordurar}{}{}{}{}{v.t.}{Untar ou sujar com gordura; besuntar.}{en.gor.du.rar}{\verboinum{1}}
\verb{engraçado}{}{}{}{}{adj.}{Que tem graça; divertido, espirituoso, jocoso.}{en.gra.ça.do}{0}
\verb{engraçamento}{}{}{}{}{s.m.}{Ato de engraçar"-se; simpatia.}{en.gra.ça.men.to}{0}
\verb{engraçar}{}{}{}{}{v.t.}{Dar graça, esplendor; realçar, enfeitar.}{en.gra.çar}{0}
\verb{engraçar}{}{}{}{}{v.pron.}{Simpatizar, agradar"-se.}{en.gra.çar}{\verboinum{3}}
\verb{engradado}{}{}{}{}{adj.}{Cercado por grade.}{en.gra.da.do}{0}
\verb{engradado}{}{}{}{}{}{Que tem forma de grade.}{en.gra.da.do}{0}
\verb{engradado}{}{}{}{}{s.m.}{Caixa ou armação de sarrafo para facilitar o transporte e proteger os objetos.}{en.gra.da.do}{0}
\verb{engradar}{}{}{}{}{v.t.}{Cercar com grades.}{en.gra.dar}{\verboinum{1}}
\verb{engrandecer}{ê}{}{}{}{v.t.}{Tornar grande; aumentar, elevar, crescer.}{en.gran.de.cer}{\verboinum{15}}
\verb{engrandecimento}{}{}{}{}{s.m.}{Ato ou efeito de engrandecer.}{en.gran.de.ci.men.to}{0}
\verb{engranzar}{}{}{}{}{v.t.}{Enfiar contas em fio.}{en.gran.zar}{\verboinum{1}}
\verb{engravatar"-se}{}{}{}{}{v.pron.}{Pôr gravata.}{en.gra.va.tar"-se}{0}
\verb{engravatar"-se}{}{}{}{}{}{Apresentar"-se bem vestido. (\textit{Engravatou"-se para a entrevista do primeiro emprego.})}{en.gra.va.tar"-se}{\verboinum{1}}
\verb{engravidar}{}{}{}{}{v.t.}{Tornar grávida; emprenhar.}{en.gra.vi.dar}{0}
\verb{engravidar}{}{}{}{}{v.i.}{Ficar grávida.}{en.gra.vi.dar}{\verboinum{1}}
\verb{engraxar}{ch}{}{}{}{v.t.}{Passar graxa e em seguida esfregar, para dar brilho e proteger o material.}{en.gra.xar}{\verboinum{1}}
\verb{engraxate}{ch}{}{}{}{s.2g.}{Indivíduo que engraxa sapatos e peças de couro.}{en.gra.xa.te}{0}
\verb{engrenagem}{}{}{"-ens}{}{s.f.}{Peça ou conjunto de peças circulares dentadas que, ligadas umas às outras, transmitem movimento de rotação em máquinas.}{en.gre.na.gem}{0}
\verb{engrenagem}{}{}{"-ens}{}{}{Ato ou efeito de engrenar.}{en.gre.na.gem}{0}
\verb{engrenar}{}{}{}{}{v.t.}{Conectar as engrenagens de um sistema de transmissão; engatar.}{en.gre.nar}{0}
\verb{engrenar}{}{}{}{}{}{Iniciar, instaurar.}{en.gre.nar}{\verboinum{1}}
\verb{engrinaldar}{}{}{}{}{v.t.}{Pôr grinalda; coroar, adornar, enfeitar.}{en.gri.nal.dar}{\verboinum{1}}
\verb{engripar}{}{}{}{}{v.i.}{Travar ou deixar de funcionar por falta de manutenção ou lubrificação.}{en.gri.par}{\verboinum{1}}
\verb{engrolar}{}{}{}{}{v.t.}{Cozinhar ou assar mal.}{en.gro.lar}{0}
\verb{engrolar}{}{}{}{}{}{Pronunciar mal.}{en.gro.lar}{0}
\verb{engrolar}{}{}{}{}{}{Fazer qualquer coisa mal.}{en.gro.lar}{\verboinum{1}}
\verb{engrossar}{}{}{}{}{v.t.}{Tornar grosso, espesso, volumoso, numeroso.}{en.gros.sar}{0}
\verb{engrossar}{}{}{}{}{}{Irritar"-se adotando modos violentos ou severos.}{en.gros.sar}{\verboinum{1}}
\verb{engrupir}{}{}{}{}{v.t.}{Passar a lábia; enganar, iludir, ludibriar.}{en.gru.pir}{\verboinum{18}}
\verb{engruvinhado}{}{}{}{}{adj.}{Que se amassou; enrugado, encarquilhado.}{en.gru.vi.nha.do}{0}
\verb{enguia}{}{Zool.}{}{}{s.f.}{Peixe de corpo alongado e pele escorregadia, que vive tanto em água doce como em água salgada.}{en.gui.a}{0}
\verb{enguiçar}{}{}{}{}{v.i.}{Sofrer desarranjo; parar por mau funcionamento; quebrar. (\textit{O carro enguiçou em cima da ponte em pleno horário de pico.})}{en.gui.çar}{\verboinum{3}}
\verb{enguiço}{}{}{}{}{s.m.}{Defeito em máquina; desarranjo.}{en.gui.ço}{0}
\verb{engulhar}{}{}{}{}{v.t.}{Causar engulho; provocar ânsia de vômito.}{en.gu.lhar}{\verboinum{1}}
\verb{engulho}{}{}{}{}{s.m.}{Ânsia de vômito; náusea, repugnância.}{en.gu.lho}{0}
\verb{enigma}{}{}{}{}{s.m.}{Coisa obscura, de difícil compreensão; mistério.}{e.nig.ma}{0}
\verb{enigma}{}{}{}{}{}{Adivinha, charada.}{e.nig.ma}{0}
\verb{enigmático}{}{}{}{}{adj.}{Relativo a enigma.}{e.nig.má.ti.co}{0}
\verb{enigmático}{}{}{}{}{}{Obscuro, indecifrável, misterioso.}{e.nig.má.ti.co}{0}
\verb{enjambrar}{}{}{}{}{v.i.}{Sair da linha do prumo; entortar, deformar.}{en.jam.brar}{\verboinum{1}}
\verb{enjaular}{}{}{}{}{v.t.}{Meter em jaula; prender.}{en.jau.lar}{\verboinum{1}}
\verb{enjeitado}{}{}{}{}{adj.}{Que não foi aceito; recusado.}{en.jei.ta.do}{0}
\verb{enjeitado}{}{}{}{}{}{Diz"-se da criança abandonada pelos pais ao nascer.}{en.jei.ta.do}{0}
\verb{enjeitamento}{}{}{}{}{s.m.}{Ato ou efeito de enjeitar; rejeição, abandono.}{en.jei.ta.men.to}{0}
\verb{enjeitar}{}{}{}{}{v.t.}{Não aceitar; rejeitar, repudiar.}{en.jei.tar}{0}
\verb{enjeitar}{}{}{}{}{}{Abandonar uma criança recém"-nascida.}{en.jei.tar}{\verboinum{1}}
\verb{enjerir}{}{}{}{}{v.pron.}{Encolher"-se por causa do frio ou de doença; entanguir"-se.}{en.je.rir"-se}{\verboinum{29}}
\verb{enjoado}{}{}{}{}{adj.}{Que sente enjoo; nauseado.}{en.jo.a.do}{0}
\verb{enjoado}{}{}{}{}{}{Que está sempre de mau humor; antipático, entojado.}{en.jo.a.do}{0}
\verb{enjoar}{}{}{}{}{v.t.}{Provocar enjoo, náusea; repugnar.}{en.jo.ar}{0}
\verb{enjoar}{}{}{}{}{}{Causar tédio; aborrecer, enfastiar.}{en.jo.ar}{\verboinum{7}}
\verb{enjoativo}{}{}{}{}{adj.}{Que causa enjoo; repugnante, nauseabundo.}{en.jo.a.ti.vo}{0}
\verb{enjoativo}{}{}{}{}{}{Entediante, cansativo, maçante.}{en.jo.a.ti.vo}{0}
\verb{enjoo}{}{}{}{}{s.m.}{Mal"-estar que precede a vontade de vomitar; náusea, engulho.}{en.jo.o}{0}
\verb{enjoo}{}{}{}{}{}{Tédio, aborrecimento, enfado.}{en.jo.o}{0}
\verb{enlaçar}{}{}{}{}{v.t.}{Prender com laço; atar, unir.}{en.la.çar}{0}
\verb{enlaçar}{}{}{}{}{}{Envolver com os braços; abraçar.}{en.la.çar}{\verboinum{3}}
\verb{enlace}{}{}{}{}{s.m.}{Ato ou efeito de enlaçar.}{en.la.ce}{0}
\verb{enlace}{}{}{}{}{}{União matrimonial; casamento.}{en.la.ce}{0}
\verb{enlambuzar}{}{}{}{}{v.t.}{Sujar de comida, graxa, cola etc.; lambuzar.}{en.lam.bu.zar}{\verboinum{1}}
\verb{enlameado}{}{}{}{}{adj.}{Coberto de lama; sujo.}{en.la.me.a.do}{0}
\verb{enlamear}{}{}{}{}{v.t.}{Cobrir de lama; sujar, manchar.}{en.la.me.ar}{\verboinum{4}}
\verb{enlanguescer}{ê}{}{}{}{v.i.}{Ficar lânguido, enfraquecido; definhar.}{en.lan.gues.cer}{\verboinum{15}}
\verb{enlatado}{}{}{}{}{adj.}{Que se enlatou.}{en.la.ta.do}{0}
\verb{enlatado}{}{}{}{}{}{Diz"-se de produto alimentício acondicionado em lata.}{en.la.ta.do}{0}
\verb{enlatamento}{}{}{}{}{s.m.}{Ato ou efeito de enlatar.}{en.la.ta.men.to}{0}
\verb{enlatar}{}{}{}{}{v.t.}{Acondicionar em latas.}{en.la.tar}{\verboinum{1}}
\verb{enlear}{}{}{}{}{v.t.}{Prender com liame; atar, ligar.}{en.le.ar}{0}
\verb{enlear}{}{}{}{}{}{Envolver, implicar.}{en.le.ar}{\verboinum{4}}
\verb{enleio}{ê}{}{}{}{s.m.}{Ato ou efeito de enlear; envolvimento.}{en.lei.o}{0}
\verb{enlevação}{}{}{"-ões}{}{s.f.}{Ato ou efeito de enlevar; enlevo, enlevamento, êxtase.}{en.le.va.ção}{0}
\verb{enlevamento}{}{}{}{}{s.m.}{Enlevação.}{en.le.va.men.to}{0}
\verb{enlevar}{}{}{}{}{v.t.}{Prender a atenção; cativar, extasiar, maravilhar.}{en.le.var}{\verboinum{1}}
\verb{enlevo}{ê}{}{}{}{s.m.}{Sensação de êxtase; arroubo, encanto.}{en.le.vo}{0}
\verb{enliçar}{}{}{}{}{v.t.}{Pôr fios de arame em tear; tramar, urdir.}{en.li.çar}{\verboinum{3}}
\verb{enlodar}{}{}{}{}{v.t.}{Sujar, cobrir de lodo, de lama; enlamear.  }{en.lo.dar}{0}
\verb{enlodar}{}{Fig.}{}{}{}{Manchar a honra de alguém; conspurcar, enlamear.  }{en.lo.dar}{\verboinum{1}}
\verb{enlouquecer}{ê}{}{}{}{v.t.}{Fazer perder a razão; endoidecer.}{en.lou.que.cer}{\verboinum{15}}
\verb{enlouquecimento}{}{}{}{}{s.m.}{Ato ou efeito de enlouquecer; insanidade mental, loucura.}{en.lou.que.ci.men.to}{0}
\verb{enluarado}{}{}{}{}{adj.}{Iluminado pelo luar.}{en.lu.a.ra.do}{0}
\verb{enlutar}{}{}{}{}{v.t.}{Cobrir de luto; consternar.}{en.lu.tar}{\verboinum{1}}
\verb{enluvar}{}{}{}{}{v.t.}{Colocar luvas em.}{en.lu.var}{\verboinum{1}}
\verb{enobrecer}{ê}{}{}{}{v.t.}{Tornar nobre, ilustre.}{e.no.bre.cer}{\verboinum{15}}
\verb{enobrecimento}{}{}{}{}{s.m.}{Ato ou efeito de enobrecer; engrandecimento.}{e.no.bre.ci.men.to}{0}
\verb{enodoar}{}{}{}{}{v.t.}{Cobrir de nódoas; manchar.}{e.no.do.ar}{\verboinum{7}}
\verb{enojado}{}{}{}{}{adj.}{Que se enojou; entediado, aborrecido.}{e.no.ja.do}{0}
%\verb{}{}{}{}{}{}{}{}{0}
\verb{enojar}{}{}{}{}{v.t.}{Causar nojo, náusea.}{e.no.jar}{0}
\verb{enojar}{}{}{}{}{}{Entendiar, aborrecer, enfadar.}{e.no.jar}{\verboinum{1}}
\verb{enojo}{ô}{}{}{}{s.m.}{Sensação de nojo; náusea.}{e.no.jo}{0}
\verb{enojo}{ô}{}{}{}{}{Aborrecimento, enfado.}{e.no.jo}{0}
\verb{enomel}{é}{}{"-éis}{}{s.m.}{Xarope de vinho com mel.}{e.no.mel}{0}
\verb{enorme}{ó}{}{}{}{adj.}{Que excede os padrões normais; fora do comum; muito grande.}{e.nor.me}{0}
\verb{enormidade}{}{}{}{}{s.f.}{Qualidade do que é enorme; grandeza, desproporção.}{e.nor.mi.da.de}{0}
\verb{enoveladeira}{ê}{}{}{}{s.f.}{Máquina de enovelar as fiações.}{e.no.ve.la.dei.ra}{0}
\verb{enovelar}{}{}{}{}{v.t.}{Dar aspecto ou forma de novelo; enrolar.}{e.no.ve.lar}{\verboinum{1}}
\verb{enquadrar}{}{}{}{}{v.t.}{Pôr em quadro; emoldurar.}{en.qua.drar}{0}
\verb{enquadrar}{}{}{}{}{}{Dar punição; castigar.}{en.qua.drar}{\verboinum{1}}
\verb{enquanto}{}{}{}{}{conj.}{No tempo que; ao passo que. (\textit{Enquanto trabalhou naquela empresa, ele nunca faltou.})}{en.quan.to}{0}
\verb{enquanto}{}{}{}{}{}{Ao mesmo tempo; durante. (\textit{Minha mãe estava preparando o jantar enquanto o bolo estava assando.})}{en.quan.to}{0}
\verb{enquete}{é}{}{}{}{s.f.}{Pesquisa de opinião pública.}{en.que.te}{0}
\verb{enquistar}{}{}{}{}{v.t.}{Formar quisto.}{en.quis.tar}{0}
\verb{enquistar}{}{}{}{}{}{Encaixar, entalhar.}{en.quis.tar}{\verboinum{1}}
\verb{enquizilar}{}{}{}{}{v.t.}{Causar incômodo; importunar, aborrecer; quizilar.}{en.qui.zi.lar}{\verboinum{1}}
\verb{enrabichado}{}{}{}{}{adj.}{Diz"-se do cabelo atado em rabicho.}{en.ra.bi.cha.do}{0}
\verb{enrabichado}{}{Fig.}{}{}{}{Apaixonado, enamorado.}{en.ra.bi.cha.do}{0}
\verb{enrabichar}{}{}{}{}{v.t.}{Atar em forma de rabicho.}{en.ra.bi.char}{0}
\verb{enrabichar}{}{Fig.}{}{}{}{Enamorar, apaixonar.}{en.ra.bi.char}{\verboinum{1}}
\verb{enraivecer}{ê}{}{}{}{v.t.}{Causar raiva; encolerizar.}{en.rai.ve.cer}{\verboinum{15}}
\verb{enraizado}{}{}{}{}{adj.}{Que se enraizou; arraigado.}{en.ra.i.za.do}{0}
\verb{enraizar}{}{}{}{}{v.t.}{Fixar raízes; arraigar.}{en.ra.i.zar}{\verboinum{8}}
\verb{enramar}{}{}{}{}{v.t.}{Cobrir, enfeitar de ramos.}{en.ra.mar}{\verboinum{1}}
\verb{enrançar}{}{}{}{}{v.t.}{Tornar rançoso; mofar.}{en.ran.çar}{\verboinum{3}}
\verb{enrascada}{}{}{}{}{s.f.}{Ato ou efeito de enrascar; apuro, dificuldade, aperto.}{en.ras.ca.da}{0}
\verb{enrascar}{}{}{}{}{v.t.}{Criar dificuldades; complicar, embaraçar.}{en.ras.car}{\verboinum{2}}
\verb{enredado}{}{}{}{}{adj.}{Preso na rede; emaranhado, embaraçado.}{en.re.da.do}{0}
\verb{enredado}{}{}{}{}{}{Complicado, envolvido, enrascado.}{en.re.da.do}{0}
\verb{enredar}{}{}{}{}{v.t.}{Prender em rede; emaranhar.}{en.re.dar}{0}
\verb{enredar}{}{}{}{}{}{Envolver, complicar, enrascar.}{en.re.dar}{0}
\verb{enredar}{}{}{}{}{}{Fazer enredo, intriga.}{en.re.dar}{\verboinum{1}}
\verb{enredeiro}{ê}{}{}{}{adj.}{Que faz intriga; mexeriqueiro.}{en.re.dei.ro}{0}
\verb{enredo}{ê}{}{}{}{s.m.}{Ato ou efeito de enredar, emaranhar.}{en.re.do}{0}
\verb{enredo}{ê}{}{}{}{}{Conjunto de incidentes e ações de uma obra de ficção; trama.}{en.re.do}{0}
\verb{enredo}{ê}{}{}{}{}{Intriga, mexerico.}{en.re.do}{0}
\verb{enregelar}{}{}{}{}{v.t.}{Tornar muito gelado; congelar, resfriar.}{en.re.ge.lar}{\verboinum{1}}
\verb{enricar}{}{}{}{}{v.t.}{Enriquecer.}{en.ri.car}{\verboinum{2}}
\verb{enrijar}{}{}{}{}{v.t.}{Enrijecer.}{en.ri.jar}{\verboinum{1}}
\verb{enrijecer}{ê}{}{}{}{v.t.}{Tornar rijo, rígido; robustecer.}{en.ri.je.cer}{\verboinum{15}}
\verb{enripar}{}{}{}{}{v.t.}{Cobrir com ripas.}{en.ri.par}{\verboinum{1}}
\verb{enriquecer}{ê}{}{}{}{v.t.}{Tornar rico; enricar.}{en.ri.que.cer}{0}
\verb{enriquecer}{ê}{}{}{}{}{Tornar melhor; desenvolver, aumentar.}{en.ri.que.cer}{\verboinum{15}}
\verb{enriquecimento}{}{}{}{}{s.m.}{Ato ou efeito de enriquecer. (\textit{O político foi acusado de enriquecimento ilícito.})}{en.ri.que.ci.men.to}{0}
\verb{enristar}{}{}{}{}{v.t.}{Colocar lança ou espada em riste; erguer, levantar.}{en.ris.tar}{0}
\verb{enristar}{}{}{}{}{}{Investir, atacar, assaltar.}{en.ris.tar}{\verboinum{1}}
\verb{enrocamento}{}{}{}{}{s.m.}{Conjunto de blocos de pedra tosca que resguardam a base dos muros do cais contra as ondas do mar.}{en.ro.ca.men.to}{0}
\verb{enrodilhar}{}{}{}{}{v.t.}{Enrolar, torcer pano ou fio em forma de rodilha. }{en.ro.di.lhar}{\verboinum{1}}
\verb{enrolado}{}{}{}{}{adj.}{A que se deu forma de rolo; enrodilhado.}{en.ro.la.do}{0}
\verb{enrolado}{}{}{}{}{}{Envolto, embrulhado.}{en.ro.la.do}{0}
\verb{enrolado}{}{Pop.}{}{}{}{Atrapalhado, confuso, complicado.}{en.ro.la.do}{0}
\verb{enrolador}{ô}{}{}{}{adj.}{Que enrola, ludibria; enganador.}{en.ro.la.dor}{0}
\verb{enrolamento}{}{}{}{}{s.m.}{Ato ou efeito de enrolar.}{en.ro.la.men.to}{0}
\verb{enrolamento}{}{}{}{}{}{Conjunto de fios enrolados em uma bobina ou motor.}{en.ro.la.men.to}{0}
\verb{enrolar}{}{}{}{}{v.t.}{Dar forma de rolo; enrodilhar, enroscar.}{en.ro.lar}{0}
\verb{enrolar}{}{}{}{}{}{Dar voltas ao redor; circundar, contornar, enlaçar.}{en.ro.lar}{0}
\verb{enrolar}{}{}{}{}{}{Envolver, embrulhar.}{en.ro.lar}{0}
\verb{enrolar}{}{}{}{}{}{Fingir, simular.}{en.ro.lar}{0}
\verb{enrolar}{}{Pop.}{}{}{}{Enganar, ludibriar, engambelar. (\textit{Ele me enrolou durante 8 anos até que resolvi desmanchar o noivado.})}{en.ro.lar}{0}
\verb{enrolar}{}{}{}{}{v.pron.}{Complicar"-se, enrascar"-se, atrapalhar"-se.}{en.ro.lar}{\verboinum{1}}
\verb{enroscamento}{}{}{}{}{s.m.}{Ato ou efeito de enroscar.}{en.ros.ca.men.to}{0}
\verb{enroscar}{}{}{}{}{v.t.}{Colocar em forma de rosca; enrolar, enrodilhar.}{en.ros.car}{0}
\verb{enroscar}{}{}{}{}{v.pron.}{Encontrar dificuldade; atrapalhar"-se.}{en.ros.car}{\verboinum{2}}
\verb{enroupar}{}{}{}{}{v.t.}{Cobrir ou prover de roupas; vestir; agasalhar.}{en.rou.par}{\verboinum{1}}
\verb{enrouquecer}{ê}{}{}{}{v.t.}{Tornar rouco; causar rouquidão.}{en.rou.que.cer}{\verboinum{15}}
\verb{enrubescer}{ê}{}{}{}{v.t.}{Fazer corar; tornar rubro; ruborizar.}{en.ru.bes.cer}{\verboinum{15}}
\verb{enrugar}{}{}{}{}{v.t.}{Tornar rugoso; encrespar, encarquilhar.}{en.ru.gar}{\verboinum{5}}
\verb{enrustido}{}{}{}{}{adj.}{Diz"-se do indivíduo que não se revela, não se expõe.}{en.rus.ti.do}{0}
\verb{enrustir}{}{}{}{}{v.t.}{Tornar oculto; esconder.}{en.rus.tir}{\verboinum{18}}
\verb{ensaboadela}{é}{}{}{}{s.f.}{Ato ou efeito de ensaboar ligeiramente.}{en.sa.bo.a.de.la}{0}
\verb{ensaboadela}{é}{Fig.}{}{}{}{Repreensão, bronca.}{en.sa.bo.a.de.la}{0}
\verb{ensaboadela}{é}{Fig.}{}{}{}{Noções sobre algum assunto.}{en.sa.bo.a.de.la}{0}
\verb{ensaboar}{}{}{}{}{v.t.}{Passar sabão ou lavar com sabão.}{en.sa.bo.ar}{0}
\verb{ensaboar}{}{Fig.}{}{}{}{Repreender, censurar.}{en.sa.bo.ar}{\verboinum{7}}
\verb{ensacador}{ô}{}{}{}{s.m.}{Indivíduo que ensaca, que coloca algo em sacos.}{en.sa.ca.dor}{0}
\verb{ensacar}{}{}{}{}{v.t.}{Pôr em saco ou saca.}{en.sa.car}{\verboinum{2}}
\verb{ensaiar}{}{}{}{}{v.t.}{Repetir para exercitar"-se ou memorizar; treinar, estudar.}{en.sai.ar}{0}
\verb{ensaiar}{}{}{}{}{}{Experimentar, provar.}{en.sai.ar}{\verboinum{1}}
\verb{ensaibrar}{}{}{}{}{v.t.}{Cobrir com saibro.}{en.sai.brar}{\verboinum{1}}
\verb{ensaio}{}{}{}{}{s.m.}{Ato ou efeito de ensaiar; treino, estudo.}{en.sai.o}{0}
\verb{ensaio}{}{}{}{}{}{Experiência, prova.}{en.sai.o}{0}
\verb{ensaio}{}{}{}{}{}{Texto que versa sobre determinado assunto, menos aprofundado que um tratado.}{en.sai.o}{0}
\verb{ensaísta}{}{}{}{}{s.2g.}{Autor de ensaios.}{en.sa.ís.ta}{0}
\verb{ensanchas}{}{}{}{}{s.f.pl.}{Oportunidade, ensejo.}{en.san.chas}{0}
\verb{ensandecer}{ê}{}{}{}{v.t.}{Fazer perder a razão; enlouquecer, endoidecer.}{en.san.de.cer}{\verboinum{15}}
\verb{ensandecido}{}{}{}{}{adj.}{Louco, endoidecido.}{en.san.de.ci.do}{0}
\verb{ensanguentar}{}{}{}{}{v.t.}{Cobrir, manchar ou encher de sangue.}{en.san.guen.tar}{\verboinum{1}}
\verb{ensarilhar}{}{}{}{}{v.t.}{Dobrar formando sarilho.}{en.sa.ri.lhar}{0}
\verb{ensarilhar}{}{}{}{}{}{Enredar, emaranhar.}{en.sa.ri.lhar}{\verboinum{1}}
\verb{enseada}{}{}{}{}{s.f.}{Pequena baía na costa do mar; angra.}{en.se.a.da}{0}
\verb{enseada}{}{}{}{}{}{Recôncavo.}{en.se.a.da}{0}
\verb{ensebado}{}{}{}{}{adj.}{Coberto de sebo.}{en.se.ba.do}{0}
\verb{ensebado}{}{}{}{}{}{Sujo, gorduroso.}{en.se.ba.do}{0}
\verb{ensebar}{}{}{}{}{v.t.}{Cobrir com sebo; engordurar.}{en.se.bar}{0}
\verb{ensebar}{}{}{}{}{}{Sujar em decorrência do uso normal.}{en.se.bar}{\verboinum{1}}
\verb{ensejar}{}{}{}{}{v.t.}{Dar oportunidade; possibilitar.}{en.se.jar}{0}
\verb{ensejar}{}{}{}{}{}{Esperar pela oportunidade; almejar.}{en.se.jar}{0}
\verb{ensejar}{}{}{}{}{}{Experimentar, tentar.}{en.se.jar}{\verboinum{1}}
\verb{ensejo}{ê}{}{}{}{s.m.}{Oportunidade, situação propícia. (\textit{Ele aproveitou o ensejo para agradecer os votos aos eleitores.})}{en.se.jo}{0}
\verb{ensiforme}{ó}{}{}{}{adj.2g.}{Que tem forma semelhante à de uma espada; xifoide.  }{en.si.for.me}{0}
\verb{ensilagem}{}{Bras.}{"-ens}{}{s.f.}{Ato ou efeito de ensilar; silagem.}{en.si.la.gem}{0}
\verb{ensilar}{}{Bras.}{}{}{v.t.}{Colocar em silos.}{en.si.lar}{\verboinum{1}}
\verb{ensimesmado}{}{}{}{}{adj.}{Absorto, introvertido.}{en.si.mes.ma.do}{0}
\verb{ensimesmar"-se}{}{}{}{}{v.pron.}{Concentrar"-se, absorver"-se, recolher"-se.}{en.si.mes.mar"-se}{\verboinum{1}}
\verb{ensinamento}{}{}{}{}{s.m.}{Ato ou efeito de ensinar; ensino.}{en.si.na.men.to}{0}
\verb{ensinamento}{}{}{}{}{}{Conjunto de conhecimentos; lição, doutrina, mandamento.}{en.si.na.men.to}{0}
\verb{ensinar}{}{}{}{}{v.t.}{Transmitir conhecimentos; educar.}{en.si.nar}{0}
\verb{ensinar}{}{}{}{}{}{Tornar conhecido; indicar.}{en.si.nar}{0}
\verb{ensinar}{}{}{}{}{}{Treinar, adestrar.}{en.si.nar}{0}
\verb{ensinar}{}{}{}{}{}{Punir, castigar.}{en.si.nar}{\verboinum{1}}
\verb{ensino}{}{}{}{}{s.m.}{Ato ou efeito de ensinar; instrução.}{en.si.no}{0}
\verb{ensino}{}{}{}{}{}{O conjunto de métodos e de pessoal envolvido na transmissão de conhecimentos.}{en.si.no}{0}
\verb{ensino}{}{}{}{}{}{A carreira de magistério.}{en.si.no}{0}
\verb{ensoberbecer}{ê}{}{}{}{v.t.}{Tornar orgulhoso, soberbo, elevado.}{en.so.ber.be.cer}{\verboinum{15}}
\verb{ensolarado}{}{}{}{}{adj.}{Em que há luz solar direta, especialmente abundante.}{en.so.la.ra.do}{0}
\verb{ensombrar}{}{}{}{}{v.t.}{Cobrir de sombras.}{en.som.brar}{0}
\verb{ensombrar}{}{}{}{}{}{Entristecer.}{en.som.brar}{\verboinum{1}}
\verb{ensopado}{}{}{}{}{adj.}{Muito molhado. (\textit{Ele tomou tanta chuva que chegou em casa ensopado.})}{en.so.pa.do}{0}
\verb{ensopado}{}{Cul.}{}{}{}{Diz"-se de prato feito de carne ou peixe preparado com molho abundante.}{en.so.pa.do}{0}
\verb{ensopar}{}{}{}{}{v.t.}{Embeber em líquido; encharcar.}{en.so.par}{0}
\verb{ensopar}{}{}{}{}{}{Converter em sopa.}{en.so.par}{\verboinum{1}}
\verb{ensurdecedor}{ô}{}{}{}{adj.}{Diz"-se de som ou ruído de volume muito alto.}{en.sur.de.ce.dor}{0}
\verb{ensurdecer}{ê}{}{}{}{v.t.}{Tornar surdo.}{en.sur.de.cer}{0}
\verb{ensurdecer}{ê}{}{}{}{}{Atordoar, aturdir, estontear.}{en.sur.de.cer}{\verboinum{15}}
\verb{ensurdecimento}{}{}{}{}{s.m.}{Ato ou efeito de ensurdecer.}{en.sur.de.ci.men.to}{0}
\verb{entablamento}{}{}{}{}{s.m.}{Conjunto de molduras que formam a parte superior de uma fachada.}{en.ta.bla.men.to}{0}
\verb{entabuar}{}{}{}{}{v.t.}{Guarnecer com tábuas.}{en.ta.bu.ar}{0}
\verb{entabuar}{}{}{}{}{v.pron.}{Endurecer.}{en.ta.bu.ar}{\verboinum{1}}
\verb{entabular}{}{}{}{}{v.t.}{Iniciar conversa, negociação etc.}{en.ta.bu.lar}{0}
\verb{entabular}{}{}{}{}{}{Pôr em ordem; dispor.}{en.ta.bu.lar}{\verboinum{1}}
\verb{entaipar}{}{}{}{}{v.t.}{Cobrir de taipas.}{en.tai.par}{0}
\verb{entaipar}{}{}{}{}{}{Encerrar, emparedar, enclausurar.}{en.tai.par}{\verboinum{1}}
\verb{entalado}{}{}{}{}{adj.}{Entre talas; apertado.}{en.ta.la.do}{0}
\verb{entalado}{}{}{}{}{}{Com a garganta obstruída; engasgado.}{en.ta.la.do}{0}
\verb{entalar}{}{}{}{}{v.t.}{Colocar entre talas.}{en.ta.lar}{0}
\verb{entalar}{}{}{}{}{}{Fazer entrar em local estreito.}{en.ta.lar}{\verboinum{1}}
\verb{entalhador}{ô}{}{}{}{adj.}{Que entalha, que esculpe em madeira.}{en.ta.lha.dor}{0}
\verb{entalhar}{}{}{}{}{v.t.}{Abrir cortes na madeira; esculpir, gravar.}{en.ta.lhar}{\verboinum{1}}
\verb{entalhe}{}{}{}{}{s.m.}{Ato ou efeito de entalhar.}{en.ta.lhe}{0}
\verb{entalho}{}{}{}{}{s.m.}{Escultura ou gravura em madeira; entalhe.}{en.ta.lho}{0}
\verb{entanguido}{}{}{}{}{adj.}{Tolhido de frio; encolhido.}{en.tan.gui.do}{0}
\verb{entanguir}{}{}{}{}{v.t.}{Tornar tolhido de frio; enregelar, encolher.}{en.tan.guir}{\verboinum{23}}
\verb{entanto}{}{}{}{}{adv.}{Nesse meio"-tempo; entrementes.}{en.tan.to}{0}
\verb{entanto}{}{}{}{}{}{Usado na expressão \textit{no entanto}: contudo, entretanto, todavia.}{en.tan.to}{0}
\verb{então}{}{}{}{}{adv.}{Nesse caso.}{en.tão}{0}
\verb{então}{}{}{}{}{}{Naquele tempo.}{en.tão}{0}
\verb{então}{}{}{}{}{s.m.}{A época passada; antanho.}{en.tão}{0}
\verb{entardecer}{ê}{}{}{}{v.i.}{Fazer"-se tarde.}{en.tar.de.cer}{\verboinum{15}}
\verb{entardecer}{ê}{}{}{}{s.m.}{O cair da tarde; ocaso.}{en.tar.de.cer}{0}
\verb{ente}{}{}{}{}{s.m.}{O que existe; coisa, objeto, matéria.}{en.te}{0}
\verb{ente}{}{}{}{}{}{Indivíduo, pessoa.}{en.te}{0}
\verb{ente}{}{}{}{}{}{O que se supõe existir.}{en.te}{0}
\verb{enteado}{}{}{}{}{s.m.}{O filho resultante do matrimônio anterior do atual cônjuge.}{en.te.a.do}{0}
\verb{entediado}{}{}{}{}{adj.}{Tomado pelo tédio; chateado, enfadado.}{en.te.di.a.do}{0}
\verb{entediar}{}{}{}{}{v.t.}{Causar tédio a; chatear, aborrecer, enfadar.}{en.te.di.ar}{\verboinum{1}}
\verb{entender}{ê}{}{}{}{v.t.}{Compreender.}{en.ten.der}{0}
\verb{entender}{ê}{}{}{}{}{Julgar, achar.}{en.ten.der}{0}
\verb{entender}{ê}{}{}{}{}{Ter conhecimento acerca de.}{en.ten.der}{0}
\verb{entender}{ê}{}{}{}{}{Ouvir claramente; perceber.}{en.ten.der}{0}
\verb{entender}{ê}{}{}{}{v.pron.}{Entrar em acordo; buscar entendimento.}{en.ten.der}{\verboinum{12}}
\verb{entender}{ê}{}{}{}{s.m.}{Opinião, juízo.}{en.ten.der}{0}
\verb{entendido}{}{}{}{}{adj.}{Que se entendeu; compreendido, combinado.}{en.ten.di.do}{0}
\verb{entendido}{}{}{}{}{}{Muito instruído; erudito.}{en.ten.di.do}{0}
\verb{entendido}{}{}{}{}{s.m.}{Indivíduo que é conhecedor de determinado assunto.}{en.ten.di.do}{0}
\verb{entendimento}{}{Filos.}{}{}{s.m.}{Faculdade de compreender as coisas; inteligência, juízo.}{en.ten.di.men.to}{0}
\verb{entendimento}{}{}{}{}{}{Opinião, juízo.}{en.ten.di.men.to}{0}
\verb{entendimento}{}{}{}{}{}{Acordo, consenso, combinação, pacto.  }{en.ten.di.men.to}{0}
\verb{entenebrecer}{ê}{}{}{}{v.t.}{Cobrir de trevas; obscurecer.}{en.te.ne.bre.cer}{0}
\verb{entenebrecer}{ê}{}{}{}{}{Afligir, entristecer.}{en.te.ne.bre.cer}{\verboinum{15}}
\verb{enterite}{}{Med.}{}{}{s.f.}{Inflamação do intestino.}{en.te.ri.te}{0}
\verb{enternecer}{ê}{}{}{}{v.t.}{Fazer uma pessoa ficar comovida, terna, branda, amorosa; abrandar.  }{en.ter.ne.cer}{0}
\verb{enternecer}{ê}{}{}{}{}{Tornar compassivo; sensibilizar.  }{en.ter.ne.cer}{\verboinum{15}}
\verb{enternecimento}{}{}{}{}{s.m.}{Ternura, meiguice.}{en.ter.ne.ci.men.to}{0}
\verb{enternecimento}{}{}{}{}{}{Dó, compaixão.}{en.ter.ne.ci.men.to}{0}
\verb{enterramento}{}{}{}{}{s.m.}{Ato ou efeito de enterrar; enterro, funeral.}{en.ter.ra.men.to}{0}
\verb{enterrar}{}{}{}{}{v.t.}{Colocar debaixo da terra.}{en.ter.rar}{0}
\verb{enterrar}{}{}{}{}{}{Sepultar, inumar.}{en.ter.rar}{0}
\verb{enterrar}{}{}{}{}{}{Cravar profundamente.}{en.ter.rar}{0}
\verb{enterrar}{}{}{}{}{}{Levar à ruína.}{en.ter.rar}{0}
\verb{enterrar}{}{}{}{}{v.pron.}{Embrenhar"-se, entranhar"-se.}{en.ter.rar}{\verboinum{1}}
\verb{enterro}{ê}{}{}{}{s.m.}{Ato ou efeito de enterrar; funeral.}{en.ter.ro}{0}
\verb{entesar}{}{}{}{}{v.t.}{Tornar teso ou tenso; retesar.}{en.te.sar}{0}
\verb{entesar}{}{}{}{}{}{Tornar reto; esticar.}{en.te.sar}{0}
\verb{entesar}{}{}{}{}{}{Endurecer, enrijecer.}{en.te.sar}{\verboinum{1}}
\verb{entesourar}{}{}{}{}{v.t.}{Acumular, juntar, amontoar.}{en.te.sou.rar}{\verboinum{1}}
\verb{entestar}{}{}{}{}{v.t.}{Confrontar, defrontar.}{en.tes.tar}{0}
\verb{entestar}{}{}{}{}{}{Ser contíguo; limitar"-se.}{en.tes.tar}{\verboinum{1}}
\verb{entibiar}{}{}{}{}{v.t.}{Tornar tíbio, frouxo.}{en.ti.bi.ar}{\verboinum{1}}
\verb{entidade}{}{}{}{}{s.f.}{Tudo que existe ou pode existir.}{en.ti.da.de}{0}
\verb{entidade}{}{}{}{}{}{A essência de uma coisa; individualidade, ser, essência.}{en.ti.da.de}{0}
\verb{entidade}{}{Bras.}{}{}{}{Instituição com finalidades específicas.}{en.ti.da.de}{0}
\verb{entisicar}{}{}{}{}{v.t.}{Tornar tísico.}{en.ti.si.car}{0}
\verb{entisicar}{}{}{}{}{}{Aborrecer, importunar, incomodar.}{en.ti.si.car}{\verboinum{2}}
\verb{entoação}{}{}{"-ões}{}{s.f.}{Ato ou efeito de entoar.}{en.to.a.ção}{0}
\verb{entoação}{}{}{"-ões}{}{}{Modulação de voz; inflexão.}{en.to.a.ção}{0}
\verb{entoar}{}{}{}{}{v.t.}{Fazer soar modulando a voz; cantar.}{en.to.ar}{0}
\verb{entoar}{}{}{}{}{}{Iniciar um canto.}{en.to.ar}{\verboinum{7}}
\verb{entocar}{}{}{}{}{v.t.}{Colocar pessoa ou animal numa toca.}{en.to.car}{0}
\verb{entocar}{}{}{}{}{}{Esconder, encafuar.}{en.to.car}{\verboinum{2}}
\verb{entojado}{}{}{}{}{adj.}{Que sofre de entojo; nauseado.}{en.to.ja.do}{0}
\verb{entojar}{}{}{}{}{v.t.}{Causar entojo; repugnar.}{en.to.jar}{0}
\verb{entojar}{}{}{}{}{}{Amolar, aborrecer.}{en.to.jar}{\verboinum{1}}
\verb{entojo}{ô}{}{}{}{s.m.}{Nojo, repugnância.}{en.to.jo}{0}
\verb{entojo}{ô}{}{}{}{}{Aversão a certos alimentos; fastio.}{en.to.jo}{0}
\verb{entomologia}{}{}{}{}{s.f.}{Ramo da biologia que estuda os insetos.}{en.to.mo.lo.gi.a}{0}
\verb{entomológico}{}{}{}{}{adj.}{Relativo a entomologia.}{en.to.mo.ló.gi.co}{0}
\verb{entomologista}{}{}{}{}{s.2g.}{Especialista em entomologia.}{en.to.mo.lo.gis.ta}{0}
\verb{entomólogo}{}{}{}{}{s.m.}{Entomologista.}{en.to.mó.lo.go}{0}
\verb{entonação}{}{}{"-ões}{}{s.f.}{Ato ou efeito de entonar; entoação.}{en.to.na.ção}{0}
\verb{entonar}{}{}{}{}{v.t.}{Entoar.}{en.to.nar}{\verboinum{1}}
\verb{entono}{ô}{}{}{}{s.m.}{Altivez, vaidade, orgulho.}{en.to.no}{0}
\verb{entontecer}{ê}{}{}{}{v.t.}{Causar tontura.}{en.ton.te.cer}{0}
\verb{entontecer}{ê}{}{}{}{}{Tornar tonto, idiota, tolo.}{en.ton.te.cer}{0}
\verb{entontecer}{ê}{}{}{}{v.i.}{Sentir tontura, ter vertigem.}{en.ton.te.cer}{\verboinum{15}}
\verb{entontecimento}{}{}{}{}{s.m.}{Ato ou efeito de entontecer.}{en.ton.te.ci.men.to}{0}
\verb{entornar}{}{}{}{}{v.t.}{Inclinar recipiente para despejar o conteúdo.}{en.tor.nar}{0}
\verb{entornar}{}{}{}{}{}{Despejar, derramar.}{en.tor.nar}{0}
\verb{entornar}{}{}{}{}{v.i.}{Tomar muita bebida alcoólica; embriagar"-se.}{en.tor.nar}{\verboinum{1}}
\verb{entorno}{ô}{}{}{}{s.m.}{Região situada em volta de algo; vizinhança.}{en.tor.no}{0}
\verb{entorpecente}{}{}{}{}{adj.2g.}{Que entorpece.}{en.tor.pe.cen.te}{0}
\verb{entorpecente}{}{}{}{}{s.m.}{Substância que age no sistema nervoso central provocando sensação de entorpecimento ou embriaguez.}{en.tor.pe.cen.te}{0}
\verb{entorpecer}{ê}{}{}{}{v.t.}{Causar torpor.}{en.tor.pe.cer}{0}
\verb{entorpecer}{ê}{}{}{}{}{Debilitar, enfraquecer, retardar.}{en.tor.pe.cer}{\verboinum{15}}
\verb{entorpecimento}{}{}{}{}{s.m.}{Ato ou efeito de entorpecer; fraqueza, preguiça, torpor.}{en.tor.pe.ci.men.to}{0}
\verb{entorse}{ó}{Med.}{}{}{s.m.}{Lesão causada por movimento ou torção violenta dos tendões de uma articulação.}{en.tor.se}{0}
\verb{entortar}{}{}{}{}{v.t.}{Tornar torto; arquear, empenar.}{en.tor.tar}{\verboinum{1}}
\verb{entrada}{}{}{}{}{s.f.}{Ato ou efeito de entrar.}{en.tra.da}{0}
\verb{entrada}{}{}{}{}{}{Lugar por onde se entra.}{en.tra.da}{0}
\verb{entrada}{}{}{}{}{}{Quantia paga no ato da compra como primeira de uma série de parcelas.}{en.tra.da}{0}
\verb{entrada}{}{}{}{}{}{Parte que entra; reentrância.}{en.tra.da}{0}
\verb{entrada}{}{}{}{}{}{Bilhete de ingresso em parques, espetáculos etc.}{en.tra.da}{0}
\verb{entrada}{}{Hist.}{}{}{}{Nome dado às expedições que partiam do litoral em direção ao interior para encontrar minas e escravizar índios.}{en.tra.da}{0}
\verb{entra"-e"-sai}{}{}{}{}{s.m.}{Movimento ininterrupto de entrada e saída de pessoas.}{en.tra"-e"-sai}{0}
\verb{entralhar}{}{}{}{}{v.t.}{Tecer as tralhas.}{en.tra.lhar}{0}
\verb{entralhar}{}{}{}{}{}{Prender na tralha.}{en.tra.lhar}{\verboinum{1}}
\verb{entrançado}{}{}{}{}{adj.}{Que forma trança.}{en.tran.ça.do}{0}
\verb{entrançado}{}{}{}{}{}{Entrelaçado, enleado.}{en.tran.ça.do}{0}
\verb{entrançar}{}{}{}{}{v.t.}{Passar um fio por outro; entrelaçar, trançar.}{en.tran.çar}{\verboinum{3}}
\verb{entrância}{}{Jur.}{}{}{s.f.}{Categoria das circunscrições judiciárias de um tribunal.}{en.trân.cia}{0}
\verb{entranha}{}{Anat.}{}{}{s.f.}{Víscera do abdômen ou do tórax.}{en.tra.nha}{0}
\verb{entranhado}{}{}{}{}{adj.}{Que se entranhou; íntimo, profundo, arraigado.}{en.tra.nha.do}{0}
\verb{entranhar}{}{}{}{}{v.t.}{Introduzir nas entranhas, fazer penetrar.}{en.tra.nhar}{0}
\verb{entranhar}{}{}{}{}{}{Embrenhar.}{en.tra.nhar}{\verboinum{1}}
\verb{entranhas}{}{}{}{}{s.f.pl.}{O ventre materno.}{en.tra.nhas}{0}
\verb{entranhas}{}{}{}{}{}{Parte de dentro da barriga.}{en.tra.nhas}{0}
\verb{entranhas}{}{}{}{}{}{Parte de dentro da terra.}{en.tra.nhas}{0}
\verb{entrante}{}{}{}{}{adj.2g.}{Que está entrando ou começando.}{en.tran.te}{0}
\verb{entrar}{}{}{}{}{v.i.}{Ir ou vir para dentro.}{en.trar}{0}
\verb{entrar}{}{}{}{}{v.t.}{Interferir em algo.}{en.trar}{0}
\verb{entrar}{}{}{}{}{}{Ser admitido.}{en.trar}{0}
\verb{entrar}{}{}{}{}{}{Fazer parte de algo.}{en.trar}{\verboinum{1}}
\verb{entravar}{}{}{}{}{v.t.}{Ter alguma coisa que dificulta ou impossibilita uma ação; impedir, obstruir, travar.}{en.tra.var}{\verboinum{1}}
\verb{entrave}{}{}{}{}{s.m.}{Coisa que dificulta ou impossibilita uma ação.}{en.tra.ve}{0}
\verb{entre}{}{}{}{}{prep.}{Estabelece relação de espaço ou de intervalo de tempo que separa dois fatos.}{en.tre}{0}
\verb{entre}{}{}{}{}{}{Estabelece relação de quantidade aproximada.}{en.tre}{0}
\verb{entre}{}{}{}{}{}{Estabelece relação de meio, de inclusão, de diferenciação etc.}{en.tre}{0}
\verb{entreaberto}{é}{}{}{}{adj.}{Que se encontra parcialmente aberto.}{en.tre.a.ber.to}{0}
\verb{entreabrir}{}{}{}{}{v.t.}{Abrir em parte.}{en.tre.a.brir}{\verboinum{18}}
\verb{entreato}{}{}{}{}{s.m.}{Intervalo entre dois atos de uma peça teatral.}{en.tre.a.to}{0}
\verb{entrecasca}{}{}{}{}{s.f.}{Parte interna da casca das árvores.}{en.tre.cas.ca}{0}
\verb{entrecasco}{}{}{}{}{s.m.}{Parte superior do casco dos animais.}{en.tre.cas.co}{0}
\verb{entrecasco}{}{}{}{}{}{Entrecasca.}{en.tre.cas.co}{0}
\verb{entrecerrar}{}{}{}{}{v.t.}{Fechar incompletamente; entrefechar.}{en.tre.cer.rar}{\verboinum{1}}
\verb{entrecho}{ê}{}{}{}{s.m.}{Conjunto dos incidentes que constituem a ação de uma obra de ficção; urdidura, intriga, fábula.}{en.tre.cho}{0}
\verb{entrechocar"-se}{}{}{}{}{v.pron.}{Chocar"-se ou bater"-se mutuamente.}{en.tre.cho.car"-se}{0}
\verb{entrechocar"-se}{}{Fig.}{}{}{}{Estar em contradição; contrariar"-se.}{en.tre.cho.car"-se}{\verboinum{2}}
\verb{entrechoque}{ó}{}{}{}{s.m.}{Choque, colisão.}{en.tre.cho.que}{0}
\verb{entrechoque}{ó}{}{}{}{}{Oposição, confronto.}{en.tre.cho.que}{0}
\verb{entrecortado}{}{}{}{}{adj.}{Cortado a intervalos; interrompido.}{en.tre.cor.ta.do}{0}
\verb{entrecortar}{}{}{}{}{v.t.}{Interromper alguma coisa a espaços de tempos.}{en.tre.cor.tar}{\verboinum{1}}
\verb{entrecosto}{ô}{}{}{}{s.m.}{Carne entre as costelas do animal, junto ao espinhaço.}{en.tre.cos.to}{0}
\verb{entrecruzar"-se}{}{}{}{}{v.pron.}{Cruzar"-se reciprocamente.}{en.tre.cru.zar"-se}{\verboinum{1}}
\verb{entrefechar}{}{}{}{}{v.t.}{Fechar um pouco.}{en.tre.fe.char}{\verboinum{1}}
\verb{entrefolha}{ô}{}{}{}{s.f.}{Folha em branco que se intercala às folhas impressas.}{en.tre.fo.lha}{0}
\verb{entrega}{é}{}{}{}{s.f.}{Ato ou efeito de ceder algo; cessão, transmissão.}{en.tre.ga}{0}
\verb{entrega}{é}{}{}{}{}{O que se entregou.}{en.tre.ga}{0}
\verb{entrega}{é}{}{}{}{}{Rendição, capitulação.}{en.tre.ga}{0}
\verb{entrega}{é}{}{}{}{}{Traição, perfídia.}{en.tre.ga}{0}
\verb{entrega}{é}{}{}{}{}{Dedicação integral; consagração.}{en.tre.ga}{0}
\verb{entrega}{é}{}{}{}{}{Revelação, denúncia.}{en.tre.ga}{0}
\verb{entregador}{ô}{}{}{}{adj.}{Diz"-se daquele que entrega.}{en.tre.ga.dor}{0}
\verb{entregador}{ô}{}{}{}{}{Traiçoeiro, traidor.}{en.tre.ga.dor}{0}
\verb{entregar}{}{}{}{}{v.t.}{Passar pessoa ou coisa para alguém.}{en.tre.gar}{0}
\verb{entregar}{}{}{}{}{}{Dizer a uma pessoa o que a outra fez ou quer fazer em segredo; delatar, denunciar, trair.}{en.tre.gar}{0}
\verb{entregar}{}{}{}{}{v.pron.}{Empregar os seus esforços para alguma finalidade; dedicar"-se.}{en.tre.gar}{0}
\verb{entregar}{}{}{}{}{}{Render"-se.}{en.tre.gar}{\verboinum{5}}
\verb{entregue}{é}{}{}{}{adj.2g.}{Que se entregou; posto nas mãos ou na posse de alguém; dado, confiado.}{en.tre.gue}{0}
\verb{entregue}{é}{}{}{}{}{Aplicado exclusiva ou especialmente a algo; dedicado, devotado.}{en.tre.gue}{0}
\verb{entregue}{é}{}{}{}{}{Absorto, ocupado.}{en.tre.gue}{0}
\verb{entregue}{é}{}{}{}{}{Fraco, cansado.}{en.tre.gue}{0}
\verb{entrelaçado}{}{}{}{}{adj.}{Que se entrelaçou; preso, enlaçado um no outro.}{en.tre.la.ça.do}{0}
\verb{entrelaçado}{}{}{}{}{}{Embaralhado, mesclado, misturado.}{en.tre.la.ça.do}{0}
\verb{entrelaçado}{}{}{}{}{s.m.}{Grupo, conjunto de coisas entrelaçadas.}{en.tre.la.ça.do}{0}
\verb{entrelaçamento}{}{}{}{}{s.m.}{Ato ou efeito de entrelaçar; junção, mistura.}{en.tre.la.ça.men.to}{0}
\verb{entrelaçamento}{}{}{}{}{}{Cruzamento.}{en.tre.la.ça.men.to}{0}
\verb{entrelaçar}{}{}{}{}{v.t.}{Juntar, prender algo, enlaçando.}{en.tre.la.çar}{0}
\verb{entrelaçar}{}{}{}{}{}{Cruzar.}{en.tre.la.çar}{\verboinum{3}}
\verb{entrelinha}{}{}{}{}{s.f.}{Espaço entre duas linhas.}{en.tre.li.nha}{0}
\verb{entrelinha}{}{}{}{}{}{Texto escrito nesse espaço.}{en.tre.li.nha}{0}
\verb{entrelinhar}{}{}{}{}{v.t.}{Escrever ou colocar em entrelinhas.}{en.tre.li.nhar}{0}
\verb{entrelinhar}{}{}{}{}{}{Traduzir, comentar texto.}{en.tre.li.nhar}{\verboinum{1}}
\verb{entreluzir}{}{}{}{}{v.i.}{Começar a luzir; luzir frouxamente.}{en.tre.lu.zir}{0}
\verb{entreluzir}{}{}{}{}{}{Formar ideia, adquirir conhecimento por meio dos sentidos; perceber. }{en.tre.lu.zir}{0}
\verb{entreluzir}{}{}{}{}{v.i.}{Deixar"-se entrever; mostrar"-se incompletamente.}{en.tre.lu.zir}{\verboinum{21}}
\verb{entremear}{}{}{}{}{v.t.}{Interromper alternadamente a sequência de alguma coisa com a introdução de outra coisa; intercalar, intermediar.}{en.tre.me.ar}{\verboinum{4}}
\verb{entremeio}{ê}{}{}{}{s.m.}{Bordado com uma fita passada pelas aberturas do meio.}{en.tre.mei.o}{0}
\verb{entremeio}{ê}{}{}{}{}{Intervalo.}{en.tre.mei.o}{0}
\verb{entrementes}{}{}{}{}{adv.}{Entretanto, naquela ou nesta ocasião; neste meio"-tempo.}{en.tre.men.tes}{0}
\verb{entremeter}{ê}{}{}{}{v.t.}{Colocar de permeio; entremear, intrometer.}{en.tre.me.ter}{0}
\verb{entremeter}{ê}{}{}{}{v.pron.}{Ter participação ou influência; interferir, intervir.}{en.tre.me.ter}{0}
\verb{entremeter}{ê}{}{}{}{}{Colocar"-se onde não deve, onde não é chamado; intrometer"-se.}{en.tre.me.ter}{\verboinum{12}}
\verb{entremez}{ê}{Art.}{}{}{s.m.}{Entre os séculos \textsc{xvi} e \textsc{xviii}, representação jocosa ou burlesca de curta duração, que servia de entreato da peça principal; farsa.}{en.tre.mez}{0}
\verb{entremostrar}{}{}{}{}{v.t.}{Mostrar em parte.}{en.tre.mos.trar}{\verboinum{1}}
\verb{entrenó}{}{Bot.}{}{}{s.m.}{Porção do caule da planta, situada entre dois nós.}{en.tre.nó}{0}
\verb{entreolhar"-se}{}{}{}{}{v.pron.}{Olhar"-se um ao outro.}{en.tre.o.lhar"-se}{\verboinum{1}}
\verb{entreouvir}{}{}{}{}{v.t.}{Ouvir vagamente, parcialmente, de forma indistinta, confusa.}{en.tre.ou.vir}{\verboinum{58}}
\verb{entreperna}{é}{}{}{}{s.f.}{A parte das calças onde se juntam as pernas.}{en.tre.per.na}{0}
\verb{entreperna}{é}{}{}{}{}{Peça da carne tirada da região entre as pernas da rês, própria para churrasco ou assado.}{en.tre.per.na}{0}
\verb{entrepor}{}{}{}{}{v.t.}{Colocar alguma coisa entre duas outras; interpor.}{en.tre.por}{\verboinum{60}}
\verb{entreposto}{ô}{}{"-s ⟨ó⟩}{"-a ⟨ó⟩}{adj.}{Que se entrepôs; interposto.}{en.tre.pos.to}{0}
\verb{entreposto}{ô}{}{"-s ⟨ó⟩}{"-a ⟨ó⟩}{s.m.}{Depósito de mercadorias.}{en.tre.pos.to}{0}
\verb{entreposto}{ô}{}{"-s ⟨ó⟩}{"-a ⟨ó⟩}{}{Casa em que se compram ou se vendem muitas mercadorias.}{en.tre.pos.to}{0}
\verb{entressachar}{}{}{}{}{v.t.}{Pôr entre outras coisas; intercalar, alternar.}{en.tres.sa.char}{\verboinum{1}}
\verb{entressafra}{}{}{}{}{s.f.}{Período entre duas safras.}{en.tres.sa.fra}{0}
\verb{entressemear}{}{}{}{}{v.t.}{Plantar, semear intercalando.}{en.tres.se.me.ar}{\verboinum{4}}
\verb{entressola}{ó}{}{}{}{s.f.}{Peça entre a sola e a palmilha de um calçado.}{en.tres.so.la}{0}
\verb{entretanto}{}{}{}{}{adv.}{Neste meio tempo, entrementes.}{en.tre.tan.to}{0}
\verb{entretanto}{}{}{}{}{conj.}{No entanto; contudo, todavia.}{en.tre.tan.to}{0}
\verb{entretecer}{ê}{}{}{}{v.t.}{Entrelaçar ao tecer; entremear.}{en.tre.te.cer}{\verboinum{15}}
\verb{entretecimento}{}{}{}{}{s.m.}{Entrelaçamento ao tecer.}{en.tre.te.ci.men.to}{0}
\verb{entretecimento}{}{}{}{}{}{Introdução ou inclusão de uma coisa em outra; intercalação, interposição.}{en.tre.te.ci.men.to}{0}
\verb{entretela}{é}{}{}{}{s.f.}{Tecido grosso colocado entre o forro e a fazenda de uma roupa.}{en.tre.te.la}{0}
\verb{entretempo}{}{}{}{}{s.m.}{Período, tempo intermediário.}{en.tre.tem.po}{0}
\verb{entretenimento}{}{}{}{}{s.m.}{Ato ou efeito de entreter, de distrair.}{en.tre.te.ni.men.to}{0}
\verb{entretenimento}{}{}{}{}{}{Aquilo que distrai; distração, divertimento.}{en.tre.te.ni.men.to}{0}
\verb{entreter}{ê}{}{}{}{v.t.}{Desviar a atenção; distrair, divertir.}{en.tre.ter}{\verboinum{39}}
\verb{entretimento}{}{}{}{}{s.m.}{Entretenimento.}{en.tre.ti.men.to}{0}
\verb{entrevado}{}{}{}{}{adj.}{Que se entrevou; paralisado, tolhido.}{en.tre.va.do}{0}
\verb{entrevado}{}{}{}{}{}{Diz"-se de quem tem seus movimentos impedidos; paralítico.}{en.tre.va.do}{0}
\verb{entrevar}{}{}{}{}{v.t.}{Perder os movimentos de parte do corpo; tornar paralítico.}{en.tre.var}{\verboinum{1}}
\verb{entrever}{ê}{}{}{}{v.t.}{Ver de modo indistinto ou rápido.}{en.tre.ver}{\verboinum{46}}
\verb{entrevero}{ê}{}{}{}{s.m.}{Confusão, desordem.}{en.tre.ve.ro}{0}
\verb{entrevero}{ê}{}{}{}{}{Briga.}{en.tre.ve.ro}{0}
\verb{entrevista}{}{}{}{}{s.f.}{Conversa em que alguém faz perguntas sobre determinado assunto a uma pessoa, que as responde.}{en.tre.vis.ta}{0}
\verb{entrevista}{}{}{}{}{}{Encontro com hora marcada.}{en.tre.vis.ta}{0}
%\verb{}{}{}{}{}{}{}{}{0}
\verb{entrevistar}{}{}{}{}{v.t.}{Fazer uma entrevista com alguém para fins de divulgação.}{en.tre.vis.tar}{0}
\verb{entrevistar}{}{}{}{}{v.pron.}{Encontrar"-se com alguém para tratar de alguma coisa.}{en.tre.vis.tar}{\verboinum{1}}
\verb{entrincheiramento}{}{}{}{}{s.m.}{Ato ou efeito de entrincheirar; abarreiramento, fortificação.}{en.trin.chei.ra.men.to}{0}
\verb{entrincheiramento}{}{}{}{}{}{Cortadura, trincheira ou série de trincheiras.}{en.trin.chei.ra.men.to}{0}
\verb{entrincheirar}{}{}{}{}{v.t.}{Proteger com trincheira; defender.}{en.trin.chei.rar}{\verboinum{1}}
\verb{entristecer}{ê}{}{}{}{v.t.}{Tornar triste; afligir, penalizar.}{en.tris.te.cer}{\verboinum{15}}
\verb{entroncado}{}{}{}{}{adj.}{Que criou tronco, que engrossou.}{en.tron.ca.do}{0}
\verb{entroncado}{}{}{}{}{}{Diz"-se de indivíduo corpulento, de ombros largos, e em geral, de estatura mediana.}{en.tron.ca.do}{0}
\verb{entroncamento}{}{}{}{}{s.m.}{Junção de dois ou mais caminhos.}{en.tron.ca.men.to}{0}
\verb{entroncar}{}{}{}{}{v.i.}{Aumentar o tronco; engrossar.}{en.tron.car}{0}
\verb{entroncar}{}{}{}{}{}{Reunir um caminho a outro.}{en.tron.car}{\verboinum{2}}
\verb{entronizar}{}{}{}{}{v.t.}{Elevar ao trono.}{en.tro.ni.zar}{0}
\verb{entronizar}{}{}{}{}{}{Pôr imagem no altar.}{en.tro.ni.zar}{0}
\verb{entronizar}{}{}{}{}{}{Exaltar.}{en.tro.ni.zar}{\verboinum{1}}
\verb{entropia}{}{Fís.}{}{}{s.f.}{Medida da degradação de energia de um sistema termodinâmico.}{en.tro.pi.a}{0}
\verb{entropia}{}{Fís.}{}{}{}{Medida da energia não disponível para a realização do trabalho.}{en.tro.pi.a}{0}
\verb{entrosado}{}{}{}{}{adj.}{Que se entrosou; adaptado, engrenado, ordenado, afinado.}{en.tro.sa.do}{0}
\verb{entrosamento}{}{}{}{}{s.m.}{Ato ou efeito de entrosar; entendimento, acordo.}{en.tro.sa.men.to}{0}
\verb{entrosamento}{}{Fig.}{}{}{}{Harmonia, sintonia.}{en.tro.sa.men.to}{0}
\verb{entrosar}{}{}{}{}{v.t.}{Fazer com que um elemento combine com outro.}{en.tro.sar}{\verboinum{1}}
\verb{entrouxar}{ch}{}{}{}{v.t.}{Reunir coisas em trouxa.}{en.trou.xar}{\verboinum{1}}
\verb{entrudo}{}{}{}{}{s.m.}{Antiga brincadeira carnavalesca, que consistia em as pessoas jogarem água, talco, tinta, umas nas outras.}{en.tru.do}{0}
\verb{entrudo}{}{Desus.}{}{}{}{Carnaval.}{en.tru.do}{0}
\verb{entulhar}{}{}{}{}{v.t.}{Encher um lugar de muitas pessoas ou coisas; abarrotar, entupir.}{en.tu.lhar}{\verboinum{1}}
\verb{entulho}{}{}{}{}{s.m.}{Restos de construção.}{en.tu.lho}{0}
\verb{entulho}{}{}{}{}{}{Coisa inútil ocupando espaço.}{en.tu.lho}{0}
\verb{entupido}{}{}{}{}{adj.}{Que se entupiu; obstruído, tapado.}{en.tu.pi.do}{0}
\verb{entupido}{}{Fig.}{}{}{}{Embaraçado, atrapalhado.}{en.tu.pi.do}{0}
\verb{entupigaitar}{}{}{}{}{v.t.}{Atrapalhar, embaraçar, confundir, desnortear, desorientar. }{en.tu.pi.gai.tar}{\verboinum{1}}
\verb{entupimento}{}{}{}{}{s.m.}{Ato ou efeito de entupir; obstrução, fechamento.}{en.tu.pi.men.to}{0}
\verb{entupimento}{}{}{}{}{}{Abarrotamento, entulhamento.}{en.tu.pi.men.to}{0}
\verb{entupir}{}{}{}{}{v.t.}{Acumular alguma coisa dentro de um cano, fechando a passagem de um líquido.}{en.tu.pir}{0}
\verb{entupir}{}{}{}{}{}{Encher muito; abarrotar, entulhar.}{en.tu.pir}{\verboinum{33}}
\verb{enturmar}{}{}{}{}{v.t.}{Fazer entrar numa turma.}{en.tur.mar}{\verboinum{1}}
\verb{enturvar}{}{}{}{}{v.t.}{Tornar turvo, escuro; ensombrar}{en.tur.var}{0}
\verb{enturvar}{}{Fig.}{}{}{v.pron.}{Amuar"-se, perturbar"-se.}{en.tur.var}{\verboinum{1}}
\verb{entusiasmar}{}{}{}{}{v.t.}{Encher alguém de entusiasmo; encantar, encorajar.}{en.tu.si.as.mar}{\verboinum{1}}
\verb{entusiasmo}{}{}{}{}{s.m.}{Estado de grande alegria e exaltação; animação.}{en.tu.si.as.mo}{0}
\verb{entusiasta}{}{}{}{}{s.2g.}{Indivíduo que se entusiasma, que se dedica vivamente a alguma coisa.}{en.tu.si.as.ta}{0}
\verb{entusiástico}{}{}{}{}{adj.}{Que tem ou manifesta entusiasmo.}{en.tu.si.ás.ti.co}{0}
\verb{enumeração}{}{}{"-ões}{}{s.f.}{Exposição ou relação de coisas uma a uma.}{e.nu.me.ra.ção}{0}
\verb{enumeração}{}{}{"-ões}{}{}{Conta, cálculo.}{e.nu.me.ra.ção}{0}
\verb{enumerar}{}{}{}{}{v.t.}{Especificar um a um.}{e.nu.me.rar}{0}
\verb{enumerar}{}{}{}{}{}{Narrar minuciosamente.}{e.nu.me.rar}{\verboinum{1}}
\verb{enunciação}{}{}{"-ões}{}{s.f.}{Ato ou efeito de enunciar; declaração, exposição.}{e.nun.ci.a.ção}{0}
\verb{enunciado}{}{}{}{}{adj.}{Que se enunciou; expresso, declarado.}{e.nun.ci.a.do}{0}
\verb{enunciado}{}{}{}{}{s.m.}{Proposição, exposição.}{e.nun.ci.a.do}{0}
\verb{enunciado}{}{}{}{}{}{Qualquer texto oral ou escrito, produzido por emissor.}{e.nun.ci.a.do}{0}
\verb{enunciar}{}{}{}{}{v.t.}{Expressar por meio de palavras o que se deseja; exprimir, manifestar.}{e.nun.ci.ar}{\verboinum{1}}
\verb{envaidecer}{ê}{}{}{}{v.t.}{Tornar vaidoso, orgulhoso; vangloriar; ensoberbecer.}{en.vai.de.cer}{\verboinum{15}}
\verb{envasilhar}{}{}{}{}{v.t.}{Despejar líquido em vasilhas, pipas, tonéis ou garrafas.}{en.va.si.lhar}{\verboinum{1}}
\verb{envelhecer}{ê}{}{}{}{v.t.}{Tornar velho ou mais velho.}{en.ve.lhe.cer}{0}
\verb{envelhecer}{ê}{}{}{}{v.i.}{Cair em desuso.}{en.ve.lhe.cer}{\verboinum{15}}
\verb{envelhecimento}{}{}{}{}{s.m.}{Ato ou efeito de envelhecer, de tornar"-se mais velho. }{en.ve.lhe.ci.men.to}{0}
\verb{envelhecimento}{}{}{}{}{}{Maturação.}{en.ve.lhe.ci.men.to}{0}
\verb{envelopar}{}{Bras.}{}{}{v.t.}{Colocar em envelope.}{en.ve.lo.par}{\verboinum{1}}
\verb{envelope}{ó}{}{}{}{s.m.}{Invólucro de papel para acondicionar correspondência ou quaisquer outros documentos que se queira enviar.}{en.ve.lo.pe}{0}
\verb{envenenado}{}{}{}{}{adj.}{Que se envenenou; intoxicado.}{en.ve.ne.na.do}{0}
\verb{envenenamento}{}{}{}{}{s.m.}{Ato ou efeito de envenenar; intoxicação.}{en.ve.ne.na.men.to}{0}
\verb{envenenar}{}{}{}{}{v.t.}{Pôr veneno; dar veneno; intoxicar.}{en.ve.ne.nar}{\verboinum{1}}
\verb{enverdecer}{ê}{}{}{}{v.t.}{Tornar verde; assumir a cor verde.}{en.ver.de.cer}{0}
\verb{enverdecer}{ê}{}{}{}{}{Cobrir de plantas e folhas verdes.}{en.ver.de.cer}{\verboinum{15}}
\verb{enveredar}{}{}{}{}{v.t.}{Entrar, seguir por uma vereda ou caminho; encaminhar"-se.}{en.ve.re.dar}{\verboinum{1}}
\verb{envergadura}{}{}{}{}{s.f.}{Distância de uma extremidade à outra das asas abertas de uma ave, ou das asas de um avião.}{en.ver.ga.du.ra}{0}
\verb{envergadura}{}{Fig.}{}{}{}{Importância, autoridade, peso, valor, mérito.}{en.ver.ga.du.ra}{0}
\verb{envergadura}{}{Fig.}{}{}{}{Competência, aptidão, talento.}{en.ver.ga.du.ra}{0}
\verb{envergar}{}{}{}{}{v.t.}{Curvar, vergar, entortar.}{en.ver.gar}{\verboinum{5}}
\verb{envergonhar}{}{}{}{}{v.t.}{Encher de vergonha; vexar.}{en.ver.go.nhar}{\verboinum{1}}
\verb{envernizar}{}{}{}{}{v.t.}{Revestir ou lustrar com verniz.}{en.ver.ni.zar}{\verboinum{1}}
\verb{enverrugar}{}{}{}{}{v.t.}{Cobrir de verrugas.}{en.ver.ru.gar}{\verboinum{5}}
\verb{envesgar}{}{}{}{}{v.t.}{Tornar vesgo, entortar os olhos, a vista.}{en.ves.gar}{\verboinum{5}}
\verb{enviado}{}{}{}{}{adj.}{Que se enviou; remetido, expedido, mandado, endereçado.}{en.vi.a.do}{0}
\verb{enviado}{}{}{}{}{s.m.}{Pessoa que leva uma carta, um objeto etc.; portador, mensageiro.}{en.vi.a.do}{0}
\verb{enviar}{}{}{}{}{v.t.}{Despachar, endereçar, remeter.}{en.vi.ar}{0}
\verb{enviar}{}{}{}{}{}{Mandar em missão; encaminhar.}{en.vi.ar}{\verboinum{1}}
\verb{envidar}{}{}{}{}{v.t.}{Empregar esforços com empenho.}{en.vi.dar}{0}
\verb{envidar}{}{}{}{}{}{Convidar ou desafiar alguém a aceitar uma aposta, a participar de um jogo.}{en.vi.dar}{\verboinum{1}}
\verb{envidraçar}{}{}{}{}{v.t.}{Pôr vidros ou vidraças.}{en.vi.dra.çar}{\verboinum{3}}
\verb{enviesado}{}{}{}{}{adj.}{Que se enviesou; inclinado, torto.}{en.vi.e.sa.do}{0}
\verb{enviesar}{}{}{}{}{v.t.}{Pôr de viés, entortar.}{en.vi.e.sar}{\verboinum{1}}
\verb{envilecer}{ê}{}{}{}{v.t.}{Tornar vil; aviltar, humilhar.}{en.vi.le.cer}{\verboinum{15}}
\verb{envio}{}{}{}{}{s.m.}{Ato ou efeito de enviar; enviamento, remessa, expedição, despacho.}{en.vi.o}{0}
\verb{enviuvar}{}{}{}{}{v.t.}{Tornar viúvo ou viúva.}{en.vi.u.var}{\verboinum{8}}
\verb{envolta}{ô}{Bras.}{}{}{s.f.}{Confusão, desordem, refrega, tumulto.}{en.vol.ta}{0}
\verb{envolto}{ô}{}{}{}{adj.}{Que se envolveu; envolvido, enrolado, embrulhado.}{en.vol.to}{0}
\verb{envoltório}{}{}{}{}{s.m.}{Coisa que envolve outra; invólucro; capa; embrulho.}{en.vol.tó.rio}{0}
\verb{envolvente}{}{}{}{}{adj.2g.}{Que envolve; envolvedor.}{en.vol.ven.te}{0}
\verb{envolvente}{}{}{}{}{}{Que seduz, encanta; atraente, sedutor, encantador.}{en.vol.ven.te}{0}
\verb{envolver}{ê}{}{}{}{v.t.}{Cobrir completamente; embrulhar, enrolar.  }{en.vol.ver}{0}
\verb{envolver}{ê}{}{}{}{}{Abarcar, abranger, compreender.}{en.vol.ver}{0}
\verb{envolver}{ê}{}{}{}{}{Seduzir, cativar.}{en.vol.ver}{0}
\verb{envolver}{ê}{}{}{}{}{Acarretar, implicar, enredar.}{en.vol.ver}{0}
\verb{envolver}{ê}{}{}{}{}{Intrometer"-se, imiscuir"-se. (\textit{Ele não queria se envolver naquele assunto.})}{en.vol.ver}{\verboinum{12}}
\verb{envolvimento}{}{}{}{}{s.m.}{Ato ou efeito de envolver; envoltura.}{en.vol.vi.men.to}{0}
\verb{envolvimento}{}{Fig.}{}{}{}{Qualquer relacionamento, especialmente amoroso ou afetivo; ligação amorosa, caso, aventura.}{en.vol.vi.men.to}{0}
\verb{enxada}{ch}{}{}{}{s.f.}{Ferramenta que consiste num cabo longo com uma lâmina de metal numa das extremidades, usada para capinar ou revolver a terra.}{en.xa.da}{0}
\verb{enxadão}{ch}{}{"-ões}{}{s.m.}{Enxada grande; alvião, marraco.}{en.xa.dão}{0}
\verb{enxadrezar}{ch}{}{}{}{v.t.}{Dividir em quadrados, à semelhança do tabuleiro de xadrez.}{en.xa.dre.zar}{\verboinum{1}}
\verb{enxadrismo}{ch}{}{}{}{s.m.}{Arte, técnica ou gosto pelo jogo de xadrez.}{en.xa.dris.mo}{0}
\verb{enxadrista}{ch}{}{}{}{s.2g.}{Jogador ou jogadora de xadrez; xadrezista.}{en.xa.dris.ta}{0}
\verb{enxaguar}{ch}{}{}{}{v.t.}{Lavar retirando o sabão.}{en.xa.guar}{\verboinum{9}}
\verb{enxame}{ch}{}{}{}{s.m.}{Conjunto de abelhas de uma colmeia. }{en.xa.me}{0}
\verb{enxamear}{ch}{}{}{}{v.i.}{Formar enxame.}{en.xa.me.ar}{0}
\verb{enxamear}{ch}{}{}{}{v.t.}{Reunir em enxame.}{en.xa.me.ar}{\verboinum{4}}
\verb{enxaqueca}{ch\ldots{}ê}{}{}{}{s.f.}{Forte dor de cabeça, periódica, que afeta um lado da cabeça, e é geralmente acompanhada de náusea.}{en.xa.que.ca}{0}
\verb{enxárcia}{ch}{}{}{}{s.f.}{Conjunto de cabos que sustentam os mastros de uma embarcação a vela.}{en.xár.cia}{0}
\verb{enxaropar}{ch}{}{}{}{v.t.}{Transformar em xarope, ou tornar tão adocicado quanto o xarope.}{en.xa.ro.par}{\verboinum{1}}
\verb{enxerga}{chê}{}{}{}{s.f.}{Colchão rústico, geralmente de palha.}{en.xer.ga}{0}
\verb{enxerga}{chê}{Por ext.}{}{}{}{Cama pobre, grosseira; catre.}{en.xer.ga}{0}
\verb{enxergão}{ch}{}{"-ões}{}{s.m.}{Colchão de palha que se coloca debaixo do colchão da cama.}{en.xer.gão}{0}
\verb{enxergar}{ch}{}{}{}{v.t.}{Alcançar com a vista; avistar, divisar, descortinar.}{en.xer.gar}{\verboinum{5}}
\verb{enxerido}{ch}{Bras.}{}{}{adj.}{Que se intromete onde não foi chamado; intrometido, xereta, bisbilhoteiro.}{en.xe.ri.do}{0}
\verb{enxerido}{ch}{}{}{}{s.m.}{Essa pessoa.}{en.xe.ri.do}{0}
\verb{enxerir}{ch}{}{}{}{v.pron.}{Meter"-se em assunto de outra pessoa, sem ter direito ou sem pedir licença; intrometer"-se. (\textit{Existem pessoas que gostam de se enxerir nos assuntos que não lhes dizem respeito.})}{en.xe.rir"-se}{\verboinum{29}}
\verb{enxertar}{ch}{}{}{}{v.t.}{Fazer enxerto.}{en.xer.tar}{0}
\verb{enxertar}{ch}{}{}{}{}{Inserir, introduzir, acrescentar.}{en.xer.tar}{\verboinum{1}}
\verb{enxertia}{ch}{}{}{}{s.f.}{Ato ou efeito de enxertar; enxerto.}{en.xer.ti.a}{0}
\verb{enxerto}{chê}{}{}{}{s.m.}{Inserir uma parte viva de uma planta (broto, ramo etc.) no tronco ou num ramo de outra, para que ela se desenvolva nesta, formando nova planta.}{en.xer.to}{0}
\verb{enxerto}{chê}{}{}{}{}{Implantar as células da pele ou parte da pele de um local do corpo de uma pessoa para outro local nessa mesma pessoa, ou do corpo de uma pessoa para outra.}{en.xer.to}{0}
\verb{enxó}{ch}{}{}{}{s.m.}{Ferramenta usada em carpintaria para desbastar a madeira.}{en.xó}{0}
\verb{enxofrar}{ch}{Quím.}{}{}{v.t.}{Misturar ou preparar um elemento químico ou composto com enxofre; sulfurar.}{en.xo.frar}{\verboinum{1}}
\verb{enxofre}{chô}{Quím.}{}{}{s.m.}{Elemento químico do grupo dos não metais, sólido, de cor amarela, usado como matéria"-prima para fungicida, pólvora, vulcanização de borracha etc. \elemento{16}{32.066}{S}.}{en.xo.fre}{0}
\verb{enxotar}{ch}{}{}{}{v.t.}{Expulsar, fazer sair pessoa ou animal com brutalidade; escorraçar.}{en.xo.tar}{\verboinum{1}}
\verb{enxoval}{ch}{}{"-ais}{}{s.m.}{Conjunto de roupas e acessórios que são usados por quem se casa, por um recém"-nascido etc.}{en.xo.val}{0}
\verb{enxovalhar}{ch}{}{}{}{v.t.}{Fazer com que uma coisa ou um lugar fique muito sujo; emporcalhar, sujar, enlamear.}{en.xo.va.lhar}{0}
\verb{enxovalhar}{ch}{Fig.}{}{}{}{Manchar o nome, a reputação, a honra de alguém; injuriar, insultar, macular.}{en.xo.va.lhar}{\verboinum{1}}
\verb{enxovia}{ch}{}{}{}{s.f.}{Cela térrea ou subterrânea, geralmente escura, úmida e suja; masmorra, calabouço.}{en.xo.vi.a}{0}
\verb{enxugadouro}{ch}{}{}{}{s.m.}{Local onde se estendem roupas para secar ou se colocam objetos para enxugar.}{en.xu.ga.dou.ro}{0}
\verb{enxugar}{ch}{}{}{}{v.t.}{Tirar a umidade; secar.}{en.xu.gar}{0}
\verb{enxugar}{ch}{}{}{}{}{Cortar, reduzir o que está em excesso.}{en.xu.gar}{0}
\verb{enxugar}{ch}{}{}{}{}{Fazer parar; interromper.}{en.xu.gar}{\verboinum{5}}
\verb{enxúndia}{ch}{}{}{}{s.f.}{Gordura de porco ou de ave; banha, unto.}{en.xún.dia}{0}
\verb{enxundioso}{ch\ldots{}ô}{}{"-osos ⟨ó⟩}{"-osa ⟨ó⟩}{adj.}{Que tem enxúndia, gordura; gorduroso, untuoso, oleoso. }{en.xun.di.o.so}{0}
\verb{enxurrada}{ch}{}{}{}{s.f.}{Grande massa de água, resultante de chuva, que corre com grande força, arrastando tudo com ela; aguaça, enxurro. }{en.xur.ra.da}{0}
\verb{enxurrada}{ch}{Por ext.}{}{}{}{Grande quantidade de coisas, abundância; série; chorrilho.}{en.xur.ra.da}{0}
\verb{enxurro}{ch}{}{}{}{s.m.}{Enxurrada.}{en.xur.ro}{0}
\verb{enxuto}{ch}{}{}{}{adj.}{Que se enxugou, que não está mais molhado ou úmido; seco.}{en.xu.to}{0}
\verb{enxuto}{ch}{Fig.}{}{}{}{Diz"-se do que teve eliminado ou reduzido o que era excessivo ou supérfluo. (\textit{Após a revisão, o texto ficou enxuto e preciso.})}{en.xu.to}{0}
\verb{enxuto}{ch}{Fig.}{}{}{}{Que tem o corpo bem proporcionado, que não é magro nem gordo; esbelto.}{en.xu.to}{0}
\verb{enzima}{}{Bioquím.}{}{}{s.f.}{Nome comum às proteínas especializadas na catálise de reações metabólicas específicas.}{en.zi.ma}{0}
\verb{eoceno}{}{Geol.}{}{}{s.m.}{Período da Era Cenozoica, posterior ao Paleoceno e anterior ao Oligoceno. (Nesta acepção usa"-se maiúscula.)}{e.o.ce.no}{0}
\verb{eoceno}{}{Geol.}{}{}{adj.}{Que se refere a esse período.}{e.o.ce.no}{0}
\verb{eólico}{}{}{}{}{adj.}{Relativo a vento.}{e.ó.li.co}{0}
\verb{éolo}{}{Poét.}{}{}{s.m.}{Vento forte.}{é.o.lo}{0}
\verb{epa}{ê}{}{}{}{interj.}{Exprime admiração, surpresa, animação, alegria etc.}{e.pa}{0}
\verb{epêntese}{}{Gram.}{}{}{s.f.}{Acréscimo de um fonema no interior de uma palavra.}{e.pên.te.se}{0}
\verb{epicarpo}{}{Bot.}{}{}{s.m.}{Camada ou película mais externa dos frutos; casca.}{e.pi.car.po}{0}
\verb{epiceno}{}{Gram.}{}{}{adj.}{Diz"-se de substantivo que tem apenas um gênero, que é só feminino ou só masculino.}{e.pi.ce.no}{0}
\verb{epicentro}{}{Geol.}{}{}{s.m.}{Ponto da superfície terrestre onde primeiramente chega a onda sísmica e onde o tremor de terra decorrente é mais intenso.}{e.pi.cen.tro}{0}
\verb{épico}{}{Liter.}{}{}{adj.}{Diz"-se de poema extenso em que se narram e se celebram feitos heroicos relacionados a guerras.}{é.pi.co}{0}
\verb{epicurismo}{}{Filos.}{}{}{s.m.}{Doutrina materialista do filósofo Epicuro (341--270 a.C.) e de seus seguidores, caracterizada pela busca da felicidade através do domínio ou da renúncia dos prazeres sensuais.}{e.pi.cu.ris.mo}{0}
\verb{epicurismo}{}{Por ext.}{}{}{}{Luxúria, sensualidade, libertinagem.}{e.pi.cu.ris.mo}{0}
\verb{epicurista}{}{}{}{}{adj.2g.}{Relativo a ou seguidor do epicurismo.}{e.pi.cu.ris.ta}{0}
\verb{epicurista}{}{Fig.}{}{}{}{Que procura o prazer em todas as suas formas.}{e.pi.cu.ris.ta}{0}
\verb{epicurista}{}{}{}{}{s.2g.}{Indivíduo que segue o epicurismo.}{e.pi.cu.ris.ta}{0}
\verb{epidemia}{}{Med.}{}{}{s.f.}{Doença contagiosa que ataca simultaneamente grande número de pessoas numa determinada região.}{e.pi.de.mi.a}{0}
\verb{epidêmico}{}{}{}{}{adj.}{Relativo a epidemia.}{e.pi.dê.mi.co}{0}
\verb{epidemiologia}{}{Med.}{}{}{s.f.}{Parte da medicina que estuda as  epidemias.}{e.pi.de.mi.o.lo.gi.a}{0}
\verb{epiderme}{é}{Anat.}{}{}{s.f.}{A camada mais externa da pele.}{e.pi.der.me}{0}
\verb{epidídimo}{}{Anat.}{}{}{s.m.}{Canal alongado e sinuoso, por onde passam as células espermáticas produzidas no testículo.}{e.pi.dí.di.mo}{0}
\verb{epifania}{}{Relig.}{}{}{s.f.}{Festa religiosa que comemora o aparecimento de Jesus aos pagãos, especialmente aos três Reis Magos.}{e.pi.fa.ni.a}{0}
\verb{epifania}{}{Por ext.}{}{}{}{Percepção, ou compreensão da natureza ou do significado de uma coisa.}{e.pi.fa.ni.a}{0}
\verb{epigástrico}{}{}{}{}{adj.}{Relativo ou pertencente ao epigástrio.}{e.pi.gás.tri.co}{0}
\verb{epiglote}{ó}{Anat.}{}{}{s.f.}{Válvula cuja função é tapar a glote durante a deglutição, impedindo a entrada de alimentos na laringe.}{e.pi.glo.te}{0}
\verb{epígono}{}{}{}{}{s.m.}{O que pertence à geração seguinte.}{e.pí.go.no}{0}
\verb{epígono}{}{}{}{}{}{Discípulo ou continuador de um artista ou de uma escola (doutrina) nas letras, artes, ciências etc.}{e.pí.go.no}{0}
\verb{epigrafar}{}{}{}{}{v.t.}{Pôr epígrafe num livro, num conto etc.}{e.pi.gra.far}{\verboinum{1}}
\verb{epígrafe}{}{}{}{}{s.f.}{Inscrição, frase, palavra ou título que serve de tema a um assunto, o resume ou revela a inspiração ou a motivação da obra à qual está aposta; mote. }{e.pí.gra.fe}{0}
\verb{epigrama}{}{}{}{}{s.m.}{Poema curto, satírico ou irônico.}{e.pi.gra.ma}{0}
\verb{epigrama}{}{}{}{}{}{Dito picante, malicioso ou mordaz; sátira.}{e.pi.gra.ma}{0}
\verb{epilepsia}{}{Med.}{}{}{s.f.}{Doença nervosa caracterizada por ataques súbitos nos quais o paciente perde a consciência e tem convulsões em intervalos irregulares de tempo.}{e.pi.lep.si.a}{0}
\verb{epiléptico}{}{}{}{}{adj.}{Epilético.}{e.pi.lép.ti.co}{0}
\verb{epilético}{}{}{}{}{adj.}{Relativo à epilepsia.}{e.pi.lé.ti.co}{0}
\verb{epilético}{}{}{}{}{s.m.}{Pessoa que sofre de epilepsia.}{e.pi.lé.ti.co}{0}
\verb{epílogo}{}{}{}{}{s.m.}{Final, desfecho de uma obra literária.}{e.pí.lo.go}{0}
\verb{epílogo}{}{Por ext.}{}{}{}{A última parte, o final ou o desfecho de um acontecimento, evento etc.}{e.pí.lo.go}{0}
\verb{episcopado}{}{}{}{}{s.m.}{Dignidade, cargo ou jurisdição de um bispo.}{e.pis.co.pa.do}{0}
\verb{episcopado}{}{}{}{}{}{Conjunto dos bispos.}{e.pis.co.pa.do}{0}
\verb{episcopal}{}{}{"-ais}{}{adj.2g.}{Relativo a bispo.}{e.pis.co.pal}{0}
\verb{episódico}{}{}{}{}{adj.}{Relativo a episódio.}{e.pi.só.di.co}{0}
\verb{episódico}{}{Por ext.}{}{}{}{Que acontece eventual ou inesperadamente; acidental, ocasional.}{e.pi.só.di.co}{0}
\verb{episódio}{}{}{}{}{s.m.}{Numa obra de ficção, incidente ligado à ação principal.}{e.pi.só.dio}{0}
\verb{episódio}{}{}{}{}{}{Fato, acontecimento, evento, caso.}{e.pi.só.dio}{0}
\verb{episódio}{}{}{}{}{}{Divisão de uma obra em série, de uma telenovela, de um filme etc.}{e.pi.só.dio}{0}
\verb{epistemologia}{}{Filos.}{}{}{s.f.}{Teoria do conhecimento; estudo crítico do conhecimento científico, de sua validade, de suas aplicações, métodos, limites e de suas relações com o sujeito.}{e.pis.te.mo.lo.gi.a}{0}
\verb{epístola}{}{Relig.}{}{}{s.f.}{Cada uma das cartas escritas pelos apóstolos de Jesus Cristo aos fiéis e incluídas no Novo Testamento.}{e.pís.to.la}{0}
\verb{epístola}{}{Por ext.}{}{}{}{Carta, correspondência, missiva.}{e.pís.to.la}{0}
\verb{epistolar}{}{}{}{}{adj.2g.}{Relativo a ou próprio de epístola, de carta. }{e.pis.to.lar}{0}
\verb{epistolar}{}{}{}{}{v.t.}{Assinalar, contar ou narrar um acontecimento, fato etc. em epístola.}{e.pis.to.lar}{\verboinum{1}}
\verb{epistolário}{}{}{}{}{s.m.}{Coleção, conjunto de epístolas ou cartas.}{e.pis.to.lá.rio}{0}
\verb{epitáfio}{}{}{}{}{s.m.}{Inscrição sobre túmulos ou monumentos funerários.}{e.pi.tá.fio}{0}
\verb{epitalâmio}{}{}{}{}{s.m.}{Poema ou canto composto para casamento.}{e.pi.ta.lâ.mio}{0}
\verb{epitelial}{}{}{"-ais}{}{adj.2g.}{Relativo ao epitélio.}{e.pi.te.li.al}{0}
\verb{epitelial}{}{Anat.}{"-ais}{}{}{Formado de epitélio.}{e.pi.te.li.al}{0}
\verb{epitélio}{}{Anat.}{}{}{s.m.}{Camada celular que reveste todas as estruturas externas e internas do corpo.}{e.pi.té.lio}{0}
\verb{epíteto}{}{}{}{}{s.m.}{Palavra ou expressão que qualifica alguém ou algo.}{e.pí.te.to}{0}
\verb{epíteto}{}{}{}{}{}{Alcunha, apelido, qualificativo.}{e.pí.te.to}{0}
\verb{epítome}{}{}{}{}{s.m.}{Resumo de uma obra extensa; sinopse, compêndio.}{e.pí.to.me}{0}
\verb{epizootia}{}{Veter.}{}{}{s.f.}{Doença que ataca muitos animais ao mesmo tempo e no mesmo lugar.}{e.pi.zo.o.ti.a}{0}
\verb{época}{}{}{}{}{s.f.}{Momento histórico que marca um novo período ou um novo desenvolvimento.}{é.po.ca}{0}
\verb{época}{}{}{}{}{}{Período, temporada, estação. (\textit{O mês de dezembro é uma época atribulada devido às festas e ao encerramento de inúmeras atividades.})}{é.po.ca}{0}
\verb{epopeia}{é}{Liter.}{}{}{s.f.}{Poema ou narrativa extensa que conta os feitos e aventuras de um herói histórico ou lendário. }{e.po.pei.a}{0}
\verb{epopeia}{é}{Fig.}{}{}{}{Série de grandes acontecimentos.}{e.po.pei.a}{0}
\verb{epopeico}{é}{}{}{}{adj.}{Relativo a epopeia.}{e.po.pei.co}{0}
\verb{epopeico}{é}{}{}{}{}{Grandioso, heroico.}{e.po.pei.co}{0}
\verb{epóxi}{cs}{}{}{}{s.m.}{Tipo de resina sintética usada como revestimento.}{e.pó.xi}{0}
\verb{epsilo}{}{}{}{}{s.m.}{Épsilon.}{ep.si.lo}{0}
\verb{epsílon}{}{}{}{}{s.m.}{Épsilon.}{ep.sí.lon}{0}
\verb{épsilon}{}{}{}{}{s.m.}{Quinta letra do alfabeto grego. }{ép.si.lon}{0}
\verb{equação}{}{Mat.}{"-ões}{}{s.f.}{Expressão algébrica de uma igualdade envolvendo grandezas conhecidas e desconhecidas.}{e.qua.ção}{0}
\verb{equação}{}{}{"-ões}{}{}{Relação, sob certas condições, entre pessoas e coisas.}{e.qua.ção}{0}
\verb{equacionar}{}{}{}{}{v.t.}{Dispor os dados de um problema em equação a fim de se encontrar uma solução.}{e.qua.ci.o.nar}{\verboinum{1}}
\verb{equador}{ô}{Geogr.}{}{}{s.m.}{Linha imaginária, perpendicular ao eixo, que divide a Terra nos hemisférios norte e sul; círculo máximo da Terra.}{e.qua.dor}{0}
\verb{equalização}{}{}{"-ões}{}{s.f.}{Ato ou efeito de equalizar; equilíbrio, uniformização.}{e.qua.li.za.ção}{0}
\verb{equalizar}{}{}{}{}{v.t.}{Tornar igual; uniformizar, igualar.}{e.qua.li.zar}{\verboinum{1}}
\verb{equânime}{}{}{}{}{adj.2g.}{Que revela equilíbrio de ânimo em qualquer situação; moderado, imparcial.}{e.quâ.ni.me}{0}
\verb{equanimidade}{}{}{}{}{s.f.}{Constância de ânimo, de temperamento; serenidade, retidão.}{e.qua.ni.mi.da.de}{0}
\verb{equatorial}{}{}{"-ais}{}{adj.2g.}{Relativo ao equador. (\textit{A Floresta Amazônica é uma região equatorial.})}{e.qua.to.ri.al}{0}
\verb{equatorial}{}{}{"-ais}{}{}{Situado no equador.}{e.qua.to.ri.al}{0}
\verb{equatoriano}{}{}{}{}{adj.}{Relativo ao Equador.}{e.qua.to.ri.a.no}{0}
\verb{equatoriano}{}{}{}{}{s.m.}{Indivíduo natural ou habitante desse país.}{e.qua.to.ri.a.no}{0}
\verb{equestre}{é}{}{}{}{adj.2g.}{Relativo a equitação ou a cavaleiro.}{e.ques.tre}{0}
\verb{equiângulo}{}{Geom.}{}{}{adj.}{Que possui os ângulos iguais.}{e.qui.ân.gu.lo}{0}
%\verb{}{}{}{}{}{}{}{}{0}
\verb{equidade}{}{}{}{}{s.f.}{Reconhecimento do direito de cada um; justiça, imparcialidade.}{e.qui.da.de}{0}
\verb{equídeo}{}{}{}{}{adj.}{Relativo a cavalo.}{e.quí.deo}{0}
\verb{equídeo}{}{Zool.}{}{}{s.m.}{Espécime da família de mamíferos à qual pertence o cavalo.}{e.quí.deo}{0}
\verb{equidistância}{}{}{}{}{s.f.}{Qualidade de equidistante; igualdade de distância.}{e.qui.dis.tân.cia}{0}
\verb{equidistante}{}{}{}{}{adj.2g.}{Localizado a igual distância.}{e.qui.dis.tan.te}{0}
\verb{equidistar}{}{}{}{}{v.t.}{Estar à igual distância de dois ou mais pontos.}{e.qui.dis.tar}{\verboinum{1}}
%\verb{}{}{}{}{}{}{}{}{0}
\verb{equilátero}{}{Geom.}{}{}{adj.}{Diz"-se da figura cujos lados são iguais.}{e.qui.lá.te.ro}{0}
\verb{equilibrado}{}{}{}{}{adj.}{Que foi posto em equilíbrio.}{e.qui.li.bra.do}{0}
\verb{equilibrado}{}{Fig.}{}{}{}{Que revela estabilidade emocional ou mental; ponderado, comedido.}{e.qui.li.bra.do}{0}
\verb{equilibrar}{}{}{}{}{v.t.}{Colocar em equilíbrio.}{e.qui.li.brar}{0}
\verb{equilibrar}{}{}{}{}{}{Contrabalançar, compensar, harmonizar. (\textit{Consegui finalmente equilibrar minhas despesas mensais.})}{e.qui.li.brar}{\verboinum{1}}
\verb{equilíbrio}{}{}{}{}{s.m.}{Estado de um corpo que se mantém em repouso sem oscilações ou desvios; estabilidade física.}{e.qui.lí.brio}{0}
\verb{equilíbrio}{}{}{}{}{}{Ponderação, comedimento, prudência.}{e.qui.lí.brio}{0}
\verb{equilibrista}{}{}{}{}{s.2g.}{Artista que exibe destreza e habilidade em exercícios e jogos que envolvem equilíbrio de corpo, de objetos etc.; malabarista, funâmbulo.}{e.qui.li.bris.ta}{0}
\verb{equimose}{ó}{Med.}{}{}{s.f.}{Mancha que se forma na pele ou nas mucosas, produzida por hemorragia resultante de contusões.}{e.qui.mo.se}{0}
\verb{equino}{}{}{}{}{adj.}{Relativo a cavalo; cavalar.}{e.qui.no}{0}
\verb{equinócio}{}{Astron.}{}{}{s.m.}{Tempo do ano em que o Sol corta o equador, fazendo com que o dia e a noite tenham igual duração, e que ocorre duas vezes por ano, na primavera e no outono.}{e.qui.nó.cio}{0}
\verb{equinodermo}{é}{Zool.}{}{}{s.m.}{Filo de animais marinhos, com tentáculos locomotores e a pele recoberta de espinhos, a que pertencem a estrela"-do"-mar e o ouriço"-do"-mar.}{e.qui.no.der.mo}{0}
\verb{equipagem}{}{}{"-ens}{}{s.f.}{Conjunto de pessoas que trabalham a bordo de navios, aviões, trens; tripulação.}{e.qui.pa.gem}{0}
\verb{equipamento}{}{}{}{}{s.m.}{Ato ou efeito de equipar.}{e.qui.pa.men.to}{0}
\verb{equipamento}{}{}{}{}{}{Conjunto de apetrechos ou instalações usadas para executar alguma atividade.}{e.qui.pa.men.to}{0}
\verb{equipar}{}{}{}{}{v.t.}{Fornecer o necessário para a execução de alguma atividade.}{e.qui.par}{\verboinum{1}}
\verb{equiparação}{}{}{"-ões}{}{s.f.}{Ato ou efeito de equiparar, igualar.}{e.qui.pa.ra.ção}{0}
\verb{equiparar}{}{}{}{}{v.t.}{Comparar pessoas ou coisas, considerando"-as idênticas; igualar.}{e.qui.pa.rar}{\verboinum{1}}
\verb{equipe}{}{}{}{}{s.f.}{Conjunto de pessoas que participam de uma atividade comum. (\textit{Quase toda a equipe de cientistas foi contaminada pelo vírus.})}{e.qui.pe}{0}
\verb{equitação}{}{}{"-ões}{}{s.f.}{Técnica ou esporte de andar a cavalo.}{e.qui.ta.ção}{0}
\verb{equitativo}{}{}{}{}{adj.}{Em que há equidade; justo, imparcial.}{e.qui.ta.ti.vo}{0}
%\verb{}{}{}{}{}{}{}{}{0}
%\verb{}{}{}{}{}{}{}{}{0}
\verb{equivalência}{}{}{}{}{s.f.}{Qualidade de equivalente; igualdade de valor.}{e.qui.va.lên.cia}{0}
%\verb{}{}{}{}{}{}{}{}{0}
\verb{equivalente}{}{}{}{}{adj.2g.}{Que tem igual valor, peso, força etc.}{e.qui.va.len.te}{0}
\verb{equivaler}{ê}{}{}{}{v.t.}{Ter o mesmo valor, peso, força etc.}{e.qui.va.ler}{\verboinum{51}}
\verb{equivocado}{}{}{}{}{adj.}{Que cometeu um engano; confuso.}{e.qui.vo.ca.do}{0}
\verb{equivocar}{}{}{}{}{v.t.}{Confundir uma coisa pela outra.}{e.qui.vo.car}{0}
\verb{equivocar}{}{}{}{}{v.pron.}{Cometer erro; enganar"-se.}{e.qui.vo.car}{\verboinum{2}}
\verb{equívoco}{}{}{}{}{adj.}{Que pode ter mais de um sentido; ambíguo, duvidoso.}{e.quí.vo.co}{0}
\verb{equívoco}{}{}{}{}{s.m.}{Engano, confusão.}{e.quí.vo.co}{0}
\verb{Er}{}{Quím.}{}{}{}{Símb. do \textit{érbio}. }{Er}{0}
\verb{era}{é}{}{}{}{s.f.}{Período de tempo que se inicia com um fato histórico notável e que estabelece uma nova ordem dos acontecimentos.}{e.ra}{0}
\verb{era}{é}{Geol.}{}{}{}{Cada uma das grandes divisões dos tempos geológicos da Terra.}{e.ra}{0}
\verb{erário}{}{}{}{}{s.m.}{Conjunto dos recursos financeiros e dos bens públicos; tesouro.}{e.rá.rio}{0}
\verb{érbio}{}{Quím.}{}{}{s.m.}{Elemento químico metálico, brilhante, prateado, sólido, mole, maleável, da família dos lantanídeos (terras"-raras); usado em ligas, reatores nucleares e \textit{lasers}. \elemento{68}{167.26}{Er}.}{ér.bio}{0}
\verb{ereção}{}{}{"-ões}{}{s.f.}{Ato ou efeito de erigir, levantar.}{e.re.ção}{0}
\verb{ereção}{}{Biol.}{"-ões}{}{}{Processo de levantamento e endurecimento do pênis devido a um aumento do fluxo sanguíneo na região.}{e.re.ção}{0}
\verb{eréctil}{}{}{}{}{}{Var. de \textit{erétil}.}{e.réc.til}{0}
\verb{erecto}{}{}{}{}{}{Var. de \textit{ereto}.}{e.rec.to}{0}
\verb{eremita}{}{Relig.}{}{}{s.2g.}{Indivíduo religioso que escolhe se afastar do convívio em sociedade e viver isolado em lugar deserto; ermitão.}{e.re.mi.ta}{0}
\verb{eremitério}{}{}{}{}{s.m.}{Local onde vivem eremitas.}{e.re.mi.té.rio}{0}
\verb{éreo}{}{}{}{}{adj.}{Feito ou revestido de bronze; brônzeo.}{é.re.o}{0}
\verb{erétil}{}{}{"-eis}{}{adj.2g.}{Que é capaz de se erguer e se manter em estado de ereção.}{e.ré.til}{0}
\verb{ereto}{é}{}{}{}{adj.}{Que se ergueu; levantado, aprumado, erigido.}{e.re.to}{0}
\verb{ergástulo}{}{}{}{}{s.m.}{Local onde se confinam os prisioneiros; cárcere, calabouço, masmorra.}{er.gás.tu.lo}{0}
\verb{ergometria}{}{}{}{}{s.f.}{Sistema de medição do trabalho muscular.}{er.go.me.tri.a}{0}
\verb{ergômetro}{}{}{}{}{s.m.}{Aparelho que mede o trabalho muscular.}{er.gô.me.tro}{0}
\verb{ergonomia}{}{}{}{}{s.f.}{Estudo das relações entre homem e máquina, visando a uma organização racional e a uma eficiência ideal nessa interação.}{er.go.no.mi.a}{0}
\verb{erguer}{ê}{}{}{}{v.t.}{Pôr em lugar alto; elevar, levantar.}{er.guer}{0}
\verb{erguer}{ê}{}{}{}{}{Construir, erigir, edificar.}{er.guer}{0}
\verb{erguer}{ê}{}{}{}{}{Levantar o ânimo; alentar.}{er.guer}{\verboinum{53}}
\verb{eriçar}{}{}{}{}{v.t.}{Fazer cabelo ou pelo erguer"-se; arrepiar, ouriçar.}{e.ri.çar}{\verboinum{3}}
\verb{erigir}{}{}{}{}{v.t.}{Construir uma obra arquitetônica, uma estátua; erguer, levantar. (\textit{O artista erigiu um monumento em homenagem ao poeta.})}{e.ri.gir}{\verboinum{22}}
\verb{erisipela}{é}{Med.}{}{}{s.f.}{Inflamação da pele caracterizada pelo surgimento de vesículas de pus e acompanhada de febre alta.}{e.ri.si.pe.la}{0}
\verb{eritema}{}{Med.}{}{}{s.m.}{Afecção da pele caracterizada por vermelhidão.}{e.ri.te.ma}{0}
\verb{ermida}{}{}{}{}{s.f.}{Pequena igreja situada em local solitário.}{er.mi.da}{0}
\verb{ermitão}{}{}{"-ões, -ães ou -ãos}{"-ã ou -oa}{s.m.}{Religioso que cuida de uma ermida; eremita.}{er.mi.tão}{0}
\verb{ermitão}{}{}{"-ões, -ães ou -ãos}{"-ã ou -oa}{}{Indivíduo que vive isolado da sociedade; anacoreta.}{er.mi.tão}{0}
\verb{ermitério}{}{}{}{}{s.m.}{Eremitério.}{er.mi.té.rio}{0}
\verb{ermo}{ê}{}{}{}{s.m.}{Lugar solitário, despovoado, abandonado.}{er.mo}{0}
\verb{erodir}{}{}{}{}{v.t.}{Provocar erosão; desgatar, corroer.}{e.ro.dir}{\verboinum{18}}
\verb{erosão}{}{}{"-ões}{}{s.f.}{Desgaste da superfície da Terra causado por agentes da natureza tais como água corrente, vento, gelo etc.}{e.ro.são}{0}
\verb{erosivo}{}{}{}{}{adj.}{Que produz erosão; corrosivo.}{e.ro.si.vo}{0}
\verb{erótico}{}{}{}{}{adj.}{Relativo ao amor.}{e.ró.ti.co}{0}
\verb{erótico}{}{}{}{}{}{Sensual, voluptuoso, lascivo.}{e.ró.ti.co}{0}
\verb{erotismo}{}{}{}{}{s.m.}{Estado de paixão, de excitação sexual; sensualidade.}{e.ro.tis.mo}{0}
\verb{erotizar}{}{}{}{}{v.t.}{Provocar erotismo, excitação em.}{e.ro.ti.zar}{\verboinum{1}}
\verb{erradicar}{}{}{}{}{v.t.}{Arrancar pela raiz; desarraigar.}{er.ra.di.car}{0}
\verb{erradicar}{}{}{}{}{}{Extirpar, eliminar. (\textit{O governo conseguiu erradicar a poliomelite do país.})}{er.ra.di.car}{\verboinum{2}}
\verb{erradio}{}{}{}{}{adj.}{Que vagueia, anda sem destino; errante.}{er.ra.di.o}{0}
\verb{errado}{}{}{}{}{adj.}{Que contém erro; incorreto, inexato.}{er.ra.do}{0}
\verb{errante}{}{}{}{}{adj.2g.}{Que não tem destino certo; erradio, vagabundo, nômade.}{er.ran.te}{0}
\verb{errar}{}{}{}{}{v.t.}{Cometer erro; enganar"-se.}{er.rar}{0}
\verb{errar}{}{}{}{}{v.i.}{Cair em culpa; falhar.}{er.rar}{0}
\verb{errar}{}{}{}{}{}{Andar sem rumo; vaguear.}{er.rar}{\verboinum{1}}
\verb{errata}{}{}{}{}{s.f.}{Lista dos erros e sua devida correção que acompanham um livro após a impressão.}{er.ra.ta}{0}
\verb{erre}{é}{}{}{}{s.m.}{Nome da letra \textit{r}.}{er.re}{0}
\verb{erriçar}{}{}{}{}{v.t.}{Eriçar.}{er.ri.çar}{\verboinum{3}}
\verb{erro}{ê}{}{}{}{s.m.}{Ato ou efeito de errar; engano, equívoco.}{er.ro}{0}
\verb{erro}{ê}{}{}{}{}{Incorreção, imperfeição, inexatidão.}{er.ro}{0}
\verb{erro}{ê}{Fís.}{}{}{}{Variação possível no resultado de uma medição devido ao limite de precisão dos instrumentos ou provocada pelo método utilizado (erro sistemático).}{er.ro}{0}
\verb{errôneo}{}{}{}{}{adj.}{Que contém erro; inexato, falso.}{er.rô.neo}{0}
\verb{eructação}{}{}{"-ões}{}{s.f.}{Emissão ruidosa dos gases do estômago pela boca; arroto.}{e.ruc.ta.ção}{0}
\verb{eructar}{}{}{}{}{v.i.}{Arrotar.}{e.ruc.tar}{\verboinum{1}}
\verb{erudição}{}{}{"-ões}{}{s.f.}{Qualidade de erudito; instrução, conhecimento.}{e.ru.di.ção}{0}
\verb{eruditismo}{}{}{}{}{s.m.}{Ostentação de erudição.}{e.ru.di.tis.mo}{0}
\verb{erudito}{}{}{}{}{adj.}{Que tem conhecimentos variados e aprofundados, especialmente em áreas da cultura e da ciência consideradas elevadas e de prestígio.}{e.ru.di.to}{0}
\verb{erupção}{}{Med.}{"-ões}{}{s.f.}{Alteração patológica da pele caracterizada por saliência ou vermelhidão.}{e.rup.ção}{0}
\verb{erupção}{}{}{"-ões}{}{}{Ato ou efeito de irromper; saída com ímpeto.}{e.rup.ção}{0}
\verb{eruptivo}{}{}{}{}{adj.}{Relativo a erupção.}{e.rup.ti.vo}{0}
\verb{erva}{é}{Bot.}{}{}{s.f.}{Planta não lenhosa e geralmente de pequeno porte, já que as partes aéreas vivem menos de um ano.}{er.va}{0}
\verb{erva"-cidreira}{é}{Bot.}{ervas"-cidreiras ⟨é⟩}{}{s.f.}{Erva aromática com flores brancas ou rosadas e propriedades medicinais; melissa.}{er.va"-ci.drei.ra}{0}
\verb{erva"-de"-passarinho}{é}{Bot.}{ervas"-de"-passarinho ⟨é⟩}{}{s.f.}{Erva parasita encontrada geralmente em árvores, muito procurada pelos passarinhos e disseminada através de seus excrementos.}{er.va"-de"-pas.sa.ri.nho}{0}
\verb{ervado}{}{}{}{}{adj.}{Cheio de erva.}{er.va.do}{0}
\verb{ervado}{}{}{}{}{}{Impregnado com suco de erva venenosa.}{er.va.do}{0}
\verb{erva"-doce}{é\ldots{}ô}{Bot.}{ervas"-doces ⟨é\ldots{}ô⟩}{}{s.f.}{Erva aromática com flores brancas e propriedades medicinais; anis.}{er.va"-do.ce}{0}
\verb{erval}{}{}{"-ais}{}{s.m.}{Plantação com predomínio de erva"-mate.}{er.val}{0}
\verb{erva"-mate}{é}{Bot.}{ervas"-mates ⟨é⟩}{}{s.f.}{Erva com propriedades medicinais estimulantes e diuréticas, usada no preparo do chimarrão.}{er.va"-ma.te}{0}
\verb{ervanário}{}{}{}{}{s.m.}{Indivíduo que coleciona, conhece ou vende plantas medicinais; herbolário, herborista.}{er.va.ná.rio}{0}
\verb{ervateiro}{ê}{}{}{}{adj.}{Relativo ao cultivo, preparação ou exportação da erva"-mate.}{er.va.tei.ro}{0}
\verb{ervateiro}{ê}{}{}{}{s.m.}{Indivíduo que negocia com erva"-mate ou se dedica a sua colheita e preparação.}{er.va.tei.ro}{0}
\verb{ervilha}{}{Bot.}{}{}{s.f.}{Nome de várias plantas da família das leguminosas.}{er.vi.lha}{0}
\verb{ervilha}{}{}{}{}{}{Vagem e semente dessa planta.}{er.vi.lha}{0}
\verb{ervoso}{ô}{}{"-osos ⟨ô⟩}{"-osa ⟨ó⟩}{adj.}{Cheio de erva; herboso.}{er.vo.so}{0}
\verb{Es}{}{Quím.}{}{}{}{Símb. do \textit{einstênio}.}{Es}{0}
\verb{ES}{}{}{}{}{}{Sigla do estado do Espírito Santo.}{ES}{0}
\verb{esbaforido}{}{}{}{}{adj.}{Que tem a respiração irregular devido a cansaço ou pressa; ofegante.}{es.ba.fo.ri.do}{0}
\verb{esbaforir}{}{}{}{}{v.pron.}{Ficar ofegante.}{es.ba.fo.rir"-se}{\verboinum{35}\verboirregular{\emph{def.} esbaforimos, esbaforis}}
\verb{esbagaçar}{}{}{}{}{v.t.}{Fazer em bagaços; despedaçar, arrebentar.}{es.ba.ga.çar}{\verboinum{3}}
\verb{esbagoar}{}{}{}{}{v.t.}{Tirar os bagos de.}{es.ba.go.ar}{\verboinum{7}}
\verb{esbandalhar}{}{}{}{}{v.t.}{Fazer em trapos; esfarrapar, despedaçar.}{es.ban.da.lhar}{\verboinum{1}}
\verb{esbanjador}{ô}{}{}{}{adj.}{Que esbanja; gastador, perdulário.}{es.ban.ja.dor}{0}
\verb{esbanjamento}{}{}{}{}{s.m.}{Ato ou efeito de esbanjar.}{es.ban.ja.men.to}{0}
\verb{esbanjar}{}{}{}{}{v.t.}{Gastar em excesso.}{es.ban.jar}{\verboinum{1}}
\verb{esbarrada}{}{}{}{}{s.f.}{Ato ou efeito de esbarrar; esbarrão.}{es.bar.ra.da}{0}
\verb{esbarrancado}{}{}{}{}{adj.}{Cheio de barrancos.}{es.bar.ran.ca.do}{0}
\verb{esbarrão}{}{}{"-ões}{}{s.m.}{Ato ou efeito de esbarrar; encontrão, tropeção.}{es.bar.rão}{0}
\verb{esbarrar}{}{}{}{}{v.t.}{Ir de encontro; topar.}{es.bar.rar}{0}
\verb{esbarrar}{}{}{}{}{}{Tropeçar.}{es.bar.rar}{\verboinum{1}}
\verb{esbarro}{}{}{}{}{s.m.}{Ato ou efeito de esbarrar; choque, esbarrão.}{es.bar.ro}{0}
\verb{esbater}{ê}{}{}{}{v.t.}{Diminuir os contrastes utilizando cores ou tons intermediários.}{es.ba.ter}{\verboinum{12}}
\verb{esbeiçar}{}{}{}{}{v.t.}{Arrancar os beiços; desbeiçar.}{es.bei.çar}{\verboinum{3}}
\verb{esbeltez}{ê}{}{}{}{s.f.}{Qualidade de esbelto; esbelteza.}{es.bel.tez}{0}
\verb{esbelteza}{ê}{}{}{}{s.f.}{Qualidade de esbelto.}{es.bel.te.za}{0}
\verb{esbelto}{é}{}{}{}{adj.}{Fino de corpo; magro.}{es.bel.to}{0}
\verb{esbelto}{é}{}{}{}{}{Elegante, gentil, distinto.}{es.bel.to}{0}
\verb{esbirro}{}{}{}{}{s.m.}{Nos tribunais, funcionário de nível subalterno.}{es.bir.ro}{0}
\verb{esboçar}{}{}{}{}{v.t.}{Traçar o esboço de.}{es.bo.çar}{0}
\verb{esboçar}{}{}{}{}{}{Deixar entrever.}{es.bo.çar}{\verboinum{3}}
\verb{esboço}{ô}{}{}{}{s.m.}{Linhas fundamentais de um desenho, gravura, pintura.}{es.bo.ço}{0}
\verb{esboço}{ô}{}{}{}{}{A obra quando em estado de esboço.}{es.bo.ço}{0}
\verb{esboço}{ô}{}{}{}{}{As primeiras noções sobre determinado assunto; rudimento, sumário.}{es.bo.ço}{0}
\verb{esbodegar}{}{Pop.}{}{}{v.t.}{Estragar, arruinar.}{es.bo.de.gar}{\verboinum{5}}
\verb{esbofar}{}{}{}{}{v.t.}{Causar grande cansaço; esfalfar.}{es.bo.far}{\verboinum{1}}
\verb{esbofetear}{}{}{}{}{v.t.}{Dar bofetadas; estapear.}{es.bo.fe.te.ar}{\verboinum{4}}
\verb{esbordoar}{}{}{}{}{v.t.}{Dar bordoadas em.}{es.bor.do.ar}{\verboinum{7}}
\verb{esbórnia}{}{}{}{}{s.f.}{Orgia, bacanal.}{es.bór.nia}{0}
\verb{esboroar}{}{}{}{}{v.t.}{Reduzir a pó; desfazer.}{es.bo.ro.ar}{\verboinum{7}}
\verb{esborrachar}{}{}{}{}{v.t.}{Rebentar apertando ou achatando.}{es.bor.ra.char}{0}
\verb{esborrachar}{}{}{}{}{}{Esmurrar, esbofetear.}{es.bor.ra.char}{\verboinum{1}}
\verb{esborrifar}{}{}{}{}{v.t.}{Molhar com borrifos; borrifar.}{es.bor.ri.far}{\verboinum{1}}
\verb{esbranquiçado}{}{}{}{}{adj.}{Cuja cor tende ao branco; alvacento, desbotado.}{es.bran.qui.ça.do}{0}
\verb{esbrasear}{}{}{}{}{v.t.}{Pôr em brasa.}{es.bra.se.ar}{0}
\verb{esbrasear}{}{Fig.}{}{}{}{Excitar, acalorar, inflamar.}{es.bra.se.ar}{\verboinum{4}}
\verb{esbravejar}{}{}{}{}{v.i.}{Tornar"-se bravo.}{es.bra.ve.jar}{0}
\verb{esbravejar}{}{}{}{}{v.t.}{Dar gritos; vociferar.}{es.bra.ve.jar}{\verboinum{1}}
\verb{esbregue}{é}{Pop.}{}{}{s.m.}{Confusão, conflito, rolo.}{es.bre.gue}{0}
\verb{esbregue}{é}{Pop.}{}{}{}{Bronca, repreensão, descompostura.}{es.bre.gue}{0}
\verb{esbrugar}{}{}{}{}{}{Var. de \textit{esburgar}.}{es.bru.gar}{0}
\verb{esbugalhado}{}{}{}{}{adj.}{Diz"-se de olho arregalado ou muito saliente.}{es.bu.ga.lha.do}{0}
\verb{esbugalhar}{}{}{}{}{v.t.}{Tirar os bugalhos de.}{es.bu.ga.lhar}{0}
\verb{esbugalhar}{}{}{}{}{}{Abrir bastante os olhos; arregalar.}{es.bu.ga.lhar}{\verboinum{1}}
\verb{esbulhar}{}{}{}{}{v.t.}{Privar da posse de; espoliar.}{es.bu.lhar}{0}
\verb{esbulhar}{}{}{}{}{}{Roubar, despojar.}{es.bu.lhar}{\verboinum{1}}
\verb{esbulho}{}{}{}{}{s.m.}{Ato ou efeito de esbulhar.}{es.bu.lho}{0}
\verb{esburacar}{}{}{}{}{v.t.}{Encher de buracos.}{es.bu.ra.car}{\verboinum{2}}
\verb{esburgar}{}{}{}{}{v.t.}{Tirar a casca de.}{es.bur.gar}{\verboinum{5}}
\verb{escabeche}{é}{Cul.}{}{}{s.m.}{Molho feito com vinagre, azeite, cebola e outros temperos refogados, usado geralmente com peixe frito.}{es.ca.be.che}{0}
\verb{escabelo}{ê}{}{}{}{s.m.}{Tipo de banco, comprido e com espaldar, e cujo assento é uma tampa que abre para dar acesso ao interior da caixa.}{es.ca.be.lo}{0}
\verb{escabelo}{ê}{}{}{}{}{Pequeno banco.}{es.ca.be.lo}{0}
\verb{escabichar}{}{}{}{}{v.t.}{Examinar minuciosa e pacientemente; investigar, escarafunchar.}{es.ca.bi.char}{\verboinum{1}}
\verb{escabiose}{ó}{Med.}{}{}{s.f.}{Doença contagiosa da pele produzida por um ácaro; sarna.}{es.ca.bi.o.se}{0}
\verb{escabioso}{ô}{}{"-osos ⟨ó⟩}{"-osa ⟨ó⟩}{adj.}{Relativo a escabiose ou sarna.}{es.ca.bi.o.so}{0}
\verb{escabioso}{ô}{}{"-osos ⟨ó⟩}{"-osa ⟨ó⟩}{}{Que tem erupções semelhantes às da sarna.}{es.ca.bi.o.so}{0}
\verb{escabreado}{}{}{}{}{adj.}{Irritado, zangado.}{es.ca.bre.a.do}{0}
\verb{escabreado}{}{}{}{}{}{Agitado, inquieto.}{es.ca.bre.a.do}{0}
\verb{escabrear}{}{}{}{}{v.t.}{Irritar, zangar, enfurecer.}{es.ca.bre.ar}{\verboinum{4}}
\verb{escabroso}{ô}{}{}{}{adj.}{Escarpado, pedregoso.}{es.ca.bro.so}{0}
\verb{escabroso}{ô}{}{}{}{}{Árduo, difícil.}{es.ca.bro.so}{0}
\verb{escabroso}{ô}{}{}{}{}{Indecoroso, imoral.}{es.ca.bro.so}{0}
\verb{escabujar}{}{}{}{}{v.i.}{Debater"-se sacudindo pés e mãos; espernear, esbracejar.}{es.ca.bu.jar}{\verboinum{1}}
\verb{escachar}{}{}{}{}{v.t.}{Abrir usando a força; rachar, fender.}{es.ca.char}{\verboinum{1}}
\verb{escada}{}{}{}{}{s.f.}{Série de degraus, de madeira, alvenaria ou outro material, que servem de passagem entre lugares em diferentes níveis ou alturas.}{es.ca.da}{0}
\verb{escada}{}{Fig.}{}{}{}{Meio que pode ser utilizado por alguém para conseguir o que deseja.}{es.ca.da}{0}
\verb{escadaria}{}{}{}{}{s.f.}{Série de escadas ou escada de grandes dimensões.}{es.ca.da.ri.a}{0}
\verb{escafandrista}{}{}{}{}{s.2g.}{Mergulhador que utiliza escafandro.}{es.ca.fan.dris.ta}{0}
\verb{escafandro}{}{}{}{}{s.m.}{Equipamento de mergulho especial para trabalhos demorados, constituído de roupa e máscara hermeticamente fechadas.}{es.ca.fan.dro}{0}
\verb{escafeder}{ê}{Pop.}{}{}{v.pron.}{Fugir às pressas; safar"-se, esgueirar"-se, sumir.  }{es.ca.fe.der"-se}{\verboinum{12}}
\verb{escala}{}{}{}{}{s.f.}{A medida graduada nos aparelhos e instrumentos de medição.}{es.ca.la}{0}
\verb{escala}{}{}{}{}{}{Cada uma das paradas intermediárias no percurso de avião ou navio.}{es.ca.la}{0}
\verb{escala}{}{}{}{}{}{Em mapas, cartas e plantas, a indicação da proporção entre a representação gráfica e a realidade.}{es.ca.la}{0}
\verb{escala}{}{}{}{}{}{Série de categorias, graus ou níveis em relação hierárquica.}{es.ca.la}{0}
\verb{escala}{}{Mús.}{}{}{}{Sequência de sons selecionados dentro de uma oitava.}{es.ca.la}{0}
\verb{escala}{}{}{}{}{}{Tabela que estabelece os serviços e as folgas dos funcionários de determinados postos de trabalho.}{es.ca.la}{0}
\verb{escalação}{}{}{"-ões}{}{s.f.}{Ato ou efeito de escalar.}{es.ca.la.ção}{0}
\verb{escalação}{}{Esport.}{"-ões}{}{}{A relação dos jogadores que compõem um time.}{es.ca.la.ção}{0}
\verb{escalada}{}{}{}{}{s.f.}{Ato ou efeito de escalar; subida.}{es.ca.la.da}{0}
\verb{escalada}{}{}{}{}{}{Aumento progressivo da intensidade de uma ofensiva militar ou de atos violentos.}{es.ca.la.da}{0}
\verb{escalada}{}{Esport.}{}{}{}{Modalidade esportiva que consiste em escalar paredes íngremes, naturais ou artificiais, utilizando"-se de técnicas como a tirolesa e o rapel.}{es.ca.la.da}{0}
\verb{escalafobético}{}{}{}{}{adj.}{Excêntrico, extravagante.}{es.ca.la.fo.bé.ti.co}{0}
\verb{escalão}{}{}{"-ões}{}{s.m.}{Cada um dos pontos de uma série progressiva.}{es.ca.lão}{0}
\verb{escalão}{}{}{"-ões}{}{}{Escala hierárquica.}{es.ca.lão}{0}
\verb{escalar}{}{}{}{}{v.t.}{Subir em lugar alto e difícil.}{es.ca.lar}{0}
\verb{escalar}{}{}{}{}{}{Indicar alguém para fazer um trabalho em dia e hora determinados.}{es.ca.lar}{\verboinum{1}}
\verb{escalavrar}{}{}{}{}{v.t.}{Arranhar, esfolar.}{es.ca.la.vrar}{\verboinum{1}}
\verb{escaldado}{}{}{}{}{adj.}{Em que se derramou água fervente.}{es.cal.da.do}{0}
\verb{escaldante}{}{}{}{}{adj.2g.}{Que escalda; muito quente.}{es.cal.dan.te}{0}
\verb{escalda"-pés}{}{}{}{}{s.m.}{Banho de imersão que se dá nos pés com água muito quente.}{es.cal.da"-pés}{0}
\verb{escaldar}{}{}{}{}{}{Causar calor.}{es.cal.dar}{0}
\verb{escaldar}{}{}{}{}{v.t.}{Queimar com líquido quente ou vapor d'água.}{es.cal.dar}{0}
\verb{escaldar}{}{}{}{}{}{Refogar.}{es.cal.dar}{0}
\verb{escaldar}{}{}{}{}{}{Punir, castigar.}{es.cal.dar}{\verboinum{1}}
\verb{escaleno}{ê}{}{}{}{adj.}{Diz"-se de triângulo em que todos os lados e ângulos têm medidas diferentes entre si.}{es.ca.le.no}{0}
\verb{escaler}{é}{}{}{}{s.m.}{Embarcação de pequeno porte, a remo ou a vela, usada em pequenos serviços.}{es.ca.ler}{0}
\verb{escalonar}{}{}{}{}{v.t.}{Dispor tropas em escalão.}{es.ca.lo.nar}{0}
\verb{escalonar}{}{}{}{}{}{Subir por etapas ou degraus; escalar.}{es.ca.lo.nar}{\verboinum{1}}
\verb{escalope}{ó}{Cul.}{}{}{s.m.}{Fatia fina e pequena de carne cortada no sentido transversal às fibras.}{es.ca.lo.pe}{0}
\verb{escalpar}{}{}{}{}{v.t.}{Arrancar a pele que reveste o crânio; escalpelar.}{es.cal.par}{\verboinum{1}}
\verb{escalpelar}{}{}{}{}{v.t.}{Abrir corte ou dissecar com escalpelo.}{es.cal.pe.lar}{0}
\verb{escalpelar}{}{Fig.}{}{}{}{Analisar minuciosamente.}{es.cal.pe.lar}{0}
\verb{escalpelar}{}{}{}{}{}{Escalpar.}{es.cal.pe.lar}{\verboinum{1}}
\verb{escalpelo}{ê}{}{}{}{s.m.}{Tipo de bisturi, usado em dissecações.}{es.cal.pe.lo}{0}
\verb{escalpo}{}{}{}{}{s.m.}{Couro cabeludo arrancado da vítima, com valor de troféu de guerra em alguns povos.}{es.cal.po}{0}
\verb{escalvado}{}{}{}{}{adj.}{Calvo, careca.}{es.cal.va.do}{0}
\verb{escalvado}{}{Fig.}{}{}{}{Sem vegetação; árido.}{es.cal.va.do}{0}
\verb{escalvar}{}{}{}{}{v.t.}{Tornar calvo.}{es.cal.var}{0}
\verb{escalvar}{}{}{}{}{}{Retirar a vegetação.}{es.cal.var}{\verboinum{1}}
\verb{escama}{}{Zool.}{}{}{s.f.}{Cada uma das pequenas lâminas que formam o revestimento do corpo de várias espécies de peixes e répteis.}{es.ca.ma}{0}
\verb{escamar}{}{}{}{}{v.t.}{Tirar ou perder as escamas; descamar.}{es.ca.mar}{0}
\verb{escamar}{}{Fig.}{}{}{v.pron.}{Zangar"-se, irritar"-se.}{es.ca.mar}{\verboinum{1}}
\verb{escambo}{}{}{}{}{s.m.}{Troca direta de mercadorias por mercadorias, sem usar qualquer tipo de moeda.}{es.cam.bo}{0}
\verb{escamoso}{ô}{}{"-osos ⟨ó⟩}{"-osa ⟨ó⟩}{adj.}{Que tem ou é coberto de escamas.}{es.ca.mo.so}{0}
\verb{escamoso}{ô}{Pop.}{"-osos ⟨ó⟩}{"-osa ⟨ó⟩}{}{Arrogante, intratável, enjoado.}{es.ca.mo.so}{0}
\verb{escamotear}{}{}{}{}{v.t.}{Esconder ou encobrir algo sutilmente.}{es.ca.mo.te.ar}{0}
\verb{escamotear}{}{}{}{}{}{Furtar com habilidade.}{es.ca.mo.te.ar}{0}
\verb{escamotear}{}{}{}{}{}{Fazer sumir sorrateiramente.}{es.ca.mo.te.ar}{\verboinum{4}}
\verb{escâncara}{}{}{}{}{s.f.}{Estado do que está à vista.}{es.cân.ca.ra}{0}
\verb{escancarado}{}{}{}{}{adj.}{Inteiramente a descoberto, que se vê ou percebe facilmente; patente, claro, evidente.}{es.can.ca.ra.do}{0}
\verb{escancarado}{}{}{}{}{}{Inteiramente aberto.}{es.can.ca.ra.do}{0}
\verb{escancarar}{}{}{}{}{v.t.}{Abrir totalmente.}{es.can.ca.rar}{0}
\verb{escancarar}{}{}{}{}{}{Tornar exposto.}{es.can.ca.rar}{\verboinum{1}}
\verb{escanchar}{}{}{}{}{v.t.}{Abrir ao meio.}{es.can.char}{0}
\verb{escanchar}{}{}{}{}{}{Sentar abrindo as pernas.}{es.can.char}{\verboinum{1}}
\verb{escandalizar}{}{}{}{}{v.t.}{Causar escândalo ou indignação.}{es.can.da.li.zar}{0}
\verb{escandalizar}{}{}{}{}{}{Causar ofensa; chocar.}{es.can.da.li.zar}{\verboinum{1}}
\verb{escândalo}{}{}{}{}{s.m.}{Ato ou fato condenável que causa indignação.}{es.cân.da.lo}{0}
\verb{escândalo}{}{}{}{}{}{Tumulto, desordem , escarcéu.}{es.cân.da.lo}{0}
\verb{escandaloso}{ô}{}{"-osos ⟨ó⟩}{"-osa ⟨ó⟩}{adj.}{Que provoca escândalo; vergonhoso, indecoroso.}{es.can.da.lo.so}{0}
\verb{escandinavo}{}{}{}{}{adj.}{Relativo à península da Escandinávia, no norte da Europa, compreendendo a Noruega, Dinamarca, Finlândia, Islândia e Suécia.}{es.can.di.na.vo}{0}
\verb{escandinavo}{}{}{}{}{s.m.}{Indivíduo natural ou habitante dessa península.}{es.can.di.na.vo}{0}
\verb{escândio}{}{Quím.}{}{}{s.m.}{Elemento químico metálico, branco"-prateado, do subgrupo do alumínio, sem grandes aplicações na indústria. \elemento{21}{44.95591}{Sc}.}{es.cân.dio}{0}
\verb{escandir}{}{}{}{}{v.t.}{Medir versos contando suas sílabas.}{es.can.dir}{0}
\verb{escandir}{}{}{}{}{}{Pronunciar palavras, frases destacando suas sílabas.}{es.can.dir}{\verboinum{18}}
\verb{escanear}{}{Informát.}{}{}{v.t.}{Examinar minuciosamente uma imagem ou texto para enviar ao computador sob forma de sinais digitais.}{es.ca.ne.ar}{\verboinum{4}}
\verb{escangalhar}{}{}{}{}{v.t.}{Fazer alguma coisa ficar sem  funcionar; estragar, arruinar.}{es.can.ga.lhar}{\verboinum{1}}
\verb{escanhoar}{}{}{}{}{v.t.}{Barbear com apuro, repassando a navalha.}{es.ca.nho.ar}{\verboinum{7}}
\verb{escanifrado}{}{}{}{}{adj.}{Que é muito magro; descarnado.}{es.ca.ni.fra.do}{0}
\verb{escanifrado}{}{}{}{}{}{Desengonçado, desajeitado.}{es.ca.ni.fra.do}{0}
\verb{escaninho}{}{}{}{}{s.m.}{Compartimento de pequeno tamanho, às vezes secreto, em caixas, cofres, gavetas, armários etc.}{es.ca.ni.nho}{0}
\verb{escaninho}{}{}{}{}{}{Lugar aculto, recôndito.}{es.ca.ni.nho}{0}
\verb{escansão}{}{}{"-ões}{}{s.f.}{Ato ou efeito de escandir, de decompor versos em elementos métricos.}{es.can.são}{0}
\verb{escansão}{}{Mús.}{"-ões}{}{}{Subida de tom.}{es.can.são}{0}
\verb{escanteio}{ê}{Esport.}{}{}{s.m.}{No futebol, falta em que um jogador da defesa lança a bola para fora do campo, pela linha de fundo.}{es.can.tei.o}{0}
\verb{escanteio}{ê}{}{}{}{}{Cobrança dessa falta, do canto do campo.}{es.can.tei.o}{0}
\verb{escanzelado}{}{}{}{}{adj.}{Magro como um cão faminto; escanifrado.}{es.can.ze.la.do}{0}
\verb{escapada}{}{}{}{}{s.f.}{Fuga feita às pressas e às ocultas; escapulida.}{es.ca.pa.da}{0}
\verb{escapadela}{é}{}{}{}{s.f.}{Escapada.}{es.ca.pa.de.la}{0}
\verb{escapamento}{}{}{}{}{s.m.}{Ato ou efeito de escapar; escape.}{es.ca.pa.men.to}{0}
\verb{escapamento}{}{}{}{}{}{Tubo pelo qual se escapa a fumaça do motor, em veículo.}{es.ca.pa.men.to}{0}
\verb{escapar}{}{}{}{}{v.t.}{Correr para longe de algum perigo; fugir.}{es.ca.par}{0}
\verb{escapar}{}{}{}{}{}{Sobreviver.}{es.ca.par}{0}
\verb{escapar}{}{}{}{}{}{Omitir, esquecer.}{es.ca.par}{0}
\verb{escapar}{}{}{}{}{}{Sair de onde está preso ou guardado.}{es.ca.par}{\verboinum{1}}
\verb{escapatória}{}{}{}{}{s.f.}{Meio de escapar; saída.}{es.ca.pa.tó.ria}{0}
\verb{escape}{}{}{}{}{s.m.}{Ação de livrar de perigo.}{es.ca.pe}{0}
\verb{escape}{}{}{}{}{}{Escapamento.}{es.ca.pe}{0}
\verb{escapismo}{}{}{}{}{s.m.}{Tendência para fugir à realidade.}{es.ca.pis.mo}{0}
\verb{escapo}{}{}{}{}{s.m.}{Dispositivo que regula os movimentos de um relógio.}{es.ca.po}{0}
\verb{escápula}{}{Anat.}{}{}{s.f.}{Cada um dos ossos que ficam no alto das costas e cobrem as costelas; omoplata.}{es.cá.pu.la}{0}
\verb{escapular}{}{}{}{}{adj.2g.}{Relativo à escápula.}{es.ca.pu.lar}{0}
\verb{escapulário}{}{}{}{}{s.m.}{Objeto de devoção formado por um cordão com dois pequenos quadrados de pano bento, com orações escritas ou uma relíquia, que os devotos trazem ao pescoço. }{es.ca.pu.lá.rio}{0}
\verb{escapulário}{}{}{}{}{}{Tira de pano que alguns religiosos trazem ao pescoço, pendente sobre o peito.}{es.ca.pu.lá.rio}{0}
\verb{escapulida}{}{}{}{}{s.f.}{Escapada.}{es.ca.pu.li.da}{0}
\verb{escapulir}{}{}{}{}{v.i.}{Sair de algum lugar sem ser notado; escafeder"-se, escapar, fugir.}{es.ca.pu.lir}{0}
\verb{escapulir}{}{}{}{}{}{Sair de onde está preso.}{es.ca.pu.lir}{\verboinum{33}}
\verb{escara}{}{Med.}{}{}{s.f.}{Crosta de ferida resultante da morte de tecidos.}{es.ca.ra}{0}
\verb{escara}{}{Bot.}{}{}{}{Marca deixada no caule pela queda de uma folha.}{es.ca.ra}{0}
\verb{escarafunchar}{}{}{}{}{v.t.}{Limpar com dedo, palito etc.; esgaravatar.}{es.ca.ra.fun.char}{0}
\verb{escarafunchar}{}{}{}{}{}{Procurar, investigar ou examinar com insistência, com paciência.}{es.ca.ra.fun.char}{0}
\verb{escarafunchar}{}{}{}{}{}{Remexer em algo, geralmente à procura de alguma coisa.}{es.ca.ra.fun.char}{\verboinum{1}}
\verb{escaramuça}{}{}{}{}{s.f.}{Combate de pouca importância.}{es.ca.ra.mu.ça}{0}
\verb{escaramuçar}{}{}{}{}{v.t.}{Combater, lutar, brigar.}{es.ca.ra.mu.çar}{\verboinum{3}}
\verb{escaravelho}{ê}{Zool.}{}{}{s.m.}{Inseto da família do besouro, que se alimenta de excremento de animais herbívoros.}{es.ca.ra.ve.lho}{0}
\verb{escarcéu}{}{}{}{}{s.m.}{Grande vaga ou onda formada por mar revolto; vagalhão.}{es.car.céu}{0}
\verb{escarcéu}{}{Fig.}{}{}{}{Ato de exagerar, ou levar demasiadamente a sério, coisa sem importância.}{es.car.céu}{0}
\verb{escarcéu}{}{Fig.}{}{}{}{Grande confusão; alvoroço, gritaria.}{es.car.céu}{0}
\verb{escarificação}{}{}{"-ões}{}{s.f.}{Ato ou efeito de escarificar.}{es.ca.ri.fi.ca.ção}{0}
\verb{escarificação}{}{Med.}{"-ões}{}{}{Série de leves incisões ou arranhaduras superficiais na pele feitas com lancetas}{es.ca.ri.fi.ca.ção}{0}
\verb{escarificar}{}{}{}{}{v.t.}{Fazer cortes leves em uma superfície.}{es.ca.ri.fi.car}{0}
\verb{escarificar}{}{}{}{}{}{Revolver superficialmente o solo.}{es.ca.ri.fi.car}{\verboinum{2}}
\verb{escarlate}{}{}{}{}{s.m.}{Cor vermelha viva e rutilante.}{es.car.la.te}{0}
\verb{escarlate}{}{}{}{}{adj.2g.}{Que tem essa cor.}{es.car.la.te}{0}
\verb{escarlatina}{}{Med.}{}{}{s.f.}{Doença infecciosa caracterizada por febre alta, manchas vermelhas na pele e descamação.}{es.car.la.ti.na}{0}
\verb{escarmentado}{}{}{}{}{adj.}{Que foi repreendido ou punido.}{es.car.men.ta.do}{0}
\verb{escarmentado}{}{}{}{}{}{Que aprendeu por experiência dolorosa ou dano recebido; experimentado, escaldado.}{es.car.men.ta.do}{0}
\verb{escarmentado}{}{}{}{}{}{Que perdeu a esperança; desiludido.}{es.car.men.ta.do}{0}
\verb{escarmentar}{}{}{}{}{v.t.}{Inflingir castigo ou punição.}{es.car.men.tar}{0}
\verb{escarmentar}{}{}{}{}{}{Repreender ou censurar de maneira enérgica.}{es.car.men.tar}{\verboinum{1}}
\verb{escarmento}{}{}{}{}{s.m.}{Ato ou efeito de escarmentar; castigo, punição.}{es.car.men.to}{0}
\verb{escarmento}{}{}{}{}{}{Repreensão ou censura rigorosa.}{es.car.men.to}{0}
\verb{escarnar}{}{}{}{}{v.t.}{Retirar a carne de um osso; descarnar, desossar.}{es.car.nar}{\verboinum{1}}
\verb{escarnecer}{ê}{}{}{}{v.t.}{Tratar ou considerar alguém ou algo com escárnio, com zombaria; troçar.}{es.car.ne.cer}{\verboinum{15}}
\verb{escarnecimento}{}{}{}{}{s.m.}{Escárnio.}{es.car.ne.ci.men.to}{0}
\verb{escarninho}{}{}{}{}{adj.}{Em que há escárnio.}{es.car.ni.nho}{0}
\verb{escarninho}{}{}{}{}{}{Zombeteiro, trocista, sarcástico.}{es.car.ni.nho}{0}
\verb{escárnio}{}{}{}{}{s.m.}{O que é feito ou dito com intenção de provocar riso ou hilariedade acerca de alguém ou algo; caçoada, zombaria.}{es.cár.nio}{0}
\verb{escárnio}{}{}{}{}{}{Menosprezo, desprezo, desdém.}{es.cár.nio}{0}
\verb{escarola}{ó}{}{}{}{s.f.}{Verdura de folhas compridas e amargas, semelhante à chicória.}{es.ca.ro.la}{0}
\verb{escarpa}{}{}{}{}{s.f.}{Terreno de declive muito forte.}{es.car.pa}{0}
\verb{escarpado}{}{}{}{}{adj.}{Que tem escarpa; íngrime.}{es.car.pa.do}{0}
\verb{escarradeira}{ê}{}{}{}{s.f.}{Recipiente onde se escarra ou cospe.}{es.car.ra.dei.ra}{0}
\verb{escarranchar}{}{}{}{}{v.t.}{Montar ou sentar abrindo bem as pernas como quem monta a cavalo.}{es.car.ran.char}{\verboinum{1}}
\verb{escarrapachar}{}{}{}{}{v.t.}{Abrir muito as pernas.}{es.car.ra.pa.char}{0}
\verb{escarrapachar}{}{}{}{}{}{Sentar"-se em posição relaxada, meio reclinado, abrindo muito as pernas; escarranchar.}{es.car.ra.pa.char}{0}
\verb{escarrapachar}{}{}{}{}{}{Cair de bruços, estatelar"-se.}{es.car.ra.pa.char}{\verboinum{1}}
\verb{escarrar}{}{}{}{}{v.t.}{Expelir escarro ou sangue pela boca.}{es.car.rar}{\verboinum{1}}
\verb{escarro}{}{}{}{}{s.m.}{Secreção proveniente da traqueia, brônquios ou pulmões.}{es.car.ro}{0}
\verb{escarvar}{}{}{}{}{v.t.}{Cavar superficialmente.}{es.car.var}{0}
\verb{escarvar}{}{}{}{}{}{Causar erosão, abalo; corroer.}{es.car.var}{\verboinum{1}}
\verb{escassear}{}{}{}{}{v.i.}{Tornar"-se escasso.}{es.cas.se.ar}{0}
\verb{escassear}{}{}{}{}{v.t.}{Dar com moderação.}{es.cas.se.ar}{\verboinum{4}}
\verb{escassez}{ê}{}{}{}{s.f.}{Pouca quantidade de alguma coisa; falta, privação, carência.}{es.cas.sez}{0}
\verb{escasso}{}{}{}{}{adj.}{Que existe em pouca quantidade; pouco, raro.}{es.cas.so}{0}
\verb{escasso}{}{}{}{}{}{Desprovido, carente, pobre.}{es.cas.so}{0}
\verb{escatologia}{}{}{}{}{s.f.}{Doutrina religiosa que trata do destino final do homem e do mundo.}{es.ca.to.lo.gi.a}{0}
\verb{escatologia}{}{}{}{}{}{Tratado sobre os excrementos.}{es.ca.to.lo.gi.a}{0}
\verb{escatológico}{}{}{}{}{adj.}{Relativo à escatologia.}{es.ca.to.ló.gi.co}{0}
\verb{escavação}{}{}{"-ões}{}{s.f.}{Trabalho de escavar.}{es.ca.va.ção}{0}
\verb{escavação}{}{}{"-ões}{}{}{Buraco, cova, furo.}{es.ca.va.ção}{0}
\verb{escavadeira}{ê}{}{}{}{s.f.}{Máquina própria para cavar e remover terra.}{es.ca.va.dei.ra}{0}
\verb{escavador}{ô}{}{}{}{adj.}{Que escava.}{es.ca.va.dor}{0}
\verb{escavador}{ô}{Fig.}{}{}{}{Investigador, pesquisador.}{es.ca.va.dor}{0}
\verb{escavar}{}{}{}{}{v.t.}{Tirar terra de algum lugar, deixando um espaço vazio.}{es.ca.var}{\verboinum{1}}
\verb{escaveirado}{}{}{}{}{adj.}{Que é muito magro, semelhante a uma caveira.}{es.ca.vei.ra.do}{0}
\verb{esclarecer}{ê}{}{}{}{v.t.}{Fazer algo ficar claro para alguém; explicar.}{es.cla.re.cer}{0}
\verb{esclarecer}{ê}{}{}{}{}{Fazer desaparecer os aspectos desconhecidos de alguma coisa; desvendar, elucidar.}{es.cla.re.cer}{0}
\verb{esclarecer}{ê}{}{}{}{}{Tornar mais instruído.}{es.cla.re.cer}{\verboinum{15}}
\verb{esclarecido}{}{}{}{}{adj.}{Que foi explicado; elucidado.}{es.cla.re.ci.do}{0}
\verb{esclarecido}{}{}{}{}{}{Diz"-se do indivíduo que é dotado de conhecimentos.}{es.cla.re.ci.do}{0}
\verb{esclarecimento}{}{}{}{}{s.m.}{Ato ou efeito de esclarecer; explicação, elucidação.}{es.cla.re.ci.men.to}{0}
\verb{esclarecimento}{}{}{}{}{}{Anotação ou comentário.}{es.cla.re.ci.men.to}{0}
\verb{esclarecimento}{}{}{}{}{}{Informação, dado.}{es.cla.re.ci.men.to}{0}
\verb{esclerosado}{}{}{}{}{adj.}{Que se esclerosou.}{es.cle.ro.sa.do}{0}
\verb{esclerosado}{}{}{}{}{}{Diz"-se de indivíduo atingido por esclerose no sistema nervoso central.}{es.cle.ro.sa.do}{0}
\verb{esclerosar}{}{Med.}{}{}{v.t.}{Produzir ou adquirir esclerose.}{es.cle.ro.sar}{\verboinum{1}}
\verb{esclerose}{ó}{Med.}{}{}{s.f.}{Endurecimento patológico de tecido do organismo.}{es.cle.ro.se}{0}
\verb{esclerótica}{}{Anat.}{}{}{s.f.}{Túnica externa branca e fibrosa do globo ocular, vulgarmente chamada de branco do olho.}{es.cle.ró.ti.ca}{0}
\verb{esclusa}{}{}{}{}{s.f.}{Represa, comporta, eclusa.}{es.clu.sa}{0}
\verb{escoadouro}{ô}{}{}{}{s.m.}{Cano ou vala que dá saída a líquidos, dejetos etc.}{es.co.a.dou.ro}{0}
\verb{escoamento}{}{}{}{}{s.m.}{Ato ou efeito de escoar.}{es.co.a.men.to}{0}
\verb{escoamento}{}{}{}{}{}{Plano inclinado por onde as águas escoam. }{es.co.a.men.to}{0}
\verb{escoamento}{}{}{}{}{}{Modo de fluir de uma corrente.}{es.co.a.men.to}{0}
\verb{escoar}{}{}{}{}{v.t.}{Fazer escorrer devagar um líquido.}{es.co.ar}{0}
\verb{escoar}{}{}{}{}{v.i.}{Ser posto em circulação.}{es.co.ar}{\verboinum{7}}
\verb{escocês}{}{}{}{}{adj.}{Relativo à Escócia.}{es.co.cês}{0}
\verb{escocês}{}{}{}{}{s.m.}{Indivíduo natural ou habitante desse país.}{es.co.cês}{0}
\verb{escoicear}{}{}{}{}{v.t.}{Bater violentamente em pessoa ou coisa com uma das patas traseiras; dar coices.}{es.coi.ce.ar}{0}
\verb{escoicear}{}{}{}{}{}{Tratar alguém com grosseria; insultar.}{es.coi.ce.ar}{\verboinum{4}}
\verb{escoimar}{}{}{}{}{v.t.}{Livrar de coima, pena ou censura.}{es.coi.mar}{0}
\verb{escoimar}{}{}{}{}{}{Limpar, livrar de impurezas}{es.coi.mar}{0}
\verb{escoimar}{}{}{}{}{v.pron.}{Furtar"-se; livrar"-se; escapar. }{es.coi.mar}{\verboinum{1}}
\verb{escol}{ó}{}{}{}{s.m.}{O que é considerado o melhor, o mais distinto.}{es.col}{0}
\verb{escola}{ó}{}{}{}{s.f.}{Estabelecimento de ensino.}{es.co.la}{0}
\verb{escola}{ó}{}{}{}{}{Conjunto de professores, alunos e funcionários de uma escola.}{es.co.la}{0}
\verb{escola}{ó}{}{}{}{}{Doutrina, teoria ou tendência.}{es.co.la}{0}
\verb{escola}{ó}{}{}{}{}{Conjunto dos que a seguem.}{es.co.la}{0}
\verb{escolado}{}{}{}{}{adj.}{Que conhece a vida, que tem experiência.}{es.co.la.do}{0}
\verb{escolado}{}{}{}{}{}{Que não se deixa enganar; esperto, sabido, vivo.}{es.co.la.do}{0}
\verb{escolar}{}{}{}{}{adj.2g.}{Relativo a escola.}{es.co.lar}{0}
\verb{escolar}{}{}{}{}{s.2g.}{Indivíduo que estuda em uma escola; aluno, estudante.}{es.co.lar}{0}
\verb{escolaridade}{}{}{}{}{s.f.}{Aprendizado ou atividade escolar.}{es.co.la.ri.da.de}{0}
\verb{escolaridade}{}{}{}{}{}{Rendimento escolar de um aluno ou de um grupo de alunos.}{es.co.la.ri.da.de}{0}
\verb{escolarizar}{}{}{}{}{v.t.}{Fazer passar por aprendizado em escola.}{es.co.la.ri.zar}{\verboinum{1}}
\verb{escolástica}{}{}{}{}{s.f.}{Pensamento cristão surgido nas escolas da Idade Média e caracterizado pela coordenação entre teologia e filosofia.}{es.co.lás.ti.ca}{0}
\verb{escolha}{ô}{}{}{}{s.f.}{Predileção, opção.}{es.co.lha}{0}
\verb{escolha}{ô}{}{}{}{}{Ato de eleger; eleição.}{es.co.lha}{0}
\verb{escolher}{ê}{}{}{}{v.t.}{Ficar com um ou mais entre muitos, de acordo com a própria vontade; decidir"-se.}{es.co.lher}{\verboinum{12}}
\verb{escolhido}{}{}{}{}{adj.}{Que se escolheu; selecionado, eleito.}{es.co.lhi.do}{0}
\verb{escolhido}{}{}{}{}{}{Preferido, predileto.}{es.co.lhi.do}{0}
\verb{escolho}{ô}{}{}{}{s.m.}{Recife ou baixio à flor da água; abrolho.}{es.co.lho}{0}
\verb{escolho}{ô}{}{}{}{}{Pequena ilha rochosa.}{es.co.lho}{0}
\verb{escoliose}{ó}{Med.}{}{}{s.f.}{Desvio lateral da coluna vertebral.}{es.co.li.o.se}{0}
\verb{escolta}{ó}{}{}{}{s.f.}{Conjunto de soldados ou de veículos que servem de defesa a pessoa ou coisa.}{es.col.ta}{0}
\verb{escoltar}{}{}{}{}{v.t.}{Acompanhar pessoa ou coisa para protegê"-la.}{es.col.tar}{\verboinum{1}}
\verb{escombros}{}{}{}{}{s.m.pl.}{Restos de coisas destruídas; destroços, ruínas.}{es.com.bros}{0}
\verb{esconde"-esconde}{}{}{}{}{s.m.}{Brincadeira de criança em que uma delas deve sair à procura das demais, que se esconderam.  }{es.con.de"-es.con.de}{0}
\verb{esconder}{ê}{}{}{}{v.t.}{Colocar pessoa ou coisa em lugar no qual possa ficar oculto; ocultar.}{es.con.der}{0}
\verb{esconder}{ê}{}{}{}{}{Deixar de revelar; manter em segredo.}{es.con.der}{\verboinum{12}}
\verb{esconderijo}{}{}{}{}{s.m.}{Lugar para se esconder.}{es.con.de.ri.jo}{0}
\verb{escondidas}{}{}{}{}{s.f.pl.}{Usado na locução adverbial \textit{às escondidas}: às ocultas, em segredo.}{es.con.di.das}{0}
\verb{esconjurar}{}{}{}{}{v.t.}{Lançar maldição, desejar malefícios a alguém; amaldiçoar.}{es.con.ju.rar}{0}
\verb{esconjurar}{}{}{}{}{}{Afastar o demônio ou os maus espíritos; exorcizar, conjurar.}{es.con.ju.rar}{\verboinum{1}}
\verb{esconjuro}{}{}{}{}{s.m.}{Ato ou efeito de esconjurar; praga, maldição, conjuro, imprecação.}{es.con.ju.ro}{0}
\verb{esconjuro}{}{}{}{}{}{Exorcismo.}{es.con.ju.ro}{0}
\verb{esconso}{}{}{}{}{adj.}{Escondido, oculto.}{es.con.so}{0}
\verb{escopeta}{ê}{}{}{}{s.f.}{Tipo de espingarda leve e de cano curto.}{es.co.pe.ta}{0}
\verb{escopo}{ô}{}{}{}{s.m.}{O ponto que se quer alcançar com o que se faz; finalidade, objetivo, intento, alvo.}{es.co.po}{0}
\verb{escopro}{ô}{}{}{}{s.m.}{Ferramenta para lavrar pedra, madeira etc.; cinzel.}{es.co.pro}{0}
\verb{escora}{ó}{}{}{}{s.f.}{Qualquer peça usada para escorar; espeque, esteio.}{es.co.ra}{0}
\verb{escora}{ó}{Fig.}{}{}{}{Proteção, amparo, arrimo.}{es.co.ra}{0}
\verb{escoramento}{}{}{}{}{s.m.}{Ato ou efeito de escorar, de colocar escoras para impedir o desabamento de uma construção.}{es.co.ra.men.to}{0}
\verb{escorar}{}{}{}{}{v.t.}{Pôr escoras; amparar, firmar, encostar.}{es.co.rar}{0}
\verb{escorar}{}{}{}{}{v.pron.}{Amparar"-se, firmar"-se, apoiar"-se.}{es.co.rar}{\verboinum{1}}
\verb{escorbuto}{}{Med.}{}{}{s.m.}{Doença causada pela falta de vitamina \textsc{c} no organismo, caracterizada por hemorragias, lesões nas gengivas, queda dos dentes e da resistência às infecções.}{es.cor.bu.to}{0}
\verb{escorchante}{}{Pop.}{}{}{adj.2g.}{Diz"-se de preço muito alto, abusivo; extorsivo.}{es.cor.chan.te}{0}
\verb{escorchar}{}{}{}{}{v.t.}{Tirar a casca; descascar, despelar.}{es.cor.char}{0}
\verb{escorchar}{}{}{}{}{}{Tirar a pele; esfolar, pelar.}{es.cor.char}{0}
\verb{escorchar}{}{}{}{}{}{Cobrar preço excessivo, abusivo; explorar, esfolar.}{es.cor.char}{\verboinum{1}}
\verb{escorço}{ô}{}{}{}{s.m.}{Desenho, pintura ou figura em miniatura.}{es.cor.ço}{0}
\verb{escorço}{ô}{}{}{}{}{Resumo, esboço, condensação.}{es.cor.ço}{0}
\verb{escore}{ó}{}{}{}{s.m.}{Resultado de um jogo ou disputa expresso em números; placar, contagem.}{es.co.re}{0}
\verb{escória}{}{}{}{}{s.f.}{Resíduo que se forma na fundição de um metal; borra.}{es.có.ria}{0}
\verb{escória}{}{Fig.}{}{}{}{Pessoa ou grupo de pessoas desprezíveis; plebe, ralé, gentalha, populacho.}{es.có.ria}{0}
\verb{escoriação}{}{}{"-ões}{}{s.f.}{Ato ou efeito de escoriar; esfoladura, ferimento, arranhadura. }{es.co.ri.a.ção}{0}
\verb{escoriar}{}{}{}{}{v.t.}{Ferir levemente, produzir escoriação; esfolar.}{es.co.ri.ar}{\verboinum{1}}
\verb{escorificar}{}{}{}{}{v.t.}{Limpar, separar as escórias dos metais; purificar.}{es.co.ri.fi.car}{\verboinum{2}}
\verb{escornear}{}{}{}{}{v.t.}{Ferir, atingir com chifres; chifrar, marrar.}{es.cor.ne.ar}{0}
\verb{escornear}{}{Fig.}{}{}{}{Tratar com violência ou desprezo; envilecer, escorraçar.}{es.cor.ne.ar}{\verboinum{4}}
\verb{escorpiano}{}{Astrol.}{}{}{s.m.}{Indivíduo que nasceu sob o signo de escorpião.}{es.cor.pi.a.no}{0}
\verb{escorpiano}{}{Astrol.}{}{}{adj.}{Relativo ou pertencente a esse signo.    }{es.cor.pi.a.no}{0}
\verb{escorpião}{}{Astron.}{}{}{s.m.}{Oitava constelação zodiacal.}{es.cor.pi.ão}{0}
\verb{escorpião}{}{Astrol.}{}{}{}{O signo do zodíaco referente a essa constelação.}{es.cor.pi.ão}{0}
\verb{escorpião}{}{Zool.}{"-ões}{}{s.m.}{Nome comum de alguns aracnídeos de hábitos noturnos, que vivem sob pedaços de tronco, e que são dotados de cauda terminada em ferrão, por meio do qual é inoculado um veneno.}{es.cor.pi.ão}{0}
\verb{escorraçar}{}{}{}{}{v.t.}{Expulsar, pôr para fora, enxotar.}{es.cor.ra.çar}{\verboinum{3}}
\verb{escorredor}{ô}{Bras.}{}{}{s.m.}{Nome comum a diversos utensílios de cozinha usados para escorrer a água de frutos, verduras, legumes, massas, talheres etc.}{es.cor.re.dor}{0}
\verb{escorrega}{é}{}{}{}{s.m.}{Escorregador.}{es.cor.re.ga}{0}
\verb{escorregadela}{é}{}{}{}{s.f.}{Ato ou efeito de escorregar de leve.}{es.cor.re.ga.de.la}{0}
\verb{escorregadela}{é}{Fig.}{}{}{}{Deslize, falta, descuido, lapso.}{es.cor.re.ga.de.la}{0}
\verb{escorregadiço}{}{}{}{}{adj.}{Diz"-se de piso, superfície etc. em que se escorrega com facilidade; resvaladio.}{es.cor.re.ga.di.ço}{0}
\verb{escorregadio}{}{}{}{}{adj.}{Escorregadiço.}{es.cor.re.ga.di.o}{0}
\verb{escorregador}{ô}{}{}{}{adj.}{Que escorrega, desliza.}{es.cor.re.ga.dor}{0}
\verb{escorregador}{ô}{}{}{}{s.m.}{Brinquedo infantil que consiste de uma superfície inclinada e lisa por onde as crianças deslizam sentadas ou deitadas; escorrega.}{es.cor.re.ga.dor}{0}
\verb{escorregão}{}{}{"-ões}{}{s.m.}{Ato ou efeito de escorregar; escorregadela.}{es.cor.re.gão}{0}
\verb{escorregar}{}{}{}{}{v.i.}{Deslizar sobre uma superfície e perder o equilíbrio, caindo ou não; resvalar.}{es.cor.re.gar}{0}
\verb{escorregar}{}{}{}{}{}{Cometer uma pequena falta, erro ou deslize.}{es.cor.re.gar}{\verboinum{5}}
\verb{escorreito}{ê}{}{}{}{adj.}{Que não apresenta falha, defeito ou lesão; são, íntegro, ileso.}{es.cor.rei.to}{0}
\verb{escorreito}{ê}{}{}{}{}{Diz"-se do estilo, linguagem, correto(a), apurado(a), castiço(a).}{es.cor.rei.to}{0}
\verb{escorrer}{ê}{}{}{}{v.t.}{Retirar o líquido de algo. (\textit{A cozinheira escorreu o macarrão.})}{es.cor.rer}{0}
\verb{escorrer}{ê}{}{}{}{v.i.}{Escoar, pingar. (\textit{O sorvete que a criança tomava escorria pela casquinha e pela sua mão.})}{es.cor.rer}{\verboinum{12}}
\verb{escorrido}{}{}{}{}{adj.}{Que escorreu, ou a que se retirou o líquido.}{es.cor.ri.do}{0}
\verb{escorrimento}{}{}{}{}{s.m.}{Ato ou efeito de escorrer.}{es.cor.ri.men.to}{0}
\verb{escorva}{ó}{}{}{}{s.f.}{Compartimento da arma em que se colocava a pólvora.}{es.cor.va}{0}
\verb{escorva}{ó}{}{}{}{}{Porção de pólvora do tubo de um foguete.}{es.cor.va}{0}
\verb{escoteirismo}{}{}{}{}{s.m.}{Escotismo.}{es.co.tei.ris.mo}{0}
\verb{escoteiro}{ê}{}{}{}{adj.}{Que anda ou viaja só, desacompanhado.}{es.co.tei.ro}{0}
\verb{escoteiro}{ê}{}{}{}{}{Que viaja sem bagagem.}{es.co.tei.ro}{0}
\verb{escoteiro}{ê}{}{}{}{s.m.}{Membro ou participante do escotismo.}{es.co.tei.ro}{0}
\verb{escotilha}{}{}{}{}{s.f.}{Abertura no convés de um navio para passagem de carga, de pessoas, arejamento e luz.}{es.co.ti.lha}{0}
\verb{escotismo}{}{}{}{}{s.m.}{Sistema educativo que tem por objetivo aperfeiçoar física e moralmente os jovens.}{es.co.tis.mo}{0}
\verb{escova}{ô}{}{}{}{s.f.}{Instrumento de pelo, cerdas ou arame com cabo de metal, plástico ou madeira, usado para limpar roupas, dentes, lustrar sapatos, pentear cabelos etc.}{es.co.va}{0}
\verb{escovação}{}{}{"-ões}{}{s.f.}{Ato ou efeito de escovar; escovagem.}{es.co.va.ção}{0}
\verb{escovadela}{é}{}{}{}{s.f.}{Ato de escovar levemente, ou de uma só vez, rapidamente.}{es.co.va.de.la}{0}
\verb{escovadela}{é}{Fig.}{}{}{}{Reprimenda, censura.}{es.co.va.de.la}{0}
\verb{escovadela}{é}{Fig.}{}{}{}{Castigo, punição.}{es.co.va.de.la}{0}
\verb{escovado}{}{}{}{}{adj.}{Que foi limpo com escova.}{es.co.va.do}{0}
\verb{escovado}{}{Pop.}{}{}{}{Diz"-se de pessoa esperta, ladina, matreira.}{es.co.va.do}{0}
\verb{escovão}{}{}{"-ões}{}{s.m.}{Escova grande para limpar ou encerar pisos.}{es.co.vão}{0}
\verb{escovar}{}{}{}{}{v.t.}{Limpar ou lustrar com escova.}{es.co.var}{\verboinum{1}}
\verb{escovinha}{}{}{}{}{s.f.}{Escova pequena.}{es.co.vi.nha}{0}
\verb{escovinha}{}{}{}{}{loc. adv.}{Palavra usada na expressão \textit{à escovinha}: cortado rente (cabelo).}{es.co.vi.nha}{0}
\verb{escravagismo}{}{}{}{}{s.m.}{Regime socioeconômico que se apoia na escravidão, ou a doutrina dos partidários desse regime.}{es.cra.va.gis.mo}{0}
\verb{escravatura}{}{}{}{}{s.f.}{Escravidão.}{es.cra.va.tu.ra}{0}
\verb{escravatura}{}{}{}{}{}{Tráfico de escravos.}{es.cra.va.tu.ra}{0}
\verb{escravidão}{}{}{"-ões}{}{s.f.}{Estado ou condição de escravo, de quem não tem liberdade; submissão, servidão, cativeiro, sujeição, escravatura.}{es.cra.vi.dão}{0}
\verb{escravismo}{}{}{}{}{s.m.}{Prática da escravidão; escravagismo.}{es.cra.vis.mo}{0}
\verb{escravista}{}{}{}{}{adj.2g.}{Relativo a ou próprio de escravo ou escravidão.}{es.cra.vis.ta}{0}
\verb{escravista}{}{}{}{}{s.2g.}{Pessoa partidária do escravagismo; escravocrata.}{es.cra.vis.ta}{0}
\verb{escravizar}{}{}{}{}{v.t.}{Tornar escravo. }{es.cra.vi.zar}{\verboinum{1}}
\verb{escravo}{}{}{}{}{adj.}{Que não tem liberdade, ou está sujeito a um senhor, como propriedade dele.}{es.cra.vo}{0}
\verb{escravo}{}{Por ext.}{}{}{}{Que está submetido a alguém ou a alguma coisa;}{es.cra.vo}{0}
\verb{escravo}{}{}{}{}{s.m.}{Pessoa que está sujeita a um senhor, como propriedade dele.}{es.cra.vo}{0}
\verb{escravo}{}{Fig.}{}{}{}{Criado, servo, serviçal.}{es.cra.vo}{0}
\verb{escravo}{}{Fig.}{}{}{}{Pessoa que trabalha excessivamente, ou vive exclusivamente para o trabalho.}{es.cra.vo}{0}
\verb{escravocracia}{}{}{}{}{s.f.}{Poder, domínio dos escravocratas.}{es.cra.vo.cra.ci.a}{0}
\verb{escravocrata}{}{}{}{}{adj.}{Em que há ou envolve escravidão; escravista, escravagista.}{es.cra.vo.cra.ta}{0}
\verb{escravocrata}{}{}{}{}{s.2g.}{Pessoa partidária da escravatura, ou que é dona de escravos; escravagista, escravista.}{es.cra.vo.cra.ta}{0}
\verb{escrete}{é}{Esport.}{}{}{s.m.}{Equipe composta pelos melhores atletas para representar, numa competição, um país, uma cidade etc.; seleção.}{es.cre.te}{0}
\verb{escrevente}{}{}{}{}{adj.2g.}{Que copia o que outra pessoa escreveu ou ditou; copista, escriturário.}{es.cre.ven.te}{0}
\verb{escrevente}{}{}{}{}{s.2g.}{Essa pessoa.}{es.cre.ven.te}{0}
\verb{escrever}{ê}{}{}{}{v.t.}{Representar por meio de letras.}{es.cre.ver}{0}
\verb{escrever}{ê}{}{}{}{}{Redigir, compor sobre uma folha.}{es.cre.ver}{\verboinum{12}}
\verb{escrevinhador}{ô}{Pejor.}{}{}{adj.}{Que escrevinha, que escreve mal.}{es.cre.vi.nha.dor}{0}
\verb{escrevinhador}{ô}{}{}{}{s.m.}{Mau escritor, que não domina a língua na qual se expressa ou que escreve sobre coisas banais, sem importância.}{es.cre.vi.nha.dor}{0}
\verb{escrevinhar}{}{}{}{}{v.i.}{Escrever mal ou de forma tosca as letras; rabiscar, garatujar.}{es.cre.vi.nhar}{0}
\verb{escrevinhar}{}{}{}{}{v.t.}{Escrever ou redigir mal; escrever sobre banalidades ou sobre coisas sem importância.}{es.cre.vi.nhar}{\verboinum{1}}
\verb{escriba}{}{}{}{}{s.m.}{Pessoa que era encarregada de copiar manuscritos ou escrever textos ditados; copista, escrivão.}{es.cri.ba}{0}
\verb{escrínio}{}{}{}{}{s.m.}{Pequena caixa ou cofre onde se guardam joias; porta"-joias.}{es.crí.nio}{0}
\verb{escrita}{}{}{}{}{s.f.}{Ato ou efeito de escrever, redigir; escritura.}{es.cri.ta}{0}
\verb{escrita}{}{}{}{}{}{Representação de palavras ou ideias por meio de signos gráficos.}{es.cri.ta}{0}
\verb{escrita}{}{}{}{}{}{O modo pessoal de escrever, de grafar; caligrafia.}{es.cri.ta}{0}
\verb{escrita}{}{}{}{}{}{Letras, escritos, caracteres.}{es.cri.ta}{0}
\verb{escrito}{}{}{}{}{adj.}{Que se escreveu, que foi registrado com sinais gráficos; redigido.}{es.cri.to}{0}
\verb{escrito}{}{}{}{}{s.m.}{Qualquer suporte no qual se escreveu, como um bilhete, documento, título, obra literária, científica etc.}{es.cri.to}{0}
\verb{escritor}{ô}{}{}{}{adj.}{Que escreve.}{es.cri.tor}{0}
\verb{escritor}{ô}{}{}{}{s.m.}{Autor de obras científicas, literárias, didáticas etc.}{es.cri.tor}{0}
\verb{escritório}{}{}{}{}{s.m.}{Cômodo de uma casa destinado à leitura e à escrita; gabinete.}{es.cri.tó.rio}{0}
\verb{escritório}{}{}{}{}{}{Sala onde as atividades profissionais são exercidas, na qual se realizam negócios ou onde os clientes são recebidos.}{es.cri.tó.rio}{0}
\verb{escritura}{}{Jur.}{}{}{s.f.}{Documento autêntico que comprova o valor de um contrato.}{es.cri.tu.ra}{0}
\verb{escrituração}{}{}{"-ões}{}{s.f.}{Ato ou efeito de escriturar, de registrar todos os fatos de uma organização para posterior verificação.}{es.cri.tu.ra.ção}{0}
\verb{escrituração}{}{}{"-ões}{}{}{A escrita de livros comerciais.}{es.cri.tu.ra.ção}{0}
\verb{escriturar}{}{}{}{}{v.t.}{Anotar de forma organizada e sistemática contas comerciais.}{es.cri.tu.rar}{0}
\verb{escriturar}{}{Jur.}{}{}{}{Lavrar um documento autêntico.}{es.cri.tu.rar}{\verboinum{1}}
\verb{escriturário}{}{}{}{}{s.m.}{Pessoa encarregada da escrituração; escrevente, copista.}{es.cri.tu.rá.rio}{0}
\verb{escrivaninha}{}{}{}{}{s.f.}{Mesa apropriada para escritórios, com apetrechos para escrever.}{es.cri.va.ni.nha}{0}
\verb{escrivão}{}{}{"-ães}{escrivã}{s.m.}{Funcionário público encarregado de escrever depoimentos, autos, atas e outros documentos de fé pública.}{es.cri.vão}{0}
\verb{escrivão}{}{}{"-ães}{escrivã}{}{Tabelião, notário.}{es.cri.vão}{0}
\verb{escrófula}{}{Med.}{}{}{s.f.}{Inchamento dos gânglios linfáticos, com formação de tumefação que pode ulcerar.}{es.cró.fu.la}{0}
\verb{escrofuloso}{ô}{}{"-osos ⟨ó⟩}{"-osa ⟨ó⟩}{adj.}{Relativo a escrófula.}{es.cro.fu.lo.so}{0}
\verb{escroque}{ó}{}{}{}{s.m.}{Indivíduo que se apodera das coisas alheias por meios fraudulentos.}{es.cro.que}{0}
\verb{escrotal}{}{}{"-ais}{}{adj.2g.}{Relativo a escroto.}{es.cro.tal}{0}
\verb{escroto}{ô}{}{}{}{s.m.}{A pele que envolve e protege as bolsas testiculares; saco.}{es.cro.to}{0}
\verb{escroto}{ô}{Chul.}{}{}{}{Pessoa ordinária, sem caráter, desprezível ou repugnante.}{es.cro.to}{0}
\verb{escroto}{ô}{}{}{}{adj.}{Sem caráter; mal"-educado; mal"-afamado; canalha; repugnante.}{es.cro.to}{0}
\verb{escrúpulo}{}{}{}{}{s.m.}{Hesitação, dúvida de consciência.}{es.crú.pu.lo}{0}
\verb{escrúpulo}{}{}{}{}{}{Cuidado que se tem ao se fazer algo para não cometer erros ou faltas; zelo, meticulosidade.}{es.crú.pu.lo}{0}
\verb{escrupuloso}{ô}{}{"-osos ⟨ó⟩}{"-osa ⟨ó⟩}{adj.}{Que tem escrúpulos; rigoroso, cuidadoso, exigente, meticuloso.}{es.cru.pu.lo.so}{0}
\verb{escrutar}{}{}{}{}{v.t.}{Investigar, sondar, perscrutar, pesquisar.}{es.cru.tar}{\verboinum{1}}
\verb{escrutinador}{ô}{}{}{}{adj.}{Que escrutina.}{es.cru.ti.na.dor}{0}
\verb{escrutinador}{ô}{}{}{}{s.m.}{Indivíduo encarregado de verificar os votos de uma eleição.}{es.cru.ti.na.dor}{0}
\verb{escrutinar}{}{}{}{}{v.i.}{Apurar, conferir os resultados de uma votação.}{es.cru.ti.nar}{\verboinum{1}}
\verb{escrutínio}{}{}{}{}{s.m.}{Ato ou efeito de escrutinar.}{es.cru.tí.nio}{0}
\verb{escrutínio}{}{}{}{}{}{Votação em urna.}{es.cru.tí.nio}{0}
\verb{escrutínio}{}{}{}{}{}{Exame cuidadoso, minucioso.}{es.cru.tí.nio}{0}
\verb{escudar}{}{}{}{}{v.t.}{Defender com escudo.}{es.cu.dar}{\verboinum{1}}
\verb{escudeiro}{ê}{Hist.}{}{}{s.m.}{Na Idade Média, pessoa que servia a um cavaleiro, carregando seu escudo e acompanhando"-o na guerra.}{es.cu.dei.ro}{0}
\verb{escudeiro}{ê}{}{}{}{}{Primeiro título de nobreza em alguns países.}{es.cu.dei.ro}{0}
\verb{escudela}{é}{}{}{}{s.f.}{Tigela rasa de madeira, geralmente arredondada.}{es.cu.de.la}{0}
\verb{escuderia}{}{Esport.}{}{}{s.f.}{Empresa proprietária de carros de corrida, fabricados especialmente para competições, e que possui equipe de pilotos e técnicos.}{es.cu.de.ri.a}{0}
\verb{escudo}{}{}{}{}{s.m.}{Arma defensiva usada para se proteger dos golpes de espada ou pedra.}{es.cu.do}{0}
\verb{escudo}{}{Fig.}{}{}{}{Amparo, proteção, defesa.}{es.cu.do}{0}
\verb{escudo}{}{}{}{}{}{Peça em que se representam os brasões de uma família nobre.}{es.cu.do}{0}
\verb{escudo}{}{}{}{}{}{Unidade monetária de Portugal.}{es.cu.do}{0}
\verb{esculachado}{}{Pop.}{}{}{adj.}{Que foi desmoralizado, avacalhado, esculhambado.}{es.cu.la.cha.do}{0}
\verb{esculachar}{}{Pop.}{}{}{v.t.}{Repreender alguém de forma grosseira e rude; desmoralizar alguém ou alguma coisa; avacalhar, esculhambar.}{es.cu.la.char}{\verboinum{1}}
\verb{esculacho}{}{Pop.}{}{}{s.m.}{Ato ou efeito de esculachar; esculhambação, avacalhação, desmoralização.}{es.cu.la.cho}{0}
\verb{esculápio}{}{Desus.}{}{}{s.m.}{Médico.}{es.cu.lá.pio}{0}
\verb{esculhambação}{}{Pop.}{"-ões}{}{s.f.}{Ato ou efeito de esculhambar; desmoralização, avacalhação, anarquia, confusão.}{es.cu.lham.ba.ção}{0}
\verb{esculhambar}{}{Pop.}{}{}{v.t.}{Esculachar; avacalhar.}{es.cu.lham.bar}{\verboinum{1}}
\verb{esculpir}{}{Art.}{}{}{v.t.}{Lavrar, entalhar figuras, formas, ornamentos na pedra, madeira, barro, cera etc.; modelar, gravar.  }{es.cul.pir}{0}
\verb{esculpir}{}{Art.}{}{}{v.i.}{Praticar ou exercer a arte da escultura.}{es.cul.pir}{\verboinum{34}}
\verb{escultor}{ô}{}{}{}{s.m.}{Artista que se dedica à escultura.}{es.cul.tor}{0}
\verb{escultura}{}{Art.}{}{}{s.f.}{A arte ou a técnica de esculpir.}{es.cul.tu.ra}{0}
\verb{escultura}{}{Art.}{}{}{}{Obra de arte esculpida, constituída por uma estrutura imóvel em três dimensões.}{es.cul.tu.ra}{0}
\verb{escultural}{}{}{"-ais}{}{adj.2g.}{Relativo a escultura.}{es.cul.tu.ral}{0}
\verb{escultural}{}{}{"-ais}{}{}{Cujas formas são perfeitas ou tão exemplares que podem servir de modelo a um escultor.}{es.cul.tu.ral}{0}
\verb{escuma}{}{}{}{}{s.f.}{Espuma.}{es.cu.ma}{0}
\verb{escumadeira}{ê}{}{}{}{s.f.}{Colher cheia de furos para tirar a escuma dos líquidos.}{es.cu.ma.dei.ra}{0}
\verb{escumar}{}{}{}{}{v.t. e v.i.}{Espumar.}{es.cu.mar}{\verboinum{1}}
\verb{escumilha}{}{}{}{}{s.f.}{Chumbo miúdo que se usa na caça a pássaros.}{es.cu.mi.lha}{0}
\verb{escumilha}{}{}{}{}{}{Tecido fino e transparente, de lã ou seda.}{es.cu.mi.lha}{0}
\verb{escuna}{}{}{}{}{s.f.}{Tipo de embarcação pequena, de dois mastros e velas latinas.}{es.cu.na}{0}
\verb{escuras}{}{}{}{}{s.f.pl.}{Palavra usada na locução \textit{às escuras}: 1. Sem iluminação.  2. Às escondidas; ocultamente.  3. Sem conhecimento do assunto em questão.}{es.cu.ras}{0}
\verb{escurecer}{ê}{}{}{}{v.t.}{Tornar escuro.}{es.cu.re.cer}{0}
\verb{escurecer}{ê}{Fig.}{}{}{}{Apagar a fama de; tornar obscuro.}{es.cu.re.cer}{0}
\verb{escurecer}{ê}{}{}{}{v.i.}{Tornar"-se escuro; anoitecer.}{es.cu.re.cer}{\verboinum{15}}
\verb{escuridão}{}{}{"-ões}{}{s.f.}{Ausência de luz.}{es.cu.ri.dão}{0}
\verb{escuridão}{}{Fig.}{"-ões}{}{}{Ausência de conhecimento; ignorância.}{es.cu.ri.dão}{0}
\verb{escuridão}{}{}{"-ões}{}{}{Tristeza profunda; melancolia, solidão.}{es.cu.ri.dão}{0}
\verb{escuro}{}{}{}{}{adj.}{Em que há pouca luz; sombrio.}{es.cu.ro}{0}
\verb{escuro}{}{Fig.}{}{}{}{Melancólico, triste.}{es.cu.ro}{0}
\verb{escuro}{}{Fig.}{}{}{}{Misterioso, suspeito.}{es.cu.ro}{0}
\verb{escuro}{}{}{}{}{}{Pouco inteligível; intrincado.}{es.cu.ro}{0}
\verb{escusa}{}{}{}{}{s.f.}{Ato ou efeito de escusar; desculpa, justificativa.}{es.cu.sa}{0}
\verb{escusado}{}{}{}{}{adj.}{Desnecessário, inútil.}{es.cu.sa.do}{0}
\verb{escusar}{}{}{}{}{v.t.}{Desculpar, perdoar.}{es.cu.sar}{0}
\verb{escusar}{}{}{}{}{}{Não precisar de; dispensar, isentar.}{es.cu.sar}{0}
\verb{escusar}{}{}{}{}{v.pron.}{Desculpar"-se, justificar"-se.}{es.cu.sar}{0}
\verb{escusar}{}{}{}{}{}{Recusar"-se, negar"-se.}{es.cu.sar}{\verboinum{1}}
\verb{escuso}{}{}{}{}{adj.}{Escondido, recôndito.}{es.cu.so}{0}
\verb{escuso}{}{}{}{}{}{Suspeito, ilícito.}{es.cu.so}{0}
\verb{escuta}{}{}{}{}{s.f.}{Ato de escutar.}{es.cu.ta}{0}
\verb{escuta}{}{}{}{}{}{Recepção de ondas eletromagnéticas provenientes de aparelhos de telecomunicação.}{es.cu.ta}{0}
\verb{escuta}{}{}{}{}{s.2g.}{Indivíduo incumbido de escutar. }{es.cu.ta}{0}
\verb{escutar}{}{}{}{}{v.t.}{Ouvir, especialmente de maneira atenciosa.}{es.cu.tar}{0}
\verb{escutar}{}{}{}{}{}{Aceitar os conselhos de; dar ouvidos a.}{es.cu.tar}{0}
\verb{escutar}{}{}{}{}{}{Espionar.}{es.cu.tar}{\verboinum{1}}
\verb{esdrúxulo}{ch}{}{}{}{adj.}{Esquisito, estranho, excêntrico.}{es.drú.xu.lo}{0}
\verb{esdrúxulo}{ch}{Gram.}{}{}{}{Diz"-se de palavra proparoxítona.}{es.drú.xu.lo}{0}
\verb{ESE}{}{}{}{}{}{Abrev. de \textit{és"-sudeste}.}{e.s.e.}{0}
\verb{esfacelar}{}{}{}{}{v.t.}{Desfazer, despedaçar, destruir.}{es.fa.ce.lar}{0}
\verb{esfacelar}{}{}{}{}{}{Causar gangrena a; gangrenar.}{es.fa.ce.lar}{0}
\verb{esfacelar}{}{}{}{}{v.pron.}{Corromper"-se, desfazer"-se.}{es.fa.ce.lar}{\verboinum{1}}
\verb{esfaimado}{}{}{}{}{adj.}{Faminto, esfomeado.}{es.fai.ma.do}{0}
\verb{esfaimar}{}{}{}{}{v.t.}{Privar de alimento; causar fome.}{es.fai.mar}{0}
\verb{esfaimar}{}{}{}{}{}{Matar de fome.}{es.fai.mar}{\verboinum{1}}
\verb{esfalfado}{}{}{}{}{adj.}{Cansado, esgotado, extenuado.}{es.fal.fa.do}{0}
\verb{esfalfamento}{}{}{}{}{s.m.}{Ato ou efeito de esfalfar; cansaço.}{es.fal.fa.men.to}{0}
\verb{esfalfar}{}{}{}{}{v.t.}{Diminuir por algum tempo a resistência de pessoa ou animal; cansar, exaurir, extenuar, fatigar.}{es.fal.far}{\verboinum{1}}
\verb{esfaquear}{}{}{}{}{v.t.}{Ferir ou matar com golpes de faca.}{es.fa.que.ar}{\verboinum{4}}
\verb{esfarelar}{}{}{}{}{v.t.}{Transformar em farelo ou em migalhas.}{es.fa.re.lar}{\verboinum{1}}
\verb{esfarrapado}{}{}{}{}{adj.}{Que tem as roupas em farrapos, rasgadas.}{es.far.ra.pa.do}{0}
\verb{esfarrapar}{}{}{}{}{v.t.}{Reduzir a farrapos; rasgar.}{es.far.ra.par}{\verboinum{1}}
\verb{esfenoide}{}{}{}{}{adj.2g.}{Em forma de cunha.}{es.fe.noi.de}{0}
\verb{esfenoide}{}{Anat.}{}{}{s.m.}{Osso ímpar, em forma de cunha, localizado na base do crânio.}{es.fe.noi.de}{0}
\verb{esfera}{é}{Geom.}{}{}{s.f.}{Sólido cuja superfície tem todos os pontos equidistantes de um mesmo ponto no interior, o centro.}{es.fe.ra}{0}
\verb{esfera}{é}{}{}{}{}{Bola, globo.}{es.fe.ra}{0}
\verb{esfera}{é}{}{}{}{}{Área em que se estende uma autoridade, um cargo, uma atribuição.}{es.fe.ra}{0}
\verb{esfera}{é}{}{}{}{}{Área, campo, setor.}{es.fe.ra}{0}
\verb{esférico}{}{}{}{}{adj.}{Que tem forma de esfera.}{es.fé.ri.co}{0}
\verb{esferográfica}{}{Bras.}{}{}{adj.}{Diz"-se de caneta em cuja ponta há uma pequena esfera de metal que controla a saída de tinta.}{es.fe.ro.grá.fi.ca}{0}
\verb{esferoide}{}{}{}{}{adj.2g.}{Diz"-se de objeto cuja forma se aproxima de uma esfera.}{es.fe.roi.de}{0}
\verb{esfiapar}{}{}{}{}{v.t.}{Transformar em fiapos; desfiar.}{es.fi.a.par}{\verboinum{1}}
\verb{esfiar}{}{}{}{}{v.t.}{Desmanchar em fios; desfiar.}{es.fi.ar}{\verboinum{1}}
\verb{esfíncter}{}{Anat.}{}{}{s.m.}{Tipo de músculo em forma de anel que abre ou fecha a passagem em certas cavidades do organismo.}{es.fínc.ter}{0}
\verb{esfinge}{}{}{}{}{s.f.}{Ser mitológico com corpo de leão alado e cabeça humana.}{es.fin.ge}{0}
\verb{esfinge}{}{}{}{}{}{A representação desse ser.}{es.fin.ge}{0}
\verb{esfinge}{}{Fig.}{}{}{}{Pessoa calada ou misteriosa.}{es.fin.ge}{0}
\verb{esfinge}{}{Zool.}{}{}{}{Designação comum de certas borboletas noturnas.}{es.fin.ge}{0}
\verb{esfirra}{}{Cul.}{}{}{s.f.}{Pastel de forno feito com massa de trigo e com recheios diversos.}{es.fir.ra}{0}
\verb{esfoguear}{}{}{}{}{v.t.}{Pôr fogo em, queimar.}{es.fo.gue.ar}{0}
\verb{esfoguear}{}{}{}{}{v.pron.}{Apressar"-se; atarantar"-se; perder a calma. }{es.fo.gue.ar}{\verboinum{4}}
\verb{esfola}{ó}{}{}{}{s.f.}{Ato ou efeito de esfolar; esfoladura.}{es.fo.la}{0}
\verb{esfoladela}{é}{}{}{}{s.f.}{Esfoladura.}{es.fo.la.de.la}{0}
\verb{esfoladela}{é}{}{}{}{}{Extorsão, logro.}{es.fo.la.de.la}{0}
\verb{esfoladura}{}{}{}{}{s.f.}{Ato ou efeito de esfolar.}{es.fo.la.du.ra}{0}
\verb{esfolar}{}{}{}{}{v.t.}{Ferir superficialmente; arranhar.}{es.fo.lar}{0}
\verb{esfolar}{}{}{}{}{}{Tirar a pele de.}{es.fo.lar}{\verboinum{1}}
\verb{esfolhar}{}{}{}{}{v.t.}{Tirar as folhas de.}{es.fo.lhar}{\verboinum{1}}
\verb{esfoliar}{}{}{}{}{v.t.}{Separar em lâminas ou folhas.}{es.fo.li.ar}{\verboinum{1}}
\verb{esfomeado}{}{}{}{}{adj.}{Que tem fome; faminto.}{es.fo.me.a.do}{0}
\verb{esfomear}{}{}{}{}{v.t.}{Provocar fome em.}{es.fo.me.ar}{0}
\verb{esfomear}{}{}{}{}{}{Privar de alimento; causar fome.}{es.fo.me.ar}{\verboinum{4}}
\verb{esforçado}{}{}{}{}{adj.}{Que se esforça em suas tarefas; diligente.}{es.for.ça.do}{0}
\verb{esforçado}{}{}{}{}{}{Forte, enérgico.}{es.for.ça.do}{0}
\verb{esforçar}{}{}{}{}{v.t.}{Dar forças a; reforçar.}{es.for.çar}{0}
\verb{esforçar}{}{}{}{}{v.pron.}{Empenhar"-se para conseguir algo.}{es.for.çar}{\verboinum{3}}
\verb{esforço}{ô}{}{}{}{s.m.}{Mobilização de forças para realizar uma tarefa ou atingir um objetivo.}{es.for.ço}{0}
\verb{esforço}{ô}{}{}{}{}{Energia, força.}{es.for.ço}{0}
\verb{esfrangalhar}{}{}{}{}{v.t.}{Transformar em frangalhos; esfarrapar, rasgar.}{es.fran.ga.lhar}{\verboinum{1}}
\verb{esfrega}{é}{}{}{}{s.f.}{Ato ou efeito de esfregar.}{es.fre.ga}{0}
\verb{esfrega}{é}{}{}{}{}{Repreensão física ou moral.}{es.fre.ga}{0}
\verb{esfregação}{}{}{"-ões}{}{s.f.}{Ato ou efeito de esfregar.}{es.fre.ga.ção}{0}
\verb{esfregador}{ô}{}{}{}{adj.}{Que esfrega, limpa superfícies.}{es.fre.ga.dor}{0}
\verb{esfregão}{}{}{"-ões}{}{s.m.}{Escova, vassoura ou pano próprio para esfregar.}{es.fre.gão}{0}
\verb{esfregar}{}{}{}{}{v.t.}{Friccionar continuamente para produzir calor, limpar, coçar.}{es.fre.gar}{\verboinum{5}}
\verb{esfriamento}{}{}{}{}{s.m.}{Ato ou efeito de esfriar.}{es.fri.a.men.to}{0}
\verb{esfriar}{}{}{}{}{v.t.}{Tornar frio.}{es.fri.ar}{0}
\verb{esfriar}{}{}{}{}{}{Tornar insensível, frio.}{es.fri.ar}{0}
\verb{esfriar}{}{}{}{}{}{Tornar fraco; afrouxar.}{es.fri.ar}{0}
\verb{esfumaçado}{}{}{}{}{adj.}{Cheio de fumaça.}{es.fu.ma.ça.do}{0}
\verb{esfumaçado}{}{}{}{}{}{Defumado.}{es.fu.ma.ça.do}{0}
\verb{esfumaçar}{}{}{}{}{v.t.}{Encher de fumaça.}{es.fu.ma.çar}{0}
\verb{esfumaçar}{}{Bras.}{}{}{}{Defumar.}{es.fu.ma.çar}{\verboinum{3}}
\verb{esfumar}{}{}{}{}{v.t.}{Desenhar com carvão.}{es.fu.mar}{0}
\verb{esfumar}{}{}{}{}{}{Realçar os traços de um desenho com esfuminho.}{es.fu.mar}{0}
\verb{esfumar}{}{}{}{}{v.pron.}{Desfazer"-se, desaparecer.}{es.fu.mar}{\verboinum{1}}
\verb{esfuminho}{}{}{}{}{s.m.}{Rolo de feltro ou pelica para realçar os traços de desenho a lápis ou carvão.}{es.fu.mi.nho}{0}
\verb{esfuziante}{}{}{}{}{adj.2g.}{Que esfuzia.}{es.fu.zi.an.te}{0}
\verb{esfuziante}{}{}{}{}{}{Muito alegre, comunicativo; radiante.}{es.fu.zi.an.te}{0}
\verb{esfuziar}{}{}{}{}{v.i.}{Zunir como balas de fuzilaria.}{es.fu.zi.ar}{\verboinum{1}}
\verb{esgadanhar}{}{}{}{}{v.t.}{Ferir com as unhas.}{es.ga.da.nhar}{\verboinum{1}}
\verb{esgadelhar}{}{}{}{}{v.t.}{Despentear, desgrenhar.}{es.ga.de.lhar}{\verboinum{1}}
\verb{esgalgado}{}{}{}{}{adj.}{Magro como um galgo.}{es.gal.ga.do}{0}
\verb{esgalhar}{}{}{}{}{v.t.}{Separar em galhos ou ramos.}{es.ga.lhar}{\verboinum{1}}
\verb{esganação}{}{}{"-ões}{}{s.f.}{Ato ou efeito de esganar.}{es.ga.na.ção}{0}
\verb{esganação}{}{}{"-ões}{}{}{Aperto, sufocação na garganta.}{es.ga.na.ção}{0}
\verb{esganado}{}{}{}{}{adj.}{Sufocado, estrangulado.}{es.ga.na.do}{0}
\verb{esganado}{}{}{}{}{}{Esfomeado, faminto.}{es.ga.na.do}{0}
\verb{esganar}{}{}{}{}{v.t.}{Estrangular, sufocar.}{es.ga.nar}{\verboinum{1}}
\verb{esganiçar}{}{}{}{}{v.t.}{Modular a voz tornando"-a estridente, imitando um cão.}{es.ga.ni.çar}{\verboinum{3}}
\verb{esgar}{}{}{}{}{s.m.}{Trejeito do rosto ou gesto de escárnio; careta.}{es.gar}{0}
\verb{esgaravatar}{}{}{}{}{v.t.}{Limpar com palito; palitar.}{es.ga.ra.va.tar}{\verboinum{1}}
\verb{esgarçar}{}{}{}{}{v.t.}{Abrir ou rasgar o tecido devido ao afastamento dos fios; desfiar.}{es.gar.çar}{\verboinum{3}}
\verb{esgazeado}{}{}{}{}{adj.}{Diz"-se de olhar inquieto, desnorteado, perturbado ou raivoso.}{es.ga.ze.a.do}{0}
\verb{esgazear}{}{}{}{}{v.t.}{Tornar a expressão dos olhos inquieta ou perturbada .}{es.ga.ze.ar}{\verboinum{4}}
\verb{esgoelar}{}{}{}{}{v.t.}{Estrangular; esganar.  	 }{es.go.e.lar}{0}
\verb{esgoelar}{}{}{}{}{v.i.}{Gritar muito. }{es.go.e.lar}{\verboinum{1}}
\verb{esgotado}{}{}{}{}{adj.}{Que se esgotou; exausto.}{es.go.ta.do}{0}
\verb{esgotamento}{}{}{}{}{s.m.}{Ato ou efeito de esgotar, exaurir.}{es.go.ta.men.to}{0}
\verb{esgotante}{}{}{}{}{adj.2g.}{Que esgota; cansativo.}{es.go.tan.te}{0}
\verb{esgotar}{}{}{}{}{v.t.}{Tirar todo o conteúdo.}{es.go.tar}{0}
\verb{esgotar}{}{}{}{}{}{Consumir, gastar, exaurir.}{es.go.tar}{0}
\verb{esgotar}{}{}{}{}{}{Tornar exausto ou muito cansado; extenuar.}{es.go.tar}{0}
\verb{esgotar}{}{}{}{}{}{Não ter mais o que dizer sobre um assunto. (\textit{Após uma intensa discussão, os diretores esgotaram o assunto.})}{es.go.tar}{\verboinum{1}}
\verb{esgoto}{ô}{}{}{}{s.m.}{Sistema subterrâneo de canalizações que recebe os detritos das residências de uma cidade e também água da chuva coletada nas ruas.}{es.go.to}{0}
\verb{esgoto}{ô}{}{}{}{}{Ato ou efeito de esgotar.}{es.go.to}{0}
\verb{esgravatar}{}{}{}{}{v.t.}{Esgaravatar, palitar.}{es.gra.va.tar}{\verboinum{1}}
\verb{esgrima}{}{Esport.}{}{}{s.f.}{Jogo com armas brancas como espada, sabre, florete etc.}{es.gri.ma}{0}
\verb{esgrimidor}{ô}{}{}{}{adj.}{Que esgrime.}{es.gri.mi.dor}{0}
\verb{esgrimidor}{ô}{}{}{}{s.m.}{Esgrimista.}{es.gri.mi.dor}{0}
\verb{esgrimir}{}{}{}{}{v.t.}{Manejar ou jogar armas brancas.}{es.gri.mir}{0}
\verb{esgrimir}{}{}{}{}{}{Agitar, brandir.}{es.gri.mir}{\verboinum{18}}
\verb{esgrimista}{}{}{}{}{s.2g.}{Indivíduo hábil no jogo ou manejo de armas brancas.}{es.gri.mis.ta}{0}
\verb{esgrouviado}{}{}{}{}{adj.}{Alto e magro.}{es.grou.vi.a.do}{0}
\verb{esgrouviado}{}{}{}{}{}{Que tem o cabelo desalinhado.}{es.grou.vi.a.do}{0}
\verb{esgrouvinhado}{}{}{}{}{adj.}{Esgrouviado.}{es.grou.vi.nha.do}{0}
\verb{esguedelhado}{}{}{}{}{adj.}{Diz"-se do cabelo não penteado, desgrenhado, desguedelhado.}{es.gue.de.lha.do}{0}
\verb{esguedelhar}{}{}{}{}{v.t.}{Emaranhar o cabelo; despentear, desguedelhar.}{es.gue.de.lhar}{\verboinum{1}}
\verb{esgueirar}{}{}{}{}{v.t.}{Desviar cautelosamente. (\textit{Esgueirou os olhos, pois ficou sem jeito.})}{es.guei.rar}{0}
\verb{esgueirar}{}{}{}{}{v.pron.}{Retirar"-se cautelosamente; escapulir, safar"-se.}{es.guei.rar}{\verboinum{1}}
\verb{esguelha}{ê}{}{}{}{s.f.}{Obliquidade, través, soslaio.}{es.gue.lha}{0}
\verb{esguelhar}{}{}{}{}{v.t.}{Colocar em posição oblíqua, de esguelha; atravessar.}{es.gue.lhar}{\verboinum{1}}
\verb{esguichar}{}{}{}{}{v.t.}{Fazer sair um líquido com força.}{es.gui.char}{\verboinum{1}}
\verb{esguicho}{}{}{}{}{s.m.}{Ato ou efeito de esguichar; jato de líquido.}{es.gui.cho}{0}
\verb{esguicho}{}{}{}{}{}{Peça que se coloca na extremidade da mangueira para esguichar água.}{es.gui.cho}{0}
\verb{esguio}{}{}{}{}{adj.}{Comprido e delgado.}{es.gui.o}{0}
\verb{eslavo}{}{}{}{}{adj.}{Relativo aos eslavos, povos indo"-europeus habitantes da Europa central e oriental, e antepassados dos russos, sérvios, poloneses, tchecos, ucranianos.}{es.la.vo}{0}
\verb{eslavo}{}{}{}{}{}{Diz"-se das línguas faladas por esses povos e do grupo linguístico formado por elas.}{es.la.vo}{0}
\verb{eslavo}{}{}{}{}{s.m.}{Indivíduo pertencente aos povos eslavos.}{es.la.vo}{0}
\verb{eslovaco}{}{}{}{}{adj.}{Relativo a Eslováquia.}{es.lo.va.co}{0}
\verb{eslovaco}{}{}{}{}{s.m.}{Indivíduo natural ou habitante desse país.}{es.lo.va.co}{0}
\verb{esloveno}{}{}{}{}{adj.}{Relativo a Eslovênia.}{es.lo.ve.no}{0}
\verb{esloveno}{}{}{}{}{s.m.}{Indivíduo natural ou habitante desse país.}{es.lo.ve.no}{0}
\verb{esmaecer}{ê}{}{}{}{v.i.}{Perder a cor; desbotar.}{es.ma.e.cer}{\verboinum{15}}
\verb{esmaecimento}{}{}{}{}{s.m.}{Ato ou efeito de esmaecer.}{es.ma.e.ci.men.to}{0}
\verb{esmagador}{ô}{}{}{}{adj.}{Que esmaga, oprime.}{es.ma.ga.dor}{0}
\verb{esmagador}{ô}{}{}{}{}{Que não admite réplica; irrefutável, indiscutível.}{es.ma.ga.dor}{0}
\verb{esmagar}{}{}{}{}{v.t.}{Comprimir até ser achatado.}{es.ma.gar}{0}
\verb{esmagar}{}{}{}{}{}{Vencer alguém por completo; abater, aniquilar.}{es.ma.gar}{\verboinum{5}}
\verb{esmaltador}{ô}{}{}{}{s.m.}{Artista que trabalha com esmalte.}{es.mal.ta.dor}{0}
\verb{esmaltagem}{}{}{"-ens}{}{s.f.}{Colocação de verniz.}{es.mal.ta.gem}{0}
\verb{esmaltar}{}{}{}{}{v.t.}{Cobrir um objeto com esmalte.}{es.mal.tar}{\verboinum{1}}
\verb{esmalte}{}{}{}{}{s.m.}{Líquido que se aplica sobre uma superfície e produz uma camada brilhante quando seca.}{es.mal.te}{0}
\verb{esmalte}{}{Anat.}{}{}{}{Revestimento da coroa dentária.}{es.mal.te}{0}
\verb{esmeralda}{}{Geol.}{}{}{s.f.}{Pedra preciosa de cor verde.}{es.me.ral.da}{0}
\verb{esmeraldino}{}{}{}{}{adj.}{Da cor da esmeralda.}{es.me.ral.di.no}{0}
\verb{esmerar}{}{}{}{}{v.t.}{Demonstrar esmero; apurar, aperfeiçoar.}{es.me.rar}{0}
\verb{esmerar}{}{}{}{}{v.pron.}{Aplicar"-se para relizar algo com qualidade, com perfeição; caprichar.}{es.me.rar}{\verboinum{1}}
\verb{esmeril}{}{}{}{}{s.m.}{Pedra usada para afiar lâminas.}{es.me.ril}{0}
\verb{esmeril}{}{}{}{}{}{Substância abrasiva usada para polir metais, pedras preciosas etc.}{es.me.ril}{0}
%\verb{}{}{}{}{}{}{}{}{0}
\verb{esmerilhar}{}{}{}{}{v.t.}{Polir com esmeril.}{es.me.ri.lhar}{\verboinum{1}}
\verb{esmero}{ê}{}{}{}{s.m.}{Cuidado especial com que se faz alguma coisa; zelo.}{es.me.ro}{0}
\verb{esmigalhar}{}{}{}{}{v.t.}{Transformar em migalhas.}{es.mi.ga.lhar}{0}
\verb{esmigalhar}{}{}{}{}{}{Reduzir a fragmentos; despedaçar.}{es.mi.ga.lhar}{\verboinum{1}}
\verb{esmiuçar}{}{}{}{}{v.t.}{Desfazer em pedaços miúdos.}{es.mi.u.çar}{0}
\verb{esmiuçar}{}{}{}{}{}{Analisar ou explicar detalhadamente.}{es.mi.u.çar}{\verboinum{3}}
\verb{esmo}{ê}{}{}{}{s.m.}{Cálculo aproximado; estimativa, conjectura.}{es.mo}{0}
\verb{esmoer}{ê}{}{}{}{v.t.}{Triturar com os dentes.}{es.mo.er}{0}
\verb{esmoer}{ê}{}{}{}{}{Ruminar, digerir.}{es.mo.er}{\verboinum{12}}
\verb{esmola}{ó}{}{}{}{s.f.}{O que se dá aos pobres por caridade.}{es.mo.la}{0}
\verb{esmolambado}{}{}{}{}{adj.}{Diz"-se daquele que tem a roupa em molambos; esfarrapado, maltrapilho.}{es.mo.lam.ba.do}{0}
\verb{esmolar}{}{}{}{}{v.t.}{Dar ou pedir esmola.}{es.mo.lar}{\verboinum{1}}
\verb{esmoleiro}{ê}{Relig.}{}{}{adj.}{Diz"-se de religioso que pedia esmolas para o convento.}{es.mo.lei.ro}{0}
\verb{esmoleiro}{ê}{}{}{}{s.m.}{Casa ou instituição em que se distribuem esmolas.}{es.mo.lei.ro}{0}
\verb{esmoleiro}{ê}{}{}{}{}{Aquele que pede esmolas; mendigo, pedinte.}{es.mo.lei.ro}{0}
\verb{esmoler}{é}{}{}{}{adj.2g.}{Diz"-se daquele que dá esmolas frequentemente; caridoso.}{es.mo.ler}{0}
\verb{esmoler}{é}{}{}{}{s.2g.}{Indivíduo que tem a incumbência de distribuir esmolas.}{es.mo.ler}{0}
\verb{esmoler}{é}{}{}{}{}{Indivíduo que vive de esmolas; pedinte; mendigo.}{es.mo.ler}{0}
\verb{esmorecer}{ê}{}{}{}{v.t.}{Fazer alguém perder o ânimo; desencorajar.}{es.mo.re.cer}{0}
\verb{esmorecer}{ê}{}{}{}{}{Diminuir, definhar.}{es.mo.re.cer}{\verboinum{15}}
\verb{esmorecido}{}{}{}{}{adj.}{Que ficou sem força ou sem ânimo; desalentado, desanimado.}{es.mo.re.ci.do}{0}
\verb{esmorecimento}{}{}{}{}{s.m.}{Qualidade daquele que desanima; abandono, desalento.}{es.mo.re.ci.men.to}{0}
\verb{esmorecimento}{}{}{}{}{}{Desfalecimento, desmaio.}{es.mo.re.ci.men.to}{0}
\verb{esmorecimento}{}{}{}{}{}{Falta de luz, de brilho.}{es.mo.re.ci.men.to}{0}
\verb{esmurrar}{}{}{}{}{v.t.}{Dar murros em pessoa ou coisa; bater, socar.}{es.mur.rar}{\verboinum{1}}
\verb{esnobar}{}{}{}{}{v.t.}{Tratar com orgulho e desprezo; fazer pouco"-caso.}{es.no.bar}{0}
\verb{esnobar}{}{}{}{}{}{Exibir.}{es.no.bar}{\verboinum{1}}
\verb{esnobe}{ó}{}{}{}{adj.2g.}{Que demonstra esnobismo, falsa superioridade; afetado, pernóstico.}{es.no.be}{0}
\verb{esnobismo}{}{}{}{}{s.m.}{Admiração excessiva pelo que está em voga.}{es.no.bis.mo}{0}
\verb{esnobismo}{}{}{}{}{}{Exagerado sentimento de superioridade.}{es.no.bis.mo}{0}
\verb{esnobismo}{}{}{}{}{}{Sofisticação exagerada.}{es.no.bis.mo}{0}
\verb{és"-nordeste}{é}{}{és"-nordestes ⟨é⟩}{}{s.m.}{Ponto do horizonte situado entre o este e o nordeste; lés"-nordeste. Abrev. \textsc{e.n.e}.}{és"-nor.des.te}{0}
\verb{esofagiano}{}{}{}{}{adj.}{Relativo ao esôfago.}{e.so.fa.gi.a.no}{0}
\verb{esofagite}{}{Med.}{}{}{s.f.}{Inflamação do esôfago.}{e.so.fa.gi.te}{0}
\verb{esôfago}{}{Anat.}{}{}{s.m.}{Canal que liga a faringe ao estômago.}{e.sô.fa.go}{0}
\verb{esotérico}{}{}{}{}{adj.}{Relativo ao esoterismo.}{e.so.té.ri.co}{0}
\verb{esotérico}{}{}{}{}{}{Que é ensinado de modo secreto, só para iniciados.}{e.so.té.ri.co}{0}
\verb{esoterismo}{}{}{}{}{s.m.}{Doutrina secreta cujos princípios são transmitidos só aos iniciados.}{e.so.te.ris.mo}{0}
\verb{espaçar}{}{}{}{}{v.t.}{Criar espaços ou intervalos.}{es.pa.çar}{0}
\verb{espaçar}{}{}{}{}{}{Impor adiamento ou dilatações; prolongar.}{es.pa.çar}{0}
\verb{espaçar}{}{}{}{}{}{Tornar menos frequente; criar interrupções.}{es.pa.çar}{\verboinum{3}}
\verb{espacejar}{}{}{}{}{v.t.}{Espaçar.}{es.pa.ce.jar}{\verboinum{1}}
\verb{espacial}{}{}{"-ais}{}{adj.2g.}{Que se estende pelo espaço.}{es.pa.ci.al}{0}
\verb{espaço}{}{}{}{}{s.m.}{Distância entre dois pontos.}{es.pa.ço}{0}
\verb{espaço}{}{}{}{}{}{Período ou intervalo de tempo.}{es.pa.ço}{0}
\verb{espaço}{}{}{}{}{}{O Universo.}{es.pa.ço}{0}
\verb{espaço}{}{}{}{}{}{Capacidade, acomodação.}{es.pa.ço}{0}
\verb{espaço}{}{}{}{}{}{Extensão indefinida.}{es.pa.ço}{0}
\verb{espaçonave}{}{}{}{}{s.f.}{Nave para viagens interplanetárias.}{es.pa.ço.na.ve}{0}
\verb{espaçoso}{ô}{}{"-osos ⟨ó⟩}{"-osa ⟨ó⟩}{adj.}{Que é amplo, extenso.}{es.pa.ço.so}{0}
\verb{espada}{}{}{}{}{s.f.}{Arma branca de lâmina de ferro ou de aço, comprida e pontiaguda, com um ou dois gumes. (\textit{Ele mantinha a espada pendendo ao longo do corpo.})}{es.pa.da}{0}
\verb{espada}{}{Fig.}{}{}{}{Força, poder militar; carreira das armas. (\textit{Se nos invadirem, serão recebidos pela espada.})}{es.pa.da}{0}
\verb{espada}{}{Fig.}{}{}{}{Luta, guerra. (\textit{O perigo era que esses meninos mal conheciam as leis da espada.})}{es.pa.da}{0}
\verb{espadachim}{}{}{"-ins}{}{adj.2g.}{Diz"-se de quem luta com espada.}{es.pa.da.chim}{0}
\verb{espadagão}{}{}{"-ões}{}{s.m.}{Grande espada.}{es.pa.da.gão}{0}
\verb{espadagão}{}{}{"-ões}{}{}{Espada velha e enferrujada; chanfalho.}{es.pa.da.gão}{0}
\verb{espadana}{}{}{}{}{s.f.}{Objeto em forma de espada.}{es.pa.da.na}{0}
\verb{espadana}{}{}{}{}{}{Jato de líquido em forma de lâmina de espada.}{es.pa.da.na}{0}
\verb{espadana}{}{}{}{}{}{Labareda.}{es.pa.da.na}{0}
\verb{espadana}{}{}{}{}{}{Barbatana de peixe.}{es.pa.da.na}{0}
\verb{espadanar}{}{}{}{}{v.t.}{Deixar cair em borbotões; soltar, lançar.}{es.pa.da.nar}{0}
\verb{espadanar}{}{}{}{}{v.i.}{Jorrar ou rebentar em espadanas; sair em borbotões; repuxar.}{es.pa.da.nar}{\verboinum{1}}
\verb{espadarte}{}{Zool.}{}{}{s.m.}{Peixe que tem na cabeça um prolongamento em forma de espada.}{es.pa.dar.te}{0}
\verb{espadas}{}{}{}{}{s.f.pl.}{Um dos quatro naipes do baralho, representado por uma ponta de lança preta.}{es.pa.das}{0}
\verb{espadaúdo}{}{}{}{}{adj.}{Que é dotado de ombros largos.}{es.pa.da.ú.do}{0}
\verb{espadim}{}{}{"-ins}{}{s.m.}{Espada pequena.}{es.pa.dim}{0}
\verb{espádua}{}{Anat.}{}{}{s.f.}{Parte do corpo que vai do alto de cada braço ao começo do pescoço; ombro.}{es.pá.du.a}{0}
\verb{espaguete}{é}{Cul.}{}{}{s.m.}{Macarrão à base de trigo em forma de fio fino sem furo.}{es.pa.gue.te}{0}
\verb{espairecer}{ê}{}{}{}{v.t.}{Livrar de preocupação; distrair, entreter.}{es.pai.re.cer}{\verboinum{15}}
\verb{espairecimento}{}{}{}{}{s.m.}{Ato ou efeito de espairecer; distração, entretenimento.}{es.pai.re.ci.men.to}{0}
\verb{espaldar}{}{}{}{}{s.m.}{Parte da cadeira ou similar em que se apoiam as costas de quem se senta.}{es.pal.dar}{0}
\verb{espalha"-brasas}{}{}{}{}{s.2g.}{Pessoa espalhafatosa, bagunceira.}{es.pa.lha"-bra.sas}{0}
\verb{espalhafato}{}{}{}{}{s.m.}{Estado em que predominam a confusão e o barulho.}{es.pa.lha.fa.to}{0}
\verb{espalhafato}{}{}{}{}{}{Ostentação exagerada.}{es.pa.lha.fa.to}{0}
\verb{espalhafatoso}{ô}{}{"-osos ⟨ó⟩}{"-osa ⟨ó⟩}{adj.}{Que age com espalhafato.}{es.pa.lha.fa.to.so}{0}
\verb{espalhafatoso}{ô}{}{"-osos ⟨ó⟩}{"-osa ⟨ó⟩}{}{Que chama a atenção pelo exagero, pela extravagância.}{es.pa.lha.fa.to.so}{0}
\verb{espalhamento}{}{}{}{}{s.m.}{Ato ou efeito de espalhar.}{es.pa.lha.men.to}{0}
\verb{espalhamento}{}{Fís.}{}{}{}{Processo pelo qual partículas ou radiações eletromagnéticas sofrem uma mudança na sua trajetória ou em sua energia ao interagirem com uma ou mais partículas ou com um campo; difusão.}{es.pa.lha.men.to}{0}
\verb{espalhar}{}{}{}{}{v.t.}{Lançar em várias direções; distribuir.}{es.pa.lhar}{0}
\verb{espalhar}{}{}{}{}{}{Propagar, difundir, dispersar.}{es.pa.lhar}{0}
\verb{espalhar}{}{}{}{}{v.pron.}{Pôr"-se à vontade.}{es.pa.lhar}{\verboinum{1}}
\verb{espalmado}{}{}{}{}{adj.}{Que é plano ou aberto como a palma da mão.}{es.pal.ma.do}{0}
\verb{espalmado}{}{}{}{}{}{Diz"-se da mão aberta e estendida.}{es.pal.ma.do}{0}
\verb{espalmado}{}{}{}{}{}{Diz"-se do metal reduzido a lâmina.}{es.pal.ma.do}{0}
\verb{espalmar}{}{}{}{}{v.t.}{Tornar plano como a palma da mão; aplanar, achatar.}{es.pal.mar}{0}
\verb{espalmar}{}{}{}{}{}{Abrir, distender.}{es.pal.mar}{0}
\verb{espalmar}{}{}{}{}{}{Reduzir o metal a lâminas ou chapas.}{es.pal.mar}{0}
\verb{espalmar}{}{}{}{}{v.i.}{Aparar a bola com as palmas das mãos.}{es.pal.mar}{\verboinum{1}}
\verb{espanador}{ô}{}{}{}{s.m.}{Instrumento formado por um cabo com penas, fios ou tiras de pano numa das pontas, usado para tirar o pó de móveis e outros objetos.}{es.pa.na.dor}{0}
\verb{espanar}{}{}{}{}{v.t.}{Tirar o pó de alguma coisa com o espanador.}{es.pa.nar}{\verboinum{1}}
\verb{espancamento}{}{}{}{}{s.m.}{Ato ou efeito de espancar; sova, surra.}{es.pan.ca.men.to}{0}
\verb{espancar}{}{}{}{}{v.t.}{Dar pancadas; surrar.}{es.pan.car}{\verboinum{2}}
\verb{espanejar}{}{}{}{}{v.t.}{Espanar.}{es.pa.ne.jar}{\verboinum{1}}
\verb{espanhol}{ó}{}{"-óis}{}{adj.}{Relativo à Espanha.}{es.pa.nhol}{0}
\verb{espanhol}{ó}{}{"-óis}{}{s.m.}{Indivíduo natural ou habitante desse país.}{es.pa.nhol}{0}
\verb{espanholismo}{}{}{}{}{s.m.}{Palavra ou construção própria do idioma espanhol.}{es.pa.nho.lis.mo}{0}
\verb{espantadiço}{}{}{}{}{adj.}{Que se espanta com facilidade.}{es.pan.ta.di.ço}{0}
\verb{espantado}{}{}{}{}{adj.}{Que se espantou; assustado.}{es.pan.ta.do}{0}
\verb{espantado}{}{}{}{}{}{Surpreendido, admirado, maravilhado.}{es.pan.ta.do}{0}
\verb{espantado}{}{}{}{}{}{Pasmado, atônito.}{es.pan.ta.do}{0}
\verb{espantalho}{}{}{}{}{s.m.}{Boneco de braços abertos que se coloca no campo para espantar as aves que prejudicam a plantação.}{es.pan.ta.lho}{0}
\verb{espantalho}{}{Fig.}{}{}{}{Indivíduo deselegante.}{es.pan.ta.lho}{0}
\verb{espantar}{}{}{}{}{v.t.}{Causar medo; assustar.}{es.pan.tar}{0}
\verb{espantar}{}{}{}{}{}{Causar admiração; maravilhar, surpreender.}{es.pan.tar}{0}
\verb{espantar}{}{}{}{}{}{Afugentar, repelir.}{es.pan.tar}{\verboinum{1}}
\verb{espanto}{}{}{}{}{s.m.}{Sentimento de se encontrar diante de alguma coisa não esperada ou imaginada; assombro, pasmo.}{es.pan.to}{0}
\verb{espantoso}{ô}{}{"-osos ⟨ó⟩}{"-osa ⟨ó⟩}{adj.}{Que causa espanto; assombroso.}{es.pan.to.so}{0}
\verb{espantoso}{ô}{}{"-osos ⟨ó⟩}{"-osa ⟨ó⟩}{}{Admirável, surpreendente.}{es.pan.to.so}{0}
\verb{esparadrapo}{}{}{}{}{s.m.}{Tira de pano, com cola em um dos lados, própria para segurar o curativo no lugar.}{es.pa.ra.dra.po}{0}
\verb{espargir}{}{}{}{}{v.t.}{Espalhar em gotículas um líquido; borrifar, aspergir.}{es.par.gir}{0}
\verb{espargir}{}{}{}{}{}{Espalhar, derramar.}{es.par.gir}{\verboinum{22}}
\verb{espargo}{}{}{}{}{}{Var. de \textit{aspargo}.}{es.par.go}{0}
\verb{esparramar}{}{}{}{}{v.t.}{Separar ou espalhar em várias direções.}{es.par.ra.mar}{0}
\verb{esparramar}{}{}{}{}{}{Derramar.}{es.par.ra.mar}{0}
\verb{esparramar}{}{}{}{}{v.pron.}{Cair ao comprido ou deitado; estatelar"-se.}{es.par.ra.mar}{\verboinum{1}}
\verb{esparrame}{}{}{}{}{s.m.}{Ato ou efeito de esparramar; dispersão, debandada.}{es.par.ra.me}{0}
\verb{esparrame}{}{}{}{}{}{Ostentação, exagero.}{es.par.ra.me}{0}
\verb{esparrame}{}{}{}{}{}{Confusão, barulho.}{es.par.ra.me}{0}
\verb{esparramo}{}{}{}{}{}{Var. de \textit{esparrame}.}{es.par.ra.mo}{0}
\verb{esparrela}{é}{}{}{}{s.f.}{Armadilha de caça.}{es.par.re.la}{0}
\verb{esparrela}{é}{}{}{}{}{Cilada, logro.}{es.par.re.la}{0}
\verb{esparso}{}{}{}{}{adj.}{Que se espargiu; espalhado, disperso.}{es.par.so}{0}
\verb{espartano}{}{}{}{}{adj.}{Relativo à Esparta, região da antiga Grécia.}{es.par.ta.no}{0}
\verb{espartano}{}{}{}{}{}{Que tem ou lembra a severidade da educação e costumes espartanos; rigoroso, austero.}{es.par.ta.no}{0}
\verb{espartano}{}{}{}{}{s.m.}{Indivíduo natural ou habitante de Esparta.}{es.par.ta.no}{0}
\verb{espartilho}{}{}{}{}{s.m.}{Cinta feminina para comprimir a cintura.}{es.par.ti.lho}{0}
\verb{esparzir}{}{}{}{}{}{Var. de \textit{espargir}.}{es.par.zir}{0}
\verb{espasmo}{}{}{}{}{s.m.}{Contração involuntária de um ou vários músculos.}{es.pas.mo}{0}
\verb{espasmódico}{}{}{}{}{adj.}{Relativo a espasmo.}{es.pas.mó.di.co}{0}
\verb{espatifar}{}{Pop.}{}{}{v.t.}{Quebrar em pedaços; despedaçar, esfacelar.}{es.pa.ti.far}{0}
\verb{espatifar}{}{}{}{}{v.pron.}{Levar um tombo.}{es.pa.ti.far}{\verboinum{1}}
\verb{espátula}{}{}{}{}{s.f.}{Espécie de faca de madeira, de metal ou de outro material, utilizada para abrir livros, ou para espalmar e amolecer preparações farmacêuticas.}{es.pá.tu.la}{0}
\verb{espaventar}{}{}{}{}{v.t.}{Causar espanto ou susto; espantar, assustar, assombrar.}{es.pa.ven.tar}{0}
\verb{espaventar}{}{}{}{}{}{Ostentar luxo.}{es.pa.ven.tar}{0}
\verb{espaventar}{}{}{}{}{}{Demonstrar orgulho ou vaidade.}{es.pa.ven.tar}{\verboinum{1}}
\verb{espavento}{}{}{}{}{s.m.}{Ato ou efeito de espaventar; assombro, espanto.}{es.pa.ven.to}{0}
\verb{espavento}{}{}{}{}{}{Ostentação.}{es.pa.ven.to}{0}
\verb{espaventoso}{ô}{}{"-osos ⟨ó⟩}{"-osa ⟨ó⟩}{adj.}{Que assusta, assombra.}{es.pa.ven.to.so}{0}
\verb{espaventoso}{ô}{}{"-osos ⟨ó⟩}{"-osa ⟨ó⟩}{}{Que chama a atenção; espalhafatoso.}{es.pa.ven.to.so}{0}
\verb{espaventoso}{ô}{}{"-osos ⟨ó⟩}{"-osa ⟨ó⟩}{}{Cheio de orgulho.}{es.pa.ven.to.so}{0}
\verb{espavorido}{}{}{}{}{adj.}{Cheio de medo; apavorado, aterrorizado.}{es.pa.vo.ri.do}{0}
\verb{espavorir}{}{}{}{}{v.t.}{Causar medo, pavor, susto; amedrontar, apavorar.}{es.pa.vo.rir}{\verboinum{35}\verboirregular{\emph{def.} espavorimos, espavoris}}
\verb{especar}{}{}{}{}{v.t.}{Sustentar com espeque; escorar.}{es.pe.car}{0}
\verb{especar}{}{}{}{}{}{Apoiar, firmar.}{es.pe.car}{0}
\verb{especar}{}{}{}{}{v.i.}{Pôr"-se parado; estacar.}{es.pe.car}{\verboinum{2}}
\verb{especial}{}{}{"-ais}{}{adj.2g.}{Que se refere a uma espécie; próprio, peculiar, específico, exclusivo.}{es.pe.ci.al}{0}
\verb{especial}{}{}{"-ais}{}{}{Excelente, extraordinário, fora do comum, distinto.}{es.pe.ci.al}{0}
\verb{especial}{}{}{"-ais}{}{}{Exclusivo, reservado, pessoal.}{es.pe.ci.al}{0}
\verb{especialidade}{}{}{}{}{s.f.}{Qualidade de especial.}{es.pe.ci.a.li.da.de}{0}
\verb{especialidade}{}{}{}{}{}{Particularidade, peculiaridade, singularidade.}{es.pe.ci.a.li.da.de}{0}
\verb{especialidade}{}{}{}{}{}{Atividade ou profissão específica; ramo profissional.}{es.pe.ci.a.li.da.de}{0}
\verb{especialidade}{}{}{}{}{}{Prato, iguaria ou acepipe característico de um lugar, de uma pessoa, de um restaurante, de uma região etc.}{es.pe.ci.a.li.da.de}{0}
\verb{especialista}{}{}{}{}{s.2g.}{Pessoa que se dedica a determinado estudo, atividade ou a um ramo de profissão; perito, conhecedor, \textit{expert}.}{es.pe.ci.a.lis.ta}{0}
\verb{especialização}{}{}{"-ões}{}{s.f.}{Ato ou efeito de especializar.}{es.pe.ci.a.li.za.ção}{0}
\verb{especializar}{}{}{}{}{v.t.}{Tornar especial; distinguir, particularizar, singularizar, especificar.}{es.pe.ci.a.li.zar}{0}
\verb{especializar}{}{}{}{}{v.pron.}{Tornar"-se especialista em alguma coisa; dedicar"-se a uma especialidade.}{es.pe.ci.a.li.zar}{\verboinum{1}}
\verb{especiaria}{}{}{}{}{s.f.}{Quaisquer produtos aromáticos, geralmente ervas, partes de plantas etc., usados para temperar a comida; tempero, condimento.}{es.pe.ci.a.ri.a}{0}
\verb{espécie}{}{}{}{}{s.f.}{Gênero, natureza, qualidade. (\textit{Qualquer espécie de escritor é capaz de fazer poesia, mas nem sempre é boa poesia.})}{es.pé.cie}{0}
\verb{espécie}{}{Biol.}{}{}{}{Grupo de indivíduos com muitas características semelhantes entre si, capazes de se entrecruzarem gerando descendentes férteis.}{es.pé.cie}{0}
\verb{especificação}{}{}{"-ões}{}{s.f.}{Ato ou efeito de especificar; discriminação, pormenorização.}{es.pe.ci.fi.ca.ção}{0}
\verb{especificar}{}{}{}{}{v.t.}{Indicar a espécie; classificar, determinar, discriminar.}{es.pe.ci.fi.car}{\verboinum{2}}
\verb{especificativo}{}{}{}{}{adj.}{Que especifica, que indica ou discrimina.}{es.pe.ci.fi.ca.ti.vo}{0}
\verb{especificidade}{}{}{}{}{s.f.}{Qualidade do que é específico; particularidade.}{es.pe.ci.fi.ci.da.de}{0}
\verb{especificidade}{}{}{}{}{}{Qualidade típica, peculiar de uma espécie.}{es.pe.ci.fi.ci.da.de}{0}
\verb{específico}{}{}{}{}{adj.}{Que é próprio de espécie.}{es.pe.cí.fi.co}{0}
\verb{específico}{}{}{}{}{}{Particular, exclusivo, peculiar, especial.}{es.pe.cí.fi.co}{0}
\verb{espécime}{}{}{}{}{s.m.}{Amostra, exemplo, exemplar, modelo, tipo.}{es.pé.ci.me}{0}
\verb{espécime}{}{Biol.}{}{}{}{Qualquer indivíduo representativo de uma espécie animal ou vegetal; espécimen.}{es.pé.ci.me}{0}
\verb{espécimen}{}{}{}{}{s.m.}{Espécime.}{es.pé.ci.men}{0}
\verb{especioso}{ô}{}{"-osos ⟨ó⟩}{"-osa ⟨ó⟩}{adj.}{Que tem a aparência falsa, enganadora; enganador, ilusório.}{es.pe.ci.o.so}{0}
\verb{especioso}{ô}{}{"-osos ⟨ó⟩}{"-osa ⟨ó⟩}{}{Formoso, belo, atraente, sedutor.}{es.pe.ci.o.so}{0}
\verb{espectador}{ô}{}{}{}{adj.}{Que vê um ato, ou assiste a um espetáculo; observador.}{es.pec.ta.dor}{0}
\verb{espectador}{ô}{}{}{}{s.m.}{Pessoa que observa um ato qualquer; testemunha.}{es.pec.ta.dor}{0}
\verb{espectador}{ô}{}{}{}{}{Pessoa que assiste a um espetáculo.}{es.pec.ta.dor}{0}
\verb{espectro}{é}{}{}{}{s.m.}{Aparição de uma pessoa morta; fantasma.}{es.pec.tro}{0}
\verb{espectro}{é}{Fís.}{}{}{}{Imagem com as cores do arco"-íris, obtida por meio da decomposição da luz solar através de um prisma.}{es.pec.tro}{0}
\verb{especulação}{}{}{"-ões}{}{s.f.}{Ato ou efeito de especular.}{es.pe.cu.la.ção}{0}
\verb{especulação}{}{}{"-ões}{}{}{Investigação estritamente teórica.}{es.pe.cu.la.ção}{0}
\verb{especulador}{ô}{}{}{}{adj.}{Que especula, investiga, analisa.}{es.pe.cu.la.dor}{0}
\verb{especulador}{ô}{}{}{}{s.m.}{Indivíduo que age de má"-fé, aproveitando"-se de uma situação para enganar os outros ou ter lucros ilícitos; aproveitador.}{es.pe.cu.la.dor}{0}
\verb{especular}{}{}{}{}{adj.2g.}{Relativo a espelho ou que possui suas propriedades de reflexão.}{es.pe.cu.lar}{0}
\verb{especular}{}{}{}{}{v.t.}{Investigar um dado assunto ou problema atenciosa e detalhadamente, do ponto de vista teórico; estudar, pesquisar, investigar.}{es.pe.cu.lar}{0}
\verb{especular}{}{}{}{}{}{Praticar especulação comercial ou financeira.}{es.pe.cu.lar}{\verboinum{1}}
\verb{especulativo}{}{}{}{}{adj.}{Em que ocorre especulação comercial ou financeira.}{es.pe.cu.la.ti.vo}{0}
\verb{especulativo}{}{}{}{}{}{Que se dedica à investigação teórica sem levar em consideração o lado prático; teórico.}{es.pe.cu.la.ti.vo}{0}
\verb{espéculo}{}{Med.}{}{}{s.m.}{Instrumento usado para expandir a entrada de certas cavidades do corpo, como a vagina, o ânus etc., para permitir o exame do seu interior.}{es.pé.cu.lo}{0}
\verb{espedaçar}{}{}{}{}{v.t.}{Fazer em pedaços; despedaçar.}{es.pe.da.çar}{\verboinum{3}}
\verb{espeleologia}{}{Geol.}{}{}{s.f.}{Estudo da formação e constituição de cavernas e grutas.}{es.pe.le.o.lo.gi.a}{0}
\verb{espeleologia}{}{Biol.}{}{}{}{Estudo dos organismos que habitam cavernas e grutas.}{es.pe.le.o.lo.gi.a}{0}
\verb{espeleólogo}{}{Geol.}{}{}{s.m.}{Indivíduo que se dedica à espeleologia.}{es.pe.le.ó.lo.go}{0}
\verb{espelhante}{}{}{}{}{adj.2g.}{Que espelha, que reflete como um espelho; polido.}{es.pe.lhan.te}{0}
\verb{espelhar}{}{}{}{}{v.t.}{Refletir como um espelho; deixar ver; retratar, refletir.}{es.pe.lhar}{0}
\verb{espelhar}{}{}{}{}{}{Tornar liso, polido como um espelho.}{es.pe.lhar}{\verboinum{1}}
\verb{espelharia}{}{}{}{}{s.f.}{Fábrica ou loja de espelhos.}{es.pe.lha.ri.a}{0}
\verb{espelho}{ê}{}{}{}{s.m.}{Superfície polida capaz de refletir imagens e luz.}{es.pe.lho}{0}
\verb{espelho}{ê}{Fig.}{}{}{}{Modelo, exemplo.}{es.pe.lho}{0}
\verb{espelho}{ê}{}{}{}{}{Representação, imagem, reflexo.}{es.pe.lho}{0}
\verb{espeloteado}{}{}{}{}{adj.}{Estouvado, confuso, desmiolado.}{es.pe.lo.te.a.do}{0}
\verb{espelunca}{}{}{}{}{s.f.}{Qualquer lugar, estabelecimento sujo, mal frequentado ou de má categoria.}{es.pe.lun.ca}{0}
\verb{espelunca}{}{}{}{}{}{Casa de jogo, geralmente clandestina.}{es.pe.lun.ca}{0}
\verb{espeque}{é}{}{}{}{s.m.}{Peça de madeira ou metal usada para escorar; escora.}{es.pe.que}{0}
\verb{espera}{é}{}{}{}{s.f.}{Ato ou efeito de esperar.}{es.pe.ra}{0}
\verb{espera}{é}{}{}{}{}{Esperança, expectativa.}{es.pe.ra}{0}
\verb{espera}{é}{}{}{}{}{Tempo durante o qual se aguarda; demora, dilação, tardança.}{es.pe.ra}{0}
\verb{esperado}{}{}{}{}{adj.}{Que se espera.}{es.pe.ra.do}{0}
\verb{esperado}{}{}{}{}{}{Provável, previsto.}{es.pe.ra.do}{0}
\verb{esperado}{}{}{}{}{s.m.}{O que é almejado ou desejado.}{es.pe.ra.do}{0}
\verb{esperança}{}{}{}{}{s.f.}{Sentimento de que aquilo que se espera e se deseja será alcançado ou se realizará.}{es.pe.ran.ça}{0}
\verb{esperança}{}{}{}{}{}{Expectativa, espera.}{es.pe.ran.ça}{0}
\verb{esperança}{}{}{}{}{}{Fé, confiança.}{es.pe.ran.ça}{0}
\verb{esperançado}{}{}{}{}{adj.}{Que tem esperança.}{es.pe.ran.ça.do}{0}
\verb{esperançar}{}{}{}{}{v.t.}{Dar esperanças; animar, estimular.}{es.pe.ran.çar}{\verboinum{3}}
\verb{esperançoso}{ô}{}{"-osos ⟨ó⟩}{"-osa ⟨ó⟩}{adj.}{Que tem esperança; esperançado, confiante.}{es.pe.ran.ço.so}{0}
\verb{esperantista}{}{}{}{}{adj.2g.}{Relativo ao esperanto.}{es.pe.ran.tis.ta}{0}
\verb{esperantista}{}{}{}{}{s.2g.}{Especialista ou falante do esperanto.}{es.pe.ran.tis.ta}{0}
\verb{esperanto}{}{}{}{}{s.m.}{Língua artificial, de fácil aprendizado, criada pelo estudioso de línguas Ludwig Zamenhof, no final do século \textsc{xix}, para ser uma língua de comunicação internacional.}{es.pe.ran.to}{0}
\verb{esperar}{}{}{}{}{v.t.}{Ter esperança; contar com alguma coisa.}{es.pe.rar}{0}
\verb{esperar}{}{}{}{}{}{Estar à espera; aguardar.}{es.pe.rar}{0}
\verb{esperar}{}{}{}{}{}{Conjeturar, supor, presumir.}{es.pe.rar}{0}
\verb{esperar}{}{}{}{}{}{Ter esperança na realização de alguma coisa desejada ou prometida.}{es.pe.rar}{0}
\verb{esperar}{}{}{}{}{v.i.}{Estar na expectativa.}{es.pe.rar}{\verboinum{1}}
\verb{esperdiçar}{}{}{}{}{v.t.}{Gastar com exagero; desperdiçar.}{es.per.di.çar}{\verboinum{3}}
\verb{esperma}{é}{}{}{}{s.m.}{Líquido produzido pelas glândulas genitais masculinas e que abriga os espermatozoides; sêmen.}{es.per.ma}{0}
\verb{espermacete}{ê}{Bioquím.}{}{}{s.m.}{Substância branca, gordurosa que se extrai da cabeça de certas espécies de baleias e tem diversos usos na indústria de cosméticos, na fabricação de velas etc.}{es.per.ma.ce.te}{0}
\verb{espermático}{}{}{}{}{adj.}{Relativo a esperma, espermatozoide ou semente.}{es.per.má.ti.co}{0}
\verb{espermatozoide}{}{Biol.}{}{}{s.m.}{Célula reprodutora masculina, dotada de flagelo; gameta masculino.}{es.per.ma.to.zoi.de}{0}
\verb{espermicida}{}{Farm.}{}{}{adj.2g.}{Diz"-se da substância que elimina espermatozoides, usada como contraceptivo local.}{es.per.mi.ci.da}{0}
\verb{espermicida}{}{}{}{}{s.m.}{Essa substância.}{es.per.mi.ci.da}{0}
\verb{espernear}{}{}{}{}{v.i.}{Agitar freneticamente as pernas; pernear.}{es.per.ne.ar}{\verboinum{4}}
\verb{espertalhão}{}{}{"-ões}{"-ona}{adj.}{Que é muito esperto, astuto ou malicioso; velhaco, enganador.}{es.per.ta.lhão}{0}
\verb{espertar}{}{}{}{}{v.t.}{Despertar, acordar.}{es.per.tar}{0}
\verb{espertar}{}{}{}{}{}{Avivar, atiçar, animar.}{es.per.tar}{\verboinum{1}}
\verb{esperteza}{ê}{}{}{}{s.f.}{Qualidade, modos ou ação de esperto.}{es.per.te.za}{0}
\verb{esperteza}{ê}{}{}{}{}{Malandragem, astúcia, malícia, manha, ardil.}{es.per.te.za}{0}
\verb{esperto}{é}{}{}{}{adj.}{Desperto, acordado.}{es.per.to}{0}
\verb{esperto}{é}{}{}{}{}{Vivo, inteligente, perspicaz, arguto.}{es.per.to}{0}
\verb{esperto}{é}{}{}{}{}{Espertalhão, finório, astuto.}{es.per.to}{0}
\verb{espessar}{}{}{}{}{v.t.}{Tornar espesso; engrossar, adensar.}{es.pes.sar}{\verboinum{1}}
\verb{espesso}{ê}{}{}{}{adj.}{Que é denso, grosso, compacto.}{es.pes.so}{0}
\verb{espesso}{ê}{}{}{}{}{Que tem a consistência cremosa, pastosa; viscoso, consistente.}{es.pes.so}{0}
\verb{espessura}{}{}{}{}{s.f.}{Qualidade do que é espesso; grossura.}{es.pes.su.ra}{0}
\verb{espessura}{}{}{}{}{}{A terceira dimensão de um sólido; densidade.}{es.pes.su.ra}{0}
\verb{espetacular}{}{}{}{}{adj.2g.}{Que é próprio de espetáculo.}{es.pe.ta.cu.lar}{0}
\verb{espetacular}{}{Por ext.}{}{}{}{Que é sensacional, surpreendente, excepcional.}{es.pe.ta.cu.lar}{0}
\verb{espetáculo}{}{}{}{}{s.m.}{Tudo aquilo que chama a atenção, que atrai o olhar.}{es.pe.tá.cu.lo}{0}
\verb{espetáculo}{}{}{}{}{}{Contemplação.}{es.pe.tá.cu.lo}{0}
\verb{espetáculo}{}{}{}{}{}{Qualquer representação de teatro, cinema, televisão; apresentação pública de música, dança etc.; \textit{show}.}{es.pe.tá.cu.lo}{0}
\verb{espetáculo}{}{}{}{}{}{Cena (discussão, bate"-boca, briga etc.) ridícula ou escandalosa. }{es.pe.tá.cu.lo}{0}
\verb{espetaculoso}{ô}{}{"-osos ⟨ó⟩}{"-osa ⟨ó⟩}{adj.}{Que dá muito na vista; que chama muito a atenção; espalhafatoso.}{es.pe.ta.cu.lo.so}{0}
\verb{espetada}{}{}{}{}{s.f.}{Ato ou efeito de espetar.}{es.pe.ta.da}{0}
\verb{espetadela}{é}{}{}{}{s.f.}{Ato ou efeito de espetar levemente; espetada, picada.}{es.pe.ta.de.la}{0}
\verb{espetadela}{é}{}{}{}{}{Fraude, engano, logro.}{es.pe.ta.de.la}{0}
\verb{espetar}{}{}{}{}{v.t.}{Furar ou transpassar com espeto ou outro objeto pontiagudo; enfiar, fincar.}{es.pe.tar}{0}
\verb{espetar}{}{}{}{}{}{Fixar, cravar, prender, transfixar.}{es.pe.tar}{\verboinum{1}}
\verb{espeto}{ê}{}{}{}{s.m.}{Utensílio de ferro com o qual se espetam vários tipos de carnes para assar.}{es.pe.to}{0}
\verb{espeto}{ê}{}{}{}{}{Qualquer vara ou pedaço de pau com uma das extremidades afiada.}{es.pe.to}{0}
\verb{espeto}{ê}{Fig.}{}{}{}{Pessoa muito alta e magra.}{es.pe.to}{0}
\verb{espevitadeira}{ê}{}{}{}{s.f.}{Tesoura usada para espevitar pavios.}{es.pe.vi.ta.dei.ra}{0}
\verb{espevitado}{}{}{}{}{adj.}{Diz"-se do pavio ou morrão que foi cortado  com espevitadeira.}{es.pe.vi.ta.do}{0}
\verb{espevitado}{}{Fig.}{}{}{}{Assanhado, animado.}{es.pe.vi.ta.do}{0}
\verb{espevitado}{}{Fig.}{}{}{}{Arrogante, pretensioso, afetado, petulante.}{es.pe.vi.ta.do}{0}
\verb{espevitar}{}{}{}{}{v.t.}{Cortar, aparar um pavio para avivar a chama.}{es.pe.vi.tar}{0}
\verb{espevitar}{}{Fig.}{}{}{}{Estimular, despertar, avivar.}{es.pe.vi.tar}{0}
\verb{espevitar}{}{}{}{}{v.pron.}{Irritar"-se, exaltar"-se, zangar"-se.}{es.pe.vi.tar}{\verboinum{1}}
\verb{espezinhar}{}{}{}{}{v.t.}{Bater com os pés; pisar, pisotear.}{es.pe.zi.nhar}{0}
\verb{espezinhar}{}{Fig.}{}{}{}{Tratar ou falar com desprezo ou desdém; desdenhar, ofender, humilhar, rebaixar.}{es.pe.zi.nhar}{\verboinum{1}}
\verb{espia}{}{}{}{}{s.2g.}{Espião.}{es.pi.a}{0}
\verb{espia}{}{}{}{}{}{Sentinela, vigia.}{es.pi.a}{0}
\verb{espiada}{}{}{}{}{s.f.}{Ato ou efeito de espiar; olhadela, espiadela. }{es.pi.a.da}{0}
\verb{espiadela}{é}{}{}{}{s.f.}{Ato ou efeito de espiar rapidamente, e de uma só vez; espiada, olhadela.}{es.pi.a.de.la}{0}
\verb{espião}{}{}{"-ões}{espiã}{adj.}{Que espiona, que é próprio para fazer espionagem.}{es.pi.ão}{0}
\verb{espião}{}{}{"-ões}{espiã}{s.m.}{Pessoa que observa as atividades, os hábitos de alguém secretamente com o objetivo de obter informações.}{es.pi.ão}{0}
\verb{espião}{}{}{"-ões}{espiã}{}{Agente secreto cuja função é obter informações confidenciais sobre determinado país, projeto etc., e revelá"-las ao governo ou organização para a qual trabalha.}{es.pi.ão}{0}
\verb{espiar}{}{}{}{}{v.t.}{Observar secretamente para conseguir informações.}{es.pi.ar}{0}
\verb{espiar}{}{Bras.}{}{}{}{Olhar, observar às ocultas ou disfarçadamente.}{es.pi.ar}{\verboinum{1}}
\verb{espicaçar}{}{}{}{}{v.t.}{Ferir com o bico; bicar.}{es.pi.ca.çar}{0}
\verb{espicaçar}{}{}{}{}{}{Furar ou picar repetidamente com instrumento agudo.}{es.pi.ca.çar}{0}
\verb{espicaçar}{}{}{}{}{}{Molestar, afligir, magoar, torturar.}{es.pi.ca.çar}{0}
\verb{espicaçar}{}{}{}{}{}{Instigar, estimular, incitar, atiçar. }{es.pi.ca.çar}{\verboinum{3}}
\verb{espichar}{}{}{}{}{v.t.}{Esticar alguma coisa; estender, alongar.}{es.pi.char}{0}
\verb{espichar}{}{}{}{}{v.pron.}{Deitar"-se, estirar"-se, refestelar"-se.}{es.pi.char}{\verboinum{1}}
\verb{espiga}{}{Bot.}{}{}{s.f.}{Tipo de inflorescência na qual as flores se encontram dispostas em torno de um eixo central.}{es.pi.ga}{0}
\verb{espiga}{}{}{}{}{}{Haste terminal de várias gramíneas, como o trigo, o milho etc., onde se acham os grãos. }{es.pi.ga}{0}
\verb{espiga}{}{}{}{}{}{Parte de uma peça que se encaixa no furo de outra.}{es.pi.ga}{0}
\verb{espigado}{}{}{}{}{adj.}{Que formou espiga.}{es.pi.ga.do}{0}
\verb{espigado}{}{Fig.}{}{}{}{Que se desenvolveu; alto, crescido.}{es.pi.ga.do}{0}
\verb{espigão}{}{}{"-ões}{}{s.m.}{Espiga grande.}{es.pi.gão}{0}
\verb{espigão}{}{}{"-ões}{}{}{Aresta de telhado, saliente e inclinada.}{es.pi.gão}{0}
\verb{espigão}{}{Bras.}{"-ões}{}{}{Edifício muito alto.}{es.pi.gão}{0}
\verb{espigar}{}{}{}{}{v.i.}{Criar espiga (trigo, milho, entre outros).}{es.pi.gar}{0}
\verb{espigar}{}{}{}{}{}{Desenvolver"-se, crescer; tornar"-se alto.}{es.pi.gar}{\verboinum{5}}
\verb{espinafrar}{}{Pop.}{}{}{v.t.}{Repreender, chamar a atenção (de alguém) de forma rude; descompor.}{es.pi.na.frar}{0}
\verb{espinafre}{}{}{}{}{s.m.}{Planta hortense, comestível, de folhas esverdeadas.}{es.pi.na.fre}{0}
\verb{espinal}{}{}{}{}{adj.2g.}{Espinhal.}{es.pi.nal}{0}
\verb{espingarda}{}{}{}{}{s.f.}{Tipo de arma de fogo, portátil e de cano longo.}{es.pin.gar.da}{0}
\verb{espingardear}{}{}{}{}{v.t.}{Alvejar ou matar com tiro de espingarda; fuzilar.}{es.pin.gar.de.ar}{\verboinum{4}}
\verb{espinha}{}{Pop.}{}{}{s.f.}{A coluna vertebral.}{es.pi.nha}{0}
\verb{espinha}{}{}{}{}{}{Osso do esqueleto dos peixes.}{es.pi.nha}{0}
\verb{espinha}{}{}{}{}{}{Pequena erupção na pele; acne.}{es.pi.nha}{0}
\verb{espinhaço}{}{Pop.}{}{}{s.m.}{A coluna vertebral; espinha dorsal.}{es.pi.nha.ço}{0}
\verb{espinhaço}{}{Pop.}{}{}{}{Dorso, costas.}{es.pi.nha.ço}{0}
\verb{espinhaço}{}{}{}{}{}{Cordilheira.}{es.pi.nha.ço}{0}
\verb{espinhal}{}{}{}{}{adj.2g.}{Relativo a espinha vertebral; espinal.}{es.pi.nhal}{0}
\verb{espinhar}{}{}{}{}{v.t.}{Ferir, furar com espinho.}{es.pi.nhar}{0}
\verb{espinhar}{}{Fig.}{}{}{}{Zangar, irritar.}{es.pi.nhar}{\verboinum{1}}
\verb{espinheiral}{}{}{"-ais}{}{s.m.}{Grande quantidade de espinheiros próximos entre si; espinhal.}{es.pi.nhei.ral}{0}
\verb{espinheiro}{ê}{}{}{}{s.m.}{Arbusto de casca fina, com os ramos cheios de espinhos curvos e as flores verde"-amareladas, nativo das Guianas e da Amazônia.}{es.pi.nhei.ro}{0}
\verb{espinhel}{é}{}{"-éis}{}{s.m.}{Aparelho de pesca formado por uma corda longa na qual se prendem a intervalos linhas armadas de anzóis. }{es.pi.nhel}{0}
\verb{espinhela}{é}{Pop.}{}{}{s.f.}{O apêndice xifoide (extremidade inferior do osso esterno).}{es.pi.nhe.la}{0}
\verb{espinhento}{}{}{}{}{adj.}{Espinhoso.}{es.pi.nhen.to}{0}
\verb{espinho}{}{}{}{}{s.m.}{Saliência rígida e pontiaguda no caule e nos ramos de certas plantas.}{es.pi.nho}{0}
\verb{espinho}{}{}{}{}{}{Cerda enrijecida de alguns animais, como o porco"-espinho e o ouriço"-do"-mar.}{es.pi.nho}{0}
\verb{espinho}{}{Fig.}{}{}{}{Complicação, entrave, embaraço, dificuldade.}{es.pi.nho}{0}
\verb{espinhoso}{ô}{}{"-osos ⟨ó⟩}{"-osa ⟨ó⟩}{adj.}{Que tem ou está coberto de espinhos ou espinhas; espinhento.}{es.pi.nho.so}{0}
\verb{espinhoso}{ô}{Fig.}{"-osos ⟨ó⟩}{"-osa ⟨ó⟩}{}{Que é difícil, complicado, embaraçoso.}{es.pi.nho.so}{0}
\verb{espiniforme}{ó}{}{}{}{adj.2g.}{Que tem forma de espinho.}{es.pi.ni.for.me}{0}
\verb{espinoteado}{}{}{}{}{adj.}{Que é estouvado, leviano, doidivanas.}{es.pi.no.te.a.do}{0}
\verb{espinotear}{}{}{}{}{v.i.}{Dar pinotes; pinotear.}{es.pi.no.te.ar}{\verboinum{4}}
\verb{espinotear}{}{}{}{}{}{Esbravejar, enfurecer"-se, encolerizar"-se.}{es.pi.no.te.ar}{0}
\verb{espiolhar}{}{}{}{}{v.t.}{Limpar de piolhos.}{es.pi.o.lhar}{\verboinum{1}}
\verb{espionagem}{}{}{"-ens}{}{s.f.}{Ato ou efeito de espionar. }{es.pi.o.na.gem}{0}
\verb{espionagem}{}{}{"-ens}{}{}{Prática ou atividade de espião.}{es.pi.o.na.gem}{0}
\verb{espionar}{}{}{}{}{v.t.}{Observar algo ou alguém secretamente; espiar, vigiar, espreitar.}{es.pi.o.nar}{0}
\verb{espionar}{}{}{}{}{v.i.}{Exercer atividade de espião.}{es.pi.o.nar}{\verboinum{1}}
\verb{espira}{}{}{}{}{s.f.}{Cada uma das voltas de uma espiral, de uma hélice ou de um parafuso.}{es.pi.ra}{0}
\verb{espiral}{}{}{"-ais}{}{adj.2g.}{Que tem forma de espiral ou a ela se assemelha; espiralado, encaracolado. }{es.pi.ral}{0}
\verb{espiral}{}{}{"-ais}{}{s.f.}{Curva plana que descreve voltas em torno de um ponto, do qual se afasta gradualmente.}{es.pi.ral}{0}
\verb{espiralado}{}{}{}{}{adj.}{Que tem forma de espiral.}{es.pi.ra.la.do}{0}
\verb{espiralar}{}{}{}{}{v.t.}{Tornar espiralado.}{es.pi.ra.lar}{\verboinum{1}}
\verb{espirar}{}{}{}{}{v.t.}{Respirar, soprar, exalar.}{es.pi.rar}{0}
\verb{espirar}{}{}{}{}{v.i.}{Estar vivo.}{es.pi.rar}{\verboinum{1}}
\verb{espírita}{}{}{}{}{adj.2g.}{Relativo ao espiritismo.}{es.pí.ri.ta}{0}
\verb{espírita}{}{}{}{}{s.2g.}{Indivíduo que segue ou pratica o espiritismo.}{es.pí.ri.ta}{0}
\verb{espiriteira}{ê}{}{}{}{s.f.}{Pequeno fogareiro a álcool. }{es.pi.ri.tei.ra}{0}
\verb{espiritismo}{}{Relig.}{}{}{s.m.}{Doutrina que acredita na reencarnação e na comunicação entre vivos e mortos. }{es.pi.ri.tis.mo}{0}
\verb{espiritista}{}{}{}{}{adj.2g. e s.2g.}{Espírita.}{es.pi.ri.tis.ta}{0}
\verb{espírito}{}{}{}{}{s.m.}{Princípio imortal do ser humano; alma.}{es.pí.ri.to}{0}
\verb{espírito}{}{}{}{}{}{Entidade sobrenatural; espectro.}{es.pí.ri.to}{0}
\verb{espírito}{}{}{}{}{}{Ideia, intenção.}{es.pí.ri.to}{0}
\verb{espírito}{}{}{}{}{}{Graça, sutileza, finura.}{es.pí.ri.to}{0}
\verb{espírito}{}{}{}{}{}{Imaginação, engenho, inteligência.}{es.pí.ri.to}{0}
\verb{espírito}{}{}{}{}{}{Líquido obtido pela destilação; álcool.}{es.pí.ri.to}{0}
\verb{espírito"-santense}{}{}{espírito"-santenses}{}{adj.2g.}{Relativo ao estado do Espírito Santo; capixaba.}{es.pí.ri.to"-san.ten.se}{0}
\verb{espírito"-santense}{}{}{espírito"-santenses}{}{s.2g.}{Indivíduo natural ou habitante desse estado.}{es.pí.ri.to"-san.ten.se}{0}
\verb{espiritual}{}{}{"-ais}{}{adj.2g.}{Relativo ao espírito.}{es.pi.ri.tu.al}{0}
\verb{espiritual}{}{}{"-ais}{}{}{Relativo a religião, a misticismo; místico, sobrenatural.}{es.pi.ri.tu.al}{0}
\verb{espiritual}{}{}{"-ais}{}{}{Imaterial, incorpóreo, impalpável.}{es.pi.ri.tu.al}{0}
\verb{espiritualidade}{}{}{}{}{s.f.}{Qualidade do que é espiritual.}{es.pi.ri.tu.a.li.da.de}{0}
\verb{espiritualidade}{}{}{}{}{}{Inteligência superior; transcendência, sublimidade.}{es.pi.ri.tu.a.li.da.de}{0}
\verb{espiritualismo}{}{}{}{}{s.m.}{Doutrina que se baseia na existência do espírito e de Deus.}{es.pi.ri.tu.a.lis.mo}{0}
\verb{espiritualista}{}{}{}{}{adj.2g.}{Relativo ao espiritualismo.}{es.pi.ri.tu.a.lis.ta}{0}
\verb{espiritualista}{}{}{}{}{s.2g.}{Indivíduo que segue os preceitos do espiritualismo.}{es.pi.ri.tu.a.lis.ta}{0}
\verb{espiritualizar}{}{}{}{}{v.t.}{Tornar espiritual, distanciando"-se dos aspectos materialistas.}{es.pi.ri.tu.a.li.zar}{\verboinum{1}}
\verb{espirituoso}{ô}{}{"-osos ⟨ó⟩}{"-osa ⟨ó⟩}{adj.}{Que provoca o riso; engraçado, vivaz.}{es.pi.ri.tu.o.so}{0}
\verb{espirituoso}{ô}{}{"-osos ⟨ó⟩}{"-osa ⟨ó⟩}{}{Diz"-se da bebida que contém álcool.}{es.pi.ri.tu.o.so}{0}
\verb{espirradeira}{ê}{Bot.}{}{}{s.f.}{Planta ornamental que contém folhas em forma de pequenas lanças e flores brancas ou vermelhas.}{es.pir.ra.dei.ra}{0}
\verb{espirrar}{}{}{}{}{v.t.}{Lançar fora de si; expelir, emitir.}{es.pir.rar}{0}
\verb{espirrar}{}{}{}{}{v.i.}{Soltar espirros; esguichar.}{es.pir.rar}{\verboinum{1}}
\verb{espirro}{}{}{}{}{s.m.}{Expiração rápida e ruidosa do ar pelo nariz e pela boca.}{es.pir.ro}{0}
\verb{esplanada}{}{}{}{}{s.f.}{Terreno largo e plano em frente a um edifício.}{es.pla.na.da}{0}
\verb{esplanada}{}{}{}{}{}{Chapada, planalto, platô.}{es.pla.na.da}{0}
\verb{esplender}{ê}{}{}{}{v.i.}{Brilhar intensamente; resplandecer.}{es.plen.der}{\verboinum{12}}
\verb{esplêndido}{}{}{}{}{adj.}{Que tem esplendor; brilhante, reluzente.}{es.plên.di.do}{0}
\verb{esplêndido}{}{}{}{}{}{Magnífico, suntuoso, grandioso.}{es.plên.di.do}{0}
\verb{esplendor}{ô}{}{}{}{s.m.}{Brilho intenso; fulgor.}{es.plen.dor}{0}
\verb{esplendoroso}{ô}{}{"-osos ⟨ó⟩}{"-osa ⟨ó⟩}{adj.}{Cheio de esplendor; resplandecente, magnífico.}{es.plen.do.ro.so}{0}
\verb{esplênico}{}{}{}{}{adj.}{Relativo ao baço.}{es.plê.ni.co}{0}
\verb{esplenite}{}{Med.}{}{}{s.f.}{Inflamação do baço.}{es.ple.ni.te}{0}
\verb{espocar}{}{}{}{}{v.i.}{Arrebentar com barulho; estourar, pipocar.}{es.po.car}{\verboinum{2}}
\verb{espojar"-se}{}{}{}{}{v.pron.}{Estender"-se e rolar no chão. (\textit{Os alunos se espojaram na quadra de esportes.})}{es.po.jar"-se}{\verboinum{1}}
\verb{espoleta}{ê}{}{}{}{s.f.}{Peça da arma destinada a inflamar a carga de pólvora dos projéteis ocos.}{es.po.le.ta}{0}
\verb{espoleta}{ê}{Pop.}{}{}{adj.}{Criança muito ativa, inquieta, peralta.}{es.po.le.ta}{0}
\verb{espoliação}{}{}{"-ões}{}{s.f.}{Ato ou efeito de espoliar; usurpação.}{es.po.li.a.ção}{0}
\verb{espoliador}{ô}{}{}{}{adj.}{Que espolia; usurpador.}{es.po.li.a.dor}{0}
\verb{espoliar}{}{}{}{}{v.t.}{Privar (alguém) de seus bens e direitos por meios ilícitos; esbulhar, saquear.}{es.po.li.ar}{\verboinum{1}}
\verb{espólio}{}{}{}{}{s.m.}{Bens e posses que são deixados por alguém que morreu; herança.}{es.pó.lio}{0}
\verb{espólio}{}{}{}{}{}{Conjunto de coisas que são tomadas do inimigo numa guerra; despojo.}{es.pó.lio}{0}
\verb{espongiário}{}{Zool.}{}{}{adj.}{Espécime de animais invertebrados rudimentares, que não possuem órgãos nem tecidos verdadeiros, e que têm por tipo as esponjas; porífero. }{es.pon.gi.á.rio}{0}
\verb{esponja}{}{Zool.}{}{}{s.f.}{Animal invertebrado rudimentar, de água doce ou salgada, cujo corpo é provido de numerosos poros.}{es.pon.ja}{0}
\verb{esponja}{}{}{}{}{}{Utensílio poroso e absorvente, usado em banho, limpeza doméstica, entre outros.}{es.pon.ja}{0}
\verb{esponjoso}{ô}{}{"-osos ⟨ó⟩}{"-osa ⟨ó⟩}{adj.}{Que tem a aparência da esponja.}{es.pon.jo.so}{0}
\verb{esponjoso}{ô}{}{"-osos ⟨ó⟩}{"-osa ⟨ó⟩}{}{Leve, poroso e absorvente.}{es.pon.jo.so}{0}
\verb{esponsais}{}{}{}{}{s.m.pl.}{Festa de casamento; bodas.}{es.pon.sais}{0}
\verb{esponsal}{}{}{"-ais}{}{adj.2g.}{Relativo a esposo.}{es.pon.sal}{0}
\verb{esponsal}{}{}{"-ais}{}{s.m.}{União de almas; afinidade.}{es.pon.sal}{0}
%\verb{}{}{}{}{}{}{}{}{0}
\verb{espontaneidade}{}{}{}{}{s.f.}{Qualidade do que é espontâneo; naturalidade.}{es.pon.ta.nei.da.de}{0}
\verb{espontâneo}{}{}{}{}{adj.}{Que não é forçado; de livre vontade; voluntário.}{es.pon.tâ.neo}{0}
\verb{espontâneo}{}{}{}{}{}{Sem artificialismos; natural, verdadeiro.}{es.pon.tâ.neo}{0}
\verb{espontar}{}{}{}{}{v.t.}{Aparar as pontas, as extremidades de; podar.}{es.pon.tar}{0}
\verb{espontar}{}{}{}{}{v.i.}{Começar a aparecer; surgir, despontar.}{es.pon.tar}{\verboinum{1}}
\verb{espora}{ó}{}{}{}{s.f.}{Ponta de metal que se prende no calcanhar do cavaleiro para incitar a montaria. }{es.po.ra}{0}
\verb{espora}{ó}{Fig.}{}{}{}{Incitamento, estímulo.}{es.po.ra}{0}
\verb{esporada}{}{}{}{}{s.f.}{Picada com espora.}{es.po.ra.da}{0}
\verb{esporada}{}{Pop.}{}{}{}{Repreensão, descompostura.}{es.po.ra.da}{0}
\verb{esporádico}{}{}{}{}{adj.}{Que ocorre poucas vezes; raro, espaçado, disperso.}{es.po.rá.di.co}{0}
\verb{esporângio}{}{Bot.}{}{}{s.m.}{Órgão que produz os esporos.}{es.po.rân.gio}{0}
\verb{esporão}{}{}{}{}{s.m.}{Espécie de unha ou saliência que nasce nas asas ou pernas de certas aves, como o galo, o pavão, entre outras.}{es.po.rão}{0}
\verb{esporar}{}{}{}{}{v.t.}{Esporear.}{es.po.rar}{\verboinum{1}}
\verb{esporear}{}{}{}{}{v.t.}{Ferir com espora.}{es.po.re.ar}{0}
\verb{esporear}{}{}{}{}{}{Excitar, estimular.}{es.po.re.ar}{\verboinum{4}}
\verb{esporo}{ó}{Biol.}{}{}{s.m.}{Célula reprodutora de certos vegetais e microrganismos, resistente ao calor e à dessecação, capaz de germinar sem fundir"-se com outras células.}{es.po.ro}{0}
\verb{esporro}{ô}{Chul.}{}{}{s.m.}{Censura áspera; reprimenda violenta; descompostura.}{es.por.ro}{0}
\verb{esporro}{ô}{Chul.}{}{}{}{Esperma, sêmen.}{es.por.ro}{0}
\verb{esporte}{ó}{}{}{}{s.m.}{Prática de exercícios físicos, individual ou em equipe; desporto. (\textit{Toda criança deve praticar um esporte para desenvolver sua capacidade motora.})}{es.por.te}{0}
\verb{esporte}{ó}{}{}{}{}{Diversão, entretenimento, recreação.}{es.por.te}{0}
\verb{esporte}{ó}{}{}{}{adj.2g.}{Diz"-se da roupa simples e confortável; informal.}{es.por.te}{0}
\verb{esportista}{}{}{}{}{s.2g.}{Indivíduo que se dedica ao esporte; desportista.}{es.por.tis.ta}{0}
\verb{esportiva}{}{Bras.}{}{}{s.f.}{Espírito esportivo.}{es.por.ti.va}{0}
\verb{esportivo}{}{}{}{}{adj.}{Relativo a esporte; desportivo.}{es.por.ti.vo}{0}
\verb{esportivo}{}{}{}{}{}{Descontraído, informal.}{es.por.ti.vo}{0}
\verb{espórtula}{}{}{}{}{s.f.}{Gratificação em dinheiro; gorjeta, donativo.}{es.pór.tu.la}{0}
\verb{esposa}{ô}{}{}{}{s.f.}{Mulher casada, em relação a seu marido.}{es.po.sa}{0}
\verb{esposar}{}{}{}{}{v.t.}{Unir em matrimônio; casar, desposar.}{es.po.sar}{\verboinum{1}}
\verb{esposo}{ô}{}{}{}{s.m.}{Homem casado, em relação à sua mulher; marido.}{es.po.so}{0}
\verb{esposório}{}{}{}{}{s.m.}{Festa de casamento; esponsais, bodas.}{es.po.só.rio}{0}
\verb{espostejar}{}{}{}{}{v.t.}{Cortar em postas ou fatias; retalhar, fatiar. (\textit{O peixeiro espostejou o cação.})}{es.pos.te.jar}{\verboinum{1}}
\verb{espraiar}{}{}{}{}{v.t.}{Lançar à praia.}{es.prai.ar}{0}
\verb{espraiar}{}{}{}{}{}{Estender, expandir, espalhar.}{es.prai.ar}{\verboinum{1}}
\verb{espreguiçadeira}{ê}{}{}{}{s.f.}{Cadeira comprida, com encosto reclinável ou reclinado, usada para fazer a sesta ou para repousar.}{es.pre.gui.ça.dei.ra}{0}
\verb{espreguiçar}{}{}{}{}{v.t.}{Tirar a preguiça; espertar.}{es.pre.gui.çar}{0}
\verb{espreguiçar}{}{}{}{}{v.pron.}{Estirar os membros, por cansaço ou sono.}{es.pre.gui.çar}{\verboinum{3}}
\verb{espreita}{ê}{}{}{}{s.f.}{Ato ou efeito de espreitar.}{es.prei.ta}{0}
\verb{espreita}{ê}{}{}{}{}{Espionagem, vigilância, tocaia.}{es.prei.ta}{0}
\verb{espreitar}{}{}{}{}{v.t.}{Observar às escondidas; vigiar, espionar.}{es.prei.tar}{\verboinum{1}}
\verb{espremedor}{ô}{}{}{}{s.m.}{Utensílio doméstico usado para espremer frutas.}{es.pre.me.dor}{0}
\verb{espremer}{ê}{}{}{}{v.t.}{Apertar para extrair suco, líquido; comprimir, premer.}{es.pre.mer}{0}
\verb{espremer}{ê}{Fig.}{}{}{}{Tirar o máximo de algo; esgotar.}{es.pre.mer}{0}
\verb{espremer}{ê}{Fig.}{}{}{}{Interrogar com insistência; coagir.}{es.pre.mer}{\verboinum{12}}
\verb{espremido}{}{}{}{}{adj.}{Que se espremeu; apertado, premido.}{es.pre.mi.do}{0}
\verb{espulgar}{}{}{}{}{v.t.}{Limpar de pulgas; catar.}{es.pul.gar}{\verboinum{5}}
\verb{espuma}{}{}{}{}{s.f.}{Conjunto de bolhas que se formam na superfície de um líquido; escuma.}{es.pu.ma}{0}
\verb{espumadeira}{ê}{}{}{}{s.f.}{Utensílio de cozinha, semelhante a uma colher comprida com furos, usado para tirar a espuma de um líquido; escumadeira.}{es.pu.ma.dei.ra}{0}
\verb{espumante}{}{}{}{}{adj.2g.}{Que forma espuma. (\textit{Hoje em dia, os vinhos espumantes são muito apreciados.})}{es.pu.man.te}{0}
\verb{espumar}{}{}{}{}{v.t.}{Cobrir de espuma.}{es.pu.mar}{0}
\verb{espumar}{}{}{}{}{v.i.}{Produzir espuma. (\textit{Esse sabão não espuma.})}{es.pu.mar}{\verboinum{1}}
\verb{espumarada}{}{}{}{}{s.f.}{Grande quantidade de espuma.}{es.pu.ma.ra.da}{0}
\verb{espumejar}{}{}{}{}{v.i.}{Lançar espuma; espumar.}{es.pu.me.jar}{\verboinum{1}}
\verb{espumoso}{ô}{}{"-osos ⟨ó⟩}{"-osa ⟨ó⟩}{adj.}{Que tem ou está cheio de espuma. (\textit{O leite tirado diretamente da teta da vaca é muito espumoso.})}{es.pu.mo.so}{0}
\verb{espúrio}{}{}{}{}{adj.}{Diz"-se do filho nascido do adultério; ilegítimo, bastardo.}{es.pú.rio}{0}
\verb{espúrio}{}{}{}{}{}{Adulterado, falsificado, alterado.}{es.pú.rio}{0}
\verb{esquadra}{}{}{}{}{s.f.}{Conjunto dos navios de guerra de um país.}{es.qua.dra}{0}
\verb{esquadra}{}{Bras.}{}{}{}{Equipe de jogadores; time.}{es.qua.dra}{0}
\verb{esquadrão}{}{}{"-ões}{}{s.m.}{Unidade de combate da cavalaria, do exército blindado ou da polícia.}{es.qua.drão}{0}
\verb{esquadrejar}{}{}{}{}{v.t.}{Serrar a madeira dentro das medidas necessárias.}{es.qua.dre.jar}{\verboinum{1}}
\verb{esquadria}{}{}{}{}{s.f.}{Ângulo reto.}{es.qua.dri.a}{0}
\verb{esquadria}{}{}{}{}{}{Conjunto de peças de madeira que formam a moldura e o acabamento de portas, janelas, venezianas.}{es.qua.dri.a}{0}
\verb{esquadrilha}{}{}{}{}{s.f.}{Pequeno grupo de navios de guerra.}{es.qua.dri.lha}{0}
\verb{esquadrilha}{}{}{}{}{}{Grupo de dois a quatro aviões. (\textit{Ontem assistimos a uma apresentação da esquadrilha da fumaça.})}{es.qua.dri.lha}{0}
\verb{esquadrinhamento}{}{}{}{}{s.m.}{Ato ou efeito de esquadrinhar; observação minuciosa; investigação.}{es.qua.dri.nha.men.to}{0}
\verb{esquadrinhar}{}{}{}{}{v.t.}{Examinar detalhadamente; pesquisar, esmiuçar.}{es.qua.dri.nhar}{0}
\verb{esquadrinhar}{}{}{}{}{}{Procurar minuciosamente; vasculhar, escarafunchar.}{es.qua.dri.nhar}{\verboinum{1}}
\verb{esquadro}{}{}{}{}{s.m.}{Instrumento usado para medir ângulos retos e traçar linhas perpendiculares.}{es.qua.dro}{0}
\verb{esquálido}{}{}{}{}{adj.}{Que perdeu as forças por doença ou desnutrição; pálido, magro, macilento.}{es.quá.li.do}{0}
\verb{esquartejamento}{}{}{}{}{s.m.}{Ato ou efeito de esquartejar, de cortar em partes.}{es.quar.te.ja.men.to}{0}
\verb{esquartejar}{}{}{}{}{v.t.}{Cortar em partes; retalhar, despedaçar.}{es.quar.te.jar}{\verboinum{1}}
\verb{esquecer}{ê}{}{}{}{v.t.}{Perder a lembrança; olvidar.}{es.que.cer}{0}
\verb{esquecer}{ê}{}{}{}{}{Deixar de lado; abandonar, relegar. (\textit{Ele esqueceu os amigos depois que se tornou famoso.})}{es.que.cer}{\verboinum{15}}
\verb{esquecidiço}{}{}{}{}{adj.}{Que se esquece facilmente das coisas; esquecido.}{es.que.ci.di.ço}{0}
\verb{esquecido}{}{}{}{}{adj.}{Que se esqueceu; perdido na lembrança.}{es.que.ci.do}{0}
\verb{esquecido}{}{}{}{}{}{Que se esquece das coisas; esquecidiço.}{es.que.ci.do}{0}
\verb{esquecimento}{}{}{}{}{s.m.}{Ato ou efeito de esquecer; olvido.}{es.que.ci.men.to}{0}
\verb{esquecimento}{}{}{}{}{}{Falta de memória.}{es.que.ci.men.to}{0}
\verb{esquelético}{}{}{}{}{adj.}{Relativo ao esqueleto.}{es.que.lé.ti.co}{0}
\verb{esquelético}{}{}{}{}{}{Muito magro.}{es.que.lé.ti.co}{0}
\verb{esqueleto}{ê}{}{}{}{s.m.}{Conjunto de ossos, cartilagens e articulações do corpo dos vertebrados.}{es.que.le.to}{0}
\verb{esqueleto}{ê}{}{}{}{}{Armação, estrutura.}{es.que.le.to}{0}
\verb{esquema}{}{}{}{}{s.m.}{Representação simplificada das relações e funções de objetos.}{es.que.ma}{0}
\verb{esquema}{}{}{}{}{}{Resumo, esboço, plano.}{es.que.ma}{0}
\verb{esquemático}{}{}{}{}{adj.}{Que registra um objeto à maneira de um esquema.}{es.que.má.ti.co}{0}
\verb{esquematizar}{}{}{}{}{v.t.}{Fazer esquema de.}{es.que.ma.ti.zar}{\verboinum{1}}
\verb{esquentado}{}{}{}{}{adj.}{Que se esquentou; aquecido, requentado.}{es.quen.ta.do}{0}
\verb{esquentado}{}{Fig.}{}{}{}{Irritadiço, irascível, exaltado.}{es.quen.ta.do}{0}
\verb{esquentar}{}{}{}{}{v.t.}{Provocar calor em; aquecer.}{es.quen.tar}{0}
\verb{esquentar}{}{}{}{}{}{Irritar, enfurecer.}{es.quen.tar}{\verboinum{1}}
\verb{esquerda}{ê}{}{}{}{s.f.}{O lado oposto ao direito.}{es.quer.da}{0}
\verb{esquerda}{ê}{}{}{}{}{A mão do lado esquerdo.}{es.quer.da}{0}
\verb{esquerda}{ê}{}{}{}{}{O conjunto de indivíduos partidários de ideologia socialista.}{es.quer.da}{0}
\verb{esquerdismo}{}{}{}{}{s.m.}{Tendência política da esquerda.}{es.quer.dis.mo}{0}
\verb{esquerdista}{}{}{}{}{s.2g.}{Indivíduo partidário de ideologia socialista.}{es.quer.dis.ta}{0}
\verb{esquerdo}{ê}{}{}{}{adj.}{Diz"-se daquilo que está do lado oposto ao direito.}{es.quer.do}{0}
\verb{esquete}{é}{}{}{}{s.m.}{Pequena cena em teatro, televisão ou rádio, geralmente de caráter cômico.}{es.que.te}{0}
\verb{esqui}{}{}{}{}{s.m.}{Chapa longa de madeira que se prende aos pés para deslizar sobre neve ou água.}{es.qui}{0}
\verb{esqui}{}{}{}{}{}{O esporte praticado com esquis.}{es.qui}{0}
\verb{esquiar}{}{}{}{}{v.i.}{Praticar o esqui.}{es.qui.ar}{\verboinum{1}}
\verb{esquife}{}{}{}{}{s.m.}{Caixão, ataúde.}{es.qui.fe}{0}
\verb{esquilo}{}{Zool.}{}{}{s.m.}{Animal roedor de cauda longa e peluda, olhos grandes e garras pontiagudas, que vive em árvores e se alimenta de castanhas e sementes.}{es.qui.lo}{0}
\verb{esquimó}{}{}{}{}{adj.2g.}{Relativo aos esquimós, povo originário da Groenlândia e da costa norte da América do Norte.}{es.qui.mó}{0}
\verb{esquimó}{}{}{}{}{s.2g.}{Indivíduo dos esquimós.}{es.qui.mó}{0}
\verb{esquimó}{}{}{}{}{s.m.}{Cada uma das línguas faladas pelos esquimós.}{es.qui.mó}{0}
\verb{esquina}{}{}{}{}{s.f.}{O canto formado por duas ruas que se cruzam.}{es.qui.na}{0}
\verb{esquipático}{}{}{}{}{adj.}{Excêntrico, extravagante, esquisito.}{es.qui.pá.ti.co}{0}
\verb{esquisitice}{}{}{}{}{s.f.}{Qualidade, ato ou dito de quem é esquisito.}{es.qui.si.ti.ce}{0}
\verb{esquisito}{}{}{}{}{adj.}{Fora do comum; estranho, excêntrico.}{es.qui.si.to}{0}
\verb{esquisito}{}{}{}{}{}{Raro, precioso, fino, requintado.}{es.qui.si.to}{0}
\verb{esquistossomo}{}{Zool.}{}{}{s.m.}{Verme encontrado em regiões tropicais, causador da esquistossomose.}{es.quis.tos.so.mo}{0}
\verb{esquistossomose}{}{Med.}{}{}{s.f.}{Doença causada pelo esquistossomo e caracterizada por diarreia e dilatação anormal do baço e do fígado.}{es.quis.tos.so.mo.se}{0}
\verb{esquiva}{}{}{}{}{s.f.}{Ato de desviar o corpo ou parte dele para evitar um golpe.}{es.qui.va}{0}
\verb{esquivança}{}{}{}{}{s.f.}{Ato ou efeito de esquivar.}{es.qui.van.ça}{0}
\verb{esquivança}{}{}{}{}{}{Indisposição para a convivência; insociabilidade.}{es.qui.van.ça}{0}
\verb{esquivar}{}{}{}{}{v.t.}{Desviar o corpo ou parte dele para evitar um golpe.}{es.qui.var}{\verboinum{1}}
\verb{esquivar}{}{}{}{}{}{Evitar, escapar, escapulir, afastar"-se.}{es.qui.var}{0}
\verb{esquivo}{}{}{}{}{adj.}{Que evita a convivência; arredio, insociável, intratável.}{es.qui.vo}{0}
\verb{esquizofrenia}{}{Med.}{}{}{s.f.}{Distúrbio psíquico caracterizado pela dissociação entre ação e pensamento, alucinações e perda de contato com a realidade.}{es.qui.zo.fre.ni.a}{0}
\verb{esquizofrênico}{}{}{}{}{adj.}{Relativo à esquizofrenia.}{es.qui.zo.frê.ni.co}{0}
\verb{esquizofrênico}{}{}{}{}{}{Que sofre de esquizofrenia.}{es.qui.zo.frê.ni.co}{0}
\verb{essa}{é}{}{}{}{pron.}{Forma feminina do pronome demonstrativo \textit{esse}.}{es.sa}{0}
\verb{essa}{é}{}{}{}{s.f.}{Estrado sobre o qual se coloca o caixão durante o funeral.}{es.sa}{0}
\verb{esse}{é}{}{}{}{s.m.}{Nome da letra \textit{s}.}{es.se}{0}
\verb{esse}{ê}{}{}{essa ⟨é⟩}{pron.}{Que está perto da pessoa a quem se fala. (\textit{Esse seu casaco é muito bonito.})}{es.se}{0}
\verb{esse}{ê}{}{}{essa ⟨é⟩}{}{Que foi dito ou escrito pouco antes. (\textit{Sinto saudades suas e é impossível disfarçar esse sentimento.})}{es.se}{0}
\verb{essência}{}{}{}{}{s.f.}{Aquilo que constitui a natureza de uma coisa.}{es.sên.cia}{0}
\verb{essência}{}{}{}{}{}{A ideia principal contida em uma obra, texto ou doutrina.}{es.sên.cia}{0}
\verb{essência}{}{}{}{}{}{Líquido extraído de uma planta ou material, que contém seu aroma ou propriedades úteis para alguma aplicação.}{es.sên.cia}{0}
\verb{essencial}{}{}{"-ais}{}{adj.2g.}{Relativo a essência.}{es.sen.ci.al}{0}
\verb{essencial}{}{}{"-ais}{}{}{Que constitui a essência.}{es.sen.ci.al}{0}
\verb{essencial}{}{}{"-ais}{}{}{Necessário, indispensável.}{es.sen.ci.al}{0}
\verb{essencial}{}{}{"-ais}{}{s.m.}{O ponto mais importante de algo.}{es.sen.ci.al}{0}
\verb{és"-sudeste}{é}{}{és"-sudestes ⟨é⟩}{}{s.m.}{Ponto do horizonte situado entre o este e o sudeste; és"-sueste. Abrev. \textsc{e.s.e}.}{és"-su.des.te}{0}
\verb{és"-sueste}{é}{}{és"-suestes ⟨é⟩}{}{s.m.}{És"-sudeste. Abrev. \textsc{e.s.e}.}{és"-su.es.te}{0}
\verb{esta}{é}{}{}{}{pron.}{Forma feminina do pronome demonstrativo \textit{este}.}{es.ta}{0}
\verb{estabanado}{}{}{}{}{adj.}{Precipitado, imprudente, desajeitado.}{es.ta.ba.na.do}{0}
\verb{estabelecer}{ê}{}{}{}{v.t.}{Tornar estável; fixar.}{es.ta.be.le.cer}{0}
\verb{estabelecer}{ê}{}{}{}{}{Instituir, fundar, criar.}{es.ta.be.le.cer}{0}
\verb{estabelecer}{ê}{}{}{}{}{Determinar, mandar, assentar.}{es.ta.be.le.cer}{\verboinum{15}}
\verb{estabelecido}{}{}{}{}{adj.}{Que se estabeleceu; instituído.}{es.ta.be.le.ci.do}{0}
\verb{estabelecido}{}{}{}{}{}{Que tem estabelecimento profissional próprio.}{es.ta.be.le.ci.do}{0}
\verb{estabelecimento}{}{}{}{}{s.m.}{Ato ou efeito de estabelecer.}{es.ta.be.le.ci.men.to}{0}
\verb{estabelecimento}{}{}{}{}{}{Entidade comercial ou industrial; instituição.}{es.ta.be.le.ci.men.to}{0}
\verb{estabilidade}{}{}{}{}{s.f.}{Qualidade de estável; firmeza.}{es.ta.bi.li.da.de}{0}
\verb{estabilidade}{}{}{}{}{}{Garantia existente em certos cargos, geralmente públicos, de que o ocupante não seja demitido, exceto por ordem judicial ou processo administrativo.}{es.ta.bi.li.da.de}{0}
\verb{estabilização}{}{}{"-ões}{}{s.f.}{Ato ou efeito de estabilizar.}{es.ta.bi.li.za.ção}{0}
\verb{estabilizador}{ô}{Quím.}{}{}{s.m.}{Substância que torna uma solução estável.}{es.ta.bi.li.za.dor}{0}
\verb{estabilizador}{ô}{Fís.}{}{}{}{Dispositivo que torna constante a tensão em um circuito elétrico, protegendo os aparelhos nele ligados dos picos e quedas de tensão.}{es.ta.bi.li.za.dor}{0}
\verb{estabilizar}{}{}{}{}{v.t.}{Tornar estável.}{es.ta.bi.li.zar}{\verboinum{1}}
\verb{estábulo}{}{}{}{}{s.m.}{Área coberta onde se recolhe o gado.}{es.tá.bu.lo}{0}
\verb{estaca}{}{}{}{}{s.f.}{Peça comprida, de madeira ou outro material, fixada no terreno e usada como parte estrutural das edificações, para demarcar terreno ou para servir de apoio para plantas.}{es.ta.ca}{0}
\verb{estacada}{}{}{}{}{s.f.}{Área demarcada por uma série de estacas.}{es.ta.ca.da}{0}
\verb{estação}{}{}{"-ões}{}{s.f.}{Lugar de parada de trens e ônibus.}{es.ta.ção}{0}
\verb{estação}{}{}{"-ões}{}{}{Cada uma das subdivisões do ano, com base em critérios astronômicos ou climáticos.}{es.ta.ção}{0}
\verb{estação}{}{}{"-ões}{}{}{Temporada, período, época.}{es.ta.ção}{0}
\verb{estação}{}{}{"-ões}{}{}{Local equipado com aparelhos de trasmissão e recepção de ondas de rádio e televisão.}{es.ta.ção}{0}
\verb{estacar}{}{}{}{}{v.t.}{Parar, interromper, suspender, imobilizar.}{es.ta.car}{0}
\verb{estacar}{}{Desus.}{}{}{}{Segurar com estacas.}{es.ta.car}{\verboinum{2}}
\verb{estacaria}{}{}{}{}{s.f.}{Conjunto de estacas.}{es.ta.ca.ri.a}{0}
\verb{estacionamento}{}{}{}{}{s.m.}{Lugar onde se estacionam veículos.}{es.ta.ci.o.na.men.to}{0}
\verb{estacionamento}{}{}{}{}{}{Ato de estacionar.}{es.ta.ci.o.na.men.to}{0}
\verb{estacionar}{}{}{}{}{v.t.}{Parar um veículo por um tempo relativamente longo.}{es.ta.ci.o.nar}{0}
\verb{estacionar}{}{}{}{}{}{Parar; permanecer parado; deter"-se.}{es.ta.ci.o.nar}{\verboinum{1}}
\verb{estacionário}{}{}{}{}{adj.}{Que não muda de posição ou de estado; imóvel, parado.}{es.ta.ci.o.ná.rio}{0}
\verb{estada}{}{}{}{}{s.f.}{Ato de estar; permanência.}{es.ta.da}{0}
\verb{estadear}{}{}{}{}{v.t.}{Exibir orgulhosamente; ostentar.}{es.ta.de.ar}{0}
\verb{estadear}{}{}{}{}{}{Tornar público com alarde; anunciar vangloriosamente.}{es.ta.de.ar}{0}
\verb{estadear}{}{}{}{}{v.i.}{Mostrar"-se de forma majestosa.}{es.ta.de.ar}{\verboinum{4}}
\verb{estadia}{}{}{}{}{s.f.}{Tempo que se passa em algum lugar; estada, permanência.}{es.ta.di.a}{0}
\verb{estadia}{}{}{}{}{}{Tempo durante o qual o navio fica no porto para carregar ou descarregar.}{es.ta.di.a}{0}
\verb{estádio}{}{}{}{}{s.m.}{Campo de jogos esportivos.}{es.tá.dio}{0}
\verb{estadista}{}{}{}{}{s.2g.}{Governante e político de atuação notável.}{es.ta.dis.ta}{0}
\verb{estado}{}{}{}{}{s.m.}{Situação em que se acha uma pessoa, um animal ou uma coisa.}{es.ta.do}{0}
\verb{estado}{}{}{}{}{}{Forma organizada de um povo, que mora no mesmo território e obedece às mesmas leis.}{es.ta.do}{0}
\verb{estado}{}{}{}{}{}{Divisão territorial de certos países.}{es.ta.do}{0}
\verb{estado"-maior}{ó}{}{estados"-maiores ⟨ó⟩}{}{s.m.}{Grupo de oficiais que assessoram um comandante no planejamento e no controle de operações militares.  }{es.ta.do"-mai.or}{0}
\verb{estado"-maior}{ó}{Fig.}{estados"-maiores ⟨ó⟩}{}{}{O conjunto das pessoas mais eminentes de um grupo, de uma classe, de uma profissão.}{es.ta.do"-mai.or}{0}
\verb{estadual}{}{}{"-ais}{}{adj.2g.}{Relativo a estado ou estados de uma federação.}{es.ta.du.al}{0}
\verb{estadunidense}{}{}{}{}{adj.2g.}{Relativo aos Estados Unidos; norte"-americano. }{es.ta.du.ni.den.se}{0}
\verb{estadunidense}{}{}{}{}{s.2g.}{Indivíduo natural ou habitante desse país.}{es.ta.du.ni.den.se}{0}
\verb{estafa}{}{}{}{}{s.f.}{Cansaço muito grande.}{es.ta.fa}{0}
\verb{estafante}{}{}{}{}{adj.2g.}{Que causa estafa; fatigante, cansativo.}{es.ta.fan.te}{0}
\verb{estafar}{}{}{}{}{v.t.}{Causar estafa, fadiga.}{es.ta.far}{\verboinum{1}}
\verb{estafermo}{ê}{}{}{}{s.m.}{Indivíduo inútil.}{es.ta.fer.mo}{0}
\verb{estafeta}{ê}{}{}{}{adj.}{Entregador de cartas e telegramas; mensageiro.}{es.ta.fe.ta}{0}
\verb{estafilococo}{}{Biol.}{}{}{s.f.}{Bactéria que se apresenta em aglomerados semelhantes a cachos de uvas, é o mais comum dos agentes purulentos, produzindo abscessos, a furunculose, o antraz, a septicemia, entre outros.}{es.ta.fi.lo.co.co}{0}
\verb{estagflação}{}{}{"-ões}{}{s.f.}{Aumento da taxa de desemprego mais aumento contínuo de preços.}{es.tag.fla.ção}{0}
\verb{estagiar}{}{}{}{}{v.i.}{Fazer estágio em algum lugar.}{es.ta.gi.ar}{\verboinum{6}}
\verb{estagiário}{}{}{}{}{adj.}{Relativo a estágio.}{es.ta.gi.á.rio}{0}
\verb{estagiário}{}{}{}{}{s.m.}{Indivíduo que faz estágio.}{es.ta.gi.á.rio}{0}
\verb{estágio}{}{}{}{}{s.m.}{Prática de uma profissão para aprendizagem ou aperfeiçoamento. }{es.tá.gio}{0}
\verb{estagnação}{}{}{"-ões}{}{s.f.}{Estado do que se encontra estagnado, sem fluir.}{es.tag.na.ção}{0}
\verb{estagnação}{}{Fig.}{"-ões}{}{}{Falta de progresso, de movimento, de atividade; paralisação.}{es.tag.na.ção}{0}
\verb{estagnação}{}{}{"-ões}{}{}{Situação em que o produto nacional não cresce à altura do potencial econômico do país.}{es.tag.na.ção}{0}
\verb{estagnar}{}{}{}{}{v.t.}{Fazer parar de fluir; estancar.}{es.tag.nar}{0}
\verb{estagnar}{}{Fig.}{}{}{}{Fazer cessar o progresso ou o funcionamento de; paralisar.}{es.tag.nar}{\verboinum{1}}
\verb{estalactite}{}{}{}{}{s.f.}{Coluna calcária que se forma a partir do teto de uma caverna.}{es.ta.lac.ti.te}{0}
\verb{estalada}{}{}{}{}{s.f.}{Som produzido por algo que estala.}{es.ta.la.da}{0}
\verb{estalada}{}{Por ext.}{}{}{}{Grande ruído ou barulho.}{es.ta.la.da}{0}
\verb{estalada}{}{Fig.}{}{}{}{Motim, desordem.}{es.ta.la.da}{0}
\verb{estalagem}{}{}{"-ens}{}{s.f.}{Hospedaria, pousada.}{es.ta.la.gem}{0}
\verb{estalagmite}{}{}{}{}{s.f.}{Coluna calcária que se forma a partir do chão de uma caverna e aponta para o teto.}{es.ta.lag.mi.te}{0}
\verb{estalajadeiro}{ê}{}{}{}{s.m.}{Dono ou administrador de estalagem.}{es.ta.la.ja.dei.ro}{0}
\verb{estalão}{}{}{"-ões}{}{s.m.}{Padrão, medida, craveira.}{es.ta.lão}{0}
\verb{estalar}{}{}{}{}{v.t.}{Produzir estalido ou ruído.}{es.ta.lar}{0}
\verb{estalar}{}{Bras.}{}{}{}{Fritar ovo com a clara e a gema juntas e inteiriças; estrelar.}{es.ta.lar}{0}
\verb{estalar}{}{}{}{}{v.i.}{Rachar, fender"-se.}{es.ta.lar}{0}
\verb{estalar}{}{}{}{}{}{Estourar, explodir.}{es.ta.lar}{\verboinum{1}}
\verb{estaleiro}{ê}{}{}{}{s.m.}{Lugar onde se constroem e consertam navios.}{es.ta.lei.ro}{0}
\verb{estalido}{}{}{}{}{s.m.}{Som breve e seco; estalo.}{es.ta.li.do}{0}
\verb{estalido}{}{}{}{}{}{Estouro.}{es.ta.li.do}{0}
\verb{estalo}{}{}{}{}{s.m.}{Ruído súbito; som de coisa quebrando ou rachando; crepitação.}{es.ta.lo}{0}
\verb{estame}{}{Bot.}{}{}{s.m.}{Parte da flor onde fica o pólen.}{es.ta.me}{0}
\verb{estampa}{}{}{}{}{s.f.}{Imagem impressa; figura.}{es.tam.pa}{0}
\verb{estampado}{}{}{}{}{adj.}{Que se estampou; gravado.}{es.tam.pa.do}{0}
\verb{estampado}{}{}{}{}{}{Que se publicou.}{es.tam.pa.do}{0}
\verb{estampado}{}{Bras.}{}{}{}{Tecido em que se estamparam padrões decorativos.}{es.tam.pa.do}{0}
\verb{estampador}{ô}{}{}{}{adj.}{Diz"-se de indivíduo cujo ofício é estampagem de gravuras, tecidos, porcelanas, entre outros.}{es.tam.pa.dor}{0}
\verb{estampagem}{}{}{"-ens}{}{s.f.}{Ato ou efeito de estampar; impressão.}{es.tam.pa.gem}{0}
\verb{estampar}{}{}{}{}{v.t.}{Imprimir letras ou desenhos sobre a superfície de alguma coisa.}{es.tam.par}{\verboinum{1}}
\verb{estamparia}{}{}{}{}{s.f.}{Fábrica, depósito ou loja de estampas.}{es.tam.pa.ri.a}{0}
\verb{estampido}{}{}{}{}{s.m.}{Barulho forte da arma de fogo que dispara.}{es.tam.pi.do}{0}
\verb{estampilha}{}{}{}{}{s.f.}{Pequena estampa.}{es.tam.pi.lha}{0}
\verb{estampilha}{}{}{}{}{}{Selo fiscal.}{es.tam.pi.lha}{0}
\verb{estampilha}{}{}{}{}{}{Selo que se coloca nos documentos forenses.}{es.tam.pi.lha}{0}
\verb{estampilhado}{}{}{}{}{adj.}{Selado com estampilha.}{es.tam.pi.lha.do}{0}
\verb{estampilhar}{}{}{}{}{v.t.}{Pôr estampilha; selar ou franquear com estampilha.}{es.tam.pi.lhar}{\verboinum{1}}
\verb{estancar}{}{}{}{}{v.t.}{Fazer (um líquido) parar de correr. }{es.tan.car}{0}
\verb{estancar}{}{}{}{}{}{Fazer parar (uma ação); cessar, deter.}{es.tan.car}{\verboinum{2}}
\verb{estância}{}{}{}{}{s.f.}{Fazenda em que se fazem plantações e se cria gado.}{es.tân.cia}{0}
\verb{estância}{}{}{}{}{}{Lugar com águas usadas para fins medicinais; estação de águas.}{es.tân.cia}{0}
\verb{estância}{}{}{}{}{}{Cada grupo de versos de um poema; estrofe.}{es.tân.cia}{0}
\verb{estancieiro}{ê}{}{}{}{s.m.}{Proprietário de fazenda.}{es.tan.ci.ei.ro}{0}
\verb{estandardizar}{}{}{}{}{v.t.}{Padronizar.}{es.tan.dar.di.zar}{\verboinum{1}}
\verb{estandarte}{}{}{}{}{s.m.}{Bandeira de guerra.}{es.tan.dar.te}{0}
\verb{estandarte}{}{}{}{}{}{Pedaço de pano, de uma ou mais cores, que serve de símbolo para uma comunidade; bandeira.}{es.tan.dar.te}{0}
\verb{estande}{}{}{}{}{s.m.}{Compartimento em feira ou exposição.}{es.tan.de}{0}
\verb{estande}{}{}{}{}{}{Local para tiro ao alvo.}{es.tan.de}{0}
\verb{estanhagem}{}{}{"-ens}{}{s.f.}{Ato ou efeito de revestir com camada de estanho.}{es.ta.nha.gem}{0}
\verb{estanhar}{}{}{}{}{v.t.}{Cobrir com camada de estanho.}{es.ta.nhar}{\verboinum{1}}
\verb{estanho}{}{Quím.}{}{}{s.m.}{Elemento químico metálico, maleável, prateado, usado como revestimento protetor para outros metais e em ligas com o bronze e o cobre ou com o chumbo e o zinco. \elemento{50}{118.71}{Sn}.}{es.ta.nho}{0}
\verb{estanque}{}{}{}{}{s.m.}{Ato ou efeito de estancar; estancamento.}{es.tan.que}{0}
\verb{estanque}{}{}{}{}{adj.2g.}{Que se estancou, se esgotou ou esvaziou; seco, enxuto.}{es.tan.que}{0}
\verb{estanque}{}{}{}{}{}{Sem furo, abertura ou fenda por onde entre ou escorra um líquido; vedado, tapado, selado, impermeabilizado.}{es.tan.que}{0}
\verb{estante}{}{}{}{}{s.f.}{Móvel com prateleiras para guardar livros, papéis, entre outros.}{es.tan.te}{0}
\verb{estante}{}{}{}{}{}{Acessório portátil, dobrável, com um suporte inclinado que serve para colocar as partituras musicais de um executante.}{es.tan.te}{0}
\verb{estapafúrdio}{}{}{}{}{adj.}{Que é extravagante, fora do comum (pessoa ou coisa); estranho, esquisito, bizarro, singular, esdrúxulo.}{es.ta.pa.fúr.dio}{0}
\verb{estapear}{}{}{}{}{v.t.}{Dar tapas em; esbofetear.}{es.ta.pe.ar}{\verboinum{4}}
\verb{estaquear}{}{}{}{}{v.t.}{Segurar, firmar com estacas; estacar; firmar.}{es.ta.que.ar}{\verboinum{4}}
\verb{estar}{}{}{}{}{v.pred.}{Passar um certo tempo em determinada condição física, emocional, material, entre outras. (\textit{Ela estava alegre por ter reencontrado a prima após tantos anos.})}{es.tar}{0}
\verb{estar}{}{}{}{}{}{Achar"-se, encontrar"-se em algum lugar. (\textit{Ela esteve comigo no hospital.})}{es.tar}{0}
\verb{estar}{}{}{}{}{}{Ficar, permanecer em determinada posição. (\textit{Nós estivemos sentados o dia inteiro esperando um comunicado da diretoria.})}{es.tar}{\verboinum{36}}
\verb{estardalhaço}{}{}{}{}{s.m.}{Gritaria, ruído, bulha, estrondo.}{es.tar.da.lha.ço}{0}
\verb{estardalhaço}{}{Fig.}{}{}{}{Ostentação espalhafatosa.}{es.tar.da.lha.ço}{0}
\verb{estarrecer}{ê}{}{}{}{v.t.}{Causar espanto, pavor ou susto; assombrar, aterrorizar, aterrar.}{es.tar.re.cer}{\verboinum{15}}
\verb{estatal}{}{}{"-ais}{}{adj.2g.}{Relativo ou pertencente ao Estado (nação).}{es.ta.tal}{0}
\verb{estatelar}{}{}{}{}{v.t.}{Estender ou atirar no chão; fazer cair por terra.}{es.ta.te.lar}{0}
\verb{estatelar}{}{}{}{}{v.pron.}{Escorregar, cair de chapa.}{es.ta.te.lar}{\verboinum{1}}
\verb{estática}{}{Fís.}{}{}{s.f.}{Ramo da mecânica que estuda o equilíbrio dos corpos sob a ação de forças.}{es.tá.ti.ca}{0}
\verb{estática}{}{}{}{}{}{Interferência, que se manifesta na forma de ruído, nos aparelhos de rádio, devida à eletricidade atmosférica.}{es.tá.ti.ca}{0}
\verb{estático}{}{}{}{}{adj.}{Que não se move, que está em repouso; parado, imóvel.}{es.tá.ti.co}{0}
\verb{estatística}{}{Mat.}{}{}{s.f.}{Ramo da matemática aplicada que tem por finalidade agrupar metodicamente e estudar séries de fatos ou de dados numéricos relacionados com a sociedade em geral.}{es.ta.tís.ti.ca}{0}
\verb{estatística}{}{}{}{}{}{O conjunto metódico de números referentes a fatos sociais, econômicos, entre outros, para avaliação e estudo.}{es.ta.tís.ti.ca}{0}
\verb{estatístico}{}{}{}{}{adj.}{Relativo a estatística.}{es.ta.tís.ti.co}{0}
\verb{estatístico}{}{}{}{}{s.m.}{Pessoa especializada em estatística.}{es.ta.tís.ti.co}{0}
\verb{estatização}{}{}{"-ões}{}{s.f.}{Ato ou efeito de estatizar, de o Estado apropriar"-se de empresa ou instituição particular.}{es.ta.ti.za.ção}{0}
\verb{estatizar}{}{}{}{}{v.t.}{Passar uma empresa ou instituição particular para o domínio do Estado.}{es.ta.ti.zar}{0}
\verb{estatizar}{}{}{}{}{}{Reservar (um ramo de atividade, recurso etc.) à exploração estatal apenas.}{es.ta.ti.zar}{\verboinum{1}}
\verb{estátua}{}{}{}{}{s.f.}{Escultura em três dimensões, que representa uma figura (homem, divindade, animal etc.).}{es.tá.tu.a}{0}
\verb{estátua}{}{Fig.}{}{}{}{Pessoa que fica parada, sem ação, impassível.}{es.tá.tu.a}{0}
\verb{estatuaria}{}{}{}{}{s.f.}{Coleção de estátuas.}{es.ta.tu.a.ri.a}{0}
\verb{estatuária}{}{}{}{}{s.f.}{Arte de criar ou fazer estátuas; escultura.}{es.ta.tu.á.ria}{0}
\verb{estatuário}{}{}{}{}{adj.}{Relativo a estátuas. }{es.ta.tu.á.rio}{0}
\verb{estatuário}{}{}{}{}{s.m.}{Pessoa que faz estátuas; escultor.	}{es.ta.tu.á.rio}{0}
\verb{estatueta}{ê}{}{}{}{s.f.}{Estátua pequena, usada em decoração.}{es.ta.tu.e.ta}{0}
\verb{estatuir}{}{}{}{}{v.t.}{Determinar por meio de estatuto, lei, decreto; decretar, deliberar, estabelecer.}{es.ta.tu.ir}{\verboinum{26}}
\verb{estatura}{}{}{}{}{s.f.}{Tamanho, altura de uma pessoa.}{es.ta.tu.ra}{0}
\verb{estatutário}{}{}{}{}{adj.}{Relativo a estatuto.}{es.ta.tu.tá.rio}{0}
\verb{estatuto}{}{}{}{}{s.m.}{Conjunto de regras de uma organização, corporação, sociedade, associação etc.}{es.ta.tu.to}{0}
\verb{estatuto}{}{Por ext.}{}{}{}{Regimento, regulamento, regra, ordenação.}{es.ta.tu.to}{0}
\verb{estável}{}{}{"-eis}{}{adj.2g.}{Que tem estabilidade; fixo, firme, seguro.}{es.tá.vel}{0}
\verb{estável}{}{}{"-eis}{}{}{Que não muda, não varia; inalterável, invariável, imutável. }{es.tá.vel}{0}
\verb{estável}{}{}{"-eis}{}{}{Que é duradouro; permanente, constante.}{es.tá.vel}{0}
\verb{estável}{}{Jur.}{"-eis}{}{}{Diz"-se de funcionário que, após determinado período, só pode ser despedido por justa causa.}{es.tá.vel}{0}
\verb{este}{é}{}{}{}{s.m.}{Ponto cardeal situado à direita do observador voltado para o norte; leste. Símb.: \textsc{l}. }{es.te}{0}
\verb{este}{é}{}{}{}{}{O vento que sopra do este.}{es.te}{0}
\verb{este}{ê}{}{}{}{}{Designa a pessoa ou coisa a que por último foi mencionada. (\textit{Participaram da disputa meninas e meninos; estes saíram vencedores.})}{es.te}{0}
\verb{este}{ê}{}{}{}{}{Designa o que vai ser dito ou escrito em seguida. (\textit{Ouça esta novidade: voltei a estudar!})}{es.te}{0}
\verb{este}{ê}{}{}{}{pron.}{Designa a pessoa ou coisa presente e próxima de quem fala. (\textit{Esta blusa que estou usando foi um presente da minha mãe.})}{es.te}{0}
\verb{estear}{}{}{}{}{v.t.}{Sustentar, segurar com esteio ou escora; escorar.}{es.te.ar}{0}
\verb{estear}{}{Fig.}{}{}{}{Amparar, sustentar, proteger.}{es.te.ar}{\verboinum{4}}
\verb{estearina}{}{Quím.}{}{}{s.f.}{Substância encontrada nas gorduras animais e vegetais mais duras, tais como o sebo e a manteiga de cacau, e usada em cosméticos, na fabricação de velas etc.}{es.te.a.ri.na}{0}
\verb{esteio}{ê}{}{}{}{s.m.}{Peça de metal, madeira, plástico etc. que serve para segurar ou escorar alguma coisa; escora.}{es.tei.o}{0}
\verb{esteio}{ê}{Fig.}{}{}{}{Amparo, apoio, proteção, arrimo.}{es.tei.o}{0}
\verb{esteira}{ê}{}{}{}{s.f.}{Tecido de junco, de palha trançada ou de outro material, usado como tapete, para dormir etc.}{es.tei.ra}{0}
\verb{esteira}{ê}{Bras.}{}{}{}{Tapete rolante, movido por engrenagens, usado para carregar pessoas, objetos etc.}{es.tei.ra}{0}
\verb{esteiro}{ê}{}{}{}{s.m.}{Parte estreita de rio ou de mar, que adentra a terra; estuário. }{es.tei.ro}{0}
\verb{estelar}{}{}{}{}{adj.2g.}{Relativo a estrelas.}{es.te.lar}{0}
\verb{estelionatário}{}{}{}{}{s.m.}{Pessoa que pratica estelionato.}{es.te.li.o.na.tá.rio}{0}
\verb{estelionato}{}{}{}{}{s.m.}{Tipo de crime em que se obtém, para si ou para outra pessoa, vantagem ilícita, em prejuízo alheio, induzindo em erro a vítima por meio de ardil ou fraude, como emitir cheque sem fundos, vender coisas alheias etc.}{es.te.li.o.na.to}{0}
\verb{estêncil}{}{}{"-eis}{}{s.m.}{Material (papel, metal, cera etc.) perfurado com um desenho ou um texto, usado para imprimir sobre uma superfície.}{es.tên.cil}{0}
\verb{estendal}{}{}{"-ais}{}{s.m.}{Lugar onde se penduram coisas para secar; varal, quarador, estendedouro.}{es.ten.dal}{0}
\verb{estendedouro}{ô}{}{}{}{s.m.}{Lugar onde se estende alguma coisa para secar; varal, estendal.}{es.ten.de.dou.ro}{0}
\verb{estender}{ê}{}{}{}{v.t.}{Colocar alguma coisa em todo o seu comprimento; esticar, alongar, estirar.}{es.ten.der}{0}
\verb{estender}{ê}{}{}{}{}{Desdobrar, desenrolar.}{es.ten.der}{0}
\verb{estender}{ê}{}{}{}{v.pron.}{Prolongar"-se, demorar"-se.}{es.ten.der}{\verboinum{12}}
\verb{estenodatilografia}{}{}{}{}{s.f.}{Sistema de escrita que combina a estenografia e a datilografia.}{es.te.no.da.ti.lo.gra.fi.a}{0}
\verb{estenografar}{}{}{}{}{v.t.}{Escrever abreviadamente, usando caracteres da estenografia; taquigrafar.}{es.te.no.gra.far}{\verboinum{1}}
\verb{estenografia}{}{}{}{}{s.f.}{Escrita abreviada e simplificada, na qual se usam sinais que permitem escrever com a mesma rapidez com que as palavras são pronunciadas; taquigrafia.}{es.te.no.gra.fi.a}{0}
\verb{estenógrafo}{}{}{}{}{s.m.}{Pessoa que pratica a estenografia profissionalmente; taquígrafo. }{es.te.nó.gra.fo}{0}
\verb{estentor}{ô}{}{}{}{s.m.}{Pessoa que tem a voz possante.}{es.ten.tor}{0}
\verb{estentor}{ô}{Por ext.}{}{}{}{Voz muito forte, possante.}{es.ten.tor}{0}
\verb{estentóreo}{}{}{}{}{adj.}{Relativo a estentor.}{es.ten.tó.re.o}{0}
\verb{estentóreo}{}{}{}{}{}{Que tem a voz possante, muito vigorosa.}{es.ten.tó.re.o}{0}
\verb{estepe}{é}{Geogr.}{}{}{s.f.}{Formação vegetal descontínua caracterizada pela predominância de plantas herbáceas.}{es.te.pe}{0}
\verb{estepe}{é}{Bras.}{}{}{s.m.}{Pneu sobressalente.}{es.te.pe}{0}
\verb{éster}{}{Quím.}{}{}{s.m.}{Classe de compostos orgânicos derivados da reação de um ácido orgânico com um álcool.}{és.ter}{0}
\verb{estercar}{}{}{}{}{v.t.}{Adubar, fertilizar a terra com esterco; estrumar.}{es.ter.car}{\verboinum{2}}
\verb{esterçar}{}{}{}{}{v.t.}{Girar o volante de um veículo para a direita ou para a esquerda.}{es.ter.çar}{\verboinum{3}}
\verb{esterco}{ê}{}{}{}{s.m.}{Excremento de animal, usado para fertilizar a terra.}{es.ter.co}{0}
\verb{esterco}{ê}{}{}{}{}{Adubo; estrume.}{es.ter.co}{0}
\verb{estéreo}{}{}{}{}{adj.}{Estereofônico.}{es.té.re.o}{0}
\verb{estéreo}{}{Por ext.}{}{}{s.m.}{Qualquer aparelho de som estereofônico.}{es.té.re.o}{0}
\verb{estereofonia}{}{}{}{}{s.f.}{Qualidade de estereofônico.}{es.te.re.o.fo.ni.a}{0}
\verb{estereofônico}{}{}{}{}{adj.}{Diz"-se de sistema acústico que dá a sensação de que o som está distribuído espacialmente; estéreo.}{es.te.re.o.fô.ni.co}{0}
\verb{estereoscópio}{}{}{}{}{s.m.}{Instrumento binocular que dá a sensação de relevo a uma imagem observada.}{es.te.re.os.có.pio}{0}
\verb{estereotipado}{}{}{}{}{adj.}{Que se estereotipou.}{es.te.re.o.ti.pa.do}{0}
\verb{estereotipado}{}{Fig.}{}{}{}{Que não tem originalidade, que não é autêntico.}{es.te.re.o.ti.pa.do}{0}
\verb{estereotipado}{}{}{}{}{}{Que não muda; fixo, inalterável, invariável.}{es.te.re.o.ti.pa.do}{0}
\verb{estereotipar}{}{}{}{}{v.t.}{Imprimir pelo processo de estereotipia.}{es.te.re.o.ti.par}{0}
\verb{estereotipar}{}{}{}{}{}{Tornar fixo, inalterável.}{es.te.re.o.ti.par}{0}
\verb{estereotipar}{}{Fig.}{}{}{}{Formar uma imagem, ideia ou opinião preconcebida sobre alguém ou alguma coisa.}{es.te.re.o.ti.par}{\verboinum{1}}
\verb{estereotipia}{}{}{}{}{s.f.}{Processo que permite a duplicação de uma composição tipográfica a partir de uma matriz.}{es.te.re.o.ti.pi.a}{0}
\verb{estereotipia}{}{}{}{}{}{O clichê conseguido a partir desse processo.}{es.te.re.o.ti.pi.a}{0}
\verb{estereótipo}{}{}{}{}{s.m.}{Impresso por estereotipia.}{es.te.re.ó.ti.po}{0}
\verb{estereótipo}{}{Fig.}{}{}{}{Lugar"-comum, clichê, chavão.}{es.te.re.ó.ti.po}{0}
\verb{estéril}{}{}{}{}{adj.2g.}{Que não produz, não dá frutos ou não procria; improdutivo, infecundo, árido, infértil.}{es.té.ril}{0}
\verb{estéril}{}{}{}{}{}{Que está livre de germes; asséptico.}{es.té.ril}{0}
\verb{estéril}{}{Fig.}{}{}{}{Que não tem criatividade ou imaginação.}{es.té.ril}{0}
\verb{esterilidade}{}{}{}{}{s.f.}{Qualidade ou condição do que é estéril; improdutividade, infecundidade.  }{es.te.ri.li.da.de}{0}
\verb{esterilidade}{}{Med.}{}{}{}{Incapacidade da mulher de conceber, ou do homem de fecundar.}{es.te.ri.li.da.de}{0}
\verb{esterilização}{}{}{"-ões}{}{s.f.}{Ato ou efeito de esterilizar.}{es.te.ri.li.za.ção}{0}
\verb{esterilizador}{}{}{}{}{adj.}{Que esteriliza; esterilizante.  }{es.te.ri.li.za.dor}{0}
\verb{esterilizador}{}{}{}{}{s.m.}{Aparelho para esterilizar, eliminar germes de objetos, ambientes, entre outros.}{es.te.ri.li.za.dor}{0}
\verb{esterilizante}{}{}{}{}{adj.2g.}{Esterilizador.}{es.te.ri.li.zan.te}{0}
\verb{esterilizar}{}{}{}{}{v.t.}{Tornar estéril, improdutivo ou infecundo (solo, animal, planta).}{es.te.ri.li.zar}{0}
\verb{esterilizar}{}{}{}{}{}{Eliminar os germes.}{es.te.ri.li.zar}{\verboinum{1}}
\verb{esterlino}{}{}{}{}{adj.}{Relativo a libra (moeda inglesa).}{es.ter.li.no}{0}
\verb{esterno}{é}{Anat.}{}{}{s.m.}{Osso que fica na parte média e anterior do peito, e que articula as costelas.}{es.ter.no}{0}
\verb{esterqueira}{ê}{}{}{}{s.f.}{Lugar onde se coloca ou se junta esterco; estrumeira.}{es.ter.quei.ra}{0}
\verb{estertor}{}{}{}{}{s.m.}{Respiração anormal, rouca e entrecortada, típica dos moribundos e dos que sofrem de doenças respiratórias.}{es.ter.tor}{0}
\verb{estertorar}{}{}{}{}{v.i.}{Estar em estertor, respirar com grande dificuldade; agonizar.}{es.ter.to.rar}{\verboinum{1}}
\verb{esteta}{é}{}{}{}{s.2g.}{Pessoa que se dedica à fruição e ao culto do belo, especialmente nas artes, colocando os valores estéticos acima de todos os outros.}{es.te.ta}{0}
\verb{esteta}{é}{}{}{}{}{Pessoa versada ou especialista em estética.}{es.te.ta}{0}
\verb{estética}{}{Filos.}{}{}{s.f.}{Parte da filosofia que estuda o belo e as obras de arte.}{es.té.ti.ca}{0}
\verb{estética}{}{}{}{}{}{Beleza física; plástica.}{es.té.ti.ca}{0}
\verb{esteticista}{}{}{}{}{adj.2g.}{Que se refere ao esteticismo.  }{es.te.ti.cis.ta}{0}
\verb{esteticista}{}{}{}{}{s.2g.}{Pessoa adepta do esteticismo nas artes, na literatura etc.}{es.te.ti.cis.ta}{0}
\verb{esteticista}{}{Bras.}{}{}{}{Profissional especialista em tratamentos de beleza, como maquiagem, limpeza de pele, penteado, emagrecimento, entre outros.}{es.te.ti.cis.ta}{0}
\verb{estético}{}{}{}{}{adj.}{Que se refere à estética, ao sentimento e à fruição do belo.}{es.té.ti.co}{0}
\verb{estético}{}{}{}{}{}{Harmonioso, belo, elegante, de bom gosto.}{es.té.ti.co}{0}
\verb{estetoscópio}{}{Med.}{}{}{s.m.}{Aparelho usado para auscultar os barulhos do organismo, como os dos pulmões, coração, entre outros, para diagnosticar doenças.}{es.te.tos.có.pio}{0}
\verb{estévia}{}{}{}{}{s.f.}{Planta largamente distribuída pelas regiões tropicais, usada na produção de adoçantes.}{es.té.via}{0}
\verb{estiada}{}{}{}{}{s.f.}{Estiagem.}{es.ti.a.da}{0}
\verb{estiagem}{}{}{}{}{s.f.}{Tempo seco e sereno, após período chuvoso.}{es.ti.a.gem}{0}
\verb{estiagem}{}{}{}{}{}{Falta de chuva; seca.}{es.ti.a.gem}{0}
\verb{estiar}{}{}{}{}{v.i.}{Cessar (a chuva); serenar (o tempo).}{es.ti.ar}{\verboinum{1}}
\verb{estibordo}{ó}{}{}{}{s.m.}{O lado direito da embarcação, de quem olha da popa para a proa; boreste.}{es.ti.bor.do}{0}
\verb{esticada}{}{Bras.}{}{}{s.f.}{Ato ou efeito de esticar; esticadela, esticamento, espichada.}{es.ti.ca.da}{0}
\verb{esticada}{}{}{}{}{}{Continuação de uma festa, reunião, viagem, em outro lugar.}{es.ti.ca.da}{0}
\verb{esticado}{}{}{}{}{adj.}{Que se esticou; liso, estendido, retesado.}{es.ti.ca.do}{0}
\verb{esticado}{}{Fig.}{}{}{}{Vestido com apuro.}{es.ti.ca.do}{0}
\verb{esticador}{ô}{}{}{}{adj.}{Que estica.}{es.ti.ca.dor}{0}
\verb{esticador}{ô}{Bras.}{}{}{s.m.}{Mourão principal, que serve para esticar os fios de arame de uma cerca. }{es.ti.ca.dor}{0}
\verb{esticar}{}{}{}{}{v.t.}{Estender fortemente; retesar, estirar.}{es.ti.car}{\verboinum{2}}
\verb{estigma}{}{}{}{}{s.m.}{Cada uma das cinco chagas de Cristo.}{es.tig.ma}{0}
\verb{estigma}{}{}{}{}{}{Marca, sinal, cicatriz.}{es.tig.ma}{0}
\verb{estigma}{}{Bot.}{}{}{}{Porção final do gineceu, responsável pela produção de uma substância doce e pegajosa a qual se aderem os grãos de pólen, que aí germinam. }{es.tig.ma}{0}
\verb{estigmatizar}{}{}{}{}{v.t.}{Marcar com ferro em brasa um estigma (sinal).}{es.tig.ma.ti.zar}{0}
\verb{estigmatizar}{}{}{}{}{}{Acusar alguém de um ato vil; censurar, condenar.}{es.tig.ma.ti.zar}{\verboinum{1}}
\verb{estilete}{ê}{}{}{}{s.m.}{Pequeno punhal de lâmina fina e longa, usado para cortar papel, plástico, couro, entre outros.}{es.ti.le.te}{0}
\verb{estilete}{ê}{Bot.}{}{}{}{Parte do pistilo que liga o interior do ovário ao estigma.}{es.ti.le.te}{0}
\verb{estilha}{}{}{}{}{s.f.}{Qualquer lasca de madeira ou de outro material; pedaço, cavaco, fragmento.}{es.ti.lha}{0}
\verb{estilhaçar}{}{}{}{}{v.t.}{Partir em estilhaços ou pedaços; despedaçar, fragmentar.}{es.ti.lha.çar}{\verboinum{3}}
\verb{estilhaço}{}{}{}{}{s.m.}{Pedaço de qualquer coisa quebrada ou partida com violência.}{es.ti.lha.ço}{0}
\verb{estilhaço}{}{}{}{}{}{Lasca ou pedaço de qualquer coisa; fragmento.}{es.ti.lha.ço}{0}
\verb{estilingue}{}{}{}{}{s.m.}{Brinquedo infantil  que consiste numa forquilha de madeira que prende dois elásticos unidos por um pedaço de couro em que se colocam pequenas pedras para serem atiradas; atiradeira, bodoque.}{es.ti.lin.gue}{0}
\verb{estilista}{}{}{}{}{s.2g.}{Pessoa que cria e desenha modelos de roupas e acessórios, geralmente de alta costura.}{es.ti.lis.ta}{0}
\verb{estilista}{}{}{}{}{}{Pessoa que escreve com apuro ou com estilo próprio.}{es.ti.lis.ta}{0}
\verb{estilística}{}{}{}{}{s.f.}{Disciplina que estuda a expressividade de uma língua e de seus recursos de estilo. }{es.ti.lís.ti.ca}{0}
\verb{estilístico}{}{}{}{}{adj.}{Relativo a ou próprio da estilística, ou do estilo. }{es.ti.lís.ti.co}{0}
\verb{estilizado}{}{}{}{}{adj.}{Que passou por estilização.}{es.ti.li.za.do}{0}
\verb{estilizado}{}{}{}{}{}{Que foi representado por meio de símbolos. }{es.ti.li.za.do}{0}
\verb{estilizar}{}{}{}{}{v.t.}{Dar aspecto estético a uma forma natural modificando"-a com adornos, desenhos etc.}{es.ti.li.zar}{\verboinum{1}}
\verb{estilo}{}{}{}{}{s.m.}{Maneira particular de se expressar, de se vestir, de viver.}{es.ti.lo}{0}
\verb{estilo}{}{}{}{}{}{Tendência artística.}{es.ti.lo}{0}
\verb{estilo}{}{}{}{}{}{Elegância.}{es.ti.lo}{0}
\verb{estilo}{}{}{}{}{}{Haste com que os antigos escreviam em tábuas cobertas de cera.}{es.ti.lo}{0}
\verb{estima}{}{}{}{}{s.f.}{Sentimento de amizade ou afeição.}{es.ti.ma}{0}
\verb{estima}{}{}{}{}{}{Cálculo aproximado; estimativa.}{es.ti.ma}{0}
\verb{estimação}{}{}{"-ões}{}{s.f.}{Estima.}{es.ti.ma.ção}{0}
\verb{estimado}{}{}{}{}{adj.}{Que se estima; prezado.}{es.ti.ma.do}{0}
\verb{estimado}{}{}{}{}{}{Avaliado.}{es.ti.ma.do}{0}
\verb{estimar}{}{}{}{}{v.t.}{Ter estima por; apreciar, gostar.}{es.ti.mar}{0}
\verb{estimar}{}{}{}{}{}{Desejar alguma coisa de bom para alguém.}{es.ti.mar}{0}
\verb{estimar}{}{}{}{}{}{Fazer o cálculo do valor ou da quantidade de alguma coisa; avaliar, calcular.}{es.ti.mar}{0}
\verb{estimar}{}{}{}{}{}{Fazer julgamento a respeito de algo com base nas evidências existentes; achar, julgar.}{es.ti.mar}{\verboinum{1}}
\verb{estimativa}{}{}{}{}{s.f.}{Ato de fazer o cálculo de alguma coisa; avaliação.}{es.ti.ma.ti.va}{0}
\verb{estimativo}{}{}{}{}{adj.}{Relativo a estima.}{es.ti.ma.ti.vo}{0}
\verb{estimativo}{}{}{}{}{}{Que constitui uma estimativa, sobre o qual se faz uma avaliação aproximada.}{es.ti.ma.ti.vo}{0}
\verb{estimativo}{}{}{}{}{}{Conjecturado com base em evidências.}{es.ti.ma.ti.vo}{0}
\verb{estimável}{}{}{"-eis}{}{adj.2g.}{Que é merecedor de estima, de apreço.}{es.ti.má.vel}{0}
\verb{estimável}{}{}{"-eis}{}{}{Que se pode estimar, calcular.}{es.ti.má.vel}{0}
\verb{estimulante}{}{}{}{}{adj.2g.}{Que estimula ou incentiva.}{es.ti.mu.lan.te}{0}
\verb{estimulante}{}{}{}{}{s.m.}{Substância que ativa uma função física ou psíquica.}{es.ti.mu.lan.te}{0}
\verb{estimular}{}{}{}{}{v.t.}{Dar incentivo; despertar o ânimo, o interesse, o brio; encorajar, incentivar.}{es.ti.mu.lar}{0}
\verb{estimular}{}{}{}{}{}{Empenhar"-se para que algo seja criado, realizado ou intensificado; impulsionar, promover.}{es.ti.mu.lar}{0}
\verb{estimular}{}{}{}{}{}{Submeter"-se à ação de um estímulo; ativar, excitar.}{es.ti.mu.lar}{0}
\verb{estimular}{}{}{}{}{}{Picar animal com aguilhão ou aguilhada, para incitá"-lo.}{es.ti.mu.lar}{\verboinum{1}}
\verb{estímulo}{}{}{}{}{s.m.}{O que incita à atividade; incentivo.}{es.tí.mu.lo}{0}
\verb{estímulo}{}{}{}{}{}{O que ativa uma função do organismo.}{es.tí.mu.lo}{0}
\verb{estio}{}{}{}{}{s.m.}{A estação mais quente do ano; verão.}{es.ti.o}{0}
\verb{estiolamento}{}{}{}{}{s.m.}{Ato ou efeito de estiolar.}{es.ti.o.la.men.to}{0}
\verb{estiolamento}{}{}{}{}{}{Definhamento, enfraquecimento, debilidade.}{es.ti.o.la.men.to}{0}
\verb{estiolar}{}{}{}{}{v.t.}{Provocar definhamento (planta).}{es.ti.o.lar}{0}
\verb{estiolar}{}{}{}{}{}{Enfraquecer.}{es.ti.o.lar}{\verboinum{1}}
\verb{estipe}{}{}{}{}{s.m.}{Caule sem galhos, marcado de riscas, como o das palmeiras.}{es.ti.pe}{0}
\verb{estipendiar}{}{}{}{}{v.t.}{Dar estipêndio; assalariar.}{es.ti.pen.di.ar}{\verboinum{6}}
\verb{estipêndio}{}{}{}{}{s.m.}{Salário, remuneração.}{es.ti.pên.dio}{0}
\verb{estipular}{}{}{}{}{v.t.}{Colocar com condição; determinar, estabelecer.}{es.ti.pu.lar}{\verboinum{1}}
\verb{estirada}{}{}{}{}{s.f.}{Caminhada longa; estirão.}{es.ti.ra.da}{0}
\verb{estiramento}{}{}{}{}{s.m.}{Distensão, em geral muscular.}{es.ti.ra.men.to}{0}
\verb{estiramento}{}{}{}{}{}{Espreguiçamento.}{es.ti.ra.men.to}{0}
\verb{estirão}{}{}{}{}{s.m.}{Distância longa.}{es.ti.rão}{0}
\verb{estirão}{}{}{}{}{}{Caminhada longa.}{es.ti.rão}{0}
\verb{estirar}{}{}{}{}{v.t.}{Alongar puxando; estender.}{es.ti.rar}{0}
\verb{estirar}{}{}{}{}{}{Alongar o corpo ou parte dele; esticar.}{es.ti.rar}{0}
\verb{estirar}{}{}{}{}{}{Causar distensão em músculo; distender.}{es.ti.rar}{0}
\verb{estirar}{}{}{}{}{}{Estender no tempo; prolongar.}{es.ti.rar}{\verboinum{1}}
\verb{estirpe}{}{}{}{}{s.f.}{Parte da planta que se desenvolve debaixo da terra; raiz.}{es.tir.pe}{0}
\verb{estirpe}{}{}{}{}{}{Origem, raça, descendência.}{es.tir.pe}{0}
\verb{estiva}{}{}{}{}{s.f.}{Carga pesada colocada nos porões dos navios.}{es.ti.va}{0}
\verb{estiva}{}{}{}{}{}{Trabalho de carregamento e descarregamento de navios.}{es.ti.va}{0}
\verb{estiva}{}{}{}{}{}{Conjunto dos trabalhadores portuários.}{es.ti.va}{0}
\verb{estivador}{ô}{}{}{}{adj.}{Diz"-se de quem carrega e descarrega navios.}{es.ti.va.dor}{0}
\verb{estival}{}{}{}{}{adj.2g.}{Próprio de estio; quente, calmoso, estivo.}{es.ti.val}{0}
\verb{estocada}{}{}{}{}{s.f.}{Golpe com a ponta aguda de uma arma.}{es.to.ca.da}{0}
\verb{estocar}{}{}{}{}{v.t.}{Guardar alguma quantidade de mercadoria para uso futuro.}{es.to.car}{\verboinum{2}}
\verb{estofa}{}{}{}{}{s.f.}{Tecido encorpado de algodão, lã, seda ou outro material, usado em decoração, como tapete, para cobrir assentos, entre outros; estofo.}{es.to.fa}{0}
\verb{estofa}{}{Fig.}{}{}{}{Classe, laia.}{es.to.fa}{0}
\verb{estofado}{}{}{}{}{adj.}{Acolchoado e forrado de tecido.}{es.to.fa.do}{0}
\verb{estofado}{}{}{}{}{s.m.}{Tecido ou revestimento grosso, encorpado.}{es.to.fa.do}{0}
\verb{estofado}{}{}{}{}{}{Móvel estofado.}{es.to.fa.do}{0}
\verb{estofador}{ô}{}{}{}{adj.}{Diz"-se de quem estofa móveis.}{es.to.fa.dor}{0}
\verb{estofamento}{}{}{}{}{s.m.}{Revestimento ou enchimento com estofo.}{es.to.fa.men.to}{0}
\verb{estofamento}{}{}{}{}{}{Material usado para estofar.}{es.to.fa.men.to}{0}
\verb{estofar}{}{}{}{}{v.t.}{Cobrir ou encher com estofo.}{es.to.far}{\verboinum{1}}
\verb{estofo}{ô}{}{}{}{s.m.}{Tecido usado em decoração.}{es.to.fo}{0}
\verb{estofo}{ô}{}{}{}{}{Enchimento para estofados.}{es.to.fo}{0}
\verb{estofo}{ô}{}{}{}{}{Firmeza, caráter.}{es.to.fo}{0}
\verb{estoicismo}{}{}{}{}{s.m.}{Austeridade de caráter; rigidez moral. }{es.toi.cis.mo}{0}
\verb{estoico}{ó}{}{}{}{adj.}{Que é digno de um herói; que implica um grande esforço; heroico.}{es.toi.co}{0}
\verb{estoico}{ó}{}{}{}{}{Relativo ao estoicismo.}{es.toi.co}{0}
\verb{estoirado}{}{}{}{}{}{Var. de \textit{estourado}.}{es.toi.ra.do}{0}
\verb{estoirar}{}{}{}{}{}{Var. de \textit{estourar}.}{es.toi.rar}{0}
\verb{estoiro}{ô}{}{}{}{}{Var. de \textit{estouro}.}{es.toi.ro}{0}
\verb{estojo}{ô}{}{}{}{s.m.}{Pequena caixa com formato e divisão de espaço interno planejados para acomodar determinados objetos.}{es.to.jo}{0}
\verb{estola}{}{}{}{}{s.f.}{Faixa usada em torno do pescoço pelos padres em liturgias.}{es.to.la}{0}
\verb{estola}{}{}{}{}{}{Espécie de xale comprido, geralmente retangular, que as mulheres usam como agasalho ou como adorno.}{es.to.la}{0}
\verb{estomacal}{}{}{}{}{adj.2g.}{Que diz respeito ao estômago.}{es.to.ma.cal}{0}
\verb{estomagado}{}{}{}{}{adj.}{Que se estomagou; indignado, irritado.}{es.to.ma.ga.do}{0}
\verb{estomagar}{}{}{}{}{v.t.}{Zangar, irritar, indignar.}{es.to.ma.gar}{0}
\verb{estomagar}{}{}{}{}{}{Ofender; escandalizar. }{es.to.ma.gar}{\verboinum{5}}
\verb{estômago}{}{}{}{}{s.m.}{Víscera onde ocorre parte da digestão dos alimentos, situada entre o esôfago e o duodeno.}{es.tô.ma.go}{0}
\verb{estomatite}{}{Med.}{}{}{s.f.}{Inflamação da membrana mucosa da boca.}{es.to.ma.ti.te}{0}
\verb{estoniano}{}{}{}{}{adj.}{Relativo a Estônia.}{es.to.ni.a.no}{0}
\verb{estoniano}{}{}{}{}{s.m.}{Indivíduo natural ou habitante desse país.}{es.to.ni.a.no}{0}
\verb{estontear}{}{}{}{}{v.t.}{Fazer ficar tonto; aturdir.}{es.ton.te.ar}{0}
\verb{estontear}{}{}{}{}{}{Deslumbrar.}{es.ton.te.ar}{\verboinum{4}}
\verb{estopa}{ô}{}{}{}{s.f.}{Tecido fabricado com resíduos de fibra têxtil penteada, geralmente algodão.}{es.to.pa}{0}
\verb{estopada}{}{}{}{}{s.f.}{Porção de estopa.}{es.to.pa.da}{0}
\verb{estopada}{}{Pop.}{}{}{}{Coisa enfadonha, chata, maçante.}{es.to.pa.da}{0}
\verb{estopim}{}{}{}{}{s.m.}{Peça de um explosivo, feita de pólvora negra, destinada a provocar a explosão da espoleta.}{es.to.pim}{0}
\verb{estopim}{}{Fig.}{}{}{}{Evento que desencadeia uma série de acontecimentos.}{es.to.pim}{0}
\verb{estoque}{ó}{}{}{}{s.m.}{Quantidade de mercadoria armazenada e destinada a uso ou comercialização.}{es.to.que}{0}
\verb{estoquista}{}{Bras.}{}{}{s.2g.}{Indivíduo que mantém um estoque de mercadorias.}{es.to.quis.ta}{0}
\verb{estore}{}{}{}{}{s.m.}{Tipo de cortina que se enrola em uma vara e se pode abaixar e levantar.}{es.to.re}{0}
\verb{estória}{}{Desus.}{}{}{s.f.}{Narrativa popular tradicional.}{es.tó.ria}{0}
\verb{estória}{}{}{}{}{}{História.}{es.tó.ria}{0}
\verb{estornar}{}{}{}{}{v.t.}{Lançar em uma conta um crédito de valor igual a um débito feito indevidamente, ou vice"-versa, com a finalidade de cancelar o lançamento anterior.}{es.tor.nar}{\verboinum{1}}
\verb{estorno}{ô}{}{}{}{s.m.}{Ato ou efeito de estornar.}{es.tor.no}{0}
\verb{estorricado}{}{}{}{}{adj.}{Muito assado; torrado.}{es.tor.ri.ca.do}{0}
\verb{estorricado}{}{}{}{}{}{Muito seco.}{es.tor.ri.ca.do}{0}
\verb{estorricar}{}{}{}{}{v.t.}{Assar em demasia; torrar.}{es.tor.ri.car}{0}
\verb{estorricar}{}{}{}{}{v.i.}{Secar demais.}{es.tor.ri.car}{\verboinum{2}}
\verb{estorvar}{}{}{}{}{v.t.}{Incomodar, importunar, aborrecer.}{es.tor.var}{\verboinum{1}}
\verb{estorvar}{}{}{}{}{}{Tirar a liberdade de movimento de; impedir, embaraçar.}{es.tor.var}{0}
\verb{estorvo}{ô}{}{}{}{s.m.}{Incômodo, obstáculo, dificuldade.}{es.tor.vo}{0}
\verb{estorvo}{ô}{}{}{}{}{Aquele ou aquilo que estorva.}{es.tor.vo}{0}
\verb{estou"-fraca}{}{Zool.}{}{}{s.f.}{Ave da família do galo, de plumagem cinzenta com pintas brancas, cuja cabeça nua é dotada de uma crista óssea; galinha"-d'angola.}{es.tou"-fra.ca}{0}
\verb{estourado}{}{}{}{}{adj.}{Que estourou.}{es.tou.ra.do}{0}
\verb{estourado}{}{Fig.}{}{}{}{Diz"-se de tempo, valor, sentimento que atingiu o limite.}{es.tou.ra.do}{0}
\verb{estourado}{}{Fig.}{}{}{}{Que age ou fala de maneira precipitada; impaciente, amalucado.}{es.tou.ra.do}{0}
\verb{estourar}{}{}{}{}{v.i.}{Rebentar fazendo barulho seco e intenso.}{es.tou.rar}{0}
\verb{estourar}{}{}{}{}{v.t.}{Romper, rebentar, explodir.}{es.tou.rar}{0}
\verb{estourar}{}{Fig.}{}{}{}{Atingir o limite.}{es.tou.rar}{\verboinum{1}}
\verb{estouro}{ô}{}{}{}{s.m.}{Ruído seco daquilo que estoura; estampido.}{es.tou.ro}{0}
\verb{estouro}{ô}{Fig.}{}{}{}{Acontecimento repentino e imprevisto.}{es.tou.ro}{0}
\verb{estouro}{ô}{Bras.}{}{}{}{Fato ou coisa espetacular, que faz muito sucesso.}{es.tou.ro}{0}
\verb{estoutro}{}{}{}{}{pron.}{Contração do pronome demonstrativo \textit{este} com o pronome indefinido \textit{outro}.}{es.tou.tro}{0}
\verb{estouvado}{}{}{}{}{adj.}{Que age de maneira impensada; leviano, imprudente.}{es.tou.va.do}{0}
\verb{estouvado}{}{}{}{}{}{Brincalhão, espirituoso, travesso.}{es.tou.va.do}{0}
\verb{estouvamento}{}{}{}{}{s.m.}{Qualidade de estouvado.}{es.tou.va.men.to}{0}
\verb{estouvamento}{}{}{}{}{}{Ato ou dito de quem é estouvado.}{es.tou.va.men.to}{0}
\verb{estrábico}{}{}{}{}{adj.}{Relativo ao estrabismo.}{es.trá.bi.co}{0}
\verb{estrábico}{}{}{}{}{}{Que sofre de estrabismo; vesgo.}{es.trá.bi.co}{0}
\verb{estrabismo}{}{Med.}{}{}{s.m.}{Anormalidade da vista caracterizada pelo desvio de um dos olhos de forma que ambos não se dirigem ao mesmo ponto simultaneamente.}{es.tra.bis.mo}{0}
\verb{estraçalhar}{}{}{}{}{v.t.}{Fazer em pedaços; quebrar, estilhaçar, despedaçar.}{es.tra.ça.lhar}{\verboinum{1}}
\verb{estrada}{}{}{}{}{s.f.}{Caminho para o trânsito de pessoas, animais, veículos.}{es.tra.da}{0}
\verb{estrada}{}{Fig.}{}{}{}{Meio para se alcançar um objetivo; modo de agir.}{es.tra.da}{0}
\verb{estrada}{}{Fig.}{}{}{}{Rumo, objetivo, direção, carreira.}{es.tra.da}{0}
\verb{estradeiro}{ê}{Bras.}{}{}{adj.}{Diz"-se de quem está quase sempre em circulação, fora de casa.}{es.tra.dei.ro}{0}
\verb{estrado}{}{}{}{}{s.m.}{Estrutura que forma um piso elevado para que algo ou alguém fique em um nível acima dos demais.}{es.tra.do}{0}
\verb{estrado}{}{}{}{}{}{Estrutura da cama sobre a qual fica o colchão.}{es.tra.do}{0}
\verb{estragado}{}{}{}{}{adj.}{Que não se encontra em bom estado; danificado, deteriorado, podre.}{es.tra.ga.do}{0}
\verb{estragado}{}{}{}{}{}{Moralmente condenável; viciado, corrupto, derrotado.}{es.tra.ga.do}{0}
\verb{estragar}{}{}{}{}{v.t.}{Danificar, avariar, deteriorar, apodrecer.}{es.tra.gar}{0}
\verb{estragar}{}{}{}{}{}{Perverter, depravar, corromper, viciar.}{es.tra.gar}{\verboinum{5}}
\verb{estrago}{}{}{}{}{s.m.}{Dano, avaria, destruição, perda.}{es.tra.go}{0}
\verb{estrago}{}{}{}{}{}{Depravação, vício, corrupção.}{es.tra.go}{0}
\verb{estralada}{}{}{}{}{s.f.}{Ato de estralar.}{es.tra.la.da}{0}
\verb{estralada}{}{}{}{}{}{Ruído, gritaria, confusão.}{es.tra.la.da}{0}
\verb{estralar}{}{}{}{}{v.t.}{Estalar.}{es.tra.lar}{\verboinum{1}}
\verb{estralejar}{}{}{}{}{v.i.}{Dar muitos estalos.}{es.tra.le.jar}{\verboinum{1}}
\verb{estralo}{}{}{}{}{s.m.}{Estalo.}{es.tra.lo}{0}
\verb{estrambólico}{}{Pop.}{}{}{adj.}{Incomum, esquisito, estrambótico.}{es.tram.bó.li.co}{0}
\verb{estrambótico}{}{}{}{}{adj.}{Incomum, extravagante, esquisito.}{es.tram.bó.ti.co}{0}
\verb{estrangeirado}{}{}{}{}{adj.}{Que imita, prefere ou se assemelha às coisas estrangeiras.}{es.tran.gei.ra.do}{0}
\verb{estrangeirice}{}{}{}{}{s.f.}{Ato ou dito que se assemelha ao dos estrangeiros.}{es.tran.gei.ri.ce}{0}
\verb{estrangeirismo}{}{Gram.}{}{}{s.m.}{Palavra ou construção semelhante ou própria de uma língua estrangeira.}{es.tran.gei.ris.mo}{0}
\verb{estrangeiro}{ê}{}{}{}{adj.}{Originário de ou referente a país diferente daquele que se está considerando.}{es.tran.gei.ro}{0}
\verb{estrangeiro}{ê}{}{}{}{}{Que é de outra região do próprio país; forasteiro.}{es.tran.gei.ro}{0}
\verb{estrangeiro}{ê}{}{}{}{s.m.}{Qualquer país diferente daquele de que se fala.}{es.tran.gei.ro}{0}
\verb{estrangulação}{}{}{"-ões}{}{s.f.}{Ato ou efeito de estrangular; estrangulamento.}{es.tran.gu.la.ção}{0}
\verb{estrangulador}{ô}{}{}{}{adj.}{Que estrangula.}{es.tran.gu.la.dor}{0}
\verb{estrangulamento}{}{}{}{}{s.m.}{Ato ou efeito de estrangular; sufocação, constrição.}{es.tran.gu.la.men.to}{0}
\verb{estrangular}{}{}{}{}{v.t.}{Apertar o pescoço impedindo a respiração; sufocar, esganar.}{es.tran.gu.lar}{0}
\verb{estrangular}{}{}{}{}{}{Comprimir, apertar.}{es.tran.gu.lar}{\verboinum{1}}
\verb{estranhar}{}{}{}{}{v.t.}{Achar estranho, diferente do habitual.}{es.tra.nhar}{0}
\verb{estranhar}{}{}{}{}{}{Causar espanto; surpreender.}{es.tra.nhar}{0}
\verb{estranhar}{}{}{}{}{}{Manifestar timidez ou repulsão em relação a; não ficar à vontade com.}{es.tra.nhar}{0}
\verb{estranhar}{}{}{}{}{v.pron.}{Entrar em discórdia.}{es.tra.nhar}{\verboinum{1}}
\verb{estranhável}{}{}{"-eis}{}{adj.2g.}{Que causa estranheza.}{es.tra.nhá.vel}{0}
\verb{estranheza}{ê}{}{}{}{s.f.}{Qualidade de estranho.}{es.tra.nhe.za}{0}
\verb{estranho}{}{}{}{}{adj.}{Fora do comum; anormal, diferente.}{es.tra.nho}{0}
\verb{estranho}{}{}{}{}{}{Misterioso, desconhecido.}{es.tra.nho}{0}
\verb{estranho}{}{}{}{}{}{Que é de fora; estrangeiro.}{es.tra.nho}{0}
\verb{estranho}{}{}{}{}{s.m.}{Indivíduo que não pertence à família ou à organização em questão.}{es.tra.nho}{0}
\verb{estranja}{}{}{}{}{s.f.}{Lugar fora do próprio país; terra estrangeira.}{es.tran.ja}{0}
\verb{estratagema}{}{}{}{}{s.m.}{Meio empregado em uma guerra para confundir ou lesar o inimigo.}{es.tra.ta.ge.ma}{0}
\verb{estratagema}{}{}{}{}{}{Plano para atingir qualquer objetivo; astúcia, estratégia, manha.}{es.tra.ta.ge.ma}{0}
\verb{estratégia}{}{}{}{}{s.f.}{Técnica de planejamento e aplicação dos recursos bélicos para atingir os objetivos de uma guerra.}{es.tra.té.gia}{0}
\verb{estratégia}{}{}{}{}{}{Habilidade em organizar as coisas e pessoas para alcançar qualquer objetivo.}{es.tra.té.gia}{0}
\verb{estratégico}{}{}{}{}{adj.}{Relativo a estratégia.}{es.tra.té.gi.co}{0}
\verb{estratégico}{}{}{}{}{}{Diz"-se de locais, coisas ou pessoas importantes para pôr em prática determinado plano e atingir os objetivos.}{es.tra.té.gi.co}{0}
\verb{estrategista}{}{}{}{}{s.2g.}{Indivíduo que entende de ou emprega estratégias.}{es.tra.te.gis.ta}{0}
\verb{estratificação}{}{}{"-ões}{}{s.f.}{Organização em camadas.}{es.tra.ti.fi.ca.ção}{0}
\verb{estratificado}{}{}{}{}{adj.}{Que sofreu estratificação; em que há camadas; disposto em camadas.}{es.tra.ti.fi.ca.do}{0}
\verb{estratificar}{}{}{}{}{v.t.}{Dispor em estratos ou camadas.}{es.tra.ti.fi.car}{\verboinum{2}}
\verb{estrato}{}{Geol.}{}{}{s.m.}{Cada uma das camadas dos terrenos sedimentares.}{es.tra.to}{0}
\verb{estrato}{}{}{}{}{}{Tipo de nuvem que se apresenta em forma de camadas horizontais e paralelas.}{es.tra.to}{0}
\verb{estrato}{}{}{}{}{}{Disposição hierárquica da sociedade; camada, nível social.}{es.tra.to}{0}
\verb{estratosfera}{é}{}{}{}{s.f.}{Camada da atmosfera situada a cerca de 12 mil metros de altitude, composta principalmente por hidrogênio.}{es.tra.tos.fe.ra}{0}
\verb{estreante}{}{}{}{}{adj.2g.}{Diz"-se de quem estreia; debutante.}{es.tre.an.te}{0}
\verb{estrear}{}{}{}{}{v.i.}{Apresentar"-se ao público pela primeira vez. (\textit{Quando ela estreou no teatro há quarenta anos, não imaginava que se transformaria em uma das maiores atrizes da televisão brasileira.})}{es.tre.ar}{0}
\verb{estrear}{}{}{}{}{v.t.}{Usar pela primeira vez.}{es.tre.ar}{0}
\verb{estrear}{}{}{}{}{}{Começar, iniciar, inaugurar.}{es.tre.ar}{\verboinum{4}}
\verb{estrebaria}{}{}{}{}{s.f.}{Local onde se guardam os cavalos; estábulo, cavalariça.}{es.tre.ba.ri.a}{0}
\verb{estrebuchamento}{}{}{}{}{s.m.}{Ato ou efeito de estrebuchar; convulsão, agitação.}{es.tre.bu.cha.men.to}{0}
\verb{estrebuchar}{}{}{}{}{v.i.}{Agitar os braços e as pernas em convulsão; mexer"-se muito; debater"-se.}{es.tre.bu.char}{\verboinum{1}}
\verb{estreia}{é}{}{}{}{s.f.}{Ato ou efeito de estrear, inaugurar.}{es.trei.a}{0}
\verb{estreia}{é}{}{}{}{}{A primeira apresentação de um artista ou de uma peça teatral, um filme etc. (\textit{A estreia mundial daquele filme levou milhões de espectadores aos cinemas.})}{es.trei.a}{0}
\verb{estreia}{é}{}{}{}{}{O primeiro uso de um objeto.}{es.trei.a}{0}
\verb{estreitamento}{}{}{}{}{s.m.}{Ato ou efeito de estreitar, apertar.}{es.trei.ta.men.to}{0}
\verb{estreitamento}{}{}{}{}{}{Redução, diminuição, restrição.}{es.trei.ta.men.to}{0}
\verb{estreitamento}{}{Fig.}{}{}{}{Fortalecimento, consolidação. (\textit{Aquele incidente só ajudou no estreitamento das relações entre os dois.})}{es.trei.ta.men.to}{0}
\verb{estreitar}{}{}{}{}{v.t.}{Tornar estreito; restringir, limitar, diminuir.}{es.trei.tar}{0}
\verb{estreitar}{}{}{}{}{}{Apertar contra o peito; abraçar fortemente. (\textit{A mãe estreitou o filho nos braços depois da longa separação.})}{es.trei.tar}{0}
\verb{estreitar}{}{}{}{}{}{Tornar mais íntimo; unir, ligar.}{es.trei.tar}{\verboinum{1}}
\verb{estreiteza}{ê}{}{}{}{s.f.}{Qualidade de estreito.}{es.trei.te.za}{0}
\verb{estreiteza}{ê}{}{}{}{}{Mesquinhez, carência, miséria.}{es.trei.te.za}{0}
\verb{estreito}{ê}{}{}{}{adj.}{Pouco largo; apertado, acanhado.}{es.trei.to}{0}
\verb{estreito}{ê}{}{}{}{}{De compreensão ou inteligência limitada; tacanho.}{es.trei.to}{0}
\verb{estreito}{ê}{Geol.}{}{}{s.m.}{Canal natural que banha duas costas, situado entre dois mares.}{es.trei.to}{0}
\verb{estrela}{ê}{Astron.}{}{}{s.f.}{Astro dotado de luz própria.}{es.tre.la}{0}
\verb{estrela}{ê}{}{}{}{}{Figura que apresenta cinco ou seis pontas.}{es.tre.la}{0}
\verb{estrela}{ê}{Fig.}{}{}{}{Atriz de muito sucesso.}{es.tre.la}{0}
\verb{estrela"-d’alva}{ê}{}{estrelas"-d'alva ⟨ê⟩}{}{s.f.}{O planeta Vênus visto de madrugada.}{es.tre.la"-d’al.va}{0}
\verb{estrela"-de"-davi}{ê}{}{estrelas"-de"-davi ⟨ê⟩}{}{s.f.}{Estrela de seis pontas, formada pela união de dois triângulos equiláteros entrelaçados ou superpostos, considerada símbolo do judaísmo.}{es.tre.la"-de"-da.vi}{0}
\verb{estrelado}{}{}{}{}{adj.}{Coberto de estrelas.}{es.tre.la.do}{0}
\verb{estrela"-do"-mar}{ê}{}{estrelas"-do"-mar ⟨ê⟩}{}{s.f.}{Nome comum aos equinodermos em forma de estrela, que rastejam no fundo do mar, com o corpo achatado em disco, do qual se projetam de 5 a 50 braços, e que são encontrados nos mares de todo o mundo.}{es.tre.la"-do"-mar}{0}
\verb{estrelar}{}{}{}{}{v.t.}{Encher de estrelas.}{es.tre.lar}{0}
\verb{estrelar}{}{Fig.}{}{}{}{Atuar em um filme em papel principal.}{es.tre.lar}{0}
\verb{estrelar}{}{}{}{}{}{Frigir um ovo sem o mexer.}{es.tre.lar}{\verboinum{1}}
\verb{estrelato}{}{}{}{}{s.m.}{O apogeu da carreira de um artista de cinema ou de teatro; fama.}{es.tre.la.to}{0}
\verb{estrelinha}{}{}{}{}{s.f.}{Sinal gráfico em forma de estrela; asterisco.}{es.tre.li.nha}{0}
\verb{estrelinha}{}{}{}{}{}{Massa para sopa em forma de estrela.}{es.tre.li.nha}{0}
\verb{estrelinha}{}{}{}{}{}{Tipo de fogo de artifício de salão.}{es.tre.li.nha}{0}
\verb{estrema}{ê}{}{}{}{s.f.}{Linha divisória; divisa, delimitação.}{es.tre.ma}{0}
\verb{estremadura}{}{}{}{}{s.f.}{Fronteira de país; limite.}{es.tre.ma.du.ra}{0}
\verb{estremar}{}{}{}{}{v.t.}{Demarcar por meio de divisa; limitar, separar.}{es.tre.mar}{\verboinum{1}}
\verb{estreme}{ê}{}{}{}{adj.}{Sem mistura; puro.}{es.tre.me}{0}
\verb{estremeção}{}{}{"-ões}{}{s.f.}{Ato ou efeito de estremecer; estremecimento, tremor.}{es.tre.me.ção}{0}
\verb{estremecer}{ê}{}{}{}{v.t.}{Fazer tremer; abalar, sacudir.}{es.tre.me.cer}{\verboinum{15}}
\verb{estremecido}{}{}{}{}{adj.}{Que estremeceu; abalado, assustado, sobressaltado.}{es.tre.me.ci.do}{0}
\verb{estremecimento}{}{}{}{}{s.m.}{Ato ou efeito de estremecer; estremeção, tremor, abalo.}{es.tre.me.ci.men.to}{0}
\verb{estremunhar}{}{}{}{}{v.i.}{Despertar de repente, ainda tonto de sono.}{es.tre.mu.nhar}{\verboinum{1}}
\verb{estrênuo}{}{}{}{}{adj.}{Que se esforça muito; pertinaz, incansável.}{es.trê.nu.o}{0}
\verb{estrepar}{}{}{}{}{v.t.}{Ferir com estrepe; espetar.}{es.tre.par}{0}
\verb{estrepar}{}{Pop.}{}{}{v.pron.}{Sair"-se mal.}{es.tre.par}{\verboinum{1}}
\verb{estrepe}{é}{}{}{}{s.m.}{Pua de madeira ou ferro; estaca.}{es.tre.pe}{0}
\verb{estrepe}{é}{}{}{}{}{Indivíduo que incomoda, importuna.}{es.tre.pe}{0}
\verb{estrepitar}{}{}{}{}{v.i.}{Soar com estrépito; estrondar.}{es.tre.pi.tar}{\verboinum{1}}
\verb{estrépito}{}{}{}{}{s.m.}{Grande barulho; estrondo.}{es.tré.pi.to}{0}
\verb{estrepitoso}{ô}{}{"-osos ⟨ó⟩}{"-osa ⟨ó⟩}{adj.}{Que produz estrépito; estrondoso, barulhento, ruidoso.}{es.tre.pi.to.so}{0}
\verb{estrepolia}{}{}{}{}{}{Var. de \textit{estripulia}.}{es.tre.po.li.a}{0}
\verb{estreptococo}{ô}{Biol.}{}{}{s.m.}{Gênero de bactérias esféricas que se apresentam em cadeia e causam infecções graves.}{es.trep.to.co.co}{0}
\verb{estreptomicina}{}{Farm.}{}{}{s.f.}{Antibiótico usado para combater o bacilo da tuberculose e de outras doenças infecciosas.}{es.trep.to.mi.ci.na}{0}
\verb{estressar}{}{}{}{}{v.t.}{Produzir estresse em; esgotar.}{es.tres.sar}{\verboinum{1}}
\verb{estresse}{é}{Med.}{}{}{s.m.}{Conjunto de reações físicas e emocionais em resposta a uma situação que envolve  grande ansiedade.}{es.tres.se}{0}
\verb{estria}{}{}{}{}{s.f.}{Pequeno sulco sobre a superfície de um corpo; linha, traço.}{es.tri.a}{0}
\verb{estriar}{}{}{}{}{v.t.}{Fazer estrias em; riscar, traçar.}{es.tri.ar}{\verboinum{1}}
\verb{estribar}{}{}{}{}{v.i.}{Firmar os pés nos estribos.}{es.tri.bar}{0}
\verb{estribar}{}{}{}{}{v.t.}{Apoiar, firmar, fundamentar.}{es.tri.bar}{\verboinum{1}}
\verb{estribeira}{ê}{}{}{}{s.f.}{Estribo de carruagem, de coche.}{es.tri.bei.ra}{0}
\verb{estribilho}{}{}{}{}{s.m.}{Verso repetido no final de cada estrofe; refrão.}{es.tri.bi.lho}{0}
\verb{estribo}{}{}{}{}{s.m.}{Cada uma das peças de metal que ficam abaixo da sela para o cavaleiro firmar os pés.}{es.tri.bo}{0}
\verb{estribo}{}{}{}{}{}{Degrau lateral de veículos como bonde, trem, entre outros.}{es.tri.bo}{0}
\verb{estribo}{}{Anat.}{}{}{}{Ossinho do ouvido médio.}{es.tri.bo}{0}
\verb{estricnina}{}{Quím.}{}{}{s.f.}{Substância venenosa extraída principalmente da noz"-vômica, usada como estimulante do sistema nervoso central.}{es.tric.ni.na}{0}
\verb{estridente}{}{}{}{}{adj.2g.}{De som muito agudo, penetrante, sibilante.}{es.tri.den.te}{0}
\verb{estridor}{ô}{}{}{}{s.m.}{Som forte, agudo; zunido, sibilo.}{es.tri.dor}{0}
\verb{estridular}{}{}{}{}{v.i.}{Produzir um som agudo e penetrante; zunir, sibilar.}{es.tri.du.lar}{0}
\verb{estridular}{}{}{}{}{v.t.}{Cantar ou falar de modo estridente.}{es.tri.du.lar}{\verboinum{1}}
\verb{estrídulo}{}{}{}{}{s.m.}{Som estridente, agudo; estridor.}{es.trí.du.lo}{0}
\verb{estrigídeo}{}{}{}{}{s.m.}{Família de aves carnívoras, com hábitos noturnos, que se caracterizam por penachos na cabeça, pequenas orelhas e plumagem, como a coruja e o mocho.}{es.tri.gí.deo}{0}
\verb{estrilar}{}{}{}{}{v.i.}{Soltar estrilo; sibilar, zunir.}{es.tri.lar}{0}
\verb{estrilar}{}{Pop.}{}{}{}{Gritar, bradar com raiva; zangar"-se.}{es.tri.lar}{\verboinum{1}}
\verb{estrilo}{}{}{}{}{s.m.}{Som estridente; zunido.}{es.tri.lo}{0}
\verb{estrilo}{}{Pop.}{}{}{}{Protesto, zanga.}{es.tri.lo}{0}
\verb{estringir}{}{}{}{}{v.t.}{Apertar fortemente; cingir, comprimir.}{es.trin.gir}{\verboinum{22}}
\verb{estripar}{}{}{}{}{v.t.}{Tirar as tripas; retalhar o ventre de. (\textit{A onça estripou o boi.})}{es.tri.par}{\verboinum{1}}
\verb{estripulia}{}{}{}{}{s.f.}{Travessura, traquinagem, desordem.}{es.tri.pu.li.a}{0}
\verb{estrito}{}{}{}{}{adj.}{Que é restrito, preciso, rigoroso, exato.}{es.tri.to}{0}
\verb{estro}{}{}{}{}{s.m.}{Inspiração, imaginação, talento; engenho poético.}{es.tro}{0}
\verb{estrofe}{ó}{}{}{}{s.f.}{Conjunto de dois ou mais versos, rimados ou não entre si, em que são divididas as composições poéticas; estância.}{es.tro.fe}{0}
\verb{estrógeno}{}{Bioquím.}{}{}{s.m.}{Hormônio natural ou sintético responsável pelo desenvolvimento de várias características femininas e fundamental no controle do ciclo ovulatório; estrogênio.}{es.tró.ge.no}{0}
\verb{estrogonofe}{ó}{Cul.}{}{}{s.m.}{Iguaria preparada com pedaços de carne bovina (ou de frango), creme de leite, vinho, \textit{champignons} e especiarias.}{es.tro.go.no.fe}{0}
\verb{estróina}{}{}{}{}{adj.2g.}{Que age de forma irresponsável ou leviana; doidivanas, desajuizado, extravagante.}{es.trói.na}{0}
\verb{estróina}{}{}{}{}{s.2g.}{Pessoa que gasta exageradamente; perdulário, dissipador.}{es.trói.na}{0}
\verb{estrompar}{}{}{}{}{v.t.}{Gastar muito alguma coisa; estragar, deteriorar.}{es.trom.par}{\verboinum{1}}
\verb{estroncar}{}{}{}{}{}{Var. de \textit{destroncar}.}{es.tron.car}{\verboinum{1}}
\verb{estrôncio}{}{Quím.}{}{}{s.m.}{Elemento químico metálico, do grupo dos alcalino"-terrosos, utilizado no tratamento dos olhos, em baterias atômicas e em pirotecnia, para atribuir coloração vermelha às chamas; é particularmente perigoso, porque pode ser assimilado pelo organismo humano e de animais, causando danos irreversíveis. \elemento{38}{87.62}{Sr}.}{es.trôn.cio}{0}
\verb{estrondar}{}{}{}{}{v.i.}{Fazer estrondo, barulho; retumbar, estrepitar.}{es.tron.dar}{\verboinum{1}}
\verb{estrondear}{}{}{}{}{v.i.}{Estrondar.}{es.tron.de.ar}{\verboinum{4}}
\verb{estrondo}{}{}{}{}{s.m.}{Barulho muito forte; fragor, estampido, estouro.}{es.tron.do}{0}
\verb{estrondoso}{ô}{}{"-osos ⟨ó⟩}{"-osa ⟨ó⟩}{adj.}{Que faz ou causa estrondo; ruidoso, estrepitoso, barulhento.}{es.tron.do.so}{0}
\verb{estropiar}{}{}{}{}{v.t.}{Cortar um membro; mutilar, aleijar.}{es.tro.pi.ar}{0}
\verb{estropiar}{}{Fig.}{}{}{}{Desfigurar, deformar ou desvirtuar (um texto, o idioma, uma peça de música etc.).}{es.tro.pi.ar}{0}
\verb{estropiar}{}{}{}{}{}{Cansar, fatigar.}{es.tro.pi.ar}{0}
\verb{estropiar}{}{}{}{}{v.pron.}{Aleijar"-se, mutilar"-se.}{es.tro.pi.ar}{\verboinum{1}}
\verb{estropício}{}{}{}{}{s.m.}{Qualquer dano ou prejuízo; estrago, malefício, transtorno. }{es.tro.pí.cio}{0}
\verb{estrugido}{}{}{}{}{s.m.}{Estrondo.}{es.tru.gi.do}{0}
\verb{estrugir}{}{}{}{}{v.t.}{Estrondear.}{es.tru.gir}{\verboinum{22}}
\verb{estrumar}{}{}{}{}{v.t.}{Estercar.}{es.tru.mar}{\verboinum{1}}
\verb{estrume}{}{}{}{}{s.m.}{Mistura feita de esterco, ramos e folhas apodrecidas, usada para fertilizar a terra; adubo.}{es.tru.me}{0}
\verb{estrumeira}{ê}{}{}{}{s.f.}{Esterqueira.}{es.tru.mei.ra}{0}
\verb{estrupício}{}{Pop.}{}{}{s.m.}{Conflito, motim, desordem, algazarra, confusão, briga, rolo.}{es.tru.pí.cio}{0}
\verb{estrupício}{}{}{}{}{}{Coisa estranha, esquisita; estrovenga.}{es.tru.pí.cio}{0}
\verb{estrutura}{}{}{}{}{s.f.}{Disposição, ordem e organização dos elementos que compõem um conjunto.}{es.tru.tu.ra}{0}
\verb{estrutura}{}{}{}{}{}{Armação, esqueleto, arcabouço.}{es.tru.tu.ra}{0}
\verb{estrutural}{}{}{"-ais}{}{adj.2g.}{Relativo a estrutura.}{es.tru.tu.ral}{0}
\verb{estruturalismo}{}{}{}{}{s.m.}{Corrente de pensamento que considera os sistemas sociais um conjunto de elementos solidários entre si formando uma estrutura.}{es.tru.tu.ra.lis.mo}{0}
\verb{estruturar}{}{}{}{}{v.t.}{Dar ou fazer estrutura.}{es.tru.tu.rar}{0}
\verb{estruturar}{}{}{}{}{v.pron.}{Adquirir estrutura; constituir"-se, organizar"-se.}{es.tru.tu.rar}{\verboinum{1}}
\verb{estuar}{}{}{}{}{v.i.}{Estar muito quente ou ardente; ferver, fervilhar.}{es.tu.ar}{0}
\verb{estuar}{}{Por ext.}{}{}{}{Agitar"-se, vibrar, palpitar.}{es.tu.ar}{\verboinum{1}}
\verb{estuário}{}{}{}{}{s.m.}{Desembocadura larga de um rio, semelhante a um golfo.}{es.tu.á.rio}{0}
\verb{estucador}{ô}{}{}{}{s.m.}{Indivíduo que estuca, que trabalha com estuque.}{es.tu.ca.dor}{0}
\verb{estucar}{}{}{}{}{v.t.}{Revestir paredes e tetos com estuque.}{es.tu.car}{\verboinum{2}}
\verb{estudado}{}{}{}{}{adj.}{Que se adquiriu por meio do estudo.}{es.tu.da.do}{0}
\verb{estudado}{}{}{}{}{}{Cuidadosamente pensado, examinado.}{es.tu.da.do}{0}
\verb{estudado}{}{}{}{}{}{Diz"-se do indivíduo que tem estudo.}{es.tu.da.do}{0}
\verb{estudantada}{}{}{}{}{s.f.}{Grupo de estudantes.}{es.tu.dan.ta.da}{0}
\verb{estudantada}{}{}{}{}{}{Brincadeira de estudantes.}{es.tu.dan.ta.da}{0}
\verb{estudante}{}{}{}{}{s.2g.}{Indivíduo que estuda, que está matriculado em algum curso; aluno.}{es.tu.dan.te}{0}
\verb{estudar}{}{}{}{}{v.t.}{Aplicar a inteligência para aprender, para adquirir conhecimento.}{es.tu.dar}{0}
\verb{estudar}{}{}{}{}{}{Observar atentamente; examinar, refletir. (\textit{O detetive estudou o caso atentamente e decidiu não aceitá"-lo.})}{es.tu.dar}{\verboinum{1}}
\verb{estúdio}{}{}{}{}{s.m.}{Oficina de artista, fotógrafo etc.; ateliê. (\textit{O estúdio do escultor estava repleto de obras inacabadas.})}{es.tú.dio}{0}
\verb{estúdio}{}{}{}{}{}{Espaço onde são feitas gravações e filmagens para a televisão e o cinema.}{es.tú.dio}{0}
\verb{estudioso}{ô}{}{"-osos ⟨ó⟩}{"-osa ⟨ó⟩}{adj.}{Que gosta de estudar; que é aplicado aos estudos.}{es.tu.di.o.so}{0}
\verb{estudo}{}{}{}{}{s.m.}{Ato ou efeito de estudar.}{es.tu.do}{0}
\verb{estudo}{}{}{}{}{}{Aquilo que resulta do ato de estudar; pesquisa, trabalho, análise. (\textit{O estudo sobre a cultura material indígena atraiu muitos curiosos na feira de ciências da escola. })}{es.tu.do}{0}
\verb{estufa}{}{}{}{}{s.f.}{Espaço fechado onde se mantém a temperatura elevada para fins de cultivo de plantas.}{es.tu.fa}{0}
\verb{estufa}{}{}{}{}{}{Aparelho destinado a aquecer as casas; aquecedor.}{es.tu.fa}{0}
\verb{estufado}{}{}{}{}{adj.}{Que foi posto em estufa para ser aquecido.}{es.tu.fa.do}{0}
\verb{estufado}{}{}{}{}{}{Que aumentou de volume; inchado, intumescido.}{es.tu.fa.do}{0}
\verb{estufar}{}{}{}{}{v.t.}{Colocar ou secar em estufa.}{es.tu.far}{0}
\verb{estufar}{}{}{}{}{}{Aumentar de volume; inchar, inflar.}{es.tu.far}{\verboinum{1}}
\verb{estugar}{}{}{}{}{v.t.}{Caminhar rapidamente; apressar o passo.}{es.tu.gar}{\verboinum{5}}
\verb{estultice}{}{}{}{}{s.f.}{Estultícia.}{es.tul.ti.ce}{0}
\verb{estultícia}{}{}{}{}{s.f.}{Qualidade de estulto; estupidez, tolice.}{es.tul.tí.cia}{0}
\verb{estulto}{}{}{}{}{adj.}{Que carece de inteligência ou bom"-senso; tolo, estúpido, insensato.}{es.tul.to}{0}
\verb{estupefação}{}{}{"-ões}{}{s.f.}{Sentimento de espanto diante do inesperado; assombro, surpresa.}{es.tu.pe.fa.ção}{0}
\verb{estupefaciente}{}{}{}{}{adj.2g.}{Que provoca estupefação; espantoso, assombroso.}{es.tu.pe.fa.ci.en.te}{0}
\verb{estupefacto}{}{}{}{}{}{Var. de \textit{estupefato}.}{es.tu.pe.fac.to}{0}
\verb{estupefato}{}{}{}{}{adj.}{Tomado de assombro; espantado, atônito, pasmado.}{es.tu.pe.fa.to}{0}
\verb{estupefazer}{ê}{}{}{}{v.t.}{Provocar grande espanto; assombrar, pasmar.}{es.tu.pe.fa.zer}{\verboinum{42}}
\verb{estupendo}{}{}{}{}{adj.}{Que causa admiração; surpreendente, fantástico, maravilhoso.}{es.tu.pen.do}{0}
\verb{estupidez}{ê}{}{}{}{s.f.}{Falta de educação; grosseria, indelicadeza.}{es.tu.pi.dez}{0}
\verb{estupidez}{ê}{}{}{}{}{Ato ou dito estúpido; tolice, asneira.}{es.tu.pi.dez}{0}
\verb{estupidificar}{}{}{}{}{v.t.}{Tornar estúpido, parvo; bestificar.}{es.tu.pi.di.fi.car}{\verboinum{2}}
\verb{estúpido}{}{}{}{}{adj.}{Que carece de inteligência; tolo, parvo.}{es.tú.pi.do}{0}
\verb{estúpido}{}{}{}{}{}{Grosseiro, rude, indelicado.}{es.tú.pi.do}{0}
\verb{estupor}{ô}{Med.}{}{}{s.m.}{Estado de inconsciência profunda em que o paciente não apresenta reações motoras nem sensibilidade.}{es.tu.por}{0}
\verb{estupor}{ô}{Fig.}{}{}{}{Estado de imobilidade provocada por grande assombro ou emoção muito forte.}{es.tu.por}{0}
\verb{estuporar}{}{}{}{}{v.t.}{Fazer cair em estupor; assombrar, assustar.}{es.tu.po.rar}{\verboinum{1}}
\verb{estuprador}{ô}{}{}{}{adj.}{Que estupra, violenta.}{es.tu.pra.dor}{0}
\verb{estuprar}{}{}{}{}{v.t.}{Cometer estupro contra; violentar sexualmente.}{es.tu.prar}{\verboinum{1}}
\verb{estupro}{}{}{}{}{s.m.}{Crime hediondo de forçar alguém a prática sexual por meio de violência ou ameaça.}{es.tu.pro}{0}
\verb{estuque}{}{}{}{}{s.m.}{Massa preparada com pó de mármore, gesso ou cal fina.}{es.tu.que}{0}
\verb{esturjão}{}{Zool.}{"-ões}{}{s.m.}{Peixe de cuja ova se faz o caviar.}{es.tur.jão}{0}
\verb{esturrar}{}{}{}{}{v.t.}{Torrar muito; queimar, estorricar.}{es.tur.rar}{\verboinum{1}}
\verb{esturricar}{}{}{}{}{}{Var. de \textit{estorricar}.}{es.tur.ri.car}{0}
\verb{esturro}{}{}{}{}{s.m.}{Estado de coisa quase queimada, estorricada. }{es.tur.ro}{0}
\verb{esvaecer}{ê}{}{}{}{v.t.}{Tornar vão, vazio; desvanecer, desfazer.}{es.va.e.cer}{0}
\verb{esvaecer}{ê}{}{}{}{v.i.}{Perder o ânimo ou as forças; desmaiar.}{es.va.e.cer}{\verboinum{15}}
\verb{esvaecimento}{}{}{}{}{s.m.}{Ato ou efeito de esvaecer; desvanecimento, enfraquecimento, desânimo.}{es.va.e.ci.men.to}{0}
\verb{esvair}{}{}{}{}{v.t.}{Fazer evaporar; dissipar, desfazer.}{es.va.ir}{0}
\verb{esvair}{}{}{}{}{v.pron.}{Esgotar"-se, exaurir"-se, desfalecer.}{es.va.ir}{\verboinum{19}}
\verb{esvanecer}{ê}{}{}{}{v.t.}{Esvaecer.}{es.va.ne.cer}{\verboinum{15}}
\verb{esvaziamento}{}{}{}{}{s.m.}{Ato ou efeito de esvaziar; esgotamento, desocupação.}{es.va.zi.a.men.to}{0}
\verb{esvaziar}{}{}{}{}{v.t.}{Tornar vazio; desocupar, esgotar.}{es.va.zi.ar}{\verboinum{1}}
\verb{esverdeado}{}{}{}{}{adj.}{Próximo ao verde. }{es.ver.de.a.do}{0}
\verb{esverdear}{}{}{}{}{v.t.}{Tornar da cor verde ou semelhante a ela.}{es.ver.de.ar}{\verboinum{4}}
\verb{esvoaçar}{}{}{}{}{v.i.}{Bater as asas com força para voar.}{es.vo.a.çar}{0}
\verb{esvoaçar}{}{}{}{}{}{Agitar ou flutuar ao vento.}{es.vo.a.çar}{\verboinum{3}}
\verb{esvurmar}{}{}{}{}{v.t.}{Espremer o pus de uma ferida; fazer supurar.}{es.vur.mar}{\verboinum{1}}
\verb{ET}{}{}{}{}{s.m.}{Abrev.de \textit{extraterrestre}.}{ET}{0}
\verb{eta}{é}{}{}{}{s.m.}{Sétima letra do alfabeto grego.}{e.ta}{0}
\verb{eta}{ê}{}{}{}{interj.}{Expressão que denota alegria, admiração, enfado.}{e.ta}{0}
\verb{etano}{}{Quím.}{}{}{s.m.}{Hidrocarboneto gasoso, inodoro e incolor.}{e.ta.no}{0}
\verb{etanol}{ó}{Quím.}{"-óis}{}{s.m.}{Álcool etílico.}{e.ta.nol}{0}
\verb{etapa}{}{}{}{}{s.f.}{Cada uma das fases de uma atividade; estágio.}{e.ta.pa}{0}
\verb{etário}{}{}{}{}{adj.}{Relativo a idade.}{e.tá.rio}{0}
\verb{etc}{}{}{}{}{}{Abrev. da locução latina \textit{et cetera}, que significa \textit{e outras coisas}.}{etc.}{0}
\verb{éter}{}{}{}{}{s.m.}{O espaço celeste; a atmosfera.}{é.ter}{0}
\verb{éter}{}{Quím.}{}{}{}{Composto orgânico cuja molécula é formada por dois radicais de hidrocarbonetos ligados a um único átomo de oxigênio.}{é.ter}{0}
\verb{etéreo}{}{}{}{}{adj.}{Relativo a ou da natureza do éter.}{e.té.re.o}{0}
\verb{etéreo}{}{Fig.}{}{}{}{Celeste, sublime, elevado.}{e.té.re.o}{0}
\verb{eternidade}{}{}{}{}{s.f.}{Qualidade do que é eterno; imortalidade.}{e.ter.ni.da.de}{0}
\verb{eternidade}{}{}{}{}{}{Segundo várias crenças religiosas, o período após a morte.}{e.ter.ni.da.de}{0}
\verb{eternidade}{}{Fig.}{}{}{}{Tempo muito longo, indefinido.}{e.ter.ni.da.de}{0}
\verb{eternizar}{}{}{}{}{v.t.}{Tornar eterno na lembrança dos seres humanos; imortalizar. (\textit{Michelângelo eternizou o rei David em uma belíssima escultura.})}{e.ter.ni.zar}{\verboinum{1}}
\verb{eterno}{é}{}{}{}{adj.}{Que não tem princípio nem fim; imortal.}{e.ter.no}{0}
\verb{eterno}{é}{}{}{}{}{Imutável, inalterável.}{e.ter.no}{0}
\verb{ética}{}{}{}{}{s.f.}{Conjunto de regras e valores a que um grupo social está submetido; moral.}{é.ti.ca}{0}
\verb{ética}{}{Filos.}{}{}{}{Ciência que estuda esse conjunto.}{é.ti.ca}{0}
\verb{ético}{}{}{}{}{adj.}{Que se refere a ou é próprio da ética.}{é.ti.co}{0}
\verb{etileno}{}{Quím.}{}{}{s.m.}{Hidrocarboneto gasoso, incolor, obtido da desidrogenação do etano ou do craqueamento da nafta, muito usado em petroquímica.}{e.ti.le.no}{0}
\verb{etílico}{}{Quím.}{}{}{adj.}{Que contém o radical etila.}{e.tí.li.co}{0}
\verb{etílico}{}{Fig.}{}{}{}{Instigado, provocado pelo álcool; alcoólico. }{e.tí.li.co}{0}
\verb{étimo}{}{}{}{}{s.m.}{Palavra que é considerada a origem de outra; etimologia.}{é.ti.mo}{0}
\verb{etimologia}{}{}{}{}{s.f.}{Ciência que estuda a origem e o significado das palavras.}{e.ti.mo.lo.gi.a}{0}
\verb{etimologia}{}{}{}{}{}{A origem, exata ou provável, de uma palavra.}{e.ti.mo.lo.gi.a}{0}
\verb{etimológico}{}{}{}{}{adj.}{Relativo a etimologia.}{e.ti.mo.ló.gi.co}{0}
\verb{etimologista}{}{}{}{}{s.2g.}{Pessoa que se dedica à etimologia; etimólogo.}{e.ti.mo.lo.gis.ta}{0}
\verb{etimólogo}{}{}{}{}{s.m.}{Etimologista.}{e.ti.mó.lo.go}{0}
\verb{etiologia}{}{}{}{}{s.f.}{Estudo sobre a origem das coisas.}{e.ti.o.lo.gi.a}{0}
\verb{etiologia}{}{}{}{}{}{Ramo da medicina que estuda a origem das doenças.}{e.ti.o.lo.gi.a}{0}
\verb{etíope}{}{}{}{}{adj.2g.}{Relativo a Etiópia.}{e.tí.o.pe}{0}
\verb{etíope}{}{}{}{}{s.2g.}{Indivíduo natural ou habitante desse país.}{e.tí.o.pe}{0}
\verb{etiqueta}{ê}{}{}{}{s.f.}{Conjunto de cerimônias usadas numa corte ou nas residências de um chefe de Estado; formalidade, estilo.}{e.ti.que.ta}{0}
\verb{etiqueta}{ê}{}{}{}{}{Pequeno rótulo afixado em um produto que fornece alguma informação sobre ele.}{e.ti.que.ta}{0}
\verb{etiquetagem}{}{}{"-ens}{}{s.f.}{Ato ou efeito de etiquetar; rotulagem, rotulação.}{e.ti.que.ta.gem}{0}
\verb{etiquetar}{}{}{}{}{v.t.}{Pôr etiqueta ou rótulo; rotular.}{e.ti.que.tar}{\verboinum{1}}
\verb{etmoide}{}{Anat.}{}{}{s.m.}{Osso do crânio que forma as fossas nasais.}{et.moi.de}{0}
\verb{etnia}{}{}{}{}{s.f.}{Grupo humano que compartilha das mesmas características biológicas e culturais; raça.}{et.ni.a}{0}
\verb{étnico}{}{}{}{}{adj.}{Que se refere a ou é próprio de uma etnia, de um povo ou de uma raça; racial.}{ét.ni.co}{0}
\verb{etnografia}{}{}{}{}{s.f.}{Estudo descritivo dos aspectos sociais, culturais, religiosos etc. de uma etnia, de um povo ou de uma raça.}{et.no.gra.fi.a}{0}
\verb{etnógrafo}{}{}{}{}{s.m.}{Pessoa que estuda ou é especialista em etnografia.}{et.nó.gra.fo}{0}
\verb{etnologia}{}{}{}{}{s.f.}{Ciência que estuda os fatos e documentos recolhidos pela etnografia, no âmbito da antropologia cultural e social.}{et.no.lo.gia}{0}
\verb{etnólogo}{}{}{}{}{s.m.}{Pessoa que se dedica ao estudo da etnologia; etnologista.}{et.nó.lo.go}{0}
\verb{etnônimo}{}{}{}{}{s.m.}{Nome de povos, tribos, etnias, castas, raças, ou qualquer outro tipo de agrupamento humano; gentílico.}{et.nô.ni.mo}{0}
\verb{etologia}{}{}{}{}{s.f.}{Ramo da biologia que estuda os hábitos e o comportamento dos animais.}{e.to.lo.gi.a}{0}
\verb{etrusco}{}{}{}{}{adj.}{Relativo a Etrúria.	 }{e.trus.co}{0}
\verb{etrusco}{}{}{}{}{s.m.}{Natural ou habitante da Etrúria; tirreno.}{e.trus.co}{0}
\verb{etrusco}{}{}{}{}{}{A língua dos etruscos.  }{e.trus.co}{0}
\verb{Eu}{}{Quím.}{}{}{}{Símb. do \textit{európio}. }{Eu}{0}
\verb{eu}{}{Gram.}{}{}{pron.}{Pronome pessoal da primeira pessoa do singular.}{eu}{0}
\verb{eu}{}{}{}{}{s.m.}{A personalidade, a individualidade da pessoa humana.}{eu}{0}
\verb{eucalipto}{}{}{}{}{s.m.}{Nome comum a vários arbustos e árvores de crescimento rápido, muito usados em reflorestamento, e também pela madeira, para produção de celulose e extração de um óleo com propriedades medicinais.}{eu.ca.lip.to}{0}
\verb{eucaliptol}{ó}{Farm.}{}{}{s.m.}{Principal componente do óleo que se extrai das folhas de eucalipto, usado como expectorante.}{eu.ca.lip.tol}{0}
\verb{eucaristia}{}{Relig.}{}{}{s.f.}{O principal sacramento dos cristãos, mediante o qual o pão e o vinho consagrados se transformam no corpo e no sangue de Cristo.}{eu.ca.ris.ti.a}{0}
\verb{eufemismo}{}{}{}{}{s.m.}{Palavra ou locução que é usada para suavizar o que se quer dizer ou expressar, substituindo outra, de forma mais agradável ou polida. }{eu.fe.mis.mo}{0}
\verb{eufonia}{}{}{}{}{s.f.}{Som agradável ao ouvido.}{eu.fo.ni.a}{0}
\verb{eufonia}{}{}{}{}{}{Combinação agradável de sons, harmonia na sucessão de vogais e consoantes numa palavra ou frase.}{eu.fo.ni.a}{0}
\verb{eufônico}{}{}{}{}{adj.}{Relativo a eufonia; que é agradável aos ouvidos.}{eu.fô.ni.co}{0}
\verb{euforia}{}{}{}{}{s.f.}{Sensação de bem"-estar, de satisfação.}{eu.fo.ri.a}{0}
\verb{euforia}{}{}{}{}{}{Grande alegria, entusiasmo; exaltação.}{eu.fo.ri.a}{0}
\verb{eufórico}{}{}{}{}{adj.}{Relativo a euforia.}{eu.fó.ri.co}{0}
\verb{eufórico}{}{Por ext.}{}{}{}{Entusiasmado, animado, arrebatado.}{eu.fó.ri.co}{0}
\verb{eugenia}{}{}{}{}{s.f.}{Ciência que estuda os meios de melhorar a espécie humana, baseada em leis genéticas.}{eu.ge.ni.a}{0}
\verb{eulalia}{}{}{}{}{s.f.}{Maneira correta de falar; boa dicção.}{eu.la.li.a}{0}
\verb{eunuco}{}{}{}{}{s.m.}{Homem castrado que guardava os haréns no Oriente.}{eu.nu.co}{0}
\verb{eunuco}{}{Por ext.}{}{}{}{Homem impotente sexualmente.}{eu.nu.co}{0}
\verb{euro}{}{}{}{}{s.m.}{Moeda usada pelos países da União Europeia, em circulação desde 1999.}{eu.ro}{0}
\verb{eurodólar}{}{Econ.}{}{}{s.m.}{O dólar americano depositado ou investido em bancos europeus.}{eu.ro.dó.lar}{0}
\verb{europeização}{}{}{"-ões}{}{s.f.}{Ato ou efeito de europeizar. }{eu.ro.pe.i.za.ção}{0}
\verb{europeizar}{}{}{}{}{v.t.}{Tornar europeu, adaptar aos costumes, ao temperamento, à maneira ou ao estilo europeu. }{eu.ro.pe.i.zar}{\verboinum{1}}
\verb{europeu}{}{}{}{europeia}{adj.}{Relativo ou pertencente à Europa.}{eu.ro.peu}{0}
\verb{europeu}{}{}{}{europeia}{s.m.}{Natural ou habitante da Europa.}{eu.ro.peu}{0}
\verb{európio}{}{Quím.}{}{}{s.m.}{Elemento químico metálico, prateado, dúctil, reativo, da família dos lantanídeos (terras"-raras); usado em aparelhos de televisão e reatores nucleares. \elemento{63}{151.964}{Eu}.}{eu.ró.pio}{0}
\verb{eutanásia}{}{}{}{}{s.f.}{Morte serena e indolor.}{eu.ta.ná.sia}{0}
\verb{eutanásia}{}{}{}{}{}{Prática, amparada legalmente em alguns países, de abreviar a vida de um paciente comprovadamente incurável, de forma a poupá"-lo de mais sofrimento. }{eu.ta.ná.sia}{0}
\verb{evacuação}{}{}{"-ões}{}{s.f.}{Ato ou efeito de evacuar.}{e.va.cu.a.ção}{0}
\verb{evacuar}{}{}{}{}{v.t.}{Sair de um lugar, deixando"-o livre ou vazio; desocupar.}{e.va.cu.ar}{0}
\verb{evacuar}{}{}{}{}{}{Fazer sair do corpo; expelir.}{e.va.cu.ar}{0}
\verb{evacuar}{}{}{}{}{v.i.}{Defecar.}{e.va.cu.ar}{\verboinum{1}}
\verb{evadir}{}{}{}{}{v.t.}{Fugir, evitar, escapar (de responsabilidade, compromisso, perigo etc.).}{e.va.dir}{0}
\verb{evadir}{}{}{}{}{v.pron.}{Fugir, escapar de forma discreta ou furtiva; desaparecer, sumir"-se.}{e.va.dir}{\verboinum{18}}
\verb{evangelho}{é}{}{}{}{s.m.}{Doutrina de Cristo.}{e.van.ge.lho}{0}
\verb{evangelho}{é}{}{}{}{}{Cada um dos quatro primeiros livros do Novo Testamento.}{e.van.ge.lho}{0}
\verb{evangélico}{}{}{}{}{adj.}{Relativo ao Evangelho, ou conforme os seus ensinamentos.}{e.van.gé.li.co}{0}
\verb{evangélico}{}{}{}{}{s.m.}{Membro da igreja evangélica.}{e.van.gé.li.co}{0}
\verb{evangelismo}{}{}{}{}{s.m.}{Doutrina religiosa que se apoia no Evangelho.}{e.van.ge.lis.mo}{0}
\verb{evangelismo}{}{}{}{}{}{Pregação e difusão do Evangelho.}{e.van.ge.lis.mo}{0}
\verb{evangelista}{}{}{}{}{adj.2g.}{Evangelizador.}{e.van.ge.lis.ta}{0}
\verb{evangelista}{}{}{}{}{}{Protestante; evangélico.}{e.van.ge.lis.ta}{0}
\verb{evangelista}{}{}{}{}{s.m.}{Autor de um dos quatro livros do Evangelho.}{e.van.ge.lis.ta}{0}
\verb{evangelização}{}{}{"-ões}{}{s.f.}{Ato ou efeito de evangelizar.}{e.van.ge.li.za.ção}{0}
\verb{evangelizador}{ô}{}{}{}{adj.}{Que evangeliza; evangelizante.  }{e.van.ge.li.za.dor}{0}
\verb{evangelizador}{ô}{}{}{}{s.m.}{Pessoa que evangeliza; evangelista.}{e.van.ge.li.za.dor}{0}
\verb{evangelizar}{}{}{}{}{v.t.}{Pregar ou difundir o Evangelho.}{e.van.ge.li.zar}{\verboinum{1}}
\verb{evaporação}{}{}{"-ões}{}{s.f.}{Ato ou efeito de evaporar.}{e.va.po.ra.ção}{0}
\verb{evaporar}{}{}{}{}{v.t.}{Converter um líquido em vapor.}{e.va.po.rar}{0}
\verb{evaporar}{}{Fig.}{}{}{}{Desaparecer ou fazer algo ou alguém desaparecer; desvanecer, dissipar, sumir.}{e.va.po.rar}{\verboinum{1}}
\verb{evasão}{}{}{"-ões}{}{s.f.}{Ato ou efeito de evadir; fuga, escapada, retirada. }{e.va.são}{0}
\verb{evasão}{}{}{"-ões}{}{}{Evasiva, subterfúgio, pretexto.}{e.va.são}{0}
\verb{evasiva}{}{}{}{}{s.f.}{Desculpa vaga e ardilosa para evitar uma resposta; subterfúgio, pretexto. (\textit{O namorado respondeu com evasivas quando a moça perguntou como fora a festa.})}{e.va.si.va}{0}
\verb{evasivo}{}{}{}{}{adj.}{Que propicia ou facilita a evasão.}{e.va.si.vo}{0}
\verb{evasivo}{}{}{}{}{}{Que se vale de subterfúgio.}{e.va.si.vo}{0}
\verb{evento}{}{}{}{}{s.m.}{Acontecimento, fato, ocorrência.}{e.ven.to}{0}
\verb{evento}{}{}{}{}{}{Oportunidade, ocasião, momento.}{e.ven.to}{0}
\verb{eventual}{}{}{"-ais}{}{adj.2g.}{Que pode ocorrer ou não, que depende de acontecimento incerto; casual, fortuito, acidental. }{e.ven.tu.al}{0}
\verb{eventualidade}{}{}{}{}{s.f.}{Qualidade do que é eventual.}{e.ven.tu.a.li.da.de}{0}
\verb{eventualidade}{}{}{}{}{}{Acontecimento inesperado ou incerto; acaso, evento, contingência. }{e.ven.tu.a.li.da.de}{0}
\verb{eversão}{}{}{"-ões}{}{s.f.}{Grande estrago, ruína; desmoronamento, assolação.}{e.ver.são}{0}
\verb{everter}{ê}{}{}{}{v.t.}{Causar destruição, ruína; destruir.}{e.ver.ter}{\verboinum{12}}
\verb{evicção}{}{}{"-ões}{}{s.f.}{Perda da posse para o verdadeiro dono, exigida juridicamente.}{e.vic.ção}{0}
\verb{evicto}{}{}{}{}{adj.}{Diz"-se de pessoa ou coisa que está sujeito à evicção.}{e.vic.to}{0}
\verb{evidência}{}{}{}{}{s.f.}{Verdade que logo se nota, sem precisar de provas.}{e.vi.dên.cia}{0}
\verb{evidência}{}{}{}{}{}{Prova clara de alguma coisa.}{e.vi.dên.cia}{0}
\verb{evidenciar}{}{}{}{}{v.t.}{Fazer alguma coisa ficar evidente; comprovar.}{e.vi.den.ci.ar}{\verboinum{6}}
\verb{evidente}{}{}{}{}{adj.2g.}{Que se sabe ser certo e verdadeiro sem se precisar de outra prova; claro, incontestável.}{e.vi.den.te}{0}
\verb{eviscerar}{}{}{}{}{v.t.}{Tirar as vísceras.}{e.vis.ce.rar}{\verboinum{1}}
\verb{evitar}{}{}{}{}{v.t.}{Fazer com que alguma coisa deixe de acontecer; impedir.}{e.vi.tar}{0}
\verb{evitar}{}{}{}{}{}{Esforçar"-se por não se encontrar com pessoa ou coisa; esquivar"-se.}{e.vi.tar}{0}
\verb{evitar}{}{}{}{}{}{Esforçar"-se por não ser obrigado a fazer alguma coisa.}{e.vi.tar}{\verboinum{1}}
\verb{evitável}{}{}{"-eis}{}{adj.2g.}{Que se pode ou deve evitar.}{e.vi.tá.vel}{0}
\verb{evocação}{}{}{"-ões}{}{s.f.}{Ato de evocar.}{e.vo.ca.ção}{0}
\verb{evocação}{}{}{"-ões}{}{}{Função da memória pela qual as lembranças são chamadas de novo à consciência.}{e.vo.ca.ção}{0}
\verb{evocar}{}{}{}{}{v.t.}{Fazer alguma coisa vir à lembrança; lembrar, recordar.}{e.vo.car}{0}
\verb{evocar}{}{}{}{}{s.m.}{Pedir que algum espírito se manifeste; chamar, invocar.}{e.vo.car}{\verboinum{2}}
\verb{evocativo}{}{}{}{}{adj.}{Que serve para evocar.}{e.vo.ca.ti.vo}{0}
\verb{evolar"-se}{}{}{}{}{v.pron.}{Desaparecer no espaço.}{e.vo.lar"-se}{0}
\verb{evolar"-se}{}{}{}{}{}{Elevar"-se, voando.}{e.vo.lar"-se}{0}
\verb{evolar"-se}{}{}{}{}{}{Desvanecer"-se, volatizar"-se.}{e.vo.lar"-se}{\verboinum{1}}
\verb{evolução}{}{}{"-ões}{}{s.f.}{Mudança vagarosa e natural de pessoa ou coisa; transformação progressiva.}{e.vo.lu.ção}{0}
\verb{evolução}{}{}{"-ões}{}{}{Movimentação harmônica em ginástica, dança, desfile etc.}{e.vo.lu.ção}{0}
\verb{evolucionar}{}{}{}{}{v.t.}{Passar por processo gradual de evolução ou transformação; evolver, evoluir.}{e.vo.lu.ci.o.nar}{0}
\verb{evolucionar}{}{}{}{}{v.i.}{Executar evoluções, movimentos, deslocamentos graduais e harmônicos.}{e.vo.lu.ci.o.nar}{\verboinum{1}}
\verb{evolucionismo}{}{}{}{}{s.m.}{Qualquer teoria fundada na noção de evolução, especialmente a dos seres vivos.}{e.vo.lu.ci.o.nis.mo}{0}
\verb{evolucionista}{}{}{}{}{adj.2g.}{Relativo ao evolucionismo.}{e.vo.lu.ci.o.nis.ta}{0}
\verb{evolucionista}{}{}{}{}{adj.2g.}{Diz"-se de partidário do evolucionismo.}{e.vo.lu.ci.o.nis.ta}{0}
\verb{evoluído}{}{}{}{}{adj.}{Que evoluiu, que progrediu.}{e.vo.lu.í.do}{0}
\verb{evoluir}{}{}{}{}{v.i.}{Passar de maneira natural e vagarosa de um estado para outro; transformar"-se progressivamente.}{e.vo.lu.ir}{0}
\verb{evoluir}{}{}{}{}{}{Fazer movimentos programados.}{e.vo.lu.ir}{\verboinum{26}}
\verb{evolutivo}{}{}{}{}{adj.}{Que é suscetível de evolução ou que produz evolução; que se transforma ou aperfeiçoa.}{e.vo.lu.ti.vo}{0}
\verb{evolver}{ê}{}{}{}{v.t.}{Desenvolver"-se gradualmente.}{e.vol.ver}{\verboinum{12}}
\verb{evulsão}{}{}{"-ões}{}{s.f.}{Ato de extrair algo violentamente.}{e.vul.são}{0}
\verb{exação}{z}{}{"-ões}{}{s.f.}{Cobrança de impostos, taxas etc.}{e.xa.ção}{0}
\verb{exação}{z}{}{"-ões}{}{}{Exatidão, precisão, correção.}{e.xa.ção}{0}
\verb{exacerbação}{z}{}{"-ões}{}{s.f.}{Ato ou efeito de exacerbar.}{e.xa.cer.ba.ção}{0}
\verb{exacerbação}{z}{}{"-ões}{}{}{Ato ou efeito de irritar, de exasperar.}{e.xa.cer.ba.ção}{0}
\verb{exacerbação}{z}{Med.}{"-ões}{}{}{Aumento da intensidade de uma doença, dos sinais e dos sintomas.}{e.xa.cer.ba.ção}{0}
\verb{exacerbar}{z}{}{}{}{v.t.}{Tornar mais intenso, mais violento.}{e.xa.cer.bar}{0}
\verb{exacerbar}{z}{}{}{}{}{Irritar.}{e.xa.cer.bar}{\verboinum{1}}
\verb{exageração}{z}{}{"-ões}{}{s.f.}{Exagero.}{e.xa.ge.ra.ção}{0}
\verb{exagerado}{z}{}{}{}{adj.}{Em que há exagero.}{e.xa.ge.ra.do}{0}
\verb{exagerado}{z}{}{}{}{s.m.}{Indivíduo que tem o hábito de exagerar.}{e.xa.ge.ra.do}{0}
\verb{exagerar}{z}{}{}{}{v.t.}{Fazer ou dizer algo com excesso.}{e.xa.ge.rar}{0}
\verb{exagerar}{z}{}{}{}{}{Aumentar em demasia.}{e.xa.ge.rar}{\verboinum{1}}
\verb{exagero}{z\ldots{}ê}{}{}{}{s.m.}{Aumento, excesso.}{e.xa.ge.ro}{0}
\verb{exagero}{z\ldots{}ê}{}{}{}{}{Qualquer coisa de valor ou dimensões muito além do normal.}{e.xa.ge.ro}{0}
\verb{exalação}{z}{}{"-ões}{}{s.f.}{Ato ou efeito de exalar.}{e.xa.la.ção}{0}
\verb{exalação}{z}{}{"-ões}{}{}{Emanação sob a forma de vapores, odores etc., de um corpo sólido ou líquido.}{e.xa.la.ção}{0}
\verb{exalação}{z}{}{"-ões}{}{}{Luminosidade rápida, proveniente de substâncias gasosas que escapam do solo e se inflamam ao entrarem em contato com o ar.}{e.xa.la.ção}{0}
\verb{exalar}{z}{}{}{}{v.t.}{Lançar de si; emanar.}{e.xa.lar}{0}
\verb{exalar}{z}{}{}{}{v.pron.}{Evaporar"-se.}{e.xa.lar}{\verboinum{1}}
\verb{exalçar}{z}{}{}{}{v.t.}{Exaltar.}{e.xal.çar}{\verboinum{3}}
\verb{exaltação}{z}{}{"-ões}{}{s.f.}{Ato ou efeito de exaltar.}{e.xal.ta.ção}{0}
\verb{exaltação}{z}{}{"-ões}{}{}{Grande excitação de ânimo; descontrole; exagero nas ideias ou atitudes; entusiasmo.}{e.xal.ta.ção}{0}
\verb{exaltação}{z}{}{"-ões}{}{}{Louvor entusiástico; glorificação.}{e.xal.ta.ção}{0}
\verb{exaltação}{z}{}{"-ões}{}{}{Irritação, exasperação.}{e.xal.ta.ção}{0}
\verb{exaltado}{z}{}{}{}{adj.}{Que se exaltou.}{e.xal.ta.do}{0}
\verb{exaltado}{z}{}{}{}{}{Que se exalta facilmente; irritado, exasperado.}{e.xal.ta.do}{0}
\verb{exaltar}{z}{}{}{}{v.t.}{Tornar sublime, grandioso; engrandecer, glorificar.}{e.xal.tar}{0}
\verb{exaltar}{z}{}{}{}{}{Louvar, enaltecer, elogiar.}{e.xal.tar}{0}
\verb{exaltar}{z}{}{}{}{}{Irritar profundamente; enfurecer.}{e.xal.tar}{0}
\verb{exaltar}{z}{}{}{}{v.pron.}{Atingir o mais alto grau de intensidade ou energia.}{e.xal.tar}{\verboinum{1}}
\verb{exame}{z}{}{}{}{s.m.}{Observação minuciosa.}{e.xa.me}{0}
\verb{exame}{z}{}{}{}{}{Teste, prova.}{e.xa.me}{0}
\verb{exame}{z}{}{}{}{}{Análise clínica ou de laboratório.}{e.xa.me}{0}
\verb{examinador}{z\ldots{}ô}{}{}{}{adj.}{Diz"-se daquele que examina.}{e.xa.mi.na.dor}{0}
\verb{examinando}{z}{}{}{}{s.m.}{O que será ou que está sendo examinado.}{e.xa.mi.nan.do}{0}
\verb{examinando}{z}{}{}{}{}{Candidato que se apresenta para ser examinado com o fim de obter grau, licença etc., caso seja aprovado no exame.}{e.xa.mi.nan.do}{0}
\verb{examinar}{z}{}{}{}{v.t.}{Fazer perguntas a uma pessoa para dar uma nota; arguir.}{e.xa.mi.nar}{0}
\verb{examinar}{z}{}{}{}{}{Observar com cuidado pessoa ou coisa para descobrir algum mau funcionamento; diagnosticar.}{e.xa.mi.nar}{0}
\verb{examinar}{z}{}{}{}{}{Olhar pessoa ou coisa com cuidado para saber como ela é; analisar.}{e.xa.mi.nar}{\verboinum{1}}
\verb{exangue}{z}{}{}{}{adj.2g.}{Que ficou sem sangue.}{e.xan.gue}{0}
\verb{exangue}{z}{}{}{}{}{Enfraquecido.}{e.xan.gue}{0}
\verb{exânime}{z}{}{}{}{adj.2g.}{Que deixou de sentir dores e parece estar dormindo; desmaiado, desfalecido.}{e.xâ.ni.me}{0}
\verb{exânime}{z}{}{}{}{}{Sem forças; combalido, enfraquecido.}{e.xâ.ni.me}{0}
\verb{exantema}{z}{}{}{}{s.f.}{Erupção cutânea.}{e.xan.te.ma}{0}
\verb{exarar}{z}{}{}{}{v.t.}{Abrir entalhe ou fazer marca em alguma coisa; entalhar, talhar, lavrar.}{e.xa.rar}{0}
\verb{exarar}{z}{}{}{}{}{Registrar por escrito; lavrar.}{e.xa.rar}{\verboinum{1}}
\verb{exasperação}{z}{}{"-ões}{}{s.f.}{Ato ou efeito de exasperar; irritação, exacerbação.}{e.xas.pe.ra.ção}{0}
\verb{exasperador}{z\ldots{}ô}{}{}{}{adj.}{Que exaspera, irrita, exalta.}{e.xas.pe.ra.dor}{0}
\verb{exasperar}{z}{}{}{}{v.t.}{Fazer perder a paciência; enfurecer, irritar,}{e.xas.pe.rar}{\verboinum{1}}
\verb{exaspero}{z\ldots{}ê}{}{}{}{s.m.}{Exasperação.}{e.xas.pe.ro}{0}
\verb{exatidão}{z}{}{"-ões}{}{s.f.}{Ausência de erro ou imprecisão.}{e.xa.ti.dão}{0}
\verb{exato}{z}{}{}{}{adj.}{Sem erro; certo, correto, preciso.}{e.xa.to}{0}
\verb{exator}{z\ldots{}ô}{}{}{}{s.m.}{Cobrador ou arrecadador de impostos e contribuições; coletor.}{e.xa.tor}{0}
\verb{exatoria}{z}{Bras.}{}{}{s.f.}{Cargo ou funções de exator.}{e.xa.to.ri.a}{0}
\verb{exatoria}{z}{}{}{}{}{Repartição fiscal para cobranças de impostos; coletoria.}{e.xa.to.ri.a}{0}
\verb{exaurir}{z}{}{}{}{v.t.}{Esgotar inteiramente.}{e.xau.rir}{0}
\verb{exaurir}{z}{}{}{}{}{Tornar exausto.}{e.xau.rir}{\verboinum{34}\verboirregular{\emph{def.} exaurimos, exauris}}
\verb{exaurível}{z}{}{"-eis}{}{adj.2g.}{Que se pode exaurir.}{e.xau.rí.vel}{0}
\verb{exaustão}{z}{}{"-ões}{}{s.f.}{Ato ou efeito de exaurir; esgotamento, estresse.}{e.xaus.tão}{0}
\verb{exaustivo}{z}{}{}{}{adj.}{Que esgota ou se destina a esgotar, que abrange até os mínimos pormenores.}{e.xaus.ti.vo}{0}
\verb{exaustivo}{z}{}{}{}{}{Extremamente fatigante.}{e.xaus.ti.vo}{0}
\verb{exausto}{z}{}{}{}{adj.}{Completamente sem forças; esgotado, estressado.}{e.xaus.to}{0}
\verb{exaustor}{z\ldots{}ô}{}{}{}{s.m.}{Aparelho que aspira o ar viciado ou engordurado de um recinto.}{e.xaus.tor}{0}
\verb{exautorar}{z}{}{}{}{v.t.}{Retirar a autoridade conferida a.}{e.xau.to.rar}{0}
\verb{exautorar}{z}{}{}{}{}{Retirar de alguém cargo, insígnias, honrarias etc.}{e.xau.to.rar}{0}
\verb{exautorar}{z}{Fig.}{}{}{}{Retirar o prestígio de.}{e.xau.to.rar}{\verboinum{1}}
\verb{exceção}{s}{}{"-ões}{}{s.f.}{Pessoa ou coisa que fica fora de alguma regra.}{ex.ce.ção}{0}
\verb{exceção}{s}{}{"-ões}{}{}{Direito particular de uma pessoa; prerrogativa, privilégio.}{ex.ce.ção}{0}
\verb{excedente}{s}{}{}{}{adj.2g.}{Que excede; excessivo.}{ex.ce.den.te}{0}
\verb{excedente}{s}{}{}{}{s.m.}{Aquilo que excede o normal, o permitido ou o necessário; excesso, sobra.}{ex.ce.den.te}{0}
\verb{exceder}{s}{}{}{}{v.t.}{Atingir um valor ou quantidade superior a.}{ex.ce.der}{\verboinum{12}}
\verb{excedível}{s}{}{"-eis}{}{adj.2g.}{Que pode ser excedido.}{ex.ce.dí.vel}{0}
\verb{excelência}{s}{}{}{}{s.f.}{Qualidade de excelente.}{ex.ce.lên.cia}{0}
\verb{excelência}{s}{}{}{}{}{Tratamento usado ao se dirigir a pessoas de altos cargos e a chefes do poder executivo.}{ex.ce.lên.cia}{0}
\verb{excelente}{s}{}{}{}{adj.2g.}{Que excele; muito bom.}{ex.ce.len.te}{0}
\verb{excelentíssimo}{s}{}{}{}{adj.}{Superlativo absoluto sintético de \textit{excelente}.}{ex.ce.len.tís.si.mo}{0}
\verb{excelentíssimo}{s}{}{}{}{}{Tratamento dado a pessoas de altos cargos e a chefes do poder executivo.}{ex.ce.len.tís.si.mo}{0}
\verb{exceler}{s}{}{}{}{v.i.}{Ser superior a, acima de ou melhor que os outros; destacar"-se.}{ex.ce.ler}{\verboinum{12}}
\verb{excelsitude}{s}{}{}{}{s.f.}{Qualidade de excelso.}{ex.cel.si.tu.de}{0}
\verb{excelso}{s\ldots{}é}{}{}{}{adj.}{Sublime, elevado, admirável.}{ex.cel.so}{0}
\verb{excentricidade}{s}{}{}{}{s.f.}{Qualidade de excêntrico.}{ex.cen.tri.ci.da.de}{0}
\verb{excentricidade}{s}{}{}{}{}{Afastamento em relação ao centro.}{ex.cen.tri.ci.da.de}{0}
\verb{excêntrico}{s}{}{}{}{adj.}{Diz"-se de indivíduo esquisito, extravagante.}{ex.cên.tri.co}{0}
\verb{excêntrico}{s}{}{}{}{}{Afastado em relação ao centro.}{ex.cên.tri.co}{0}
\verb{excêntrico}{s}{}{}{}{}{Diz"-se de círculos, ângulos que não têm o mesmo centro.}{ex.cên.tri.co}{0}
\verb{excepcional}{s}{}{"-ais}{}{adj.2g.}{Que constitui exceção; diferente, privilegiado, excêntrico, incomum.}{ex.cep.ci.o.nal}{0}
\verb{excepcional}{s}{}{"-ais}{}{}{Excelente, extraordinário.}{ex.cep.ci.o.nal}{0}
\verb{excepcional}{s}{}{"-ais}{}{}{Diz"-se de indivíduo portador de deficiência física, mental ou sensorial.}{ex.cep.ci.o.nal}{0}
\verb{excerto}{s\ldots{}ê}{}{}{}{s.m.}{Trecho selecionado de uma obra; fragmento.}{ex.cer.to}{0}
\verb{excessivo}{s}{}{}{}{adj.}{Em quantidade maior do que o necessário; demasiado, exagerado.}{ex.ces.si.vo}{0}
\verb{excesso}{s\ldots{}é}{}{}{}{s.m.}{Aquilo que ultrapassa o normal, o permitido, o necessário.}{ex.ces.so}{0}
\verb{excesso}{s\ldots{}é}{}{}{}{}{Diferença para mais resultante da comparação entre duas quantidades.}{ex.ces.so}{0}
\verb{exceto}{s\ldots{}é}{}{}{}{prep.}{Com exclusão de; salvo, afora.}{ex.ce.to}{0}
\verb{excetuar}{s}{}{}{}{v.t.}{Excluir, isentar.}{ex.ce.tu.ar}{\verboinum{1}}
\verb{excipiente}{s}{}{}{}{s.m.}{Substância geralmente líquida e sem efeito que serve de veículo a certas fórmulas e medicamentos.}{ex.ci.pi.en.te}{0}
\verb{excitabilidade}{s}{}{}{}{s.f.}{Qualidade de excitável.}{ex.ci.ta.bi.li.da.de}{0}
\verb{excitação}{s}{}{"-ões}{}{s.f.}{Ato ou efeito de excitar; exaltação, irritação, agitação.}{ex.ci.ta.ção}{0}
\verb{excitamento}{s}{}{}{}{s.m.}{Excitação.}{ex.ci.ta.men.to}{0}
\verb{excitante}{s}{}{}{}{adj.2g.}{Que excita.}{ex.ci.tan.te}{0}
\verb{excitar}{s}{}{}{}{v.t.}{Ativar a ação de.}{ex.ci.tar}{0}
\verb{excitar}{s}{}{}{}{}{Estimular, animar.}{ex.ci.tar}{0}
\verb{excitar}{s}{}{}{}{}{Encolerizar, irritar.}{ex.ci.tar}{\verboinum{1}}
\verb{excitável}{s}{}{"-eis}{}{adj.2g.}{Que se pode excitar.}{ex.ci.tá.vel}{0}
\verb{exclamação}{s}{}{"-ões}{}{s.f.}{Ato de exclamar.}{ex.cla.ma.ção}{0}
\verb{exclamar}{s}{}{}{}{v.t.}{Pronunciar em voz alta e em tom de alegria, raiva, dor.}{ex.cla.mar}{\verboinum{1}}
\verb{exclamativo}{s}{}{}{}{adj.}{Que manifesta exclamação.}{ex.cla.ma.ti.vo}{0}
\verb{excludente}{s}{}{}{}{adj.2g.}{Que exclui.}{ex.clu.den.te}{0}
\verb{excluído}{s}{}{}{}{adj.}{Que se excluiu; que é ou foi deixado de fora.}{ex.clu.í.do}{0}
\verb{excluir}{s}{}{}{}{v.t.}{Pôr de lado; separar.}{ex.clu.ir}{0}
\verb{excluir}{s}{}{}{}{}{Eliminar, omitir, afastar.}{ex.clu.ir}{0}
\verb{excluir}{s}{}{}{}{}{Ser incompatível com.}{ex.clu.ir}{0}
\verb{excluir}{s}{}{}{}{v.pron.}{Colocar"-se de fora; isentar"-se, privar"-se.}{ex.clu.ir}{\verboinum{26}}
\verb{exclusão}{s}{}{"-ões}{}{s.f.}{Ato ou efeito de excluir.}{ex.clu.são}{0}
\verb{exclusive}{s}{}{}{}{adv.}{Sem a inclusão de.}{ex.clu.si.ve}{0}
\verb{exclusividade}{s}{}{}{}{s.f.}{Qualidade de exclusivo.}{ex.clu.si.vi.da.de}{0}
\verb{exclusivismo}{s}{}{}{}{s.m.}{Sistema ou doutrina que tende a praticar a exclusão.}{ex.clu.si.vis.mo}{0}
\verb{exclusivista}{s}{}{}{}{adj.2g.}{Relativo a ou partidário do exclusivismo.}{ex.clu.si.vis.ta}{0}
\verb{exclusivista}{s}{}{}{}{}{Inflexível, individualista.}{ex.clu.si.vis.ta}{0}
\verb{exclusivo}{s}{}{}{}{adj.}{Que exclui.}{ex.clu.si.vo}{0}
\verb{exclusivo}{s}{}{}{}{}{Cujo acesso ou usufruto é permitido somente a certas pessoas; privativo, restrito.}{ex.clu.si.vo}{0}
\verb{excogitar}{s}{}{}{}{v.t.}{Imaginar, inventar, cogitar.}{ex.co.gi.tar}{0}
\verb{excogitar}{s}{}{}{}{}{Pesquisar, investigar, esquadrinhar.}{ex.co.gi.tar}{0}
\verb{excogitar}{s}{}{}{}{v.i.}{Meditar, refletir.}{ex.co.gi.tar}{\verboinum{1}}
\verb{excomungado}{s}{}{}{}{adj.}{Que sofreu excomunhão.}{ex.co.mun.ga.do}{0}
\verb{excomungado}{s}{}{}{}{}{Detestável, maldito, péssimo.}{ex.co.mun.ga.do}{0}
\verb{excomungado}{s}{}{}{}{s.m.}{O diabo.}{ex.co.mun.ga.do}{0}
\verb{excomungar}{s}{}{}{}{v.t.}{Expulsar da Igreja Católica.}{ex.co.mun.gar}{0}
\verb{excomungar}{s}{}{}{}{}{Amaldiçoar, esconjurar.}{ex.co.mun.gar}{\verboinum{5}}
\verb{excomunhão}{s}{}{"-ões}{}{s.f.}{Pena da Igreja Católica que consiste em privar alguém dos bens espirituais comuns aos fiéis.}{ex.co.mu.nhão}{0}
\verb{excomunhão}{s}{}{"-ões}{}{}{Ato ou efeito de excomungar.}{ex.co.mu.nhão}{0}
\verb{excreção}{s}{}{"-ões}{}{s.f.}{Ato pelo qual o organismo dos seres vivos elimina substâncias inúteis ao seu funcionamento.}{ex.cre.ção}{0}
\verb{excreção}{s}{}{"-ões}{}{}{O material que foi excretado; dejeto, excreto.}{ex.cre.ção}{0}
\verb{excrementício}{s}{}{}{}{adj.}{Relativo a excremento ou a excreção.}{ex.cre.men.tí.cio}{0}
\verb{excremento}{s}{}{}{}{s.m.}{Material excretado pelo organismo dos seres vivos; excreção.}{ex.cre.men.to}{0}
\verb{excrescência}{s}{}{}{}{s.f.}{Saliência, proeminência.}{ex.cres.cên.cia}{0}
\verb{excrescência}{s}{}{}{}{}{Superfluidade.}{ex.cres.cên.cia}{0}
\verb{excrescência}{s}{Med.}{}{}{}{Tumor na superfície de um órgão, tecido ou mucosa.}{ex.cres.cên.cia}{0}
\verb{excretar}{s}{}{}{}{v.t.}{Expelir, evacuar, eliminar.}{ex.cre.tar}{\verboinum{1}}
\verb{excretor}{s\ldots{}ô}{}{}{}{adj.}{Que excreta.}{ex.cre.tor}{0}
\verb{excruciante}{s}{}{}{}{adj.2g.}{Que excrucia; doloroso, pungente.}{ex.cru.ci.an.te}{0}
\verb{excruciar}{s}{}{}{}{v.t.}{Afligir em demasia; torturar.}{ex.cru.ci.ar}{\verboinum{6}}
\verb{excursão}{s}{}{"-ões}{}{s.f.}{Passeio ou viagem com fins recreativos e geralmente acompanhado por um guia.}{ex.cur.são}{0}
\verb{excursão}{s}{}{"-ões}{}{}{Incursão em território inimigo; invasão.}{ex.cur.são}{0}
\verb{excursão}{s}{Fig.}{"-ões}{}{}{Digressão, divagação.}{ex.cur.são}{0}
\verb{excursionar}{s}{}{}{}{v.i.}{Fazer excursão.}{ex.cur.si.o.nar}{\verboinum{1}}
\verb{excursionista}{s}{}{}{}{s.2g.}{Indivíduo que faz ou promove excursões de recreação ou estudo.}{ex.cur.si.o.nis.ta}{0}
\verb{execração}{z}{}{"-ões}{}{s.f.}{Ato ou efeito de execrar; aversão, horror.}{e.xe.cra.ção}{0}
\verb{execração}{z}{}{"-ões}{}{}{Maldição, imprecação.}{e.xe.cra.ção}{0}
\verb{execração}{z}{Relig.}{"-ões}{}{}{Perda da condição de ungido.}{e.xe.cra.ção}{0}
\verb{execrando}{z}{}{}{}{adj.}{Digno de execração; execrável, abominável.}{e.xe.cran.do}{0}
\verb{execrar}{z}{}{}{}{v.t.}{Ter horror a; abominar, amaldiçoar.}{e.xe.crar}{0}
\verb{execrar}{z}{}{}{}{}{Desejar mal a.}{e.xe.crar}{\verboinum{1}}
\verb{execrável}{z}{}{"-eis}{}{adj.2g.}{Digno de execração; abominável.}{e.xe.crá.vel}{0}
\verb{execução}{z}{}{"-ões}{}{s.f.}{Ato ou efeito de executar.}{e.xe.cu.ção}{0}
\verb{execução}{z}{}{"-ões}{}{}{Cumprimento da pena de morte.}{e.xe.cu.ção}{0}
\verb{executante}{z}{}{}{}{adj.2g.}{Que executa.}{e.xe.cu.tan.te}{0}
\verb{executar}{z}{}{}{}{v.t.}{Realizar, efetivar, cumprir.}{e.xe.cu.tar}{0}
\verb{executar}{z}{}{}{}{}{Interpretar obra musical, canto, papel.}{e.xe.cu.tar}{0}
\verb{executar}{z}{}{}{}{}{Matar por ordem judicial; cumprir a pena de morte.}{e.xe.cu.tar}{\verboinum{1}}
\verb{executivo}{z}{}{}{}{adj.}{Que executa; executor.}{e.xe.cu.ti.vo}{0}
\verb{executivo}{z}{}{}{}{}{Decidido, resoluto.}{e.xe.cu.ti.vo}{0}
\verb{executivo}{z}{}{}{}{s.m.}{Alto funcionário de uma empresa, geralmente com poder decisório.}{e.xe.cu.ti.vo}{0}
\verb{executor}{z\ldots{}ô}{}{}{}{adj.}{Que executa.}{e.xe.cu.tor}{0}
\verb{executório}{z}{}{}{}{adj.}{Que se pode executar.}{e.xe.cu.tó.rio}{0}
\verb{exegese}{z\ldots{}é}{}{}{}{s.f.}{Comentário, explicação ou interpretação detalhada de textos.}{e.xe.ge.se}{0}
\verb{exegeta}{z\ldots{}é}{}{}{}{s.m.}{Indivíduo que se dedica à exegese.}{e.xe.ge.ta}{0}
\verb{exegética}{z}{}{}{}{s.f.}{Ramo da teologia que se dedica à explanação e à interpretação da Bíblia.}{e.xe.gé.ti.ca}{0}
\verb{exegética}{z}{Por ext.}{}{}{}{Explanação e interpretação de qualquer matéria.}{e.xe.gé.ti.ca}{0}
\verb{exemplar}{z}{}{}{}{adj.2g.}{Que serve de exemplo; modelar.}{e.xem.plar}{0}
\verb{exemplar}{z}{}{}{}{s.2g.}{Cada um dos indivíduos da mesma espécie; espécime.}{e.xem.plar}{0}
\verb{exemplar}{z}{}{}{}{}{Cada uma das unidades de alguma coisa impressa.}{e.xem.plar}{0}
\verb{exemplaridade}{z}{}{}{}{s.f.}{Qualidade do que é exemplar, daquilo que serve como modelo.}{e.xem.pla.ri.da.de}{0}
\verb{exemplário}{z}{}{}{}{s.m.}{Conjunto, coleção ou livro de exemplo.}{e.xem.plá.rio}{0}
\verb{exemplificação}{z}{}{"-ões}{}{s.f.}{Ato ou efeito de exemplificar.}{e.xem.pli.fi.ca.ção}{0}
\verb{exemplificação}{z}{}{"-ões}{}{}{Porção de palavras ou frases que se menciona para demonstrar alguma coisa; elucidação por meio de exemplos.}{e.xem.pli.fi.ca.ção}{0}
\verb{exemplificar}{z}{}{}{}{v.t.}{Dar um exemplo para explicar alguma coisa.}{e.xem.pli.fi.car}{\verboinum{2}}
\verb{exemplo}{z}{}{}{}{s.m.}{Tudo o que pode ou deve ser imitado; modelo.}{e.xem.plo}{0}
\verb{exemplo}{z}{}{}{}{}{Tudo o que pode servir de lição.}{e.xem.plo}{0}
\verb{exemplo}{z}{}{}{}{}{Fato que se usa como prova de uma asserção.}{e.xem.plo}{0}
\verb{exemplo}{z}{}{}{}{}{Palavra, frase ou passagem de um autor, citada para confirmar regra, apoiar afirmação ou definição etc.}{e.xem.plo}{0}
\verb{exéquias}{z}{}{}{}{s.f.pl.}{Cerimônias ou honras fúnebres.}{e.xé.qui.as}{0}
\verb{exequível}{z}{}{"-eis}{}{adj.2g.}{Que se pode realizar; executável, viável.}{e.xe.quí.vel}{0}
\verb{exercer}{z}{}{}{}{v.t.}{Fazer o trabalho de uma profissão; atuar, desempenhar.}{e.xer.cer}{0}
\verb{exercer}{z}{}{}{}{}{Fazer uso de algum direito; exercitar, praticar.}{e.xer.cer}{0}
\verb{exercer}{z}{}{}{}{}{Fazer alguma força se manifestar sobre pessoa ou coisa.}{e.xer.cer}{\verboinum{15}}
\verb{exercício}{z}{}{}{}{s.m.}{Ação de exercer; desempenho, prática.}{e.xer.cí.cio}{0}
\verb{exercício}{z}{}{}{}{}{Atividade do corpo para treino ou benefício da saúde.}{e.xer.cí.cio}{0}
\verb{exercício}{z}{}{}{}{}{Atividade que serve para aprender alguma coisa; lição, trabalho.}{e.xer.cí.cio}{0}
\verb{exercício}{z}{}{}{}{}{Cada uma das etapas de uma atividade econômica.}{e.xer.cí.cio}{0}
\verb{exercitar}{z}{}{}{}{v.t.}{Praticar, professar, exercer.}{e.xer.ci.tar}{0}
\verb{exercitar}{z}{}{}{}{}{Pôr em ação; fazer valer.}{e.xer.ci.tar}{0}
\verb{exercitar}{z}{}{}{}{}{Adestrar; habilitar.}{e.xer.ci.tar}{0}
\verb{exercitar}{z}{}{}{}{}{Dedicar"-se a; cultivar.}{e.xer.ci.tar}{0}
\verb{exercitar}{z}{}{}{}{v.pron.}{Adestrar"-se mediante estudo ou exercício.}{e.xer.ci.tar}{\verboinum{1}}
\verb{exército}{z}{}{}{}{s.m.}{A força armada terrestre de uma nação.}{e.xér.ci.to}{0}
\verb{exército}{z}{}{}{}{}{Grande quantidade de pessoas; multidão.}{e.xér.ci.to}{0}
\verb{exibição}{z}{}{"-ões}{}{s.f.}{Ato ou efeito de exibir.}{e.xi.bi.ção}{0}
\verb{exibição}{z}{}{"-ões}{}{}{Ação de mostrar ou demonstrar algo, especialmente ao público.}{e.xi.bi.ção}{0}
\verb{exibicionismo}{z}{}{}{}{s.m.}{Mania de exibir"-se.}{e.xi.bi.ci.o.nis.mo}{0}
\verb{exibicionista}{z}{}{}{}{adj.2g.}{Diz"-se daquele que gosta de se exibir.}{e.xi.bi.ci.o.nis.ta}{0}
\verb{exibido}{z}{}{}{}{adj.}{Que foi exposto ou mostrado.}{e.xi.bi.do}{0}
\verb{exibido}{z}{}{}{}{}{Diz"-se daquele que procura insistentemente chamar a atenção sobre si; exibicionista.}{e.xi.bi.do}{0}
\verb{exibir}{z}{}{}{}{v.t.}{Mostrar pessoa ou coisa com orgulho.}{e.xi.bir}{0}
\verb{exibir}{z}{}{}{}{v.pron.}{Chamar a atenção de alguém com alguma coisa; alardear"-se, ostentar"-se.}{e.xi.bir}{\verboinum{18}}
\verb{exigência}{z}{}{}{}{s.f.}{Ato ou efeito de exigir; pedido; instância; reclamação.}{e.xi.gên.cia}{0}
\verb{exigente}{z}{}{}{}{adj.2g.}{Que pede ou deseja com insistência, com obstinação.}{e.xi.gen.te}{0}
\verb{exigente}{z}{}{}{}{}{Que quer sempre o melhor; difícil de satisfazer, de contentar.}{e.xi.gen.te}{0}
\verb{exigir}{z}{}{}{}{v.t.}{Reclamar com base em direito fundado ou suposto. }{e.xi.gir}{0}
\verb{exigir}{z}{}{}{}{}{Impor obrigação a.}{e.xi.gir}{0}
\verb{exigir}{z}{}{}{}{}{Determinar; prescrever.}{e.xi.gir}{0}
\verb{exigir}{z}{}{}{}{}{Precisar; demandar.}{e.xi.gir}{\verboinum{22}}
\verb{exigível}{z}{}{"-eis}{}{adj.2g.}{Que se pode ou deve exigir.}{e.xi.gí.vel}{0}
\verb{exíguo}{z}{}{}{}{adj.}{Diminuto, pequeno.}{e.xí.gu.o}{0}
\verb{exíguo}{z}{}{}{}{}{Escasso.}{e.xí.gu.o}{0}
\verb{exilado}{z}{}{}{}{adj.}{Diz"-se de quem foi expulso de sua pátria; banido, degredado, desterrado.}{e.xi.la.do}{0}
\verb{exilar}{z}{}{}{}{v.t.}{Mandar alguém embora de seu país; banir, degredar, deportar.}{e.xi.lar}{\verboinum{1}}
\verb{exílio}{z}{}{}{}{s.m.}{Castigo de ser exilado; degredo, desterro, expatriação.}{e.xí.lio}{0}
\verb{exílio}{z}{}{}{}{}{Lugar onde vive a pessoa exilada.}{e.xí.lio}{0}
\verb{exímio}{z}{}{}{}{adj.}{Que tem muita habilidade; competente, eminente, excelente.}{e.xí.mio}{0}
\verb{eximir}{z}{}{}{}{v.t.}{Tirar alguma responsabilidade de alguém; isentar, dispensar, desobrigar.}{e.xi.mir}{\verboinum{18}}
\verb{existência}{z}{}{}{}{s.f.}{O fato de existir.}{e.xis.tên.cia}{0}
\verb{existência}{z}{}{}{}{}{A vida ou a maneira de viver.}{e.xis.tên.cia}{0}
\verb{existencial}{z}{}{"-ais}{}{adj.2g.}{Referente a existência.}{e.xis.ten.ci.al}{0}
\verb{existencialismo}{z}{}{}{}{s.m.}{Doutrina que afirma a importância da existência.}{e.xis.ten.ci.a.lis.mo}{0}
\verb{existencialista}{z}{}{}{}{adj.2g.}{Próprio do existencialismo.}{e.xis.ten.ci.a.lis.ta}{0}
\verb{existencialista}{z}{}{}{}{}{Diz"-se de quem defende o existencialismo.}{e.xis.ten.ci.a.lis.ta}{0}
\verb{existente}{z}{}{}{}{adj.2g.}{Que existe, que está em algum lugar.}{e.xis.ten.te}{0}
\verb{existente}{z}{}{}{}{}{Diz"-se daquele que vive, que subsiste ou permanece vivo; vivente.}{e.xis.ten.te}{0}
\verb{existir}{z}{}{}{}{v.i.}{Ter existência real; haver, ter.}{e.xis.tir}{0}
\verb{existir}{z}{}{}{}{}{Viver.}{e.xis.tir}{0}
\verb{existir}{z}{}{}{}{}{Durar, subsistir.}{e.xis.tir}{\verboinum{18}}
\verb{êxito}{z}{}{}{}{s.m.}{Consequência feliz; resultado bom.}{ê.xi.to}{0}
\verb{êxito}{z}{}{}{}{}{Sucesso, voga.}{ê.xi.to}{0}
\verb{êxodo}{z}{}{}{}{s.m.}{Saída de um povo ou de parte dele do lugar onde vive.}{ê.xo.do}{0}
\verb{exoesqueleto}{z\ldots{}ê}{Zool.}{}{}{s.m.}{Esqueleto que cobre o corpo pelo lado de fora e é trocado muitas vezes durante o desenvolvimento do animal.}{e.xo.es.que.le.to}{0}
\verb{exogamia}{z}{}{}{}{s.f.}{Regime social em que os matrimônios se efetuam com membros de tribo estranha, ou dentro da mesma tribo, com os de outra família ou de outro clã.}{e.xo.ga.mi.a}{0}
\verb{exógamo}{z}{}{}{}{adj.}{Relativo à exogamia.}{e.xó.ga.mo}{0}
\verb{exógamo}{z}{}{}{}{}{Diz"-se daquele que se casa fora da aldeia, ou fora de seu clão ou de sua família.}{e.xó.ga.mo}{0}
\verb{exógeno}{z}{}{}{}{adj.}{Que se origina fora do organismo.}{e.xó.ge.no}{0}
\verb{exoneração}{z}{}{"-ões}{}{s.f.}{Ato ou efeito de exonerar.}{e.xo.ne.ra.ção}{0}
\verb{exoneração}{z}{}{"-ões}{}{}{Demissão, destituição, dispensa.}{e.xo.ne.ra.ção}{0}
\verb{exoneração}{z}{}{"-ões}{}{}{Isenção, desobrigação.}{e.xo.ne.ra.ção}{0}
\verb{exonerar}{z}{}{}{}{v.t.}{Dispensar, destituir de emprego; demitir.}{e.xo.ne.rar}{0}
\verb{exonerar}{z}{}{}{}{}{Desobrigar, isentar.}{e.xo.ne.rar}{\verboinum{1}}
\verb{exorar}{z}{}{}{}{v.t.}{Invocar.}{e.xo.rar}{0}
\verb{exorar}{z}{}{}{}{}{Implorar de modo ansioso; rogar.}{e.xo.rar}{\verboinum{1}}
\verb{exorbitância}{z}{}{}{}{s.f.}{Qualidade de exorbitante.}{e.xor.bi.tân.cia}{0}
\verb{exorbitância}{z}{}{}{}{}{Excesso, exagero.}{e.xor.bi.tân.cia}{0}
\verb{exorbitância}{z}{}{}{}{}{Preço excessivo.}{e.xor.bi.tân.cia}{0}
\verb{exorbitante}{z}{}{}{}{adj.2g.}{Que sai fora da órbita.}{e.xor.bi.tan.te}{0}
\verb{exorbitante}{z}{}{}{}{}{Que ultrapassa os limites do justo ou razoável; excessivo.}{e.xor.bi.tan.te}{0}
\verb{exorbitar}{z}{}{}{}{v.t.}{Tirar da órbita.}{e.xor.bi.tar}{0}
\verb{exorbitar}{z}{}{}{}{}{Desviar"-se de uma norma ou exceder os justos limites, o razoável.}{e.xor.bi.tar}{\verboinum{1}}
\verb{exorcismar}{z}{}{}{}{v.t.}{Exorcisar.}{e.xor.cis.mar}{\verboinum{1}}
\verb{exorcismo}{z}{}{}{}{s.m.}{Oração para esconjurar o demônio ou os maus espíritos.}{e.xor.cis.mo}{0}
\verb{exorcista}{z}{}{}{}{s.2g.}{Indivíduo que faz exorcismo.}{e.xor.cis.ta}{0}
\verb{exorcizar}{z}{}{}{}{v.t.}{Usar de exorcismo para expulsar os maus espíritos ou demônios; esconjurar.}{e.xor.ci.zar}{\verboinum{1}}
\verb{exórdio}{z}{}{}{}{s.m.}{Introdução de discurso; preâmbulo.}{e.xór.dio}{0}
\verb{exortação}{z}{}{"-ões}{}{s.f.}{Ato ou efeito de exortar; encorajamento, estímulo.}{e.xor.ta.ção}{0}
\verb{exortação}{z}{}{"-ões}{}{}{Conselho, advertência.}{e.xor.ta.ção}{0}
\verb{exortar}{z}{}{}{}{v.t.}{Dar estímulo a; animar, estimular.}{e.xor.tar}{0}
\verb{exortar}{z}{}{}{}{}{Induzir alguém a fazer ou pensar determinada coisa; persuadir.}{e.xor.tar}{\verboinum{1}}
\verb{exortativo}{z}{}{}{}{adj.}{Que serve ou é próprio para exortar.}{e.xor.ta.ti.vo}{0}
\verb{exosfera}{z\ldots{}é}{}{}{}{s.f.}{A parte mais externa da atmosfera de um planeta.}{e.xos.fe.ra}{0}
\verb{exotérico}{z}{}{}{}{adj.}{Que pode ser ensinado ao grande público.}{e.xo.té.ri.co}{0}
\verb{exotérico}{z}{}{}{}{}{Comum, trivial.}{e.xo.té.ri.co}{0}
\verb{exoterismo}{z}{}{}{}{s.m.}{Qualidade de ser exotérico.}{e.xo.te.ris.mo}{0}
\verb{exótico}{z}{}{}{}{adj.}{Que é diferente de tudo o que se conhece.}{e.xó.ti.co}{0}
\verb{exotismo}{z}{}{}{}{s.m.}{Qualidade de exótico; extravagância, excentricidade.}{e.xo.tis.mo}{0}
\verb{expandir}{s}{}{}{}{v.t.}{Fazer alguma coisa ocupar mais espaço; ampliar, dilatar.}{ex.pan.dir}{0}
\verb{expandir}{s}{}{}{}{}{Fazer algo ficar conhecido; difundir, divulgar.}{ex.pan.dir}{0}
\verb{expandir}{s}{}{}{}{v.pron.}{Contar o que preocupa; abrir"-se, desabafar"-se.}{ex.pan.dir}{\verboinum{18}}
\verb{expansão}{s}{}{"-ões}{}{s.f.}{Ato ou efeito de expandir.}{ex.pan.são}{0}
\verb{expansão}{s}{}{"-ões}{}{}{Desenvolvimento em volume ou em superfície.}{ex.pan.são}{0}
\verb{expansão}{s}{}{"-ões}{}{}{Procedimento espontâneo de alegria, amizade, fraqueza, confiança; expansividade. }{ex.pan.são}{0}
\verb{expansão}{s}{}{"-ões}{}{}{Crescimento, desenvolvimento, progresso.}{ex.pan.são}{0}
\verb{expansibilidade}{s}{}{}{}{s.f.}{Qualidade de expansível.}{ex.pan.si.bi.li.da.de}{0}
\verb{expansibilidade}{s}{}{}{}{}{Propriedade que têm os gases de dilatar"-se, ocupando sempre o maior espaço.}{ex.pan.si.bi.li.da.de}{0}
\verb{expansionismo}{s}{}{}{}{s.m.}{Política de um país, empresa etc., de aumentar seus domínios.}{ex.pan.si.o.nis.mo}{0}
\verb{expansionista}{s}{}{}{}{adj.2g.}{Relativo ao expansionismo.}{ex.pan.si.o.nis.ta}{0}
\verb{expansionista}{s}{}{}{}{adj.2g.}{Diz"-se de partidário do expansionismo.}{ex.pan.si.o.nis.ta}{0}
\verb{expansivo}{s}{}{}{}{adj.}{Que é comunicativo, extrovertido.}{ex.pan.si.vo}{0}
\verb{expatriação}{s}{}{"-ões}{}{s.f.}{Ato ou efeito de expatriar; exílio, desterro.}{ex.pa.tri.a.ção}{0}
\verb{expatriar}{s}{}{}{}{v.t.}{Mandar alguém embora de seu país; degredar, deportar, desterrar.}{ex.pa.tri.ar}{\verboinum{6}}
\verb{expectador}{s\ldots{}ô}{}{}{}{adj.}{Diz"-se de quem tem ou está na expectativa.}{ex.pec.ta.dor}{0}
\verb{expectante}{s}{}{}{}{adj.2g.}{Que espera, que tem expectativa.}{ex.pec.tan.te}{0}
\verb{expectativa}{s}{}{}{}{s.f.}{Esperança de que alguma coisa venha a acontecer.}{ex.pec.ta.ti.va}{0}
\verb{expectável}{s}{}{"-eis}{}{adj.2g.}{Que se pode esperar; provável.}{ex.pec.tá.vel}{0}
\verb{expectoração}{s}{}{"-ões}{}{s.f.}{Expulsão de secreção das vias respiratórias.}{ex.pec.to.ra.ção}{0}
\verb{expectoração}{s}{}{"-ões}{}{}{Catarro.}{ex.pec.to.ra.ção}{0}
\verb{expectorante}{s}{}{}{}{s.m.}{Remédio que ajuda a expectorar.}{ex.pec.to.ran.te}{0}
\verb{expectorar}{s}{}{}{}{v.t.}{Soltar do peito ou lançar pela boca.}{ex.pec.to.rar}{\verboinum{1}}
\verb{expedição}{s}{}{"-ões}{}{s.f.}{Ato ou efeito de expedir.}{ex.pe.di.ção}{0}
\verb{expedição}{s}{}{"-ões}{}{}{Grupo de pessoas que vai explorar uma região.}{ex.pe.di.ção}{0}
\verb{expedição}{s}{}{"-ões}{}{}{Parte de uma empresa encarregada de mandar a correspondência e as mercadorias.}{ex.pe.di.ção}{0}
\verb{expedicionário}{s}{}{}{}{adj.}{Que participa de uma expedição.}{ex.pe.di.ci.o.ná.rio}{0}
\verb{expedicionário}{s}{}{}{}{s.m.}{Soldado que participou da Segunda Guerra Mundial.}{ex.pe.di.ci.o.ná.rio}{0}
\verb{expedidor}{s\ldots{}ô}{}{}{}{adj.}{Diz"-se de emissor; remetente.}{ex.pe.di.dor}{0}
\verb{expediente}{s}{}{}{}{s.m.}{Horário de funcionamento de uma empresa.}{ex.pe.di.en.te}{0}
\verb{expediente}{s}{}{}{}{}{Trabalho que se realiza nesse horário.}{ex.pe.di.en.te}{0}
\verb{expediente}{s}{}{}{}{}{Meio de se livrar de uma dificuldade.}{ex.pe.di.en.te}{0}
\verb{expedir}{s}{}{}{}{v.t.}{Fazer seguir carga ou correspondência para algum lugar; enviar, mandar, remeter.}{ex.pe.dir}{\verboinum{20}}
\verb{expedito}{s}{}{}{}{adj.}{Que é desembaraçado, desinibido, ativo, diligente.}{ex.pe.di.to}{0}
\verb{expelir}{s}{}{}{}{v.t.}{Lançar para fora; expulsar.}{ex.pe.lir}{\verboinum{29}}
\verb{expender}{s}{}{}{}{v.t.}{Expor minuciosamente.}{ex.pen.der}{0}
\verb{expender}{s}{}{}{}{}{Explicar, ponderando ou analisando.}{ex.pen.der}{0}
\verb{expender}{s}{}{}{}{}{Gastar.}{ex.pen.der}{\verboinum{12}}
\verb{expensas}{s}{}{}{}{s.f.pl.}{Despesas, custos.}{ex.pen.sas}{0}
\verb{experiência}{s}{}{}{}{s.f.}{Conjunto de conhecimentos trazidos por uma longa prática.}{ex.pe.ri.ên.cia}{0}
\verb{experiência}{s}{}{}{}{}{Fato que serve de lição para o futuro; experiência de vida.}{ex.pe.ri.ên.cia}{0}
\verb{experiência}{s}{}{}{}{}{Tentativa de provar uma teoria; experimento.}{ex.pe.ri.ên.cia}{0}
\verb{experiente}{s}{}{}{}{adj.2g.}{Que praticou muito alguma coisa e passou a ter muitos conhecimentos sobre ela.}{ex.pe.ri.en.te}{0}
\verb{experimentação}{s}{}{"-ões}{}{s.f.}{Ato ou efeito de experimentar; experiência.}{ex.pe.ri.men.ta.ção}{0}
\verb{experimentado}{s}{}{}{}{adj.}{Que já foi tentado ou testado.}{ex.pe.ri.men.ta.do}{0}
\verb{experimentado}{s}{}{}{}{}{Que foi submetido a prova.}{ex.pe.ri.men.ta.do}{0}
\verb{experimentado}{s}{}{}{}{}{Experiente, versado.}{ex.pe.ri.men.ta.do}{0}
\verb{experimental}{s}{}{"-ais}{}{adj.2g.}{Que se baseia na experiência.}{ex.pe.ri.men.tal}{0}
\verb{experimentar}{s}{}{}{}{v.t.}{Controlar e provar alguma coisa em laboratório.}{ex.pe.ri.men.tar}{0}
\verb{experimentar}{s}{}{}{}{}{Fazer alguma coisa pela primeira vez para ver se ela serve ou agrada; provar, testar.}{ex.pe.ri.men.tar}{0}
\verb{experimentar}{s}{}{}{}{}{Passar por uma situação agradável ou desagradável.}{ex.pe.ri.men.tar}{\verboinum{1}}
\verb{experimento}{s}{}{}{}{s.m.}{Teste para descobrir ou verificar algum fenômeno.}{ex.pe.ri.men.to}{0}
\verb{expert}{}{}{}{}{s.2g.}{Especialista.}{\textit{expert}}{0}
\verb{experto}{s\ldots{}é}{}{}{}{adj.}{Que entende muito de determinado assunto; especialista, perito.}{ex.per.to}{0}
\verb{expetativa}{s}{}{}{}{}{Var. de \textit{expectativa}.}{ex.pe.ta.ti.va}{0}
\verb{expetorante}{s}{}{}{}{}{Var. de \textit{expectorante}.}{ex.pe.to.ran.te}{0}
\verb{expetorar}{s}{}{}{}{}{Var. de \textit{expectorar}.}{ex.pe.to.rar}{0}
\verb{expiação}{s}{}{"-ões}{}{s.f.}{Ato ou efeito de expiar; castigo, penitência, cumprimento de pena.}{ex.pi.a.ção}{0}
\verb{expiar}{s}{}{}{}{v.t.}{Livrar uma culpa, um crime, por meio de penitência ou cumprindo pena.}{ex.pi.ar}{0}
\verb{expiar}{s}{}{}{}{}{Sofrer as consequências de.}{ex.pi.ar}{\verboinum{6}}
\verb{expiatório}{s}{}{}{}{adj.}{Que serve para expiar.}{ex.pi.a.tó.rio}{0}
\verb{expiração}{s}{}{"-ões}{}{s.f.}{Expulsão do ar dos pulmões.}{ex.pi.ra.ção}{0}
\verb{expiração}{s}{}{"-ões}{}{}{Fim de prazo estipulado.}{ex.pi.ra.ção}{0}
\verb{expirante}{s}{}{}{}{adj.2g.}{Que expira; moribundo.}{ex.pi.ran.te}{0}
\verb{expirante}{s}{}{}{}{}{Que está próximo do fim.}{ex.pi.ran.te}{0}
\verb{expirar}{s}{}{}{}{v.i.}{Fazer o ar sair dos pulmões.}{ex.pi.rar}{0}
\verb{expirar}{s}{}{}{}{}{Morrer.}{ex.pi.rar}{0}
\verb{expirar}{s}{}{}{}{}{Chegar ao fim de determinado tempo; acabar, terminar.}{ex.pi.rar}{\verboinum{1}}
\verb{explanação}{s}{}{"-ões}{}{s.f.}{Ato ou efeito de explanar; explicação.}{ex.pla.na.ção}{0}
\verb{explanar}{s}{}{}{}{v.t.}{Narrar ou explicar com detalhes.}{ex.pla.nar}{\verboinum{1}}
\verb{explanatório}{s}{}{}{}{adj.}{Que serve para explanar.}{ex.pla.na.tó.rio}{0}
\verb{expletivo}{s}{Gram.}{}{}{adj.}{Que serve para completar.}{ex.ple.ti.vo}{0}
\verb{expletivo}{s}{}{}{}{}{Diz"-se das palavras ou expressões que, desnecessárias ao sentido da frase, lhe dão, todavia, mais força ou graça. }{ex.ple.ti.vo}{0}
\verb{explicação}{s}{}{"-ões}{}{s.f.}{Ato ou efeito de explicar; esclarecimento, explanação.}{ex.pli.ca.ção}{0}
\verb{explicação}{s}{}{"-ões}{}{}{Comentário, análise.}{ex.pli.ca.ção}{0}
\verb{explicação}{s}{}{"-ões}{}{}{Justificativa.}{ex.pli.ca.ção}{0}
\verb{explicador}{s\ldots{}ô}{}{}{}{adj.}{Que explica, que torna mais compreensível algo que parecia ininteligível.}{ex.pli.ca.dor}{0}
\verb{explicar}{s}{}{}{}{v.t.}{Tornar compreensível ou mais claro; esclarecer, elucidar.}{ex.pli.car}{0}
\verb{explicar}{s}{}{}{}{}{Dar o sentido de; interpretar.}{ex.pli.car}{0}
\verb{explicar}{s}{}{}{}{}{Justificar, desculpar.}{ex.pli.car}{\verboinum{2}}
\verb{explicativo}{s}{}{}{}{adj.}{Que serve para explicar; elucidativo.}{ex.pli.ca.ti.vo}{0}
\verb{explicável}{s}{}{"-eis}{}{adj.2g.}{Que se pode explicar.}{ex.pli.cá.vel}{0}
\verb{explicitar}{s}{}{}{}{v.t.}{Tornar claramente expresso.}{ex.pli.ci.tar}{\verboinum{1}}
\verb{explícito}{s}{}{}{}{adj.}{Claramente expresso; manifesto, preciso.}{ex.plí.ci.to}{0}
\verb{explodir}{s}{}{}{}{v.i.}{Rebentar com estrondo; estourar.}{ex.plo.dir}{0}
\verb{explodir}{s}{}{}{}{}{Expandir"-se ruidosamente; irromper.}{ex.plo.dir}{\verboinum{34}\verboirregular{\emph{def.} explode, explodem}}
\verb{exploração}{s}{}{"-ões}{}{s.f.}{Ato ou efeito de explorar.}{ex.plo.ra.ção}{0}
\verb{exploração}{s}{}{"-ões}{}{}{Pesquisa, investigação.}{ex.plo.ra.ção}{0}
\verb{exploração}{s}{}{"-ões}{}{}{Abuso da boa"-fé de outra(s) pessoa(s).}{ex.plo.ra.ção}{0}
\verb{explorador}{s\ldots{}ô}{}{}{}{adj.}{Que explora.}{ex.plo.ra.dor}{0}
\verb{explorador}{s\ldots{}ô}{}{}{}{}{Que pesquisa, investiga.}{ex.plo.ra.dor}{0}
\verb{explorador}{s\ldots{}ô}{}{}{}{}{Que abusa da boa"-fé de outra(s) pessoa(s).}{ex.plo.ra.dor}{0}
\verb{explorar}{s}{}{}{}{v.t.}{Procurar descobrir; pesquisar, investigar.}{ex.plo.rar}{0}
\verb{explorar}{s}{}{}{}{}{Fazer produzir; cultivar.}{ex.plo.rar}{0}
\verb{explorar}{s}{}{}{}{}{Abusar da boa"-fé; iludir, ludibriar.}{ex.plo.rar}{\verboinum{1}}
\verb{exploratório}{s}{}{}{}{adj.}{Relativo a exploração.}{ex.plo.ra.tó.rio}{0}
\verb{explorável}{s}{}{"-eis}{}{adj.2g.}{Que pode ser explorado.}{ex.plo.rá.vel}{0}
\verb{explosão}{s}{}{"-ões}{}{s.f.}{Abalo violento seguido de detonação; estouro.}{ex.plo.são}{0}
\verb{explosivo}{s}{}{}{}{adj.}{Diz"-se da substância que produz explosão.}{ex.plo.si.vo}{0}
\verb{expoente}{s}{Mat.}{}{}{s.m.}{Em uma potência, o número que indica o grau a que uma quantidade é elevada}{ex.po.en.te}{0}
\verb{expoente}{s}{Fig.}{}{}{}{Representante notável de uma profissão ou de um ramo de saber.}{ex.po.en.te}{0}
\verb{exponencial}{s}{Mat.}{"-ais}{}{adj.2g.}{Diz"-se da equação cujas incógnitas figuram em expoentes.}{ex.po.nen.ci.al}{0}
\verb{expor}{s}{}{}{}{}{Sujeitar à ação de; submeter.}{ex.por}{\verboinum{60}}
\verb{expor}{s}{}{}{}{}{Explicar, explanar, narrar.}{ex.por}{0}
\verb{expor}{s}{}{}{}{v.t.}{Pôr à vista; mostrar, apresentar.}{ex.por}{0}
\verb{exportação}{s}{}{"-ões}{}{s.f.}{Ato ou efeito de exportar.}{ex.por.ta.ção}{0}
\verb{exportação}{s}{}{"-ões}{}{}{As mercadorias exportadas.}{ex.por.ta.ção}{0}
\verb{exportador}{s\ldots{}ô}{}{}{}{adj.}{Diz"-se do negociante ou da empresa que exporta.}{ex.por.ta.dor}{0}
\verb{exportar}{s}{}{}{}{v.t.}{Vender ou transportar produtos para fora do país.}{ex.por.tar}{\verboinum{1}}
\verb{exposição}{s}{}{"-ões}{}{s.f.}{Ato ou efeito de expor; apresentação.}{ex.po.si.ção}{0}
\verb{exposição}{s}{}{"-ões}{}{}{Exibição de produtos ao público. (\textit{A exposição de orquídeas desse ano trouxe um público três vezes maior do que a do ano passado.})}{ex.po.si.ção}{0}
\verb{expositivo}{s}{}{}{}{adj.}{Relativo a exposição.}{ex.po.si.ti.vo}{0}
\verb{expositivo}{s}{}{}{}{}{Que expõe, apresenta; descritivo.}{ex.po.si.ti.vo}{0}
\verb{expositor}{s\ldots{}ô}{}{}{}{s.m.}{Indivíduo ou empresa que expõe. (\textit{Meu vizinho é um grande expositor de flores.})}{ex.po.si.tor}{0}
\verb{exposto}{s\ldots{}ô}{}{"-s ⟨s\ldots{}ó⟩}{"-a ⟨s\ldots{}ó⟩}{adj.}{Que se expõe; que está à mostra; visível.}{ex.pos.to}{0}
\verb{exposto}{s\ldots{}ô}{}{"-s ⟨s\ldots{}ó⟩}{"-a ⟨s\ldots{}ó⟩}{}{Diz"-se de criança abandonada pelos pais; enjeitado.}{ex.pos.to}{0}
\verb{expressão}{s}{}{"-ões}{}{s.f.}{Ato ou efeito de exprimir.}{ex.pres.são}{0}
\verb{expressão}{s}{}{"-ões}{}{}{Manifestação do pensamento por meio da palavra escrita ou falada ou por meio de gestos.}{ex.pres.são}{0}
\verb{expressão}{s}{}{"-ões}{}{}{Energia ou entonação especial com que se pronuncia uma palavra ou uma frase; expressividade.}{ex.pres.são}{0}
\verb{expressar}{s}{}{}{}{v.t.}{Demonstrar algo por gesto ou palavra; manifestar, exprimir.}{ex.pres.sar}{\verboinum{1}}
\verb{expressionismo}{s}{Art.}{}{}{s.m.}{Movimento artístico que surgiu em reação ao Impressionismo e que procura retratar as emoções e respostas subjetivas que objetos e acontecimentos despertam no artista.}{ex.pres.si.o.nis.mo}{0}
\verb{expressionista}{s}{}{}{}{adj.2g.}{Relativo ao Expressionismo.}{ex.pres.si.o.nis.ta}{0}
\verb{expressionista}{s}{}{}{}{s.2g.}{Artista que segue os padrões desse movimento.}{ex.pres.si.o.nis.ta}{0}
\verb{expressividade}{s}{}{}{}{s.f.}{Qualidade do que é expressivo.}{ex.pres.si.vi.da.de}{0}
\verb{expressividade}{s}{}{}{}{}{Energia, eloquência, ênfase.}{ex.pres.si.vi.da.de}{0}
\verb{expressivo}{s}{}{}{}{adj.}{Que tem o poder de se expressar.}{ex.pres.si.vo}{0}
\verb{expressivo}{s}{}{}{}{}{Significativo.}{ex.pres.si.vo}{0}
\verb{expresso}{s\ldots{}é}{}{}{}{adj.}{Que se exprimiu; enunciado, manifestado.}{ex.pres.so}{0}
\verb{expresso}{s\ldots{}é}{}{}{}{}{Que não admite oposição; categórico.}{ex.pres.so}{0}
\verb{expresso}{s\ldots{}é}{}{}{}{}{Enviado sem delongas, rapidamente.}{ex.pres.so}{0}
\verb{expresso}{s\ldots{}é}{}{}{}{}{Diz"-se do café preparado na hora, em máquinas apropriadas.}{ex.pres.so}{0}
\verb{expresso}{s\ldots{}é}{}{}{}{}{Diz"-se do trem ou ônibus que não para em todas as estações.}{ex.pres.so}{0}
\verb{exprimir}{s}{}{}{}{v.t.}{Dar a entender; manifestar, revelar.}{ex.pri.mir}{0}
\verb{exprimir}{s}{}{}{}{v.pron.}{Falar com liberdade; comunicar"-se.}{ex.pri.mir}{\verboinum{18}}
\verb{exprimível}{s}{}{"-eis}{}{adj.2g.}{Que se pode exprimir, enunciar.}{ex.pri.mí.vel}{0}
\verb{exprobar}{s}{}{}{}{}{Var. de \textit{exprobrar}.}{ex.pro.bar}{0}
\verb{exprobrar}{s}{}{}{}{v.t.}{Censurar violentamente; repreender, criticar.}{ex.pro.brar}{\verboinum{1}}
\verb{expropriação}{s}{}{"-ões}{}{s.f.}{Ato ou efeito de expropriar; desapropriação.}{ex.pro.pri.a.ção}{0}
\verb{expropriar}{s}{}{}{}{v.t.}{Privar legalmente alguém da posse de sua propriedade; desapropriar.}{ex.pro.pri.ar}{\verboinum{1}}
\verb{expugnar}{s}{}{}{}{v.t.}{Tomar à força de armas; conquistar, dominar.}{ex.pug.nar}{\verboinum{1}}
\verb{expulsão}{s}{}{"-ões}{}{s.f.}{Ato ou efeito de expulsar; retirada ou saída à força.}{ex.pul.são}{0}
\verb{expulsar}{s}{}{}{}{v.t.}{Retirar à força de; repelir.}{ex.pul.sar}{0}
\verb{expulsar}{s}{}{}{}{}{Lançar fora; expelir.}{ex.pul.sar}{\verboinum{1}}
\verb{expulsivo}{s}{}{}{}{adj.}{Que provoca ou facilita a expulsão.}{ex.pul.si.vo}{0}
\verb{expulso}{s}{}{}{}{adj.}{Que se expulsou; retirado ou expelido à força.}{ex.pul.so}{0}
\verb{expurgar}{s}{}{}{}{v.t.}{Fazer purgar; eliminar a sujeira; limpar.}{ex.pur.gar}{0}
\verb{expurgar}{s}{}{}{}{}{Corrigir, emendar.}{ex.pur.gar}{\verboinum{5}}
\verb{expurgo}{s}{}{}{}{s.m.}{Ato ou efeito de expurgar; limpeza.}{ex.pur.go}{0}
\verb{exsudação}{s}{}{"-ões}{}{s.f.}{Ato ou efeito de exsudar; transpiração.}{ex.su.da.ção}{0}
\verb{exsudar}{s}{}{}{}{v.t.}{Segregar líquido em forma de gotas ou de suor.}{ex.su.dar}{\verboinum{1}}
\verb{exsudato}{s}{Med.}{}{}{s.m.}{Líquido seroso e purulento, produzido como reação a inflamações.}{ex.su.da.to}{0}
\verb{exsurgir}{s}{}{}{}{v.i.}{Levantar"-se, erguer"-se.}{ex.sur.gir}{\verboinum{22}}
\verb{êxtase}{s}{}{}{}{s.m.}{Arrebatamento do espírito por efeito de exaltação ou contemplação mística.}{êx.ta.se}{0}
\verb{êxtase}{s}{}{}{}{}{Enlevo, encantamento, embevecimento.}{êx.ta.se}{0}
\verb{extasiado}{s}{}{}{}{adj.}{Que se extasiou; arrebatado, enlevado, encantado.}{ex.ta.si.a.do}{0}
\verb{extasiar}{s}{}{}{}{v.t.}{Causar êxtase; arrebatar, enlevar.}{ex.ta.si.ar}{\verboinum{1}}
\verb{extático}{s}{}{}{}{adj.}{Que se encontra em estado de êxtase; arrebatado, absorto, enlevado.}{ex.tá.ti.co}{0}
\verb{extemporâneo}{s}{}{}{}{adj.}{Que está fora do tempo próprio; inoportuno.}{ex.tem.po.râ.neo}{0}
\verb{extensão}{s}{}{"-ões}{}{s.f.}{Ato ou efeito de estender; amplidão, tamanho.}{ex.ten.são}{0}
\verb{extensão}{s}{}{"-ões}{}{}{Dimensão de algo em qualquer direção.}{ex.ten.são}{0}
\verb{extensão}{s}{}{"-ões}{}{}{Unidade que se acrescenta à principal; continuação.}{ex.ten.são}{0}
\verb{extensão}{s}{}{"-ões}{}{}{Ramal telefônico de uma mesma linha.}{ex.ten.são}{0}
\verb{extensível}{s}{}{"-eis}{}{adj.2g.}{Que se pode estender.}{ex.ten.sí.vel}{0}
\verb{extensivo}{s}{}{}{}{adj.}{Que se pode estender.}{ex.ten.si.vo}{0}
\verb{extensivo}{s}{}{}{}{}{Que se aplica a mais de um caso.}{ex.ten.si.vo}{0}
\verb{extenso}{s}{}{}{}{adj.}{Que tem extensão; comprido, longo.}{ex.ten.so}{0}
\verb{extenso}{s}{}{}{}{}{De grande dimensão; amplo, largo.}{ex.ten.so}{0}
\verb{extensor}{s\ldots{}ô}{}{}{}{adj.}{Que faz estender ou estica com facilidade.}{ex.ten.sor}{0}
\verb{extenuação}{s}{}{"-ões}{}{s.f.}{Estado de quem se extenuou; enfraquecimento, cansaço, debilidade.}{ex.te.nu.a.ção}{0}
\verb{extenuar}{s}{}{}{}{v.t.}{Esgotar as forças de; fatigar, exaurir.}{ex.te.nu.ar}{\verboinum{1}}
\verb{exterior}{s\ldots{}ô}{}{}{}{}{As nações estrangeiras. (\textit{Meu pai nunca havia viajado para o exterior.})}{ex.te.ri.or}{0}
\verb{exterior}{s\ldots{}ô}{}{}{}{s.m.}{A parte externa.}{ex.te.ri.or}{0}
\verb{exterior}{s\ldots{}ô}{}{}{}{adj.2g.}{Que está do lado de fora; externo.}{ex.te.ri.or}{0}
\verb{exterioridade}{s}{}{}{}{s.f.}{Qualidade do que é exterior.}{ex.te.ri.o.ri.da.de}{0}
\verb{exteriorizar}{s}{}{}{}{v.t.}{Dar a conhecer; externar, manifestar.}{ex.te.ri.o.ri.zar}{\verboinum{1}}
\verb{exterminação}{s}{}{"-ões}{}{s.f.}{Ato ou efeito de exterminar; extermínio.}{ex.ter.mi.na.ção}{0}
\verb{exterminador}{s\ldots{}ô}{}{}{}{adj.}{Que extermina; destruidor.}{ex.ter.mi.na.dor}{0}
\verb{exterminar}{s}{}{}{}{v.t.}{Destruir com mortandade; aniquilar, eliminar.}{ex.ter.mi.nar}{\verboinum{1}}
\verb{extermínio}{s}{}{}{}{s.m.}{Ato ou efeito de exterminar; destruição, chacina, aniquilamento.}{ex.ter.mí.nio}{0}
\verb{externa}{s\ldots{}é}{}{}{}{s.f.}{Qualquer emissão ou gravação feita em local externo, fora de um estúdio.}{ex.ter.na}{0}
\verb{externar}{s}{}{}{}{v.t.}{Tornar externo; manifestar, exteriorizar. (\textit{O homenageado não conseguiu externar o que sentiu no momento de receber o prêmio.})}{ex.ter.nar}{\verboinum{1}}
\verb{externato}{s}{}{}{}{s.m.}{Estabelecimento de ensino em que os alunos assistem às aulas e voltam para casa todos os dias.}{ex.ter.na.to}{0}
\verb{externo}{s\ldots{}é}{}{}{}{adj.}{Que está do lado de fora; exterior.}{ex.ter.no}{0}
\verb{externo}{s\ldots{}é}{}{}{}{}{Diz"-se do aluno que não mora no colégio onde estuda.}{ex.ter.no}{0}
\verb{externo}{s\ldots{}é}{}{}{}{}{Diz"-se do medicamento que se aplica na superfície do corpo.}{ex.ter.no}{0}
\verb{extinção}{s}{}{"-ões}{}{s.f.}{Ato ou efeito de extinguir; cessação, desaparecimento.}{ex.tin.ção}{0}
\verb{extinguir}{s}{}{}{}{v.t.}{Tornar extinto; apagar. (\textit{Os bombeiros conseguiram extinguir o fogo.})}{ex.tin.guir}{0}
\verb{extinguir}{s}{}{}{}{}{Desaparecer completamente; aniquilar, destruir.}{ex.tin.guir}{\verboinum{23}}
\verb{extinguível}{s}{}{"-eis}{}{adj.2g.}{Que se pode extinguir.}{ex.tin.guí.vel}{0}
\verb{extinto}{s}{}{}{}{adj.}{Que deixou de existir; morto, acabado, findo.}{ex.tin.to}{0}
\verb{extintor}{s\ldots{}ô}{}{}{}{s.m.}{Aparelho usado para extinguir incêndios.}{ex.tin.tor}{0}
\verb{extirpação}{s}{}{"-ões}{}{s.f.}{Ato ou efeito de extirpar; extração.}{ex.tir.pa.ção}{0}
\verb{extirpar}{s}{}{}{}{v.t.}{Arrancar pela raiz; extrair, extinguir, eliminar.}{ex.tir.par}{\verboinum{1}}
\verb{extorquir}{s}{}{}{}{v.t.}{Conseguir algo de alguém por meio de violência, ameaça ou ardil.}{ex.tor.quir}{\verboinum{34}\verboirregular{\emph{def.} extorquimos, extorquis}}
\verb{extorsão}{s}{}{"-ões}{}{s.f.}{Ato ou efeito de extorquir; usurpação.}{ex.tor.são}{0}
\verb{extorsionário}{s}{}{}{}{s.m.}{Indivíduo que pratica extorsão; chantagista, usurpador.}{ex.tor.si.o.ná.rio}{0}
\verb{extorsivo}{s}{}{}{}{adj.}{Em que há extorsão.}{ex.tor.si.vo}{0}
\verb{extorsivo}{s}{}{}{}{}{Diz"-se do preço muito acima do valor justo.}{ex.tor.si.vo}{0}
\verb{extra}{ês/ ou /és}{}{}{}{adj.2g.}{Que está além do padrão normal; adicional, suplementar.}{ex.tra}{0}
\verb{extra}{ês/ ou /és}{}{}{}{}{De qualidade superior; extraordinário.}{ex.tra}{0}
\verb{extração}{s}{}{"-ões}{}{s.f.}{Ato ou efeito de extrair; extirpação.}{ex.tra.ção}{0}
\verb{extração}{s}{}{"-ões}{}{}{Sorteio dos números da loteria.}{ex.tra.ção}{0}
\verb{extraconjugal}{s}{}{"-ais}{}{adj.2g.}{Que está fora dos direitos e dos deveres do matrimônio; extramatrimonial.}{ex.tra.con.ju.gal}{0}
\verb{extradição}{s}{}{"-ões}{}{s.f.}{Ato ou efeito de extraditar.}{ex.tra.di.ção}{0}
\verb{extraditar}{s}{}{}{}{v.t.}{Devolver um criminoso a pedido das autoridades de seu país de origem.}{ex.tra.di.tar}{\verboinum{1}}
%\verb{}{}{}{}{}{}{}{}{0}
\verb{extrafino}{s}{}{}{}{adj.}{Diz"-se de produto de qualidade superior.}{ex.tra.fi.no}{0}
\verb{extrair}{s}{}{}{}{v.t.}{Tirar para fora; arrancar.}{ex.tra.ir}{0}
\verb{extrair}{s}{Mat.}{}{}{}{Determinar a raiz de um número.}{ex.tra.ir}{\verboinum{19}}
\verb{extrajudicial}{s}{}{"-ais}{}{adj.2g.}{Que se passa fora do juízo, sem processo ou formalidade judicial; extrajudiciário.}{ex.tra.ju.di.ci.al}{0}
\verb{extrajudiciário}{s}{}{}{}{adj.}{Extrajudicial.}{ex.tra.ju.di.ci.á.rio}{0}
\verb{extramatrimonial}{s}{}{"-ais}{}{adj.2g.}{Extraconjugal.}{ex.tra.ma.tri.mo.ni.al}{0}
\verb{extranumerário}{s}{}{}{}{adj.}{Que está além do número certo e determinado.}{ex.tra.nu.me.rá.rio}{0}
\verb{extranumerário}{s}{}{}{}{}{Que não faz parte do quadro efetivo de funcionários de uma empresa.}{ex.tra.nu.me.rá.rio}{0}
\verb{extraoficial}{s}{}{extraoficiais ⟨s⟩}{}{adj.2g.}{Que não tem origem oficial.}{ex.tra.o.fi.ci.al}{0}
\verb{extraoficial}{s}{}{extraoficiais ⟨s⟩}{}{}{Que não pertence aos negócios públicos.}{ex.tra.o.fi.ci.al}{0}
\verb{extraordinário}{s}{}{}{}{adj.}{Fora do comum; não ordinário.}{ex.tra.or.di.ná.rio}{0}
\verb{extraordinário}{s}{}{}{}{}{Raro, excepcional, notável.}{ex.tra.or.di.ná.rio}{0}
\verb{extrapolar}{s}{}{}{}{v.t.}{Ultrapassar os limites; exceder.}{ex.tra.po.lar}{\verboinum{1}}
\verb{extraprograma}{s}{}{}{}{adj.2g.}{Que não faz parte do programa.}{ex.tra.pro.gra.ma}{0}
\verb{extraterreno}{s}{}{}{}{adj.}{De fora da Terra.}{ex.tra.ter.re.no}{0}
\verb{extraterrestre}{s\ldots{}é}{}{}{}{adj.2g.}{Diz"-se de ser que se origina de fora da Terra.}{ex.tra.ter.res.tre}{0}
\verb{extraterritorial}{s}{}{"-ais}{}{adj.2g.}{Localizado fora do território.}{ex.tra.ter.ri.to.ri.al}{0}
\verb{extrativo}{s}{}{}{}{adj.}{Relativo a extração.}{ex.tra.ti.vo}{0}
\verb{extrativo}{s}{}{}{}{}{Que trabalha por meio de processos de extração.}{ex.tra.ti.vo}{0}
\verb{extrato}{s}{}{}{}{s.m.}{Aquilo que se extraiu de alguma coisa.}{ex.tra.to}{0}
\verb{extrato}{s}{}{}{}{}{Documento em que se apresentam várias atividades resumidas. (\textit{Meu extrato bancário acusou o débito de várias tarifas cuja origem eu desconheço.})}{ex.tra.to}{0}
\verb{extrato}{s}{}{}{}{}{Essência aromática; perfume.}{ex.tra.to}{0}
\verb{extrauterino}{s}{}{extrauterinos}{}{adj.}{Que está ou se realizou fora do útero. }{ex.tra.u.te.ri.no}{0}
\verb{extravagância}{s}{}{}{}{s.f.}{Qualidade de extravagante; excentricidade, capricho.}{ex.tra.va.gân.cia}{0}
\verb{extravagante}{s}{}{}{}{adj.2g.}{Que está fora do uso comum; inabitual, excêntrico, singular.}{ex.tra.va.gan.te}{0}
\verb{extravasamento}{s}{}{}{}{s.m.}{Ato ou efeito de extravasar; transbordamento, derramamento.}{ex.tra.va.sa.men.to}{0}
\verb{extravasar}{s}{}{}{}{v.t.}{Fazer transbordar; derramar.}{ex.tra.va.sar}{0}
\verb{extravasar}{s}{Fig.}{}{}{}{Tornar claro; manifestar.}{ex.tra.va.sar}{\verboinum{1}}
\verb{extraviado}{s}{}{}{}{adj.}{Que se extraviou; desnorteado, perdido.}{ex.tra.vi.a.do}{0}
\verb{extraviado}{s}{Fig.}{}{}{}{Que saiu do bom caminho; desencaminhado, pervertido.}{ex.tra.vi.a.do}{0}
\verb{extraviar}{s}{}{}{}{v.t.}{Tirar do caminho; desviar.}{ex.tra.vi.ar}{0}
\verb{extraviar}{s}{Fig.}{}{}{}{Tirar do bom caminho; perverter.}{ex.tra.vi.ar}{\verboinum{1}}
\verb{extravio}{s}{}{}{}{s.m.}{Ato ou efeito de extraviar; perda, sumiço. (\textit{A transportadora se responsabiliza pelo extravio da mercadoria.})}{ex.tra.vi.o}{0}
\verb{extravio}{s}{Fig.}{}{}{}{Corrupção moral; perversão.}{ex.tra.vi.o}{0}
\verb{extrema"-direita}{s}{Esport.}{extremas"-direitas ⟨s⟩}{}{s.2g.}{No futebol, jogador que atua ofensivamente pela extremidade direita da linha dianteira (camisa 7); ponta"-direita.  }{ex.tre.ma"-di.rei.ta}{0}
\verb{extrema"-direita}{s}{}{extremas"-direitas ⟨s⟩}{}{}{Posição política que defende o uso da força para a centralização do poder público e do poder econômico nas mãos de uns poucos selecionados.}{ex.tre.ma"-di.rei.ta}{0}
\verb{extremado}{s}{}{}{}{adj.}{Que é fora do comum; extraordinário, excepcional.}{ex.tre.ma.do}{0}
\verb{extrema"-esquerda}{s\ldots{}ê}{Esport.}{extremas"-esquerdas ⟨ê⟩}{}{s.2g.}{No futebol, jogador que atua ofensivamente pela extremidade esquerda da linha dianteira (camisa 11); ponta"-esquerda.  }{ex.tre.ma"-es.quer.da}{0}
\verb{extrema"-esquerda}{s\ldots{}ê}{}{extremas"-esquerdas ⟨ê⟩}{}{}{Posição política que defende a centralização do poder público e do poder econômico para promover a distribuição obrigatória da riqueza entre todos os participantes de uma nação.}{ex.tre.ma"-es.quer.da}{0}
\verb{extremar}{s}{}{}{}{v.t.}{Tornar extremo; assinalar, exaltar.}{ex.tre.mar}{\verboinum{1}}
\verb{extrema"-unção}{s}{}{extremas"-unções \textit{ou}  extrema"-unções}{}{s.f.}{Ato religioso de passar um óleo próprio sobre o doente e o abençoar, e que constitui um dos sete sacramentos da Igreja católica; unção dos enfermos.}{ex.tre.ma"-un.ção}{0}
\verb{extremidade}{s}{}{}{}{s.f.}{Cada uma das partes em que termina um corpo; ponta, limite.}{ex.tre.mi.da.de}{0}
\verb{extremismo}{s}{}{}{}{s.m.}{Sistema político que recorre a soluções extremas para a resolução dos problemas sociais.}{ex.tre.mis.mo}{0}
\verb{extremista}{s}{}{}{}{adj.2g.}{Relativo ao extremismo.}{ex.tre.mis.ta}{0}
\verb{extremista}{s}{}{}{}{s.2g.}{Indivíduo partidário do extremismo.}{ex.tre.mis.ta}{0}
\verb{extremo}{s}{}{}{}{adj.}{Que está no ponto mais afastado; remoto, distante.}{ex.tre.mo}{0}
\verb{extremo}{s}{}{}{}{}{Que atingiu o ponto mais alto; máximo.}{ex.tre.mo}{0}
\verb{extremo}{s}{}{}{}{s.m.}{Ponto mais remoto; extremidade.}{ex.tre.mo}{0}
\verb{extremosa}{s\ldots{}ó}{Bot.}{}{}{s.f.}{Árvore ornamental de até 10 m com madeira de boa qualidade e belas flores brancas ou róseas com pétalas frisadas e recurvadas.}{ex.tre.mo.sa}{0}
\verb{extremoso}{s\ldots{}ô}{}{"-osos ⟨s\ldots{}ó⟩}{"-osa ⟨s\ldots{}ó⟩}{adj.}{Que demonstra grande carinho; afetuoso, terno.}{ex.tre.mo.so}{0}
\verb{extrínseco}{s}{}{}{}{adj.}{Que não faz parte da essência de alguma coisa; de fora; exterior.}{ex.trín.se.co}{0}
\verb{extroversão}{s}{}{"-ões}{}{s.f.}{Qualidade de extrovertido, de comunicativo; expansão.}{ex.tro.ver.são}{0}
\verb{extroverter}{s}{}{}{}{v.pron.}{Tornar"-se extrovertido.}{ex.tro.ver.ter"-se}{\verboinum{12}}
\verb{extrovertido}{s}{}{}{}{adj.}{Que gosta de se comunicar; expansivo, sociável.}{ex.tro.ver.ti.do}{0}
\verb{exu}{ch}{Relig.}{}{}{s.m.}{Nos cultos afro"-brasileiros, orixá que serve de intermediário entre outros orixás e os homens.}{e.xu}{0}
\verb{exu}{ch}{Relig.}{}{}{}{Nesses mesmos cultos, espírito inferior que pendula entre o Bem e o Mal.}{e.xu}{0}
\verb{exuberância}{z}{}{}{}{s.f.}{Qualidade de exuberante; abundância, copiosidade, fartura.}{e.xu.be.rân.cia}{0}
\verb{exuberante}{z}{}{}{}{adj.2g.}{Em que há abundância; farto, copioso.}{e.xu.be.ran.te}{0}
\verb{exuberante}{z}{Fig.}{}{}{}{Cheio de vida; animado.}{e.xu.be.ran.te}{0}
\verb{exuberar}{z}{}{}{}{v.t.}{Ter em excesso; superabundar.}{e.xu.be.rar}{\verboinum{1}}
\verb{exultação}{z}{}{"-ões}{}{s.f.}{Ato ou efeito de exultar; intensa alegria; júbilo.}{e.xul.ta.ção}{0}
\verb{exultante}{z}{}{}{}{adj.2g.}{Que exulta, regozija.}{e.xul.tan.te}{0}
\verb{exultar}{z}{}{}{}{v.i.}{Demonstrar intensa alegria; regozijar"-se.}{e.xul.tar}{\verboinum{1}}
\verb{exumação}{z}{}{"-ões}{}{s.f.}{Ato ou efeito de exumar; desenterramento.}{e.xu.ma.ção}{0}
\verb{exumar}{z}{}{}{}{v.t.}{Tirar da sepultura; desenterrar.}{e.xu.mar}{\verboinum{1}}
\verb{ex"-voto}{es\ldots{}ó}{}{ex"-votos ⟨es\ldots{}ó⟩}{}{s.m.}{Quadro, imagem, pintura etc., que se oferece e expõe numa igreja ou numa capela em agradecimento a uma graça alcançada ou pagamento de promessa.}{ex"-vo.to}{0}
