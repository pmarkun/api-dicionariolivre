\verb{o}{ó/ ou /ô}{}{}{}{s.m.}{Décima quinta letra e quarta vogal do alfabeto português.}{o}{0}
\verb{o}{ô}{Gram.}{}{a}{pron.}{Pronome pessoal da 3ª pessoa, masculino.}{o}{0}
\verb{o}{ô}{Gram.}{}{a}{}{Artigo masculino definido singular.}{o}{0}
\verb{ó}{}{}{}{}{interj.}{Expressão usada para chamar, atrair a atenção, invocar.}{ó}{0}
\verb{ô}{}{}{}{}{interj.}{Expressão usada para chamar a atenção de alguém.}{ô}{0}
\verb{O}{}{}{}{}{}{Com ponto, abrev. de \textit{oeste}.}{O}{0}
\verb{O}{}{Quím.}{}{}{}{Símb. do \textit{oxigênio}.}{O}{0}
\verb{oásis}{}{}{}{}{s.m.}{Nos desertos, terreno fértil e coberto de vegetação, que permite a fixação do homem.}{o.á.sis}{0}
\verb{oásis}{}{Fig.}{}{}{}{Lugar belo, agradável, convidativo.}{o.á.sis}{0}
\verb{oba}{ô}{}{}{}{interj.}{Expressão que denota alegria, admiração.}{o.ba}{0}
\verb{obcecação}{}{}{"-ões}{}{s.f.}{Ato ou efeito de obcecar.}{ob.ce.ca.ção}{0}
\verb{obcecação}{}{}{"-ões}{}{}{Obscurecimento da inteligência, da razão; insistência excessiva.}{ob.ce.ca.ção}{0}
\verb{obcecação}{}{Med.}{"-ões}{}{}{Cegueira parcial.}{ob.ce.ca.ção}{0}
\verb{obcecado}{}{}{}{}{adj.}{Que está com a razão ou a inteligência obscurecida; desvairado.}{ob.ce.ca.do}{0}
\verb{obcecado}{}{}{}{}{}{Inflexível, obstinado, teimoso.}{ob.ce.ca.do}{0}
\verb{obcecado}{}{}{}{}{}{Que está temporariamente cego, ofuscado.}{ob.ce.ca.do}{0}
\verb{obcecante}{}{}{}{}{adj.2g.}{Que obceca; que cega; ofuscante.}{ob.ce.can.te}{0}
\verb{obcecar}{}{}{}{}{v.t.}{Tornar cego; ofuscar.}{ob.ce.car}{0}
\verb{obcecar}{}{}{}{}{}{Obscurecer a razão; desvairar, perturbar.}{ob.ce.car}{0}
\verb{obcecar}{}{}{}{}{}{Persistir no erro; obstinar.}{ob.ce.car}{\verboinum{2}}
\verb{obedecer}{ê}{}{}{}{v.t.}{Sujeitar"-se à vontade de outrem; submeter"-se.}{o.be.de.cer}{0}
\verb{obedecer}{ê}{}{}{}{}{Estar sob o comando ou a autoridade de outrem.}{o.be.de.cer}{0}
\verb{obedecer}{ê}{}{}{}{}{Estar submetido a uma força ou influência.}{o.be.de.cer}{0}
\verb{obedecer}{ê}{}{}{}{}{Deixar"-se conduzir; não resistir; ceder.}{o.be.de.cer}{0}
\verb{obedecer}{ê}{}{}{}{}{Estar de acordo com algo; funcionar corretamente.}{o.be.de.cer}{\verboinum{15}}
\verb{obediência}{}{}{}{}{s.f.}{Ato ou efeito de obedecer.}{o.be.di.ên.cia}{0}
\verb{obediência}{}{}{}{}{}{Disposição para obedecer; submissão, sujeição.}{o.be.di.ên.cia}{0}
\verb{obediência}{}{}{}{}{}{Estado daquele que obedece; docilidade, humildade.}{o.be.di.ên.cia}{0}
\verb{obediente}{}{}{}{}{adj.2g.}{Que obedece; submisso.}{o.be.di.en.te}{0}
\verb{obediente}{}{}{}{}{}{Que é excessivamente dócil; humilde, vassalo.}{o.be.di.en.te}{0}
\verb{obelisco}{}{}{}{}{s.m.}{Monumento vertical, com base quadrangular, formato de agulha, e que tem, no ponto culminante, uma pirâmide quadrangular.}{o.be.lis.co}{0}
\verb{obelisco}{}{}{}{}{}{Qualquer objeto alto e alongado.}{o.be.lis.co}{0}
\verb{obesidade}{}{}{}{}{s.f.}{Excesso de peso provocado por aumento das reservas de gordura no corpo.}{o.be.si.da.de}{0}
\verb{obeso}{ê}{}{}{}{adj.}{Que sofre de obesidade; que tem gordura em excesso; muito gordo.}{o.be.so}{0}
\verb{óbice}{}{}{}{}{s.m.}{Aquilo que impede; empecilho, estorvo, obstáculo.}{ó.bi.ce}{0}
\verb{óbito}{}{}{}{}{s.m.}{Morte de pessoa; falecimento, passamento.}{ó.bi.to}{0}
\verb{obituário}{}{}{}{}{adj.}{Relativo a óbito, falecimento.}{o.bi.tu.á.rio}{0}
\verb{obituário}{}{}{}{}{}{Registro ou livro em que se lançam os nomes dos mortos, o dia da morte etc.; necrológio.}{o.bi.tu.á.rio}{0}
\verb{objeção}{}{}{"-ões}{}{s.f.}{Ato ou efeito de objetar; contestação.}{ob.je.ção}{0}
\verb{objeção}{}{}{"-ões}{}{}{Argumento que se opõe a uma proposição, a uma afirmação. }{ob.je.ção}{0}
\verb{objeção}{}{}{"-ões}{}{}{Obstáculo, dificuldade, empecilho.}{ob.je.ção}{0}
\verb{objetar}{}{}{}{}{v.t.}{Contrapor um argumento a outro; apresentar uma razão contrária.}{ob.je.tar}{0}
\verb{objetar}{}{}{}{}{}{Apresentar dificuldades, empecilhos.}{ob.je.tar}{0}
\verb{objetar}{}{}{}{}{}{Ser contrário; opor"-se.}{ob.je.tar}{\verboinum{1}}
\verb{objetiva}{}{}{}{}{s.f.}{Sistema óptico ou conjunto de lentes fotográficas, cinematográficas etc. que fornecem imagens sem deformação.}{ob.je.ti.va}{0}
\verb{objetivar}{}{}{}{}{v.t.}{Ter como objetivo, como finalidade; pretender.}{ob.je.ti.var}{0}
\verb{objetivar}{}{}{}{}{}{Tornar objetivo; considerar como existente; materializar.}{ob.je.ti.var}{\verboinum{1}}
\verb{objetividade}{}{}{}{}{s.f.}{Qualidade ou condição do que é objetivo; imparcialidade.}{ob.je.ti.vi.da.de}{0}
\verb{objetividade}{}{}{}{}{}{Ausência de opinião preconcebida.}{ob.je.ti.vi.da.de}{0}
\verb{objetividade}{}{}{}{}{}{Existência real ou representação fiel do que se concebeu no espírito.}{ob.je.ti.vi.da.de}{0}
\verb{objetivismo}{}{Filos.}{}{}{s.m.}{Doutrina que só considera a realidade sensível, pregando a supremacia dos fenômenos objetivos sobre a experiência subjetiva.}{ob.je.ti.vis.mo}{0}
\verb{objetivo}{}{}{}{}{adj.}{Relativo a objeto.}{ob.je.ti.vo}{0}
\verb{objetivo}{}{}{}{}{}{Que existe independentemente do pensamento; sem interferência pessoal.}{ob.je.ti.vo}{0}
\verb{objetivo}{}{}{}{}{}{Sem rodeios ou circunlóquios; direto, conciso.}{ob.je.ti.vo}{0}
\verb{objetivo}{}{}{}{}{s.m.}{Alvo, fim, propósito, intento.}{ob.je.ti.vo}{0}
\verb{objeto}{é}{}{}{}{s.m.}{O que se apresenta à vista; o que é apreendido pelo sujeito do conhecimento. (\textit{É importante saber reconhecer os objetos a nossa volta.})}{ob.je.to}{0}
\verb{objeto}{é}{}{}{}{}{Tudo o que fornece matéria a uma ciência, a uma arte, a uma obra literária. (\textit{O objeto dos estudos dele é a música.})}{ob.je.to}{0}
\verb{objeto}{é}{}{}{}{}{Finalidade, objetivo. (\textit{Será nosso objeto descobrir a cura dessa doença.})}{ob.je.to}{0}
\verb{objeto}{é}{}{}{}{}{Bem material fabricado para atender a determinado uso. (\textit{Nós precisamos de objetos claros para a decoração da sala.})}{ob.je.to}{0}
\verb{objurgar}{}{}{}{}{v.t.}{Repreender asperamente; censurar, criticar.}{ob.jur.gar}{0}
\verb{objurgar}{}{}{}{}{}{Lançar em rosto; acusar.}{ob.jur.gar}{\verboinum{5}}
\verb{oblação}{}{}{"-ões}{}{s.f.}{Ato pelo qual se oferta algo a Deus ou aos santos; oferecimento.}{o.bla.ção}{0}
\verb{oblação}{}{}{"-ões}{}{}{O objeto dessa oferta; oferenda.}{o.bla.ção}{0}
\verb{oblato}{}{}{}{}{s.m.}{Leigo que se oferecia para servir em uma ordem religiosa.}{o.bla.to}{0}
\verb{oblato}{}{}{}{}{adj.}{Que é achatado nos polos. }{o.bla.to}{0}
\verb{oblíqua}{}{Geom.}{}{}{s.f.}{Reta que forma com outra, ou com um plano, um ângulo agudo ou obtuso.}{o.blí.qua}{0}
\verb{obliquar}{}{}{}{}{v.i.}{Andar em direção oblíqua, de través.}{o.bli.quar}{0}
\verb{obliquar}{}{Fig.}{}{}{}{Proceder com malícia; dissimular, tergiversar.}{o.bli.quar}{\verboinum{1}}
\verb{obliquidade}{}{}{}{}{s.f.}{Posição do que é oblíquo; inclinação de uma superfície ou de uma linha em relação a outra.}{o.bli.qui.da.de}{0}
\verb{obliquidade}{}{Fig.}{}{}{}{Desvio do padrão moral.}{o.bli.qui.da.de}{0}
\verb{obliquidade}{}{Fig.}{}{}{}{Falta de objetividade; dissimulação, evasiva.}{o.bli.qui.da.de}{0}
\verb{oblíquo}{}{}{}{}{adj.}{Inclinado sobre um plano; não perpendicular.}{o.blí.quo}{0}
\verb{oblíquo}{}{Fig.}{}{}{}{De soslaio; enviesado, dissimulado.}{o.blí.quo}{0}
\verb{oblíquo}{}{Geom.}{}{}{}{Diz"-se do sólido cujo eixo não é perpendicular à base.}{o.blí.quo}{0}
\verb{oblíquo}{}{Gram.}{}{}{}{Diz"-se do pronome pessoal que exerce a função de complemento verbal.}{o.blí.quo}{0}
\verb{obliteração}{}{}{"-ões}{}{s.f.}{Ato ou efeito de obliterar, destruir; extinção.}{o.bli.te.ra.ção}{0}
\verb{obliteração}{}{Fig.}{"-ões}{}{}{Esquecimento, olvido.}{o.bli.te.ra.ção}{0}
\verb{obliteração}{}{Med.}{"-ões}{}{}{Fechamento de uma cavidade; obstrução.}{o.bli.te.ra.ção}{0}
\verb{obliterado}{}{}{}{}{adj.}{Que se obliterou; desaparecido, extinto.}{o.bli.te.ra.do}{0}
\verb{obliterar}{}{}{}{}{v.t.}{Fazer desaparecer progressivamente; destruir pouco a pouco; extinguir.}{o.bli.te.rar}{0}
\verb{obliterar}{}{Fig.}{}{}{}{Fazer esquecer; apagar da memória.}{o.bli.te.rar}{0}
\verb{obliterar}{}{Med.}{}{}{}{Fechar uma cavidade; obstruir.}{o.bli.te.rar}{\verboinum{1}}
\verb{oblongo}{}{}{}{}{adj.}{De forma alongada; cujo comprimento é maior que a largura; oval, elíptico.}{o.blon.go}{0}
\verb{obnubilação}{}{Med.}{"-ões}{}{s.f.}{Perturbação da consciência caracterizada por obscurecimento do pensamento e por lentidão das respostas.}{ob.nu.bi.la.ção}{0}
\verb{obnubilação}{}{Por ext.}{"-ões}{}{}{Obscurecimento, ofuscação.}{ob.nu.bi.la.ção}{0}
\verb{obnubilar}{}{}{}{}{v.t.}{Tornar obscuro; escurecer, toldar.}{ob.nu.bi.lar}{0}
\verb{obnubilar}{}{Med.}{}{}{}{Causar obnubilação.}{ob.nu.bi.lar}{\verboinum{1}}
\verb{oboé}{}{Mús.}{}{}{s.m.}{Instrumento de sopro, feito de madeira, com palheta dupla e tubo ligeiramente cônico.}{o.bo.é}{0}
\verb{oboísta}{}{}{}{}{s.2g.}{Músico que toca oboé.}{o.bo.í.sta}{0}
\verb{óbolo}{}{}{}{}{s.m.}{Donativo de pequeno valor feito aos pobres; esmola.}{ó.bo.lo}{0}
\verb{obra}{ó}{}{}{}{s.f.}{Coisa que se faz. (\textit{As pessoas estão trabalhando na obra do metrô.})}{o.bra}{0}
\verb{obra}{ó}{}{}{}{}{Coisa que se fez. (\textit{Aquela peça foi obra de um escultor famoso.})}{o.bra}{0}
\verb{obra}{ó}{}{}{}{}{Aquilo que ainda está sendo construído. }{o.bra}{0}
\verb{obra}{ó}{}{}{}{}{Ação que se pratica. (\textit{Isso é parte das obras de caridade da igreja.})}{o.bra}{0}
\verb{obra"-prima}{ó}{}{obras"-primas ⟨ó⟩}{}{s.f.}{Obra perfeita; a melhor obra de um autor.}{o.bra"-pri.ma}{0}
\verb{obra"-prima}{ó}{}{obras"-primas ⟨ó⟩}{}{}{Obra que reúne toda a produção e a visão de uma época, gênero, estilo ou autor.}{o.bra"-pri.ma}{0}
\verb{obra"-prima}{ó}{Hist.}{obras"-primas ⟨ó⟩}{}{}{Na Idade Média, primeira obra que devia realizar todo artesão aspirante a mestre em uma cooperativa.}{o.bra"-pri.ma}{0}
\verb{obrar}{}{}{}{}{v.t.}{Converter em obra; realizar uma ação; fazer; fabricar.}{o.brar}{0}
\verb{obrar}{}{}{}{}{v.i.}{Proceder, haver"-se, trabalhar.}{o.brar}{0}
\verb{obrar}{}{}{}{}{}{Surtir, produzir efeito.}{o.brar}{0}
\verb{obrar}{}{Pop.}{}{}{}{Defecar, evacuar, descomer.}{o.brar}{\verboinum{1}}
\verb{obreia}{ê}{Relig.}{}{}{s.f.}{Folha de massa de que se faz a hóstia para a comunhão.}{o.brei.a}{0}
\verb{obreia}{ê}{}{}{}{}{Folha fina de massa de farinha de trigo que serve para envolver remédios ou colar papéis.}{o.brei.a}{0}
\verb{obreira}{ê}{}{}{}{s.f.}{Mulher que exerce um trabalho; operária.}{o.brei.ra}{0}
\verb{obreiro}{ê}{}{}{}{adj.}{Que trabalha; operário.}{o.brei.ro}{0}
\verb{ob"-reptício}{}{}{}{}{adj.}{Dissimulado, ardiloso, astucioso.}{ob"-rep.tí.cio}{0}
\verb{obrigação}{}{}{"-ões}{}{s.f.}{Ato de obrigar; fato de estar obrigado a fazer uma ação.}{o.bri.ga.ção}{0}
\verb{obrigação}{}{}{"-ões}{}{}{Tarefa necessária; encargo, compromisso, dever.}{o.bri.ga.ção}{0}
\verb{obrigação}{}{}{"-ões}{}{}{Motivo de reconhecimento; benefício, favor.}{o.bri.ga.ção}{0}
\verb{obrigação}{}{}{"-ões}{}{}{Emprego, ofício, profissão.}{o.bri.ga.ção}{0}
\verb{obrigação}{}{Jur.}{"-ões}{}{}{Título de dívida.}{o.bri.ga.ção}{0}
\verb{obrigado}{}{}{}{}{adj.}{Forçado pelas circunstâncias; coagido, pressionado.}{o.bri.ga.do}{0}
\verb{obrigado}{}{}{}{}{}{Agradecido, grato.}{o.bri.ga.do}{0}
\verb{obrigado}{}{}{}{}{interj.}{Expressão que denota agradecimento.}{o.bri.ga.do}{0}
\verb{obrigar}{}{}{}{}{v.t.}{Impor como dever; compelir.}{o.bri.gar}{0}
\verb{obrigar}{}{}{}{}{}{Coagir, pressionar, forçar.}{o.bri.gar}{0}
\verb{obrigar}{}{}{}{}{}{Prender por reconhecimento ou por afeição; tornar grato.}{o.bri.gar}{0}
\verb{obrigar}{}{}{}{}{}{Sujeitar a qualquer condição; oferecer.}{o.bri.gar}{0}
\verb{obrigar}{}{}{}{}{}{Contrair dívida; hipotecar, penhorar.}{o.bri.gar}{\verboinum{5}}
\verb{obrigatoriedade}{}{}{}{}{s.f.}{Qualidade do que é obrigatório.}{o.bri.ga.to.ri.e.da.de}{0}
\verb{obrigatoriedade}{}{}{}{}{}{Aquilo que é forçoso, imposto por lei.}{o.bri.ga.to.ri.e.da.de}{0}
\verb{obrigatório}{}{}{}{}{adj.}{Que envolve obrigação, imposição.}{o.bri.ga.tó.rio}{0}
\verb{obrigatório}{}{}{}{}{}{Imposto por lei, por pressão moral ou convenção social.}{o.bri.ga.tó.rio}{0}
\verb{obrigatório}{}{}{}{}{}{Necessário, indispensável, inevitável.}{o.bri.ga.tó.rio}{0}
\verb{ob"-rogar}{}{Jur.}{}{}{v.t.}{Contrapor uma lei a outra, anulando esta última; derrogar.}{ob"-ro.gar}{\verboinum{5}}
\verb{obscenidade}{}{}{}{}{s.f.}{Qualidade ou caráter de obsceno.}{obs.ce.ni.da.de}{0}
\verb{obscenidade}{}{}{}{}{}{Palavra, ato, gesto ou pensamento obscenos; sujeira, imoralidade.}{obs.ce.ni.da.de}{0}
\verb{obsceno}{ê}{}{}{}{adj.}{Contrário ao pudor; imoral, sujo, torpe.}{obs.ce.no}{0}
\verb{obsceno}{ê}{}{}{}{}{Que diz ou comete obscenidades.}{obs.ce.no}{0}
\verb{obscurantismo}{}{}{}{}{s.m.}{Estado de ignorância; falta de instrução}{obs.cu.ran.tis.mo}{0}
\verb{obscurantismo}{}{}{}{}{}{Atitude ou doutrina contrária à difusão dos conhecimentos intelectuais e materiais às classes populares.}{obs.cu.ran.tis.mo}{0}
\verb{obscurantista}{}{}{}{}{adj.2g.}{Que é contra o progresso material ou intelectual.}{obs.cu.ran.tis.ta}{0}
\verb{obscurecer}{ê}{}{}{}{adj.}{Tornar obscuro; privar de luz; diminuir a claridade.}{obs.cu.re.cer}{0}
\verb{obscurecer}{ê}{}{}{}{}{Tornar confuso, incompreensível; toldar.}{obs.cu.re.cer}{0}
\verb{obscurecer}{ê}{}{}{}{}{Tornar sombrio; entristecer.}{obs.cu.re.cer}{\verboinum{15}}
\verb{obscurecimento}{}{}{}{}{s.m.}{Ato ou efeito de obscurecer; ausência de luz; escuridão.}{obs.cu.re.ci.men.to}{0}
\verb{obscurecimento}{}{Fig.}{}{}{}{Ato de tornar incompreensível, impreciso.}{obs.cu.re.ci.men.to}{0}
\verb{obscuridade}{}{}{}{}{s.f.}{Estado do que é obscuro; ausência de luz; escuridão.}{obs.cu.ri.da.de}{0}
\verb{obscuridade}{}{}{}{}{}{Falta de clareza; confusão.}{obs.cu.ri.da.de}{0}
\verb{obscuridade}{}{}{}{}{}{Condição humilde; ignorância.}{obs.cu.ri.da.de}{0}
\verb{obscuro}{}{}{}{}{adj.}{Que está sem luz; que não tem claridade; escuro, sombrio.}{obs.cu.ro}{0}
\verb{obscuro}{}{}{}{}{}{Que é difícil de compreender; ininteligível, confuso, indistinto.}{obs.cu.ro}{0}
\verb{obscuro}{}{}{}{}{}{Pouco conhecido; ignorado, humilde.}{obs.cu.ro}{0}
\verb{obsedante}{}{}{}{}{adj.2g.}{Que obseda, importuna.}{ob.se.dan.te}{0}
\verb{obsedar}{}{}{}{}{v.t.}{Importunar com insistência; molestar.}{ob.se.dar}{0}
\verb{obsedar}{}{}{}{}{}{Preocupar constantemente; causar ideia fixa; obcecar.}{ob.se.dar}{\verboinum{1}}
\verb{obsequiar}{}{}{}{}{v.t.}{Fazer obséquios; favorecer, presentear.}{ob.se.qui.ar}{0}
\verb{obsequiar}{}{}{}{}{}{Tornar grato; cativar.}{ob.se.qui.ar}{\verboinum{1}}
\verb{obséquio}{}{}{}{}{s.m.}{Aquilo que se faz por alguém sem interesse; favor, serviço.}{ob.sé.quio}{0}
\verb{obséquio}{}{}{}{}{}{Benefício, amabilidade, benevolência.}{ob.sé.quio}{0}
\verb{obsequiosidade}{}{}{}{}{}{Afabilidade, benevolência, fineza.}{ob.se.qui.o.si.da.de}{0}
\verb{obsequiosidade}{}{}{}{}{s.f.}{Qualidade ou caráter de obsequioso.}{ob.se.qui.o.si.da.de}{0}
\verb{obsequioso}{ô}{}{"-osos ⟨ó⟩}{"-osa ⟨ó⟩}{adj.}{Que tem o hábito de ajudar, de prestar favores; prestativo.}{ob.se.qui.o.so}{0}
\verb{obsequioso}{ô}{}{"-osos ⟨ó⟩}{"-osa ⟨ó⟩}{}{Que exagera na polidez ou nos agrados a outrem; serviçal.}{ob.se.qui.o.so}{0}
\verb{observação}{}{}{"-ões}{}{s.f.}{Ato ou efeito de observar, de olhar atentamente; exame, nota.}{ob.ser.va.ção}{0}
\verb{observação}{}{}{"-ões}{}{}{Cumprimento, observância de regras, leis, normas.}{ob.ser.va.ção}{0}
\verb{observação}{}{}{"-ões}{}{}{Repreensão leve; admoestação.}{ob.ser.va.ção}{0}
\verb{observador}{ô}{}{}{}{adj.}{Que observa, que presta atenção; espectador.}{ob.ser.va.dor}{0}
\verb{observador}{ô}{}{}{}{}{Que cumpre uma regra, lei ou promessa; respeitador, cumpridor.}{ob.ser.va.dor}{0}
\verb{observador}{ô}{}{}{}{}{Que é encarregado de fazer observações científicas; crítico.}{ob.ser.va.dor}{0}
\verb{observância}{}{}{}{}{s.f.}{Ato ou efeito de observar, de praticar fielmente alguma coisa.}{ob.ser.vân.cia}{0}
\verb{observância}{}{}{}{}{}{Cumprimento rigoroso do que prescreve uma lei, uma norma.}{ob.ser.vân.cia}{0}
\verb{observar}{}{}{}{}{v.t.}{Examinar cuidadosamente; olhar atentamente; estudar.}{ob.ser.var}{0}
\verb{observar}{}{}{}{}{}{Respeitar as prescrições; cumprir, obedecer.}{ob.ser.var}{0}
\verb{observar}{}{}{}{}{}{Espiar, espreitar.}{ob.ser.var}{0}
\verb{observar}{}{}{}{}{}{Fazer ver; advertir, admoestar.}{ob.ser.var}{\verboinum{1}}
\verb{observatório}{}{}{}{}{s.m.}{Instituição ou serviço encarregado de fazer observações astronômicas ou meteorológicas.}{ob.ser.va.tó.rio}{0}
\verb{observável}{}{}{"-eis}{}{adj.2g.}{Que pode ou merece ser observado.}{ob.ser.vá.vel}{0}
\verb{observável}{}{Fís.}{"-eis}{}{}{Diz"-se da grandeza passível de uma medida direta. }{ob.ser.vá.vel}{0}
\verb{obsessão}{}{}{"-ões}{}{s.f.}{Ato ou efeito de obsedar, importunar; impertinência, perseguição.}{ob.ses.são}{0}
\verb{obsessão}{}{Fig.}{"-ões}{}{}{Preocupação constante; ideia fixa; mania.}{ob.ses.são}{0}
\verb{obsessão}{}{Relig.}{"-ões}{}{}{Perseguição de espíritos inferiores.}{ob.ses.são}{0}
\verb{obsessivo}{}{}{}{}{adj.}{Que causa obsessão; obsessor.}{ob.ses.si.vo}{0}
\verb{obsessivo}{}{}{}{}{}{Em que há obsessão; muito preocupante.}{ob.ses.si.vo}{0}
\verb{obsesso}{é}{}{}{}{adj.}{Cheio de aflição; atormentado, importunado.}{ob.ses.so}{0}
\verb{obsesso}{é}{Relig.}{}{}{}{Que se sente perseguido por espíritos inferiores.}{ob.ses.so}{0}
\verb{obsessor}{ô}{}{}{}{adj.}{Que causa obsessão; que importuna.}{ob.ses.sor}{0}
\verb{obsessor}{ô}{Relig.}{}{}{}{Diz"-se do espírito inferior que persegue.}{ob.ses.sor}{0}
\verb{obsidiar}{}{}{}{}{v.t.}{Fazer cerco; assediar.}{ob.si.di.ar}{0}
\verb{obsidiar}{}{}{}{}{}{Observar o comportamento ou os atos de alguém; espiar.}{ob.si.di.ar}{0}
\verb{obsidiar}{}{}{}{}{}{Importunar, obcecar, perturbar.}{ob.si.di.ar}{\verboinum{1}}
\verb{obsoleto}{ê}{}{}{}{adj.}{Que caiu em desuso; antiquado, arcaico.}{ob.so.le.to}{0}
\verb{obstaculizar}{}{}{}{}{v.t.}{Criar obstáculos; dificultar, obstar.}{obs.ta.cu.li.zar}{\verboinum{1}}
\verb{obstáculo}{}{}{}{}{s.m.}{Aquilo que dificulta alguma coisa; estorvo, embaraço.}{obs.tá.cu.lo}{0}
\verb{obstáculo}{}{}{}{}{}{Barreira, empecilho, óbice.}{obs.tá.cu.lo}{0}
\verb{obstante}{}{}{}{}{adj.2g.}{Que obsta, impede.}{obs.tan.te}{0}
\verb{obstante}{}{}{}{}{loc. prep.}{(\textit{não obstante}) Apesar de.}{obs.tan.te}{0}
\verb{obstar}{}{}{}{}{v.t.}{Servir de obstáculo; causar impedimento; estorvar, embaraçar.}{obs.tar}{0}
\verb{obstar}{}{}{}{}{}{Fazer oposição; opor"-se.}{obs.tar}{\verboinum{1}}
\verb{obstetra}{é}{}{}{}{s.2g.}{Médico especialista em gravidez e parto.}{obs.te.tra}{0}
\verb{obstétrica}{}{Med.}{}{}{s.f.}{Obstetrícia.}{obs.té.tri.ca}{0}
\verb{obstetrícia}{}{Med.}{}{}{s.f.}{Parte da medicina que trata da gravidez e do parto; obstétrica.}{obs.te.trí.cia}{0}
\verb{obstetrício}{}{}{}{}{adj.}{Obstétrico.}{obs.te.trí.cio}{0}
\verb{obstétrico}{}{}{}{}{adj.}{Relativo à obstetrícia; obstetrício.}{obs.té.tri.co}{0}
\verb{obstinação}{}{}{"-ões}{}{s.f.}{Ato ou efeito de obstinar; persistência, tenacidade.}{obs.ti.na.ção}{0}
\verb{obstinação}{}{}{"-ões}{}{}{Teima, birra.}{obs.ti.na.ção}{0}
\verb{obstinado}{}{}{}{}{adj.}{Que não se deixa vencer; irredutível, inflexível.}{obs.ti.na.do}{0}
\verb{obstinado}{}{}{}{}{adj.}{Teimoso, pertinaz, firme.}{obs.ti.na.do}{0}
\verb{obstinar}{}{}{}{}{v.t.}{Tornar obstinado.}{obs.ti.nar}{0}
\verb{obstinar}{}{}{}{}{v.pron.}{Aferrar"-se a uma ideia; teimar, relutar.}{obs.ti.nar}{\verboinum{1}}
\verb{obstrução}{}{}{"-ões}{}{s.f.}{Ato ou efeito de obstruir; bloqueio.}{obs.tru.ção}{0}
\verb{obstrução}{}{Med.}{"-ões}{}{}{Impedimento parcial ou total nos vasos ou canais orgânicos.}{obs.tru.ção}{0}
\verb{obstruir}{}{}{}{}{v.t.}{Impedir com obstáculos; estorvar, embaraçar.}{obs.tru.ir}{0}
\verb{obstruir}{}{}{}{}{}{Fechar, entupir, tapar.}{obs.tru.ir}{\verboinum{26}}
\verb{obtemperar}{}{}{}{}{v.t.}{Responder com modéstia; ponderar.}{ob.tem.pe.rar}{0}
\verb{obtemperar}{}{}{}{}{}{Submeter, aquiescer, sujeitar.}{ob.tem.pe.rar}{\verboinum{1}}
\verb{obtenção}{}{}{"-ões}{}{s.f.}{Ato ou efeito de obter; aquisição, consecução.}{ob.ten.ção}{0}
\verb{obtenível}{}{}{"-eis}{}{adj.2g.}{Que pode ser obtido, adquirido.}{ob.te.ní.vel}{0}
\verb{obter}{ê}{}{}{}{v.t.}{Conseguir, alcançar o que se deseja.}{ob.ter}{0}
\verb{obter}{ê}{}{}{}{}{Adquirir, conquistar, ganhar.}{ob.ter}{\verboinum{39}}
\verb{obturação}{}{}{"-ões}{}{s.f.}{Ato ou efeito de obturar; obstrução.}{ob.tu.ra.ção}{0}
\verb{obturação}{}{}{"-ões}{}{}{Obstrução de cavidade dentária cariada.}{ob.tu.ra.ção}{0}
\verb{obturador}{ô}{}{}{}{adj.}{Que obtura, fecha, obstrui.}{ob.tu.ra.dor}{0}
\verb{obturador}{ô}{}{}{}{s.m.}{Dispositivo odontológico usado para obturar dentes cariados.}{ob.tu.ra.dor}{0}
\verb{obturador}{ô}{}{}{}{}{Dispositivo que fecha o tubo da objetiva da máquina fotográfica, para interceptar a entrada dos raios luminosos na câmara escura.}{ob.tu.ra.dor}{0}
\verb{obturar}{}{}{}{}{v.t.}{Tapar, fechar, obstruir.}{ob.tu.rar}{0}
\verb{obturar}{}{}{}{}{}{Obstruir cavidades de dentes cariados.}{ob.tu.rar}{\verboinum{1}}
\verb{obtusângulo}{}{Geom.}{}{}{adj.}{Diz"-se do triângulo que possui um ângulo obtuso.}{ob.tu.sân.gu.lo}{0}
\verb{obtusidade}{}{}{}{}{s.f.}{Qualidade de obtuso; estupidez, insensibilidade, ignorância.}{ob.tu.si.da.de}{0}
\verb{obtuso}{}{}{}{}{adj.}{Que não é agudo; rombo, arredondado. }{ob.tu.so}{0}
\verb{obtuso}{}{}{}{}{}{Rude, estúpido, ignorante.}{ob.tu.so}{0}
\verb{obtuso}{}{}{}{}{}{Pouco sensível; pouco apurado.}{ob.tu.so}{0}
\verb{obumbrar}{}{}{}{}{v.t.}{Cobrir de sombras; anuviar, nublar.}{o.bum.brar}{0}
\verb{obumbrar}{}{Fig.}{}{}{}{Disfarçar, ocultar, velar.}{o.bum.brar}{\verboinum{1}}
\verb{obus}{}{}{}{}{s.m.}{Projétil oco de forma cilíndrica, lançado por uma peça de artilharia; bomba.}{o.bus}{0}
\verb{obus}{}{}{}{}{}{Pequena peça de artilharia, semelhante a um morteiro comprido, com a qual se atiram granadas, bombas e fogos de artifício.}{o.bus}{0}
\verb{obviar}{}{}{}{}{v.t.}{Atenuar efeitos; remediar, atalhar.}{ob.vi.ar}{0}
\verb{obviar}{}{}{}{}{}{Apresentar resistência; opor"-se, obstar.}{ob.vi.ar}{\verboinum{6}}
\verb{óbvio}{}{}{}{}{adj.}{Que não requer esclarecimento ou explicação; evidente, patente.}{ób.vio}{0}
\verb{óbvio}{}{}{}{}{}{Que não se pode duvidar; incontestável. }{ób.vio}{0}
\verb{oca}{ó}{}{}{}{s.f.}{Construção de madeira ou barro, coberta com fibras vegetais, usada pelos índios como moradia; palhoça, cabana. }{o.ca}{0}
\verb{ocapi}{}{Zool.}{}{}{s.m.}{Mamífero africano da família das girafas, com o pescoço e as pernas mais curtas e de colorido uniforme no corpo.}{o.ca.pi}{0}
\verb{ocara}{}{Bras.}{}{}{s.f.}{Terreno circular no interior da aldeia indígena.}{o.ca.ra}{0}
\verb{ocarina}{}{Mús.}{}{}{s.f.}{Instrumento de sopro de forma ovoide, feito de porcelana, barro ou pedra, com pequena embocadura, oito furos, cujos sons se assemelham aos da flauta.}{o.ca.ri.na}{0}
\verb{ocasião}{}{}{"-ões}{}{s.f.}{Oportunidade favorável à realização de algo; motivo, ensejo, razão.}{o.ca.si.ão}{0}
\verb{ocasião}{}{}{"-ões}{}{}{Encontro de circunstâncias; conjuntura, situação.}{o.ca.si.ão}{0}
\verb{ocasião}{}{}{"-ões}{}{}{Tempo disponível; lazer.}{o.ca.si.ão}{0}
\verb{ocasional}{}{}{"-ais}{}{adj.2g.}{Que acontece por acaso; eventual, acidental, casual.}{o.ca.si.o.nal}{0}
\verb{ocasionar}{}{}{}{}{v.t.}{Dar motivo; causar, originar.}{o.ca.si.o.nar}{0}
\verb{ocasionar}{}{}{}{}{}{Dar oportunidade; proporcionar, propiciar.}{o.ca.si.o.nar}{0}
\verb{ocasionar}{}{}{}{}{v.pron.}{Suceder, ocorrer, acontecer.}{o.ca.si.o.nar}{\verboinum{1}}
\verb{ocaso}{}{}{}{}{s.m.}{O desaparecimento do sol no horizonte; pôr do sol, poente, ocidente.}{o.ca.so}{0}
\verb{ocaso}{}{Fig.}{}{}{}{Fase de decadência; declínio, fim, ruína.}{o.ca.so}{0}
\verb{occipício}{}{Anat.}{}{}{s.m.}{Parte inferior e posterior da cabeça.}{oc.ci.pí.cio}{0}
\verb{occipital}{}{Anat.}{"-ais}{}{adj.2g.}{Relativo ao occipício.}{oc.ci.pi.tal}{0}
\verb{oceânico}{}{}{}{}{adj.}{Relativo ao oceano ou à Oceania.}{o.ce.â.ni.co}{0}
\verb{oceânico}{}{}{}{}{}{Diz"-se do ser que vive no oceano.}{o.ce.â.ni.co}{0}
\verb{oceano}{}{}{}{}{s.m.}{Cada uma das vastas extensões de água que cobrem a maior parte da Terra e que constituem entidades geográficas isoladas em regiões.}{o.ce.a.no}{0}
\verb{oceano}{}{Fig.}{}{}{}{Grande quantidade; imensidão.}{o.ce.a.no}{0}
\verb{oceanografia}{}{}{}{}{s.f.}{Ciência que estuda os componentes físico, químico e biológico das águas e dos fundos dos oceanos e mares.}{o.ce.a.no.gra.fi.a}{0}
\verb{oceanográfico}{}{}{}{}{adj.}{Relativo à oceanografia.}{o.ce.a.no.grá.fi.co}{0}
\verb{oceanógrafo}{}{}{}{}{s.m.}{Estudioso que se dedica à oceanografia.}{o.ce.a.nó.gra.fo}{0}
\verb{ocidental}{}{}{}{}{adj.2g.}{Relativo a ou situado no Ocidente.}{o.ci.den.tal}{0}
\verb{ocidental}{}{}{}{}{s.2g.}{Indivíduo natural ou habitante dos países do Ocidente.}{o.ci.den.tal}{0}
\verb{ocidentalizar}{}{}{}{}{v.t.}{Adaptar à civilização do Ocidente.}{o.ci.den.ta.li.zar}{\verboinum{1}}
\verb{ocidente}{}{}{}{}{s.m.}{O lado do horizonte onde o sol se põe; poente, ocaso, este.}{o.ci.den.te}{0}
\verb{ocidente}{}{}{}{}{}{Hemisfério oeste da Terra, em relação ao meridiano de Greenwich.}{o.ci.den.te}{0}
\verb{ocidente}{}{}{}{}{}{Conjunto dos povos que habitam essa parte ou as regiões dessa parte da Terra.}{o.ci.den.te}{0}
\verb{ócio}{}{}{}{}{s.m.}{Falta de ocupação; inatividade.}{ó.cio}{0}
\verb{ócio}{}{}{}{}{}{Descanso do trabalho; folga, repouso.}{ó.cio}{0}
\verb{ócio}{}{}{}{}{}{Tempo à disposição; vagar, lazer.}{ó.cio}{0}
\verb{ociosidade}{}{}{}{}{s.f.}{Qualidade ou estado de ocioso; inatividade.}{o.ci.o.si.da.de}{0}
\verb{ociosidade}{}{}{}{}{}{Ausência de empenho, de disposição; preguiça, indolência. }{o.ci.o.si.da.de}{0}
\verb{ocioso}{ô}{}{"-osos ⟨ó⟩}{"-osa ⟨ó⟩}{}{Que não serve para nada; inútil, improdutivo.}{o.ci.o.so}{0}
\verb{ocioso}{ô}{}{"-osos ⟨ó⟩}{"-osa ⟨ó⟩}{}{Que não faz falta; desnecessário, supérfluo.}{o.ci.o.so}{0}
\verb{ocioso}{ô}{}{"-osos ⟨ó⟩}{"-osa ⟨ó⟩}{adj.}{Que não tem ocupação; inativo.}{o.ci.o.so}{0}
\verb{ocioso}{ô}{}{"-osos ⟨ó⟩}{"-osa ⟨ó⟩}{}{Vadio, preguiçoso, indolente.}{o.ci.o.so}{0}
\verb{oclusão}{}{}{"-ões}{}{s.f.}{Ato ou efeito de fechar; fechamento, cerramento.}{o.clu.são}{0}
\verb{oclusão}{}{}{"-ões}{}{}{Estado do que se encontra fechado; obstrução.}{o.clu.são}{0}
\verb{oclusão}{}{}{"-ões}{}{}{Apagamento total ou parcial; obscurecimento, escurecimento.}{o.clu.são}{0}
\verb{oclusivo}{}{}{}{}{adj.}{Que causa oclusão.}{o.clu.si.vo}{0}
\verb{oclusivo}{}{Gram.}{}{}{}{Diz"-se do segmento consonantal cuja articulação está relacionada à interrupção brusca da passagem de ar pela cavidade bucal.}{o.clu.si.vo}{0}
\verb{ocluso}{}{}{}{}{adj.}{Em que há oclusão; fechado, tapado.}{o.clu.so}{0}
\verb{oco}{ô}{}{}{}{adj.}{Que é vazio por dentro; vão, escavado.}{o.co}{0}
\verb{oco}{ô}{}{}{}{}{Que não tem miolo ou medula.}{o.co}{0}
\verb{oco}{ô}{}{}{}{}{Que tem pouca importância; fútil, insignificante.}{o.co}{0}
\verb{ocorrência}{}{}{}{}{s.f.}{O que ocorre; acontecimento, fato, evento.}{o.cor.rên.cia}{0}
\verb{ocorrência}{}{}{}{}{}{Circunstância fortuita; acaso, eventualidade.}{o.cor.rên.cia}{0}
\verb{ocorrente}{}{}{}{}{adj.2g.}{Que ocorre, sucede; convergente.}{o.cor.ren.te}{0}
\verb{ocorrer}{ê}{}{}{}{v.i.}{Acontecer, suceder, dar"-se.}{o.cor.rer}{0}
\verb{ocorrer}{ê}{}{}{}{}{Aparecer subitamente; sobrevir.}{o.cor.rer}{0}
\verb{ocorrer}{ê}{}{}{}{}{Vir à memória ou ao pensamento; lembrar, acudir.}{o.cor.rer}{\verboinum{12}}
\verb{ocorrido}{}{}{}{}{adj.}{Que ocorreu; acontecido, sucedido.}{o.cor.ri.do}{0}
\verb{ocorrido}{}{}{}{}{s.m.}{Acontecimento, evento.}{o.cor.ri.do}{0}
\verb{ocra}{ó}{}{}{}{s.f.}{Ocre.}{o.cra}{0}
\verb{ocre}{ó}{}{}{}{s.m.}{Tipo de argila colorida pela presença de óxido de ferro que apresenta tonalidades pardacentas.}{o.cre}{0}
\verb{ocre}{ó}{}{}{}{}{A cor dessa terra.}{o.cre}{0}
\verb{octaedro}{é}{Geom.}{}{}{s.m.}{Poliedro que contém oito faces.}{oc.ta.e.dro}{0}
\verb{octana}{}{Quím.}{}{}{s.f.}{Unidade em que se mede a octanagem de um combustível.}{oc.ta.na}{0}
\verb{octanagem}{}{}{"-ens}{}{s.f.}{Medida utilizada para avaliar a qualidade dos combustíveis usados em motores de explosão.}{oc.ta.na.gem}{0}
\verb{octano}{}{Quím.}{}{}{s.m.}{Hidrocarboneto existente no petróleo, usado como solvente e intermediário químico.}{oc.ta.no}{0}
%\verb{}{}{}{}{}{}{}{}{0}
%\verb{}{}{}{}{}{}{}{}{0}
\verb{octingentésimo}{}{}{}{}{num.}{Ordinal e fracionário correspondente a 800; oitocentésimo.}{oc.tin.gen.té.si.mo}{0}
\verb{octogenário}{}{}{}{}{adj.}{Que tem oitenta unidades.}{oc.to.ge.ná.rio}{0}
\verb{octogenário}{}{}{}{}{}{Que está na casa dos oitenta anos de idade.}{oc.to.ge.ná.rio}{0}
\verb{octogenário}{}{}{}{}{s.m.}{Indivíduo que está nessa faixa etária.}{oc.to.ge.ná.rio}{0}
\verb{octogésimo}{}{}{}{}{num.}{Ordinal e fracionário correspondente a 80.}{oc.to.gé.si.mo}{0}
\verb{octogonal}{}{}{"-ais}{}{adj.2g.}{Relativo a octógono.}{oc.to.go.nal}{0}
\verb{octogonal}{}{}{"-ais}{}{}{Que possui oito lados.}{oc.to.go.nal}{0}
\verb{octógono}{}{Geom.}{}{}{s.m.}{Polígono que possui oito ângulos, consequentemente, oito lados.   }{oc.tó.go.no}{0}
\verb{octossílabo}{}{Gram.}{}{}{adj.}{Diz"-se do verso que possui oito sílabas.}{oc.tos.sí.la.bo}{0}
\verb{octuplicar}{}{}{}{}{v.t.}{Multiplicar por oito; tornar oito vezes maior.}{oc.tu.pli.car}{\verboinum{2}}
\verb{óctuplo}{}{}{}{}{num.}{Que contém oito vezes o mesmo elemento ou quantidade.  }{óc.tu.plo}{0}
\verb{óctuplo}{}{}{}{}{s.m.}{Quantidade oito vezes maior que outra.}{óc.tu.plo}{0}
\verb{ocular}{}{}{}{}{adj.2g.}{Relativo ao olho ou à vista.}{o.cu.lar}{0}
\verb{ocular}{}{}{}{}{}{Diz"-se da testemunha que presenciou um fato.}{o.cu.lar}{0}
\verb{ocular}{}{}{}{}{s.f.}{Parte de um sistema óptico que serve para examinar a imagem fornecida pela objetiva.}{o.cu.lar}{0}
\verb{oculista}{}{}{}{}{s.2g.}{Médico especialista em doenças dos olhos; oftalmologista.}{o.cu.lis.ta}{0}
\verb{oculista}{}{}{}{}{}{Indivíduo que fabrica ou vende óculos.}{o.cu.lis.ta}{0}
\verb{óculo}{}{}{}{}{s.m.}{Instrumento composto de lentes que auxiliam a visão; luneta.}{ó.cu.lo}{0}
\verb{óculo}{}{}{}{}{}{Abertura circular na parede de um edifício, para entrada do ar e da luz.}{ó.cu.lo}{0}
\verb{óculo}{}{}{}{}{}{Armação usada para proteger os olhos dos motociclistas.}{ó.cu.lo}{0}
\verb{óculos}{}{}{}{}{s.m.pl.}{Objeto que consiste em um par de lentes sustentadas em frente dos olhos por uma armação, usado para corrigir a visão ou atenuar a intensidade luminosa.}{ó.cu.los}{0}
\verb{ocultação}{}{}{"-ões}{}{s.f.}{Ato ou efeito de ocultar, esconder; encobrimento.}{o.cul.ta.ção}{0}
\verb{ocultação}{}{}{"-ões}{}{}{Dissimulação, disfarce.}{o.cul.ta.ção}{0}
\verb{ocultar}{}{}{}{}{v.t.}{Não deixar ver; encobrir, esconder.}{o.cul.tar}{0}
\verb{ocultar}{}{}{}{}{}{Não demonstrar; dissimular, disfarçar.}{o.cul.tar}{0}
\verb{ocultar}{}{}{}{}{}{Deixar de mencionar; sonegar, guardar.}{o.cul.tar}{\verboinum{1}}
\verb{ocultas}{}{}{}{}{s.f.pl.}{Usado na locução \textit{às ocultas}: de modo oculto; às escondidas.}{o.cul.tas}{0}
\verb{ocultismo}{}{}{}{}{s.m.}{Conjunto das ciências ocultas que tratam de fenômenos sem explicação pelas leis naturais; esoterismo, magia, astrologia.}{o.cul.tis.mo}{0}
\verb{ocultismo}{}{}{}{}{}{Crença na ação ou na influência de poderes sobrenaturais.}{o.cul.tis.mo}{0}
\verb{ocultista}{}{}{}{}{adj.2g.}{Relativo ao ocultismo.}{o.cul.tis.ta}{0}
\verb{ocultista}{}{}{}{}{s.2g.}{Indivíduo que estuda ou pratica o ocultismo.}{o.cul.tis.ta}{0}
\verb{oculto}{}{}{}{}{adj.}{Que está encoberto, escondido.}{o.cul.to}{0}
\verb{oculto}{}{}{}{}{}{Que não se conhece; inexplorado.}{o.cul.to}{0}
\verb{oculto}{}{}{}{}{}{Misterioso, sobrenatural, secreto.}{o.cul.to}{0}
\verb{oculto}{}{Gram.}{}{}{}{Diz"-se do sujeito que não está expresso na oração, mas que pode ser identificado.}{o.cul.to}{0}
\verb{ocupação}{}{}{"-ões}{}{s.f.}{Ato ou efeito de ocupar.}{o.cu.pa.ção}{0}
\verb{ocupação}{}{}{"-ões}{}{}{Ação de trabalhar em algo; emprego, ofício, serviço.}{o.cu.pa.ção}{0}
\verb{ocupação}{}{}{"-ões}{}{}{Obrigação a cumprir em determinada atividade profissional; cargo, função.}{o.cu.pa.ção}{0}
\verb{ocupação}{}{}{"-ões}{}{}{Tarefa, atividade, afazeres.}{o.cu.pa.ção}{0}
\verb{ocupacional}{}{}{"-ais}{}{adj.2g.}{Relativo a ocupação, atividade.}{o.cu.pa.ci.o.nal}{0}
\verb{ocupacional}{}{}{"-ais}{}{}{Diz"-se da técnica terapêutica que busca interessar o paciente por determinado tipo de trabalho ou atividade.}{o.cu.pa.ci.o.nal}{0}
\verb{ocupado}{}{}{}{}{adj.}{Que tem muito trabalho a fazer; atarefado.}{o.cu.pa.do}{0}
\verb{ocupado}{}{}{}{}{}{Que não está disponível, nem livre; preenchido.}{o.cu.pa.do}{0}
\verb{ocupado}{}{}{}{}{}{De que se tomou posse ou se recebeu por concessão; dominado.}{o.cu.pa.do}{0}
\verb{ocupante}{}{}{}{}{adj.2g.}{Que ocupa, que se apodera de algo.}{o.cu.pan.te}{0}
\verb{ocupar}{}{}{}{}{v.t.}{Preencher um espaço de tempo ou de lugar.}{o.cu.par}{0}
\verb{ocupar}{}{}{}{}{}{Tomar posse; apoderar, tomar, conquistar.}{o.cu.par}{0}
\verb{ocupar}{}{}{}{}{}{Dar trabalho; atarefar, empregar.}{o.cu.par}{0}
\verb{ocupar}{}{}{}{}{}{Prender ou atrair a atenção; entreter.}{o.cu.par}{0}
\verb{ocupar}{}{}{}{}{v.pron.}{Entregar"-se com zelo; dedicar"-se.}{o.cu.par}{\verboinum{1}}
\verb{odalisca}{}{}{}{}{s.f.}{Escrava de harém a serviço das mulheres do sultão.}{o.da.lis.ca}{0}
\verb{odalisca}{}{}{}{}{}{Concubina do sultão.}{o.da.lis.ca}{0}
\verb{ode}{ó}{Gram.}{}{}{s.f.}{Poema lírico composto de estrofes simétricas, destinado a celebrar grandes eventos ou pessoas notáveis com tom entusiástico e alegre.}{o.de}{0}
\verb{ode}{ó}{}{}{}{}{Na Grécia Antiga, poema composto para ser cantado.}{o.de}{0}
\verb{odiar}{}{}{}{}{v.t.}{Ter ódio ou raiva; detestar, abominar.}{o.di.ar}{0}
\verb{odiar}{}{}{}{}{}{Sentir aversão ou repulsa; aborrecer profundamente.}{o.di.ar}{0}
\verb{odiar}{}{}{}{}{}{Ter inimizade; indispor, intrigar.}{o.di.ar}{\verboinum{6}}
\verb{odiento}{}{}{}{}{adj.}{Que tem ódio ou rancor; rancoroso.}{o.di.en.to}{0}
\verb{odiento}{}{}{}{}{}{Que revela ódio; odioso.}{o.di.en.to}{0}
\verb{ódio}{}{}{}{}{s.m.}{Sentimento de profunda aversão; repugnância, repulsa.}{ó.dio}{0}
\verb{ódio}{}{}{}{}{}{Rancor violento e duradouro; ira, raiva.}{ó.dio}{0}
\verb{ódio}{}{}{}{}{}{Antipatia, desprezo, malquerença.}{ó.dio}{0}
\verb{odiosidade}{}{}{}{}{s.f.}{Qualidade de odioso.}{o.di.o.si.da.de}{0}
\verb{odioso}{ô}{}{"-osos ⟨ó⟩}{"-osa ⟨ó⟩}{adj.}{Que provoca ódio, indignação; detestável, execrável.}{o.di.o.so}{0}
\verb{odioso}{ô}{}{"-osos ⟨ó⟩}{"-osa ⟨ó⟩}{}{Que inspira aversão; repulsivo, insuportável.}{o.di.o.so}{0}
\verb{odisseia}{é}{}{}{}{s.f.}{Viagem cheia de aventuras extraordinárias e inesperadas, por alusão à obra homônima de Homero, poeta grego da Antiguidade.}{o.dis.sei.a}{0}
\verb{odisseia}{é}{Por ext.}{}{}{}{Qualquer série de acontecimentos e aventuras estranhas e variadas.}{o.dis.sei.a}{0}
\verb{odonato}{}{Zool.}{}{}{s.m.}{Espécime da ordem dos odonatos, composta de insetos com quatro asas, longas e enervadas, olhos compostos grandes, e cujo representante mais comum é a libélula.}{o.do.na.to}{0}
\verb{odontologia}{}{}{}{}{s.f.}{Ramo da medicina que tem por objeto o estudo e o tratamento dos dentes e dos tecidos dentários.}{o.don.to.lo.gi.a}{0}
\verb{odontológico}{}{}{}{}{adj.}{Relativo à odontologia.}{o.don.to.ló.gi.co}{0}
\verb{odontologista}{}{}{}{}{s.2g.}{Especialista que se dedica ao estudo e ao tratamento dos dentes; dentista; odontólogo.}{o.don.to.lo.gis.ta}{0}
\verb{odontólogo}{}{}{}{}{s.m.}{Odontologista.}{o.don.tó.lo.go}{0}
\verb{odor}{ô}{}{}{}{s.m.}{Emanação que se desprende dos corpos e que é percebida pelo olfato; cheiro.}{o.dor}{0}
\verb{odor}{ô}{}{}{}{}{Fragrância suave e agradável; aroma, perfume, olor.}{o.dor}{0}
\verb{odorante}{}{}{}{}{adj.2g.}{Que espalha odor, geralmente agradável; odorífero. }{o.do.ran.te}{0}
\verb{odorífero}{}{}{}{}{adj.}{Que exala odor; odorante, fragrante.}{o.do.rí.fe.ro}{0}
\verb{odorífico}{}{}{}{}{adj.}{Odorante.}{o.do.rí.fi.co}{0}
\verb{odoroso}{ô}{}{"-osos ⟨ó⟩}{"-osa ⟨ó⟩}{adj.}{Odorante.}{o.do.ro.so}{0}
\verb{odre}{ô}{}{}{}{s.m.}{Saco feito de pele ou de couro que serve para transportar líquidos.}{o.dre}{0}
\verb{odre}{ô}{Fig.}{}{}{}{Indivíduo extremamente gordo.}{o.dre}{0}
\verb{odre}{ô}{Fig.}{}{}{}{Indivíduo que bebe com frequência; beberrão.}{o.dre}{0}
\verb{oés"-nordeste}{é}{}{oés"-nordestes ⟨é⟩}{}{s.m.}{Ponto do horizonte entre o oeste e o nordeste. Abrev. \textsc{o.n.e}. ou \textsc{w.n.e}.}{o.és"-nor.des.te}{0}
\verb{oés"-nordeste}{é}{}{oés"-nordestes ⟨é⟩}{}{}{Vento que sopra desse ponto.}{o.és"-nor.des.te}{0}
\verb{oés"-noroeste}{é}{}{oés"-noroestes ⟨é⟩}{}{s.m.}{Ponto do horizonte entre o oeste e o noroeste. Abrev. \textsc{o.n.o}. ou \textsc{w.n.w}.}{o.és"-no.ro.es.te}{0}
\verb{oés"-noroeste}{é}{}{oés"-noroestes ⟨é⟩}{}{}{Vento que sopra desse ponto.}{o.és"-no.ro.es.te}{0}
\verb{oés"-sudoeste}{é}{}{oés"-sudoestes ⟨é⟩}{}{s.m.}{Ponto do horizonte entre o oeste e o sudoeste. Abrev. \textsc{o.s.o}. ou \textsc{w.s.w}.}{o.és"-su.do.es.te}{0}
\verb{oés"-sudoeste}{é}{}{oés"-sudoestes ⟨é⟩}{}{}{Vento que sopra desse ponto.}{o.és"-su.do.es.te}{0}
\verb{oés"-sueste}{é}{}{oés"-suestes ⟨é⟩}{}{s.m.}{Ponto do horizonte entre o oeste e o sudeste. Abrev. \textsc{o.s.e}. ou \textsc{w.s.e}.}{o.és"-su.es.te}{0}
\verb{oés"-sueste}{é}{}{oés"-suestes ⟨é⟩}{}{}{Vento que sopra desse ponto.}{o.és"-su.es.te}{0}
\verb{oeste}{é}{}{}{}{s.m.}{Ponto no horizonte onde o sol se põe; poente, ocidente.}{o.es.te}{0}
\verb{oeste}{é}{}{}{}{}{Um dos quatro pontos cardeais. Abrev. \textsc{o}.}{o.es.te}{0}
\verb{oeste}{é}{}{}{}{}{Vento que sopra do poente.}{o.es.te}{0}
\verb{ofegante}{}{}{}{}{adj.2g.}{Que respira com dificuldade; arfante, arquejante.}{o.fe.gan.te}{0}
\verb{ofegante}{}{}{}{}{}{Cansado, exausto, exaurido.}{o.fe.gan.te}{0}
\verb{ofegar}{}{}{}{}{v.i.}{Respirar com dificuldade; arquejar, arfar.}{o.fe.gar}{0}
\verb{ofegar}{}{}{}{}{}{Estar muito ansioso; anelar.}{o.fe.gar}{\verboinum{5}}
\verb{ofender}{ê}{}{}{}{v.t.}{Causar desgosto; magoar, desagradar.}{o.fen.der}{0}
\verb{ofender}{ê}{}{}{}{}{Causar ferimento; lesar, machucar.}{o.fen.der}{0}
\verb{ofender}{ê}{}{}{}{}{Atentar contra regras ou preceitos; escandalizar, contrariar.}{o.fen.der}{0}
\verb{ofender}{ê}{}{}{}{}{Estuprar, violentar.}{o.fen.der}{\verboinum{12}}
\verb{ofendido}{}{}{}{}{adj.}{Que recebeu ofensa; ultrajado, magoado.}{o.fen.di.do}{0}
\verb{ofensa}{}{}{}{}{s.f.}{Palavra ou ato que atinge alguém em sua dignidade ou honra; afronta, ultraje.}{o.fen.sa}{0}
\verb{ofensa}{}{}{}{}{}{Desgosto, aborrecimento, mágoa da pessoa ofendida.}{o.fen.sa}{0}
\verb{ofensa}{}{}{}{}{}{Desconsideração, desacato, menosprezo.}{o.fen.sa}{0}
\verb{ofensa}{}{}{}{}{}{Violação de uma regra ou norma; falta, pecado.}{o.fen.sa}{0}
\verb{ofensiva}{}{}{}{}{s.f.}{Ação visando ao ataque; assalto, investida.}{o.fen.si.va}{0}
\verb{ofensiva}{}{}{}{}{}{Iniciativa no ataque.}{o.fen.si.va}{0}
\verb{ofensivo}{}{}{}{}{}{Que provoca dano moral, magoa; prejudicial.}{o.fen.si.vo}{0}
\verb{ofensivo}{}{}{}{}{adj.}{Que é próprio de ataque; agressivo.}{o.fen.si.vo}{0}
\verb{ofensor}{ô}{}{}{}{adj.}{Que ofende, agride.}{o.fen.sor}{0}
\verb{oferecer}{ê}{}{}{}{v.t.}{Apresentar alguma coisa, querendo que a pessoa fique com ela. (\textit{Eu ofereci um doce à criança.})}{o.fe.re.cer}{0}
\verb{oferecer}{ê}{}{}{}{}{Querer ser aceito para alguma coisa; apresentar"-se. (\textit{Eu me ofereci para aquele emprego.})}{o.fe.re.cer}{\verboinum{15}}
\verb{oferecimento}{}{}{}{}{s.m.}{Ato ou efeito de oferecer.}{o.fe.re.ci.men.to}{0}
\verb{oferecimento}{}{}{}{}{}{Disposição favorável a execução de algo; proposta. }{o.fe.re.ci.men.to}{0}
\verb{oferecimento}{}{}{}{}{}{Dedicatória de uma obra a alguém.}{o.fe.re.ci.men.to}{0}
\verb{oferenda}{}{}{}{}{s.f.}{Aquilo que se oferece; presente, dádiva, oferta.}{o.fe.ren.da}{0}
\verb{oferenda}{}{Relig.}{}{}{}{Oferta feita à divindade; oblação.}{o.fe.ren.da}{0}
\verb{oferta}{é}{}{}{}{s.f.}{Ato ou efeito de oferecer; oferta.}{o.fer.ta}{0}
\verb{oferta}{é}{}{}{}{}{Aquilo que se oferece; oferecimento.}{o.fer.ta}{0}
\verb{oferta}{é}{Relig.}{}{}{}{Doação oferecida aos santos ou a Deus; oblação.}{o.fer.ta}{0}
\verb{oferta}{é}{Econ.}{}{}{}{Quantidade de um bem ou de um serviço que se oferece no mercado para venda.}{o.fer.ta}{0}
\verb{ofertar}{}{}{}{}{v.t.}{Fazer ou apresentar uma oferta; doar.}{o.fer.tar}{\verboinum{1}}
\verb{ofertório}{}{Relig.}{}{}{s.m.}{Parte da missa católica em que o padre oferece a Deus o vinho e o pão.}{o.fer.tó.rio}{0}
\verb{office"-boy}{}{}{}{}{s.m.}{Indivíduo de qualquer idade, empregado num escritório para fazer pequenas tarefas de rua e internas; contínuo.}{\textit{office"-boy}}{0}
\verb{off"-line}{}{}{}{}{adj.}{Diz"-se do equipamento que não está conectado, que está fora do ar.}{\textit{off"-line}}{0}
\verb{offset}{}{}{}{}{s.m.}{Processo de impressão indireta, litográfica, em que a imagem é gravada numa chapa de metal flexível, e, em seguida, transferida para o papel por meio de um cilindro revestido de borracha.}{\textit{offset}}{0}
\verb{oficial}{}{}{"-ais}{}{adj.2g.}{Que é feito por uma autoridade. (\textit{O registro de nascimento é um documento oficial.})}{o.fi.ci.al}{0}
\verb{oficial}{}{}{"-ais}{}{}{Que tem caráter formal e solene. (\textit{O presidente fez uma visita oficial aos países vizinhos.})}{o.fi.ci.al}{0}
\verb{oficial}{}{}{"-ais}{}{}{Militar com posto superior ao de aspirante ou de guarda"-marinha. (\textit{Ele já tem um posto de oficial.})}{o.fi.ci.al}{0}
\verb{oficial}{}{}{"-ais}{}{}{Operário especializado, de função superior à de servente. (\textit{Ele trabalha como oficial nas obras de construção do prédio.})}{o.fi.ci.al}{0}
\verb{oficialato}{}{}{}{}{s.m.}{Cargo, dignidade ou função de oficial das Forças Armadas.}{o.fi.ci.a.la.to}{0}
\verb{oficial"-general}{}{}{oficiais"-generais}{}{s.m.}{Na hierarquia do Exército e da Aeronáutica, posto superior ao de coronel, e, na Marinha, superior ao de capitão"-de"-mar"-e"-guerra.}{o.fi.ci.al"-ge.ne.ral}{0}
\verb{oficial"-general}{}{}{oficiais"-generais}{}{}{Militar que ocupa esse posto.}{o.fi.ci.al"-ge.ne.ral}{0}
\verb{oficialidade}{}{}{}{}{s.f.}{Conjunto de oficiais das Forças Armadas ou de apenas de uma de suas unidades.}{o.fi.ci.a.li.da.de}{0}
\verb{oficialização}{}{}{"-ões}{}{s.f.}{Ato ou efeito de tornar ou dar caráter oficial.}{o.fi.ci.a.li.za.ção}{0}
\verb{oficializado}{}{}{}{}{adj.}{Que se tornou oficial.}{o.fi.ci.a.li.za.do}{0}
\verb{oficializar}{}{}{}{}{v.t.}{Tornar oficial.}{o.fi.ci.a.li.zar}{0}
\verb{oficializar}{}{}{}{}{}{Dar caráter ou sanção oficial.}{o.fi.ci.a.li.zar}{\verboinum{1}}
\verb{oficiar}{}{}{}{}{v.i.}{Celebrar algum ofício divino; celebrar a missa.}{o.fi.ci.ar}{0}
\verb{oficiar}{}{}{}{}{v.t.}{Endereçar um ofício, uma comunicação a alguém.}{o.fi.ci.ar}{\verboinum{6}}
\verb{oficina}{}{}{}{}{s.f.}{Local onde se trabalha ou onde se exerce um ofício.}{o.fi.ci.na}{0}
\verb{oficina}{}{}{}{}{}{Local onde se consertam automóveis.}{o.fi.ci.na}{0}
\verb{oficina}{}{}{}{}{}{Curso intensivo, de curta duração, em que técnicas, habilidades, artes etc. são demonstradas e aplicadas; \textit{workshop}.}{o.fi.ci.na}{0}
\verb{ofício}{}{}{}{}{s.m.}{Atividade especializada; profissão, trabalho.}{o.fí.cio}{0}
\verb{ofício}{}{}{}{}{}{Tarefa com que uma pessoa se compromete; incumbência, encargo.}{o.fí.cio}{0}
\verb{ofício}{}{}{}{}{}{Forma de correspondência ou comunicação no serviço público oficial entre autoridades.}{o.fí.cio}{0}
\verb{ofício}{}{}{}{}{}{Conjunto de orações e de cerimônias religiosas; missa.}{o.fí.cio}{0}
\verb{oficioso}{ô}{}{"-osos ⟨ó⟩}{"-osa ⟨ó⟩}{adj.}{Que emana de fontes oficiais, mas sem caráter oficial.}{o.fi.ci.o.so}{0}
\verb{oficioso}{ô}{}{"-osos ⟨ó⟩}{"-osa ⟨ó⟩}{}{Que costuma prestar serviços; prestativo, serviçal.}{o.fi.ci.o.so}{0}
\verb{ofídico}{}{}{}{}{adj.}{Relativo a ofídio, serpente.}{o.fí.di.co}{0}
\verb{ofídio}{}{Zool.}{}{}{s.m.}{Ordem de répteis com o corpo coberto de escamas, sem pernas e de olhos sem movimento; serpente, cobra.  }{o.fí.dio}{0}
\verb{ofsete}{é}{}{}{}{s.m.}{Forma aportuguesada de \textit{off"-set}.}{of.se.te}{0}
\verb{oftálmico}{}{}{}{}{adj.}{Relativo aos olhos.}{of.tál.mi.co}{0}
\verb{oftalmologia}{}{Med.}{}{}{s.f.}{Ramo da medicina que estuda os olhos e as afecções oculares.}{of.tal.mo.lo.gi.a}{0}
\verb{oftalmológico}{}{}{}{}{adj.}{Relativo à oftalmologia.}{of.tal.mo.ló.gi.co}{0}
\verb{oftalmologista}{}{}{}{}{s.2g.}{Médico especialista em oftalmologia; oculista.}{of.tal.mo.lo.gis.ta}{0}
\verb{oftalmopatia}{}{Med.}{}{}{s.f.}{Nome comum dado às doenças e afecções oculares.}{of.tal.mo.pa.ti.a}{0}
\verb{ofuscação}{}{}{"-ões}{}{s.f.}{Ato ou efeito de ofuscar, encobrir; obscurecimento.}{o.fus.ca.ção}{0}
\verb{ofuscar}{}{}{}{}{v.t.}{Impedir de ver ou de ser visto; ocultar, encobrir.}{o.fus.car}{0}
\verb{ofuscar}{}{}{}{}{}{Turvar a vista; deslumbrar, desorientar.}{o.fus.car}{0}
\verb{ofuscar}{}{}{}{}{}{Tornar menos distinto; empanar, embaçar.}{o.fus.car}{0}
\verb{ofuscar}{}{Fig.}{}{}{}{Fazer perder o prestígio; desvirtuar, diminuir.}{o.fus.car}{\verboinum{2}}
\verb{ogiva}{}{}{}{}{s.f.}{Figura arquitetônica formada pelo cruzamento de dois arcos que se encontram e formam um ângulo mais ou menos agudo na parte superior.}{o.gi.va}{0}
\verb{ogiva}{}{}{}{}{}{Parte anterior de um projétil, de forma cônica.}{o.gi.va}{0}
\verb{ogiva}{}{}{}{}{}{Parte frontal de um foguete ou míssil, que carrega a carga nuclear ou atômica.}{o.gi.va}{0}
\verb{ogival}{}{}{"-ais}{}{adj.2g.}{Relativo a ogiva.}{o.gi.val}{0}
\verb{ogival}{}{}{"-ais}{}{}{Que tem forma de ogiva; cônico.}{o.gi.val}{0}
\verb{ogro}{ô}{}{}{}{s.m.}{Ser monstruoso fantástico no qual se fala para assustar as crianças.}{o.gro}{0}
\verb{ogum}{}{Relig.}{}{}{s.m.}{No candomblé, orixá masculino que preside as lutas e as guerras.}{o.gum}{0}
\verb{oh}{}{}{}{}{interj.}{Expressão que denota surpresa, espanto, tristeza, admiração, pena etc.}{oh}{0}
\verb{ohm}{}{Fís.}{}{}{s.m.}{No Sistema Internacional de unidades (\textsc{si}), unidade de resistência ou impedância elétrica.}{ohm}{0}
\verb{oi}{}{}{}{}{interj.}{Expressão com que se chama, ou se saúda alguém.}{oi}{0}
\verb{oitão}{}{}{"-ões}{}{s.m.}{Cada uma das paredes laterais de uma casa, situada nas linhas de divisa do lote.}{oi.tão}{0}
\verb{oitava}{}{}{}{}{s.f.}{Cada uma das oito partes iguais em que uma coisa pode ser dividida; a oitava parte.}{oi.ta.va}{0}
\verb{oitava}{}{Gram.}{}{}{}{Estrofe ou poema de oito versos.}{oi.ta.va}{0}
\verb{oitava}{}{Mús.}{}{}{}{Espaço entre duas notas musicais de mesmo nome, sendo uma mais grave e a outra mais aguda.}{oi.ta.va}{0}
\verb{oitava"-de"-final}{}{Esport.}{oitavas"-de"-final}{}{s.f.}{Nos torneios por eliminação, fase na qual oito duplas de equipes concorrentes disputam entre si, em oito jogos, a classificação às quartas"-de"-final.}{oi.ta.va"-de"-fi.nal}{0}
\verb{oitavado}{}{}{}{}{adj.}{Que apresenta oito faces ou lados; octogonal.}{oi.ta.va.do}{0}
\verb{oitavo}{}{}{}{}{num.}{Ordinal e fracionário correspondente a 8.}{oi.ta.vo}{0}
\verb{oiteiro}{ê}{}{}{}{}{Var. de \textit{outeiro}.}{oi.tei.ro}{0}
\verb{oitenta}{}{}{}{}{num.}{Nome dado à quantidade expressa pelo número 80.  }{oi.ten.ta}{0}
\verb{oiti}{}{Bot.}{}{}{s.m.}{Árvore rosácea de frutos comestíveis.   }{oi.ti}{0}
\verb{oiticica}{}{Bot.}{}{}{s.f.}{Árvore rosácea cujas sementes são usadas na fabricação de óleo.  }{oi.ti.ci.ca}{0}
\verb{oitiva}{}{}{}{}{s.f.}{Audição, ouvido.}{oi.ti.va}{0}
\verb{oitiva}{}{}{}{}{}{Usado na expressão \textit{de oitiva}: por ouvir dizer; de ouvido.}{oi.ti.va}{0}
\verb{oito}{ô}{}{}{}{num.}{Nome dado à quantidade expressa pelo número 8.  }{oi.to}{0}
\verb{oitocentésimo}{}{}{}{}{num.}{Ordinal e fracionário correspondente a 800; octingentésimo.}{oi.to.cen.té.si.mo}{0}
\verb{oitocentos}{}{}{}{}{num.}{Nome dado à quantidade expressa pelo número 800.  }{oi.to.cen.tos}{0}
\verb{ojeriza}{}{}{}{}{s.f.}{Sentimento de aversão, má vontade; antipatia, nojo.}{o.je.ri.za}{0}
\verb{ola}{ô}{}{}{}{s.f.}{Em jogos esportivos, movimento em que a torcida, ao levantar"-se com os braços erguidos e sentar"-se gradualmente, realiza um movimento harmônico que lembra o das ondas do mar. }{o.la}{0}
\verb{olá}{}{}{}{}{interj.}{Expressão usada como chamamento ou saudação.}{o.lá}{0}
\verb{olaria}{}{}{}{}{s.f.}{Fábrica de vasilhames de barro, manilhas, telhas, tijolos etc.}{o.la.ri.a}{0}
\verb{olaria}{}{}{}{}{}{Arte ou ofício de oleiro.}{o.la.ri.a}{0}
\verb{olé}{}{Esport.}{}{}{s.m.}{Série de dribles ou de passes entre os jogadores de uma equipe que deixa a outra equipe desnorteada. }{o.lé}{0}
\verb{olé}{}{}{}{}{interj.}{Exclamação com que a torcida aplaude essas jogadas.}{o.lé}{0}
\verb{oleado}{}{}{}{}{adj.}{Que contém óleo; oleoso.}{o.le.a.do}{0}
\verb{oleado}{}{}{}{}{s.m.}{Pano preparado com substância impermeável; lona, encerado.}{o.le.a.do}{0}
\verb{oleaginoso}{ô}{}{"-osos ⟨ó⟩}{"-osa ⟨ó⟩}{adj.}{Que contém óleo; oleoso.}{o.le.a.gi.no.so}{0}
\verb{oleaginoso}{ô}{}{"-osos ⟨ó⟩}{"-osa ⟨ó⟩}{}{Diz"-se do fruto que contém ou fornece óleos.}{o.le.a.gi.no.so}{0}
\verb{olear}{}{}{}{}{v.t.}{Untar com óleo ou com substância oleosa.}{o.le.ar}{0}
\verb{olear}{}{}{}{}{}{Impregnar de óleo ou substância oleosa.}{o.le.ar}{\verboinum{4}}
\verb{oleicultor}{ô}{}{}{}{s.m.}{Indivíduo que se dedica ao cultivo de oliveiras.}{o.lei.cul.tor}{0}
\verb{oleicultura}{}{}{}{}{s.f.}{Cultivo de oliveiras.}{o.lei.cul.tu.ra}{0}
\verb{oleicultura}{}{}{}{}{}{Indústria, tratamento e conservação do azeite.}{o.lei.cul.tu.ra}{0}
\verb{oleiro}{ê}{}{}{}{s.m.}{Indivíduo que trabalha em olaria, fazendo e vendendo objetos de cerâmica ou barro; ceramista.}{o.lei.ro}{0}
\verb{oleiro}{ê}{Zool.}{}{}{}{João"-de"-barro.}{o.lei.ro}{0}
\verb{olente}{}{}{}{}{adj.2g.}{Que cheira; fragrante, perfumado, aromático.}{o.len.te}{0}
\verb{óleo}{}{}{}{}{s.m.}{Líquido gorduroso de origem animal, vegetal ou mineral. (\textit{O óleo de fígado de bacalhau sempre foi um bom remédio. Usa"-se muito o óleo de soja. O óleo diesel é usado como combustível para caminhões e ônibus.})}{ó.leo}{0}
\verb{óleo}{}{}{}{}{}{Perfume que se obtém pela maceração de flores; essência. (\textit{Óleo de rosas.})}{ó.leo}{0}
\verb{oleoduto}{}{}{}{}{s.m.}{Tubo, longo e largo, destinado a conduzir petróleo ou seus derivados líquidos dos poços aos depósitos ou às refinarias. }{o.le.o.du.to}{0}
\verb{oleosidade}{}{}{}{}{s.f.}{Qualidade do que é oleoso, gorduroso.}{o.le.o.si.da.de}{0}
\verb{oleoso}{ô}{}{"-osos ⟨ó⟩}{"-osa ⟨ó⟩}{adj.}{Que contém óleo; gorduroso.}{o.le.o.so}{0}
\verb{olfação}{}{}{"-ões}{}{s.f.}{Ação de cheirar; exercício do sentido do olfato.}{ol.fa.ção}{0}
\verb{olfativo}{}{}{}{}{adj.}{Relativo ao olfato.}{ol.fa.ti.vo}{0}
\verb{olfato}{}{}{}{}{s.m.}{Sentido com que se distinguem os odores.}{ol.fa.to}{0}
\verb{olfato}{}{}{}{}{}{Cheiro, faro.}{ol.fa.to}{0}
\verb{olhada}{}{}{}{}{s.f.}{Ato de olhar; espiada.}{o.lha.da}{0}
\verb{olhadela}{é}{}{}{}{s.f.}{Ato de olhar rapidamente; espiadela.}{o.lha.de.la}{0}
\verb{olhado}{}{}{}{}{adj.}{Visto ou encarado de forma especial; observado, considerado.}{o.lha.do}{0}
\verb{olhado}{}{}{}{}{s.m.}{Quebranto, mau"-olhado.}{o.lha.do}{0}
\verb{olhar}{}{}{}{}{v.t.}{Dirigir os olhos para pessoa ou coisa.  (\textit{Ele olhou para todos antes de sair.})}{o.lhar}{0}
\verb{olhar}{}{}{}{}{}{Estar com os olhos presos em pessoa ou coisa; contemplar, mirar. (\textit{As crianças não paravam de olhar para os brinquedos da loja.})}{o.lhar}{\verboinum{1}}
\verb{olhar}{}{}{}{}{}{Tomar conta de pessoa ou coisa; vigiar, cuidar. (\textit{Os avós olhavam as crianças quando os pais saíam.})}{o.lhar}{0}
\verb{olhar}{}{}{}{}{s.m.}{Movimento dos olhos. (\textit{Quando eles chegaram, todos desviaram o olhar para eles.})}{o.lhar}{0}
\verb{olhar}{}{}{}{}{}{Aspecto dos olhos. (\textit{Eles têm um olhar tristonho.})}{o.lhar}{0}
\verb{olheiras}{}{}{}{}{s.f.pl.}{Manchas escuras ao redor ou embaixo dos olhos.}{o.lhei.ras}{0}
\verb{olheiro}{ê}{}{}{}{s.m.}{Indivíduo que observa ou vigia certos trabalhos; vigia, informante, observador. }{o.lhei.ro}{0}
\verb{olheiro}{ê}{}{}{}{}{Olho"-d'água; nascente.}{o.lhei.ro}{0}
\verb{olho}{ô}{}{ ⟨ó⟩}{}{s.m.}{Cada um dos órgãos em forma de globo com os quais as pessoas e os animais enxergam; vista. (\textit{A moça tinha olhos verdes.})}{o.lho}{0}
\verb{olho}{ô}{Fig.}{ ⟨ó⟩}{}{}{Abertura arredondada; buraco, orifício. (\textit{Eles viam quem chegava pelo olho da fechadura.})}{o.lho}{0}
\verb{olho"-d’água}{ô}{}{olhos"-d'água ⟨ó⟩}{}{s.m.}{Nascente de água que brota do solo; olheiro.}{o.lho"-d’á.gua}{0}
\verb{olho"-de"-boi}{ô}{}{olhos"-de"-boi ⟨ó⟩}{}{s.m.}{Claraboia.}{o.lho"-de"-boi}{0}
\verb{olho"-de"-boi}{ô}{Pop.}{olhos"-de"-boi ⟨ó⟩}{}{}{Saliência exagerada do globo ocular; exoftalmia.}{o.lho"-de"-boi}{0}
\verb{olho"-de"-boi}{ô}{Bras.}{olhos"-de"-boi ⟨ó⟩}{}{}{Primeiro selo postal brasileiro, emitido em 1843, cujo desenho lembra um olho. }{o.lho"-de"-boi}{0}
\verb{olho"-de"-boi}{ô}{Bot.}{olhos"-de"-boi ⟨ó⟩}{}{}{Trepadeira lenhosa, nativa da Guiana e do Brasil, pilosa, de folhas trifolioladas, flores violáceas, vagens com três sementes grandes, é cultivada como ornamental e pelas sementes, usadas na confecção de colares e amuletos.}{o.lho"-de"-boi}{0}
\verb{olho"-de"-cabra}{ô}{}{olhos"-de"-cabra ⟨ó⟩}{}{s.m.}{Selo postal brasileiro, emitido em 1845, menor que o olho"-de"-boi.}{o.lho"-de"-ca.bra}{0}
\verb{olho"-de"-cabra}{ô}{Bot.}{olhos"-de"-cabra ⟨ó⟩}{}{}{Nome comum a plantas leguminosas com sementes vermelhas e pretas.}{o.lho"-de"-ca.bra}{0}
\verb{olho"-de"-gato}{ô}{}{olhos"-de"-gato ⟨ó⟩}{}{s.m.}{Sinalização luminosa instalada ao longo de estradas de rodagem, que reflete feixes de luz de faróis de automóveis. }{o.lho"-de"-ga.to}{0}
\verb{olho"-de"-gato}{ô}{Geol.}{olhos"-de"-gato ⟨ó⟩}{}{}{Quartzo com agulhas de amianto.}{o.lho"-de"-ga.to}{0}
\verb{olho"-de"-sogra}{ô\ldots{}ó}{Cul.}{olhos"-de"-sogra ⟨ó\ldots{}ó⟩}{}{s.m.}{Doce feito com uma ameixa coberta de calda caramelada e recheada com massa feita de ovos e coco.}{o.lho"-de"-so.gra}{0}
\verb{olhudo}{}{}{}{}{adj.}{Diz"-se do que tem olho grande ou saliente.}{o.lhu.do}{0}
\verb{oligarca}{}{}{}{}{s.2g.}{Membro ou partidário de uma oligarquia.}{o.li.gar.ca}{0}
\verb{oligarquia}{}{}{}{}{s.f.}{Governo de poucas pessoas de um partido, de uma classe ou de uma família.  }{o.li.gar.qui.a}{0}
\verb{oligárquico}{}{}{}{}{adj.}{Relativo ou pertencente a oligarquia.}{o.li.gár.qui.co}{0}
\verb{oligoceno}{}{Geol.}{}{}{s.m.}{Época geológica do sistema terciário, posterior ao Eoceno e anterior ao Mioceno, cujo marco inicial é datado, aproximadamente, de 36 milhões de anos atrás. (Nesta acepção com maiúscula.)}{o.li.go.ce.no}{0}
\verb{oligoceno}{}{Geol.}{}{}{adj.}{Relativo a essa época geológica da Terra.}{o.li.go.ce.no}{0}
\verb{oligoelemento}{}{Biol.}{}{}{s.m.}{Qualquer elemento químico que, em pequena quantidade, seja fundamental à vida.}{o.li.go.e.le.men.to}{0}
\verb{oligofrenia}{}{Med.}{}{}{s.f.}{Deficiência do desenvolvimento mental, congênita ou adquirida, que compromete a capacidade intelectual.}{o.li.go.fre.ni.a}{0}
\verb{oligofrênico}{}{}{}{}{adj.}{Relativo a oligofrenia.}{o.li.go.frê.ni.co}{0}
\verb{oligofrênico}{}{}{}{}{s.m.}{Indivíduo que sofre de oligofrenia.}{o.li.go.frê.ni.co}{0}
\verb{oligopólio}{}{Econ.}{}{}{s.m.}{Situação de mercado em que apenas poucas empresas detêm o monopólio da oferta, resultando geralmente na ausência de competição e consequente não variação do preço do serviço ou produto, o que pode lesar o consumidor.}{o.li.go.pó.lio}{0}
\verb{oligoqueta}{ê}{Zool.}{}{}{adj.}{Classe de anelídeos hermafroditas, de corpo segmentado, sem diferenciação da região cefálica, que habitam solo úmido ou água doce, e cujos representantes mais comuns são as minhocas.}{o.li.go.que.ta}{0}
\verb{oligoqueta}{ê}{Zool.}{}{}{adj.2g.}{Relativo aos oligoquetas.}{o.li.go.que.ta}{0}
\verb{olimpíada}{}{}{}{}{s.f.}{Período de quatro anos entre jogos olímpicos consecutivos.}{o.lim.pí.a.da}{0}
\verb{olimpíada}{}{}{}{}{}{No plural, os jogos olímpicos modernos, torneio esportivo internacional reunindo os melhores atletas de diversas modalidades, realizado a cada quatro anos.}{o.lim.pí.a.da}{0}
\verb{olímpico}{}{}{}{}{adj.}{Relativo ou pertencente a Olímpia, cidade da Grécia.}{o.lím.pi.co}{0}
\verb{olímpico}{}{}{}{}{}{Relativo aos jogos olímpicos.}{o.lím.pi.co}{0}
\verb{olímpico}{}{Fig.}{}{}{}{Sublime, grandioso, majestoso, divino.}{o.lím.pi.co}{0}
\verb{Olimpo}{}{Mit.}{}{}{s.m.}{Lugar onde habitavam as divindades greco"-latinas. }{O.lim.po}{0}
\verb{Olimpo}{}{Fig.}{}{}{}{Céu, paraíso.}{O.lim.po}{0}
\verb{Olimpo}{}{Por ext.}{}{}{}{Conjunto das divindades greco"-latinas.}{O.lim.po}{0}
\verb{oliva}{}{}{}{}{s.f.}{Fruto da oliveira; azeitona.}{o.li.va}{0}
\verb{oliva}{}{}{}{}{}{O pé dessa fruta.}{o.li.va}{0}
\verb{oliva}{}{Fig.}{}{}{}{Qualquer objeto em forma de azeitona.}{o.li.va}{0}
\verb{oliva}{}{}{}{}{adj.}{Que tem a cor verde da azeitona.}{o.li.va}{0}
\verb{oliváceo}{}{}{}{}{adj.}{Que tem a cor da azeitona.}{o.li.vá.ceo}{0}
\verb{olival}{}{}{"-ais}{}{s.m.}{Coletivo de olivas; oliveiral, olivedo.}{o.li.val}{0}
\verb{olivedo}{ê}{}{}{}{s.m.}{Coletivo de olivas; olival.}{o.li.ve.do}{0}
\verb{oliveira}{ê}{Bot.}{}{}{s.f.}{Árvore cujo fruto é a azeitona ou a oliva.}{o.li.vei.ra}{0}
\verb{oliveiral}{}{}{"-ais}{}{s.m.}{Coletivo de oliveiras; olival, alivedo. }{o.li.vei.ral}{0}
\verb{olivicultor}{ô}{}{}{}{s.m.}{Indivíduo que se dedica à olivicultura. }{o.li.vi.cul.tor}{0}
\verb{olivicultura}{}{}{}{}{s.f.}{Cultivo de oliveiras.}{o.li.vi.cul.tu.ra}{0}
\verb{olmo}{ô}{Bot.}{}{}{s.m.}{Nome comum às árvores da família das ulmáceas, distribuídas pelas regiões temperadas, cuja madeira é utilizada, com algumas espécies cultivadas pela sombra e como ornamentais.}{ol.mo}{0}
\verb{olor}{ô}{}{}{}{s.m.}{Odor delicado e agradável, aroma, fragrância, perfume.}{o.lor}{0}
\verb{oloroso}{ô}{}{"-osos ⟨ó⟩}{"-osa ⟨ó⟩}{adj.}{Que tem olor; odoroso, aromático, perfumado.}{o.lo.ro.so}{0}
\verb{olvidar}{}{}{}{}{v.t.}{Não se lembrar; esquecer.}{ol.vi.dar}{\verboinum{1}}
\verb{olvido}{}{}{}{}{s.m.}{Ato ou efeito de olvidar; esquecimento, olvidamento. }{ol.vi.do}{0}
\verb{omani}{}{}{}{}{adj.2g.}{Relativo a Omã.}{o.ma.ni}{0}
\verb{omani}{}{}{}{}{s.2g.}{Indivíduo natural ou habitante desse país.}{o.ma.ni}{0}
\verb{ombrear}{}{}{}{}{v.t.}{Ficar ombro a ombro com outra pessoa; igualar"-se.   }{om.bre.ar}{\verboinum{4}}
\verb{ombreira}{ê}{}{}{}{s.f.}{Acessório de vestuário correspondente aos ombros ou aplicado sobre eles.}{om.brei.ra}{0}
\verb{ombreira}{ê}{}{}{}{}{Cada uma das peças verticais dos vãos de portas ou janelas que sustentam as vergas; umbral.}{om.brei.ra}{0}
\verb{ombro}{}{Anat.}{}{}{s.m.}{Parte do corpo que liga o braço ao tórax; espádua.   }{om.bro}{0}
\verb{ombudsman}{}{}{}{}{s.2g.}{Em alguns regimes democráticos, funcionário do governo encarregado de defender os direitos dos cidadãos, recebendo e investigando queixas e denúncias contra os órgãos públicos.}{\textit{ombudsman}}{0}
\verb{ombudsman}{}{Por ext.}{}{}{}{Indivíduo encarregado de observar e apontar as falhas de uma empresa pública ou privada, estabelecendo o canal de comunicação entre a empresa e o consumidor ou público.}{\textit{ombudsman}}{0}
\verb{ômega}{}{}{}{}{s.m.}{Vigésima quarta e última letra do alfabeto grego.}{ô.me.ga}{0}
\verb{ômega}{}{Fig.}{}{}{}{Final; fim.}{ô.me.ga}{0}
\verb{omeleta}{ê}{Lus.}{}{}{s.f.}{Omelete.}{o.me.le.ta}{0}
\verb{omelete}{é}{Cul.}{}{}{s.2g.}{Fritada de ovos batidos, na qual se pode acrescentar inúmeros temperos ou ingredientes.}{o.me.le.te}{0}
\verb{ômicron}{}{}{}{}{s.m.}{Décima quinta letra do alfabeto grego.}{ô.mi.cron}{0}
\verb{ominoso}{ô}{}{"-osos ⟨ó⟩}{"-osa ⟨ó⟩}{adj.}{Que traz mau agouro; agourento, aziago, funesto, nefasto.}{o.mi.no.so}{0}
\verb{ominoso}{ô}{Por ext.}{"-osos ⟨ó⟩}{"-osa ⟨ó⟩}{}{Execrável, detestável, abominável.}{o.mi.no.so}{0}
\verb{omissão}{}{}{"-ões}{}{s.f.}{Ato ou efeito de omitir.}{o.mis.são}{0}
\verb{omissão}{}{}{"-ões}{}{}{Lacuna, falta.}{o.mis.são}{0}
\verb{omissão}{}{}{"-ões}{}{}{Negligência, descuido.}{o.mis.são}{0}
\verb{omissão}{}{}{"-ões}{}{}{Ato ou efeito de não cumprir o que moral ou juridicamente deve ser feito.}{o.mis.são}{0}
\verb{omisso}{}{}{}{}{adj.}{Que se omite.}{o.mis.so}{0}
\verb{omisso}{}{}{}{}{}{Negligente, descuidado, desleixado.}{o.mis.so}{0}
\verb{omitir}{}{}{}{}{v.t.}{Deixar de incluir alguma coisa; não mencionar; deixar de lado.}{o.mi.tir}{0}
\verb{omitir}{}{}{}{}{v.pron.}{Deixar de fazer aquilo que deveria ser feito; deixar de atuar ou de se manifestar.}{o.mi.tir}{\verboinum{18}}
\verb{omoplata}{}{Anat.}{}{}{s.f.}{Osso chato e triangular, que compõe a articulação do ombro, situado na parte posterior deste; escápula.}{o.mo.pla.ta}{0}
\verb{onagro}{}{Zool.}{}{}{s.m.}{Jumento selvagem, encontrado nos desertos asiáticos.  }{o.na.gro}{0}
\verb{onagro}{}{}{}{}{}{Antiga máquina de guerra, que arremessava projéteis.}{o.na.gro}{0}
\verb{onanismo}{}{}{}{}{s.m.}{Ato, masculino, de se masturbar solitariamente.    }{o.na.nis.mo}{0}
\verb{onanismo}{}{}{}{}{}{De acordo com a Bíblia, coito interrompido no momento da ejaculação para evitar a fecundação.}{o.na.nis.mo}{0}
\verb{onça}{}{}{}{}{s.f.}{Antiga unidade de medida de peso, equivalente a aproximadamente 28 g.}{on.ça}{0}
\verb{onça}{}{Zool.}{}{}{s.f.}{Nome comum aos felídeos brasileiros selvagens de grande porte.}{on.ça}{0}
\verb{onça}{}{}{}{}{}{Unidade de capacidade anglo"-saxã equivalente a 29,572 ml.}{on.ça}{0}
\verb{onça}{}{Zool.}{}{}{}{Onça"-pintada.}{on.ça}{0}
\verb{onça}{}{Fig.}{}{}{}{Pessoa valente, corajosa, intrépida.}{on.ça}{0}
\verb{onça}{}{Fig.}{}{}{}{Pessoa muito feia.}{on.ça}{0}
\verb{onça"-pintada}{}{Zool.}{onças"-pintadas}{}{s.f.}{Felino de grande porte, encontrado em quase toda a América, pesando até 150 kg e com até 1,8 m de comprimento, com manchas negras por todo o corpo, de coloração amarelada, e que se alimenta de caça e pesca.}{on.ça"-pin.ta.da}{0}
\verb{oncologia}{}{Med.}{}{}{s.f.}{Ramo da medicina que se dedica ao estudo e ao tratamento dos tumores.}{on.co.lo.gi.a}{0}
\verb{oncologista}{}{}{}{}{s.2g.}{Especialista em oncologia; cancerologista.}{on.co.lo.gis.ta}{0}
\verb{onda}{}{}{}{}{s.f.}{Qualquer massa de água (de mar, lago ou rio) que se eleva e abaixa, deslocando"-se geralmente pela ação do vento; vaga.}{on.da}{0}
\verb{onda}{}{}{}{}{}{Movimento semelhante ao das águas do mar.}{on.da}{0}
\verb{onda}{}{Fig.}{}{}{}{Grande quantidade de coisas.}{on.da}{0}
\verb{onde}{}{}{}{}{adv.}{No lugar em que; em que lugar.}{on.de}{0}
\verb{onde}{}{}{}{}{pron.}{Em que.  }{on.de}{0}
\verb{ondeado}{}{}{}{}{adj.}{Que tem ondas ou é disposto à feição delas; ondulado, ondado, ondeante.  }{on.de.a.do}{0}
\verb{ondear}{}{}{}{}{v.i.}{Mover"-se formando ondas.}{on.de.ar}{\verboinum{4}}
\verb{ondulação}{}{}{"-ões}{}{s.f.}{Movimento oscilatório das ondas ou semelhante ao delas.    }{on.du.la.ção}{0}
\verb{ondulação}{}{}{"-ões}{}{}{Série de saliências e de depressões numa superfície.   }{on.du.la.ção}{0}
\verb{ondulado}{}{}{}{}{adj.}{Cuja superfície apresenta ondulações; ondeado, ondeante.}{on.du.la.do}{0}
\verb{ondulante}{}{}{}{}{adj.2g.}{Que faz ou é disposto em ondas; ondeado, sinuoso, tortuoso.}{on.du.lan.te}{0}
\verb{ondular}{}{}{}{}{v.t.}{Ondear.}{on.du.lar}{\verboinum{1}}
\verb{ondulatório}{}{}{}{}{adj.}{Que se propaga por meio de ondas; ondeante. }{on.du.la.tó.rio}{0}
\verb{onduloso}{ô}{}{"-osos ⟨ó⟩}{"-osa ⟨ó⟩}{adj.}{Ondeado.}{on.du.lo.so}{0}
\verb{ONE}{}{}{}{}{}{Abrev. de \textit{oés"-nordeste}.}{O.N.E.}{0}
\verb{onerar}{}{}{}{}{v.t.}{Impor ônus, sujeitar a obrigação.}{o.ne.rar}{\verboinum{1}}
\verb{onerar}{}{}{}{}{v.t.}{Carregar, sobrecarregar.}{o.ne.rar}{0}
\verb{onerar}{}{}{}{}{}{Oprimir, vexar.}{o.ne.rar}{0}
\verb{onerar}{}{}{}{}{}{Impor tributos pesados.}{o.ne.rar}{0}
\verb{oneroso}{ô}{}{"-osos ⟨ó⟩}{"-osa ⟨ó⟩}{adj.}{Que impõe ônus, encargo; que ocasiona gastos.}{o.ne.ro.so}{0}
\verb{oneroso}{ô}{}{"-osos ⟨ó⟩}{"-osa ⟨ó⟩}{}{Incômodo, opressor, vexatório.}{o.ne.ro.so}{0}
\verb{ONG}{}{}{}{}{s.f.}{Sigla de Organização Não Governamental, designação genérica para qualquer entidade jurídica, de caráter privado, sem fins lucrativos, cujo trabalho esteja vinculado ao interesse público e à autonomia dos grupos sociais em questões onde o Estado é omisso ou ineficiente.}{ONG}{0}
\verb{ônibus}{}{}{}{}{s.m.}{Veículo grande que transporta passageiros dentro de uma mesma cidade, ou entre cidades.}{ô.ni.bus}{0}
\verb{onicofagia}{}{Med.}{}{}{s.f.}{Vício ou hábito de roer as unhas.}{o.ni.co.fa.gi.a}{0}
\verb{oniforme}{ó}{}{}{}{adj.}{Que tem ou pode assumir todas as formas.}{o.ni.for.me}{0}
\verb{onipotência}{}{}{}{}{s.f.}{Qualidade ou caráter de onipotente, todo"-poderoso.}{o.ni.po.tên.cia}{0}
\verb{onipotência}{}{}{}{}{}{Atributo de Deus.}{o.ni.po.tên.cia}{0}
\verb{onipotência}{}{}{}{}{}{Autoridade, soberania absoluta.}{o.ni.po.tên.cia}{0}
\verb{onipotente}{}{}{}{}{adj.2g.}{Que tudo pode, que tem poder absoluto e ilimitado; todo"-poderoso. }{o.ni.po.ten.te}{0}
\verb{onipotente}{}{}{}{}{s.m.}{Grafado com maiúscula, Deus; Altíssimo, Todo"-Poderoso.}{o.ni.po.ten.te}{0}
\verb{onipresença}{}{}{}{}{s.f.}{Qualidade ou condição de onipresente; ubiquidade.}{o.ni.pre.sen.ça}{0}
\verb{onipresente}{}{}{}{}{adj.2g.}{Diz"-se do que está presente em todos os lugares ao mesmo tempo; ubíquo.}{o.ni.pre.sen.te}{0}
\verb{onírico}{}{}{}{}{adj.}{Relativo ou próprio dos sonhos ou da sua natureza.}{o.ní.ri.co}{0}
\verb{onisciência}{}{}{}{}{s.f.}{Qualidade de onisciente.}{o.nis.ci.ên.cia}{0}
\verb{onisciência}{}{}{}{}{}{Saber absoluto, conhecimento sobre todas as coisas.}{o.nis.ci.ên.cia}{0}
\verb{onisciência}{}{}{}{}{}{Nas religiões monoteístas, um dos atributos de Deus, o saber ilimitado.}{o.nis.ci.ên.cia}{0}
\verb{onisciente}{}{}{}{}{adj.2g.}{Que tem saber ilimitado, absoluto, pleno; que tem ciência de todas as coisas.}{o.nis.ci.en.te}{0}
\verb{onívoro}{}{Ecol.}{}{}{adj.}{Diz"-se do animal que se alimenta tanto de matéria vegetal como animal.}{o.ní.vo.ro}{0}
\verb{onívoro}{}{}{}{}{}{Que come tudo ou de tudo.}{o.ní.vo.ro}{0}
\verb{ônix}{cs}{}{}{}{s.m.}{Variedade de ágata cujas camadas apresentam várias cores.}{ô.nix}{0}
\verb{ônix}{cs}{}{}{}{}{Tipo de mármore listrado, de várias cores.}{ô.nix}{0}
\verb{on"-line}{}{}{}{}{adv.}{Em conexão com ou entre sistemas de processamento ou transmissão de informação.}{\textit{on"-line}}{0}
\verb{on"-line}{}{Informát.}{}{}{adj.}{Diz"-se de equipamento que se encontra conectado a um sistema computacional ou acessível para utilização por este.}{\textit{on"-line}}{0}
\verb{ONO}{}{}{}{}{}{Abrev. de \textit{oés"-noroeste}.}{O.N.O.}{0}
\verb{onomástica}{}{}{}{}{s.f.}{Relação de nomes próprios.}{o.no.más.ti.ca}{0}
\verb{onomástica}{}{}{}{}{}{Estudo que explica a origem dos nomes próprios.}{o.no.más.ti.ca}{0}
\verb{onomástico}{}{}{}{}{adj.}{Relativo aos nomes próprios ou ao seu estudo.}{o.no.más.ti.co}{0}
\verb{onomástico}{}{}{}{}{}{Que contém nomes próprios.}{o.no.más.ti.co}{0}
\verb{onomástico}{}{}{}{}{s.m.}{Onomástica.}{o.no.más.ti.co}{0}
\verb{onomatopaico}{}{}{}{}{adj.}{Em que há onomatopeia.}{o.no.ma.to.pai.co}{0}
\verb{onomatopeia}{é}{}{}{}{s.f.}{Palavra cuja pronúncia imita o som natural daquilo que ela significa.}{o.no.ma.to.pei.a}{0}
\verb{onomatopeico}{é}{}{}{}{adj.}{Onomatopaico.}{o.no.ma.to.pei.co}{0}
\verb{ontem}{}{}{}{}{adv.}{No dia imediatamente anterior ao de hoje.}{on.tem}{0}
\verb{ontem}{}{Por ext.}{}{}{}{No tempo que passou.}{on.tem}{0}
\verb{ontogênese}{}{Biol.}{}{}{s.f.}{Série de transformações por que passa o indivíduo, desde a fecundação do ovo até ao completo desenvolvimento; ontogenia.  }{on.to.gê.ne.se}{0}
\verb{ontogenia}{}{Biol.}{}{}{s.f.}{Ontogênese.  }{on.to.ge.ni.a}{0}
\verb{ontologia}{}{Filos.}{}{}{s.f.}{Parte da filosofia que trata do ser como ser, isto é, independente de suas determinações particulares.}{on.to.lo.gi.a}{0}
\verb{ontológico}{}{}{}{}{adj.}{Relativo à ontologia.}{on.to.ló.gi.co}{0}
\verb{ônus}{}{}{}{}{s.m.}{Aquilo que sobrecarrega; carga, peso.}{ô.nus}{0}
\verb{ônus}{}{}{}{}{}{Encargo ou obrigação pesada, de cumprimento difícil ou desagradável.}{ô.nus}{0}
\verb{ônus}{}{Fig.}{}{}{}{Encargo, obrigação; dever.}{ô.nus}{0}
\verb{onze}{}{}{}{}{num.}{Nome dado à quantidade expressa pelo número 11.}{on.ze}{0}
\verb{onze}{}{}{}{}{}{Décimo primeiro.}{on.ze}{0}
\verb{onze}{}{}{}{}{s.m.}{Algarismo representativo do número onze.}{on.ze}{0}
\verb{onze}{}{}{}{}{}{O time de futebol, a equipe.}{on.ze}{0}
\verb{onze"-horas}{ó}{}{}{}{s.f.}{Erva da família nativa do Brasil, de caule e folhas suculentos, com flores violáceas que desabrocham no fim da manhã, cultivada como ornamental.}{on.ze"-ho.ras}{0}
\verb{onzena}{}{}{}{}{s.f.}{Juro excessivo.}{on.ze.na}{0}
\verb{onzena}{}{}{}{}{}{Porção de onze objetos. }{on.ze.na}{0}
\verb{onzenário}{}{}{}{}{adj.}{Relativo à onzena.}{on.ze.ná.rio}{0}
\verb{onzenário}{}{}{}{}{s.m.}{Usuário, agiota.}{on.ze.ná.rio}{0}
\verb{oosfera}{é}{}{}{}{s.f.}{Célula sexual feminina, esférica, com reservas nutritivas, que ocorre nos vegetais.}{o.os.fe.ra}{0}
\verb{opa}{ô}{}{}{}{interj.}{Expressão que denota espanto, surpresa, admiração.}{o.pa}{0}
\verb{opacidade}{}{}{}{}{s.f.}{Qualidade do que é opaco; ausência de transparência.}{o.pa.ci.da.de}{0}
\verb{opacidade}{}{Fig.}{}{}{}{Lugar sombrio; sombra densa.}{o.pa.ci.da.de}{0}
\verb{opaco}{}{}{}{}{adj.}{Que não deixa atravessar a luz; que não é transparente; turvo.}{o.pa.co}{0}
\verb{opaco}{}{Por ext.}{}{}{}{Espesso, denso.}{o.pa.co}{0}
\verb{opaco}{}{Fig.}{}{}{}{Incompreensível, obscuro.}{o.pa.co}{0}
\verb{opala}{}{}{}{}{s.f.}{Pedra dura, transparente ou opaca, apresentando variedade de cores quando exposta à luz.}{o.pa.la}{0}
\verb{opala}{}{Bras.}{}{}{}{Espécie de tecido de algodão.}{o.pa.la}{0}
\verb{opalescência}{}{}{}{}{s.f.}{Reflexo opalino.}{o.pa.les.cên.cia}{0}
\verb{opalescência}{}{}{}{}{}{Tinta ou cor opalina.}{o.pa.les.cên.cia}{0}
\verb{opalescente}{}{}{}{}{adj.2g.}{Em que há opalescência.}{o.pa.les.cen.te}{0}
\verb{opalescente}{}{}{}{}{}{Opalino.}{o.pa.les.cen.te}{0}
\verb{opalina}{}{}{}{}{s.f.}{Vidro fosco, mas translúcido, que se emprega na fabricação de objetos decorativos. }{o.pa.li.na}{0}
\verb{opalina}{}{Por ext.}{}{}{}{Objeto confeccionado com esse vidro.}{o.pa.li.na}{0}
\verb{opalino}{}{}{}{}{adj.}{Que apresenta a tonalidade leitosa, azulada, como a do mineral opala.}{o.pa.li.no}{0}
\verb{opalino}{}{}{}{}{}{Que apresenta reflexos irisados como os da opala.}{o.pa.li.no}{0}
\verb{opção}{}{}{"-ões}{}{s.f.}{Ato, faculdade de optar; livre escolha; preferência.}{op.ção}{0}
\verb{opção}{}{Por ext.}{"-ões}{}{}{Aquilo por que se opta; alternativa.}{op.ção}{0}
\verb{opcional}{}{}{"-ais}{}{adj.2g.}{Que pode ser objeto de opção; por que se pode optar.}{op.ci.o.nal}{0}
\verb{open market}{}{}{}{}{s.m.}{Operação de compra e venda de títulos do governo pelo Banco Central para controlar o volume de moeda em circulação.}{\textit{open market}}{0}
\verb{ópera}{}{}{}{}{s.f.}{Obra dramática musicada, geralmente desprovida de partes faladas, composta de recitativos, coro, às vezes de balé, e acompanhada de orquestra.}{ó.pe.ra}{0}
\verb{ópera}{}{}{}{}{}{Teatro onde se representam óperas.}{ó.pe.ra}{0}
\verb{ópera"-bufa}{}{}{}{}{s.f.}{Ópera de origem italiana, de assunto jocoso, com personagens burlescos, ligeira e espirituosa.}{ó.pe.ra"-bu.fa}{0}
\verb{operação}{}{}{"-ões}{}{s.f.}{Ato ou efeito de operar; ação de um poder ou faculdade de que resulta certo efeito.}{o.pe.ra.ção}{0}
\verb{operação}{}{}{"-ões}{}{}{Trabalho de cirurgia.}{o.pe.ra.ção}{0}
\verb{operação}{}{}{"-ões}{}{}{Cálculo matemático.}{o.pe.ra.ção}{0}
\verb{operação}{}{}{"-ões}{}{}{Manobra ou combate militar.}{o.pe.ra.ção}{0}
\verb{operação}{}{}{"-ões}{}{}{Transação comercial.}{o.pe.ra.ção}{0}
\verb{operacional}{}{}{"-ais}{}{adj.2g.}{Relativo a operação.}{o.pe.ra.ci.o.nal}{0}
\verb{operacional}{}{}{"-ais}{}{}{Pronto a entrar em atividade, em funcionamento, a realizar uma operação.}{o.pe.ra.ci.o.nal}{0}
\verb{ópera"-cômica}{}{}{}{}{s.f.}{Ópera de caráter cômico, e na qual os episódios cantados alternam com as partes faladas.}{ó.pe.ra"-cô.mi.ca}{0}
\verb{operado}{}{}{}{}{adj.}{Que se executou, se realizou; feito, produzido.}{o.pe.ra.do}{0}
\verb{operado}{}{}{}{}{}{Que foi submetido a intervenção cirúrgica.}{o.pe.ra.do}{0}
\verb{operado}{}{}{}{}{s.m.}{Indivíduo que se submeteu a uma operação cirúrgica.}{o.pe.ra.do}{0}
\verb{operador}{ô}{}{}{}{adj.}{Que opera, que executa ou indica uma operação.}{o.pe.ra.dor}{0}
\verb{operador}{ô}{}{}{}{s.m.}{Indivíduo ou coisa que opera, realiza algo, executa uma ação.}{o.pe.ra.dor}{0}
\verb{operador}{ô}{}{}{}{}{Profissional que se dedica à prática de cirurgias.}{o.pe.ra.dor}{0}
\verb{operadora}{ô}{}{}{}{s.f.}{Qualquer empresa que explora determinadas áreas de prestação de serviços.}{o.pe.ra.do.ra}{0}
\verb{operante}{}{}{}{}{adj.2g.}{Que opera, realiza, produz.}{o.pe.ran.te}{0}
\verb{operar}{}{}{}{}{v.t.}{Exercer ação, função, atividade ou ofício; executar, obrar.}{o.pe.rar}{0}
\verb{operar}{}{}{}{}{}{Provocar uma reação, produzir, surtir um efeito.}{o.pe.rar}{0}
\verb{operar}{}{}{}{}{}{Submeter a uma operação cirúrgica.}{o.pe.rar}{0}
\verb{operar}{}{}{}{}{}{Fazer funcionar; manobrar, acionar.}{o.pe.rar}{0}
\verb{operar}{}{}{}{}{v.pron.}{Suceder, ocorrer, realizar"-se.}{o.pe.rar}{\verboinum{1}}
\verb{operária}{}{}{}{}{s.f.}{Mulher que trabalha mediante salário, especialmente a que exerce trabalhos manuais ou mecânicos numa indústria.}{o.pe.rá.ria}{0}
\verb{operária}{}{Zool.}{}{}{}{Abelha estéril, com o abdômen terminado em ponta dotada de ferrão, responsável pela maioria das atividades da colônia.}{o.pe.rá.ria}{0}
\verb{operariado}{}{}{}{}{s.m.}{A classe operária; o segmento social constituído pelos operários.}{o.pe.ra.ri.a.do}{0}
\verb{operário}{}{}{}{}{s.m.}{Trabalhador que, mediante salário, exerce uma ocupação manual.}{o.pe.rá.rio}{0}
\verb{operário}{}{}{}{}{}{Trabalhador manual ou mecânico nas grandes indústrias.}{o.pe.rá.rio}{0}
\verb{operário}{}{Fig.}{}{}{}{Indivíduo que colabora na realização de uma ideia, plano, campanha ou apostolado.}{o.pe.rá.rio}{0}
\verb{operário}{}{Fig.}{}{}{}{Autor, artífice.}{o.pe.rá.rio}{0}
\verb{operário}{}{}{}{}{adj.}{Relativo ou constituído por operários.}{o.pe.rá.rio}{0}
\verb{operatório}{}{}{}{}{adj.}{Relativo a operações cirúrgicas.}{o.pe.ra.tó.rio}{0}
\verb{operatório}{}{}{}{}{}{Operante.}{o.pe.ra.tó.rio}{0}
\verb{operável}{}{}{"-eis}{}{adj.2g.}{Que pode ser operado.}{o.pe.rá.vel}{0}
\verb{opérculo}{}{}{}{}{s.m.}{Coisa que fecha.    }{o.pér.culo}{0}
\verb{opérculo}{}{Zool.}{}{}{}{Substância córnea ou calcárea que tapa a abertura das conchas univalves. }{o.pér.culo}{0}
\verb{opérculo}{}{}{}{}{}{Tampa de turíbulo.    }{o.pér.culo}{0}
\verb{opereta}{ê}{}{}{}{s.f.}{Gênero leve de teatro musicado, sobre assunto cômico e sentimental, e no qual as estrofes, cantadas, alternam com as partes faladas.}{o.pe.re.ta}{0}
\verb{opereta}{ê}{}{}{}{}{Pequena peça desse gênero.}{o.pe.re.ta}{0}
\verb{operosidade}{}{}{}{}{s.f.}{Qualidade ou condição do que é operoso.}{o.pe.ro.si.da.de}{0}
\verb{operoso}{ô}{}{"-osos ⟨ó⟩}{"-osa ⟨ó⟩}{adj.}{Que opera, que produz ou causa efeito; produtivo.}{o.pe.ro.so}{0}
\verb{operoso}{ô}{}{"-osos ⟨ó⟩}{"-osa ⟨ó⟩}{}{Que é trabalhoso, laborioso, difícil.}{o.pe.ro.so}{0}
\verb{opiáceo}{}{}{}{}{adj.}{Relativo ao ópio.}{o.pi.á.ceo}{0}
\verb{opiáceo}{}{}{}{}{}{Que contém ou é preparado com ópio.}{o.pi.á.ceo}{0}
\verb{opilação}{}{}{"-ões}{}{s.f.}{Ato ou efeito de opilar; obstrução, entupimento.}{o.pi.la.ção}{0}
\verb{opilação}{}{}{"-ões}{}{}{Bloqueio de uma abertura ou de um ducto natural; entupimento.}{o.pi.la.ção}{0}
\verb{opilação}{}{Pop.}{"-ões}{}{}{Doença causada pela ancilostomíase.}{o.pi.la.ção}{0}
\verb{opilado}{}{Fig.}{}{}{}{Cerrado, fechado, inchado.}{o.pi.la.do}{0}
\verb{opilado}{}{}{}{}{}{Diz"-se de doente que sofre de ancilostomíase.}{o.pi.la.do}{0}
\verb{opilado}{}{}{}{}{s.m.}{Indivíduo que sofre de ancilostomíase.}{o.pi.la.do}{0}
\verb{opilado}{}{}{}{}{adj.}{Que se opilou; obstruído, bloqueado.}{o.pi.la.do}{0}
\verb{opilar}{}{}{}{}{}{Tornar opilado, inchado, intumescido.}{o.pi.lar}{0}
\verb{opilar}{}{}{}{}{v.t.}{Causar opilação; obstruir, entupir.}{o.pi.lar}{0}
\verb{opilar}{}{}{}{}{v.pron.}{Sofrer de opilação, de ancilostomíase.}{o.pi.lar}{\verboinum{1}}
\verb{opimo}{}{}{}{}{adj.}{Que é excelente, de grande valor, abundante, fértil, rico.}{o.pi.mo}{0}
\verb{opinante}{}{}{}{}{adj.2g.}{Que opina, que emite uma opinião.}{o.pi.nan.te}{0}
\verb{opinar}{}{}{}{}{v.t.}{Emitir opinião, dar parecer.}{o.pi.nar}{0}
\verb{opinar}{}{}{}{}{}{Considerar após reflexão; entender, julgar.}{o.pi.nar}{0}
\verb{opinar}{}{}{}{}{}{Votar favoravelmente; estar de acordo.}{o.pi.nar}{\verboinum{1}}
\verb{opinativo}{}{}{}{}{adj.}{Que depende de opinião; baseado em opinião particular.}{o.pi.na.ti.vo}{0}
\verb{opinativo}{}{}{}{}{}{Que é discutível, duvidoso, incerto.}{o.pi.na.ti.vo}{0}
\verb{opinião}{}{}{"-ões}{}{s.f.}{Modo de ver, de pensar, de deliberar.}{o.pi.ni.ão}{0}
\verb{opinião}{}{}{"-ões}{}{}{Julgamento pessoal que se tem sobre determinado assunto.}{o.pi.ni.ão}{0}
\verb{opinião}{}{}{"-ões}{}{}{Juízo, reputação.}{o.pi.ni.ão}{0}
\verb{opinião}{}{}{"-ões}{}{}{Ideia, doutrina, princípio.}{o.pi.ni.ão}{0}
\verb{opinião}{}{}{"-ões}{}{}{Ideia sem fundamento; presunção.}{o.pi.ni.ão}{0}
\verb{opinião}{}{}{"-ões}{}{}{Capricho, birra, teimosia.}{o.pi.ni.ão}{0}
\verb{opiniático}{}{}{}{}{adj.}{Opinioso.}{o.pi.ni.á.ti.co}{0}
\verb{opinioso}{ô}{}{"-osos ⟨ó⟩}{"-osa ⟨ó⟩}{adj.}{Que defende ou se apega obstinadamente a suas opiniões; obstinado, teimoso.}{o.pi.ni.o.so}{0}
\verb{opinioso}{ô}{}{"-osos ⟨ó⟩}{"-osa ⟨ó⟩}{}{Que demonstra convencimento, vaidade, orgulho; presunçoso.}{o.pi.ni.o.so}{0}
\verb{ópio}{}{}{}{}{s.m.}{Substância que se extrai dos frutos imaturos de várias espécies de papoulas, de ação analgésica, narcótica e hipnótica, usada também na produção de morfina e heroína.}{ó.pio}{0}
\verb{ópio}{}{Fig.}{}{}{}{Aquilo que produz adormecimento, embrutecimento, entorpecimento.}{ó.pio}{0}
\verb{opíparo}{}{}{}{}{adj.}{Que é esplêndido, pomposo, suntuoso.}{o.pí.pa.ro}{0}
\verb{oponente}{}{}{}{}{adj.2g.}{Que se opõe; contrário, oposto.}{o.po.nen.te}{0}
\verb{opor}{}{}{}{}{v.t.}{Pôr defronte, colocar de maneira que forme obstáculo.}{o.por}{0}
\verb{opor}{}{}{}{}{}{Pôr de maneira que forme contraste; pôr em paralelo.}{o.por}{0}
\verb{opor}{}{}{}{}{}{Objetar, contrapor.}{o.por}{0}
\verb{opor}{}{}{}{}{v.pron.}{Dar combate; resistir.}{o.por}{0}
\verb{opor}{}{}{}{}{}{Recusar"-se, negar"-se.}{o.por}{\verboinum{60}}
\verb{oportunidade}{}{}{}{}{s.f.}{Qualidade, caráter do que é oportuno.}{o.por.tu.ni.da.de}{0}
\verb{oportunidade}{}{}{}{}{}{Circunstância favorável para realização de algo; ensejo.}{o.por.tu.ni.da.de}{0}
\verb{oportunidade}{}{}{}{}{}{Conveniência, interesse, utilidade.}{o.por.tu.ni.da.de}{0}
\verb{oportunismo}{}{}{}{}{s.m.}{Acomodação e aproveitamento das circunstâncias para se chegar mais facilmente a algum resultado.}{o.por.tu.nis.mo}{0}
\verb{oportunista}{}{}{}{}{adj.2g.}{Que aproveita as oportunidades.}{o.por.tu.nis.ta}{0}
\verb{oportunista}{}{}{}{}{s.2g.}{Indivíduo que aproveita as oportunidades.}{o.por.tu.nis.ta}{0}
\verb{oportunista}{}{Pop.}{}{}{}{Indivíduo que se aproveita dos outros, ou que tira, sempre que possível, vantagens pessoais de situações.}{o.por.tu.nis.ta}{0}
\verb{oportuno}{}{}{}{}{adj.}{Que vem a tempo, a propósito ou quando convém; apropriado.}{o.por.tu.no}{0}
\verb{oportuno}{}{}{}{}{}{Que é cômodo, favorável.}{o.por.tu.no}{0}
\verb{oposição}{}{}{"-ões}{}{s.f.}{Dificuldade, impedimento, obstáculo que se opõe à realização de alguma coisa.}{o.po.si.ção}{0}
\verb{oposição}{}{}{"-ões}{}{}{Partido político contrário ao governo.}{o.po.si.ção}{0}
\verb{oposição}{}{}{"-ões}{}{}{Vontade contrária.}{o.po.si.ção}{0}
\verb{oposição}{}{}{"-ões}{}{}{Contestação, réplica, objeção.}{o.po.si.ção}{0}
\verb{oposição}{}{}{"-ões}{}{}{Grau marcante de diferença entre coisas da mesma natureza, passíveis de comparação; contraste.}{o.po.si.ção}{0}
\verb{oposicionismo}{}{}{}{}{s.m.}{Sistema de opor"-se a tudo. }{o.po.si.ci.o.nis.mo}{0}
\verb{oposicionismo}{}{}{}{}{}{Facção política que combate o governo; oposição.}{o.po.si.ci.o.nis.mo}{0}
\verb{oposicionista}{}{}{}{}{adj.2g.}{Que se opõe a qualquer coisa.}{o.po.si.ci.o.nis.ta}{0}
\verb{oposicionista}{}{}{}{}{}{Que combate o governo.}{o.po.si.ci.o.nis.ta}{0}
\verb{oposicionista}{}{}{}{}{s.2g.}{Indivíduo que faz oposição.}{o.po.si.ci.o.nis.ta}{0}
\verb{oposicionista}{}{}{}{}{}{Indivíduo que é contrário ao governo.}{o.po.si.ci.o.nis.ta}{0}
\verb{opositivo}{}{}{}{}{adj.}{Que envolve oposição; oposto.}{o.po.si.ti.vo}{0}
\verb{opositor}{ô}{}{}{}{adj.}{Que se opõe; contrário, adversário.}{o.po.si.tor}{0}
\verb{opositor}{ô}{}{}{}{s.m.}{Concorrente, candidato.}{o.po.si.tor}{0}
\verb{oposto}{ô}{}{"-s ⟨ó⟩}{"-a ⟨ó⟩}{adj.}{Que fica em frente.}{o.pos.to}{0}
\verb{oposto}{ô}{}{"-s ⟨ó⟩}{"-a ⟨ó⟩}{}{Inverso, contrário.}{o.pos.to}{0}
\verb{oposto}{ô}{}{"-s ⟨ó⟩}{"-a ⟨ó⟩}{s.m.}{Aquilo que é contrário.}{o.pos.to}{0}
\verb{opressão}{}{}{"-ões}{}{s.f.}{Ato ou efeito de oprimir.}{o.pres.são}{0}
\verb{opressão}{}{}{"-ões}{}{}{Tirania.}{o.pres.são}{0}
\verb{opressão}{}{}{"-ões}{}{}{Dificuldade em respirar; sufocação.}{o.pres.são}{0}
\verb{opressivo}{}{}{}{}{adj.}{Que oprime ou serve para oprimir.}{o.pres.si.vo}{0}
\verb{opresso}{é}{}{}{}{adj.}{Oprimido.}{o.pres.so}{0}
\verb{opressor}{ô}{}{}{}{adj.}{Que oprime ou serve para oprimir.}{o.pres.sor}{0}
\verb{opressor}{ô}{}{}{}{s.m.}{Indivíduo que oprime; tirano.}{o.pres.sor}{0}
\verb{oprimente}{}{}{}{}{adj.2g.}{Opressor.}{o.pri.men.te}{0}
\verb{oprimido}{}{}{}{}{adj.}{Que sofre opressão; humilhado.}{o.pri.mi.do}{0}
\verb{oprimir}{}{}{}{}{v.t.}{Apertar, esmagar.}{o.pri.mir}{0}
\verb{oprimir}{}{}{}{}{}{Exercer dominação de forma cruel; tiranizar.}{o.pri.mir}{0}
\verb{oprimir}{}{}{}{}{}{Humilhar, vexar.}{o.pri.mir}{0}
\verb{oprimir}{}{}{}{}{}{Afligir, angustiar.}{o.pri.mir}{0}
\verb{oprimir}{}{}{}{}{}{Sufocar, abafar.}{o.pri.mir}{\verboinum{18}}
\verb{opróbrio}{}{}{}{}{s.m.}{Humilhação em grau extremo.}{o.pró.brio}{0}
\verb{optar}{}{}{}{}{v.t.}{Escolher entre duas ou mais alternativas.}{op.tar}{\verboinum{1}}
\verb{optativo}{}{}{}{}{adj.}{Que envolve escolha, opção.}{op.ta.ti.vo}{0}
\verb{optativo}{}{}{}{}{}{Que exprime vontade.}{op.ta.ti.vo}{0}
\verb{óptica}{}{Fís.}{}{}{s.f.}{Ramo da física que estuda fenômenos relacionados com a luz e com a visão.}{óp.ti.ca}{0}
\verb{óptica}{}{Fig.}{}{}{}{Modo particular de compreender ou considerar algo.}{óp.ti.ca}{0}
\verb{óptica}{}{}{}{}{}{Estabelecimento especializado em óculos, lunetas e instrumentos ópticos.}{óp.ti.ca}{0}
\verb{óptico}{}{}{}{}{adj.}{Relativo à óptica.}{óp.ti.co}{0}
\verb{óptico}{}{}{}{}{}{Relativo aos olhos ou à visão.}{óp.ti.co}{0}
\verb{optometria}{}{}{}{}{s.f.}{Exame da visão para detectar problemas e eventualmente prescrever lentes corretivas.}{op.to.me.tri.a}{0}
\verb{opugnar}{}{}{}{}{v.t.}{Atacar, investir, acometer.}{o.pug.nar}{0}
\verb{opugnar}{}{Fig.}{}{}{}{Rejeitar, combater ideias, doutrinas.}{o.pug.nar}{\verboinum{1}}
\verb{opulência}{}{}{}{}{s.f.}{Riqueza, fartura, abundância, luxo.}{o.pu.lên.cia}{0}
\verb{opulentar}{}{}{}{}{v.t.}{Tornar opulento, grandioso, abundante.}{o.pu.len.tar}{\verboinum{1}}
\verb{opulento}{}{}{}{}{adj.}{Rico, suntuoso, luxuoso, abundante.}{o.pu.len.to}{0}
\verb{opus}{}{Mús.}{}{}{s.m.}{Obra musical incluída em índice catalográfico com todas as obras do autor.}{\textit{opus}}{0}
\verb{opúsculo}{}{}{}{}{s.m.}{Pequeno livro ou folheto de conteúdo artístico relevante.}{o.pús.cu.lo}{0}
\verb{ora}{ó}{}{}{}{adv.}{Neste momento; agora. (\textit{Antes não queriam, ora querem.})}{o.ra}{0}
\verb{ora}{ó}{}{}{}{conj.}{Conjunção coordenativa aditiva, indicando sequência lógica do discurso. (\textit{O cão é fiel ao dono, ora, seu cão será fiel a você também.})}{o.ra}{0}
\verb{ora}{ó}{}{}{}{}{Conjunção coordenativa alternativa, repetida no início das duas frases coordenadas. (\textit{Eles ora vêm, ora vão.})}{o.ra}{0}
\verb{ora}{ó}{Bras.}{}{}{interj.}{Exprime impaciência, espanto, menosprezo. (\textit{Ora, que me interessa isso?})}{o.ra}{0}
\verb{oração}{}{}{"-ões}{}{s.f.}{Enunciação dirigida a uma divindade; prece, reza.}{o.ra.ção}{0}
\verb{oração}{}{Gram.}{"-ões}{}{}{Frase que contém um verbo.}{o.ra.ção}{0}
\verb{oracional}{}{Gram.}{"-ais}{}{adj.2g.}{Relativo a oração.}{o.ra.ci.o.nal}{0}
\verb{oracular}{}{}{}{}{adj.2g.}{Relativo a oráculo.}{o.ra.cu.lar}{0}
\verb{oráculo}{}{}{}{}{s.m.}{Na Antiguidade, resposta de uma divindade a quem a consultava.}{o.rá.cu.lo}{0}
\verb{oráculo}{}{}{}{}{}{A divindade consultada ou o sacerdote encarregado da consulta à divindade e transmissão de suas respostas.}{o.rá.cu.lo}{0}
\verb{oráculo}{}{}{}{}{}{O local, o santuário onde se realiza essa consulta.}{o.rá.cu.lo}{0}
\verb{orador}{ô}{}{}{}{s.m.}{Indivíduo que fala em público, geralmente com eloquência.}{o.ra.dor}{0}
\verb{orago}{}{}{}{}{s.m.}{Santo padroeiro.}{o.ra.go}{0}
\verb{oral}{}{}{"-ais}{}{adj.2g.}{Relativo à boca.}{o.ral}{0}
\verb{oral}{}{Gram.}{"-ais}{}{}{Expresso em viva voz (em oposição a escrito).}{o.ral}{0}
\verb{oralidade}{}{}{}{}{s.f.}{Qualidade de oral.}{o.ra.li.da.de}{0}
\verb{orangotango}{}{Zool.}{}{}{s.m.}{Macaco de grande porte, braços compridos, pelos longos e que se alimenta de frutos.}{o.ran.go.tan.go}{0}
\verb{orar}{}{}{}{}{v.i.}{Proferir oração; rezar.}{o.rar}{0}
\verb{orar}{}{}{}{}{}{Falar em público; discursar.}{o.rar}{\verboinum{1}}
\verb{oratória}{}{}{}{}{s.f.}{Arte de falar bem em público.}{o.ra.tó.ria}{0}
\verb{oratório}{}{}{}{}{adj.}{Relativo a oratória ou a orador.}{o.ra.tó.rio}{0}
\verb{oratório}{}{}{}{}{s.m.}{Cômodo da casa reservado à oração.}{o.ra.tó.rio}{0}
\verb{orbe}{ó}{Astron.}{}{}{s.f.}{Corpo celeste esférico.}{or.be}{0}
\verb{orbe}{ó}{Por ext.}{}{}{}{A Terra, o mundo, o Universo.}{or.be}{0}
\verb{orbicular}{}{}{}{}{adj.2g.}{Que tem forma de globo; circular, esférico.}{or.bi.cu.lar}{0}
\verb{órbita}{}{Astron.}{}{}{s.f.}{Trajetória descrita por um corpo celeste.}{ór.bi.ta}{0}
\verb{órbita}{}{Anat.}{}{}{}{Cavidade onde ficam alojados os olhos.}{ór.bi.ta}{0}
\verb{orbital}{}{}{"-ais}{}{adj.2g.}{Relativo a órbita.}{or.bi.tal}{0}
\verb{orbitário}{}{}{}{}{adj.}{Relativo a órbita; orbital.}{or.bi.tá.rio}{0}
\verb{orca}{ó}{Zool.}{}{}{s.f.}{Mamífero aquático de grande porte, dorso preto, ventre branco e grande nadadeira dorsal.}{or.ca}{0}
\verb{orca}{ó}{}{}{}{}{Vaso de cerâmica semelhante a uma ânfora.}{or.ca}{0}
\verb{orçamentário}{}{}{}{}{adj.}{Relativo a orçamento.}{or.ça.men.tá.rio}{0}
\verb{orçamento}{}{}{}{}{s.m.}{Ato ou efeito de orçar.}{or.ça.men.to}{0}
\verb{orçamento}{}{}{}{}{}{Cálculo que prevê o custo de uma compra, obra ou serviço.}{or.ça.men.to}{0}
\verb{orçamento}{}{}{}{}{}{Cálculo da receita e despesa.}{or.ça.men.to}{0}
\verb{orçar}{}{}{}{}{v.t.}{Calcular um custo.}{or.çar}{0}
\verb{orçar}{}{}{}{}{}{Atingir determinado montante de custo.}{or.çar}{\verboinum{3}}
\verb{ordeiro}{ê}{}{}{}{adj.}{Que prega ou pratica a ordem, a disciplina, a organização.}{or.dei.ro}{0}
\verb{ordem}{ó}{}{"-ens}{}{s.f.}{Organização conveniente das coisas.}{or.dem}{0}
\verb{ordem}{ó}{}{"-ens}{}{}{Disciplina.}{or.dem}{0}
\verb{ordem}{ó}{}{"-ens}{}{}{Associação de profissionais liberais.}{or.dem}{0}
\verb{ordem}{ó}{}{"-ens}{}{}{Determinação dada por alguém investido de autoridade.}{or.dem}{0}
\verb{ordem}{ó}{}{"-ens}{}{}{Classe, categoria de coisas ou seres vivos em uma organização taxonômica.}{or.dem}{0}
\verb{ordem}{ó}{}{"-ens}{}{}{Congregação religiosa.}{or.dem}{0}
\verb{ordem}{ó}{}{"-ens}{}{}{Sacramento que confere o direito de exercer funções de sacerdote.}{or.dem}{0}
\verb{ordem}{ó}{Mat.}{"-ens}{}{}{Lugar ocupado por cada algarismo em um número.}{or.dem}{0}
\verb{ordenação}{}{}{"-ões}{}{s.f.}{Ato ou efeito de ordenar.}{or.de.na.ção}{0}
\verb{ordenação}{}{}{"-ões}{}{}{Cerimônia de colação de ordens eclesiásticas.}{or.de.na.ção}{0}
\verb{ordenada}{}{Mat.}{}{}{s.f.}{A segunda coordenada do plano cartesiano, correspondente ao eixo vertical e representada pela variável \textit{y}.}{or.de.na.da}{0}
\verb{ordenado}{}{}{}{}{adj.}{Disposto em ordem; arranjado.}{or.de.na.do}{0}
\verb{ordenado}{}{}{}{}{}{Que recebeu ordens eclesiásticas.}{or.de.na.do}{0}
\verb{ordenado}{}{}{}{}{s.m.}{Pagamento periódico por um trabalho; salário, vencimento.}{or.de.na.do}{0}
\verb{ordenança}{}{Desus.}{}{}{s.2g.}{Soldado que fica à disposição de um oficial do exército.}{or.de.nan.ça}{0}
\verb{ordenança}{}{Desus.}{}{}{s.f.}{Ato ou efeito de ordenar.}{or.de.nan.ça}{0}
\verb{ordenar}{}{}{}{}{v.t.}{Pôr em ordem.}{or.de.nar}{0}
\verb{ordenar}{}{}{}{}{}{Determinar, mandar.}{or.de.nar}{0}
\verb{ordenar}{}{}{}{}{}{Conferir o sacramento da ordem.}{or.de.nar}{\verboinum{1}}
\verb{ordenha}{}{}{}{}{s.f.}{Ato ou efeito de ordenhar.}{or.de.nha}{0}
\verb{ordenhar}{}{}{}{}{v.t.}{Tirar o leite espremendo as tetas; mungir.}{or.de.nhar}{\verboinum{1}}
\verb{ordinal}{}{}{"-ais}{}{adj.2g.}{Relativo a ordem.}{or.di.nal}{0}
\verb{ordinal}{}{Gram.}{"-ais}{}{}{Diz"-se de numeral que exprime ordem.}{or.di.nal}{0}
\verb{ordinando}{}{}{}{}{adj.}{Diz"-se de quem está se preparando para receber ordens eclesiásticas.}{or.di.nan.do}{0}
\verb{ordinária}{}{}{}{}{s.f.}{Pensão alimentícia.}{or.di.ná.ria}{0}
\verb{ordinária}{}{}{}{}{}{Gasto diário, mensal ou anual.}{or.di.ná.ria}{0}
\verb{ordinário}{}{}{}{}{adj.}{Comum, cotidiano, habitual.}{or.di.ná.rio}{0}
\verb{ordinário}{}{}{}{}{}{De pouca qualidade; vulgar, medíocre.}{or.di.ná.rio}{0}
\verb{ordinário}{}{}{}{}{}{De pouco valor moral; sem caráter; mesquinho, tratante.}{or.di.ná.rio}{0}
\verb{ordoviciano}{}{Geol.}{}{}{adj.}{Diz"-se do período do tempo geológico caracterizado pelo desenvolvimento de animais inferiores, até os crustáceos.}{or.do.vi.ci.a.no}{0}
\verb{orégano}{}{Bot.}{}{}{s.m.}{Planta muito usada como condimento culinário e com óleo essencial com propriedades medicinais.}{o.ré.ga.no}{0}
\verb{orégão}{}{}{"-ãos}{}{s.m.}{Orégano.}{o.ré.gão}{0}
\verb{orelha}{ê}{Anat.}{}{}{s.f.}{A parte exterior do ouvido.}{o.re.lha}{0}
\verb{orelha}{ê}{}{}{}{}{A parte do martelo que serve para arrancar pregos.}{o.re.lha}{0}
\verb{orelha}{ê}{}{}{}{}{Extremidades dobráveis das capas dos livros, que eventualmente carregam informação impressa e que servem para marcar as páginas.}{o.re.lha}{0}
\verb{orelhada}{}{}{}{}{s.f.}{Pancada ou puxão de orelha.}{o.re.lha.da}{0}
\verb{orelha"-de"-pau}{ê}{Bot.}{orelhas"-de"-pau ⟨ê⟩}{}{s.f.}{Espécie de cogumelo que cresce em troncos de árvores; urupê.}{o.re.lha"-de"-pau}{0}
\verb{orelhão}{}{Pop.}{"-ões}{}{s.m.}{Cabine de telefone público cuja forma lembra a de uma orelha.}{o.re.lhão}{0}
\verb{orelhudo}{}{}{}{}{adj.}{Que tem orelhas grandes.}{o.re.lhu.do}{0}
\verb{orelhudo}{}{}{}{}{}{Teimoso, estúpido.}{o.re.lhu.do}{0}
\verb{orfanato}{}{}{}{}{s.m.}{Estabelecimento que dá assistência a órfãos ou crianças abandonadas.}{or.fa.na.to}{0}
\verb{orfandade}{}{}{}{}{s.f.}{Condição de órfão.}{or.fan.da.de}{0}
\verb{orfandade}{}{}{}{}{}{O conjunto dos órfãos.}{or.fan.da.de}{0}
\verb{orfandade}{}{Fig.}{}{}{}{Desamparo, privação.}{or.fan.da.de}{0}
\verb{órfão}{}{}{"-ãos}{órfã}{adj.}{Que perdeu o pai ou a mãe, ou os dois.}{ór.fão}{0}
\verb{órfão}{}{Fig.}{"-ãos}{órfã}{}{Abandonado, desamparado.}{ór.fão}{0}
\verb{orfeão}{}{Mús.}{"-ões}{}{s.m.}{Agremiação ou escola dedicada ao canto coral.}{or.fe.ão}{0}
\verb{orfeônico}{}{}{}{}{adj.}{Relativo a orfeão.}{or.fe.ô.ni.co}{0}
\verb{organdi}{}{}{}{}{s.m.}{Tecido bastante leve, feito geralmente de algodão ou seda.}{or.gan.di}{0}
\verb{organela}{é}{Biol.}{}{}{s.f.}{Formação celular, relativamente permanente e com função definida, limitada, em geral, por membranas.}{or.ga.ne.la}{0}
\verb{orgânico}{}{}{}{}{adj.}{Característico ou derivado de organismos vivos.}{or.gâ.ni.co}{0}
\verb{orgânico}{}{}{}{}{}{Inerente a um organismo, um ser ou uma instituição.}{or.gâ.ni.co}{0}
\verb{orgânico}{}{Quím.}{}{}{}{Diz"-se dos compostos que têm carbono em sua constituição molecular.}{or.gâ.ni.co}{0}
\verb{organismo}{}{}{}{}{s.m.}{O conjunto dos órgãos e sistemas que constituem um ser vivo.}{or.ga.nis.mo}{0}
\verb{organismo}{}{}{}{}{}{O corpo físico, sob o aspecto fisiológico.}{or.ga.nis.mo}{0}
\verb{organismo}{}{}{}{}{}{Organização, entidade, instituição.}{or.ga.nis.mo}{0}
\verb{organista}{}{}{}{}{s.2g.}{Indivíduo que toca órgão.}{or.ga.nis.ta}{0}
\verb{organização}{}{}{"-ões}{}{s.f.}{Ato ou efeito de organizar.}{or.ga.ni.za.ção}{0}
\verb{organização}{}{}{"-ões}{}{}{Arrumação, arranjo, estruturação.}{or.ga.ni.za.ção}{0}
\verb{organização}{}{}{"-ões}{}{}{Instituição pública ou particular.}{or.ga.ni.za.ção}{0}
\verb{organizado}{}{}{}{}{adj.}{Disposto em ordem ou segundo algum critério.}{or.ga.ni.za.do}{0}
\verb{organizado}{}{}{}{}{}{Planejado para uma boa realização.}{or.ga.ni.za.do}{0}
\verb{organizado}{}{}{}{}{}{Estruturado, fundamentado.}{or.ga.ni.za.do}{0}
\verb{organizador}{ô}{}{}{}{adj.}{Que organiza, planeja, constitui.}{or.ga.ni.za.dor}{0}
\verb{organizar}{}{}{}{}{v.t.}{Pôr em ordem; arrumar, ordenar.}{or.ga.ni.zar}{0}
\verb{organizar}{}{}{}{}{}{Planejar, criar, formar.}{or.ga.ni.zar}{0}
\verb{organizar}{}{}{}{}{v.pron.}{Constituir"-se em grupo.}{or.ga.ni.zar}{\verboinum{1}}
\verb{organograma}{}{}{}{}{s.m.}{Representação gráfica das etapas de um trabalho ou dos cargos e funcionários de uma organização.}{or.ga.no.gra.ma}{0}
\verb{organza}{}{}{}{}{s.f.}{Tecido fino e transparente feito geralmente com fio de seda.}{or.gan.za}{0}
\verb{órgão}{}{}{"-ãos}{}{s.m.}{Em um organismo, cada parte que tem uma função específica.}{ór.gão}{0}
\verb{órgão}{}{}{"-ãos}{}{}{Instituição, entidade, geralmente de caráter social.}{ór.gão}{0}
\verb{órgão}{}{}{"-ãos}{}{}{Publicação que representa os interesses de um grupo; jornal, revista.}{ór.gão}{0}
\verb{órgão}{}{Mús.}{"-ãos}{}{}{Instrumento composto de tubos por onde flui ar comprimido produzindo som.}{ór.gão}{0}
\verb{orgasmo}{}{}{}{}{s.m.}{Momento de maior intensidade do prazer sexual em uma relação.}{or.gas.mo}{0}
\verb{orgia}{}{}{}{}{s.f.}{Festa eufórica geralmente com muita bebida e liberdade sexual.}{or.gi.a}{0}
\verb{orgíaco}{}{}{}{}{adj.}{Relativo a orgia.}{or.gí.a.co}{0}
\verb{orgulhar}{}{}{}{}{v.t.}{Proporcionar orgulho; envaidecer.}{or.gu.lhar}{0}
\verb{orgulhar}{}{}{}{}{v.pron.}{Ostentar méritos e conhecimentos; vangloriar"-se, jactanciar"-se.}{or.gu.lhar}{\verboinum{1}}
\verb{orgulho}{}{}{}{}{s.m.}{Sentimento positivo em relação a si mesmo; dignidade pessoal.}{or.gu.lho}{0}
\verb{orgulho}{}{}{}{}{}{Falta de humildade; arrogância, imodéstia.}{or.gu.lho}{0}
\verb{orgulhoso}{ô}{}{"-osos ⟨ó⟩}{"-osa ⟨ó⟩}{adj.}{Que tem ou manifesta orgulho.}{or.gu.lho.so}{0}
\verb{orientação}{}{}{"-ões}{}{s.f.}{Ato ou efeito de orientar.}{o.ri.en.ta.ção}{0}
\verb{orientação}{}{}{"-ões}{}{}{Posição em relação aos pontos cardeais; direção.}{o.ri.en.ta.ção}{0}
\verb{orientação}{}{}{"-ões}{}{}{Tendência, inclinação, impulso, propensão.}{o.ri.en.ta.ção}{0}
\verb{orientador}{ô}{}{}{}{adj.}{Que orienta; guia, dirigente, condutor.}{o.ri.en.ta.dor}{0}
\verb{oriental}{}{}{"-ais}{}{adj.2g.}{Relativo ao Oriente.}{o.ri.en.tal}{0}
\verb{oriental}{}{}{"-ais}{}{s.2g.}{Natural ou habitante de um país do Oriente.}{o.ri.en.tal}{0}
\verb{orientar}{}{}{}{}{v.t.}{Indicar o rumo; nortear.}{o.ri.en.tar}{0}
\verb{orientar}{}{}{}{}{}{Aconselhar, encaminhar, guiar, conduzir.}{o.ri.en.tar}{0}
\verb{orientar}{}{}{}{}{v.pron.}{Situar"-se.}{o.ri.en.tar}{\verboinum{1}}
\verb{oriente}{}{}{}{}{s.m.}{Lugar onde nasce o sol; leste, nascente.}{o.ri.en.te}{0}
\verb{orifício}{}{}{}{}{s.m.}{Pequeno buraco; furo.}{o.ri.fí.cio}{0}
\verb{origami}{}{Art.}{}{}{s.m.}{Arte de origem japonesa em que se produzem figuras da natureza apenas dobrando folhas de papel.}{\textit{origami}}{0}
\verb{origem}{}{}{"-ens}{}{s.f.}{Ponto de partida; começo, princípio.}{o.ri.gem}{0}
\verb{origem}{}{}{"-ens}{}{}{Procedência, proveniência.}{o.ri.gem}{0}
\verb{origem}{}{}{"-ens}{}{}{Causa, motivo, razão.}{o.ri.gem}{0}
\verb{original}{}{}{"-ais}{}{adj.2g.}{Relativo a origem.}{o.ri.gi.nal}{0}
\verb{original}{}{}{"-ais}{}{}{Genuíno, natural, autêntico.}{o.ri.gi.nal}{0}
\verb{original}{}{}{"-ais}{}{}{Diferente em alguns ou muitos aspectos daquilo que já existe; singular.}{o.ri.gi.nal}{0}
\verb{originalidade}{}{}{}{}{s.f.}{Qualidade de original.}{o.ri.gi.na.li.da.de}{0}
\verb{originar}{}{}{}{}{v.t.}{Dar origem; causar.}{o.ri.gi.nar}{\verboinum{1}}
\verb{originário}{}{}{}{}{adj.}{Procedente, proveniente, oriundo.}{o.ri.gi.ná.rio}{0}
\verb{originário}{}{}{}{}{}{Descendente.}{o.ri.gi.ná.rio}{0}
\verb{oriundo}{}{}{}{}{adj.}{Proveniente, originário.}{o.ri.un.do}{0}
\verb{oriundo}{}{}{}{}{}{Descendente.}{o.ri.un.do}{0}
\verb{orixá}{ch}{}{}{}{s.m.}{Designação comum às divindades do candomblé.}{o.ri.xá}{0}
\verb{orixalá}{ch}{Relig.}{}{}{s.m.}{Nome de um grande orixá, sincretizado como Jesus Cristo.}{o.ri.xa.lá}{0}
\verb{orizicultor}{ô}{}{}{}{adj.}{Dedicado à orizicultura; rizicultor.}{o.ri.zi.cul.tor}{0}
\verb{orizicultura}{}{}{}{}{s.f.}{Cultura de arroz; rizicultura.}{o.ri.zi.cul.tu.ra}{0}
\verb{orla}{ó}{}{}{}{s.f.}{Margem, borda, beira.}{or.la}{0}
\verb{orla}{ó}{}{}{}{}{Tira, faixa.}{or.la}{0}
\verb{orlar}{}{}{}{}{v.t.}{Margear, rodear.}{or.lar}{0}
\verb{orlar}{}{}{}{}{}{Guarnecer com orla.}{or.lar}{\verboinum{1}}
\verb{ornamentação}{}{}{"-ões}{}{s.f.}{Ato ou efeito de ornamentar; decoração.}{or.na.men.ta.ção}{0}
\verb{ornamental}{}{}{"-ais}{}{adj.2g.}{Relativo a ornamento ou que serve de ornamento; decorativo.}{or.na.men.tal}{0}
\verb{ornamentar}{}{}{}{}{v.t.}{Colocar ornamentos; enfeitar, ornar.}{or.na.men.tar}{\verboinum{1}}
\verb{ornamento}{}{}{}{}{s.m.}{Adorno, enfeite, ornato.}{or.na.men.to}{0}
\verb{ornamento}{}{}{}{}{}{Pessoa eminente ou ilustre.}{or.na.men.to}{0}
\verb{ornar}{}{}{}{}{v.t.}{Ornamentar.}{or.nar}{\verboinum{1}}
\verb{ornato}{}{}{}{}{s.m.}{Ornamento, ornamentação.}{or.na.to}{0}
\verb{ornear}{}{}{}{}{v.i.}{Soltar ornejos; ornejar, zurrar.}{or.ne.ar}{\verboinum{4}}
\verb{ornejar}{}{}{}{}{v.i.}{Soltar ornejos; zurrar.}{or.ne.jar}{\verboinum{1}}
\verb{ornejo}{ê}{}{}{}{s.m.}{Voz do burro; zurro.}{or.ne.jo}{0}
\verb{ornitologia}{}{}{}{}{s.f.}{Ramo da zoologia que estuda as aves.}{or.ni.to.lo.gi.a}{0}
\verb{ornitológico}{}{}{}{}{adj.}{Relativo à ornitologia.}{or.ni.to.ló.gi.co}{0}
\verb{ornitólogo}{}{}{}{}{s.m.}{Indivíduo especializado em ornitologia.}{or.ni.tó.lo.go}{0}
\verb{ornitorrinco}{}{Zool.}{}{}{s.m.}{Mamífero ovíparo com bico de pato, cloaca e patas com membranas, encontrado na região australiana.}{or.ni.tor.rin.co}{0}
\verb{orosfera}{é}{Geogr.}{}{}{s.f.}{A parte externa consolidada da Terra; crosta da Terra, crosta terrestre, litosfera.}{o.ros.fe.ra}{0}
\verb{orquestra}{é}{}{}{}{s.f.}{Conjunto de músicos com seus respectivos instrumentos que executam peça musical ou acompanham um cantor.}{or.ques.tra}{0}
\verb{orquestração}{}{}{"-ões}{}{s.f.}{Ato ou efeito de orquestrar.}{or.ques.tra.ção}{0}
\verb{orquestrador}{ô}{}{}{}{adj.}{Que adapta uma obra musical para ser executada por uma orquestra.}{or.ques.tra.dor}{0}
\verb{orquestral}{}{}{"-ais}{}{adj.2g.}{Relativo a orquestra ou a música de orquestra.}{or.ques.tral}{0}
\verb{orquestrar}{}{}{}{}{v.t.}{Compor ou adaptar obra musical para ser executada por uma orquestra.}{or.ques.trar}{\verboinum{1}}
\verb{orquidácea}{}{Bot.}{}{}{s.f.}{Espécime das orquidáceas, família que reúne diversas plantas com flores solitárias, muitas delas cultivadas como ornamentais.}{or.qui.dá.cea}{0}
\verb{orquidário}{}{}{}{}{s.m.}{Aglomerado ou viveiro de orquídeas.}{or.qui.dá.rio}{0}
\verb{orquídea}{}{Bot.}{}{}{s.f.}{Designação comum às plantas da família das orquidáceas, muitas delas cultivadas como ornamentais pela beleza de suas flores.}{or.quí.dea}{0}
\verb{orquídea}{}{}{}{}{}{A flor dessas plantas.}{or.quí.dea}{0}
\verb{ortodontia}{}{}{}{}{s.f.}{Ramo da odontologia que tem como objeto a manutenção e a correção do alinhamento dos dentes.}{or.to.don.ti.a}{0}
\verb{ortodontista}{}{}{}{}{s.2g.}{Indivíduo especialista em ortodontia.}{or.to.don.tis.ta}{0}
\verb{ortodoxia}{cs}{}{}{}{s.f.}{Qualidade de ortodoxo.}{or.to.do.xi.a}{0}
\verb{ortodoxo}{ócs}{}{}{}{adj.}{Que segue e defende dogmas e normas tradicionais.}{or.to.do.xo}{0}
\verb{ortodoxo}{ócs}{Pop.}{}{}{}{Que não gosta de novidades ou de mudanças dos padrões e ideias.}{or.to.do.xo}{0}
\verb{ortoepia}{}{Gram.}{}{}{s.f.}{Ortoépia.}{or.to.e.pi.a}{0}
\verb{ortoépia}{}{Gram.}{}{}{s.f.}{Estudo de caráter normativo da forma considerada culta de pronunciar as palavras.}{or.to.é.pia}{0}
\verb{ortofonia}{}{Med.}{}{}{s.f.}{Tratamento para corrigir a articulação da fala ou eliminar os chamados vícios de pronúncia.}{or.to.fo.ni.a}{0}
\verb{ortogonal}{}{}{"-ais}{}{adj.2g.}{Diz"-se de retas que formam um ângulo de 90 graus.}{or.to.go.nal}{0}
\verb{ortografar}{}{}{}{}{v.t.}{Escrever segundo as regras ortográficas.}{or.to.gra.far}{0}
\verb{ortografar}{}{}{}{}{}{Grafar.}{or.to.gra.far}{\verboinum{1}}
\verb{ortografia}{}{Gram.}{}{}{s.f.}{Parte da gramática que ensina a escrever corretamente as palavras.}{or.to.gra.fi.a}{0}
\verb{ortografia}{}{}{}{}{}{Maneira de representar as palavras por meio da escrita; grafia.}{or.to.gra.fi.a}{0}
\verb{ortográfico}{}{}{}{}{adj.}{Relativo a ortografia.}{or.to.grá.fi.co}{0}
\verb{ortopedia}{}{Med.}{}{}{s.f.}{Especialidade médica que se dedica ao estudo e tratamento do sistema locomotor e da coluna vertebral.}{or.to.pe.di.a}{0}
\verb{ortopédico}{}{}{}{}{adj.}{Relativo a ortopedia.}{or.to.pé.di.co}{0}
\verb{ortopedista}{}{}{}{}{s.2g.}{Especialista em ortopedia.}{or.to.pe.dis.ta}{0}
\verb{ortóptero}{}{Zool.}{}{}{s.m.}{Espécime dos ortópteros, ordem de insetos terrestres, alados ou ápteros, com aparelho bucal mastigador, cosmopolitas, e cujos representantes mais comuns são os gafanhotos.}{or.tóp.te.ro}{0}
\verb{ortóptero}{}{}{}{}{adj.}{Relativo aos ortópteros.}{or.tóp.te.ro}{0}
\verb{orvalhar}{}{}{}{}{v.t.}{Molhar ou umedecer com orvalho; cobrir de orvalho.}{or.va.lhar}{0}
\verb{orvalhar}{}{}{}{}{}{Borrifar ou aspergir com gotas de qualquer líquido.}{or.va.lhar}{0}
\verb{orvalhar}{}{}{}{}{v.i.}{Chuviscar, garoar.}{or.va.lhar}{\verboinum{1}}
\verb{orvalho}{}{}{}{}{s.m.}{Umidade da atmosfera, que se condensa, principalmente durante a noite, e se deposita, em forma de gotículas, sobre qualquer superfície fria.}{or.va.lho}{0}
\verb{orvalho}{}{}{}{}{}{Chuvisco, garoa.}{or.va.lho}{0}
\verb{Os}{}{Quím.}{}{}{}{Símb. do \textit{ósmio}.}{Os}{0}
\verb{oscilação}{}{}{"-ões}{}{s.f.}{Ato ou efeito de oscilar.}{os.ci.la.ção}{0}
\verb{oscilação}{}{}{"-ões}{}{}{Movimento de um corpo que passa e torna a passar alternadamente pelas mesmas posições.}{os.ci.la.ção}{0}
\verb{oscilação}{}{}{"-ões}{}{}{Movimento de vaivém; balanço.}{os.ci.la.ção}{0}
\verb{oscilação}{}{Fig.}{"-ões}{}{}{Incerteza, dúvida, indecisão.}{os.ci.la.ção}{0}
\verb{oscilante}{}{}{}{}{adj.2g.}{Que oscila; pendular.}{os.ci.lan.te}{0}
\verb{oscilante}{}{Fig.}{}{}{}{Indeciso, incerto.}{os.ci.lan.te}{0}
\verb{oscilar}{}{}{}{}{v.i.}{Mover"-se de um lado para outro; balançar"-se.}{os.ci.lar}{0}
\verb{oscilar}{}{}{}{}{}{Deslocar"-se alternadamente de um lado para outro, em sentidos opostos.}{os.ci.lar}{0}
\verb{oscilar}{}{}{}{}{}{Abalar"-se,tremer.}{os.ci.lar}{0}
\verb{oscilar}{}{}{}{}{}{Hesitar, vascilar.}{os.ci.lar}{\verboinum{1}}
\verb{oscilatório}{}{}{}{}{adj.}{Oscilante.}{os.ci.la.tó.rio}{0}
\verb{osciloscópio}{}{Fís.}{}{}{s.m.}{Instrumento usado para detectar e observar oscilações.}{os.ci.los.có.pio}{0}
\verb{oscitar}{}{}{}{}{v.i.}{Bocejar.}{os.ci.tar}{\verboinum{1}}
\verb{oscular}{}{}{}{}{v.t.}{Beijar.}{os.cu.lar}{\verboinum{1}}
\verb{ósculo}{}{}{}{}{s.m.}{Beijo.}{ós.cu.lo}{0}
\verb{ósculo}{}{}{}{}{}{Beijo dado ou recebido como sinal de paz e amizade.}{ós.cu.lo}{0}
\verb{ósculo}{}{Zool.}{}{}{}{Orifício de saída da água de uma esponja.}{ós.cu.lo}{0}
\verb{OSE}{}{}{}{}{}{Abrev. de \textit{oés"-sueste}.}{O.S.E.}{0}
\verb{osga}{ó}{Zool.}{}{}{s.f.}{Lagartixa.}{os.ga}{0}
\verb{osga}{ó}{Fig.}{}{}{}{Repulsa, asco.}{os.ga}{0}
\verb{ósmio}{}{Quím.}{}{}{s.m.}{Elemento químico metálico, branco"-azulado, utilizado em ligas muito duras, com o irídio e a platina. \elemento{76}{190.23}{Os}.}{ós.mio}{0}
\verb{osmose}{ó}{Bioquím.}{}{}{s.f.}{Fluxo do solvente de uma solução pouco concentrada, em direção a outra mais concentrada, que se dá através de uma membrana impermeável.}{os.mo.se}{0}
\verb{OSO}{}{}{}{}{}{Abrev. de \textit{oés"-sudoeste}.}{O.S.O.}{0}
\verb{ossada}{}{}{}{}{s.f.}{Grande quantidade de ossos.}{os.sa.da}{0}
\verb{ossada}{}{}{}{}{}{Os ossos de um cadáver.}{os.sa.da}{0}
\verb{ossada}{}{}{}{}{}{Armação, estrutura, esqueleto.}{os.sa.da}{0}
\verb{ossada}{}{Fig.}{}{}{}{Restos, destroços.}{os.sa.da}{0}
\verb{ossaria}{}{}{}{}{s.f.}{Grande quantidade de ossos; ossada.}{os.sa.ri.a}{0}
\verb{ossaria}{}{}{}{}{}{Local onde se depositam os ossos humanos extraídos dos cemitérios; ossuário.}{os.sa.ri.a}{0}
\verb{ossário}{}{}{}{}{s.m.}{Local onde se depositam os ossos dos finados em cemitérios; ossuário.}{os.sá.rio}{0}
\verb{ossário}{}{}{}{}{}{Sepulcro com muitos cadáveres.}{os.sá.rio}{0}
\verb{ossatura}{}{}{}{}{s.f.}{Conjunto de ossos; esqueleto de um homem ou de um animal.}{os.sa.tu.ra}{0}
\verb{ossatura}{}{}{}{}{}{Armação, estrutura, carcaça.}{os.sa.tu.ra}{0}
\verb{ósseo}{}{}{}{}{adj.}{Relativo ao osso, ou da natureza dele.}{ós.seo}{0}
\verb{ósseo}{}{}{}{}{}{Que tem ossos.}{ós.seo}{0}
\verb{ossículo}{}{}{}{}{s.m.}{Pequeno osso.}{os.sí.cu.lo}{0}
\verb{ossículo}{}{}{}{}{}{Caroço de frutos quando pequeno e não divisível.}{os.sí.cu.lo}{0}
\verb{ossificação}{}{}{"-ões}{}{s.f.}{Ato ou efeito de ossificar.}{os.si.fi.ca.ção}{0}
\verb{ossificação}{}{}{"-ões}{}{}{Formação de ossos ou do sistema ósseo.}{os.si.fi.ca.ção}{0}
\verb{ossificar}{}{}{}{}{v.t.}{Converter em osso.}{os.si.fi.car}{0}
\verb{ossificar}{}{}{}{}{}{Endurecer como osso.}{os.si.fi.car}{\verboinum{2}}
\verb{osso}{ô}{}{}{}{s.m.}{Parte dura e sólida que forma o esqueleto do corpo do homem e dos animais vertebrados.}{os.so}{0}
\verb{osso}{ô}{Fig.}{}{}{}{Dificuldade.}{os.so}{0}
\verb{ossuário}{}{}{}{}{s.m.}{Local onde se depositam os ossos dos finados em cemitérios; ossário.}{os.su.á.rio}{0}
\verb{ossuário}{}{}{}{}{}{Sepulcro com muitos cadáveres.}{os.su.á.rio}{0}
\verb{ossudo}{}{}{}{}{adj.}{Que tem ossos grandes.}{os.su.do}{0}
\verb{ossudo}{}{}{}{}{}{Que tem os ossos muito salientes.}{os.su.do}{0}
\verb{osteíte}{}{Med.}{}{}{s.f.}{Inflamação do tecido ósseo.}{os.te.í.te}{0}
\verb{ostensivo}{}{}{}{}{adj.}{Que se quer mostrar ou ostentar.}{os.ten.si.vo}{0}
\verb{ostensivo}{}{}{}{}{}{Aparente, evidente.}{os.ten.si.vo}{0}
\verb{ostensório}{}{}{}{}{s.m.}{Peça para apresentar ou expor a hóstia consagrada para os fiéis.}{os.ten.só.rio}{0}
\verb{ostentação}{}{}{"-ões}{}{s.f.}{Ato ou efeito de ostentar.}{os.ten.ta.ção}{0}
\verb{ostentação}{}{}{"-ões}{}{}{Exibicionismo.}{os.ten.ta.ção}{0}
\verb{ostentação}{}{}{"-ões}{}{}{Pompa, luxo.}{os.ten.ta.ção}{0}
\verb{ostentador}{ô}{}{}{}{adj.}{Que ostenta.}{os.ten.ta.dor}{0}
\verb{ostentador}{ô}{}{}{}{s.m.}{Indivíduo que age ou fala com ostentação.}{os.ten.ta.dor}{0}
\verb{ostentar}{}{}{}{}{v.t.}{Exibir com luxo.}{ostentar}{\verboinum{1}}
\verb{ostentoso}{ô}{}{"-osos ⟨ó⟩}{"-osa ⟨ó⟩}{adj.}{Que é feito com ostentação.}{os.ten.to.so}{0}
\verb{ostentoso}{ô}{}{"-osos ⟨ó⟩}{"-osa ⟨ó⟩}{}{Que tem pompa; luxuoso.}{os.ten.to.so}{0}
\verb{osteologia}{}{}{}{}{s.f.}{Ciência que estuda os ossos.}{os.te.o.lo.gi.a}{0}
\verb{osteomielite}{}{Med.}{}{}{s.f.}{Inflamação da medula dos ossos.}{os.te.o.mi.e.li.te}{0}
\verb{osteopatia}{}{Med.}{}{}{s.f.}{Doença dos ossos.}{os.te.o.pa.ti.a}{0}
\verb{osteoporose}{ó}{Med.}{}{}{s.f.}{Fragilidade óssea devida à diminuição da densidade dos ossos.}{os.te.o.po.ro.se}{0}
\verb{ostra}{ô}{Zool.}{}{}{s.f.}{Nome comum aos moluscos bivalves, marinhos, que formam colônias e vivem fixos em qualquer substrato firme, mesmo uns nos outros, com várias espécies comestíveis e algumas produtoras de pérolas.}{os.tra}{0}
\verb{ostra}{ô}{Pop.}{}{}{}{Indivíduo importuno, que vive agarrado ou seguindo outrem.}{os.tra}{0}
\verb{ostracismo}{}{}{}{}{s.m.}{Na Grécia antiga, desterro temporário determinado em plebiscito contra um cidadão.}{os.tra.cis.mo}{0}
\verb{ostracismo}{}{Por ext.}{}{}{}{Afastamento das funções políticas.}{os.tra.cis.mo}{0}
\verb{ostracismo}{}{}{}{}{}{Exclusão, banimento.}{os.tra.cis.mo}{0}
\verb{ostreicultor}{ô}{}{}{}{s.m.}{Indivíduo que pratica a ostreicultura.}{os.tre.i.cul.tor}{0}
\verb{ostreicultura}{}{}{}{}{s.f.}{Cultura de ostras.}{os.tre.i.cul.tu.ra}{0}
\verb{ostreira}{ê}{}{}{}{s.f.}{Lugar destinado à criação de ostras.}{os.trei.ra}{0}
\verb{ostrogodo}{ô}{}{}{}{adj.}{Relativo ou pertencente aos ostrogodos, um dos povos germânicos que derrubaram o império romano.}{os.tro.go.do}{0}
\verb{ostrogodo}{ô}{}{}{}{s.m.}{Indivíduo dos ostrogodos.}{os.tro.go.do}{0}
\verb{otalgia}{}{Med.}{}{}{s.f.}{Dor de ouvido.}{o.tal.gi.a}{0}
\verb{otário}{}{Pop.}{}{}{s.m.}{Indivíduo tolo, simplório, fácil de ser enganado.}{o.tá.rio}{0}
\verb{ótica}{}{}{}{}{}{Var. de \textit{óptica}.}{ó.ti.ca}{0}
\verb{ótico}{}{}{}{}{adj.}{Relativo a orelha.}{ó.ti.co}{0}
\verb{ótico}{}{}{}{}{}{Var. de \textit{óptico}.}{ó.ti.co}{0}
\verb{otimismo}{}{}{}{}{s.m.}{Atitude de quem acha que tudo vai bem, pode dar certo ou ser contornado ou melhorado.}{o.ti.mis.mo}{0}
\verb{otimista}{}{}{}{}{adj.2g.}{Que tem otimismo.}{o.ti.mis.ta}{0}
\verb{otimista}{}{}{}{}{s.2g.}{Pessoa que tem ou revela otimismo.}{o.ti.mis.ta}{0}
\verb{otimização}{}{}{"-ões}{}{s.f.}{Aproveitamento máximo da capacidade de alguém ou de alguma coisa.}{o.ti.mi.za.ção}{0}
\verb{otimizar}{}{}{}{}{v.t.}{Fazer com que uma coisa ou pessoa use sua capacidade ao máximo.}{o.ti.mi.zar}{\verboinum{1}}
\verb{ótimo}{}{}{}{}{adj.}{Muito bom, excelente; o melhor possível.}{ó.ti.mo}{0}
\verb{otite}{}{Med.}{}{}{s.f.}{Inflamação do ouvido.}{o.ti.te}{0}
\verb{otomano}{}{}{}{}{adj.}{Relativo à Turquia (Ásia e Europa); turco.  }{o.to.ma.no}{0}
\verb{otomano}{}{}{}{}{s.m.}{Indivíduo natural ou habitante desse país.  }{o.to.ma.no}{0}
\verb{otorrino}{}{Pop.}{}{}{s.m.}{Forma reduzida de otorrinolaringologista.}{o.tor.ri.no}{0}
\verb{otorrinolaringologia}{}{Med.}{}{}{s.f.}{Especialidade da medicina que estuda e trata das doenças que afetam o ouvido, o nariz e a garganta.}{o.tor.ri.no.la.rin.go.lo.gi.a}{0}
\verb{otorrinolaringologista}{}{Med.}{}{}{s.2g.}{Médico especializado em otorrinolaringologia.}{o.tor.ri.no.la.rin.go.lo.gis.ta}{0}
\verb{ou}{}{}{}{}{conj.}{Liga termos ou orações, indicando alternância ou exclusão. (\textit{Ele tem de querer suco ou água; não há outras opções.})}{ou}{0}
\verb{ou}{}{}{}{}{}{De outra forma; isto é. (\textit{Ele me disse que não gosta de futebol, ou que prefere outros esportes.})}{ou}{0}
\verb{ourela}{é}{}{}{}{s.f.}{Borda de um tecido, que lhe serve de acabamento.}{ou.re.la}{0}
\verb{ourela}{é}{}{}{}{}{Margem, beira.}{ou.re.la}{0}
\verb{ouriçado}{}{}{}{}{}{Aborrecido, nervoso, irritado.}{ou.ri.ça.do}{0}
\verb{ouriçado}{}{}{}{}{adj.}{Que tem forma de ouriço.}{ou.ri.ça.do}{0}
\verb{ouriçado}{}{}{}{}{}{Desalinhado, arrepiado, encrespado.}{ou.ri.ça.do}{0}
\verb{ouriçado}{}{}{}{}{}{Agitado, animado, excitado.}{ou.ri.ça.do}{0}
\verb{ouriçar}{}{}{}{}{v.t.}{Fazer os cabelos ou pelos levantarem; eriçar, arrepiar.}{ou.ri.çar}{0}
\verb{ouriçar}{}{Fig.}{}{}{}{Entusiasmar, deixar alguém muito animado; excitar, agitar, exaltar.}{ou.ri.çar}{\verboinum{3}}
\verb{ouriço}{}{Zool.}{}{}{s.m.}{Ouriço"-cacheiro. }{ou.ri.ço}{0}
\verb{ouriço}{}{Zool.}{}{}{}{Ouriço"-do"-mar.}{ou.ri.ço}{0}
\verb{ouriço}{}{}{}{}{}{Casca dura ou espinhosa de alguns frutos.}{ou.ri.ço}{0}
\verb{ouriço"-cacheiro}{ê}{Zool.}{ouriços"-cacheiros}{}{s.m.}{Pequeno mamífero arborícola, de até 40 cm de comprimento, com o corpo coberto de espinhos (pelos modificados) e cauda preênsil, encontrado em grande parte de América do Sul; ouriço.}{ou.ri.ço"-ca.chei.ro}{0}
\verb{ouriço"-do"-mar}{}{Zool.}{ouriços"-do"-mar}{}{s.m.}{Nome comum aos equinodermos da classe dos equinoides, invertebrados marinhos de corpo esférico, com simetria radial, dotados de espinhos móveis, pés ambulacrários, e que se alimentam raspando algas das rochas.}{ou.ri.ço"-do"-mar}{0}
\verb{ourives}{}{}{}{}{s.2g.}{Pessoa que fabrica ou vende joias e objetos de ouro, prata etc.}{ou.ri.ves}{0}
\verb{ourivesaria}{}{}{}{}{s.f.}{Arte ou comércio de ourives.}{ou.ri.ve.sa.ri.a}{0}
\verb{ourivesaria}{}{}{}{}{}{Oficina ou loja de ourives.}{ou.ri.ve.sa.ri.a}{0}
\verb{ouro}{ô}{Quím.}{}{}{s.m.}{Elemento químico metálico, amarelo,  muito brilhante, dúctil,  maleável,  utilizado na fabricação de joias e moedas e em odontologia, quase sempre sob a forma de ligas com outros elementos, como a prata e o cobre. \elemento{79}{196.96655}{Au}.}{ou.ro}{0}
\verb{ouro}{ô}{Por ext.}{}{}{}{Riqueza, opulência.}{ou.ro}{0}
\verb{ouropel}{é}{}{"-éis}{}{s.m.}{Lâmina fina de metal que imita ouro; ouro falso.  (\textit{Diversos enfeites de ouropel adornavam a sala.})}{ou.ro.pel}{0}
\verb{ouros}{ô}{}{}{}{s.m.pl.}{Um dos quatro naipes do baralho, representado por um losango vermelho.}{ou.ros}{0}
\verb{ousadia}{}{}{}{}{s.f.}{Qualidade de quem enfrenta perigos e grandes dificuldades sem medo; coragem, arrojo.}{ou.sa.di.a}{0}
\verb{ousadia}{}{}{}{}{}{Atrevimento, audácia, petulância.}{ou.sa.di.a}{0}
\verb{ousado}{}{}{}{}{adj.}{Que tem ousadia; corajoso, arrojado.}{ou.sa.do}{0}
\verb{ousado}{}{}{}{}{}{Que desrespeita as regras de comportamento e convívio da sociedade; atrevido, audacioso, malcriado, insolente.}{ou.sa.do}{0}
\verb{ousar}{}{}{}{}{v.t.}{Não ter medo de fazer alguma coisa arriscada; arrojar"-se, atrever"-se.}{ou.sar}{0}
\verb{ousar}{}{}{}{}{}{Desrespeitar as regras de comportamento e convívio da sociedade; atrever"-se.}{ou.sar}{\verboinum{1}}
\verb{outão}{}{}{"-ões}{}{s.m.}{Oitão.}{ou.tão}{0}
\verb{outdoor}{}{}{}{}{s.m.}{Painel publicitário de grandes dimensões, geralmente exposto à margem das vias públicas.}{\textit{outdoor}}{0}
\verb{outeiro}{ê}{}{}{}{s.m.}{Monte pequeno, ou pequena elevação de terreno, menor que um morro; colina.}{ou.tei.ro}{0}
\verb{outonada}{}{}{}{}{s.f.}{A estação de outono.}{ou.to.na.da}{0}
\verb{outonada}{}{}{}{}{}{A colheita de outono.}{ou.to.na.da}{0}
\verb{outonal}{}{}{"-ais}{}{adj.2g.}{Relativo ou próprio do outono; outoniço.}{ou.to.nal}{0}
\verb{outoniço}{}{}{}{}{adj.}{Outonal.}{ou.to.ni.ço}{0}
\verb{outono}{}{}{}{}{s.m.}{Estação do ano que antecede o inverno e sucede ao verão, e que, no hemisfério sul, compreende o período de 21 de março a 20 de junho.}{ou.to.no}{0}
\verb{outono}{}{Por ext.}{}{}{}{A época da colheita.}{ou.to.no}{0}
\verb{outono}{}{Fig.}{}{}{}{Período da vida que antecede a velhice ou a meia"-idade.}{ou.to.no}{0}
\verb{outono}{}{Fig.}{}{}{}{Ocaso, decadência, declínio.}{ou.to.no}{0}
\verb{outorga}{ó}{}{}{}{s.f.}{Ato ou efeito de outorgar; consentimento, concessão.}{ou.tor.ga}{0}
\verb{outorgante}{}{}{}{}{adj.2g.}{Que outorga, concede ou transfere; outorgador.}{ou.tor.gan.te}{0}
\verb{outorgar}{}{}{}{}{v.t.}{Dar, conceder alguma coisa a alguém.}{ou.tor.gar}{\verboinum{5}}
\verb{output}{}{Informát.}{}{}{s.m.}{Resultado, dados obtidos após o processamento.}{\textit{output}}{0}
\verb{outrem}{}{}{}{}{pron.}{Outra(s) pessoa(s), que não é conhecidas. (\textit{Mesmo tendo um médico particular, preferiu tratar"-se com outrem.})}{ou.trem}{0}
\verb{outro}{}{}{}{}{pron.}{Que é diferente da pessoa ou da coisa que foi mencionada; que não é o mesmo; diverso, diferente.}{ou.tro}{0}
\verb{outro}{}{}{}{}{}{Mais um.}{ou.tro}{0}
\verb{outrora}{ó}{}{}{}{adv.}{Em outro tempo, em tempo passado; antigamente.}{ou.tro.ra}{0}
\verb{outrossim}{}{}{}{}{adv.}{Igualmente; da mesma maneira; também.}{ou.tros.sim}{0}
\verb{outubro}{}{}{}{}{s.m.}{O décimo mês do ano civil.}{ou.tu.bro}{0}
\verb{ouvido}{}{}{}{}{adj.}{Que se ouviu; escutado.}{ou.vi.do}{0}
\verb{ouvido}{}{Anat.}{}{}{s.m.}{Conjunto dos órgãos da audição; orelha.}{ou.vi.do}{0}
\verb{ouvido}{}{}{}{}{}{Sentido pelo qual se percebem os sons e os ruídos; audição.}{ou.vi.do}{0}
\verb{ouvidor}{ô}{}{}{}{s.m.}{Que ouve; ouvinte.}{ou.vi.dor}{0}
\verb{ouvidoria}{}{}{}{}{s.f.}{Cargo, função de ouvidor.}{ou.vi.do.ri.a}{0}
\verb{ouvinte}{}{}{}{}{s.2g.}{Pessoa que ouve quem está falando. }{ou.vin.te}{0}
\verb{ouvir}{}{}{}{}{v.t.}{Perceber algum som ou ruído; escutar.}{ou.vir}{0}
\verb{ouvir}{}{}{}{}{}{Atender um pedido; seguir um conselho ou opinião.}{ou.vir}{\verboinum{58}}
\verb{ova}{ó}{}{}{}{s.f.}{O ovário dos peixes.}{o.va}{0}
\verb{ova}{ó}{}{}{}{}{Usada na locução \textit{Uma ova!}, para exprimir protesto, objeção, revolta, repulsa etc.}{o.va}{0}
\verb{ovação}{}{}{"-ões}{}{s.f.}{Aclamação pública; aplausos ou honras dirigidas a alguém.}{o.va.ção}{0}
\verb{ovacionar}{}{}{}{}{v.t.}{Fazer ovação; aplaudir, aclamar.}{o.va.ci.o.nar}{\verboinum{1}}
\verb{oval}{}{}{"-ais}{}{adj.2g.}{Que tem forma de ovo; ovoide.}{o.val}{0}
\verb{ovalar}{}{}{}{}{adj.2g.}{Oval.}{o.va.lar}{0}
\verb{ovalar}{}{}{}{}{v.t.}{Dar forma oval, tornar oval.}{o.va.lar}{\verboinum{1}}
\verb{ovante}{}{}{}{}{adj.2g.}{Vitorioso, triunfante.}{o.van.te}{0}
\verb{ovar}{}{}{}{}{v.i.}{Pôr ovos.}{o.var}{0}
\verb{ovar}{}{}{}{}{}{Criar ovas ou ovos.}{o.var}{\verboinum{1}}
\verb{ovariano}{}{}{}{}{adj.}{Relativo a ovário.}{o.va.ri.a.no}{0}
\verb{ovário}{}{Anat.}{}{}{s.m.}{Órgão do corpo da mulher e da fêmea dos animais que produz os óvulos destinados à fecundação.}{o.vá.rio}{0}
\verb{ovário}{}{Bot.}{}{}{}{Nas plantas, parte da flor onde se formam os óvulos e que depois dá origem ao fruto.}{o.vá.rio}{0}
\verb{oveiro}{ê}{}{}{}{s.m.}{Ovário das aves.}{o.vei.ro}{0}
\verb{oveiro}{ê}{Lus.}{}{}{s.m.}{Recipiente para servir os ovos à mesa.}{o.vei.ro}{0}
\verb{ovelha}{ê}{Zool.}{}{}{s.f.}{Animal mamífero, ruminante, que tem o corpo coberto de lã; a fêmea do carneiro.}{o.ve.lha}{0}
\verb{ovelhum}{}{}{"-uns}{}{adj.}{Relativo a ovelhas, carneiros e cordeiros; ovino.}{o.ve.lhum}{0}
\verb{overdose}{ó}{}{}{}{s.f.}{Dose muito grande, tóxica, de um remédio ou droga.}{\textit{overdose}}{0}
\verb{overloque}{ó}{}{}{}{s.f.}{Peça de máquina de costura própria para dar acabamento em tecidos.}{o.ver.lo.que}{0}
\verb{overloquista}{}{}{}{}{s.2g.}{Pessoa que trabalha com overloque.}{o.ver.lo.quis.ta}{0}
\verb{overnight}{}{}{}{}{adj.}{Diz"-se de aplicações financeiras com rendimento diário, que podem ser resgatadas no dia seguinte àquele em que foram feitas.}{\textit{overnight}}{0}
\verb{overnight}{}{}{}{}{s.m.}{O mercado dessas aplicações.}{\textit{overnight}}{0}
\verb{oviário}{}{}{}{}{s.m.}{Curral de ovelhas; ovil.}{o.vi.á.rio}{0}
\verb{oviário}{}{}{}{}{}{Rebanho de ovelhas.}{o.vi.á.rio}{0}
\verb{ovil}{}{}{"-is}{}{s.m.}{Curral de ovelhas; oviário, redil.}{o.vil}{0}
\verb{ovino}{}{}{}{}{adj.}{Que se refere a ovelhas, carneiros ou cordeiros; ovelhum.}{o.vi.no}{0}
\verb{ovinocultor}{ô}{}{}{}{s.m.}{Indivíduo que cria ovelhas.}{o.vi.no.cul.tor}{0}
\verb{ovinocultura}{}{}{}{}{s.f.}{Criação de ovelhas.}{o.vi.no.cul.tu.ra}{0}
\verb{ovíparo}{}{Zool.}{}{}{adj.}{Diz"-se de animal que põe ovos, que se reproduz por meio de ovos.}{o.ví.pa.ro}{0}
\verb{ovíparo}{}{}{}{}{s.m.}{Esse animal.}{o.ví.pa.ro}{0}
\verb{óvni}{}{}{}{}{s.m.}{Nome comum aos objetos voadores não identificados, supostamente de origem extraterrestre; disco voador, ufo.}{óv.ni}{0}
\verb{ovo}{ô}{}{"-s ⟨ó⟩}{}{s.m.}{Corpo arredondado, envolvido por uma casca, que abriga um novo animal em formação e que contém uma reserva nutritiva para alimentá"-lo até o nascimento.}{o.vo}{0}
\verb{ovo}{ô}{}{"-s ⟨ó⟩}{}{}{Célula originada da fecundação, da união do óvulo (a célula reprodutora feminina) com o espermatozoide (a célula reprodutora masculina).}{o.vo}{0}
\verb{ovo}{ô}{}{"-s ⟨ó⟩}{}{}{Usado na expressão \textit{pisar em ovos}, agir com todo cuidado, com cautela.}{o.vo}{0}
\verb{ovoide}{}{}{}{}{adj.}{Que tem forma semelhante à do ovo; oval.}{o.voi.de}{0}
\verb{ovovivíparo}{}{Zool.}{}{}{adj.}{Diz"-se de animal cujo ovo fica dentro da mãe, sem retirar alimento do organismo dela, mas da reserva nutritiva que esse ovo possui.}{o.vo.vi.ví.pa.ro}{0}
\verb{ovovivíparo}{}{}{}{}{s.m.}{Esse animal.}{o.vo.vi.ví.pa.ro}{0}
\verb{ovulação}{}{}{"-ões}{}{s.f.}{Processo de formação e liberação de um óvulo maduro do útero.}{o.vu.la.ção}{0}
\verb{ovular}{}{}{}{}{v.i.}{Produzir óvulos.}{o.vu.lar}{\verboinum{1}}
\verb{ovular}{}{}{}{}{adj.2g.}{Relativo ao óvulo.}{o.vu.lar}{0}
\verb{ovular}{}{}{}{}{}{Que apresenta a forma de um ovo de galinha; oval.}{o.vu.lar}{0}
\verb{óvulo}{}{Biol.}{}{}{s.m.}{Gameta feminino que, fecundado, se transforma em ovo.}{ó.vu.lo}{0}
\verb{óvulo}{}{Bot.}{}{}{}{Pequeno órgão contido no ovário das flores, envolvido por tecidos nutritivos e protetores, que contém a oosfera, de onde sairá a semente fecundada.}{ó.vu.lo}{0}
\verb{óvulo}{}{}{}{}{}{Pequena cápsula ovoide com substância medicamentosa destinada à aplicação vaginal.}{ó.vu.lo}{0}
\verb{oxalá}{ch}{}{}{}{interj.}{Expressão que denota desejo de que certa coisa aconteça; tomara, queira Deus.}{o.xa.lá}{0}
\verb{oxalá}{}{Relig.}{}{}{s.m.}{No candomblé, orixá da criação, da procriação, que se identifica com Jesus Cristo.}{o.xa.lá}{0}
\verb{oxidação}{cs}{}{"-ões}{}{s.f.}{Ato ou efeito de oxidar. }{o.xi.da.ção}{0}
\verb{oxidação}{cs}{Quím.}{"-ões}{}{}{Reação química, frequentemente provocada pelo oxigênio, que ocasiona perda de elétrons de um átomo ou molécula e consequente aumento de sua carga.}{o.xi.da.ção}{0}
\verb{oxidante}{cs}{}{}{}{adj.2g.}{Que tem a propriedade de oxidar; agente da oxidação.}{o.xi.dan.te}{0}
\verb{oxidar}{cs}{}{}{}{v.t.}{Converter em óxido.}{o.xi.dar}{0}
\verb{oxidar}{cs}{}{}{}{}{Causar uma reação de oxidação em certas substâncias.}{o.xi.dar}{0}
\verb{oxidar}{cs}{}{}{}{}{Tornar enferrujado.}{o.xi.dar}{\verboinum{1}}
\verb{óxido}{cs}{Quím.}{}{}{s.m.}{Composto formado pela combinação de oxigênio com outro elemento.}{ó.xi.do}{0}
\verb{oxigenação}{cs}{}{"-ões}{}{s.f.}{Ato ou efeito de oxigenar; fixação do oxigênio pela hemoglobina.}{o.xi.ge.na.ção}{0}
\verb{oxigenado}{cs}{}{}{}{adj.}{Diz"-se da substância combinada com oxigênio.}{o.xi.ge.na.do}{0}
\verb{oxigenado}{cs}{}{}{}{}{Que teve o ar renovado, purificado.}{o.xi.ge.na.do}{0}
\verb{oxigenar}{cs}{}{}{}{v.t.}{Combinar com oxigênio.}{o.xi.ge.nar}{0}
\verb{oxigenar}{cs}{Fig.}{}{}{}{Renovar o ar; estimular, revigorar.}{o.xi.ge.nar}{\verboinum{1}}
\verb{oxigênio}{cs}{Quím.}{}{}{s.m.}{Elemento químico do grupo dos não metais, gasoso em temperatura ambiente, incolor, inodoro, abundante na natureza, indispensável à vida e muito usado na indústria. \elemento{8}{15.9994}{O}.}{o.xi.gê.nio}{0}
\verb{oxigênio}{cs}{Por ext.}{}{}{}{Ar puro, não poluído.}{o.xi.gê.nio}{0}
\verb{oximoro}{cs\ldots{}ô}{Gram.}{}{}{s.m.}{Figura que consiste em se combinarem palavras de sentido oposto que parecem excluir"-se mutuamente; paradoxo.}{o.xi.mo.ro}{0}
\verb{oxítono}{cs}{Gram.}{}{}{adj.}{Diz"-se do vocábulo de duas sílabas ou mais cujo acento de tonicidade recai na última sílaba.}{o.xí.to.no}{0}
\verb{oxiúro}{cs}{Zool.}{}{}{s.m.}{Verme parasita do intestino do homem.}{o.xi.ú.ro}{0}
\verb{oxiurose}{cs\ldots{}ó}{Med.}{}{}{s.f.}{Doença parasitária causada por infestação de oxiúros.}{o.xi.u.ro.se}{0}
\verb{oxossi}{chó}{Relig.}{}{}{s.m.}{No candomblé, orixá da caça, irmão de Ogum.}{o.xos.si}{0}
\verb{oxum}{ch}{Relig.}{}{}{s.f.}{No candomblé, orixá feminino cuja encarnação no Brasil são as águas doces.}{o.xum}{0}
\verb{ozônio}{}{Quím.}{}{}{s.m.}{Substância gasosa azulada, constituída de três átomos de oxigênio, sendo uma variedade alotrópica deste; forma uma camada que protege a Terra contra as radiações ultravioleta do sol.}{o.zô.nio}{0}
\verb{ozonizar}{}{}{}{}{v.t.}{Fazer o ozônio agir sobre a água para esterilizá"-la.}{o.zo.ni.zar}{\verboinum{1}}
