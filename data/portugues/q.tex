\verb{q}{}{}{}{}{s.m.}{Décima sétima letra do alfabeto português.}{q}{0}
\verb{Q}{}{Mat.}{}{}{}{Símb. do conjunto dos números racionais. }{Q}{0}
\verb{QG}{}{}{}{}{}{Sigla de \textit{quartel"-general}.}{Q.G.}{0}
\verb{QI}{}{}{}{}{}{Sigla de \textit{quociente intelectual}, que é a proporção entre a inteligência de um indivíduo, determinada de acordo com alguma medida mental, e a inteligência normal ou média para sua idade; quociente de inteligência; coeficiente de inteligência.}{Q.I.}{0}
\verb{quacre}{}{}{}{}{s.m.}{Membro de um grupo religioso de tradição protestante, chamado Sociedade dos Amigos.}{qua.cre}{0}
\verb{quadra}{}{}{}{}{s.f.}{Série de quatro.}{qua.dra}{0}
\verb{quadra}{}{}{}{}{}{Área destinada à prática de esportes.}{qua.dra}{0}
\verb{quadra}{}{}{}{}{}{Quarteirão de casas.}{qua.dra}{0}
\verb{quadra}{}{}{}{}{}{A distância entre uma esquina e outra no mesmo lado de uma rua.}{qua.dra}{0}
\verb{quadra}{}{Liter.}{}{}{}{Compartimento ou terreno com a forma aproximada de um quadrilátero.}{qua.dra}{0}
\verb{quadra}{}{Liter.}{}{}{}{Estrofe de quatro versos; quarteto.}{qua.dra}{0}
\verb{quadra}{}{Fig.}{}{}{}{Ocasião, tempo, estação do ano, fase, idade, época da vida.}{qua.dra}{0}
\verb{quadra}{}{Bras.}{}{}{}{Medida linear equivalente a 132 metros.}{qua.dra}{0}
\verb{quadrado}{}{}{}{}{adj.}{Que tem quatro lados iguais formando ângulos retos.}{qua.dra.do}{0}
\verb{quadrado}{}{Fig.}{}{}{}{Que é preso aos padrões tradicionais; antiquado.}{qua.dra.do}{0}
\verb{quadrado}{}{Mat.}{}{}{s.m.}{O resultado de um número multiplicado por si mesmo.}{qua.dra.do}{0}
\verb{quadrado}{}{Geom.}{}{}{}{Figura geométrica cujos lados são iguais entre si e cujos ângulos são retos.}{qua.dra.do}{0}
\verb{quadragenário}{}{}{}{}{adj.}{Que tem quarenta unidades.}{qua.dra.ge.ná.rio}{0}
\verb{quadragenário}{}{}{}{}{adj.}{Que está na casa dos quarenta anos de idade; quarentão.}{qua.dra.ge.ná.rio}{0}
\verb{quadragenário}{}{}{}{}{s.m.}{Indivíduo que está nessa faixa etária. }{qua.dra.ge.ná.rio}{0}
\verb{quadragésima}{}{}{}{}{s.f.}{Período de quarenta dias.}{qua.dra.gé.si.ma}{0}
\verb{quadragésima}{}{Relig.}{}{}{}{Quaresma.}{qua.dra.gé.si.ma}{0}
\verb{quadragesimal}{}{}{"-ais}{}{adj.2g.}{Relativo à quadragésima ou quaresma; quaresmal.}{qua.dra.ge.si.mal}{0}
\verb{quadragésimo}{}{}{}{}{num.}{Ordinal e fracionário correspondente a 40.}{qua.dra.gé.si.mo}{0}
\verb{quadragésimo}{}{}{}{}{s.m.}{A quadragésima parte de um todo.}{qua.dra.gé.si.mo}{0}
%\verb{}{}{}{}{}{}{}{}{0}
\verb{quadrangular}{}{}{}{}{adj.2g.}{Que tem quatro ângulos.}{qua.dran.gu.lar}{0}
\verb{quadrangular}{}{}{}{}{s.m.}{Torneio esportivo com quatro participantes.}{qua.dran.gu.lar}{0}
\verb{quadrângulo}{}{Geom.}{}{}{s.m.}{Figura quadrangular, com quatro ângulos.}{qua.drân.gu.lo}{0}
\verb{quadrante}{}{Geom.}{}{}{s.m.}{Quarta parte de um círculo ou de sua circunferência.}{qua.dran.te}{0}
\verb{quadrante}{}{}{}{}{}{Mostrador de relógio.}{qua.dran.te}{0}
\verb{quadrar}{}{}{}{}{v.t.}{Dar forma quadrada.}{qua.drar}{0}
\verb{quadrar}{}{}{}{}{v.i.}{Ser conveniente, satisfatório; agradar, convir.}{qua.drar}{\verboinum{1}}
%\verb{}{}{}{}{}{}{}{}{0}
\verb{quadratim}{}{}{"-ins}{}{s.m.}{Em indústria gráfica, pequena barra de metal, mais baixa e do mesmo corpo que os tipos de imprensa, que serve para completar uma linha em branco.}{qua.dra.tim}{0}
\verb{quadratim}{}{}{"-ins}{}{}{Medida que corresponde ao número de pontos de um quadratim.}{qua.dra.tim}{0}
%\verb{}{}{}{}{}{}{}{}{0}
\verb{quadratura}{}{Geom.}{}{}{s.f.}{Processo de construir um quadrado equivalente a uma figura dada.}{qua.dra.tu.ra}{0}
\verb{quadratura}{}{Astron.}{}{}{}{Posição de dois astros em relação à Terra quando suas direções formam um ângulo reto.}{qua.dra.tu.ra}{0}
\verb{quadríceps}{}{Anat.}{}{}{adj.2g.}{Diz"-se de músculo da coxa.}{qua.drí.ceps}{0}
%\verb{}{}{}{}{}{}{}{}{0}
%\verb{}{}{}{}{}{}{}{}{0}
\verb{quadricolor}{ô}{}{}{}{adj.2g.}{Que tem quatro cores diferentes.}{qua.dri.co.lor}{0}
\verb{quadricórneo}{}{Zool.}{}{}{adj.}{Que tem quatro antenas ou cornos.}{qua.dri.cór.neo}{0}
\verb{quadrícula}{}{}{}{}{s.f.}{Pequeno quadrado ou retângulo.}{qua.drí.cu.la}{0}
\verb{quadrícula}{}{}{}{}{}{Pequena quadra, quadrículo.}{qua.drí.cu.la}{0}
\verb{quadriculado}{}{}{}{}{adj.}{Que é pautado ou dividido em quadrículas; quadricular.}{qua.dri.cu.la.do}{0}
\verb{quadricular}{}{}{}{}{adj.2g.}{Quadriculado.}{qua.dri.cu.lar}{0}
\verb{quadricular}{}{}{}{}{v.t.}{Dar a forma de quadrícula.}{qua.dri.cu.lar}{0}
\verb{quadricular}{}{}{}{}{}{Dividir em quadrículas.}{qua.dri.cu.lar}{\verboinum{1}}
\verb{quadrículo}{}{}{}{}{s.m.}{Pequeno quadrado. }{qua.drí.cu.lo}{0}
\verb{quadrículo}{}{}{}{}{}{Pequena quadra, quadrícula.}{qua.drí.cu.lo}{0}
\verb{quadridentado}{}{}{}{}{adj.}{Que tem quatro dentes ou quatro pontas.}{qua.dri.den.ta.do}{0}
\verb{quadridigitado}{}{Anat.}{}{}{adj.}{Diz"-se do membro terminado por quatro dedos.}{qua.dri.di.gi.ta.do}{0}
\verb{quadridigitado}{}{Biol.}{}{}{}{Diz"-se da folha cujo pecíolo termina por quatro folíolos.}{qua.dri.di.gi.ta.do}{0}
\verb{quadrienal}{}{}{"-ais}{}{adj.2g.}{Que aparece ou ocorre de quatro em quatro anos.}{qua.dri.e.nal}{0}
\verb{quadriênio}{}{}{}{}{s.m.}{Período de quatro anos.}{qua.dri.ê.nio}{0}
\verb{quadrifólio}{}{Bot.}{}{}{adj.}{Que tem quatro folhas.}{qua.dri.fó.lio}{0}
%\verb{}{}{}{}{}{}{}{}{0}
\verb{quadriga}{}{}{}{}{s.f.}{Antigo carro de quatro rodas puxado por quatro cavalos atrelados lado a lado.}{qua.dri.ga}{0}
\verb{quadrigêmeo}{}{}{}{}{adj.}{Referente a cada um dos quatro irmãos gêmeos, ou a todos eles.}{qua.dri.gê.meo}{0}
\verb{quadrigêmeo}{}{}{}{}{s.m.}{Cada um dos quatro irmãos gêmeos; quádruplo.}{qua.dri.gê.meo}{0}
\verb{quadril}{}{Anat.}{"-is}{}{s.m.}{Região lateral do corpo humano, da cintura até a extremidade superior da coxa; anca.}{qua.dril}{0}
\verb{quadril}{}{}{"-is}{}{}{Alcatra (de gado).}{qua.dril}{0}
%\verb{}{}{}{}{}{}{}{}{0}
\verb{quadrilátero}{}{}{}{}{adj.}{Que tem quatro lados.}{qua.dri.lá.te.ro}{0}
\verb{quadrilátero}{}{Mat.}{}{}{s.m.}{Figura plana de quatro lados.}{qua.dri.lá.te.ro}{0}
\verb{quadrilha}{}{}{}{}{s.f.}{Dança figurada executada por duas alas de pares, praticada sobretudo durante as festas juninas.}{qua.dri.lha}{0}
\verb{quadrilha}{}{}{}{}{}{Música para essa dança.}{qua.dri.lha}{0}
\verb{quadrilha}{}{}{}{}{}{Grupo de pessoas desonestas ou ladrões que obedecem a um chefe; súcia, corja.}{qua.dri.lha}{0}
%\verb{}{}{}{}{}{}{}{}{0}
\verb{quadrilongo}{}{}{}{}{adj.}{Que tem quatro lados, paralelos dois a dois, sendo dois deles mais longos que os outros dois. }{qua.dri.lon.go}{0}
\verb{quadrilongo}{}{}{}{}{s.m.}{Figura quadrilonga; retângulo.  }{qua.dri.lon.go}{0}
\verb{quadrimestral}{}{}{"-ais}{}{adj.2g.}{Relativo a quadrimestre.}{qua.dri.mes.tral}{0}
\verb{quadrimestral}{}{}{"-ais}{}{}{Que acontece ou se faz de quatro em quatro meses.}{qua.dri.mes.tral}{0}
\verb{quadrimestre}{é}{}{}{}{s.m.}{Período de quatro meses. }{qua.dri.mes.tre}{0}
\verb{quadrimotor}{ô}{}{}{}{adj. e s.m.  }{Diz"-se de ou aeroplano com quatro motores.}{qua.dri.mo.tor}{0}
\verb{quadringentésimo}{}{}{}{}{num.}{Ordinal e fracionário correspondente a 400.}{qua.drin.gen.té.si.mo}{0}
\verb{quadringentésimo}{}{}{}{}{}{Diz"-se de cada uma das quatrocentas partes em que um todo pode ser dividido.}{qua.drin.gen.té.si.mo}{0}
\verb{quadringentésimo}{}{}{}{}{s.m.}{A quadringentésima parte de um todo.}{qua.drin.gen.té.si.mo}{0}
\verb{quadrinho}{}{}{}{}{s.m.}{Cada um dos pequenos quadros que compõem uma história em quadrinhos.}{qua.dri.nho}{0}
\verb{quadrinhos}{}{}{}{}{s.m.pl.}{Sequência de desenhos figurativos, dispostos em pequenas molduras, que tem por objetivo contar uma história.; história em quadrinhos. }{qua.dri.nhos}{0}
%\verb{}{}{}{}{}{}{}{}{0}
\verb{quadro}{}{}{}{}{s.m.}{Obra de arte feita sobre uma superfície plana.}{qua.dro}{0}
\verb{quadro}{}{}{}{}{}{Divisão de uma peça teatral ou sudivisão de um ato com diferente cenário.}{qua.dro}{0}
\verb{quadro}{}{}{}{}{}{Quadro"-negro, lousa.}{qua.dro}{0}
\verb{quadro}{}{}{}{}{}{Lista ou conjunto dos membros de uma corporação, sociedade, empresa, repartição etc.}{qua.dro}{0}
\verb{quadro}{}{}{}{}{}{Descrição de um fato ou acontecimento; resenha. }{qua.dro}{0}
\verb{quadro}{}{Med.}{}{}{}{O conjunto dos sinais e sintomas apresentados por um doente.}{qua.dro}{0}
\verb{quadro"-de"-giz}{}{}{quadros"-de"-giz}{}{s.m.}{Quadro"-negro.}{qua.dro"-de"-giz}{0}
\verb{quadro"-negro}{ê}{}{quadros"-negros ⟨ê⟩}{}{s.m.}{Painel quadrangular de madeira ou ardósia, usado nas escolas, para anotações, cálculos, traçados, explicações do professor, exercícios dos alunos, entre outros; lousa; quadro"-de"-giz.}{qua.dro"-ne.gro}{0}
\verb{quadrúmano}{}{}{}{}{adj.}{Que tem quatro mãos.}{qua.drú.ma.no}{0}
\verb{quadrúpede}{}{}{}{}{adj.2g.}{Que tem quatro pés.}{qua.drú.pe.de}{0}
\verb{quadrúpede}{}{Zool.}{}{}{s.m.}{Mamífero que tem quatro patas.}{qua.drú.pe.de}{0}
\verb{quadrúpede}{}{Fig.}{}{}{s.2g.}{Indivíduo ignorante, estúpido, néscio.}{qua.drú.pe.de}{0}
\verb{quadruplicar}{}{}{}{}{v.t.}{Multiplicar por quatro; tornar quatro vezes maior.}{qua.dru.pli.car}{\verboinum{2}}
\verb{quádruplo}{}{}{}{}{num.}{Que vale quatro vezes mais que outro.}{quá.dru.plo}{0}
\verb{quádruplo}{}{}{}{}{s.m.}{Número quádruplo de outro; quantidade quatro vezes maior que outra.}{quá.dru.plo}{0}
\verb{quádruplo}{}{}{}{}{}{Cada um dos quatro irmãos gêmeos; quadrigêmeo.}{quá.dru.plo}{0}
\verb{qual}{}{Gram.}{"-ais}{}{pron.}{Nas orações interrogativas, indaga sobre que pessoa ou que coisa, dentre duas ou mais, designando quantidade ou natureza.}{qual}{0}
\verb{qual}{}{}{"-ais}{}{}{Refere"-se a algo ou alguém citado anteriormente. (nesta acepção sempre precedido do artigo definido)}{qual}{0}
\verb{qual}{}{}{"-ais}{}{conj.}{Indica comparação; como, assim como, tal qual.}{qual}{0}
\verb{qual}{}{}{"-ais}{}{interj.}{Exprime dúvida ou negação. }{qual}{0}
\verb{qualidade}{}{}{}{}{}{Em uma escala de valores, o que torna alguma coisa superior à média; superioridade, excelência de alguém ou de algo.}{qua.li.da.de}{0}
\verb{qualidade}{}{}{}{}{s.f.}{Característica de alguém ou algo; conjunto das modalidades sob as quais se apresenta; propriedade; atributo.}{qua.li.da.de}{0}
\verb{qualidade}{}{}{}{}{}{Disposição moral ou intelectual.}{qua.li.da.de}{0}
\verb{qualidade}{}{}{}{}{}{Condição social, civil, jurídica; espécie, casta, laia.}{qua.li.da.de}{0}
\verb{qualificação}{}{}{"-ões}{}{s.f.}{Ato ou efeito de qualificar; atribuir uma qualidade, um título; classificar.}{qua.li.fi.ca.ção}{0}
\verb{qualificação}{}{}{"-ões}{}{}{Conjunto de condições referentes à formação profissional e experiência, necessário para o exercício de um cargo ou de uma função; habilitação.}{qua.li.fi.ca.ção}{0}
\verb{qualificado}{}{}{}{}{adj.}{Que tem certas qualidades, determinados conhecimentos ou atributos; apto.}{qua.li.fi.ca.do}{0}
\verb{qualificado}{}{Jur.}{}{}{}{Diz"-se de crime especialmente agravado em razão de certas circunstâncias expressas em lei.}{qua.li.fi.ca.do}{0}
\verb{qualificar}{}{}{}{}{}{Atribuir qualidade a; apreciar; classificar, avaliar, considerar.}{qua.li.fi.car}{0}
\verb{qualificar}{}{}{}{}{v.t.}{Considerar qualificado, apto.}{qua.li.fi.car}{0}
\verb{qualificar}{}{}{}{}{}{Autorizar.}{qua.li.fi.car}{\verboinum{2}}
\verb{qualificativo}{}{}{}{}{adj.}{Que qualifica ou serve para qualificar; qualitativo.}{qua.li.fi.ca.ti.vo}{0}
%\verb{}{}{}{}{}{}{}{}{0}
\verb{qualitativo}{}{}{}{}{adj.}{Relativo a qualidade; qualificativo.}{qua.li.ta.ti.vo}{0}
\verb{qualitativo}{}{}{}{}{}{Que exprime ou determina qualidade.}{qua.li.ta.ti.vo}{0}
\verb{qualquer}{é}{}{quaisquer ⟨é⟩}{}{pron.}{Designa coisa, lugar ou indivíduo indeterminado; algum.}{qual.quer}{0}
\verb{qualquer}{é}{}{quaisquer ⟨é⟩}{}{}{Um ou outro; este, esse ou aquele.}{qual.quer}{0}
\verb{qualquer}{é}{Fig.}{quaisquer ⟨é⟩}{}{}{Comum, simples, mero, sem importância, indiferente.}{qual.quer}{0}
\verb{quando}{}{}{}{}{adv.}{Em que tempo, em que ocasião.}{quan.do}{0}
\verb{quando}{}{}{}{}{conj.}{No momento em que, no tempo em que.}{quan.do}{0}
\verb{quando}{}{}{}{}{}{Ainda que, mesmo que.}{quan.do}{0}
\verb{quantia}{}{}{}{}{s.f.}{Soma, total ou quantidade de dinheiro; importância.}{quan.ti.a}{0}
\verb{quantidade}{}{}{}{}{s.f.}{Qualidade daquilo que é capaz de aumento ou de diminuição.}{quan.ti.da.de}{0}
\verb{quantidade}{}{}{}{}{}{Grande porção de pessoas ou coisas que podem ser contadas, pesadas, medidas; abundância.}{quan.ti.da.de}{0}
\verb{quantificação}{}{}{"-ões}{}{s.f.}{Ato ou efeito de quantificar.}{quan.ti.fi.ca.ção}{0}
\verb{quantificação}{}{Fís.}{"-ões}{}{}{A passagem da descrição clássica e contínua de um sistema para a descrição quântica. }{quan.ti.fi.ca.ção}{0}
\verb{quantificado}{}{}{}{}{adj.}{Em que se determinou a quantidade.}{quan.ti.fi.ca.do}{0}
\verb{quantificar}{}{}{}{}{v.t.}{Determinar a quantidade ou o valor de.}{quan.ti.fi.car}{0}
\verb{quantificar}{}{Fís.}{}{}{}{Efetuar a quantificação de um sistema.}{quan.ti.fi.car}{\verboinum{2}}
\verb{quantitativo}{}{}{}{}{adj.}{Que se refere a quantidade.}{quan.ti.ta.ti.vo}{0}
\verb{quantitativo}{}{}{}{}{}{Que indica quantidades numéricas.}{quan.ti.ta.ti.vo}{0}
\verb{quanto}{}{}{}{}{pron.}{Que quantidade; que quantia.}{quan.to}{0}
\verb{quanto}{}{}{}{}{}{Tudo que; tudo quanto.}{quan.to}{0}
\verb{quanto}{}{}{}{}{adv.}{Com que intensidade; quão intensamente.}{quan.to}{0}
\verb{quanto}{}{}{}{}{loc. adv.}{(\textit{quanto a}) Em relação a; com referência a.}{quan.to}{0}
\verb{quantum}{}{Fís.}{quanta}{}{s.m.}{Quantidade unitária de energia ou de matéria, determinada como sua menor porção.}{\textit{quantum}}{0}
\verb{quão}{}{}{}{}{adv.}{Exprime intensidade; quanto, como.}{quão}{0}
\verb{quarador}{ô}{}{}{}{s.m.}{Lugar onde se coloca a roupa para branquear, limpar, expondo"-a ao sol; coradouro.}{qua.ra.dor}{0}
\verb{quarar}{}{Pop.}{}{}{v.t.}{Clarear roupa ao sol; corar.}{qua.rar}{\verboinum{1}}
\verb{quarenta}{}{}{}{}{num.}{Nome dado à quantidade expressa pelo número 40.}{qua.ren.ta}{0}
\verb{quarentão}{}{Pop.}{"-ões}{quarentona}{adj.}{Diz"-se do indivíduo que completou quarenta anos de idade ou que está nessa faixa etária; quadragenário.}{qua.ren.tão}{0}
\verb{quarentão}{}{}{"-ões}{quarentona}{}{Indivíduo que está nessa faixa etária. }{qua.ren.tão}{0}
\verb{quarentena}{}{}{}{}{s.f.}{Período de 40 dias; quaresma.}{qua.ren.te.na}{0}
\verb{quarentena}{}{Med.}{}{}{}{Espaço de tempo em que a pessoa  fica isolada para observação de possíveis doenças contagiosas.}{qua.ren.te.na}{0}
\verb{quarentena}{}{}{}{}{}{Período de abstinência sexual.}{qua.ren.te.na}{0}
\verb{quaresma}{é}{Relig.}{}{}{s.f.}{Período de 40 dias que se inicia na quarta"-feira de Cinzas e segue até o domingo de Páscoa, durante o qual católicos e ortodoxos devem cumprir penitência, abstendo"-se de comer carne, dentre outros preceitos; quarentena.}{qua.res.ma}{0}
\verb{quaresmal}{}{}{"-ais}{}{adj.2g.}{Relativo a quaresma.}{qua.res.mal}{0}
\verb{quaresmeira}{ê}{Bot.}{}{}{s.f.}{Árvore ornamental própria de parques e jardins, cujas flores, roxas ou púrpuras, desabrocham no período da quaresma.}{qua.res.mei.ra}{0}
\verb{quark}{}{Fís.}{}{}{s.m.}{Partícula elementar da qual são formados os prótons e os nêutrons.}{\textit{quark}}{0}
\verb{quarta}{}{}{}{}{s.f.}{Uma das quatro partes iguais em que se pode dividir qualquer unidade.     }{quar.ta}{0}
\verb{quartã}{}{Med.}{}{}{adj.}{Diz"-se da febre intermitente que se repete de quatro em quatro dias, como sintoma da malária.}{quar.tã}{0}
\verb{quarta}{}{}{}{}{}{Cântaro, vaso de barro, bilha.}{quar.ta}{0}
\verb{quarta}{}{}{}{}{}{Forma reduzida de \textit{quarta"-feira}.  }{quar.ta}{0}
\verb{quarta"-de"-final}{}{Esport.}{quartas"-de"-final}{}{s.f.}{Num torneio disputado por eliminação, etapa em que se realizam quatro jogos, com oito times buscando a classificação às semifinais. }{quar.ta"-de"-fi.nal}{0}
\verb{quarta"-feira}{ê}{}{quartas"-feiras}{}{s.f.}{O quarto dia da semana.}{quar.ta"-fei.ra}{0}
\verb{quartanista}{}{}{}{}{adj.2g.}{Diz"-se do estudante que frequenta o quarto ano de uma escola, especialmente em faculdades.}{quar.ta.nis.ta}{0}
\verb{quarteirão}{}{}{"-ões}{}{s.m.}{Conjunto de edificações que forma um quadrilátero de que cada um dos lados dá para uma rua; quadra.}{quar.tei.rão}{0}
\verb{quarteirão}{}{}{"-ões}{}{}{A quarta parte de cem; vinte e cinco.}{quar.tei.rão}{0}
\verb{quartel}{é}{}{"-éis}{}{s.m.}{Edifício destinado ao alojamento de tropas de soldados; caserna.}{quar.tel}{0}
\verb{quartel}{é}{}{"-éis}{}{s.m.}{A quarta parte de um todo; quarta.}{quar.tel}{0}
\verb{quartel}{é}{Fig.}{"-éis}{}{}{Espaço de tempo; período, época.}{quar.tel}{0}
\verb{quartelada}{}{}{}{}{s.f.}{Rebelião ou motim militar, sem base ideológica e social, que visa à tomada do poder.}{quar.te.la.da}{0}
\verb{quarteleiro}{ê}{}{}{}{s.m.}{Soldado encarregado de guardar o armamento e os uniformes de um corpo de tropas.}{quar.te.lei.ro}{0}
\verb{quartel"-general}{é}{}{quartéis"-generais\textit{] [Sigla: \textsc{q.g}.}}{}{s.m.}{Repartição militar dirigida por um oficial general e destinada a executar, transmitir e fazer cumprir ordens do ministro, quanto ao movimento, organização e disciplina militar.}{quar.tel"-ge.ne.ral}{0}
\verb{quartel"-general}{é}{}{quartéis"-generais\textit{] [Sigla: \textsc{q.g}.}}{}{}{O lugar ocupado pelos oficiais generais e seu estado"-maior.}{quar.tel"-ge.ne.ral}{0}
\verb{quartel"-general}{é}{}{quartéis"-generais\textit{] [Sigla: \textsc{q.g}.}}{}{}{A casa ou lugar de residência do general e de onde ele expede as ordens aos corpos que lhe estão subordinados.}{quar.tel"-ge.ne.ral}{0}
\verb{quartel"-general}{é}{Fig.}{quartéis"-generais\textit{] [Sigla: \textsc{q.g}.}}{}{}{Lugar de reuniões, de abrigo.}{quar.tel"-ge.ne.ral}{0}
\verb{quarteto}{ê}{}{}{}{s.m.}{Estrofe de quatro versos; quadra.}{quar.te.to}{0}
\verb{quarteto}{ê}{Mús.}{}{}{}{Música composta para quarteto de cordas.}{quar.te.to}{0}
\verb{quarteto}{ê}{Mús.}{}{}{}{Conjunto formado por dois violinos, viola e violoncelo.}{quar.te.to}{0}
\verb{quarteto}{ê}{Pop.}{}{}{}{Reunião ou conjunto de quatro pessoas.}{quar.te.to}{0}
\verb{quartinha}{}{}{}{}{s.f.}{Recipiente de barro usado para guardar água fresca; moringa.}{quar.ti.nha}{0}
\verb{quartinho}{}{}{}{}{s.m.}{Quarto pequeno; cubículo.}{quar.ti.nho}{0}
\verb{quartinho}{}{Bras.}{}{}{}{Privada, latrina.}{quar.ti.nho}{0}
\verb{quarto}{}{}{}{}{num.}{Ordinal e fracionário correspondente a 4. (\textit{O candidato ficou em quarto lugar. })}{quar.to}{0}
\verb{quarto}{}{}{}{}{s.m.}{Cada um dos cômodos da casa que servem para dormir; aposento.  }{quar.to}{0}
\verb{quarto}{}{Astron.}{}{}{}{A quarta parte do período gasto pela Lua para dar uma volta completa em torno da Terra.}{quar.to}{0}
\verb{quartos}{}{Pop.}{}{}{s.m.pl.}{Parte do corpo entre o alto da coxa e a cintura; quadris, ancas.}{quar.tos}{0}
\verb{quartzo}{}{Geol.}{}{}{s.m.}{Óxido de silício cristalizado, mineral geralmente branco ou transparente, encontrado em numerosas rochas.}{quar.tzo}{0}
\verb{quasar}{}{Astron.}{}{}{s.m.}{Fonte de rádio de origem cósmica, com aparência estelar, que emite ondas de rádio superiores à de qualquer galáxia.}{qua.sar}{0}
\verb{quase}{}{}{}{}{adv.}{Aproximadamente; perto.}{qua.se}{0}
\verb{quase}{}{}{}{}{}{Por pouco; por um triz.}{qua.se}{0}
\verb{quase}{}{}{}{}{}{Pouco menos; quando muito.}{qua.se}{0}
\verb{quássia}{}{Bot.}{}{}{s.f.}{Árvore nativa de regiões tropicais, cuja madeira, extremamente amarga, é utilizada para combater males do estômago.}{quás.sia}{0}
\verb{quaternário}{}{}{}{}{adj.}{Formado por quatro unidades.}{qua.ter.ná.rio}{0}
\verb{quaternário}{}{}{}{}{}{Que tem quatro lados ou faces.}{qua.ter.ná.rio}{0}
\verb{quaternário}{}{Geol.}{}{}{}{Diz"-se do período geológico atual, da era cenozoica, que compreende as épocas pleistocena e holocena.}{qua.ter.ná.rio}{0}
\verb{quati}{}{Zool.}{}{}{s.m.}{Pequeno mamífero carnívoro, nativo da América do Sul, de focinho longo e anéis escuros na cauda, que vive em árvores.}{qua.ti}{0}
\verb{quatorze}{ô}{}{}{}{}{Var. de \textit{catorze}.}{qua.tor.ze}{0}
\verb{quatriênio}{}{}{}{}{}{Var. de \textit{quadriênio}.}{qua.tri.ê.nio}{0}
\verb{quatrilhão}{}{}{"-ões}{}{num.}{Mil trilhões.}{qua.tri.lhão}{0}
\verb{quatrilião}{}{}{}{}{}{Var. de \textit{quatrilhão}.}{qua.tri.li.ão}{0}
\verb{quatro}{}{}{}{}{num.}{Nome dado à quantidade expressa pelo número 4.}{qua.tro}{0}
\verb{quatrocentão}{}{Bras.}{"-ões}{}{adj.}{Que tem quatrocentos anos; tradicional.}{qua.tro.cen.tão}{0}
\verb{quatrocentos}{}{}{}{}{num.}{Nome dado à quantidade expressa pelo número 400.}{qua.tro.cen.tos}{0}
\verb{quatro"-olhos}{ó}{Pop.}{}{}{adj.2g.}{Diz"-se da pessoa que usa óculos.}{qua.tro"-o.lhos}{0}
\verb{que}{}{}{}{}{pron.}{Qual coisa; o que. (\textit{Que fazes aí?})}{que}{0}
\verb{quê}{}{}{}{}{s.m.}{Nome da letra \textit{q}.}{quê}{0}
\verb{quê}{}{}{}{}{s.m.}{Alguma coisa; qualquer coisa.}{quê}{0}
\verb{quê}{}{}{}{}{interj.}{Expressão que denota espanto, surpresa, incredulidade.}{quê}{0}
\verb{que}{}{}{}{}{conj.}{Palavra que começa uma frase em que se afirma alguma coisa. (\textit{Insista que o diretor o atende.})}{que}{0}
\verb{que}{}{}{}{}{}{Qual pessoa ou coisa. (\textit{Que presente quer ganhar?})}{que}{0}
\verb{que}{}{}{}{}{}{O qual. (\textit{Comprei o livro que estava querendo.})}{que}{0}
\verb{que}{}{}{}{}{adv.}{Em que medida; como; tão; quão. (\textit{Que belo trabalho!})}{que}{0}
\verb{que}{}{}{}{}{}{Palavra que começa uma frase em que se deseja alguma coisa. (\textit{Que seja feliz!})}{que}{0}
\verb{quebra}{é}{}{}{}{s.f.}{Ato ou efeito de quebrar, fragmentar.}{que.bra}{0}
\verb{quebra}{é}{}{}{}{}{Dobradura em tecido ou papel; vinco, dobra.}{que.bra}{0}
\verb{quebra}{é}{}{}{}{}{Interrupção, rompimento, ruptura.}{que.bra}{0}
\verb{quebra}{é}{}{}{}{}{Transgressão de regulamento; violação, infração.}{que.bra}{0}
\verb{quebra}{é}{}{}{}{}{O que se dá além do combinado numa transação.}{que.bra}{0}
\verb{quebra"-cabeça}{é\ldots{}ê}{}{quebra"-cabeças ⟨é\ldots{}ê⟩}{}{s.m.}{Aquilo que preocupa, que é complicado.}{que.bra"-ca.be.ça}{0}
\verb{quebra"-cabeça}{é\ldots{}ê}{}{quebra"-cabeças ⟨é\ldots{}ê⟩}{}{}{Tarefa ou problema de difícil resolução.}{que.bra"-ca.be.ça}{0}
\verb{quebra"-cabeça}{é\ldots{}ê}{}{quebra"-cabeças ⟨é\ldots{}ê⟩}{}{}{Jogo que consiste em juntar peças diferentes que estão dispersas com o fim de montar uma figura.}{que.bra"-ca.be.ça}{0}
\verb{quebrada}{}{}{}{}{s.f.}{Declive de montanha; ladeira, encosta.}{que.bra.da}{0}
\verb{quebrada}{}{}{}{}{}{Curva de caminho; volta, capão.}{que.bra.da}{0}
\verb{quebrada}{}{Pop.}{}{}{}{Região ou bairro mal urbanizado de uma cidade.}{que.bra.da}{0}
\verb{quebradeira}{ê}{Bras.}{}{}{s.f.}{Falta de dinheiro; pindaíba.}{que.bra.dei.ra}{0}
\verb{quebradeira}{ê}{}{}{}{}{Falência em massa.}{que.bra.dei.ra}{0}
\verb{quebradeira}{ê}{}{}{}{}{Sensação de moleza; cansaço físico; prostração.}{que.bra.dei.ra}{0}
\verb{quebradiço}{}{}{}{}{adj.}{Que é fácil de quebrar, frágil. }{que.bra.di.ço}{0}
\verb{quebradiço}{}{}{}{}{}{Débil, tênue.}{que.bra.di.ço}{0}
\verb{quebrado}{}{}{}{}{adj.}{Que foi feito em pedaços, partido, separado.}{que.bra.do}{0}
\verb{quebrado}{}{}{}{}{}{Que quebrou; falido, arruinado, sem dinheiro. }{que.bra.do}{0}
\verb{quebrado}{}{Fig.}{}{}{}{Desalentado, abatido, cansado, desfalecido. }{que.bra.do}{0}
\verb{quebrado}{}{}{}{}{}{Fraco, magoado. }{que.bra.do}{0}
\verb{quebrado}{}{}{}{}{}{Que sofre de hérnia intestinal ou quebradura; rendido.  }{que.bra.do}{0}
\verb{quebrado}{}{}{}{}{}{Violado por trangressão; infringido.}{que.bra.do}{0}
\verb{quebradura}{}{}{}{}{s.f.}{Ato ou efeito de quebrar; quebra.}{que.bra.du.ra}{0}
\verb{quebradura}{}{}{}{}{}{Hérnia. }{que.bra.du.ra}{0}
\verb{quebra"-galho}{é}{}{quebra"-galhos ⟨é⟩}{}{s.m.}{Pessoa, coisa ou recurso que ajuda a encontrar a solução para uma dificuldade.}{que.bra"-ga.lho}{0}
\verb{quebra"-gelos}{é\ldots{}ê}{}{}{}{s.m.}{Navio destinado a abrir caminho em águas cobertas de gelo.}{que.bra"-ge.los}{0}
\verb{quebra"-luz}{é}{}{quebra"-luzes ⟨é⟩}{}{s.m.}{1. Peça para preservar os olhos da luz forte de vela, candeeiro ou lâmpada; a cúpula do abajur.}{que.bra"-luz}{0}
\verb{quebra"-mar}{é}{}{quebra"-mares ⟨é⟩}{}{s.m.}{Muralha ou barreira natural destinada a oferecer resistência ao embate das ondas ou à força das correntes marítimas.}{que.bra"-mar}{0}
\verb{quebra"-molas}{é\ldots{}ó}{Bras.}{}{}{s.m.}{Obstáculo colocado nas ruas ou estradas para forçar o motorista a diminuir a velocidade do veículo; tartaruga, lombada.}{que.bra"-mo.las}{0}
\verb{quebra"-nozes}{é\ldots{}ó}{}{}{}{s.m.}{Instrumento de metal em forma de pinça ou alicate utilizado para quebrar a casca de nozes, amêndoas etc.}{que.bra"-no.zes}{0}
\verb{quebra"-nozes}{é\ldots{}ó}{Zool.}{}{}{}{Nome dado a diversos pássaros, de bico resistente, que se alimentam de nozes, e habitam a Ásia e a Europa.}{que.bra"-no.zes}{0}
\verb{quebrantar}{}{}{}{}{v.t.}{Quebrar, abater, arrasar.}{que.bran.tar}{0}
\verb{quebrantar}{}{Fig.}{}{}{}{Infringir, violar, ultrapassar.}{que.bran.tar}{0}
\verb{quebrantar}{}{}{}{}{}{Enfraquecer, debilitar.}{que.bran.tar}{0}
\verb{quebrantar}{}{}{}{}{v.i.}{Acalmar, suavizar, abrandar.}{que.bran.tar}{0}
\verb{quebrantar}{}{}{}{}{}{Sofrer a ação do quebranto.}{que.bran.tar}{\verboinum{1}}
\verb{quebranto}{}{}{}{}{s.m.}{Abatimento, enfraquecimento, fraqueza, prostração.}{que.bran.to}{0}
\verb{quebranto}{}{}{}{}{}{Estado de indisposição, atribuído pela crendice popular ao olhar de algumas pessoas; mau"-olhado.}{que.bran.to}{0}
\verb{quebra"-pau}{é}{Pop.}{quebra"-paus ⟨é⟩}{}{s.m.}{Discussão violenta, briga, conflito.}{que.bra"-pau}{0}
\verb{quebra"-pedra}{é\ldots{}é}{Bot.}{quebra"-pedras}{}{s.f.}{Erva com propriedades medicinais, utilizada como diurético e para dissolver cálculos renais.}{que.bra"-pe.dra}{0}
\verb{quebra"-quebra}{é\ldots{}é}{}{quebra"-quebras ⟨é\ldots{}é⟩}{}{s.m.}{Desordem popular, tumulto, confusão, arruaça, em que há depredações.}{que.bra"-que.bra}{0}
\verb{quebra"-quebra}{é\ldots{}é}{Cul.}{quebra"-quebras ⟨é\ldots{}é⟩}{}{}{Tipo de sequilho muito macio que derrete na boca.}{que.bra"-que.bra}{0}
\verb{quebra"-queixo}{é\ldots{}ch}{Bras.}{quebra"-queixos ⟨é\ldots{}ch⟩}{}{s.m.}{Doce de consistência elástica e grudenta; puxa"-puxa.}{que.bra"-quei.xo}{0}
\verb{quebrar}{}{}{}{}{v.t.}{Reduzir a pedaços; partir, romper. (\textit{A criança derrubou o vaso e o quebrou.})}{que.brar}{0}
\verb{quebrar}{}{}{}{}{}{Fazer algo contra; infringir, transgredir, violar. (\textit{O jogador ficou na reserva por ter quebrado algumas regras.})}{que.brar}{0}
\verb{quebrar}{}{}{}{}{}{Interromper. (\textit{Sua risada quebrou o clima tenso a sala.})}{que.brar}{0}
\verb{quebrar}{}{}{}{}{v.i.}{Deixar de funcionar. (\textit{A máquina de lavar roupas quebrou.})}{que.brar}{\verboinum{1}}
\verb{quebra"-vento}{é}{}{quebra"-ventos ⟨é⟩}{}{s.m.}{Pequena janela móvel do vidro lateral dianteiro de certos veículos que serve para desviar o vento para onde se quer.}{que.bra"-ven.to}{0}
\verb{quebreira}{ê}{Pop.}{}{}{s.f.}{Canseira, moleza, prostração, quebradeira.}{que.brei.ra}{0}
\verb{queda}{é}{}{}{}{s.f.}{Ato de cair; caída, tombo.  }{que.da}{0}
\verb{queda"-d'água}{é}{}{quedas"-d'água ⟨é⟩}{}{s.f.}{Lugar em que a queda de um rio é acentuadamente vertical; cachoeira, cascata, catarata, salto.}{que.da"-d'á.gua}{0}
\verb{queda"-de"-braço}{é}{}{quedas"-de"-braço ⟨é⟩}{}{s.m.}{Jogo para medir força, em que cada um dos dois competidores, de mãos dadas, tenta encostar o antebraço do outro na superfície sobre a qual apoiam o cotovelo; braço"-de"-ferro.}{que.da"-de"-bra.ço}{0}
\verb{queda"-de"-braço}{é}{Fig.}{quedas"-de"-braço ⟨é⟩}{}{}{Luta, embate.}{que.da"-de"-bra.ço}{0}
\verb{quedar}{}{}{}{}{v.i.}{Ficar, permanecer; deter"-se.    }{que.dar}{\verboinum{1}}
\verb{quede}{ê/ ou /é}{}{}{}{adv.}{Expressão interrogativa equivalente a \textit{cadê, onde está}; quedê. }{que.de}{0}
\verb{quedê}{}{}{}{}{}{Var. de \textit{quede}.}{que.dê}{0}
\verb{quedo}{ê}{}{}{}{adj.}{Quieto, sem movimento.}{que.do}{0}
\verb{quedo}{ê}{}{}{}{}{Sossegado, tranquilo, sereno, calmo.  }{que.do}{0}
\verb{quefazeres}{ê}{}{}{}{s.m.pl.}{Ocupações, negócios; afazeres.  }{que.fa.ze.res}{0}
\verb{queijada}{}{Cul.}{}{}{s.m.}{Pequena torta achatada feita de farinha de trigo com recheio de queijo, coco, ovos e açúcar.}{quei.ja.da}{0}
\verb{queijadinha}{}{Cul.}{}{}{s.f.}{Espécie de doce feito de farinha de trigo com recheio de queijo, coco, ovos e açúcar, assada em pequenas formas.        }{quei.ja.di.nha}{0}
\verb{queijaria}{}{}{}{}{s.f.}{Casa ou lugar onde se fabricam queijos; queijeira.      }{quei.ja.ri.a}{0}
\verb{queijeira}{ê}{}{}{}{s.f.}{Queijaria.}{quei.jei.ra}{0}
\verb{queijeira}{ê}{}{}{}{}{Mulher que faz ou vende queijos.  }{quei.jei.ra}{0}
\verb{queijeiro}{ê}{}{}{}{s.m.}{Indivíduo que faz ou vende queijos.  }{quei.jei.ro}{0}
\verb{queijo}{ê}{}{}{}{s.m.}{Alimento feito a partir do leite coalhado.     }{quei.jo}{0}
\verb{queijo"-de"-minas}{}{Bras.}{queijos"-de"-minas}{}{s.m.}{Queijo cilíndrico, esbranquiçado, de massa crua homogênea e consistência variável, muito consumido no Brasil.}{quei.jo"-de"-mi.nas}{0}
\verb{queima}{}{}{}{}{s.f.}{Ato ou efeito de queimar.   }{quei.ma}{0}
\verb{queima}{}{}{}{}{}{Venda a preços muitos baixos; liquidação.    }{quei.ma}{0}
\verb{queimação}{}{}{"-ões}{}{s.f.}{Ato ou efeito de queimar.}{quei.ma.ção}{0}
\verb{queimação}{}{}{"-ões}{}{}{Coisa que molesta; impertinência. }{quei.ma.ção}{0}
\verb{queimação}{}{}{"-ões}{}{}{Azia.   }{quei.ma.ção}{0}
\verb{queimada}{}{}{}{}{s.f.}{Queima de mato.}{quei.ma.da}{0}
\verb{queimada}{}{}{}{}{}{Terreno em que se queimou o mato.}{quei.ma.da}{0}
\verb{queimada}{}{}{}{}{}{Brincadeira de criança, entre duas equipes, que consiste em atirar a bola para bater em um dos adversários, que, ao ser atingido e não agarrar a bola, deve sair do jogo.  }{quei.ma.da}{0}
\verb{queimado}{}{}{}{}{adj.}{Que se queimou.}{quei.ma.do}{0}
\verb{queimado}{}{Pop.}{}{}{}{Zangado, aborrecido. (\textit{O rapaz ficou queimado com a brincadeira de mal gosto.})}{quei.ma.do}{0}
\verb{queimado}{}{Pop.}{}{}{}{Bronzeado pelo sol. (\textit{As crianças estão queimadas de sol.})}{quei.ma.do}{0}
\verb{queimador}{ô}{}{}{}{adj.}{Que queima.}{quei.ma.dor}{0}
\verb{queimador}{ô}{}{}{}{s.m.}{Cada uma das bocas em que se queima o gás de um fogão; bico.  }{quei.ma.dor}{0}
\verb{queimadura}{}{}{}{}{s.f.}{Ferimento causado pelo fogo.  }{quei.ma.du.ra}{0}
\verb{queimar}{}{}{}{}{}{Vender alguma mercadoria por preço muito baixo.}{quei.mar}{0}
\verb{queimar}{}{}{}{}{}{Machucar o corpo com alguma coisa quente.}{quei.mar}{0}
\verb{queimar}{}{}{}{}{v.t.}{Destruir com o fogo.}{quei.mar}{0}
\verb{queimar}{}{Fig.}{}{}{}{Fazer alguém perder o prestígio; fritar.}{quei.mar}{\verboinum{1}}
\verb{queima"-roupa}{ê\ldots{}ô}{}{}{}{s.f.}{Usado na locução \textit{à queima"-roupa}: de muito perto; cara"-a"-cara.}{quei.ma"-rou.pa}{0}
\verb{queixa}{ch}{}{}{}{s.f.}{Ato de se queixar; reclamação.}{quei.xa}{0}
\verb{queixa"-crime}{ch}{Jur.}{queixas"-crimes \textit{ou} queixas"-crime ⟨ch⟩}{}{s.f.}{Petição com que se inicia um processo por ofensa.}{quei.xa"-cri.me}{0}
\verb{queixada}{ch}{}{}{}{s.f.}{Mandíbula.}{quei.xa.da}{0}
\verb{queixada}{ch}{}{}{}{}{Queixo grande, proeminente.}{quei.xa.da}{0}
\verb{queixada}{ch}{Zool.}{}{}{s.m.}{Mamífero de pelagem avermelhada, semelhante ao javali.}{quei.xa.da}{0}
\verb{queixal}{ch}{}{"-ais}{}{adj.2g.}{Que se refere ou pertence ao queixo.}{quei.xal}{0}
\verb{queixal}{ch}{}{"-ais}{}{s.m.}{Dente molar.}{quei.xal}{0}
\verb{queixar"-se}{ch}{}{}{}{v.pron.}{Mostrar a dor ou o desgosto que sente; lamuriar"-se, lastimar"-se.}{quei.xar"-se}{0}
\verb{queixar"-se}{ch}{}{}{}{}{Falar a outra pessoa sobre a dor ou o desgosto que sente; reclamar.}{quei.xar"-se}{\verboinum{1}}
\verb{queixo}{ch}{Anat.}{}{}{s.m.}{Parte do rosto abaixo do lábio inferior; mento.}{quei.xo}{0}
\verb{queixoso}{chô}{}{"-osos ⟨chó⟩}{"-osa ⟨chó⟩}{adj.}{Em que há queixa.}{quei.xo.so}{0}
\verb{queixoso}{chô}{}{"-osos ⟨chó⟩}{"-osa ⟨chó⟩}{s.m.}{Indivíduo que vai reclamar a uma autoridade.}{quei.xo.so}{0}
\verb{queixudo}{ch}{Pop.}{}{}{adj.}{Que tem o queixo grande ou proeminente.}{quei.xu.do}{0}
\verb{queixume}{ch}{}{}{}{s.m.}{Queixa muito triste; lamentação, lamento, lamúria.}{quei.xu.me}{0}
\verb{quejando}{}{}{}{}{pron.}{Da mesma natureza ou qualidade.}{que.jan.do}{0}
\verb{quejando}{}{}{}{}{}{Semelhante a outrem.}{que.jan.do}{0}
\verb{quejando}{}{}{}{}{}{Que tal.}{que.jan.do}{0}
\verb{quelídeo}{}{Zool.}{}{}{s.m.}{Espécime dos quelídeos, família de quelônios de água doce cuja cabeça e pescoço podem ser mais longos que a carapaça.}{que.lí.deo}{0}
\verb{quelídeo}{}{}{}{}{adj.}{Relativo aos quelídeos.}{que.lí.deo}{0}
\verb{quelônio}{}{}{}{}{s.m.}{Espécime dos quelônios, ordem de répteis de casco rígido, que inclui as tartarugas, os cágados e os jabutis.  }{que.lô.nio}{0}
\verb{quelônio}{}{Zool.}{}{}{adj.}{Relativo aos quelônios. }{que.lô.nio}{0}
\verb{quem}{}{}{}{}{pron.}{Que pessoa(s).    (\textit{Quem vem para o jantar?})}{quem}{0}
\verb{quem}{}{}{}{}{}{O qual. (\textit{A escritora respondeu para quem lhe enviou cartas.})}{quem}{0}
\verb{quem}{}{}{}{}{}{Qualquer pessoa.}{quem}{0}
\verb{quendô}{}{Esport.}{}{}{s.m.}{Arte marcial de origem japonesa, em que os oponentes, protegidos com capacete e peitilho, utilizam espadas de bambu.}{quen.dô}{0}
\verb{queniano}{}{}{}{}{adj.}{Relativo ao Quênia.}{que.ni.a.no}{0}
\verb{queniano}{}{}{}{}{}{Indivíduo natural ou habitante desse país. }{que.ni.a.no}{0}
\verb{quentão}{}{}{"-ões}{}{s.m.}{Bebida feita com aguardente, gengibre e canela, adoçada e servida quente.}{quen.tão}{0}
\verb{quente}{}{}{}{}{adj.2g.}{Em que existe calor.}{quen.te}{0}
\verb{quente}{}{}{}{}{}{Que protege o corpo contra o frio. }{quen.te}{0}
\verb{quente}{}{}{}{}{}{Ardente, cálido.  }{quen.te}{0}
\verb{quentinha}{}{Pop.}{}{}{s.f.}{Embalagem de alumínio para acondicionar e transportar comida.  }{quen.ti.nha}{0}
\verb{quentura}{}{}{}{}{s.f.}{Quantidade de calor. (\textit{Verifique a quentura da água antes de entrar no banho.})}{quentura}{0}
\verb{quepe}{é}{}{}{}{s.m.}{Boné usado por militares em vários países.   }{que.pe}{0}
\verb{quer}{é}{}{}{}{conj.}{Não importa que\ldots{} ou que. (\textit{Quer você queira, quer não queira, deverá ir à cerimônia.})}{quer}{0}
\verb{queratina}{}{Bioquím.}{}{}{s.f.}{Proteína fibrosa e pouco solúvel sintetizada por todos os vertebrados, e o principal componente estrutural dos cabelos, unhas, escamas, chifres, lã, cascos e penas.}{que.ra.ti.na}{0}
\verb{querela}{é}{}{}{}{s.f.}{Falta de acordo entre pessoas ou grupos sobre determinado ponto; contenda, pendência, disputa.}{que.re.la}{0}
\verb{querela}{é}{Jur.}{}{}{}{Denúncia apresentada em juízo contra alguém; litígio.}{que.re.la}{0}
\verb{querelante}{}{}{}{}{adj.2g.}{Que querela, que move ação penal contra outrem.}{que.re.lan.te}{0}
\verb{querelar}{}{}{}{}{v.t.}{Começar um processo contra alguém.}{que.re.lar}{0}
\verb{querelar}{}{}{}{}{v.pron.}{Queixar"-se.}{que.re.lar}{\verboinum{1}}
\verb{querença}{}{}{}{}{s.f.}{Ação ou efeito de querer.}{que.ren.ça}{0}
\verb{querença}{}{}{}{}{}{Boa ou má vontade para com alguém.}{que.ren.ça}{0}
\verb{querença}{}{}{}{}{}{Afeição.}{que.ren.ça}{0}
\verb{querência}{}{Bras.}{}{}{s.f.}{Lugar ou paradeiro onde habitualmente o gado pasta, ou onde foi criado.}{que.rên.cia}{0}
\verb{querência}{}{Por ext.}{}{}{}{Lugar onde alguém nasceu ou mora.}{que.rên.cia}{0}
\verb{querer}{ê}{}{}{}{v.t.}{Sentir vontade de; desejar. (\textit{Minha amiga e eu queremos apresentar o trabalho na biblioteca.})}{que.rer}{0}
\verb{querer}{ê}{}{}{}{}{Ter carinho por; amar, gostar de. (\textit{Acabou de conhecê"-la e já lhe quer muito.})}{que.rer}{\verboinum{49}}
\verb{querido}{}{}{}{}{adj. e s.m.  }{A quem se quer bem; amado, estimado.}{que.ri.do}{0}
\verb{querido}{}{}{}{}{s.m.}{Pessoa a quem se quer bem; bem. }{que.ri.do}{0}
\verb{quermesse}{é}{}{}{}{s.f.}{Festa, em geral para ajudar as obras de uma igreja, com leilão de prendas, rifas, jogos e barracas de bebidas e comidas.}{quer.mes.se}{0}
\verb{quero"-quero}{é\ldots{}é}{Zool.}{quero"-queros ⟨é\ldots{}é⟩}{}{s.m.}{Ave de penas cinzentas, bico fino e comprido e um topete na nuca, que vive em campos de pastagens e banhados.}{que.ro"-que.ro}{0}
\verb{querosene}{}{Quím.}{}{}{s.m.}{Líquido que vem do petróleo e serve de combustível.}{que.ro.se.ne}{0}
\verb{querubim}{}{}{"-ins}{}{s.m.}{Anjo}{que.ru.bim}{0}
\verb{querubim}{}{Relig.}{"-ins}{}{}{Anjo da primeira hierarquia.}{que.ru.bim}{0}
\verb{querubim}{}{Fig.}{"-ins}{}{}{Criança muito bonita.}{que.ru.bim}{0}
\verb{quesito}{}{}{}{}{s.m.}{Pergunta de questionário.}{que.si.to}{0}
\verb{quesito}{}{}{}{}{}{Requisito.}{que.si.to}{0}
\verb{questão}{}{}{"-ões}{}{s.f.}{Coisa de que se trata; assunto, tema.}{ques.tão}{0}
\verb{questão}{}{}{"-ões}{}{}{Problema que se deve resolver; dificuldade.}{ques.tão}{0}
\verb{questão}{}{}{"-ões}{}{}{Parte de prova que precisa de uma resposta.}{ques.tão}{0}
\verb{questão}{}{Jur.}{"-ões}{}{}{Assunto que se leva para o juiz decidir.}{ques.tão}{0}
\verb{questionador}{}{}{}{}{}{}{ques.ti.o.na.dor}{0}
\verb{questionamento}{}{}{}{}{s.m.}{Ato ou efeito de discordar; discordância, controvérsia, discussão, interrogatório.}{ques.ti.o.na.men.to}{0}
\verb{questionar}{}{}{}{}{v.t.}{Levantar uma dúvida sobre; controverter; discutir; interrogar. (\textit{Os candidatos questionavam os critérios de avaliação.})}{ques.ti.o.nar}{0}
\verb{questionário}{}{}{}{}{s.m.}{Lista de questões ou perguntas de uma prova ou de uma pesquisa.}{ques.ti.o.ná.rio}{0}
\verb{questionável}{}{}{"-eis}{}{adj.2g.}{Que se pode questionar.}{ques.ti.o.ná.vel}{0}
\verb{questionável}{}{}{"-eis}{}{}{Duvidoso, discutível, problemático.}{ques.ti.o.ná.vel}{0}
\verb{questiúncula}{}{}{}{}{s.f.}{Pequena questão, sem importância.}{ques.ti.ún.cu.la}{0}
\verb{questiúncula}{}{}{}{}{}{Briga por motivo sem importância.}{ques.ti.ún.cu.la}{0}
\verb{qui}{}{}{}{}{s.m.}{Vigésima segunda letra do alfabeto grego.}{qui}{0}
\verb{quiabeiro}{ê}{Bot.}{}{}{s.m.}{Planta hortense que produz frutos estreitos e alongados, os quiabos.}{qui.a.bei.ro}{0}
\verb{quiabo}{}{}{}{}{s.m.}{Fruto do quiabeiro.}{qui.a.bo}{0}
\verb{quibe}{}{Cul.}{}{}{s.m.}{Comida árabe, feita com carne moída, trigo integral e temperos, podendo ser consumida crua, frita ou assada.}{qui.be}{0}
\verb{quibebe}{é}{Cul.}{}{}{s.m.}{Comida pastosa feita de legume ou de fruta; purê.}{qui.be.be}{0}
\verb{quibebe}{é}{}{}{}{}{Purê de abóbora}{qui.be.be}{0}
\verb{quiçá}{}{}{}{}{adv.}{Talvez; pode ser que.}{qui.çá}{0}
\verb{quicar}{}{}{}{}{v.i.}{Repicar (bola).}{qui.car}{0}
\verb{quicar}{}{}{}{}{v.t.}{Fazer saltar.}{qui.car}{0}
\verb{quicar}{}{}{}{}{}{Bater um no outro (bola de gude).}{qui.car}{\verboinum{2}}
%\verb{}{}{}{}{}{}{}{}{0}
\verb{quíchua}{}{}{}{}{adj.2g.}{Relativo ou pertencente aos quíchuas, povo do Peru.}{quí.chu.a}{0}
\verb{quíchua}{}{}{}{}{s.m.}{Língua falada por esse povo.}{quí.chu.a}{0}
\verb{quício}{}{}{}{}{s.m.}{Gonzo de porta.}{quí.cio}{0}
\verb{quietação}{}{}{"-ões}{}{s.f.}{Ação ou efeito de quitar.}{qui.e.ta.ção}{0}
\verb{quietação}{}{}{"-ões}{}{}{Repouso, sossego, tranquilidade, quietude.}{qui.e.ta.ção}{0}
\verb{quietar}{}{}{}{}{v.t.}{Fazer ficar quieto; acalmar, tranquilizar.}{qui.e.tar}{\verboinum{1}}
\verb{quietismo}{}{Relig.}{}{}{s.m.}{Doutrina mística que considera a anulação da vontade e a completa indiferença da alma como meio eficaz para o aperfeiçoamento moral.}{qui.e.tis.mo}{0}
\verb{quietismo}{}{}{}{}{}{Contemplação inativa, sem obras exteriores.}{qui.e.tis.mo}{0}
\verb{quietismo}{}{}{}{}{}{Imobilidade, indiferença, apatia.}{qui.e.tis.mo}{0}
\verb{quieto}{é}{}{}{}{adj.}{Que não se mexe; imóvel, parado, quedo.}{qui.e.to}{0}
\verb{quieto}{é}{}{}{}{}{Que se comporta sem nenhum exagero; calmo, sossegado.}{qui.e.to}{0}
\verb{quietude}{}{}{}{}{s.f.}{Estado de quieto; quietação. (\textit{Gosto da quietude das noites no campo.})}{qui.e.tu.de}{0}
\verb{quilate}{}{}{}{}{s.m.}{Cada uma das vinte e quatro partes de uma liga que tem o metal precioso. (\textit{Ouro 18 quilates.})}{qui.la.te}{0}
\verb{quilha}{}{}{}{}{s.f.}{Peça de madeira ou metal que se coloca sob o fundo de uma embarcação para dar estabilidade.}{qui.lha}{0}
\verb{quiliare}{}{}{}{}{s.m.}{Medida de mil ares.}{qui.li.a.re}{0}
\verb{quilo}{}{}{}{}{s.m.}{Quilograma. Símb.: kg.}{qui.lo}{0}
\verb{quilobit}{}{Informát.}{}{}{s.m.}{Unidade de medida de quantidade de informação correspondente a 1.024 \textit{bits}.}{\textit{quilobit}}{0}
\verb{quilobyte}{}{Informát.}{}{}{s.m.}{Unidade de medida de quantidade de informação correspondente a 1.024 \textit{bytes}.}{\textit{quilobyte}}{0}
\verb{quilograma}{}{}{}{}{s.m.}{Unidade padrão, equivalente a mil gramas, usada para medir  a massa de um corpo. Símb.: kg. (\textit{Foram usados 100 quilogramas de ferro na reforma do muro.})}{qui.lo.gra.ma}{0}
\verb{quilo"-hertz}{é}{Fís.}{}{}{s.m.}{Unidade de medida de frequência equivalente a mil hertz. Símb.: kHz}{qui.lo"-hertz}{0}
\verb{quilolitro}{}{}{}{}{s.m.}{Unidade de capacidade equivalente a mil litros. Símb.: kl.}{qui.lo.li.tro}{0}
\verb{quilombo}{}{}{}{}{s.m.}{Povoado onde moravam os escravos fugidos.}{qui.lom.bo}{0}
\verb{quilombola}{ó}{}{}{}{s.2g.}{Indivíduo que morava no quilombo.}{qui.lom.bo.la}{0}
\verb{quilometragem}{}{}{"-ens}{}{s.f.}{Contagem de quilômetros.}{qui.lo.me.tra.gem}{0}
\verb{quilometrar}{}{}{}{}{v.t.}{Medir em quilômetros.}{qui.lo.me.trar}{\verboinum{1}}
\verb{quilométrico}{}{}{}{}{adj.}{Relativo a quilômetro.}{qui.lo.mé.tri.co}{0}
\verb{quilômetro}{}{}{}{}{s.m.}{Extensão de mil metros. Símb.: km.}{qui.lô.me.tro}{0}
\verb{quilowatt}{}{}{}{}{s.m.}{Unidade de medida de potência elétrica equivalente a mil watts. Símb.: kW.}{qui.lo.watt}{0}
\verb{quilowatt"-hora}{ó}{}{quilowatts"-hora ⟨ó⟩}{}{s.m.}{Unidade de medida de energia usada para designar o consumo de uma instalação elétrica por hora. Símb.: kWh.}{qui.lo.watt"-ho.ra}{0}
\verb{quimbanda}{}{Relig.}{}{}{s.f.}{Culto afro"-brasileiro que tem rituais para resolver dificuldades materiais.}{quim.ban.da}{0}
\verb{quimbundo}{}{}{}{}{adj.}{Relativo ou pertencente aos quimbundos, negros bantos de Angola.}{quim.bun.do}{0}
\verb{quimbundo}{}{}{}{}{s.m.}{Indivíduo dos quimbundos.}{quim.bun.do}{0}
\verb{quimbundo}{}{}{}{}{}{A língua dos quimbundos.}{quim.bun.do}{0}
\verb{quimera}{é}{}{}{}{s.f.}{Coisa impossível de se realizar; fantasia, sonho.}{qui.me.ra}{0}
\verb{quimérico}{}{}{}{}{adj.}{Em que há quimera.}{qui.mé.ri.co}{0}
\verb{quimérico}{}{}{}{}{}{Fantástico, utópico.}{qui.mé.ri.co}{0}
\verb{química}{}{}{}{}{s.f.}{Ciência que estuda a combinação dos elementos e suas propriedades.}{quí.mi.ca}{0}
\verb{químico}{}{}{}{}{adj.}{Relativo a química.}{quí.mi.co}{0}
\verb{químico}{}{}{}{}{s.m.}{Especialista em química.}{quí.mi.co}{0}
\verb{quimioterapia}{}{}{}{}{s.f.}{Tratamento feito com produtos químicos.}{qui.mi.o.te.ra.pi.a}{0}
\verb{quimioterápico}{}{}{}{}{adj.}{Relativo a quimioterapia.}{qui.mi.o.te.rá.pi.co}{0}
\verb{quimo}{}{}{}{}{s.m.}{Pasta a que se reduzem os alimentos, depois de haverem recebido no estômago a primeira elaboração.}{qui.mo}{0}
\verb{quimono}{}{}{}{}{s.m.}{Roupa japonesa na forma de uma túnica longa, cruzada na frente e com faixa na cintura.}{qui.mo.no}{0}
\verb{quina}{}{}{}{}{s.f.}{Ângulo ou vértice saliente; canto.}{qui.na}{0}
\verb{quina}{}{}{}{}{s.f.}{Grupo de cinco objetos, em geral iguais.}{qui.na}{0}
\verb{quina}{}{Bot.}{}{}{s.f.}{Nome comum a várias plantas arborescentes, dentre elas uma cuja casca tem propriedades medicinais.}{qui.na}{0}
\verb{quinado}{}{}{}{}{adj.}{Relativo a quina.}{qui.na.do}{0}
\verb{quinau}{}{}{}{}{s.m.}{Ato ou efeito de corrigir; corretivo.}{qui.nau}{0}
\verb{quindim}{}{Cul.}{"-ins}{}{s.m.}{Tipo de doce feito com gemas de ovos, coco e açúcar.}{quin.dim}{0}
\verb{quingentésimo}{}{}{}{}{num.}{Ordinal e fracionário correspondente a 500.   }{quin.gen.té.si.mo}{0}
\verb{quinhão}{}{}{"-ões}{}{s.m.}{A parte de um todo que cabe a cada um daqueles por quem se divide; cota.}{qui.nhão}{0}
\verb{quinhentos}{}{}{}{}{num.}{Nome dado à quantidade expressa pelo número 500.}{qui.nhen.tos}{0}
\verb{quinhoeiro}{ê}{}{}{}{s.m.}{Indivíduo que recebeu ou tem quinhão; sócio.}{qui.nho.ei.ro}{0}
\verb{quinina}{}{Quím.}{}{}{s.f.}{Alcaloide extraído da casca da quina.}{qui.ni.na}{0}
\verb{quinino}{}{}{}{}{s.m.}{Remédio que se obtém da casca da quina; sulfato de quinina.}{qui.ni.no}{0}
\verb{quino}{}{}{}{}{s.m.}{Loto, víspora.}{qui.no}{0}
\verb{quinquagenário}{}{}{}{}{adj.}{Que tem 50 anos ou está nessa faixa etária; cinquentão.}{quin.qua.ge.ná.rio}{0}
\verb{quinquagenário}{}{}{}{}{s.m.}{Indivíduo que tem 50 anos ou está nessa faixa etária. }{quin.qua.ge.ná.rio}{0}
\verb{quinquagésimo}{}{}{}{}{num.}{Ordinal e fracionário correspondente a 50.  }{quin.qua.gé.si.mo}{0}
\verb{quinquenal}{}{}{"-ais}{}{adj.2g.}{Que dura ou se estende por cinco anos.}{quin.que.nal}{0}
\verb{quinquenal}{}{}{"-ais}{}{}{Que ocorre ou se realiza a cada cinco anos.}{quin.que.nal}{0}
\verb{quinquênio}{}{}{}{}{s.m.}{Período de cinco anos; lustro.}{quin.quê.nio}{0}
\verb{quinquídio}{}{}{}{}{s.m.}{Período de cinco dias.}{quin.quí.dio}{0}
\verb{quinquilharia}{}{}{}{}{s.f.}{Coisa de pouco valor; bagatela, bugiganga, ninharia.}{quin.qui.lha.ri.a}{0}
\verb{quinta}{}{}{}{}{s.f.}{Forma reduzida de \textit{quinta"-feira}.}{quin.ta}{0}
\verb{quinta}{}{}{}{}{}{Propriedade rural, com casa de habitação.}{quin.ta}{0}
\verb{quinta"-coluna}{}{}{quinta"-colunas}{}{s.2g.}{Pessoa que age em favor de um país beligerante, ou que está para entrar em guerra, em detrimento do país que o abriga; espião, traidor.}{quin.ta"-co.lu.na}{0}
\verb{quinta"-coluna}{}{}{quinta"-colunas}{}{s.f.}{A classe dos quinta"-colunas.}{quin.ta"-co.lu.na}{0}
\verb{quinta"-colunista}{}{}{quinta"-colunistas}{}{adj.2g.}{Relativo à quinta"-coluna.}{quin.ta"-co.lu.nis.ta}{0}
\verb{quinta"-essência}{}{}{quinta"-essências}{}{s.f.}{Extrato levado ao último apuramento.}{quin.ta"-es.sên.cia}{0}
\verb{quinta"-essência}{}{}{quinta"-essências}{}{}{A própria essência; o que há de mais puro; requinte.}{quin.ta"-es.sên.cia}{0}
\verb{quinta"-essência}{}{}{quinta"-essências}{}{}{Auge; o mais alto grau.}{quin.ta"-es.sên.cia}{0}
\verb{quinta"-essência}{}{}{quinta"-essências}{}{}{Na Alquimia, a parte mais pura de uma substância, obtida após cinco destilações.}{quin.ta"-es.sên.cia}{0}
\verb{quinta"-feira}{ê}{}{quintas"-feiras}{}{s.f.}{O quinto dia da semana.}{quin.ta"-fei.ra}{0}
\verb{quintal}{}{}{"-ais}{}{s.m.}{Pequeno terreno atrás ou junto de uma casa, utilizado geralmente como jardim ou horta.}{quin.tal}{0}
\verb{quintal}{}{}{"-ais}{}{}{Pequena quinta; pequena propriedade.}{quin.tal}{0}
\verb{quintanista}{}{}{}{}{adj.2g.}{Diz"-se de estudante que frequenta o quinto ano de um curso superior.}{quin.ta.nis.ta}{0}
\verb{quintessência}{}{}{}{}{}{Var. de \textit{quinta"-essência}.}{quin.tes.sên.cia}{0}
\verb{quinteto}{ê}{Mús.}{}{}{s.m.}{Composição musical feita para cinco vozes ou cinco instrumentos.}{quin.te.to}{0}
\verb{quinteto}{ê}{}{}{}{}{Estrofe de cinco versos, geralmente heptassílabos; quintilha. }{quin.te.to}{0}
\verb{quinteto}{ê}{}{}{}{}{Grupo de cinco pessoas ou coisas.}{quin.te.to}{0}
\verb{quintilha}{}{Gram.}{}{}{s.f.}{Estrofe de cinco versos, geralmente heptassílabos; quinteto.}{quin.ti.lha}{0}
\verb{quintilhão}{}{}{"-ões}{}{num.}{Cardinal equivalente a mil quatrilhões. }{quin.ti.lhão}{0}
\verb{quintilião}{}{}{}{}{}{Var. de \textit{quintilhão}.}{quin.ti.li.ão}{0}
\verb{quinto}{}{}{}{}{num.}{Numa sequência, o que ocupa a posição do número 5.}{quin.to}{0}
\verb{quinto}{}{}{}{}{}{Ordinal e fracionário correspondente a 5.  }{quin.to}{0}
\verb{quintuplicar}{}{}{}{}{v.t.}{Multiplicar por cinco.}{quin.tu.pli.car}{0}
\verb{quintuplicar}{}{}{}{}{}{Tornar cinco vezes maior.}{quin.tu.pli.car}{\verboinum{2}}
\verb{quíntuplo}{}{}{}{}{num.}{Cinco vezes maior.}{quín.tu.plo}{0}
\verb{quíntuplo}{}{}{}{}{}{Quantidade cinco vezes maior.}{quín.tu.plo}{0}
\verb{quinze}{}{}{}{}{num.}{Nome dado à quantidade expressa pelo número 15.}{quin.ze}{0}
\verb{quinzena}{}{}{}{}{s.f.}{Espaço de quinze dias; período de duas semanas.}{quin.ze.na}{0}
\verb{quinzena}{}{}{}{}{}{Grupo de quinze seres ou coisas.}{quin.ze.na}{0}
\verb{quinzena}{}{}{}{}{}{Pagamento do trabalho de quinze dias.}{quin.ze.na}{0}
\verb{quinzenal}{}{}{"-ais}{}{adj.2g.}{Relativo a quinzena.}{quin.ze.nal}{0}
\verb{quinzenal}{}{}{"-ais}{}{}{Que ocorre ou se realiza de 15 em 15 dias.}{quin.ze.nal}{0}
\verb{quinzenário}{}{}{}{}{s.m.}{Periódico publicado de 15 em 15 dias.}{quin.ze.ná.rio}{0}
\verb{quiosque}{ó}{}{}{}{s.m.}{Pequeno pavilhão de madeira, em lugares públicos, geralmente destinado à venda de cigarros, jornais, revistas, refrigerantes.}{qui.os.que}{0}
\verb{quiosque}{ó}{}{}{}{}{Pavilhão aberto que ornamenta praças e jardins.}{qui.os.que}{0}
\verb{quiproquó}{}{}{}{}{s.m.}{Erro que consiste em se tomar uma coisa por outra; engano, equívoco.}{qui.pro.quó}{0}
\verb{quiproquó}{}{}{}{}{}{Situação cômica ou confusão criada por esse engano.}{qui.pro.quó}{0}
\verb{quirela}{é}{}{}{}{}{Var. de \textit{quirera}.}{qui.re.la}{0}
\verb{quirera}{é}{}{}{}{s.f.}{Milho partido que se dá a pequenas aves e pássaros.}{qui.re.ra}{0}
\verb{quirera}{é}{Fig.}{}{}{}{Dinheiro miúdo; trocado.}{qui.re.ra}{0}
\verb{quiromancia}{}{}{}{}{s.f.}{Adivinhação do futuro pelo exame das linhas da palma da mão.}{qui.ro.man.ci.a}{0}
\verb{quiromante}{}{}{}{}{s.2g.}{Pessoa que adivinha o futuro examinando as linhas da palma da mão.}{qui.ro.man.te}{0}
\verb{quiroprática}{}{}{}{}{s.f.}{Arte e técnica de aplicar massagens.}{qui.ro.prá.ti.ca}{0}
\verb{quiróptero}{}{Zool.}{}{}{s.m.}{Espécime dos quirópteros, ordem dos mamíferos que inclui os morcegos.}{qui.róp.te.ro}{0}
\verb{quisto}{}{Med.}{}{}{s.m.}{Tumor formado por um saco ou vesícula que contém um líquido ou uma substância mole; cisto.}{quis.to}{0}
\verb{quitação}{}{}{"-ões}{}{s.f.}{Ato ou efeito de quitar.}{qui.ta.ção}{0}
\verb{quitação}{}{}{"-ões}{}{}{Ato pelo qual alguém salda uma dívida, desobrigando"-se perante o credor.}{qui.ta.ção}{0}
\verb{quitação}{}{}{"-ões}{}{}{Documento ou recibo de pagamento.}{qui.ta.ção}{0}
\verb{quitanda}{}{}{}{}{s.f.}{Estabelecimento onde se vendem frutas, verduras, ovos e cereais, entre outros.}{qui.tan.da}{0}
\verb{quitanda}{}{}{}{}{}{Pequena mercearia; venda, tenda.}{qui.tan.da}{0}
\verb{quitandeiro}{ê}{}{}{}{s.m.}{Dono de quitanda.}{qui.tan.dei.ro}{0}
\verb{quitandeiro}{ê}{}{}{}{}{Vendedor ambulante de hortaliças e frutas; verdureiro.}{qui.tan.dei.ro}{0}
\verb{quitar}{}{}{}{}{v.t.}{Saldar (uma dívida); tornar quite; desobrigar.}{qui.tar}{0}
\verb{quitar}{}{}{}{}{}{Poupar, evitar.}{qui.tar}{\verboinum{1}}
\verb{quite}{}{}{}{}{adj.}{Livre de dívida, obrigação ou compromisso; desobrigado, desembaraçado.}{qui.te}{0}
\verb{quite}{}{}{}{}{}{Empatado, igualado.}{qui.te}{0}
\verb{quitina}{}{Zool.}{}{}{s.f.}{Substância insolúvel que reveste certos artrópodes como aranhas e crustáceos e também constitui a parede celular de certos fungos.}{qui.ti.na}{0}
\verb{quitinete}{é}{Bras.}{}{}{s.f.}{Pequeno apartamento, constituído de um único cômodo, com uma cozinha minúscula e um banheiro; \textit{kitchenette}.}{qui.ti.ne.te}{0}
\verb{quitute}{}{Cul.}{}{}{s.m.}{Comida refinada e apetitosa; iguaria, manjar.}{qui.tu.te}{0}
\verb{quituteiro}{ê}{}{}{}{s.m.}{Indivíduo que sabe preparar quitutes.}{qui.tu.tei.ro}{0}
%\verb{}{}{}{}{}{}{}{}{0}
\verb{quiuí}{}{Bot.}{}{}{s.m.}{Trepadeira de folhagem densa, que produz frutos comestíveis, de casca marrom, coberta de pelos e polpa verde amarelado; \textit{kiwi}.}{qui.uí}{0}
\verb{quiuí}{}{}{}{}{}{O fruto dessa planta.  }{qui.uí}{0}
\verb{quivi}{}{Zool.}{}{}{s.m.}{Espécie de ave terrestre da Nova Zelândia, com bico longo e delgado e corpo recoberto de penugem.}{qui.vi}{0}
\verb{quivi}{}{}{}{}{}{Var. de \textit{quiuí}.}{qui.vi}{0}
\verb{quixaba}{ch}{}{}{}{s.f.}{Fruto comestível da quixabeira, de coloração roxo"-escura.}{qui.xa.ba}{0}
\verb{quixabeira}{ch}{Bot.}{}{}{s.f.}{Árvore frutífera da caatinga, com folhas pequenas e frutos comestíveis, os quais o gado come em período de seca.}{qui.xa.bei.ra}{0}
\verb{quixotada}{ch}{}{}{}{s.f.}{Ato ingênuo, sonhador, romântico.}{qui.xo.ta.da}{0}
\verb{quixotada}{ch}{}{}{}{}{Bravata ridícula; fanfarronada.}{qui.xo.ta.da}{0}
\verb{quixote}{chó}{}{}{}{s.m.}{Diz"-se do indivíduo ingênuo e sonhador que se mete em questões que não lhe dizem respeito e geralmente se dá mal, por alusão a D. Quixote, personagem da obra \textit{Dom Quixote de La Mancha}, de Miguel de Cervantes, século \textsc{xvi}.}{qui.xo.te}{0}
\verb{quixotesco}{ch\ldots{}ê}{}{}{}{adj.}{Relativo a D. Quixote, personagem da obra \textit{Dom Quixote de La Mancha}.}{qui.xo.tes.co}{0}
\verb{quixotesco}{ch\ldots{}ê}{}{}{}{}{Que é irrealizável na prática; utópico.}{qui.xo.tes.co}{0}
\verb{quixotice}{ch}{}{}{}{s.f.}{Quixotada.}{qui.xo.ti.ce}{0}
\verb{quixotismo}{ch}{}{}{}{s.m.}{Comportamento semelhante a D. Quixote.}{qui.xo.tis.mo}{0}
\verb{quixotismo}{ch}{}{}{}{}{Excesso de cavalheirismo.}{qui.xo.tis.mo}{0}
\verb{quixotismo}{ch}{}{}{}{}{Pretensão de coragem e grandes aventuras; fanfarronice.}{qui.xo.tis.mo}{0}
\verb{quizila}{}{}{}{}{s.f.}{Aversão espontânea e gratuita; repugnância, antipatia.}{qui.zi.la}{0}
\verb{quizila}{}{}{}{}{}{Sensação de impaciência; aborrecimento, chateação.}{qui.zi.la}{0}
\verb{quizila}{}{}{}{}{}{Conflito de interesses; rixa, inimizade.}{qui.zi.la}{0}
\verb{quizilar}{}{}{}{}{v.t.}{Causar incômodo; importunar, aborrecer.}{qui.zi.lar}{\verboinum{1}}
\verb{quizilento}{}{}{}{}{adj.}{Que faz quizila; antipático, inoportuno.}{qui.zi.len.to}{0}
\verb{quizília}{}{}{}{}{}{Var. de \textit{quizila}.}{qui.zí.lia}{0}
\verb{quociente}{}{Mat.}{}{}{s.m.}{Resultado de uma divisão, indicando quantas vezes o divisor se contém no dividendo.}{quo.ci.en.te}{0}
\verb{quórum}{ó}{}{"-uns}{}{s.m.}{Número mínimo necessário de membros para que uma assembleia possa deliberar e tomar decisões válidas. }{quó.rum}{0}
\verb{quota}{ó}{}{}{}{}{Var. de \textit{cota}.}{quo.ta}{0}
\verb{quota"-parte}{}{}{}{}{}{Var. de \textit{cota"-parte}.}{quo.ta"-par.te}{0}
\verb{quotidiano}{}{}{}{}{}{Var. de \textit{cotidiano}.}{quo.ti.di.a.no}{0}
\verb{quotista}{}{}{}{}{}{Var. de \textit{cotista}.}{quo.tis.ta}{0}
\verb{quotização}{}{}{}{}{}{Var. de \textit{cotização}.}{quo.ti.za.ção}{0}
\verb{quotizar}{}{}{}{}{}{Var. de \textit{cotizar}.}{quo.ti.zar}{0}
