\verb{d}{}{}{}{}{s.m.}{Quarta letra do alfabeto português.}{d}{0}
%\verb{}{}{}{}{}{}{}{}{0}
%\verb{}{}{}{}{}{}{}{}{0}
%\verb{}{}{}{}{}{}{}{}{0}
\verb{da}{}{}{}{}{}{Contração da preposição \textit{de} com o artigo \textit{a}. (\textit{Os alunos estão chegando da aula de Ciências.})}{da}{0}
\verb{dáblio}{}{}{}{}{s.m.}{Nome da letra \textit{w}.}{dá.blio}{0}
\verb{dactilografar}{}{}{}{}{}{Var. de \textit{datilografar}.}{dac.ti.lo.gra.far}{0}
\verb{dactilografia}{}{}{}{}{}{Var. de \textit{datilografia}.}{dac.ti.lo.gra.fi.a}{0}
\verb{dactilográfico}{}{}{}{}{}{Var. de \textit{datilográfico}.}{dac.ti.lo.grá.fi.co}{0}
\verb{dactilógrafo}{}{}{}{}{}{Var. de \textit{datilógrafo}.}{dac.ti.ló.gra.fo}{0}
\verb{dactiloscopia}{}{}{}{}{}{Var. de \textit{datiloscopia}.}{dac.ti.los.co.pi.a}{0}
\verb{dactiloscopista}{}{}{}{}{}{Var. de \textit{datiloscopista}.}{dac.ti.los.co.pis.ta}{0}
\verb{dadaísmo}{}{Liter.}{}{}{s.m.}{Movimento artístico"-literário iniciado em 1916 por Tristan Tzara, que tinha por princípio o apelo ao subconsciente e a provocação, para negar todas as formas de arte e denunciar o absurdo reinante no mundo. }{da.da.ís.mo}{0}
\verb{dádiva}{}{}{}{}{s.f.}{Aquilo que se dá sem querer pagamento; dom, graça, presente.}{dá.di.va}{0}
\verb{dadivoso}{ô}{}{"-osos ⟨ó⟩}{"-osa ⟨ó⟩}{adj.}{Que gosta de presentear; generoso.}{da.di.vo.so}{0}
\verb{dado}{}{}{}{}{adj.}{Que se deu; concedido.}{da.do}{0}
\verb{dado}{}{}{}{}{s.m.}{Peça de jogo com seis lados quadrados, numerados.}{da.do}{0}
\verb{dado}{}{}{}{}{}{Que se mostra amigo; afável, tratável.}{da.do}{0}
\verb{dado}{}{}{}{}{s.m.}{Cada uma das informações disponíveis para análise.}{da.do}{0}
\verb{daí}{}{}{}{}{}{Contração da preposição \textit{de} com o advérbio \textit{aí}.}{da.í}{0}
\verb{Dalai"-lama}{}{Relig.}{dalai"-lamas}{}{s.m.}{Chefe supremo do lamaísmo.}{Da.lai"-la.ma}{0}
\verb{dalém}{}{}{}{}{}{Contração da preposição \textit{de} com o advérbio \textit{além}.}{da.lém}{0}
\verb{dali}{}{}{}{}{}{Contração da preposição \textit{de} com o advérbio \textit{ali}.}{da.li}{0}
\verb{dália}{}{Bot.}{}{}{s.f.}{Nome comum às plantas herbáceas nativas das regiões montanhosas da América, muito cultivadas pela beleza de suas flores. }{dá.lia}{0}
\verb{dália}{}{}{}{}{}{A flor dessas plantas.}{dá.lia}{0}
\verb{dálmata}{}{}{}{}{adj.}{Da Dalmácia, região da Europa.}{dál.ma.ta}{0}
\verb{dálmata}{}{}{}{}{s.m.}{Raça de cães de pelagem branca com pintas pretas, possivelmente originário da Dalmácia.}{dál.ma.ta}{0}
\verb{daltônico}{}{}{}{}{adj.}{Relativo a daltonismo.}{dal.tô.ni.co}{0}
\verb{daltônico}{}{}{}{}{s.m.}{Indivíduo que sofre de daltonismo.}{dal.tô.ni.co}{0}
\verb{daltonismo}{}{Med.}{}{}{s.m.}{Incapacidade de diferenciar cores, especialmente o vermelho do verde.}{dal.to.nis.mo}{0}
\verb{dama}{}{}{}{}{s.f.}{Mulher nobre, bem"-educada.}{da.ma}{0}
\verb{dama}{}{}{}{}{}{Em uma dança, mulher que forma par com um cavalheiro.}{da.ma}{0}
\verb{dama}{}{}{}{}{}{Carta do baralho em que aparece a figura de uma mulher; rainha.}{da.ma}{0}
\verb{dama}{}{}{}{}{}{Peça do jogo de xadrez que se posiciona entre o rei e um dos bispos e que se pode movimentar em qualquer direção, ao longo de casas livres; rainha. }{da.ma}{0}
\verb{damas}{}{}{}{}{s.m.}{Jogo composto de 24 peças que se movem em um tabuleiro de 64 casas.}{da.mas}{0}
\verb{damasceno}{}{}{}{}{adj.}{Relativo a Damasco, capital da Síria.}{da.mas.ce.no}{0}
\verb{damasceno}{}{}{}{}{s.m.}{Indivíduo natural ou habitante dessa cidade.}{da.mas.ce.no}{0}
\verb{damasco}{}{}{}{}{s.m.}{Fruto do damasqueiro, de casca amarelo"-avermelhada, macia e brilhante; abricó.}{da.mas.co}{0}
\verb{damasco}{}{}{}{}{}{Tecido grosso de seda com desenhos.}{da.mas.co}{0}
\verb{damasqueiro}{ê}{Bot.}{}{}{s.m.}{Árvore cujo fruto é o damasco.}{da.mas.quei.ro}{0}
\verb{danação}{}{}{"-ões}{}{s.f.}{Ato de danar; fúria, raiva.}{da.na.ção}{0}
\verb{danado}{}{}{}{}{adj.}{Que foi atacado pela doença da raiva; hidrófobo, raivoso.}{da.na.do}{0}
\verb{danado}{}{}{}{}{}{Furioso, irado, zangado.}{da.na.do}{0}
\verb{danado}{}{}{}{}{}{Que foi condenado às penas do inferno; amaldiçoado, maldito.}{da.na.do}{0}
\verb{danado}{}{}{}{}{}{Que é muito esperto; travesso, arteiro.}{da.na.do}{0}
\verb{danar}{}{}{}{}{v.t.}{Causar irritação; encolerizar, enraivecer.}{da.nar}{0}
\verb{danar}{}{}{}{}{}{Causar dano; prejudicar, estragar.}{da.nar}{0}
\verb{danar}{}{}{}{}{}{Tornar hidrófobo, raivoso.}{da.nar}{0}
\verb{danar}{}{}{}{}{}{Corromper física ou moralmente; perverter.}{da.nar}{\verboinum{1}}
\verb{dança}{}{}{}{}{s.f.}{Ato ou efeito de dançar.}{dan.ça}{0}
\verb{dança}{}{}{}{}{}{Série de passos e gestos cadenciados ao som de uma música.}{dan.ça}{0}
\verb{dançador}{ô}{}{}{}{adj.}{Que gosta de dançar; dançarino.}{dan.ça.dor}{0}
\verb{dançante}{}{}{}{}{adj.2g.}{Em que há dança. (\textit{Participamos, no último domingo, de um jantar dançante.})}{dan.çan.te}{0}
\verb{dançar}{}{}{}{}{v.t.}{Executar os movimentos próprios de uma determinada dança; bailar. }{dan.çar}{0}
\verb{dançar}{}{}{}{}{}{Girar sobre uma superfície, sobre a água ou no ar.}{dan.çar}{0}
\verb{dançar}{}{Pop.}{}{}{v.i.}{Sair"-se mal. (\textit{Ele achou que conseguiria sair com a menina, mas dançou.})}{dan.çar}{\verboinum{3}}
\verb{dançarino}{}{}{}{}{s.m.}{Indivíduo que dança por profissão.}{dan.ça.ri.no}{0}
\verb{dançarino}{}{}{}{}{}{Indivíduo que dança muito bem.}{dan.ça.ri.no}{0}
\verb{danceteria}{}{}{}{}{s.f.}{Casa noturna com bar e pista de dança; discoteca.}{dan.ce.te.ri.a}{0}
\verb{dancing}{}{}{}{}{s.m.}{Estabelecimento onde se bebe e dança, ou onde se aprende a dançar com profissionais; salão de dança.}{\textit{dancing}}{0}
\verb{dândi}{}{}{}{}{s.m.}{Homem que se veste com muito apuro, com muito requinte.}{dân.di}{0}
\verb{dandismo}{}{}{}{}{s.m.}{Modo afetado de se comportar ou se vestir.}{dan.dis.mo}{0}
\verb{danificar}{}{}{}{}{v.t.}{Causar dano ou prejuízo; estragar, prejudicar.}{da.ni.fi.car}{\verboinum{2}}
\verb{daninho}{}{}{}{}{adj.}{Que prejudica o desenvolvimento de algo; nocivo.}{da.ni.nho}{0}
\verb{dano}{}{}{}{}{s.m.}{Mal ou ofensa que se faz a alguém.}{da.no}{0}
\verb{dano}{}{}{}{}{}{Prejuízo, perda, estrago.}{da.no}{0}
\verb{danoso}{ô}{}{"-osos ⟨ó⟩}{"-osa ⟨ó⟩}{adj.}{Que causa dano; prejudicial, nocivo.}{da.no.so}{0}
\verb{dantes}{}{}{}{}{adv.}{No tempo passado; antigamente, antes.}{dan.tes}{0}
\verb{dantesco}{ê}{}{}{}{adj.}{Que causa grande horror; diabólico, medonho, pavoroso.}{dan.tes.co}{0}
\verb{daquele}{ê}{}{}{}{}{Contração da preposição \textit{de} com o pronome \textit{aquele}.}{da.que.le}{0}
\verb{daquém}{}{}{}{}{}{Contração da preposição \textit{de} com o advérbio \textit{aquém}.}{da.quém}{0}
\verb{daqui}{}{}{}{}{}{Contração da preposição \textit{de} com o advérbio \textit{aqui}.}{da.qui}{0}
\verb{daquilo}{}{}{}{}{}{Contração da preposição \textit{de} com o pronome \textit{aquilo}.}{da.qui.lo}{0}
\verb{dar}{}{}{}{}{v.t.}{Ceder gratuitamente o que se possui; doar, conceder.}{dar}{0}
\verb{dar}{}{}{}{}{}{Bater, soar. (\textit{O relógio deu oito horas.})}{dar}{0}
\verb{dar}{}{}{}{}{}{Ministrar, aplicar. (\textit{O enfermeiro deu 12 gotas do remédio ao paciente.})}{dar}{0}
\verb{dar}{}{}{}{}{}{Ser suficiente; bastar. (\textit{O salário dele não dá até o fim do mês.})}{dar}{0}
\verb{dar}{}{}{}{}{}{Produzir, gerar. (\textit{Essa vaca dá cinco litros de leite por dia.})}{dar}{0}
\verb{dar}{}{}{}{}{}{Efetuar, executar, praticar.}{dar}{0}
\verb{dar}{}{}{}{}{v.pron.}{Combinar"-se; entender"-se. (\textit{Eles se dão muito bem.})}{dar}{\verboinum{37}}
\verb{dardejar}{}{}{}{}{v.t.}{Atirar, arremessar dardos.}{dar.de.jar}{0}
\verb{dardejar}{}{}{}{}{}{Brilhar muito; cintilar, chamejar.}{dar.de.jar}{\verboinum{1}}
\verb{dardo}{}{}{}{}{s.m.}{Arma de arremesso, curta e delgada, terminada em ponta de ferro.}{dar.do}{0}
\verb{darma}{}{}{}{}{s.m.}{Conjunto de preceitos morais e religiosos das filosofias e religiões indianas.}{dar.ma}{0}
\verb{darmstácio}{}{Quím.}{}{}{s.m.}{Elemento químico artificial sintetizado por fusão de baixo aquecimento e que fora denominado provisóriamente ``uninílio'' (Unn). \elemento{110}{(269)}{Ds}.}{darms.tá.cio}{0}
\verb{darwinismo}{}{}{}{}{s.m.}{Teoria evolucionista baseada nas ideias do naturalista inglês Charles Robert Darwin, no século \textsc{xix}, em que se propõem mecanismos baseados na seleção natural para explicar a origem e a transformação das espécies ao longo do tempo.}{dar.wi.nis.mo}{0}
\verb{dasipodídeo}{}{Zool.}{}{}{s.m.}{Espécime dos dasipodídeos, família de mamíferos xenartros, providos de garras e corpo protegido por carapaças, cujo representante é o tatu.}{da.si.po.dí.deo}{0}
\verb{dasipodídeo}{}{Zool.}{}{}{adj.}{Relativo aos dasipodídeos.}{da.si.po.dí.deo}{0}
\verb{data}{}{}{}{}{s.f.}{Indicação da época, do ano, do mês ou do dia de um acontecimento. (\textit{A data de seu aniversário coincidia com a de seu pai.})}{da.ta}{0}
\verb{data"-base}{}{}{}{}{s.f.}{Data considerada marco para a concessão de reajustes salariais de certa categoria profissional.}{da.ta"-ba.se}{0}
\verb{datação}{}{}{"-ões}{}{s.f.}{Ato ou efeito de datar.}{da.ta.ção}{0}
\verb{datação}{}{}{"-ões}{}{}{Processo de determinação da idade de formações geológicas, minerais etc.}{da.ta.ção}{0}
\verb{datar}{}{}{}{}{v.t.}{Indicar a data de algo.}{da.tar}{0}
\verb{datar}{}{}{}{}{}{Existir desde certa época. (\textit{Aquele forte data do início do século.})}{da.tar}{\verboinum{1}}
\verb{datilografar}{}{}{}{}{v.t.}{Escrever um texto com máquina de escrever.}{da.ti.lo.gra.far}{\verboinum{1}}
\verb{datilografia}{}{}{}{}{s.f.}{Técnica de escrever à máquina.}{da.ti.lo.gra.fi.a}{0}
\verb{datilógrafo}{}{}{}{}{s.m.}{Indivíduo que escreve à máquina.}{da.ti.ló.gra.fo}{0}
\verb{datiloscopia}{}{}{}{}{s.f.}{Processo de identificação de pessoas através das impressões digitais.}{da.ti.los.co.pi.a}{0}
\verb{datiloscopista}{}{}{}{}{s.2g.}{Indivíduo especializado em datiloscopia.}{da.ti.los.co.pis.ta}{0}
\verb{dativo}{}{Jur.}{}{}{adj.}{Que foi nomeado por magistrado e não por lei.}{da.ti.vo}{0}
\verb{dativo}{}{Gram.}{}{}{s.m.}{Um dos casos sintáticos, morfologicamente marcados, de algumas línguas como o latim.}{da.ti.vo}{0}
\verb{dC}{}{}{}{}{}{Abrev. de \textit{depois de Cristo}.}{d.C.}{0}
\verb{DDD}{}{}{}{}{}{Sigla de \textit{Discagem Direta a Distância}.}{DDD}{0}
\verb{DDI}{}{}{}{}{}{Sigla de \textit{Discagem Direta Internacional}.}{DDI}{0}
\verb{DDT}{}{}{}{}{s.m.}{Dedetê, inseticida, principalmente contra baratas.}{DDT}{0}
\verb{de}{}{}{}{}{prep.}{Indica diferentes relações: de posse, origem, lugar de onde, matéria etc.}{de}{0}
\verb{dê}{}{}{}{}{s.m.}{O nome da letra \textit{d}.}{dê}{0}
\verb{deambular}{}{}{}{}{v.i.}{Caminhar sem rumo certo; vaguear, perambular.}{de.am.bu.lar}{\verboinum{1}}
\verb{deão}{}{}{"-ãos, -ães \textit{ou} -ões.}{}{s.m.}{Dignitário eclesiástico que dirige o capítulo; decano.}{de.ão}{0}
\verb{debacle}{}{}{}{}{s.m.}{Mau resultado; fracasso, queda.}{de.ba.cle}{0}
\verb{debaixo}{ch}{}{}{}{adv.}{Em lugar, posição ou condição inferior.}{de.bai.xo}{0}
\verb{debaixo}{ch}{}{}{}{}{Na dependência. (\textit{A mãe tinha os filhos sempre debaixo de seus olhos.})}{de.bai.xo}{0}
\verb{debalde}{}{}{}{}{adv.}{Sem nenhum resultado; em vão. (\textit{Esforcei"-me debalde para a prova.})}{de.bal.de}{0}
\verb{debandada}{}{}{}{}{s.f.}{Saída com pressa e desordem.}{de.ban.da.da}{0}
\verb{debandar}{}{}{}{}{v.i.}{Pôr"-se em debandada; dispersar"-se desordenadamente.}{de.ban.dar}{\verboinum{1}}
\verb{debate}{}{}{}{}{s.m.}{Ato de debater; discussão.}{de.ba.te}{0}
\verb{debatedor}{ô}{}{}{}{adj.}{Diz"-se de mediador num debate.}{de.ba.te.dor}{0}
\verb{debatedor}{ô}{}{}{}{}{Diz"-se de indivíduo que formula as perguntas num debate.}{de.ba.te.dor}{0}
\verb{debater}{ê}{}{}{}{v.t.}{Falar sobre um assunto com outra pessoa, devendo cada uma delas dar a sua opinião.}{de.ba.ter}{0}
\verb{debater}{ê}{}{}{}{v.pron.}{Agitar"-se muito.}{de.ba.ter}{\verboinum{12}}
\verb{debelar}{}{}{}{}{v.t.}{Vencer em luta armada; derrotar, sujeitar.}{de.be.lar}{0}
\verb{debelar}{}{}{}{}{}{Anular ação ou efeito de algo considerado maléfico; extinguir, reprimir.}{de.be.lar}{\verboinum{1}}
\verb{debênture}{}{}{}{}{s.f.}{Título de dívida que se paga aos poucos.}{de.bên.tu.re}{0}
\verb{debicar}{}{}{}{}{v.t.}{Tirar ou puxar com o bico.}{de.bi.car}{0}
\verb{debicar}{}{}{}{}{}{Comer pequena porção de algo; beliscar.}{de.bi.car}{\verboinum{2}}
\verb{débil}{}{}{"-eis}{}{adj.2g.}{Em que não há força física, vigor ou saúde; fraco.}{dé.bil}{0}
\verb{débil}{}{}{"-eis}{}{}{Diz"-se de indivíduo que é débil mental; pessoa que tem um atraso no desenvolvimento da mente.}{dé.bil}{0}
\verb{debilidade}{}{}{}{}{s.f.}{Qualidade ou estado de débil; falta de vigor ou energia; fraqueza.}{de.bi.li.da.de}{0}
\verb{debilitação}{}{}{"-ões}{}{s.f.}{Ato ou efeito de debilitar; perda de forças; enfraquecimento.}{de.bi.li.ta.ção}{0}
\verb{debilitante}{}{}{}{}{adj.2g.}{Que debilita; enfraquecedor.}{de.bi.li.tan.te}{0}
\verb{debilitar}{}{}{}{}{v.t.}{Tirar a força do corpo de alguém; enfraquecer.}{de.bi.li.tar}{\verboinum{1}}
\verb{debiloide}{}{}{}{}{adj.2g.}{Diz"-se de quem é um tanto débil mental.}{de.bi.loi.de}{0}
\verb{debique}{}{}{}{}{s.m.}{Ato de debicar; zombaria, troça.}{de.bi.que}{0}
\verb{debitar}{}{}{}{}{v.t.}{Escrever o valor de uma dívida em alguma conta.}{de.bi.tar}{\verboinum{1}}
\verb{débito}{}{}{}{}{s.m.}{Dinheiro que se deve; dívida.}{dé.bi.to}{0}
\verb{deblaterar}{}{}{}{}{v.t.}{Falar ou clamar com violência contra pessoas ou coisas.}{de.bla.te.rar}{\verboinum{1}}
\verb{debochado}{}{}{}{}{adj.}{Diz"-se de indivíduo entregue ao deboche; devasso, libertino.}{de.bo.cha.do}{0}
\verb{debochado}{}{Pop.}{}{}{}{Que costuma debochar dos outros; gozador, zombeteiro.}{de.bo.cha.do}{0}
\verb{debochar}{}{}{}{}{v.t.}{Entregar"-se à devassidão, aos prazeres carnais, ao vício; corromper.}{de.bo.char}{0}
\verb{debochar}{}{}{}{}{}{Falar de uma pessoa ou coisa com desprezo; zombar, ridicularizar, troçar.}{de.bo.char}{\verboinum{1}}
\verb{deboche}{ó}{}{}{}{s.m.}{Ato de debochar; zombaria.}{de.bo.che}{0}
\verb{debrear}{}{}{}{}{v.t.}{Embrear.}{de.bre.ar}{\verboinum{4}}
\verb{debruar}{}{}{}{}{v.t.}{Costurar debrum em roupa.}{de.bru.ar}{\verboinum{1}}
\verb{debruçar}{}{}{}{}{v.t.}{Colocar alguém com o peito e o rosto para baixo.}{de.bru.çar}{0}
\verb{debruçar}{}{}{}{}{}{Tombar para a frente; inclinar"-se.}{de.bru.çar}{\verboinum{3}}
\verb{debrum}{}{}{"-uns}{}{s.m.}{Tira que se costura dobrada sobre a borda de um roupa para que ela não desfie.}{de.brum}{0}
\verb{debulha}{}{}{}{}{s.f.}{Extração de grãos, bagos ou sementes.}{de.bu.lha}{0}
\verb{debulha}{}{}{}{}{}{Extração da casca de cereais, frutas, legumes etc.}{de.bu.lha}{0}
\verb{debulhadora}{ô}{}{}{}{s.f.}{Máquina de debulhar cereais.}{de.bu.lha.do.ra}{0}
\verb{debulhar}{}{}{}{}{v.t.}{Extrair grãos ou sementes.}{de.bu.lhar}{0}
\verb{debulhar}{}{}{}{}{v.pron.}{Desmanchar"-se; desatar"-se. (\textit{A menina debulhou"-se em lágrimas.})}{de.bu.lhar}{\verboinum{1}}
\verb{debutante}{}{}{}{}{adj.2g.}{Diz"-se de quem se inicia em algo.}{de.bu.tan.te}{0}
\verb{debutante}{}{}{}{}{s.f.}{Moça que estreia na vida social.}{de.bu.tan.te}{0}
\verb{debutar}{}{}{}{}{v.i.}{Iniciar"-se em alguma atividade.}{de.bu.tar}{0}
\verb{debutar}{}{}{}{}{}{Estrear"-se na vida social.}{de.bu.tar}{\verboinum{1}}
\verb{debuxar}{ch}{}{}{}{v.t.}{Fazer o debuxo de algo; desenhar, esboçar.}{de.bu.xar}{\verboinum{1}}
\verb{debuxo}{ch}{}{}{}{s.m.}{Desenho de um objeto em suas linhas gerais; esboço.}{de.bu.xo}{0}
\verb{década}{}{}{}{}{s.f.}{Espaço de dez anos; decênio.}{dé.ca.da}{0}
\verb{decadência}{}{}{}{}{s.f.}{Estado daquele ou daquilo que decai; degradação, declínio.}{de.ca.dên.cia}{0}
\verb{decadente}{}{}{}{}{adj.2g.}{Que está em decadência, que definha ou se corrompe.}{de.ca.den.te}{0}
\verb{decaedro}{é}{Geom.}{}{}{s.m.}{Poliedro de dez faces.}{de.ca.e.dro}{0}
\verb{decágono}{}{Geom.}{}{}{s.m.}{Polígono que tem dez lados e dez ângulos.}{de.cá.go.no}{0}
\verb{decagrama}{}{}{}{}{s.m.}{Medida de peso equivalente a dez gramas.}{de.ca.gra.ma}{0}
\verb{decaída}{}{}{}{}{s.f.}{Efeito de decair; decaimento.}{de.ca.í.da}{0}
\verb{decaída}{}{}{}{}{}{Mulher que caiu na prostituição; meretriz.}{de.ca.í.da}{0}
\verb{decair}{}{}{}{}{v.i.}{Entrar em decadência; declinar.}{de.ca.ir}{0}
\verb{decair}{}{}{}{}{}{Sofrer perdas; baixar, diminuir.}{de.ca.ir}{\verboinum{19}}
\verb{decalcar}{}{}{}{}{v.t.}{Copiar um desenho através de um papel transparente ou comprimindo"-o sobre uma superfície.}{de.cal.car}{0}
\verb{decalcar}{}{Fig.}{}{}{}{Imitar servilmente, quase copiando.}{de.cal.car}{\verboinum{2}}
\verb{decalcomania}{}{}{}{}{s.f.}{Processo de transportar desenhos ou imagens coloridas de um papel para outro papel, ou para uma superfície, calcando"-os diretamente e retirando"-os depois de umedecidos.}{de.cal.co.ma.ni.a}{0}
\verb{decalitro}{}{}{}{}{s.m.}{Medida de capacidade equivalente a dez litros.}{de.ca.li.tro}{0}
\verb{decálogo}{}{Relig.}{}{}{s.m.}{Conjunto dos dez mandamentos bíblicos da lei de Deus.}{de.cá.lo.go}{0}
\verb{decalque}{}{}{}{}{s.m.}{Ato ou efeito de decalcar; copiar.}{de.cal.que}{0}
\verb{decalque}{}{}{}{}{}{Desenho que se faz decalcando.}{de.cal.que}{0}
\verb{decâmetro}{}{}{}{}{s.m.}{Unidade de comprimento equivalente a dez metros.}{de.câ.me.tro}{0}
\verb{decanato}{}{}{}{}{s.m.}{Cada uma das três divisões de um signo do zodíaco.}{de.ca.na.to}{0}
\verb{decanato}{}{}{}{}{}{Cargo ou dignidade de deão ou decano.}{de.ca.na.to}{0}
\verb{decano}{}{}{}{}{s.m.}{O membro mais velho de uma organização.}{de.ca.no}{0}
\verb{decantação}{}{}{"-ões}{}{s.f.}{Exaltação em verso ou canto; celebração.}{de.can.ta.ção}{0}
\verb{decantação}{}{}{"-ões}{}{}{Filtragem das impurezas de um líquido.}{de.can.ta.ção}{0}
\verb{decantado}{}{}{}{}{adj.}{Que é famoso, notável.}{de.can.ta.do}{0}
\verb{decantar}{}{}{}{}{v.t.}{Separar, por ação da gravidade, impurezas sólidas presentes em um líquido.}{de.can.tar}{0}
\verb{decantar}{}{}{}{}{}{Exaltar em cantos ou versos.}{de.can.tar}{\verboinum{1}}
\verb{decapitação}{}{}{"-ões}{}{s.f.}{Ato ou efeito de decapitar; degolação, decepamento.}{de.ca.pi.ta.ção}{0}
\verb{decapitar}{}{}{}{}{v.t.}{Cortar a cabeça de pessoa ou animal; degolar.}{de.ca.pi.tar}{\verboinum{1}}
\verb{decápode}{}{}{}{}{adj.}{Que tem dez pés, patas ou qualquer outra espécie de membro locomotor.}{de.cá.po.de}{0}
\verb{decasségui}{}{}{}{}{s.m.}{Descendente de japonês trabalhando temporariamente no Japão.}{de.cas.sé.gui}{0}
\verb{decassílabo}{}{Gram.}{}{}{adj.}{Que tem dez sílabas.}{de.cas.sí.la.bo}{0}
\verb{decatlo}{}{}{}{}{s.m.}{Conjunto de dez provas atléticas.}{de.ca.tlo}{0}
\verb{decenal}{}{}{"-ais}{}{adj.2g.}{Que abrange um período de dez anos.}{de.ce.nal}{0}
\verb{decenal}{}{}{"-ais}{}{}{Que acontece a cada dez anos.}{de.ce.nal}{0}
\verb{decência}{}{}{}{}{s.f.}{Conformidade com os padrões morais e éticos da sociedade; dignidade.}{de.cên.cia}{0}
\verb{decência}{}{}{}{}{}{Atitude de modéstia; honradez, honestidade.}{de.cên.cia}{0}
\verb{decêndio}{}{}{}{}{s.m.}{Espaço de dez dias.}{de.cên.dio}{0}
\verb{decênio}{}{}{}{}{s.m.}{Espaço de dez anos; década.}{de.cê.nio}{0}
\verb{decente}{}{}{}{}{adj.2g.}{Que obedece aos costumes da sociedade.}{de.cen.te}{0}
\verb{decepar}{}{}{}{}{v.t.}{Cortar parte de um corpo; mutilar.}{de.ce.par}{0}
\verb{decepar}{}{}{}{}{}{Decapitar, degolar.}{de.ce.par}{\verboinum{1}}
\verb{decepção}{}{}{"-ões}{}{s.f.}{Sentimento de tristeza, descontentamento ou frustração pela ocorrência de fato inesperado, que representa um mal; desilusão, desapontamento.}{de.cep.ção}{0}
\verb{decepcionar}{}{}{}{}{v.t.}{Causar decepção a alguém; desencantar, desiludir, frustrar.}{de.cep.ci.o.nar}{\verboinum{1}}
\verb{decerto}{é}{}{}{}{adv.}{Sem nenhuma dúvida, certamente.}{de.cer.to}{0}
\verb{decibel}{é}{}{"-éis}{}{s.m.}{Unidade de medida de intensidade do som.}{de.ci.bel}{0}
\verb{decidido}{}{}{}{}{adj.}{Sobre o que se tomou decisão; resolvido, definido, determinado.}{de.ci.di.do}{0}
\verb{decidido}{}{}{}{}{}{Que toma decisões; determinado, resoluto. }{de.ci.di.do}{0}
\verb{decidir}{}{}{}{}{v.t.}{Escolher o que se vai fazer; resolver.}{de.ci.dir}{0}
\verb{decidir}{}{}{}{}{}{Dar a solução final a alguma coisa; solucionar.}{de.ci.dir}{0}
\verb{decidir}{}{}{}{}{}{Dar preferência a uma pessoa ou coisa, desprezando outras.}{de.ci.dir}{\verboinum{18}}
\verb{decíduo}{}{Biol.}{}{}{adj.}{Que cai ou se solta em uma estação específica ou em certa fase do desenvolvimento.}{de.cí.du.o}{0}
\verb{decifrar}{}{}{}{}{v.t.}{Compreender texto ou mensagem codificada ou de difícil leitura; decodificar, interpretar.}{de.ci.frar}{\verboinum{1}}
\verb{decigrama}{}{}{}{}{s.m.}{Unidade de medida de massa equivalente a um décimo de um grama.}{de.ci.gra.ma}{0}
\verb{decilitro}{}{}{}{}{s.m.}{Unidade de medida de volume correspondente a um décimo de um litro.}{de.ci.li.tro}{0}
\verb{décima}{}{}{}{}{s.f.}{Cada uma das dez partes iguais em que se divide um todo.}{dé.ci.ma}{0}
\verb{décima}{}{Liter.}{}{}{}{Estrofe de dez versos.}{dé.ci.ma}{0}
\verb{decimal}{}{}{}{}{adj.2g.}{Relativo a dez.}{de.ci.mal}{0}
\verb{decimal}{}{Mat.}{}{}{}{Diz"-se do sistema de numeração em que cada algarismo pode assumir dez valores.}{de.ci.mal}{0}
\verb{decimal}{}{Mat.}{}{}{}{Diz"-se dos números de qualquer casa à direita da vírgula.}{de.ci.mal}{0}
\verb{decímetro}{}{}{}{}{s.m.}{Unidade de medida de comprimento equivalente a um décimo de um metro.}{de.cí.me.tro}{0}
\verb{décimo}{}{}{}{}{num.}{Ordinal e fracionário correspondente a dez.}{dé.ci.mo}{0}
\verb{décimo}{}{}{}{}{s.m.}{A décima parte.}{dé.ci.mo}{0}
\verb{decisão}{}{}{}{}{s.f.}{Ato ou efeito de decidir.}{de.ci.são}{0}
\verb{decisão}{}{}{}{}{}{Resolução, opção, definição, deliberação. (\textit{Finalmente chegaram a uma decisão.})}{de.ci.são}{0}
\verb{decisão}{}{}{}{}{}{Coragem, firmeza. (\textit{Seus gestos mostravam decisão.})}{de.ci.são}{0}
\verb{decisivo}{}{}{}{}{adj.}{Que decide.}{de.ci.si.vo}{0}
\verb{decisivo}{}{}{}{}{}{Definitivo, categórico.}{de.ci.si.vo}{0}
\verb{decisório}{}{}{}{}{adj.}{Que decide; decisivo.}{de.ci.só.rio}{0}
\verb{decisório}{}{}{}{}{}{Que tem poder de decidir.}{de.ci.só.rio}{0}
\verb{declamação}{}{}{"-ões}{}{s.f.}{Ato, efeito ou técnica de declamar.}{de.cla.ma.ção}{0}
\verb{declamador}{ô}{}{}{}{adj.}{Que declama; recitador.}{de.cla.ma.dor}{0}
\verb{declamar}{}{}{}{}{v.t.}{Recitar em voz alta. (\textit{Ele gosta de declamar poesias.})}{de.cla.mar}{\verboinum{1}}
\verb{declamatório}{}{}{}{}{adj.}{Relativo a declamação.}{de.cla.ma.tó.rio}{0}
\verb{declaração}{}{}{"-ões}{}{s.f.}{Ato ou efeito de declarar.}{de.cla.ra.ção}{0}
\verb{declaração}{}{}{"-ões}{}{}{Manifestação oral ou escrita; revelação, anúncio.}{de.cla.ra.ção}{0}
\verb{declarado}{}{}{}{}{adj.}{Que se declarou; anunciado, manifestado.}{de.cla.ra.do}{0}
\verb{declarante}{}{}{}{}{adj.2g.}{Que declara, que manifesta.}{de.cla.ran.te}{0}
\verb{declarar}{}{}{}{}{v.t.}{Dar ao conhecimento alheio; manifestar, anunciar.}{de.cla.rar}{0}
\verb{declarar}{}{}{}{}{}{Nomear, eleger.}{de.cla.rar}{0}
\verb{declarar}{}{}{}{}{}{Tornar um bem de conhecimento público para a fiscalização.}{de.cla.rar}{0}
\verb{declarar}{}{}{}{}{v.pron.}{Pronunciar"-se, manifestar"-se.}{de.cla.rar}{0}
\verb{declarar}{}{}{}{}{}{Revelar sentimento.}{de.cla.rar}{\verboinum{1}}
\verb{declarativo}{}{}{}{}{adj.}{Relativo a declaração.}{de.cla.ra.ti.vo}{0}
\verb{declarativo}{}{Gram.}{}{}{}{Diz"-se de sentença que se refere objetivamente a um fato sem emitir opinião.}{de.cla.ra.ti.vo}{0}
\verb{declinação}{}{}{"-ões}{}{s.f.}{Ato ou efeito de declinar.}{de.cli.na.ção}{0}
\verb{declinação}{}{}{"-ões}{}{}{Decadência, diminuição.}{de.cli.na.ção}{0}
\verb{declinação}{}{Gram.}{"-ões}{}{}{Flexão de caso em substantivos, adjetivos e pronomes. (\textit{A língua latina apresenta declinação.})}{de.cli.na.ção}{0}
\verb{declinação}{}{Gram.}{"-ões}{}{}{Cada grupo de palavras que declina da mesma maneira. (\textit{Essa palavra pertence à segunda declinação.})}{de.cli.na.ção}{0}
\verb{declinar}{}{}{}{}{v.i.}{Afastar"-se de um ponto.}{de.cli.nar}{0}
\verb{declinar}{}{}{}{}{}{Diminuir, decair.}{de.cli.nar}{0}
\verb{declinar}{}{}{}{}{v.t.}{Recusar, desistir, eximir"-se.}{de.cli.nar}{0}
\verb{declinar}{}{Gram.}{}{}{}{Enunciar as flexões de substantivos, adjetivos e pronomes.}{de.cli.nar}{\verboinum{1}}
\verb{declinável}{}{}{}{}{adj.2g.}{Que pode ser declinado.}{de.cli.ná.vel}{0}
\verb{declinável}{}{Gram.}{}{}{}{Diz"-se de palavras em que há flexão de caso.}{de.cli.ná.vel}{0}
\verb{declínio}{}{}{}{}{s.m.}{Decadência, diminuição.}{de.clí.nio}{0}
\verb{declive}{}{}{}{}{s.m.}{Inclinação de cima para baixo; descida.}{de.cli.ve}{0}
\verb{decocção}{}{}{"-ões}{}{s.f.}{Processo de extração de substâncias dos vegetais através de fervura prolongada em água.}{de.coc.ção}{0}
\verb{decodificação}{}{}{"-ões}{}{s.f.}{Ato ou efeito de decodificar.}{de.co.di.fi.ca.ção}{0}
\verb{decodificador}{ô}{}{}{}{adj.}{Que decodifica.}{de.co.di.fi.ca.dor}{0}
\verb{decodificador}{ô}{}{}{}{s.m.}{Sistema, dispositivo ou aparelho que recompõe mensagens ou sinais transmitidos em um processo de comunicação.}{de.co.di.fi.ca.dor}{0}
\verb{decodificar}{}{}{}{}{v.t.}{Transformar mensagem codificada em mensagem inteligível; decifrar.}{de.co.di.fi.car}{\verboinum{1}}
\verb{decolagem}{}{}{}{}{s.f.}{Ato ou efeito de decolar.}{de.co.la.gem}{0}
\verb{decolar}{}{}{}{}{v.i.}{Levantar voo.}{de.co.lar}{\verboinum{1}}
\verb{decomponente}{}{}{}{}{adj.2g.}{Que decompõe.}{de.com.po.nen.te}{0}
\verb{decomponível}{}{}{"-eis}{}{adj.2g.}{Que pode ser decomposto.}{de.com.po.ní.vel}{0}
\verb{decompor}{}{}{}{}{v.t.}{Separar as partes que compõem um todo; desmontar, analisar.}{de.com.por}{0}
\verb{decompor}{}{}{}{}{v.i.}{Entrar em processo de putrefação; apodrecer.}{de.com.por}{\verboinum{60}}
\verb{decomposição}{}{}{"-ões}{}{s.f.}{Ato ou efeito de decompor.}{de.com.po.si.ção}{0}
\verb{decomposto}{ô}{}{"-s ⟨ó⟩}{"-a ⟨ó⟩}{adj.}{Que se decompôs.}{de.com.pos.to}{0}
\verb{decomposto}{ô}{}{"-s ⟨ó⟩}{"-a ⟨ó⟩}{}{Apodrecido.}{de.com.pos.to}{0}
\verb{decoração}{}{}{"-ões}{}{s.f.}{Ato ou efeito de decorar.}{de.co.ra.ção}{0}
\verb{decoração}{}{}{"-ões}{}{}{Conjunto de enfeites, adornos.}{de.co.ra.ção}{0}
\verb{decorador}{ô}{}{}{}{s.m.}{Indivíduo especializado na decoração de ambientes.}{de.co.ra.dor}{0}
\verb{decorar}{}{}{}{}{v.t.}{Colocar elementos decorativos; enfeitar. (\textit{Preciso decorar a sala.})}{de.co.rar}{0}
\verb{decorar}{}{}{}{}{}{Servir de elemento decorativo. (\textit{Os quadros antigos decoram a biblioteca.})}{de.co.rar}{\verboinum{1}}
\verb{decorar}{}{}{}{}{v.t.}{Reter na memória; memorizar. (\textit{Vamos decorar a tabuada do nove.})}{de.co.rar}{\verboinum{1}}
\verb{decorativo}{}{}{}{}{adj.}{Relativo a ou que serve de decoração.}{de.co.ra.ti.vo}{0}
\verb{decoreba}{é}{Pejor.}{}{}{s.f.}{Ação de memorizar informações, geralmente sem estabelecer relações entre elas, com a finalidade de prestar exames.}{de.co.re.ba}{0}
\verb{decoro}{ô}{}{}{}{s.m.}{Decência, compostura, dignidade, seriedade.}{de.co.ro}{0}
\verb{decoroso}{ô}{}{"-osos ⟨ó⟩}{"-osa ⟨ó⟩}{adj.}{Que tem decoro; decente, honrado.}{de.co.ro.so}{0}
\verb{decorrência}{}{}{}{}{s.f.}{Ato ou efeito de decorrer; consequência.}{de.cor.rên.cia}{0}
\verb{decorrente}{}{}{}{}{adj.2g.}{Que decorre. (\textit{Houve prejuízos grandes decorrentes das enchentes do mês de janeiro.})}{de.cor.ren.te}{0}
\verb{decorrer}{ê}{}{}{}{v.i.}{Passar, transcorrer, desenrolar"-se. (\textit{As negociações decorreram normalmente.})}{de.cor.rer}{0}
\verb{decorrer}{ê}{}{}{}{}{Suceder, acontecer, ocorrer. (\textit{Algo estranho decorreu aqui ontem.})}{de.cor.rer}{0}
\verb{decorrer}{ê}{}{}{}{v.t.}{Ter origem em; ter como causa; derivar. (\textit{A violência no Brasil decorre da concentração de renda.})}{de.cor.rer}{\verboinum{12}}
\verb{decotado}{}{}{}{}{adj.}{Diz"-se de peça de vestuário que tem decote.}{de.co.ta.do}{0}
\verb{decotar}{}{}{}{}{v.t.}{Fazer decote em peça de vestuário.}{de.co.tar}{0}
\verb{decotar}{}{}{}{}{}{Cortar, aparar ramos de árvores.}{de.co.tar}{\verboinum{1}}
\verb{decote}{ó}{}{}{}{s.m.}{Recorte ou abertura larga na parte de cima de peça de vestuário.}{de.co.te}{0}
\verb{decote}{ó}{}{}{}{}{Ato ou efeito de decotar, aparar; corte, poda.}{de.co.te}{0}
\verb{decrépito}{}{}{}{}{adj.}{Que tem muito tempo de vida ou muito tempo de uso; caduco.}{de.cré.pi.to}{0}
\verb{decrepitude}{}{}{}{}{s.f.}{Qualidade de decrépito.}{de.cre.pi.tu.de}{0}
\verb{decrescente}{}{}{}{}{adj.2g.}{Que se encontra em processo de decrescimento, declínio ou diminuição.}{de.cres.cen.te}{0}
\verb{decrescer}{ê}{}{}{}{v.i.}{Diminuir em quantidade, intensidade, dimensão ou duração.}{de.cres.cer}{\verboinum{15}}
\verb{decréscimo}{}{}{}{}{s.m.}{Ato ou efeito de decrescer. (\textit{Um decréscimo populacional ocorreu durante a guerra.})}{de.crés.ci.mo}{0}
\verb{decréscimo}{}{}{}{}{}{Quantificação de um decréscimo. (\textit{O decréscimo salarial foi de 16\%.})}{de.crés.ci.mo}{0}
\verb{decretar}{}{}{}{}{v.t.}{Determinar ou ordenar por lei ou decreto.}{de.cre.tar}{\verboinum{1}}
\verb{decreto}{é}{}{}{}{s.m.}{Decisão de chefe de Estado.}{de.cre.to}{0}
\verb{decreto"-lei}{é}{}{}{}{s.m.}{Decreto que tem força ou efeito de lei.}{de.cre.to"-lei}{0}
\verb{decúbito}{}{}{}{}{s.m.}{Posição de um corpo na horizontal.}{de.cú.bi.to}{0}
\verb{decuplar}{}{}{}{}{v.t.}{Decuplicar.}{de.cu.plar}{\verboinum{1}}
\verb{decuplicar}{}{}{}{}{v.t.}{Tornar dez vezes maior.}{de.cu.pli.car}{\verboinum{2}}
\verb{décuplo}{}{}{}{}{num.}{Igual a dez vezes determinada quantidade.}{dé.cu.plo}{0}
\verb{decurso}{}{}{}{}{s.m.}{Ato ou efeito de decorrer.}{de.cur.so}{0}
\verb{dedada}{}{}{}{}{s.f.}{Porção de qualquer coisa que se retira com um dedo.}{de.da.da}{0}
\verb{dedada}{}{}{}{}{}{Toque ou batida aplicada com o dedo.}{de.da.da}{0}
\verb{dedada}{}{}{}{}{}{Sinal deixado com o dedo em alguma superfície.}{de.da.da}{0}
\verb{dedal}{}{}{"-ais}{}{s.m.}{Tipo de capuz que se coloca no dedo para protegê"-lo quando se lida com agulha ao costurar.}{de.dal}{0}
\verb{dedaleira}{ê}{}{}{}{s.f.}{Estojo próprio para guardar dedais.}{de.da.lei.ra}{0}
\verb{dédalo}{}{}{}{}{s.m.}{Emaranhado confuso de caminhos; labirinto.}{dé.da.lo}{0}
\verb{dedão}{}{}{"-ões}{}{s.m.}{O dedo polegar.}{de.dão}{0}
\verb{dedar}{}{Pop.}{}{}{v.t.}{Dedurar.}{de.dar}{\verboinum{1}}
\verb{dedeira}{ê}{}{}{}{s.f.}{Proteção para os dedos da mão.}{de.dei.ra}{0}
\verb{dedetizar}{}{}{}{}{v.t.}{Aplicar \textsc{ddt}, inseticida ou outro tratamento contra insetos. (\textit{Vamos dedetizar a loja para acabar com as baratas.})}{de.de.ti.zar}{\verboinum{1}}
\verb{dedicação}{}{}{"-ões}{}{s.f.}{Ato ou efeito de dedicar; empenho, sacrifício, disposição.}{de.di.ca.ção}{0}
\verb{dedicado}{}{}{}{}{adj.}{Empenhado, prestativo.}{de.di.ca.do}{0}
\verb{dedicar}{}{}{}{}{v.t.}{Destinar, oferecer, ofertar, devotar.}{de.di.car}{0}
\verb{dedicar}{}{}{}{}{v.pron.}{Empenhar"-se, entregar"-se, sacrificar"-se.}{de.di.car}{\verboinum{2}}
\verb{dedicatória}{}{}{}{}{s.f.}{Inscrição manuscrita com a qual se dedica geralmente um livro ou uma fotografia a alguém.}{de.di.ca.tó.ria}{0}
\verb{dedignar"-se}{}{}{}{}{v.pron.}{Considerar com menosprezo; recusar com desdém.}{de.dig.nar"-se}{\verboinum{1}}
\verb{dedilhar}{}{}{}{}{v.t.}{Fazer soar as cordas de instrumento musical, puxando"-as e soltando"-as com os dedos.}{de.di.lhar}{\verboinum{1}}
\verb{dedo}{ê}{Anat.}{}{}{s.m.}{Cada um dos prolongamentos articulados nas extremidades das mãos e dos pés da espécie humana e das patas de alguns animais.}{de.do}{0}
\verb{dedo"-du"-ro}{ê}{Pop.}{}{}{adj.}{Que delata, denuncia; delator, alcaguete.}{de.do"-du"-ro}{0}
\verb{dedução}{}{}{"-ões}{}{s.f.}{Conclusão lógica de um raciocínio.}{de.du.ção}{0}
\verb{dedução}{}{}{"-ões}{}{}{Subtração, diminuição, abatimento.}{de.du.ção}{0}
\verb{dedurar}{}{Pop.}{}{}{v.t.}{Delatar, denunciar, alcaguetar.}{de.du.rar}{\verboinum{1}}
\verb{dedutível}{}{}{"-eis}{}{adj.2g.}{Que se pode deduzir.}{de.du.tí.vel}{0}
\verb{dedutivo}{}{}{}{}{adj.}{Relativo a ou que procede por dedução.}{de.du.ti.vo}{0}
\verb{deduzir}{}{}{}{}{v.t.}{Tirar como consequência; concluir, inferir.}{de.du.zir}{0}
\verb{deduzir}{}{}{}{}{}{Abater, diminuir, subtrair.}{de.du.zir}{\verboinum{21}}
\verb{defasado}{}{}{}{}{adj.}{Em que há defasagem.}{de.fa.sa.do}{0}
\verb{defasagem}{}{Fís.}{}{}{s.f.}{Diferença de fase.}{de.fa.sa.gem}{0}
\verb{defasagem}{}{Fig.}{}{}{}{Descompasso, discrepância, atraso.}{de.fa.sa.gem}{0}
\verb{defasar}{}{}{}{}{}{Ficar para trás; atrasar. (\textit{As dificuldades defasaram sua promoção.})}{de.fa.sar}{0}
\verb{defasar}{}{}{}{}{v.t.}{Pôr fora de fase.}{de.fa.sar}{\verboinum{1}}
\verb{default}{}{Informát.}{}{}{s.m.}{Valor previamente estabelecido em um programa para ser assumido na falta de uma escolha do usuário.}{\textit{default}}{0}
\verb{defecação}{}{}{"-ões}{}{s.f.}{Ato de defecar.}{de.fe.ca.ção}{0}
\verb{defecar}{}{}{}{}{v.i.}{Eliminar fezes; cagar.}{de.fe.car}{\verboinum{2}}
\verb{defecção}{}{}{"-ões}{}{s.f.}{Abandono de um compromisso, uma causa ou uma opinião; deserção.}{de.fec.ção}{0}
\verb{defectível}{}{}{"-eis}{}{adj.2g.}{Que tem defeito; imperfeito, defeituoso, falível.}{de.fec.tí.vel}{0}
\verb{defectivo}{}{}{}{}{adj.}{Que apresenta defeito; defeituoso, imperfeito.}{de.fec.ti.vo}{0}
\verb{defeito}{ê}{}{}{}{s.m.}{Imperfeição, deficiência, deformidade.}{de.fei.to}{0}
\verb{defeito}{ê}{}{}{}{}{Imperfeição moral; vício.}{de.fei.to}{0}
\verb{defeituoso}{ô}{}{"-osos ⟨ó⟩}{"-osa ⟨ó⟩}{adj.}{Que apresenta defeito.}{de.fei.tu.o.so}{0}
\verb{defender}{ê}{}{}{}{v.t.}{Proteger contra ataque ou perigo. (\textit{Ele já é grande e sabe se defender.})}{de.fen.der}{0}
\verb{defender}{ê}{}{}{}{}{Argumentar em favor de. (\textit{O advogado defendeu muito bem o deputado.})}{de.fen.der}{0}
\verb{defender}{ê}{}{}{}{}{Preservar. (\textit{Devemos defender a Amazônia.})}{de.fen.der}{\verboinum{12}}
\verb{defensável}{}{}{"-eis}{}{adj.2g.}{Que se pode defender.}{de.fen.sá.vel}{0}
\verb{defensiva}{}{}{}{}{s.f.}{Posição de quem se defende.}{de.fen.si.va}{0}
\verb{defensivo}{}{}{}{}{adj.}{Que defende ou serve para defender.}{de.fen.si.vo}{0}
\verb{defensor}{ô}{}{}{}{adj.}{Que defende, protege.}{de.fen.sor}{0}
\verb{deferência}{}{}{}{}{s.f.}{Consideração, respeito, reverência.}{de.fe.rên.cia}{0}
\verb{deferente}{}{}{}{}{adj.2g.}{Respeitoso, atencioso, reverente.}{de.fe.ren.te}{0}
\verb{deferimento}{}{}{}{}{s.m.}{Ato de deferir; aprovação.}{de.fe.ri.men.to}{0}
\verb{deferir}{}{}{}{}{v.t.}{Atender, anuir, conceder.}{de.fe.rir}{\verboinum{29}}
\verb{defesa}{ê}{}{}{}{s.f.}{Ato ou efeito de defender.}{de.fe.sa}{0}
\verb{defesa}{ê}{}{}{}{}{Argumento para contestar uma acusação.}{de.fe.sa}{0}
\verb{defesa}{ê}{}{}{}{}{Meio ou recurso para se defender.}{de.fe.sa}{0}
\verb{defeso}{ê}{}{}{}{adj.}{Proibido, impedido, vedado.}{de.fe.so}{0}
\verb{deficiência}{}{}{}{}{s.f.}{Falha, imperfeição, defeito, carência.}{de.fi.ci.ên.cia}{0}
\verb{deficiente}{}{}{}{}{adj.2g.}{Falho, imperfeito, carente.}{de.fi.ci.en.te}{0}
\verb{déficit}{}{}{}{}{s.m.}{Situação em que a despesa é maior que a receita.}{dé.fi.cit}{0}
\verb{déficit}{}{}{}{}{}{Aquilo que falta para completar uma necessidade.}{dé.fi.cit}{0}
\verb{deficitário}{}{}{}{}{adj.}{Que apresenta déficit.}{de.fi.ci.tá.rio}{0}
\verb{definhado}{}{}{}{}{adj.}{Debilitado, enfraquecido, emagrecido, murcho.}{de.fi.nha.do}{0}
\verb{definhar}{}{}{}{}{v.t.}{Tornar magro; enfraquecer.}{de.fi.nhar}{0}
\verb{definhar}{}{}{}{}{v.i.}{Consumir"-se lenta e continuamente.}{de.fi.nhar}{0}
\verb{definhar}{}{}{}{}{}{Murchar, secar.}{de.fi.nhar}{\verboinum{1}}
\verb{definição}{}{}{"-ões}{}{}{Enunciado construído para definir algo.}{de.fi.ni.ção}{0}
\verb{definição}{}{}{"-ões}{}{s.f.}{Ato de definir.}{de.fi.ni.ção}{0}
\verb{definição}{}{}{"-ões}{}{}{Resolução, decisão.}{de.fi.ni.ção}{0}
\verb{definido}{}{}{}{}{adj.}{Determinado, demarcado, exato.}{de.fi.ni.do}{0}
\verb{definir}{}{}{}{}{v.t.}{Enunciar as características específicas de algo.}{de.fi.nir}{0}
\verb{definir}{}{}{}{}{}{Explicar o significado de.}{de.fi.nir}{0}
\verb{definir}{}{}{}{}{}{Determinar com precisão.}{de.fi.nir}{0}
\verb{definir}{}{}{}{}{}{Decidir.}{de.fi.nir}{0}
\verb{definir}{}{}{}{}{v.pron.}{Tomar uma resolução; determinar"-se.}{de.fi.nir}{\verboinum{18}}
\verb{definitivo}{}{}{}{}{adj.}{Categórico, inapelável, decisivo, final.}{de.fi.ni.ti.vo}{0}
\verb{deflação}{}{}{"-ões}{}{s.f.}{Diminuição dos preços dos produtos.}{de.fla.ção}{0}
\verb{deflacionar}{}{}{}{}{v.t.}{Provocar a deflação de.}{de.fla.ci.o.nar}{\verboinum{1}}
\verb{deflacionista}{}{}{}{}{adj.2g.}{Relativo a deflação.}{de.fla.ci.o.nis.ta}{0}
\verb{deflagração}{}{}{"-ões}{}{s.f.}{Ato ou efeito de deflagrar.}{de.fla.gra.ção}{0}
\verb{deflagrar}{}{}{}{}{v.i.}{Inflamar"-se.}{de.fla.grar}{0}
\verb{deflagrar}{}{}{}{}{v.t.}{Provocar, atear.}{de.fla.grar}{\verboinum{1}}
\verb{defloração}{}{}{"-ões}{}{s.f.}{Queda das flores.}{de.flo.ra.ção}{0}
\verb{defloração}{}{}{"-ões}{}{}{Retirada ou perda da virgindade.}{de.flo.ra.ção}{0}
\verb{defloramento}{}{}{}{}{s.m.}{Defloração.}{de.flo.ra.men.to}{0}
\verb{deflorar}{}{}{}{}{v.t.}{Tirar as flores de; desflorar.}{de.flo.rar}{0}
\verb{deflorar}{}{}{}{}{}{Tirar a virgindade.}{de.flo.rar}{\verboinum{1}}
\verb{defluir}{}{}{}{}{v.t.}{Decorrer, derivar.}{de.flu.ir}{0}
\verb{defluir}{}{}{}{}{v.i.}{Fluir, derivar.}{de.flu.ir}{\verboinum{26}}
\verb{defluxo}{cs}{Pop.}{}{}{s.m.}{Secreção nasal anormal; coriza.}{de.flu.xo}{0}
\verb{deformação}{}{}{"-ões}{}{s.f.}{Ato ou efeito de deformar.}{de.for.ma.ção}{0}
\verb{deformar}{}{}{}{}{v.t.}{Alterar a forma; modificar, deturpar, desfigurar.}{de.for.mar}{0}
\verb{deformar}{}{}{}{}{v.i.}{Perder a forma; sofrer alteração.}{de.for.mar}{\verboinum{1}}
\verb{deformidade}{}{}{}{}{s.f.}{Desfiguração, defeito, vício.}{de.for.mi.da.de}{0}
\verb{defraudação}{}{}{"-ões}{}{s.f.}{Ato ou efeito de defraudar.}{de.frau.da.ção}{0}
\verb{defraudar}{}{}{}{}{v.t.}{Espoliar através de fraude; fraudar.}{de.frau.dar}{0}
\verb{defraudar}{}{}{}{}{}{Burlar, enganar, lesar.}{de.frau.dar}{\verboinum{1}}
\verb{defrontar}{}{}{}{}{v.t.}{Pôr"-se defronte.}{de.fron.tar}{0}
\verb{defrontar}{}{}{}{}{}{Enfrentar, arrostar, encarar.}{de.fron.tar}{0}
\verb{defrontar}{}{}{}{}{}{Topar, encontrar.}{de.fron.tar}{\verboinum{1}}
\verb{defronte}{}{}{}{}{adv.}{Em frente; frente a frente.}{de.fron.te}{0}
\verb{defumar}{}{}{}{}{v.t.}{Secar ou aromatizar na fumaça.}{de.fu.mar}{\verboinum{1}}
\verb{defunto}{}{}{}{}{adj.}{Que faleceu.}{de.fun.to}{0}
\verb{degelar}{}{}{}{}{v.t.}{Fazer derreter; descongelar.}{de.ge.lar}{0}
\verb{degelar}{}{}{}{}{v.i.}{Derreter"-se, descongelar"-se.}{de.ge.lar}{\verboinum{1}}
\verb{degelo}{ê}{}{}{}{s.m.}{Ato ou efeito de degelar.}{de.ge.lo}{0}
\verb{degeneração}{}{}{"-ões}{}{s.f.}{Ato ou efeito de degenerar.}{de.ge.ne.ra.ção}{0}
\verb{degeneração}{}{}{"-ões}{}{}{Perda ou alteração no ser vivo das qualidades de sua espécie.}{de.ge.ne.ra.ção}{0}
\verb{degeneração}{}{}{"-ões}{}{}{Mudança para um estado pior; declínio.}{de.ge.ne.ra.ção}{0}
\verb{degeneração}{}{Fig.}{"-ões}{}{}{Estado de depravação.}{de.ge.ne.ra.ção}{0}
\verb{degenerado}{}{}{}{}{adj.}{Diz"-se de ser vivo que perdeu ou teve alteradas as qualidades próprias de sua espécie.}{de.ge.ne.ra.do}{0}
\verb{degenerado}{}{}{}{}{}{Que passou a um estado ou condição qualitativamente inferior; decaído.}{de.ge.ne.ra.do}{0}
\verb{degenerado}{}{}{}{}{}{Diz"-se de indivíduo depravado, corrompido.}{de.ge.ne.ra.do}{0}
\verb{degenerar}{}{}{}{}{v.i.}{Perder as características que são boas.}{de.ge.ne.rar}{\verboinum{1}}
\verb{degenerativo}{}{}{}{}{adj.}{Que produz degeneração.}{de.ge.ne.ra.ti.vo}{0}
\verb{degenerescência}{}{}{}{}{s.f.}{Alteração das características naturais de um corpo organizado.}{de.ge.ne.res.cên.cia}{0}
%\verb{}{}{}{}{}{}{}{}{0}
\verb{deglutição}{}{}{"-ões}{}{s.f.}{Ato ou efeito de deglutir, de engolir.}{de.glu.ti.ção}{0}
\verb{deglutir}{}{}{}{}{v.t.}{Passar da boca para o estômago; engolir.}{de.glu.tir}{\verboinum{18}}
\verb{degola}{ó}{}{}{}{s.f.}{Ato ou efeito de degolar; corte, incisão no pescoço.}{de.go.la}{0}
\verb{degola}{ó}{}{}{}{}{Decepamento da cabeça; decapitação.}{de.go.la}{0}
\verb{degola}{ó}{Pop.}{}{}{}{Exclusão em massa; afastamento, corte.}{de.go.la}{0}
\verb{degolação}{}{}{"-ões}{}{s.f.}{Degola.}{de.go.la.ção}{0}
\verb{degolar}{}{}{}{}{v.t.}{Cortar o pescoço.}{de.go.lar}{0}
\verb{degolar}{}{}{}{}{}{Cortar, decepar a cabeça; decapitar.}{de.go.lar}{\verboinum{1}}
\verb{degradação}{}{}{"-ões}{}{s.f.}{Estado de pessoa que perdeu a honra e a dignidade.}{de.gra.da.ção}{0}
\verb{degradação}{}{}{"-ões}{}{}{Rebaixamento de grau e função.}{de.gra.da.ção}{0}
\verb{degradado}{}{}{}{}{adj.}{Que foi privado de dignidade; rebaixado.}{de.gra.da.do}{0}
\verb{degradante}{}{}{}{}{adj.2g.}{Que provoca degradação; aviltante, humilhante, deteriorante.}{de.gra.dan.te}{0}
\verb{degradar}{}{}{}{}{v.t.}{Fazer alguém perder a honra ou a dignidade; aviltar.}{de.gra.dar}{0}
\verb{degradar}{}{}{}{}{}{Tirar ou privar de grau ou função; rebaixar.}{de.gra.dar}{\verboinum{1}}
\verb{degrau}{}{}{}{}{s.m.}{Cada uma das partes de uma escada em que se põe o pé para subir ou descer.}{de.grau}{0}
\verb{degredado}{}{}{}{}{adj.}{Diz"-se daquele que sofreu pena de degredo; exilado.}{de.gre.da.do}{0}
\verb{degredar}{}{}{}{}{v.t.}{Impor a pena de degredo a alguém; desterrar, exilar.}{de.gre.dar}{\verboinum{1}}
\verb{degredo}{ê}{}{}{}{s.m.}{Pena de desterro que a justiça impõe a criminosos; exílio.}{de.gre.do}{0}
\verb{degredo}{ê}{}{}{}{}{Lugar onde se cumpre a pena de degredo.}{de.gre.do}{0}
\verb{degringolada}{}{}{}{}{s.f.}{Queda precipitada.}{de.grin.go.la.da}{0}
\verb{degringolada}{}{Fig.}{}{}{}{Decadência, queda, ruína.}{de.grin.go.la.da}{0}
\verb{degringolar}{}{}{}{}{v.t.}{Arruinar"-se ou permanecer em situação de ruína, de decadência; desandar, decair.}{de.grin.go.lar}{\verboinum{1}}
\verb{degustação}{}{}{"-ões}{}{s.f.}{Avaliação pelo paladar.}{de.gus.ta.ção}{0}
\verb{degustação}{}{}{"-ões}{}{}{Experiência aprazível.}{de.gus.ta.ção}{0}
\verb{degustador}{ô}{}{}{}{adj.}{Que degusta; provador.}{de.gus.ta.dor}{0}
\verb{degustar}{}{}{}{}{v.t.}{Avaliar pelo paladar o sabor de algo; provar.}{de.gus.tar}{0}
\verb{degustar}{}{Fig.}{}{}{}{Apreciar com suavidade e atenção.}{de.gus.tar}{\verboinum{1}}
\verb{deidade}{}{}{}{}{s.f.}{Divindade; deus ou deusa.}{dei.da.de}{0}
\verb{deidade}{}{Fig.}{}{}{}{Mulher muito bonita.}{dei.da.de}{0}
\verb{deificação}{}{}{"-ões}{}{s.f.}{Ato ou efeito de deificar; divinização.}{de.i.fi.ca.ção}{0}
\verb{deificar}{}{}{}{}{v.t.}{Atribuir natureza divina; endeusar, divinizar.}{de.i.fi.car}{\verboinum{2}}
\verb{deiscência}{}{Bot.}{}{}{s.f.}{Abertura espontânea de órgãos ou partes vegetais ao alcançarem a maturidade.}{de.is.cên.cia}{0}
\verb{deiscente}{}{Bot.}{}{}{adj.2g.}{Diz"-se de órgão vegetal que apresenta deiscência.}{de.is.cen.te}{0}
\verb{deísmo}{}{Filos.}{}{}{s.m.}{Doutrina que afirma a existência de Deus a partir de evidências racionais; difere do teísmo por considerar que a participação de Deus no mundo se restringiu ao momento de sua criação.}{de.ís.mo}{0}
\verb{deitar}{}{}{}{}{v.t.}{Estender pessoa ou coisa ao comprido.}{dei.tar}{0}
\verb{deitar}{}{}{}{}{}{Fazer alguma coisa cair em determinado lugar.}{dei.tar}{0}
\verb{deitar}{}{}{}{}{v.i.}{Ir para a cama; deitar"-se.}{dei.tar}{\verboinum{1}}
\verb{deixa}{ch}{}{}{}{s.f.}{Ato ou efeito de deixar.}{dei.xa}{0}
\verb{deixa}{ch}{}{}{}{}{Legado, herança.}{dei.xa}{0}
\verb{deixa}{ch}{}{}{}{}{Palavra ou gesto de um ator que indica que é hora de outro falar ou entrar.}{dei.xa}{0}
\verb{deixar}{ch}{}{}{}{v.t.}{Ir embora, separando"-se de pessoa ou coisa; retirar"-se. (\textit{O aluno deixou a sala.})}{dei.xar}{0}
\verb{deixar}{ch}{}{}{}{}{Afastar"-se de pessoa ou coisa para sempre; abandonar, largar. (\textit{Aquele homem deixou a mulher.})}{dei.xar}{0}
\verb{deixar}{ch}{}{}{}{}{Ir para fora; sair.}{dei.xar}{0}
\verb{deixar}{ch}{}{}{}{}{Não proibir que se faça alguma coisa; consentir, permitir.}{dei.xar}{0}
\verb{deixar}{ch}{}{}{}{}{Fazer que pessoa ou coisa fique em determinado estado; tornar.}{dei.xar}{\verboinum{1}}
\verb{dejeção}{}{}{"-ões}{}{s.f.}{Saída das fezes.}{de.je.ção}{0}
\verb{dejejum}{}{}{}{}{}{Var. de \textit{desjejum}.}{de.je.jum}{0}
\verb{dejetar}{}{}{}{}{v.i.}{Fazer a dejeção; defecar.}{de.je.tar}{\verboinum{1}}
\verb{dejeto}{é}{}{}{}{s.m.}{Ato de evacuar excrementos.}{de.je.to}{0}
\verb{dejeto}{é}{}{}{}{}{Conjunto de matérias fecais expelidas por vez; dejeção.}{de.je.to}{0}
\verb{dela}{é}{}{}{}{}{Contração da preposição \textit{de} com o pronome pessoal \textit{ela}.}{de.la}{0}
\verb{delação}{}{}{"-ões}{}{s.f.}{Ato ou efeito de delatar; acusação secreta; denúncia.}{de.la.ção}{0}
\verb{delambido}{}{}{}{}{adj.}{Diz"-se de indivíduo afetado, presunçoso.}{de.lam.bi.do}{0}
\verb{delambido}{}{}{}{}{}{Sem vivacidade ou cor; deslavado, chocho.}{de.lam.bi.do}{0}
\verb{delatar}{}{}{}{}{v.t.}{Denunciar um crime ou um criminoso; acusar, revelar.}{de.la.tar}{0}
\verb{delatar}{}{}{}{}{}{Deixar perceber; evidenciar.}{de.la.tar}{\verboinum{1}}
\verb{delator}{ô}{}{}{}{adj.}{Que delata; acusador, denunciante.}{de.la.tor}{0}
\verb{dele}{ê}{}{}{}{}{Contração da preposição \textit{de} com o pronome pessoal \textit{ele}.}{de.le}{0}
\verb{delegação}{}{}{"-ões}{}{s.f.}{Ato ou efeito de delegar.}{de.le.ga.ção}{0}
\verb{delegação}{}{}{"-ões}{}{}{Comissão dada a uma ou mais pessoas para representar outras.}{de.le.ga.ção}{0}
\verb{delegacia}{}{}{}{}{s.f.}{Cargo de um delegado; delegação.}{de.le.ga.ci.a}{0}
\verb{delegacia}{}{}{}{}{}{Repartição; área de atuação de um delegado, geralmente de repartição pública.}{de.le.ga.ci.a}{0}
\verb{delegacia}{}{}{}{}{}{Repartição de um delegado de polícia.}{de.le.ga.ci.a}{0}
\verb{delegado}{}{}{}{}{s.m.}{Indivíduo que recebeu o direito de representar uma autoridade.}{de.le.ga.do}{0}
\verb{delegar}{}{}{}{}{v.t.}{Transmitir poderes a alguém.}{de.le.gar}{0}
\verb{delegar}{}{}{}{}{}{Encarregar, incumbir.}{de.le.gar}{\verboinum{5}}
\verb{deleitar}{}{}{}{}{v.t.}{Causar grande prazer a alguém; agradar, deliciar.}{de.lei.tar}{\verboinum{1}}
\verb{deleite}{}{}{}{}{s.m.}{Sensação ou sentimento de prazer; satisfação.}{de.lei.te}{0}
\verb{deleitoso}{ô}{}{"-osos ⟨ó⟩}{"-osa ⟨ó⟩}{adj.}{Que causa deleite; delicioso.}{de.lei.to.so}{0}
\verb{deletar}{}{Informát.}{}{}{v.t.}{Apagar algo do computador.}{de.le.tar}{\verboinum{1}}
\verb{deletério}{}{}{}{}{adj.}{Que destrói ou danifica; prejudicial, danoso.}{de.le.té.rio}{0}
\verb{deletério}{}{}{}{}{}{Que corrompe ou desmoraliza.}{de.le.té.rio}{0}
\verb{delével}{}{}{"-eis}{}{adj.2g.}{Que se pode apagar.}{de.lé.vel}{0}
\verb{delfim}{}{Zool.}{"-ins}{}{s.m.}{Golfinho.}{del.fim}{0}
\verb{delfim}{}{}{"-ins}{}{}{Título do antigo herdeiro do trono francês.}{del.fim}{0}
\verb{delfinídeo}{}{Zool.}{}{}{s.m.}{Espécime da família dos delfinídeos, mamíferos marinhos como a orca, o boto e alguns golfinhos, de bico longo, reduzido ou ausente, maxilas com dentes e orifício respiratório único.}{del.fi.ní.deo}{0}
\verb{delfinídeo}{}{Zool.}{}{}{adj.}{Relativo aos delfinídeos.}{del.fi.ní.deo}{0}
\verb{delgado}{}{}{}{}{adj.}{De pouca espessura; fino.}{del.ga.do}{0}
\verb{delgado}{}{}{}{}{}{De corpo magro; esbelto.}{del.ga.do}{0}
\verb{delgado}{}{}{}{}{}{Agudo, fino. (\textit{A moça tem uma voz delgada.})}{del.ga.do}{0}
\verb{delgado}{}{}{}{}{}{Pontudo, afiado. (\textit{Os índios portavam flechas delgadas.})}{del.ga.do}{0}
\verb{delibação}{}{}{"-ões}{}{s.f.}{Ato ou efeito de delibar; prova.}{de.li.ba.ção}{0}
\verb{deliberação}{}{}{"-ões}{}{s.f.}{Ato ou efeito de deliberar; resolução.}{de.li.be.ra.ção}{0}
\verb{deliberar}{}{}{}{}{v.t.}{Decidir após discussão ou exame prolongado.}{de.li.be.rar}{\verboinum{1}}
\verb{deliberativo}{}{}{}{}{adj.}{Que tem poderes para deliberar.}{de.li.be.ra.ti.vo}{0}
\verb{delicadeza}{ê}{}{}{}{s.f.}{Qualidade de ser delicado.}{de.li.ca.de.za}{0}
\verb{delicadeza}{ê}{}{}{}{}{Atitude gentil; fineza, gentileza.}{de.li.ca.de.za}{0}
\verb{delicado}{}{}{}{}{adj.}{Que tem pouca resistência; débil, frágil.}{de.li.ca.do}{0}
\verb{delicado}{}{}{}{}{}{Que pode ofender ou prejudicar; embaraçoso. (\textit{Esse é um assunto delicado.})}{de.li.ca.do}{0}
\verb{delicado}{}{}{}{}{}{Em que há leveza e elegância.}{de.li.ca.do}{0}
\verb{delicado}{}{}{}{}{}{Em que se nota muita sensibilidade e carinho. (\textit{Aquela moça é uma pessoa delicada.})}{de.li.ca.do}{0}
\verb{delícia}{}{}{}{}{s.f.}{Prazer intenso; deleite.}{de.lí.cia}{0}
\verb{delícia}{}{}{}{}{}{Coisa que agrada muito}{de.lí.cia}{0}
\verb{deliciar}{}{}{}{}{v.t.}{Dar um grande prazer a alguém; agradar, deleitar, encantar. (\textit{Eu me delicio com essa torta de morango.})}{de.li.ci.ar}{\verboinum{6}}
\verb{delicioso}{ô}{}{"-osos ⟨ó⟩}{"-osa ⟨ó⟩}{adj.}{Que causa delícia; extremamente agradável.}{de.li.ci.o.so}{0}
\verb{delimitar}{}{}{}{}{v.t.}{Marcar os limites de alguma coisa; demarcar, limitar.}{de.li.mi.tar}{\verboinum{1}}
\verb{delineador}{ô}{}{}{}{adj.}{Que delineia; que esboça ou planeja.}{de.li.ne.a.dor}{0}
\verb{delineador}{ô}{}{}{}{}{Diz"-se de líquido espesso e colorido usado em maquiagem.}{de.li.ne.a.dor}{0}
\verb{delinear}{}{}{}{}{v.t.}{Desenhar os contornos, os traços gerais de algo; esboçar, traçar.}{de.li.ne.ar}{0}
\verb{delinear}{}{Fig.}{}{}{}{Planejar, tramar.}{de.li.ne.ar}{\verboinum{4}}
\verb{delinquência}{}{}{}{}{s.f.}{Desobediência a leis, regulamentos ou padrões morais; infração.}{de.lin.quên.cia}{0}
\verb{delinquente}{}{}{}{}{adj.2g.}{Que fez uma ação contra a lei.}{de.lin.quen.te}{0}
\verb{delinquir}{}{}{}{}{v.i.}{Agir contra a lei.}{de.lin.quir}{\verboinum{34}\verboirregular{\emph{def.} delinquimos, delinquis}}
\verb{delíquio}{}{}{}{}{s.m.}{Perda dos sentidos; desmaio, síncope.}{de.lí.quio}{0}
\verb{delir}{}{}{}{}{v.t.}{Dissolver, desfazer alguma coisa em líquido.}{de.lir}{0}
\verb{delir}{}{Fig.}{}{}{}{Fazer desaparecer, apagar, extinguir.}{de.lir}{\verboinum{34}\verboirregular{\emph{def.} delimos, delis}}
\verb{delirante}{}{}{}{}{adj.2g.}{Próprio de quem delira; extravagante, insensato; caótico.}{de.li.ran.te}{0}
\verb{delirante}{}{Pop.}{}{}{}{Que é surpreendente, extraordinário.}{de.li.ran.te}{0}
\verb{delirar}{}{}{}{}{v.i.}{Estar em estado de alucinação ou inquietação mental.}{de.li.rar}{0}
\verb{delirar}{}{}{}{}{}{Mostrar a emoção com muita intensidade; entusiasmar"-se, exaltar"-se.}{de.li.rar}{\verboinum{1}}
\verb{delírio}{}{}{}{}{s.m.}{Perturbação mental que cria visões e perda do contato com a realidade.}{de.lí.rio}{0}
\verb{delírio}{}{}{}{}{}{Sentimento de profundo entusiasmo; exaltação.}{de.lí.rio}{0}
\verb{delito}{}{}{}{}{s.m.}{Qualquer ato que constitua uma infração às leis estabelecidas; crime, infração.}{de.li.to}{0}
\verb{delituoso}{ô}{}{"-osos ⟨ó⟩}{"-osa ⟨ó⟩}{adj.}{Em que há, ou que constitui delito.}{de.li.tu.o.so}{0}
\verb{delivery}{}{}{}{}{s.m.}{Entrega de mercadorias.}{\textit{delivery}}{0}
\verb{delonga}{}{}{}{}{s.f.}{Ato ou efeito de delongar; adiamento, atraso, demora.}{de.lon.ga}{0}
\verb{delongar}{}{}{}{}{v.t.}{Tornar longo, demorado; demorar, adiar, retardar.}{de.lon.gar}{\verboinum{5}}
\verb{delta}{é}{}{}{}{s.m.}{Quarta letra do alfabeto grego, cuja maiúscula tem a forma de triângulo.}{del.ta}{0}
\verb{delta}{é}{}{}{}{}{Terreno triangular entre dois braços de um rio, junto à sua foz.}{del.ta}{0}
\verb{deltoide}{}{}{}{}{adj.}{Que tem a forma de um delta.}{del.toi.de}{0}
\verb{deltoide}{}{Anat.}{}{}{s.m.}{Um dos músculos do ombro, de forma triangular.}{del.toi.de}{0}
\verb{demagogia}{}{}{}{}{s.f.}{Atividade de demagogo.}{de.ma.go.gi.a}{0}
\verb{demagógico}{}{}{}{}{adj.}{Que manifesta demagogia.}{de.ma.gó.gi.co}{0}
\verb{demagogo}{ô}{}{}{}{s.m.}{Político que engana o povo com promessas e mentiras.}{de.ma.go.go}{0}
\verb{demais}{}{}{}{}{adv.}{Em excesso.}{de.mais}{0}
\verb{demais}{}{}{}{}{pron.}{Os outros; os restantes.}{de.mais}{0}
\verb{demanda}{}{}{}{}{s.f.}{Ato de demandar.}{de.man.da}{0}
\verb{demanda}{}{Jur.}{}{}{}{Processo judicial.}{de.man.da}{0}
\verb{demanda}{}{}{}{}{}{Facilidade de venda; procura, saída.}{de.man.da}{0}
\verb{demandar}{}{}{}{}{v.t.}{Dirigir"-se para algum lugar; deslocar"-se, procurar.}{de.man.dar}{0}
\verb{demandar}{}{}{}{}{}{Instaurar processo judicial contra pessoa ou coisa.}{de.man.dar}{0}
\verb{demandar}{}{}{}{}{}{Necessitar, precisar, requerer. }{de.man.dar}{\verboinum{1}}
\verb{demão}{}{}{"-ãos}{}{s.f.}{Cada uma das vezes seguidas com que se pinta uma parede.}{de.mão}{0}
\verb{demarcação}{}{}{"-ões}{}{s.f.}{Determinação de limites por meio de marcos ou balizas; delimitação.}{de.mar.ca.ção}{0}
\verb{demarcação}{}{}{"-ões}{}{}{Separação, distinção.}{de.mar.ca.ção}{0}
\verb{demarcar}{}{}{}{}{v.t.}{Marcar os limites de algo; estremar, delimitar.}{de.mar.car}{0}
\verb{demarcar}{}{}{}{}{}{Separar, distinguir.}{de.mar.car}{\verboinum{2}}
\verb{demasia}{}{}{}{}{s.f.}{Aquilo que é demais; excesso, sobra, resto.}{de.ma.si.a}{0}
\verb{demasiado}{}{}{}{}{adj.}{Em quantidade maior que a necessária; excessivo.}{de.ma.si.a.do}{0}
\verb{demasiar"-se}{}{}{}{}{v.pron.}{Ir além dos limites razoáveis; exceder"-se.}{de.ma.si.ar"-se}{\verboinum{6}}
\verb{demência}{}{}{}{}{s.f.}{Qualquer deterioração mental.}{de.mên.cia}{0}
\verb{demência}{}{Pop.}{}{}{}{Loucura, doidice.}{de.mên.cia}{0}
\verb{demente}{}{}{}{}{adj.2g.}{Que sofre de demência; doido, insano, louco.}{de.men.te}{0}
\verb{demérito}{}{}{}{}{s.m.}{Falta ou perda de mérito ou merecimento.}{de.mé.ri.to}{0}
\verb{demissão}{}{}{"-ões}{}{s.f.}{Ato ou efeito de demitir; destituição.}{de.mis.são}{0}
\verb{demissionário}{}{}{}{}{adj.}{Que se demitiu; que se afastou de cargo ou função.}{de.mis.si.o.ná.rio}{0}
\verb{demitir}{}{}{}{}{v.t.}{Mandar alguém embora de um emprego; despedir, exonerar.}{de.mi.tir}{\verboinum{18}}
\verb{demo}{}{}{}{}{s.m.}{Demônio.}{de.mo}{0}
\verb{demo}{}{Pop.}{}{}{}{Diz"-se de fita, disquete ou \textsc{cd} que serve de demonstração.}{de.mo}{0}
\verb{democracia}{}{}{}{}{s.f.}{Governo do povo; governo em que o povo exerce a soberania.}{de.mo.cra.ci.a}{0}
\verb{democracia}{}{}{}{}{}{Doutrina ou regime político que se caracteriza pela liberdade do ato eleitoral, pela divisão dos poderes e pelo controle da autoridade, dos poderes de decisão e de execução.}{de.mo.cra.ci.a}{0}
\verb{democrata}{}{}{}{}{adj.}{Que pratica a democracia.}{de.mo.cra.ta}{0}
\verb{democrático}{}{}{}{}{adj.}{Relativo a democracia.}{de.mo.crá.ti.co}{0}
\verb{democrático}{}{}{}{}{}{Que se adapta aos interesses do povo.}{de.mo.crá.ti.co}{0}
\verb{democratização}{}{}{"-ões}{}{s.f.}{Ato ou efeito de democratizar.}{de.mo.cra.ti.za.ção}{0}
\verb{democratizar}{}{}{}{}{v.t.}{Conduzir à democracia; tornar democrata.}{de.mo.cra.ti.zar}{0}
\verb{democratizar}{}{}{}{}{}{Tornar popular, colocar ao alcance do povo, da maioria da população.}{de.mo.cra.ti.zar}{\verboinum{1}}
\verb{demografia}{}{}{}{}{s.f.}{Estudo das características de uma população.}{de.mo.gra.fi.a}{0}
\verb{demolição}{}{}{"-ões}{}{s.f.}{Ato ou efeito de demolir; destruição.}{de.mo.li.ção}{0}
\verb{demolir}{}{}{}{}{v.t.}{Desfazer uma construção; derrubar, destruir.}{de.mo.lir}{0}
\verb{demolir}{}{}{}{}{}{Reduzir a nada; arruinar, aniquilar.}{de.mo.lir}{\verboinum{34}\verboirregular{\emph{def.} demolimos, demolis}}
\verb{demoníaco}{}{}{}{}{adj.}{Relativo a demônio; diabólico.}{de.mo.ní.a.co}{0}
\verb{demônio}{}{}{}{}{s.m.}{Nas religiões judaica e cristã, o anjo mau que se revoltou contra Deus e procura levar o homem ao pecado; espírito do mal; Satanás, Diabo.}{de.mô.nio}{0}
\verb{demonismo}{}{}{}{}{s.m.}{Crença em demônios.}{de.mo.nis.mo}{0}
\verb{demonstração}{}{}{"-ões}{}{s.f.}{Ato ou efeito de demonstrar; comprovação, manifestação, apresentação.}{de.mons.tra.ção}{0}
\verb{demonstrar}{}{}{}{}{v.t.}{Fazer alguém saber de algum sentimento ou qualidade que se tem; manifestar, mostrar.}{de.mons.trar}{0}
\verb{demonstrar}{}{}{}{}{}{Tornar evidente através de provas; comprovar.}{de.mons.trar}{0}
\verb{demonstrar}{}{}{}{}{}{Dar uma prova prática de alguma coisa.}{de.mons.trar}{\verboinum{1}}
\verb{demonstrativo}{}{}{}{}{adj.}{Que serve para demonstrar; que prova.}{de.mons.tra.ti.vo}{0}
\verb{demonstrativo}{}{Gram.}{}{}{}{Diz"-se do pronome que serve para situar pessoa ou coisa em relação à pessoa que fala.}{de.mons.tra.ti.vo}{0}
\verb{demora}{ó}{}{}{}{s.f.}{Ato de demorar; atraso, delonga.}{de.mo.ra}{0}
\verb{demorado}{}{}{}{}{adj.}{Que demora; que permanece por um longo tempo.}{de.mo.ra.do}{0}
\verb{demorado}{}{}{}{}{}{Que chega com atraso; tardio.}{de.mo.ra.do}{0}
\verb{demorar}{}{}{}{}{v.i.}{Gastar tempo.}{de.mo.rar}{0}
\verb{demorar}{}{}{}{}{}{Exigir algum tempo para se fazer.}{de.mo.rar}{0}
\verb{demorar}{}{}{}{}{v.pron.}{Ficar algum tempo em algum lugar.}{de.mo.rar}{\verboinum{1}}
\verb{demover}{ê}{}{}{}{v.t.}{Provocar o deslocamento de algo ou de si mesmo.}{de.mo.ver}{0}
\verb{demover}{ê}{}{}{}{}{Fazer renunciar a uma pretensão; dissuadir.}{de.mo.ver}{\verboinum{12}}
\verb{demudado}{}{}{}{}{adj.}{Que se demudou; alterado, transformado.}{de.mu.da.do}{0}
\verb{dendê}{}{}{}{}{s.m.}{Dendezeiro.}{den.dê}{0}
\verb{dendê}{}{}{}{}{}{Fruto do dendezeiro.}{den.dê}{0}
\verb{dendê}{}{}{}{}{}{Óleo extraído da polpa desse fruto.}{den.dê}{0}
\verb{dendezeiro}{ê}{Bot.}{}{}{s.m.}{Palmeira originária da África, de folhas compridas e frutos amarelos, que fornecem um óleo de sabor adocicado e cheiro típico, usado na alimentação.}{den.de.zei.ro}{0}
\verb{denegação}{}{}{"-ões}{}{s.f.}{Ato de denegar; recusa, negação, indeferimento.}{de.ne.ga.ção}{0}
\verb{denegar}{}{}{}{}{v.t.}{Dizer que não é verdade; negar.}{de.ne.gar}{0}
\verb{denegar}{}{}{}{}{}{Não dar; recusar, negar.}{de.ne.gar}{0}
\verb{denegar}{}{}{}{}{}{Desatender, indeferir.}{de.ne.gar}{\verboinum{5}}
\verb{denegrir}{}{}{}{}{v.t.}{Tornar negro; escurecer.}{de.ne.grir}{0}
\verb{denegrir}{}{Fig.}{}{}{}{Manchar a reputação; infamar, difamar.}{de.ne.grir}{\verboinum{30}}
\verb{dengo}{}{Bras.}{}{}{s.m.}{Comportamento insinuante; melindre, denguice.}{den.go}{0}
\verb{dengo}{}{}{}{}{}{Comportamento astucioso; manha.}{den.go}{0}
\verb{dengo}{}{}{}{}{}{Birra ou lamentação de criança.}{den.go}{0}
\verb{dengoso}{ô}{}{"-osos ⟨ó⟩}{"-osa ⟨ó⟩}{adj.}{Cheio de dengo.}{den.go.so}{0}
\verb{dengue}{}{Bras.}{}{}{s.m.}{Denguice, dengo.}{den.gue}{0}
\verb{dengue}{}{Med.}{}{}{s.f.}{Doença infecciosa causada por vírus e transmitida pela picada de certo mosquito, caracterizada por febre alta, fadiga e dor no corpo e na cabeça.}{den.gue}{0}
\verb{denguice}{}{}{}{}{s.f.}{Dengo.}{den.gui.ce}{0}
\verb{denodado}{}{}{}{}{adj.}{Cheio de denodo; ousado, corajoso.}{de.no.da.do}{0}
\verb{denodo}{ô}{}{}{}{s.m.}{Ousadia, coragem.}{de.no.do}{0}
\verb{denominação}{}{}{"-ões}{}{s.f.}{Ato ou efeito de denominar.}{de.no.mi.na.ção}{0}
\verb{denominação}{}{}{"-ões}{}{}{Nome, designação.}{de.no.mi.na.ção}{0}
\verb{denominador}{ô}{}{}{}{adj.}{Que denomina; nomeador.}{de.no.mi.na.dor}{0}
\verb{denominador}{ô}{Mat.}{}{}{}{Diz"-se do número que, em uma fração, está situado abaixo do traço.}{de.no.mi.na.dor}{0}
\verb{denominar}{}{}{}{}{v.t.}{Nomear, designar.}{de.no.mi.nar}{\verboinum{1}}
\verb{denominativo}{}{}{}{}{adj.}{Que denomina, nomeia.}{de.no.mi.na.ti.vo}{0}
\verb{denotação}{}{}{"-ões}{}{s.f.}{Ato ou efeito de denotar; designação.}{de.no.ta.ção}{0}
\verb{denotação}{}{Gram.}{"-ões}{}{}{Significado objetivo de uma palavra, sem os traços semânticos secundários e subjetivos.}{de.no.ta.ção}{0}
\verb{denotar}{}{}{}{}{v.t.}{Mostrar, indicar, demonstrar.}{de.no.tar}{\verboinum{1}}
\verb{densidade}{}{}{}{}{s.f.}{Qualidade de denso; espessura, viscosidade.}{den.si.da.de}{0}
\verb{densidade}{}{Fís.}{}{}{}{Relação entre massa e volume de uma substância.}{den.si.da.de}{0}
\verb{denso}{}{}{}{}{adj.}{Cerrado, carregado, espesso, viscoso.}{den.so}{0}
\verb{denso}{}{}{}{}{}{Que tem muita massa em relação ao volume.}{den.so}{0}
\verb{dentada}{}{}{}{}{s.f.}{Corte, ferida ou marca feita com os dentes.}{den.ta.da}{0}
\verb{dentado}{}{}{}{}{adj.}{Provido de dentes.}{den.ta.do}{0}
\verb{dentado}{}{}{}{}{}{Diz"-se de superfície que apresenta extremidade ou margem recortada em forma de dentes.}{den.ta.do}{0}
\verb{dentadura}{}{}{}{}{s.f.}{O conjunto dos dentes de um indivíduo.}{den.ta.du.ra}{0}
\verb{dentadura}{}{}{}{}{}{Peça montada com dentes artificiais para ser usada por pessoas que perderam os dentes.}{den.ta.du.ra}{0}
\verb{dental}{}{}{"-ais}{}{adj.2g.}{Relativo a dente ou a conjunto de dentes.}{den.tal}{0}
\verb{dentar}{}{}{}{}{v.t.}{Morder.}{den.tar}{0}
\verb{dentar}{}{}{}{}{v.i.}{Começar a desenvolver os dentes.}{den.tar}{\verboinum{1}}
\verb{dentário}{}{}{}{}{adj.}{Relativo a dente. (\textit{No semestre passado, submeti"-me a um tratamento dentário.})}{den.tá.rio}{0}
\verb{dente}{}{}{}{}{s.m.}{Cada uma das estruturas duras presentes nos maxilares que serve para morder e triturar os alimentos.}{den.te}{0}
\verb{dente}{}{}{}{}{}{Saliências ou pontas presentes nas engrenagens e em certos objetos, as quais apresentam a forma de dentes.}{den.te}{0}
\verb{denteado}{}{}{}{}{adj.}{Provido de dentes; dentado.}{den.te.a.do}{0}
\verb{dentear}{}{}{}{}{v.t.}{Moldar ou recortar de modo que fique com saliências semelhantes a dentes.}{den.te.ar}{\verboinum{4}}
\verb{dentição}{}{}{"-ões}{}{s.f.}{Conjunto dos dentes de um indivíduo.}{den.ti.ção}{0}
\verb{dentição}{}{}{"-ões}{}{}{Processo de formação e crescimento dos dentes.}{den.ti.ção}{0}
\verb{denticulado}{}{}{}{}{adj.}{Que apresenta dentículos.}{den.ti.cu.la.do}{0}
\verb{dentículo}{}{}{}{}{s.m.}{Dente pequeno.}{den.tí.cu.lo}{0}
\verb{dentículo}{}{Art.}{}{}{}{Pequeno ornato em forma de dente.}{den.tí.cu.lo}{0}
\verb{dentículo}{}{Bot.}{}{}{}{Recorte em forma de dente presente na margem das folhas de alguns vegetais.}{den.tí.cu.lo}{0}
\verb{dentifrício}{}{}{}{}{s.m.}{Produto, geralmente um creme, usado para fazer a higiene dos dentes.}{den.ti.frí.cio}{0}
\verb{dentina}{}{Anat.}{}{}{s.f.}{Tecido rico em cálcio e de cor branca que recobre a polpa dos dentes.}{den.ti.na}{0}
\verb{dentista}{}{}{}{}{s.2g.}{Indivíduo que se dedica ao tratamento e à conservação dos dentes.}{den.tis.ta}{0}
\verb{dentre}{}{}{}{}{prep.}{Contração das preposições \textit{de} e \textit{entre}; do meio de.}{den.tre}{0}
\verb{dentro}{}{}{}{}{adv.}{Na parte interior.}{den.tro}{0}
\verb{dentuça}{}{}{}{}{s.f.}{O conjunto dos dentes da frente de um indivíduo, quando salientes.}{den.tu.ça}{0}
\verb{dentuço}{}{Bras.}{}{}{adj.}{Que possui os dentes da frente salientes ou destacados.}{den.tu.ço}{0}
\verb{denudar}{}{}{}{}{v.t.}{Ficar nu; desnudar.}{de.nu.dar}{\verboinum{1}}
\verb{denudar}{}{Fig.}{}{}{}{Tornar aparente; revelar.}{de.nu.dar}{0}
\verb{denúncia}{}{}{}{}{s.f.}{Ato ou efeito de denunciar; delação.}{de.nún.cia}{0}
\verb{denunciador}{ô}{}{}{}{adj.}{Que apresenta denúncias; denunciante.}{de.nun.ci.a.dor}{0}
\verb{denunciante}{}{}{}{}{adj.2g.}{Que apresenta denúncias; denunciador.}{de.nun.ci.an.te}{0}
\verb{denunciar}{}{}{}{}{v.t.}{Tornar conhecido; anunciar, difundir.}{de.nun.ci.ar}{0}
\verb{denunciar}{}{}{}{}{}{Atribuir responsabilidade por fato ilícito ou comprometedor.}{de.nun.ci.ar}{0}
\verb{denunciar}{}{}{}{}{}{Evidenciar, mostrar.}{de.nun.ci.ar}{\verboinum{1}}
\verb{denunciativo}{}{}{}{}{adj.}{Que denuncia.}{de.nun.ci.a.ti.vo}{0}
%\verb{}{}{}{}{}{}{}{}{0}
\verb{deparar}{}{}{}{}{v.t.}{Encontrar de modo inesperado; topar, defrontar.}{de.pa.rar}{0}
\verb{deparar}{}{}{}{}{}{Fazer aparecer repentinamente.}{de.pa.rar}{\verboinum{1}}
\verb{departamento}{}{}{}{}{s.m.}{Cada uma das subdivisões de uma instituição pública ou privada.}{de.par.ta.men.to}{0}
\verb{departamento}{}{}{}{}{}{Cada uma das subdivisões administrativas de alguns países.}{de.par.ta.men.to}{0}
\verb{depauperar}{}{}{}{}{v.t.}{Causar a diminuição dos recursos econômicos; empobrecer.}{de.pau.pe.rar}{0}
\verb{depauperar}{}{}{}{}{}{Causar debilidade ou esgotamento físico.}{de.pau.pe.rar}{\verboinum{1}}
\verb{depenar}{}{}{}{}{v.t.}{Arrancar as penas.}{de.pe.nar}{0}
\verb{depenar}{}{Fig.}{}{}{}{Arrancar pelos ou cabelos.}{de.pe.nar}{0}
\verb{depenar}{}{}{}{}{}{Tirar grande parte ou a totalidade dos haveres de alguém ou das peças componentes de alguma máquina, especialmente carros.}{de.pe.nar}{\verboinum{1}}
\verb{dependência}{}{}{}{}{s.f.}{Qualidade de dependente.}{de.pen.dên.cia}{0}
\verb{dependência}{}{}{}{}{}{Cada aposento de uma casa.}{de.pen.dên.cia}{0}
\verb{dependência}{}{}{}{}{}{Cada uma das unidades de um conjunto composto de várias edificações, como empresas, clubes, hotéis.}{de.pen.dên.cia}{0}
\verb{dependente}{}{}{}{}{adj.2g.}{Que depende.}{de.pen.den.te}{0}
\verb{dependente}{}{Jur.}{}{}{s.2g.}{Pessoa sem condições financeiras de subsistência dependendo de outra para efeitos legais.}{de.pen.den.te}{0}
\verb{depender}{ê}{}{}{}{v.t.}{Estar subordinado a.}{de.pen.der}{0}
\verb{depender}{ê}{}{}{}{}{Necessitar do auxílio ou proteção de.}{de.pen.der}{\verboinum{12}}
\verb{dependura}{}{}{}{}{s.f.}{Ato ou efeito de dependurar; pendura.}{de.pen.du.ra}{0}
\verb{dependurar}{}{}{}{}{v.t.}{Suspender e prender a certa altura do chão; pendurar.}{de.pen.du.rar}{\verboinum{1}}
\verb{depenicar}{}{}{}{}{v.t.}{Tirar as penas ou os pelos; depenar.}{de.pe.ni.car}{0}
\verb{depenicar}{}{}{}{}{}{Tirar pequenas porções para comer; beliscar.}{de.pe.ni.car}{\verboinum{2}}
\verb{deperecer}{ê}{}{}{}{v.i.}{Finar"-se pouco a pouco; definhar.}{de.pe.re.cer}{\verboinum{15}}
\verb{depilação}{}{}{"-ões}{}{s.f.}{Ato ou efeito de depilar; retirada dos pelos.}{de.pi.la.ção}{0}
\verb{depilador}{ô}{}{}{}{s.m.}{Profissional que trabalha com depilação.}{de.pi.la.dor}{0}
\verb{depilador}{ô}{}{}{}{}{Aparelho usado para fazer depilação.}{de.pi.la.dor}{0}
\verb{depilar}{}{}{}{}{v.t.}{Arrancar, remover ou cortar os pelos do corpo.}{de.pi.lar}{\verboinum{1}}
\verb{depilatório}{}{}{}{}{adj.}{Que depila.}{de.pi.la.tó.rio}{0}
\verb{depilatório}{}{}{}{}{s.m.}{Cosmético usado para remover os pelos.}{de.pi.la.tó.rio}{0}
\verb{deplorar}{}{}{}{}{v.t.}{Manifestar sofrimento ou incômodo; lastimar.}{de.plo.rar}{\verboinum{1}}
\verb{deplorável}{}{}{"-eis}{}{adj.2g.}{Lamentável, lastimável, abominável.}{de.plo.rá.vel}{0}
\verb{depoente}{}{Jur.}{}{}{s.2g.}{Indivíduo que depõe em juízo.}{de.po.en.te}{0}
\verb{depoimento}{}{}{}{}{s.m.}{Ato de depor.}{de.po.i.men.to}{0}
\verb{depoimento}{}{}{}{}{}{A informação prestada por aquele que depõe.}{de.po.i.men.to}{0}
\verb{depois}{}{}{}{}{adv.}{Posteriormente no tempo ou no espaço.}{de.pois}{0}
\verb{depois}{}{}{}{}{}{Além disso; ademais.}{de.pois}{0}
\verb{depor}{ô}{}{}{}{v.t.}{Pôr de lado; deixar, largar.}{de.por}{0}
\verb{depor}{ô}{}{}{}{}{Despojar de cargo; exonerar. (\textit{O general depôs o presidente da república.})}{de.por}{0}
\verb{depor}{ô}{}{}{}{}{Declarar em juízo.}{de.por}{0}
\verb{depor}{ô}{}{}{}{v.i.}{Prestar declarações em juízo.}{de.por}{\verboinum{60}}
\verb{deportação}{}{}{"-ões}{}{s.f.}{Ato ou efeito de deportar; exílio, banimento, desterro.}{de.por.ta.ção}{0}
\verb{deportar}{}{}{}{}{v.t.}{Levar ou mandar alguém para fora de uma sociedade; expatriar, banir, exilar.}{de.por.tar}{\verboinum{1}}
\verb{deposição}{}{}{"-ões}{}{s.f.}{Ato ou efeito de depor; tirar do poder.}{de.po.si.ção}{0}
\verb{depositante}{}{}{}{}{s.2g.}{Indivíduo que realiza um depósito.}{de.po.si.tan.te}{0}
\verb{depositar}{}{}{}{}{v.t.}{Pôr em depósito; guardar.}{de.po.si.tar}{0}
\verb{depositar}{}{}{}{}{}{Pôr em segurança; confiar.}{de.po.si.tar}{0}
\verb{depositar}{}{}{}{}{}{Entregar ou colocar de maneira solene.}{de.po.si.tar}{\verboinum{1}}
\verb{depositário}{}{}{}{}{s.m.}{Indivíduo que recebe em depósito.}{de.po.si.tá.rio}{0}
\verb{depósito}{}{}{}{}{s.m.}{Ato ou efeito de depositar.}{de.pó.si.to}{0}
\verb{depósito}{}{}{}{}{}{Lugar onde se deposita algo.}{de.pó.si.to}{0}
\verb{depósito}{}{}{}{}{}{Armazém ou reservatório.}{de.pó.si.to}{0}
\verb{depósito}{}{}{}{}{}{Sedimento, borra.}{de.pó.si.to}{0}
\verb{depravação}{}{}{"-ões}{}{s.f.}{Ato ou efeito de depravar; perversão.}{de.pra.va.ção}{0}
\verb{depravação}{}{Fig.}{"-ões}{}{}{Decadência, degeneração, declínio.}{de.pra.va.ção}{0}
\verb{depravado}{}{}{}{}{adj.}{Pervertido, malvado, perverso, corrompido.}{de.pra.va.do}{0}
\verb{depravar}{}{}{}{}{v.t.}{Estragar, danificar.}{de.pra.var}{0}
\verb{depravar}{}{Fig.}{}{}{}{Perverter, corromper.}{de.pra.var}{0}
\verb{depravar}{}{}{}{}{}{Falsificar.}{de.pra.var}{\verboinum{1}}
\verb{deprecar}{}{}{}{}{v.t.}{Pedir com submissão; implorar, suplicar.}{de.pre.car}{\verboinum{2}}
\verb{depreciação}{}{Fig.}{"-ões}{}{}{Menosprezo, desprezo, desdém.}{de.pre.ci.a.ção}{0}
\verb{depreciação}{}{}{"-ões}{}{s.f.}{Ato ou efeito de depreciar.}{de.pre.ci.a.ção}{0}
\verb{depreciar}{}{}{}{}{v.t.}{Diminuir o valor ou o preço; desvalorizar.}{de.pre.ci.ar}{0}
\verb{depreciar}{}{Fig.}{}{}{}{Rebaixar, desprezar, desestimar.}{de.pre.ci.ar}{\verboinum{1}}
\verb{depreciativo}{}{}{}{}{adj.}{Que envolve depreciação.}{de.pre.ci.a.ti.vo}{0}
\verb{depredação}{}{}{"-ões}{}{s.f.}{Ato ou efeito de depredar; saque, pilhagem.}{de.pre.da.ção}{0}
\verb{depredar}{}{}{}{}{v.t.}{Saquear, pilhar, estragar.}{de.pre.dar}{\verboinum{1}}
\verb{depreender}{ê}{}{}{}{v.t.}{Compreender, perceber, entender.}{de.pre.en.der}{0}
\verb{depreender}{ê}{}{}{}{}{Deduzir, inferir, concluir.}{de.pre.en.der}{\verboinum{12}}
\verb{depressa}{é}{}{}{}{adv.}{Com rapidez; sem demora.}{de.pres.sa}{0}
\verb{depressão}{}{}{"-ões}{}{s.f.}{Ato ou efeito de deprimir.}{de.pres.são}{0}
\verb{depressão}{}{}{"-ões}{}{}{Tristeza profunda; abatimento.}{de.pres.são}{0}
\verb{depressão}{}{Geogr.}{"-ões}{}{}{Área de altitude menor em relação às áreas vizinhas.}{de.pres.são}{0}
\verb{depressivo}{}{}{}{}{adj.}{Que se encontra em estado de depressão.}{de.pres.si.vo}{0}
\verb{depressivo}{}{}{}{}{}{Que provoca depressão.}{de.pres.si.vo}{0}
\verb{depressivo}{}{}{}{}{}{Diz"-se de pessoa que tem tendência a entrar em estado de depressão.}{de.pres.si.vo}{0}
\verb{deprimente}{}{}{}{}{adj.2g.}{Que provoca depressão.}{de.pri.men.te}{0}
\verb{deprimir}{}{}{}{}{v.t.}{Causar depressão ou angústia; abater, angustiar.}{de.pri.mir}{0}
\verb{deprimir}{}{}{}{}{}{Enfraquecer, debilitar.}{de.pri.mir}{0}
\verb{deprimir}{}{}{}{}{}{Rebaixar, humilhar.}{de.pri.mir}{\verboinum{18}}
\verb{depuração}{}{}{"-ões}{}{s.f.}{Ato ou efeito de depurar; purificação.}{de.pu.ra.ção}{0}
\verb{depurar}{}{}{}{}{v.t.}{Tornar ou ficar mais puro; purificar.}{de.pu.rar}{\verboinum{1}}
\verb{depurativo}{}{}{}{}{s.m.}{Medicamento que provoca a eliminação de resíduos e toxinas.}{de.pu.ra.ti.vo}{0}
\verb{depurativo}{}{}{}{}{adj.}{Que depura.}{de.pu.ra.ti.vo}{0}
\verb{deputação}{}{}{"-ões}{}{s.f.}{Ato ou efeito de deputar.}{de.pu.ta.ção}{0}
\verb{deputação}{}{}{"-ões}{}{}{Reunião das pessoas encarregadas de determinada missão.}{de.pu.ta.ção}{0}
\verb{deputado}{}{}{}{}{s.m.}{Membro eleito do poder legislativo.}{de.pu.ta.do}{0}
\verb{deputado}{}{}{}{}{}{Indivíduo que age comissionado para tratar dos negócios de outrem; delegado.}{de.pu.ta.do}{0}
\verb{deputar}{}{}{}{}{v.t.}{Transmitir poderes a alguém; delegar.}{de.pu.tar}{\verboinum{1}}
\verb{deque}{é}{}{}{}{s.m.}{Piso de pavimento descoberto, geralmente feito de tábuas e em áreas próximas à água.}{de.que}{0}
\verb{deriva}{}{}{}{}{s.f.}{Usada na expressão \textit{à deriva}: ao sabor da corrente; sem rumo; desgovernado.}{de.ri.va}{0}
\verb{derivação}{}{}{"-ões}{}{s.f.}{Ato ou efeito de derivar.}{de.ri.va.ção}{0}
\verb{derivação}{}{Gram.}{"-ões}{}{}{Processo pelo qual se formam palavras a partir de outras através da adição de um sufixo ou um prefixo.}{de.ri.va.ção}{0}
\verb{derivada}{}{Mat.}{}{}{s.f.}{Limite da razão incremental quando o acréscimo da variável independente tende a zero.}{de.ri.va.da}{0}
\verb{derivado}{}{}{}{}{adj.}{Oriundo de uma transformação.}{de.ri.va.do}{0}
\verb{derivado}{}{Gram.}{}{}{}{Diz"-se de vocábulo formado a partir de outro por processo de derivação.}{de.ri.va.do}{0}
\verb{derivar}{}{}{}{}{v.t.}{Alterar o rumo; desviar.}{de.ri.var}{0}
\verb{derivar}{}{}{}{}{}{Ser proveniente de; originar"-se, proceder, descender.}{de.ri.var}{\verboinum{1}}
\verb{derivativo}{}{}{}{}{adj.}{Relativo a derivação.}{de.ri.va.ti.vo}{0}
\verb{derivativo}{}{}{}{}{s.m.}{Ocupação para distrair ou divertir, geralmente em uma situação de tristeza ou monotonia.}{de.ri.va.ti.vo}{0}
\verb{derma}{é}{}{}{}{s.m.}{Pele, derme.}{der.ma}{0}
\verb{dermatite}{}{Med.}{}{}{s.f.}{Inflamação da pele.}{der.ma.ti.te}{0}
\verb{dermatologia}{}{Med.}{}{}{s.f.}{Ramo da medicina que estuda as lesões e as doenças da pele.}{der.ma.to.lo.gi.a}{0}
\verb{dermatologista}{}{}{}{}{s.2g.}{Especialista em dermatologia.}{der.ma.to.lo.gis.ta}{0}
\verb{dermatose}{ó}{Med.}{}{}{s.f.}{Denominação genérica das doenças de pele, especialmente as sem inflamação.}{der.ma.to.se}{0}
\verb{derme}{é}{}{}{}{s.f.}{Pele.}{der.me}{0}
\verb{derradeiro}{ê}{}{}{}{adj.}{Que vem atrás; último.}{der.ra.dei.ro}{0}
\verb{derradeiro}{ê}{}{}{}{}{Final, extremo.}{der.ra.dei.ro}{0}
\verb{derrama}{}{}{}{}{s.f.}{Tributo local proporcional aos rendimentos do contribuinte.}{der.ra.ma}{0}
\verb{derrama}{}{Hist.}{}{}{}{No Brasil colonial, cobrança dos impostos atrasados nas regiões de mineração.}{der.ra.ma}{0}
\verb{derramamento}{}{}{}{}{s.m.}{Ato ou efeito de derramar; derrame.}{der.ra.ma.men.to}{0}
\verb{derramar}{}{}{}{}{v.t.}{Fazer cair ou deixar cair.}{der.ra.mar}{0}
\verb{derramar}{}{}{}{}{}{Espalhar, espargir.}{der.ra.mar}{\verboinum{1}}
\verb{derrame}{}{}{}{}{s.m.}{Ato ou efeito de derramar; derramamento.}{der.ra.me}{0}
\verb{derrame}{}{Med.}{}{}{}{Hemorragia interna, especialmente cerebral.}{der.ra.me}{0}
\verb{derrapada}{}{}{}{}{s.f.}{Ato ou efeito de derrapar; derrapagem, deslizamento.}{der.ra.pa.da}{0}
\verb{derrapagem}{}{}{}{}{s.f.}{Ato ou efeito de derrapar; derrapada, deslizamento.}{der.ra.pa.gem}{0}
\verb{derrapar}{}{}{}{}{v.i.}{Escorregar para o lado; deslizar, desgovernar"-se.}{der.ra.par}{\verboinum{1}}
\verb{derreado}{}{}{}{}{adj.}{Arqueado, curvado, dobrado.}{der.re.a.do}{0}
\verb{derreado}{}{}{}{}{}{Muito cansado; extenuado.}{der.re.a.do}{0}
\verb{derrear}{}{}{}{}{v.t.}{Fazer vergar; curvar.}{der.re.ar}{0}
\verb{derrear}{}{}{}{}{}{Machucar, maltratar.}{der.re.ar}{0}
\verb{derrear}{}{}{}{}{}{Esgotar, prostrar, extenuar.}{der.re.ar}{\verboinum{4}}
\verb{derredor}{ó}{}{}{}{adv.}{Em redor, em volta.}{der.re.dor}{0}
\verb{derretedura}{}{}{}{}{s.f.}{Derretimento.}{der.re.te.du.ra}{0}
\verb{derreter}{ê}{}{}{}{v.t.}{Tornar líquido; fundir.}{der.re.ter}{0}
\verb{derreter}{ê}{}{}{}{}{Dissolver.}{der.re.ter}{0}
\verb{derreter}{ê}{Fig.}{}{}{}{Comover, enternecer.}{der.re.ter}{\verboinum{12}}
\verb{derretimento}{}{}{}{}{s.m.}{Ato ou efeito de derreter; derretedura, liquefação.}{der.re.ti.men.to}{0}
\verb{derretimento}{}{Fig.}{}{}{}{Comoção intensa, desvanecimento, afetação.}{der.re.ti.men.to}{0}
\verb{derribar}{}{}{}{}{v.t.}{Derrubar.}{der.ri.bar}{\verboinum{1}}
\verb{derriço}{}{Pop.}{}{}{s.m.}{Namoro, galanteio.}{der.ri.ço}{0}
\verb{derrisão}{}{}{"-ões}{}{s.f.}{Riso irônico, sarcástico e zombeteiro, geralmente acompanhado de desprezo.}{der.ri.são}{0}
\verb{derrocada}{}{}{}{}{s.f.}{Ato ou efeito de derrocar; destruição, desabamento.}{der.ro.ca.da}{0}
\verb{derrocar}{}{}{}{}{v.t.}{Pôr abaixo; destruir.}{der.ro.car}{0}
\verb{derrocar}{}{Fig.}{}{}{}{Humilhar, abater.}{der.ro.car}{\verboinum{2}}
\verb{derrogar}{}{}{}{}{v.t.}{Anular, abolir.}{der.ro.gar}{\verboinum{5}}
\verb{derrota}{ó}{}{}{}{s.f.}{Ato ou efeito de derrotar; perda, fracasso.}{der.ro.ta}{0}
\verb{derrotar}{}{}{}{}{v.t.}{Vencer em batalha, competição ou discussão.}{der.ro.tar}{0}
\verb{derrotar}{}{}{}{}{}{Fatigar, prostrar, exaurir.}{der.ro.tar}{0}
\verb{derrotar}{}{}{}{}{}{Desestimular, desencorajar.}{der.ro.tar}{\verboinum{1}}
\verb{derrotismo}{}{}{}{}{s.m.}{Atitude de quem só espera derrota; pessimismo.}{der.ro.tis.mo}{0}
\verb{derrotista}{}{}{}{}{adj.2g.}{Inclinado ao derrotismo.}{der.ro.tis.ta}{0}
\verb{derrubada}{}{}{}{}{s.f.}{Ato ou efeito de derrubar.}{der.ru.ba.da}{0}
\verb{derrubada}{}{}{}{}{}{Ato de remover as árvores de uma área, preparando"-a para o cultivo.}{der.ru.ba.da}{0}
\verb{derrubar}{}{}{}{}{v.t.}{Fazer cair.}{der.ru.bar}{0}
\verb{derrubar}{}{}{}{}{}{Lançar por terra; arriar.}{der.ru.bar}{0}
\verb{derrubar}{}{}{}{}{}{Demolir, destruir.}{der.ru.bar}{\verboinum{1}}
\verb{derruído}{}{}{}{}{adj.}{Desmoronado, destruído.}{der.ru.í.do}{0}
\verb{derruir}{}{}{}{}{v.t.}{Desmoronar, derribar, destruir.}{der.ru.ir}{\verboinum{26}}
\verb{dervixe}{ch}{}{}{}{s.m.}{Religioso muçulmano.}{der.vi.xe}{0}
\verb{desabado}{}{}{}{}{adj.}{Que desabou, desmoronou.}{de.sa.ba.do}{0}
\verb{desabado}{}{}{}{}{}{Diz"-se de chapéu que possui aba caída, em geral larga.}{de.sa.ba.do}{0}
\verb{desabafar}{}{}{}{}{v.t.}{Retirar aquilo que abafa; descobrir, desagasalhar.}{de.sa.ba.far}{0}
\verb{desabafar}{}{}{}{}{}{Dizer o que sente ou pensa.}{de.sa.ba.far}{\verboinum{1}}
\verb{desabafo}{}{}{}{}{s.m.}{Ato ou efeito de desabafar; confissão.}{de.sa.ba.fo}{0}
\verb{desabafo}{}{}{}{}{}{Manifestação ou ocorrência que satisfaz um desejo que estivera impossibilitado de se realizar; desafogo.}{de.sa.ba.fo}{0}
\verb{desabalado}{}{}{}{}{adj.}{Em que há muita pressa; precipitado, desembestado.}{de.sa.ba.la.do}{0}
\verb{desabalar}{}{}{}{}{v.i.}{Correr desenfreadamente; locomover"-se com precipitação.}{de.sa.ba.lar}{\verboinum{1}}
\verb{desabamento}{}{}{}{}{s.m.}{Ato ou efeito de desabar; desmoronamento, queda.}{de.sa.ba.men.to}{0}
\verb{desabar}{}{}{}{}{v.i.}{Cair com violência.}{de.sa.bar}{\verboinum{1}}
\verb{desabilitar}{}{}{}{}{v.t.}{Tornar inábil; fazer perder a habilidade.}{de.sa.bi.li.tar}{\verboinum{1}}
\verb{desabitado}{}{}{}{}{adj.}{Que não tem habitantes; deserto, ermo.}{de.sa.bi.ta.do}{0}
\verb{desabitar}{}{}{}{}{v.t.}{Deixar sem habitantes, sem moradores; despovoar.}{de.sa.bi.tar}{\verboinum{1}}
\verb{desabituar}{}{}{}{}{v.t.}{Fazer alguém perder algum costume; desacostumar.}{de.sa.bi.tu.ar}{\verboinum{1}}
\verb{desabonar}{}{}{}{}{v.t.}{Desacreditar; fazer perder o crédito ou a reputação; depreciar.}{de.sa.bo.nar}{\verboinum{1}}
\verb{desabono}{ô}{}{}{}{s.m.}{Ato ou efeito de desabonar; descrédito, depreciação.}{de.sa.bo.no}{0}
\verb{desabotoar}{}{}{}{}{v.t.}{Abrir as pétalas; desabrochar.}{de.sa.bo.to.ar}{0}
\verb{desabotoar}{}{}{}{}{}{Desapertar os botões de roupa para fora de suas casas.}{de.sa.bo.to.ar}{\verboinum{7}}
\verb{desabrido}{}{}{}{}{adj.}{Em que há falta de gentileza e respeito; áspero, grosseiro, violento.}{de.sa.bri.do}{0}
\verb{desabrigado}{}{}{}{}{adj.}{Que não tem abrigo.}{de.sa.bri.ga.do}{0}
\verb{desabrigado}{}{}{}{}{}{Desamparado, desprotegido.}{de.sa.bri.ga.do}{0}
\verb{desabrigar}{}{}{}{}{v.t.}{Tirar o abrigo; deixar exposto ao tempo.}{de.sa.bri.gar}{0}
\verb{desabrigar}{}{}{}{}{}{Desamparar, desproteger.}{de.sa.bri.gar}{\verboinum{5}}
\verb{desabrigo}{}{}{}{}{s.m.}{Falta de abrigo.}{de.sa.bri.go}{0}
\verb{desabrigo}{}{Fig.}{}{}{}{Desamparo, abandono.}{de.sa.bri.go}{0}
\verb{desabrimento}{}{}{}{}{s.m.}{Estado ou condição de desabrido; rispidez, rudeza no trato ou no procedimento.}{de.sa.bri.men.to}{0}
\verb{desabrochar}{}{}{}{}{v.t.}{Fazer a flor se abrir.}{de.sa.bro.char}{0}
\verb{desabrochar}{}{}{}{}{}{Mostrar, revelar.}{de.sa.bro.char}{0}
\verb{desabrochar}{}{}{}{}{v.i.}{Desenvolver"-se.}{de.sa.bro.char}{\verboinum{1}}
\verb{desabusado}{}{}{}{}{adj.}{Que é abusado; de caráter atrevido, insolente.}{de.sa.bu.sa.do}{0}
\verb{desacatar}{}{}{}{}{v.t.}{Faltar com o devido respeito; tratar com indelicadeza; desrespeitar.}{de.sa.ca.tar}{\verboinum{1}}
\verb{desacato}{}{}{}{}{s.m.}{Ato de desacatar; desrespeito.}{de.sa.ca.to}{0}
\verb{desacautelado}{}{}{}{}{adj.}{Que não tem cautela; descuidado.}{de.sa.cau.te.la.do}{0}
\verb{desacautelar}{}{}{}{}{v.t.}{Proceder sem cautela, sem cuidado.}{de.sa.cau.te.lar}{\verboinum{1}}
\verb{desaceitar}{}{}{}{}{v.t.}{Não aceitar; rejeitar.}{de.sa.cei.tar}{\verboinum{1}}
\verb{desacelerar}{}{}{}{}{v.t.}{Reduzir a velocidade; retardar.}{de.sa.ce.le.rar}{\verboinum{1}}
\verb{desacertado}{}{}{}{}{adj.}{Não acertado; errado.}{de.sa.cer.ta.do}{0}
\verb{desacertado}{}{}{}{}{}{Que se mostra inconveniente, inoportuno.}{de.sa.cer.ta.do}{0}
\verb{desacertado}{}{}{}{}{}{Que não apresenta sensatez; despropositado.}{de.sa.cer.ta.do}{0}
\verb{desacertar}{}{}{}{}{v.t.}{Tirar da ordem ou do acerto.}{de.sa.cer.tar}{0}
\verb{desacertar}{}{}{}{}{v.i.}{Proceder erradamente.}{de.sa.cer.tar}{\verboinum{1}}
\verb{desacerto}{ê}{}{}{}{s.m.}{Falta de acerto; erro.}{de.sa.cer.to}{0}
\verb{desacerto}{ê}{}{}{}{}{Tolice, asneira.}{de.sa.cer.to}{0}
\verb{desacolchetar}{}{}{}{}{v.t.}{Abrir, desprendendo os colchetes.}{de.sa.col.che.tar}{\verboinum{1}}
\verb{desacomodado}{}{}{}{}{adj.}{Que está fora do seu lugar; desarrumado, desordenado.}{de.sa.co.mo.da.do}{0}
\verb{desacomodar}{}{}{}{}{v.t.}{Tirar pessoa ou coisa do lugar em que foi colocada; desalojar.}{de.sa.co.mo.dar}{\verboinum{1}}
\verb{desacompanhado}{}{}{}{}{adj.}{Que está sem companhia; só, isolado, solitário.}{de.sa.com.pa.nha.do}{0}
\verb{desacompanhar}{}{}{}{}{v.t.}{Deixar de acompanhar.}{de.sa.com.pa.nhar}{0}
\verb{desacompanhar}{}{}{}{}{}{Recusar apoio.}{de.sa.com.pa.nhar}{0}
\verb{desacompanhar}{}{}{}{}{}{Perder interesse por algo.}{de.sa.com.pa.nhar}{\verboinum{1}}
\verb{desaconchegar}{}{}{}{}{v.t.}{Tirar o aconchego; desconchegar.}{de.sa.con.che.gar}{\verboinum{5}}
\verb{desaconselhar}{}{}{}{}{v.t.}{Aconselhar alguém a que não faça determinada coisa; dissuadir.}{de.sa.con.se.lhar}{\verboinum{1}}
\verb{desacorçoar}{}{}{}{}{v.t.}{Fazer perder a esperança; desapontar, desanimar; descoroçoar; desacoroçoar.}{de.sa.cor.ço.ar}{\verboinum{7}}
\verb{desacordado}{}{}{}{}{adj.}{Que perdeu os sentidos; desmaiado.}{de.sa.cor.da.do}{0}
\verb{desacordar}{}{}{}{}{v.t.}{Pôr em desacordo ou em oposição.}{de.sa.cor.dar}{0}
\verb{desacordar}{}{}{}{}{}{Fazer perder os sentidos.}{de.sa.cor.dar}{\verboinum{1}}
\verb{desacordo}{ô}{}{}{}{s.m.}{Falta de acordo; desarmonia, divergência, discordância.}{de.sa.cor.do}{0}
\verb{desacoroçoar}{}{}{}{}{v.t.}{Descorçoar.}{de.sa.co.ro.ço.ar}{\verboinum{7}}
\verb{desacorrentar}{}{}{}{}{v.t.}{Desligar da corrente.}{de.sa.cor.ren.tar}{0}
\verb{desacorrentar}{}{}{}{}{}{Desprender, soltar.}{de.sa.cor.ren.tar}{\verboinum{1}}
\verb{desacostumar}{}{}{}{}{v.t.}{Fazer perder um hábito ou costume; desabituar.}{de.sa.cos.tu.mar}{\verboinum{1}}
\verb{desacreditado}{}{}{}{}{adj.}{Que perdeu o crédito ou a confiança.}{de.sa.cre.di.ta.do}{0}
\verb{desacreditar}{}{}{}{}{v.t.}{Fazer perder o crédito ou a reputação; difamar, desmerecer, depreciar.}{de.sa.cre.di.tar}{\verboinum{1}}
\verb{desadorar}{}{}{}{}{v.t.}{Recusar"-se a adorar; não adorar.}{de.sa.do.rar}{0}
\verb{desadorar}{}{}{}{}{}{Detestar, abominar.}{de.sa.do.rar}{0}
\verb{desadorar}{}{}{}{}{}{Reprovar, desaprovar.}{de.sa.do.rar}{\verboinum{1}}
\verb{desafeição}{}{}{"-ões}{}{s.f.}{Falta de afeto; desamor.}{de.sa.fei.ção}{0}
\verb{desafeiçoar}{}{}{}{}{v.t.}{Modificar as feições; desfigurar.}{de.sa.fei.ço.ar}{0}
\verb{desafeiçoar}{}{}{}{}{}{Fazer perder o afeto, a amizade ou o amor por alguém ou algo.}{de.sa.fei.ço.ar}{\verboinum{7}}
\verb{desafeito}{ê}{}{}{}{adj.}{Desacostumado.}{de.sa.fei.to}{0}
\verb{desaferrar}{}{}{}{}{v.t.}{Soltar o que estava preso com o ferro ou o que estava seguro.}{de.sa.fer.rar}{0}
\verb{desaferrar}{}{}{}{}{}{Fazer desistir; dissuadir.}{de.sa.fer.rar}{\verboinum{1}}
\verb{desaferrolhar}{}{}{}{}{v.t.}{Abrir, correndo o ferrolho.}{de.sa.fer.ro.lhar}{\verboinum{1}}
\verb{desafetação}{}{}{"-ões}{}{s.f.}{Ausência de afetação; espontaneidade na maneira de ser, falar e agir; naturalidade, simplicidade.}{de.sa.fe.ta.ção}{0}
\verb{desafeto}{é}{}{}{}{s.m.}{Ausência de afeto ou afeição.}{de.sa.fe.to}{0}
\verb{desafeto}{é}{}{}{}{adj.}{Que desperta antipatia.}{de.sa.fe.to}{0}
\verb{desafiar}{}{}{}{}{v.t.}{Propor duelo ou combate; provocar.}{de.sa.fi.ar}{0}
\verb{desafiar}{}{}{}{}{}{Estar sem medo diante de uma coisa perigosa; afrontar, enfrentar.}{de.sa.fi.ar}{\verboinum{1}}
\verb{desafinação}{}{}{"-ões}{}{s.f.}{Ato ou efeito de desafinar; falta de harmonia.}{de.sa.fi.na.ção}{0}
\verb{desafinado}{}{Mús.}{}{}{adj.}{Que saiu do tom próprio; não afinado; desentoado.}{de.sa.fi.na.do}{0}
\verb{desafinar}{}{}{}{}{v.t.}{Fazer perder a afinação; executar notas erradas.}{de.sa.fi.nar}{\verboinum{1}}
\verb{desafio}{}{}{}{}{s.m.}{Convite para lutar; provocação.}{de.sa.fi.o}{0}
\verb{desafio}{}{}{}{}{}{Situação ou grande problema a ser vencido ou superado; tarefa difícil de ser executada. }{de.sa.fi.o}{0}
\verb{desafio}{}{Mús.}{}{}{}{Discussão em verso entre dois cantadores.}{de.sa.fi.o}{0}
\verb{desafivelar}{}{}{}{}{v.t.}{Abrir ou soltar; desprendendo de fivela ou presilha.}{de.sa.fi.ve.lar}{\verboinum{1}}
\verb{desafogado}{}{}{}{}{adj.}{Que está aliviado, desembaraçado.}{de.sa.fo.ga.do}{0}
\verb{desafogado}{}{}{}{}{}{Que é amplo, bem arejado.}{de.sa.fo.ga.do}{0}
\verb{desafogar}{}{}{}{}{v.t.}{Livrar do que aperta, sufoca ou oprime.}{de.sa.fo.gar}{0}
\verb{desafogar}{}{}{}{}{}{Manifestar francamente o que sente.}{de.sa.fo.gar}{\verboinum{5}}
\verb{desafogo}{ô}{}{}{}{s.m.}{Ato ou efeito de desafogar; alívio.}{de.sa.fo.go}{0}
\verb{desaforado}{}{}{}{}{adj.}{Diz"-se de quem ofende pela falta de respeito; atrevido, insolente.}{de.sa.fo.ra.do}{0}
\verb{desaforado}{}{Jur.}{}{}{}{Que está livre ou isento do pagamento de pensão.}{de.sa.fo.ra.do}{0}
\verb{desaforar}{}{}{}{}{v.t.}{Isentar do pagamento de um foro.}{de.sa.fo.rar}{0}
\verb{desaforar}{}{}{}{}{}{Tornar insolente, atrevido.}{de.sa.fo.rar}{\verboinum{1}}
\verb{desaforo}{ô}{}{}{}{s.m.}{Palavra ou ato que mostra a falta de respeito por outra pessoa; insolência, atrevimento.}{de.sa.fo.ro}{0}
\verb{desafortunado}{}{}{}{}{adj.}{Diz"-se de quem é marcado pela má sorte, pelo fracasso, pela desgraça.}{de.sa.for.tu.na.do}{0}
\verb{desafronta}{}{}{}{}{s.f.}{Reparação de uma afronta, de uma ofensa; vingança.}{de.sa.fron.ta}{0}
\verb{desafrontar}{}{}{}{}{v.t.}{Obter reparação de uma afronta, de uma ofensa, feita a alguém ou a si próprio; vingar.}{de.sa.fron.tar}{\verboinum{1}}
\verb{desagasalhado}{}{}{}{}{adj.}{Que não está coberto por agasalho.}{de.sa.ga.sa.lha.do}{0}
\verb{desagasalhado}{}{Fig.}{}{}{}{Desamparado, desprotegido.}{de.sa.ga.sa.lha.do}{0}
\verb{desagasalhar}{}{}{}{}{v.t.}{Tirar o agasalho; descobrir.}{de.sa.ga.sa.lhar}{0}
\verb{desagasalhar}{}{Fig.}{}{}{}{Deixar sem abrigo.}{de.sa.ga.sa.lhar}{\verboinum{1}}
\verb{deságio}{}{Econ.}{}{}{s.m.}{Depreciação do valor nominal de um título ou do preço de uma mercadoria em relação ao seu valor de mercado.}{de.sá.gio}{0}
\verb{deságio}{}{}{}{}{}{Desvalorização ou depreciação da moeda.}{de.sá.gio}{0}
\verb{desagradar}{}{}{}{}{v.t.}{Causar desagrado a alguém; aborrecer, descontentar, desgostar.}{de.sa.gra.dar}{\verboinum{1}}
\verb{desagradável}{}{}{"-eis}{}{adj.2g.}{Que não agrada; que causa desprazer; que impressiona mal.}{de.sa.gra.dá.vel}{0}
\verb{desagrado}{}{}{}{}{s.m.}{Ato ou efeito de desagradar; falta de agrado; desprazer.}{de.sa.gra.do}{0}
\verb{desagravar}{}{}{}{}{v.t.}{Fazer o necessário para que a ofensa lhe seja perdoada; reparar.}{de.sa.gra.var}{\verboinum{1}}
\verb{desagravo}{}{}{}{}{s.m.}{Reparação de afronta, de injúria.}{de.sa.gra.vo}{0}
\verb{desagregação}{}{}{"-ões}{}{s.f.}{Ato ou efeito de desagregar.}{de.sa.gre.ga.ção}{0}
\verb{desagregação}{}{}{"-ões}{}{}{Separação de partes que estavam agregadas.}{de.sa.gre.ga.ção}{0}
\verb{desagregação}{}{Fig.}{"-ões}{}{}{Dispersão, dissolução, desorganização.}{de.sa.gre.ga.ção}{0}
\verb{desagregar}{}{}{}{}{v.t.}{Separar um grupo de pessoas ou animais; desorganizar, desunir, dispersar.}{de.sa.gre.gar}{\verboinum{5}}
\verb{desaguadoiro}{ô}{}{}{}{}{Var. de \textit{desaguadouro}.}{de.sa.gua.doi.ro}{0}
\verb{desaguadouro}{ô}{}{}{}{s.m.}{Lugar para onde ou por onde se escoam águas; rego de escoamento, vala.}{de.sa.gua.dou.ro}{0}
\verb{desaguar}{}{}{}{}{v.t.}{Despejar água.}{de.sa.guar}{0}
\verb{desaguar}{}{}{}{}{}{Enxugar, secar.}{de.sa.guar}{0}
\verb{desaguar}{}{}{}{}{v.i.}{Lançar as suas águas em outra extensão de águas.}{de.sa.guar}{\verboinum{9}\verboirregular{aguar}}
\verb{desaguisado}{}{}{}{}{s.m.}{Conflito entre pessoas; rixa, contenda.}{de.sa.gui.sa.do}{0}
\verb{desaguisado}{}{}{}{}{}{Confusão, desordem.}{de.sa.gui.sa.do}{0}
\verb{desaire}{}{}{}{}{s.m.}{Falta de decoro; inconveniência.}{de.sai.re}{0}
\verb{desairoso}{ô}{}{"-osos ⟨ó⟩}{"-osa ⟨ó⟩}{adj.}{Que tem, ou em que há desaire; indecoroso, inconveniente.}{de.sai.ro.so}{0}
\verb{desajeitado}{}{}{}{}{adj.}{Sem jeito para alguma coisa; desastrado, inábil.}{de.sa.jei.ta.do}{0}
\verb{desajeitar}{}{}{}{}{v.t.}{Fazer perder o jeito, a forma.}{de.sa.jei.tar}{0}
\verb{desajeitar}{}{}{}{}{}{Pôr fora de ordem; desarranjar, desarrumar.}{de.sa.jei.tar}{\verboinum{1}}
\verb{desajudar}{}{}{}{}{v.t.}{Faltar com a ajuda, com o auxílio.}{de.sa.ju.dar}{0}
\verb{desajudar}{}{}{}{}{}{Prejudicar, estorvar.}{de.sa.ju.dar}{\verboinum{1}}
\verb{desajuizado}{}{}{}{}{adj.}{Diz"-se de quem dá mostras de não ter juízo; inconsequente, insensato, imprudente.}{de.sa.ju.i.za.do}{0}
\verb{desajustado}{}{}{}{}{adj.}{Que se desajustou; enguiçado.}{de.sa.jus.ta.do}{0}
\verb{desajustado}{}{}{}{}{}{Que não se adapta ao meio em que vive.}{de.sa.jus.ta.do}{0}
\verb{desajustamento}{}{}{}{}{s.m.}{Falta de adaptação ao meio em que vive; desajuste.}{de.sa.jus.ta.men.to}{0}
\verb{desajustar}{}{}{}{}{v.t.}{Desfazer um acordo.}{de.sa.jus.tar}{0}
\verb{desajustar}{}{}{}{}{}{Desorganizar.}{de.sa.jus.tar}{0}
\verb{desajustar}{}{}{}{}{}{Desequilibrar emocionalmente.}{de.sa.jus.tar}{\verboinum{1}}
\verb{desajuste}{}{}{}{}{s.m.}{Rompimento de acordo.}{de.sa.jus.te}{0}
\verb{desajuste}{}{}{}{}{}{Mau alinhamento.}{de.sa.jus.te}{0}
\verb{desajuste}{}{}{}{}{}{Desorganização, desarranjo.}{de.sa.jus.te}{0}
\verb{desajuste}{}{}{}{}{}{Desequilíbrio emocional.}{de.sa.jus.te}{0}
\verb{desalentado}{}{}{}{}{adj.}{Sem ânimo ou alento; desanimado.}{de.sa.len.ta.do}{0}
\verb{desalentado}{}{}{}{}{}{Cansado, fatigado.}{de.sa.len.ta.do}{0}
\verb{desalentador}{ô}{}{}{}{adj.}{Diz"-se do que desalenta, que faz perder o ânimo, que tira a vontade de agir. }{de.sa.len.ta.dor}{0}
\verb{desalentar}{}{}{}{}{v.t.}{Fazer perder o alento, o ânimo; esmorecer, desencorajar.}{de.sa.len.tar}{\verboinum{1}}
\verb{desalento}{}{}{}{}{s.m.}{Estado de quem se mostra sem alento; desânimo, abatimento.}{de.sa.len.to}{0}
\verb{desalinhado}{}{}{}{}{adj.}{Sem alinho; descuidado, desordenado.}{de.sa.li.nha.do}{0}
\verb{desalinhar}{}{}{}{}{v.t.}{Tirar do alinhamento.}{de.sa.li.nhar}{0}
\verb{desalinhar}{}{}{}{}{}{Desarranjar, desordenar.}{de.sa.li.nhar}{\verboinum{1}}
\verb{desalinhavar}{}{}{}{}{v.t.}{Tirar os alinhavos de costura, roupa etc.}{de.sa.li.nha.var}{\verboinum{1}}
\verb{desalinho}{}{}{}{}{s.m.}{Falta de alinho; descuido no traje; desarranjo, desordem.}{de.sa.li.nho}{0}
\verb{desalinho}{}{}{}{}{}{Perturbação de ânimo.}{de.sa.li.nho}{0}
\verb{desalmado}{}{}{}{}{adj.}{Que demonstra maus sentimentos; cruel, perverso, desumano.}{de.sal.ma.do}{0}
\verb{desalojado}{}{}{}{}{adj.}{Que foi tirado do lugar que ocupava; despejado.}{de.sa.lo.ja.do}{0}
\verb{desalojar}{}{}{}{}{v.t.}{Fazer sair de alojamento; de lugar em que está ou em que reside; retirar.}{de.sa.lo.jar}{0}
\verb{desalojar}{}{}{}{}{}{Forçar a abandonar um posto, uma posição; expulsar.}{de.sa.lo.jar}{\verboinum{1}}
\verb{desalterar}{}{}{}{}{v.t.}{Saciar a sede; dessedentar.}{de.sal.te.rar}{0}
\verb{desalterar}{}{}{}{}{}{Tornar menos alterado, mais calmo; aliviar, tranquilizar.}{de.sal.te.rar}{\verboinum{1}}
\verb{desamar}{}{}{}{}{v.t.}{Deixar de amar.}{de.sa.mar}{\verboinum{1}}
\verb{desamarrar}{}{}{}{}{v.t.}{Soltar o que estava amarrado; desprender, desatar.}{de.sa.mar.rar}{\verboinum{1}}
\verb{desamarrotar}{}{}{}{}{v.t.}{Tornar liso, sem vincos ou dobras; alisar, desenrugar.}{de.sa.mar.ro.tar}{\verboinum{1}}
\verb{desamassar}{}{}{}{}{v.t.}{Desfazer o amassado; desamarrotar, alisar.}{de.sa.mas.sar}{\verboinum{1}}
\verb{desambição}{}{}{"-ões}{}{s.f.}{Ausência de ambição; falta de interesse; desprendimento, desapego.}{de.sam.bi.ção}{0}
\verb{desambicioso}{ô}{}{"-osos ⟨ó⟩}{"-osa ⟨ó⟩}{adj.}{Que não tem ambição; desinteressado, despretensioso.}{de.sam.bi.ci.o.so}{0}
\verb{desambientado}{}{}{}{}{adj.}{Que está fora de seu ambiente.}{de.sam.bi.en.ta.do}{0}
\verb{desambientado}{}{}{}{}{}{Que ainda não se adaptou ou não se pode adaptar ao novo meio onde vive.}{de.sam.bi.en.ta.do}{0}
\verb{desambientar}{}{}{}{}{v.t.}{Tirar pessoa, animal ou coisa de seu ambiente.}{de.sam.bi.en.tar}{\verboinum{1}}
\verb{desamontoar}{}{}{}{}{v.t.}{Desfazer um amontoado; separar coisas que estão acumuladas em monte.}{de.sa.mon.to.ar}{\verboinum{7}}
\verb{desamor}{ô}{}{}{}{s.m.}{Perda ou ausência de amor; desafeição, desprezo, indiferença.}{de.sa.mor}{0}
\verb{desamparado}{}{}{}{}{adj.}{Deixado ao desamparo; abandonado.}{de.sam.pa.ra.do}{0}
\verb{desamparado}{}{}{}{}{}{Solitário, ermo.}{de.sam.pa.ra.do}{0}
\verb{desamparar}{}{}{}{}{v.t.}{Privar de ajuda ou de escora.}{de.sam.pa.rar}{\verboinum{1}}
\verb{desamparo}{}{}{}{}{s.m.}{Falta de amparo; abandono.}{de.sam.pa.ro}{0}
\verb{desamuar}{}{}{}{}{v.t.}{Tirar o amuo; alegrar, desemburrar.}{de.sa.mu.ar}{\verboinum{1}}
\verb{desancar}{}{}{}{}{v.t.}{Espancar.}{de.san.car}{0}
\verb{desancar}{}{}{}{}{}{Criticar severamente.}{de.san.car}{\verboinum{2}}
\verb{desancorar}{}{}{}{}{v.t.}{Levantar a âncora.}{de.san.co.rar}{\verboinum{1}}
\verb{desandar}{}{}{}{}{v.t.}{Fazer andar para trás.}{de.san.dar}{0}
\verb{desandar}{}{}{}{}{}{Entrar em decadência.}{de.san.dar}{0}
\verb{desandar}{}{}{}{}{}{Adquirir maus hábitos.}{de.san.dar}{0}
\verb{desandar}{}{}{}{}{}{Não obter a consistência desejada; talhar. (\textit{A maionese desandou.})}{de.san.dar}{\verboinum{1}}
\verb{desanexar}{cs}{}{}{}{v.t.}{Separar aquilo que estava anexado; desligar, desmembrar.}{de.sa.ne.xar}{\verboinum{1}}
\verb{desanimação}{}{}{"-ões}{}{s.f.}{Falta de animação; desânimo, desalento.}{de.sa.ni.ma.ção}{0}
\verb{desanimador}{ô}{}{}{}{adj.}{Que desanima, tira o alento, o estímulo.}{de.sa.ni.ma.dor}{0}
\verb{desanimar}{}{}{}{}{v.t.}{Fazer perder o ânimo ou a animação.}{de.sa.ni.mar}{\verboinum{1}}
\verb{desânimo}{}{}{}{}{s.m.}{Estado de quem se mostra desanimado, desestimulado.}{de.sâ.ni.mo}{0}
\verb{desaninhar}{}{}{}{}{v.t.}{Tirar do ninho.}{de.sa.ni.nhar}{0}
\verb{desaninhar}{}{}{}{}{}{Fazer sair; desalojar.}{de.sa.ni.nhar}{\verboinum{1}}
\verb{desanuviar}{}{}{}{}{v.t.}{Limpar de nuvens; tornar límpido, sem nuvens.}{de.sa.nu.vi.ar}{0}
\verb{desanuviar}{}{Fig.}{}{}{}{Serenar, tranquilizar.}{de.sa.nu.vi.ar}{\verboinum{1}}
\verb{desapaixonado}{ch}{}{}{}{adj.}{Que não tem, ou em que não há paixão; isento.}{de.sa.pai.xo.na.do}{0}
\verb{desapaixonado}{ch}{}{}{}{}{Que age com imparcialidade; imparcial.}{de.sa.pai.xo.na.do}{0}
\verb{desaparafusar}{}{}{}{}{v.t.}{Desprender os parafusos de alguma coisa; desparafusar, desatarraxar.}{de.sa.pa.ra.fu.sar}{\verboinum{1}}
\verb{desaparecer}{ê}{}{}{}{v.i.}{Deixar de ser visto em algum lugar; sumir.}{de.sa.pa.re.cer}{0}
\verb{desaparecer}{ê}{}{}{}{}{Deixar de existir; morrer.}{de.sa.pa.re.cer}{\verboinum{15}}
\verb{desaparecido}{}{}{}{}{adj.}{Que desapareceu, que deixou de estar à vista.}{de.sa.pa.re.ci.do}{0}
\verb{desaparecido}{}{}{}{}{}{Que deixou de existir; extinto.}{de.sa.pa.re.ci.do}{0}
\verb{desaparecimento}{}{}{}{}{s.m.}{Ato ou efeito de desaparecer, de deixar de ser visto.}{de.sa.pa.re.ci.men.to}{0}
\verb{desaparecimento}{}{}{}{}{}{Extravio ou roubo.}{de.sa.pa.re.ci.men.to}{0}
\verb{desaparecimento}{}{}{}{}{}{Morte, falecimento.}{de.sa.pa.re.ci.men.to}{0}
\verb{desaparecimento}{}{}{}{}{}{Ato ou efeito de deixar de existir, de manifestar"-se.}{de.sa.pa.re.ci.men.to}{0}
\verb{desaparelhado}{}{}{}{}{adj.}{Desprevenido, desprovido.}{de.sa.pa.re.lha.do}{0}
\verb{desaparelhado}{}{}{}{}{}{Desguarnecido.}{de.sa.pa.re.lha.do}{0}
\verb{desaparição}{}{}{"-ões}{}{s.f.}{Desaparecimento.}{de.sa.pa.ri.ção}{0}
\verb{desapegado}{}{}{}{}{adj.}{Que não tem afeição, apego por alguém ou alguma coisa.}{de.sa.pe.ga.do}{0}
\verb{desapegado}{}{}{}{}{}{Que não se interessa por algo; desprendido, indiferente. (\textit{O moço é desapegado do dinheiro.})}{de.sa.pe.ga.do}{0}
\verb{desapegar}{}{}{}{}{v.t.}{Tornar menos afeiçoado.}{de.sa.pe.gar}{0}
\verb{desapegar}{}{}{}{}{}{Fazer perder o envolvimento, a dependência ou o compromisso com algo; afastar, libertar.}{de.sa.pe.gar}{\verboinum{5}}
\verb{desapego}{ê}{}{}{}{s.m.}{Qualidade ou estado de pessoa desapegada, que revela desamor por alguém.}{de.sa.pe.go}{0}
\verb{desapego}{ê}{}{}{}{}{Qualidade ou estado de quem demonstra indiferença, desinteresse, desprendimento por algo.}{de.sa.pe.go}{0}
\verb{desaperceber}{ê}{}{}{}{v.t.}{Fazer ficar sem as coisas de que precisa para se manter.}{de.sa.per.ce.ber}{\verboinum{12}}
\verb{desapercebido}{}{}{}{}{adj.}{Que não está preparado, sem munições; desmunido.}{de.sa.per.ce.bi.do}{0}
\verb{desapercebido}{}{}{}{}{}{Que está fora de sua guarda; desacautelado, desprevenido.}{de.sa.per.ce.bi.do}{0}
\verb{desapercebido}{}{}{}{}{}{Que não foi percebido; não observado.}{de.sa.per.ce.bi.do}{0}
\verb{desapertar}{}{}{}{}{v.t.}{Tornar folgado; afrouxar, alargar.}{de.sa.per.tar}{0}
\verb{desapertar}{}{Fig.}{}{}{}{Aliviar daquilo que preocupa, angustia.}{de.sa.per.tar}{0}
\verb{desapertar}{}{Fig.}{}{}{}{Tirar de dificuldade financeira.}{de.sa.per.tar}{\verboinum{1}}
\verb{desaperto}{ê}{}{}{}{s.m.}{Ato ou efeito de desapertar; afrouxamento, desarrocho.}{de.sa.per.to}{0}
\verb{desaperto}{ê}{}{}{}{}{Alívio daquilo que oprime.}{de.sa.per.to}{0}
\verb{desaperto}{ê}{}{}{}{}{Alívio de dificuldade financeira.}{de.sa.per.to}{0}
\verb{desapiedado}{}{}{}{}{adj.}{Que não tem piedade ou compaixão; desumano, cruel.}{de.sa.pi.e.da.do}{0}
\verb{desapiedar}{}{}{}{}{v.t.}{Fazer perder a piedade; tornar indiferente à dor e ao sofrimento alheio.}{de.sa.pi.e.dar}{\verboinum{1}}
\verb{desapoiar}{}{}{}{}{v.t.}{Tirar o apoio; não concordar; desaprovar.}{de.sa.poi.ar}{\verboinum{1}}
\verb{desapontado}{}{}{}{}{adj.}{Que sofreu desapontamento; desiludido, decepcionado.}{de.sa.pon.ta.do}{0}
\verb{desapontamento}{}{}{}{}{s.m.}{Surpresa desagradável; decepção, desilusão.}{de.sa.pon.ta.men.to}{0}
\verb{desapontar}{}{}{}{}{v.t.}{Causar desapontamento; desiludir, decepcionar.}{de.sa.pon.tar}{\verboinum{1}}
\verb{desapossar}{}{}{}{}{v.t.}{Tirar a posse; despojar.}{de.sa.pos.sar}{\verboinum{1}}
\verb{desapreço}{ê}{}{}{}{s.m.}{Falta de apreço; desconsideração, menosprezo.}{de.sa.pre.ço}{0}
\verb{desaprender}{ê}{}{}{}{v.t.}{Esquecer o que se aprendeu.}{de.sa.pren.der}{\verboinum{12}}
\verb{desapropriação}{}{}{"-ões}{}{s.f.}{Ato ou efeito de desapropriar; expropriação.}{de.sa.pro.pri.a.ção}{0}
\verb{desapropriado}{}{}{}{}{adj.}{Que foi objeto de desapropriação; expropriado.}{de.sa.pro.pri.a.do}{0}
\verb{desapropriar}{}{}{}{}{v.t.}{Tirar ou fazer perder a propriedade; expropriar.}{de.sa.pro.pri.ar}{\verboinum{1}}
\verb{desaprovação}{}{}{"-ões}{}{s.f.}{Ato ou efeito de desaprovar; censura, reprovação.}{de.sa.pro.va.ção}{0}
\verb{desaprovar}{}{}{}{}{v.t.}{Não aprovar; censurar, reprovar.}{de.sa.pro.var}{\verboinum{1}}
\verb{desaproveitar}{}{}{}{}{v.t.}{Não aproveitar; desperdiçar.}{de.sa.pro.vei.tar}{\verboinum{1}}
\verb{desaprumar}{}{}{}{}{v.t.}{Tirar do prumo; fazer perder o equilíbrio.}{de.sa.pru.mar}{0}
\verb{desaprumar}{}{Fig.}{}{}{}{Abater, humilhar, desconcertar.}{de.sa.pru.mar}{\verboinum{1}}
\verb{desaprumo}{}{}{}{}{s.m.}{Ato ou efeito de desaprumar; falta de prumo; desequilíbrio.}{de.sa.pru.mo}{0}
\verb{desarmamento}{}{}{}{}{s.m.}{Ato ou efeito de desarmar.}{de.sar.ma.men.to}{0}
\verb{desarmamento}{}{}{}{}{}{Redução das forças militares e das armas de uma nação.}{de.sar.ma.men.to}{0}
\verb{desarmar}{}{}{}{}{v.t.}{Tirar as armas de alguém.}{de.sar.mar}{0}
\verb{desarmar}{}{}{}{}{}{Tirar a munição da arma.}{de.sar.mar}{0}
\verb{desarmar}{}{Fig.}{}{}{}{Apaziguar, serenar, acalmar.}{de.sar.mar}{\verboinum{1}}
\verb{desarmonia}{}{}{}{}{s.f.}{Falta de harmonia; dissonância.}{de.sar.mo.ni.a}{0}
\verb{desarmonia}{}{}{}{}{}{Desacordo, desavença, divergência.}{de.sar.mo.ni.a}{0}
\verb{desarmonizar}{}{}{}{}{v.t.}{Tirar a proporção, o equilíbrio, a harmonia.}{de.sar.mo.ni.zar}{0}
\verb{desarmonizar}{}{}{}{}{}{Pôr em desacordo; discordar.}{de.sar.mo.ni.zar}{\verboinum{1}}
\verb{desarraigar}{}{}{}{}{v.t.}{Arrancar pela raiz; extirpar, desenraizar.}{de.sar.ra.i.gar}{\verboinum{5}}
\verb{desarranjar}{}{}{}{}{v.t.}{Tirar do arranjo, da ordem costumeira; desordenar, alterar, transtornar.}{de.sar.ran.jar}{\verboinum{1}}
\verb{desarranjo}{}{}{}{}{s.m.}{Desordem, desarrumação, desalinho.}{de.sar.ran.jo}{0}
\verb{desarrazoado}{}{}{}{}{adj.}{Que não é razoável; despropositado,  injusto.}{de.sar.ra.zo.a.do}{0}
\verb{desarrazoar}{}{}{}{}{v.i.}{Dizer ou proceder de modo oposto à razão.}{de.sar.ra.zo.ar}{\verboinum{7}}
\verb{desarrear}{}{}{}{}{v.t.}{Tirar os arreios.}{de.sar.re.ar}{\verboinum{4}}
\verb{desarrochar}{}{}{}{}{v.t.}{Soltar o que está arrochado; desapertar, afrouxar.}{de.sar.ro.char}{\verboinum{1}}
\verb{desarrolhar}{}{}{}{}{v.t.}{Tirar a rolha. (\textit{Meu pai desarrolhou a garrafa de vinho.})}{de.sar.ro.lhar}{\verboinum{1}}
\verb{desarrumação}{}{}{"-ões}{}{s.f.}{Ato ou efeito de desarrumar.}{de.sar.ru.ma.ção}{0}
\verb{desarrumação}{}{}{"-ões}{}{}{Desordem, confusão, desarranjo.}{de.sar.ru.ma.ção}{0}
\verb{desarrumar}{}{}{}{}{v.t.}{Pôr fora do lugar; desarranjar, desordenar.}{de.sar.ru.mar}{\verboinum{1}}
\verb{desarticulação}{}{}{"-ões}{}{s.f.}{Ato ou efeito de desarticular; falta de articulação.}{de.sar.ti.cu.la.ção}{0}
\verb{desarticulado}{}{}{}{}{adj.}{Que se desarticulou; desconjuntado.}{de.sar.ti.cu.la.do}{0}
\verb{desarticular}{}{}{}{}{v.t.}{Desfazer uma organização; desunir, desmantelar.}{de.sar.ti.cu.lar}{0}
\verb{desarticular}{}{Med.}{}{}{}{Deslocar um osso do corpo; torcer, luxar.}{de.sar.ti.cu.lar}{\verboinum{1}}
\verb{desarvorado}{}{}{}{}{adj.}{Sem rumo certo; sem governo; desnorteado.}{de.sar.vo.ra.do}{0}
\verb{desarvorado}{}{}{}{}{}{Que fugiu desordenadamente.}{de.sar.vo.ra.do}{0}
\verb{desarvorar}{}{}{}{}{v.t.}{Tirar os mastros de uma embarcação.}{de.sar.vo.rar}{0}
\verb{desarvorar}{}{}{}{}{v.pron.}{Desorientar"-se; desnortear"-se.}{de.sar.vo.rar}{\verboinum{1}}
\verb{desasado}{}{}{}{}{adj.}{Que teve as asas partidas ou quebradas.}{de.sa.sa.do}{0}
\verb{desasado}{}{Fig.}{}{}{}{De aspecto desajeitado; derreado, estropiado.}{de.sa.sa.do}{0}
\verb{desasnar}{}{Fig.}{}{}{v.t.}{Tirar da ignorância; dar instrução; ensinar.}{de.sas.nar}{\verboinum{1}}
\verb{desasseio}{ê}{}{}{}{s.m.}{Falta de asseio, de limpeza; sujeira.}{de.sas.sei.o}{0}
\verb{desassimilação}{}{Biol.}{"-ões}{}{s.f.}{Processo de eliminação de substâncias absorvidas por um organismo vivo; catabolismo.}{de.sas.si.mi.la.ção}{0}
\verb{desassimilar}{}{}{}{}{v.t.}{Fazer parar a assimilação; alterar.}{de.sas.si.mi.lar}{\verboinum{1}}
\verb{desassisado}{}{}{}{}{adj.}{Que não tem siso; desajuizado, desatinado.}{de.sas.si.sa.do}{0}
\verb{desassistido}{}{}{}{}{adj.}{Que não tem amparo, ajuda; desprotegido.}{de.sas.sis.ti.do}{0}
\verb{desassociar}{}{}{}{}{v.t.}{Desligar o vínculo; desunir.}{de.sas.so.ci.ar}{\verboinum{1}}
\verb{desassombrado}{}{}{}{}{adj.}{Sem medo; destemido, corajoso, ousado.}{de.sas.som.bra.do}{0}
\verb{desassombrado}{}{Desus.}{}{}{}{Exposto ao sol; claro, iluminado.}{de.sas.som.bra.do}{0}
\verb{desassombrar}{}{}{}{}{v.t.}{Fazer perder o medo; encorajar.}{de.sas.som.brar}{0}
\verb{desassombrar}{}{Desus.}{}{}{}{Tirar da sombra; iluminar.}{de.sas.som.brar}{\verboinum{1}}
\verb{desassombro}{}{}{}{}{s.m.}{Falta do assombro; coragem, intrepidez, ousadia.}{de.sas.som.bro}{0}
\verb{desassossegar}{}{}{}{}{v.t.}{Tirar o sossego; inquietar, intranquilizar.}{de.sas.sos.se.gar}{\verboinum{5}}
\verb{desassossego}{ê}{}{}{}{s.m.}{Falta de sossego; inquietação, intranquilidade.}{de.sas.sos.se.go}{0}
\verb{desastrado}{}{}{}{}{adj.}{Que não tem habilidade; desajeitado, estabanado. (\textit{Meu irmão sempre foi desastrado para lavar louça.})}{de.sas.tra.do}{0}
\verb{desastrado}{}{}{}{}{}{Que resultou em desastre; catastrófico, infeliz.}{de.sas.tra.do}{0}
\verb{desastre}{}{}{}{}{s.m.}{Acidente que causa grandes perdas; calamidade.}{de.sas.tre}{0}
\verb{desastre}{}{}{}{}{}{Desgraça, fatalidade.}{de.sas.tre}{0}
\verb{desastre}{}{}{}{}{}{Fracasso, insucesso, fiasco.}{de.sas.tre}{0}
\verb{desastroso}{ô}{}{"-osos ⟨ó⟩}{"-osa ⟨ó⟩}{adj.}{Que produz desastre; ruinoso.}{de.sas.tro.so}{0}
\verb{desatar}{}{}{}{}{v.t.}{Desfazer um nó; desmanchar, desprender.}{de.sa.tar}{\verboinum{1}}
\verb{desatarraxar}{ch}{}{}{}{v.t.}{Desprender tarraxas ou parafusos; desparafusar.}{de.sa.tar.ra.xar}{0}
\verb{desatarraxar}{ch}{}{}{}{}{Abrir alguma coisa, girando a tampa.}{de.sa.tar.ra.xar}{\verboinum{1}}
\verb{desataviado}{}{}{}{}{adj.}{Sem atavios, enfeites; singelo, simples.}{de.sa.ta.vi.a.do}{0}
\verb{desataviar}{}{}{}{}{v.t.}{Tirar os atavios; desenfeitar.}{de.sa.ta.vi.ar}{\verboinum{1}}
\verb{desatenção}{}{}{"-ões}{}{s.f.}{Falta de atenção; desconcentração, distração.}{de.sa.ten.ção}{0}
\verb{desatenção}{}{}{"-ões}{}{}{Falta de cortesia; descaso, desrespeito.}{de.sa.ten.ção}{0}
\verb{desatencioso}{ô}{}{"-osos ⟨ó⟩}{"-osa ⟨ó⟩}{adj.}{Que não presta atenção; desconcentrado, distraído.}{de.sa.ten.ci.o.so}{0}
\verb{desatencioso}{ô}{}{"-osos ⟨ó⟩}{"-osa ⟨ó⟩}{}{Descortês, indelicado.}{de.sa.ten.ci.o.so}{0}
\verb{desatender}{ê}{}{}{}{v.t.}{Não dar atenção; desconsiderar, ignorar.}{de.sa.ten.der}{\verboinum{12}}
\verb{desatento}{}{}{}{}{adj.}{Que não presta atenção; distraído.}{de.sa.ten.to}{0}
\verb{desaterrar}{}{}{}{}{v.t.}{Retirar terra ou entulho de um aterro.}{de.sa.ter.rar}{\verboinum{1}}
\verb{desatinado}{}{}{}{}{adj.}{Que não tem tino, juízo; fora de si; desvairado, enlouquecido.}{de.sa.ti.na.do}{0}
\verb{desatinar}{}{}{}{}{v.t.}{Fazer perder o juízo; enlouquecer, endoidecer.}{de.sa.ti.nar}{\verboinum{1}}
\verb{desatino}{}{}{}{}{s.m.}{Falta de juízo; desvario, loucura.}{de.sa.ti.no}{0}
\verb{desatino}{}{}{}{}{}{Ato ou palavras próprias de quem perdeu o juízo; contrassenso, disparate. (\textit{No estado em que ele está, é capaz de cometer um desatino.})}{de.sa.ti.no}{0}
\verb{desativar}{}{}{}{}{v.t.}{Tornar inativo; fazer parar de funcionar.}{de.sa.ti.var}{\verboinum{1}}
\verb{desatolar}{}{}{}{}{v.t.}{Tirar do atoleiro.}{de.sa.to.lar}{\verboinum{1}}
\verb{desatracar}{}{}{}{}{v.t.}{Desamarrar o que prende uma embarcação para que ela possa navegar.}{de.sa.tra.car}{\verboinum{2}}
\verb{desatravancar}{}{}{}{}{v.t.}{Remover o que impede a passagem; desobstruir, desimpedir.}{de.sa.tra.van.car}{\verboinum{2}}
\verb{desatrelar}{}{}{}{}{v.t.}{Soltar a trela; desengatar, desprender.}{de.sa.tre.lar}{\verboinum{1}}
\verb{desatualizar}{}{}{}{}{v.t.}{Fazer perder a atualidade; tornar ultrapassado.}{de.sa.tu.a.li.zar}{\verboinum{1}}
\verb{desautorar}{}{}{}{}{v.t.}{Tirar as honras ou a dignidade de alguém; rebaixar.}{de.sau.to.rar}{\verboinum{1}}
\verb{desautorização}{}{}{"-ões}{}{s.f.}{Ato ou efeito de desautorizar; descrédito, desprestígio.}{de.sau.to.ri.za.ção}{0}
\verb{desautorizar}{}{}{}{}{v.t.}{Tirar a autoridade; desprestigiar, desacreditar.}{de.sau.to.ri.zar}{\verboinum{1}}
\verb{desavença}{}{}{}{}{s.f.}{Falta de concordância; discórdia, dissensão.}{de.sa.ven.ça}{0}
\verb{desavergonhado}{}{}{}{}{adj.}{Que não tem vergonha; descarado, despudorado; insolente.}{de.sa.ver.go.nha.do}{0}
\verb{desavindo}{}{}{}{}{adj.}{Que anda em desavença; brigado, indisposto.}{de.sa.vin.do}{0}
\verb{desavir}{}{}{}{}{v.t.}{Pôr em desavença; fazer brigar; indispor.}{de.sa.vir}{\verboinum{56}}
\verb{desavisado}{}{}{}{}{adj.}{Que não tem cuidado; imprudente, leviano.}{de.sa.vi.sa.do}{0}
\verb{desaviso}{}{}{}{}{s.f.}{Imprudência, leviandade.}{de.sa.vi.so}{0}
\verb{desazado}{}{}{}{}{adj.}{Inapto, descuidado, desmazelado.}{de.sa.za.do}{0}
\verb{desazado}{}{}{}{}{}{Descabido, impróprio.}{de.sa.za.do}{0}
\verb{desazo}{}{}{}{}{s.m.}{Inaptidão, descuido, desmazelo.}{de.sa.zo}{0}
\verb{desbancar}{}{}{}{}{v.t.}{Superar, exceder, suplantar, vencer.}{des.ban.car}{0}
\verb{desbancar}{}{}{}{}{}{Ganhar o dinheiro da banca.}{des.ban.car}{\verboinum{2}}
\verb{desbaratamento}{}{}{}{}{s.m.}{Ato ou efeito de desbaratar.}{des.ba.ra.ta.men.to}{0}
\verb{desbaratar}{}{}{}{}{v.t.}{Vencer, derrotar, destroçar.}{des.ba.ra.tar}{0}
\verb{desbaratar}{}{}{}{}{}{Esbanjar, desperdiçar, malgastar.}{des.ba.ra.tar}{\verboinum{1}}
\verb{desbarato}{}{}{}{}{s.m.}{Desbaratamento.}{des.ba.ra.to}{0}
\verb{desbarrancado}{}{Bras.}{}{}{s.m.}{Precipício, abismo, despenhadeiro.}{des.bar.ran.ca.do}{0}
\verb{desbarrancar}{}{}{}{}{v.i.}{Escorrer morro abaixo por efeito de erosão.}{des.bar.ran.car}{0}
\verb{desbarrancar}{}{}{}{}{}{Escavar profundamente; desaterrar.}{des.bar.ran.car}{\verboinum{2}}
\verb{desbastar}{}{}{}{}{v.t.}{Tornar menos basto; tornar mais ralo; afinar.}{des.bas.tar}{0}
\verb{desbastar}{}{}{}{}{}{Polir, aperfeiçoar.}{des.bas.tar}{\verboinum{1}}
\verb{desbaste}{}{}{}{}{s.m.}{Ato ou efeito de desbastar.}{des.bas.te}{0}
\verb{desbeiçar}{}{}{}{}{v.t.}{Cortar o beiço.}{des.bei.çar}{0}
\verb{desbeiçar}{}{}{}{}{}{Cortar ou quebrar as bordas de xícaras, vasilhas etc.}{des.bei.çar}{\verboinum{3}}
\verb{desbloquear}{}{}{}{}{v.t.}{Desfazer o bloqueio.}{des.blo.que.ar}{0}
\verb{desbloquear}{}{}{}{}{}{Destravar, desimpedir.}{des.blo.que.ar}{\verboinum{4}}
\verb{desbloqueio}{ê}{}{}{}{s.m.}{Ato ou efeito de desbloquear.}{des.blo.quei.o}{0}
\verb{desbocado}{}{}{}{}{adj.}{Que usa linguagem obscena, desapropriada ou inconveniente.}{des.bo.ca.do}{0}
\verb{desbotado}{}{}{}{}{adj.}{Diz"-se de cor que perdeu a vivacidade ou a tonalidade original; pálido, esmaecido.}{des.bo.ta.do}{0}
\verb{desbotado}{}{}{}{}{}{Diz"-se daquilo que perdeu a cor original.}{des.bo.ta.do}{0}
\verb{desbotamento}{}{}{}{}{s.m.}{Ato ou efeito de desbotar.}{des.bo.ta.men.to}{0}
\verb{desbotar}{}{}{}{}{v.t.}{Fazer perder a vivacidade da cor ou o brilho.}{des.bo.tar}{0}
\verb{desbotar}{}{}{}{}{v.i.}{Perder a viveza da cor.}{des.bo.tar}{\verboinum{1}}
\verb{desbragado}{}{}{}{}{adj.}{Diz"-se de indivíduo impudico, indecoroso, libertino, desregrado.}{des.bra.ga.do}{0}
\verb{desbragamento}{}{}{}{}{s.m.}{Qualidade de desbragado.}{des.bra.ga.men.to}{0}
\verb{desbragar}{}{}{}{}{v.t.}{Tornar impudico, libertino, desregrado.}{des.bra.gar}{\verboinum{5}}
\verb{desbravar}{}{}{}{}{v.t.}{Preparar terreno para cultura.}{des.bra.var}{0}
\verb{desbravar}{}{}{}{}{}{Explorar lugares ou terras desconhecidas.}{des.bra.var}{0}
\verb{desbravar}{}{}{}{}{}{Tornar manso; amansar.}{des.bra.var}{\verboinum{1}}
\verb{desbriado}{}{}{}{}{adj.}{Sem brio; desavergonhado.}{des.bri.a.do}{0}
\verb{desbriar}{}{}{}{}{v.t.}{Tirar o brio.}{des.bri.ar}{\verboinum{1}}
\verb{desbrio}{}{}{}{}{s.m.}{Falta de brio, vergonha.}{des.bri.o}{0}
\verb{desbundar}{}{}{}{}{v.t.}{Causar impacto, admiração; deslumbrar.}{des.bun.dar}{0}
\verb{desbundar}{}{}{}{}{}{Perder o autocontrole; descomedir"-se.}{des.bun.dar}{\verboinum{1}}
\verb{desbunde}{}{}{}{}{s.m.}{Ato ou efeito de desbundar.}{des.bun.de}{0}
\verb{desburocratizar}{}{}{}{}{v.t.}{Fazer perder o caráter ou os hábitos burocráticos.}{des.bu.ro.cra.ti.zar}{\verboinum{1}}
\verb{descabeçado}{}{Bras.}{}{}{adj.}{Diz"-se de indivíduo sem juízo; maluco.}{des.ca.be.ça.do}{0}
\verb{descabelado}{}{}{}{}{adj.}{Despenteado, desgrenhado.}{des.ca.be.la.do}{0}
\verb{descabelado}{}{}{}{}{}{Cujo cabelo foi arrancado.}{des.ca.be.la.do}{0}
\verb{descabelar}{}{}{}{}{v.t.}{Desgrenhar, desalinhar os cabelos.}{des.ca.be.lar}{0}
\verb{descabelar}{}{}{}{}{}{Arrancar os cabelos.}{des.ca.be.lar}{0}
\verb{descabelar}{}{Fig.}{}{}{v.pron.}{Desesperar"-se.}{des.ca.be.lar}{\verboinum{1}}
\verb{descabido}{}{}{}{}{adj.}{Sem cabimento; impróprio, inconveniente.}{des.ca.bi.do}{0}
\verb{descadastrar}{}{}{}{}{v.t.}{Retirar de um cadastro.}{des.ca.das.trar}{\verboinum{1}}
\verb{descadeirado}{}{}{}{}{adj.}{Que tem dor nas cadeiras.}{des.ca.dei.ra.do}{0}
\verb{descadeirar}{}{}{}{}{v.t.}{Provocar dor nas cadeiras.}{des.ca.dei.rar}{0}
\verb{descadeirar}{}{}{}{}{v.pron.}{Requebrar as cadeiras; rebolar.}{des.ca.dei.rar}{\verboinum{1}}
\verb{descafeinado}{}{}{}{}{adj.}{Do qual se retirou a cafeína.}{des.ca.fe.i.na.do}{0}
\verb{descaída}{}{}{}{}{s.f.}{Ato ou efeito de descair.}{des.ca.í.da}{0}
\verb{descaída}{}{}{}{}{}{Erro, lapso, descuido.}{des.ca.í.da}{0}
\verb{descaimento}{}{}{}{}{s.m.}{Ato ou efeito de descair.}{des.ca.i.men.to}{0}
\verb{descaimento}{}{Fig.}{}{}{}{Abatimento, prostração.}{des.ca.i.men.to}{0}
\verb{descaimento}{}{Fig.}{}{}{}{Decadência, degeneração, declinação.}{des.ca.i.men.to}{0}
\verb{descair}{}{}{}{}{v.t.}{Deixar cair ou pender.}{des.ca.ir}{0}
\verb{descair}{}{}{}{}{v.i.}{Inclinar"-se lentamente.}{des.ca.ir}{0}
\verb{descair}{}{}{}{}{}{Sofrer diminuição; decair.}{des.ca.ir}{\verboinum{19}}
\verb{descalabro}{}{}{}{}{s.m.}{Dano, perda, ruína, desgraça.}{des.ca.la.bro}{0}
\verb{descalabro}{}{}{}{}{}{Escândalo, vergonha.}{des.ca.la.bro}{0}
\verb{descalçadela}{é}{}{}{}{s.f.}{Repreensão, descompostura.}{des.cal.ça.de.la}{0}
\verb{descalçar}{}{}{}{}{v.t.}{Tirar calça, calçado ou luva.}{des.cal.çar}{0}
\verb{descalçar}{}{}{}{}{}{Tirar o calço ou apoio.}{des.cal.çar}{0}
\verb{descalçar}{}{Fig.}{}{}{}{Retirar o auxílio; desamparar.}{des.cal.çar}{\verboinum{3}}
\verb{descalço}{}{}{}{}{adj.}{Com os pés nus ou só de meias.}{des.cal.ço}{0}
\verb{descalço}{}{Pop.}{}{}{}{Desprevenido financeiramente.}{des.cal.ço}{0}
\verb{descalibrado}{}{}{}{}{adj.}{Que está fora do calibre, regulagem ou especificação correta.}{des.ca.li.bra.do}{0}
\verb{descalibrar}{}{}{}{}{v.t.}{Retirar do calibre, regulagem ou especificação.}{des.ca.li.brar}{\verboinum{1}}
\verb{descalvado}{}{}{}{}{adj.}{Sem cabelos; calvo, careca, escalvado.}{des.cal.va.do}{0}
\verb{descalvado}{}{Fig.}{}{}{}{Sem vegetação; árido, calvo, escalvado, estéril.}{des.cal.va.do}{0}
\verb{descalvar}{}{}{}{}{v.t.}{Tornar careca.}{des.cal.var}{0}
\verb{descalvar}{}{Fig.}{}{}{}{Tirar a vegetação, tornar estéril.}{des.cal.var}{\verboinum{1}}
\verb{descamação}{}{}{"-ões}{}{s.f.}{Ato ou efeito de descamar.}{des.ca.ma.ção}{0}
\verb{descamar}{}{}{}{}{v.t.}{Tirar as escamas.}{des.ca.mar}{0}
\verb{descamar}{}{}{}{}{v.i.}{Perder as escamas.}{des.ca.mar}{\verboinum{1}}
\verb{descambar}{}{}{}{}{v.i.}{Desabar, despencar.}{des.cam.bar}{0}
\verb{descambar}{}{}{}{}{}{Descer, declinar, descair.}{des.cam.bar}{\verboinum{1}}
\verb{descaminhar}{}{}{}{}{v.t.}{Desencaminhar.}{des.ca.mi.nhar}{\verboinum{1}}
\verb{descaminho}{}{}{}{}{s.m.}{Ato ou efeito de descaminhar; sumiço.}{des.ca.mi.nho}{0}
\verb{descamisado}{}{}{}{}{adj.}{Que não tem camisa.}{des.ca.mi.sa.do}{0}
\verb{descamisado}{}{}{}{}{}{Roto, esfarrapado.}{des.ca.mi.sa.do}{0}
\verb{descamisar}{}{}{}{}{v.t.}{Tirar a camisa.}{des.ca.mi.sar}{\verboinum{1}}
\verb{descampado}{}{}{}{}{adj.}{Desabitado, desabrigado.}{des.cam.pa.do}{0}
\verb{descampado}{}{}{}{}{s.m.}{Área aberta, desabitada e sem cultura.}{des.cam.pa.do}{0}
\verb{descansado}{}{}{}{}{adj.}{Não cansado; despreocupado, tranquilo, sossegado.}{des.can.sa.do}{0}
\verb{descansado}{}{}{}{}{}{Lento, pausado, vagaroso.}{des.can.sa.do}{0}
\verb{descansar}{}{}{}{}{v.t.}{Dar descanso.}{des.can.sar}{0}
\verb{descansar}{}{}{}{}{v.i.}{Repousar, tomar descanso.}{des.can.sar}{0}
\verb{descansar}{}{}{}{}{}{Livrar"-se de aflições ou preocupações.}{des.can.sar}{0}
\verb{descansar}{}{}{}{}{}{Morrer.}{des.can.sar}{\verboinum{1}}
\verb{descanso}{}{}{}{}{s.m.}{Repouso, folga, ócio, sossego.}{des.can.so}{0}
\verb{descanso}{}{}{}{}{}{Alívio, consolo.}{des.can.so}{0}
\verb{descanso}{}{}{}{}{}{Sono.}{des.can.so}{0}
\verb{descanso}{}{}{}{}{}{Objeto sobre o qual outro se apoia.}{des.can.so}{0}
\verb{descapitalização}{}{Econ.}{"-ões}{}{s.f.}{Ato ou efeito de descapitalizar.}{des.ca.pi.ta.li.za.ção}{0}
\verb{descapitalizar}{}{Econ.}{}{}{v.t.}{Perder ou gastar capital ou bem de valor.}{des.ca.pi.ta.li.zar}{\verboinum{1}}
\verb{descaracterização}{}{}{"-ões}{}{s.f.}{Ato ou efeito de descaracterizar.}{des.ca.rac.te.ri.za.ção}{0}
\verb{descaracterizar}{}{}{}{}{v.t.}{Tirar ou perder uma ou mais qualidades que distinguem algo ou alguém dos demais.}{des.ca.rac.te.ri.zar}{\verboinum{1}}
\verb{descarado}{}{}{}{}{adj.}{Que indica ausência de constrangimento ou de preocupação em disfarçar algo que pode ser considerado negativo; cínico, atrevido, imprudente.}{des.ca.ra.do}{0}
\verb{descaramento}{}{}{}{}{s.m.}{Qualidade de descarado; falta de vergonha.}{des.ca.ra.men.to}{0}
\verb{descaraterizar}{}{}{}{}{v.t.}{Descaracterizar.}{des.ca.ra.te.ri.zar}{\verboinum{1}}
\verb{descarga}{}{}{}{}{s.f.}{Ato ou efeito de descarregar.}{des.car.ga}{0}
\verb{descarga}{}{}{}{}{}{Retirada da carga.}{des.car.ga}{0}
\verb{descarga}{}{}{}{}{}{Tiro ou sequência de tiros de arma de fogo.}{des.car.ga}{0}
\verb{descarga}{}{}{}{}{}{Em motores a explosão, escapamento dos gases resultantes da combustão.}{des.car.ga}{0}
\verb{descarga}{}{Fís.}{}{}{}{Condução de eletricidade através de um corpo.}{des.car.ga}{0}
\verb{descarga}{}{}{}{}{}{Sistema hidráulico que despeja água em um vaso sanitário para limpá"-lo.}{des.car.ga}{0}
\verb{descargo}{}{}{}{}{s.m.}{Desencargo.}{des.car.go}{0}
\verb{descarnado}{}{}{}{}{adj.}{Que tem pouca carne; muito magro.}{des.car.na.do}{0}
\verb{descarnar}{}{}{}{}{v.t.}{Separar a carne dos ossos.}{des.car.nar}{0}
\verb{descarnar}{}{}{}{}{v.i.}{Ficar muito magro; emagrecer em demasia.}{des.car.nar}{\verboinum{1}}
\verb{descaro}{}{}{}{}{s.m.}{Descaramento.}{des.ca.ro}{0}
\verb{descaroçador}{ô}{}{}{}{s.m.}{Dispositivo para retirar o caroço de alimentos, especialmente frutos.}{des.ca.ro.ça.dor}{0}
\verb{descaroçar}{}{}{}{}{v.t.}{Retirar o(s) caroço(s).}{des.ca.ro.çar}{\verboinum{1}}
\verb{descarregamento}{}{}{}{}{s.m.}{Retirada da carga; descarga.}{des.car.re.ga.men.to}{0}
\verb{descarregar}{}{}{}{}{v.t.}{Tirar a carga.}{des.car.re.gar}{0}
\verb{descarregar}{}{}{}{}{}{Retirar a munição de arma de fogo.}{des.car.re.gar}{0}
\verb{descarregar}{}{}{}{}{}{Disparar uma arma de fogo.}{des.car.re.gar}{0}
\verb{descarregar}{}{Fig.}{}{}{}{Tranquilizar, aliviar, sossegar.}{des.car.re.gar}{0}
\verb{descarregar}{}{}{}{}{}{Retirar carga elétrica.}{des.car.re.gar}{\verboinum{5}}
\verb{descarrilamento}{}{}{}{}{s.m.}{Ato ou efeito de descarrilar.}{des.car.ri.la.men.to}{0}
\verb{descarrilar}{}{}{}{}{v.t.}{Tirar ou sair dos trilhos.}{des.car.ri.lar}{\verboinum{1}}
\verb{descarrilhamento}{}{}{}{}{s.m.}{Ato ou efeito de descarrilhar; descarrilamento.}{des.car.ri.lha.men.to}{0}
\verb{descarrilhar}{}{}{}{}{v.t.}{Descarrilar.}{des.car.ri.lhar}{\verboinum{1}}
\verb{descartar}{}{}{}{}{v.t.}{Rejeitar carta do baralho.}{des.car.tar}{0}
\verb{descartar}{}{}{}{}{}{Retirar da mão uma ou mais cartas do baralho de maneira compulsória.}{des.car.tar}{0}
\verb{descartar}{}{}{}{}{}{Não levar em conta; ignorar, afastar.}{des.car.tar}{0}
\verb{descartar}{}{}{}{}{}{Jogar fora após o uso.}{des.car.tar}{\verboinum{1}}
\verb{descartável}{}{}{"-eis}{}{adj.2g.}{Que se pode ou se deve descartar.}{des.car.tá.vel}{0}
\verb{descartável}{}{}{"-eis}{}{}{Diz"-se de materiais, especialmente embalagens, que têm vida útil curta ou que não devem ser reaproveitados após sua utilização.}{des.car.tá.vel}{0}
\verb{descarte}{}{}{}{}{s.m.}{Ato ou efeito de descartar.}{des.car.te}{0}
\verb{descarte}{}{}{}{}{}{Conjunto de cartas do baralho que foram descartadas.}{des.car.te}{0}
\verb{descasado}{}{}{}{}{adj.}{Que se desligou do cônjuge; separado.}{des.ca.sa.do}{0}
\verb{descasado}{}{}{}{}{}{Que se afastou de seus pares ou semelhantes; desemparelhado.}{des.ca.sa.do}{0}
\verb{descasar}{}{}{}{}{v.t.}{Anular o contrato de casamento.}{des.ca.sar}{0}
\verb{descasar}{}{}{}{}{}{Separar pessoas casadas ou animais acasalados.}{des.ca.sar}{0}
\verb{descasar}{}{}{}{}{}{Desemparelhar.}{des.ca.sar}{\verboinum{1}}
\verb{descascador}{ô}{}{}{}{s.m.}{Máquina para descascar cereais.}{des.cas.ca.dor}{0}
\verb{descascamento}{}{}{}{}{s.m.}{Ato ou efeito de descascar.}{des.cas.ca.men.to}{0}
\verb{descascar}{}{}{}{}{v.t.}{Tirar ou perder a casca.}{des.cas.car}{0}
\verb{descascar}{}{Bras.}{}{}{}{Repreender, censurar.}{des.cas.car}{\verboinum{2}}
\verb{descaso}{}{}{}{}{s.m.}{Falta de atenção; desprezo, inadvertência.}{des.ca.so}{0}
\verb{descavalgar}{}{}{}{}{v.t.}{Descer de montaria; apear, desmontar.}{des.ca.val.gar}{\verboinum{1}}
\verb{descendência}{}{}{}{}{s.f.}{Conjunto das pessoas que descendem de um progenitor comum.}{des.cen.dên.cia}{0}
\verb{descendente}{}{}{}{}{s.2g.}{Indivíduo que descende de outro.}{des.cen.den.te}{0}
\verb{descendente}{}{}{}{}{adj.2g.}{Que desce.}{des.cen.den.te}{0}
\verb{descender}{ê}{}{}{}{v.t.}{Originar"-se, provir por geração.}{des.cen.der}{\verboinum{12}}
\verb{descenso}{}{}{}{}{s.m.}{Ato de descer; descida.}{des.cen.so}{0}
\verb{descentralização}{}{}{"-ões}{}{s.f.}{Ato ou efeito de descentralizar.}{des.cen.tra.li.za.ção}{0}
\verb{descentralizar}{}{}{}{}{v.t.}{Afastar ou retirar do centro.}{des.cen.tra.li.zar}{0}
\verb{descentralizar}{}{}{}{}{}{Delegar atribuições com autonomia administrativa.}{des.cen.tra.li.zar}{\verboinum{1}}
\verb{descentrar}{}{}{}{}{v.t.}{Afastar do centro geométrico.}{des.cen.trar}{\verboinum{1}}
\verb{descer}{ê}{}{}{}{v.t.}{Mover do alto para baixo.}{des.cer}{0}
\verb{descer}{ê}{}{}{}{}{Abaixar.}{des.cer}{0}
\verb{descer}{ê}{}{}{}{}{Apear de montaria.}{des.cer}{0}
\verb{descer}{ê}{}{}{}{}{Sair de um lugar alto.}{des.cer}{0}
\verb{descer}{ê}{}{}{}{}{Diminuir, baixar.}{des.cer}{0}
\verb{descer}{ê}{}{}{}{v.i.}{Vir do alto para baixo.}{des.cer}{\verboinum{15}}
\verb{descerrar}{}{}{}{}{v.t.}{Abrir.}{des.cer.rar}{0}
\verb{descerrar}{}{}{}{}{}{Descobrir, revelar, divulgar.}{des.cer.rar}{\verboinum{1}}
\verb{descida}{}{}{}{}{s.f.}{Ato de descer.}{des.ci.da}{0}
\verb{descida}{}{}{}{}{}{Trecho inclinado de terreno que se percorre de cima para baixo.}{des.ci.da}{0}
\verb{desclassificação}{}{}{"-ões}{}{s.f.}{Ato ou efeito de desclassificar.}{des.clas.si.fi.ca.ção}{0}
\verb{desclassificado}{}{}{}{}{adj.}{Que não obteve classificação ou sofreu desclassificação.}{des.clas.si.fi.ca.do}{0}
\verb{desclassificado}{}{}{}{}{}{Diz"-se de indivíduo que tem conceito negativo perante os outros; desacreditado.}{des.clas.si.fi.ca.do}{0}
\verb{desclassificar}{}{}{}{}{v.t.}{Retirar de uma classe ou categoria de classificação.}{des.clas.si.fi.car}{0}
\verb{desclassificar}{}{}{}{}{}{Eliminar concorrente em competição ou concurso.}{des.clas.si.fi.car}{\verboinum{2}}
\verb{descoberta}{é}{}{}{}{s.f.}{Ato ou efeito de descobrir.}{des.co.ber.ta}{0}
\verb{descoberta}{é}{}{}{}{}{Aquilo que se descobriu por acaso ou como resultado de pesquisa; invenção.}{des.co.ber.ta}{0}
\verb{descoberto}{é}{}{}{}{adj.}{Que não está coberto; nu, evidente.}{des.co.ber.to}{0}
\verb{descobridor}{ô}{}{}{}{s.m.}{Indivíduo que faz descobertas ou fez determinada descoberta.}{des.co.bri.dor}{0}
\verb{descobrimento}{}{}{}{}{s.m.}{Ato ou efeito de descobrir.}{des.co.bri.men.to}{0}
\verb{descobrir}{}{}{}{}{v.t.}{Tirar tampa ou cobertura; destapar.}{des.co.brir}{0}
\verb{descobrir}{}{}{}{}{}{Achar, encontrar.}{des.co.brir}{0}
\verb{descobrir}{}{}{}{}{}{Inventar, criar.}{des.co.brir}{0}
\verb{descobrir}{}{}{}{}{}{Revelar, manifestar.}{des.co.brir}{0}
\verb{descobrir}{}{}{}{}{v.pron.}{Tirar chapéu ou qualquer tipo de cobertura da cabeça.}{des.co.brir}{\verboinum{31}}
\verb{descoco}{ô}{}{}{}{s.m.}{Insolência, atrevimento, disparate.}{des.co.co}{0}
\verb{descodificação}{}{}{"-ões}{}{s.f.}{Decodificação.}{des.co.di.fi.ca.ção}{0}
\verb{descodificador}{ô}{}{}{}{s.m.}{Decodificador.}{des.co.di.fi.ca.dor}{0}
\verb{descodificar}{}{}{}{}{v.t.}{Decodificar.}{des.co.di.fi.car}{\verboinum{1}}
\verb{descolar}{}{}{}{}{v.t.}{Separar aquilo que estava colado; despegar.}{des.co.lar}{0}
\verb{descolar}{}{Pop.}{}{}{}{Obter, conseguir, arranjar.}{des.co.lar}{\verboinum{1}}
\verb{descoloração}{}{}{"-ões}{}{s.f.}{Ato ou efeito de descolorar, descolorir.}{des.co.lo.ra.ção}{0}
\verb{descolorar}{}{}{}{}{v.t.}{Tirar ou perder a cor; descorar.}{des.co.lo.rar}{\verboinum{1}}
\verb{descolorir}{}{}{}{}{v.t.}{Descolorar, descorar.}{des.co.lo.rir}{0}
\verb{descolorir}{}{}{}{}{}{Tirar a expressividade; empobrecer.}{des.co.lo.rir}{\verboinum{34}}
\verb{descomedido}{}{}{}{}{adj.}{Sem comedimento; inconveniente, imprudente.}{des.co.me.di.do}{0}
\verb{descomedimento}{}{}{}{}{s.m.}{Falta de comedimento; inconveniência, imprudência.}{des.co.me.di.men.to}{0}
\verb{descomedir}{}{}{}{}{v.pron.}{Passar dos limites; mostrar"-se inconveniente; exceder"-se.}{des.co.me.dir"-se}{\verboinum{20}}
\verb{descomer}{ê}{Pop.}{}{}{v.t.}{Defecar, expelir.}{des.co.mer}{\verboinum{12}}
\verb{descompassado}{}{}{}{}{adj.}{Sem regularidade, medida.}{des.com.pas.sa.do}{0}
\verb{descompassado}{}{}{}{}{}{Inconveniente, descomedido.}{des.com.pas.sa.do}{0}
\verb{descompassar}{}{}{}{}{v.t.}{Fazer sair do compasso, da medida.}{des.com.pas.sar}{0}
\verb{descompassar}{}{}{}{}{}{Tornar inconveniente, descomedido.}{des.com.pas.sar}{\verboinum{1}}
\verb{descompasso}{}{}{}{}{s.m.}{Falta de medida, de regularidade.}{des.com.pas.so}{0}
\verb{descompasso}{}{}{}{}{}{Desacordo, divergência, descomedimento.}{des.com.pas.so}{0}
\verb{descomplicar}{}{}{}{}{v.t.}{Desfazer um problema ou uma complicação; simplificar.}{des.com.pli.car}{\verboinum{1}}
\verb{descompor}{}{}{}{}{v.t.}{Desordenar, desarranjar.}{des.com.por}{0}
\verb{descompor}{}{}{}{}{}{Decompor, desfazer.}{des.com.por}{0}
\verb{descompor}{}{}{}{}{}{Despir.}{des.com.por}{0}
\verb{descompor}{}{}{}{}{v.pron.}{Destratar, censurar.}{des.com.por}{\verboinum{60}}
\verb{descompostura}{}{}{}{}{s.f.}{Ato ou efeito de descompor.}{des.com.pos.tu.ra}{0}
\verb{descompostura}{}{}{}{}{}{Falta de compostura; desalinho, desarranjo.}{des.com.pos.tu.ra}{0}
\verb{descompostura}{}{}{}{}{}{Reprimenda, censura, injúria.}{des.com.pos.tu.ra}{0}
\verb{descomunal}{}{}{"-ais}{}{adj.2g.}{Fora do comum; extraordinário, imenso.}{des.co.mu.nal}{0}
\verb{desconceituar}{}{}{}{}{v.t.}{Tirar ou perder o bom conceito, a reputação; desacreditar, desclassificar.}{des.con.cei.tu.ar}{\verboinum{1}}
\verb{desconcentrar}{}{}{}{}{v.t.}{Tirar de um estado de concentração.}{des.con.cen.trar}{\verboinum{1}}
\verb{desconcertado}{}{}{}{}{adj.}{Descomposto, desregrado, descomedido.}{des.con.cer.ta.do}{0}
\verb{desconcertado}{}{}{}{}{}{Embaraçado, perturbado, atrapalhado.}{des.con.cer.ta.do}{0}
\verb{desconcertante}{}{}{}{}{adj.2g.}{Que desconcerta, desorienta.}{des.con.cer.tan.te}{0}
\verb{desconcertar}{}{}{}{}{v.t.}{Fazer perder a ordem; desarranjar.}{des.con.cer.tar}{0}
\verb{desconcertar}{}{}{}{}{v.pron.}{Perturbar"-se, embaraçar"-se.}{des.con.cer.tar}{\verboinum{1}}
\verb{desconcerto}{ê}{}{}{}{s.m.}{Ato ou efeito de desconcertar.}{des.con.cer.to}{0}
\verb{desconcerto}{ê}{}{}{}{}{Desordem, desarmonia, transtorno.}{des.con.cer.to}{0}
\verb{desconchavar}{}{}{}{}{v.t.}{Desligar, descombinar.}{des.con.cha.var}{0}
\verb{desconchavar}{}{}{}{}{v.i.}{Fazer ou dizer disparates, despropósitos.}{des.con.cha.var}{\verboinum{1}}
\verb{desconchavo}{}{}{}{}{s.m.}{Ato de desconchavar; tolice, disparate.}{des.con.cha.vo}{0}
\verb{desconexão}{cs}{}{"-ões}{}{s.f.}{Ausência de conexão.}{des.co.ne.xão}{0}
\verb{desconexo}{écs}{}{}{}{adj.}{Sem conexão; desunido, solto, incoerente.}{des.co.ne.xo}{0}
\verb{desconfiado}{}{}{}{}{adj.}{Que não confia.}{des.con.fi.a.do}{0}
\verb{desconfiado}{}{}{}{}{}{Que se melindra com facilidade.}{des.con.fi.a.do}{0}
\verb{desconfiança}{}{}{}{}{s.f.}{Qualidade de desconfiado.}{des.con.fi.an.ça}{0}
\verb{desconfiança}{}{}{}{}{}{Ausência de confiança.}{des.con.fi.an.ça}{0}
\verb{desconfiar}{}{}{}{}{v.t.}{Ter uma determinada suposição; supor, conjeturar.}{des.con.fi.ar}{0}
\verb{desconfiar}{}{}{}{}{}{Não ter confiança; suspeitar, duvidar.}{des.con.fi.ar}{0}
\verb{desconfiar}{}{}{}{}{}{Melindrar"-se, zangar"-se.}{des.con.fi.ar}{\verboinum{1}}
\verb{desconfiômetro}{}{Pop.}{}{}{s.m.}{Capacidade de perceber quando se está sendo inconveniente.}{des.con.fi.ô.me.tro}{0}
\verb{desconforme}{ó}{}{}{}{adj.}{Que não está em conformidade; discordante.}{des.con.for.me}{0}
\verb{desconforme}{ó}{}{}{}{}{Desproporcionado, colossal, descomunal.}{des.con.for.me}{0}
\verb{desconfortável}{}{}{"-eis}{}{adj.2g.}{Que não oferece conforto.}{des.con.for.tá.vel}{0}
\verb{desconfortável}{}{Fig.}{"-eis}{}{}{Constrangedor, embaraçoso.}{des.con.for.tá.vel}{0}
\verb{desconforto}{ô}{}{}{}{s.m.}{Ausência de conforto.}{des.con.for.to}{0}
\verb{desconforto}{ô}{}{}{}{}{Desalento, aflição.}{des.con.for.to}{0}
\verb{descongelamento}{}{}{}{}{s.m.}{Ato ou efeito de descongelar; degelo.}{des.con.ge.la.men.to}{0}
\verb{descongelar}{}{}{}{}{v.t.}{Tornar líquido aquilo que estava congelado; derreter, degelar.}{des.con.ge.lar}{0}
\verb{descongelar}{}{}{}{}{}{Fazer subir a temperatura de alimentos conservados em temperaturas abaixo de zero e torná"-los próprios para o uso ou o consumo.}{des.con.ge.lar}{\verboinum{1}}
\verb{descongestionamento}{}{}{}{}{s.m.}{Ato ou efeito de descongestionar.}{des.con.ges.ti.o.na.men.to}{0}
\verb{descongestionante}{}{}{}{}{adj.2g.}{Que tem a propriedade de descongestionar.}{des.con.ges.ti.o.nan.te}{0}
\verb{descongestionar}{}{}{}{}{v.t.}{Eliminar a congestão.}{des.con.ges.ti.o.nar}{0}
\verb{descongestionar}{}{}{}{}{}{Desinchar, desintumescer.}{des.con.ges.ti.o.nar}{0}
\verb{descongestionar}{}{}{}{}{}{Desobstruir, desacumular.}{des.con.ges.ti.o.nar}{0}
\verb{descongestionar}{}{}{}{}{}{Restabelecer a fluidez do tráfego de veículos.}{des.con.ges.ti.o.nar}{\verboinum{1}}
\verb{desconhecer}{ê}{}{}{}{v.t.}{Ignorar, não ter informações a respeito.}{des.co.nhe.cer}{\verboinum{15}}
\verb{desconhecido}{}{}{}{}{}{Diz"-se de pessoa que não tem relações sociais.}{des.co.nhe.ci.do}{0}
\verb{desconhecido}{}{}{}{}{adj.}{Ignorado, incógnito.}{des.co.nhe.ci.do}{0}
\verb{desconhecimento}{}{}{}{}{s.m.}{Ausência de conhecimento.}{des.co.nhe.ci.men.to}{0}
\verb{desconjuntado}{}{}{}{}{adj.}{Desarticulado, desmontado, desmantelado, separado.}{des.con.jun.ta.do}{0}
\verb{desconjuntar}{}{}{}{}{}{Desarticular, desorganizar, desmanchar.}{des.con.jun.tar}{\verboinum{1}}
\verb{desconjuntar}{}{}{}{}{v.t.}{Tirar fora das juntas; desunir, separar.}{des.con.jun.tar}{0}
\verb{desconjurar}{}{}{}{}{v.t.}{Desacatar, ofender.}{des.con.ju.rar}{0}
\verb{desconjurar}{}{}{}{}{}{Fazer prometer ou jurar; esconjurar.}{des.con.ju.rar}{\verboinum{1}}
\verb{desconsertar}{}{}{}{}{v.t.}{Desconjuntar, estragar.}{des.con.ser.tar}{\verboinum{1}}
\verb{desconsideração}{}{}{"-ões}{}{s.f.}{Ausência de consideração; desrespeito, ofensa.}{des.con.si.de.ra.ção}{0}
\verb{desconsiderar}{}{}{}{}{v.t.}{Deixar de considerar; desprezar, ignorar.}{des.con.si.de.rar}{0}
\verb{desconsiderar}{}{}{}{}{}{Tratar sem consideração; desvalorizar.}{des.con.si.de.rar}{\verboinum{1}}
\verb{desconsolação}{}{}{"-ões}{}{s.f.}{Falta de consolação; desconsolo.}{des.con.so.la.ção}{0}
\verb{desconsolado}{}{}{}{}{adj.}{Triste, consternado.}{des.con.so.la.do}{0}
\verb{desconsolar}{}{}{}{}{v.t.}{Causar desconsolação; entristecer, afligir.}{des.con.so.lar}{\verboinum{1}}
\verb{desconsolo}{ô}{}{}{}{s.m.}{Desconsolação.}{des.con.so.lo}{0}
\verb{descontar}{}{}{}{}{v.t.}{Pagar ou receber um título com desconto antes do vencimento.}{des.con.tar}{0}
\verb{descontar}{}{}{}{}{}{Subtrair de uma quantia ou quantidade; deduzir, abater.}{des.con.tar}{0}
\verb{descontar}{}{}{}{}{}{Não levar em conta; desconsiderar.}{des.con.tar}{0}
\verb{descontar}{}{Pop.}{}{}{}{Revidar, desforrar, responder.}{des.con.tar}{\verboinum{1}}
\verb{descontentamento}{}{}{}{}{s.m.}{Falta de contentamento; desgosto.}{des.con.ten.ta.men.to}{0}
\verb{descontentar}{}{}{}{}{v.t.}{Tornar descontente; aborrecer, desgostar.}{des.con.ten.tar}{\verboinum{1}}
\verb{descontente}{}{}{}{}{adj.2g.}{Triste, insatisfeito, aborrecido.}{des.con.ten.te}{0}
\verb{descontinuar}{}{}{}{}{v.t.}{Interromper, suspender.}{des.con.ti.nu.ar}{\verboinum{1}}
\verb{descontinuidade}{}{}{}{}{s.f.}{Qualidade de descontínuo.}{des.con.ti.nu.i.da.de}{0}
\verb{descontínuo}{}{}{}{}{adj.}{Não contínuo; interrompido.}{des.con.tí.nu.o}{0}
\verb{desconto}{}{}{}{}{s.m.}{Ato ou efeito de descontar.}{des.con.to}{0}
\verb{desconto}{}{}{}{}{}{Abatimento, dedução.}{des.con.to}{0}
\verb{descontração}{}{}{"-ões}{}{s.f.}{Ato ou efeito de descontrair.}{des.con.tra.ção}{0}
\verb{descontraído}{}{}{}{}{adj.}{Sem constrangimento ou timidez; espontâneo, simples, informal.}{des.con.tra.í.do}{0}
\verb{descontrair}{}{}{}{}{v.t.}{Cessar a contração; relaxar.}{des.con.tra.ir}{0}
\verb{descontrair}{}{}{}{}{}{Diminuir ou eliminar o constrangimento; tornar natural, espontâneo, informal.}{des.con.tra.ir}{\verboinum{19}}
\verb{descontratar}{}{}{}{}{v.t.}{Anular um contrato.}{des.con.tra.tar}{\verboinum{1}}
\verb{descontrolar}{}{}{}{}{v.t.}{Fazer perder o controle; desgovernar, desequilibrar.}{des.con.tro.lar}{0}
\verb{descontrolar}{}{}{}{}{v.i.}{Perder o controle; desequilibrar"-se.}{des.con.tro.lar}{\verboinum{1}}
\verb{descontrole}{ô}{}{}{}{s.m.}{Ausência de controle.}{des.con.tro.le}{0}
\verb{desconversar}{}{Bras.}{}{}{v.i.}{Durante uma conversa, mudar de assunto deliberadamente.}{des.con.ver.sar}{\verboinum{1}}
\verb{descorado}{}{}{}{}{adj.}{Que perdeu a cor; pálido.}{des.co.ra.do}{0}
\verb{descorar}{}{}{}{}{v.t.}{Fazer perder a cor.}{des.co.rar}{0}
\verb{descorar}{}{}{}{}{v.i.}{Perder a cor; empalidecer.}{des.co.rar}{\verboinum{1}}
\verb{descorçoar}{}{}{}{}{v.t.}{Descoroçoar.}{des.cor.ço.ar}{\verboinum{7}}
\verb{descoroçoar}{}{}{}{}{v.t.}{Tirar o ânimo; desacorçoar.}{des.co.ro.ço.ar}{0}
\verb{descoroçoar}{}{}{}{}{v.i.}{Perder o ânimo.}{des.co.ro.ço.ar}{\verboinum{7}}
\verb{descortês}{}{}{}{}{adj.}{Sem cortesia; indelicado, grosseiro.}{des.cor.tês}{0}
\verb{descortesia}{}{}{}{}{s.f.}{Ato ou dito descortês; indelicadeza, grosseria.}{des.cor.te.si.a}{0}
\verb{descortinar}{}{}{}{}{v.t.}{Mostrar, correndo a cortina.}{des.cor.ti.nar}{0}
\verb{descortinar}{}{}{}{}{}{Tornar visível; revelar, patentear.}{des.cor.ti.nar}{0}
\verb{descortinar}{}{}{}{}{}{Avistar, enxergar, distinguir, descobrir.}{des.cor.ti.nar}{\verboinum{1}}
\verb{descortino}{}{Bras.}{}{}{s.m.}{Ato de descortinar.}{des.cor.ti.no}{0}
\verb{descoser}{ê}{}{}{}{v.t.}{Desfazer a costura; descosturar.}{des.co.ser}{\verboinum{12}}
\verb{descosido}{}{}{}{}{adj.}{Cuja costura se desfez.}{des.co.si.do}{0}
\verb{descosido}{}{Fig.}{}{}{}{Desconjuntado, solto, sem nexo.}{des.co.si.do}{0}
\verb{descosturar}{}{}{}{}{v.t.}{Desfazer a costura; descoser.}{des.cos.tu.rar}{\verboinum{1}}
\verb{descredenciar}{}{}{}{}{v.t.}{Tirar as credenciais.}{des.cre.den.ci.ar}{\verboinum{1}}
\verb{descrédito}{}{}{}{}{s.m.}{Falta de crédito; má fama.}{des.cré.di.to}{0}
\verb{descrença}{}{}{}{}{s.f.}{Falta de crença; incredulidade.}{des.cren.ça}{0}
\verb{descrente}{}{}{}{}{adj.2g.}{Incrédulo.}{des.cren.te}{0}
\verb{descrer}{ê}{}{}{}{v.t.}{Não crer, não ter fé.}{des.crer}{0}
\verb{descrer}{ê}{}{}{}{}{Não aceitar; negar.}{des.crer}{\verboinum{13}}
\verb{descrever}{ê}{}{}{}{v.t.}{Apresentar as características ou os detalhes de algo.}{des.cre.ver}{0}
\verb{descrever}{ê}{}{}{}{}{Contar detalhadamente; narrar.}{des.cre.ver}{\verboinum{12}}
\verb{descrição}{}{}{"-ões}{}{s.f.}{Ato ou efeito de descrever.}{des.cri.ção}{0}
\verb{descriminação}{}{Jur.}{"-ões}{}{s.f.}{Ato ou efeito de descriminar.}{des.cri.mi.na.ção}{0}
\verb{descriminalizar}{}{Jur.}{}{}{v.t.}{Isentar de culpa; absolver.}{des.cri.mi.na.li.zar}{\verboinum{1}}
\verb{descriminar}{}{Jur.}{}{}{v.t.}{Descriminalizar.}{des.cri.mi.nar}{\verboinum{1}}
\verb{descristianizar}{}{}{}{}{v.t.}{Tirar a qualidade de cristão.}{des.cris.ti.a.ni.zar}{\verboinum{1}}
\verb{descritível}{}{}{"-eis}{}{adj.2g.}{Que pode ser descrito.}{des.cri.tí.vel}{0}
\verb{descritivo}{}{}{}{}{adj.}{Relativo a descrição.}{des.cri.ti.vo}{0}
\verb{descritivo}{}{}{}{}{}{Em que há descrição.}{des.cri.ti.vo}{0}
\verb{descrito}{}{}{}{}{adj.}{Representado em detalhes.}{des.cri.to}{0}
\verb{descritor}{ô}{}{}{}{adj.}{Que expõe em detalhes; narrador.}{des.cri.tor}{0}
\verb{descruzar}{}{}{}{}{v.t.}{Desfazer o cruzamento.}{des.cru.zar}{\verboinum{1}}
\verb{descuidado}{}{}{}{}{adj.}{Que não tem cuidado; negligente.}{des.cui.da.do}{0}
\verb{descuidar}{}{}{}{}{v.t.}{Tratar sem cuidado.}{des.cui.dar}{0}
\verb{descuidar}{}{}{}{}{v.pron.}{Desprevinir"-se, relaxar"-se, esquecer"-se.}{des.cui.dar}{\verboinum{1}}
\verb{descuidista}{}{}{}{}{adj.2g.}{Descuidado, negligente, incauto.}{des.cui.dis.ta}{0}
\verb{descuidista}{}{Pop.}{}{}{s.2g.}{Ladrão que atua mediante o descuido da vítima.}{des.cui.dis.ta}{0}
\verb{descuido}{}{}{}{}{s.m.}{Falta de cuidado; negligência.}{des.cui.do}{0}
\verb{descuido}{}{}{}{}{}{Lapso, distração.}{des.cui.do}{0}
\verb{descuidoso}{ô}{}{"-osos ⟨ó⟩}{"-osa ⟨ó⟩}{adj.}{Descuidado, negligente.}{des.cui.do.so}{0}
\verb{descuidoso}{ô}{}{"-osos ⟨ó⟩}{"-osa ⟨ó⟩}{}{Tranquilo, descansado.}{des.cui.do.so}{0}
\verb{desculpa}{}{}{}{}{s.f.}{Ato ou efeito de desculpar.}{des.cul.pa}{0}
\verb{desculpa}{}{}{}{}{}{Perdão, absolvição.}{des.cul.pa}{0}
\verb{desculpa}{}{}{}{}{}{Justificação, justificativa.}{des.cul.pa}{0}
\verb{desculpa}{}{}{}{}{}{Pretexto.}{des.cul.pa}{0}
\verb{desculpar}{}{}{}{}{v.t.}{Perdoar, absolver.}{des.cul.par}{0}
\verb{desculpar}{}{}{}{}{}{Relevar.}{des.cul.par}{0}
\verb{desculpar}{}{}{}{}{v.pron.}{Expor as razões que justificam um ato; justificar"-se.}{des.cul.par}{0}
\verb{desculpar}{}{}{}{}{}{Alegar como pretexto; escusar.}{des.cul.par}{\verboinum{1}}
\verb{descumprir}{}{}{}{}{v.t.}{Deixar de cumprir.}{des.cum.prir}{\verboinum{18}}
\verb{descupinizar}{}{}{}{}{v.t.}{Eliminar o(s) cupim(ns) de.}{des.cu.pi.ni.zar}{\verboinum{1}}
\verb{descurar}{}{}{}{}{v.t.}{Desleixar, descuidar, abandonar.}{des.cu.rar}{\verboinum{1}}
\verb{desdar}{}{}{}{}{v.t.}{Desatar nó ou laço.}{des.dar}{0}
\verb{desdar}{}{Desus.}{}{}{}{Tomar de volta algo que havia sido dado.}{des.dar}{0}
\verb{desdar}{}{}{}{}{v.pron.}{Soltar"-se, desatar"-se.}{des.dar}{\verboinum{37}}
\verb{desde}{ê}{}{}{}{prep.}{A partir de, a começar de, a contar de (no espaço e no tempo).}{des.de}{0}
\verb{desdém}{}{}{"-éns}{}{s.m.}{Desprezo causado pelo orgulho.}{des.dém}{0}
\verb{desdenhar}{}{}{}{}{v.t.}{Mostrar a pouca importância que se dá a pessoa ou coisa; desprezar.}{des.de.nhar}{\verboinum{1}}
\verb{desdenhoso}{ô}{}{"-osos ⟨ó⟩}{"-osa ⟨ó⟩}{adj.}{Cheio de desdém; menosprezador.}{des.de.nho.so}{0}
\verb{desdentado}{}{}{}{}{adj.}{Que não tem dentes.}{des.den.ta.do}{0}
\verb{desdita}{}{}{}{}{s.f.}{Má sorte, desgraça, infelicidade.}{des.di.ta}{0}
\verb{desditoso}{ô}{}{"-osos ⟨ó⟩}{"-osa ⟨ó⟩}{adj.}{Que foi atingido pela desdita; infeliz.}{des.di.to.so}{0}
\verb{desdizer}{ê}{}{}{}{v.t.}{Dizer o contrário do que foi afirmado; desmentir}{des.di.zer}{\verboinum{41}}
\verb{desdobramento}{}{}{}{}{s.m.}{Ato ou efeito de desdobrar; desenvolvimento, consequência, incremento.}{des.do.bra.men.to}{0}
\verb{desdobrar}{}{}{}{}{v.t.}{Abrir as dobras de alguma coisa.}{des.do.brar}{0}
\verb{desdobrar}{}{}{}{}{}{Dividir alguma coisa em duas partes.}{des.do.brar}{0}
\verb{desdobrar}{}{}{}{}{v.pron.}{Trabalhar muito; esforçar"-se.}{des.do.brar}{\verboinum{1}}
\verb{desdoirar}{}{}{}{}{}{Var. de \textit{desdourar}.}{des.doi.rar}{0}
\verb{desdourar}{}{}{}{}{v.t.}{Fazer perder o dourado.}{des.dou.rar}{0}
\verb{desdourar}{}{Fig.}{}{}{}{Fazer perder o brilho, o encanto, o mérito.}{des.dou.rar}{\verboinum{1}}
\verb{desdouro}{ô}{}{}{}{s.m.}{Ato ou efeito de desdourar.}{des.dou.ro}{0}
\verb{desdouro}{ô}{}{}{}{}{Descrédito, desonra.}{des.dou.ro}{0}
\verb{deseducar}{}{}{}{}{v.t.}{Fazer perder a educação.}{de.se.du.car}{0}
\verb{deseducar}{}{}{}{}{}{Educar mal.}{de.se.du.car}{\verboinum{2}}
\verb{desejar}{}{}{}{}{v.t.}{Sentir vontade de ter alguém ou alguma coisa; ambicionar, querer.}{de.se.jar}{0}
\verb{desejar}{}{}{}{}{}{Querer que alguém tenha alguma coisa; almejar. (\textit{Os pais desejam aos filhos um futuro feliz.})}{de.se.jar}{\verboinum{1}}
\verb{desejo}{ê}{}{}{}{s.m.}{Vontade de conseguir algo; ambição, anseio, aspiração.}{de.se.jo}{0}
\verb{desejoso}{ô}{}{"-osos ⟨ó⟩}{"-osa ⟨ó⟩}{adj.}{Que tem desejo; ávido, cobiçoso.}{de.se.jo.so}{0}
\verb{deselegância}{}{}{}{}{s.f.}{Falta de elegância; desalinho; desajeitamento; indelicadeza; incorreção.}{de.se.le.gân.cia}{0}
\verb{deselegante}{}{}{}{}{adj.2g.}{Que não tem elegância; desalinhado; indelicado.}{de.se.le.gan.te}{0}
\verb{desmaranhar}{}{}{}{}{v.t.}{Desmanchar o emaranhado; desembaraçar.}{des.ma.ra.nhar}{0}
\verb{desmaranhar}{}{}{}{}{}{Esclarecer, decifrar, explicar algo emaranhado.}{des.ma.ra.nhar}{\verboinum{1}}
\verb{desembaçar}{}{}{}{}{v.t.}{Desfazer nó ou embaraço.}{de.sem.ba.çar}{0}
\verb{desembaçar}{}{}{}{}{}{Tornar claro, compreensível; decifrar.}{de.sem.ba.çar}{0}
\verb{desembaçar}{}{}{}{}{}{Fazer perder a timidez; desinibir.}{de.sem.ba.çar}{\verboinum{3}}
\verb{desembaciar}{}{}{}{}{v.t.}{Devolver ou recuperar o brilho, a transparência; limpar, polir.}{de.sem.ba.ci.ar}{\verboinum{1}}
\verb{desembainhar}{}{}{}{}{v.t.}{Tirar a arma da bainha.}{de.sem.ba.i.nhar}{0}
\verb{desembainhar}{}{}{}{}{}{Desmanchar a bainha de uma roupa.}{de.sem.ba.i.nhar}{\verboinum{1}}
\verb{desembalar}{}{}{}{}{v.t.}{Retirar de embalagem.}{de.sem.ba.lar}{0}
\verb{desembalar}{}{}{}{}{}{Fazer perder o embalo, a velocidade; desacelerar.}{de.sem.ba.lar}{\verboinum{1}}
\verb{desembaraçado}{}{}{}{}{adj.}{Livre de embaraços.}{de.sem.ba.ra.ça.do}{0}
\verb{desembaraçado}{}{}{}{}{}{Ativo, diligente, desinibido, espontâneo.}{de.sem.ba.ra.ça.do}{0}
\verb{desembaraçar}{}{}{}{}{v.t.}{Desfazer nó ou embaraço.}{de.sem.ba.ra.çar}{0}
\verb{desembaraçar}{}{}{}{}{}{Livrar de embaraço; desimpedir.}{de.sem.ba.ra.çar}{0}
\verb{desembaraçar}{}{}{}{}{}{Tornar claro, compreensível; decifrar.}{de.sem.ba.ra.çar}{0}
\verb{desembaraçar}{}{}{}{}{v.pron.}{Perder a timidez, o acanhamento; desinibir.}{de.sem.ba.ra.çar}{\verboinum{3}}
\verb{desembaraço}{}{}{}{}{s.m.}{Ato ou efeito de desembaraçar; ausência de embaraço, de obstáculo; desempedimento.}{de.sem.ba.ra.ço}{0}
\verb{desembaraço}{}{}{}{}{}{Ausência de acanhamento; desenvoltura, desinibição.}{de.sem.ba.ra.ço}{0}
\verb{desembaraço}{}{}{}{}{}{Valentia diante do perigo; coragem.}{de.sem.ba.ra.ço}{0}
\verb{desembaralhar}{}{}{}{}{v.t.}{Pôr em ordem o que está confuso; desembaraçar.}{de.sem.ba.ra.lhar}{\verboinum{1}}
\verb{desembarcadouro}{ô}{}{}{}{s.m.}{Local de desembarque.}{de.sem.bar.ca.dou.ro}{0}
\verb{desembarcar}{}{}{}{}{v.t.}{Fazer pessoa ou coisa sair de um veículo.}{de.sem.bar.car}{\verboinum{2}}
\verb{desembargador}{ô}{Jur.}{}{}{s.m.}{Juiz do Tribunal de Justiça ou do Tribunal de Apelação.}{de.sem.bar.ga.dor}{0}
\verb{desembargar}{}{}{}{}{v.t.}{Levantar, tirar o embargo ou o empedimento; desembaraçar, despachar.}{de.sem.bar.gar}{\verboinum{5}}
\verb{desembargo}{}{}{}{}{s.m.}{Ato ou efeito de desembargar; despacho, desempedimento.}{de.sem.bar.go}{0}
\verb{desembarque}{}{}{}{}{s.m.}{Retirada ou saída de pessoa(s) ou objeto(s) de um veículo.}{de.sem.bar.que}{0}
\verb{desembarque}{}{}{}{}{}{Local desse movimento.}{de.sem.bar.que}{0}
\verb{desembestado}{}{}{}{}{adj.}{Que desembestou; desenfreado, desabalado.}{de.sem.bes.ta.do}{0}
\verb{desembestar}{}{}{}{}{v.i.}{Sair correndo apressadamente.}{de.sem.bes.tar}{\verboinum{1}}
\verb{desembocadura}{}{}{}{}{s.f.}{Lugar em que o rio deságua; foz.}{de.sem.bo.ca.du.ra}{0}
\verb{desembocar}{}{}{}{}{v.i.}{Ter o final de seu curso em determinado lugar.}{de.sem.bo.car}{\verboinum{2}}
\verb{desembolsar}{}{}{}{}{v.t.}{Tirar da bolsa ou do bolso.}{de.sem.bol.sar}{\verboinum{1}}
\verb{desembolso}{ô}{}{}{}{s.m.}{Valor que se desembolsou; despesa, gasto.}{de.sem.bol.so}{0}
\verb{desembrear}{}{}{}{}{v.t.}{Deixar um veículo sem marcha engatada; desengatar.}{de.sem.bre.ar}{\verboinum{4}}
\verb{desembrulhar}{}{}{}{}{v.t.}{Desfazer o embrulho de alguma coisa.}{de.sem.bru.lhar}{\verboinum{1}}
\verb{desembuchar}{}{}{}{}{v.t.}{Tirar ou expelir do bucho, do estômago.}{de.sem.bu.char}{0}
\verb{desembuchar}{}{}{}{}{}{Expor, falar algo que não se queria ou não se ousava dizer, contar; desabafar. }{de.sem.bu.char}{\verboinum{1}}
\verb{desemburrar}{}{}{}{}{v.t.}{Livrar da ignorância; tornar mais instruído.}{de.sem.bur.rar}{0}
\verb{desemburrar}{}{}{}{}{}{Dar boas maneiras; educar, polir.}{de.sem.bur.rar}{0}
\verb{desemburrar}{}{}{}{}{}{Fazer perder o enfezamento; desamuar, alegrar.}{de.sem.bur.rar}{\verboinum{1}}
\verb{desemoldurar}{}{}{}{}{v.t.}{Retirar a moldura; desenquadrar.}{de.se.mol.du.rar}{\verboinum{1}}
\verb{desempacar}{}{}{}{}{v.t.}{Fazer andar um animal que não quer sair do lugar.}{de.sem.pa.car}{\verboinum{2}}
\verb{desempacotar}{}{}{}{}{v.t.}{Retirar o conteúdo de pacote, embrulho; desembalar.}{de.sem.pa.co.tar}{0}
\verb{desempacotar}{}{}{}{}{}{Desfazer pacote, embrulho; desembrulhar.}{de.sem.pa.co.tar}{\verboinum{1}}
\verb{desempanar}{}{}{}{}{v.t.}{Restituir o brilho, dar lustre; desembaçar.}{de.sem.pa.nar}{0}
\verb{desempanar}{}{}{}{}{}{Desvendar, esclarecer.}{de.sem.pa.nar}{\verboinum{1}}
\verb{desemparelhar}{}{}{}{}{v.t.}{Separar o que estava emparelhado, desfazer um par ou parelha.}{de.sem.pa.re.lhar}{\verboinum{1}}
\verb{desempatar}{}{}{}{}{v.t.}{Sair de um resultado em que nem se ganha, nem se perde; decidir, resolver.}{de.sem.pa.tar}{\verboinum{1}}
\verb{desempate}{}{}{}{}{s.m.}{Ato ou efeito de desempatar; decisão.}{de.sem.pa.te}{0}
\verb{desempenado}{}{}{}{}{adj.}{Que não está empenado; direito.}{de.sem.pe.na.do}{0}
\verb{desempenar}{}{}{}{}{v.t.}{Fazer deixar de ser torto; endireitar.}{de.sem.pe.nar}{\verboinum{1}}
\verb{desempenhar}{}{}{}{}{v.t.}{Resgatar, recuperar o que se havia penhorado.}{de.sem.pe.nhar}{0}
\verb{desempenhar}{}{}{}{}{}{Fazer alguma atividade que é sua obrigação; cumprir, executar; exercer.}{de.sem.pe.nhar}{0}
\verb{desempenhar}{}{}{}{}{}{Representar em cena.}{de.sem.pe.nhar}{\verboinum{1}}
\verb{desempenho}{}{}{}{}{s.m.}{Ato ou efeito de desempenhar; atuação.}{de.sem.pe.nho}{0}
\verb{desempeno}{}{}{}{}{s.m.}{Ato ou efeito de desempenar.}{de.sem.pe.no}{0}
\verb{desempeno}{}{}{}{}{}{Elegância, aprumo.}{de.sem.pe.no}{0}
\verb{desemperrar}{}{}{}{}{v.t.}{Soltar algo que estava emperrado.}{de.sem.per.rar}{\verboinum{1}}
\verb{desempilhar}{}{}{}{}{v.t.}{Desarrumar, desfazer a organização.}{de.sem.pi.lhar}{\verboinum{1}}
\verb{desempoar}{}{}{}{}{v.t.}{Tirar o pó.}{de.sem.po.ar}{0}
\verb{desempoar}{}{Fig.}{}{}{}{Livrar de preconceitos; esclarecer.}{de.sem.po.ar}{0}
\verb{desempoar}{}{Fig.}{}{}{}{Tornar modesto.}{de.sem.po.ar}{\verboinum{7}}
\verb{desempoçar}{}{}{}{}{v.t.}{Desfazer as poças.}{de.sem.po.çar}{0}
\verb{desempoçar}{}{}{}{}{}{Tirar água do poço.}{de.sem.po.çar}{\verboinum{3}}
\verb{desempoeirar}{}{}{}{}{v.t.}{Tirar a poeira.}{de.sem.po.ei.rar}{0}
\verb{desempoeirar}{}{Fig.}{}{}{}{Livrar de preconceito; esclarecer.}{de.sem.po.ei.rar}{\verboinum{1}}
\verb{desempoleirar}{}{}{}{}{v.t.}{Tirar do poleiro.}{de.sem.po.lei.rar}{\verboinum{1}}
\verb{desempossar}{}{}{}{}{v.t.}{Privar da posse; desapossar.}{de.sem.pos.sar}{\verboinum{1}}
\verb{desempregado}{}{}{}{}{adj.}{Que não tem emprego; desocupado.}{de.sem.pre.ga.do}{0}
\verb{desempregar}{}{}{}{}{v.t.}{Tirar o emprego; despedir, exonerar.}{de.sem.pre.gar}{\verboinum{1}}
\verb{desemprego}{ê}{}{}{}{s.m.}{Falta de emprego.}{de.sem.pre.go}{0}
\verb{desemprego}{ê}{}{}{}{}{Estado de quem procura emprego e não encontra.}{de.sem.pre.go}{0}
\verb{desencabar}{}{}{}{}{v.t.}{Soltar do cabo.}{de.sen.ca.bar}{\verboinum{1}}
\verb{desencabeçar}{}{}{}{}{v.t.}{Tirar uma ideia da cabeça; dissuadir.}{de.sen.ca.be.çar}{0}
\verb{desencabeçar}{}{}{}{}{}{Fazer perder a cabeça; perverter.}{de.sen.ca.be.çar}{\verboinum{3}}
\verb{desencabrestar}{}{}{}{}{v.t.}{Tirar do cabresto.}{de.sen.ca.bres.tar}{\verboinum{1}}
\verb{descadeamento}{}{}{}{}{s.m.}{Ato ou efeito de desencadear.}{des.ca.de.a.men.to}{0}
\verb{descadeamento}{}{}{}{}{}{Soltura, separação, desligamento.}{des.ca.de.a.men.to}{0}
\verb{descadeamento}{}{}{}{}{}{Falta de ordenação; desconexão.}{des.ca.de.a.men.to}{0}
\verb{desencadear}{}{}{}{}{v.t.}{Soltar aquilo que estava preso por cadeias.}{de.sen.ca.de.ar}{0}
\verb{desencadear}{}{}{}{}{}{Dar vazão; libertar.}{de.sen.ca.de.ar}{0}
\verb{desencadear}{}{}{}{}{v.i.}{Cair com força (chuva).}{de.sen.ca.de.ar}{\verboinum{4}}
\verb{desencadernar}{}{}{}{}{v.t.}{Desfazer a encadernação.}{de.sen.ca.der.nar}{0}
\verb{desencadernar}{}{Por ext.}{}{}{}{Desconjuntar, desmantelar.}{de.sen.ca.der.nar}{\verboinum{1}}
\verb{desencaixar}{ch}{}{}{}{v.t.}{Fazer sair do encaixe.}{de.sen.cai.xar}{0}
\verb{desencaixar}{ch}{}{}{}{}{Tirar da caixa.}{de.sen.cai.xar}{\verboinum{1}}
\verb{desencaixe}{ch}{}{}{}{s.m.}{Ato ou efeito de desencaixar.}{de.sen.cai.xe}{0}
\verb{desencaixotar}{ch}{}{}{}{v.t.}{Tirar de caixa ou caixote.}{de.sen.cai.xo.tar}{\verboinum{1}}
\verb{desencalacrar}{}{Pop.}{}{}{v.t.}{Livrar de calacre, de dificuldades.}{de.sen.ca.la.crar}{\verboinum{1}}
\verb{desencalhar}{}{}{}{}{v.t.}{Tirar ou sair do encalhe.}{de.sen.ca.lhar}{0}
\verb{desencalhar}{}{}{}{}{}{Fazer retomar o funcionamento ou o andamento.}{de.sen.ca.lhar}{\verboinum{1}}
\verb{desencalhe}{}{}{}{}{s.m.}{Ato ou efeito de desencalhar.}{de.sen.ca.lhe}{0}
\verb{desencaminhamento}{}{}{}{}{s.m.}{Ato ou efeito de desencaminhar.}{de.sen.ca.mi.nha.men.to}{0}
\verb{desencaminhar}{}{}{}{}{v.t.}{Desviar do caminho; extraviar.}{de.sen.ca.mi.nhar}{0}
\verb{desencaminhar}{}{Fig.}{}{}{}{Afastar daquilo que é considerado moralmente bom; perverter, corromper.}{de.sen.ca.mi.nhar}{\verboinum{1}}
\verb{desencantamento}{}{}{}{}{s.m.}{Ato ou efeito de desencantar; desencanto.}{de.sen.can.ta.men.to}{0}
\verb{desencantamento}{}{}{}{}{}{Estado daquele que se frustrou; desgosto, decepção.}{de.sen.can.ta.men.to}{0}
\verb{desencantar}{}{}{}{}{v.t.}{Tirar o encanto.}{de.sen.can.tar}{0}
\verb{desencantar}{}{}{}{}{}{Causar decepção; desiludir.}{de.sen.can.tar}{\verboinum{1}}
\verb{desencanto}{}{}{}{}{s.m.}{Desencantamento.}{de.sen.can.to}{0}
\verb{desencapar}{}{}{}{}{v.t.}{Tirar a capa; desembrulhar.}{de.sen.ca.par}{0}
\verb{desencapar}{}{}{}{}{}{Retirar o capote; desencapotar.}{de.sen.ca.par}{\verboinum{1}}
\verb{desencarapinhar}{}{}{}{}{v.t.}{Tornar o cabelo liso; desencrespar, alisar.}{de.sen.ca.ra.pi.nhar}{\verboinum{1}}
\verb{desencarcerar}{}{}{}{}{v.t.}{Libertar do cárcere.}{de.sen.car.ce.rar}{\verboinum{1}}
\verb{desencardir}{}{}{}{}{v.t.}{Limpar a sujeira; alvejar, clarear.}{de.sen.car.dir}{\verboinum{18}}
\verb{desencargo}{}{}{}{}{s.m.}{Ato de desobrigar de um encargo.}{de.sen.car.go}{0}
\verb{desencargo}{}{}{}{}{}{Diminuição da ansiedade; alívio.}{de.sen.car.go}{0}
\verb{desencarnar}{}{}{}{}{v.t.}{Tirar a carne; descarnar.}{de.sen.car.nar}{0}
\verb{desencarnar}{}{Relig.}{}{}{}{Separar a alma do corpo; morrer.}{de.sen.car.nar}{\verboinum{1}}
\verb{desencarquilhar}{}{}{}{}{v.t.}{Remover carquilhas, rugas; alisar, desenrugar.}{de.sen.car.qui.lhar}{\verboinum{1}}
\verb{desencarregar}{}{}{}{}{v.t.}{Livrar de culpa; absolver.}{de.sen.car.re.gar}{0}
\verb{desencarregar}{}{}{}{}{}{Livrar de obrigação ou encargo; desobrigar.}{de.sen.car.re.gar}{0}
\verb{desencarregar}{}{}{}{}{}{Destituir de cargo ou emprego.}{de.sen.car.re.gar}{\verboinum{1}}
\verb{desencarrilar}{}{}{}{}{v.t.}{Descarrilar.}{de.sen.car.ri.lar}{\verboinum{1}}
\verb{desencarrilhar}{}{}{}{}{v.t.}{Descarrilar.}{de.sen.car.ri.lhar}{\verboinum{1}}
\verb{desencasquetar}{}{}{}{}{v.t.}{Tirar o casquete.}{de.sen.cas.que.tar}{0}
\verb{desencasquetar}{}{Pop.}{}{}{}{Tirar uma ideia fixa ou uma mania; dissuadir.}{de.sen.cas.que.tar}{\verboinum{1}}
\verb{desencastoar}{}{}{}{}{v.t.}{Tirar o castão.}{de.sen.cas.to.ar}{0}
\verb{desencastoar}{}{}{}{}{}{Tirar do engaste; desengastar.}{de.sen.cas.to.ar}{\verboinum{7}}
\verb{desencavar}{}{Bras.}{}{}{v.t.}{Descobrir.}{de.sen.ca.var}{\verboinum{1}}
\verb{desencilhar}{}{}{}{}{v.t.}{Tirar a cilha.}{de.sen.ci.lhar}{0}
\verb{desencilhar}{}{Bras.}{}{}{}{Tirar os arreios.}{de.sen.ci.lhar}{\verboinum{1}}
\verb{desencobrir}{}{}{}{}{v.t.}{Tirar aquilo que cobre; descobrir.}{de.sen.co.brir}{\verboinum{31}}
\verb{desencolher}{ê}{}{}{}{v.t.}{Esticar o que esteve encolhido.}{de.sen.co.lher}{0}
\verb{desencolher}{ê}{}{}{}{v.pron.}{Retomar o tamanho original.}{de.sen.co.lher}{0}
\verb{desencolher}{ê}{}{}{}{}{Perder o acanhamento.}{de.sen.co.lher}{\verboinum{12}}
\verb{desencomendar}{}{}{}{}{v.t.}{Cancelar uma encomenda.}{de.sen.co.men.dar}{\verboinum{1}}
\verb{desencontrado}{}{}{}{}{adj.}{Que vai em direção oposta; contrário.}{de.sen.con.tra.do}{0}
\verb{desencontrado}{}{}{}{}{}{Não condizente; discordante, inconsistente.}{de.sen.con.tra.do}{0}
\verb{desencontrar}{}{}{}{}{v.t.}{Não estar no mesmo lugar e momento; perder.}{de.sen.con.trar}{0}
\verb{desencontrar}{}{}{}{}{v.i.}{Ser incompatível ou oposto; divergir.}{de.sen.con.trar}{\verboinum{1}}
\verb{desencontro}{}{}{}{}{s.m.}{Ato ou efeito de desencontrar.}{de.sen.con.tro}{0}
\verb{desencontro}{}{Fig.}{}{}{}{Divergência, discrepância.}{de.sen.con.tro}{0}
\verb{desencorajamento}{}{}{}{}{s.m.}{Ato ou efeito de desencorajar.}{de.sen.co.ra.ja.men.to}{0}
\verb{desencorajar}{}{}{}{}{v.t.}{Tirar a coragem, o ânimo; desestimular.}{de.sen.co.ra.jar}{\verboinum{1}}
\verb{desencordoar}{}{}{}{}{v.t.}{Tirar a(s) corda(s).}{de.sen.cor.do.ar}{\verboinum{7}}
\verb{desencorpar}{}{}{}{}{v.t.}{Diminuir a consistência ou o volume.}{de.sen.cor.par}{\verboinum{1}}
\verb{desencostar}{}{}{}{}{v.t.}{Afastar ou desviar do encosto; desapoiar.}{de.sen.cos.tar}{\verboinum{1}}
\verb{desencovar}{}{}{}{}{v.t.}{Tirar da cova.}{de.sen.co.var}{0}
\verb{desencovar}{}{}{}{}{}{Descobrir, desencavar.}{de.sen.co.var}{\verboinum{1}}
\verb{desencravar}{}{}{}{}{v.t.}{Tirar o que estava encravado; despregar.}{de.sen.cra.var}{\verboinum{1}}
\verb{desencrespar}{}{}{}{}{v.t.}{Alisar, amaciar.}{de.sen.cres.par}{0}
\verb{desencrespar}{}{}{}{}{v.pron.}{Desanuviar"-se, serenar.}{de.sen.cres.par}{\verboinum{1}}
\verb{desencurvar}{}{}{}{}{v.t.}{Desfazer a curvatura; endireitar.}{de.sen.cur.var}{\verboinum{1}}
\verb{desendividar}{}{}{}{}{v.t.}{Pagar dívidas; quitar.}{de.sen.di.vi.dar}{\verboinum{1}}
\verb{desenevoar}{}{}{}{}{v.t.}{Limpar as névoas ou as nuvens; aclarar, desanuviar.}{de.se.ne.vo.ar}{\verboinum{7}}
\verb{desenfadar}{}{}{}{}{v.t.}{Tirar o enfado; distrair, divertir.}{de.sen.fa.dar}{\verboinum{1}}
\verb{desenfado}{}{}{}{}{s.m.}{Alívio do enfado; divertimento, passatempo.}{de.sen.fa.do}{0}
\verb{desenfaixar}{ch}{}{}{}{v.t.}{Tirar as faixas.}{de.sen.fai.xar}{\verboinum{1}}
\verb{desenfardar}{}{}{}{}{v.t.}{Tirar do fardo; desembrulhar.}{de.sen.far.dar}{\verboinum{1}}
\verb{desenfastiar}{}{}{}{}{v.t.}{Despertar o apetite; tirar o fastio.}{de.sen.fas.ti.ar}{0}
\verb{desenfastiar}{}{}{}{}{}{Alegrar, distrair.}{de.sen.fas.ti.ar}{\verboinum{6}}
\verb{desenfeitar}{}{}{}{}{v.t.}{Tirar o(s) enfeite(s).}{de.sen.fei.tar}{\verboinum{1}}
\verb{desenfeitiçar}{}{}{}{}{v.t.}{Tirar o feitiço; desencantar.}{de.sen.fei.ti.çar}{0}
\verb{desenfeitiçar}{}{Fig.}{}{}{}{Tirar a paixão.}{de.sen.fei.ti.çar}{\verboinum{3}}
\verb{desenfeixar}{ch}{}{}{}{v.t.}{Tirar do feixe.}{de.sen.fei.xar}{0}
\verb{desenfeixar}{ch}{}{}{}{}{Desmanchar, separar.}{de.sen.fei.xar}{\verboinum{1}}
\verb{desenferrujar}{}{}{}{}{v.t.}{Tirar a ferrugem.}{de.sen.fer.ru.jar}{0}
\verb{desenferrujar}{}{Fig.}{}{}{}{Fazer exercícios físicos, especialmente depois de período de inatividade.}{de.sen.fer.ru.jar}{\verboinum{1}}
\verb{desenfiar}{}{}{}{}{v.t.}{Tirar do fio ou da linha.}{de.sen.fi.ar}{0}
\verb{desenfiar}{}{}{}{}{}{Tirar linha da agulha.}{de.sen.fi.ar}{\verboinum{6}}
\verb{desenfreado}{}{}{}{}{adj.}{Sem freio.}{de.sen.fre.a.do}{0}
\verb{desenfreado}{}{}{}{}{}{Arrebatado, incontrolável, desmedido.}{de.sen.fre.a.do}{0}
\verb{desenfrear}{}{}{}{}{v.t.}{Tirar o freio; soltar.}{de.sen.fre.ar}{0}
\verb{desenfrear}{}{}{}{}{v.pron.}{Arremessar"-se com ímpeto.}{de.sen.fre.ar}{0}
\verb{desenfrear}{}{}{}{}{}{Enfurecer"-se, irritar"-se.}{de.sen.fre.ar}{\verboinum{4}}
\verb{desenfronhar}{}{}{}{}{v.t.}{Tirar da fronha.}{de.sen.fro.nhar}{0}
\verb{desenfronhar}{}{Por ext.}{}{}{}{Desnudar, despir.}{de.sen.fro.nhar}{\verboinum{1}}
\verb{desenfurnar}{}{}{}{}{v.t.}{Tirar da furna.}{de.sen.fur.nar}{0}
\verb{desenfurnar}{}{}{}{}{v.pron.}{Voltar ao convívio social.}{de.sen.fur.nar}{\verboinum{1}}
\verb{desengaiolar}{}{}{}{}{v.t.}{Tirar da gaiola.}{de.sen.gai.o.lar}{0}
\verb{desengaiolar}{}{}{}{}{}{Soltar da prisão; libertar.}{de.sen.gai.o.lar}{\verboinum{1}}
\verb{desengajado}{}{}{}{}{adj.}{Sem compromisso de caráter ideológico.}{de.sen.ga.ja.do}{0}
\verb{desengajar}{}{}{}{}{v.t.}{Liberar de compromisso; descontratar, desobrigar.}{de.sen.ga.jar}{\verboinum{1}}
\verb{desenganado}{}{}{}{}{adj.}{Desesperançado, desiludido.}{de.sen.ga.na.do}{0}
\verb{desenganado}{}{}{}{}{}{Sem esperança de vida.}{de.sen.ga.na.do}{0}
\verb{desenganar}{}{}{}{}{v.t.}{Tirar o engano ou a esperança ilusória.}{de.sen.ga.nar}{0}
\verb{desenganar}{}{}{}{}{}{Tirar esperança, geralmente de salvação.}{de.sen.ga.nar}{\verboinum{1}}
\verb{desenganchar}{}{}{}{}{v.t.}{Desprender, soltar.}{de.sen.gan.char}{\verboinum{1}}
\verb{desengano}{}{}{}{}{s.m.}{Ato ou efeito de desenganar; desilusão.}{de.sen.ga.no}{0}
\verb{desengarrafar}{}{}{}{}{v.t.}{Tirar da garrafa.}{de.sen.gar.ra.far}{0}
\verb{desengarrafar}{}{}{}{}{}{Cessar engarrafamento (de tráfego); desobstruir.}{de.sen.gar.ra.far}{\verboinum{1}}
\verb{desengasgar}{}{}{}{}{v.t.}{Tirar o engasgamento.}{de.sen.gas.gar}{\verboinum{1}}
\verb{desengastar}{}{}{}{}{v.t.}{Tirar do engaste.}{de.sen.gas.tar}{\verboinum{1}}
\verb{desengatar}{}{}{}{}{v.t.}{Tirar do engate.}{de.sen.ga.tar}{0}
\verb{desengatar}{}{}{}{}{}{Desengrenar.}{de.sen.ga.tar}{\verboinum{1}}
\verb{desengate}{}{}{}{}{s.m.}{Ato ou efeito de desengatar.}{de.sen.ga.te}{0}
\verb{desengatilhar}{}{}{}{}{v.t.}{Desarmar o gatilho de arma de fogo.}{de.sen.ga.ti.lhar}{0}
\verb{desengatilhar}{}{}{}{}{}{Acionar o gatilho de arma de fogo; disparar.}{de.sen.ga.ti.lhar}{\verboinum{1}}
\verb{desengomar}{}{}{}{}{v.t.}{Tirar a goma.}{de.sen.go.mar}{\verboinum{1}}
\verb{desengonçado}{}{}{}{}{adj.}{Fora dos engonços, das dobradiças.}{de.sen.gon.ça.do}{0}
\verb{desengonçado}{}{}{}{}{}{Sem aprumo ou elegância de movimentos; desajeitado, descoordenado.}{de.sen.gon.ça.do}{0}
\verb{desengonçar}{}{}{}{}{v.t.}{Tirar dos engonços.}{de.sen.gon.çar}{0}
\verb{desengonçar}{}{Fig.}{}{}{}{Mover"-se desajeitadamente como se estivesse desconjuntado.}{de.sen.gon.çar}{\verboinum{3}}
\verb{desengordurar}{}{}{}{}{v.t.}{Tirar gordura ou mancha(s) de gordura.}{de.sen.gor.du.rar}{\verboinum{1}}
\verb{desengraçado}{}{}{}{}{adj.}{Sem graça, desgracioso, insípido.}{de.sen.gra.ça.do}{0}
\verb{desengraxar}{ch}{}{}{}{v.t.}{Tirar a graxa ou o polimento.}{de.sen.gra.xar}{\verboinum{1}}
\verb{desengrenar}{}{}{}{}{v.t.}{Em automóvel ou máquina, desfazer a conexão mecânica entre o motor e o sistema de tração ou as peças que ele aciona.}{de.sen.gre.nar}{\verboinum{1}}
\verb{desengrossar}{}{}{}{}{v.t.}{Tornar menos grosso; adelgaçar.}{de.sen.gros.sar}{0}
\verb{desengrossar}{}{}{}{}{v.i.}{Desinchar, desintumescer.}{de.sen.gros.sar}{\verboinum{1}}
\verb{desenguiçar}{}{}{}{}{v.t.}{Tirar o enguiço.}{de.sen.gui.çar}{\verboinum{3}}
\verb{desenhar}{}{}{}{}{v.t.}{Fazer desenho.}{de.se.nhar}{0}
\verb{desenhar}{}{}{}{}{}{Projetar, delinear, conceber, representar.}{de.se.nhar}{\verboinum{1}}
\verb{desenhista}{}{}{}{}{s.2g.}{Indivíduo que desenha profissionalmente ou que domina a técnica do desenho.}{de.se.nhis.ta}{0}
\verb{desenho}{}{}{}{}{s.m.}{Representação de formas em uma superfície por meio de linhas e texturas gráficas.}{de.se.nho}{0}
\verb{desenho}{}{}{}{}{}{Esboço, delineamento.}{de.se.nho}{0}
\verb{desenho}{}{}{}{}{}{Projeto, plano.}{de.se.nho}{0}
\verb{desenlaçar}{}{}{}{}{v.t.}{Tirar do laço; soltar.}{de.sen.la.çar}{0}
\verb{desenlaçar}{}{}{}{}{v.pron.}{Livrar"-se, libertar"-se, desprender"-se.}{de.sen.la.çar}{\verboinum{3}}
\verb{desenlace}{}{}{}{}{s.m.}{Ato ou efeito de desenlaçar.}{de.sen.la.ce}{0}
\verb{desenlace}{}{}{}{}{}{Desfecho, remate, solução.}{de.sen.la.ce}{0}
\verb{desenlamear}{}{}{}{}{v.t.}{Tirar a lama.}{de.sen.la.me.ar}{0}
\verb{desenlamear}{}{}{}{}{}{Restabelecer a honra; limpar.}{de.sen.la.me.ar}{\verboinum{4}}
\verb{desenlatar}{}{}{}{}{v.t.}{Tirar da lata.}{de.sen.la.tar}{\verboinum{1}}
\verb{desenlear}{}{}{}{}{v.t.}{Tirar o enleio; soltar, desprender.}{de.sen.le.ar}{0}
\verb{desenlear}{}{}{}{}{}{Desemaranhar, desenredar.}{de.sen.le.ar}{0}
\verb{desenlear}{}{Fig.}{}{}{}{Livrar de dificuldade.}{de.sen.le.ar}{\verboinum{4}}
\verb{desenodoar}{}{}{}{}{v.t.}{Tirar as nódoas; limpar.}{de.se.no.do.ar}{\verboinum{7}}
\verb{desenovelar}{}{}{}{}{v.t.}{Desfazer um novelo; desenrolar.}{de.se.no.ve.lar}{0}
\verb{desenovelar}{}{}{}{}{v.pron.}{Estender"-se, desdobrar"-se.}{de.se.no.ve.lar}{\verboinum{1}}
\verb{desenquadrar}{}{}{}{}{v.t.}{Tirar de quadro ou moldura.}{de.sen.qua.drar}{\verboinum{1}}
\verb{desenraizar}{}{}{}{}{v.t.}{Arrancar pela raiz.}{de.sen.ra.i.zar}{\verboinum{1}}
\verb{desenrascar}{}{}{}{}{v.t.}{Tirar de dificuldade.}{de.sen.ras.car}{\verboinum{2}}
\verb{desenredar}{}{}{}{}{v.t.}{Desembaraçar, desemaranhar.}{de.sen.re.dar}{0}
\verb{desenredar}{}{}{}{}{}{Resolver ou esclarecer situação ou negócio complicado.}{de.sen.re.dar}{\verboinum{1}}
\verb{desenrijar}{}{}{}{}{v.t.}{Tirar a rigidez.}{de.sen.ri.jar}{\verboinum{1}}
\verb{desenrolar}{}{}{}{}{v.t.}{Desfazer ou tirar de rolo; estender.}{de.sen.ro.lar}{0}
\verb{desenrolar}{}{}{}{}{}{Expor, explicar.}{de.sen.ro.lar}{0}
\verb{desenrolar}{}{}{}{}{v.pron.}{Mostrar"-se, apresentar"-se, estender"-se.}{de.sen.ro.lar}{\verboinum{1}}
\verb{desenroscar}{}{}{}{}{v.t.}{Desenrolar, estirar.}{de.sen.ros.car}{0}
\verb{desenroscar}{}{}{}{}{}{Desparafusar.}{de.sen.ros.car}{\verboinum{2}}
\verb{desenrugar}{}{}{}{}{v.t.}{Tirar as rugas; alisar, desfranzir.}{de.sen.ru.gar}{\verboinum{5}}
\verb{desensaboar}{}{}{}{}{v.t.}{Tirar o sabão.}{de.sen.sa.bo.ar}{\verboinum{7}}
\verb{desensacar}{}{}{}{}{v.t.}{Retirar do saco ou da saca.}{de.sen.sa.car}{\verboinum{2}}
\verb{desensinar}{}{}{}{}{v.t.}{Fazer alguém esquecer o que tinha aprendido; desaprender.}{de.sen.si.nar}{\verboinum{1}}
\verb{desentaipar}{}{}{}{}{v.t.}{Liberar de impedimento; desobstruir, soltar.}{de.sen.tai.par}{0}
\verb{desentaipar}{}{}{}{}{}{Revelar.}{de.sen.tai.par}{0}
\verb{desentaipar}{}{}{}{}{}{Repara ofensa, insulto; desagravar.}{de.sen.tai.par}{\verboinum{1}}
\verb{desentalar}{}{}{}{}{v.t.}{Tirar a tala.}{de.sen.ta.lar}{0}
\verb{desentalar}{}{}{}{}{}{Soltar uma pessoa ou uma coisa presa num lugar apertado; desvencilhar.}{de.sen.ta.lar}{\verboinum{1}}
\verb{desentediar}{}{}{}{}{v.t.}{Livrar de tédio, de fastio; alegrar, distrair.}{de.sen.te.di.ar}{\verboinum{6}}
\verb{desentender}{ê}{}{}{}{v.t.}{Não entender, não aprender pela razão.}{de.sen.ten.der}{0}
\verb{desentender}{ê}{}{}{}{}{Fingir não entender.}{de.sen.ten.der}{0}
\verb{desentender}{ê}{}{}{}{v.pron.}{Pôr"-se em desavença; discordar.}{de.sen.ten.der}{\verboinum{12}}
\verb{desentendido}{}{}{}{}{adj.}{Que não entendeu ou fingiu não haver entendido.}{de.sen.ten.di.do}{0}
\verb{desentendimento}{}{}{}{}{s.m.}{Falta de entendimento, de percepção.}{de.sen.ten.di.men.to}{0}
\verb{desentendimento}{}{}{}{}{}{Briga, desavença.}{de.sen.ten.di.men.to}{0}
\verb{desenterrar}{}{}{}{}{v.t.}{Tirar pessoa ou coisa de dentro da terra.}{de.sen.ter.rar}{0}
\verb{desenterrar}{}{}{}{}{}{Tirar do esquecimento.}{de.sen.ter.rar}{\verboinum{1}}
\verb{desentoado}{}{}{}{}{adj.}{Que desentoou; fora do tom; desafinado.}{de.sen.to.a.do}{0}
\verb{desentoar}{}{}{}{}{v.t.}{Desafinar.}{de.sen.to.ar}{\verboinum{7}}
\verb{desentocar}{}{}{}{}{v.t.}{Fazer sair da toca ou da cova; desencovar.}{de.sen.to.car}{0}
\verb{desentocar}{}{Por ext.}{}{}{}{Fazer sair do isolamento, de local onde se está habitualmente encerrado.}{de.sen.to.car}{\verboinum{2}}
\verb{desentorpecer}{ê}{}{}{}{v.t.}{Fazer voltar o movimento e a sensibilidade de parte do corpo.}{de.sen.tor.pe.cer}{\verboinum{15}}
\verb{desentortar}{}{}{}{}{v.t.}{Endireitar o que está torto, curvado; aprumar.}{de.sen.tor.tar}{0}
\verb{desentortar}{}{Fig.}{}{}{}{Corrigir, emendar.}{de.sen.tor.tar}{\verboinum{1}}
\verb{desentranhar}{}{}{}{}{v.t.}{Tirar alguma coisa do ventre.}{de.sen.tra.nhar}{0}
\verb{desentranhar}{}{}{}{}{}{Arrancar o intestino de um ser vivo; estripar.}{de.sen.tra.nhar}{0}
\verb{desentranhar}{}{}{}{}{}{Retirar do seu íntimo; relembrar.}{de.sen.tra.nhar}{\verboinum{1}}
\verb{desentristecer}{ê}{}{}{}{v.t.}{Fazer perder a tristeza; reanimar, alegrar.}{de.sen.tris.te.cer}{\verboinum{15}}
\verb{desentrosamento}{}{}{}{}{s.m.}{Ato ou efeito de desentrosar; desligamento.}{de.sen.tro.sa.men.to}{0}
\verb{desentrosamento}{}{}{}{}{}{Falta de encaixe.}{de.sen.tro.sa.men.to}{0}
\verb{desentrosamento}{}{}{}{}{}{Divergência de opinião; desentendimento.}{de.sen.tro.sa.men.to}{0}
\verb{desentulhar}{}{}{}{}{v.t.}{Tirar da tulha.}{de.sen.tu.lhar}{0}
\verb{desentulhar}{}{}{}{}{}{Remover o entulho, o lixo; desobstruir.}{de.sen.tu.lhar}{\verboinum{1}}
\verb{desentupir}{}{}{}{}{v.t.}{Tirar o que impede a passagem de um líquido por algum lugar.}{de.sen.tu.pir}{\verboinum{18}}
\verb{desenvencilhar}{}{}{}{}{}{Var. de \textit{desvencilhar}.}{de.sen.ven.ci.lhar}{0}
\verb{desenvenenar}{}{}{}{}{v.t.}{Anular ou curar os efeitos do veneno; desintoxicar.}{de.sen.ve.ne.nar}{0}
\verb{desenvenenar}{}{}{}{}{}{Tornar não venenoso, tirar o veneno.}{de.sen.ve.ne.nar}{\verboinum{1}}
\verb{desenvolto}{ô}{}{}{}{adj.}{Que se comporta sem acanhamento; desinibido.}{de.sen.vol.to}{0}
\verb{desenvoltura}{}{}{}{}{s.f.}{Ausência de acanhamento; desembaraço, desinibição.}{de.sen.vol.tu.ra}{0}
\verb{desenvolver}{ê}{}{}{}{v.t.}{Fazer um ser vivo crescer.}{de.sen.vol.ver}{0}
\verb{desenvolver}{ê}{}{}{}{}{Fazer alguma coisa progredir, melhorar, aumentar.}{de.sen.vol.ver}{\verboinum{12}}
\verb{desenvolvido}{}{}{}{}{adj.}{Que se desenvolveu; crescido, forte.}{de.sen.vol.vi.do}{0}
\verb{desenvolvido}{}{}{}{}{}{Adiantado, avançado.}{de.sen.vol.vi.do}{0}
\verb{desenvolvimento}{}{}{}{}{s.m.}{Ato ou efeito de desenvolver; crescimento.}{de.sen.vol.vi.men.to}{0}
\verb{desenvolvimento}{}{}{}{}{}{Adiantamento, progresso.}{de.sen.vol.vi.men.to}{0}
\verb{desenxabido}{ch}{}{}{}{adj.}{Sem sabor; insípido.}{de.sen.xa.bi.do}{0}
\verb{desenxabido}{ch}{}{}{}{}{Sem graça ou sem animação; monótono.}{de.sen.xa.bi.do}{0}
\verb{desenxavido}{ch}{}{}{}{}{Var. de \textit{desenxabido}.}{de.sen.xa.vi.do}{0}
\verb{desequilibrado}{}{}{}{}{adj.}{Que se desequilibrou; falta de equilíbrio físico, de regularidade entre as partes; irregular.}{de.se.qui.li.bra.do}{0}
\verb{desequilibrado}{}{Fig.}{}{}{}{Desprovido de harmonia, de similaridade; desproporcional.}{de.se.qui.li.bra.do}{0}
\verb{desequilibrado}{}{Pop.}{}{}{}{Diz"-se daquele que não tem ou perdeu seu equilíbrio mental; alienado, louco.}{de.se.qui.li.bra.do}{0}
\verb{desequilibrar}{}{}{}{}{v.t.}{Fazer uma pessoa ou coisa perder o equilíbrio.}{de.se.qui.li.brar}{0}
\verb{desequilibrar}{}{}{}{}{}{Fazer alguém perder a estabilidade mental ou emocional.}{de.se.qui.li.brar}{\verboinum{1}}
\verb{desequilíbrio}{}{}{}{}{s.m.}{Falta de equilíbrio.}{de.se.qui.lí.brio}{0}
\verb{desequilíbrio}{}{}{}{}{}{Perturbação mental.}{de.se.qui.lí.brio}{0}
\verb{deserção}{}{}{"-ões}{}{s.f.}{Ato ou efeito de desertar; extinção dos habitantes; despovoação.}{de.ser.ção}{0}
\verb{deserção}{}{}{"-ões}{}{}{Abandono especialmente de corporação militar.}{de.ser.ção}{0}
\verb{deserção}{}{}{"-ões}{}{}{Afastamento de determinada coisa a que se estava ligado por dever ou por laço de natureza particular; fuga, renúncia.}{de.ser.ção}{0}
\verb{deserdado}{}{}{}{}{adj.}{Diz"-se de indivíduo que foi privado do direito de herança.}{de.ser.da.do}{0}
\verb{deserdado}{}{}{}{}{}{Diz"-se de indivíduo que foi privado de bens, vantagens ou qualidades; desfavorecido.}{de.ser.da.do}{0}
\verb{deserdar}{}{}{}{}{v.t.}{Retirar o nome de alguém da lista de herdeiros.}{de.ser.dar}{0}
\verb{deserdar}{}{Por ext.}{}{}{}{Privar de certos bens ou vantagens de natureza diversa; desfavorecer.}{de.ser.dar}{\verboinum{1}}
\verb{desertar}{}{}{}{}{v.t.}{Tornar deserto, ermo; abandonar, despovoar.}{de.ser.tar}{0}
\verb{desertar}{}{}{}{}{}{Fugir do serviço militar.}{de.ser.tar}{0}
\verb{desertar}{}{}{}{}{}{Desistir de alguma coisa; abandonar, renunciar.}{de.ser.tar}{\verboinum{1}}
\verb{desértico}{}{}{}{}{adj.}{Relativo a deserto.}{de.sér.ti.co}{0}
\verb{desértico}{}{}{}{}{}{Despovoado.}{de.sér.ti.co}{0}
\verb{desertificação}{}{}{"-ões}{}{s.f.}{Transformação de uma região fértil em deserto pela ação de fatores naturais ou humanos.}{de.ser.ti.fi.ca.ção}{0}
\verb{deserto}{é}{}{}{}{s.m.}{Grande extensão de terra muito seca, sem plantas ou de vegetação muito pobre, coberta de areia.}{de.ser.to}{0}
\verb{deserto}{é}{}{}{}{adj.}{Despovoado.}{de.ser.to}{0}
\verb{deserto}{é}{}{}{}{}{Vazio, ermo.}{de.ser.to}{0}
\verb{desertor}{ô}{}{}{}{adj.}{Diz"-se de indivíduo que pratica a deserção.}{de.ser.tor}{0}
\verb{desertor}{ô}{}{}{}{}{Diz"-se de militar que abandona as fileiras do exército.}{de.ser.tor}{0}
\verb{desesperação}{}{}{"-ões}{}{s.f.}{Desespero.}{de.ses.pe.ra.ção}{0}
\verb{desesperado}{}{}{}{}{adj.}{Que perdeu a esperança; entregue ao desespero.}{de.ses.pe.ra.do}{0}
\verb{desesperado}{}{}{}{}{}{Arrebatado, precipitado.}{de.ses.pe.ra.do}{0}
\verb{desesperado}{}{}{}{}{s.m.}{Indivíduo que perdeu a esperança.}{de.ses.pe.ra.do}{0}
\verb{desesperança}{}{}{}{}{s.f.}{Falta de confiança no futuro; desânimo.}{de.ses.pe.ran.ça}{0}
\verb{desesperançar}{}{}{}{}{v.t.}{Fazer perder a esperança.}{de.ses.pe.ran.çar}{\verboinum{3}}
\verb{desesperar}{}{}{}{}{v.t.}{Tirar a esperança de alguém; angustiar, desalentar, desanimar.}{de.ses.pe.rar}{0}
\verb{desesperar}{}{}{}{}{v.i.}{Perder a esperança.}{de.ses.pe.rar}{0}
\verb{desesperar}{}{}{}{}{v.pron.}{Enraivecer"-se; encolerizar"-se.   }{de.ses.pe.rar}{\verboinum{1}}
\verb{desespero}{ê}{}{}{}{s.m.}{Estado de grande aflição; angústia, desânimo.}{de.ses.pe.ro}{0}
\verb{desestabilizar}{}{}{}{}{v.t.}{Fazer perder a estabilidade, a segurança.}{de.ses.ta.bi.li.zar}{\verboinum{1}}
\verb{desestima}{}{}{}{}{s.f.}{Falta de estima, de amor"-próprio; desamor, menosprezo.}{de.ses.ti.ma}{0}
\verb{desestimar}{}{}{}{}{v.t.}{Não estimar; depreciar, desprezar.}{de.ses.ti.mar}{\verboinum{1}}
\verb{desestimular}{}{}{}{}{v.t.}{Levar alguém a perder a vontade de fazer alguma coisa; desalentar, desanimar, desencorajar. (\textit{As notas baixas desestimulam o aluno a estudar.})}{de.ses.ti.mu.lar}{\verboinum{1}}
\verb{desestímulo}{}{}{}{}{s.m.}{Falta ou perda de estímulo; desencorajamento.}{de.ses.tí.mu.lo}{0}
\verb{desestruturar}{}{}{}{}{v.t.}{Desfazer a estrutura, a sustentação; desmanchar, desorganizar.}{de.ses.tru.tu.rar}{0}
\verb{desestruturar}{}{Por ext.}{}{}{}{Fazer perder o referencial; abalar, perturbar.}{de.ses.tru.tu.rar}{\verboinum{1}}
\verb{desfaçatez}{ê}{}{}{}{s.f.}{Falta de vergonha; cinismo.}{des.fa.ça.tez}{0}
\verb{desfalcado}{}{}{}{}{adj.}{Que se desfalcou; reduzido, diminuído, incompleto.}{des.fal.ca.do}{0}
\verb{desfalcar}{}{}{}{}{v.t.}{Tirar parte de alguma coisa; diminuir, reduzir; subtrair.}{des.fal.car}{0}
\verb{desfalcar}{}{}{}{}{}{Roubar.}{des.fal.car}{\verboinum{2}}
\verb{desfalecer}{ê}{}{}{}{v.i.}{Perder os sentidos; desmaiar.}{des.fa.le.cer}{0}
\verb{desfalecer}{ê}{}{}{}{}{Diminuir de intensidade; enfraquecer.}{des.fa.le.cer}{\verboinum{15}}
\verb{desfalecimento}{}{}{}{}{s.m.}{Perda de ânimo; esmorecimento.}{des.fa.le.ci.men.to}{0}
\verb{desfalecimento}{}{}{}{}{}{Desmaio.}{des.fa.le.ci.men.to}{0}
\verb{desfalque}{}{}{}{}{s.m.}{Perda de parte de alguma coisa; diminuição, redução.}{des.fal.que}{0}
\verb{desfalque}{}{}{}{}{}{Desvio de parte do dinheiro de algum lugar; rombo.}{des.fal.que}{0}
\verb{desfastio}{}{}{}{}{s.m.}{Ausência de fastio; apetite.}{des.fas.ti.o}{0}
\verb{desfastio}{}{}{}{}{}{Prazer de viver.}{des.fas.ti.o}{0}
\verb{desfavelar}{}{}{}{}{v.t.}{Tirar a favela de um lugar.}{des.fa.ve.lar}{0}
\verb{desfavelar}{}{}{}{}{}{Retirar, mudar de favela.}{des.fa.ve.lar}{\verboinum{1}}
\verb{desfavor}{ô}{}{}{}{s.m.}{Desprezo, desconsideração.}{des.fa.vor}{0}
\verb{desfavor}{ô}{}{}{}{}{Malquerença, inimizade.}{des.fa.vor}{0}
\verb{desfavorável}{}{}{"-eis}{}{adj.2g.}{Que não é favorável; adverso, contrário.}{des.fa.vo.rá.vel}{0}
\verb{desfavorecer}{ê}{}{}{}{v.t.}{Privar alguém de algo que lhe seria vantajoso; prejudicar.}{des.fa.vo.re.cer}{0}
\verb{desfavorecer}{ê}{}{}{}{}{Contrariar os interesses; desajudar.}{des.fa.vo.re.cer}{\verboinum{15}}
\verb{desfazer}{ê}{}{}{}{v.t.}{Desmanchar o que foi feito.}{des.fa.zer}{0}
\verb{desfazer}{ê}{}{}{}{}{Diminuir a importância de uma pessoa ou coisa; desprezar.}{des.fa.zer}{\verboinum{42}}
\verb{desfechar}{}{}{}{}{v.t.}{Disparar arma de fogo; descarregar.}{des.fe.char}{0}
\verb{desfechar}{}{}{}{}{}{Dar um golpe; desferir.}{des.fe.char}{\verboinum{1}}
\verb{desfecho}{ê}{}{}{}{s.m.}{Acontecimento final; conclusão, desenlace.}{des.fe.cho}{0}
\verb{desfeita}{ê}{}{}{}{s.f.}{Ação que ofende a outra pessoa; grosseria, insulto, ofensa.}{des.fei.ta}{0}
\verb{desfeitear}{}{}{}{}{v.t.}{Fazer desfeita; insultar, ofender.}{des.fei.te.ar}{\verboinum{4}}
\verb{desfeito}{ê}{}{}{}{adj.}{Que desfez; desmanchado.}{des.fei.to}{0}
\verb{desferir}{}{}{}{}{v.t.}{Lançar um objeto para atingir pessoa, animal ou coisa; arremessar, atirar.}{des.fe.rir}{0}
\verb{desferir}{}{}{}{}{}{Dar um golpe; desfechar.}{des.fe.rir}{\verboinum{18}}
\verb{desferrar}{}{}{}{}{v.t.}{Tirar a ferradura de um animal.}{des.fer.rar}{\verboinum{1}}
\verb{desfiar}{}{}{}{}{v.t.}{Puxar os fios de um pano; destecer.}{des.fi.ar}{0}
\verb{desfiar}{}{}{}{}{}{Puxar uma por uma as contas de terço ou rosário.}{des.fi.ar}{0}
\verb{desfiar}{}{}{}{}{}{Narrar.}{des.fi.ar}{\verboinum{6}}
\verb{desfibrado}{}{}{}{}{adj.}{Sem fibras.}{des.fi.bra.do}{0}
\verb{desfibrado}{}{}{}{}{}{Que perdeu a coragem; desanimado, fraco.}{des.fi.bra.do}{0}
\verb{desfibrado}{}{}{}{}{}{Desfiado.}{des.fi.bra.do}{0}
\verb{desfibrar}{}{}{}{}{v.t.}{Tirar as fibras.}{des.fi.brar}{0}
\verb{desfibrar}{}{}{}{}{}{Fazer perder a energia; enfraquecer.}{des.fi.brar}{0}
\verb{desfibrar}{}{}{}{}{}{Desfiar.}{des.fi.brar}{\verboinum{1}}
\verb{desfigurar}{}{}{}{}{v.t.}{Mudar o aspecto de uma pessoa ou coisa; alterar; deformar.}{des.fi.gu.rar}{\verboinum{1}}
\verb{desfilada}{}{}{}{}{s.f.}{Corrida impetuosa.}{des.fi.la.da}{0}
\verb{desfiladeiro}{ê}{}{}{}{s.m.}{Passagem estreita entre montanhas; garganta.}{des.fi.la.dei.ro}{0}
\verb{desfilar}{}{}{}{}{v.i.}{Andar ou marchar um após o outro.}{des.fi.lar}{0}
\verb{desfilar}{}{}{}{}{}{Apresentar"-se em passarela.}{des.fi.lar}{0}
\verb{desfilar}{}{}{}{}{v.t.}{Exibir algo como num desfile.}{des.fi.lar}{\verboinum{1}}
\verb{desfile}{}{}{}{}{s.m.}{Ato ou efeito de desfilar.}{des.fi.le}{0}
\verb{desfile}{}{}{}{}{}{Parada militar.}{des.fi.le}{0}
\verb{desfile}{}{}{}{}{}{Apresentação de coleção de moda.}{des.fi.le}{0}
\verb{desfile}{}{}{}{}{}{Apresentação de escola de samba.}{des.fi.le}{0}
\verb{desfitar}{}{}{}{}{v.t.}{Deixar de fitar; desviar os olhos, a vista.}{des.fi.tar}{\verboinum{1}}
\verb{desfloração}{}{}{"-ões}{}{s.f.}{Queda das flores.}{des.flo.ra.ção}{0}
\verb{desfloração}{}{}{"-ões}{}{}{Violação da virgindade.}{des.flo.ra.ção}{0}
\verb{desfloramento}{}{}{}{}{s.m.}{Desfloração.}{des.flo.ra.men.to}{0}
\verb{desflorar}{}{}{}{}{v.t.}{Fazer perder as flores; deflorar.}{des.flo.rar}{0}
\verb{desflorar}{}{}{}{}{}{Fazer uma mulher ter o seu primeiro ato sexual.}{des.flo.rar}{\verboinum{1}}
\verb{desflorestamento}{}{}{}{}{s.m.}{Destruição de floresta.}{des.flo.res.ta.men.to}{0}
\verb{desflorestamento}{}{}{}{}{}{Derrubada de árvores.}{des.flo.res.ta.men.to}{0}
\verb{desflorestar}{}{}{}{}{v.t.}{Pôr abaixo grande porção de árvores; desmatar.}{des.flo.res.tar}{\verboinum{1}}
\verb{desfocar}{}{}{}{}{v.t.}{Tornar turvo, fora de foco.}{des.fo.car}{0}
\verb{desfocar}{}{}{}{}{}{Tirar o foco.}{des.fo.car}{\verboinum{2}}
\verb{desfolhamento}{}{}{}{}{s.m.}{Perda ou retirada de folhas ou pétalas.}{des.fo.lha.men.to}{0}
\verb{desfolhar}{}{}{}{}{v.t.}{Tirar as folhas ou as pétalas.}{des.fo.lhar}{\verboinum{1}}
\verb{desforço}{ô}{}{}{}{s.m.}{Ato ou efeito de desforçar; desafronta, vingança.}{des.for.ço}{0}
\verb{desforço}{ô}{Jur.}{}{}{}{Medida que visa conservar ou integrar alguém na posse de bem que se lhe tirou de forma injusta ou violenta.}{des.for.ço}{0}
\verb{desforra}{ó}{}{}{}{s.f.}{Ato ou efeito de desforrar; vingança.}{des.for.ra}{0}
\verb{desforrar}{}{}{}{}{v.t.}{Tirar o forro.}{des.for.rar}{0}
\verb{desforrar}{}{}{}{}{}{Fazer alguma coisa para responder a uma ofensa ou derrota; vingar.}{des.for.rar}{\verboinum{1}}
\verb{desfraldar}{}{}{}{}{v.t.}{Soltar bandeira ou vela de barco ao vento; arvorar.}{des.fral.dar}{\verboinum{1}}
\verb{desfranzir}{}{}{}{}{v.t.}{Desmanchar o franzido.}{des.fran.zir}{0}
\verb{desfranzir}{}{}{}{}{}{Desenrugar a testa, o semblante.}{des.fran.zir}{\verboinum{18}}
\verb{desfrutar}{}{}{}{}{v.t.}{Ter alguma coisa boa para seu uso; gozar, usufruir.}{des.fru.tar}{\verboinum{1}}
\verb{desfrutável}{}{}{"-eis}{}{adj.2g.}{Que se pode desfrutar.}{des.fru.tá.vel}{0}
\verb{desfrutável}{}{}{"-eis}{}{}{Que se presta a zombarias.}{des.fru.tá.vel}{0}
\verb{desfrute}{}{}{}{}{s.m.}{Ato ou efeito de desfrutar; gozo.}{des.fru.te}{0}
\verb{desfruto}{}{}{}{}{s.m.}{Ato ou efeito de desfrutar.}{des.fru.to}{0}
\verb{desfruto}{}{}{}{}{}{Ato de aproveitar oportunidade ou vantagem; fruição, usufruto.}{des.fru.to}{0}
\verb{desfruto}{}{Pop.}{}{}{}{Exposição ao ridículo; escárnio, zombaria.}{des.fru.to}{0}
\verb{desgalhar}{}{}{}{}{v.t.}{Cortar os galhos.}{des.ga.lhar}{\verboinum{1}}
\verb{desgarrar}{}{}{}{}{v.t.}{Levar alguma coisa para o lugar errado; desviar.}{des.gar.rar}{0}
\verb{desgarrar}{}{}{}{}{}{Afastar de alguém, de um grupo ou de rebanho; separar.}{des.gar.rar}{\verboinum{1}}
\verb{desgastar}{}{}{}{}{v.t.}{Gastar alguma coisa pelo uso; consumir.}{des.gas.tar}{\verboinum{1}}
\verb{desgaste}{}{}{}{}{s.m.}{Ato ou efeito de desgastar.}{des.gas.te}{0}
\verb{desgaste}{}{}{}{}{}{Alteração ou redução da forma, por fricção ou atrito; corrosão.}{des.gas.te}{0}
\verb{desgaste}{}{}{}{}{}{Envelhecimento, ruína.}{des.gas.te}{0}
\verb{desgaste}{}{}{}{}{}{Abatimento, enfraquecimento.}{des.gas.te}{0}
\verb{desgelar}{}{}{}{}{}{Var. de \textit{degelar}.}{des.ge.lar}{0}
\verb{desgostar}{}{}{}{}{v.t.}{Causar desgosto a alguém; aborrecer, desagradar.}{des.gos.tar}{\verboinum{1}}
\verb{desgosto}{ô}{}{}{}{s.m.}{Falta de prazer; desprazer, tristeza, aborrecimento.}{des.gos.to}{0}
\verb{desgostoso}{ô}{}{"-osos ⟨ó⟩}{"-osa ⟨ó⟩}{adj.}{Que sente desgosto; descontente, triste.}{des.gos.to.so}{0}
\verb{desgostoso}{ô}{}{"-osos ⟨ó⟩}{"-osa ⟨ó⟩}{}{Que denota desgosto, descontentamento.}{des.gos.to.so}{0}
\verb{desgostoso}{ô}{}{"-osos ⟨ó⟩}{"-osa ⟨ó⟩}{}{Que tem gosto ou sabor desagradável.}{des.gos.to.so}{0}
\verb{desgovernar}{}{}{}{}{v.t.}{Governar mal.}{des.go.ver.nar}{0}
\verb{desgovernar}{}{}{}{}{}{Desviar do caminho.}{des.go.ver.nar}{0}
\verb{desgovernar}{}{}{}{}{}{Desperdiçar, gastar.}{des.go.ver.nar}{0}
\verb{desgovernar}{}{Bras.}{}{}{v.pron.}{Desorientar"-se, desnortear"-se.}{des.go.ver.nar}{\verboinum{1}}
\verb{desgoverno}{ê}{}{}{}{s.m.}{Mau governo.}{des.go.ver.no}{0}
\verb{desgoverno}{ê}{}{}{}{}{Desperdício.}{des.go.ver.no}{0}
\verb{desgoverno}{ê}{}{}{}{}{Desregramento, desorientação.}{des.go.ver.no}{0}
\verb{desgraça}{}{}{}{}{s.f.}{Acontecimento desastroso ou fatal; má sorte, desventura.}{des.gra.ça}{0}
\verb{desgraça}{}{}{}{}{}{Penúria, miséria.}{des.gra.ça}{0}
\verb{desgraçado}{}{}{}{}{adj.}{Infeliz, desventurado.}{des.gra.ça.do}{0}
\verb{desgraçado}{}{}{}{}{}{Miserável, muito pobre.}{des.gra.ça.do}{0}
\verb{desgraçado}{}{}{}{}{}{Vil, desprezível, infame.}{des.gra.ça.do}{0}
\verb{desgraçado}{}{Bras.}{}{}{}{Travesso, levado, arteiro.}{des.gra.ça.do}{0}
\verb{desgraçar}{}{}{}{}{v.t.}{Tornar desgraçado.}{des.gra.çar}{\verboinum{3}}
\verb{desgraceira}{ê}{}{}{}{s.f.}{Desgraça contínua.}{des.gra.cei.ra}{0}
\verb{desgracioso}{ô}{}{"-osos ⟨ó⟩}{"-osa ⟨ó⟩}{adj.}{Que não tem graça; deselegante.}{des.gra.ci.o.so}{0}
\verb{desgrenhado}{}{}{}{}{adj.}{Diz"-se de cabelo despenteado, emaranhado.}{des.gre.nha.do}{0}
\verb{desgrenhar}{}{}{}{}{v.t.}{Emaranhar, despentear.}{des.gre.nhar}{\verboinum{1}}
\verb{desgrudar}{}{}{}{}{v.t.}{Soltar aquilo que estava grudado; descolar.}{des.gru.dar}{0}
\verb{desgrudar}{}{}{}{}{}{Afastar"-se, tornar"-se independente.}{des.gru.dar}{\verboinum{1}}
\verb{desguarnecer}{ê}{}{}{}{v.t.}{Tirar a guarnição.}{des.guar.ne.cer}{0}
\verb{desguarnecer}{ê}{}{}{}{}{Privar de forças militares ou de munição.}{des.guar.ne.cer}{0}
\verb{desguarnecer}{ê}{}{}{}{}{Tirar os enfeites; desornar.}{des.guar.ne.cer}{\verboinum{15}}
\verb{desguiar}{}{Pop.}{}{}{v.i.}{Ir embora.}{des.gui.ar}{\verboinum{1}}
\verb{desiderato}{}{}{}{}{s.m.}{Objeto do desejo, aquilo que se deseja.}{de.si.de.ra.to}{0}
\verb{desídia}{}{}{}{}{s.f.}{Preguiça, indolência.}{de.sí.dia}{0}
\verb{desídia}{}{}{}{}{}{Negligência, descaso.}{de.sí.dia}{0}
\verb{desidioso}{ô}{}{"-osos ⟨ó⟩}{"-osa ⟨ó⟩}{adj.}{Relativo a desídia.}{de.si.di.o.so}{0}
\verb{desidratação}{}{}{"-ões}{}{s.f.}{Ato ou efeito de desidratar.}{de.si.dra.ta.ção}{0}
\verb{desidratação}{}{Med.}{"-ões}{}{}{Estado patológico resultante da perda não compensada de água do organismo.}{de.si.dra.ta.ção}{0}
\verb{desidratado}{}{}{}{}{adj.}{Que se desidratou.}{de.si.dra.ta.do}{0}
\verb{desidratar}{}{}{}{}{v.t.}{Extrair, através de processos específicos, a água de um composto.}{de.si.dra.tar}{0}
\verb{desidratar}{}{Med.}{}{}{}{Causar ou entrar em estado de desidratação.}{de.si.dra.tar}{\verboinum{1}}
\verb{design}{}{}{}{}{s.m.}{O projeto visual de um objeto qualquer.}{\textit{design}}{0}
\verb{designação}{}{}{"-ões}{}{s.f.}{Ato ou efeito de designar.}{de.sig.na.ção}{0}
\verb{designação}{}{}{"-ões}{}{}{Denominação.}{de.sig.na.ção}{0}
\verb{designar}{}{}{}{}{v.t.}{Indicar, nomear.}{de.sig.nar}{0}
\verb{designar}{}{}{}{}{}{Ser símbolo de; representar.}{de.sig.nar}{0}
\verb{designar}{}{}{}{}{}{Determinar, marcar, assinalar.}{de.sig.nar}{\verboinum{1}}
\verb{designativo}{}{}{}{}{adj.}{Que designa.}{de.sig.na.ti.vo}{0}
\verb{designer}{}{}{}{}{s.m.}{Profissional que cria o \textit{design} de objetos, utensílios, publicações.}{\textit{designer}}{0}
\verb{desígnio}{}{}{}{}{s.m.}{Propósito, intenção, projeto, vontade.}{de.síg.nio}{0}
\verb{desigual}{}{}{"-ais}{}{adj.2g.}{Diferente, diverso.}{de.si.gual}{0}
\verb{desigual}{}{}{"-ais}{}{}{Irregular.}{de.si.gual}{0}
\verb{desigual}{}{}{"-ais}{}{}{Variável, instável, inconstante.}{de.si.gual}{0}
\verb{desigual}{}{}{"-ais}{}{}{Desproporcional, desequilibrado.}{de.si.gual}{0}
\verb{desigualar}{}{}{}{}{v.t.}{Estabelecer a diferença; distinguir, diferenciar.}{de.si.gua.lar}{0}
\verb{desigualar}{}{}{}{}{}{Tornar desigual.}{de.si.gua.lar}{0}
\verb{desigualar}{}{}{}{}{}{Ser desigual.}{de.si.gua.lar}{\verboinum{1}}
\verb{desigualdade}{}{}{}{}{s.f.}{Qualidade de desigual.}{de.si.gual.da.de}{0}
\verb{desiludido}{}{}{}{}{adj.}{Que se desiludiu; decepcionado.}{de.si.lu.di.do}{0}
\verb{desiludir}{}{}{}{}{v.t.}{Fazer perder ilusões; desenganar.}{de.si.lu.dir}{0}
\verb{desiludir}{}{}{}{}{}{Causar decepção.}{de.si.lu.dir}{\verboinum{18}}
\verb{desilusão}{}{}{"-ões}{}{s.f.}{Ato ou efeito de desiludir; decepção, desengano.}{de.si.lu.são}{0}
\verb{desimpedido}{}{}{}{}{adj.}{Livre, desobstruído.}{de.sim.pe.di.do}{0}
\verb{desimpedido}{}{}{}{}{}{Não sujeito a compromisso.}{de.sim.pe.di.do}{0}
\verb{desimpedir}{}{}{}{}{v.t.}{Tirar o impedimento; desobstruir.}{de.sim.pe.dir}{\verboinum{20}}
\verb{desinchar}{}{}{}{}{v.t.}{Desfazer ou diminuir o inchaço.}{de.sin.char}{0}
\verb{desinchar}{}{}{}{}{}{Tirar o orgulho, a vaidade; humilhar.}{de.sin.char}{\verboinum{1}}
\verb{desincompatibilizar}{}{}{}{}{v.t.}{Eliminar a causa da incompatibilidade.}{de.sin.com.pa.ti.bi.li.zar}{0}
\verb{desincompatibilizar}{}{}{}{}{v.pron.}{Deixar de estar incompatibilizado.}{de.sin.com.pa.ti.bi.li.zar}{\verboinum{1}}
\verb{desincorporar}{}{}{}{}{v.t.}{Separar, desligar.}{de.sin.cor.po.rar}{0}
\verb{desincorporar}{}{}{}{}{}{Tirar de uma corporação.}{de.sin.cor.po.rar}{\verboinum{1}}
\verb{desincumbir}{}{}{}{}{v.t.}{Fazer uma obrigação, uma incumbência; cumprir.}{de.sin.cum.bir}{\verboinum{18}}
\verb{desindexar}{cs}{}{}{}{v.t.}{Desfazer a indexação.}{de.sin.de.xar}{0}
\verb{desindexar}{cs}{}{}{}{}{Eliminar o reajuste por índice.}{de.sin.de.xar}{0}
\verb{desindexar}{cs}{}{}{}{}{Eliminar a relação entre valores.}{de.sin.de.xar}{\verboinum{1}}
\verb{desinência}{}{Gram.}{}{}{s.f.}{Parte final da palavra, que pode indicar o gênero e o número dos nomes, e o tempo, o modo, a pessoa e o número dos verbos.}{de.si.nên.cia}{0}
\verb{desinência}{}{}{}{}{}{Fim, extremidade.}{de.si.nên.cia}{0}
\verb{desinfecção}{}{}{"-ões}{}{s.f.}{Ato ou efeito de desinfeccionar.}{de.sin.fec.ção}{0}
\verb{desinfeliz}{}{Pop.}{}{}{adj.}{Infeliz.}{de.sin.fe.liz}{0}
\verb{desinfestar}{}{}{}{}{v.t.}{Eliminar aquilo que infesta.}{de.sin.fes.tar}{\verboinum{1}}
\verb{desinfetante}{}{}{}{}{adj.2g.}{Que desinfeta.}{de.sin.fe.tan.te}{0}
\verb{desinfetar}{}{}{}{}{v.t.}{Eliminar aquilo que infecta, que causa infecção; sanear.}{de.sin.fe.tar}{\verboinum{1}}
\verb{desinflação}{}{Econ.}{"-ões}{}{s.f.}{Redução dos fatores que provocam inflação.}{de.sin.fla.ção}{0}
\verb{desinflacionar}{}{Econ.}{}{}{v.t.}{Pôr em prática uma política de desinflação.}{de.sin.fla.ci.o.nar}{\verboinum{1}}
\verb{desinflamar}{}{}{}{}{v.t.}{Cessar a inflamação.}{de.sin.fla.mar}{\verboinum{1}}
\verb{desinformação}{}{}{"-ões}{}{s.f.}{Ato ou efeito de desinformar.}{de.sin.for.ma.ção}{0}
\verb{desinformação}{}{}{"-ões}{}{}{Informação deformada, falseada ou tendenciosa que visa provocar no destinatário uma opinião predeterminada e artificial.}{de.sin.for.ma.ção}{0}
\verb{desinformado}{}{}{}{}{adj.}{Mal informado ou sem nenhuma informação.}{de.sin.for.ma.do}{0}
\verb{desinformar}{}{}{}{}{v.t.}{Omitir informações ou fornecer informações falsas.}{de.sin.for.mar}{\verboinum{1}}
\verb{desinibido}{}{}{}{}{adj.}{Sem inibições.}{de.si.ni.bi.do}{0}
\verb{desinibir}{}{}{}{}{v.t.}{Eliminar aquilo que inibe.}{de.si.ni.bir}{0}
\verb{desinibir}{}{}{}{}{}{Tirar a timidez.}{de.si.ni.bir}{\verboinum{18}}
\verb{desinquietar}{}{}{}{}{v.t.}{Tirar do sossego; inquietar, incomodar.}{de.sin.qui.e.tar}{\verboinum{1}}
\verb{desinquieto}{é}{}{}{}{adj.}{Inquieto, agitado.}{de.sin.qui.e.to}{0}
\verb{desinsofrido}{}{}{}{}{adj.}{Inquieto, insofrido, impaciente.}{de.sin.so.fri.do}{0}
\verb{desintegração}{}{}{"-ões}{}{s.f.}{Ato ou efeito de desintegrar.}{de.sin.te.gra.ção}{0}
\verb{desintegração}{}{Fís.}{"-ões}{}{}{Processo, espontâneo ou provocado, em que um núcleo atômico emite uma partícula.}{de.sin.te.gra.ção}{0}
\verb{desintegrar}{}{}{}{}{v.t.}{Desmanchar alguma coisa em partes muito pequenas; desfazer, destruir, decompor.}{de.sin.te.grar}{\verboinum{1}}
\verb{desinteligência}{}{}{}{}{s.f.}{Discrepância entre pontos de vista; desacordo, desentendimento.}{de.sin.te.li.gên.cia}{0}
\verb{desinteligência}{}{}{}{}{}{Falta de amizade; hostilidade, malquerença.}{de.sin.te.li.gên.cia}{0}
\verb{desinteligência}{}{}{}{}{}{Falta de inteligência.}{de.sin.te.li.gên.cia}{0}
\verb{desinteressado}{}{}{}{}{adj.}{Que perdeu o interesse; indiferente.}{de.sin.te.res.sa.do}{0}
\verb{desinteressado}{}{}{}{}{}{Em que se nota a vontade de fazer alguma coisa sem pedir pagamento; desprendido, generoso.}{de.sin.te.res.sa.do}{0}
\verb{desinteressante}{}{}{}{}{adj.2g.}{Que deixa de agradar por não ser do gosto da pessoa.}{de.sin.te.res.san.te}{0}
\verb{desinteressar}{}{}{}{}{v.t.}{Fazer alguém perder o interesse por pessoa ou coisa.}{de.sin.te.res.sar}{\verboinum{1}}
\verb{desinteresse}{ê}{}{}{}{s.m.}{Inexistência de curiosidade, de gosto por alguém ou alguma coisa; indiferença, imparcialidade.}{de.sin.te.res.se}{0}
\verb{desinteresse}{ê}{}{}{}{}{Falta de empenho, de cuidado; descuido, negligência.}{de.sin.te.res.se}{0}
\verb{desinteresse}{ê}{}{}{}{}{Ausênia de interesse com relação a bens materiais.}{de.sin.te.res.se}{0}
\verb{desintoxicação}{cs}{}{"-ões}{}{s.f.}{Eliminação de toxinas ou de elementos tóxicos do organismo.}{de.sin.to.xi.ca.ção}{0}
\verb{desintoxicar}{cs}{}{}{}{v.t.}{Fazer sair o veneno que se acha no corpo de pessoa ou animal.}{de.sin.to.xi.car}{\verboinum{2}}
\verb{desintumescer}{ê}{}{}{}{v.t.}{Fazer o volume de alguma coisa ficar menor; desinchar.}{de.sin.tu.mes.cer}{\verboinum{15}}
\verb{desirmanado}{}{}{}{}{adj.}{Que se desuniu de coisa ou pessoa com que estava irmanado; desemparelhado, separado.}{de.sir.ma.na.do}{0}
\verb{desirmanar}{}{}{}{}{v.t.}{Desunir irmãos por motivo de briga, discórdia.}{de.sir.ma.nar}{0}
\verb{desirmanar}{}{}{}{}{}{Fazer romper laços de amizade; desentender.}{de.sir.ma.nar}{0}
\verb{desirmanar}{}{}{}{}{}{Tornar incompleto o que forma um jogo, um conjunto; desemparelhar.}{de.sir.ma.nar}{\verboinum{1}}
\verb{desistência}{}{}{}{}{s.f.}{Ato ou efeito de desistir; abdicação de algo que se desejava; abstinência, renúncia.}{de.sis.tên.cia}{0}
\verb{desistir}{}{}{}{}{v.t.}{Não querer mais; renunciar.}{de.sis.tir}{\verboinum{18}}
\verb{desjejuar}{}{}{}{}{v.i.}{Fazer a primeira refeição do dia.}{des.je.ju.ar}{0}
\verb{desjejuar}{}{}{}{}{v.t.}{Dar de comer.}{des.je.ju.ar}{\verboinum{1}}
\verb{desjejum}{}{}{"-uns}{}{s.m.}{A primeira refeição do dia.}{des.je.jum}{0}
\verb{desjungir}{}{}{}{}{v.t.}{Soltar de jugo ou canga; desatrelar.}{des.jun.gir}{0}
\verb{desjungir}{}{Por ext.}{}{}{}{Desligar, desunir, separar.}{des.jun.gir}{\verboinum{22}}
\verb{deslaçar}{}{}{}{}{v.t.}{Desprender o que se tinha enlaçado.}{des.la.çar}{\verboinum{3}}
\verb{deslacrar}{}{}{}{}{v.t.}{Remover o lacre.}{des.la.crar}{\verboinum{1}}
\verb{deslanchar}{}{}{}{}{v.t.}{Pôr em marcha, em atividade, geralmente de modo repentino  e brusco; arrancar.}{des.lan.char}{0}
\verb{deslanchar}{}{Fig.}{}{}{}{Dar seguimento; prosseguir.}{des.lan.char}{0}
\verb{deslanchar}{}{}{}{}{}{Fazer progredir; evoluir.}{des.lan.char}{0}
\verb{deslanchar}{}{}{}{}{v.i.}{Ir embora; partir.}{des.lan.char}{\verboinum{1}}
\verb{deslavado}{}{}{}{}{adj.}{Que se deslavou; cuja cor se perdeu; desbotado.}{des.la.va.do}{0}
\verb{deslavado}{}{Fig.}{}{}{}{De comportamento atrevido, petulante; descarado.}{des.la.va.do}{0}
\verb{desleal}{}{}{"-ais}{}{adj.2g.}{Que apresenta conduta traiçoeira; falso, desonesto.}{des.le.al}{0}
\verb{desleal}{}{}{"-ais}{}{}{Que é infiel.}{des.le.al}{0}
\verb{deslealdade}{}{}{}{}{s.f.}{Ausência de lealdade; falsidade, traição. }{des.le.al.da.de}{0}
\verb{desleitar}{}{}{}{}{v.t.}{Deixar de aleitar; desmamar.}{des.lei.tar}{0}
\verb{desleitar}{}{}{}{}{}{Retirar de peles e couros o excesso de tanino e gelatina.}{des.lei.tar}{\verboinum{1}}
\verb{desleixado}{ch}{}{}{}{adj.}{Que faz as coisas com desleixo; descuidado, negligente, relaxado.}{des.lei.xa.do}{0}
\verb{desleixado}{ch}{}{}{}{}{Em que há desleixo; descuidado.}{des.lei.xa.do}{0}
\verb{desleixar}{ch}{}{}{}{v.t.}{Tratar sem cuidado; descuidar, negligenciar.}{des.lei.xar}{\verboinum{1}}
\verb{desleixo}{ch}{}{}{}{s.m.}{Falta de ordem e cuidado; descuido, negligência, relaxamento.}{des.lei.xo}{0}
\verb{deslembrança}{}{}{}{}{s.f.}{Ato ou efeito de deslembrar; ausência de lembrança; esquecimento.}{des.lem.bran.ça}{0}
\verb{deslembrar}{}{}{}{}{v.t.}{Deixar de lembrar; esquecer.}{des.lem.brar}{\verboinum{1}}
\verb{desligado}{}{}{}{}{s.f.}{Que se encontra distante; desunido, afastado.}{des.li.ga.do}{0}
\verb{desligado}{}{}{}{}{}{Que não está ligado ou funcionando; apagado.}{des.li.ga.do}{0}
\verb{desligado}{}{}{}{}{}{Que é desatento ou distraído.}{des.li.ga.do}{0}
\verb{desligamento}{}{}{}{}{s.m.}{Ato ou efeito de desligar; separação.}{des.li.ga.men.to}{0}
\verb{desligamento}{}{}{}{}{}{Interrupção de funcionamento.}{des.li.ga.men.to}{0}
\verb{desligamento}{}{}{}{}{}{Ato de desvincular daquilo a que estava preso por obrigação, compromisso, laço afetivo etc.}{des.li.ga.men.to}{0}
\verb{desligamento}{}{}{}{}{}{Ato de destituir de um emprego, cargo ou função; exoneração.}{des.li.ga.men.to}{0}
\verb{desligamento}{}{}{}{}{}{Qualidade de quem é desligado, distraído.}{des.li.ga.men.to}{0}
\verb{desligar}{}{}{}{}{v.t.}{Desfazer a ligação; desjuntar, separar.}{des.li.gar}{0}
\verb{desligar}{}{}{}{}{}{Fazer deixar de funcionar; apagar.}{des.li.gar}{0}
\verb{desligar}{}{}{}{}{v.pron.}{Despreocupar"-se, distrair"-se.}{des.li.gar}{\verboinum{5}}
\verb{deslindar}{}{}{}{}{v.t.}{Descobrir os segredos de alguma coisa; destrinchar, esclarecer.}{des.lin.dar}{\verboinum{1}}
\verb{deslinde}{}{}{}{}{s.m.}{Ato ou efeito de deslindar; pesquisa, investigação.}{des.lin.de}{0}
\verb{deslizamento}{}{}{}{}{s.m.}{Deslocamento de terra.}{des.li.za.men.to}{0}
\verb{deslizar}{}{}{}{}{v.t.}{Fazer alguam coisa se mover em cima de outra.}{des.li.zar}{0}
\verb{deslizar}{}{}{}{}{}{Cometer falhas.}{des.li.zar}{\verboinum{1}}
\verb{deslize}{}{}{}{}{s.m.}{Erro feito por falta de atenção; descuido, engano, falha.}{des.li.ze}{0}
\verb{deslocado}{}{}{}{}{adj.}{Que mudou ou está fora de seu lugar.}{des.lo.ca.do}{0}
\verb{deslocado}{}{}{}{}{}{Fora de propósito.}{des.lo.ca.do}{0}
\verb{deslocado}{}{}{}{}{}{Desarticulado.}{des.lo.ca.do}{0}
\verb{deslocamento}{}{}{}{}{s.m.}{Ato ou efeito de deslocar.}{des.lo.ca.men.to}{0}
\verb{deslocamento}{}{}{}{}{}{Mudança de um lugar para outro.}{des.lo.ca.men.to}{0}
\verb{deslocamento}{}{}{}{}{}{Mudança de direção; desvio.}{des.lo.ca.men.to}{0}
\verb{deslocar}{}{}{}{}{v.t.}{Tirar do lugar em que se encontrava.}{des.lo.car}{0}
\verb{deslocar}{}{}{}{}{}{Fazer mudar de lugar; afastar; desviar.}{des.lo.car}{0}
\verb{deslocar}{}{Med.}{}{}{}{Fazer um osso do corpo sair do lugar em que se une a outro; desarticular; luxar.}{des.lo.car}{\verboinum{2}}
\verb{deslumbrado}{}{}{}{}{adj.}{Que se deslumbrou; maravilhado, fascinado.}{des.lum.bra.do}{0}
\verb{deslumbrado}{}{}{}{}{s.m.}{Indivíduo que por tudo se deslumbra ou encanta.}{des.lum.bra.do}{0}
\verb{deslumbramento}{}{}{}{}{s.m.}{Ofuscação momentânea causada por uma luz muito forte.}{des.lum.bra.men.to}{0}
\verb{deslumbramento}{}{Fig.}{}{}{}{Sedução, fascinação, encanto.}{des.lum.bra.men.to}{0}
\verb{deslumbrante}{}{}{}{}{adj.2g.}{Que turva a vista por excesso de luz ou brilho ou por outros fatores; ofuscante.}{des.lum.bran.te}{0}
\verb{deslumbrante}{}{Fig.}{}{}{}{Que maravilha, que impressiona por suas qualidades raras ou superiores; fascinante.}{des.lum.bran.te}{0}
\verb{deslumbrante}{}{}{}{}{}{Que revela suntuosidade; luxuoso.}{des.lum.bran.te}{0}
\verb{deslumbrar}{}{}{}{}{v.t.}{Fazer alguém ficar muito admirado; encantar, fascinar, maravilhar.}{des.lum.brar}{0}
\verb{deslumbrar}{}{}{}{}{}{Deixar alguém cego por alguns momentos com a luz direta nos olhos; ofuscar.}{des.lum.brar}{\verboinum{1}}
\verb{deslustrar}{}{}{}{}{v.t.}{Fazer alguma coisa perder o brilho, o lustre; despolir.}{des.lus.trar}{0}
\verb{deslustrar}{}{}{}{}{}{Fazer algum acontecimento perder o brilho; macular; manchar.}{des.lus.trar}{0}
\verb{deslustrar}{}{}{}{}{}{Fazer uma pessoa ou uma instituição perder a honra; desonrar, infamar.}{des.lus.trar}{\verboinum{1}}
\verb{deslustre}{}{}{}{}{s.m.}{Falta de lustre, brilho ou polimento.}{des.lus.tre}{0}
\verb{deslustre}{}{}{}{}{}{Descrédito, desonra.}{des.lus.tre}{0}
\verb{desmagnetizar}{}{}{}{}{v.t.}{Retirar as propriedades magnéticas; desimantar.}{des.mag.ne.ti.zar}{\verboinum{1}}
\verb{desmaiado}{}{}{}{}{adj.}{Que tem pouco brilho; desbotado.}{des.mai.a.do}{0}
\verb{desmaiado}{}{}{}{}{}{Que perdeu os sentidos; desfalecido.}{des.mai.a.do}{0}
\verb{desmaiar}{}{}{}{}{v.t.}{Fazer perder a cor, o brilho; desbotar.}{des.mai.ar}{0}
\verb{desmaiar}{}{}{}{}{}{Perder os sentidos; desfalecer.}{des.mai.ar}{\verboinum{6}}
\verb{desmaio}{}{}{}{}{s.m.}{Estado em que a pessoa sofre a perda dos sentidos; desfalecimento.}{des.mai.o}{0}
\verb{desmama}{}{}{}{}{s.f.}{Suspensão da amamentação.}{des.ma.ma}{0}
\verb{desmamar}{}{}{}{}{v.t.}{Fazer deixar o hábito de mamar.}{des.ma.mar}{\verboinum{1}}
\verb{desmame}{}{}{}{}{}{Var. de \textit{desmama}.}{des.ma.me}{0}
\verb{desmancha"-prazeres}{ê}{}{}{}{s.2g.2n}{Indivíduo que sempre dá um jeito de acabar com a alegria de outro(s).}{des.man.cha"-pra.ze.res}{0}
\verb{desmanchar}{}{}{}{}{v.t.}{Tirar as partes de um conjunto, dando fim ao que existia; demolir, destruir.}{des.man.char}{0}
\verb{desmanchar}{}{}{}{}{}{Fazer qualquer tipo de acordo deixar de existir; desfazer, romper.}{des.man.char}{0}
\verb{desmanchar}{}{}{}{}{}{Fazer alguma coisa desaparecer, misturada com outra; dissolver, solver.}{des.man.char}{\verboinum{1}}
\verb{desmanche}{}{}{}{}{s.m.}{Ato ou efeito de desmontar, de desmantelar mecanismos, engenhos ou máquinas.}{des.man.che}{0}
\verb{desmanche}{}{}{}{}{}{Desmonte ilícito de veículos.}{des.man.che}{0}
\verb{desmancho}{}{}{}{}{s.m.}{Ato ou efeito de desmanchar; desarranjo, transtorno.}{des.man.cho}{0}
\verb{desmandar}{}{}{}{}{v.t.}{Mandar fazer o contrário do que tinha mandado antes.}{des.man.dar}{\verboinum{1}}
\verb{desmando}{}{}{}{}{s.m.}{Ato ou efeito de desmandar; indisciplina; transgressão de ordens.}{des.man.do}{0}
\verb{desmando}{}{}{}{}{}{Excesso na maneira de proceder ou de tratar; abuso.}{des.man.do}{0}
\verb{desmantelar}{}{}{}{}{v.t.}{Destruir alguma coisa separando as suas partes; desconjuntar.}{des.man.te.lar}{0}
\verb{desmantelar}{}{}{}{}{}{Desorganizar.}{des.man.te.lar}{\verboinum{1}}
\verb{desmarcado}{}{}{}{}{adj.}{Fora das marcas.}{des.mar.ca.do}{0}
\verb{desmarcado}{}{}{}{}{}{Desmedido, enorme.}{des.mar.ca.do}{0}
\verb{desmarcado}{}{}{}{}{}{Revogado, desfeito, cancelado.}{des.mar.ca.do}{0}
\verb{desmarcar}{}{}{}{}{v.t.}{Tirar as marcas de alguma coisa.}{des.mar.car}{0}
\verb{desmarcar}{}{}{}{}{}{Tornar enorme, desmedido.}{des.mar.car}{0}
\verb{desmarcar}{}{}{}{}{}{Revogar, cancelar, anular.}{des.mar.car}{\verboinum{2}}
\verb{desmascarar}{}{}{}{}{v.t.}{Tirar a máscara do rosto.}{des.mas.ca.rar}{0}
\verb{desmascarar}{}{}{}{}{}{Revelar.}{des.mas.ca.rar}{\verboinum{1}}
\verb{desmastrear}{}{}{}{}{v.t.}{Fazer perder o(s) mastro(s).}{des.mas.tre.ar}{0}
\verb{desmastrear}{}{}{}{}{}{Desgovernar; desorientar.}{des.mas.tre.ar}{\verboinum{4}}
\verb{desmatar}{}{}{}{}{v.t.}{Cortar as árvores de um lugar; desflorestar.}{des.ma.tar}{\verboinum{1}}
\verb{desmazelado}{}{}{}{}{adj.}{Que não toma cuidado com as coisas; desleixado, negligente, relaxado.}{des.ma.ze.la.do}{0}
\verb{desmazelar"-se}{}{}{}{}{v.pron.}{Tornar"-se desmazelado; desleixar"-se, descuidar"-se.}{des.ma.ze.lar"-se}{\verboinum{1}}
\verb{desmazelo}{ê}{}{}{}{s.m.}{Falta de cuidado com as coisas; desleixo, negligência, relaxamento.}{des.ma.ze.lo}{0}
\verb{desmedido}{}{}{}{}{adj.}{Que passa dos limites; exagerado, excessivo.}{des.me.di.do}{0}
\verb{desmedir}{}{}{}{}{v.pron.}{Descomedir"-se; exceder"-se.}{des.me.dir"-se}{\verboinum{20}\verboirregular{desmedimo"-nos, desmedis"-vos}}
\verb{desmembrado}{}{}{}{}{adj.}{Que se desmembrou; separado; desapegado.}{des.mem.bra.do}{0}
\verb{desmembrado}{}{Fig.}{}{}{}{Diz"-se de indivíduo sem forças, sem ânimo; prostrado.}{des.mem.bra.do}{0}
\verb{desmembramento}{}{}{}{}{s.m.}{Ato ou efeito de desmembrar.}{des.mem.bra.men.to}{0}
\verb{desmembramento}{}{}{}{}{}{Desligamento ou amputação de membros do corpo.}{des.mem.bra.men.to}{0}
\verb{desmembramento}{}{}{}{}{}{Divisão, em várias partes, daquilo que formava uma unidade; partilha, separação.}{des.mem.bra.men.to}{0}
\verb{desmembrar}{}{}{}{}{v.t.}{Separar ou cortar os membros de um corpo; mutilar, amputar.}{des.mem.brar}{0}
\verb{desmembrar}{}{}{}{}{}{Promover a separação ou divisão, em partes, de algo que formava uma unidade; dividir, desdobrar. }{des.mem.brar}{\verboinum{1}}
\verb{desmemoriado}{}{}{}{}{adj.}{Que perdeu a memória.}{des.me.mo.ri.a.do}{0}
\verb{desmemoriar}{}{}{}{}{v.t.}{Fazer perder a memória.}{des.me.mo.ri.ar}{\verboinum{6}}
\verb{desmentido}{}{}{}{}{s.m.}{Declaração com que se desmente.}{des.men.ti.do}{0}
\verb{desmentido}{}{}{}{}{adj.}{Que se desmentiu.}{des.men.ti.do}{0}
\verb{desmentir}{}{}{}{}{v.t.}{Declarar que alguém mentiu.}{des.men.tir}{0}
\verb{desmentir}{}{}{}{}{}{Negar o que foi dito.}{des.men.tir}{\verboinum{18}}
\verb{desmerecer}{ê}{}{}{}{v.t.}{Deixar de ser digno de uma pessoa ou uma coisa.}{des.me.re.cer}{0}
\verb{desmerecer}{ê}{}{}{}{}{Fazer uma pessoa ou coisa perder o respeito, a estima.}{des.me.re.cer}{\verboinum{15}}
\verb{desmerecimento}{}{}{}{}{s.m.}{Falta ou perda de merecimento; demérito.}{des.me.re.ci.men.to}{0}
\verb{desmerecimento}{}{}{}{}{}{Perda de confiança ou prestígio.}{des.me.re.ci.men.to}{0}
\verb{desmesurado}{}{}{}{}{adj.}{Que está acima da medida habitual; enorme, exagerado, desmedido.}{des.me.su.ra.do}{0}
\verb{desmilinguido}{}{}{}{}{adj.}{Que se desmilinguiu; enfraquecido, debilitado.}{des.mi.lin.gui.do}{0}
%\verb{}{}{}{}{}{}{}{}{0}
\verb{desmilitarização}{}{}{"-ões}{}{s.f.}{Ato ou efeito de desmilitarizar; desarmamento.}{des.mi.li.ta.ri.za.ção}{0}
\verb{desmilitarizar}{}{}{}{}{v.t.}{Tirar ou perder o caráter militar.}{des.mi.li.ta.ri.zar}{0}
\verb{desmilitarizar}{}{}{}{}{}{Abolir a atividade militar.}{des.mi.li.ta.ri.zar}{\verboinum{1}}
\verb{desmiolado}{}{}{}{}{adj.}{Que não tem juízo; insensato, irresponsável.}{des.mi.o.la.do}{0}
\verb{desmistificação}{}{}{"-ões}{}{s.f.}{Ato ou efeito de desmistificar; destituição do caráter místico ou misterioso.}{des.mis.ti.fi.ca.ção}{0}
\verb{desmistificar}{}{}{}{}{v.t.}{Desfazer a mistificação.}{des.mis.ti.fi.car}{0}
\verb{desmistificar}{}{}{}{}{}{Desmascarar.}{des.mis.ti.fi.car}{\verboinum{2}}
\verb{desmitificação}{}{}{"-ões}{}{s.f.}{Ato ou efeito de desmitificar; perda do caráter mítico ou lendário.}{des.mi.ti.fi.ca.ção}{0}
\verb{desmitificar}{}{}{}{}{v.t.}{Tirar de algo ou alguém o caráter de mítico.}{des.mi.ti.fi.car}{\verboinum{2}}
\verb{desmobilhar}{}{}{}{}{v.t.}{Tirar a mobília.}{des.mo.bi.lhar}{\verboinum{1}}
\verb{desmobiliar}{}{}{}{}{}{Var. de \textit{desmobilhar}.}{des.mo.bi.li.ar}{0}
\verb{desmobilização}{}{}{"-ões}{}{s.f.}{Ato ou efeito de desmobilizar; impedimento da mobilização.}{des.mo.bi.li.za.ção}{0}
\verb{desmobilizar}{}{}{}{}{v.t.}{Fazer deixar de estar mobilizado.}{des.mo.bi.li.zar}{0}
\verb{desmobilizar}{}{}{}{}{}{Fazer retornar à vida civil.}{des.mo.bi.li.zar}{0}
\verb{desmobilizar}{}{}{}{}{}{Impedir a mobilização.}{des.mo.bi.li.zar}{\verboinum{1}}
\verb{desmontado}{}{}{}{}{adj.}{Que se desmontou; que descavalgou; apeado.}{des.mon.ta.do}{0}
\verb{desmontado}{}{}{}{}{}{Que foi desfeito em partes; desarmado, desmantelado.}{des.mon.ta.do}{0}
\verb{desmontado}{}{}{}{}{}{Que se arruinou; destruído, arrasado.}{des.mon.ta.do}{0}
\verb{desmontado}{}{}{}{}{}{Transtornado; desnorteado.}{des.mon.ta.do}{0}
\verb{desmontar}{}{}{}{}{v.t.}{Descer ou apear de um animal.}{des.mon.tar}{0}
\verb{desmontar}{}{}{}{}{}{Retirar uma por uma as partes que formam alguma coisa; desarmar.}{des.mon.tar}{0}
\verb{desmontar}{}{}{}{}{}{Causar ruína; destruir, arrasar.}{des.mon.tar}{0}
\verb{desmontar}{}{}{}{}{}{Causar embaraço; transtornar, desnortear.}{des.mon.tar}{\verboinum{1}}
\verb{desmonte}{}{}{}{}{s.m.}{Ato ou efeito de desmontar; apeamento de uma montaria.}{des.mon.te}{0}
\verb{desmonte}{}{}{}{}{}{Ato ou efeito de desmontar; de separar as partes que formavam um conjunto; desmancho.}{des.mon.te}{0}
\verb{desmonte}{}{}{}{}{}{Derruimento de morro; desmoronamento.}{des.mon.te}{0}
\verb{desmonte}{}{}{}{}{}{Extração de minérios de jazidas.}{des.mon.te}{0}
\verb{desmoralização}{}{}{"-ões}{}{s.f.}{Ato ou efeito de desmoralizar; perda do senso moral.}{des.mo.ra.li.za.ção}{0}
\verb{desmoralização}{}{}{"-ões}{}{}{Perda da boa reputação.}{des.mo.ra.li.za.ção}{0}
\verb{desmoralizado}{}{}{}{}{adj.}{Que perdeu a força moral; desacreditado.}{des.mo.ra.li.za.do}{0}
\verb{desmoralizado}{}{}{}{}{}{Pervertido, corrupto.}{des.mo.ra.li.za.do}{0}
\verb{desmoralizar}{}{}{}{}{v.t.}{Fazer uma pessoa ou uma organização perder o respeito e a estima; desacreditar, desprestigiar.}{des.mo.ra.li.zar}{0}
\verb{desmoralizar}{}{}{}{}{}{Fazer alguém perder a coragem, a força moral.}{des.mo.ra.li.zar}{\verboinum{1}}
\verb{desmoronamento}{}{}{}{}{s.m.}{Ato ou efeito desmoronar; queda, derrubada.}{des.mo.ro.na.men.to}{0}
\verb{desmoronar}{}{}{}{}{v.i.}{Vir abaixo; desabar, ruir. }{des.mo.ro.nar}{\verboinum{1}}
\verb{desmotivado}{}{}{}{}{adj.}{Que não tem motivação ou estímulo; desaimado, desinteressado.}{des.mo.ti.va.do}{0}
\verb{desmotivado}{}{}{}{}{}{Sem fundamento, sem motivo.}{des.mo.ti.va.do}{0}
\verb{desmotivar}{}{}{}{}{v.t.}{Tirar o atrativo ou o interesse; fazer perder a motivação; desestimular.}{desmotivar}{\verboinum{1}}
\verb{desmunhecado}{}{}{}{}{adj.}{Diz"-se de quem desmunheca; efeminado.}{des.mu.nhe.ca.do}{0}
\verb{desmunhecar}{}{}{}{}{v.t.}{Quebrar a mão de uma pessoa na altura do pulso.}{des.mu.nhe.car}{0}
\verb{desmunhecar}{}{Pop.}{}{}{v.i.}{Ficar com jeito de mulher.}{des.mu.nhe.car}{\verboinum{2}}
\verb{desnacionalização}{}{}{"-ões}{}{s.f.}{Ato ou efeito de desnacionalizar.}{des.na.ci.o.na.li.za.ção}{0}
\verb{desnacionalização}{}{}{"-ões}{}{}{Perda da nacionalidade originária ou adquirida.}{des.na.ci.o.na.li.za.ção}{0}
\verb{desnacionalizar}{}{}{}{}{v.t.}{Tirar o caráter ou a feição nacional.}{des.na.ci.o.na.li.zar}{\verboinum{1}}
\verb{desnasalar}{}{Gram.}{}{}{v.t.}{Fazer perder o timbre nasal.}{des.na.sa.lar}{\verboinum{1}}
\verb{desnastrar}{}{}{}{}{v.t.}{Destrançar.}{des.nas.trar}{\verboinum{1}}
\verb{desnatadeira}{ê}{}{}{}{s.f.}{Máquina que separa a gordura do leite e a concentra em forma de nata.}{des.na.ta.dei.ra}{0}
\verb{desnatado}{}{}{}{}{adj.}{Diz"-se do leite a que se tirou a nata; desengordurado.}{des.na.ta.do}{0}
\verb{desnatar}{}{}{}{}{v.t.}{Tirar a nata do leite.}{des.na.tar}{\verboinum{1}}
\verb{desnaturado}{}{}{}{}{adj.}{Que se desnaturou; cuja natureza ou características foram profundamente alteradas.}{des.na.tu.ra.do}{0}
\verb{desnaturado}{}{}{}{}{}{Diz"-se de indivíduo desumano, cruel.}{des.na.tu.ra.do}{0}
\verb{desnaturalização}{}{}{"-ões}{}{s.f.}{Ato ou efeito de desnaturalizar.}{des.na.tu.ra.li.za.ção}{0}
\verb{desnaturalização}{}{}{"-ões}{}{}{Perda da nacionalidade e dos direitos adquiridos por naturalização.}{des.na.tu.ra.li.za.ção}{0}
\verb{desnaturalizar}{}{}{}{}{v.t.}{Tirar a nacionalidade.}{des.na.tu.ra.li.zar}{\verboinum{1}}
\verb{desnaturar}{}{}{}{}{v.t.}{Fazer pessoa ou coisa perder o que é natural dela.}{des.na.tu.rar}{\verboinum{1}}
\verb{desnecessário}{}{}{}{}{adj.}{Que se pode dispensar; dispensável.}{des.ne.ces.sá.rio}{0}
\verb{desnecessidade}{}{}{}{}{s.f.}{Falta de necessidade ou de utilidade; inutilidade.}{des.ne.ces.si.da.de}{0}
\verb{desnível}{}{}{"-eis}{}{s.m.}{Diferença de nível em uma superfície.}{des.ní.vel}{0}
\verb{desnível}{}{}{"-eis}{}{}{Diferença, desigualdade em relação a uma escala de valores.}{des.ní.vel}{0}
\verb{desnivelar}{}{}{}{}{v.t.}{Fazer ficar fora do nível.}{des.ni.ve.lar}{\verboinum{1}}
\verb{desnodoar}{}{}{}{}{}{Var. de \textit{desenodoar}.}{des.no.do.ar}{0}
\verb{desnorteado}{}{}{}{}{adj.}{Que perdeu o sentido da direção; desorientado, sem rumo; tonto, perdido.}{des.nor.te.a.do}{0}
\verb{desnortear}{}{}{}{}{v.t.}{Fazer perder o rumo; desorientar.}{des.nor.te.ar}{\verboinum{4}}
\verb{desnovelar}{}{}{}{}{v.t.}{Desenovelar.}{des.no.ve.lar}{\verboinum{1}}
\verb{desnudamento}{}{}{}{}{s.m.}{Ato ou efeito de desnudar; despimento do corpo ou de parte do corpo.}{des.nu.da.men.to}{0}
\verb{desnudamento}{}{}{}{}{}{Perda de proteção.}{des.nu.da.men.to}{0}
\verb{desnudamento}{}{}{}{}{}{Ato ou efeito de tornar manifesto; revelação, manifestação.}{des.nu.da.men.to}{0}
\verb{desnudamento}{}{}{}{}{}{Despojamento, abandono.}{des.nu.da.men.to}{0}
\verb{desnudar}{}{}{}{}{v.t.}{Descobrir o corpo ou parte do corpo.}{des.nu.dar}{0}
\verb{desnudar}{}{}{}{}{}{Despojar algo do que o cobre ou protege.}{des.nu.dar}{0}
\verb{desnudar}{}{Fig.}{}{}{}{Tornar manifesto; patentear, revelar.}{des.nu.dar}{0}
\verb{desnudar}{}{}{}{}{v.pron.}{Despir"-se, despojar"-se.}{des.nu.dar}{\verboinum{1}}
\verb{desnudo}{}{}{}{}{adj.}{Sem roupa; despido.}{des.nu.do}{0}
\verb{desnutrição}{}{}{"-ões}{}{s.f.}{Ato ou efeito de desnutrir.}{des.nu.tri.ção}{0}
\verb{desnutrição}{}{}{"-ões}{}{}{Falta de nutrição; carência alimentar.}{des.nu.tri.ção}{0}
\verb{desnutrição}{}{}{"-ões}{}{}{Enfraquecimento ou emagrecimento por falta de nutrição.}{des.nu.tri.ção}{0}
\verb{desnutrido}{}{}{}{}{adj.}{Que se desnutriu; que deixou de se alimentar ou se nutre de forma inadequada. }{des.nu.tri.do}{0}
\verb{desnutrido}{}{}{}{}{}{Que é magro ou fraco por carência alimentar.}{des.nu.tri.do}{0}
\verb{desnutrir}{}{}{}{}{v.t.}{Nutrir mal, de forma inadequada; deixar de nutrir.}{des.nu.trir}{0}
\verb{desnutrir}{}{}{}{}{}{Emagrecer.}{des.nu.trir}{\verboinum{18}}
\verb{desobedecer}{ê}{}{}{}{v.t.}{Deixar de fazer o que lhe mandam; desacatar, desrespeitar.}{de.so.be.de.cer}{\verboinum{15}}
\verb{desobediência}{}{}{}{}{s.f.}{Falta de obediência; insubordinação.}{de.so.be.di.ên.cia}{0}
\verb{desobediente}{}{}{}{}{adj.2g.}{Que desobedece; que não acata ordens, comandos ou prescrições.}{de.so.be.di.en.te}{0}
\verb{desobrigação}{}{}{"-ões}{}{s.f.}{Ato ou efeito de desobrigar; liberação, descompromisso.}{de.so.bri.ga.ção}{0}
\verb{desobrigar}{}{}{}{}{v.t.}{Livrar alguém de alguma coisa que deveria fazer; dispensar, liberar.}{de.so.bri.gar}{0}
\verb{desobrigar}{}{}{}{}{v.pron.}{Fazer o que dever ser feito; cumprir; desempenhar"-se.}{de.so.bri.gar}{\verboinum{5}}
\verb{desobstrução}{}{}{"-ões}{}{s.f.}{Ato ou efeito de desobstruir; desimpedimento, descongestionamento.}{de.sobs.tru.ção}{0}
\verb{desobstruir}{}{}{}{}{v.t.}{Tirar o que fecha ou dificulta a passagem por um lugar; desbloquear, desimpedir.}{de.sobs.tru.ir}{\verboinum{18}}
\verb{desocupação}{}{}{"-ões}{}{s.f.}{Falta de ocupação; ociosidade.}{de.so.cu.pa.ção}{0}
\verb{desocupado}{}{}{}{}{adj.}{Que não está sendo ocupado; livre, disponível.}{de.so.cu.pa.do}{0}
\verb{desocupado}{}{}{}{}{}{Que não tem trabalho ou ocupação; ocioso.}{de.so.cu.pa.do}{0}
\verb{desocupado}{}{}{}{}{}{Que está com o tempo livre.}{de.so.cu.pa.do}{0}
\verb{desocupar}{}{}{}{}{v.t.}{Deixar livre o lugar em que estava; liberar.}{de.so.cu.par}{0}
\verb{desocupar}{}{}{}{}{}{Tirar alguma coisa para deixar um lugar vazio.}{de.so.cu.par}{0}
\verb{desocupar}{}{}{}{}{}{Liberar de trabalho, tarefa, serviço etc.}{de.so.cu.par}{\verboinum{1}}
\verb{desodorante}{}{}{}{}{adj.2g.}{Diz"-se de produto que se usa para tirar o mau cheiro.}{de.so.do.ran.te}{0}
\verb{desodorar}{}{}{}{}{v.t.}{Tirar o mau cheiro.}{de.so.do.rar}{\verboinum{1}}
\verb{desodorizar}{}{}{}{}{v.t.}{Desodorar.}{de.so.do.ri.zar}{\verboinum{1}}
\verb{desoficializar}{}{}{}{}{v.t.}{Tirar o caráter oficial.}{de.so.fi.ci.a.li.zar}{\verboinum{1}}
\verb{desolação}{}{}{"-ões}{}{s.f.}{Isolamento, desamparo.}{de.so.la.ção}{0}
\verb{desolação}{}{}{"-ões}{}{}{Estrago causado por calamidade; ruína.}{de.so.la.ção}{0}
\verb{desolação}{}{}{"-ões}{}{}{Grande tristeza; consternação, aflição.}{de.so.la.ção}{0}
\verb{desolado}{}{}{}{}{adj.}{Cheio de tristeza; inconsolável.}{de.so.la.do}{0}
\verb{desolado}{}{}{}{}{}{Sem moradores e com tudo destruído; arruinado, devastado.}{de.so.la.do}{0}
\verb{desolador}{ô}{}{}{}{adj.}{Que provoca desolação.}{de.so.la.dor}{0}
\verb{desolar}{}{}{}{}{v.t.}{Fazer alguém ficar triste; consternar, desesperar.}{de.so.lar}{0}
\verb{desolar}{}{}{}{}{}{Deixar em ruínas; destruir, devastar.}{de.so.lar}{\verboinum{1}}
\verb{desonerar}{}{}{}{}{v.t.}{Livrar de obrigação ou incumbência; desobrigar, isentar.}{de.so.ne.rar}{0}
\verb{desonerar}{}{Fig.}{}{}{}{Aliviar.}{de.so.ne.rar}{\verboinum{1}}
\verb{desonestidade}{}{}{}{}{s.f.}{Falta de honradez, integridade; sinceridade.}{de.so.nes.ti.da.de}{0}
\verb{desonesto}{é}{}{}{}{adj.}{Que não é honesto; que denota intenção de enganar; falso.}{de.so.nes.to}{0}
\verb{desonesto}{é}{}{}{}{}{Que é contrário à lei; ilegal.}{de.so.nes.to}{0}
\verb{desonra}{}{}{}{}{s.f.}{Falta de honra; vergonha, impudor.}{de.son.ra}{0}
\verb{desonradez}{ê}{}{}{}{s.f.}{Desonra.}{de.son.ra.dez}{0}
\verb{desonrado}{}{}{}{}{adj.}{Que não tem honra; que perdeu a honra.}{de.son.ra.do}{0}
\verb{desonrar}{}{}{}{}{v.t.}{Ofender a honra de alguém ou de algo.}{de.son.rar}{0}
\verb{desonrar}{}{}{}{}{}{Perder a honra.}{de.son.rar}{\verboinum{1}}
\verb{desonroso}{ô}{}{"-osos ⟨ó⟩}{"-osa ⟨ó⟩}{adj.}{Que causa ou em que há desonra; aviltante, degradante.}{de.son.ro.so}{0}
\verb{desopilação}{}{}{"-ões}{}{s.f.}{Ato ou efeito de desopilar; desobstrução, desimpedimento.}{de.so.pi.la.ção}{0}
\verb{desopilação}{}{}{"-ões}{}{}{Relaxamento de um estado de tensão; alívio.}{de.so.pi.la.ção}{0}
\verb{desopilar}{}{}{}{}{v.t.}{Desobstruir.}{de.so.pi.lar}{0}
\verb{desopilar}{}{}{}{}{}{Aliviar das tensões; alegrar.}{de.so.pi.lar}{\verboinum{1}}
\verb{desopressão}{}{}{"-ões}{}{s.f.}{Alívio da opressão; liberação.}{de.so.pres.são}{0}
\verb{desoprimir}{}{}{}{}{v.t.}{Livrar do que oprime, do que pesa.}{de.so.pri.mir}{0}
\verb{desoprimir}{}{}{}{}{}{Libertar de uma tirania, de um regime opressor. }{de.so.pri.mir}{\verboinum{18}}
\verb{desoras}{ó}{}{}{}{s.f.pl.}{Usado na locução adverbial \textit{a desoras}: fora de hora; tarde da noite.}{de.so.ras}{0}
\verb{desordeiro}{ê}{}{}{}{adj.}{Que faz ou promove desordens; arruaçeiro.}{de.sor.dei.ro}{0}
\verb{desordeiro}{ê}{}{}{}{s.m.}{Essa pessoa.}{de.sor.dei.ro}{0}
\verb{desordem}{ó}{}{"-ens}{}{s.f.}{Falta de ordem; desorganização.}{de.sor.dem}{0}
\verb{desordem}{ó}{}{"-ens}{}{}{Distúrbio, tumulto, confusão.}{de.sor.dem}{0}
\verb{desordenado}{}{}{}{}{adj.}{Que não tem ordem ou não está ordenado; desarranjado, desorganizado.}{de.sor.de.na.do}{0}
\verb{desordenar}{}{}{}{}{v.t.}{Tirar da ordem; desarranjar, desorganizar.}{de.sor.de.nar}{\verboinum{1}}
\verb{desorganização}{}{}{"-ões}{}{s.f.}{Falta de organização; desordem. }{de.sor.ga.ni.za.ção}{0}
\verb{desorganizado}{}{}{}{}{adj.}{Que não tem organização; desordenado.  }{de.sor.ga.ni.za.do}{0}
\verb{desorganizar}{}{}{}{}{v.t.}{Desfazer a organização de alguma coisa; desordenar.}{de.sor.ga.ni.zar}{\verboinum{1}}
\verb{desorientação}{}{}{"-ões}{}{s.f.}{Ato ou efeito de desorientar.}{de.so.ri.en.ta.ção}{0}
\verb{desorientação}{}{}{"-ões}{}{}{Falta de orientação; desnorteamento.}{de.so.ri.en.ta.ção}{0}
\verb{desorientação}{}{}{"-ões}{}{}{Desvairamento, insensatez, desconcerto, confusão.}{de.so.ri.en.ta.ção}{0}
\verb{desorientar}{}{}{}{}{v.t.}{Fazer perder o rumo, a direção, a orientação; desnortear.}{de.so.ri.en.tar}{0}
\verb{desorientar}{}{Fig.}{}{}{}{Perturbar, tornar confuso, hesitante; desconcertar, embaraçar.}{de.so.ri.en.tar}{\verboinum{1}}
\verb{desossar}{}{}{}{}{v.t.}{Retirar os ossos; separar os ossos. (\textit{O açougueiro desossou o carneiro.})}{de.sos.sar}{\verboinum{1}}
\verb{desova}{ó}{}{}{}{s.f.}{Ato ou efeito de desovar, de pôr ovos, especialmente de peixes; desovamento.}{de.so.va}{0}
\verb{desova}{ó}{Por ext.}{}{}{}{A época da postura de ovos.}{de.so.va}{0}
\verb{desovar}{}{}{}{}{v.i.}{Pôr ovos, especialmente os de peixe.}{de.so.var}{\verboinum{1}}
\verb{desoxidar}{cs}{Quím.}{}{}{v.t.}{Retirar o óxido, ou retirar o oxigênio de um óxido.}{de.so.xi.dar}{0}
\verb{desoxidar}{cs}{Por ext.}{}{}{}{Retirar a ferrugem.}{de.so.xi.dar}{\verboinum{1}}
\verb{despachado}{}{}{}{}{adj.}{Que se despachou ou que recebeu despacho.}{des.pa.cha.do}{0}
\verb{despachado}{}{}{}{}{}{Que foi mandado embora; demitido, expulso, dispensado.}{des.pa.cha.do}{0}
\verb{despachado}{}{Bras.}{}{}{}{Sem cerimônias, franco, desembaraçado, desinibido.}{des.pa.cha.do}{0}
\verb{despachante}{}{}{}{}{s.m.}{Pessoa encarregada de despachar mercadorias ou de encaminhar papéis e documentos em repartições públicas, firmas etc.}{des.pa.chan.te}{0}
\verb{despachar}{}{}{}{}{v.t.}{Pôr despacho em.}{des.pa.char}{0}
\verb{despachar}{}{}{}{}{}{Deliberar, decidir, resolver.}{des.pa.char}{0}
\verb{despachar}{}{}{}{}{}{Expedir, enviar, remeter.}{des.pa.char}{0}
\verb{despachar}{}{}{}{}{}{Mandar embora; dispensar, despedir.}{des.pa.char}{\verboinum{1}}
\verb{despacho}{}{}{}{}{s.m.}{Resolução de uma autoridade pública em requerimento ou petição.}{des.pa.cho}{0}
\verb{despacho}{}{}{}{}{}{Ato ou efeito de despachar.}{des.pa.cho}{0}
\verb{despacho}{}{Relig.}{}{}{}{Ato de depositar uma oferenda para Exu.}{des.pa.cho}{0}
\verb{desparafusar}{}{}{}{}{v.t.}{Tirar os parafusos; desaparafusar.}{des.pa.ra.fu.sar}{\verboinum{1}}
\verb{despautério}{}{}{}{}{s.m.}{Grande disparate; besteira, tolice, despropósito.}{des.pau.té.rio}{0}
\verb{despedaçar}{}{}{}{}{v.t.}{Reduzir a pedaços; quebrar, espedaçar, rasgar, dilacerar.}{des.pe.da.çar}{0}
\verb{despedaçar}{}{Fig.}{}{}{}{Causar dor, aflição a alguém ou a si mesmo; pungir, afligir.}{des.pe.da.çar}{\verboinum{3}}
\verb{despedida}{}{}{}{}{s.f.}{Ato ou efeito de despedir.}{des.pe.di.da}{0}
\verb{despedir}{}{}{}{}{v.t.}{Fazer sair; dispensar, despachar.}{des.pe.dir}{0}
\verb{despedir}{}{}{}{}{}{Mandar embora; demitir.}{des.pe.dir}{0}
\verb{despedir}{}{}{}{}{}{Lançar, soltar, exalar.}{des.pe.dir}{0}
\verb{despedir}{}{}{}{}{v.i.}{Terminar, cessar.}{des.pe.dir}{0}
\verb{despedir}{}{}{}{}{v.pron.}{Cumprimentar para retirar"-se.}{des.pe.dir}{\verboinum{18}}
\verb{despegar}{}{}{}{}{v.t.}{Desunir, descolar, desapegar.}{des.pe.gar}{\verboinum{5}}
\verb{despeitado}{}{}{}{}{adj.}{Que tem ou revela despeito; magoado, ressentido, invejoso, contrariado.}{des.pei.ta.do}{0}
\verb{despeitado}{}{}{}{}{s.m.}{Essa pessoa.}{des.pei.ta.do}{0}
\verb{despeitar}{}{}{}{}{v.t.}{Causar despeito a alguém ou a si mesmo; irritar. }{des.pei.tar}{0}
\verb{despeitar}{}{}{}{}{}{Tratar com despeito.}{des.pei.tar}{\verboinum{1}}
\verb{despeito}{ê}{}{}{}{s.m.}{Desgosto, acompanhado de raiva, provocado por decepção, ofensa, amor"-próprio ferido ou inveja; ressentimento, rancor.}{des.pei.to}{0}
\verb{despejado}{}{}{}{}{adj.}{Que se despejou; derramado, esvaziado, evacuado, entornado.}{des.pe.ja.do}{0}
\verb{despejado}{}{}{}{}{}{Diz"-se daquele que sofreu ação de despejo, que teve que desocupar um imóvel por decisão judicial.}{des.pe.ja.do}{0}
\verb{despejar}{}{}{}{}{v.t.}{Derramar, fazer cair do vaso ou do recipiente em que se encontra.}{des.pe.jar}{0}
\verb{despejar}{}{}{}{}{}{Promover despejo.}{des.pe.jar}{\verboinum{1}}
\verb{despejo}{ê}{}{}{}{s.m.}{Ato ou efeito de despejar.}{des.pe.jo}{0}
\verb{despejo}{ê}{}{}{}{}{Lixo, dejeto.}{des.pe.jo}{0}
\verb{despejo}{ê}{Jur.}{}{}{}{Desocupação de imóvel determinada por ordem judicial.}{des.pe.jo}{0}
\verb{despelar}{}{}{}{}{v.t.}{Tirar a pele ou a casca; descascar.}{des.pe.lar}{\verboinum{1}}
\verb{despencar}{}{}{}{}{v.i.}{Cair de grande altura; tombar.}{des.pen.car}{0}
\verb{despencar}{}{}{}{}{v.t.}{Tirar, soltar, separar da penca ou do cacho.}{des.pen.car}{\verboinum{1}}
\verb{despender}{ê}{}{}{}{v.t.}{Fazer despesa; gastar, consumir.}{des.pen.der}{0}
\verb{despender}{ê}{Fig.}{}{}{}{Usar; empregar, gastar. (\textit{O menino despendeu toda sua energia naquela prova.})}{des.pen.der}{\verboinum{12}}
\verb{despendurar}{}{}{}{}{v.t.}{Tirar do lugar o que estava pendurado.}{des.pen.du.rar}{\verboinum{1}}
\verb{despenhadeiro}{ê}{}{}{}{s.m.}{Encosta muito inclinada, de grande profundidade; precipício, abismo, garganta, barranco, ribanceira.}{des.pe.nha.dei.ro}{0}
\verb{despenhar}{}{}{}{}{v.t.}{Lançar, cair de grande altura.}{des.pe.nhar}{\verboinum{1}}
\verb{despensa}{}{}{}{}{s.f.}{Armário ou divisão de uma casa onde são guardados os mantimentos.}{des.pen.sa}{0}
\verb{despenseiro}{ê}{}{}{}{s.m.}{Pessoa encarregada da despensa.}{des.pen.sei.ro}{0}
\verb{despentear}{}{}{}{}{v.t.}{Desmanchar, estragar o penteado.}{des.pen.te.ar}{\verboinum{4}}
\verb{desperceber}{ê}{}{}{}{v.t.}{Não perceber; não notar, não dar atenção.}{des.per.ce.ber}{\verboinum{12}}
\verb{despercebido}{}{}{}{}{adj.}{Que não se viu nem se ouviu; que não foi notado.}{des.per.ce.bi.do}{0}
\verb{desperdiçado}{}{}{}{}{adj.}{Que foi gasto, usado sem proveito ou utilidade; esbanjado.}{des.per.di.ça.do}{0}
\verb{desperdiçar}{}{}{}{}{v.t.}{Gastar ou usar mais do que o necessário ou sem proveito; esbanjar, esperdiçar.  }{des.per.di.çar}{\verboinum{3}}
\verb{desperdício}{}{}{}{}{s.m.}{Ato ou efeito de desperdiçar, de gastar ou usar mais do que o necessário; esbanjamento.}{des.per.dí.cio}{0}
\verb{despersonalizado}{}{}{}{}{adj.}{Que se despersonalizou, perdeu ou teve alterada a sua personalidade, sua individualidade; descaracterizado.}{des.per.so.na.li.za.do}{0}
\verb{despersonalizado}{}{}{}{}{}{Que se tornou impessoal.}{des.per.so.na.li.za.do}{0}
\verb{despersonalizar}{}{}{}{}{v.t.}{Fazer perder ou alterar a personalidade.}{des.per.so.na.li.zar}{\verboinum{1}}
\verb{despersuadir}{}{}{}{}{v.t.}{Fazer mudar de opinião, de intento; dissuadir.}{des.per.su.a.dir}{\verboinum{18}}
\verb{despersuasão}{}{}{"-ões}{}{s.f.}{Ato ou efeito de despersuadir; dissuasão.}{des.per.su.a.são}{0}
\verb{despertador}{ô}{}{}{}{adj.}{Que desperta.}{des.per.ta.dor}{0}
\verb{despertador}{ô}{}{}{}{s.m.}{Rélogio com dispositivo que soa um alarme em determinada hora.}{des.per.ta.dor}{0}
\verb{despertar}{}{}{}{}{v.t.}{Tirar do sono; acordar.}{des.per.tar}{0}
\verb{despertar}{}{}{}{}{}{Dar origem a; estimular.}{des.per.tar}{0}
\verb{despertar}{}{}{}{}{v.i.}{Aparecer, despontar, revelar"-se.}{des.per.tar}{\verboinum{1}}
\verb{despertar}{}{}{}{}{s.m.}{O ato de despertar.}{des.per.tar}{0}
\verb{desperto}{é}{}{}{}{adj.}{Que despertou; acordado, despertado.}{des.per.to}{0}
\verb{despesa}{ê}{}{}{}{s.f.}{Aquilo que se gastou ou consumiu.}{des.pe.sa}{0}
\verb{despesa}{ê}{}{}{}{}{Ato ou efeito de despender.}{des.pe.sa}{0}
\verb{despetalar}{}{}{}{}{v.t.}{Tirar, arrancar as pétalas.}{des.pe.ta.lar}{\verboinum{1}}
\verb{despicar}{}{}{}{}{v.t.}{Vingar, desagravar, desafrontar, desforrar.}{des.pi.car}{\verboinum{2}}
\verb{despiciendo}{}{}{}{}{adj.}{Desprezível.}{des.pi.ci.en.do}{0}
\verb{despido}{}{}{}{}{adj.}{Sem roupas; nu, desnudo.}{des.pi.do}{0}
\verb{despido}{}{Fig.}{}{}{}{Livre, desprovido, isento, despojado.}{des.pi.do}{0}
\verb{despique}{}{}{}{}{s.m.}{Ato ou efeito de despicar; desagravo, desforra.}{des.pi.que}{0}
\verb{despir}{}{}{}{}{v.t.}{Tirar a roupa; desnudar.}{des.pir}{\verboinum{29}}
\verb{despistar}{}{}{}{}{v.t.}{Fazer alguém perder a pista ou o rastro; desnortear, desorientar.}{des.pis.tar}{0}
\verb{despistar}{}{Fig.}{}{}{}{Iludir, enganar encobrindo as suspeitas.}{des.pis.tar}{\verboinum{1}}
\verb{desplante}{}{}{}{}{s.m.}{Atitude que revela atrevimento; descaramento, audácia, ousadia.}{des.plan.te}{0}
\verb{desplumar}{}{}{}{}{v.t.}{Retirar as plumas ou penas; depenar.}{des.plu.mar}{\verboinum{1}}
\verb{despojado}{}{}{}{}{adj.}{Que se despojou ou foi despojado; espoliado, roubado, saqueado.}{des.po.ja.do}{0}
\verb{despojado}{}{}{}{}{}{Que não tem ambição; desprendido, desambicioso. }{des.po.ja.do}{0}
\verb{despojado}{}{Por ext.}{}{}{}{Sem enfeites; simples, enxuto.}{des.po.ja.do}{0}
\verb{despojar}{}{}{}{}{v.t.}{Privar da posse; espoliar, desapossar, saquear, roubar.}{des.po.jar}{\verboinum{1}}
\verb{despojo}{ô}{}{"-s ⟨ó⟩}{}{s.m.}{Ato ou efeito de despojar; despojamento.}{des.po.jo}{0}
\verb{despojo}{ô}{}{"-s ⟨ó⟩}{}{}{Tudo aquilo que se tomou ao inimigo; presa, espólio.}{des.po.jo}{0}
\verb{despojos}{ó}{}{}{}{s.m.pl.}{Tudo aquilo que sobra; restos, fragmentos, sobras.}{des.po.jos}{0}
\verb{despolpar}{}{}{}{}{v.t.}{Tirar a polpa.}{des.pol.par}{\verboinum{1}}
\verb{despoluir}{}{}{}{}{v.t.}{Retirar ou diminuir a poluição; purificar.}{des.po.lu.ir}{\verboinum{26}}
\verb{despontar}{}{}{}{}{v.t.}{Retirar, gastar ou cortar a ponta. }{des.pon.tar}{0}
\verb{despontar}{}{}{}{}{v.i.}{Começar a aparecer; surgir, nascer, brotar, revelar"-se.}{des.pon.tar}{\verboinum{1}}
\verb{desporte}{ó}{}{}{}{s.m.}{Esporte.}{des.por.te}{0}
\verb{desportista}{}{}{}{}{adj.2g.}{Relativo a desportismo.}{des.por.tis.ta}{0}
\verb{desportista}{}{}{}{}{s.2g.}{Pessoa que pratica um esporte; esportista.}{des.por.tis.ta}{0}
\verb{desportivo}{}{}{}{}{adj.}{Relativo a desporto.}{des.por.ti.vo}{0}
\verb{desporto}{ô}{}{"-s ⟨ó⟩}{}{s.m.}{Esporte.}{des.por.to}{0}
\verb{desposar}{}{}{}{}{v.t.}{Unir"-se em casamento, contrair matrimônio; casar"-se, esposar.}{des.po.sar}{\verboinum{1}}
\verb{déspota}{}{}{}{}{adj.2g.}{Que exerce autoridade absoluta e arbitrária; despótico, tirano, opressor, dominador. }{dés.po.ta}{0}
\verb{déspota}{}{}{}{}{s.2g.}{Governante que exerce poder absoluto; tirano.}{dés.po.ta}{0}
\verb{déspota}{}{Por ext.}{}{}{}{Pessoa que impõe suas vontades de maneira autoritária.}{dés.po.ta}{0}
\verb{despótico}{}{}{}{}{adj.}{Próprio de déspota; tirânico, opressor, dominador.}{des.pó.ti.co}{0}
\verb{despotismo}{}{Pejor.}{}{}{s.m.}{Forma de governo arbitrária, autoritária e absoluta.}{des.po.tis.mo}{0}
\verb{despotismo}{}{}{}{}{}{Ato despótico; tirania, opressão.}{des.po.tis.mo}{0}
\verb{despovoado}{}{}{}{}{adj.}{Que não é povoado, que não tem habitantes nem casas; ermo, deserto, desabitado.}{des.po.vo.a.do}{0}
\verb{despovoar}{}{}{}{}{v.t.}{Tornar despovoado, desabitado; acabar com a população ou reduzi"-la.}{des.po.vo.ar}{\verboinum{7}}
\verb{desprazer}{ê}{}{}{}{s.f.}{Falta de prazer; desgosto, desagrado, desprazimento, descontentamento.}{des.pra.zer}{0}
\verb{desprazer}{ê}{}{}{}{v.i.}{Causar desagrado; desagradar.}{des.pra.zer}{\verboinum{14}}
\verb{desprecatar"-se}{}{}{}{}{v.pron.}{Não tomar as devidas precauções ou cautelas; descuidar"-se, desprevenir"-se, desacautelar"-se.  }{des.pre.ca.tar"-se}{\verboinum{1}}
\verb{desprecaver}{ê}{}{}{}{v.t.}{Não se precaver; desprevenir, desacautelar.}{des.pre.ca.ver}{\verboinum{12}}
\verb{despregar}{}{}{}{}{v.t.}{Retirar as pregas; desenrugar.}{des.pre.gar}{\verboinum{5}}
\verb{despregar}{}{}{}{}{v.t.}{Arrancar, separar o que estava pregado, colado; descravar, despegar.}{des.pre.gar}{\verboinum{5}}
\verb{despreguiçar}{}{}{}{}{v.t.}{Espreguiçar.}{des.pre.gui.çar}{\verboinum{3}}
\verb{desprender}{ê}{}{}{}{v.t.}{Soltar o que estava preso; desatar, desamarrar, despregar. }{des.pren.der}{\verboinum{12}}
\verb{desprendido}{}{}{}{}{adj.}{Que se desprendeu.}{des.pren.di.do}{0}
\verb{desprendido}{}{}{}{}{}{Que tem ou revela desprendimento, abnegação; altruísta, abnegado.}{des.pren.di.do}{0}
\verb{desprendimento}{}{}{}{}{s.m.}{Ato ou efeito de desprender.}{des.pren.di.men.to}{0}
\verb{desprendimento}{}{}{}{}{}{Atitude de quem é abnegado; desapego.}{des.pren.di.men.to}{0}
\verb{despreocupação}{}{}{"-ões}{}{s.f.}{Estado ou atitude de quem é ou está despreocupado. }{des.pre.o.cu.pa.ção}{0}
\verb{despreocupado}{}{}{}{}{adj.}{Que não tem preocupação ou não é preocupado. }{des.pre.o.cu.pa.do}{0}
\verb{despreocupar}{}{}{}{}{v.t.}{Livrar de preocupação; tranquilizar.}{des.pre.o.cu.par}{\verboinum{1}}
\verb{despreparado}{}{}{}{}{adj.}{Que não tem preparo ou não se preparou.}{des.pre.pa.ra.do}{0}
\verb{despreparo}{}{}{}{}{s.m.}{Falta de preparo, de conhecimento ou de competência.}{des.pre.pa.ro}{0}
\verb{desprestigiar}{}{}{}{}{v.t.}{Tirar o prestígio; desacreditar.}{des.pres.ti.gi.ar}{\verboinum{1}}
\verb{desprestígio}{}{}{}{}{s.m.}{Falta ou perda de prestígio.}{des.pres.tí.gio}{0}
\verb{despretensão}{}{}{"-ões}{}{s.f.}{Falta de pretensão; modéstia, singeleza, despresunção.}{des.pre.ten.são}{0}
\verb{despretensioso}{ô}{}{"-osos ⟨ó⟩}{"-osa ⟨ó⟩}{adj.}{Que não tem pretensão; franco, modesto, singelo.}{des.pre.ten.si.o.so}{0}
\verb{desprevenido}{}{}{}{}{adj.}{Não prevenido; desprecavido, desacautelado.}{des.pre.ve.ni.do}{0}
\verb{desprevenido}{}{Pop.}{}{}{}{Sem dinheiro no bolso.}{des.pre.ve.ni.do}{0}
\verb{desprevenir}{}{}{}{}{v.t.}{Não prevenir; desacautelar, desprecaver.}{des.pre.ve.nir}{\verboinum{30}}
\verb{desprezar}{}{}{}{}{v.t.}{Não prezar, não dar importância.}{des.pre.zar}{0}
\verb{desprezar}{}{}{}{}{}{Recusar desvalorizando.}{des.pre.zar}{0}
\verb{desprezar}{}{}{}{}{}{Não levar em conta; desconsiderar.}{des.pre.zar}{\verboinum{1}}
\verb{desprezível}{}{}{"-eis}{}{adj.2g.}{Que merece desprezo; vil, abjeto, miserável, vergonhoso.  }{des.pre.zí.vel}{0}
\verb{desprezo}{ê}{}{}{}{s.m.}{Falta de apreço ou consideração.}{des.pre.zo}{0}
\verb{desprimor}{ô}{}{}{}{s.m.}{Falta de primor, de perfeição, de esmero.}{des.pri.mor}{0}
\verb{desprimoroso}{ô}{}{"-osos ⟨ó⟩}{"-osa ⟨ó⟩}{s.m.}{Que não tem primor; imperfeito.}{des.pri.mo.ro.so}{0}
\verb{desproporção}{}{}{"-ões}{}{s.f.}{Falta de proporção; desacordo, desconformidade.}{des.pro.por.ção}{0}
\verb{desproporcionado}{}{}{}{}{adj.}{Que não é proporcionado; desigual, desproporcional. }{des.pro.por.ci.o.na.do}{0}
\verb{desproporcional}{}{}{"-ais}{}{adj.2g.}{Desproporcionado.}{des.pro.por.ci.o.nal}{0}
\verb{despropositado}{}{}{}{}{adj.}{Que não tem propósito; inconveniente, impróprio.}{des.pro.po.si.ta.do}{0}
\verb{despropositar}{}{}{}{}{v.i.}{Cometer despropósitos; desatinar, disparatar.}{des.pro.po.si.tar}{\verboinum{1}}
\verb{despropósito}{}{}{}{}{s.m.}{Falta de propósito, tudo aquilo que é feito ou dito de forma inoportuna; disparate, desatino, absurdo.}{des.pro.pó.si.to}{0}
\verb{desproteção}{}{}{"-ões}{}{s.f.}{Falta de proteção; abandono, desamparo.}{des.pro.te.ção}{0}
\verb{desproteger}{ê}{}{}{}{v.t.}{Retirar a proteção de.}{des.pro.te.ger}{\verboinum{16}}
\verb{desprotegido}{}{}{}{}{adj.}{Que não tem proteção; desamparado, desassistido.}{des.pro.te.gi.do}{0}
\verb{desproveito}{ê}{}{}{}{s.m.}{Falta de aproveitamento; desperdício.}{des.pro.vei.to}{0}
\verb{desprover}{ê}{}{}{}{v.t.}{Retirar as provisões, privar de tudo aquilo que é necessário.}{des.pro.ver}{\verboinum{47}}
\verb{desprovido}{}{}{}{}{adj.}{Que não tem provisões ou coisas necessárias; falto, vazio.}{des.pro.vi.do}{0}
\verb{desprovido}{}{}{}{}{}{Sem recursos, dinheiro etc.; desprevenido.}{des.pro.vi.do}{0}
\verb{despudor}{ô}{}{}{}{s.m.}{Falta de pudor, de vergonha; impudência, impudor.}{des.pu.dor}{0}
\verb{despudorado}{}{}{}{}{adj.}{Que não tem pudor, vergonha; impudente, desavergonhado.}{des.pu.do.ra.do}{0}
\verb{despudorado}{}{}{}{}{s.m.}{Essa pessoa.}{des.pu.do.ra.do}{0}
\verb{desqualificado}{}{}{}{}{adj.}{Que perdeu as qualidades que o distinguiam; depreciado.}{des.qua.li.fi.ca.do}{0}
\verb{desqualificado}{}{}{}{}{}{Diz"-se daquele que foi eliminado de uma disputa; desclassificado.}{des.qua.li.fi.ca.do}{0}
\verb{desqualificar}{}{}{}{}{v.t.}{Tirar ou fazer perder as boas qualidades.}{des.qua.li.fi.car}{0}
\verb{desqualificar}{}{}{}{}{}{Eliminar de uma disputa; desclassificar.}{des.qua.li.fi.car}{\verboinum{1}}
\verb{desquitado}{}{}{}{}{adj.}{Que se separou por desquite.}{des.qui.ta.do}{0}
\verb{desquitar}{}{}{}{}{v.t.}{Separar um casal por desquite.}{des.qui.tar}{\verboinum{1}}
\verb{desquite}{}{Jur.}{}{}{s.m.}{Ato de separação do casal e de seus bens sem desfazer o casamento.}{des.qui.te}{0}
\verb{desraigar}{}{}{}{}{v.t.}{Desarraigar.}{des.rai.gar}{\verboinum{5}}
\verb{desratização}{}{}{"-ões}{}{s.f.}{Ato ou efeito de desratizar.}{des.ra.ti.za.ção}{0}
\verb{desratizar}{}{}{}{}{v.t.}{Eliminar os ratos de um lugar.}{des.ra.ti.zar}{\verboinum{1}}
\verb{desregrado}{}{}{}{}{adj.}{Que não segue as regras estabelecidas.}{des.re.gra.do}{0}
\verb{desregrado}{}{}{}{}{}{Sem controle sobre seus gastos; descomedido, perdulário, dissipador.}{des.re.gra.do}{0}
\verb{desregrado}{}{}{}{}{}{Sem moral; libertino, devasso.}{des.re.gra.do}{0}
\verb{desregramento}{}{}{}{}{s.m.}{Ato ou efeito de desregrar; descomedimento, abuso.}{des.re.gra.men.to}{0}
\verb{desregrar}{}{}{}{}{v.t.}{Tirar da regra.}{des.re.grar}{0}
\verb{desregrar}{}{}{}{}{}{Tornar descomedido.}{des.re.grar}{\verboinum{1}}
\verb{desregular}{}{}{}{}{v.t.}{Fazer com que fique desregulado.}{des.re.gu.lar}{\verboinum{1}}
\verb{desrespeitar}{}{}{}{}{v.t.}{Faltar ao respeito com.}{des.res.pei.tar}{\verboinum{1}}
\verb{desrespeito}{ê}{}{}{}{s.m.}{Falta de respeito.}{des.res.pei.to}{0}
\verb{desrespeitoso}{ô}{}{"-osos ⟨ó⟩}{"-osa ⟨ó⟩}{adj.}{Que não tem respeito; não respeitoso, desrespeitador.}{des.res.pei.to.so}{0}
\verb{dessa}{é}{}{}{}{}{Contração da preposição \textit{de} com o pronome demonstrativo \textit{essa}.}{des.sa}{0}
\verb{dessalinizar}{}{}{}{}{v.t.}{Separar e retirar o sal de algo.}{des.sa.li.ni.zar}{\verboinum{1}}
\verb{dessangrar}{}{}{}{}{v.t.}{Retirar todo o sangue; sangrar.}{des.san.grar}{\verboinum{1}}
\verb{dessarte}{}{}{}{}{adv.}{Destarte.}{des.sar.te}{0}
\verb{desse}{ê}{}{}{}{}{Contração da preposição \textit{de} com o pronome demonstrativo \textit{esse}.}{des.se}{0}
\verb{dessecação}{}{}{"-ões}{}{s.f.}{Ato ou efeito de dessecar, de retirar a umidade; desidratação.}{des.se.ca.ção}{0}
\verb{dessecar}{}{}{}{}{v.t.}{Secar por completo, retirar toda a umidade; enxugar.}{des.se.car}{\verboinum{2}}
\verb{dessedentar}{}{}{}{}{v.t.}{Saciar, matar a sede.}{des.se.den.tar}{\verboinum{1}}
\verb{dessemelhança}{}{}{}{}{s.f.}{Falta de semelhança; desigualdade, diferença. }{des.se.me.lhan.ça}{0}
\verb{dessemelhante}{}{}{}{}{adj.2g.}{Não semelhante; diferente, desigual.}{des.se.me.lhan.te}{0}
\verb{dessemelhar}{}{}{}{}{v.t.}{Tornar dessemelhante, desigual; diferençar.}{des.se.me.lhar}{\verboinum{1}}
\verb{desserviço}{}{}{}{}{s.m.}{Mau serviço; desfavor, prejuízo.}{des.ser.vi.ço}{0}
\verb{desservir}{}{Desus.}{}{}{v.t.}{Prestar um desserviço, causar prejuízo; prejudicar.}{des.ser.vir}{\verboinum{29}}
\verb{dessoldar}{}{}{}{}{v.t.}{Tirar a solda.}{des.sol.dar}{\verboinum{1}}
\verb{dessorar}{}{}{}{}{v.pron.}{Converter"-se, transformar"-se em soro.}{des.so.rar}{0}
\verb{dessorar}{}{}{}{}{v.t.}{Retirar, separar o soro.}{des.so.rar}{\verboinum{1}}
\verb{desta}{é}{}{}{}{}{Contração da preposição \textit{de} com o pronome demonstrativo \textit{esta}.}{des.ta}{0}
\verb{destabocado}{}{Pop.}{}{}{adj.}{Que não tem timidez nem acanhamento; ousado, atrevido, audacioso.}{des.ta.bo.ca.do}{0}
\verb{destabocado}{}{}{}{}{}{Que brinca e fala muito; tagarela, brincalhão.}{des.ta.bo.ca.do}{0}
\verb{destacado}{}{}{}{}{adj.}{Que se destacou; separado, isolado.}{des.ta.ca.do}{0}
\verb{destacado}{}{}{}{}{}{Que (se) sobressai; saliente.}{des.ta.ca.do}{0}
\verb{destacamento}{}{}{}{}{s.m.}{Conjunto de unidades de exército, sob comando único, designado para missão definida e temporária.}{des.ta.ca.men.to}{0}
\verb{destacar}{}{}{}{}{v.t.}{Separar, desligar.}{des.ta.car}{0}
\verb{destacar}{}{}{}{}{}{Fazer sobressair; dar destaque.}{des.ta.car}{\verboinum{2}}
\verb{destacável}{}{}{"-eis}{}{adj.2g.}{Que se pode destacar, separar. }{des.ta.cá.vel}{0}
\verb{destampado}{}{}{}{}{adj.}{Que se destampou, a que se retirou a tampa ou tampo; aberto.}{des.tam.pa.do}{0}
\verb{destampar}{}{}{}{}{v.t.}{Tirar a tampa ou o tampo; abrir, descobrir, destapar.}{des.tam.par}{\verboinum{1}}
\verb{destampatório}{}{Bras.}{}{}{s.m.}{Despropósito, descomedimento, disparate.}{des.tam.pa.tó.rio}{0}
\verb{destampatório}{}{}{}{}{}{Discussão violenta.}{des.tam.pa.tó.rio}{0}
\verb{destapar}{}{}{}{}{v.t.}{Abrir, descobrir o que estava tapado; destampar.}{des.ta.par}{\verboinum{1}}
\verb{destaque}{}{}{}{}{s.m.}{Ato ou efeito de destacar.}{des.ta.que}{0}
\verb{destaque}{}{}{}{}{}{Qualidade daquilo que (se) sobressai.}{des.ta.que}{0}
\verb{destaque}{}{}{}{}{}{Assunto relevante.}{des.ta.que}{0}
\verb{destarte}{}{}{}{}{adv.}{Deste modo; assim; em consequência; diante disto; dessarte.}{des.tar.te}{0}
\verb{deste}{ê}{}{"-s ⟨ê⟩}{"-a ⟨é⟩}{}{Contração da preposição \textit{de} com o pronome demonstrativo \textit{este}.}{des.te}{0}
\verb{destelhar}{}{}{}{}{v.t.}{Retirar ou arrancar as telhas de uma construção.}{des.te.lhar}{\verboinum{1}}
\verb{destemeroso}{ô}{}{"-osos ⟨ó⟩}{"-osa ⟨ó⟩}{adj.}{Sem temor; intrépido, corajoso, destemido, valente.}{des.te.me.ro.so}{0}
\verb{destemido}{}{}{}{}{adj.}{Que não tem temor; intrépido, corajoso, impávido, destemeroso.}{des.te.mi.do}{0}
\verb{destemor}{ô}{}{}{}{s.m.}{Ausência de temor; coragem, valentia, audácia, intrepidez.}{des.te.mor}{0}
\verb{destemperado}{}{}{}{}{adj.}{Descomedido, imoderado, disparatado.}{des.tem.pe.ra.do}{0}
\verb{destemperança}{}{}{}{}{s.f.}{Ausência de temperança; intemperança.}{des.tem.pe.ran.ça}{0}
\verb{destemperar}{}{}{}{}{v.t.}{Fazer perder a têmpera (do aço).}{des.tem.pe.rar}{0}
\verb{destemperar}{}{}{}{}{}{Adicionar água para alterar o sabor.}{des.tem.pe.rar}{0}
\verb{destemperar}{}{}{}{}{}{Desorganizar, alterar.}{des.tem.pe.rar}{0}
\verb{destemperar}{}{}{}{}{v.pron.}{Desatinar, descomedir"-se.}{des.tem.pe.rar}{0}
\verb{destemperar}{}{}{}{}{}{Desafinar.}{des.tem.pe.rar}{\verboinum{1}}
\verb{destempero}{ê}{}{}{}{s.m.}{Despropósito, disparate.}{des.tem.pe.ro}{0}
\verb{desterrar}{}{}{}{}{v.t.}{Fazer sair da terra em que vive; exilar, banir, deportar.}{des.ter.rar}{0}
\verb{desterrar}{}{}{}{}{}{Afugentar, repelir, afastar.}{des.ter.rar}{\verboinum{1}}
\verb{desterro}{ê}{}{}{}{s.m.}{Ato ou efeito de desterrar.}{des.ter.ro}{0}
\verb{desterro}{ê}{}{}{}{}{Pena de degredo.}{des.ter.ro}{0}
\verb{desterro}{ê}{}{}{}{}{Lugar onde vive o indivíduo desterrado.}{des.ter.ro}{0}
\verb{desterro}{ê}{}{}{}{}{Isolamento, solidão.}{des.ter.ro}{0}
\verb{destilação}{}{}{"-ões}{}{s.f.}{Ato de destilar.}{des.ti.la.ção}{0}
\verb{destilação}{}{Quím.}{"-ões}{}{}{Processo de separação de substâncias em que se evapora e condensa um líquido, tornando"-o puro.}{des.ti.la.ção}{0}
\verb{destilação}{}{}{"-ões}{}{}{Estabelecimento em que se faz destilação.}{des.ti.la.ção}{0}
\verb{destilado}{}{}{}{}{adj.}{Que se destilou.}{des.ti.la.do}{0}
\verb{destilado}{}{}{}{}{}{Diz"-se de bebida alcoólica que passou por processo de destilação, como cachaça, vodca, rum, uísque.}{des.ti.la.do}{0}
\verb{destilador}{ô}{}{}{}{s.m.}{Aparelho de destilação constituído por caldeira e condensador; alambique.}{des.ti.la.dor}{0}
\verb{destilar}{}{}{}{}{v.t.}{Evaporar e condensar um líquido para separá"-lo de outras substâncias.}{des.ti.lar}{\verboinum{1}}
\verb{destilaria}{}{}{}{}{s.f.}{Estabelecimento onde se faz destilação, especialmente de bebidas.}{des.ti.la.ri.a}{0}
\verb{destinação}{}{}{"-ões}{}{s.f.}{Ato de destinar.}{des.ti.na.ção}{0}
\verb{destinação}{}{}{"-ões}{}{}{Destino, direção.}{des.ti.na.ção}{0}
\verb{destinar}{}{}{}{}{v.t.}{Determinar, fixar antecipadamente.}{des.ti.nar}{0}
\verb{destinar}{}{}{}{}{}{Resolver, decidir.}{des.ti.nar}{0}
\verb{destinar}{}{}{}{}{v.pron.}{Consagrar"-se, dedicar"-se.}{des.ti.nar}{\verboinum{1}}
\verb{destinatário}{}{}{}{}{s.m.}{Indivíduo a quem se destina algo.}{des.ti.na.tá.rio}{0}
\verb{destingir}{}{}{}{}{v.t.}{Tirar a cor; descorar, desbotar.}{des.tin.gir}{\verboinum{22}}
\verb{destino}{}{}{}{}{s.m.}{Lugar para onde algo ou alguém se dirige; direção.}{des.ti.no}{0}
\verb{destino}{}{}{}{}{}{Fim para o qual se designa algo.}{des.ti.no}{0}
\verb{destino}{}{}{}{}{}{Fatalidade a que estariam sujeitas as pessoas e as coisas do mundo; fado, fortuna.}{des.ti.no}{0}
\verb{destino}{}{}{}{}{}{Aquilo que há de acontecer; futuro.}{des.ti.no}{0}
\verb{destituição}{}{}{"-ões}{}{s.f.}{Ato ou efeito de destituir; demissão.}{des.ti.tu.i.ção}{0}
\verb{destituição}{}{}{"-ões}{}{}{Ausência, carência, desamparo, falta.}{des.ti.tu.i.ção}{0}
\verb{destituir}{}{}{}{}{v.t.}{Demitir, exonerar.}{des.ti.tu.ir}{0}
\verb{destituir}{}{}{}{}{}{Privar.}{des.ti.tu.ir}{\verboinum{26}}
\verb{destoar}{}{}{}{}{v.i.}{Sair do tom; desafinar.}{des.to.ar}{0}
\verb{destoar}{}{}{}{}{v.t.}{Divergir, discordar.}{des.to.ar}{0}
\verb{destoar}{}{}{}{}{}{Não combinar com.}{des.to.ar}{\verboinum{7}}
\verb{destocar}{}{}{}{}{v.t.}{Arrancar tocos de árvores para utilizar o terreno.}{des.to.car}{\verboinum{1}}
\verb{destocar}{}{Bras.}{}{}{v.t.}{Tirar animal da toca; desentocar.}{des.to.car}{\verboinum{1}}
\verb{destocar}{}{}{}{}{v.t.}{Separar, abrir.}{des.to.car}{\verboinum{1}}
\verb{destorcer}{ê}{}{}{}{v.t.}{Endireitar.}{des.tor.cer}{0}
\verb{destorcer}{ê}{}{}{}{}{Virar para o lado oposto.}{des.tor.cer}{\verboinum{15}}
\verb{destra}{é}{}{}{}{s.f.}{A mão direita.}{des.tra}{0}
\verb{destramar}{}{}{}{}{v.t.}{Desfazer a trama; destecer.}{des.tra.mar}{\verboinum{1}}
\verb{destrambelhado}{}{}{}{}{adj.}{Diz"-se de indivíduo desorganizado ou adoidado.}{des.tram.be.lha.do}{0}
\verb{destrambelhado}{}{}{}{}{}{De quem se tirou o trambelho.}{des.tram.be.lha.do}{0}
\verb{destrambelhar}{}{}{}{}{v.i.}{Desarranjar"-se.}{des.tram.be.lhar}{0}
\verb{destrambelhar}{}{}{}{}{v.t.}{Tirar o trambelho.}{des.tram.be.lhar}{\verboinum{1}}
\verb{destrancar}{}{}{}{}{v.t.}{Tirar a tranca.}{des.tran.car}{\verboinum{2}}
\verb{destrançar}{}{}{}{}{v.t.}{Desfazer as tranças; desmanchar.}{des.tran.çar}{\verboinum{3}}
\verb{destratar}{}{Bras.}{}{}{v.t.}{Maltratar com palavras; insultar.}{des.tra.tar}{\verboinum{1}}
\verb{destravancar}{}{}{}{}{v.t.}{Desatravancar, desobstruir.}{des.tra.van.car}{\verboinum{2}}
\verb{destravar}{}{}{}{}{v.t.}{Soltar, desentravar.}{des.tra.var}{0}
\verb{destravar}{}{}{}{}{}{Desprender de trava ou travão.}{des.tra.var}{\verboinum{1}}
\verb{destreinar}{}{}{}{}{v.t.}{Fazer perder o treino; desabituar.}{des.trei.nar}{\verboinum{1}}
\verb{destreza}{ê}{}{}{}{s.f.}{Habilidade, aptidão.}{des.tre.za}{0}
\verb{destreza}{ê}{}{}{}{}{Perspicácia, astúcia, sagacidade.}{des.tre.za}{0}
\verb{destreza}{ê}{}{}{}{}{Qualidade de quem é destro.}{des.tre.za}{0}
\verb{destrinçar}{}{}{}{}{v.t.}{Separar os fios; desenredar.}{des.trin.çar}{0}
\verb{destrinçar}{}{}{}{}{}{Resolver, desenredar um problema.}{des.trin.çar}{0}
\verb{destrinçar}{}{}{}{}{}{Explicar minuciosamente; esmiuçar.}{des.trin.çar}{\verboinum{3}}
\verb{destrinchar}{}{Bras.}{}{}{v.t.}{Destrinçar.}{des.trin.char}{\verboinum{1}}
\verb{destripar}{}{}{}{}{v.t.}{Tirar as tripas; estripar.}{des.tri.par}{\verboinum{1}}
\verb{destro}{é}{}{}{}{adj.}{Relativo ao lado do corpo humano cujos membros são, na maioria dos indivíduos, mais ágeis; direito.}{des.tro}{0}
\verb{destro}{é}{}{}{}{}{Que fica do lado direito.}{des.tro}{0}
\verb{destro}{é}{}{}{}{}{Que tem destreza; ágil.}{des.tro}{0}
\verb{destrocar}{}{}{}{}{v.t.}{Desfazer uma troca.}{des.tro.car}{\verboinum{2}}
\verb{destroçar}{}{}{}{}{v.t.}{Despedaçar, destruir, devastar.}{des.tro.çar}{0}
\verb{destroçar}{}{}{}{}{}{Derrotar.}{des.tro.çar}{0}
\verb{destroçar}{}{}{}{}{}{Dispersar.}{des.tro.çar}{\verboinum{3}}
\verb{destroço}{ô}{}{}{}{s.m.}{Ato ou efeito de destroçar.}{des.tro.ço}{0}
\verb{destroço}{ô}{}{}{}{}{Restos do que foi destroçado; ruína.}{des.tro.ço}{0}
\verb{destróier}{}{}{}{}{s.m.}{Navio de combate relativamente veloz e equipado com armamento de médio calibre; contratorpedeiro.}{des.trói.er}{0}
\verb{destronar}{}{}{}{}{v.t.}{Tirar do trono; destituir.}{des.tro.nar}{0}
\verb{destronar}{}{Fig.}{}{}{}{Rebaixar, humilhar.}{des.tro.nar}{\verboinum{1}}
\verb{destroncar}{}{}{}{}{v.t.}{Separar do tronco; decepar.}{des.tron.car}{0}
\verb{destroncar}{}{Med.}{}{}{}{Tirar membro da articulação, geralmente provocando dor forte; deslocar.}{des.tron.car}{\verboinum{2}}
\verb{destruição}{}{}{"-ões}{}{s.f.}{Ato ou efeito de destruir.}{des.tru.i.ção}{0}
\verb{destruidor}{ô}{}{}{}{adj.}{Que destrói; destrutivo, destrutor.}{des.tru.i.dor}{0}
\verb{destruir}{}{}{}{}{v.t.}{Exterminar, aniquilar, matar.}{des.tru.ir}{0}
\verb{destruir}{}{}{}{}{}{Devastar, destroçar.}{des.tru.ir}{0}
\verb{destruir}{}{}{}{}{}{Demolir, arruinar, desmanchar.}{des.tru.ir}{0}
\verb{destruir}{}{}{}{}{}{Desorganizar, transtornar, desfazer.}{des.tru.ir}{\verboinum{26}}
\verb{destrutível}{}{}{"-eis}{}{adj.2g.}{Que pode ser destruído.}{des.tru.tí.vel}{0}
\verb{destrutivo}{}{}{}{}{adj.}{Que destrói; destruidor.}{des.tru.ti.vo}{0}
\verb{desumanidade}{}{}{}{}{s.f.}{Qualidade de desumano; crueldade.}{de.su.ma.ni.da.de}{0}
\verb{desumanidade}{}{}{}{}{}{Ato desumano.}{de.su.ma.ni.da.de}{0}
\verb{desumanizar}{}{}{}{}{v.t.}{Tornar desumano; desumanar.}{de.su.ma.ni.zar}{\verboinum{1}}
\verb{desumano}{}{}{}{}{adj.}{Que não é humano; bestial, desnaturado, cruel.}{de.su.ma.no}{0}
\verb{desunião}{}{}{"-ões}{}{s.f.}{Ausência de união.}{de.su.ni.ão}{0}
\verb{desunião}{}{}{"-ões}{}{}{Discórdia, desavença.}{de.su.ni.ão}{0}
\verb{desunião}{}{}{"-ões}{}{}{Separação.}{de.su.ni.ão}{0}
\verb{desunir}{}{}{}{}{v.t.}{Desfazer a união; separar.}{de.su.nir}{0}
\verb{desunir}{}{}{}{}{}{Gerar discórdia ou desacordo.}{de.su.nir}{\verboinum{18}}
\verb{desusado}{}{}{}{}{adj.}{Que não se usa mais.}{de.su.sa.do}{0}
\verb{desusado}{}{}{}{}{}{Que não costuma ser usado; inusitado, insólito.}{de.su.sa.do}{0}
\verb{desuso}{}{}{}{}{s.m.}{Falta de uso ou de costume.}{de.su.so}{0}
\verb{desvairado}{}{}{}{}{adj.}{Que perdeu o juízo; alucinado, desorientado.}{des.vai.ra.do}{0}
\verb{desvairado}{}{}{}{}{}{Que denota desatino.}{des.vai.ra.do}{0}
\verb{desvairamento}{}{}{}{}{s.m.}{Desvario, alucinação.}{des.vai.ra.men.to}{0}
\verb{desvairar}{}{}{}{}{v.t.}{Alucinar"-se.}{des.vai.rar}{0}
\verb{desvairar}{}{}{}{}{}{Praticar desatino.}{des.vai.rar}{\verboinum{1}}
\verb{desvalia}{}{}{}{}{s.f.}{Falta de valia; desvalimento.}{des.va.li.a}{0}
\verb{desvalido}{}{}{}{}{adj.}{Que não tem valia ou valimento.}{des.va.li.do}{0}
\verb{desvalido}{}{}{}{}{}{Miserável, desgraçado.}{des.va.li.do}{0}
\verb{desvalido}{}{}{}{}{}{Desamparado, desprotegido, abandonado.}{des.va.li.do}{0}
\verb{desvalioso}{ô}{}{"-osos ⟨ó⟩}{"-osa ⟨ó⟩}{adj.}{Que não é valioso; sem valia.}{des.va.li.o.so}{0}
\verb{desvalor}{ô}{}{}{}{s.m.}{Falta ou ausência de valor.}{des.va.lor}{0}
\verb{desvalorização}{}{}{"-ões}{}{s.f.}{Perda de valor.}{des.va.lo.ri.za.ção}{0}
\verb{desvalorizar}{}{}{}{}{v.t.}{Tirar ou perder o valor.}{des.va.lo.ri.zar}{\verboinum{1}}
\verb{desvanecer}{ê}{}{}{}{v.t.}{Fazer desaparecer; extinguir.}{desvanecer}{0}
\verb{desvanecer}{ê}{}{}{}{}{Causar vaidade, orgulho.}{desvanecer}{0}
\verb{desvanecer}{ê}{}{}{}{v.pron.}{Esmorecer, desbotar.}{desvanecer}{0}
\verb{desvanecer}{ê}{}{}{}{}{Mostrar"-se vaidoso; ufanar"-se.}{desvanecer}{\verboinum{15}}
\verb{desvanecido}{}{}{}{}{adj.}{Extinto, desfeito.}{des.va.ne.ci.do}{0}
\verb{desvanecido}{}{}{}{}{}{Desbotado.}{des.va.ne.ci.do}{0}
\verb{desvanecido}{}{}{}{}{}{Orgulhoso, vaidoso.}{des.va.ne.ci.do}{0}
\verb{desvanecimento}{}{}{}{}{s.m.}{Ato ou efeito de desvanecer.}{des.va.ne.ci.men.to}{0}
\verb{desvanecimento}{}{}{}{}{}{Vaidade, presunção, orgulho.}{des.va.ne.ci.men.to}{0}
\verb{desvantagem}{}{}{"-ens}{}{s.f.}{Falta de vantagem; inferioridade.}{des.van.ta.gem}{0}
\verb{desvantajoso}{ô}{}{"-osos ⟨ó⟩}{"-osa ⟨ó⟩}{adj.}{Que oferece desvantagem; não vantajoso.}{des.van.ta.jo.so}{0}
\verb{desvão}{}{}{"-ãos}{}{s.m.}{Vão entre o forro e o telhado da casa; sótão.}{des.vão}{0}
\verb{desvão}{}{}{"-ãos}{}{}{Recanto oculto; esconderijo.}{des.vão}{0}
\verb{desvario}{}{}{}{}{s.m.}{Gesto de loucura; desatino, delírio.}{des.va.ri.o}{0}
\verb{desvelar}{}{}{}{}{v.t.}{Tirar o véu; revelar, descobrir.}{des.ve.lar}{0}
\verb{desvelar}{}{}{}{}{v.pron.}{Ter desvelo; zelar com atenção.}{des.ve.lar}{\verboinum{1}}
\verb{desvelo}{ê}{}{}{}{s.m.}{Grande cuidado; dedicação, vigilância.}{des.ve.lo}{0}
\verb{desvencilhar}{}{}{}{}{v.t.}{Soltar do vencilho; desprender.}{des.ven.ci.lhar}{0}
\verb{desvencilhar}{}{}{}{}{v.pron.}{Livrar"-se.}{des.ven.ci.lhar}{\verboinum{1}}
\verb{desvendar}{}{}{}{}{v.t.}{Revelar, descobrir.}{des.ven.dar}{0}
\verb{desvendar}{}{}{}{}{}{Tirar a venda dos olhos; destapar.}{des.ven.dar}{\verboinum{1}}
\verb{desventura}{}{}{}{}{s.f.}{Falta de ventura; infelicidade, desgraça.}{des.ven.tu.ra}{0}
\verb{desventurado}{}{}{}{}{adj.}{Infeliz, desgraçado.}{des.ven.tu.ra.do}{0}
\verb{desvestir}{}{}{}{}{v.t.}{Tirar do corpo; despir.}{des.ves.tir}{\verboinum{18}}
\verb{desviado}{}{}{}{}{adj.}{Remoto, afastado, longínquo.}{des.vi.a.do}{0}
\verb{desviar}{}{}{}{}{v.t.}{Mudar a direção.}{des.vi.ar}{0}
\verb{desviar}{}{}{}{}{}{Afastar, atalhar, evitar.}{des.vi.ar}{0}
\verb{desviar}{}{}{}{}{}{Realocar dinheiro fraudulentamente.}{des.vi.ar}{0}
\verb{desviar}{}{}{}{}{}{Tirar do bom caminho; desencaminhar.}{des.vi.ar}{\verboinum{1}}
\verb{desvincar}{}{}{}{}{v.t.}{Tirar os vincos; alisar, desenrugar.}{des.vin.car}{\verboinum{2}}
\verb{desvincular}{}{}{}{}{v.t.}{Desfazer os vínculos; desligar, liberar, desatar.}{des.vin.cu.lar}{\verboinum{1}}
\verb{desvio}{}{}{}{}{s.m.}{Ato ou efeito de desviar.}{des.vi.o}{0}
\verb{desvio}{}{}{}{}{}{Linha secundária em rodovias ou ferrovias para evitar tráfego ou interrupção na via principal.}{des.vi.o}{0}
\verb{desvio}{}{}{}{}{}{Afastamento do assunto; digressão.}{des.vi.o}{0}
\verb{desvio}{}{}{}{}{}{Realocação fraudulenta de dinheiro.}{des.vi.o}{0}
\verb{desvirar}{}{}{}{}{v.t.}{Tornar à posição normal.}{des.vi.rar}{\verboinum{1}}
\verb{desvirginar}{}{}{}{}{v.t.}{Tirar a virgindade; deflorar.}{des.vir.gi.nar}{\verboinum{1}}
\verb{desvirilizar}{}{}{}{}{v.t.}{Tirar a virilidade.}{des.vi.ri.li.zar}{\verboinum{1}}
\verb{desvirtuamento}{}{}{}{}{s.m.}{Ato ou efeito de desvirtuar.}{des.vir.tu.a.men.to}{0}
\verb{desvirtuar}{}{}{}{}{v.t.}{Tirar a virtude, o valor.}{des.vir.tu.ar}{0}
\verb{desvirtuar}{}{}{}{}{}{Interpretar desfavoravelmente; distorcer.}{des.vir.tu.ar}{\verboinum{1}}
\verb{desvitalizar}{}{}{}{}{v.t.}{Tirar a vitalidade.}{des.vi.ta.li.zar}{\verboinum{1}}
\verb{detalhar}{}{}{}{}{v.t.}{Expor minuciosamente; particularizar.}{de.ta.lhar}{\verboinum{1}}
\verb{detalhe}{}{}{}{}{s.m.}{Minúcia, pormenor, particularidade.}{de.ta.lhe}{0}
\verb{detecção}{}{}{"-ões}{}{s.f.}{Ato ou efeito de detectar.}{de.tec.ção}{0}
\verb{detectar}{}{}{}{}{v.t.}{Perceber a existência de algo que não está manifesto.}{de.tec.tar}{\verboinum{1}}
\verb{detective}{}{}{}{}{}{Var. de \textit{detetive}.}{de.tec.ti.ve}{0}
\verb{detector}{ô}{}{}{}{s.m.}{Aparelho capaz de detectar a presença de objetos ou fenômenos específicos.}{de.tec.tor}{0}
\verb{detença}{}{}{}{}{s.f.}{Demora, delonga.}{de.ten.ça}{0}
\verb{detenção}{}{}{"-ões}{}{s.f.}{Ato de deter.}{de.ten.ção}{0}
\verb{detenção}{}{}{"-ões}{}{}{Prisão provisória.}{de.ten.ção}{0}
\verb{detenção}{}{}{"-ões}{}{}{Possessão ilegítima.}{de.ten.ção}{0}
\verb{detento}{}{Bras.}{}{}{s.m.}{Indivíduo que se encontra detido ou preso.}{de.ten.to}{0}
\verb{detentor}{ô}{}{}{}{adj.}{Que detém.}{de.ten.tor}{0}
\verb{deter}{ê}{}{}{}{v.t.}{Fazer parar; interromper.}{de.ter}{0}
\verb{deter}{ê}{}{}{}{}{Suspender, reprimir.}{de.ter}{0}
\verb{deter}{ê}{}{}{}{}{Fazer demorar; reter.}{de.ter}{0}
\verb{deter}{ê}{}{}{}{}{Reter em seu poder.}{de.ter}{0}
\verb{deter}{ê}{}{}{}{}{Determinar a prisão provisória.}{de.ter}{0}
\verb{deter}{ê}{}{}{}{v.pron.}{Ocupar"-se demoradamente; demorar"-se.}{de.ter}{\verboinum{39}}
\verb{detergente}{}{}{}{}{s.m.}{Substância que remove materiais gordurosos.}{de.ter.gen.te}{0}
\verb{detergir}{}{}{}{}{v.t.}{Limpar por meio de substâncias químicas.}{de.ter.gir}{\verboinum{22}}
\verb{deterioração}{}{}{"-ões}{}{s.f.}{Ato ou efeito de deteriorar; degeneração, ruína.}{de.te.ri.o.ra.ção}{0}
\verb{deteriorar}{}{}{}{}{v.t.}{Estragar, danificar, apodrecer.}{de.te.ri.o.rar}{0}
\verb{deteriorar}{}{}{}{}{}{Corromper, decair, agravar, piorar, estragar.}{de.te.ri.o.rar}{\verboinum{1}}
\verb{determinação}{}{}{"-ões}{}{s.f.}{Ato ou efeito de determinar.}{de.ter.mi.na.ção}{0}
\verb{determinação}{}{}{"-ões}{}{}{Decisão, resolução, ordem.}{de.ter.mi.na.ção}{0}
\verb{determinação}{}{}{"-ões}{}{}{Capacidade pessoal de decisão ou de persistência.}{de.ter.mi.na.ção}{0}
\verb{determinado}{}{}{}{}{adj.}{Estabelecido, fixo, definido.}{de.ter.mi.na.do}{0}
\verb{determinado}{}{}{}{}{}{Decidido, certo, resoluto.}{de.ter.mi.na.do}{0}
\verb{determinante}{}{}{}{}{adj.2g.}{Que determina; decisivo.}{de.ter.mi.nan.te}{0}
\verb{determinar}{}{}{}{}{v.t.}{Fixar, marcar.}{de.ter.mi.nar}{0}
\verb{determinar}{}{}{}{}{}{Prescrever, estabelecer.}{de.ter.mi.nar}{0}
\verb{determinar}{}{}{}{}{}{Resolver, decidir.}{de.ter.mi.nar}{0}
\verb{determinar}{}{}{}{}{}{Causar, motivar.}{de.ter.mi.nar}{\verboinum{1}}
\verb{determinativo}{}{}{}{}{adj.}{Que determina.}{de.ter.mi.na.ti.vo}{0}
\verb{determinativo}{}{}{}{}{}{Restritivo, definitivo.}{de.ter.mi.na.ti.vo}{0}
\verb{determinismo}{}{}{}{}{s.m.}{Crença segundo a qual todos os acontecimentos do Universo estão sujeitos a leis preterderminadas, que nega o livre"-arbítrio.   }{de.ter.mi.nis.mo}{0}
\verb{determinista}{}{}{}{}{adj.2g.}{Relativo ao determinismo.}{de.ter.mi.nis.ta}{0}
\verb{determinista}{}{}{}{}{}{Adepto do determinismo.}{de.ter.mi.nis.ta}{0}
\verb{detestar}{}{}{}{}{v.t.}{Ter aversão; odiar.}{de.tes.tar}{\verboinum{1}}
\verb{detestável}{}{}{"-eis}{}{adj.2g.}{Abominável, péssimo, insuportável.}{de.tes.tá.vel}{0}
\verb{detetive}{}{}{}{}{s.2g.}{Profissional que investiga crimes ou fatos da vida particular de outras pessoas.}{de.te.ti.ve}{0}
\verb{detido}{}{}{}{}{adj.}{Que está preso em caráter provisório.}{de.ti.do}{0}
\verb{detido}{}{}{}{}{}{Demorado.}{de.ti.do}{0}
\verb{detonação}{}{}{"-ões}{}{s.f.}{Ato ou efeito de detonar.}{de.to.na.ção}{0}
\verb{detonador}{ô}{}{}{}{s.m.}{Aparelho ou substância que detona cargas explosivas.}{de.to.na.dor}{0}
\verb{detonar}{}{}{}{}{v.t.}{Fazer explodir.}{de.to.nar}{\verboinum{1}}
\verb{detração}{}{}{"-ões}{}{s.f.}{Ato ou efeito de detrair; difamação, depreciação, maledicência.}{de.tra.ção}{0}
\verb{detrair}{}{}{}{}{v.t.}{Depreciar o mérito ou a reputação; difamar, detratar.}{de.tra.ir}{\verboinum{19}}
\verb{detrás}{}{}{}{}{adv.}{Na parte posterior.}{de.trás}{0}
\verb{detrás}{}{}{}{}{}{Em seguida, depois.}{de.trás}{0}
\verb{detratar}{}{Bras.}{}{}{v.t.}{Detrair.}{de.tra.tar}{\verboinum{1}}
\verb{detrator}{ô}{}{}{}{adj.}{Que detrata ou detrai.}{de.tra.tor}{0}
\verb{detrimento}{}{}{}{}{s.m.}{Prejuízo, dano, perda.}{de.tri.men.to}{0}
\verb{detrito}{}{}{}{}{s.m.}{Resíduo, resto.}{de.tri.to}{0}
\verb{deturpação}{}{}{"-ões}{}{s.f.}{Ato ou efeito de deturpar.}{de.tur.pa.ção}{0}
\verb{deturpar}{}{}{}{}{v.t.}{Tornar torpe, feio; desfigurar.}{de.tur.par}{0}
\verb{deturpar}{}{}{}{}{}{Viciar, corromper, estragar, manchar, adulterar.}{de.tur.par}{0}
\verb{deturpar}{}{}{}{}{}{Modificar de maneira tendenciosa; distorcer.}{de.tur.par}{\verboinum{1}}
\verb{deus}{}{Relig.}{}{}{s.m.}{Entidade sobrenatural detentora de poder sobre o destino dos homens e da natureza e geralmente objeto de culto; divindade.}{deus}{0}
\verb{deusa}{}{}{}{}{s.f.}{Divindade feminina.}{deu.sa}{0}
\verb{deusa}{}{Fig.}{}{}{}{Mulher muito bela.}{deu.sa}{0}
\verb{deus"-dará}{}{}{}{}{}{Usado na expressão \textit{ao deus"-dará}: a esmo, ao acaso, à deriva.}{deus"-da.rá}{0}
\verb{deus"-nos"-acuda}{}{}{}{}{s.m.}{Confusão, tumulto, desordem.}{deus"-nos"-a.cu.da}{0}
\verb{deutério}{}{Quím.}{}{}{s.m.}{Isótopo do hidrogênio com número de massa igual a 2.}{deu.té.rio}{0}
\verb{deuteronômio}{}{}{}{}{s.m.}{O quinto livro do Antigo Testamento e parte do Pentateuco.}{deu.te.ro.nô.mio}{0}
\verb{devagar}{}{}{}{}{adv.}{Lentamente, vagarosamente, sem pressa ou rapidez.}{de.va.gar}{0}
\verb{devanear}{}{}{}{}{v.t.}{Imaginar, sonhar, fantasiar.}{de.va.ne.ar}{0}
\verb{devanear}{}{}{}{}{v.i.}{Dizer coisas sem nexo; delirar.}{de.va.ne.ar}{\verboinum{4}}
\verb{devaneio}{ê}{}{}{}{s.m.}{Fantasia, sonho.}{de.va.nei.o}{0}
\verb{devassa}{}{}{}{}{s.f.}{Ato ou efeito de devassar.}{de.vas.sa}{0}
\verb{devassa}{}{}{}{}{}{Apuração detalhada de um ato criminoso.}{de.vas.sa}{0}
\verb{devassar}{}{}{}{}{v.t.}{Invadir para ver o que se passa em lugar proibido.}{de.vas.sar}{0}
\verb{devassar}{}{}{}{}{}{Submeter a devassa; investigar, esquadrinhar, pesquisar.}{de.vas.sar}{0}
\verb{devassar}{}{}{}{}{}{Penetrar na essência; desvendar.}{de.vas.sar}{\verboinum{1}}
\verb{devassidão}{}{}{"-ões}{}{s.f.}{Qualidade de devasso; libertinagem.}{de.vas.si.dão}{0}
\verb{devasso}{}{}{}{}{adj.}{Libertino, dissoluto, depravado.}{de.vas.so}{0}
\verb{devastação}{}{}{"-ões}{}{s.f.}{Ato ou efeito de devastar; destruição.}{de.vas.ta.ção}{0}
\verb{devastar}{}{}{}{}{v.t.}{Destruir, arruinar, danificar.}{de.vas.tar}{\verboinum{1}}
\verb{deve}{é}{}{}{}{s.m.}{Débito lançado no livro comercial de uma instituição.}{de.ve}{0}
\verb{deve}{é}{}{}{}{}{A coluna desse livro onde se registram os débitos.}{de.ve}{0}
\verb{devedor}{ô}{}{}{}{adj.}{Que deve para alguém.}{de.ve.dor}{0}
\verb{devedor}{ô}{}{}{}{}{Que representa débito. (\textit{Saldo devedor.})}{de.ve.dor}{0}
\verb{dever}{ê}{}{}{}{v.t.}{Ter obrigação de. (\textit{Você deve estar aqui amanhã.})}{de.ver}{0}
\verb{dever}{ê}{}{}{}{}{Ter necessidade de; precisar. (\textit{Devo comer uma torta de limão agora.})}{de.ver}{0}
\verb{dever}{ê}{}{}{}{}{Ter de pagar. (\textit{Ainda devo 20 reais a ele.})}{de.ver}{0}
\verb{dever}{ê}{}{}{}{}{Estar em agradecimento. (\textit{Devo tudo isso ao meu grande amigo.})}{de.ver}{0}
\verb{dever}{ê}{}{}{}{}{Haver possibilidade de. (\textit{Amanhã deve fazer sol.})}{de.ver}{0}
\verb{dever}{ê}{}{}{}{v.i.}{Ter dívidas. (\textit{Já paguei muito mas ainda devo.})}{de.ver}{\verboinum{12}}
\verb{dever}{ê}{}{}{}{s.m.}{Tarefa, obrigação. (\textit{Faça logo o seu dever.})}{de.ver}{0}
\verb{deveras}{é}{}{}{}{adv.}{Em alto grau; muito, realmente.}{de.ve.ras}{0}
\verb{deverbal}{}{Gram.}{"-ais}{}{adj.2g.}{Diz"-se de palavra formada por derivação a partir de um verbo.}{de.ver.bal}{0}
\verb{devido}{}{}{}{}{adj.}{Que se deve.}{de.vi.do}{0}
\verb{devido}{}{}{}{}{s.m.}{Aquilo que é direito ou dever; o legítimo.}{de.vi.do}{0}
\verb{devido}{}{}{}{}{}{Aquilo que se deve.}{de.vi.do}{0}
\verb{devido}{}{}{}{}{}{Usado na expressão \textit{devido a}: por causa de.}{de.vi.do}{0}
\verb{devoção}{}{}{"-ões}{}{s.f.}{Apego fervoroso à religião.}{de.vo.ção}{0}
\verb{devoção}{}{}{"-ões}{}{}{Prática religiosa.}{de.vo.ção}{0}
\verb{devoção}{}{}{"-ões}{}{}{Afeto, dedicação, afeição.}{de.vo.ção}{0}
\verb{devolução}{}{}{"-ões}{}{s.f.}{Ato ou efeito de devolver.}{de.vo.lu.ção}{0}
\verb{devoluto}{}{}{}{}{adj.}{Sem habitantes; vago, vazio.}{de.vo.lu.to}{0}
\verb{devoluto}{}{Jur.}{}{}{}{Adquirido por devolução.}{de.vo.lu.to}{0}
\verb{devolver}{ê}{}{}{}{v.t.}{Dar ou enviar alguma coisa de volta.}{de.vol.ver}{0}
\verb{devolver}{ê}{}{}{}{}{Dizer em resposta; replicar.}{de.vol.ver}{0}
\verb{devolver}{ê}{}{}{}{}{Recusar o recebimento de algo.}{de.vol.ver}{\verboinum{12}}
\verb{devoniano}{}{Geol.}{}{}{adj.}{Diz"-se de período da era paleozoica posterior ao siluriano e anterior ao carbonífero.}{de.vo.ni.a.no}{0}
\verb{devorador}{ô}{}{}{}{adj.}{Que devora ou consome.}{de.vo.ra.dor}{0}
\verb{devorador}{ô}{}{}{}{}{Comilão, glutão.}{de.vo.ra.dor}{0}
\verb{devorar}{}{}{}{}{v.t.}{Engolir avidamente.}{de.vo.rar}{0}
\verb{devorar}{}{}{}{}{}{Carcomer, destruir, consumir.}{de.vo.rar}{0}
\verb{devorar}{}{}{}{}{}{Percorrer com rapidez.}{de.vo.rar}{\verboinum{1}}
\verb{devotamento}{}{}{}{}{s.m.}{Ato de devotar; dedicação.}{de.vo.ta.men.to}{0}
\verb{devotar}{}{}{}{}{v.t.}{Oferecer em voto, promessa solene.}{de.vo.tar}{0}
\verb{devotar}{}{}{}{}{}{Dedicar, consagrar.}{de.vo.tar}{\verboinum{1}}
\verb{devoto}{ó}{}{}{}{adj.}{Que tem devoção; religioso, beato.}{de.vo.to}{0}
\verb{dextrose}{es\ldots{}ó}{Quím.}{}{}{s.f.}{Glicose.}{dex.tro.se}{0}
\verb{dez}{é}{}{}{}{num.}{Nome dado à quantidade expressa pelo número 10.}{dez}{0}
\verb{dezembro}{}{}{}{}{s.m.}{O décimo segundo mês do ano civil.}{de.zem.bro}{0}
\verb{dezena}{}{}{}{}{s.f.}{Conjunto de dez unidades de qualquer coisa.}{de.ze.na}{0}
\verb{dezena}{}{Mat.}{}{}{}{A segunda posição a contar da direita no sistema numérico decimal.}{de.ze.na}{0}
\verb{dezena}{}{}{}{}{}{Período de dez dias.}{de.ze.na}{0}
\verb{dezenove}{ó}{}{}{}{num.}{Nome dado à quantidade expressa pelo número 19.}{de.ze.no.ve}{0}
\verb{dezesseis}{}{}{}{}{num.}{Nome dado à quantidade expressa pelo número 16.}{de.zes.seis}{0}
\verb{dezessete}{é}{}{}{}{num.}{Nome dado à quantidade expressa pelo número 17.}{de.zes.se.te}{0}
\verb{dezoito}{ô}{}{}{}{num.}{Nome dado à quantidade expressa pelo número 18.}{de.zoi.to}{0}
\verb{DF}{}{}{}{}{}{Sigla de Distrito Federal.}{DF}{0}
\verb{dia}{}{}{}{}{s.m.}{Período de tempo em que há claridade solar, entre o nascer e o pôr do sol.}{di.a}{0}
\verb{dia}{}{}{}{}{}{Período de 24 horas, equivalente a uma rotação completa da Terra sobre seu próprio eixo.}{di.a}{0}
\verb{dia}{}{}{}{}{}{O tempo presente; atualidade. (\textit{Estes são os assuntos do dia.})}{di.a}{0}
\verb{dia a dia}{}{}{dias a dias \textit{ou} dia a dias}{}{s.m.}{A sucessão dos dias; o cotidiano, a rotina.}{di.a a di.a}{0}
\verb{diabete}{é}{Med.}{}{}{s.2g.}{Diabetes.}{di.a.be.te}{0}
\verb{diabetes}{é}{Med.}{}{}{s.2g.}{Doença metabólica caracterizada pela deficiência de insulina e aumento da taxa de glicose no sangue.}{di.a.be.tes}{0}
\verb{diabético}{}{}{}{}{adj.}{Relativo a diabetes.}{di.a.bé.ti.co}{0}
\verb{diabético}{}{}{}{}{s.m.}{Indivíduo que sofre de diabetes.}{di.a.bé.ti.co}{0}
\verb{diabo}{}{Relig.}{}{}{s.m.}{O espírito ou a personificação do mal; demônio, satanás.}{di.a.bo}{0}
\verb{diabo}{}{}{}{}{interj.}{Expressão que denota raiva ou impaciência.}{di.a.bo}{0}
\verb{diabólico}{}{}{}{}{adj.}{Relativo a diabo.}{di.a.bó.li.co}{0}
\verb{diabólico}{}{}{}{}{}{Terrivel, infernal, obscuro, complicado.}{di.a.bó.li.co}{0}
\verb{diabrete}{ê}{}{}{}{s.m.}{Pequeno diabo.}{di.a.bre.te}{0}
\verb{diabrete}{ê}{Pop.}{}{}{}{Criança travessa, indisciplinada.}{di.a.bre.te}{0}
\verb{diabrura}{}{}{}{}{s.f.}{Ato ou coisa própria de diabo.}{di.a.bru.ra}{0}
\verb{diabrura}{}{Pop.}{}{}{}{Travessura, traquinice.}{di.a.bru.ra}{0}
\verb{diacho}{}{}{}{}{s.m.}{Forma eufêmica da palavra \textit{diabo}.}{di.a.cho}{0}
\verb{diacho}{}{}{}{}{interj.}{Expressão que denota raiva ou impaciência.}{di.a.cho}{0}
\verb{diaconato}{}{}{}{}{s.m.}{Dignidade ou função de diácono.}{di.a.co.na.to}{0}
\verb{diácono}{}{}{}{}{s.m.}{Clérigo que recebeu as segundas ordens, imediatamente inferior ao padre.}{di.á.co.no}{0}
\verb{diacrítico}{}{Gram.}{}{}{adj.}{Diz"-se de sinal gráfico que confere à letra um valor especial ou diferenciado.}{di.a.crí.ti.co}{0}
\verb{diacronia}{}{Gram.}{}{}{s.f.}{Abordagem descritiva de uma língua ao longo de seu desenvolvimento histórico, em oposição a \textit{sincronia}.}{di.a.cro.ni.a}{0}
\verb{diacrônico}{}{}{}{}{adj.}{Relativo a diacronia.}{di.a.crô.ni.co}{0}
\verb{diadema}{}{}{}{}{s.m.}{Adorno fartamente ornamentado que reis e rainhas usam na cabeça; coroa.}{di.a.de.ma}{0}
\verb{diadema}{}{}{}{}{}{Ornato ou joia com que as mulheres adornam o toucado ou a fronte.}{di.a.de.ma}{0}
\verb{diadema}{}{Fig.}{}{}{}{A dignidade real.}{di.a.de.ma}{0}
\verb{diáfano}{}{}{}{}{adj.}{Que deixa a luz passar; transparente.}{di.á.fa.no}{0}
\verb{diáfano}{}{Fig.}{}{}{}{Muito magro ou sem substância; vago.}{di.á.fa.no}{0}
\verb{diafragma}{}{Anat.}{}{}{s.m.}{Grande músculo disposto horizontalmente e  que separa a cavidade torácica da cavidade abdominal.}{di.a.frag.ma}{0}
\verb{diafragma}{}{Med.}{}{}{}{Membrana elástica com anel flexível usada no fundo da vagina como método anticoncepcional.}{di.a.frag.ma}{0}
\verb{diafragma}{}{Fís.}{}{}{}{Dispositivo com abertura variável que controla a intensidade luminosa em um sistema óptico, como as lentes objetivas das câmaras.}{di.a.frag.ma}{0}
\verb{diafragmático}{}{}{}{}{adj.}{Relativo a diafragma.}{di.a.frag.má.ti.co}{0}
\verb{diagnose}{ó}{Med.}{}{}{s.f.}{Diagnóstico.}{di.ag.no.se}{0}
\verb{diagnosticar}{}{}{}{}{v.t.}{Fazer um diagnóstico.}{di.ag.nos.ti.car}{0}
\verb{diagnosticar}{}{}{}{}{}{Estabelecer como diagnóstico.}{di.ag.nos.ti.car}{\verboinum{2}}
\verb{diagnóstico}{}{Med.}{}{}{s.m.}{Determinação de uma doença através da observação dos sintomas e de exames clínicos, radiológicos e laboratoriais.}{di.ag.nós.ti.co}{0}
\verb{diagonal}{}{}{"-ais}{}{adj.2g.}{Inclinado, oblíquo.}{di.a.go.nal}{0}
\verb{diagonal}{}{}{"-ais}{}{s.f.}{A direção transversal.}{di.a.go.nal}{0}
\verb{diagrama}{}{}{}{}{s.m.}{Representação gráfica de um fenômeno.}{di.a.gra.ma}{0}
\verb{diagramação}{}{}{"-ões}{}{s.f.}{Ato ou efeito de diagramar.}{di.a.gra.ma.ção}{0}
\verb{diagramador}{ô}{}{}{}{s.m.}{Profissional que cuida da diagramação dos impressos.}{di.a.gra.ma.dor}{0}
\verb{diagramar}{}{}{}{}{v.t.}{Projetar e executar a disposição dos elementos (textos, figuras, espaços) de um impresso, escolhendo o tamanho das letras, a medida das colunas.}{di.a.gra.mar}{\verboinum{1}}
\verb{dial}{}{}{}{}{s.m.}{Dispositivo de sintonia em um receptor de ondas de rádio, composto de botão giratório, capacitor de sintonia e escala de frequência.}{di.al}{0}
\verb{dialetal}{}{}{"-ais}{}{adj.2g.}{Relativo a dialeto.}{di.a.le.tal}{0}
\verb{dialética}{}{Filos.}{}{}{s.f.}{Método lógico ou forma de argumentação que, mediante a contraposição dos diferentes juízos emitidos acerca de uma questão, procura alcançar a verdade dela.}{di.a.lé.ti.ca}{0}
\verb{dialético}{}{}{}{}{adj.}{Relativo a dialética.}{di.a.lé.ti.co}{0}
\verb{dialético}{}{}{}{}{}{Caracterizado pela dialética.}{di.a.lé.ti.co}{0}
\verb{dialeto}{é}{}{}{}{s.m.}{Cada uma das variantes regionais ou sociais de uma língua, com características fonológicas, lexicais, sintáticas e semânticas próprias mas compreensível por falantes de outras variantes da mesma língua.}{di.a.le.to}{0}
\verb{diálise}{}{Bioquím.}{}{}{s.f.}{Processo de separação de substâncias que passam ou não por membranas semipermeáveis.}{di.á.li.se}{0}
\verb{diálise}{}{Med.}{}{}{}{Técnica que visa suplementar as falhas da função renal de certos indivíduos que não conseguem eliminar água e produtos de excreção do sangue.}{di.á.li.se}{0}
\verb{dialogal}{}{}{"-ais}{}{adj.2g.}{Relativo a diálogo.}{di.a.lo.gal}{0}
\verb{dialogar}{}{}{}{}{v.i.}{Falar em turnos alternados; conversar.}{di.a.lo.gar}{0}
\verb{dialogar}{}{}{}{}{}{Buscar entendimento (pessoas, grupos, nações) na solução de problemas comuns.}{di.a.lo.gar}{\verboinum{5}}
\verb{diálogo}{}{}{}{}{s.m.}{Ato ou efeito de dialogar; conversação.}{di.á.lo.go}{0}
\verb{diálogo}{}{}{}{}{}{Obra literária ou científica escrita em forma de diálogo.}{di.á.lo.go}{0}
\verb{diamante}{}{}{}{}{s.m.}{Mineral composto de carbono puro cristalizado, duro e brilhante.}{di.a.man.te}{0}
\verb{diamantífero}{}{}{}{}{adj.}{Diz"-se de região ou terreno em que há diamantes.}{di.a.man.tí.fe.ro}{0}
\verb{diamantino}{}{}{}{}{adj.}{Feito de ou semelhante ao diamante.}{di.a.man.ti.no}{0}
\verb{diamantino}{}{}{}{}{}{Muito estimável; precioso.}{di.a.man.ti.no}{0}
\verb{diametral}{}{}{"-ais}{}{adj.2g.}{Relativo a diâmetro.}{di.a.me.tral}{0}
\verb{diâmetro}{}{Geom.}{}{}{s.m.}{Segmento de reta que passa por dois pontos da circunferência e pelo seu centro.}{di.â.me.tro}{0}
\verb{diante}{}{}{}{}{adv.}{Na frente (de), em presença (de).}{di.an.te}{0}
\verb{dianteira}{ê}{}{}{}{s.f.}{A parte mais à frente de algo; frente, vanguarda.}{di.an.tei.ra}{0}
\verb{dianteiro}{ê}{}{}{}{adj.}{Que está na frente ou em primeiro lugar.}{di.an.tei.ro}{0}
\verb{diapasão}{}{Mús.}{"-ões}{}{s.m.}{Peça que emite som em uma frequência fixa e constante, usado para afinar instrumentos musicais.}{di.a.pa.são}{0}
\verb{diapasão}{}{}{"-ões}{}{}{A nota emitida por essa peça.}{di.a.pa.são}{0}
\verb{diapositivo}{}{}{}{}{s.m.}{Foto transparente positiva que pode ser projetada; \textit{slide}.}{di.a.po.si.ti.vo}{0}
\verb{diária}{}{}{}{}{s.f.}{Preço para cada dia de hospedagem em hotéis ou de internamento em hospitais.}{di.á.ria}{0}
\verb{diária}{}{}{}{}{}{Pagamento por um dia de trabalho.}{di.á.ria}{0}
\verb{diário}{}{}{}{}{adj.}{Que ocorre ou que se faz todos os dias; cotidiano, diurnal.}{di.á.rio}{0}
\verb{diário}{}{}{}{}{s.m.}{Registro dos acontecimentos ou dos sentimentos vividos, feito todos os dias.}{di.á.rio}{0}
\verb{diário}{}{}{}{}{}{Jornal publicado todos os dias.}{di.á.rio}{0}
\verb{diarista}{}{}{}{}{s.2g.}{Trabalhador que ganha por dia trabalhado.}{di.a.ris.ta}{0}
\verb{diarreia}{é}{Med.}{}{}{s.f.}{Evacuação frequente fora do normal, com fezes líquidas e abundantes.}{di.ar.rei.a}{0}
\verb{diarreico}{é}{}{}{}{adj.}{Relativo a diarreia.}{di.ar.rei.co}{0}
\verb{diarreico}{é}{}{}{}{}{Que sofre de diarreia.}{di.ar.rei.co}{0}
\verb{dias}{}{}{}{}{s.m.pl.}{Tempo de vida. (\textit{Compôs até o fim de seus dias.})}{di.as}{0}
\verb{diáspora}{}{Hist.}{}{}{s.f.}{A dispersão dos judeus pelo mundo através dos séculos.}{di.ás.po.ra}{0}
\verb{diáspora}{}{Por ext.}{}{}{}{Dispersão de quaisquer povos por motivos políticos, religiosos ou de perseguição.}{di.ás.po.ra}{0}
\verb{diástole}{}{Med.}{}{}{s.f.}{Movimento de dilatação do coração, que ocorre após a contração.}{di.ás.to.le}{0}
\verb{diatérmico}{}{}{}{}{adj.}{Diz"-se dos corpos que transmitem calor.}{di.a.tér.mi.co}{0}
\verb{diatribe}{}{}{}{}{s.f.}{Discurso escrito ou oral com a finalidade de atacar ou injuriar alguém, feito geralmente em tom violento.}{di.a.tri.be}{0}
\verb{diatribe}{}{}{}{}{}{Discussão exaltada.}{di.a.tri.be}{0}
\verb{dica}{}{Bras.}{}{}{s.f.}{Informação pontual e geralmente pouco conhecida de bastante utilidade na solução de um problema.}{di.ca}{0}
\verb{dição}{}{}{}{}{}{Var. de \textit{dicção}.}{di.ção}{0}
\verb{dicaz}{}{}{}{}{adj.2g.}{Que é severo na crítica; mordaz, sarcástico.}{di.caz}{0}
\verb{dicção}{}{}{"-ões}{}{s.f.}{Modo de pronunciar as palavras; articulação.}{dic.ção}{0}
\verb{dichote}{ó}{}{}{}{s.m.}{Dito irônico, depreciativo; zombaria, remoque.}{di.cho.te}{0}
\verb{dicionário}{}{}{}{}{s.m.}{Livro que apresenta as palavras e as expressões de uma língua ou os termos próprios de uma ciência ou arte, colocados em ordem alfabética ou por assunto e seguidos de sua definição.}{di.ci.o.ná.rio}{0}
\verb{dicionarista}{}{}{}{}{s.2g.}{Autor de dicionário; lexicógrafo.}{di.ci.o.na.ris.ta}{0}
\verb{dicionarizar}{}{}{}{}{v.t.}{Incluir vocábulo em dicionário.}{di.ci.o.na.ri.zar}{\verboinum{1}}
\verb{dicotilédone}{}{}{}{}{adj.2g.}{Dicotiledôneo.}{di.co.ti.lé.do.ne}{0}
\verb{dicotiledônea}{}{Bot.}{}{}{s.f.}{Espécime das dicotiledôneas, classe das angiospermas caracterizada por plantas cuja semente possui dois cotilédones; magnoliópsida. }{di.co.ti.le.dô.nea}{0}
\verb{dicotiledôneo}{}{Bot.}{}{}{adj.}{Diz"-se do embrião que apresenta dois cotilédones; dicotilédone.}{di.co.ti.le.dô.neo}{0}
\verb{dicotomia}{}{}{}{}{s.f.}{Divisão em dois ramos; bifurcação.}{di.co.to.mi.a}{0}
\verb{dicotômico}{}{}{}{}{adj.}{Dividido ou subdividido em dois; bifurcado.}{di.co.tô.mi.co}{0}
\verb{dictério}{}{}{}{}{s.m.}{Dito sarcástico; zombaria, dichote.}{dic.té.rio}{0}
\verb{didata}{}{}{}{}{s.2g.}{Indivíduo que ensina, instrui.}{di.da.ta}{0}
\verb{didática}{}{}{}{}{s.f.}{Metodologia ou técnica de ensinar.}{di.dá.ti.ca}{0}
\verb{didático}{}{}{}{}{adj.}{Relativo a ensino ou a aprendizagem. (\textit{Minha professora costuma adotar bons livros didáticos para as aulas.})}{di.dá.ti.co}{0}
\verb{didatismo}{}{}{}{}{s.m.}{Qualidade ou característica do que é didático; pedagogismo.}{di.da.tis.mo}{0}
\verb{diedro}{é}{Geom.}{}{}{s.m.}{Ângulo formado por duas faces planas.}{di.e.dro}{0}
\verb{dielétrico}{}{Fís.}{}{}{adj.}{Diz"-se da substância ou do objeto capaz de isolar a eletricidade.}{di.e.lé.tri.co}{0}
\verb{diérese}{}{Gram.}{}{}{s.f.}{Forma pausada de se pronunciar um ditongo, passando"-o a um hiato.}{di.é.re.se}{0}
\verb{diérese}{}{Med.}{}{}{}{Separação cirúrgica de tecidos orgânicos que estão unidos.}{di.é.re.se}{0}
\verb{diesel}{}{}{}{}{s.m.}{Motor de combustão, alimentado a óleo, que funciona por meio de pressão.}{\textit{diesel}}{0}
\verb{diesel}{}{Por ext.}{}{}{}{O óleo que alimenta esse tipo de motor.}{\textit{diesel}}{0}
\verb{diet}{}{}{}{}{adj.2g.}{Diz"-se do alimento de baixo valor calórico ou que emprega adoçante em substituição ao açúcar.}{\textit{diet}}{0}
\verb{dieta}{é}{}{}{}{s.f.}{Programa de alimentação em que se restringem produtos prejudiciais à saúde do indivíduo.}{di.e.ta}{0}
\verb{dietética}{}{}{}{}{s.f.}{Ramo da medicina que estuda as dietas.}{di.e.té.ti.ca}{0}
\verb{dietético}{}{}{}{}{adj.}{Relativo a dieta.}{di.e.té.ti.co}{0}
\verb{dietista}{}{}{}{}{s.2g.}{Profissional especializado em nutrição, em dietas; nutricionista.}{di.e.tis.ta}{0}
\verb{difamação}{}{}{"-ões}{}{s.f.}{Ato ou efeito de difamar; calúnia, descrédito.}{di.fa.ma.ção}{0}
\verb{difamador}{ô}{}{}{}{adj.}{Que difama; caluniador.}{di.fa.ma.dor}{0}
\verb{difamante}{}{}{}{}{adj.2g.}{Difamador.}{di.fa.man.te}{0}
\verb{difamar}{}{}{}{}{v.t.}{Ofender a reputação; falar mal; caluniar, desacreditar.}{di.fa.mar}{\verboinum{1}}
\verb{difamatório}{}{}{}{}{adj.}{Que contém difamação.}{di.fa.ma.tó.rio}{0}
\verb{diferença}{}{}{}{}{s.f.}{Característica do que é diferente, diverso; desigualdade, dessemelhança, disparidade.}{di.fe.ren.ça}{0}
\verb{diferença}{}{Mat.}{}{}{}{Resultado da subtração.}{di.fe.ren.ça}{0}
\verb{diferençar}{}{}{}{}{v.t.}{Diferenciar.}{di.fe.ren.çar}{\verboinum{3}}
\verb{diferenciação}{}{}{"-ões}{}{s.f.}{Ato ou efeito de diferenciar; discriminação, discernimento.}{di.fe.ren.ci.a.ção}{0}
\verb{diferencial}{}{}{"-ais}{}{adj.2g.}{Relativo a diferença.}{di.fe.ren.ci.al}{0}
\verb{diferencial}{}{}{"-ais}{}{}{Que indica ou estabelece diferença.}{di.fe.ren.ci.al}{0}
\verb{diferencial}{}{}{"-ais}{}{s.m.}{Mecanismo do automóvel que transmite o movimento do motor às rodas traseiras, permitindo o equilíbrio na passagem das curvas.}{di.fe.ren.ci.al}{0}
\verb{diferenciar}{}{}{}{}{v.t.}{Estabelecer diferença entre pessoas ou coisas; distinguir, discriminar, diferençar.}{di.fe.ren.ci.ar}{\verboinum{1}}
\verb{diferente}{}{}{}{}{adj.2g.}{Que apresenta diferença; desigual, diverso, dessemelhante.}{di.fe.ren.te}{0}
\verb{diferente}{}{}{}{}{}{Que sofreu mudança; modificado, alterado. (\textit{A professora apareceu hoje na sala com um corte de cabelo diferente. })}{di.fe.ren.te}{0}
\verb{diferimento}{}{}{}{}{s.m.}{Ato ou efeito de diferir; demora, adiamento.}{di.fe.ri.men.to}{0}
\verb{diferir}{}{}{}{}{v.t.}{Ser diferente; distinguir"-se.}{di.fe.rir}{0}
\verb{diferir}{}{}{}{}{}{Discordar, divergir. (\textit{Minha opinião difere da sua em relação à política.})}{di.fe.rir}{0}
\verb{diferir}{}{}{}{}{}{Adiar, retardar.}{di.fe.rir}{\verboinum{29}}
\verb{difícil}{}{}{"-eis}{}{adj.2g.}{Que exige esforço para ser feito; árduo, trabalhoso, custoso.}{di.fí.cil}{0}
\verb{difícil}{}{}{"-eis}{}{}{Que é pouco provável; incerto, duvidoso. (\textit{Os meteorologistas acham difícil que chova nos próximos dois meses.})}{di.fí.cil}{0}
\verb{difícil}{}{}{"-eis}{}{}{Que não é fácil de contentar; exigente, intratável. (\textit{Minha irmã é uma pessoa difícil de se conviver.})}{di.fí.cil}{0}
\verb{dificuldade}{}{}{}{}{s.f.}{Aquilo que torna algo difícil, que atrapalha; obstáculo, empecilho, estorvo. (\textit{Os viajantes tiveram que enfrentar uma série de dificuldades até chegarem ao seu destino.})}{di.fi.cul.da.de}{0}
\verb{dificultar}{}{}{}{}{v.t.}{Tornar difícil; impedir, atrapalhar.}{di.fi.cul.tar}{\verboinum{1}}
\verb{dificultoso}{ô}{}{"-osos ⟨ó⟩}{"-osa ⟨ó⟩}{adj.}{Que apresenta muitas dificuldades; árduo.}{di.fi.cul.to.so}{0}
\verb{difração}{}{Fís.}{"-ões}{}{s.f.}{Desvio dos raios luminosos ao incidirem sobre um corpo opaco.}{di.fra.ção}{0}
\verb{difteria}{}{Med.}{}{}{s.f.}{Doença infecto"-contagiosa, causada por uma certa bactéria, que ataca a garganta.}{dif.te.ri.a}{0}
\verb{diftérico}{}{}{}{}{adj.}{Relativo a difteria.}{dif.té.ri.co}{0}
\verb{difundir}{}{}{}{}{v.t.}{Tornar algo largamente conhecido; propagar, espalhar, divulgar.}{di.fun.dir}{0}
\verb{difundir}{}{}{}{}{}{Derramar, espargir, estender.}{di.fun.dir}{\verboinum{18}}
\verb{difusão}{}{}{"-ões}{}{s.f.}{Ato ou efeito de difundir; propagação, divulgação.}{di.fu.são}{0}
\verb{difuso}{}{}{}{}{adj.}{Em que há difusão; espalhado, disseminado.}{di.fu.so}{0}
\verb{digerir}{}{}{}{}{v.t.}{Fazer a digestão. (\textit{Para que o organismo consiga digerir os alimentos com mais facilidade, é importante que eles sejam bem mastigados antes.})}{di.ge.rir}{\verboinum{29}}
\verb{digestão}{}{Biol.}{"-ões}{}{s.f.}{Transformação dos alimentos em substâncias que o organismo possa absorver e assimilar.}{di.ges.tão}{0}
\verb{digestivo}{}{}{}{}{adj.}{Relativo a digestão.}{di.ges.ti.vo}{0}
\verb{digestivo}{}{Biol.}{}{}{}{Diz"-se do sistema que tem por função básica realizar a digestão.}{di.ges.ti.vo}{0}
\verb{digestivo}{}{}{}{}{}{Diz"-se da substância que facilita a digestão.}{di.ges.ti.vo}{0}
\verb{digesto}{é}{Jur.}{}{}{s.m.}{Compilação de regras, decisões ou prescrições jurídicas.}{di.ges.to}{0}
\verb{digesto}{é}{}{}{}{}{Publicação composta de resumos de livros.}{di.ges.to}{0}
\verb{digitação}{}{}{"-ões}{}{s.f.}{Ato ou efeito de digitar.}{di.gi.ta.ção}{0}
\verb{digitado}{}{}{}{}{adj.}{Que se digitou; que foi introduzido no computador por meio de um teclado.}{di.gi.ta.do}{0}
\verb{digitado}{}{}{}{}{}{Que tem forma de dedo; digitiforme.}{di.gi.ta.do}{0}
\verb{digitador}{ô}{}{}{}{s.m.}{Indivíduo que digita; operador de teclado. (\textit{Meu primo é digitador de uma empresa de informática.})}{di.gi.ta.dor}{0}
\verb{digital}{}{}{"-ais}{}{adj.2g.}{Relativo a dedo. (\textit{A polícia colheu as impressões digitais dos suspeitos.})}{di.gi.tal}{0}
\verb{digital}{}{}{"-ais}{}{}{Relativo a dígito. (\textit{Comprei um relógio digital para minha mãe.})}{di.gi.tal}{0}
\verb{digitalizar}{}{Informát.}{}{}{v.t.}{Converter imagem ou dados analógicos em dados legíveis pelo computador.}{di.gi.ta.li.zar}{\verboinum{1}}
\verb{digitar}{}{Informát.}{}{}{v.t.}{Registrar dados no computador por meio do teclado; teclar.}{di.gi.tar}{\verboinum{1}}
\verb{digitiforme}{ó}{}{}{}{adj.}{Que tem a forma de dedo; digitado.}{di.gi.ti.for.me}{0}
\verb{dígito}{}{}{}{}{s.m.}{Qualquer dos algarismos arábicos de 0 a 9.}{dí.gi.to}{0}
\verb{dígito}{}{Informát.}{}{}{}{Qualquer sinal que se digita num computador.}{dí.gi.to}{0}
\verb{digladiar}{}{}{}{}{v.i.}{Combater corpo a corpo com a espada; lutar, brigar.}{di.gla.di.ar}{\verboinum{1}}
\verb{dignar"-se}{}{}{}{}{v.pron.}{Ter a bondade de fazer algo; haver por bem; condescender.}{dig.nar"-se}{\verboinum{1}}
\verb{dignidade}{}{}{}{}{s.f.}{Qualidade de quem é digno; nobreza, respeitabilidade.}{dig.ni.da.de}{0}
\verb{dignidade}{}{}{}{}{}{Cargo elevado; honraria, grandeza.}{dig.ni.da.de}{0}
\verb{dignificar}{}{}{}{}{v.t.}{Tornar digno; enobrecer, nobilitar.}{dig.ni.fi.car}{\verboinum{2}}
\verb{dignitário}{}{}{}{}{s.m.}{Indivíduo que exerce cargo elevado.}{dig.ni.tá.rio}{0}
\verb{digno}{}{}{}{}{adj.}{Que revela caráter elevado; honrado, honesto.}{dig.no}{0}
\verb{digno}{}{}{}{}{}{Merecedor, credor. (\textit{Aquele homem é digno de confiança.})}{dig.no}{0}
\verb{digno}{}{}{}{}{}{Adequado, apropriado, conveniente.}{dig.no}{0}
\verb{dígrafo}{}{}{}{}{s.m.}{Conjunto de duas letras que representam um só fonema. (\textit{A palavra "choque" tem um dígrafo no começo de cada sílaba.})}{dí.gra.fo}{0}
\verb{digrama}{}{}{}{}{s.m.}{Dígrafo.}{di.gra.ma}{0}
\verb{digressão}{}{}{"-ões}{}{s.f.}{Desvio de assunto; divagação.}{di.gres.são}{0}
\verb{digressão}{}{}{"-ões}{}{}{Subterfúgio, pretexto, evasiva.}{di.gres.são}{0}
\verb{digressivo}{}{}{}{}{adj.}{Em que há digressão; dispersivo, evasivo.}{di.gres.si.vo}{0}
\verb{dilação}{}{}{"-ões}{}{s.f.}{Demora, adiamento, delonga.}{di.la.ção}{0}
\verb{dilacerante}{}{}{}{}{adj.2g.}{Que dilacera; torturante, aflitivo.}{di.la.ce.ran.te}{0}
\verb{dilacerar}{}{}{}{}{v.t.}{Despedaçar com violência; lacerar.}{di.la.ce.rar}{0}
\verb{dilacerar}{}{}{}{}{}{Afligir, torturar.}{di.la.ce.rar}{\verboinum{1}}
\verb{dilapidação}{}{}{"-ões}{}{s.f.}{Ato ou efeito de dilapidar; gasto excessivo, esbanjamento.}{di.la.pi.da.ção}{0}
\verb{dilapidar}{}{}{}{}{v.t.}{Gastar excessivamente; esbanjar, desperdiçar, arruinar. (\textit{Em poucos anos, ele dilapidou a fortuna deixada por seu pai.})}{di.la.pi.dar}{\verboinum{1}}
\verb{dilatação}{}{}{"-ões}{}{s.f.}{Ato ou efeito de dilatar; aumento, alargamento, expansão.}{di.la.ta.ção}{0}
\verb{dilatado}{}{}{}{}{adj.}{Que sofreu dilatação; aumentado, ampliado, extenso.}{di.la.ta.do}{0}
\verb{dilatar}{}{}{}{}{v.t.}{Aumentar o volume ou as dimensões de algo; ampliar, alargar, estender.}{di.la.tar}{\verboinum{1}}
\verb{dileção}{}{}{"-ões}{}{s.f.}{Afeição especial; predileção, estima.}{di.le.ção}{0}
\verb{dilema}{}{}{}{}{s.m.}{Situação difícil com duas saídas igualmente insatisfatórias.}{di.le.ma}{0}
\verb{diletante}{}{}{}{}{adj.2g.}{Que faz alguma coisa por gosto e não por obrigação; amador.}{di.le.tan.te}{0}
\verb{diletantismo}{}{}{}{}{s.m.}{Qualidade ou caráter de diletante; amadorismo.}{di.le.tan.tis.mo}{0}
\verb{dileto}{é}{}{}{}{adj.}{Muito querido; amado, predileto.}{di.le.to}{0}
\verb{diligência}{}{}{}{}{s.f.}{Presteza em fazer algo; empenho, zelo.}{di.li.gên.cia}{0}
\verb{diligência}{}{}{}{}{}{Antiga carruagem usada para transportar passageiros.}{di.li.gên.cia}{0}
\verb{diligenciar}{}{}{}{}{v.t.}{Empregar meios para realizar algo; empenhar"-se, esforçar"-se.}{di.li.gen.ci.ar}{\verboinum{1}}
\verb{diligente}{}{}{}{}{adj.2g.}{Que faz as coisas com dedicação; zeloso, aplicado, cuidadoso.}{di.li.gen.te}{0}
\verb{diluente}{}{}{}{}{adj.2g.}{Diz"-se de substância própria para diluir.}{di.lu.en.te}{0}
\verb{diluição}{}{}{"-ões}{}{s.f.}{Ato ou efeito de diluir, dissolver.}{di.lu.i.ção}{0}
\verb{diluído}{}{}{}{}{adj.}{Que se tornou pouco concentrado, pouco perceptível. (\textit{A tinta deve ser diluída em água para poder penetrar na tela.})}{di.lu.í.do}{0}
\verb{diluir}{}{}{}{}{v.t.}{Tornar mais fluido, menos concentrado; dissolver, desmanchar.}{di.lu.ir}{\verboinum{26}}
\verb{diluviano}{}{}{}{}{adj.}{Relativo ao dilúvio universal.}{di.lu.vi.a.no}{0}
\verb{diluviano}{}{Fig.}{}{}{}{Muito abundante; copioso, torrencial.}{di.lu.vi.a.no}{0}
\verb{dilúvio}{}{Por ext.}{}{}{}{Grande chuva; inundação.}{di.lú.vio}{0}
\verb{dilúvio}{}{}{}{}{s.m.}{Segundo a Bíblia, inundação que encobriu toda a superfície terrestre.}{di.lú.vio}{0}
\verb{dimanar}{}{}{}{}{v.i.}{Correr mansamente; fluir, manar.}{di.ma.nar}{\verboinum{1}}
\verb{dimensão}{}{}{"-ões}{}{s.f.}{Extensão em qualquer sentido; medida, tamanho.}{di.men.são}{0}
\verb{dimensionar}{}{}{}{}{v.t.}{Calcular as dimensões; planejar.}{di.men.si.o.nar}{\verboinum{1}}
\verb{diminuendo}{}{Mat.}{}{}{s.m.}{Número do qual se subtrai outro; minuendo. (\textit{Quando se subtrai cinco de nove, nove é o diminuendo.})}{di.mi.nu.en.do}{0}
\verb{diminuição}{}{Mat.}{"-ões}{}{}{Operação em que se retira uma quantidade de outra; subtração.}{di.mi.nu.i.ção}{0}
\verb{diminuição}{}{}{"-ões}{}{s.f.}{Ato ou efeito de diminuir; redução.}{di.mi.nu.i.ção}{0}
\verb{diminuir}{}{}{}{}{v.t.}{Tornar menor; reduzir a quantidade.}{di.mi.nu.ir}{0}
\verb{diminuir}{}{}{}{}{}{Perder a força ou a intensidade; baixar.}{di.mi.nu.ir}{0}
\verb{diminuir}{}{Mat.}{}{}{}{Subtrair, deduzir.}{di.mi.nu.ir}{\verboinum{26}}
\verb{diminutivo}{}{Gram.}{}{}{}{Diz"-se do grau que expressa diminuição, redução.}{di.mi.nu.ti.vo}{0}
\verb{diminutivo}{}{}{}{}{adj.}{Que diminui ou encerra diminuição.}{di.mi.nu.ti.vo}{0}
\verb{diminuto}{}{}{}{}{adj.}{De tamanho bem reduzido; muito pequeno.}{di.mi.nu.to}{0}
\verb{dinamarquês}{}{}{}{}{adj.}{Relativo a Dinamarca.}{di.na.mar.quês}{0}
\verb{dinamarquês}{}{}{}{}{s.m.}{Indivíduo natural ou habitante desse país.}{di.na.mar.quês}{0}
\verb{dinâmica}{}{Fís.}{}{}{s.f.}{Parte da mecânica que estuda as relações entre os movimentos dos corpos e a força que os produz.}{di.nâ.mi.ca}{0}
\verb{dinâmico}{}{}{}{}{adj.}{Que apresenta grande atividade ou movimento; ativo, trabalhador.}{di.nâ.mi.co}{0}
\verb{dinamismo}{}{}{}{}{s.m.}{Qualidade de dinâmico; atividade intensa; energia.}{di.na.mis.mo}{0}
\verb{dinamitar}{}{}{}{}{v.t.}{Fazer explodir por meio de dinamite.}{di.na.mi.tar}{\verboinum{1}}
\verb{dinamite}{}{}{}{}{s.f.}{Explosivo de grande poder à base de nitroglicerina.}{di.na.mi.te}{0}
\verb{dinamização}{}{}{"-ões}{}{s.f.}{Ato ou efeito de dinamizar; ativação.}{di.na.mi.za.ção}{0}
\verb{dinamizar}{}{}{}{}{v.t.}{Aumentar a atividade; energizar.}{di.na.mi.zar}{\verboinum{1}}
\verb{dínamo}{}{}{}{}{s.m.}{Máquina que converte energia mecânica em energia elétrica; gerador.}{dí.na.mo}{0}
\verb{dinamometria}{}{}{}{}{s.f.}{Medição realizada com dinamômetro.}{di.na.mo.me.tri.a}{0}
\verb{dinamômetro}{}{}{}{}{s.m.}{Instrumento utilizado para medir as forças.}{di.na.mô.me.tro}{0}
\verb{dinar}{}{}{}{}{s.m.}{Unidade monetária e moeda utilizada na Argélia, no Iraque, na Jordânia, na  Tunísia e em outros países.}{di.nar}{0}
\verb{dinastia}{}{}{}{}{s.f.}{Sucessão de soberanos que pertencem a uma mesma família.}{di.nas.ti.a}{0}
\verb{dindinho}{}{Pop.}{}{}{s.m.}{Padrinho, avô.}{din.di.nho}{0}
\verb{dinheirada}{}{}{}{}{s.f.}{Dinheirama.}{di.nhei.ra.da}{0}
\verb{dinheirama}{}{}{}{}{s.f.}{Grande quantidade de dinheiro; dinheirada.}{di.nhei.ra.ma}{0}
\verb{dinheirão}{}{Pop.}{}{}{s.m.}{Quantia alta de dinheiro; fortuna. (\textit{Esse carro custa um dinheirão!})}{di.nhei.rão}{0}
\verb{dinheiro}{ê}{}{}{}{}{Riqueza, fortuna, bens.}{di.nhei.ro}{0}
\verb{dinheiro}{ê}{}{}{}{s.m.}{Moeda corrente utilizada em transações e em compra e venda de mercadorias.}{di.nhei.ro}{0}
\verb{dinossauro}{}{Zool.}{}{}{s.m.}{Réptil fóssil, de tamanho variado, que viveu na Terra milhões de anos antes do surgimento do ser humano. }{di.nos.sau.ro}{0}
\verb{dintel}{é}{}{"-éis}{}{s.m.}{Peça de madeira que se coloca sobre portas e janelas.}{din.tel}{0}
\verb{diocesano}{}{}{}{}{adj.}{Relativo a diocese.}{di.o.ce.sa.no}{0}
\verb{diocesano}{}{}{}{}{}{Indivíduo pertencente a uma diocese.}{di.o.ce.sa.no}{0}
\verb{diocese}{é}{}{}{}{s.f.}{Região dirigida por um bispo; bispado.}{di.o.ce.se}{0}
\verb{diodo}{ô}{}{}{}{s.m.}{Válvula eletrônica que possui apenas dois eletrodos, um negativo e um positivo.}{di.o.do}{0}
\verb{dionisíaco}{}{Mit.}{}{}{adj.}{Relativo ao deus grego Dioniso ou ao deus romano Baco, ligado às festas campestres e ao vinho.}{di.o.ni.sí.a.co}{0}
\verb{dionisíaco}{}{}{}{}{}{De natureza desinibida, arrebatada, instintiva.}{di.o.ni.sí.a.co}{0}
\verb{dióxido}{cs}{Quím.}{}{}{s.m.}{Óxido que possui somente dois átomos de oxigênio; bióxido.}{di.ó.xi.do}{0}
\verb{diple}{}{Paleo.}{}{}{s.f.}{Diacrítico em forma de \textsc{v} horizontal com a abertura para a esquerda (>) ou direita (<), usado para marcar trechos escritos; diplo.}{di.ple}{0}
\verb{diploma}{ô}{}{}{}{s.m.}{Documento que comprova as habilitações de alguém ou que confere o grau de escolaridade a quem terminou um curso.}{di.plo.ma}{0}
\verb{diplomação}{}{}{"-ões}{}{s.f.}{Ato ou efeito de se conferir diploma.}{di.plo.ma.ção}{0}
\verb{diplomacia}{}{}{}{}{s.f.}{Ciência que trata das relações internacionais.}{di.plo.ma.ci.a}{0}
\verb{diplomacia}{}{}{}{}{}{Habilidade de resolver questões; tato, finura.}{di.plo.ma.ci.a}{0}
\verb{diplomado}{}{}{}{}{adj.}{Diz"-se daquele que recebeu diploma; formado.}{di.plo.ma.do}{0}
\verb{diplomando}{}{}{}{}{adj.}{Diz"-se daquele que terminou o curso e está prestes a receber o diploma.}{di.plo.man.do}{0}
\verb{diplomar}{}{}{}{}{v.t.}{Conferir diploma a quem tem direito a ele.}{di.plo.mar}{0}
\verb{diplomar}{}{}{}{}{v.pron.}{Receber diploma após graduar"-se.}{di.plo.mar}{\verboinum{1}}
\verb{diplomata}{}{}{}{}{s.2g.}{Indivíduo que faz parte do corpo diplomático de uma nação.}{di.plo.ma.ta}{0}
\verb{diplomata}{}{}{}{}{}{Indivíduo hábil no trato de questões e situações difíceis.}{di.plo.ma.ta}{0}
\verb{diplomático}{}{}{}{}{adj.}{Relativo a diplomacia.}{di.plo.má.ti.co}{0}
\verb{diplomático}{}{}{}{}{}{Cortês, polido, discreto.}{di.plo.má.ti.co}{0}
\verb{dipsomania}{}{Med.}{}{}{s.f.}{Tendência doentia para o consumo exagerado de bebidas alcoólicas; alcoolismo.}{dip.so.ma.ni.a}{0}
\verb{dipsomaníaco}{}{}{}{}{adj.}{Relativo a dipsomania.}{dip.so.ma.ní.a.co}{0}
\verb{dipsomaníaco}{}{}{}{}{s.m.}{Indivíduo que sofre de dipsomania; alcoólatra.}{dip.so.ma.ní.a.co}{0}
\verb{díptero}{}{Zool.}{}{}{adj.}{Diz"-se dos insetos que possuem apenas duas asas como as moscas e os mosquitos.}{díp.te.ro}{0}
\verb{dique}{}{}{}{}{s.m.}{Construção feita para represar águas correntes; açude, represa, barragem.}{di.que}{0}
\verb{direção}{}{}{"-ões}{}{s.f.}{Ato ou efeito de dirigir.}{di.re.ção}{0}
\verb{direção}{}{}{"-ões}{}{}{Conjunto das autoridades que comandam; diretoria. (\textit{A direção da escola resolveu não falar sobre o assunto.})}{di.re.ção}{0}
\verb{direção}{}{}{"-ões}{}{}{Cargo de diretor.}{di.re.ção}{0}
\verb{direção}{}{}{"-ões}{}{}{Lado para o qual se dirige; rumo, rota, orientação.}{di.re.ção}{0}
\verb{direção}{}{}{"-ões}{}{}{Conjunto de mecanismos que permitem conduzir um veículo para onde se deseja ir.}{di.re.ção}{0}
\verb{direcionar}{}{}{}{}{v.t.}{Dar a direção desejada; conduzir, orientar.}{di.re.ci.o.nar}{\verboinum{1}}
\verb{direita}{ê}{}{}{}{s.f.}{A mão direita; destra.}{di.rei.ta}{0}
\verb{direita}{ê}{}{}{}{}{Lado direito.}{di.rei.ta}{0}
\verb{direita}{ê}{}{}{}{}{Conjunto de pessoas que defendem uma política conservadora baseada em princípios de hierarquia e tradição.}{di.rei.ta}{0}
\verb{direitista}{}{}{}{}{s.2g.}{Indivíduo que apoia a direita na política; conservador, reacionário.}{di.rei.tis.ta}{0}
\verb{direito}{ê}{}{}{}{adj.}{Que está do lado oposto ao do coração.}{di.rei.to}{0}
\verb{direito}{ê}{}{}{}{}{Que apresenta bom comportamento; justo, honrado.}{di.rei.to}{0}
\verb{direito}{ê}{}{}{}{s.m.}{Aquilo que é facultado a um indivíduo por força de leis ou dos costumes. (\textit{Eu tenho o direito de não opinar sobre esse caso.})}{di.rei.to}{0}
\verb{direito}{ê}{Jur.}{}{}{}{Conjunto de leis e normas que devem ser respeitadas em uma sociedade.}{di.rei.to}{0}
\verb{direito}{ê}{}{}{}{}{Ciência que estuda esse conjunto de leis.}{di.rei.to}{0}
\verb{direito}{ê}{}{}{}{}{Lado principal de alguma coisa, oposto ao avesso.}{di.rei.to}{0}
\verb{direito}{ê}{}{}{}{adv.}{Em linha reta; diretamente. (\textit{As pessoas precisam aprender a sentar direito, com as costas retas.})}{di.rei.to}{0}
\verb{direitura}{}{}{}{}{s.f.}{Qualidade do que é reto, direito; retidão.}{di.rei.tu.ra}{0}
\verb{diretiva}{}{}{}{}{s.f.}{Conjunto de normas; linha diretriz; orientação.}{di.re.ti.va}{0}
\verb{diretivo}{}{}{}{}{adj.}{Que dirige, orienta; orientação.}{di.re.ti.vo}{0}
\verb{direto}{é}{}{}{}{adj.}{Em linha reta; reto, direito.}{di.re.to}{0}
\verb{direto}{é}{}{}{}{adv.}{Sem intermediário; imediato. (\textit{Nesse mercado, pode"-se comprar legumes direto do produtor.})}{di.re.to}{0}
\verb{diretor}{ô}{}{}{}{adj.}{Que dirige; orientador, dirigente.}{di.re.tor}{0}
\verb{diretor}{ô}{}{}{}{s.m.}{Indivíduo que dirige uma organização.}{di.re.tor}{0}
\verb{diretoria}{}{}{}{}{s.f.}{Conjunto dos membros que dirigem uma organização; direção. (\textit{A diretoria do banco se reuniu para escolher o novo presidente.})}{di.re.to.ri.a}{0}
\verb{diretório}{}{}{}{}{s.m.}{Comissão que dirige uma agremiação política ou um partido nacional ou regional. (\textit{O diretório nacional do partido decidiu pela expulsão de dois membros.})}{di.re.tó.rio}{0}
\verb{diretório}{}{Informát.}{}{}{}{Listagem do índice dos arquivos armazenados em um disco do computador.}{di.re.tó.rio}{0}
\verb{diretriz}{}{}{}{}{s.f.}{Linha que determina o traçado de uma estrada.}{di.re.triz}{0}
\verb{diretriz}{}{}{}{}{}{Norma de conduta; orientação, diretiva.}{di.re.triz}{0}
\verb{dirigente}{}{}{}{}{s.2g.}{Aquele que dirige uma atividade ou uma organização. (\textit{Os dirigentes do clube resolveram contratar mais reforços para o time.})}{di.ri.gen.te}{0}
\verb{dirigir}{}{}{}{}{v.t.}{Fazer algo ou alguém chegar ao lugar de destino; encaminhar.}{di.ri.gir}{0}
\verb{dirigir}{}{}{}{}{}{Guiar um veículo.}{di.ri.gir}{0}
\verb{dirigir}{}{}{}{}{}{Administrar, comandar, governar.}{di.ri.gir}{0}
\verb{dirigir}{}{}{}{}{}{Falar, dizer, conversar. (\textit{Ele não me dirigiu a palavra em nenhum momento.})}{di.ri.gir}{0}
\verb{dirigir}{}{}{}{}{}{Voltar, volver.}{di.ri.gir}{\verboinum{22}}
\verb{dirigível}{}{}{"-eis}{}{adj.2g.}{Que pode ser dirigido.}{di.ri.gí.vel}{0}
\verb{dirigível}{}{}{"-eis}{}{s.m.}{Balão dotado de propulsores e de sistema de direção.}{di.ri.gí.vel}{0}
\verb{dirimente}{}{}{}{}{adj.2g.}{Que dirime; conclusivo, decisivo.}{di.ri.men.te}{0}
\verb{dirimente}{}{Jur.}{}{}{}{Que isenta de pena; que exclui a culpabilidade.}{di.ri.men.te}{0}
\verb{dirimir}{}{}{}{}{v.t.}{Conseguir a solução; resolver, decidir.}{di.ri.mir}{0}
\verb{dirimir}{}{}{}{}{}{Tornar nulo; extinguir, desfazer.}{di.ri.mir}{\verboinum{18}}
\verb{discar}{}{}{}{}{v.t.}{Girar o disco do telefone para fazer ligação.}{dis.car}{0}
\verb{discar}{}{}{}{}{}{Usar o telefone; telefonar.}{dis.car}{\verboinum{2}}
\verb{discente}{}{}{}{}{adj.2g.}{Que aprende.}{dis.cen.te}{0}
\verb{discente}{}{}{}{}{}{Relativo a aluno.}{dis.cen.te}{0}
\verb{discernimento}{}{}{}{}{s.m.}{Ação ou faculdade de discernir; entendimento, critério.}{dis.cer.ni.men.to}{0}
\verb{discernir}{}{}{}{}{v.t.}{Perceber a diferença; conhecer, discriminar, distinguir.}{dis.cer.nir}{\verboinum{29}\verboirregular{\emph{def.} discernimos, discernis}}
\verb{disciforme}{ó}{}{}{}{adj.2g.}{Que tem a forma plana e circular de um disco; discoide.}{dis.ci.for.me}{0}
\verb{disciplina}{}{}{}{}{s.f.}{Conjunto de regras e normas para manter o bom funcionamento em uma organização.}{dis.ci.pli.na}{0}
\verb{disciplina}{}{}{}{}{}{Submissão ou respeito a um regulamento.}{dis.ci.pli.na}{0}
\verb{disciplina}{}{}{}{}{}{Cada uma das matérias que se ensinam em uma escola.}{dis.ci.pli.na}{0}
\verb{disciplinado}{}{}{}{}{adj.}{Que obedece à disciplina; bem"-comportado; obediente.}{dis.ci.pli.na.do}{0}
\verb{disciplinador}{ô}{}{}{}{adj.}{Que faz observar a disciplina.}{dis.ci.pli.na.dor}{0}
\verb{disciplinar}{}{}{}{}{adj.2g.}{Relativo a disciplina.}{dis.ci.pli.nar}{0}
\verb{disciplinar}{}{}{}{}{v.t.}{Impor a disciplina; fazer obedecer; sujeitar.}{dis.ci.pli.nar}{\verboinum{1}}
\verb{disciplinas}{}{}{}{}{s.f.pl.}{Cordas ou correntes com que religiosos e penitentes se flagelam.}{dis.ci.pli.nas}{0}
\verb{discípulo}{}{}{}{}{s.m.}{Indivíduo que recebe ensinamento de outro; aluno.}{dis.cí.pu.lo}{0}
\verb{discípulo}{}{}{}{}{}{Indivíduo que segue conselhos ou exemplos de um mestre ou líder; adepto, seguidor.}{dis.cí.pu.lo}{0}
\verb{disco}{}{}{}{}{s.m.}{Objeto chato e circular no qual se grava algo para ser reproduzido depois.}{dis.co}{0}
\verb{disco}{}{Esport.}{}{}{}{Objeto chato e redondo utilizado em esportes de lançamento.}{dis.co}{0}
\verb{disco}{}{Astron.}{}{}{}{A superfície aparente dos astros.}{dis.co}{0}
\verb{discóbolo}{}{Hist.}{}{}{s.m.}{Na antiga Grécia, atleta que arremessava o disco nos jogos esportivos.}{dis.có.bo.lo}{0}
\verb{discoide}{}{}{}{}{adj.2g.}{Que tem a forma de um disco; disciforme.}{dis.coi.de}{0}
\verb{discordância}{}{}{}{}{s.f.}{Falta de concordância; desarmonia, divergência.}{dis.cor.dân.cia}{0}
\verb{discordante}{}{}{}{}{adj.2g.}{Que está em desacordo; divergente, dissonante.}{dis.cor.dan.te}{0}
\verb{discordar}{}{}{}{}{v.t.}{Não concordar; divergir, destoar.}{dis.cor.dar}{\verboinum{1}}
\verb{discorde}{ó}{}{}{}{adj.2g.}{Que discorda; divergente, discordante, dissonante.}{dis.cor.de}{0}
\verb{discórdia}{}{}{}{}{s.f.}{Falta de acordo; desarmonia, divergência, desavença.}{dis.cór.dia}{0}
\verb{discorrer}{ê}{}{}{}{v.t.}{Falar sobre um tema; expor, discursar.}{dis.cor.rer}{\verboinum{12}}
\verb{discoteca}{é}{}{}{}{s.f.}{Coleção de discos de música.}{dis.co.te.ca}{0}
\verb{discoteca}{é}{}{}{}{}{Lugar em que se dança ao som de música gravada; danceteria.}{dis.co.te.ca}{0}
\verb{discotecário}{}{}{}{}{s.m.}{Indivíduo responsável por uma discoteca, que escolhe os discos que serão tocados.}{dis.co.te.cá.rio}{0}
\verb{discrepância}{}{}{}{}{s.f.}{Ato ou efeito de discrepar; divergência, desacordo, disparidade.}{dis.cre.pân.cia}{0}
\verb{discrepante}{}{}{}{}{adj.2g.}{Que discrepa; divergente, discordante.}{dis.cre.pan.te}{0}
\verb{discrepar}{}{}{}{}{v.t.}{Ser diferente; diferir, divergir.}{dis.cre.par}{\verboinum{1}}
\verb{discretear}{}{}{}{}{v.t.}{Conversar com discrição.}{dis.cre.te.ar}{\verboinum{4}}
\verb{discreto}{é}{}{}{}{adj.}{Que não chama a atenção.}{dis.cre.to}{0}
\verb{discreto}{é}{}{}{}{}{Que não se intromete em ou não revela segredos da vida alheia.}{dis.cre.to}{0}
\verb{discreto}{é}{}{}{}{}{De pouca intensidade; brando.}{dis.cre.to}{0}
\verb{discreto}{é}{Mat.}{}{}{}{Diz"-se de uma grandeza constituída por unidades distintas ou valores inteiros. }{dis.cre.to}{0}
\verb{discrição}{}{}{"-ões}{}{s.f.}{Qualidade daquilo que é discreto.}{dis.cri.ção}{0}
\verb{discricionário}{}{}{}{}{adj.}{Livre de condições ou restrições; ilimitado.}{dis.cri.ci.o.ná.rio}{0}
\verb{discriminação}{}{}{"-ões}{}{s.f.}{Ato ou efeito de discriminar.}{dis.cri.mi.na.ção}{0}
\verb{discriminador}{ô}{}{}{}{adj.}{Que discrimina.}{dis.cri.mi.na.dor}{0}
\verb{discriminar}{}{}{}{}{v.t.}{Perceber diferenças; distinguir.}{dis.cri.mi.nar}{0}
\verb{discriminar}{}{}{}{}{}{Separar ou classificar por algum critério; especificar, listar.}{dis.cri.mi.nar}{0}
\verb{discriminar}{}{Por ext.}{}{}{}{Tratar de maneira desigual, geralmente mal e de modo injusto, um indivíduo ou um grupo, em razão de características étnicas, físicas, sociais ou ideológicas.}{dis.cri.mi.nar}{\verboinum{1}}
\verb{discursar}{}{}{}{}{v.i.}{Fazer um discurso; falar em público.}{dis.cur.sar}{\verboinum{1}}
\verb{discurseira}{ê}{}{}{}{s.f.}{Discurso longo e cansativo; discursório.}{dis.cur.sei.ra}{0}
\verb{discursivo}{}{}{}{}{adj.}{Que procede através do raciocínio; que não é intuitivo; dedutivo. (\textit{As questões discursivas da prova eram extremamente longas.})}{dis.cur.si.vo}{0}
\verb{discurso}{}{}{}{}{s.m.}{Fala dirigida a um público.}{dis.cur.so}{0}
\verb{discurso}{}{}{}{}{}{Ato de comunicação verbal; expressão.}{dis.cur.so}{0}
\verb{discursório}{}{}{}{}{s.m.}{Discurseira.}{dis.cur.só.rio}{0}
\verb{discussão}{}{}{"-ões}{}{s.f.}{Ato ou efeito de discutir.}{dis.cus.são}{0}
\verb{discussão}{}{}{"-ões}{}{}{Debate, polêmica, disputa.}{dis.cus.são}{0}
\verb{discutir}{}{}{}{}{v.t.}{Levantar questões e opiniões a respeito de algo; debater.}{dis.cu.tir}{0}
\verb{discutir}{}{}{}{}{}{Trocar insultos; desentender"-se.}{dis.cu.tir}{\verboinum{18}}
\verb{discutível}{}{}{"-eis}{}{adj.2g.}{Que pode ser discutido; que deixa dúvidas; contestável, incerto.}{dis.cu.tí.vel}{0}
\verb{disenteria}{}{Med.}{}{}{s.f.}{Inflamação do intestino que se manifesta por dores e diarreia.}{di.sen.te.ri.a}{0}
\verb{diserto}{é}{}{}{}{adj.}{Que se exprime bem, com facilidade; eloquente.}{di.ser.to}{0}
\verb{disfarçar}{}{}{}{}{v.t.}{Alterar a voz ou o modo de agir para não ser reconhecido; dissimular; mascarar.}{dis.far.çar}{0}
\verb{disfarçar}{}{}{}{}{}{Encobrir, tapar, ocultar.}{dis.far.çar}{0}
\verb{disfarçar}{}{}{}{}{v.pron.}{Vestir"-se de modo diferente para não ser reconhecido.}{dis.far.çar}{\verboinum{3}}
\verb{disfarce}{}{}{}{}{s.m.}{Aquilo que serve para disfarçar; dissimulação.}{dis.far.ce}{0}
\verb{disforme}{ó}{}{}{}{adj.2g.}{Que apresenta formas irregulares; desproporcionado.}{dis.for.me}{0}
\verb{disfunção}{}{Med.}{"-ões}{}{s.f.}{Perturbação de uma função orgânica; distúrbio.}{dis.fun.ção}{0}
\verb{disjunção}{}{}{"-ões}{}{s.f.}{Ato ou efeito de disjungir; separação, desunião, divisão.}{dis.jun.ção}{0}
\verb{disjungir}{}{}{}{}{v.t.}{Separar, desunir, desprender.}{dis.jun.gir}{\verboinum{34}}
\verb{disjuntivo}{}{}{}{}{adj.}{Que separa, desune.}{dis.jun.ti.vo}{0}
\verb{disjunto}{}{}{}{}{adj.}{Que não está junto; separado, desunido.}{dis.jun.to}{0}
\verb{disjuntor}{ô}{Fís.}{}{}{s.m.}{Interruptor que se desliga automaticamente, interrompendo um circuito elétrico quando ocorre sobrecarga de corrente.}{dis.jun.tor}{0}
\verb{dislate}{}{}{}{}{s.m.}{Dito ou afirmação tola; disparate, despropósito.}{dis.la.te}{0}
\verb{dismenorreia}{é}{Med.}{}{}{s.f.}{Menstruação difícil, dolorosa.}{dis.me.nor.rei.a}{0}
\verb{díspar}{}{}{}{}{adj.2g.}{Que não é par; desigual, diferente. (\textit{Eles são irmãos, mas têm atitudes totalmente díspares.})}{dís.par}{0}
\verb{disparada}{}{}{}{}{s.f.}{Corrida desenfreada, em alta velocidade.}{dis.pa.ra.da}{0}
\verb{disparado}{}{}{}{}{adj.}{Que disparou; acelerado, desembestado.}{dis.pa.ra.do}{0}
\verb{disparar}{}{}{}{}{v.t.}{Acionar o gatilho de arma de fogo;  atirar, desfechar.}{dis.pa.rar}{0}
\verb{disparar}{}{}{}{}{}{Soltar com força, arremessar, arrojar.}{dis.pa.rar}{0}
\verb{disparar}{}{}{}{}{v.i.}{Partir apressadamente; correr em alta velocidade.}{dis.pa.rar}{0}
\verb{disparar}{}{}{}{}{}{Pôr um circuito em funcionamento. (\textit{O alarme do carro disparou.})}{dis.pa.rar}{\verboinum{1}}
\verb{disparatado}{}{}{}{}{adj.}{Em que há disparate; sem sentido; absurdo. }{dis.pa.ra.ta.do}{0}
\verb{disparatar}{}{}{}{}{v.i.}{Dizer ou cometer disparates.}{dis.pa.ra.tar}{\verboinum{1}}
\verb{disparate}{}{}{}{}{s.m.}{Palavra ou ação sem sentido; despropósito, desatino, dislate.}{dis.pa.ra.te}{0}
\verb{disparidade}{}{}{}{}{s.f.}{Qualidade do que é díspar; desigualdade, dessemelhança, diferença.}{dis.pa.ri.da.de}{0}
\verb{disparo}{}{}{}{}{s.m.}{Ato ou efeito de disparar.}{dis.pa.ro}{0}
\verb{disparo}{}{}{}{}{}{Detonação, tiro. (\textit{Minha irmã disse que ouviu alguns disparos de revólver ontem.})}{dis.pa.ro}{0}
\verb{dispêndio}{}{}{}{}{s.m.}{Aquilo que se gasta; consumo, despesa.}{dis.pên.dio}{0}
\verb{dispendioso}{ô}{}{"-osos ⟨ó⟩}{"-osa ⟨ó⟩}{adj.}{Que exige grande despesa; custoso, caro.}{dis.pen.di.o.so}{0}
\verb{dispensa}{}{}{}{}{s.f.}{Permissão para deixar de executar algum serviço; licença, autorização.}{dis.pen.sa}{0}
\verb{dispensar}{}{}{}{}{v.t.}{Permitir que alguém deixe de fazer um serviço a que estava obrigado; liberar, desobrigar.}{dis.pen.sar}{0}
\verb{dispensar}{}{}{}{}{}{Mandar embora; despedir, demitir.}{dis.pen.sar}{\verboinum{1}}
\verb{dispensário}{}{}{}{}{s.m.}{Instituição beneficente onde se atende a pacientes pobres, dando"-lhes consultas médicas, remédios, alimentos etc.}{dis.pen.sá.rio}{0}
\verb{dispensável}{}{}{"-eis}{}{adj.2g.}{Que se pode dispensar; desnecessário, prescindível.}{dis.pen.sá.vel}{0}
\verb{dispepsia}{}{Med.}{}{}{s.f.}{Dificuldade em digerir os alimentos; má digestão.}{dis.pep.si.a}{0}
\verb{dispéptico}{}{}{}{}{adj.}{Relativo a dispepsia.}{dis.pép.ti.co}{0}
\verb{dispéptico}{}{}{}{}{}{Que sofre de dispepsia.}{dis.pép.ti.co}{0}
\verb{dispersão}{}{}{"-ões}{}{s.f.}{Ato ou efeito de dispersar.}{dis.per.são}{0}
\verb{dispersar}{}{}{}{}{v.t.}{Fazer um conjunto de pessoas ou coisas se desmanchar; pôr em debandada; afugentar.}{dis.per.sar}{0}
\verb{dispersar}{}{}{}{}{}{Disseminar, espalhar, dissipar.}{dis.per.sar}{\verboinum{1}}
\verb{dispersivo}{}{}{}{}{adj.}{Que provoca dispersão; que afugenta.}{dis.per.si.vo}{0}
\verb{dispersivo}{}{}{}{}{}{Que se desconcentra facilmente; distraído, desatento.}{dis.per.si.vo}{0}
\verb{disperso}{é}{}{}{}{adj.}{Que sofreu dispersão; desordenado, espalhado.}{dis.per.so}{0}
\verb{displasia}{}{Med.}{}{}{s.f.}{Desenvolvimento anormal de órgão ou tecido, provocando deformidades.}{dis.pla.si.a}{0}
\verb{display}{}{}{}{}{s.m.}{Peça promocional destinada a expor e anunciar determinado produto; expositor, mostruário.}{\textit{display}}{0}
\verb{display}{}{}{}{}{}{Anúncio colocado em balcões ou vitrinas para atrair fregueses. }{\textit{display}}{0}
\verb{display}{}{}{}{}{}{Dispositivo de aparelho eletrônico que exibe informações sobre seu funcionamento; visor, monitor, mostrador.}{\textit{display}}{0}
\verb{displicência}{}{}{}{}{s.f.}{Falta de interesse; descaso, negligência.}{dis.pli.cên.cia}{0}
\verb{displicente}{}{}{}{}{adj.2g.}{Que demonstra descaso; negligente, desinteressado.}{dis.pli.cen.te}{0}
\verb{dispneia}{é}{Med.}{}{}{s.f.}{Dificuldade de respirar; falta de ar.}{disp.nei.a}{0}
\verb{dispneico}{é}{}{}{}{adj.}{Relativo a dispneia.}{disp.nei.co}{0}
\verb{dispneico}{é}{}{}{}{}{Que sofre de dispneia.}{disp.nei.co}{0}
\verb{disponibilidade}{}{}{}{}{s.f.}{Qualidade ou estado do que é ou está disponível.}{dis.po.ni.bi.li.da.de}{0}
\verb{disponibilidade}{}{}{}{}{}{Condição do funcionário público ou militar que está afastado do exercício de suas funções, mas continua à disposição da administração.}{dis.po.ni.bi.li.da.de}{0}
\verb{disponibilizar}{}{}{}{}{v.t.}{Tornar disponível; pôr à disposição.}{dis.po.ni.bi.li.zar}{\verboinum{1}}
\verb{disponível}{}{}{"-eis}{}{adj.2g.}{Que está à disposição; livre, desocupado, desembaraçado.}{dis.po.ní.vel}{0}
\verb{dispor}{}{}{}{}{v.t.}{Colocar em lugar apropriado; arranjar, arrumar, ordenar.}{dis.por}{0}
\verb{dispor}{}{}{}{}{}{Ter à disposição; poder contar; possuir.}{dis.por}{0}
\verb{dispor}{}{}{}{}{}{Induzir, preceituar, orientar.}{dis.por}{0}
\verb{dispor}{}{}{}{}{}{Dar as regras; prescrever, determinar.}{dis.por}{0}
\verb{dispor}{}{}{}{}{v.pron.}{Colocar"-se à disposição; oferecer"-se.}{dis.por}{\verboinum{60}}
\verb{disposição}{}{}{"-ões}{}{s.f.}{Ato ou efeito de dispor; arrumação, arranjo.}{dis.po.si.ção}{0}
\verb{disposição}{}{}{"-ões}{}{}{Estado de espírito; humor, temperamento. (\textit{Hoje eu não estou com disposição para arrumar a casa.})}{dis.po.si.ção}{0}
\verb{disposição}{}{}{"-ões}{}{}{Predisposição, aptidão, tendência.}{dis.po.si.ção}{0}
\verb{disposição}{}{}{"-ões}{}{}{Prescrição legal; determinação, dispositivo.}{dis.po.si.ção}{0}
\verb{dispositivo}{}{}{}{}{s.m.}{Mecanismo adaptado a uma máquina com função específica.}{dis.po.si.ti.vo}{0}
\verb{dispositivo}{}{}{}{}{}{Preceito legal; determinação, disposição.}{dis.po.si.ti.vo}{0}
\verb{disposto}{ô}{}{"-s ⟨ó⟩}{"-a ⟨ó⟩}{adj.}{Que se dispôs.}{dis.pos.to}{0}
\verb{disposto}{ô}{}{"-s ⟨ó⟩}{"-a ⟨ó⟩}{}{Que demonstra boa disposição de ânimo; animado. (\textit{Hoje acordei disposto a arrumar meus livros.})}{dis.pos.to}{0}
\verb{disposto}{ô}{}{"-s ⟨ó⟩}{"-a ⟨ó⟩}{}{Inclinado, propenso. (\textit{Minha tia está disposta a vender a casa e comprar um \textit{trailer}).}}{dis.pos.to}{0}
\verb{disprósio}{}{Quím.}{}{}{s.m.}{Elemento químico metálico, brilhante, prateado, mole, da família dos lantanídeos (terras"-raras), usado em aparelhos de televisão e reatores nucleares. \elemento{66}{162.5}{Dy}.}{dis.pró.sio}{0}
\verb{disputa}{}{}{}{}{s.f.}{Falta de acordo; contenda, discussão.}{dis.pu.ta}{0}
\verb{disputa}{}{}{}{}{}{Competição esportiva; partida, jogo.}{dis.pu.ta}{0}
\verb{disputar}{}{}{}{}{v.t.}{Lutar ou competir pela posse; pleitear.}{dis.pu.tar}{\verboinum{1}}
\verb{disquete}{é}{Informát.}{}{}{s.m.}{Disco magnético de material flexível, próprio para guardar dados e programas de computador.}{dis.que.te}{0}
\verb{disritmia}{}{Med.}{}{}{s.f.}{Perturbação do ritmo cardíaco ou cerebral.}{dis.rit.mi.a}{0}
\verb{dissabor}{ô}{}{}{}{s.m.}{Desgosto de quem foi contrariado; mágoa, contrariedade, desprazer.}{dis.sa.bor}{0}
\verb{dissecação}{}{}{"-ões}{}{s.f.}{Ato ou efeito de dissecar; dissecção.}{dis.se.ca.ção}{0}
\verb{dissecação}{}{Fig.}{"-ões}{}{}{Exame minucioso, detalhado.}{dis.se.ca.ção}{0}
\verb{dissecar}{}{}{}{}{v.t.}{Separar partes de um órgão ou de um cadáver para estudo.}{dis.se.car}{\verboinum{2}}
\verb{dissecção}{}{}{"-ões}{}{s.f.}{Dissecação.}{dis.sec.ção}{0}
\verb{disseminação}{}{}{"-ões}{}{s.f.}{Ato ou efeito de disseminar; espalhamento, difusão.}{dis.se.mi.na.ção}{0}
\verb{disseminar}{}{}{}{}{v.t.}{Fazer deslocar"-se para longe; espalhar, difundir.}{dis.se.mi.nar}{\verboinum{1}}
\verb{dissensão}{}{}{"-ões}{}{s.f.}{Falta de concordância; divergência, discrepância.}{dis.sen.são}{0}
\verb{dissentir}{}{}{}{}{v.t.}{Ter opinião contrária; não concordar; divergir, discrepar.}{dis.sen.tir}{\verboinum{29}}
%\verb{}{}{}{}{}{}{}{}{0}
\verb{dissertação}{}{}{"-ões}{}{s.f.}{Ato ou efeito de dissertar; exposição, argumentação.}{dis.ser.ta.ção}{0}
\verb{dissertação}{}{}{"-ões}{}{}{Texto em que se disserta; tese, ensaio.}{dis.ser.ta.ção}{0}
\verb{dissertar}{}{}{}{}{v.t.}{Falar ou escrever sobre determinado assunto; discorrer.}{dis.ser.tar}{\verboinum{1}}
\verb{dissidência}{}{}{}{}{s.f.}{Ato de parte de um grupo separar"-se do todo por motivos de divergência; dissensão, cisma, cisão.}{dis.si.dên.cia}{0}
\verb{dissidência}{}{}{}{}{}{A parte do grupo que se separou.}{dis.si.dên.cia}{0}
\verb{dissidente}{}{}{}{}{adj.2g.}{Que tem opiniões divergentes e se separa do grupo a que pertence.}{dis.si.den.te}{0}
\verb{dissídio}{}{}{}{}{s.m.}{Conflito de interesses; divergência, dissensão.}{dis.sí.dio}{0}
\verb{dissídio}{}{}{}{}{}{Acordo coletivo entre patrões e empregados para reajuste salarial.}{dis.sí.dio}{0}
\verb{dissílabo}{}{Gram.}{}{}{adj.}{Que tem duas sílabas.}{dis.sí.la.bo}{0}
\verb{dissimilação}{}{Gram.}{"-ões}{}{s.f.}{Ato ou efeito de tornar alguma coisa dessemelhante de outra; diferenciação.}{dis.si.mi.la.ção}{0}
\verb{dissimilar}{}{}{}{}{v.t.}{Tornar dessemelhante; diferenciar.}{dis.si.mi.lar}{\verboinum{1}}
\verb{dissimilar}{}{}{}{}{adj.2g.}{De natureza diferente; dessemelhante; heterogêneo.}{dis.si.mi.lar}{0}
\verb{dissimulação}{}{}{"-ões}{}{s.f.}{Ato ou efeito de dissimular; disfarce, fingimento.}{dis.si.mu.la.ção}{0}
\verb{dissimulado}{}{}{}{}{adj.}{Que tem o hábito de dissimular; fingido, disfarçado.}{dis.si.mu.la.do}{0}
\verb{dissimular}{}{}{}{}{v.t.}{Não deixar que algo seja descoberto ou reconhecido; ocultar, esconder, camuflar.}{dis.si.mu.lar}{0}
\verb{dissimular}{}{}{}{}{}{Não revelar os sentimentos; fingir, simular.}{dis.si.mu.lar}{\verboinum{1}}
\verb{dissipação}{}{}{"-ões}{}{s.f.}{Ato ou efeito de dissipar; esbanjamento, desperdício.}{dis.si.pa.ção}{0}
\verb{dissipador}{ô}{}{}{}{adj.}{Que desperdiça; esbanjador, perdulário.}{dis.si.pa.dor}{0}
\verb{dissipar}{}{}{}{}{v.t.}{Gastar rapidamente alguma coisa sem necessidade; esbanjar, desperdiçar.}{dis.si.par}{0}
\verb{dissipar}{}{}{}{}{}{Fazer desaparecer; desvanecer, desfazer.}{dis.si.par}{0}
\verb{dissipar}{}{}{}{}{}{Arruinar por excesso; perder, estragar.}{dis.si.par}{\verboinum{1}}
\verb{disso}{}{}{}{}{}{Contração da preposição \textit{de} com o pronome \textit{isso}.}{dis.so}{0}
\verb{dissociação}{}{}{"-ões}{}{s.f.}{Ato ou efeito de dissociar; separação,  desagregação.}{dis.so.ci.a.ção}{0}
\verb{dissociar}{}{}{}{}{v.t.}{Desfazer uma associação; separar, desagregar.}{dis.so.ci.ar}{\verboinum{1}}
\verb{dissolução}{}{}{"-ões}{}{s.f.}{Ato ou efeito de dissolver; decomposição, desagregação.}{dis.so.lu.ção}{0}
\verb{dissolução}{}{Fís. e Quím.}{"-ões}{}{}{Liquefação de um sólido em contato com um líquido.}{dis.so.lu.ção}{0}
\verb{dissolução}{}{Fig.}{"-ões}{}{}{Deterioração dos costumes; devassidão, depravação.}{dis.so.lu.ção}{0}
\verb{dissoluto}{}{}{}{}{adj.}{Que apresenta maus costumes; depravado, devasso, libertino. (\textit{O moço levava uma vida dissoluta segundo os costumes da época.})}{dis.so.lu.to}{0}
\verb{dissolúvel}{}{}{"-eis}{}{adj.2g.}{Que se pode dissolver; solúvel.}{dis.so.lú.vel}{0}
\verb{dissolvência}{}{}{}{}{s.f.}{Dissolução.}{dis.sol.vên.cia}{0}
\verb{dissolvente}{}{}{}{}{adj.2g.}{Que tem a propriedade de dissolver; solvente.}{dis.sol.ven.te}{0}
\verb{dissolver}{ê}{}{}{}{v.t.}{Desfazer uma substância sólida em um meio líquido; liquefazer, diluir.}{dis.sol.ver}{0}
\verb{dissolver}{ê}{}{}{}{}{Desagregar, desunir, separar.}{dis.sol.ver}{0}
\verb{dissolver}{ê}{}{}{}{}{Fazer desaparecer; desmanchar, desfazer.}{dis.sol.ver}{\verboinum{12}}
\verb{dissonância}{}{}{}{}{s.f.}{Qualidade de dissonante; desafinação, desarmonia.}{dis.so.nân.cia}{0}
\verb{dissonante}{}{}{}{}{adj.2g.}{Que soa mal; desafinado, discordante.}{dis.so.nan.te}{0}
\verb{dissuadir}{}{}{}{}{v.t.}{Fazer mudar de opinião; desaconselhar. (\textit{Conseguimos dissuadi"-lo de parar de estudar.})}{dis.su.a.dir}{\verboinum{18}}
\verb{dissuasão}{}{}{"-ões}{}{s.f.}{Ato ou efeito de dissuadir.}{dis.su.a.são}{0}
\verb{distal}{}{Anat.}{}{}{adj.2g.}{Diz"-se do ponto que fica mais distante da linha que passa pelo meio do corpo de alto a baixo.}{dis.tal}{0}
\verb{distância}{}{}{}{}{s.f.}{Espaço que separa duas coisas, pessoas ou épocas; afastamento.}{dis.tân.cia}{0}
\verb{distanciamento}{}{}{}{}{s.m.}{Ato ou efeito de distanciar; afastamento.}{dis.tan.ci.a.men.to}{0}
\verb{distanciar}{}{}{}{}{v.t.}{Mover para longe; afastar, separar.}{dis.tan.ci.ar}{\verboinum{1}}
\verb{distante}{}{}{}{}{adj.2g.}{Que está afastado no tempo ou no espaço; longe, remoto.}{dis.tan.te}{0}
\verb{distar}{}{}{}{}{v.t.}{Ficar a certa distância de algo. (\textit{A cidade do Rio de Janeiro dista 400 km da cidade de São Paulo.})}{dis.tar}{\verboinum{1}}
\verb{distender}{ê}{}{}{}{v.t.}{Tornar longo; esticar, estirar.}{dis.ten.der}{0}
\verb{distender}{ê}{Med.}{}{}{}{Causar distensão.}{dis.ten.der}{\verboinum{12}}
\verb{distensão}{}{}{"-ões}{}{s.f.}{Ato ou efeito de distender.}{dis.ten.são}{0}
\verb{distensão}{}{Med.}{"-ões}{}{}{Estiramento ou torsão violenta dos ligamentos de uma articulação. (\textit{O jogador sofreu distensão na panturrilha.})}{dis.ten.são}{0}
\verb{dístico}{}{Gram.}{}{}{s.m.}{Estrofe mínima, constituída de dois versos.}{dís.ti.co}{0}
\verb{distinção}{}{}{"-ões}{}{s.f.}{Ato ou efeito de distinguir; diferenciação, discriminação.}{dis.tin.ção}{0}
\verb{distinção}{}{}{"-ões}{}{}{Aquilo que diferencia uma pessoa ou coisa de outra; característica, peculiaridade.}{dis.tin.ção}{0}
\verb{distinção}{}{}{"-ões}{}{}{Boa educação; fineza, cortesia.}{dis.tin.ção}{0}
\verb{distinção}{}{}{"-ões}{}{}{A classificação mais alta em um exame; honraria, condecoração. (\textit{O candidato ao título de mestre foi aprovado com distinção pela banca examinadora.})}{dis.tin.ção}{0}
\verb{distinguir}{}{}{}{}{v.t.}{Perceber determinada coisa através dos sentidos; separar, diferenciar.}{dis.tin.guir}{0}
\verb{distinguir}{}{}{}{}{}{Detalhar, discriminar, discernir.}{dis.tin.guir}{0}
\verb{distinguir}{}{}{}{}{}{Conceder uma honraria, uma condecoração; notabilizar, premiar.}{dis.tin.guir}{\verboinum{23}}
\verb{distintivo}{}{}{}{}{adj.}{Que serve para distinguir, diferenciar.}{dis.tin.ti.vo}{0}
\verb{distintivo}{}{}{}{}{s.m.}{Insígnia ou emblema que serve para identificar uma profissão, um cargo, uma patente.}{dis.tin.ti.vo}{0}
\verb{distinto}{}{}{}{}{adj.}{Que não é igual; diferente, diverso.}{dis.tin.to}{0}
\verb{distinto}{}{}{}{}{}{Que se destaca; nítido, perceptível.}{dis.tin.to}{0}
\verb{distinto}{}{}{}{}{}{Nobre, cortês, fino.}{dis.tin.to}{0}
\verb{disto}{}{}{}{}{}{Contração da preposição \textit{de} com o pronome \textit{isto}.}{dis.to}{0}
\verb{distonia}{}{Med.}{}{}{s.f.}{Distúrbio ou alteração da tonicidade de qualquer tecido orgânico.}{dis.to.ni.a}{0}
\verb{distorção}{}{}{"-ões}{}{s.f.}{Ato ou efeito de distorcer; deformação, alteração.}{dis.tor.ção}{0}
\verb{distorcer}{ê}{}{}{}{v.t.}{Alterar o sentido ou a forma de algo; deformar, desvirtuar.}{dis.tor.cer}{\verboinum{15}}
\verb{distração}{}{}{"-ões}{}{s.f.}{Falta de concentração; descuido, desatenção.}{dis.tra.ção}{0}
\verb{distração}{}{}{"-ões}{}{}{Recreação, divertimento, passatempo.}{dis.tra.ção}{0}
\verb{distraído}{}{}{}{}{adj.}{Que perde a concentração; desatento, alheado.}{dis.tra.í.do}{0}
\verb{distrair}{}{}{}{}{v.t.}{Desviar a atenção de algo; desconcentrar.}{dis.tra.ir}{0}
\verb{distrair}{}{}{}{}{}{Divertir, entreter, recrear.}{dis.tra.ir}{\verboinum{19}}
\verb{distratar}{}{}{}{}{v.t.}{Desfazer um trato, anular, rescindir.}{dis.tra.tar}{\verboinum{1}}
\verb{distrato}{}{}{}{}{s.m.}{Ato de distratar; rescisão, anulação.}{dis.tra.to}{0}
\verb{distribuição}{}{}{"-ões}{}{s.f.}{Ato ou efeito de distribuir; divisão, disposição.}{dis.tri.bu.i.ção}{0}
\verb{distribuidor}{ô}{}{}{}{adj.}{Que distribui, que entrega.}{dis.tri.bu.i.dor}{0}
\verb{distribuidor}{ô}{}{}{}{s.m.}{Nos motores a explosão, mecanismo que distribui a corrente para as velas de ignição.}{dis.tri.bu.i.dor}{0}
\verb{distribuir}{}{}{}{}{v.t.}{Enviar para diversas direções; entregar, espalhar.}{dis.tri.bu.ir}{0}
\verb{distribuir}{}{}{}{}{}{Pôr em ordem; classificar.}{dis.tri.bu.ir}{\verboinum{26}}
\verb{distributivo}{}{}{}{}{adj.}{Que distribui ou que indica distribuição.}{dis.tri.bu.ti.vo}{0}
\verb{distributivo}{}{Mat.}{}{}{}{Diz"-se da propriedade que relaciona as operações de multiplicação e adição.}{dis.tri.bu.ti.vo}{0}
\verb{distrito}{}{}{}{}{s.m.}{Divisão de município a cargo de autoridade administrativa, judicial ou fiscal.}{dis.tri.to}{0}
\verb{distrofia}{}{Med.}{}{}{s.f.}{Enfraquecimento progressivo dos músculos por problemas de nutrição.}{dis.tro.fi.a}{0}
\verb{distúrbio}{}{}{}{}{s.m.}{Perturbação da ordem pública; motim, desordem.}{dis.túr.bio}{0}
\verb{distúrbio}{}{Med.}{}{}{}{Mau funcionamento de alguma parte do organismo; disfunção.}{dis.túr.bio}{0}
\verb{dita}{}{}{}{}{s.f.}{Boa sorte; ventura, felicidade.}{di.ta}{0}
\verb{ditado}{}{}{}{}{adj.}{Que se ditou para ser escrito.}{di.ta.do}{0}
\verb{ditado}{}{}{}{}{s.m.}{Texto que se dita a alunos como exercício de ortografia ou caligrafia.}{di.ta.do}{0}
\verb{ditado}{}{}{}{}{}{Provérbio, adágio.}{di.ta.do}{0}
\verb{ditador}{ô}{}{}{}{s.m.}{Indivíduo que preside uma ditadura; chefe autoritário.}{di.ta.dor}{0}
\verb{ditadura}{}{}{}{}{s.f.}{Forma de governo em que o chefe de Estado tem poderes ilimitados; despotismo, autoritarismo.}{di.ta.du.ra}{0}
\verb{ditame}{}{}{}{}{s.m.}{Aquilo que a razão e a consciência ditam; conselho, preceito, critério.}{di.ta.me}{0}
\verb{ditar}{}{}{}{}{v.t.}{Dizer algo que outrem deve escrever.}{di.tar}{0}
\verb{ditar}{}{}{}{}{}{Prescrever, impor.}{di.tar}{\verboinum{1}}
\verb{ditatorial}{}{}{"-ais}{}{adj.2g.}{Relativo a ditador ou a ditadura.}{di.ta.to.ri.al}{0}
\verb{dito}{}{}{}{}{adj.}{Que se disse; mencionado, referido.}{di.to}{0}
\verb{dito}{}{}{}{}{s.m.}{Máxima, ditado, adágio.}{di.to}{0}
\verb{dito"-cujo}{}{Pop.}{ditos"-cujos}{}{s.m.}{Qualquer indivíduo de quem não se sabe ou não se quer dizer o nome; sujeito, fulano, camarada, cujo.}{di.to"-cu.jo}{0}
\verb{ditongação}{}{Gram.}{"-ões}{}{s.f.}{Variação fonética que consiste na união, em uma mesma sílaba, de uma vogal e uma semivogal, formando um ditongo.}{di.ton.ga.ção}{0}
\verb{ditongar}{}{}{}{}{v.t.}{Transformar uma vogal ou um hiato em ditongo.}{di.ton.gar}{\verboinum{5}}
\verb{ditongo}{}{Gram.}{}{}{s.m.}{Emissão de uma vogal mais uma semivogal em uma mesma sílaba.}{di.ton.go}{0}
\verb{ditoso}{ô}{}{"-osos ⟨ó⟩}{"-osa ⟨ó⟩}{adj.}{Que tem boa dita; feliz, venturoso.}{di.to.so}{0}
\verb{DIU}{}{}{}{}{s.m.}{Sigla de \textit{Dispositivo Intrauterino}, contraceptivo que se coloca no útero.}{DIU}{0}
\verb{diurese}{é}{Med.}{}{}{s.f.}{Secreção urinária, natural ou provocada.}{di.u.re.se}{0}
\verb{diurético}{}{}{}{}{adj.}{Diz"-se da substância que facilita a secreção da urina.}{di.u.ré.ti.co}{0}
\verb{diurno}{}{}{}{}{adj.}{Que se passa durante o dia.}{di.ur.no}{0}
\verb{diuturno}{}{}{}{}{adj.}{Que tem longa duração; prolongado.}{di.u.tur.no}{0}
\verb{divã}{}{}{}{}{s.m.}{Espécie de sofá, sem encosto nem braços.}{di.vã}{0}
\verb{diva}{}{}{}{}{s.f.}{Divindade feminina; deusa.}{di.va}{0}
\verb{diva}{}{Fig.}{}{}{}{Cantora de ópera notável.}{di.va}{0}
\verb{divagação}{}{}{"-ões}{}{s.f.}{Ato ou efeito de divagar; digressão. }{di.va.ga.ção}{0}
\verb{divagação}{}{}{"-ões}{}{}{Devaneio, fantasia, sonho.}{di.va.ga.ção}{0}
\verb{divagar}{}{}{}{}{v.t.}{Desviar"-se do assunto; fazer digressões.}{di.va.gar}{0}
\verb{divagar}{}{}{}{}{}{Devanear, fantasiar, sonhar.}{di.va.gar}{\verboinum{5}}
\verb{divergência}{}{}{}{}{s.f.}{Ato ou efeito de divergir; desvio, afastamento.}{di.ver.gên.cia}{0}
\verb{divergência}{}{}{}{}{}{Desacordo, dissensão, discrepância.}{di.ver.gên.cia}{0}
\verb{divergente}{}{}{}{}{adj.2g.}{Que diverge, que se afasta; discordante, oposto.}{di.ver.gen.te}{0}
\verb{divergir}{}{}{}{}{v.t.}{Afastar"-se de um ponto; separar"-se, desviar"-se.}{di.ver.gir}{0}
\verb{divergir}{}{}{}{}{}{Ter opinião diferente; discordar, dissentir.}{di.ver.gir}{\verboinum{25}}
\verb{diversão}{}{}{"-ões}{}{s.f.}{Ato ou efeito de divertir; entretenimento, distração, divertimento.}{di.ver.são}{0}
\verb{diversidade}{}{}{}{}{s.f.}{Qualidade do que é diverso; variedade, multiplicidade. (\textit{No mercado central, há uma grande diversidade de frutas e cereais.})}{di.ver.si.da.de}{0}
\verb{diversificar}{}{}{}{}{v.t.}{Tornar diverso, diferente; variar.}{di.ver.si.fi.car}{\verboinum{2}}
\verb{diverso}{é}{}{}{}{adj.}{Que apresenta vários aspectos; diferente, distinto.}{di.ver.so}{0}
\verb{diversos}{é}{}{}{}{pron.}{Mais de um; vários, alguns.}{di.ver.sos}{0}
\verb{diverticulite}{}{Med.}{}{}{s.f.}{Inflamação de um ou mais divertículos.}{di.ver.ti.cu.li.te}{0}
\verb{divertículo}{}{Anat.}{}{}{s.m.}{Apêndice oco, em forma de bolsa, que está em comunicação com alguns órgãos ocos, como o esôfago ou o intestino.}{di.ver.tí.cu.lo}{0}
\verb{divertido}{}{}{}{}{adj.}{Que diverte; alegre, engraçado.}{di.ver.ti.do}{0}
\verb{divertimento}{}{}{}{}{s.m.}{Aquilo que diverte; entretenimento, recreação, diversão.}{di.ver.ti.men.to}{0}
\verb{divertir}{}{}{}{}{v.t.}{Entreter com brincadeiras; fazer rir; distrair, recrear.}{di.ver.tir}{\verboinum{29}}
\verb{dívida}{}{}{}{}{s.f.}{Quantia de dinheiro que se deve; débito.}{dí.vi.da}{0}
\verb{dívida}{}{}{}{}{}{Obrigação moral; dever.}{dí.vi.da}{0}
\verb{dividendo}{}{Mat.}{}{}{s.m.}{Número que deve ser dividido por outro na operação de divisão.}{di.vi.den.do}{0}
\verb{dividendo}{}{}{}{}{}{Lucro, cota, porcentagem.}{di.vi.den.do}{0}
\verb{dividir}{}{}{}{}{v.t.}{Separar em partes; repartir.}{di.vi.dir}{0}
\verb{dividir}{}{Mat.}{}{}{}{Efetuar uma operação de divisão.}{di.vi.dir}{\verboinum{18}}
\verb{divinação}{}{}{"-ões}{}{s.f.}{Arte de adivinhar; adivinhação, pressentimento.}{di.vi.na.ção}{0}
\verb{divinal}{}{}{"-ais}{}{adj.2g.}{Divino.}{di.vi.nal}{0}
\verb{divinatório}{}{}{}{}{adj.}{Relativo a adivinhação.}{di.vi.na.tó.rio}{0}
\verb{divindade}{}{}{}{}{s.f.}{Qualidade de divino.}{di.vin.da.de}{0}
\verb{divindade}{}{}{}{}{}{Deus ou deusa.}{di.vin.da.de}{0}
\verb{divinizar}{}{}{}{}{v.t.}{Atribuir caráter divino; adorar.}{di.vi.ni.zar}{0}
\verb{divinizar}{}{Fig.}{}{}{}{Tornar sublime; exaltar.}{di.vi.ni.zar}{\verboinum{1}}
\verb{divino}{}{}{}{}{adj.}{Próprio de Deus.}{di.vi.no}{0}
\verb{divino}{}{Por ext.}{}{}{}{Sublime, perfeito, maravilhoso.}{di.vi.no}{0}
\verb{divisa}{}{}{}{}{s.f.}{Linha divisória entre territórios; fronteira, limite.}{di.vi.sa}{0}
\verb{divisa}{}{}{}{}{}{Sinal distintivo; emblema.}{di.vi.sa}{0}
\verb{divisa}{}{}{}{}{}{Cada um dos galões que representam as patentes militares.}{di.vi.sa}{0}
\verb{divisão}{}{}{"-ões}{}{s.f.}{Ato ou efeito de dividir; segmentação.}{di.vi.são}{0}
\verb{divisão}{}{Mat.}{"-ões}{}{}{Operação que consiste em determinar quantas vezes um número está contido em outro.}{di.vi.são}{0}
\verb{divisão}{}{}{"-ões}{}{}{Parte de um exército ou de uma esquadra.}{di.vi.são}{0}
\verb{divisar}{}{}{}{}{v.t.}{Distinguir pela vista; enxergar, avistar.}{di.vi.sar}{\verboinum{1}}
\verb{divisas}{}{}{}{}{s.f.pl.}{Dinheiro passível de ser convertido em moedas estrangeiras, por governos e por entidades privadas nas transações comerciais.}{di.vi.sas}{0}
\verb{divisibilidade}{}{}{}{}{s.f.}{Qualidade do que é divisível.}{di.vi.si.bi.li.da.de}{0}
\verb{divisionário}{}{}{}{}{adj.}{Relativo a divisão militar.}{di.vi.si.o.ná.rio}{0}
\verb{divisionário}{}{}{}{}{}{Diz"-se da moeda que corresponde à divisão da unidade monetária, geralmente destinada a troco.}{di.vi.si.o.ná.rio}{0}
\verb{divisível}{}{}{"-eis}{}{adj.2g.}{Que pode ser dividido por outro número de modo exato.}{di.vi.sí.vel}{0}
\verb{divisor}{ô}{}{}{}{adj.}{Que divide, segmenta.}{di.vi.sor}{0}
\verb{divisor}{ô}{Mat.}{}{}{s.m.}{Número pelo qual se divide outro.}{di.vi.sor}{0}
\verb{divisória}{}{}{}{}{s.f.}{Parede fina que divide um compartimento; biombo.}{di.vi.só.ria}{0}
\verb{divisório}{}{}{}{}{adj.}{Relativo a divisão.}{di.vi.só.rio}{0}
\verb{divisório}{}{}{}{}{}{Que divide ou serve para dividir.}{di.vi.só.rio}{0}
\verb{divorciar}{}{}{}{}{v.t.}{Decretar o divórcio.}{di.vor.ci.ar}{0}
\verb{divorciar}{}{}{}{}{v.pron.}{Separar"-se judicialmente. (\textit{Minha amiga se divorciou do marido após 15 anos de casamento.})}{di.vor.ci.ar}{\verboinum{1}}
\verb{divórcio}{}{}{}{}{s.m.}{Dissolução do casamento, feito de acordo com a lei, ficando as pessoas livres para se casar novamente.}{di.vór.cio}{0}
\verb{divorcista}{}{}{}{}{adj.2g.}{Que é a favor da regulamentação do divórcio.}{di.vor.cis.ta}{0}
\verb{divulgação}{}{}{"-ões}{}{s.f.}{Ato ou efeito de divulgar; difusão, propagação.}{di.vul.ga.ção}{0}
\verb{divulgador}{ô}{}{}{}{adj.}{Que divulga; propagandista.}{di.vul.ga.dor}{0}
\verb{divulgar}{}{}{}{}{v.t.}{Tornar conhecido; difundir, propagar.}{di.vul.gar}{\verboinum{5}}
\verb{dizer}{ê}{}{}{}{v.t.}{Exprimir através de palavras; enunciar, comunicar.}{di.zer}{0}
\verb{dizer}{ê}{}{}{}{}{Falar, proferir, discorrer.}{di.zer}{\verboinum{41}}
\verb{dízima}{}{}{}{}{s.f.}{Imposto ou contribuição correspondente à décima parte do rendimento de uma pessoa.}{dí.zi.ma}{0}
\verb{dízima}{}{Mat.}{}{}{}{Fração decimal que resulta de uma fração ordinária.}{dí.zi.ma}{0}
\verb{dizimação}{}{}{"-ões}{}{s.f.}{Ato ou efeito de dizimar; extermínio, destruição.}{di.zi.ma.ção}{0}
\verb{dizimar}{}{}{}{}{v.t.}{Destruir quase completamente; exterminar, matar.}{di.zi.mar}{\verboinum{1}}
\verb{dízimo}{}{}{}{}{s.m.}{A décima parte.}{dí.zi.mo}{0}
\verb{dízimo}{}{}{}{}{}{Quantidade que se doa para fins religiosos.}{dí.zi.mo}{0}
\verb{diz"-que"-diz"-que}{}{}{}{}{s.m.}{Boato, falatório, intriga.}{diz"-que"-diz"-que}{0}
\verb{dj}{}{}{}{}{s.m.}{Indivíduo que seleciona e toca discos para dançar em um baile, boate ou danceteria; discotecário.}{dj}{0}
\verb{DNA}{}{Biol.}{}{}{s.m.}{Abrev. do inglês \textit{DesoxyriboNucleic Acid}, \textsc{adn}; molécula cuja estrutura é a de uma dupla hélice e que contém as informações genéticas de um organismo.}{dna}{0}
\verb{dó}{}{Mús.}{}{}{s.m.}{A primeira nota musical na escala de \textit{dó}.}{dó}{0}
\verb{dó}{}{}{}{}{s.m.}{Sentimento de pena, compaixão, piedade.}{dó}{0}
\verb{do}{}{}{}{}{}{Contração da preposição \textit{de} com o artigo \textit{o}.}{do}{0}
\verb{doação}{}{}{"-ões}{}{s.f.}{Ato ou efeito de doar, ofertar.}{do.a.ção}{0}
\verb{doação}{}{}{"-ões}{}{}{Aquilo que se doa; donativo.}{do.a.ção}{0}
\verb{doador}{ô}{}{}{}{adj.}{Que faz doação; que transfere ou concede algo.}{do.a.dor}{0}
\verb{doar}{}{}{}{}{v.t.}{Conceder gratuitamente sem esperar nada em troca; dar.}{do.ar}{\verboinum{7}}
\verb{dobar}{}{}{}{}{v.t.}{Enrolar fio de meada; enovelar.}{do.bar}{0}
\verb{dobar}{}{}{}{}{}{Dar voltas; rodopiar, voltear.}{do.bar}{\verboinum{1}}
\verb{dóbermã}{}{}{}{}{s.m.}{Cão de guarda de origem alemã, de grande porte, preto ou com tons de marrom.}{dó.ber.mã}{0}
\verb{doblez}{ê}{}{}{}{s.f.}{Dissimulação de sentimento; fingimento, hipocrisia; dobrez.}{do.blez}{0}
\verb{dobra}{ó}{}{}{}{s.f.}{Parte de um objeto que, virada, fica por cima de outra parte.}{do.bra}{0}
\verb{dobra}{ó}{}{}{}{}{Vinco, prega.}{do.bra}{0}
\verb{dobradiça}{}{}{}{}{s.f.}{Peça de metal composta de duas chapas unidas por um eixo, e sobre a qual gira uma janela ou uma porta.}{do.bra.di.ça}{0}
\verb{dobradiço}{}{}{}{}{adj.}{Que se dobra com facilidade; flexível.}{do.bra.diço}{0}
\verb{dobradinha}{}{Cul.}{}{}{s.f.}{Guisado feito com o bucho do boi e feijão branco.}{do.bra.di.nha}{0}
\verb{dobradinha}{}{}{}{}{}{Conjunto de dois elementos; dupla.}{do.bra.di.nha}{0}
\verb{dobrado}{}{}{}{}{adj.}{Que se dobrou.}{do.bra.do}{0}
\verb{dobrado}{}{}{}{}{}{Que se multiplicou por dois; duplicado.}{do.bra.do}{0}
\verb{dobradura}{}{}{}{}{s.f.}{Ato ou efeito de dobrar; curvatura.}{do.bra.du.ra}{0}
\verb{dobramento}{}{}{}{}{s.m.}{Dobradura.}{do.bra.men.to}{0}
\verb{dobrar}{}{}{}{}{v.t.}{Tornar duas vezes maior; duplicar.}{do.brar}{0}
\verb{dobrar}{}{}{}{}{}{Fazer dobra em tecido ou papel, virando uma parte sobre a outra.}{do.brar}{0}
\verb{dobrar}{}{}{}{}{}{Mudar a direção; virar.}{do.brar}{0}
\verb{dobrar}{}{}{}{}{}{Fazer soar o sino. (\textit{O ano passado li "Por quem os sinos dobram" de Hemingway.})}{do.brar}{0}
\verb{dobrar}{}{}{}{}{}{Fazer ceder; coagir, dominar.}{do.brar}{\verboinum{1}}
\verb{dobre}{ó}{}{}{}{adj.}{Que se apresenta dobrado; duplicado.}{do.bre}{0}
\verb{dobre}{ó}{}{}{}{s.m.}{Toque especial do sino em certos ritos litúrgicos.}{do.bre}{0}
\verb{dobrez}{ê}{}{}{}{s.f.}{Doblez.}{do.brez}{0}
\verb{dobro}{ô}{}{}{}{s.m.}{Quantidade que equivale duas vezes a uma outra; duplo.}{do.bro}{0}
\verb{doca}{ó}{}{}{}{s.f.}{Construção em um porto, destinada a recolher embarcações para fugir do mau tempo ou para carregar e descarregar mercadorias.}{do.ca}{0}
\verb{doçaria}{}{}{}{}{s.f.}{Local onde se fabricam ou se vendem doces.}{do.ça.ri.a}{0}
\verb{doce}{ô}{}{}{}{adj.}{Que tem sabor açucarado ou melado.}{do.ce}{0}
\verb{doce}{ô}{}{}{}{}{Que é temperado com açúcar, mel ou adoçante.}{do.ce}{0}
\verb{doce}{ô}{Fig.}{}{}{}{Que demonstra docilidade; afetuoso, terno.}{do.ce}{0}
\verb{doce}{ô}{}{}{}{}{Preparado culinário à base de açúcar, mel ou adoçante; guloseima.}{do.ce}{0}
\verb{doceira}{ê}{}{}{}{s.f.}{Mulher que faz ou vende doces; confeiteira.}{do.cei.ra}{0}
\verb{doceiro}{ê}{}{}{}{s.m.}{Indivíduo que faz ou vende doces; confeiteiro.}{do.cei.ro}{0}
\verb{docência}{}{}{}{}{s.f.}{Qualidade de docente.}{do.cên.cia}{0}
\verb{docência}{}{}{}{}{}{Ensino, magistério, professorado.}{do.cên.cia}{0}
%\verb{}{}{}{}{}{}{}{}{0}
\verb{docente}{}{}{}{}{adj.2g.}{Relativo a ensino ou àquele que ensina.}{do.cen.te}{0}
\verb{docente}{}{}{}{}{s.2g.}{Professor, mestre.}{do.cen.te}{0}
%\verb{}{}{}{}{}{}{}{}{0}
\verb{dócil}{}{}{"-eis}{}{adj.2g.}{Que aprende com facilidade; brando, manso.}{dó.cil}{0}
\verb{docilidade}{}{}{}{}{s.f.}{Qualidade do que é dócil; obediência, submissão.}{do.ci.li.da.de}{0}
\verb{documentação}{}{}{"-ões}{}{s.f.}{Ato ou efeito de documentar.}{do.cu.men.ta.ção}{0}
\verb{documentação}{}{}{"-ões}{}{}{Conjunto de documentos relativos a um assunto.}{do.cu.men.ta.ção}{0}
\verb{documental}{}{}{"-ais}{}{adj.2g.}{Relativo a documento.}{do.cu.men.tal}{0}
\verb{documentar}{}{}{}{}{v.t.}{Provar com documentos.}{do.cu.men.tar}{\verboinum{1}}
\verb{documentário}{}{}{}{}{adj.}{Que tem valor ou caráter de documento.}{do.cu.men.tá.rio}{0}
\verb{documentário}{}{}{}{}{s.m.}{Filme que registra, interpreta e comenta fatos ou situações.}{do.cu.men.tá.rio}{0}
\verb{documento}{}{}{}{}{s.m.}{Qualquer escrito ou objeto que elucide, testemunhe ou comprove cientificamente algum fato, acontecimento, dito etc.}{do.cu.men.to}{0}
\verb{doçura}{}{}{}{}{s.f.}{Qualidade ou gosto de doce.}{do.çu.ra}{0}
\verb{doçura}{}{Fig.}{}{}{}{Suavidade, brandura, candura.}{do.çu.ra}{0}
\verb{dodecaedro}{é}{Geom.}{}{}{s.m.}{Poliedro de 12 faces.}{do.de.ca.e.dro}{0}
\verb{dodecafônico}{}{}{}{}{adj.}{Relativo a dodecafonismo.}{do.de.ca.fô.ni.co}{0}
\verb{dodecafonismo}{}{Mús.}{}{}{s.m.}{Técnica de composição musical que emprega os 12 sons da escala cromática.}{do.de.ca.fo.nis.mo}{0}
\verb{dodecágono}{}{Geom.}{}{}{s.m.}{Polígono que tem 12 ângulos e, consequentemente, 12 lados.  }{do.de.cá.go.no}{0}
\verb{dodecassílabo}{}{Gram.}{}{}{adj.}{Que tem 12 sílabas.  }{do.de.cas.sí.la.bo}{0}
\verb{dodói}{}{Pop.}{}{}{adj.}{Que está doente.}{do.dói}{0}
\verb{dodói}{}{Pop.}{}{}{}{Ferimento, machucado.}{do.dói}{0}
\verb{dodói}{}{Pop.}{}{}{}{Doença, moléstia.}{do.dói}{0}
\verb{doença}{}{}{}{}{s.f.}{Estado de perturbação mais ou menos grave do organismo; moléstia, enfermidade.}{do.en.ça}{0}
\verb{doente}{}{}{}{}{adj.2g.}{Que está com a saúde alterada, fragilizada; enfermo, fraco.}{do.en.te}{0}
\verb{doentio}{}{}{}{}{adj.}{Que adoece com facilidade; débil, enfermiço.}{do.en.ti.o}{0}
\verb{doentio}{}{}{}{}{}{Que aparenta estar doente; mórbido.}{do.en.ti.o}{0}
\verb{doer}{ê}{}{}{}{v.i.}{Causar dor, pena, angústia.}{do.er}{0}
\verb{doer}{ê}{}{}{}{}{Estar dolorido. (\textit{Quando eu me levanto, minhas pernas doem.})}{do.er}{\verboinum{17}}
\verb{doesto}{ê}{}{}{}{s.m.}{Acusação injusta; insulto, afronta.}{do.es.to}{0}
\verb{doge}{ó}{}{}{}{s.m.}{Magistrado eleito das antigas repúblicas de Gênova e Veneza, cidades italianas.}{do.ge}{0}
\verb{dogesa}{ê}{}{}{}{s.f.}{Mulher do doge ou que exercia o cargo de doge.}{do.ge.sa}{0}
\verb{dogma}{ó}{}{}{}{s.m.}{Princípio de fé fundamental de uma doutrina religiosa que é aceito sem discussão.}{dog.ma}{0}
\verb{dogmático}{}{}{}{}{adj.}{Relativo a dogma ou a dogmatismo.}{dog.má.ti.co}{0}
\verb{dogmático}{}{}{}{}{}{De caráter indiscutível; definitivo, decisivo.}{dog.má.ti.co}{0}
\verb{dogmatismo}{}{}{}{}{s.m.}{Sistema em que não se aceita discussão sobre o que se afirma; autoritarismo.}{dog.ma.tis.mo}{0}
\verb{dogmatizar}{}{}{}{}{v.t.}{Proclamar uma afirmação como dogma; ensinar de modo autoritário.}{dog.ma.ti.zar}{\verboinum{1}}
\verb{doideira}{ê}{}{}{}{s.f.}{Doidice.}{doi.dei.ra}{0}
\verb{doidice}{}{}{}{}{s.f.}{Ato ou dito de doido; loucura, maluquice.}{doi.di.ce}{0}
\verb{doidivanas}{}{}{}{}{s.2g.}{Indivíduo imprudente, leviano.}{doi.di.va.nas}{0}
\verb{doido}{}{}{}{}{adj.}{Que perdeu o juízo; maluco, louco, insano.}{doi.do}{0}
\verb{doido}{}{Fig.}{}{}{}{Arrebatado, exaltado, entusiasmado. (\textit{Ele estava doido para ver aquele filme novamente.})}{doi.do}{0}
\verb{doído}{}{}{}{}{adj.}{Que dói; dorido, dolorido.}{do.í.do}{0}
\verb{doído}{}{}{}{}{}{Magoado, ofendido, sensibilizado.}{do.í.do}{0}
\verb{dois}{}{}{}{}{num.}{Nome dado à quantidade expressa pelo número 2.}{dois}{0}
\verb{dólar}{}{}{}{}{s.m.}{Unidade monetária e moeda dos Estados Unidos, da Austrália, do Canadá e de outros países.}{dó.lar}{0}
\verb{dolarização}{}{}{"-ões}{}{s.f.}{Substituição de moeda nacional pelo dólar dos Estados Unidos nas transações comerciais.}{do.la.ri.za.ção}{0}
\verb{dolarizar}{}{}{}{}{v.t.}{Praticar a dolarização.}{do.la.ri.zar}{\verboinum{1}}
\verb{doleiro}{ê}{}{}{}{s.m.}{Agente de câmbio que opera no chamado mercado paralelo comprando e vendendo dólares estadunidenses.}{do.lei.ro}{0}
\verb{dolência}{}{}{}{}{s.f.}{Qualidade ou estado de dolente; aflição, lástima.}{do.lên.cia}{0}
\verb{dolente}{}{}{}{}{adj.2g.}{Que revela sofrimento e tristeza; magoado, lastimoso.}{do.len.te}{0}
\verb{dólmã}{}{}{}{}{s.m.}{Casaco militar curto, ajustado à cintura.}{dól.mã}{0}
\verb{dólmen}{}{}{}{}{s.m.}{Monumento pré"-histórico formado por uma grande pedra chata colocada sobre duas outras verticais.}{dól.men}{0}
\verb{dolo}{ó}{}{}{}{s.m.}{Ato criminoso praticado com intenção de prejudicar; má"-fé.}{do.lo}{0}
\verb{dolorido}{}{}{}{}{adj.}{Que dói; doído.}{do.lo.ri.do}{0}
\verb{dolorido}{}{}{}{}{}{Lastimoso, triste, magoado.}{do.lo.ri.do}{0}
\verb{dolorosa}{ó}{}{}{}{s.f.}{A conta que se deve pagar, principalmente em restaurantes.}{do.lo.ro.sa}{0}
\verb{doloroso}{ô}{}{"-osos ⟨ó⟩}{"-osa ⟨ó⟩}{adj.}{Que provoca dor.}{do.lo.ro.so}{0}
\verb{doloroso}{ô}{}{"-osos ⟨ó⟩}{"-osa ⟨ó⟩}{}{Que causa desgosto; lastimoso.}{do.lo.ro.so}{0}
\verb{doloso}{ô}{}{"-osos ⟨ó⟩}{"-osa ⟨ó⟩}{adj.}{Em que há dolo.}{do.lo.so}{0}
\verb{doloso}{ô}{Jur.}{"-osos ⟨ó⟩}{"-osa ⟨ó⟩}{}{Diz"-se do crime praticado com deliberação.}{do.lo.so}{0}
\verb{dom}{}{}{}{}{s.m.}{Qualidade inata; dádiva, talento.}{dom}{0}
\verb{dom}{}{}{}{}{s.m.}{Forma de tratamento dada a reis, nobres e eclesiásticos.}{dom}{0}
\verb{domação}{}{}{"-ões}{}{s.f.}{Ato ou efeito de domar; domesticação.}{do.ma.ção}{0}
\verb{domador}{ô}{}{}{}{adj.}{Que amansa ou domestica.}{do.ma.dor}{0}
\verb{domar}{}{}{}{}{v.t.}{Domesticar animais selvagens; amansar.}{do.mar}{\verboinum{1}}
\verb{doméstica}{}{}{}{}{s.f.}{Mulher que presta serviços relativos a manutenção da casa; empregada, criada.}{do.més.ti.ca}{0}
\verb{domesticação}{}{}{"-ões}{}{s.f.}{Ato ou efeito de domesticar; domação, amansamento.}{do.mes.ti.ca.ção}{0}
\verb{domesticar}{}{}{}{}{v.t.}{Amansar animais selvagens para fazê"-los conviver com o ser humano; domar.}{do.mes.ti.car}{\verboinum{2}}
\verb{doméstico}{}{}{}{}{adj.}{Relativo a lar e a família; caseiro, familiar.}{do.més.ti.co}{0}
\verb{doméstico}{}{}{}{}{}{Relativo à vida interna de um país. (\textit{Essa companhia aérea só opera com voos domésticos.})}{do.més.ti.co}{0}
\verb{doméstico}{}{}{}{}{}{Diz"-se do animal que vive ou é criado dentro de casa.}{do.més.ti.co}{0}
\verb{doméstico}{}{}{}{}{s.m.}{Indivíduo contratado para fazer os serviços de manutenção de uma casa.}{do.més.ti.co}{0}
\verb{domiciliado}{}{}{}{}{adj.}{Que fixou residência; residente.}{do.mi.ci.li.a.do}{0}
\verb{domiciliar}{}{}{}{}{adj.2g.}{Relativo a domicílio; domiciliário.}{do.mi.ci.li.ar}{0}
\verb{domiciliar}{}{}{}{}{v.t.}{Acolher em domicílio.}{do.mi.ci.li.ar}{0}
\verb{domiciliar}{}{}{}{}{v.pron.}{Fixar residência; morar.}{do.mi.ci.li.ar}{\verboinum{1}}
\verb{domiciliário}{}{}{}{}{adj.}{Relativo a domicílio; domiciliar.}{do.mi.ci.li.á.rio}{0}
\verb{domicílio}{}{}{}{}{s.m.}{Lugar de residência; habitação fixa; casa.}{do.mi.cí.lio}{0}
\verb{dominação}{}{}{"-ões}{}{s.f.}{Ato ou efeito de dominar; subjugação, domínio.}{do.mi.na.ção}{0}
\verb{dominador}{ô}{}{}{}{adj.}{Que domina; que detém o poder; a autoridade.}{do.mi.na.dor}{0}
\verb{dominância}{}{}{}{}{s.f.}{Qualidade de dominante; predominância, influência.}{do.mi.nân.cia}{0}
\verb{dominante}{}{}{}{}{adj.2g.}{Que domina; preponderante, influente. (\textit{Para determinar a cor da pele, o gene dominante é o da cor negra.})}{do.mi.nan.te}{0}
\verb{dominar}{}{}{}{}{v.t.}{Exercer domínio; subjugar, sujeitar. (\textit{O império romano dominou grande parte do mundo conhecido daquela época.})}{do.mi.nar}{0}
\verb{dominar}{}{}{}{}{}{Conter, reprimir, refrear. (\textit{O vaqueiro conseguiu dominar o cavalo arredio.})}{do.mi.nar}{0}
\verb{dominar}{}{}{}{}{}{Sobressair, distinguir, prevalecer. (\textit{Sua voz poderosa dominava a multidão que o ouvia.})}{do.mi.nar}{\verboinum{1}}
\verb{domingo}{}{}{}{}{s.m.}{O primeiro dia da semana.}{do.min.go}{0}
\verb{domingueiro}{ê}{}{}{}{adj.}{Relativo a domingo.}{do.min.guei.ro}{0}
\verb{domingueiro}{ê}{}{}{}{}{Diz"-se da roupa que se veste aos domingos.}{do.min.guei.ro}{0}
\verb{dominical}{}{}{"-ais}{}{adj.2g.}{Relativo a domingo.}{do.mi.ni.cal}{0}
\verb{dominicano}{}{Relig.}{}{}{adj.}{Relativo à Ordem de São Domingos.}{do.mi.ni.ca.no}{0}
\verb{dominicano}{}{}{}{}{}{Relativo à República Dominicana ou a São Domingos, nas Antilhas.}{do.mi.ni.ca.no}{0}
\verb{dominicano}{}{Relig.}{}{}{s.m.}{Frade da Ordem de São Domingos.}{do.mi.ni.ca.no}{0}
\verb{dominicano}{}{}{}{}{}{Indivíduo natural ou habitante da República Dominicana.}{do.mi.ni.ca.no}{0}
\verb{domínio}{}{}{}{}{s.m.}{Poder que se tem sobre um indivíduo ou uma coisa; dominação, autoridade, preponderância.}{do.mí.nio}{0}
\verb{domínio}{}{}{}{}{}{Grande extensão de terras pertencente a um indivíduo; possessão.}{do.mí.nio}{0}
\verb{domínio}{}{}{}{}{}{Conjunto de assuntos relativos a uma arte ou ciência.}{do.mí.nio}{0}
\verb{dominó}{}{}{}{}{s.m.}{Jogo com 28 peças de forma retangular, com pontos que variam de um a seis marcados nelas, formando diversas combinações.}{do.mi.nó}{0}
\verb{dominó}{}{}{}{}{}{Veste comprida, com capuz e mangas, usada como fantasia de carnaval.}{do.mi.nó}{0}
\verb{domo}{ô}{}{}{}{s.m.}{Parte superior e externa de um edifício, de forma convexa ou esférica, e que, internamente, corresponde a uma cúpula; zimbório.}{do.mo}{0}
\verb{dona}{ô}{}{}{}{s.f.}{Mulher a quem pertence algo; proprietária, senhora.}{do.na}{0}
\verb{dona}{ô}{}{}{}{}{Tratamento concedido às mulheres casadas ou de condição social elevada.}{do.na}{0}
\verb{dona}{ô}{Bras.}{}{}{}{Mulher, esposa.}{do.na}{0}
\verb{dona"-de"-casa}{}{}{donas"-de"-casa}{}{s.f.}{Mulher que dirige ou administra a casa em que mora.}{do.na"-de"-ca.sa}{0}
\verb{donaire}{}{}{}{}{s.m.}{Graça no andar ou nos gestos; elegância, garbo.}{do.nai.re}{0}
\verb{donatário}{}{Hist.}{}{}{s.m.}{Dono de capitania hereditária.}{do.na.tá.rio}{0}
\verb{donativo}{}{}{}{}{s.m.}{Objeto de uma doação; oferta, dádiva.}{do.na.ti.vo}{0}
\verb{donde}{}{Pop.}{}{}{adv.}{Indica procedência, origem; de qual lugar; de que lugar.   }{don.de}{0}
\verb{donde}{}{}{}{}{}{Indica origem; causa; de quê.}{don.de}{0}
\verb{donde}{}{}{}{}{}{Indica conclusão; daí.}{don.de}{0}
\verb{dondoca}{ó}{}{}{}{s.f.}{Mulher de boa situação social, que não precisa fazer esforço para se sustentar; fútil.}{don.do.ca}{0}
\verb{doninha}{}{Zool.}{}{}{s.f.}{Pequeno animal carnívoro, de corpo longo e esguio e de pernas curtas, semelhante ao furão.}{do.ni.nha}{0}
\verb{dono}{ô}{}{}{}{s.m.}{Indivíduo a quem pertence algo; proprietário, senhor.}{do.no}{0}
\verb{donzela}{é}{}{}{}{s.f.}{Mulher jovem que ainda não teve relações sexuais.}{don.ze.la}{0}
\verb{dopado}{}{}{}{}{adj.}{Que está sob o efeito de drogas estimulantes ou entorpecentes; drogado.}{do.pa.do}{0}
\verb{dopagem}{}{}{}{}{s.f.}{Ato ou efeito de dopar animal ou ser humano, utilizando substâncias que provoquem alterações no sistema nervoso como estimulantes ou entorpecentes.}{do.pa.gem}{0}
\verb{dopar}{}{}{}{}{v.t.}{Injetar substância estimulante ou entorpecente; drogar.}{do.par}{\verboinum{1}}
\verb{doping}{}{Esport.}{}{}{s.m.}{Uso ilícito de drogas estimulantes para melhorar o desempenho de atletas ou animais de corrida.}{\textit{doping}}{0}
\verb{dor}{ô}{}{}{}{s.f.}{Sensação desagradável causada por ferimento ou lesão.}{dor}{0}
\verb{dor}{ô}{}{}{}{}{Tristeza, pesar, sofrimento.}{dor}{0}
\verb{doravante}{}{}{}{}{adv.}{De agora em diante; para o futuro.}{do.ra.van.te}{0}
\verb{dor"-de"-cotovelo}{ô\ldots{}ê}{Pop.}{dores"-de"-cotovelo ⟨ô\ldots{}ê⟩}{}{s.f.}{Ciúme ou despeito; sofrimento por decepção amorosa.}{dor"-de"-co.to.ve.lo}{0}
\verb{dorido}{}{}{}{}{adj.}{Que sente dor; dolorido.}{do.ri.do}{0}
\verb{dorido}{}{}{}{}{}{Magoado, triste, consternado.}{do.ri.do}{0}
\verb{dormência}{}{}{}{}{s.f.}{Estado de quem dorme ou está entorpecido; entorpecimento.}{dor.mên.cia}{0}
\verb{dormente}{}{}{}{}{adj.2g.}{Que perdeu a sensibilidade; entorpecido.}{dor.men.te}{0}
\verb{dormente}{}{}{}{}{s.m.}{Cada uma das vigas colocadas transversalmente à via férrea e sobre as quais se estendem os trilhos.}{dor.men.te}{0}
\verb{dormida}{}{}{}{}{s.f.}{Ato ou efeito de dormir; sono.}{dor.mi.da}{0}
\verb{dormida}{}{}{}{}{}{Pousada para passar a noite.}{dor.mi.da}{0}
\verb{dormideira}{ê}{Bot.}{}{}{s.f.}{Planta da família das leguminosas, cujas folhas têm a propriedade de se retraírem quando tocadas; sensitiva.  }{dor.mi.dei.ra}{0}
\verb{dorminhoco}{ô}{}{"-s ⟨ó⟩}{"-a ⟨ó⟩}{adj.}{Que é muito sonolento ou dorme muito.}{dor.mi.nho.co}{0}
\verb{dormir}{}{}{}{}{v.i.}{Estar em repouso; descansar no sono.}{dor.mir}{0}
\verb{dormir}{}{}{}{}{}{Mover"-se com tanta rapidez que parece estar parado.}{dor.mir}{\verboinum{31}}
\verb{dormitar}{}{}{}{}{v.i.}{Dormir levemente; cochilar.}{dor.mi.tar}{\verboinum{1}}
\verb{dormitório}{}{}{}{}{s.m.}{Aposento onde dormem muitas pessoas.}{dor.mi.tó.rio}{0}
\verb{dormitório}{}{}{}{}{}{Quarto de dormir.}{dor.mi.tó.rio}{0}
\verb{dormitório}{}{}{}{}{}{Mobília usada para esse quarto.}{dor.mi.tó.rio}{0}
\verb{dorna}{ó}{}{}{}{s.f.}{Tina sem tampa e com boca larga, destinada a pisar uvas.}{dor.na}{0}
\verb{dorsal}{}{}{"-ais}{}{adj.2g.}{Relativo a dorso.}{dor.sal}{0}
\verb{dorso}{ô}{Anat.}{}{}{s.m.}{Face posterior de qualquer parte do corpo.}{dor.so}{0}
\verb{dorso}{ô}{Anat.}{}{}{}{A parte de trás do tronco do ser humano e dos animais; costas.}{dor.so}{0}
\verb{dorso}{ô}{}{}{}{}{Parte de trás de alguma coisa; verso, lombada.}{dor.so}{0}
\verb{DOS}{ó}{Informát.}{}{}{s.m.}{Sistema operacional em disco para microcomputadores.}{DOS}{0}
\verb{dosagem}{}{}{}{}{s.f.}{Ato ou efeito de dosar; dose.}{do.sa.gem}{0}
\verb{dosar}{}{}{}{}{v.t.}{Determinar a proporção certa.}{do.sar}{\verboinum{1}}
\verb{dose}{ó}{}{}{}{s.f.}{Quantidade determinada de uma substância; porção.}{do.se}{0}
\verb{dossel}{é}{}{"-éis}{}{s.m.}{Cobertura forrada e franjada que se põe sobre tronos, altares e camas.}{dos.sel}{0}
\verb{dossiê}{}{}{}{}{s.m.}{Conjunto de documentos relativos a um processo, que servem para se provar algo.}{dos.si.ê}{0}
\verb{dotação}{}{}{"-ões}{}{s.f.}{Ato ou efeito de dotar.}{do.ta.ção}{0}
\verb{dotação}{}{}{"-ões}{}{}{Quantia destinada à manutenção de um indivíduo ou uma instituição; renda, verba.}{do.ta.ção}{0}
\verb{dotado}{}{}{}{}{adj.}{Que recebeu dote.}{do.ta.do}{0}
\verb{dotado}{}{}{}{}{}{Que possui dote natural; talentoso.}{do.ta.do}{0}
\verb{dotar}{}{}{}{}{v.t.}{Conceder dote.}{do.tar}{0}
\verb{dotar}{}{}{}{}{}{Favorecer com um dom natural; beneficiar.}{do.tar}{0}
\verb{dotar}{}{}{}{}{}{Destinar dotação para uma entidade.}{do.tar}{\verboinum{1}}
\verb{dote}{ó}{}{}{}{s.m.}{Conjunto de riquezas que uma mulher levava para casar ou entrar num convento.}{do.te}{0}
\verb{dote}{ó}{}{}{}{}{Qualidade que se tem desde o nascimento; dom, talento.}{do.te}{0}
\verb{doudeira}{ê}{}{}{}{}{Var. de \textit{doideira}.}{dou.dei.ra}{0}
\verb{doudice}{}{}{}{}{}{Var. de \textit{doidice}.}{dou.di.ce}{0}
\verb{doudivanas}{}{}{}{}{}{Var. de \textit{doidivanas}.}{dou.di.va.nas}{0}
\verb{doudo}{ô}{}{}{}{}{Var. de \textit{doido}.}{dou.do}{0}
\verb{dourado}{}{}{}{}{adj.}{Coberto de ouro.}{dou.ra.do}{0}
\verb{dourado}{}{}{}{}{}{Da cor do ouro; amarelo.}{dou.ra.do}{0}
\verb{dourador}{ô}{}{}{}{s.m.}{Operário ou artista que se dedica à douradura.}{dou.ra.dor}{0}
\verb{douradura}{}{}{}{}{s.f.}{Ato ou efeito de dourar, de revestir com camada ou folha de ouro.}{dou.ra.du.ra}{0}
\verb{dourar}{}{}{}{}{v.t.}{Cobrir com uma camada ou folha de ouro.}{dou.rar}{0}
\verb{dourar}{}{}{}{}{}{Dar a cor do ouro a alguma coisa; amarelar.}{dou.rar}{\verboinum{1}}
\verb{douto}{ô}{}{}{}{adj.}{Diz"-se de indivíduo que é muito culto.}{dou.to}{0}
\verb{doutor}{ô}{}{}{}{s.m.}{Indivíduo que recebeu o maior título em um curso do ensino superior.}{dou.tor}{0}
\verb{doutor}{ô}{}{}{}{}{Indivíduo que se formou em medicina; médico.}{dou.tor}{0}
\verb{doutor}{ô}{Pop.}{}{}{}{Homem importante.}{dou.tor}{0}
\verb{doutorado}{}{}{}{}{s.m.}{Graduação de doutor.}{dou.to.ra.do}{0}
\verb{doutorado}{}{}{}{}{}{Curso de pós"-graduação após o mestrado.}{dou.to.ra.do}{0}
\verb{doutoral}{}{}{"-ais}{}{adj.2g.}{Relativo a doutor.}{dou.to.ral}{0}
\verb{doutoral}{}{Pejor.}{"-ais}{}{}{Pedante, pretensioso.}{dou.to.ral}{0}
\verb{doutoramento}{}{}{}{}{s.m.}{Conclusão do curso de doutorado.}{dou.to.ra.men.to}{0}
\verb{doutorando}{}{}{}{}{s.m.}{Indivíduo que estuda em curso de doutorado.}{dou.to.ran.do}{0}
\verb{doutorar}{}{}{}{}{v.t.}{Dar o título de doutor a quem terminou o curso.}{dou.to.rar}{\verboinum{1}}
\verb{doutrina}{}{}{}{}{s.f.}{Conjunto de ideias que formam um determinado sistema de conhecimentos religiosos, filosóficos, políticos ou científicos.}{dou.tri.na}{0}
\verb{doutrinação}{}{}{"-ões}{}{s.f.}{Ato ou efeito de doutrinar; catequese, ensinamento.}{dou.tri.na.ção}{0}
\verb{doutrinal}{}{}{"-ais}{}{adj.2g.}{Doutrinário.}{dou.tri.nal}{0}
\verb{doutrinal}{}{}{"-ais}{}{s.m.}{Catecismo.}{dou.tri.nal}{0}
\verb{doutrinar}{}{}{}{}{v.t.}{Ensinar uma doutrina a alguém.}{dou.tri.nar}{\verboinum{1}}
\verb{doutrinário}{}{}{}{}{adj.}{Relativo a ou que encerra doutrina; doutrinal.}{dou.tri.ná.rio}{0}
\verb{download}{}{Informát.}{}{}{s.m.}{Operação realizada para se copiar um arquivo da Internet para o computador.}{\textit{download}}{0}
\verb{doze}{ô}{}{}{}{num.}{Nome dado à quantidade expressa pelo número 12.}{do.ze}{0}
\verb{dracma}{}{}{}{}{s.f.}{Padrão monetário da Grécia.}{drac.ma}{0}
\verb{draconiano}{}{}{}{}{adj.}{Que é muito rigoroso, severo.}{dra.co.ni.a.no}{0}
\verb{draga}{}{}{}{}{s.f.}{Máquina que serve para tirar a areia ou a lama do fundo de rios, lagos e mares.}{dra.ga}{0}
\verb{dragagem}{}{}{}{}{s.f.}{Ato ou efeito de dragar; limpeza, desobstrução.}{dra.ga.gem}{0}
\verb{dragão}{}{}{"-ões}{}{s.m.}{Monstro imaginário que cospe fogo, representado com uma cauda comprida, garras e asas.}{dra.gão}{0}
\verb{dragar}{}{}{}{}{v.t.}{Usar a draga para limpar rios ou lagos.}{dra.gar}{\verboinum{5}}
\verb{drágea}{}{Farm.}{}{}{s.f.}{Pastilha medicamentosa revestida de uma substância endurecida, geralmente adocicada; pílula.}{drá.gea}{0}
\verb{dragona}{ô}{}{}{}{s.f.}{Adorno com franjas nos ombros de uniforme militar. }{dra.go.na}{0}
\verb{drama}{}{}{}{}{s.m.}{Qualquer peça ou composição teatral.}{dra.ma}{0}
\verb{drama}{}{}{}{}{}{Peça teatral em que se misturam o trágico e o cômico.}{dra.ma}{0}
\verb{drama}{}{}{}{}{}{Acontecimento que causa muita tristeza.}{dra.ma}{0}
\verb{dramalhão}{}{}{"-ões}{}{s.m.}{Peça ou filme de pouco valor artístico, cheio de lances trágicos e artificiosos ou que expõe atos de perversidade requintada.}{dra.ma.lhão}{0}
\verb{dramático}{}{}{}{}{adj.}{Relativo a drama.}{dra.má.ti.co}{0}
\verb{dramático}{}{}{}{}{}{Que causa forte emoção; patético, comovente.}{dra.má.ti.co}{0}
\verb{dramatização}{}{}{"-ões}{}{s.f.}{Transformação em drama ou em peça teatral; encenação. }{dra.ma.ti.za.ção}{0}
\verb{dramatizar}{}{}{}{}{v.t.}{Dar forma de drama a um assunto.}{dra.ma.ti.zar}{\verboinum{1}}
\verb{dramatizar}{}{}{}{}{}{Tornar dramático, comovente.}{dra.ma.ti.zar}{0}
\verb{dramaturgia}{}{}{}{}{s.f.}{Arte dramática ou de compor peças teatrais.}{dra.ma.tur.gi.a}{0}
\verb{dramaturgo}{}{}{}{}{s.m.}{Autor de peças teatrais.}{dra.ma.tur.go}{0}
\verb{drapear}{}{}{}{}{v.i.}{Movimentar"-se com ondulações, quando batido pelo vento.}{dra.pe.ar}{0}
\verb{drapear}{}{}{}{}{v.t.}{Fazer pregas onduladas em tecido.}{dra.pe.ar}{\verboinum{4}}
\verb{drapejar}{}{}{}{}{}{Var. de \textit{drapear}.}{dra.pe.jar}{\verboinum{1}}
\verb{drástico}{}{}{}{}{adj.}{Que age ou funciona com energia; enérgico, radical.}{drás.ti.co}{0}
\verb{drenagem}{}{}{}{}{s.f.}{Ato ou efeito de drenar; escoamento.}{dre.na.gem}{0}
\verb{drenar}{}{}{}{}{v.t.}{Fazer canais para retirar o excesso de água de um lugar.}{dre.nar}{\verboinum{1}}
\verb{dreno}{ê}{Med.}{}{}{s.m.}{Tubo para permitir que um líquido saia de uma cavidade do corpo para o exterior.}{dre.no}{0}
\verb{driblar}{}{}{}{}{v.t.}{Enganar o adversário com a bola no jogo de futebol, movendo o corpo; fintar.}{dri.blar}{\verboinum{1}}
\verb{drible}{}{}{}{}{s.m.}{Ato ou efeito de driblar, de gingar o corpo, controlando a bola com o pé; finta.}{dri.ble}{0}
\verb{drinque}{}{}{}{}{s.m.}{Bebida alcoólica; aperitivo.}{drin.que}{0}
\verb{drive}{}{Informát.}{}{}{s.m.}{Dispositivo eletroeletrônico composto de placa controladora, motor e cabeças de leitura e gravação, usado para ler ou gravar dados sobre algum meio magnético.}{\textit{drive}}{0}
\verb{drive"-in}{}{}{}{}{s.m.}{Qualquer estabelecimento onde cada freguês entra e é servido dentro do próprio automóvel.}{\textit{drive"-in}}{0}
\verb{driver}{}{Informát.}{}{}{s.m.}{Arquivo de computador que controla as operações de entrada e saída de dados.}{\textit{driver}}{0}
\verb{droga}{ó}{}{}{}{s.f.}{Substância empregada como ingrediente em farmácia, indústria química, tinturaria etc. }{dro.ga}{0}
\verb{droga}{ó}{}{}{}{}{Substância alucinógena.}{dro.ga}{0}
\verb{droga}{ó}{Pop.}{}{}{}{Coisa ruim, sem valor.}{dro.ga}{0}
\verb{droga}{ó}{}{}{}{interj.}{Expressão que indica frustração ou desânimo.}{dro.ga}{0}
\verb{drogado}{}{}{}{}{adj.}{Diz"-se de indivíduo que usa ou está sob efeito de drogas.}{dro.ga.do}{0}
\verb{drogar}{}{}{}{}{v.t.}{Ministrar medicamento.}{dro.gar}{0}
\verb{drogar}{}{}{}{}{v.pron.}{Fazer uso de drogas.}{dro.gar}{\verboinum{5}}
\verb{drogaria}{}{}{}{}{s.f.}{Estabelecimento onde se vendem medicamentos; farmácia.}{dro.ga.ri.a}{0}
\verb{droguista}{}{}{}{}{adj.2g.}{Diz"-se de vendedor de drogas.}{dro.guis.ta}{0}
\verb{droguista}{}{}{}{}{}{Diz"-se de proprietário de drogaria.}{dro.guis.ta}{0}
\verb{dromedário}{}{Zool.}{}{}{s.m.}{Espécie de camelo que tem uma só corcova. }{dro.me.dá.rio}{0}
\verb{drope}{ó}{}{}{}{s.m.}{Espécie de bala ou caramelo em forma de pequeno disco.}{dro.pe}{0}
\verb{dropes}{}{}{}{}{s.m.pl.}{Drope.}{dro.pes}{0}
\verb{druida}{}{}{}{}{s.m.}{Antigo sacerdote celta.}{dru.i.da}{0}
\verb{druídico}{}{}{}{}{adj.}{Relativo a druida.}{dru.í.di.co}{0}
\verb{drupa}{}{}{}{}{s.f.}{Fruto carnoso provido de um núcleo muito duro, como o pêssego e a manga.}{dru.pa}{0}
\verb{dual}{}{}{"-ais}{}{adj.2g.}{Relativo a dois.}{du.al}{0}
\verb{dualidade}{}{}{}{}{s.f.}{Caráter daquilo que é dual ou duplo.}{du.a.li.da.de}{0}
\verb{dualismo}{}{}{}{}{s.m.}{Doutrina que admite a coexistência de dois princípios opostos.}{du.a.lis.mo}{0}
\verb{duas}{}{}{}{}{num.}{Feminino de \textit{dois}.}{du.as}{0}
\verb{dubiedade}{}{}{}{}{s.f.}{Qualidade do que é dúbio; ambiguidade.}{du.bi.e.da.de}{0}
\verb{dúbio}{}{}{}{}{adj.}{Que pode ser entendido de mais de uma maneira; ambíguo, incerto, duvidoso.}{dú.bio}{0}
\verb{dubitativo}{}{}{}{}{s.f.}{Que expressa dúvida; cético, incrédulo.}{du.bi.ta.ti.vo}{0}
\verb{dubitável}{}{}{"-eis}{}{adj.2g.}{Que se pode pôr em dúvida, sujeito a desconfiança; ambíguo, incerto.}{du.bi.tá.vel}{0}
\verb{dublagem}{}{}{"-ens}{}{s.f.}{Substituição da língua original por outra na parte falada ou cantada de um filme.}{du.bla.gem}{0}
\verb{dublar}{}{}{}{}{v.t.}{Fazer a dublagem de.}{du.blar}{\verboinum{1}}
\verb{dublê}{}{}{}{}{s.m.}{Indivíduo que substitui um ator, especialmente nas cenas perigosas.}{du.blê}{0}
\verb{dúbnio}{}{Quím.}{}{}{s.m.}{Elemento químico sintético da família dos actinídeos, de número atômico 105. \elemento{105}{(262)}{Db}.}{dúb.nio}{0}
\verb{ducado}{}{}{}{}{s.m.}{Território de um duque.}{du.ca.do}{0}
\verb{ducado}{}{}{}{}{}{Título de duque.}{du.ca.do}{0}
\verb{ducado}{}{}{}{}{}{Moeda de ouro ou prata, de diferentes valores, países e épocas.}{du.ca.do}{0}
\verb{ducal}{}{}{"-ais}{}{adj.2g.}{Relativo a duque.}{du.cal}{0}
\verb{ducentésimo}{}{}{}{}{num.}{Ordinal e fracionário correspondente a 200.}{du.cen.té.si.mo}{0}
\verb{ducha}{}{}{}{}{s.f.}{Jorro de água que se dirige sobre uma pessoa, para fins terapêuticos ou higiênicos. }{du.cha}{0}
\verb{ducha}{}{}{}{}{}{Banho que se toma nesse chuveiro.}{du.cha}{0}
\verb{dúctil}{}{}{"-eis}{}{adj.2g.}{Que pode ser reduzido a fios muito finos, sem romper"-se.}{dúc.til}{0}
\verb{dúctil}{}{}{"-eis}{}{}{Flexível, maleável, elástico.}{dúc.til}{0}
\verb{dúctil}{}{}{"-eis}{}{}{Que é transigente; dócil.}{dúc.til}{0}
\verb{ductilidade}{}{}{}{}{s.f.}{Qualidade ou propriedade do que é dúctil, flexível; maleabilidade, flexibilidade. }{duc.ti.li.da.de}{0}
\verb{ducto}{}{}{}{}{s.m.}{Qualquer tubo condutor num organismo animal.}{duc.to}{0}
\verb{ducto}{}{}{}{}{}{Qualquer conduto ou tubulação destinada a conduzir fluidos a grandes distâncias.}{duc.to}{0}
\verb{duelar}{}{}{}{}{adj.2g.}{Relativo a duelo.}{du.e.lar}{0}
\verb{duelar}{}{}{}{}{v.t.}{Bater"-se em duelo.}{du.e.lar}{\verboinum{1}}
\verb{duelista}{}{}{}{}{adj.2g.}{Diz"-se de indivíduo que se bate em duelo.}{du.e.lis.ta}{0}
\verb{duelo}{é}{}{}{}{s.m.}{Combate entre duas pessoas, sob certas regras, com armas mortíferas.}{du.e.lo}{0}
\verb{duende}{}{}{}{}{s.m.}{Entidade das lendas europeias, de aspecto humano, orelhas pontudas e pequenina estatura, que geralmente usa seus poderes em travessuras noturnas para assustar os moradores das casas.}{du.en.de}{0}
\verb{dueto}{ê}{}{}{}{s.m.}{Música para duas vozes ou instrumentos.}{du.e.to}{0}
\verb{dueto}{ê}{}{}{}{}{Execução musical de dois solistas.}{du.e.to}{0}
\verb{dueto}{ê}{}{}{}{}{Coreografia para dois bailarinos.}{du.e.to}{0}
\verb{dulcificação}{}{}{"-ões}{}{s.f.}{Ato ou efeito de dulcificar; adoçamento.}{dul.ci.fi.ca.ção}{0}
\verb{dulcificante}{}{}{}{}{adj.2g.}{Que dulcifica; adoçante.}{dul.ci.fi.can.te}{0}
\verb{dulcificar}{}{}{}{}{v.t.}{Tornar doce.}{dul.ci.fi.car}{0}
\verb{dulcificar}{}{}{}{}{}{Suavizar, abrandar.}{dul.ci.fi.car}{\verboinum{2}}
\verb{dulçor}{ô}{}{}{}{s.m.}{Doçura.}{dul.çor}{0}
\verb{dulçoroso}{ô}{}{"-osos ⟨ó⟩}{"-osa ⟨ó⟩}{adj.}{Que tem doçura.}{dul.ço.ro.so}{0}
\verb{dum}{}{}{}{}{}{Contração da preposição \textit{de} com o artigo \textit{um}.}{dum}{0}
\verb{duma}{}{}{}{}{}{Feminino de \textit{dum}.}{du.ma}{0}
\verb{duma}{}{}{}{}{s.f.}{Assembleia dos representantes do povo na Rússia czarista.}{du.ma}{0}
\verb{dumping}{}{}{}{}{s.m.}{Venda de produtos a preços abaixo do mercado, especialmente na exportação.}{\textit{dumping}}{0}
\verb{duna}{}{}{}{}{s.f.}{Montanha de areia formada pelo vento.}{du.na}{0}
\verb{dundum}{}{}{"-uns}{}{s.f.}{Bala de cápsula modificada de maneira que produza ferimentos sempre muito graves.}{dun.dum}{0}
\verb{duo}{}{}{}{}{s.m.}{Dueto.}{du.o}{0}
\verb{duodecimal}{}{}{"-ais}{}{adj.2g.}{Que se conta por séries de 12.}{du.o.de.ci.mal}{0}
\verb{duodecimal}{}{}{"-ais}{}{}{Que tem por base o número 12.}{du.o.de.ci.mal}{0}
\verb{duodécimo}{}{}{}{}{num.}{Décimo segundo.}{du.o.dé.ci.mo}{0}
\verb{duodécimo}{}{}{}{}{s.m.}{Cada uma das 12 partes em que se pode dividir um todo.}{du.o.dé.ci.mo}{0}
\verb{duodeno}{}{Anat.}{}{}{s.m.}{A primeira parte do intestino, ligada com o estômago.}{du.o.de.no}{0}
\verb{dupla}{}{}{}{}{s.f.}{Grupo de duas pessoas ou coisas.}{du.pla}{0}
\verb{duplex}{écs}{}{}{}{num.}{Duplo, dúplice.}{du.plex}{0}
\verb{duplex}{écs}{}{}{}{s.m.}{Apartamento que ocupa dois andares; dúplex.}{du.plex}{0}
\verb{dúplex}{cs}{}{}{}{num. e s.m.}{Duplex.}{dú.plex}{0}
\verb{duplicação}{}{}{"-ões}{}{s.f.}{Ato ou efeito de duplicar; dobro, repetição.}{du.pli.ca.ção}{0}
\verb{duplicar}{}{}{}{}{v.t.}{Multiplicar por dois.}{du.pli.car}{0}
\verb{duplicar}{}{}{}{}{}{Dobrar de tamanho, quantidade.}{du.pli.car}{\verboinum{2}}
\verb{duplicata}{}{}{}{}{s.f.}{Reprodução, cópia.}{du.pli.ca.ta}{0}
\verb{duplicata}{}{}{}{}{}{Título de crédito nominativo que obriga o devedor a pagar, no prazo marcado, o valor da fatura.}{du.pli.ca.ta}{0}
\verb{dúplice}{}{}{}{}{num.}{Multiplicado por dois.}{dú.pli.ce}{0}
\verb{dúplice}{}{}{}{}{adj.2g.}{Falso, dissimulado.}{dú.pli.ce}{0}
\verb{dúplice}{}{}{}{}{}{Com duas camadas.}{dú.pli.ce}{0}
\verb{duplicidade}{}{}{}{}{s.f.}{Estado ou qualidade do que se apresenta com duas características, duas funções correlacionadas.}{du.pli.ci.da.de}{0}
\verb{duplicidade}{}{}{}{}{}{Característica de quem se apresenta de maneira diferente do que realmente é; dissimulação, fingimento.}{du.pli.ci.da.de}{0}
\verb{duplo}{}{}{}{}{num.}{Que equivale a duas vezes o outro; dobrado.}{du.plo}{0}
\verb{duplo}{}{}{}{}{s.m.}{Quantidade que equivale a duas vezes a outra; dobro.}{du.plo}{0}
\verb{duplo}{}{}{}{}{}{Pessoa ou coisa muito semelhante a outra, como se fosse uma réplica dessa outra.}{du.plo}{0}
\verb{duque}{}{}{}{}{s.m.}{Indivíduo que tem o título de nobreza entre o de marquês e o de arquiduque.}{du.que}{0}
\verb{duquesa}{ê}{}{}{}{s.f.}{Mulher que tem o título de nobreza entre o de marquesa e o de arquiduquesa.}{du.que.sa}{0}
\verb{dura}{}{}{}{}{s.f.}{Duração.}{du.ra}{0}
\verb{durabilidade}{}{}{}{}{s.f.}{Qualidade do que é durável; duração, resistência.}{du.ra.bi.li.da.de}{0}
\verb{duração}{}{}{"-ões}{}{s.f.}{Tempo que alguma coisa dura.}{du.ra.ção}{0}
\verb{duradoiro}{ô}{}{}{}{}{Var. de \textit{duradouro}.}{du.ra.doi.ro}{0}
\verb{duradouro}{ô}{}{}{}{adj.}{Que dura muito.}{du.ra.dou.ro}{0}
\verb{duralumínio}{}{Quím.}{}{}{s.m.}{Liga de alumínio com magnésio, manganês e cobre, que se caracteriza por ser torneada e resistir a ácidos diluídos e água salgada.}{du.ra.lu.mí.nio}{0}
\verb{dura"-máter}{}{Anat.}{duras"-máteres}{}{s.f.}{A membrana mais externa do cérebro e da medula.}{du.ra"-má.ter}{0}
\verb{durame}{}{}{}{}{s.m.}{Parte do lenho das árvores formada de células mortas e sem substâncias nutritivas de reserva; fica no centro do tronco e é quase sempre mais escura.}{du.ra.me}{0}
\verb{durâmen}{}{}{}{}{}{Var. de \textit{durame}.}{du.râ.men}{0}
\verb{durante}{}{}{}{}{prep.}{Enquanto alguma coisa acontece.}{du.ran.te}{0}
\verb{durão}{}{}{"-ões}{}{adj.}{Bem duro.}{du.rão}{0}
\verb{durão}{}{Pop.}{"-ões}{}{}{Diz"-se de indivíduo que possui grande resistência física, moral etc.; forte, firme, exigente, valente.}{du.rão}{0}
\verb{durar}{}{}{}{}{v.i.}{Existir por determinado tempo.}{du.rar}{\verboinum{1}}
\verb{durável}{}{}{"-eis}{}{adj.2g.}{Que dura muito tempo; duradouro.}{du.rá.vel}{0}
\verb{durex}{écs}{}{}{}{s.m.}{Nome comercial de uma fita adesiva.}{du.rex}{0}
\verb{dureza}{ê}{}{}{}{s.f.}{Qualidade daquilo que é duro; rigidez.}{du.re.za}{0}
\verb{dureza}{ê}{Pop.}{}{}{}{Falta de dinheiro.}{du.re.za}{0}
\verb{duro}{}{}{}{}{adj.}{Que é difícil de riscar ou de furar; resistente, rijo.}{du.ro}{0}
\verb{duro}{}{}{}{}{}{Em que existe muita dificuldade; árduo, custoso, trabalhoso.}{du.ro}{0}
\verb{duro}{}{}{}{}{}{Difícil de comover; insensível.}{du.ro}{0}
\verb{dúvida}{}{}{}{}{s.f.}{Ato ou efeito de duvidar, de sentir desconfiança, hesitação, incerteza.}{dú.vi.da}{0}
\verb{duvidar}{}{}{}{}{v.t.}{Pensar que existe a possibilidade de ser enganado; desconfiar, suspeitar.}{du.vi.dar}{0}
\verb{duvidar}{}{}{}{}{}{Não aceitar uma afirmação; descrer, não acreditar.}{du.vi.dar}{\verboinum{1}}
\verb{duvidoso}{ô}{}{"-osos ⟨ó⟩}{"-osa ⟨ó⟩}{adj.}{Em que há dúvida; incerto. }{du.vi.do.so}{0}
\verb{duvidoso}{ô}{}{"-osos ⟨ó⟩}{"-osa ⟨ó⟩}{}{De que se deve duvidar; arriscado, inseguro, perigoso.}{du.vi.do.so}{0}
\verb{duzentos}{}{}{}{}{num.}{Nome dado à quantidade expressa pelo número 200.}{du.zen.tos}{0}
\verb{dúzia}{}{}{}{}{s.f.}{Conjunto de 12 coisas da mesma espécie.}{dú.zia}{0}
\verb{dvd}{}{Informát.}{}{}{s.m.}{Sistema de gravação digital de som e imagem em meio óptico de grande densidade.}{dvd}{0}
\verb{dvd}{}{}{}{}{}{Disco utilizado nesse sistema de gravação.}{dvd}{0}
\verb{dvd}{}{}{}{}{}{Aparelho doméstico utilizado na reprodução de programas ou filmes gravados nesse sistema.}{dvd}{0}
\verb{Dy}{}{Quím.}{}{}{}{Símb. do \textit{disprósio}. }{Dy}{0}
\verb{dzeta}{ê}{}{}{}{s.m.}{Sexta letra do alfabeto grego.}{dze.ta}{0}
