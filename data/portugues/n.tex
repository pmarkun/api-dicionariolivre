\verb{n}{}{}{}{}{s.m.}{Décima quarta letra do alfabeto português.}{n}{0}
\verb{N}{}{}{}{}{}{Com ponto, abrev. de \textit{norte}.}{N}{0}
\verb{N}{}{Quím.}{}{}{}{Símb. do \textit{nitrogênio}.}{N}{0}
\verb{Na}{}{Quím.}{}{}{}{Símb. do \textit{sódio}.}{Na}{0}
\verb{nababesco}{ê}{}{}{}{s.m.}{Próprio de nababo.}{na.ba.bes.co}{0}
\verb{nababesco}{ê}{}{}{}{}{Luxuoso, suntuoso.}{na.ba.bes.co}{0}
\verb{nababo}{}{}{}{}{s.m.}{Pessoa que vive com grande luxo e riqueza. (\textit{Ele tem uma vida de nababo.})}{na.ba.bo}{0}
\verb{nababo}{}{Desus.}{}{}{}{Título que se dava ao príncipe ou governador de província, na Índia.}{na.ba.bo}{0}
\verb{nabiça}{}{Bot.}{}{}{s.f.}{Ramo do ramo que ainda não atingiu completo desenvolvimento.}{na.bi.ça}{0}
\verb{nabiça}{}{}{}{}{}{Nabo pequeno.}{na.bi.ça}{0}
\verb{nabo}{}{}{}{}{s.m.}{Verdura da mesma família da couve, de raiz grossa, comprida ou redonda. (\textit{Ele não gosta de comer nabos.})}{na.bo}{0}
\verb{nabo}{}{}{}{}{}{Raiz dessa planta.}{na.bo}{0}
\verb{nacada}{}{}{}{}{s.f.}{Pedaço grande arrancado de alguma coisa. (\textit{Ele tirou uma nacada da minha maçã.})}{na.ca.da}{0}
\verb{nação}{}{}{"-ões}{}{s.f.}{Conjunto organizado de pessoas que moram no mesmo território e têm os mesmos costumes e tradições. (\textit{O Brasil é uma das nações da América Latina.})}{na.ção}{0}
\verb{nácar}{}{}{}{}{s.m.}{Substância branca e brilhante de que se revestem interiormente algumas conchas. (\textit{É liso e brilhante como o nácar das conchas.})}{ná.car}{0}
\verb{nácar}{}{}{}{}{}{Cor"-de"-rosa.}{ná.car}{0}
\verb{nacarado}{}{}{}{}{adj.}{Que tem a cor, o aspecto, o brilho do nácar.}{na.ca.ra.do}{0}
\verb{nacarado}{}{}{}{}{}{Carminado, rosado. (\textit{É um abajur meio nacarado, bonito para quarto de menina.})}{na.ca.ra.do}{0}
\verb{nacarar}{}{}{}{}{v.t.}{Dar aspecto de nácar; cobrir com nácar. (\textit{A peça ficou bonita de pois que a nacaramos.})}{na.ca.rar}{0}
\verb{nacarar}{}{}{}{}{}{Tornar rosado; ruborizar. (\textit{Ficou tão envergonhado, que até seu rosto nacarou"-se.})}{na.ca.rar}{\verboinum{1}}
\verb{nacela}{é}{}{}{}{s.f.}{Cabine de avião.}{na.ce.la}{0}
\verb{nacional}{}{}{"-ais}{}{adj.2g.}{Que se refere ou pertence a uma nação. (\textit{O produto nacional é tão bom como qualquer outro.})}{na.ci.o.nal}{0}
\verb{nacional}{}{}{"-ais}{}{}{Que é próprio de uma nação. (\textit{No Brasil, o carnaval é uma festa nacional.})}{na.ci.o.nal}{0}
\verb{nacional}{}{}{"-ais}{}{s.m.}{Indivíduo natural de um país. (\textit{Os nacionais dos Estados Unidos são chamados de norte"-americanos.})}{na.ci.o.nal}{0}
\verb{nacionalidade}{}{}{}{}{s.f.}{Qualidade de nacional.}{na.ci.o.na.li.da.de}{0}
\verb{nacionalidade}{}{}{}{}{}{Origem nacional de uma pessoa ou coisa; naturalidade. (\textit{Ele tem uma nacionalidade diferente da minha; acho que é espanhol.})}{na.ci.o.na.li.da.de}{0}
\verb{nacionalidade}{}{}{}{}{}{Conjunto de caracteres próprios de uma nação. (\textit{O carnaval e o futebol fazem parte da nossa nacionalidade.})}{na.ci.o.na.li.da.de}{0}
\verb{nacionalismo}{}{}{}{}{s.m.}{Política segundo a qual se devem nacionalizar todas as atividades de um país. (\textit{O nacionalismo deve permanecer nas atividades relativas à saúde, à segurança e à educação.})}{na.ci.o.na.lis.mo}{0}
\verb{nacionalismo}{}{}{}{}{}{Preferência por tudo o que é próprio da nação a que se pertence. (\textit{Não se deve exagerar no nacionalismo, pois muita coisa feita no exterior também é boa.})}{na.ci.o.na.lis.mo}{0}
\verb{nacionalismo}{}{}{}{}{}{Aspiração de um povo que reivindica o direito de formar uma nação independente. (\textit{Sem o nacionalismo do povo, não havia como fazer a independência.})}{na.ci.o.na.lis.mo}{0}
\verb{nacionalista}{}{}{}{}{adj.2g.}{Relativo à independência e aos interesses nacionais. (\textit{Os movimentos nacionalistas do século \textsc{xviii}) culminaram na independência do Brasil.}}{na.ci.o.na.lis.ta}{0}
\verb{nacionalista}{}{}{}{}{}{Patriótico. (\textit{Cantar o hino antes dos jogos internacionais é um comportamento nacionalista.})}{na.ci.o.na.lis.ta}{0}
\verb{nacionalista}{}{}{}{}{}{Que pratica o nacionalismo.}{na.ci.o.na.lis.ta}{0}
\verb{nacionalista}{}{}{}{}{s.2g.}{Pessoa adepta do nacionalismo. (\textit{Os nacionalistas quando exageram em suas posições tornam"-se perigosos.})}{na.ci.o.na.lis.ta}{0}
\verb{nacionalização}{}{}{"-ões}{}{s.f.}{Ação ou efeito de nacionalizar.}{na.ci.o.na.li.za.ção}{0}
\verb{nacionalizar}{}{}{}{}{v.t.}{Tornar nacional. (\textit{Uma das preocupações que tínhamos era nacionalizar a produção de programas de computador.})}{na.ci.o.na.li.zar}{0}
\verb{nacionalizar}{}{}{}{}{}{Dar feição nacional. (\textit{Os turistas que chegam aqui, logo se nacionalizam e saem dando seus passinhos de samba.})}{na.ci.o.na.li.zar}{0}
\verb{nacionalizar}{}{}{}{}{v.pron.}{Naturalizar"-se.}{na.ci.o.na.li.zar}{\verboinum{1}}
\verb{nacional"-socialismo}{}{}{}{}{s.m.}{Política adotada pelos alemães até a Segunda Grande Guerra, que culminou na matança de milhões de estrangeiros inocentes; nazismo.}{na.ci.o.nal"-so.ci.a.lis.mo}{0}
\verb{naco}{}{}{}{}{s.m.}{Pedaço arrancado de alguma coisa. (\textit{Comi um naco de pão antes do almoço.})}{na.co}{0}
\verb{nada}{}{}{}{}{pron.}{Nenhuma coisa; nenhuma atitude; nenhuma ideia. (\textit{Nada pode ser feito para ajudá"-lo.})}{na.da}{0}
\verb{nada}{}{}{}{}{adv.}{Nem um pouco. (\textit{Eles não ficaram nada amigos depois da briga.})}{na.da}{0}
\verb{nada}{}{}{}{}{s.m.}{A não existência, o vazio; a ausência de qualquer coisa ou de qualquer sensação. (\textit{Eu não via, não escutava, era um nada que me arrepiava de medo.})}{na.da}{0}
\verb{nadadeira}{ê}{}{}{}{s.f.}{Parte do corpo dos peixes que serve para nadar; barbatana. (\textit{A cozinheira retirou as escamas e as nadadeiras do peixe para assá"-lo.})}{na.da.dei.ra}{0}
\verb{nadadeira}{ê}{}{}{}{}{Calçado de borracha, com a parte da frente alongada e plana, próprio para nado; pé"-de"-pato. (\textit{O salva"-vidas calçou suas nadadeiras e foi logo resgatar o banhista.})}{na.da.dei.ra}{0}
\verb{nadador}{ô}{}{}{}{adj.}{Que tem a habilidade de nadar. (\textit{O pássaro é um animal volátil, o peixe é um animal nadador e o réptil é um animal rasteiro.})}{na.da.dor}{0}
\verb{nadador}{ô}{}{}{}{s.m.}{Indivíduo que pratica a natação. (\textit{O Brasil tem excelentes nadadores, participando das competições internacionais.})}{na.da.dor}{0}
\verb{nadar}{}{}{}{}{v.i.}{Mover"-se na água com o impulso do corpo. (\textit{Os cães, mesmo muito novos, já sabem nadar.})}{na.dar}{0}
\verb{nadar}{}{Fig.}{}{}{}{Ter em abundância alguma riqueza. (\textit{Aquele homem nada em dinheiro.})}{na.dar}{\verboinum{1}}
\verb{nádega}{}{}{}{}{s.f.}{Cada uma das partes carnudas e arredondadas da porção de trás e superior das coxas. (\textit{Injeção na nádega fica menos dolorida do que no braço.})}{ná.de.ga}{0}
\verb{nádegas}{}{}{}{}{s.f.pl.}{O conjunto das duas nádegas.}{ná.de.gas}{0}
\verb{nadir}{}{Astron.}{}{}{s.m.}{Ponto que fica na posição oposta ao zênite. (\textit{Vamos fazer um buraco em direção ao nadir, até chegar no Japão.})}{na.dir}{0}
\verb{nadir}{}{Fig.}{}{}{}{Ponto mais baixo a que se pode chegar. (\textit{Esse foi o nadir, a pior fase de minha vida profissional.})}{na.dir}{0}
\verb{nado}{}{}{}{}{s.m.}{Ato ou efeito de nadar.}{na.do}{0}
\verb{nado}{}{}{}{}{}{Modo de nadar. (\textit{Ela é da equipe do nado sincronizado.})}{na.do}{0}
\verb{nafta}{}{Quím.}{}{}{s.f.}{Betume líquido, resíduo da destilação do petróleo. (\textit{A Petrobras reduz preços da nafta e do querosene no dia primeiro deste mês.})}{naf.ta}{0}
\verb{naftaleno}{}{Quím.}{}{}{s.m.}{Hidrocarboneto aromático constituinte essencial da naftalina.}{naf.ta.le.no}{0}
\verb{naftalina}{}{}{}{}{s.f.}{Produto de cheiro forte, em forma de pequenas bolas, usado para proteger livros e roupas contra traças e baratas. (\textit{A roupa do armário ficou com cheiro de naftalina.})}{naf.ta.li.na}{0}
\verb{nagô}{}{}{}{}{adj.}{Relativo aos nagôs ou a seus descendentes. (\textit{Alguns orixás nagôs oferecem banquetes anuais à gente da casa.})}{na.gô}{0}
\verb{nagô}{}{}{}{}{s.2g.}{Indivíduo ou descendente dos nagôs; antiga designação de qualquer negro escravizado, comerciado na antiga Costa dos Escravos e que falava o iorubá. (\textit{Os nagôs, no Brasil, mantêm suas tradições religiosas.})}{na.gô}{0}
\verb{nagô}{}{}{}{}{s.m.}{Língua falada por esse povo.}{na.gô}{0}
\verb{náiada}{}{}{}{}{}{Var. de \textit{náiade}.}{nái.a.da}{0}
\verb{náiade}{}{Mit.}{}{}{s.f.}{Divindade, de origem grega, que cuida dos rios e das fontes; ninfa da água.}{nái.a.de}{0}
\verb{náilon}{}{}{}{}{s.m.}{Aportuguesamento da palavra inglesa \textit{nylon}; material sintético de largo uso na indústria têxtil.}{nái.lon}{0}
\verb{náilon}{}{}{}{}{}{Tecido feito com esse material. (\textit{As meias de seda foram substituídas pelas de náilon.})}{nái.lon}{0}
\verb{naipe}{}{}{}{}{s.m.}{Sinal que distingue cada um dos quatro grupos de cartas do baralho. (\textit{Os naipes do baralho são: copas, espadas, ouros e paus.})}{nai.pe}{0}
\verb{naipe}{}{Mús.}{}{}{}{Cada um dos grupos de instrumentos em que se divide uma orquestra. (\textit{O naipe dos violinos fica à esquerda no palco.})}{nai.pe}{0}
\verb{naipe}{}{Fig.}{}{}{}{Condição ou qualidade de uma pessoa ou de um grupo de pessoas; categoria, classe. (\textit{Meu amigo é um artista plástico de primeiro naipe.})}{nai.pe}{0}
\verb{naja}{}{Zool.}{}{}{s.f.}{Espécie de cobra venenosa, característica das regiões tropicais da Ásia e da África, que ergue a cabeça e incha a parte do corpo próxima à cabeça quando fica enfurecida. (\textit{O desenho da naja foi encontrado em uma cidade das mais antigas civilizações já descobertas pelos arqueólogos.})}{na.ja}{0}
%\verb{}{}{}{}{}{}{}{}{0}
%\verb{}{}{}{}{}{}{}{}{0}
%\verb{}{}{}{}{}{}{}{}{0}
\verb{nambu}{}{}{}{}{}{Var. de \textit{inhambu}.}{nam.bu}{0}
\verb{namibiano}{}{}{}{}{adj.}{Relativo à Namíbia (sudoeste da África).}{na.mi.bi.a.no}{0}
\verb{namibiano}{}{}{}{}{s.m.}{Indivíduo natural ou habitante desse país; namíbio.}{na.mi.bi.a.no}{0}
\verb{namíbio}{}{}{}{}{adj. e s.m.  }{Ver: \textit{namibiano}.}{na.mí.bio}{0}
\verb{namorada}{}{}{}{}{s.f.}{Mulher que mantém relacionamento amoroso estável com alguém.}{na.mo.ra.da}{0}
\verb{namoradeira}{ê}{}{}{}{s.f.}{Mulher que gosta de namorar; que namora muito.}{na.mo.ra.dei.ra}{0}
\verb{namorado}{}{}{}{}{s.m.}{Homem que mantém relacionamento amoroso estável com alguém.}{na.mo.ra.do}{0}
\verb{namorado}{}{Zool.}{}{}{}{Certo peixe marinho, que ocorre do Espírito Santo a Santa Catarina, de dorso pardo e ventre claro com pintas esbranquiçadas no corpo.}{na.mo.ra.do}{0}
\verb{namorador}{ô}{}{}{}{s.m.}{Homem que gosta de namorar; que namora muito.}{na.mo.ra.dor}{0}
\verb{namorar}{}{}{}{}{v.t.}{Manter relacionamento amoroso estável com alguém.}{na.mo.rar}{0}
\verb{namorar}{}{}{}{}{}{Seduzir; atrair; apaixonar.}{na.mo.rar}{0}
\verb{namorar}{}{}{}{}{}{Desejar muito.}{na.mo.rar}{\verboinum{1}}
\verb{namoricar}{}{}{}{}{v.t.}{Namorar por pouco tempo, ou sem intenções mais sérias.}{na.mo.ri.car}{\verboinum{2}}
\verb{namorico}{}{}{}{}{s.m.}{Namoro de ocasião, que não é para durar muito.}{na.mo.ri.co}{0}
\verb{namoriscar}{}{}{}{}{v.t.}{Namoricar.}{na.mo.ris.car}{\verboinum{2}}
\verb{namorisco}{}{}{}{}{s.m.}{Namorico.}{na.mo.ris.co}{0}
\verb{namoro}{ô}{}{}{}{s.m.}{Ato ou efeito de namorar.}{na.mo.ro}{0}
\verb{namoro}{ô}{}{}{}{}{Galanteio, corte.}{na.mo.ro}{0}
\verb{namoro}{ô}{}{}{}{}{Relação amorosa entre duas pessoas.}{na.mo.ro}{0}
\verb{nana}{}{}{}{}{s.f.}{Canção para ninar; cantiga.        }{na.na}{0}
\verb{nanar}{}{Pop.}{}{}{v.i.}{Pegar no sono; dormir, adormecer. }{na.nar}{\verboinum{1}}
\verb{nanico}{}{}{}{}{adj.}{Que tem o corpo pequeno.}{na.ni.co}{0}
\verb{nanico}{}{}{}{}{}{Pouco desenvolvido; acanhado.}{na.ni.co}{0}
\verb{nanismo}{}{Med.}{}{}{s.m.}{Conjunto de caracteres de indivíduos de pequena estatura como os anões e os pigmeus.}{na.nis.mo}{0}
\verb{nanquim}{}{}{}{}{s.m.}{Tinta preta, de origem chinesa, própria para desenho.}{nan.quim}{0}
\verb{não}{}{}{}{}{adv.}{Palavra que exprime negação.}{não}{0}
\verb{não}{}{}{}{}{s.m.}{Recusa enfática; negativa.}{não}{0}
\verb{não intervenção}{}{}{não intervenções}{}{s.f.}{Princípio jurídico internacional que determina que um Estado não tem o direito de intervir na política interna ou externa de outro.}{não in.ter.ven.ção}{0}
\verb{não"-me"-toques}{ó}{Bot.}{}{}{s.m.}{Arbusto com espinhos aguçados e agrupados.}{não"-me"-to.ques}{0}
\verb{não"-me"-toques}{ó}{Fig.}{}{}{}{Pessoa cheia de melindres.}{não"-me"-to.ques}{0}
\verb{não violência}{}{}{}{}{s.f.}{Abstenção do uso de quaisquer métodos violentos.}{não vi.o.lên.cia}{0}
\verb{napa}{}{}{}{}{s.f.}{Material sintético que imita o couro.}{na.pa}{0}
\verb{napalm}{}{Quím.}{}{}{s.m.}{Material inflamável gelatinizado empregado em bombas incendiárias e lança"-chamas.}{na.palm}{0}
\verb{napoleônico}{}{}{}{}{adj.}{Relativo a Napoleão Bonaparte, imperador da França no período 1804--1815, ou ao seu sistema político e militar. }{na.po.le.ô.ni.co}{0}
\verb{napolitano}{}{}{}{}{adj.}{Relativo a Nápoles, cidade da Itália.}{na.po.li.ta.no}{0}
\verb{napolitano}{}{}{}{}{s.m.}{Indivíduo natural ou habitante dessa cidade.}{na.po.li.ta.no}{0}
\verb{naquele}{ê}{}{}{}{}{Contração da preposição \textit{em} com o pronome \textit{aquele}.}{na.que.le}{0}
\verb{naqueloutro}{}{}{}{}{}{Contração da preposição \textit{em} com os pronomes \textit{aquele} e \textit{outro}.}{na.que.lou.tro}{0}
\verb{naquilo}{}{}{}{}{}{Contração da preposição \textit{em} com o pronome \textit{aquilo}.}{na.qui.lo}{0}
\verb{narceja}{ê}{Zool.}{}{}{s.f.}{Pequena ave pernalta que vive em brejos.}{nar.ce.ja}{0}
\verb{narcisismo}{}{}{}{}{s.m.}{Amor excessivo a si mesmo ou à própria imagem; autoadmiração.}{nar.ci.sis.mo}{0}
\verb{narciso}{}{Bot.}{}{}{s.m.}{Planta ornamental que possui flores vermelhas aromáticas.}{nar.ci.so}{0}
\verb{narciso}{}{Fig.}{}{}{}{Homem muito vaidoso, apaixonado por si mesmo.}{nar.ci.so}{0}
\verb{narcose}{ó}{}{}{}{s.f.}{Letargia ou sonolência provocada por narcótico.}{nar.co.se}{0}
\verb{narcótico}{}{Quím.}{}{}{adj.}{Diz"-se da substância que entorpece os sentidos ou faz dormir; entorpecente.}{nar.có.ti.co}{0}
\verb{narcotismo}{}{}{}{}{s.m.}{Conjuntos dos efeitos provocados pelo uso de narcóticos.}{nar.co.tis.mo}{0}
\verb{narcotizar}{}{}{}{}{v.t.}{Aplicar narcótico; entorpecer, drogar.}{nar.co.ti.zar}{\verboinum{1}}
\verb{narcotráfico}{}{}{}{}{s.m.}{Tráfico de narcóticos, de drogas.}{nar.co.trá.fi.co}{0}
\verb{nardo}{}{Bot.}{}{}{s.m.}{Planta gramínea de cujo rizoma se extrai um aroma usado em perfumaria e no fabrico de incenso.}{nar.do}{0}
\verb{nardo}{}{}{}{}{}{O perfume extraído desse rizoma.}{nar.do}{0}
\verb{narguilé}{}{}{}{}{s.m.}{Cachimbo muito usado no Oriente,  constituído de um tubo longo, um fornilho e um pequeno vaso contendo água perfumada, por onde passa a fumaça antes de chegar à boca.}{nar.gui.lé}{0}
\verb{narigada}{}{}{}{}{s.f.}{Pancada com o nariz.}{na.ri.ga.da}{0}
\verb{narigão}{}{}{"-ões}{}{s.m.}{Nariz muito grande.}{na.ri.gão}{0}
\verb{narigudo}{}{}{}{}{adj.}{Que tem nariz grande.}{na.ri.gu.do}{0}
\verb{narina}{}{Anat.}{}{}{s.f.}{Cada uma das duas aberturas do nariz; venta. }{na.ri.na}{0}
\verb{nariz}{}{Anat.}{}{}{s.m.}{Parte do rosto, entre a testa e a boca, onde se encontra o sentido do olfato.}{na.riz}{0}
\verb{narração}{}{}{"-ões}{}{s.f.}{Ato ou efeito de narrar; relato.}{nar.ra.ção}{0}
\verb{narração}{}{}{"-ões}{}{}{Exposição escrita ou oral de um fato; narrativa.}{nar.ra.ção}{0}
\verb{narração}{}{Liter.}{"-ões}{}{}{Parte de uma obra literária em que se retratam acontecimentos, em oposição àquelas em que se descrevem pessoas, lugares ou objetos. }{nar.ra.ção}{0}
\verb{narrador}{ô}{}{}{}{s.m.}{Pessoa que narra, conta, relata.}{nar.ra.dor}{0}
\verb{narrador}{ô}{Liter.}{}{}{}{Aquele que conta a história, fazendo parte dela ou não. (\textit{O narrador desse conto também é um personagem da história.})}{nar.ra.dor}{0}
\verb{narrar}{}{}{}{}{v.t.}{Contar ou descrever um fato com detalhes.}{nar.rar}{0}
\verb{narrar}{}{}{}{}{}{Relatar, referir.}{nar.rar}{\verboinum{1}}
\verb{narrativa}{}{}{}{}{s.f.}{Conjunto de fatos narrados; conto, história, narração.}{nar.ra.ti.va}{0}
\verb{narrativo}{}{}{}{}{adj.}{Relativo a narração.}{nar.ra.ti.vo}{0}
\verb{narrativo}{}{}{}{}{}{Em forma de narração.}{nar.ra.ti.vo}{0}
\verb{nasal}{}{}{"-ais}{}{adj.2g.}{Relativo ao nariz.}{na.sal}{0}
\verb{nasal}{}{}{"-ais}{}{}{Diz"-se do som em cuja produção o ar é expirado parte pelo nariz, parte pela boca.}{na.sal}{0}
\verb{nasalar}{}{}{}{}{v.t.}{Nasalizar.}{na.sa.lar}{\verboinum{1}}
\verb{nasalidade}{}{}{}{}{s.f.}{Qualidade ou característica do que é nasal.}{na.sa.li.da.de}{0}
\verb{nasalização}{}{}{"-ões}{}{s.f.}{Ato ou efeito de nasalizar.}{na.sa.li.za.ção}{0}
\verb{nasalizar}{}{}{}{}{v.t.}{Tornar nasal; nasalar.}{na.sa.li.zar}{\verboinum{1}}
\verb{nascedoiro}{ô}{}{}{}{}{Var. de \textit{nascedouro}.}{nas.ce.doi.ro}{0}
\verb{nascedouro}{ô}{}{}{}{s.m.}{Lugar onde se nasce.}{nas.ce.dou.ro}{0}
\verb{nascedouro}{ô}{}{}{}{}{Princípio, origem.}{nas.ce.dou.ro}{0}
\verb{nascença}{}{}{}{}{s.f.}{Ato ou efeito de nascer; nascimento.}{nas.cen.ça}{0}
\verb{nascença}{}{}{}{}{}{Origem, princípio, nascedouro.}{nas.cen.ça}{0}
\verb{nascença}{}{}{}{}{}{Usado na expressão \textit{de nascença}:  inato, congênito.}{nas.cen.ça}{0}
\verb{nascente}{}{}{}{}{adj.2g.}{Que nasce.}{nas.cen.te}{0}
\verb{nascente}{}{}{}{}{s.m.}{Lugar no horizonte onde o sol aparece; oriente, leste.}{nas.cen.te}{0}
\verb{nascente}{}{}{}{}{}{Lugar onde nasce um rio.}{nas.cen.te}{0}
\verb{nascer}{ê}{}{}{}{v.i.}{Sair do ventre materno ou do ovo; vir à luz; vir ao mundo.}{nas.cer}{0}
\verb{nascer}{ê}{}{}{}{}{Sair da semente; brotar.}{nas.cer}{0}
\verb{nascer}{ê}{}{}{}{}{Surgir, aparecer, manifestar"-se.}{nas.cer}{\verboinum{15}}
\verb{nascida}{}{Pop.}{}{}{s.f.}{Furúnculo, ferida.}{nas.ci.da}{0}
\verb{nascido}{}{}{}{}{adj.}{Que nasceu; que acabou de nascer.}{nas.ci.do}{0}
\verb{nascimento}{}{}{}{}{s.m.}{Ato ou efeito de nascer; nascença.}{nas.ci.men.to}{0}
\verb{nascimento}{}{}{}{}{}{Começo, princípio, origem.}{nas.ci.men.to}{0}
\verb{nascituro}{}{}{}{}{adj.}{Diz"-se daquele que está para nascer.}{nas.ci.tu.ro}{0}
\verb{nastro}{}{}{}{}{s.m.}{Fita estreita de tecido; trena.}{nas.tro}{0}
\verb{nata}{}{}{}{}{s.f.}{Parte gordurosa do leite que se forma à superfície.}{na.ta}{0}
\verb{nata}{}{Fig.}{}{}{}{A melhor parte de alguma coisa.}{na.ta}{0}
\verb{natação}{}{}{"-ões}{}{s.f.}{Ato ou efeito de nadar.}{na.ta.ção}{0}
\verb{natação}{}{Esport.}{"-ões}{}{}{Modalidade esportiva que consiste em nadar, compreendendo vários estilos como costas, peito, borboleta etc.}{na.ta.ção}{0}
\verb{natal}{}{}{"-ais}{}{adj.2g.}{Relativo ao nascimento ou ao local em que se nasceu.}{na.tal}{0}
\verb{natal}{}{Relig.}{"-ais}{}{s.m.}{Dia em que os cristãos comemoram o nascimento de Jesus Cristo, 25 de dezembro. (Nesta acepção usa"-se com maiúscula).}{na.tal}{0}
\verb{natalense}{}{}{}{}{adj.}{Relativo a Natal, capital do Rio Grande do Norte.}{na.ta.len.se}{0}
\verb{natalense}{}{}{}{}{s.2g.}{Indivíduo natural ou habitante dessa cidade.}{na.ta.len.se}{0}
\verb{natalício}{}{}{}{}{adj.}{Relativo ao dia do nascimento.}{na.ta.lí.cio}{0}
\verb{natalício}{}{}{}{}{s.m.}{O dia do nascimento; aniversário.}{na.ta.lí.cio}{0}
\verb{natalidade}{}{}{}{}{s.f.}{Porcentagem de nascimentos em uma região durante determinado tempo. (\textit{O índice de natalidade em países africanos é muito alto.})}{na.ta.li.da.de}{0}
\verb{natalino}{}{}{}{}{adj.}{Referente ao Natal, data em que se comemora o nascimento de Jesus Cristo. (\textit{No período natalino, as pessoas costumam ficar mais sensíveis ao sentimento de solidariedade.})}{na.ta.li.no}{0}
\verb{natatório}{}{}{}{}{adj.}{Que serve para nadar ou para auxiliar a natação.}{na.ta.tó.rio}{0}
\verb{natimorto}{ô}{Med.}{}{}{adj.}{Diz"-se do feto que nasceu morto.}{na.ti.mor.to}{0}
\verb{natividade}{}{Relig.}{}{}{s.f.}{Nascimento, especialmente o de Jesus Cristo e o dos santos.}{na.ti.vi.da.de}{0}
\verb{nativismo}{}{}{}{}{s.m.}{Sentimento de defesa àquele que é natural da terra e de aversão ao que é estrangeiro; nacionalismo.}{na.ti.vis.mo}{0}
\verb{nativista}{}{}{}{}{adj.2g.}{Relativo ao nativismo.}{na.ti.vis.ta}{0}
\verb{nativista}{}{}{}{}{s.2g.}{Indivíduo que defende o que é próprio de sua terra e tem aversão a estrangeiros; nacionalista.}{na.ti.vis.ta}{0}
\verb{nativo}{}{}{}{}{adj.}{Que nasce, procede; oriundo, proveniente.}{na.ti.vo}{0}
\verb{nativo}{}{}{}{}{s.m.}{Indivíduo que nasce em determinada terra; indígena, aborígene.}{na.ti.vo}{0}
\verb{nato}{}{}{}{}{adj.}{Que é natural, nativo.}{na.to}{0}
\verb{nato}{}{}{}{}{}{De nascença; congênito.}{na.to}{0}
\verb{natura}{}{}{}{}{s.f.}{Natureza.}{na.tu.ra}{0}
\verb{natural}{}{}{"-ais}{}{adj.2g.}{Relativo à natureza ou produzido por ela.}{na.tu.ral}{0}
\verb{natural}{}{}{"-ais}{}{}{Próprio, inerente, inato.}{na.tu.ral}{0}
\verb{natural}{}{}{"-ais}{}{}{Que se explica com facilidade; espontâneo.}{na.tu.ral}{0}
\verb{natural}{}{}{"-ais}{}{}{Nascido em determinado lugar.}{na.tu.ral}{0}
\verb{naturalidade}{}{}{}{}{}{Lugar de nascimento. (\textit{Minha mãe é de naturalidade curitibana.})}{na.tu.ra.li.da.de}{0}
\verb{naturalidade}{}{}{}{}{s.f.}{Qualidade do que é natural; espontaneidade, simplicidade.}{na.tu.ra.li.da.de}{0}
\verb{naturalismo}{}{Art.}{}{}{s.m.}{Escola artístico"-literária que busca retratar a realidade com rigor fotográfico.}{na.tu.ra.lis.mo}{0}
\verb{naturalista}{}{}{}{}{adj.2g.}{Relativo ao naturalismo.}{na.tu.ra.lis.ta}{0}
\verb{naturalista}{}{}{}{}{s.2g.}{Especialista que se dedica às ciências naturais.}{na.tu.ra.lis.ta}{0}
\verb{naturalista}{}{}{}{}{}{Indivíduo que segue o naturalismo.}{na.tu.ra.lis.ta}{0}
\verb{naturalização}{}{}{"-ões}{}{s.f.}{Ato ou efeito de naturalizar.}{na.tu.ra.li.za.ção}{0}
\verb{naturalizar}{}{}{}{}{v.t.}{Conceder a um estrangeiro os direitos que possuem os nascidos do país.}{na.tu.ra.li.zar}{0}
\verb{naturalizar}{}{}{}{}{v.pron.}{Obter os direitos de cidadão do país. (\textit{Meu pai, que era espanhol, naturalizou"-se brasileiro.})}{na.tu.ra.li.zar}{\verboinum{1}}
\verb{naturalmente}{}{}{}{}{adv.}{De modo natural; espontaneamente.}{na.tu.ral.men.te}{0}
\verb{naturalmente}{}{}{}{}{}{Com certeza; evidentemente.}{na.tu.ral.men.te}{0}
\verb{natureba}{é}{Pop.}{}{}{adj.}{Diz"-se daquele que pratica ou defende a alimentação natural.}{na.tu.re.ba}{0}
\verb{natureza}{ê}{}{}{}{s.f.}{Conjunto dos seres e elementos que formam o Universo.}{na.tu.re.za}{0}
\verb{natureza}{ê}{}{}{}{}{Conjunto das características inerentes de um ser; índole, temperamento.}{na.tu.re.za}{0}
\verb{natureza}{ê}{}{}{}{}{Espécie, qualidade, tipo.}{na.tu.re.za}{0}
\verb{natureza"-morta}{ê\ldots{}ó}{Art.}{naturezas"-mortas ⟨ê\ldots{}ó⟩}{}{s.f.}{Gênero de pintura em que são representados seres inanimados como objetos, frutas, legumes etc.}{na.tu.re.za"-mor.ta}{0}
\verb{naturismo}{}{}{}{}{s.m.}{Conjunto de ideias que preconizam um modo de vida conforme as leis da natureza.}{na.tu.ris.mo}{0}
\verb{naturista}{}{}{}{}{adj.2g.}{Diz"-se daquele que defende ou pratica o naturismo.}{na.tu.ris.ta}{0}
\verb{nau}{}{}{}{}{s.f.}{Grande embarcação; navio.}{nau}{0}
\verb{naufragar}{}{}{}{}{v.i.}{Afundar nas águas; ir a pique; soçobrar. (\textit{O Titanic naufragou após bater em um iceberg.})}{nau.fra.gar}{0}
\verb{naufragar}{}{}{}{}{}{Sofrer um naufrágio.}{nau.fra.gar}{0}
\verb{naufragar}{}{Fig.}{}{}{}{Fracassar, malograr, falhar.}{nau.fra.gar}{\verboinum{5}}
\verb{naufrágio}{}{}{}{}{s.m.}{Ato ou efeito de naufragar.}{nau.frá.gio}{0}
\verb{naufrágio}{}{Fig.}{}{}{}{Fracasso, malogro, ruína.}{nau.frá.gio}{0}
\verb{náufrago}{}{}{}{}{adj.}{Diz"-se do indivíduo que naufragou.}{náu.fra.go}{0}
\verb{naupatia}{}{Med.}{}{}{s.f.}{Mal"-estar, enjoo causado por viagem marítima.}{nau.pa.ti.a}{0}
\verb{nauruano}{}{}{}{}{adj.}{Relativo à República de Nauru, ilha ao norte da Oceania.}{nau.ru.a.no}{0}
\verb{nauruano}{}{}{}{}{s.m.}{Indivíduo natural ou habitante dessa república.}{nau.ru.a.no}{0}
\verb{náusea}{}{}{}{}{s.f.}{Mal"-estar acompanhado de tontura e enjoo, produzidos pelo balanço de embarcação, de avião etc.}{náu.se.a}{0}
\verb{náusea}{}{}{}{}{}{Sentimento de repugnância; asco, nojo.}{náu.se.a}{0}
\verb{nauseabundo}{}{}{}{}{adj.}{Que produz náusea, vômito.}{nau.se.a.bun.do}{0}
\verb{nauseante}{}{}{}{}{adj.2g.}{Nauseabundo.}{nau.se.an.te}{0}
\verb{nausear}{}{}{}{}{v.t.}{Causar náusea; enjoar, repugnar.}{nausear}{\verboinum{4}}
\verb{nauta}{}{}{}{}{s.m.}{Navegante.}{nau.ta}{0}
\verb{náutica}{}{}{}{}{s.f.}{Arte e ciência de navegar; navegação.}{náu.ti.ca}{0}
\verb{náutico}{}{}{}{}{adj.}{Relativo a navegação.}{náu.ti.co}{0}
\verb{náutilo}{}{Zool.}{}{}{s.m.}{Espécie de molusco de concha externa, espiral, encontrado em águas profundas dos oceanos Índico e Pacífico.}{náu.ti.lo}{0}
\verb{naval}{}{}{"-ais}{}{adj.2g.}{Relativo a navios ou à marinha de guerra.}{na.val}{0}
\verb{navalha}{}{}{}{}{s.f.}{Instrumento cortante, próprio para barbear, formado de uma lâmina muito afiada e um cabo que guarda o fio dessa lâmina quando fecha.}{na.va.lha}{0}
\verb{navalhada}{}{}{}{}{s.f.}{Golpe desferido com navalha.}{na.va.lha.da}{0}
\verb{navalhar}{}{}{}{}{v.t.}{Cortar ou golpear com navalha.}{na.va.lhar}{\verboinum{1}}
\verb{nave}{}{}{}{}{s.f.}{Embarcação, navio, nau.}{na.ve}{0}
\verb{nave}{}{}{}{}{}{A parte da igreja desde a entrada até o altar principal.}{na.ve}{0}
\verb{navegabilidade}{}{}{}{}{s.f.}{Qualidade ou estado do que é navegável.}{na.ve.ga.bi.li.da.de}{0}
\verb{navegação}{}{}{"-ões}{}{s.f.}{Ato ou efeito de navegar.}{na.ve.ga.ção}{0}
\verb{navegação}{}{}{"-ões}{}{}{Técnica ou ciência de navegar; náutica.}{na.ve.ga.ção}{0}
\verb{navegador}{ô}{}{}{}{s.m.}{Indivíduo que navega; navegante.}{na.ve.ga.dor}{0}
\verb{navegador}{ô}{}{}{}{}{Indivíduo especializado nos cálculos necessários para navegar.}{na.ve.ga.dor}{0}
\verb{navegante}{}{}{}{}{adj.2g.}{Relativo a nauta.}{na.ve.gan.te}{0}
\verb{navegante}{}{}{}{}{s.2g.}{Indivíduo que navega; navegador.}{na.ve.gan.te}{0}
\verb{navegar}{}{}{}{}{v.i.}{Deslocar"-se na água ou no ar.}{na.ve.gar}{0}
\verb{navegar}{}{Informát.}{}{}{}{Acessar hipertextos pela \textit{internet}.}{na.ve.gar}{\verboinum{5}}
\verb{navegável}{}{}{"-eis}{}{adj.2g.}{Que se pode percorrer de navio ou barco.}{na.ve.gá.vel}{0}
\verb{naveta}{ê}{}{}{}{s.f.}{Recipiente pequeno usado nas igrejas para guardar o incenso queimado nos turíbulos.}{na.ve.ta}{0}
\verb{naveta}{ê}{}{}{}{}{Lançadeira de máquina de costura, que tem forma de barco.}{na.ve.ta}{0}
\verb{navio}{}{}{}{}{s.m.}{Embarcação de grande porte, utilizada para navegar em águas profundas.}{na.vi.o}{0}
\verb{navio"-escola}{ó}{}{navios"-escola ⟨ó⟩}{}{s.m.}{Navio destinado à aprendizagem  de alunos de escola mercante.}{na.vi.o"-es.co.la}{0}
\verb{navio"-tanque}{}{}{navios"-tanques \textit{ou} navios"-tanque}{}{s.m.}{Navio destinado ao transporte de carga líquida, geralmente água ou combustíveis.}{na.vi.o"-tan.que}{0}
\verb{nazareno}{}{}{}{}{adj.}{Relativo a Nazaré, cidade da Palestina, atual Israel.}{na.za.re.no}{0}
\verb{nazareno}{}{}{}{}{s.m.}{Indivíduo natural ou habitante dessa cidade.}{na.za.re.no}{0}
\verb{nazareno}{}{Relig.}{}{}{}{Denominação dada a Jesus Cristo pelos judeus, posteriormente estendida aos seus primeiros seguidores.}{na.za.re.no}{0}
\verb{nazifascista}{}{}{}{}{adj.2g.}{Relativo ao mesmo tempo ao nazismo e ao fascismo.}{na.zi.fas.cis.ta}{0}
\verb{nazifascista}{}{}{}{}{s.2g.}{Adepto das ideias nazistas e fascistas.}{na.zi.fas.cis.ta}{0}
\verb{nazismo}{}{Hist.}{}{}{s.m.}{Conjunto das doutrinas políticas totalitárias do partido do movimento nacional"-socialista alemão, fundado e liderado por Adolph Hitler, implantado na Alemanha em 1920 e extinto com a derrota desse país na Segunda Guerra Mundial.}{na.zis.mo}{0}
\verb{nazista}{}{}{}{}{adj.2g.}{Relativo ao nazismo.}{na.zis.ta}{0}
\verb{nazista}{}{}{}{}{}{Adepto ou seguidor do nazismo.}{na.zis.ta}{0}
\verb{Nb}{}{Quím.}{}{}{}{Símb. do \textit{nióbio}.}{Nb}{0}
\verb{Nd}{}{Quím.}{}{}{}{Símb. do \textit{neodímio}.}{Nd}{0}
\verb{NE}{}{}{}{}{}{Abrev. de \textit{nordeste}.}{n.e.}{0}
\verb{Ne}{}{Quím.}{}{}{}{Símb. do \textit{neônio}.}{Ne}{0}
\verb{neblina}{}{}{}{}{s.f.}{Névoa densa que se forma a baixa altura; nevoeiro, cerração.}{ne.bli.na}{0}
\verb{neblinar}{}{}{}{}{v.i.}{Cair neblina; chuviscar, garoar.}{ne.bli.nar}{\verboinum{1}}
\verb{nebulização}{}{}{"-ões}{}{s.f.}{Ato de converter um líquido em nuvem de vapor; pulverização. }{ne.bu.li.za.ção}{0}
\verb{nebulizador}{ô}{}{}{}{s.m.}{Instrumento usado para aspergir líquido em gotículas; vaporizador, borrifador, atomizador. }{ne.bu.li.za.dor}{0}
\verb{nebulizar}{}{}{}{}{v.t.}{Transformar um líquido em nuvens de vapor; vaporizar, pulverizar.}{ne.bu.li.zar}{\verboinum{1}}
\verb{nebulosa}{ó}{}{}{}{s.f.}{Nuvem de matéria interestelar que se apresenta como mancha branca.}{ne.bu.lo.sa}{0}
\verb{nebulosidade}{}{}{}{}{s.f.}{Qualidade ou estado de nebuloso.}{ne.bu.lo.si.da.de}{0}
\verb{nebulosidade}{}{Fig.}{}{}{}{Ausência de clareza ou precisão.}{ne.bu.lo.si.da.de}{0}
\verb{nebuloso}{ô}{}{"-osos ⟨ó⟩}{"-osa ⟨ó⟩}{adj.}{Coberto de névoas ou nuvens; nublado, nevoento.}{ne.bu.lo.so}{0}
\verb{neca}{é}{Pop.}{}{}{pron.}{Coisa alguma; nada.}{ne.ca}{0}
\verb{necedade}{}{}{}{}{s.f.}{Qualidade ou dito de néscio; estupidez, ignorância, disparate.}{ne.ce.da.de}{0}
\verb{necessário}{}{}{}{}{adj.}{Que não pode faltar; imprescindível, indispensável.}{ne.ces.sá.rio}{0}
\verb{necessário}{}{}{}{}{}{Que não se pode evitar; forçoso, fatal, inevitável.}{ne.ces.sá.rio}{0}
\verb{necessidade}{}{}{}{}{s.f.}{Aquilo que é absolutamente necessário, que faz falta.}{ne.ces.si.da.de}{0}
\verb{necessidade}{}{}{}{}{}{Falta de recursos para sobreviver; indigência, precisão, pobreza.}{ne.ces.si.da.de}{0}
\verb{necessitado}{}{}{}{}{adj.}{Que precisa do necessário para viver; indigente, miserável, pobre.}{ne.ces.si.ta.do}{0}
\verb{necessitar}{}{}{}{}{v.t.}{Ter necessidade; carecer, precisar.}{ne.ces.si.tar}{\verboinum{1}}
\verb{necrofagia}{}{}{}{}{s.f.}{Ato de alimentar"-se da carne de animais mortos ou de substâncias em decomposição.}{ne.cro.fa.gi.a}{0}
\verb{necrófago}{}{}{}{}{adj.}{Que se alimenta da carne de animais mortos ou de substâncias em decomposição.}{ne.cró.fa.go}{0}
\verb{necrofilia}{}{}{}{}{s.f.}{Atração sexual doentia por cadáveres.}{ne.cro.fi.li.a}{0}
\verb{necrófilo}{}{}{}{}{adj.}{Que sente desejo sexual doentio por cadáveres.}{ne.cró.fi.lo}{0}
\verb{necrofobia}{}{}{}{}{s.f.}{Medo doentio da morte ou dos mortos.}{ne.cro.fo.bi.a}{0}
\verb{necrologia}{}{}{}{}{s.f.}{Relação de notícias sobre pessoas falecidas; obituário.}{ne.cro.lo.gi.a}{0}
\verb{necrológico}{}{}{}{}{adj.}{Relativo a necrologia ou a necrológio.}{ne.cro.ló.gi.co}{0}
\verb{necrológio}{}{}{}{}{s.m.}{Elogio a respeito de alguém falecido.}{ne.cro.ló.gio}{0}
\verb{necromancia}{}{}{}{}{s.f.}{Adivinhação do futuro por meio da invocação dos mortos.}{ne.cro.man.ci.a}{0}
\verb{necromante}{}{}{}{}{s.2g.}{Indivíduo que pratica a necromancia.}{ne.cro.man.te}{0}
\verb{necrópole}{}{}{}{}{s.f.}{Local onde se enterram os mortos; cemitério.}{ne.cró.po.le}{0}
\verb{necropsia}{}{}{}{}{}{Var. de \textit{necrópsia}.}{ne.crop.si.a}{0}
\verb{necrópsia}{}{Med.}{}{}{s.f.}{Exame feito em cadáver a fim de se conhecerem as causas da morte; autópsia.}{ne.cróp.sia}{0}
\verb{necrosar}{}{Med.}{}{}{v.t.}{Causar ou sofrer necrose; gangrenar.}{ne.cro.sar}{\verboinum{1}}
\verb{necrose}{ó}{Med.}{}{}{s.f.}{Morte de um tecido do organismo; gangrena.}{ne.cro.se}{0}
\verb{necrotério}{}{}{}{}{s.m.}{Local onde são recolhidos os cadáveres para identificação ou autópsia.}{ne.cro.té.rio}{0}
\verb{néctar}{}{}{}{}{s.m.}{Líquido açucarado produzido pelas flores e que constitui o elemento principal do mel das abelhas.}{néc.tar}{0}
\verb{nectarina}{}{}{}{}{s.f.}{Variedade de pêssego com casca lisa, sem pelos e polpa macia à qual o caroço não adere.}{nec.ta.ri.na}{0}
\verb{nédio}{}{}{}{}{adj.}{De pele lustrosa, devido à gordura; luzidio.}{né.dio}{0}
\verb{neerlandês}{}{}{}{}{adj.}{Relativo à Holanda (Países Baixos); holandês.}{ne.er.lan.dês}{0}
\verb{neerlandês}{}{}{}{}{s.m.}{Indivíduo natural ou habitante desse país.}{ne.er.lan.dês}{0}
\verb{nefando}{}{}{}{}{adj.}{Indigno de ser nomeado; abominável, execrável.}{ne.fan.do}{0}
\verb{nefasto}{}{}{}{}{adj.}{Que traz mau agouro; funesto, sinistro.}{ne.fas.to}{0}
\verb{nefelibata}{}{}{}{}{adj.2g.}{Que vive nas nuvens.}{ne.fe.li.ba.ta}{0}
\verb{nefelibata}{}{}{}{}{}{Diz"-se do indivíduo idealista, sonhador, utópico.}{ne.fe.li.ba.ta}{0}
\verb{nefralgia}{}{Med.}{}{}{s.f.}{Dor nos rins.}{ne.fral.gi.a}{0}
\verb{nefrectomia}{}{Med.}{}{}{s.f.}{Retirada cirúrgica total ou parcial de rim.}{ne.frec.to.mi.a}{0}
\verb{nefrite}{}{Med.}{}{}{s.f.}{Inflamação dos rins.}{ne.fri.te}{0}
\verb{nefrologia}{}{Med.}{}{}{s.f.}{Ramo da medicina que trata das doenças renais.}{ne.fro.lo.gi.a}{0}
\verb{nefrose}{ó}{Med.}{}{}{s.f.}{Afecção renal degenerativa.}{ne.fro.se}{0}
\verb{nefrotomia}{}{Med.}{}{}{s.f.}{Incisão cirúrgica no rim.}{ne.fro.to.mi.a}{0}
\verb{negaça}{}{}{}{}{s.f.}{Ato de quem finge recusar uma proposta; negativa, engodo, logro.}{ne.ga.ça}{0}
\verb{negação}{}{}{"-ões}{}{s.f.}{Ato ou efeito de negar; negativa.}{ne.ga.ção}{0}
\verb{negação}{}{}{"-ões}{}{}{Falta de aptidão; incapacidade, nulidade.}{ne.ga.ção}{0}
\verb{negacear}{}{}{}{}{v.t.}{Fingir que se recusa uma proposta; lograr, enganar.}{ne.ga.ce.ar}{\verboinum{4}}
\verb{negar}{}{}{}{}{v.t.}{Afirmar que algo não é verdadeiro ou inexiste.}{ne.gar}{0}
\verb{negar}{}{}{}{}{}{Não conceder; recusar.}{ne.gar}{0}
\verb{negar}{}{}{}{}{}{Contestar, refutar, desmentir. (\textit{O aluno negou que estivesse colando na prova.})}{ne.gar}{\verboinum{5}}
\verb{negativa}{}{}{}{}{s.f.}{Expressão ou gesto com que se nega; recusa, negação.}{ne.ga.ti.va}{0}
\verb{negativismo}{}{}{}{}{s.m.}{Atitude de se negar ou se opor sistematicamente.}{ne.ga.ti.vis.mo}{0}
\verb{negativo}{}{}{}{}{adj.}{Que exprime ou indica negação.}{ne.ga.ti.vo}{0}
\verb{negativo}{}{}{}{}{}{Contrário ao que se esperava; contraproducente.}{ne.ga.ti.vo}{0}
\verb{negativo}{}{}{}{}{}{Que indica falta, ausência. (\textit{O exame de sangue deu negativo para aquela doença.})}{ne.ga.ti.vo}{0}
\verb{negativo}{}{}{}{}{s.m.}{Chapa fotográfica em que os tons claros ou escuros de um objeto aparecem invertidos.}{ne.ga.ti.vo}{0}
\verb{negável}{}{}{"-eis}{}{adj.2g.}{Que se pode negar.}{ne.gá.vel}{0}
\verb{négligé}{}{}{}{}{s.m.}{Roupão feminino de tecido fino e transparente.}{\textit{négligé}}{0}
\verb{negligência}{}{}{}{}{s.f.}{Falta de cuidado; descuido, desleixo.}{ne.gli.gên.cia}{0}
\verb{negligenciar}{}{}{}{}{v.t.}{Tratar com negligência; descuidar, desleixar.}{ne.gli.gen.ci.ar}{0}
\verb{negligenciar}{}{}{}{}{}{Deixar de cumprir bem um compromisso ou obrigação.}{ne.gli.gen.ci.ar}{\verboinum{6}}
\verb{negligente}{}{}{}{}{adj.2g.}{Que não tem cuidado; desleixado, descuidado.}{ne.gli.gen.te}{0}
\verb{nego}{ê}{}{}{}{s.m.}{Tratamento carinhoso; amigo, camarada, companheiro.}{ne.go}{0}
\verb{negociação}{}{}{"-ões}{}{s.f.}{Ato ou efeito de negociar; negócio, ajuste.}{ne.go.ci.a.ção}{0}
\verb{negociante}{}{}{}{}{adj.2g.}{Indivíduo que negocia; comerciante.}{ne.go.ci.an.te}{0}
\verb{negociar}{}{}{}{}{}{Estabelecer acordo; combinar.}{ne.go.ci.ar}{\verboinum{6}}
\verb{negociar}{}{}{}{}{}{Contratar, ajustar, diligenciar.}{ne.go.ci.ar}{0}
\verb{negociar}{}{}{}{}{v.t.}{Lidar com negócios; comerciar.}{ne.go.ci.ar}{0}
\verb{negociata}{}{}{}{}{s.f.}{Negócio suspeito, desonesto; trapaça.}{ne.go.ci.a.ta}{0}
\verb{negociável}{}{}{"-eis}{}{adj.2g.}{Que se pode negociar; comerciável.}{ne.go.ci.á.vel}{0}
\verb{negócio}{}{}{}{}{s.m.}{Ato ou efeito de negociar; comércio, tráfico.}{ne.gó.cio}{0}
\verb{negócio}{}{}{}{}{}{Casa de negócio; empresa.}{ne.gó.cio}{0}
\verb{negócio}{}{}{}{}{}{Assunto, fato, caso.}{ne.gó.cio}{0}
\verb{negocista}{}{}{}{}{s.2g.}{Indivíduo que pratica negócios escusos, negociatas.}{ne.go.cis.ta}{0}
\verb{negra}{ê}{}{}{}{s.f.}{Mulher da raça negra.}{ne.gra}{0}
\verb{negra}{ê}{}{}{}{}{Partida que desempata uma competição.}{ne.gra}{0}
\verb{negrada}{}{}{}{}{s.f.}{Grupo de pessoas.}{ne.gra.da}{0}
\verb{negralhão}{}{}{"-ões}{}{s.m.}{Indivíduo da raça negra que apresenta elevada estatura.}{ne.gra.lhão}{0}
\verb{negrão}{}{}{"-ões}{}{s.m.}{Certa variedade de uva tinta.}{ne.grão}{0}
%\verb{}{}{}{}{}{}{}{}{0}
%\verb{}{}{}{}{}{}{}{}{0}
%\verb{}{}{}{}{}{}{}{}{0}
\verb{negregado}{}{}{}{}{adj.}{Que inspira pena, piedade; infeliz, desventurado.}{ne.gre.ga.do}{0}
\verb{negreiro}{ê}{Hist.}{}{}{adj.}{Diz"-se do navio que trazia os negros aprisionados na África para serem vendidos como escravos.}{ne.grei.ro}{0}
\verb{negrejar}{}{}{}{}{v.i.}{Ser ou aparentar ser negro.}{ne.gre.jar}{0}
\verb{negrejar}{}{}{}{}{}{Tornar"-se negro; escurecer.}{ne.gre.jar}{\verboinum{1}}
\verb{negridão}{}{}{"-ões}{}{s.f.}{Qualidade ou característica de negro; negrura.}{ne.gri.dão}{0}
\verb{negrita}{}{}{}{}{}{Var. de \textit{negrito}.}{ne.gri.ta}{0}
\verb{negrito}{}{}{}{}{s.m.}{Tipo de letra de traços escuros e grossos, usada em textos impressos; negrita.}{ne.gri.to}{0}
\verb{negritude}{}{}{}{}{s.f.}{Estado ou condição das pessoas da raça negra. }{ne.gri.tu.de}{0}
\verb{negro}{}{}{}{}{s.m.}{Diz"-se do indivíduo pertencente à etnia negra.}{ne.gro}{0}
\verb{negro}{}{}{}{}{adj.}{Da cor do piche ou do carvão; preto.}{ne.gro}{0}
\verb{negroide}{}{}{}{}{adj.2g.}{Diz"-se do indivíduo que apresenta características da raça negra.}{ne.groi.de}{0}
\verb{negror}{ô}{}{}{}{s.m.}{Escuridão densa; negrume.}{ne.gror}{0}
\verb{negrume}{}{}{}{}{s.m.}{Falta de luz; escuridão, trevas.}{ne.gru.me}{0}
\verb{negrura}{}{}{}{}{s.f.}{Qualidade de negro; negridão.}{ne.gru.ra}{0}
\verb{nele}{ê}{}{}{}{}{Contração da preposição \textit{em} com o pronome \textit{ele}.}{ne.le}{0}
\verb{nelore}{ó}{}{}{}{adj.2g.}{Diz"-se da raça de gado zebu originária da Índia.}{ne.lo.re}{0}
\verb{nem}{}{}{}{}{conj.}{E não. (\textit{Ele não trouxe nem a caneta nem o caderno de anotações.})}{nem}{0}
\verb{nem}{}{}{}{}{}{E sem. (\textit{Ele ficou sem pai nem mãe.})}{nem}{0}
\verb{nem}{}{}{}{}{adv.}{Não. (\textit{Ele nem imaginou que poderia ficar gripado com a chuva de ontem.})}{nem}{0}
\verb{nematódeo}{}{Zool.}{}{}{s.m.}{Espécime dos nematódeos, filo de vermes de vida livre, com corpo cilíndrico e simetria radial, encontrados na água e no solo, e que parasitam muitos animais e plantas.}{ne.ma.tó.deo}{0}
\verb{nenê}{}{}{}{}{s.m.}{Criança de colo, de poucos meses; criancinha; neném.}{ne.nê}{0}
\verb{neném}{}{}{}{}{}{Var. de \textit{nenê}.}{ne.ném}{0}
\verb{nenhum}{}{}{}{}{pron.}{Nem um; nem mesmo um. (\textit{Ainda não chegou nenhum fax daquela empresa respondendo à minha solicitação.})}{ne.nhum}{0}
\verb{nenhum}{}{}{}{}{}{Nulo; inexistente.}{ne.nhum}{0}
\verb{nenhures}{}{}{}{}{pron.}{Em nenhuma parte.}{ne.nhu.res}{0}
\verb{nênia}{}{}{}{}{s.f.}{Canto melancólico; elegia, lamento.}{nê.nia}{0}
\verb{nenúfar}{}{Bot.}{}{}{s.f.}{Planta aquática muito cultivada como ornamental devido a suas belas flores e folhas grandes; lótus, ninfeia.}{ne.nú.far}{0}
\verb{Neoclassicismo}{}{Liter.}{}{}{s.m.}{Movimento artístico"-literário inspirado nos ideais e modelos do Classicismo greco"-romano e do Renascimento, e que se desenvolveu ao longo do século \textsc{xviii} e na primeira metade do século \textsc{xix}.}{Ne.o.clas.si.cis.mo}{0}
\verb{neoclássico}{}{}{}{}{adj.}{Relativo ao Neoclassicismo.}{ne.o.clás.si.co}{0}
\verb{neoclássico}{}{}{}{}{s.m.}{Indivíduo que é adepto do Neoclassicismo.}{ne.o.clás.si.co}{0}
\verb{neodímio}{}{Quím.}{}{}{s.m.}{Elemento químico metálico, prateado, reativo, da família dos lantanídeos (terras"-raras); usado na fabricação de vidros especiais, pedras de isqueiro e certos pigmentos de esmalte. \elemento{60}{144.24}{Nd}.}{ne.o.dí.mio}{0}
\verb{neófito}{}{}{}{}{s.m.}{Indivíduo que acaba de ingressar em uma organização; novato, principiante.}{ne.ó.fi.to}{0}
\verb{neoformação}{}{Biol.}{"-ões}{}{s.f.}{Formação de tecido orgânico que restaura o tecido lesado por feridas ou traumatismos.}{ne.o.for.ma.ção}{0}
\verb{neolatino}{}{}{}{}{adj.}{Diz"-se das línguas modernas originárias do latim ou dos povos cuja cultura procede da antiga civilização romana. (\textit{O português é uma língua neolatina.})}{ne.o.la.ti.no}{0}
\verb{neoliberalismo}{}{}{}{}{s.m.}{Doutrina político"-econômica que defende a absoluta liberdade de mercado e uma restrição à intervenção estatal sobre a economia.}{ne.o.li.be.ra.lis.mo}{0}
\verb{neolítico}{}{Arqueol.}{}{}{adj.}{Diz"-se do período da pedra polida.}{ne.o.lí.ti.co}{0}
\verb{neologismo}{}{}{}{}{s.m.}{Palavra ou expressão nova criada na língua ou antiga empregada com sentido novo.}{ne.o.lo.gis.mo}{0}
\verb{neologista}{}{}{}{}{adj.2g.}{Diz"-se do indivíduo que faz uso constante de neologismos.}{ne.o.lo.gis.ta}{0}
\verb{neon}{}{}{}{}{s.m.}{Letreiro luminoso em que se usa o gás neônio.}{ne.on}{0}
\verb{néon}{}{}{}{}{}{Var. de \textit{neon}.}{né.on}{0}
\verb{neônio}{}{Quím.}{}{}{s.m.}{Elemento químico da família dos gases nobres, incolor, existente em pequena proporção na atmosfera, usado na fabricação de anúncios luminosos e válvulas eletrônicas espaciais. \elemento{10}{20.1797}{Ne}.}{ne.ô.nio}{0}
\verb{neoplasia}{}{Med.}{}{}{s.f.}{Processo patológico resultante do desenvolvimento de um neoplasma; neoformação.  }{ne.o.pla.si.a}{0}
\verb{neoplasma}{}{Med.}{}{}{s.m.}{Crescimento anormal de um tecido orgânico, podendo ser benigno ou maligno; tumor.}{ne.o.plas.ma}{0}
\verb{neozelandês}{}{}{}{}{adj.}{Relativo à Nova Zelândia (Oceania).}{ne.o.ze.lan.dês}{0}
\verb{neozelandês}{}{}{}{}{s.m.}{Indivíduo natural ou habitante desse país.}{ne.o.ze.lan.dês}{0}
\verb{nepalês}{}{}{}{}{adj.}{Relativo ao Nepal (Ásia Central).}{ne.pa.lês}{0}
\verb{nepalês}{}{}{}{}{s.m.}{Indivíduo natural ou habitante desse país.}{ne.pa.lês}{0}
\verb{nepotismo}{}{}{}{}{s.m.}{Favoritismo que autoridades públicas concedem a parentes; apadrinhamento.}{ne.po.tis.mo}{0}
\verb{nereida}{ê}{Mit.}{}{}{s.f.}{Na mitologia grega, ninfa que preside ao mar.}{ne.rei.da}{0}
\verb{nereide}{}{}{}{}{}{Var. de \textit{nereida}.}{ne.rei.de}{0}
\verb{neres}{é}{Pop.}{}{}{pron.}{Coisa alguma; nada.}{ne.res}{0}
\verb{nervo}{ê}{Anat.}{}{}{s.m.}{Conjunto de células nervosas, em forma de cordão, que conduz impulsos de uma parte para outra do corpo.}{ner.vo}{0}
\verb{nervosidade}{}{}{}{}{s.f.}{Qualidade ou estado do que é nervoso.}{ner.vo.si.da.de}{0}
\verb{nervosismo}{}{}{}{}{s.m.}{Grande agitação; excitação, irritabilidade.}{ner.vo.sis.mo}{0}
\verb{nervoso}{ô}{}{"-osos ⟨ó⟩}{"-osa ⟨ó⟩}{adj.}{Relativo aos nervos.}{ner.vo.so}{0}
\verb{nervoso}{ô}{}{"-osos ⟨ó⟩}{"-osa ⟨ó⟩}{}{Que é provido de nervos.}{ner.vo.so}{0}
\verb{nervoso}{ô}{}{"-osos ⟨ó⟩}{"-osa ⟨ó⟩}{}{Que tem os nervos exasperados; agitado, exaltado, irritado.}{ner.vo.so}{0}
\verb{nervudo}{}{}{}{}{adj.}{Que tem nervos fortes; vigoroso, forte.}{ner.vu.do}{0}
\verb{nervura}{}{Bot.}{}{}{s.f.}{Conjunto de pequenas fibras que percorrem a superfície de pétalas e folhas.}{ner.vu.ra}{0}
\verb{nervura}{}{Zool.}{}{}{}{Fio que sustenta a asa dos insetos.}{ner.vu.ra}{0}
\verb{nescidade}{}{}{}{}{s.f.}{Qualidade ou dito de néscio; estupidez, ignorância, disparate; necedade.}{nes.ci.da.de}{0}
\verb{néscio}{}{}{}{}{adj.}{Diz"-so indivíduo em que falta inteligência; estúpido, tolo.}{nés.cio}{0}
\verb{nesga}{ê}{}{}{}{s.f.}{Pequena parte de qualquer espaço; bocado, pedaço.}{nes.ga}{0}
\verb{nêspera}{}{}{}{}{s.f.}{Fruto comestível da nespereira, pequeno e de cor amarela, semelhante à ameixa. }{nês.pe.ra}{0}
\verb{nespereira}{ê}{Bot.}{}{}{s.f.}{Árvore da família das rosáceas,  nativa da Ásia e muito cultivada no Brasil, e cujo fruto é a nêspera.}{nes.pe.rei.ra}{0}
\verb{nesse}{ê}{}{}{}{}{Contração da preposição \textit{em} com o pronome \textit{esse}.}{nes.se}{0}
\verb{nessoutro}{}{}{}{}{}{Contração da preposição \textit{em} com os pronomes \textit{esse} e \textit{outro}.}{nes.sou.tro}{0}
\verb{neste}{ê}{}{}{}{}{Contração da preposição \textit{em} com o pronome \textit{este}.}{nes.te}{0}
\verb{nestoutro}{}{}{}{}{}{Contração da preposição \textit{em} com os pronomes \textit{este} e \textit{outro}.}{nes.tou.tro}{0}
\verb{neto}{é}{}{}{}{s.m.}{O filho em relação aos pais de seu pai ou de sua mãe.}{ne.to}{0}
\verb{netúnio}{}{Quím.}{}{}{s.m.}{Elemento químico radioativo, do grupo dos actinídeos, obtido artificialmente. \elemento{93}{237}{Np}.}{ne.tú.nio}{0}
\verb{Netuno}{}{Astron.}{}{}{s.m.}{O oitavo planeta em relação ao Sol.}{Ne.tu.no}{0}
\verb{Netuno}{}{Mit.}{}{}{}{Na mitologia romana, divindade que preside ao mar.}{Ne.tu.no}{0}
\verb{neural}{}{}{"-ais}{}{adj.2g.}{Relativo a nervos.}{neu.ral}{0}
\verb{neural}{}{}{"-ais}{}{}{Próprio dos nervos.}{neu.ral}{0}
\verb{neuralgia}{}{Med.}{}{}{s.f.}{Dor em nervo ou em suas ramificações; nevralgia.}{neu.ral.gi.a}{0}
\verb{neurastenia}{}{Med.}{}{}{s.f.}{Estado de exaustão dos nervos acompanhado de fraqueza e irritabilidade.}{neu.ras.te.ni.a}{0}
\verb{neurastênico}{}{}{}{}{adj.}{Relativo a neurastenia.}{neu.ras.tê.ni.co}{0}
\verb{neurastênico}{}{}{}{}{}{Que sofre de neurastenia.}{neu.ras.tê.ni.co}{0}
\verb{neurite}{}{Med.}{}{}{s.f.}{Inflamação de nervo; nevrite.}{neu.ri.te}{0}
\verb{neurocirurgia}{}{Med.}{}{}{s.f.}{Cirurgia praticada nos centros nervosos do corpo.}{neu.ro.ci.rur.gi.a}{0}
\verb{neurocirurgião}{}{}{"-ões}{}{s.m.}{Médico"-cirurgião especializado em neurocirurgia.}{neu.ro.ci.rur.gi.ão}{0}
\verb{neurologia}{}{}{}{}{s.f.}{Parte da medicina que trata das doenças do sistema nervoso.}{neu.ro.lo.gi.a}{0}
\verb{neurológico}{}{}{}{}{adj.}{Relativo a neurologia.}{neu.ro.ló.gi.co}{0}
\verb{neurologista}{}{}{}{}{s.2g.}{Especialista que se dedica à neurologia.}{neu.ro.lo.gis.ta}{0}
\verb{neurônio}{}{Anat.}{}{}{s.m.}{Célula básica do sistema nervoso.}{neu.rô.nio}{0}
\verb{neuropatia}{}{Med.}{}{}{s.f.}{Nome comum dado às doenças do sistema nervoso.}{neu.ro.pa.ti.a}{0}
\verb{neurose}{ó}{Med.}{}{}{s.f.}{Doença mental caracterizada por perturbações de comportamento como ansiedade, depressão, insegurança, fobias etc.}{neu.ro.se}{0}
\verb{neurótico}{}{}{}{}{adj.}{Relativo a neurose.}{neu.ró.ti.co}{0}
\verb{neurótico}{}{}{}{}{}{Diz"-se do indivíduo que sofre de neurose.}{neu.ró.ti.co}{0}
\verb{neurotóxico}{cs/ ou /ss}{}{}{}{adj.}{Diz"-se da substância que afeta o sistema nervoso e o controle muscular.}{neu.ro.tó.xi.co}{0}
\verb{neurotransmissor}{ô}{Anat.}{}{}{s.m.}{Substância química produzida pelos neurônios que é responsável pela transmissão do impulso nervoso.}{neu.ro.trans.mis.sor}{0}
\verb{neutral}{}{}{"-ais}{}{adj.2g.}{Que não se posiciona.}{neu.tral}{0}
\verb{neutral}{}{}{"-ais}{}{}{Que julga com imparcialidade.}{neu.tral}{0}
\verb{neutral}{}{}{"-ais}{}{}{Que não se envolve com alguém ou algo.}{neu.tral}{0}
\verb{neutralidade}{}{}{}{}{s.f.}{Condição daquele que permanece neutro.}{neu.tra.li.da.de}{0}
\verb{neutralização}{}{}{"-ões}{}{s.f.}{Ato ou efeito de neutralizar.}{neu.tra.li.za.ção}{0}
\verb{neutralizar}{}{}{}{}{v.t.}{Fazer que pessoa ou coisa fique sem atividade; anular, inutilizar.}{neu.tra.li.zar}{\verboinum{1}}
\verb{neutro}{}{}{}{}{adj.}{Que não fica nem a favor nem contra em uma discussão.}{neu.tro}{0}
\verb{neutro}{}{}{}{}{}{Que não pertence a uma parte nem a outra.}{neu.tro}{0}
\verb{neutro}{}{}{}{}{}{Que não favorece nem a um nem a outro.}{neu.tro}{0}
\verb{nêutron}{}{}{}{}{s.m.}{Partícula de carga nula constituinte do núcleo do átomo.}{nêu.tron}{0}
\verb{nevada}{}{}{}{}{s.f.}{Queda ou formação de neve.}{ne.va.da}{0}
\verb{nevado}{}{}{}{}{adj.}{Coberto de neve. }{ne.va.do}{0}
\verb{nevado}{}{}{}{}{}{Que tem aparência de neve.}{ne.va.do}{0}
\verb{nevado}{}{Cul.}{}{}{}{Diz"-se de bolo coberto com clara de ovo batida até ficar branca como a neve.}{ne.va.do}{0}
\verb{nevar}{}{}{}{}{v.i.}{Cair neve.}{ne.var}{\verboinum{1}}
\verb{nevasca}{}{}{}{}{s.f.}{Tempestade de neve.}{ne.vas.ca}{0}
\verb{neve}{é}{}{}{}{s.f.}{Grande quantidade de flocos ou cristais de gelo que caem das nuvens.}{ne.ve}{0}
\verb{neviscar}{}{}{}{}{v.t.}{Nevar pouco.}{ne.vis.car}{\verboinum{2}}
\verb{nevo}{é}{Anat.}{}{}{s.m.}{Alteração localizada na pele, por excesso de pigmentação, de vasos ou de tecido conjuntivo e epidérmico.}{ne.vo}{0}
\verb{névoa}{}{}{}{}{s.f.}{Vapor d'água que sobe para a atmosfera, de intensidade menor que um nevoeiro.}{né.vo.a}{0}
\verb{nevoeiro}{ê}{}{}{}{s.m.}{Nebulosidade que se constitui de grande número de gotículas de água suspensas na camada mais baixa da atmosfera e que difere da nuvem apenas por estar mais perto da superfície terrestre.}{ne.vo.ei.ro}{0}
\verb{nevoento}{}{}{}{}{adj.}{Cheio de névoas; nebuloso.}{ne.vo.en.to}{0}
\verb{nevralgia}{}{Med.}{}{}{s.f.}{Dor que se estende ao longo do nervo.}{ne.vral.gi.a}{0}
\verb{nevrálgico}{}{}{}{}{adj.}{Relativo a nevralgia.}{ne.vrál.gi.co}{0}
\verb{nevrite}{}{Med.}{}{}{s.f.}{Inflamação de um nervo.}{ne.vri.te}{0}
\verb{nevrologia}{}{}{}{}{}{Var. de \textit{neurologia}.}{ne.vro.lo.gi.a}{0}
\verb{nevrose}{ó}{}{}{}{}{Var. de \textit{neurose}.}{ne.vro.se}{0}
\verb{nevrótico}{}{}{}{}{}{Var. de \textit{neurótico}.}{ne.vró.ti.co}{0}
\verb{newton}{}{Fís.}{}{}{s.m.}{Unidade de força do Sistema Internacional de unidades (\textsc{si}) que imprime uma aceleração de um metro por segundo ao quadrado a um corpo de um quilograma de massa, na direção e sentido da força.}{new.ton}{0}
\verb{nexo}{écs}{}{}{}{s.m.}{Junção entre duas ou mais coisas; ligação, vínculo.}{ne.xo}{0}
\verb{nexo}{écs}{}{}{}{}{Ligação entre situações, acontecimentos ou ideias; coerência.}{ne.xo}{0}
\verb{nhá}{}{}{}{}{s.f.}{Forma reduzida de \textit{sinhá}; senhora.}{nhá}{0}
%\verb{}{}{}{}{}{}{}{}{0}
%\verb{}{}{}{}{}{}{}{}{0}
\verb{nhambu}{}{}{}{}{}{Var. de \textit{inhambu}.}{nham.bu}{0}
\verb{nhandu}{}{Zool.}{}{}{s.m.}{Ave de grande porte, sem cauda, de cabeça e pescoço cinzentos e corpo branco; ema.}{nhan.du}{0}
\verb{nheengatu}{}{}{}{}{s.m.}{Língua desenvolvida a partir do tupinambá, falada ao longo de todo o vale amazônico brasileiro até a fronteira com o Peru, na Colômbia e na Venezuela; língua geral amazônica.}{nhe.en.ga.tu}{0}
\verb{nhenhenhém}{}{}{}{}{s.m.}{Resmungo.}{nhe.nhe.nhém}{0}
\verb{nhenhenhém}{}{}{}{}{}{Falatório interminável.}{nhe.nhe.nhém}{0}
\verb{nhô}{}{}{}{}{s.m.}{Forma reduzida de \textit{nhonhô}.}{nhô}{0}
\verb{nhonhô}{}{}{}{}{s.m.}{Tratamento que os escravos davam aos senhores; nhô.}{nho.nhô}{0}
\verb{nhoque}{ó}{Cul.}{}{}{s.m.}{Massa alimentícia típica da cozinha italiana, cortada em fragmentos arredondados e feita de farinha de trigo, batata, ovos e queijo.}{nho.que}{0}
\verb{nhoque}{ó}{Cul.}{}{}{}{Prato feito com essa massa cozida, molho de tomate e queijo parmesão ralado.}{nho.que}{0}
\verb{Ni}{}{Quím.}{}{}{}{Símb. do \textit{níquel}.}{Ni}{0}
\verb{nica}{}{}{}{}{s.f.}{Melindre exagerado; impertinência.}{ni.ca}{0}
\verb{nica}{}{}{}{}{}{Coisa supérflua; futilidade.}{ni.ca}{0}
\verb{nica}{}{}{}{}{}{Coisa sem importância; quantia insignificante; ninharia.}{ni.ca}{0}
\verb{nicaraguense}{}{}{}{}{adj.}{Relativo à Nicarágua.}{ni.ca.ra.guen.se}{0}
\verb{nicaraguense}{}{}{}{}{s.m.}{Indivíduo natural ou habitante desse país.}{ni.ca.ra.guen.se}{0}
\verb{nicho}{}{}{}{}{s.m.}{Vão em parede ou muro onde se colocam imagens.}{ni.cho}{0}
\verb{nicho}{}{}{}{}{}{Mercado especializado que geralmente oferece novas oportunidades de negócios.}{ni.cho}{0}
\verb{nicótico}{}{}{}{}{adj.}{Relativo a fumo; nicotínico.}{ni.có.ti.co}{0}
\verb{nicotina}{}{}{}{}{s.f.}{Substância encontrada nas folhas de tabaco, líquida, incolor e venenosa.}{ni.co.ti.na}{0}
\verb{nicotínico}{}{}{}{}{adj.}{Nicótico.}{ni.co.tí.ni.co}{0}
\verb{nidificar}{}{}{}{}{v.i.}{Fazer ninho; aninhar.}{ni.di.fi.car}{\verboinum{2}}
\verb{nife}{}{}{}{}{s.m.}{Núcleo central da Terra, supostamente constituído de ferro e níquel.}{ni.fe}{0}
\verb{nigeriano}{}{}{}{}{adj.}{Relativo ao rio Níger (África).}{ni.ge.ri.a.no}{0}
\verb{nigeriano}{}{}{}{}{}{Relativo à Nigéria (África Ocidental).}{ni.ge.ri.a.no}{0}
\verb{nigeriano}{}{}{}{}{s.m.}{Indivíduo natural ou habitante desse país.}{ni.ge.ri.a.no}{0}
\verb{nigromancia}{}{}{}{}{s.f.}{Necromancia.}{ni.gro.man.ci.a}{0}
\verb{nigromante}{}{}{}{}{s.2g.}{Necromante.}{ni.gro.man.te}{0}
\verb{niilismo}{}{}{}{}{s.m.}{Redução a nada; aniquilamento.}{ni.i.lis.mo}{0}
\verb{niilismo}{}{}{}{}{}{Ponto de vista que considera que as crenças e os valores tradicionais são infundados e que não há nenhum sentido ou utilidade na existência.}{ni.i.lis.mo}{0}
\verb{niilista}{}{}{}{}{adj.2g.}{Relativo ao niilismo.}{ni.i.lis.ta}{0}
\verb{niilista}{}{}{}{}{}{Diz"-se de quem adota essa doutrina ou ponto de vista.}{ni.i.lis.ta}{0}
\verb{nimbar}{}{}{}{}{v.t.}{Cercar de nimbo ou auréola; aureolar.}{nim.bar}{0}
\verb{nimbar}{}{Fig.}{}{}{}{Tornar sublime; enaltecer.}{nim.bar}{\verboinum{1}}
\verb{nimbo}{}{}{}{}{s.m.}{Círculo de luz que cinge a cabeça das imagens de Cristo e dos santos; auréola.}{nim.bo}{0}
\verb{nimbo}{}{}{}{}{}{Nuvem densa e cinzenta, de baixa altitude e contornos mal definidos, que facilmente se precipita em chuva ou neve.}{nim.bo}{0}
\verb{nímio}{}{}{}{}{adj.}{Que é excessivo, demasiado.}{ní.mio}{0}
\verb{ninar}{}{}{}{}{v.t.}{Fazer adormecer; acalentar; embalar.}{ni.nar}{\verboinum{1}}
\verb{ninfa}{}{Mit.}{}{}{s.f.}{Na mitologia grega, divindade dos rios, dos bosques e dos montes.}{nin.fa}{0}
\verb{ninfa}{}{Fig.}{}{}{}{Mulher jovem e formosa.}{nin.fa}{0}
\verb{ninfa}{}{Zool.}{}{}{}{Forma intermediária entre a larva e o inseto adulto.}{nin.fa}{0}
\verb{ninfeácea}{}{Bot.}{}{}{s.f.}{Planta aquática que tem folhas de haste longa e grandes flores. }{nin.fe.á.cea}{0}
\verb{ninfeta}{ê}{}{}{}{s.f.}{Menina adolescente voltada para o sexo ou que desperta desejo sexual.}{nin.fe.ta}{0}
\verb{ninfomania}{}{}{}{}{s.f.}{Desejo sexual exagerado nas mulheres.}{nin.fo.ma.ni.a}{0}
\verb{ninfomaníaca}{}{}{}{}{s.f.}{Mulher que tem desejo sexual exagerado.}{nin.fo.ma.ní.a.ca}{0}
\verb{ninfomaníaco}{}{}{}{}{adj.}{Relativo à ninfomania.  }{nin.fo.ma.ní.a.co}{0}
\verb{ninguém}{}{}{}{}{pron.}{Nenhuma pessoa.}{nin.guém}{0}
\verb{ninguém}{}{}{}{}{s.2g.}{Indivíduo de pouco ou nenhum valor, merecimento, importância.}{nin.guém}{0}
\verb{ninhada}{}{}{}{}{s.f.}{Conjunto das avezinhas contidas em um ninho.}{ni.nha.da}{0}
\verb{ninhada}{}{}{}{}{}{Total de filhotes que a fêmea do animal pariu de uma só vez.}{ni.nha.da}{0}
\verb{ninharia}{}{}{}{}{s.f.}{Coisa muito pequena, insignificante.}{ni.nha.ri.a}{0}
\verb{ninho}{}{}{}{}{s.m.}{Estrutura construída pelas aves para a postura de ovos e a criação dos filhotes.}{ni.nho}{0}
\verb{ninho}{}{}{}{}{}{A casa de habitação; lar.}{ni.nho}{0}
\verb{nióbio}{}{Quím.}{}{}{s.m.}{Elemento químico metálico, branco, brilhante, utilizado em ligas com o ferro e o níquel. \elemento{41}{92.90638}{Nb}.}{ni.ó.bio}{0}
\verb{niple}{}{}{}{}{s.m.}{Peça cilíndrica com rosca na parte externa das duas extremidades, usada para unir dois tubos.}{ni.ple}{0}
\verb{nipônico}{}{}{}{}{adj. e s.m.  }{Japonês.}{ni.pô.ni.co}{0}
\verb{níquel}{}{Quím.}{"-eis}{}{s.m.}{Elemento químico metálico, branco"-prateado, denso, dúctil, maleável, bom condutor de calor e eletricidade, usado em ligas e na indústria de vidro e cerâmica. \elemento{28}{58.6934}{Ni}.}{ní.quel}{0}
\verb{níquel}{}{}{"-eis}{}{}{Moeda feita com esse metal.}{ní.quel}{0}
\verb{níquel}{}{Fig.}{"-eis}{}{}{Dinheiro.}{ní.quel}{0}
\verb{niquelagem}{}{Quím.}{"-ens}{}{s.f.}{Ação de niquelar,  de recobrir uma superfície metálica por uma película de níquel, para proteger uma peça, servindo geralmente de base para diversos tipos de acabamentos.}{ni.que.la.gem}{0}
\verb{niquelar}{}{Quím.}{}{}{v.t.}{Cobrir com uma camada de níquel.}{ni.que.lar}{0}
\verb{niquelar}{}{}{}{}{}{Dar a aparência de níquel a.}{ni.que.lar}{\verboinum{1}}
\verb{nirvana}{}{}{}{}{s.m.}{No budismo, estado de felicidade plena, obtido pela meditação.}{nir.va.na}{0}
\verb{nissei}{}{}{}{}{adj.2g.}{Diz"-se de indivíduo que é filho de pais japoneses nascido na América.}{nis.sei}{0}
\verb{nisso}{}{}{}{}{}{Contração da preposição \textit{em} com o pronome demonstrativo \textit{isso}.}{nis.so}{0}
\verb{nisto}{}{}{}{}{}{Contração da preposição \textit{em} com o pronome \textit{isto}.  }{nis.to}{0}
\verb{nitente}{}{}{}{}{adj.2g.}{Que brilha; resplandecente.}{ni.ten.te}{0}
\verb{niteroiense}{}{}{}{}{adj.2g.}{Relativo a Niterói (\textsc{rj}).}{ni.te.roi.en.se}{0}
\verb{niteroiense}{}{}{}{}{s.2g.}{Indivíduo natural ou habitante dessa cidade.}{ni.te.roi.en.se}{0}
\verb{nitidez}{ê}{}{}{}{s.f.}{Qualidade do que é nítido; clareza.}{ni.ti.dez}{0}
\verb{nítido}{}{}{}{}{adj.}{Em que há clareza, limpidez, transparência.}{ní.ti.do}{0}
\verb{nítido}{}{}{}{}{}{Que é fácil de entender; compreensível.}{ní.ti.do}{0}
\verb{nitrato}{}{Quím.}{}{}{s.m.}{Qualquer sal derivado do ácido nítrico.}{ni.tra.to}{0}
\verb{nítrico}{}{Quím.}{}{}{adj.}{Relativo ao nitro.}{ní.tri.co}{0}
\verb{nítrico}{}{Quím.}{}{}{}{Diz"-se de um ácido muito reativo, formado  pela combinação de um átomo de hidrogênio, um de nitrogênio e três de oxigênio, muito usado na indústria; azótico.}{ní.tri.co}{0}
\verb{nitrido}{}{}{}{}{s.m.}{Ato ou efeito de nitrir, de relinchar; rincho.}{ni.tri.do}{0}
\verb{nitrir}{}{}{}{}{v.i.}{Emitir nitridos ou relinchos; relinchar.}{ni.trir}{\verboinum{18}}
\verb{nitrito}{}{Fís. e Quím.}{}{}{s.m.}{Qualquer sal derivado do ácido nitroso.}{ni.tri.to}{0}
\verb{nitro}{}{Quím.}{}{}{s.m.}{Designação comum do nitrato de potássio; salitre.}{ni.tro}{0}
\verb{nitrogenado}{}{}{}{}{adj.}{Que contém nitrogênio.}{ni.tro.ge.na.do}{0}
\verb{nitrogênio}{}{Quím.}{}{}{s.m.}{Elemento químico do grupo dos não metais, gasoso, incolor, inodoro, abundante na atmosfera, usado na indústria de explosivos e em indústrias que necessitam de atmosfera inerte; azoto. \elemento{7}{14.00674}{N}.}{ni.tro.gê.nio}{0}
\verb{nitroglicerina}{}{Quím.}{}{}{s.f.}{Combinação dos ácidos nítrico e sulfúrico, líquido, oleoso e altamente explosivo.}{ni.tro.gli.ce.ri.na}{0}
\verb{nitroso}{}{Quím.}{}{}{adj.}{Diz"-se de ácido que possui em sua molécula  um átomo de hidrogênio, um de nitrogênio e dois de oxigênio, usado na preparação de corantes.}{ni.tro.so}{0}
\verb{nível}{}{}{"-eis}{}{s.m.}{Instrumento para verificar a horizontalidade de um plano.}{ní.vel}{0}
\verb{nível}{}{}{"-eis}{}{}{Elevação relativa de uma linha ou de um plano horizontal.}{ní.vel}{0}
\verb{nível}{}{}{"-eis}{}{}{Cada uma das subdivisões do ensino escolar brasileiro.}{ní.vel}{0}
\verb{nível}{}{Fig.}{"-eis}{}{}{Qualidade, padrão; gabarito.}{ní.vel}{0}
\verb{nível}{}{Fig.}{"-eis}{}{}{Situação, estado.}{ní.vel}{0}
\verb{nivelação}{}{}{"-ões}{}{s.f.}{Ato ou efeito de nivelar; nivelamento, equiparação, igualação.}{ni.ve.la.ção}{0}
\verb{nivelamento}{}{}{}{}{s.m.}{Verificação da horizontalidade de uma superfície.}{ni.ve.la.men.to}{0}
\verb{nivelamento}{}{}{}{}{}{Eliminação de desníveis.}{ni.ve.la.men.to}{0}
\verb{nivelamento}{}{}{}{}{}{Equiparação.}{ni.ve.la.men.to}{0}
\verb{nivelar}{}{}{}{}{v.t.}{Medir ou verificar a horizontalidade de uma superfície com um nível.}{ni.ve.lar}{0}
\verb{nivelar}{}{}{}{}{}{Tornar horizontal; colocar no mesmo nível; aplainar.}{ni.ve.lar}{0}
\verb{nivelar}{}{}{}{}{}{Tornar igual; igualar.}{ni.ve.lar}{\verboinum{1}}
\verb{níveo}{}{}{}{}{adj.}{Relativo a neve.}{ní.veo}{0}
\verb{níveo}{}{}{}{}{}{Que tem a cor da neve; branco, nevado.}{ní.veo}{0}
\verb{NNE}{}{}{}{}{}{Abrev. de \textit{nor"-nordeste}.}{n.n.e.}{0}
\verb{NNO}{}{}{}{}{}{Abrev. de \textit{nor"-noroeste}.}{n.n.o.}{0}
\verb{NNW}{}{}{}{}{}{Abrev. de \textit{nor"-noroeste}.}{n.n.w.}{0}
\verb{No}{}{Quím.}{}{}{}{Símb. do \textit{nobélio}.}{No}{0}
\verb{no}{}{}{}{}{}{Contração da preposição \textit{em} com o artigo \textit{o}. (\textit{Gosto de viajar no banco de trás dos automóveis.})}{no}{0}
\verb{NO}{}{}{}{}{}{Abrev. de \textit{noroeste}.}{n.o.}{0}
\verb{nó}{}{}{}{}{s.m.}{Entrelaçamento de duas extremidades a fim de uni"-las, marcá"-las ou encurtá"-las.}{nó}{0}
\verb{nó}{}{}{}{}{}{Cada um dos pontos de junção dos ramos de uma árvore.}{nó}{0}
\verb{nó}{}{}{}{}{}{Unidade de velocidade de uma embarcação.}{nó}{0}
\verb{no}{}{}{}{}{}{Forma que o pronome \textit{o} assume depois de formas verbais terminadas em ditongos nasais, como -ão, -am, õe, -em. (\textit{Encontraram"-no já sem forças.})}{no}{0}
\verb{nobélio}{}{Quím.}{}{}{s.m.}{Elemento químico radioativo, do grupo dos actinídeos, obtido artificialmente. \elemento{102}{(259)}{No}.}{no.bé.lio}{0}
\verb{nobiliário}{}{}{}{}{s.m.}{Registro das famílias nobres de um país.}{no.bi.li.á.rio}{0}
\verb{nobiliário}{}{}{}{}{adj.}{Relativo à nobreza.}{no.bi.li.á.rio}{0}
\verb{nobiliarquia}{}{}{}{}{s.f.}{Livro que trata da origem, brasões e serviços da nobreza.}{no.bi.li.ar.qui.a}{0}
\verb{nobilitação}{}{}{"-ões}{}{s.f.}{Ato ou efeito de nobilitar; enobrecimento.}{no.bi.li.ta.ção}{0}
\verb{nobilitar}{}{}{}{}{v.t.}{Tornar nobre; engrandecer.}{no.bi.li.tar}{0}
\verb{nobilitar}{}{}{}{}{}{Enaltecer, celebrar.}{no.bi.li.tar}{\verboinum{1}}
\verb{nobre}{ó}{}{}{}{adj.2g.}{Que tem boas qualidades; bom, generoso.}{no.bre}{0}
\verb{nobre}{ó}{}{}{}{}{Que mostra essas boas qualidades; digno, louvável.}{no.bre}{0}
\verb{nobre}{ó}{}{}{}{}{De boa origem; excelente. }{no.bre}{0}
\verb{nobre}{ó}{}{}{}{}{Que tem um título que passa de pai para filho dado pelo rei; fidalgo.}{no.bre}{0}
\verb{nobreza}{ê}{}{}{}{s.f.}{Qualidade de ser nobre.}{no.bre.za}{0}
\verb{nobreza}{ê}{}{}{}{}{Classe a que pertencem os nobres; fidalguia.}{no.bre.za}{0}
\verb{noção}{}{}{"-ões}{}{s.f.}{Conhecimento sobre determinado assunto; ideia.}{no.ção}{0}
\verb{noção}{}{}{"-ões}{}{}{Reconhecimento do que se deve fazer; senso, sentido.}{no.ção}{0}
\verb{nocaute}{}{}{}{}{s.m.}{No boxe, momento em que o lutador é posto fora de combate.}{no.cau.te}{0}
\verb{nocautear}{}{}{}{}{v.t.}{Derrotar um lutador de forma a deixá"-lo sem condições de continuar.}{no.cau.te.ar}{\verboinum{4}}
\verb{nocividade}{}{}{}{}{s.f.}{Característica do que é nocivo.}{no.ci.vi.da.de}{0}
\verb{nocivo}{}{}{}{}{adj.}{Que faz mal; prejudicial.}{no.ci.vo}{0}
\verb{noctâmbulo}{}{}{}{}{adj.}{Diz"-se de quem é sonâmbulo.}{noc.tâm.bu.lo}{0}
\verb{noctâmbulo}{}{}{}{}{}{Que anda à noite.}{noc.tâm.bu.lo}{0}
\verb{noctívago}{}{}{}{}{adj.}{Que tem o costume de sair à noite.}{noc.tí.va.go}{0}
\verb{nodal}{}{}{"-ais}{}{adj.2g.}{Relativo a nó.}{no.dal}{0}
\verb{nó"-de"-adão}{}{}{nós"-de"-adão}{}{s.m.}{A saliência da cartilagem tireoide.}{nó"-de"-a.dão}{0}
\verb{nodo}{ó}{Med.}{}{}{s.m.}{Pequeno tumor; saliência, protuberância.}{no.do}{0}
\verb{nódoa}{}{}{}{}{s.f.}{Sinal deixado por alguma coisa que suja; mácula, mancha.}{nó.do.a}{0}
\verb{nodoso}{ô}{}{"-osos ⟨ó⟩}{"-osa ⟨ó⟩}{adj.}{Que tem nós ou saliências; proeminente.}{no.do.so}{0}
\verb{nódulo}{}{}{}{}{}{Pequena protuberância em tecido animal ou vegetal.}{nó.du.lo}{0}
\verb{nódulo}{}{}{}{}{s.m.}{Pequeno nó.}{nó.du.lo}{0}
\verb{nogado}{}{Cul.}{}{}{s.m.}{Doce de nozes ou de amêndoas misturadas com caramelo ou mel, vendido, em geral, em tabletes ou como recheio de bombom, de chocolate etc.; nugá.  }{no.ga.do}{0}
\verb{nogal}{}{}{"-ais}{}{s.m.}{Nogueiral.}{no.gal}{0}
\verb{nogueira}{ê}{Bot.}{}{}{s.f.}{Árvore europeia, de flores arrumadas em longas espigas pendentes, e cujos frutos comestíveis, as nozes, são muito apreciados, e que possui madeira é boa para móveis.}{no.guei.ra}{0}
\verb{nogueiral}{}{}{"-ais}{}{s.m.}{Coletivo de nogueira.}{no.guei.ral}{0}
\verb{noitada}{}{}{}{}{s.f.}{Espaço ou duração de uma noite; noite.}{noi.ta.da}{0}
\verb{noitada}{}{}{}{}{}{Folia ou divertimento que dura a noite inteira ou grande parte da noite.}{noi.ta.da}{0}
\verb{noite}{}{}{}{}{s.f.}{Período de tempo em que o Sol ilumina o outro lado da Terra.}{noi.te}{0}
\verb{noite}{}{}{}{}{}{Horário em que está escuro por falta de luz solar, e em que geralmente as pessoas descansam ou dormem.}{noi.te}{0}
\verb{noitibó}{}{Zool.}{}{}{s.m.}{Curiango.}{noi.ti.bó}{0}
\verb{noitibó}{}{Fig.}{}{}{}{Indivíduo pouco sociável, ou que tem o hábito de só sair à noite.}{noi.ti.bó}{0}
\verb{noitinha}{}{}{}{}{s.f.}{Chegada da noite; anoitecer.}{noi.ti.nha}{0}
\verb{noiva}{}{}{}{}{s.f.}{Mulher prometida em casamento.}{noi.va}{0}
\verb{noivado}{}{}{}{}{s.m.}{Compromisso de casamento.}{noi.va.do}{0}
\verb{noivado}{}{}{}{}{}{Espaço de tempo entre esse compromisso e o casamento.}{noi.va.do}{0}
\verb{noivar}{}{}{}{}{v.i.}{Assumir publicamente o compromisso de se casar com determinada pessoa.}{noi.var}{\verboinum{1}}
\verb{noivo}{}{}{}{}{s.m.}{Homem que vai casar, que fez a promessa solene de casamento.}{noi.vo}{0}
\verb{noivo}{}{}{}{}{s.m.pl.}{O casal no dia de seu casamento.}{noi.vo}{0}
\verb{nojento}{}{}{}{}{adj.}{Que causa nojo; repugnante, asqueroso.}{no.jen.to}{0}
\verb{nojento}{}{}{}{}{}{Diz"-se de indivíduo que sente nojo de tudo, que facilmente se enoja.}{no.jen.to}{0}
\verb{nojo}{ô}{}{}{}{s.m.}{Mal"-estar causado por alguma coisa que desagrada muito; náusea, enjoo.}{no.jo}{0}
\verb{nojo}{ô}{}{}{}{}{Sentimento de revolta contra alguma ação maldosa; repulsa, repugnância.}{no.jo}{0}
\verb{nolição}{}{}{"-ões}{}{s.f.}{Ato de negar, de recusar, de não querer; recusa.}{no.li.ção}{0}
\verb{nômada}{}{}{}{}{adj.2g.}{Nômade.}{nô.ma.da}{0}
\verb{nômade}{}{}{}{}{adj.2g.}{Que se muda de um lugar para outro, sem ter moradia fixa.}{nô.ma.de}{0}
\verb{nomadismo}{}{}{}{}{s.m.}{Modo de vida dos nômades.}{no.ma.dis.mo}{0}
\verb{nome}{}{}{}{}{s.m.}{Termo com o qual se designam pessoas, coisas ou animais.}{no.me}{0}
\verb{nome}{}{}{}{}{}{Bom conceito; reputação.}{no.me}{0}
\verb{nome}{}{}{}{}{}{Apelido.}{no.me}{0}
\verb{nomeação}{}{}{"-ões}{}{s.f.}{Ato ou efeito de nomear ou ser nomeado.}{no.me.a.ção}{0}
\verb{nomeação}{}{}{"-ões}{}{}{Ato formal de atribuir cargo público a alguém, efetuado por autoridade competente.}{no.me.a.ção}{0}
\verb{nomeação}{}{}{"-ões}{}{}{Designação de alguém para cargo ou função privada.}{no.me.a.ção}{0}
\verb{nomeada}{}{}{}{}{s.f.}{Estado de quem é muito conhecido; fama, reputação, renome.}{no.me.a.da}{0}
\verb{nomear}{}{}{}{}{v.t.}{Dizer o nome de pessoa ou coisa; citar.}{no.me.ar}{0}
\verb{nomear}{}{}{}{}{}{Dar um cargo a alguém.}{no.me.ar}{\verboinum{4}}
\verb{nomenclatura}{}{}{}{}{s.f.}{Conjunto de termos próprios de uma ciência ou arte; terminologia.}{no.men.cla.tu.ra}{0}
\verb{nominação}{}{}{"-ões}{}{s.f.}{Figura retórica que consiste em dar uma denominação a algo que não tenha nome.}{no.mi.na.ção}{0}
\verb{nominal}{}{}{"-ais}{}{adj.2g.}{Relativo a nome.}{no.mi.nal}{0}
\verb{nominal}{}{}{"-ais}{}{}{Que só existe como nome, que não é real.}{no.mi.nal}{0}
\verb{nominata}{}{}{}{}{s.f.}{Lista ou relação de nomes.}{no.mi.na.ta}{0}
\verb{nominativo}{}{}{}{}{adj.}{Que denomina; nominal.}{no.mi.na.ti.vo}{0}
\verb{nominativo}{}{}{}{}{}{Que traz o nome do proprietário ou favorecido.}{no.mi.na.ti.vo}{0}
\verb{nominativo}{}{Gram.}{}{}{s.m.}{Um dos casos sintáticos morfologicamente marcados de algumas línguas, como o latim.}{no.mi.na.ti.vo}{0}
\verb{nonada}{}{}{}{}{s.f.}{Coisa sem préstimo ou valor; insignificância, ninharia.}{no.na.da}{0}
\verb{nonagenário}{}{}{}{}{adj.}{Que está na faixa dos noventa anos de idade.}{no.na.ge.ná.rio}{0}
\verb{nonagésimo}{}{}{}{}{num.}{Ordinal e fracionário correspondente a 90.        }{no.na.gé.si.mo}{0}
\verb{nonato}{}{}{}{}{adj.}{Diz"-se de criança que não nasceu de parto natural, mas por operação cesariana.}{no.na.to}{0}
\verb{nonato}{}{}{}{}{}{Diz"-se de animal tirado do ventre da mãe depois que esta morreu.}{no.na.to}{0}
\verb{nongentésimo}{}{}{}{}{num.}{Noningentésimo.        }{non.gen.té.si.mo}{0}
\verb{noningentésimo}{}{}{}{}{num.}{Ordinal e fracionário correspondente a 900; nongentésimo.}{no.nin.gen.té.si.mo}{0}
\verb{nono}{}{}{}{}{num.}{Ordinal e fracionário correspondente a 9.}{no.no}{0}
\verb{nonsense}{}{}{}{}{s.m.}{Palavra ou ação sem sentido, contrária ao bom"-senso.}{\textit{nonsense}}{0}
\verb{nônuplo}{}{}{}{}{s.m.}{Quantidade nove vezes maior que outra.}{nô.nu.plo}{0}
\verb{nora}{ó}{}{}{}{s.f.}{Designativo da esposa em relação ao pai ou à mãe de seu marido.}{no.ra}{0}
\verb{nordeste}{é}{}{}{}{s.m.}{Ponto do horizonte situado a igual distância do norte e do este. Abrev. \textsc{n.e}.}{nor.des.te}{0}
\verb{nordeste}{é}{}{}{}{}{O vento que sopra dessa direção.}{nor.des.te}{0}
\verb{nordeste}{é}{}{}{}{}{Região geográfica e administrativa do Brasil que inclui os estados de Alagoas, Bahia, Ceará, Maranhão, Paraíba, Pernambuco, Piauí, Sergipe e Rio Grande do Norte.}{nor.des.te}{0}
\verb{nordestino}{}{}{}{}{adj.}{Relativo à região Nordeste do Brasil.}{nor.des.ti.no}{0}
\verb{nordestino}{}{}{}{}{s.m.}{Indivíduo natural ou habitante dessa região.}{nor.des.ti.no}{0}
\verb{nórdico}{}{}{}{}{adj.}{Relativo aos países do norte da Europa (Dinamarca, Finlândia, Islândia e Suécia).}{nór.di.co}{0}
\verb{nórdico}{}{}{}{}{s.m.}{Indivíduo natural ou habitante desses países.}{nór.di.co}{0}
\verb{nórdico}{}{}{}{}{adj.}{Diz"-se de língua indo"-europeia de ramo germânico, que foi falada na Escandinávia e deu origem às línguas escandinavas modernas (norueguês, dinamarquês, islandês, sueco).}{nór.di.co}{0}
\verb{norma}{ó}{}{}{}{s.f.}{Aquilo que se estabelece como orientação a ser seguida para se fazer alguma coisa; modelo, padrão.}{nor.ma}{0}
\verb{norma}{ó}{}{}{}{}{Aquilo que serve para disciplinar o comportamento das pessoas; princípio, regra.}{nor.ma}{0}
\verb{normal}{}{}{"-ais}{}{adj.2g.}{Que obedece a norma, regra.}{nor.mal}{0}
\verb{normal}{}{}{"-ais}{}{}{Natural, habitual.}{nor.mal}{0}
\verb{normalidade}{}{}{}{}{s.f.}{Qualidade ou estado do que é normal.}{nor.ma.li.da.de}{0}
\verb{normalista}{}{}{}{}{adj.2g.}{Que frequenta um curso de uma escola normal.}{nor.ma.lis.ta}{0}
\verb{normalização}{}{}{"-ões}{}{s.f.}{Ato ou efeito de normalizar.}{nor.ma.li.za.ção}{0}
\verb{normalizar}{}{}{}{}{v.t.}{Fazer alguma coisa ficar normal; regularizar.}{nor.ma.li.zar}{\verboinum{1}}
\verb{normando}{}{}{}{}{adj.}{Relativo à Normandia (França).}{nor.man.do}{0}
\verb{normando}{}{}{}{}{}{Diz"-se do dialeto do francês falado nessa região.}{nor.man.do}{0}
\verb{normando}{}{}{}{}{s.m.}{Indivíduo natural ou habitante da Normandia.}{nor.man.do}{0}
\verb{normativo}{}{}{}{}{adj.}{Relativo a norma.}{nor.ma.ti.vo}{0}
\verb{normativo}{}{}{}{}{}{Que serve de norma.}{nor.ma.ti.vo}{0}
\verb{normativo}{}{}{}{}{}{Que estabelece normas ou padrões de comportamento; que determina o que é correto, bom.}{nor.ma.ti.vo}{0}
\verb{normatizar}{}{}{}{}{v.t.}{Criar normas; normalizar.}{nor.ma.ti.zar}{\verboinum{1}}
\verb{normógrafo}{}{}{}{}{s.m.}{Lâmina de celuloide em que se vazam as letras do alfabeto para servirem de molde nos desenhos e letreiros.}{nor.mó.gra.fo}{0}
\verb{nor"-nordeste}{ó\ldots{}é}{}{}{}{s.m.}{Ponto do horizonte a meia distância angular do norte e do nordeste.}{nor"-nor.des.te}{0}
\verb{nor"-nordeste}{ó\ldots{}é}{}{}{}{}{Vento que sopra desse rumo.}{nor"-nor.des.te}{0}
\verb{nor"-noroeste}{ó\ldots{}é}{Astron.}{}{}{s.m.}{Ponto do horizonte a meia distância angular do norte e do noroeste.}{nor"-no.ro.es.te}{0}
\verb{nor"-noroeste}{ó\ldots{}é}{}{}{}{}{Vento que sopra desse rumo.}{nor"-no.ro.es.te}{0}
\verb{noroeste}{é}{Astron.}{}{}{s.m.}{Ponto do horizonte situado a quarenta e cinco graus do norte e do oeste.}{no.ro.es.te}{0}
\verb{noroeste}{é}{}{}{}{}{Vento que sopra desse ponto.}{no.ro.es.te}{0}
\verb{nortada}{}{}{}{}{s.f.}{Vento frio que sopra do norte.}{nor.ta.da}{0}
\verb{norte}{ó}{}{}{}{s.m.}{Ponto cardeal que fica em frente do observador que dá a direita ao lado onde nasce o sol. Abrev. \textsc{n}.}{nor.te}{0}
\verb{norte}{ó}{}{}{}{}{O vento que sopra dessa direção.}{nor.te}{0}
\verb{norte}{ó}{}{}{}{}{Região ou regiões que ficam nessa direção.}{nor.te}{0}
\verb{norte}{ó}{}{}{}{}{Rumo, direção.}{nor.te}{0}
\verb{norte}{ó}{}{}{}{}{Região geográfica e administrativa do Brasil que inclui os estados do Acre, Amazonas, Amapá, Pará, Rondônia, Roraima, e Tocantins.}{nor.te}{0}
\verb{norte}{ó}{}{}{}{}{O polo norte.}{nor.te}{0}
\verb{norteador}{ô}{}{}{}{adj.}{Que norteia; orientador.}{nor.te.a.dor}{0}
\verb{norte"-americano}{ó}{}{norte"-americanos ⟨ó⟩}{}{adj.}{Relativo aos Estados Unidos.}{nor.te"-a.me.ri.ca.no}{0}
\verb{norte"-americano}{ó}{}{norte"-americanos ⟨ó⟩}{}{s.2g.}{Indivíduo natural ou habitante desse país.}{nor.te"-a.me.ri.ca.no}{0}
\verb{nortear}{}{}{}{}{v.t.}{Indicar a alguém a direção a tomar; dirigir, guiar.}{nor.te.ar}{0}
\verb{nortear}{}{}{}{}{}{Dar conselho a alguém; aconselhar, esclarecer}{nor.te.ar}{\verboinum{4}}
\verb{norte"-coreano}{ó}{}{norte"-coreanos ⟨ó⟩}{}{adj.}{Relativo à Coreia do Norte.}{nor.te"-co.re.a.no}{0}
\verb{norte"-coreano}{ó}{}{norte"-coreanos ⟨ó⟩}{}{s.m.}{Indivíduo natural ou habitante desse país.}{nor.te"-co.re.a.no}{0}
\verb{norte"-rio"-grandense}{ó}{}{norte"-rio"-grandenses ⟨ó⟩}{}{adj.2g.}{Relativo ao Rio Grande do Norte; rio"-grandense"-do"-norte; potiguar.}{nor.te"-ri.o"-gran.den.se}{0}
\verb{norte"-rio"-grandense}{ó}{}{norte"-rio"-grandenses ⟨ó⟩}{}{s.2g.}{Indivíduo natural ou habitante desse estado. }{nor.te"-ri.o"-gran.den.se}{0}
\verb{nortista}{}{}{}{}{adj.2g.}{Relativo à região ou ao conjunto de regiões que se situa a norte.}{nor.tis.ta}{0}
\verb{nortista}{}{}{}{}{s.2g.}{Indivíduo natural ou habitante dessa região.}{nor.tis.ta}{0}
\verb{noruega}{é}{}{}{}{adj.}{Diz"-se da terra fresca e úmida na encosta de uma montanha e que recebe pouco sol.}{no.ru.e.ga}{0}
\verb{norueguês}{}{}{}{}{adj.}{Relativo à Noruega.}{no.ru.e.guês}{0}
\verb{norueguês}{}{}{}{}{s.m.}{Indivíduo natural ou habitante desse país.}{no.ru.e.guês}{0}
\verb{norueguês}{}{}{}{}{}{A língua falada pelos noruegueses.}{no.ru.e.guês}{0}
\verb{nos}{}{}{}{}{pron.}{Forma oblíqua do pronome \textit{nós}; a nós, para nós. (\textit{A diretora esperou"-nos por um longo tempo, até que terminássemos nossa lição.})}{nos}{0}
\verb{nós}{}{}{}{}{pron.}{Eu e os outros. (\textit{Nós estamos fazendo uma pesquisa de geografia.})}{nós}{0}
\verb{nosocômio}{}{}{}{}{s.m.}{Estabelecimento em que os doentes ficam para se tratar; hospital.}{no.so.cô.mio}{0}
\verb{nosofobia}{}{}{}{}{s.f.}{Medo de adoecer.}{no.so.fo.bi.a}{0}
\verb{nosomania}{}{}{}{}{s.f.}{Hipocondria.}{no.so.ma.ni.a}{0}
\verb{nossa}{ó}{}{}{}{interj.}{Palavra que expressa espanto ou admiração.}{nos.sa}{0}
\verb{nosso}{ó}{}{}{}{pron.}{O que nos pertence.}{nos.so}{0}
\verb{nostalgia}{}{}{}{}{s.f.}{Saudades de algo, de um estado, de uma forma de existência que se deixou de ter, desejo de voltar ao passado.}{nos.tal.gi.a}{0}
\verb{nostalgia}{}{}{}{}{}{Melancolia profunda causada pelo afastamento da terra natal.}{nos.tal.gi.a}{0}
\verb{nostálgico}{}{}{}{}{adj.}{Em que há nostalgia; saudoso.}{nos.tál.gi.co}{0}
\verb{nota}{ó}{}{}{}{s.f.}{Comunicação escrita para alguma finalidade.}{no.ta}{0}
\verb{nota}{ó}{}{}{}{}{Anotação de aula.}{no.ta}{0}
\verb{nota}{ó}{}{}{}{}{Cada um dos valores com que o professor classifica o desempenho escolar dos alunos.}{no.ta}{0}
\verb{nota}{ó}{}{}{}{}{Cada som musical tocado ou escrito.}{no.ta}{0}
\verb{nota}{ó}{}{}{}{}{Dinheiro em papel; cédula.}{no.ta}{0}
\verb{notabilidade}{}{}{}{}{s.f.}{Qualidade do que é notável, digno de atenção.}{no.ta.bi.li.da.de}{0}
\verb{notabilidade}{}{}{}{}{}{Pessoa ilustre; notável.}{no.ta.bi.li.da.de}{0}
\verb{notabilizar}{}{}{}{}{v.t.}{Tornar notável.}{no.ta.bi.li.zar}{\verboinum{1}}
\verb{notação}{}{}{"-ões}{}{s.f.}{Ato ou efeito de notar, de representar por meio de símbolos ou caracteres.}{no.ta.ção}{0}
\verb{notadamente}{}{}{}{}{adv.}{De maneira especial; especialmente.}{no.ta.da.men.te}{0}
\verb{notar}{}{}{}{}{v.t.}{Tomar conhecimento de uma pessoa ou coisa; observar, perceber, reparar.}{no.tar}{\verboinum{1}}
\verb{notariado}{}{}{}{}{s.m.}{Ofício ou funções de notário ou de tabelião.}{no.ta.ri.a.do}{0}
\verb{notário}{}{}{}{}{s.m.}{Escrivão público que recebe e redige escrituras, atos e contratos; tabelião.}{no.tá.rio}{0}
\verb{notável}{}{}{"-eis}{}{adj.2g.}{Que é muito conhecido e admirado; eminente, ilustre.}{no.tá.vel}{0}
\verb{notável}{}{}{"-eis}{}{}{Que merece ser notado; importante.}{no.tá.vel}{0}
\verb{notebook}{}{}{}{}{s.m.}{Computador de dimensões próximas às de um livro de tamanho médio.}{\textit{notebook}}{0}
\verb{notícia}{}{}{}{}{s.f.}{Informação ou conhecimento sobre pessoa ou coisa.}{no.tí.cia}{0}
\verb{notícia}{}{}{}{}{}{Fato comentado em jornais.}{no.tí.cia}{0}
\verb{notícia}{}{}{}{}{}{Novidade.}{no.tí.cia}{0}
\verb{noticiar}{}{}{}{}{v.t.}{Dar notícia; comunicar, divulgar.}{no.ti.ci.ar}{\verboinum{6}}
\verb{noticiário}{}{}{}{}{s.m.}{Conjunto de notícias dadas por um meio de comunicação.}{no.ti.ci.á.rio}{0}
\verb{noticiário}{}{}{}{}{}{Programa em que são dadas essas notícias.}{no.ti.ci.á.rio}{0}
\verb{noticiarista}{}{}{}{}{s.2g.}{Indivíduo que noticia, que dá notícias.}{no.ti.ci.a.ris.ta}{0}
\verb{noticiarista}{}{}{}{}{}{Redator de notícias.}{no.ti.ci.a.ris.ta}{0}
\verb{noticioso}{ô}{}{"-osos ⟨ó⟩}{"-osa ⟨ó⟩}{adj.}{Relativo a ou que contém notícia.}{no.ti.ci.o.so}{0}
\verb{noticioso}{ô}{}{"-osos ⟨ó⟩}{"-osa ⟨ó⟩}{}{Que fornece grande número de notícias.}{no.ti.ci.o.so}{0}
\verb{noticioso}{ô}{Bras.}{"-osos ⟨ó⟩}{"-osa ⟨ó⟩}{s.m.}{Programa radiofônico ou televisivo de transmissão de notícias.}{no.ti.ci.o.so}{0}
\verb{notificação}{}{}{"-ões}{}{s.f.}{Ato ou efeito de notificar; comunicação.}{no.ti.fi.ca.ção}{0}
\verb{notificação}{}{Jur.}{"-ões}{}{}{Ordem judicial para que se faça ou não alguma coisa; intimação.}{no.ti.fi.ca.ção}{0}
\verb{notificar}{}{}{}{}{v.t.}{Fazer uma pessoa saber de alguma coisa; comunicar, noticiar, inteirar.}{no.ti.fi.car}{0}
\verb{notificar}{}{}{}{}{}{Fazer uma pessoa saber de alguma coisa por força de uma decisão da Justiça.}{no.ti.fi.car}{\verboinum{2}}
\verb{notívago}{}{}{}{}{adj.}{Que tem o costume de sair à noite.}{no.tí.va.go}{0}
\verb{noto}{ó}{}{}{}{s.m.}{O vento sul.}{no.to}{0}
\verb{notoriedade}{}{}{}{}{s.f.}{Qualidade de notório; fama, publicidade.}{no.to.ri.e.da.de}{0}
\verb{notório}{}{}{}{}{adj.}{Conhecido por todos; público.}{no.tó.rio}{0}
\verb{nótula}{}{}{}{}{s.f.}{Pequena nota.}{nó.tu.la}{0}
\verb{nótula}{}{}{}{}{}{Pequeno comentário.}{nó.tu.la}{0}
\verb{noturno}{}{}{}{}{adj.}{Que acontece à noite.}{no.tur.no}{0}
\verb{noturno}{}{}{}{}{}{Que tem o costume de sair à noite; notívago.}{no.tur.no}{0}
\verb{noutada}{}{}{}{}{}{Var. de \textit{noitada}.}{nou.ta.da}{0}
\verb{noute}{}{}{}{}{}{Var. de \textit{noite}.}{nou.te}{0}
\verb{noutro}{}{}{}{}{}{Combinação da preposição \textit{em} com o pronome indefinido \textit{outro}; em alguma pessoa ou coisa determinada.}{nou.tro}{0}
\verb{nova}{ó}{}{}{}{s.f.}{Informação que acaba de chegar; notícia, novidade.}{no.va}{0}
\verb{nova"-iorquino}{}{}{}{}{adj.}{Relativo a Nova Iorque (\textsc{e.u.a}.).}{no.va"-i.or.qui.no}{0}
\verb{nova"-iorquino}{}{}{}{}{}{Indivíduo natural ou habitante dessa cidade.}{no.va"-i.or.qui.no}{0}
\verb{novamente}{}{}{}{}{adv.}{Outra vez; de novo.}{no.va.men.te}{0}
\verb{novato}{}{}{}{}{adj.}{Que está começando em alguma atividade.}{no.va.to}{0}
\verb{nove}{ó}{}{}{}{num.}{Nome dado à quantidade expressa pelo número 9.  }{no.ve}{0}
\verb{novecentos}{}{}{}{}{num.}{Nome dado à quantidade expressa pelo número 900.  }{no.ve.cen.tos}{0}
\verb{nove"-horas}{ó\ldots{}ó}{Pop.}{}{}{s.f.pl.}{Usado na expressão \textit{cheio de nove"-horas}: diz"-se de coisa excessivamente trabalhosa, complicada, enfeitada, rebuscada; cheio de novidades; cheio de frescura.}{no.ve"-ho.ras}{0}
\verb{novel}{é}{}{"-éis}{}{adj.2g.}{De pouca idade; novo.}{no.vel}{0}
\verb{novel}{é}{}{"-éis}{}{}{Que começa alguma atividade; novato, principiante.}{no.vel}{0}
\verb{novela}{é}{}{}{}{s.f.}{História que é apresentada em capítulos pelo rádio ou pela televisão.}{no.ve.la}{0}
\verb{noveleiro}{ê}{}{}{}{adj.}{Que aprecia novelas.}{no.ve.lei.ro}{0}
\verb{noveleiro}{ê}{}{}{}{}{Que gosta de dar notícias; novidadeiro.}{no.ve.lei.ro}{0}
\verb{novelesco}{ê}{}{}{}{adj.}{Próprio de novela.}{no.ve.les.co}{0}
\verb{novelista}{}{}{}{}{adj.2g.}{Que escreve novela.}{no.ve.lis.ta}{0}
\verb{novelo}{ê}{}{}{}{s.m.}{Bola feita de fios enrolados.}{no.ve.lo}{0}
\verb{novembro}{}{}{}{}{s.m.}{O décimo primeiro mês do ano civil.}{no.vem.bro}{0}
\verb{novena}{}{}{}{}{s.f.}{O espaço de nove dias.}{no.ve.na}{0}
\verb{novena}{}{}{}{}{}{Reza feita durante nove dias consecutivos.}{no.ve.na}{0}
\verb{novenário}{}{}{}{}{s.m.}{Livro de novenas.}{no.ve.ná.rio}{0}
\verb{novênio}{}{}{}{}{s.m.}{Período de nove anos.}{no.vê.nio}{0}
\verb{noventa}{}{}{}{}{num.}{Nome dado à quantidade expressa pelo número 90.  }{no.ven.ta}{0}
\verb{noviça}{}{}{}{}{s.f.}{Mulher que está se preparando num convento para se tornar freira.}{no.vi.ça}{0}
\verb{noviciado}{}{}{}{}{s.m.}{Preparação para ser noviço.}{no.vi.ci.a.do}{0}
\verb{noviciar}{}{}{}{}{v.i.}{Praticar o noviciado.}{no.vi.ci.ar}{0}
\verb{noviciar}{}{}{}{}{v.t.}{Fazer os primeiros exercícios; iniciar"-se, estrear"-se.}{no.vi.ci.ar}{\verboinum{6}}
\verb{noviço}{}{}{}{}{s.m.}{Homem que se prepara para seguir uma ordem religiosa.}{no.vi.ço}{0}
\verb{noviço}{}{}{}{}{adj.}{Inexperiente, novato.}{no.vi.ço}{0}
\verb{novidade}{}{}{}{}{s.f.}{Qualidade de novo.}{no.vi.da.de}{0}
\verb{novidade}{}{}{}{}{}{Produto ou artigo lançado no mercado.}{no.vi.da.de}{0}
\verb{novidade}{}{}{}{}{}{Primeira informação sobre um acontecimento recente, um fato.}{no.vi.da.de}{0}
\verb{novidadeiro}{ê}{}{}{}{adj.}{Diz"-se daquele que gosta de contar novidades, de dar notícias.}{no.vi.da.dei.ro}{0}
\verb{novilatino}{}{}{}{}{adj.}{Neolatino.}{no.vi.la.ti.no}{0}
\verb{novilha}{}{}{}{}{s.f.}{Vaca nova; bezerra.}{no.vi.lha}{0}
\verb{novilho}{}{}{}{}{s.m.}{Boi ainda novo; garrote.}{no.vi.lho}{0}
\verb{novilúnio}{}{}{}{}{s.m.}{Lua nova.}{no.vi.lú.nio}{0}
\verb{novilúnio}{}{}{}{}{}{Período de lua nova.}{no.vi.lú.nio}{0}
\verb{novo}{ô}{}{"-s ⟨ó⟩}{"-a ⟨ó⟩}{adj.}{De pouca idade; jovem.}{no.vo}{0}
\verb{novo}{ô}{}{"-s ⟨ó⟩}{"-a ⟨ó⟩}{}{Que foi feito ou comprado há pouco tempo; recente.}{no.vo}{0}
\verb{novo}{ô}{}{"-s ⟨ó⟩}{"-a ⟨ó⟩}{}{Que ainda não foi usado.}{no.vo}{0}
\verb{novo}{ô}{}{"-s ⟨ó⟩}{"-a ⟨ó⟩}{}{Moderno, original.}{no.vo}{0}
\verb{novocaína}{}{}{}{}{s.f.}{Nome comercial de um anestésico não tóxico, usado especialmente em anestesia local.}{no.vo.ca.í.na}{0}
\verb{novo"-rico}{ô}{}{novos"-ricos ⟨ó⟩}{}{s.m.}{Indivíduo que acabou de ganhar muito dinheiro e está deslumbrado.}{no.vo"-ri.co}{0}
%\verb{}{}{}{}{}{}{}{}{0}
\verb{noz}{ó}{}{}{}{s.f.}{Fruto da nogueira, oleoso e de casca dura.}{noz}{0}
\verb{noz}{ó}{}{}{}{}{Fruto que tem uma só semente, coberta por casca muito dura. (\textit{As avelãs e amêndoas são nozes.})}{noz}{0}
\verb{noz"-moscada}{ó}{}{nozes"-moscadas ⟨ó⟩}{}{s.f.}{Semente que se usa como tempero e traz um sabor picante.}{noz"-mos.ca.da}{0}
\verb{noz"-vômica}{ó}{Bot.}{nozes"-vômicas ⟨ó⟩}{}{s.f.}{Árvore de cujo fruto se extrai a estricnina e uma tintura de propriedades medicinais.}{noz"-vô.mi.ca}{0}
\verb{Np}{}{Quím.}{}{}{}{Símb. do \textit{netúnio}.}{Np}{0}
\verb{nu}{}{}{}{}{s.m.}{Décima terceira letra do alfabeto grego.}{nu}{0}
\verb{nu}{}{}{}{nua}{adj.}{Que está sem roupa; despido, pelado.}{nu}{0}
\verb{nu}{}{}{}{nua}{}{Descoberto; exposto.}{nu}{0}
\verb{nuança}{}{}{}{}{s.f.}{Cada uma das diversas gradações de uma cor; matiz, tonalidade, meio"-tom.}{nu.an.ça}{0}
\verb{nuança}{}{}{}{}{}{Diferença delicada entre coisas do mesmo gênero.}{nu.an.ça}{0}
\verb{nuança}{}{}{}{}{}{Grau de força ou de doçura que convém dar aos sons.}{nu.an.ça}{0}
\verb{nuançar}{}{}{}{}{v.t.}{Variar ou dar diferentes graduações às cores; matizar.}{nu.an.çar}{\verboinum{3}}
\verb{nuance}{}{}{}{}{s.f.}{Nuança.}{nu.an.ce}{0}
\verb{nubente}{}{}{}{}{s.2g.}{Indivíduo que está para se casar; noivo.}{nu.ben.te}{0}
\verb{núbil}{}{}{"-eis}{}{adj.2g.}{Que está em idade de casar.}{nú.bil}{0}
\verb{nublado}{}{}{}{}{adj.}{Coberto de nuvens.}{nu.bla.do}{0}
\verb{nublar}{}{}{}{}{v.t.}{Cobrir de nuvens.}{nu.blar}{\verboinum{1}}
\verb{nuca}{}{Anat.}{}{}{s.f.}{A parte de trás do pescoço.}{nu.ca}{0}
\verb{nucleado}{}{}{}{}{adj.}{Que tem núcleo.}{nu.cle.a.do}{0}
\verb{nuclear}{}{}{}{}{adj.2g.}{Relativo a núcleo.}{nu.cle.ar}{0}
\verb{nuclear}{}{}{}{}{}{Relativo ao núcleo do átomo.}{nu.cle.ar}{0}
\verb{núcleo}{}{}{}{}{s.m.}{Parte ou ponto que fica no centro de alguma coisa.}{nú.cleo}{0}
\verb{nudação}{}{}{"-ões}{}{s.f.}{Ato ou efeito de desnudar; nudez.}{nu.da.ção}{0}
\verb{nudez}{ê}{}{}{}{s.f.}{Estado de quem está nu.}{nu.dez}{0}
\verb{nudez}{ê}{}{}{}{}{Ausência de enfeites.}{nu.dez}{0}
\verb{nudismo}{}{}{}{}{s.m.}{Prática de viver sem roupas.}{nu.dis.mo}{0}
\verb{nudista}{}{}{}{}{s.2g.}{Indivíduo que pratica o nudismo.}{nu.dis.ta}{0}
\verb{nuga}{}{}{}{}{s.f.}{Coisa insignificante, sem valor.}{nu.ga}{0}
\verb{nugá}{}{Cul.}{}{}{s.m.}{Doce de nozes ou de amêndoas misturadas com caramelo ou mel, vendido, em geral, em tabletes ou como recheio de bombom, de chocolate etc.; nogado.  }{nu.gá}{0}
%\verb{}{}{}{}{}{}{}{}{0}
%\verb{}{}{}{}{}{}{}{}{0}
\verb{nulidade}{}{}{}{}{s.f.}{Qualidade de não ter valor.}{nu.li.da.de}{0}
\verb{nulificar}{}{}{}{}{v.t.}{Anular.}{nu.li.fi.car}{\verboinum{2}}
\verb{nulo}{}{}{}{}{adj.}{Equivalente a nada; nenhum.}{nu.lo}{0}
\verb{nulo}{}{}{}{}{}{Sem nenhum valor.}{nu.lo}{0}
\verb{num}{}{}{}{}{}{Contração da preposição \textit{em} com o artigo indefinido \textit{um}.}{num}{0}
\verb{num}{}{}{}{}{}{Em algo indeterminado ou que é mencionado pela primeira vez.}{num}{0}
\verb{num}{}{}{}{}{}{Em algo indeterminado, mas pertencente a uma classe ou categoria já mencionada em outra oração.}{num}{0}
\verb{num}{}{}{}{}{}{Em um único.}{num}{0}
\verb{nume}{}{}{}{}{s.m.}{Ser ou potência divina; divindade.}{nu.me}{0}
\verb{nume}{}{Fig.}{}{}{}{Inspiração poética advinda do poder divino.}{nu.me}{0}
\verb{numeração}{}{}{"-ões}{}{s.f.}{Ato ou efeito de numerar.}{nu.me.ra.ção}{0}
\verb{numeração}{}{}{"-ões}{}{}{Parte da aritmética que ensina a formar, ler e escrever os números.}{nu.me.ra.ção}{0}
\verb{numerador}{ô}{}{}{}{adj.}{Que numera.}{nu.me.ra.dor}{0}
\verb{numerador}{ô}{}{}{}{s.m.}{Número que indica quantas partes iguais são tomadas da divisão de uma quantidade.}{nu.me.ra.dor}{0}
\verb{numeral}{}{Gram.}{"-ais}{}{s.m.}{Classe de palavra que quantifica, ordena  ou distribui objetos.}{nu.me.ral}{0}
\verb{numeral}{}{}{"-ais}{}{adj.2g.}{Relativo a número.}{nu.me.ral}{0}
\verb{numerar}{}{}{}{}{v.t.}{Marcar alguma coisa com um número.}{nu.me.rar}{\verboinum{1}}
\verb{numerário}{}{}{}{}{adj.}{Relativo a dinheiro.}{nu.me.rá.rio}{0}
\verb{numerável}{}{}{"-eis}{}{adj.2g.}{Que se pode numerar.}{nu.me.rá.vel}{0}
\verb{numérico}{}{}{}{}{adj.}{Relativo a números.}{nu.mé.ri.co}{0}
\verb{numérico}{}{}{}{}{}{Que indica número; numeral.}{nu.mé.ri.co}{0}
\verb{número}{}{}{}{}{s.m.}{Cada uma das quantidades determinadas.}{nú.me.ro}{0}
\verb{número}{}{}{}{}{}{Quantidade indeterminada de alguma coisa.}{nú.me.ro}{0}
\verb{número}{}{}{}{}{}{Conjunto de exemplares de uma publicação periódica.}{nú.me.ro}{0}
\verb{número}{}{}{}{}{}{Cada exemplar de uma publicação periódica.}{nú.me.ro}{0}
\verb{número}{}{}{}{}{}{Cada uma das partes de um programa artístico.}{nú.me.ro}{0}
\verb{número}{}{}{}{}{}{Categoria gramatical que indica a existência de um ou de mais de um. }{nú.me.ro}{0}
\verb{numerologia}{}{}{}{}{s.f.}{Estudo da influência oculta dos números no destino das pessoas.}{nu.me.ro.lo.gi.a}{0}
\verb{numerologista}{}{}{}{}{s.2g.}{Especialista em numerologia.}{nu.me.ro.lo.gis.ta}{0}
\verb{numeroso}{ô}{}{"-osos ⟨ó⟩}{"-osa ⟨ó⟩}{adj.}{Que apresenta grande quantidade; abundante.}{nu.me.ro.so}{0}
\verb{numismata}{}{}{}{}{s.2g.}{Indivíduo versado em numismática.}{nu.mis.ma.ta}{0}
\verb{numismática}{}{}{}{}{s.f.}{Estudo das moedas e medalhas.}{nu.mis.má.ti.ca}{0}
\verb{nunca}{}{}{}{}{adv.}{Em nenhum momento; jamais.}{nun.ca}{0}
\verb{nunciatura}{}{}{}{}{s.f.}{Ofício, função ou dignidade de núncio apostólico.}{nun.ci.a.tu.ra}{0}
\verb{nunciatura}{}{}{}{}{}{A residência do núncio.}{nun.ci.a.tu.ra}{0}
\verb{nunciatura}{}{}{}{}{}{Tribunal eclesiástico submetido ao núncio.}{nun.ci.a.tu.ra}{0}
\verb{núncio}{}{}{}{}{s.m.}{Embaixador do Papa.}{nún.cio}{0}
\verb{núncio}{}{}{}{}{}{Anunciador, mensageiro; precursor.}{nún.cio}{0}
\verb{nuncupação}{}{Jur.}{"-ões}{}{s.f.}{Nomeação de herdeiro, feita de viva voz pelo testador; testamento oral.}{nun.cu.pa.ção}{0}
\verb{nuncupativo}{}{Jur.}{}{}{adj.}{Diz"-se de ato jurídico feito oralmente e não por escrito.}{nun.cu.pa.ti.vo}{0}
\verb{nuncupativo}{}{}{}{}{}{Diz"-se de herdeiro nomeado oralmente.}{nun.cu.pa.ti.vo}{0}
\verb{nuncupativo}{}{}{}{}{}{Que é só de nome; não real; nominal.}{nun.cu.pa.ti.vo}{0}
\verb{nupcial}{}{}{"-ais}{}{adj.2g.}{Relativo a núpcias, ao matrimônio.}{nup.ci.al}{0}
\verb{núpcias}{}{}{}{}{s.f.pl.}{Casamento.}{núp.cias}{0}
\verb{nutrição}{}{}{"-ões}{}{s.f.}{Ato ou efeito de nutrir.}{nu.tri.ção}{0}
\verb{nutrição}{}{}{"-ões}{}{}{Processo de ingestão, digestão e absorção de alimentos.}{nu.tri.ção}{0}
\verb{nutricional}{}{}{"-ais}{}{adj.2g.}{Relativo a nutrição.}{nu.tri.ci.o.nal}{0}
\verb{nutricionismo}{}{}{}{}{s.m.}{Estudo da nutrição e das propriedades dos alimentos.}{nu.tri.ci.o.nis.mo}{0}
\verb{nutricionista}{}{}{}{}{s.2g.}{Profissional especializado em nutrição.}{nu.tri.ci.o.nis.ta}{0}
\verb{nutrido}{}{}{}{}{adj.}{Provido de alimento; alimentado, sustentado.}{nu.tri.do}{0}
\verb{nutrido}{}{}{}{}{}{Corpulento, robusto.}{nu.tri.do}{0}
\verb{nutrido}{}{}{}{}{}{Diz"-se de fogo forte e persistente.}{nu.tri.do}{0}
\verb{nutriente}{}{}{}{}{s.m.}{Substância que nutre o organismo.}{nu.tri.en.te}{0}
\verb{nutrimento}{}{}{}{}{s.m.}{Ato ou efeito de nutrir; nutrição.}{nu.tri.men.to}{0}
\verb{nutrimento}{}{}{}{}{}{Aquilo que nutre, que alimenta; alimento, sustento.}{nu.tri.men.to}{0}
\verb{nutrimento}{}{}{}{}{}{Cada um dos componentes nutrientes de um alimento.}{nu.tri.men.to}{0}
\verb{nutrir}{}{}{}{}{v.t.}{Prover de substâncias necessárias ao metabolismo; alimentar.}{nu.trir}{0}
\verb{nutrir}{}{}{}{}{}{Guardar algum sentimento consigo; acalentar, cultivar.}{nu.trir}{\verboinum{18}}
\verb{nutritivo}{}{}{}{}{adj.}{Que serve para alimentar, nutrir; nutriente.}{nu.tri.ti.vo}{0}
\verb{nutriz}{}{}{}{}{s.f.}{Mulher que amamenta; ama de leite.}{nu.triz}{0}
\verb{nutriz}{}{}{}{}{adj.}{Que nutre, que sustenta.}{nu.triz}{0}
\verb{nuvem}{}{}{"-ens}{}{s.f.}{Vapor de água suspenso na atmosfera.}{nu.vem}{0}
\verb{nuvem}{}{}{"-ens}{}{}{Grande quantidade de coisas que aparecem no ar.}{nu.vem}{0}
\verb{NW}{}{}{}{}{}{Abrev. de \textit{noroeste}.}{n.w.}{0}
\verb{nylon}{}{}{}{}{}{Var. de \textit{náilon}.}{\textit{nylon}}{0}
