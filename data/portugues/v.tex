\verb{v}{}{}{}{}{s.m.}{Vigésima segunda letra do alfabeto português.}{v}{0}
\verb{V}{}{}{}{}{}{Algarismo romano equivalente a \textit{5}.}{V}{0}
\verb{V}{}{Quím.}{}{}{}{Símb. do \textit{vanádio}.}{V}{0}
\verb{vã}{}{}{}{}{adj.}{}{vã}{0}
\verb{vaca}{}{}{}{}{s.f.}{A fêmea do boi ou do touro.}{va.ca}{0}
\verb{vaca}{}{Pop.}{}{}{}{Mulher considerada leviana por causa de certos comportamentos.}{va.ca}{0}
\verb{vaca"-fria}{}{}{}{}{s.f.}{Usado na expressão \textit{voltar à vaca"-fria}; retomar assunto já tratado e interrompido.}{va.ca"-fri.a}{0}
\verb{vacância}{}{}{}{}{s.f.}{Estado daquilo que se encontra vago.}{va.cân.cia}{0}
\verb{vacância}{}{}{}{}{}{Tempo durante o qual algo permanece vago.}{va.cân.cia}{0}
\verb{vacante}{}{}{}{}{adj.2g.}{Que se encontra vago.}{va.can.te}{0}
\verb{vaca"-preta}{ê}{}{vacas"-pretas ⟨ê⟩}{}{s.f.}{Mistura feita com sorvete e refrigerante de cola.}{va.ca"-pre.ta}{0}
\verb{vacaria}{}{}{}{}{s.f.}{Coletivo de vacas. (\textit{A vacaria ficou no pasto.})}{va.ca.ri.a}{0}
\verb{vacaria}{}{}{}{}{}{Estabelecimento onde se criam vacas para ordenha.}{va.ca.ri.a}{0}
\verb{vacilação}{}{}{"-ões}{}{s.f.}{Ato ou efeito de vacilar; balanço, oscilação.}{va.ci.la.ção}{0}
\verb{vacilação}{}{Fig.}{"-ões}{}{}{Hesitação, dúvida.}{va.ci.la.ção}{0}
\verb{vacilante}{}{}{}{}{adj.2g.}{Que não tem firmeza; oscilante, instável, hesitante.}{va.ci.lan.te}{0}
\verb{vacilar}{}{}{}{}{v.i.}{Balançar por falta de firmeza; oscilar.}{va.ci.lar}{0}
\verb{vacilar}{}{Fig.}{}{}{}{Estar ou mostrar"-se inseguro.}{va.ci.lar}{\verboinum{1}}
\verb{vacilo}{}{}{}{}{s.m.}{Ato ou efeito de vacilar; hesitação.}{va.ci.lo}{0}
\verb{vacilo}{}{Pop.}{}{}{}{Deslize, erro, mancada.}{va.ci.lo}{0}
\verb{vacina}{}{}{}{}{s.f.}{Substância fabricada a partir de micróbios, ou das substâncias que eles produzem, que, introduzida no organismo, provoca a formação de anticorpos contra doenças infecciosas.}{va.ci.na}{0}
\verb{vacinação}{}{}{"-ões}{}{s.f.}{Ato ou efeito de vacinar.}{va.ci.na.ção}{0}
\verb{vacinação}{}{}{"-ões}{}{}{Administração de vacina em grande número de indivíduos.}{va.ci.na.ção}{0}
\verb{vacinar}{}{}{}{}{v.t.}{Aplicar vacina em um organismo a fim de imunizá"-lo contra doenças infecciosas.}{va.ci.nar}{\verboinum{1}}
\verb{vacuidade}{}{}{}{}{s.f.}{Estado ou qualidade do que está vazio; ausência, vácuo, falta.}{va.cu.i.da.de}{0}
\verb{vacuidade}{}{Fig.}{}{}{}{Ausência de modéstia; vaidade, presunção.}{va.cu.i.da.de}{0}
\verb{vacum}{}{}{"-uns}{}{adj.2g.}{Diz"-se do gado composto de vacas, bois, novilhos, bezerros.}{va.cum}{0}
\verb{vácuo}{}{}{}{}{s.m.}{Espaço não ocupado por coisa alguma; vacuidade.}{vá.cu.o}{0}
\verb{vácuo}{}{Fís.}{}{}{}{Região espacial em que a pressão é inferior à pressão atmosférica.}{vá.cu.o}{0}
\verb{vácuo}{}{}{}{}{}{Zona da atmosfera afetada por correntes descendentes.}{vá.cu.o}{0}
\verb{vadear}{}{}{}{}{v.t.}{Atravessar rios, riachos ou córregos pelo lugar mais raso.}{va.de.ar}{\verboinum{4}}
\verb{vadeável}{}{}{"-eis}{}{adj.2g.}{Diz"-se do rio ou córrego que se pode vadear.}{va.de.á.vel}{0}
\verb{vade"-mécum}{}{}{vade"-mécuns}{}{s.f.}{Livro de conteúdo prático e de uso muito frequente que o usuário costuma carregar consigo.}{va.de"-mé.cum}{0}
\verb{vade"-mécum}{}{}{vade"-mécuns}{}{}{Agenda.}{va.de"-mé.cum}{0}
\verb{vadiação}{}{}{"-ões}{}{s.f.}{Ato ou efeito de vadiar, de andar sem rumo certo.}{va.di.a.ção}{0}
\verb{vadiagem}{}{}{"-ens}{}{s.f.}{Ato ou efeito de vadiar; vadiação.}{va.di.a.gem}{0}
\verb{vadiagem}{}{}{"-ens}{}{}{Vida de vadio; ociosidade, vagabundagem, malandragem.}{va.di.a.gem}{0}
\verb{vadiagem}{}{Jur.}{"-ens}{}{}{Contravenção penal que consiste em um indivíduo levar uma vida ociosa, sem renda própria, estando apto para trabalhar.}{va.di.a.gem}{0}
\verb{vadiar}{}{}{}{}{v.i.}{Andar sem rumo certo; vaguear.}{va.di.ar}{0}
\verb{vadiar}{}{}{}{}{}{Viver na ociosidade; não trabalhar; vagabundear.}{va.di.ar}{0}
\verb{vadiar}{}{}{}{}{}{Entregar"-se à diversão; brincar.}{va.di.ar}{\verboinum{6}}
\verb{vadio}{}{}{}{}{adj.}{Que não tem domicílio certo; errante.}{va.di.o}{0}
\verb{vadio}{}{}{}{}{}{Que vive na ociosidade; desocupado, vagabundo.}{va.di.o}{0}
\verb{vadio}{}{}{}{}{}{Que não trabalha ou não se empenha em trabalhar; preguiçoso, malandro.}{va.di.o}{0}
\verb{vaga}{}{}{}{}{s.f.}{Ato ou efeito de vagar, de desocupar.}{va.ga}{0}
\verb{vaga}{}{}{}{}{s.f.}{Onda grande, em mar alto e agitado.}{va.ga}{0}
\verb{vaga}{}{}{}{}{}{Cargo ou função que esteja em disponibilidade.}{va.ga}{0}
\verb{vaga}{}{Fig.}{}{}{}{Multidão tumultuosa; turba.}{va.ga}{0}
\verb{vagabundagem}{}{}{"-ens}{}{s.f.}{Situação ou estado de quem passa o tempo sem ocupação, sem responsabilidade; vadiagem, ociosidade.}{va.ga.bun.da.gem}{0}
\verb{vagabundar}{}{}{}{}{v.i.}{Vagabundear.}{va.ga.bun.dar}{\verboinum{1}}
\verb{vagabundear}{}{}{}{}{v.i.}{Viver ociosamente; vagabundar.}{va.ga.bun.de.ar}{0}
\verb{vagabundear}{}{}{}{}{}{Andar sem rumo; vadiar.}{va.ga.bun.de.ar}{\verboinum{4}}
\verb{vagabundo}{}{}{}{}{adj.}{Que vagueia; errante, nômade.}{va.ga.bun.do}{0}
\verb{vagabundo}{}{}{}{}{}{Que não trabalha; vadio, desocupado.}{va.ga.bun.do}{0}
\verb{vagabundo}{}{}{}{}{}{Inconstante, leviano.}{va.ga.bun.do}{0}
\verb{vagabundo}{}{}{}{}{}{De má qualidade; ordinário, reles.}{va.ga.bun.do}{0}
\verb{vagalhão}{}{}{"-ões}{}{s.m.}{Onda de grande tamanho.}{va.ga.lhão}{0}
\verb{vaga"-lume}{}{Zool.}{vaga"-lumes}{}{s.m.}{Nome comum dado a insetos que são capazes de emitir luminescência; pirilampo.}{va.ga"-lu.me}{0}
\verb{vagamundo}{}{}{}{}{adj.}{Que vagueia; errante, vagabundo.}{va.ga.mun.do}{0}
\verb{vagão}{}{}{"-ões}{}{s.m.}{Veículo ferroviário usado para transportar passageiros, gado ou mercadoria.}{va.gão}{0}
\verb{vagar}{}{}{}{}{}{Espalhar"-se, derramar"-se.}{va.gar}{\verboinum{5}}
\verb{vagar}{}{}{}{}{v.i.}{Andar sem rumo; vaguear, errar.}{va.gar}{0}
\verb{vagar}{}{}{}{}{v.i.}{Estar ou ficar vago; desocupar.}{va.gar}{\verboinum{5}}
\verb{vagar}{}{}{}{}{s.m.}{Falta de pressa; lentidão, vagareza.}{va.gar}{0}
\verb{vagareza}{ê}{}{}{}{s.f.}{Falta de pressa; lentidão, morosidade.}{va.ga.re.za}{0}
\verb{vagaroso}{ô}{}{"-osos ⟨ó⟩}{"-osa ⟨ó⟩}{adj.}{Que age devagar; lento, demorado.}{va.ga.ro.so}{0}
\verb{vagaroso}{ô}{}{"-osos ⟨ó⟩}{"-osa ⟨ó⟩}{}{Indolente, moroso, lânguido.}{va.ga.ro.so}{0}
\verb{vagem}{}{}{"-ens}{}{s.m.}{Invólucro das sementes ou grãos das plantas leguminosas.}{va.gem}{0}
\verb{vagido}{}{}{}{}{s.m.}{Choro de criança recém"-nascida.}{va.gi.do}{0}
\verb{vagido}{}{}{}{}{}{Voz chorosa; lamento, gemido.}{va.gi.do}{0}
\verb{vagina}{}{Anat.}{}{}{s.f.}{Canal feminino, que vai desde a abertura da vulva até o colo do útero.}{va.gi.na}{0}
\verb{vaginal}{}{}{"-ais}{}{adj.2g.}{Relativo à vagina.}{va.gi.nal}{0}
\verb{vaginismo}{}{Med.}{}{}{s.m.}{Contração espasmódica e involuntária dos músculos da vagina no momento da penetração do pênis, impossibilitando ou tornando o ato sexual doloroso.}{va.gi.nis.mo}{0}
\verb{vagir}{}{}{}{}{v.i.}{Dar vagidos.}{va.gir}{0}
\verb{vagir}{}{}{}{}{}{Chorar, gemer, lamentar.}{va.gir}{\verboinum{34}}
\verb{vago}{}{}{}{}{adj.}{Não preenchido; desocupado, vazio.}{va.go}{0}
\verb{vago}{}{}{}{}{adj.}{Que vagueia; errante, vagabundo, nômade.}{va.go}{0}
\verb{vago}{}{}{}{}{}{Que não tem moradores ou habitantes; desabitado.}{va.go}{0}
\verb{vago}{}{}{}{}{}{Que não apresenta traços ou características nítidas; impreciso, incerto, ambíguo.}{va.go}{0}
\verb{vagonete}{ê}{}{}{}{s.m.}{Pequeno vagão para transportar terra, materiais de construção, minérios etc. nas grandes obras, como túneis ou represas.}{va.go.ne.te}{0}
\verb{vaguear}{}{}{}{}{v.i.}{Andar sem rumo; errar, perambular, vagar.}{va.gue.ar}{0}
\verb{vaguear}{}{}{}{}{}{Ter vida ociosa; vagabundear.}{va.gue.ar}{\verboinum{4}}
\verb{vaia}{}{}{}{}{s.f.}{Manifestação pública de desagrado ou desprezo; apupo, zombaria.}{vai.a}{0}
\verb{vaiar}{}{}{}{}{v.t.}{Manifestar publicamente desagrado ou despeito; apupar, zombar.}{vai.ar}{\verboinum{1}}
\verb{vaidade}{}{}{}{}{s.f.}{Desejo infundado de merecer a atenção e a admiração dos outros.}{vai.da.de}{0}
\verb{vaidade}{}{}{}{}{}{Ideia exageradamente positiva que se atribui à própria aparência ou a alguma qualidade intelectual; presunção, imodéstia.}{vai.da.de}{0}
\verb{vaidade}{}{}{}{}{}{Coisa vã, ilusória, fútil.}{vai.da.de}{0}
\verb{vaidoso}{ô}{}{"-osos ⟨ó⟩}{"-osa ⟨ó⟩}{adj.}{Que tem vaidade; fútil, presunçoso.}{va.i.do.so}{0}
\verb{vai"-e"-vem}{}{}{}{}{s.m.}{Vaivém.}{vai"-e"-vem}{0}
\verb{vaivém}{}{}{"-éns}{}{s.m.}{Movimento de ir e vir repetidas vezes.}{vai.vém}{0}
\verb{vaivém}{}{}{"-éns}{}{}{Deslocamento de pessoas ou veículos de um lado para o outro.}{vai.vém}{0}
\verb{vala}{}{}{}{}{s.f.}{Escavação longa, de profundidade e largura médias, com diversas finalidades como, por exemplo, defesa de fortificações ou escoamento de águas pluviais.}{va.la}{0}
\verb{valado}{}{}{}{}{s.m.}{Vala rasa, rodeada de tapume ou sebe, que cerca uma propriedade rural.}{va.la.do}{0}
\verb{valado}{}{}{}{}{}{Sulco para escoamento de água; rego.}{va.la.do}{0}
\verb{valdevinos}{}{}{}{}{s.m.}{Indivíduo sem ocupação; vadio, vagabundo.}{val.de.vi.nos}{0}
\verb{valdevinos}{}{}{}{}{}{Indivíduo amalucado; doidivanas.}{val.de.vi.nos}{0}
\verb{vale}{}{}{}{}{s.m.}{Depressão do terreno entre montes ou na base de um monte.}{va.le}{0}
\verb{vale}{}{}{}{}{s.m.}{Adiantamento de determinada soma de dinheiro ou retirada eventual de um caixa.}{va.le}{0}
\verb{vale}{}{}{}{}{}{Várzea ou planície à beira de um rio.}{va.le}{0}
\verb{valência}{}{Quím.}{}{}{s.f.}{Capacidade ou tendência de combinação que um átomo de substância simples tem em relação ao número de átomos de hidrogênio. }{va.lên.cia}{0}
\verb{valentão}{}{}{"-ões}{}{adj.}{Que é muito valente; corajoso, intrépido.}{va.len.tão}{0}
\verb{valentão}{}{}{"-ões}{}{}{Que é dado a bravatas; fanfarrão, jactancioso.}{va.len.tão}{0}
\verb{valente}{}{}{}{}{adj.2g.}{Que não receia o perigo; intrépido, corajoso.}{va.len.te}{0}
\verb{valente}{}{}{}{}{}{Que tem valia, força; vigoroso, robusto.}{va.len.te}{0}
\verb{valente}{}{}{}{}{}{Que revela decisão; enérgico, sério.}{va.len.te}{0}
\verb{valentia}{}{}{}{}{s.f.}{Qualidade de valente; intrepidez, coragem.}{va.len.ti.a}{0}
\verb{valentia}{}{}{}{}{}{Ato de valor; proeza, façanha.}{va.len.ti.a}{0}
\verb{valentia}{}{}{}{}{}{Energia, força, vigor.}{va.len.ti.a}{0}
\verb{valer}{ê}{}{}{}{v.t.}{Ter determinado preço ou valor.}{va.ler}{0}
\verb{valer}{ê}{}{}{}{}{Ser uma boa recompensa por alguma coisa; compensar, pagar.}{va.ler}{0}
\verb{valer}{ê}{}{}{}{}{Ser proveitoso para alguém; ajudar, auxiliar.}{va.ler}{0}
\verb{valer}{ê}{}{}{}{}{Ser a prova de algum direito; ter validade, ter valor.}{va.ler}{0}
\verb{valer}{ê}{}{}{}{v.pron.}{Usar pessoa ou coisa para seu próprio benefício; aproveitar, tirar proveito.}{va.ler}{\verboinum{51}}
\verb{vale"-refeição}{}{}{vale"-refeições}{}{s.m.}{Vale que o empregador fornece ao funcionário para ser utilizado no pagamento de refeições em estabelecimentos credenciados. }{va.le"-re.fei.ção}{0}
\verb{valeriana}{}{Bot.}{}{}{s.f.}{Planta herbácea nativa de regiões de clima temperado, de flores brancas ou cor"-de"-rosa cultivadas como ornamentais ou por suas propriedades sedativas e antiespasmódicas.}{va.le.ri.a.na}{0}
\verb{valeta}{ê}{}{}{}{s.f.}{Pequena vala aberta na beira de estradas e ruas para escoamento de água.}{va.le.ta}{0}
\verb{valete}{é}{}{}{}{s.m.}{Na Idade Média, jovem escudeiro colocado ao lado de um senhor para ser iniciado como cavaleiro.}{va.le.te}{0}
\verb{valete}{é}{}{}{}{}{Nas cartas de baralho, figura desse jovem, que, na maioria dos jogos, é imediatamente inferior à dama e ao rei.}{va.le.te}{0}
\verb{vale"-transporte}{ó}{}{vale"-transportes ⟨ó⟩}{}{s.m.}{Vale que o empregador fornece ao funcionário para pagamento de transporte coletivo.}{va.le"-trans.por.te}{0}
\verb{valetudinário}{}{}{}{}{adj.}{Que tem saúde frágil; que está sempre adoentado; enfermiço.}{va.le.tu.di.ná.rio}{0}
\verb{vale"-tudo}{}{}{}{}{s.m.}{Contexto em que se emprega qualquer medida para se chegar a uma solução.}{va.le"-tu.do}{0}
\verb{vale"-tudo}{}{Esport.}{}{}{}{Tipo de luta livre em que são válidos golpes muito violentos.}{va.le"-tu.do}{0}
\verb{valhacoito}{ô}{}{}{}{s.m.}{Lugar seguro onde se encontra abrigo; refúgio, asilo, esconderijo.}{va.lha.coi.to}{0}
\verb{valia}{}{}{}{}{s.f.}{Valor intrínseco a um objeto.}{va.li.a}{0}
\verb{valia}{}{}{}{}{}{Serventia, préstimo, utilidade.}{va.li.a}{0}
\verb{valia}{}{}{}{}{}{Mérito, reconhecimento, merecimento.}{va.li.a}{0}
\verb{validação}{}{}{"-ões}{}{s.f.}{Ato ou efeito de validar, de tornar válido; legitimar.}{va.li.da.ção}{0}
\verb{validação}{}{}{"-ões}{}{}{Confirmação de um ato, de acordo com as determinações legais.}{va.li.da.ção}{0}
\verb{validade}{}{}{}{}{s.f.}{Qualidade ou caráter do que é válido.}{va.li.da.de}{0}
\verb{validade}{}{}{}{}{}{Legitimidade, valor, reconhecimento.}{va.li.da.de}{0}
\verb{validar}{}{}{}{}{v.t.}{Tornar ou declarar válido, legal; legitimar.}{va.li.dar}{\verboinum{1}}
\verb{validez}{ê}{}{}{}{s.f.}{Estado ou qualidade do que é válido; validade, legitimidade.}{va.li.dez}{0}
\verb{valido}{}{}{}{}{adj.}{Que está sob a proteção de alguém; socorrido, amparado.}{va.li.do}{0}
\verb{valido}{}{}{}{}{}{Que recebe estima; querido, prezado.}{va.li.do}{0}
\verb{válido}{}{}{}{}{adj.}{Que tem validade ou valor legal; legítimo.}{vá.li.do}{0}
\verb{válido}{}{}{}{}{}{Que surte efeito; eficaz, eficiente.}{vá.li.do}{0}
\verb{válido}{}{}{}{}{}{Que tem valor; merecido, reconhecido.}{vá.li.do}{0}
\verb{válido}{}{}{}{}{}{Que tem saúde; são, sadio.}{vá.li.do}{0}
\verb{valimento}{}{}{}{}{s.m.}{Ato ou efeito de valer; validade, legitimidade.}{va.li.men.to}{0}
\verb{valimento}{}{}{}{}{}{Merecimento, valia, reconhecimento.}{va.li.men.to}{0}
\verb{valioso}{ô}{}{"-oso ⟨ó⟩}{"-osa ⟨ó⟩}{adj.}{Que tem muito valor; precioso, caro.}{va.li.o.so}{0}
\verb{valioso}{ô}{}{"-oso ⟨ó⟩}{"-osa ⟨ó⟩}{}{Que é digno de apreço e merecimento; estimado.}{va.li.o.so}{0}
\verb{valioso}{ô}{}{"-oso ⟨ó⟩}{"-osa ⟨ó⟩}{}{Que é de grande serventia; proveitoso, útil.}{va.li.o.so}{0}
\verb{valise}{}{}{}{}{s.f.}{Pequena mala de mão; maleta.}{va.li.se}{0}
\verb{valo}{}{}{}{}{s.m.}{Muro ou paliçada que defende um campo entrincheirado.}{va.lo}{0}
\verb{valo}{}{}{}{}{}{Fosso, vala.}{va.lo}{0}
\verb{valor}{ô}{}{}{}{s.m.}{Importância que pessoa ou coisa tem por suas qualidades.}{va.lor}{0}
\verb{valor}{ô}{}{}{}{}{Condição da coisa que continua dando determinado direito; validade.}{va.lor}{0}
\verb{valor}{ô}{}{}{}{}{Capacidade de enfrentar o perigo sem medo; bravura, valentia.}{va.lor}{0}
\verb{valorização}{}{}{"-ões}{}{s.f.}{Ato ou efeito de valorizar.}{va.lo.ri.za.ção}{0}
\verb{valorização}{}{}{"-ões}{}{}{Elevação deliberada do preço de uma mercadoria, acima do valor atribuído pela lei de oferta e procura.}{va.lo.ri.za.ção}{0}
\verb{valorizar}{}{}{}{}{v.t.}{Reconhecer o valor de algo ou alguém; dar importância.}{va.lo.ri.zar}{0}
\verb{valorizar}{}{}{}{}{}{Aumentar o valor ou o preço.}{va.lo.ri.zar}{0}
\verb{valorizar}{}{}{}{}{}{Revelar o mérito; destacar.}{va.lo.ri.zar}{\verboinum{1}}
\verb{valoroso}{ô}{}{"-osos ⟨ó⟩}{"-osa ⟨ó⟩}{adj.}{Que tem muito valor; valioso.}{va.lo.ro.so}{0}
\verb{valoroso}{ô}{}{"-osos ⟨ó⟩}{"-osa ⟨ó⟩}{}{Que demonstra bravura; corajoso, destemido.}{va.lo.ro.so}{0}
\verb{valoroso}{ô}{}{"-osos ⟨ó⟩}{"-osa ⟨ó⟩}{}{Que revela força; enérgico, esforçado.}{va.lo.ro.so}{0}
\verb{valsa}{}{}{}{}{s.f.}{Dança de salão, em três tempos, lenta, moderada ou rápida, executada por casais e rodopiada.}{val.sa}{0}
\verb{valsa}{}{Mús.}{}{}{}{Música para essa dança.}{val.sa}{0}
\verb{valsar}{}{}{}{}{v.i.}{Dançar valsas.}{val.sar}{\verboinum{1}}
\verb{valsista}{}{}{}{}{adj.2g.}{Que dança valsas muito bem.}{val.sis.ta}{0}
\verb{valva}{}{Bot.}{}{}{}{Parte do órgão de uma planta que se abre para permitir a saída de grãos; vagem.}{val.va}{0}
\verb{valva}{}{Zool.}{}{}{s.f.}{Cada uma das peças sólidas que compõem o corpo de um molusco.}{val.va}{0}
\verb{válvula}{}{}{}{}{s.f.}{Dispositivo que fecha um tubo, deixando que, sob pressão, o seu conteúdo passe apenas numa direção.}{vál.vu.la}{0}
\verb{vampirismo}{}{}{}{}{s.m.}{Crença na existência de vampiros.}{vam.pi.ris.mo}{0}
\verb{vampirismo}{}{}{}{}{}{Característica de vampiro; voracidade, avidez.}{vam.pi.ris.mo}{0}
\verb{vampiro}{}{}{}{}{s.m.}{Morto que, segundo tradição lendária, sai do túmulo à noite para sugar o sangue dos vivos; é vulnerável à luz do sol, a alho e a símbolos cristãos como a cruz e a água benta.}{vam.pi.ro}{0}
\verb{vampiro}{}{Fig.}{}{}{}{Indivíduo que enriquece explorando outrem.}{vam.pi.ro}{0}
\verb{vampiro}{}{Zool.}{}{}{}{Nome comum dado aos morcegos hematófagos, transmissores da raiva aos bovinos e que eventualmente ataca o homem.}{vam.pi.ro}{0}
\verb{vanádio}{}{Quím.}{}{}{s.m.}{Elemento químico metálico, branco"-prateado, maleável e dúctil, usado em ligas com o aço, na fabricação de catalisadores, cerâmica, tintas e corantes. \elemento{23}{50.9415}{V}.:}{va.ná.dio}{0}
\verb{vandalismo}{}{}{}{}{s.m.}{Destruição de monumentos ou outro patrimônio público ou privado por ignorância ou selvageria.}{van.da.lis.mo}{0}
\verb{vandalismo}{}{}{}{}{}{Ato próprio dos vândalos, povo germânico do século \textsc{v} que atacava, provocando ruína e devastação por onde passava.}{van.da.lis.mo}{0}
\verb{vândalo}{}{}{}{}{adj.}{Que comete atos de vandalismo, de destruição do patrimônio alheio ou público; bárbaro.}{vân.da.lo}{0}
\verb{vândalo}{}{}{}{}{}{Relativo aos vândalos.}{vân.da.lo}{0}
\verb{vândalo}{}{}{}{}{s.m.}{Indivíduo pertencente aos vândalos, povo germânico da Antiguidade que devastou o sul da Europa e o norte da África no século \textsc{v}.}{vân.da.lo}{0}
\verb{vanglória}{}{}{}{}{s.f.}{Convencimento infundado dos próprios méritos ou qualidades; vaidade, jactância.}{van.gló.ria}{0}
\verb{vangloriar}{}{}{}{}{v.t.}{Inspirar vanglória em alguém; envaidecer.}{van.glo.ri.ar}{0}
\verb{vangloriar}{}{}{}{}{v.pron.}{Ostentar os próprios méritos; gabar"-se, ufanar"-se.}{van.glo.ri.ar}{\verboinum{1}}
\verb{vanguarda}{}{}{}{}{s.f.}{Posição que encabeça uma sequência; dianteira, frente.}{van.guar.da}{0}
\verb{vanguarda}{}{Art.}{}{}{}{Movimento artístico renovador, que geralmente rompe com aquele que o precedeu.}{van.guar.da}{0}
\verb{vanguardista}{}{}{}{}{adj.2g.}{Que é próprio de movimento artístico renovador.}{van.guar.dis.ta}{0}
\verb{vanguardista}{}{}{}{}{}{Que segue esse tipo de movimento artístico.}{van.guar.dis.ta}{0}
\verb{vanilina}{}{Quím.}{}{}{s.f.}{Substância extraída de plantas da família das orquidáceas, usada em perfumes, aromatizantes etc.}{va.ni.li.na}{0}
\verb{vantagem}{}{}{"-ens}{}{s.f.}{Qualidade do que é superior ou do que está à frente; superioridade, dianteira.}{van.ta.gem}{0}
\verb{vantagem}{}{}{"-ens}{}{}{Benefício, privilégio, ganho.}{van.ta.gem}{0}
\verb{vantagem}{}{}{"-ens}{}{}{Utilidade, proveito, serventia.}{van.ta.gem}{0}
\verb{vantajoso}{ô}{}{"-oso ⟨ó⟩}{"-osa ⟨ó⟩}{adj.}{Que apresenta vantagem; proveitoso, lucrativo.}{van.ta.jo.so}{0}
\verb{vante}{}{}{}{}{s.m.}{A metade dianteira do navio.}{van.te}{0}
\verb{vão}{}{}{"-ãos}{vã}{s.m.}{Espaço vazio entre dois pontos. (\textit{A luz entrava pelo vão da porta.})}{vão}{0}
\verb{vão}{}{}{"-ãos}{vã}{adj.}{Sem sentido; inútil. (\textit{Lutar contra o tempo é uma luta vã.})}{vão}{0}
\verb{vão}{}{}{"-ãos}{vã}{}{Usado na expressão \textit{em vão}: inutilmente. (\textit{O viajante procurava em vão por um abrigo.})}{vão}{0}
\verb{vapor}{ô}{}{}{}{s.m.}{Gás resultante de um líquido ou de um sólido. }{va.por}{0}
\verb{vapor}{ô}{}{}{}{}{Conjunto de partículas gasosas que se difundem e ficam suspensas no ar.}{va.por}{0}
\verb{vapor}{ô}{}{}{}{}{Navio ou barco movido por máquina a vapor.}{va.por}{0}
\verb{vapor}{ô}{}{}{}{}{Trem movido por máquina a vapor.}{va.por}{0}
\verb{vaporar}{}{}{}{}{v.i.}{Converter"-se em vapor; vaporizar, evaporar.}{va.po.rar}{0}
\verb{vaporar}{}{}{}{}{v.t.}{Exalar vapor, fragrância; recender.}{va.po.rar}{\verboinum{1}}
\verb{vaporização}{}{}{"-ões}{}{s.f.}{Ato ou efeito de vaporizar; evaporação.}{va.po.ri.za.ção}{0}
\verb{vaporização}{}{Fís.}{"-ões}{}{}{Passagem de um corpo do estado líquido para o estado gasoso.}{va.po.ri.za.ção}{0}
\verb{vaporizador}{ô}{}{}{}{adj.}{Que vaporiza, pulveriza.}{va.po.ri.za.dor}{0}
\verb{vaporizador}{ô}{}{}{}{}{Diz"-se do instrumento que reduz os líquidos a vapor ou a gotículas minúsculas; pulverizador.}{va.po.ri.za.dor}{0}
\verb{vaporizar}{}{}{}{}{v.t.}{Converter em vapor; evaporar, vaporar.}{va.po.ri.zar}{0}
\verb{vaporizar}{}{}{}{}{}{Converter em forma de gotículas; pulverizar, borrifar.}{va.po.ri.zar}{\verboinum{1}}
\verb{vaporoso}{ô}{}{"-osos ⟨ó⟩}{"-osa ⟨ó⟩}{adj.}{Que contém ou produz vapores.}{va.po.ro.so}{0}
\verb{vaporoso}{ô}{}{"-osos ⟨ó⟩}{"-osa ⟨ó⟩}{}{Que tem aparência gasosa; leve.}{va.po.ro.so}{0}
\verb{vaporoso}{ô}{}{"-osos ⟨ó⟩}{"-osa ⟨ó⟩}{}{Extremamente delicado; diáfano, tênue.}{va.po.ro.so}{0}
\verb{vaporoso}{ô}{Fig.}{"-osos ⟨ó⟩}{"-osa ⟨ó⟩}{}{Que é difícil de compreender; nebuloso, obscuro.}{va.po.ro.so}{0}
\verb{vaqueiro}{ê}{}{}{}{s.m.}{Condutor ou guardador de gado vacum.}{va.quei.ro}{0}
\verb{vaquejada}{}{}{}{}{s.f.}{Ato de reunir o gado espalhado para marcação, apartação etc.}{va.que.ja.da}{0}
\verb{vaquejada}{}{}{}{}{}{Espécie de rodeio em que dois vaqueiros a cavalo devem derrubar um boi puxando"-o pelo rabo.}{va.que.ja.da}{0}
\verb{vaquejar}{}{}{}{}{v.t.}{Reunir o gado em rebanho; costear.}{va.que.jar}{\verboinum{1}}
\verb{vaqueta}{ê}{}{}{}{s.f.}{Vareta do guarda"-sol; baqueta.}{va.que.ta}{0}
\verb{vaquinha}{}{}{}{}{s.f.}{Rateio entre um grupo de pessoas para o pagamento de uma despesa comum.}{va.qui.nha}{0}
\verb{vaquinha}{}{}{}{}{}{Vaca pequena.}{va.qui.nha}{0}
\verb{vara}{}{}{}{}{s.f.}{Ramo fino e comprido, de bambu, árvore ou arbusto, que dobra com facilidade.}{va.ra}{0}
\verb{vara}{}{}{}{}{}{Cargo ou funções de juiz.}{va.ra}{0}
\verb{vara}{}{}{}{}{}{Cada uma das divisões de jurisdição, nas comarcas onde há mais de um juiz de direito. }{va.ra}{0}
\verb{vara}{}{}{}{}{}{Grupo de porcos; porcada, manada.}{va.ra}{0}
\verb{varada}{}{}{}{}{s.f.}{Golpe ou pancada de vara.}{va.ra.da}{0}
\verb{varado}{}{}{}{}{adj.}{Que foi espancado ou surrado com vara.}{va.ra.do}{0}
\verb{varado}{}{}{}{}{}{Furado de lado a lado; trespassado.}{va.ra.do}{0}
\verb{varado}{}{Pop.}{}{}{}{Estupefato, espantado, atônito.}{va.ra.do}{0}
\verb{varadouro}{ô}{}{}{}{s.m.}{Lugar raso junto ao litoral onde as embarcações costumam encalhar.}{va.ra.dou.ro}{0}
\verb{varal}{}{}{"-ais}{}{s.m.}{Arame, corda ou fio de plástico mantido por postes a certa altura, onde se estende a roupa lavada para secar.}{va.ral}{0}
\verb{varal}{}{}{"-ais}{}{}{Cada uma das duas peças de madeira que saem de cada parte lateral de um veículo ou máquina agrícola e entre as quais se atrela um animal de tração.}{va.ral}{0}
\verb{varanda}{}{}{}{}{s.f.}{Terraço com cobertura que fica à frente e em volta das casas; avarandado.}{va.ran.da}{0}
\verb{varanda}{}{}{}{}{}{Balcão ou sacada coberta; alpendre.}{va.ran.da}{0}
\verb{varandim}{}{}{"-ins}{}{s.m.}{Varanda estreira.}{va.ran.dim}{0}
\verb{varandim}{}{}{"-ins}{}{}{Grade baixa e elegante em que se apoia o peitoril nas janelas de sacada.}{va.ran.dim}{0}
\verb{varão}{}{}{"-ões}{}{s.m.}{Indivíduo do sexo masculino.}{va.rão}{0}
\verb{varão}{}{}{"-ões}{}{}{Homem que atingiu a idade adulta.}{va.rão}{0}
\verb{varão}{}{}{"-ões}{}{}{Homem digno de respeito, valoroso.}{va.rão}{0}
\verb{varapau}{}{}{}{}{s.m.}{Pau comprido que serve de apoio; cajado.}{va.ra.pau}{0}
\verb{varapau}{}{Fig.}{}{}{}{Pessoa muita magra e alta.}{va.ra.pau}{0}
\verb{varar}{}{}{}{}{v.t.}{Golpear, flagelar com vara.}{va.rar}{0}
\verb{varar}{}{}{}{}{}{Perfurar de lado a lado; atravessar, transpassar.}{va.rar}{0}
\verb{varar}{}{}{}{}{v.i.}{Sair de forma impetuosa.}{va.rar}{\verboinum{1}}
\verb{varejão}{"-ões}{}{}{}{s.m.}{Grande local onde se vendem mercadorias a granel.}{va.re.jão}{0}
\verb{varejar}{}{}{}{}{v.t.}{Agitar ou sacudir com vara.}{va.re.jar}{0}
\verb{varejar}{}{}{}{}{}{Bater com violência; dar pancadas.}{va.re.jar}{0}
\verb{varejar}{}{}{}{}{}{Destruir, devastar, assolar.}{va.re.jar}{\verboinum{1}}
\verb{varejeira}{ê}{Zool.}{}{}{s.f.}{Nome comum dado a moscas que depositam seus ovos em tecidos vivos ou mortos de animais ou em alimentos em decomposição.}{va.re.jei.ra}{0}
\verb{varejista}{}{}{}{}{adj.2g.}{Relativo ao comércio a varejo.}{va.re.jis.ta}{0}
\verb{varejista}{}{}{}{}{}{Que negocia a varejo.}{va.re.jis.ta}{0}
\verb{varejo}{ê}{}{}{}{s.m.}{Comércio de mercadorias em pequenas quantidades.}{va.re.jo}{0}
\verb{varejo}{ê}{}{}{}{}{Fogo de artilharia ou fuzilaria.}{va.re.jo}{0}
\verb{varela}{é}{}{}{}{s.f.}{Vareta.}{va.re.la}{0}
\verb{vareta}{ê}{}{}{}{s.f.}{Pequena vara; varela.}{va.re.ta}{0}
\verb{vareta}{ê}{}{}{}{}{Haste de metal ou madeira que compõe a armação do guarda"-chuva.}{va.re.ta}{0}
\verb{vargedo}{ê}{}{}{}{s.m.}{Várzea grande, extensa.}{var.ge.do}{0}
\verb{vargem}{}{}{"-ens}{}{s.f.}{Várzea.}{var.gem}{0}
\verb{variabilidade}{}{}{}{}{s.f.}{Qualidade do que é variável; volubilidade, inconstância.}{va.ri.a.bi.li.da.de}{0}
\verb{variação}{}{}{"-ões}{}{s.f.}{Ato ou efeito de variar.}{va.ri.a.ção}{0}
\verb{variação}{}{}{"-ões}{}{}{Modificação, alteração, mudança.}{va.ri.a.ção}{0}
\verb{variação}{}{}{"-ões}{}{}{Inconstância de princípios, sistemas etc.}{va.ri.a.ção}{0}
\verb{variado}{}{}{}{}{adj.}{Que sofreu variação.}{va.ri.a.do}{0}
\verb{variado}{}{}{}{}{}{Diferente de outro em espécie; diverso.}{va.ri.a.do}{0}
\verb{variado}{}{}{}{}{}{Que não apresenta estabilidade; instável, inconstante.}{va.ri.a.do}{0}
\verb{variado}{}{}{}{}{}{Que sofre perturbações mentais; alucinado, delirante, desvairado.}{va.ri.a.do}{0}
\verb{variante}{}{}{}{}{adj.2g.}{Que varia, difere.}{va.ri.an.te}{0}
\verb{variante}{}{}{}{}{s.f.}{Variação, diferença.}{va.ri.an.te}{0}
\verb{variante}{}{Gram.}{}{}{}{Cada uma das diferentes formas de um vocábulo apresentar"-se.}{va.ri.an.te}{0}
\verb{variar}{}{}{}{}{v.t.}{Tornar vário; diversificar.}{va.ri.ar}{0}
\verb{variar}{}{}{}{}{}{Alternar.}{va.ri.ar}{0}
\verb{variar}{}{}{}{}{v.i.}{Sofrer mudanças; mudar.}{va.ri.ar}{0}
\verb{variar}{}{}{}{}{}{Delirar, endoidecer.}{va.ri.ar}{\verboinum{6}}
\verb{variável}{}{}{"-eis}{}{adj.2g.}{Que pode ser variado, mudado; instável.}{va.ri.á.vel}{0}
\verb{variável}{}{Mat.}{"-eis}{}{s.f.}{Quantidade que, num mesmo cálculo, pode tomar sucessivamente diferentes valores.}{va.ri.á.vel}{0}
\verb{varicela}{é}{Med.}{}{}{s.f.}{Doença infecciosa e muito contagiosa, caracterizada por febre e erupção de pequenas bolhas que secam em poucos dias; catapora.}{va.ri.ce.la}{0}
\verb{varicocele}{é}{Med.}{}{}{s.f.}{Dilatação das veias do cordão espermático, causando tumor no escroto.}{va.ri.co.ce.le}{0}
\verb{varicoso}{ô}{}{"-osos ⟨ó⟩}{"-osa ⟨ó⟩}{adj.}{Diz"-se do vaso sanguíneo dilatado que forma varizes.}{va.ri.co.so}{0}
\verb{variedade}{}{}{}{}{s.f.}{Qualidade do que é vário ou variável.}{va.ri.e.da.de}{0}
\verb{variedade}{}{}{}{}{}{Conjunto de coisas que se diversificam em uma classe; diversidade, diferença, multiplicidade.}{va.ri.e.da.de}{0}
\verb{variedade}{}{}{}{}{}{Cada uma dessas coisas separadamente; variante, variação.}{va.ri.e.da.de}{0}
\verb{variedade}{}{}{}{}{}{Inconstância, instabilidade, volubilidade.}{va.ri.e.da.de}{0}
\verb{variedades}{}{}{}{}{s.f.pl.}{Conjunto de assuntos vários em literatura ou em jornalismo, e de apresentações em teatro, televisão, boates etc.}{va.ri.e.da.des}{0}
\verb{variegado}{}{}{}{}{adj.}{Diversificado, sortido, variado.}{va.ri.e.ga.do}{0}
\verb{variegado}{}{}{}{}{}{De cores diversas; matizado.}{va.ri.e.ga.do}{0}
\verb{variegar}{}{}{}{}{v.t.}{Tornar diversificado ou variado; diversificar, alternar, variar.}{va.ri.e.gar}{\verboinum{5}}
\verb{vário}{}{}{}{}{adj.}{De diversas matizes, cores, feitios, tipos etc.}{vá.rio}{0}
\verb{vário}{}{}{}{}{}{Caracterizado pela diversidade; múltiplo, numeroso.}{vá.rio}{0}
\verb{vário}{}{}{}{}{}{Que não é estável; inconstante, mutável.}{vá.rio}{0}
\verb{vário}{}{}{}{}{}{Fora de si; desvairado, delirante.}{vá.rio}{0}
\verb{vário}{}{}{}{}{pron.}{(\textit{vários})Muitos, diversos, alguns.}{vá.rio}{0}
\verb{varíola}{}{Med.}{}{}{s.f.}{Doença infecto contagiosa e epidêmica, caracterizada por febre alta, dores difusas e formação de pústulas; bexiga.}{va.rí.o.la}{0}
\verb{variolar}{}{}{}{}{adj.2g.}{Relativo à varíola.}{va.ri.o.lar}{0}
\verb{varioloso}{ô}{}{"-osos ⟨ó⟩}{"-osa ⟨ó⟩}{adj.}{Que sofre de varíola.}{va.ri.o.lo.so}{0}
\verb{variz}{}{Med.}{}{}{s.f.}{Dilatação permanente de uma veia ou artéria, geralmente da perna, causada por acumulação de sangue.}{va.riz}{0}
\verb{varonia}{}{}{}{}{s.f.}{Qualidade ou condição de varão.}{va.ro.ni.a}{0}
\verb{varonia}{}{}{}{}{}{Sucessão ou descendência por linhagem paterna.}{va.ro.ni.a}{0}
\verb{varonil}{}{}{"-is}{}{adj.2g.}{Relativo a varão; viril.}{va.ro.nil}{0}
\verb{varonil}{}{}{"-is}{}{}{Cheio de coragem; valoroso, heroico.}{va.ro.nil}{0}
\verb{varonil}{}{}{"-is}{}{}{Enérgico, firme, determinado.}{va.ro.nil}{0}
\verb{varrão}{}{}{"-ões}{}{s.m.}{Porco escolhido para ser o reprodutor, sendo poupado de sofrer castração.}{var.rão}{0}
\verb{varredor}{ô}{}{}{}{adj.}{Que varre.}{var.re.dor}{0}
\verb{varredura}{}{}{}{}{s.f.}{Ato ou efeito de varrer; varrição.}{var.re.du.ra}{0}
\verb{varredura}{}{}{}{}{}{Lixo que se acumula ao varrer.}{var.re.du.ra}{0}
\verb{varredura}{}{Informát.}{}{}{}{Exploração ponto a ponto de objeto ou imagem por meio de um feixe de elétrons ou de luz de pequena abertura.}{var.re.du.ra}{0}
\verb{varredura}{}{}{}{}{}{Ação de fazer passar feixe de radar ou sonar por um determinado setor a fim de identificar possíveis alvos.}{var.re.du.ra}{0}
\verb{varrer}{ê}{}{}{}{v.i.}{Limpar com vassoura.}{var.rer}{0}
\verb{varrer}{ê}{}{}{}{}{Tornar vazio; esgotar, exaurir.}{var.rer}{0}
\verb{varrer}{ê}{}{}{}{}{Fazer espalhar"-se; dispersar.}{var.rer}{0}
\verb{varrer}{ê}{}{}{}{}{Fazer desaparecer; eliminar, excluir.}{var.rer}{\verboinum{12}}
\verb{varrido}{}{}{}{}{adj.}{Que foi limpo com vassoura.}{var.ri.do}{0}
\verb{varrido}{}{}{}{}{}{Que perdeu o juízo; louco, alienado, maluco.}{var.ri.do}{0}
\verb{varsoviano}{}{}{}{}{adj.}{Relativo a Varsóvia, capital da Polônia.}{var.so.vi.a.no}{0}
\verb{varsoviano}{}{}{}{}{}{Indivíduo natural ou habitante dessa cidade.}{var.so.vi.a.no}{0}
\verb{várzea}{}{}{}{}{s.f.}{Terreno plano em vale extenso; vargem, planície.}{vár.ze.a}{0}
\verb{várzea}{}{}{}{}{}{Campo extenso, cultivável, junto aos rios e ribeirões.}{vár.ze.a}{0}
\verb{várzea}{}{}{}{}{}{Campo de futebol situado em terreno baldio, utilizado por jogadores amadores.}{vár.ze.a}{0}
\verb{vasa}{}{}{}{}{s.f.}{Lama que se deposita no fundo das águas.}{va.sa}{0}
\verb{vasa}{}{}{}{}{}{Terreno lodoso; lamaçal.}{va.sa}{0}
\verb{vasa}{}{Fig.}{}{}{}{Degradação moral; depravação, torpeza.}{va.sa}{0}
\verb{vasca}{}{}{}{}{s.f.}{Convulsão forte; ânsia excessiva.}{vas.ca}{0}
\verb{vasco}{}{}{}{}{}{Var. de \textit{basco}.}{vas.co}{0}
\verb{vascolejar}{}{}{}{}{v.t.}{Agitar, chocalhar, remexer.}{vas.co.le.jar}{0}
\verb{vascolejar}{}{}{}{}{}{Tornar inquieto; perturbar, abalar.}{vas.co.le.jar}{\verboinum{1}}
\verb{vasconço}{}{}{}{}{adj.}{Basco.}{vas.con.ço}{0}
\verb{vasconço}{}{Fig.}{}{}{}{Língua ininteligível, confusa.}{vas.con.ço}{0}
\verb{vascular}{}{}{}{}{adj.2g.}{Relativo aos vasos, particularmente aos vasos sanguíneos.}{vas.cu.lar}{0}
\verb{vascularização}{}{}{"-ões}{}{s.f.}{Formação e desenvolvimento de vasos sanguíneos e linfáticos num tecido que não os continha. }{vas.cu.la.ri.za.ção}{0}
\verb{vascularização}{}{}{"-ões}{}{}{O conjunto dos vasos de um órgão.}{vas.cu.la.ri.za.ção}{0}
\verb{vasculhador}{ô}{}{}{}{s.m.}{Vassoura de cabo comprido, que serve para limpar forros e tetos; vasculho.}{vas.cu.lha.dor}{0}
\verb{vasculhar}{}{}{}{}{v.t.}{Varrer com vasculho.}{vas.cu.lhar}{0}
\verb{vasculhar}{}{}{}{}{}{Pesquisar, investigar.}{vas.cu.lhar}{\verboinum{1}}
\verb{vasculho}{}{}{}{}{}{Varredouro para fornos.}{vas.cu.lho}{0}
\verb{vasculho}{}{}{}{}{s.m.}{Vassoura de teto; vasculhador.}{vas.cu.lho}{0}
\verb{vasectomia}{}{Med.}{}{}{s.f.}{Operação cirúrgica para esterilização masculina.}{va.sec.to.mi.a}{0}
\verb{vaselina}{}{}{}{}{s.f.}{Substância oleosa obtida de petróleo, utilizada como lubrificante ou como excipiente de pomadas.}{va.se.li.na}{0}
\verb{vasilha}{}{}{}{}{s.f.}{Qualquer vaso usado para guardar líquidos.}{va.si.lha}{0}
\verb{vasilha}{}{}{}{}{}{Pipa, tonel, barril.}{va.si.lha}{0}
\verb{vasilha}{}{}{}{}{}{Tipo de recipiente que serve para guardar ou conter líquidos ou sólidos.}{va.si.lha}{0}
\verb{vasilhame}{}{}{}{}{s.m.}{Conjunto de vasilhas.}{va.si.lha.me}{0}
\verb{vasilhame}{}{}{}{}{}{Recipiente para conter líquidos.}{va.si.lha.me}{0}
\verb{vaso}{}{}{}{}{s.m.}{Objeto com espaço interno, mais alto que largo, com a boca maior que o fundo e de forma arredondada, usado para diversas finalidades.}{va.so}{0}
\verb{vaso}{}{Anat.}{}{}{}{Tubo do corpo por onde corre o sangue ou outro líquido.}{va.so}{0}
\verb{vasoconstrição}{}{Med.}{"-ões}{}{s.f.}{Diminuição do diâmetro dos vasos sanguíneos.}{va.so.cons.tri.ção}{0}
\verb{vasodilatação}{}{Med.}{"-ões}{}{s.f.}{Aumento do diâmetro dos vasos sanguíneos.}{va.so.di.la.ta.ção}{0}
\verb{vasomotor}{ô}{Med.}{}{}{adj.}{Diz"-se dos nervos cuja ação determina a contração ou dilatação dos vasos sanguíneos.}{va.so.mo.tor}{0}
\verb{vasqueiro}{ê}{}{}{}{adj.}{Que é difícil de conseguir ou de encontrar; escasso, raro.}{vas.quei.ro}{0}
\verb{vassalagem}{}{}{"-ens}{}{s.f.}{Estado ou condição de vassalo.}{vas.sa.la.gem}{0}
\verb{vassalagem}{}{}{"-ens}{}{}{Tributo pago pelo vassalo ao suserano.}{vas.sa.la.gem}{0}
\verb{vassalagem}{}{Por ext.}{"-ens}{}{}{Sujeição, submissão.}{vas.sa.la.gem}{0}
\verb{vassalagem}{}{Fig.}{"-ens}{}{}{Obediência, tributo, homenagem.}{vas.sa.la.gem}{0}
\verb{vassalo}{}{}{}{}{s.m.}{No sistema feudal, indivíduo que, mediante juramento de fé e fidelidade a um suserano, dele se tornava dependente, rendendo"-lhe preito e tributo.}{vas.sa.lo}{0}
\verb{vassalo}{}{}{}{}{}{Indivíduo que é súdito de um soberano.}{vas.sa.lo}{0}
\verb{vassalo}{}{}{}{}{}{Indivíduo que paga tributo a alguém.}{vas.sa.lo}{0}
\verb{vassalo}{}{Fig.}{}{}{adj.}{Que é dependente, submisso.}{vas.sa.lo}{0}
\verb{vassoura}{ô}{}{}{}{s.f.}{Utensílio doméstico para varrer o chão, constituído de ramos, fios de piaçaba ou fios plásticos, presos a um cabo longo.}{vas.sou.ra}{0}
\verb{vassoura}{ô}{Bot.}{}{}{}{Planta cujos ramos são utilizados na confecção de vassouras.}{vas.sou.ra}{0}
\verb{vassourada}{}{}{}{}{s.f.}{Pancada com vassoura.}{vas.sou.ra.da}{0}
\verb{vassourada}{}{}{}{}{}{Aquilo que se varre com um só movimento da vassoura.}{vas.sou.ra.da}{0}
\verb{vassourar}{}{}{}{}{v.t.}{Varrer com vassoura.}{vas.sou.rar}{\verboinum{1}}
\verb{vassoureiro}{ê}{}{}{}{s.m.}{Fabricante ou vendedor de vassouras.}{vas.sou.rei.ro}{0}
\verb{vastidão}{}{}{"-ões}{}{s.f.}{Qualidade do que é vasto.}{vas.ti.dão}{0}
\verb{vastidão}{}{}{"-ões}{}{}{Espaço de grande dimensão; amplidão.}{vas.ti.dão}{0}
\verb{vastidão}{}{Fig.}{"-ões}{}{}{Magnitude, grandeza, importância.}{vas.ti.dão}{0}
\verb{vasto}{}{}{}{}{adj.}{Que é muito extenso; amplo, dilatado.}{vas.to}{0}
\verb{vasto}{}{Fig.}{}{}{}{Que é considerável, grande, importante.}{vas.to}{0}
\verb{vatapá}{}{Cul.}{}{}{s.m.}{Prato tipico da cozinha baiana, muito apimentado, feito com peixe ou galinha, a que se adiciona leite de coco, camarões secos e frescos, pão de véspera, amendoim e castanha"-de"-caju torrados e moídos, e que se tempera com azeite"-de"-dendê.}{va.ta.pá}{0}
\verb{vate}{}{}{}{}{s.m.}{Indivíduo que faz vaticínio, predição; profeta, vidente.}{va.te}{0}
\verb{vate}{}{}{}{}{}{Poeta, bardo.}{va.te}{0}
\verb{vaticano}{}{}{}{}{adj.}{Relativo ao Vaticano, estado soberano cujo chefe é o papa.}{va.ti.ca.no}{0}
\verb{vaticano}{}{}{}{}{s.m.}{Palácio do papa, em Roma.}{va.ti.ca.no}{0}
\verb{vaticinar}{}{}{}{}{v.t.}{Predizer, adivinhar, prenunciar.}{va.ti.ci.nar}{\verboinum{1}}
\verb{vaticínio}{}{}{}{}{s.m.}{Ato ou efeito de vaticinar; profecia, predição.}{va.ti.cí.nio}{0}
\verb{vau}{}{}{}{}{s.m.}{Local raso de um rio, mar ou lagoa, por onde se pode passar a pé ou a cavalo.}{vau}{0}
\verb{vaza}{}{}{}{}{s.f.}{A coleção das cartas jogadas por todos os jogadores em uma rodada e que é recolhida pelo ganhador.}{va.za}{0}
\verb{vazado}{}{}{}{}{adj.}{Que é derramado, despejado.}{va.za.do}{0}
\verb{vazado}{}{}{}{}{}{Atravessado, perfurado.}{va.za.do}{0}
\verb{vazadoiro}{ô}{}{}{}{}{Var. de \textit{vazadouro}.}{va.za.doi.ro}{0}
\verb{vazador}{ô}{}{}{}{adj.}{Que vaza.}{va.za.dor}{0}
\verb{vazador}{ô}{}{}{}{s.m.}{Instrumento usado para furar panos, papelão ou couro.}{va.za.dor}{0}
\verb{vazadouro}{ô}{}{}{}{s.m.}{Lugar onde se despejam detritos, ou onde se vaza qualquer líquido.}{va.za.dou.ro}{0}
\verb{vazamento}{}{}{}{}{s.m.}{Ato ou efeito de vazar.}{va.za.men.to}{0}
\verb{vazante}{}{}{}{}{adj.2g.}{Que vaza.}{va.zan.te}{0}
\verb{vazante}{}{}{}{}{s.f.}{Refluxo.}{va.zan.te}{0}
\verb{vazante}{}{}{}{}{}{Período em que um rio apresenta o menor volume de águas.}{va.zan.te}{0}
\verb{vazante}{}{}{}{}{}{Escoamento, saída, vazão.}{va.zan.te}{0}
\verb{vazão}{}{}{"-ões}{}{s.f.}{Ato ou efeito de vazar.}{va.zão}{0}
\verb{vazão}{}{}{"-ões}{}{}{Movimento de saída; deslocamento, escoamento.}{va.zão}{0}
\verb{vazão}{}{Fig.}{"-ões}{}{}{Venda, comercialização.}{va.zão}{0}
\verb{vazar}{}{}{}{}{v.t.}{Fazer uma obra com o metal derretido; fundir.}{va.zar}{0}
\verb{vazar}{}{}{}{}{}{Passar através da superfície de alguma coisa; atravessar, transpassar.}{va.zar}{0}
\verb{vazar}{}{}{}{}{}{Chegar uma informação sigilosa ao conhecimento de outras pessoas, por denúncia, engano, indiscrição ou negligência.}{va.zar}{0}
\verb{vazar}{}{}{}{}{v.i.}{Deixar escapar algum líquido.}{va.zar}{0}
\verb{vazar}{}{}{}{}{}{Derramar"-se por algum lugar.}{va.zar}{\verboinum{1}}
\verb{vazio}{}{}{}{}{adj.}{Que não contém nada; que só contém ar.}{va.zi.o}{0}
\verb{vazio}{}{}{}{}{}{Que se encontra desocupado, despovoado, desabitado.}{va.zi.o}{0}
\verb{vazio}{}{}{}{}{}{Que é frívolo, vão, fútil.}{va.zi.o}{0}
\verb{vazio}{}{}{}{}{}{Que é destituído de inteligência.}{va.zi.o}{0}
\verb{vazio}{}{Por ext.}{}{}{}{Falto, destituído, desprovido.}{va.zi.o}{0}
\verb{vazio}{}{Mat.}{}{}{}{Diz"-se do conjunto que não comporta nenhum elemento.}{va.zi.o}{0}
\verb{vazio}{}{}{}{}{s.m.}{Espaço vazio.}{va.zi.o}{0}
\verb{vazio}{}{Fig.}{}{}{}{Sentimento angustiante, produzido por saudade, privação ou ausência.}{va.zi.o}{0}
\verb{vê}{}{}{}{}{s.m.}{Nome da letra \textit{v}.}{vê}{0}
\verb{veado}{}{Zool.}{}{}{s.m.}{Mamífero ruminante, veloz, geralmente de coloração acastanhada, que apresenta chifres simples ou ramificados.}{ve.a.do}{0}
\verb{veado}{}{Pop.}{}{}{}{Homossexual masculino.}{ve.a.do}{0}
\verb{vector}{ô}{}{}{}{}{Var. de \textit{vetor}.}{vec.tor}{0}
\verb{vedação}{}{}{"-ões}{}{s.f.}{Ato ou efeito de vedar; proibição, impedimento.}{ve.da.ção}{0}
\verb{vedação}{}{}{"-ões}{}{}{Aquilo que serve para vedar.}{ve.da.ção}{0}
\verb{vedar}{}{}{}{}{v.t.}{Impedir, proibir, interditar.}{ve.dar}{0}
\verb{vedar}{}{}{}{}{}{Estorvar, tolher.}{ve.dar}{0}
\verb{vedar}{}{}{}{}{}{Impedir que corra; estancar.}{ve.dar}{0}
\verb{vedar}{}{}{}{}{}{Fechar, tapar.}{ve.dar}{\verboinum{1}}
\verb{vedeta}{ê}{}{}{}{s.f.}{Guarita de sentinela em lugar alto.}{ve.de.ta}{0}
\verb{vedeta}{ê}{}{}{}{}{Guarda avançada.}{ve.de.ta}{0}
\verb{vedeta}{ê}{}{}{}{}{Vedete.}{ve.de.ta}{0}
\verb{vedete}{é}{}{}{}{s.f.}{Atriz de destaque no teatro de revista.}{ve.de.te}{0}
\verb{vedete}{é}{Por ext.}{}{}{}{Artista principal de um espetáculo teatral ou cinemetográfico.}{ve.de.te}{0}
\verb{vedete}{é}{Fig.}{}{}{}{Indivíduo que está em evidência.}{ve.de.te}{0}
\verb{vedetismo}{}{}{}{}{s.m.}{Comportamento de vedete; estrelismo.}{ve.de.tis.mo}{0}
\verb{veeiro}{ê}{}{}{}{s.m.}{Linha pela qual uma pedra se parte quando batida.}{ve.ei.ro}{0}
\verb{veeiro}{ê}{}{}{}{}{Parte da mina onde se acha o mineral; filão.}{ve.ei.ro}{0}
\verb{veemência}{}{}{}{}{s.f.}{Qualidade ou estado de veemente.}{ve.e.mên.cia}{0}
\verb{veemência}{}{}{}{}{}{Impetuosidade, vivacidade, eloquência.}{ve.e.mên.cia}{0}
\verb{veemente}{}{}{}{}{adj.2g.}{Que é impetuoso, animado, arrojado.}{ve.e.men.te}{0}
\verb{veemente}{}{}{}{}{}{Que é forte, vigoroso.}{ve.e.men.te}{0}
\verb{veemente}{}{}{}{}{}{Que é entusiástico, fervoroso.}{ve.e.men.te}{0}
\verb{vegetação}{}{}{"-ões}{}{s.f.}{Ato ou efeito de vegetar.}{ve.ge.ta.ção}{0}
\verb{vegetação}{}{}{"-ões}{}{}{Força vegetativa.}{ve.ge.ta.ção}{0}
\verb{vegetação}{}{}{"-ões}{}{}{Conjunto de plantas que cobre uma região.}{ve.ge.ta.ção}{0}
\verb{vegetal}{}{}{"-ais}{}{adj.2g.}{Relativo às plantas.}{ve.ge.tal}{0}
\verb{vegetal}{}{}{"-ais}{}{}{Proveniente dos vegetais.}{ve.ge.tal}{0}
\verb{vegetal}{}{}{"-ais}{}{s.m.}{Ser vivo, geralmente clorofilado e fixado ao solo; planta.}{ve.ge.tal}{0}
\verb{vegetar}{}{}{}{}{v.i.}{Crescer e desenvolver"-se (as plantas).}{ve.ge.tar}{0}
\verb{vegetar}{}{Fig.}{}{}{}{Viver como um vegetal; ser incapaz de mover"-se ou sentir emoções.}{ve.ge.tar}{\verboinum{1}}
\verb{vegetarianismo}{}{}{}{}{s.m.}{Regime alimentar baseado unicamente no uso de vegetais.}{ve.ge.ta.ri.a.nis.mo}{0}
\verb{vegetariano}{}{}{}{}{adj.}{Diz"-se daquele que se alimenta exclusivamente de vegetais.}{ve.ge.ta.ri.a.no}{0}
\verb{vegetativo}{}{}{}{}{adj.}{Que faz vegetar.}{ve.ge.ta.ti.vo}{0}
\verb{vegetativo}{}{}{}{}{}{Relativo ao crescimento, em animais e plantas.}{ve.ge.ta.ti.vo}{0}
\verb{vegetativo}{}{}{}{}{}{Que funciona sem a participação da vontade e da consciência.}{ve.ge.ta.ti.vo}{0}
\verb{vegetomineral}{}{}{"-ais}{}{adj.2g.}{Que é composto de substâncias vegetais e minerais.}{ve.ge.to.mi.ne.ral}{0}
\verb{veia}{ê}{Anat.}{}{}{s.f.}{Vaso que transporta o sangue dos órgãos para o coração.}{vei.a}{0}
\verb{veia}{ê}{}{}{}{}{Veio mineral.}{vei.a}{0}
\verb{veia}{ê}{Fig.}{}{}{}{Tendência, vocação.}{vei.a}{0}
\verb{veia}{ê}{Bot.}{}{}{}{Cada uma das nervuras secundárias das folhas dos vegetais.}{vei.a}{0}
\verb{veiculação}{}{}{"-ões}{}{s.f.}{Ato ou efeito de veicular.}{ve.i.cu.la.ção}{0}
\verb{veiculação}{}{}{"-ões}{}{}{Deslocamento de um lugar para outro por  meio de veículos.}{ve.i.cu.la.ção}{0}
\verb{veiculação}{}{}{"-ões}{}{}{Divulgação de uma mensagem publicitária por meio de veículos de comunicação.}{ve.i.cu.la.ção}{0}
\verb{veicular}{}{}{}{}{adj.2g.}{Relativo a veículo.}{ve.i.cu.lar}{0}
\verb{veicular}{}{}{}{}{v.t.}{Transportar em veículo.}{ve.i.cu.lar}{0}
\verb{veicular}{}{}{}{}{}{Levar de um lugar para outro; transportar, conduzir.}{ve.i.cu.lar}{0}
\verb{veicular}{}{}{}{}{}{Propagar, transmitir.}{ve.i.cu.lar}{0}
\verb{veicular}{}{}{}{}{}{Promover a divulgação de um anúncio numa campanha publicitária.}{ve.i.cu.lar}{\verboinum{1}}
\verb{veículo}{}{}{}{}{s.m.}{Qualquer meio de transporte.}{ve.í.cu.lo}{0}
\verb{veículo}{}{}{}{}{}{Tudo aquilo que transmite ou conduz.}{ve.í.cu.lo}{0}
\verb{veículo}{}{}{}{}{}{O que auxilia ou promove.}{ve.í.cu.lo}{0}
\verb{veículo}{}{}{}{}{}{Excipiente líquido.}{ve.í.cu.lo}{0}
\verb{veiga}{ê}{}{}{}{s.f.}{Campo fértil e cultivado; várzea.}{vei.ga}{0}
\verb{veio}{ê}{}{}{}{s.m.}{Faixa de terra ou de rocha que se diferencia da que a ladeia pela natureza e pela cor.}{vei.o}{0}
\verb{veio}{ê}{}{}{}{}{Parte da mina onde se acha o mineral; filão.}{vei.o}{0}
\verb{veio}{ê}{Por ext.}{}{}{}{Estria, risco.}{vei.o}{0}
\verb{veio}{ê}{Fig.}{}{}{}{Fundamento, base, essência.}{vei.o}{0}
\verb{vela}{é}{Por ext.}{}{}{}{Embarcação movida a vela.}{ve.la}{0}
\verb{vela}{é}{}{}{}{s.f.}{Conjunto de peças de tecido forte que se prende aos mastros da embarcação, para que o vento a impulsione.}{ve.la}{0}
\verb{vela}{é}{}{}{}{s.f.}{Ato de velar; vigília. }{ve.la}{0}
\verb{vela}{é}{}{}{}{}{Rolo de cera com pavio que se acende e que serve para iluminar.}{ve.la}{0}
\verb{velado}{}{}{}{}{adj.}{Que está coberto com véu.}{ve.la.do}{0}
\verb{velado}{}{}{}{}{}{Oculto, encoberto, escondido.}{ve.la.do}{0}
\verb{velado}{}{}{}{}{}{Que é objeto de vigília.}{ve.la.do}{0}
\verb{velame}{}{}{}{}{s.m.}{O conjunto das velas de uma embarcação.}{ve.la.me}{0}
\verb{velar}{}{}{}{}{v.t.}{Cobrir com véu.}{ve.lar}{0}
\verb{velar}{}{}{}{}{}{Encobrir alguma coisa; esconder, ocultar.}{ve.lar}{0}
\verb{velar}{}{}{}{}{}{Ficar olhando uma pessoa para que nada aconteça a ela; cuidar, vigiar.}{ve.lar}{0}
\verb{velar}{}{}{}{}{}{Ficar longo tempo junto de um morto.}{ve.lar}{0}
\verb{velar}{}{}{}{}{}{Estar de guarda ou de sentinela.}{ve.lar}{0}
\verb{velar}{}{}{}{}{}{Ficar escuro por receber luz.  }{ve.lar}{0}
\verb{velar}{}{}{}{}{}{Passar a noite acordado.}{ve.lar}{\verboinum{1}}
\verb{velar}{}{}{}{}{adj.2g.}{Relativo ao palato, céu da boca.}{ve.lar}{0}
\verb{velarizar}{}{Gram.}{}{}{v.t.}{Dar caráter velar a um fonema.}{ve.la.ri.zar}{\verboinum{1}}
\verb{velcro}{é}{}{}{}{s.m.}{Tecido fabricado em tiras duplas aderentes, usado como fecho.}{vel.cro}{0}
\verb{veleidade}{}{}{}{}{s.f.}{Vontade imperfeita; intenção passageira.}{ve.lei.da.de}{0}
\verb{veleidade}{}{}{}{}{}{Pretensão, intenção.}{ve.lei.da.de}{0}
\verb{veleidade}{}{}{}{}{}{Quimera, fantasia.}{ve.lei.da.de}{0}
\verb{veleidade}{}{}{}{}{}{Leviandade, irreflexão.}{ve.lei.da.de}{0}
\verb{veleiro}{ê}{}{}{}{s.m.}{Barco a vela.}{ve.lei.ro}{0}
\verb{velejar}{}{}{}{}{v.i.}{Navegar a vela.}{ve.le.jar}{\verboinum{1}}
\verb{velha}{é}{}{}{}{}{Mulher de idade avançada.}{ve.lha}{0}
\verb{velha}{é}{Pop.}{}{}{}{Forma carinhosa de chamar a mãe ou a esposa.}{ve.lha}{0}
\verb{velhacaria}{}{}{}{}{s.f.}{Comportamento de pessoa que é velhaca; patifaria.}{ve.lha.ca.ri.a}{0}
\verb{velhaco}{}{}{}{}{adj.}{Que ludibria de propósito ou por má índole.}{ve.lha.co}{0}
\verb{velhaco}{}{}{}{}{}{Que é traiçoeiro ou fraudulento.}{ve.lha.co}{0}
\verb{velhaco}{}{}{}{}{s.m.}{Indivíduo que se utiliza de má"-fé e que engana e prejudica outrem.}{ve.lha.co}{0}
\verb{velhada}{}{}{}{}{s.f.}{Grupo de velhos.}{ve.lha.da}{0}
\verb{velhaquear}{}{}{}{}{v.i.}{Proceder como velhaco; praticar velhacaria.}{ve.lha.que.ar}{\verboinum{4}}
\verb{velharia}{}{}{}{}{s.f.}{O que é próprio de pessoa velha.}{ve.lha.ri.a}{0}
\verb{velharia}{}{}{}{}{}{Objeto antigo.}{ve.lha.ri.a}{0}
\verb{velharia}{}{}{}{}{}{Costume antiquado.}{ve.lha.ri.a}{0}
\verb{velhice}{}{}{}{}{s.f.}{Estado ou condição de velho.}{ve.lhi.ce}{0}
\verb{velhice}{}{}{}{}{}{Idade avançada.}{ve.lhi.ce}{0}
\verb{velhice}{}{}{}{}{}{Os velhos.}{ve.lhi.ce}{0}
\verb{velho}{é}{}{}{}{adj.}{Que é muito idoso.}{ve.lho}{0}
\verb{velho}{é}{}{}{}{}{Que é de época remota; antigo.}{ve.lho}{0}
\verb{velho}{é}{}{}{}{}{Gasto pelo uso; muito usado.}{ve.lho}{0}
\verb{velho}{é}{}{}{}{}{Que é antiquado, obsoleto.}{ve.lho}{0}
\verb{velho}{é}{}{}{}{s.m.}{Homem de idade avançada.}{ve.lho}{0}
\verb{velho}{é}{Pop.}{}{}{}{Forma carinhosa de chamar o pai ou o marido.}{ve.lho}{0}
\verb{velhote}{ó}{}{}{velhota ⟨ó⟩}{adj.}{Um tanto velho; velhusco.}{ve.lho.te}{0}
\verb{velhote}{ó}{}{}{velhota ⟨ó⟩}{s.m.}{Homem já um tanto velho, alegre, bem disposto.}{ve.lho.te}{0}
\verb{velhusco}{}{}{}{}{adj.}{Velho; velhote.}{ve.lhus.co}{0}
\verb{velhusco}{}{}{}{}{s.m.}{Homem já um tanto velho, alegre, bem disposto.}{ve.lhus.co}{0}
\verb{velo}{é}{}{}{}{s.m.}{Lã de carneiro, ovelha ou cordeiro.}{ve.lo}{0}
\verb{velo}{é}{}{}{}{}{Pele desses animais, com lã; velocino.}{ve.lo}{0}
\verb{velo}{é}{}{}{}{}{A lã dessa pele.}{ve.lo}{0}
\verb{velocidade}{}{}{}{}{s.f.}{Qualidade de veloz; rapidez, ligeireza.}{ve.lo.ci.da.de}{0}
\verb{velocidade}{}{}{}{}{}{Movimento rápido.}{ve.lo.ci.da.de}{0}
\verb{velocidade}{}{Fís.}{}{}{}{Distância percorrida na unidade de tempo.}{ve.lo.ci.da.de}{0}
\verb{velocímetro}{}{}{}{}{s.m.}{Instrumento para medir velocidade.}{ve.lo.cí.me.tro}{0}
\verb{velocino}{}{}{}{}{s.m.}{Pele de carneiro, ovelha ou cordeiro, com lã.}{ve.lo.ci.no}{0}
\verb{velocípede}{}{}{}{}{s.m.}{Veículo com três rodas, que se empurra com os pés.}{ve.lo.cí.pe.de}{0}
\verb{velódromo}{}{}{}{}{s.m.}{Pista para corridas de bicicletas.}{ve.ló.dro.mo}{0}
\verb{velório}{}{}{}{}{s.m.}{Ato de velar, em companhia de parentes e amigos, um defunto.}{ve.ló.rio}{0}
\verb{velório}{}{}{}{}{}{Local onde esse ato se realiza.}{ve.ló.rio}{0}
\verb{veloso}{ô}{}{"-osos ⟨ó⟩}{"-osa ⟨ó⟩}{adj.}{Que tem velo; lanoso, felpudo.}{ve.lo.so}{0}
\verb{veloz}{ó}{}{}{}{adj.2g.}{Que anda ou corre com rapidez.}{ve.loz}{0}
\verb{veloz}{ó}{}{}{}{}{Que passa ou se move depressa.}{ve.loz}{0}
\verb{veludo}{}{}{}{}{s.m.}{Tecido, natural ou sintético, de seda, algodão ou lã, que tem o avesso liso e lado de fora coberto de pelos cerrados e curtos.}{ve.lu.do}{0}
\verb{veludo}{}{Por ext.}{}{}{}{Objeto ou superfície macia.}{ve.lu.do}{0}
\verb{veludoso}{ô}{}{"-osos ⟨ó⟩}{"-osa ⟨ó⟩}{adj.}{Que é macio e lustroso como o veludo.}{ve.lu.do.so}{0}
\verb{velutíneo}{}{}{}{}{adj.}{Veludoso.}{ve.lu.tí.neo}{0}
\verb{venábulo}{}{}{}{}{s.m.}{Espécie de lança antiga, para caça de feras.}{ve.ná.bu.lo}{0}
\verb{venábulo}{}{Fig.}{}{}{}{Meio de defesa; expediente.}{ve.ná.bu.lo}{0}
\verb{venado}{}{}{}{}{adj.}{Que tem veias.}{ve.na.do}{0}
\verb{venal}{}{}{"-ais}{}{adj.2g.}{Relativo a venda.}{ve.nal}{0}
\verb{venal}{}{}{"-ais}{}{}{Que pode ser vendido.}{ve.nal}{0}
\verb{venal}{}{Fig.}{"-ais}{}{}{Que se corrompe por dinheiro; corrupto.}{ve.nal}{0}
\verb{venal}{}{}{"-ais}{}{adj.2g.}{Relativo a veia; venoso.}{ve.nal}{0}
\verb{venalidade}{}{}{}{}{s.f.}{Qualidade do que pode ser vendido.}{ve.na.li.da.de}{0}
\verb{venatório}{}{}{}{}{adj.}{Relativo a caça.}{ve.na.tó.rio}{0}
\verb{vencedor}{ô}{}{}{}{adj.}{Que vence ou venceu; vitorioso.}{ven.ce.dor}{0}
\verb{vencedor}{ô}{}{}{}{s.m.}{Indivíduo ou coisa vitoriosa.}{ven.ce.dor}{0}
\verb{vencer}{ê}{}{}{}{v.t.}{Conseguir vitória; triunfar; obter vantagem.}{ven.cer}{0}
\verb{vencer}{ê}{}{}{}{}{Levar a cabo; executar, realizar.}{ven.cer}{0}
\verb{vencer}{ê}{}{}{}{}{Percorrer ultrapassando; cobrir.}{ven.cer}{0}
\verb{vencer}{ê}{}{}{}{}{Dominar, submeter.}{ven.cer}{0}
\verb{vencer}{ê}{}{}{}{v.i.}{Chegar ao fim do prazo de validade.}{ven.cer}{\verboinum{15}}
\verb{vencido}{}{}{}{}{adj.}{Que sofreu derrota; derrotado.}{ven.ci.do}{0}
\verb{vencido}{}{}{}{}{}{Que se venceu, que teve o prazo expirado.}{ven.ci.do}{0}
\verb{vencido}{}{}{}{}{s.m.}{Indivíduo que foi derrotado.}{ven.ci.do}{0}
\verb{vencido}{}{Fig.}{}{}{}{Indivíduo que perdeu a coragem, o ânimo.}{ven.ci.do}{0}
\verb{vencilho}{}{}{}{}{s.m.}{Atilho de vime, verga, palha, com que se prendem videiras, amarram"-se feixes etc.}{ven.ci.lho}{0}
\verb{vencimento}{}{}{}{}{s.m.}{Ato ou efeito de vencer.}{ven.ci.men.to}{0}
\verb{vencimento}{}{}{}{}{}{Ato de terminar o prazo para o pagamento de um título ou para o cumprimento de qualquer encargo.}{ven.ci.men.to}{0}
\verb{vencimento}{}{}{}{}{}{Salário ou ordenado de um emprego ou cargo público.(Mais usado no plural.)}{ven.ci.men.to}{0}
\verb{vencível}{}{}{"-eis}{}{adj.2g.}{Que se pode vencer.}{ven.cí.vel}{0}
\verb{vencível}{}{}{"-eis}{}{}{Que tem vencimento em certa data.}{ven.cí.vel}{0}
\verb{venda}{}{}{}{}{s.f.}{Ato ou efeito de vender.}{ven.da}{0}
\verb{venda}{}{}{}{}{}{Loja de secos e molhados; armazém.}{ven.da}{0}
\verb{venda}{}{}{}{}{}{Bar, botequim, taberna.}{ven.da}{0}
\verb{venda}{}{}{}{}{s.f.}{Faixa de pano com que se cobrem os olhos.}{ven.da}{0}
\verb{vendagem}{}{}{"-ens}{}{}{Ato ou efeito de vender.}{ven.da.gem}{0}
\verb{vendagem}{}{}{}{}{s.f.}{Ato de vendar os olhos.}{ven.da.gem}{0}
\verb{vendagem}{}{}{"-ens}{}{s.f.}{Porcentagem recebida por quem vende por comissão.}{ven.da.gem}{0}
\verb{vendar}{}{}{}{}{v.t.}{Cobrir os olhos com venda.}{ven.dar}{\verboinum{1}}
\verb{vendaval}{}{}{"-ais}{}{s.m.}{Vento forte e tempestuoso; temporal.}{ven.da.val}{0}
\verb{vendaval}{}{Fig.}{"-ais}{}{}{Tumulto, desordem.}{ven.da.val}{0}
\verb{vendável}{}{}{"-eis}{}{adj.2g.}{Que tem boa venda; que se vende com facilidade.}{ven.dá.vel}{0}
\verb{vendedor}{ô}{}{}{}{adj.}{Que vende.}{ven.de.dor}{0}
\verb{vendedor}{ô}{}{}{}{s.m.}{Indivíduo que vende ou que tem por profissão vender.}{ven.de.dor}{0}
\verb{vendeiro}{ê}{}{}{}{s.m.}{Dono de venda.}{ven.dei.ro}{0}
\verb{vender}{ê}{}{}{}{v.t.}{Entregar alguma coisa a alguém em troca de dinheiro.}{ven.der}{0}
\verb{vender}{ê}{}{}{}{v.pron.}{Receber dinheiro para fazer alguma coisa desonesta; deixar"-se subornar.}{ven.der}{\verboinum{12}}
\verb{vendeta}{ê}{}{}{}{s.f.}{Sentimento de vingança, entre famílias, provocado por um assassinato ou uma ofensa.}{ven.de.ta}{0}
\verb{vendeta}{ê}{Por ext.}{}{}{}{Vingança.}{ven.de.ta}{0}
\verb{vendido}{}{}{}{}{adj.}{Que se vendeu.}{ven.di.do}{0}
\verb{vendido}{}{}{}{}{}{Adquirido por venda.}{ven.di.do}{0}
\verb{vendido}{}{Fig.}{}{}{}{Subornado, traído.}{ven.di.do}{0}
\verb{vendido}{}{}{}{}{s.m.}{Indivíduo que se vendeu.}{ven.di.do}{0}
\verb{vendilhão}{}{}{"-ões}{}{s.m.}{Indivíduo que vende suas mercadorias pelas ruas, sem um lugar fixo.}{ven.di.lhão}{0}
\verb{vendilhão}{}{}{"-ões}{}{}{Dono de venda.}{ven.di.lhão}{0}
\verb{vendilhão}{}{Fig.}{"-ões}{}{}{Indivíduo que trafica com coisas de ordem moral.}{ven.di.lhão}{0}
\verb{vendível}{}{}{"-eis}{}{adj.2g.}{Que pode ser vendido; próprio para venda.}{ven.dí.vel}{0}
\verb{venefício}{}{}{}{}{s.m.}{Preparo de veneno.}{ve.ne.fí.cio}{0}
\verb{venefício}{}{}{}{}{}{Ação de envenenar alguém.}{ve.ne.fí.cio}{0}
\verb{venéfico}{}{}{}{}{adj.}{Que tem veneno; venenoso.}{ve.né.fi.co}{0}
\verb{veneno}{}{}{}{}{s.m.}{Substância que perturba ou interrompe as funções vitais do organismo.}{ve.ne.no}{0}
\verb{veneno}{}{Fig.}{}{}{}{Aquilo que causa prejuízo moral.}{ve.ne.no}{0}
\verb{veneno}{}{}{}{}{}{Maldade, perversidade, deturpação.}{ve.ne.no}{0}
\verb{venenoso}{ô}{}{"-osos ⟨ó⟩}{"-osa ⟨ó⟩}{adj.}{Que contém veneno; tóxico.}{ve.ne.no.so}{0}
\verb{venenoso}{ô}{Fig.}{"-osos ⟨ó⟩}{"-osa ⟨ó⟩}{}{Maldoso, perverso, maligno.}{ve.ne.no.so}{0}
\verb{venera}{é}{}{}{}{s.f.}{Insígnia dos condecorados em uma ordem militar.}{ve.ne.ra}{0}
\verb{venera}{é}{}{}{}{}{Qualquer condecoração.}{ve.ne.ra}{0}
\verb{veneração}{}{}{"-ões}{}{s.f.}{Ato de venerar.}{ve.ne.ra.ção}{0}
\verb{veneração}{}{}{"-ões}{}{}{Respeito, devoção, adoração.}{ve.ne.ra.ção}{0}
\verb{venerando}{}{}{}{}{adj.}{Que é digno de veneração; venerável, respeitável.}{ve.ne.ran.do}{0}
\verb{venerar}{}{}{}{}{v.t.}{Prestar culto religioso.}{ve.ne.rar}{0}
\verb{venerar}{}{}{}{}{}{Respeitar profundamente; reverenciar.}{ve.ne.rar}{\verboinum{1}}
\verb{venerável}{}{}{"-eis}{}{adj.2g.}{Digno de veneração; respeitável.}{ve.ne.rá.vel}{0}
\verb{venéreo}{}{}{}{}{adj.}{Relativo à prática sexual.}{ve.né.re.o}{0}
\verb{venéreo}{}{}{}{}{}{Diz"-se de doença que afeta os órgãos genitais.}{ve.né.re.o}{0}
\verb{veneta}{ê}{}{}{}{s.f.}{Acesso repentino de raiva ou loucura.}{ve.ne.ta}{0}
\verb{veneta}{ê}{}{}{}{}{Impulso repentino; capricho, mania.}{ve.ne.ta}{0}
\verb{veneziana}{}{}{}{}{s.f.}{Janela cujas folhas são compostas de lâminas inclinadas que formam frestas, escurecendo o ambiente sem impedir a ventilação.}{ve.ne.zi.a.na}{0}
\verb{veneziano}{}{}{}{}{adj.}{Relativo a Veneza, cidade da Itália.}{ve.ne.zi.a.no}{0}
\verb{veneziano}{}{}{}{}{s.m.}{Indivíduo natural ou habitante dessa cidade.}{ve.ne.zi.a.no}{0}
\verb{venezuelano}{}{}{}{}{adj.}{Relativo à Venezuela.}{ve.ne.zu.e.la.no}{0}
\verb{venezuelano}{}{}{}{}{s.m.}{Indivíduo natural ou habitante desse país.}{ve.ne.zu.e.la.no}{0}
\verb{vênia}{}{}{}{}{s.f.}{Reverência, saudação respeitosa.}{vê.nia}{0}
\verb{vênia}{}{}{}{}{}{Licença, permissão.}{vê.nia}{0}
\verb{vênia}{}{}{}{}{}{Perdão, indulgência, absolvição.}{vê.nia}{0}
\verb{venial}{}{}{"-ais}{}{adj.2g.}{Perdoável, desculpável.}{ve.ni.al}{0}
\verb{venoso}{ô}{}{"-osos ⟨ó⟩}{"-osa ⟨ó⟩}{adj.}{Relativo a veia.}{ve.no.so}{0}
\verb{venoso}{ô}{}{"-osos ⟨ó⟩}{"-osa ⟨ó⟩}{}{Que tem veias.}{ve.no.so}{0}
\verb{venta}{}{}{}{}{s.f.}{Cada uma das aberturas nasais; narina.}{ven.ta}{0}
\verb{ventana}{}{Desus.}{}{}{s.f.}{Janela.}{ven.ta.na}{0}
\verb{ventania}{}{}{}{}{s.f.}{Vento forte e contínuo.}{ven.ta.ni.a}{0}
\verb{ventanilha}{}{}{}{}{s.f.}{Cada uma das bocas da mesa de bilhar, por onde cai a bola; caçapa.}{ven.ta.ni.lha}{0}
\verb{ventar}{}{}{}{}{v.i.}{Soprar o vento.}{ven.tar}{\verboinum{1}}
\verb{ventarola}{ó}{}{}{}{s.f.}{Tipo de leque sem varetas, geralmente de forma circular, e que se fecha parcialmente.}{ven.ta.ro.la}{0}
\verb{ventas}{}{}{}{}{s.f.pl.}{O nariz.}{ven.tas}{0}
\verb{ventilação}{}{}{"-ões}{}{s.f.}{Ato ou efeito de ventilar.}{ven.ti.la.ção}{0}
\verb{ventilação}{}{}{"-ões}{}{}{Passagem ou movimentação contínua do ar em um ambiente; arejamento.}{ven.ti.la.ção}{0}
\verb{ventilação}{}{}{"-ões}{}{}{Abertura pela qual o ar entra ou sai de um ambiente, compartimento, caixa.}{ven.ti.la.ção}{0}
\verb{ventilador}{ô}{}{}{}{adj.}{Que ventila.}{ven.ti.la.dor}{0}
\verb{ventilador}{ô}{}{}{}{s.m.}{Aparelho dotado de motor e hélices ou turbinas que renova ou movimenta o ar de um ambiente.}{ven.ti.la.dor}{0}
\verb{ventilar}{}{}{}{}{v.t.}{Movimentar o ar; fazer ou introduzir vento.}{ven.ti.lar}{0}
\verb{ventilar}{}{}{}{}{}{Renovar o ar.}{ven.ti.lar}{0}
\verb{ventilar}{}{}{}{}{}{Expor ao vento; arejar.}{ven.ti.lar}{0}
\verb{ventilar}{}{Fig.}{}{}{}{Debater, esclarecer, discutir uma questão.}{ven.ti.lar}{0}
\verb{ventilar}{}{}{}{}{v.pron.}{Abanar"-se.}{ven.ti.lar}{\verboinum{1}}
\verb{vento}{}{}{}{}{s.m.}{Ar em movimento.}{ven.to}{0}
\verb{vento}{}{}{}{}{}{Fenômeno atmosférico caracterizado pelo deslocamento de massas de ar causado pelas diferenças de pressão atmosférica.}{ven.to}{0}
\verb{vento}{}{}{}{}{}{Movimento do ar causado pelo deslocamento dos corpos ou pela ação de hélices, leques.}{ven.to}{0}
\verb{ventoinha}{}{}{}{}{s.f.}{Cata"-vento.}{ven.to.i.nha}{0}
\verb{ventoinha}{}{}{}{}{}{Peça rotativa de um ventilador, que impulsiona o ar.}{ven.to.i.nha}{0}
\verb{ventosa}{ó}{Zool.}{}{}{s.f.}{Parte de certos animais, em forma de pequena abóbada com centro retrátil, que serve para fixação.}{ven.to.sa}{0}
\verb{ventosa}{ó}{Por ext.}{}{}{}{Aparelho que cria pressão e sucção, usado para desentupir encanamentos.}{ven.to.sa}{0}
\verb{ventosa}{ó}{}{}{}{}{Objeto cônico que se aplica sobre a pele em certos tratamentos.}{ven.to.sa}{0}
\verb{ventosidade}{}{}{}{}{s.f.}{Acúmulo de gases no estômago ou intestinos; flatulência.}{ven.to.si.da.de}{0}
\verb{ventoso}{ô}{}{"-osos ⟨ó⟩}{"-osa ⟨ó⟩}{adj.}{Diz"-se de localidade ou período em que venta muito.}{ven.to.so}{0}
\verb{ventoso}{ô}{}{"-osos ⟨ó⟩}{"-osa ⟨ó⟩}{}{Exposto ao vento.}{ven.to.so}{0}
\verb{ventral}{}{}{"-ais}{}{adj.2g.}{Relativo ao ventre.}{ven.tral}{0}
\verb{ventre}{}{}{}{}{s.m.}{Cavidade abdominal; abdômen.}{ven.tre}{0}
\verb{ventre}{}{}{}{}{}{O estômago ou o intestino; barriga.}{ven.tre}{0}
\verb{ventre}{}{}{}{}{}{Útero.}{ven.tre}{0}
\verb{ventre}{}{Fig.}{}{}{}{Parte interior; âmago.}{ven.tre}{0}
\verb{ventrículo}{}{Anat.}{}{}{s.m.}{Pequena cavidade, especialmente as do coração e do cérebro.}{ven.trí.cu.lo}{0}
\verb{ventriloquia}{}{}{}{}{s.f.}{Habilidade de falar com movimentos imperceptíveis dos lábios, dando"-se a impressão de que a fala vem de boneco ou de outra pessoa.}{ven.tri.lo.qui.a}{0}
\verb{ventríloquo}{}{}{}{}{s.m.}{Indivíduo que domina a ventriloquia.}{ven.trí.lo.quo}{0}
\verb{ventrudo}{}{}{}{}{adj.}{Que tem ventre grande; barrigudo, pançudo.}{ven.tru.do}{0}
\verb{ventura}{}{}{}{}{s.f.}{Acaso, destino, sorte.}{ven.tu.ra}{0}
\verb{ventura}{}{}{}{}{}{Boa sorte, boa fortuna.}{ven.tu.ra}{0}
\verb{ventura}{}{}{}{}{}{Felicidade.}{ven.tu.ra}{0}
\verb{ventura}{}{}{}{}{}{Risco, perigo.}{ven.tu.ra}{0}
\verb{venturoso}{ô}{}{"-osos ⟨ó⟩}{"-osa ⟨ó⟩}{adj.}{Cheio de ventura; afortunado, sortudo, feliz.}{ven.tu.ro.so}{0}
\verb{venturoso}{ô}{}{"-osos ⟨ó⟩}{"-osa ⟨ó⟩}{}{Em que há incerteza, risco, perigo.}{ven.tu.ro.so}{0}
\verb{Vênus}{}{Astron.}{}{}{s.f.}{Segundo planeta do sistema solar em ordem de afastamento do Sol, cuja órbita se encontra entre a de Mercúrio e a da Terra.}{Vê.nus}{0}
\verb{venusto}{}{}{}{}{adj.}{Muito belo; formoso, elegante.}{ve.nus.to}{0}
\verb{ver}{ê}{}{}{}{v.t.}{Perceber pela visão; avistar, enxergar.}{ver}{0}
\verb{ver}{ê}{}{}{}{}{Assistir, presenciar, testemunhar.}{ver}{0}
\verb{ver}{ê}{}{}{}{}{Encontrar"-se com; avistar"-se com.}{ver}{0}
\verb{ver}{ê}{}{}{}{}{Perceber, reparar, notar.}{ver}{0}
\verb{ver}{ê}{}{}{}{v.pron.}{Reconhecer"-se.}{ver}{\verboinum{46}}
\verb{veracidade}{}{}{}{}{s.f.}{Qualidade do que é verdadeiro.}{ve.ra.ci.da.de}{0}
\verb{veracidade}{}{}{}{}{}{Capacidade pessoal de ser verdadeiro, de dizer a verdade.}{ve.ra.ci.da.de}{0}
\verb{veranear}{}{}{}{}{v.i.}{Passar o verão, especialmente estando de folga e em lugar particularmente agradável.}{ve.ra.ne.ar}{\verboinum{4}}
\verb{veraneio}{ê}{}{}{}{s.m.}{Ato de veranear.}{ve.ra.nei.o}{0}
\verb{veranico}{}{Bras.}{}{}{s.f.}{Período de estiagem que ocorre durante a estação chuvosa, geralmente com calor intenso.}{ve.ra.ni.co}{0}
\verb{veranico}{}{}{}{}{}{Verão ameno, com calor brando.}{ve.ra.ni.co}{0}
\verb{veranista}{}{}{}{}{s.2g.}{Indivíduo que veraneia.}{ve.ra.nis.ta}{0}
\verb{verão}{}{}{"-ões}{}{s.m.}{Estação quente do ano, com início no dia 21 de dezembro e término no dia 20 de março.}{ve.rão}{0}
\verb{veraz}{}{}{}{}{adj.2g.}{Que diz ou contém verdade; verdadeiro.}{ve.raz}{0}
\verb{verba}{é}{}{}{}{s.f.}{Cláusula ou artigo de um documento.}{ver.ba}{0}
\verb{verba}{é}{}{}{}{}{Nota, apontamento, comentário.}{ver.ba}{0}
\verb{verba}{é}{}{}{}{}{Quantia para fins determinados.}{ver.ba}{0}
\verb{verba}{é}{Por ext.}{}{}{}{Qualquer quantia; dinheiro.}{ver.ba}{0}
\verb{verbal}{}{}{"-ais}{}{adj.2g.}{Relativo à comunicação linguística.}{ver.bal}{0}
\verb{verbal}{}{}{"-ais}{}{}{Expresso oralmente, em geral por oposição àquilo que é expresso por escrito.}{ver.bal}{0}
\verb{verbal}{}{Gram.}{"-ais}{}{}{Relativo a verbo.}{ver.bal}{0}
\verb{verbalismo}{}{}{}{}{s.m.}{Transmissão de conhecimentos feita unicamente pela explicação oral.}{ver.ba.lis.mo}{0}
\verb{verbalismo}{}{}{}{}{}{Tendência literária que cultua a palavra e a eloquência vazia.}{ver.ba.lis.mo}{0}
\verb{verbalista}{}{}{}{}{adj.2g.}{Relativo ao verbalismo.}{ver.ba.lis.ta}{0}
\verb{verbalista}{}{}{}{}{s.2g.}{Escritor ligado ou adepto do verbalismo.}{ver.ba.lis.ta}{0}
\verb{verbalização}{}{}{"-ões}{}{s.f.}{Ato ou efeito de verbalizar.}{ver.ba.li.za.ção}{0}
\verb{verbalizar}{}{}{}{}{v.t.}{Tornar verbal.}{ver.ba.li.zar}{0}
\verb{verbalizar}{}{}{}{}{}{Expor algo falando.}{ver.ba.li.zar}{\verboinum{1}}
\verb{verbena}{}{Bot.}{}{}{s.f.}{Planta herbácea ou arbustiva com flores perfumadas, das quais se faz chá e licor.}{ver.be.na}{0}
\verb{verberar}{}{}{}{}{v.t.}{Fustigar, flagelar.}{ver.be.rar}{0}
\verb{verberar}{}{}{}{}{}{Censurar energicamente; reprovar.}{ver.be.rar}{\verboinum{1}}
\verb{verberativo}{}{}{}{}{adj.}{Próprio para verberar.}{ver.be.ra.ti.vo}{0}
\verb{verbetar}{}{}{}{}{v.t.}{Escrever sob a forma de verbete.}{ver.be.tar}{\verboinum{1}}
\verb{verbete}{ê}{}{}{}{s.m.}{Nota, apontamento.}{ver.be.te}{0}
\verb{verbete}{ê}{}{}{}{}{O papel onde se registra essa nota.}{ver.be.te}{0}
\verb{verbete}{ê}{}{}{}{}{O conjunto das acepções, explicações e exemplos de uso de um lexema ou termo em um dicionário, enciclopédia, glossário.}{ver.be.te}{0}
\verb{verbo}{é}{Gram.}{}{}{s.m.}{Classe de palavras que designa ação ou estado.}{ver.bo}{0}
\verb{verbo}{é}{}{}{}{}{Palavra; linguagem.}{ver.bo}{0}
\verb{verborragia}{}{}{}{}{s.f.}{Palavras em grande quantidade, geralmente com pouco conteúdo.}{ver.bor.ra.gi.a}{0}
\verb{verborreia}{é}{}{}{}{s.f.}{Verborragia.}{ver.bor.rei.a}{0}
\verb{verbosidade}{}{}{}{}{s.f.}{Qualidade de verboso; fluência oral.}{ver.bo.si.da.de}{0}
\verb{verboso}{ô}{}{"-osos ⟨ó⟩}{"-osa ⟨ó⟩}{adj.}{Que fala bastante ou com facilidade; loquaz, eloquente.}{ver.bo.so}{0}
\verb{verdade}{}{}{}{}{s.f.}{Conformidade com o real.}{ver.da.de}{0}
\verb{verdade}{}{}{}{}{}{Veracidade.}{ver.da.de}{0}
\verb{verdade}{}{}{}{}{}{Coisa certa ou verdadeira.}{ver.da.de}{0}
\verb{verdade}{}{}{}{}{}{Representação real da natureza.}{ver.da.de}{0}
\verb{verdadeiro}{ê}{}{}{}{adj.}{Em que há verdade.}{ver.da.dei.ro}{0}
\verb{verdadeiro}{ê}{}{}{}{}{Que expressa a verdade.}{ver.da.dei.ro}{0}
\verb{verdadeiro}{ê}{}{}{}{}{Real, certo, exato.}{ver.da.dei.ro}{0}
\verb{verdadeiro}{ê}{}{}{}{}{Genuíno.}{ver.da.dei.ro}{0}
\verb{verde}{ê}{}{}{}{s.m.}{A cor das folhas da maioria das árvores.}{ver.de}{0}
\verb{verde}{ê}{}{}{}{adj.2g.}{Que tem essa cor.}{ver.de}{0}
\verb{verde}{ê}{}{}{}{}{Diz"-se das frutas que ainda não estão maduras.}{ver.de}{0}
\verb{verde}{ê}{}{}{}{}{Diz"-se de madeira ou lenha que não está seca.}{ver.de}{0}
\verb{verde"-amarelo}{ê\ldots{}é}{}{ verde"-amarelos ⟨ê\ldots{}é⟩}{}{adj.}{Que tem as cores verde e amarela.}{ver.de"-a.ma.re.lo}{0}
\verb{verde"-amarelo}{ê\ldots{}é}{Fig.}{ verde"-amarelos ⟨ê\ldots{}é⟩}{}{}{Relativo ao Brasil.}{ver.de"-a.ma.re.lo}{0}
\verb{verde"-amarelo}{ê\ldots{}é}{Fig.}{ verde"-amarelos ⟨ê\ldots{}é⟩}{}{}{Patriótico em relação ao Brasil.}{ver.de"-a.ma.re.lo}{0}
\verb{verdecer}{ê}{}{}{}{v.i.}{Tomar a cor verde; verdejar.}{ver.de.cer}{\verboinum{15}}
\verb{verde"-claro}{ê}{}{verde"-claros ⟨ê⟩}{}{adj.}{Que tem tonalidade entre o verde e o branco.}{ver.de"-cla.ro}{0}
\verb{verde"-escuro}{ê}{}{verde"-escuros ⟨ê⟩}{}{adj.}{Que tem tonalidade entre o verde e o preto.}{ver.de"-es.cu.ro}{0}
\verb{verde"-esmeralda}{ê}{}{}{}{adj.2g.}{Que tem tonalidade de verde semelhante à da esmeralda.}{ver.de"-es.me.ral.da}{0}
\verb{verde"-gaio}{ê}{}{do s.m.: verdes"-gaios ⟨ê⟩}{}{adj.2g.}{Verde"-claro.}{ver.de"-gai.o}{0}
\verb{verde"-gaio}{ê}{}{do s.m.: verdes"-gaios ⟨ê⟩}{}{s.m.}{Certo tipo de música e dança popular.}{ver.de"-gai.o}{0}
\verb{verdejante}{}{}{}{}{adj.2g.}{Que verdeja; verdoso, verde.}{ver.de.jan.te}{0}
\verb{verdejar}{}{}{}{}{v.i.}{Tomar a cor verde.}{ver.de.jar}{\verboinum{1}}
\verb{verde"-mar}{ê}{}{}{}{adj.2g.}{Verde"-claro.}{ver.de"-mar}{0}
\verb{verde"-montanha}{ê}{}{}{}{adj.2g.}{Verde"-escuro ou verde tirante a azul.}{ver.de"-mon.ta.nha}{0}
\verb{verde"-oliva}{ê}{}{}{}{adj.2g.}{Da cor verde"-escura semelhante à da azeitona.}{ver.de"-o.li.va}{0}
\verb{verdoengo}{}{}{}{}{adj.}{Esverdeado.}{ver.do.en.go}{0}
\verb{verdoengo}{}{Bras.}{}{}{}{Diz"-se da fruta que não está bem madura.}{ver.do.en.go}{0}
\verb{verdolengo}{}{}{}{}{adj.}{Verdoengo.}{ver.do.len.go}{0}
\verb{verdor}{ô}{}{}{}{s.m.}{Qualidade de verde.}{ver.dor}{0}
\verb{verdor}{ô}{}{}{}{}{A cor verde dos vegetais.}{ver.dor}{0}
\verb{verdor}{ô}{}{}{}{}{Viço ou vigor de juventude.}{ver.dor}{0}
\verb{verdor}{ô}{}{}{}{}{Inexperiência característica da juventude.}{ver.dor}{0}
\verb{verdoso}{ô}{}{"-osos ⟨ó⟩}{"-osa ⟨ó⟩}{adj.}{Verdejante.}{ver.do.so}{0}
\verb{verdugo}{}{}{}{}{s.m.}{Carrasco, algoz.}{ver.du.go}{0}
\verb{verdugo}{}{Fig.}{}{}{}{Indivíduo muito cruel.}{ver.du.go}{0}
\verb{verdura}{}{}{}{}{s.f.}{Os vegetais.}{ver.du.ra}{0}
\verb{verdura}{}{}{}{}{}{Hortaliça.}{ver.du.ra}{0}
\verb{verdura}{}{}{}{}{}{Verdor.}{ver.du.ra}{0}
\verb{verdura}{}{}{}{}{}{O vigor ou a inexperiência característica da juventude.}{ver.du.ra}{0}
\verb{verdureiro}{ê}{}{}{}{s.m.}{Comerciante de verduras.}{ver.du.rei.ro}{0}
\verb{vereador}{ô}{}{}{}{s.m.}{Membro do poder legislativo municipal.}{ve.re.a.dor}{0}
\verb{vereança}{}{Bras.}{}{}{s.f.}{Ato ou efeito de verear.}{ve.re.an.ça}{0}
\verb{verear}{}{}{}{}{v.i.}{Exercer o cargo de vereador.}{ve.re.ar}{\verboinum{4}}
\verb{vereda}{ê}{}{}{}{s.f.}{Caminho estreito; trilha.}{ve.re.da}{0}
\verb{vereda}{ê}{Fig.}{}{}{}{Rumo, direção, caminho.}{ve.re.da}{0}
\verb{veredicto}{}{}{}{}{}{Var. de \textit{veredito}.}{ve.re.dic.to}{0}
\verb{veredito}{}{}{}{}{s.m.}{Decisão de júri ou tribunal judiciário; sentença.}{ve.re.di.to}{0}
\verb{veredito}{}{Por ext.}{}{}{}{Juízo proferido sobre qualquer assunto.}{ve.re.di.to}{0}
\verb{verga}{ê}{}{}{}{s.f.}{Vara fina e flexível.}{ver.ga}{0}
\verb{verga}{ê}{}{}{}{}{Barra de metal delgada.}{ver.ga}{0}
\verb{verga}{ê}{}{}{}{}{Peça horizontal de pedra ou madeira que se coloca sobre as ombreiras de porta ou janela.}{ver.ga}{0}
\verb{verga}{ê}{}{}{}{}{O sulco produzido pelo arado na terra.}{ver.ga}{0}
\verb{vergalhada}{}{}{}{}{s.f.}{Golpe desferido com o vergalho; chicotada.}{ver.ga.lha.da}{0}
\verb{vergalhão}{}{}{"-ões}{}{s.m.}{Barra de aço utilizada em estruturas de concreto.}{ver.ga.lhão}{0}
\verb{vergalhar}{}{}{}{}{v.t.}{Bater ou surrar com o vergalho; chicotear.}{ver.ga.lhar}{\verboinum{1}}
\verb{vergalho}{}{}{}{}{s.m.}{O órgão genital dos bois e cavalos, depois de cortado e seco.}{ver.ga.lho}{0}
\verb{vergalho}{}{}{}{}{}{Chicote feito desse membro.}{ver.ga.lho}{0}
\verb{vergalho}{}{Por ext.}{}{}{}{Qualquer chicote, chibata ou azorrague.}{ver.ga.lho}{0}
\verb{vergão}{}{}{"-ões}{}{s.m.}{Verga grande ou grossa.}{ver.gão}{0}
\verb{vergão}{}{}{"-ões}{}{}{Marca, vinco, arranhão na pele provocados por chicotada ou alguma outra causa.}{ver.gão}{0}
\verb{vergar}{}{}{}{}{v.t.}{Dobrar, curvar, envergar.}{ver.gar}{0}
\verb{vergar}{}{}{}{}{}{Submeter, sujeitar.}{ver.gar}{0}
\verb{vergar}{}{}{}{}{}{Abater, humilhar.}{ver.gar}{0}
\verb{vergar}{}{}{}{}{v.i.}{Ceder ao peso de alguma coisa.}{ver.gar}{0}
\verb{vergar}{}{}{}{}{}{Ceder à influência de alguém.}{ver.gar}{\verboinum{5}}
\verb{vergasta}{}{}{}{}{s.f.}{Pequena verga.}{ver.gas.ta}{0}
\verb{vergasta}{}{}{}{}{}{Vara delgada e flexível, usada para açoitar.}{ver.gas.ta}{0}
\verb{vergasta}{}{Por ext.}{}{}{}{Qualquer chicote, chibata, azorrague.}{ver.gas.ta}{0}
\verb{vergasta}{}{Fig.}{}{}{}{Castigo, flagelo.}{ver.gas.ta}{0}
\verb{vergastada}{}{}{}{}{s.f.}{Pancada com vergasta; chicotada.}{ver.gas.ta.da}{0}
\verb{vergastar}{}{}{}{}{v.t.}{Bater com vergasta; açoitar, fustigar.}{ver.gas.tar}{\verboinum{1}}
\verb{vergel}{é}{}{"-éis}{}{s.m.}{Terreno plantado de árvores frutíferas; jardim, pomar.}{ver.gel}{0}
\verb{vergonha}{}{}{}{}{s.f.}{Sentimento de insegurança causado por medo do ridículo e do julgamento dos outros; timidez.}{ver.go.nha}{0}
\verb{vergonha}{}{}{}{}{}{Sentimento causado pela inferioridade, indecência ou indignidade.}{ver.go.nha}{0}
\verb{vergonha}{}{}{}{}{}{Qualquer ato indecoroso, desonesto.}{ver.go.nha}{0}
\verb{vergonhoso}{ô}{}{"-osos ⟨ó⟩}{"-osa ⟨ó⟩}{adj.}{Que tem vergonha; tímido, acanhado.}{ver.go.nho.so}{0}
\verb{vergonhoso}{ô}{}{"-osos ⟨ó⟩}{"-osa ⟨ó⟩}{}{Que causa desonra; indigno.}{ver.go.nho.so}{0}
\verb{vergonhoso}{ô}{}{"-osos ⟨ó⟩}{"-osa ⟨ó⟩}{}{Que é obsceno, indecoroso.}{ver.go.nho.so}{0}
\verb{vergôntea}{}{}{}{}{s.f.}{Ramo fino de árvore ou arbusto; rebento, broto.}{ver.gôn.tea}{0}
\verb{verídico}{}{}{}{}{adj.}{Que diz ou expressa a verdade.}{ve.rí.di.co}{0}
\verb{verídico}{}{}{}{}{}{Que é verdadeiro, autêntico, real.}{ve.rí.di.co}{0}
\verb{verificação}{}{}{"-ões}{}{s.f.}{Ato ou efeito de verificar; averiguação.}{ve.ri.fi.ca.ção}{0}
\verb{verificação}{}{}{"-ões}{}{}{Cumprimento, realização.}{ve.ri.fi.ca.ção}{0}
\verb{verificação}{}{}{"-ões}{}{}{Prova, demonstração.}{ve.ri.fi.ca.ção}{0}
\verb{verificar}{}{}{}{}{v.t.}{Provar a verdade de algo; confirmar, comprovar.}{ve.ri.fi.car}{0}
\verb{verificar}{}{}{}{}{}{Averiguar, investigar.}{ve.ri.fi.car}{0}
\verb{verificar}{}{}{}{}{v.pron.}{Acontecer na realidade; cumprir"-se. investigar"-se.}{ve.ri.fi.car}{\verboinum{2}}
\verb{verme}{é}{Zool.}{}{}{s.m.}{Nome comum dado a muitos animais invertebrados, que apresentam corpo alongado, mole, desprovido de apêndices articulados e esqueleto interno.}{ver.me}{0}
\verb{verme}{é}{Fig.}{}{}{}{Aquilo que consome, corrói intimamente, como se fosse um verme parasito.}{ver.me}{0}
\verb{verme}{é}{Fig.}{}{}{}{Indivíduo vil, desprezível.}{ver.me}{0}
\verb{vermelhão}{}{}{"-ões}{}{s.m.}{Sulfato de mercúrio pulverizado, de cor vermelha, usado na fabricação de tinta.}{ver.me.lhão}{0}
\verb{vermelhão}{}{Fig.}{"-ões}{}{}{Rubor, vermelhidão.}{ver.me.lhão}{0}
\verb{vermelhar}{}{}{}{}{v.t.}{Tornar vermelho ou avermelhado.}{ver.me.lhar}{0}
\verb{vermelhar}{}{}{}{}{v.i.}{Ter ou ostentar a cor vermelha.}{ver.me.lhar}{\verboinum{1}}
\verb{vermelhidão}{}{}{"-ões}{}{s.f.}{Qualidade do que é vermelho.}{ver.me.lhi.dão}{0}
\verb{vermelhidão}{}{}{"-ões}{}{}{Mancha vermelha na pele; vermelhão.}{ver.me.lhi.dão}{0}
\verb{vermelho}{ê}{}{}{}{s.m.}{A cor vermelha.}{ver.me.lho}{0}
\verb{vermelho}{ê}{}{}{}{adj.}{Que tem a cor do sangue, da papoula ou do rubi.}{ver.me.lho}{0}
\verb{vermelho}{ê}{}{}{}{}{Diz"-se dessa cor.}{ver.me.lho}{0}
\verb{vermelho}{ê}{}{}{}{}{Corado, rubro.}{ver.me.lho}{0}
\verb{vermicida}{}{}{}{}{adj.2g.}{Que mata ou destrói vermes.}{ver.mi.ci.da}{0}
\verb{vermicida}{}{}{}{}{s.m.}{Substância vermicida.}{ver.mi.ci.da}{0}
\verb{vermiforme}{ó}{}{}{}{adj.2g.}{Que tem aspecto ou forma de verme.}{ver.mi.for.me}{0}
\verb{vermífugo}{}{}{}{}{adj.}{Que afugenta os vermes.}{ver.mí.fu.go}{0}
\verb{vermífugo}{}{}{}{}{s.m.}{Remédio que destrói vermes.}{ver.mí.fu.go}{0}
\verb{verminação}{}{}{"-ões}{}{s.f.}{Produção de vermes nos intestinos.}{ver.mi.na.ção}{0}
\verb{verminose}{ó}{}{}{}{s.f.}{Infecção produzida por vermes.}{ver.mi.no.se}{0}
\verb{verminoso}{ô}{}{"-osos ⟨ó⟩}{"-osa ⟨ó⟩}{adj.}{Em que há vermes.}{ver.mi.no.so}{0}
\verb{verminoso}{ô}{}{"-osos ⟨ó⟩}{"-osa ⟨ó⟩}{}{Gerado pelos vermes.}{ver.mi.no.so}{0}
\verb{vermute}{}{}{}{}{s.m.}{Vinho licoroso, branco ou tinto, preparado com extrato de plantas aromáticas e amargas.}{ver.mu.te}{0}
\verb{vernaculizar}{}{}{}{}{v.t.}{Tornar vernáculo.}{ver.na.cu.li.zar}{\verboinum{1}}
\verb{vernáculo}{}{}{}{}{adj.}{Próprio de um país ou de uma nação; nacional.}{ver.ná.cu.lo}{0}
\verb{vernáculo}{}{}{}{}{}{Diz"-se de linguagem genuína, correta, isenta de estrangeirismos.}{ver.ná.cu.lo}{0}
\verb{vernáculo}{}{}{}{}{s.m.}{O idioma próprio de um país.}{ver.ná.cu.lo}{0}
\verb{vernal}{}{}{"-ais}{}{adj.2g.}{Relativo à primavera.}{ver.nal}{0}
\verb{vernal}{}{}{"-ais}{}{}{Diz"-se dos vegetais que rebentam na primavera.}{ver.nal}{0}
\verb{vernissage}{}{}{}{}{s.m.}{Inauguração de uma exposição de obras de arte.}{ver.nis.sa.ge}{0}
\verb{verniz}{}{}{}{}{s.m.}{Tinta incolor usada para proteger e dar brilho em metais, madeiras, couros etc.}{ver.niz}{0}
\verb{verniz}{}{Fig.}{}{}{}{Polidez superficial de maneiras.}{ver.niz}{0}
\verb{vero}{é}{}{}{}{adj.}{Verdadeiro, real, autêntico.}{ve.ro}{0}
\verb{verônica}{}{}{}{}{s.f.}{Pano em que, segundo a tradição cristã, uma mulher de nome Verônica enxugou o rosto de Jesus quando carregava a cruz ao calvário, tendo ficado ali gravada a sua figura.}{ve.rô.ni.ca}{0}
\verb{verônica}{}{}{}{}{}{A imagem do rosto de Cristo gravada em metal.}{ve.rô.ni.ca}{0}
\verb{verossímil}{}{}{"-eis}{}{adj.2g.}{Que parece verdadeiro.}{ve.ros.sí.mil}{0}
\verb{verossímil}{}{}{"-eis}{}{}{Que é possível ou provável por não contrariar a verdade; plausível.}{ve.ros.sí.mil}{0}
\verb{verossimilhança}{}{}{}{}{s.f.}{Qualidade de verossímil.}{ve.ros.si.mi.lhan.ça}{0}
\verb{verossimilhança}{}{Liter.}{}{}{}{Conformidade entre fatos, ideias e outros aspectos de uma obra literária, que cria um efeito de semelhança com a realidade, ainda que os elementos imaginários ou fantásticos sejam determinantes no texto; coerência.}{ve.ros.si.mi.lhan.ça}{0}
\verb{verossimilhante}{}{}{}{}{adj.2g.}{Verossímil.}{ve.ros.si.mi.lhan.te}{0}
\verb{verrina}{}{}{}{}{s.f.}{Cada um dos discursos pronunciados pelo político e orador romano Cícero contra Caio Verres.}{ver.ri.na}{0}
\verb{verrina}{}{Por ext.}{}{}{}{Discurso de crítica e de forte censura.}{ver.ri.na}{0}
\verb{verruga}{}{}{}{}{s.f.}{Tumor benigno da pele, em forma de saliência plana ou volumosa, de origem virótica.}{ver.ru.ga}{0}
\verb{verrugoso}{ô}{}{"-osos ⟨ó⟩}{"-osa ⟨ó⟩}{adj.}{Que tem verrugas.}{ver.ru.go.so}{0}
\verb{verruma}{}{}{}{}{s.f.}{Instrumento de aço que tem a sua extremidade inferior aberta em espiral e terminada em ponta, usado para abrir furos na madeira.}{ver.ru.ma}{0}
\verb{verrumar}{}{}{}{}{v.t.}{Fazer furo com verruma.}{ver.ru.mar}{0}
\verb{verrumar}{}{Fig.}{}{}{}{Meditar, pensar.}{ver.ru.mar}{0}
\verb{verrumar}{}{Fig.}{}{}{}{Afligir, torturar.}{ver.ru.mar}{\verboinum{1}}
\verb{versado}{}{}{}{}{adj.}{Que é objeto de estudo ou discussão; tratado, estudado.}{ver.sa.do}{0}
\verb{versado}{}{}{}{}{}{Aquele que sabe muito, que é entendido em algum assunto; conhecedor, perito.}{ver.sa.do}{0}
\verb{versal}{}{}{"-ais}{}{s.f.}{Caráter, tipo ou letra maiúscula.}{ver.sal}{0}
\verb{versalete}{ê}{}{}{}{s.m.}{Letra que tem a mesma forma das maiúsculas escrita no mesmo tamanho das minúsculas.}{ver.sa.le.te}{0}
\verb{versão}{}{}{"-ões}{}{s.f.}{Ato ou efeito de verter ou de voltar.}{ver.são}{0}
\verb{versão}{}{}{"-ões}{}{}{Tradução literal de um texto.}{ver.são}{0}
\verb{versão}{}{}{"-ões}{}{}{Cada uma das várias interpretações do mesmo ponto.}{ver.são}{0}
\verb{versão}{}{}{"-ões}{}{}{Tradução de letra de música adaptada à melodia original.}{ver.são}{0}
\verb{versar}{}{}{}{}{v.t.}{Volver, manejar, examinar.}{ver.sar}{0}
\verb{versar}{}{}{}{}{}{Praticar, exercitar, estudar.}{ver.sar}{0}
\verb{versar}{}{}{}{}{}{Ponderar, considerar.}{ver.sar}{0}
\verb{versar}{}{}{}{}{}{Ter por tema ou objeto; discorrer.}{ver.sar}{\verboinum{1}}
\verb{versátil}{}{}{"-eis}{}{adj.2g.}{Que é mutável, volúvel.}{ver.sá.til}{0}
\verb{versátil}{}{}{"-eis}{}{}{Que se move facilmente.}{ver.sá.til}{0}
\verb{versátil}{}{}{"-eis}{}{}{Que possui várias qualidades ou habilidades.}{ver.sá.til}{0}
\verb{versátil}{}{}{"-eis}{}{}{Que tem utilidade variada.}{ver.sá.til}{0}
\verb{versatilidade}{}{}{}{}{s.f.}{Qualidade do que é versátil.}{ver.sa.ti.li.da.de}{0}
\verb{versejar}{}{}{}{}{v.i.}{Compor versos; versar.}{ver.se.jar}{0}
\verb{versejar}{}{}{}{}{}{Fazer versos de pouco valor poético.}{ver.se.jar}{0}
\verb{versejar}{}{}{}{}{v.t.}{Recitar, declamar.}{ver.se.jar}{\verboinum{1}}
\verb{versículo}{}{}{}{}{s.m.}{Cada um dos curtos parágrafos que dividem um texto sagrado. }{ver.sí.cu.lo}{0}
\verb{versificação}{}{}{"-ões}{}{s.f.}{Ato ou efeito de versificar, de pôr em versos.}{ver.si.fi.ca.ção}{0}
\verb{versificação}{}{}{"-ões}{}{}{Técnica, arte de compor versos.}{ver.si.fi.ca.ção}{0}
\verb{versificador}{ô}{}{}{}{adj.}{Que versifica, que faz versos.}{ver.si.fi.ca.dor}{0}
\verb{versificador}{ô}{}{}{}{s.m.}{Indivíduo que versifica.}{ver.si.fi.ca.dor}{0}
\verb{versificar}{}{}{}{}{v.t.}{Pôr em versos ou compor versos; versar, versejar.}{ver.si.fi.car}{\verboinum{2}}
\verb{versista}{}{}{}{}{adj.2g.}{Aquele que verseja, que compõe versos.}{ver.sis.ta}{0}
\verb{versista}{}{}{}{}{s.2g.}{Indivíduo que verseja.}{ver.sis.ta}{0}
\verb{verso}{é}{}{}{}{s.m.}{Cada uma das linhas constitutivas de um poema; a unidade rítmica de uma poesia.}{ver.so}{0}
\verb{verso}{é}{}{}{}{}{O gênero poético.}{ver.so}{0}
\verb{verso}{é}{}{}{}{}{Poesia, versificação.}{ver.so}{0}
\verb{verso}{é}{Pop.}{}{}{}{Quadra ou estrofe qualquer.}{ver.so}{0}
\verb{verso}{é}{}{}{}{s.m.}{Página de uma folha de papel que fica oposta à da frente.}{ver.so}{0}
\verb{verso}{é}{}{}{}{}{O lado ou face posterior de qualquer objeto.}{ver.so}{0}
\verb{verso}{é}{}{}{}{}{O lado oposto ao principal; reverso.}{ver.so}{0}
\verb{verso}{é}{Bot.}{}{}{}{Face inferior das folhas dos vegetais.}{ver.so}{0}
\verb{versus}{é}{}{}{}{prep.}{Contra.}{\textit{versus}}{0}
\verb{vértebra}{}{Anat.}{}{}{s.f.}{Cada um dos ossos que, junto com outros elementos anatômicos, formam a coluna vertebral do homem e de outros vertebrados.}{vér.te.bra}{0}
\verb{vertebrado}{}{}{}{}{adj.}{Que possui vértebras.}{ver.te.bra.do}{0}
\verb{vertebrado}{}{}{}{}{}{Relativo aos vertebrados.}{ver.te.bra.do}{0}
\verb{vertebrado}{}{Zool.}{}{}{s.m.}{Espécime dos vertebrados, grupo de animais cordados, caracterizados pela presença de coluna vertebral segmentada e de crânio que protege o cérebro.}{ver.te.bra.do}{0}
\verb{vertebral}{}{}{"-ais}{}{adj.2g.}{Relativo às vértebras.}{ver.te.bral}{0}
\verb{vertebral}{}{}{"-ais}{}{}{Que é formado por vértebras.}{ver.te.bral}{0}
\verb{vertedouro}{ô}{}{}{}{s.m.}{Espécie de pá de madeira com que se tira água de dentro das embarcações.}{ver.te.dou.ro}{0}
\verb{vertente}{}{}{}{}{adj.2g.}{Que verte, jorra.}{ver.ten.te}{0}
\verb{vertente}{}{}{}{}{}{De que se trata; a respeito do que se fala.}{ver.ten.te}{0}
\verb{vertente}{}{}{}{}{s.f.}{Declive lateral de uma elevação por onde correm as águas; encosta.}{ver.ten.te}{0}
\verb{vertente}{}{}{}{}{}{Cada uma das superfícies de um telhado.}{ver.ten.te}{0}
\verb{verter}{ê}{}{}{}{v.t.}{Fazer correr um líquido para fora do recipiente que o encerra; entornar, derramar.}{ver.ter}{0}
\verb{verter}{ê}{}{}{}{}{Deixar sair com força; jorrar, espalhar.}{ver.ter}{0}
\verb{verter}{ê}{}{}{}{}{Fazer brotar; derivar, originar.}{ver.ter}{0}
\verb{verter}{ê}{}{}{}{}{Traduzir de uma língua para outra.}{ver.ter}{\verboinum{12}}
\verb{vertical}{}{}{"-ais}{}{adj.2g.}{Perpendicular ao plano do horizonte.}{ver.ti.cal}{0}
\verb{vertical}{}{}{"-ais}{}{}{Que segue a direção do fio do prumo; direito, aprumado.}{ver.ti.cal}{0}
\verb{vertical}{}{}{"-ais}{}{}{Que está colocado no vértice.}{ver.ti.cal}{0}
\verb{vertical}{}{}{"-ais}{}{}{Relativo a pessoas de níveis diferentes.}{ver.ti.cal}{0}
\verb{verticalidade}{}{}{}{}{s.f.}{Qualidade do que é vertical ou se encontra nessa posição.}{ver.ti.ca.li.da.de}{0}
\verb{vértice}{}{}{}{}{s.m.}{A altura máxima; o ponto culminante; ápice, cume.}{vér.ti.ce}{0}
\verb{vértice}{}{}{}{}{}{O ponto oposto mais afastado da base de uma figura.}{vér.ti.ce}{0}
\verb{vértice}{}{}{}{}{}{Ponto em que se encontram as linhas que formam o ângulo.}{vér.ti.ce}{0}
\verb{verticilo}{}{Bot.}{}{}{s.m.}{Conjunto de ramos, folhas ou peças florais que se dispõem em um mesmo plano de um eixo.}{ver.ti.ci.lo}{0}
\verb{vertigem}{}{}{}{}{s.f.}{Sensação de rotação dos objetos e do meio ambiente ao redor do indivíduo; tontura, tonteira.}{ver.ti.gem}{0}
\verb{vertigem}{}{}{}{}{}{Desmaio, síncope, desfalecimento.}{ver.ti.gem}{0}
\verb{vertigem}{}{Fig.}{}{}{}{Loucura súbita; desvario.}{ver.ti.gem}{0}
\verb{vertiginoso}{ô}{}{"-osos ⟨ó⟩}{"-osa ⟨ó⟩}{adj.}{Que causa vertigens.}{ver.ti.gi.no.so}{0}
\verb{vertiginoso}{ô}{}{"-osos ⟨ó⟩}{"-osa ⟨ó⟩}{}{Que sofre de vertigens.}{ver.ti.gi.no.so}{0}
\verb{vertiginoso}{ô}{}{"-osos ⟨ó⟩}{"-osa ⟨ó⟩}{}{Que gira muito rápido; velocíssimo.}{ver.ti.gi.no.so}{0}
\verb{verve}{é}{}{}{}{s.f.}{Calor de imaginação que anima o artista, o orador, o poeta.}{ver.ve}{0}
\verb{verve}{é}{}{}{}{}{Facilidade em falar ou escrever; vigor.}{ver.ve}{0}
\verb{vesânia}{}{Med.}{}{}{s.f.}{Nome genérico de diversas formas de perturbação mental; loucura.}{ve.sâ.nia}{0}
\verb{vesano}{}{}{}{}{adj.}{Que sofre de alguma perturbação mental; alienado, louco, demente.}{ve.sa.no}{0}
\verb{vesgo}{ê}{}{}{}{s.m.}{Que sofre de desvio de um ou de ambos os olhos; estrábico.}{ves.go}{0}
\verb{vesgo}{ê}{}{}{}{}{Que não é perpendicular ou reto; oblíquo, torto.}{ves.go}{0}
\verb{vesguear}{}{}{}{}{v.i.}{Ser vesgo; sofrer de estrabismo.}{ves.gue.ar}{0}
\verb{vesguear}{}{}{}{}{}{Olhar com o canto do olho, de soslaio.}{ves.gue.ar}{\verboinum{4}}
\verb{vesguice}{}{}{}{}{s.f.}{Condição ou estado de vesgo; estrabismo.}{ves.gui.ce}{0}
\verb{vesical}{}{Anat.}{"-ais}{}{adj.2g.}{Relativo à bexiga.}{ve.si.cal}{0}
\verb{vesicante}{}{}{}{}{adj.2g.}{Diz"-se de certas substâncias que produzem bolhas na pele.}{ve.si.can.te}{0}
\verb{vesícula}{}{Anat.}{}{}{s.f.}{Saco membranoso parecido com uma bexiga.}{ve.sí.cu.la}{0}
\verb{vesícula}{}{}{}{}{}{Pequena erupção na pele, cheia de líquido incolor; bolha.}{ve.sí.cu.la}{0}
\verb{vesícula}{}{Zool.}{}{}{}{Pequeno saco cheio de ar que se encontra nos peixes e que os torna mais leves e ligeiros, quando querem subir ou descer na água. }{ve.sí.cu.la}{0}
\verb{vesicular}{}{}{}{}{adj.2g.}{Relativo à vesícula.}{ve.si.cu.lar}{0}
\verb{vespa}{ê}{Zool.}{}{}{s.f.}{Inseto, semelhante à abelha, de abdômen listrado de preto e amarelo.}{ves.pa}{0}
\verb{vespa}{ê}{Fig.}{}{}{}{Pessoa de caráter intratável, áspero, mordaz.}{ves.pa}{0}
\verb{vespeiro}{ê}{}{}{}{s.m.}{Ninho de vespas.}{ves.pei.ro}{0}
\verb{vespeiro}{ê}{Fig.}{}{}{}{Local ou situação em que repentinamente surgem perigos, intrigas, desavenças.}{ves.pei.ro}{0}
\verb{vésper}{}{}{}{}{s.m.}{O planeta Vênus, quando aparece à tarde; estrela vespertina.}{vés.per}{0}
\verb{vésper}{}{}{}{}{}{O lado em que o Sol se põe; ocaso, ocidente, poente.}{vés.per}{0}
\verb{véspera}{}{}{}{}{}{A tarde.}{vés.pe.ra}{0}
\verb{véspera}{}{}{}{}{s.f.}{Dia imediatamente antecedente àquele de que se trata.}{vés.pe.ra}{0}
\verb{vesperal}{}{}{"-ais}{}{adj.2g.}{Que se realiza ou acontece à tarde; vespertino.}{ves.pe.ral}{0}
\verb{vespertino}{}{}{}{}{adj.}{Relativo à tarde.}{ves.per.ti.no}{0}
\verb{vespertino}{}{}{}{}{s.m.}{Jornal que circula à tarde ou à noite.}{ves.per.ti.no}{0}
\verb{vestal}{}{}{"-ais}{}{s.f.}{Sacerdotisa de Vesta, deusa dos antigos romanos.}{ves.tal}{0}
\verb{vestal}{}{Fig.}{"-ais}{}{}{Mulher casta, honesta, virgem.}{ves.tal}{0}
\verb{veste}{é}{}{}{}{s.f.}{Peça de roupa que veste o corpo de uma pessoa; vestimenta.}{ves.te}{0}
\verb{véstia}{}{}{}{}{s.f.}{Espécie de casaco ou jaqueta curta, que não se aperta na cintura.}{vés.tia}{0}
\verb{vestiário}{}{}{}{}{s.m.}{Compartimento em que trocam a roupa e guardam momentaneamente seus pertences os membros de uma empresa, associação, companhia teatral.}{ves.ti.á.rio}{0}
\verb{vestibulando}{}{}{}{}{s.m.}{Estudante que se prepara para prestar o exame vestibular.}{ves.ti.bu.lan.do}{0}
\verb{vestibular}{}{}{}{}{s.m.}{Exame a que se submetem os alunos egressos do ensino médio para admissão nos cursos superiores.}{ves.ti.bu.lar}{0}
\verb{vestibular}{}{}{}{}{adj.2g.}{Relativo a vestíbulo.}{ves.ti.bu.lar}{0}
\verb{vestíbulo}{}{}{}{}{s.m.}{Espaço entre a via pública e a entrada de um edifício; pátio, átrio.}{ves.tí.bu.lo}{0}
\verb{vestíbulo}{}{}{}{}{}{Entrada principal de uma construção.}{ves.tí.bu.lo}{0}
\verb{vestíbulo}{}{Anat.}{}{}{}{Cavidade que dá acesso a um órgão oco.}{ves.tí.bu.lo}{0}
\verb{vestíbulo}{}{Anat.}{}{}{}{Uma das partes componentes do ouvido interno.}{ves.tí.bu.lo}{0}
\verb{vestido}{}{}{}{}{adj.}{Que está coberto com vestes.}{ves.ti.do}{0}
\verb{vestido}{}{}{}{}{s.m.}{Vestimenta feminina que se constitui de saia e blusa numa única peça.}{ves.ti.do}{0}
\verb{vestido}{}{Desus.}{}{}{}{Veste, vestimenta.}{ves.ti.do}{0}
\verb{vestidura}{}{}{}{}{s.f.}{Tudo o que serve para vestir o corpo; veste, traje, roupa.}{ves.ti.du.ra}{0}
\verb{vestidura}{}{}{}{}{}{Aquilo que reveste; cobertura, manto.}{ves.ti.du.ra}{0}
\verb{vestígio}{}{}{}{}{s.m.}{Marca deixada pelos pés no lugar por onde se passa; pegada, pisada, rastro.}{ves.tí.gio}{0}
\verb{vestígio}{}{}{}{}{}{Qualquer marca que sirva para indicar algo que ocorreu; indício, indicação, marca.}{ves.tí.gio}{0}
\verb{vestígio}{}{Fig.}{}{}{}{Aquilo que restou de algo que não existe mais; restos, resquícios.}{ves.tí.gio}{0}
\verb{vestimenta}{}{}{}{}{s.f.}{Peça de roupa que veste qualquer parte do corpo; veste, vestidura, vestido.}{ves.ti.men.ta}{0}
\verb{vestimenta}{}{}{}{}{}{Paramento sacerdotal usado em certas cerimônias.}{ves.ti.men.ta}{0}
\verb{vestir}{}{}{}{}{v.t.}{Cobrir com roupa ou veste.}{ves.tir}{0}
\verb{vestir}{}{}{}{}{}{Trajar, usar vestuário.}{ves.tir}{\verboinum{29}}
\verb{vestuário}{}{}{}{}{s.m.}{Conjunto das peças de roupa que se vestem; indumentária, traje.}{ves.tu.á.rio}{0}
\verb{vetar}{}{}{}{}{v.t.}{Opor veto; negar.}{ve.tar}{0}
\verb{vetar}{}{}{}{}{}{Proibir, impedir, interditar.}{ve.tar}{\verboinum{1}}
\verb{veterano}{}{}{}{}{adj.}{Diz"-se de militar antigo ou reformado.}{ve.te.ra.no}{0}
\verb{veterano}{}{}{}{}{}{Aquele que exerce um cargo há muito tempo; aquele que tem experiência ou tarimba em alguma atividade.}{ve.te.ra.no}{0}
\verb{veterano}{}{}{}{}{}{Diz"-se de estudante que frequenta as últimas séries de uma faculdade ou escola.}{ve.te.ra.no}{0}
\verb{veterinária}{}{}{}{}{s.f.}{Ciência que se dedica ao diagnóstico e ao tratamento das doenças dos animais.}{ve.te.ri.ná.ria}{0}
\verb{veterinário}{}{}{}{}{adj.}{Relativo à veterinária.}{ve.te.ri.ná.rio}{0}
\verb{veterinário}{}{}{}{}{s.m.}{Especialista que exerce a medicina veterinária.}{ve.te.ri.ná.rio}{0}
\verb{veto}{é}{}{}{}{s.m.}{Ato pelo qual o chefe do poder executivo recusa sanção a um projeto de lei aprovado pelas câmaras legislativas.}{ve.to}{0}
\verb{veto}{é}{}{}{}{}{Proibição, suspensão, oposição.}{ve.to}{0}
\verb{vetor}{ô}{}{}{}{s.m.}{Portador, condutor, carregador.}{ve.tor}{0}
\verb{vetor}{ô}{Geom.}{}{}{}{Segmento de reta orientado em grandeza, direção e sentido.}{ve.tor}{0}
\verb{vetorial}{}{}{"-ais}{}{adj.2g.}{Relativo a vetor.}{ve.to.ri.al}{0}
\verb{vetusto}{}{}{}{}{adj.}{Provindo de época remota; velho, antigo.}{ve.tus.to}{0}
\verb{vetusto}{}{}{}{}{}{Deteriorado pelo tempo; obsoleto, ultrapassado.}{ve.tus.to}{0}
\verb{vetusto}{}{}{}{}{}{Respeitável pela idade; venerável.}{ve.tus.to}{0}
\verb{véu}{}{}{}{}{s.m.}{Qualquer tecido que serve para ocultar, vendar, envolver ou cobrir alguma coisa.}{véu}{0}
\verb{véu}{}{}{}{}{}{Tecido finíssimo com que as senhoras cobrem o rosto ou que lhes serve de adorno.}{véu}{0}
\verb{véu}{}{}{}{}{}{Tecido retangular que as religiosas usam para cobrir a cabeça.}{véu}{0}
\verb{vexação}{ch}{}{"-ões}{}{s.f.}{Ato ou efeito de vexar; humilhação, tormento.}{ve.xa.ção}{0}
\verb{vexame}{ch}{}{}{}{s.m.}{Aquilo que causa vergonha; pudor, pejo.}{ve.xa.me}{0}
\verb{vexame}{ch}{}{}{}{}{Ultraje, afronta, humilhação.}{ve.xa.me}{0}
\verb{vexame}{ch}{}{}{}{}{Pressa, urgência, afã.}{ve.xa.me}{0}
\verb{vexar}{ch}{}{}{}{v.t.}{Causar pudor; envergonhar.}{ve.xar}{0}
\verb{vexar}{ch}{}{}{}{}{Causar tormento; afligir, humilhar.}{ve.xar}{0}
\verb{vexar}{ch}{}{}{}{}{Acelerar, apressar.}{ve.xar}{\verboinum{1}}
\verb{vexativo}{ch}{}{}{}{adj.}{Vexatório.}{ve.xa.ti.vo}{0}
\verb{vexatório}{ch}{}{}{}{adj.}{Que causa vexame; vexativo, humilhante.}{ve.xa.tó.rio}{0}
\verb{vez}{ê}{}{}{}{s.f.}{Ocasião em que se pode fazer algo; circunstância, ensejo, oportunidade.}{vez}{0}
\verb{vez}{ê}{}{}{}{}{Momento em que ocorre uma alternância; turno.}{vez}{0}
\verb{vez}{ê}{}{}{}{}{Alternativa, opção.}{vez}{0}
\verb{vezeiro}{ê}{}{}{}{adj.}{Que tem vezo; habituado, acostumado a fazer algo.}{ve.zei.ro}{0}
\verb{vezo}{ê}{}{}{}{s.m.}{Hábito ou costume criticável.}{ve.zo}{0}
\verb{via}{}{}{}{}{s.f.}{Caminho por onde se vai ou se é levado; rua, avenida, rodovia, pista. (\textit{A via expressa das rodovias deve ficar livre para ultrapassagens.})}{vi.a}{0}
\verb{via}{}{}{}{}{}{Direção, rumo, itinerário.}{vi.a}{0}
\verb{via}{}{}{}{}{}{Cada uma das cópias de um documento.}{vi.a}{0}
\verb{viabilidade}{}{}{}{}{s.f.}{Qualidade do que é viável, realizável.}{vi.a.bi.li.da.de}{0}
\verb{viabilizar}{}{}{}{}{v.t.}{Tornar viável, realizável.}{vi.a.bi.li.zar}{\verboinum{1}}
\verb{viação}{}{}{"-ões}{}{s.f.}{Maneira ou meio de deslocamento por vias ou caminhos.}{vi.a.ção}{0}
\verb{viação}{}{}{"-ões}{}{}{Conjunto de estradas ou caminhos públicos de um país.}{vi.a.ção}{0}
\verb{viação}{}{}{"-ões}{}{}{Serviço ou empresa de transporte de carga ou de passageiros.}{vi.a.ção}{0}
\verb{viaduto}{}{}{}{}{s.m.}{Em ferrovias e rodovias, passagem construída para transpor rios, vales, depressões etc.}{vi.a.du.to}{0}
\verb{viagem}{}{}{"-ens}{}{s.f.}{Deslocamento de um lugar a outro razoavelmente distante.}{vi.a.gem}{0}
\verb{viagem}{}{Pop.}{"-ens}{}{}{Êxtase provocado pela ingestão ou administração de droga, entorpecente ou tóxico.}{vi.a.gem}{0}
\verb{viajado}{}{}{}{}{adj.}{Que viajou muito, visitou e conheceu muitos lugares.}{vi.a.ja.do}{0}
\verb{viajado}{}{}{}{}{}{Que é muito percorrido, andado.}{vi.a.ja.do}{0}
\verb{viajante}{}{}{}{}{adj.2g.}{Que viaja.}{vi.a.jan.te}{0}
\verb{viajar}{}{}{}{}{v.i.}{Fazer uma viagem.}{vi.a.jar}{0}
\verb{viajar}{}{}{}{}{v.t.}{Percorrer, visitar, transitar.}{vi.a.jar}{\verboinum{1}}
\verb{vianda}{}{}{}{}{s.f.}{Alimento ou refeição, especialmente de carne.}{vi.an.da}{0}
\verb{vianda}{}{}{}{}{}{Utensílio para transportar refeições; marmita.}{vi.an.da}{0}
\verb{viandante}{}{}{}{}{adj.2g.}{Que vianda ou viaja; viajante.}{vi.an.dan.te}{0}
\verb{viandar}{}{}{}{}{v.i.}{Viajar, peregrinar.}{vi.an.dar}{\verboinum{1}}
\verb{viário}{}{}{}{}{adj.}{Referente a via ou a viação.}{vi.á.rio}{0}
\verb{via"-sacra}{}{Relig.}{vias"-sacras}{}{s.f.}{Série de 14 estações que representam as etapas do percurso de Jesus Cristo carregando a cruz, desde o palácio de Pôncio Pilatos, onde ele foi condenado, até o Gólgota, lugar da crucificação.}{vi.a"-sa.cra}{0}
\verb{via"-sacra}{}{}{vias"-sacras}{}{}{O conjunto de orações que são ditas diante dessas estações.}{vi.a"-sa.cra}{0}
\verb{via"-sacra}{}{Fig.}{vias"-sacras}{}{}{Sacrifício, padecimento.}{vi.a"-sa.cra}{0}
\verb{viático}{}{}{}{}{s.m.}{Provisão de gêneros ou dinheiro para viagem.}{vi.á.ti.co}{0}
\verb{viático}{}{Relig.}{}{}{}{Sacramento da Eucaristia administrado a enfermos que não podem sair de casa.}{vi.á.ti.co}{0}
\verb{viatura}{}{}{}{}{s.f.}{Qualquer veículo.}{vi.a.tu.ra}{0}
\verb{viável}{}{}{"-eis}{}{adj.2g.}{Diz"-se de caminho que pode ser percorrido.}{vi.á.vel}{0}
\verb{viável}{}{}{"-eis}{}{}{Possível, exequível.}{vi.á.vel}{0}
\verb{víbora}{}{Zool.}{}{}{s.f.}{Cobra venenosa de cabeça grande, olhos com pupila vertical e cauda cônica.}{ví.bo.ra}{0}
\verb{víbora}{}{Fig.}{}{}{}{Pessoa traiçoeira ou de má índole.}{ví.bo.ra}{0}
\verb{vibração}{}{}{"-ões}{}{s.f.}{Ato ou efeito de vibrar.}{vi.bra.ção}{0}
\verb{vibração}{}{}{"-ões}{}{}{Oscilação, balanço, tremor.}{vi.bra.ção}{0}
\verb{vibração}{}{Fig.}{"-ões}{}{}{Entusiasmo.}{vi.bra.ção}{0}
\verb{vibrador}{ô}{}{}{}{adj.}{Que vibra ou faz vibrar.}{vi.bra.dor}{0}
\verb{vibrador}{ô}{}{}{}{s.m.}{Aparelho que produz vibrações.}{vi.bra.dor}{0}
\verb{vibrafone}{}{Mús.}{}{}{s.m.}{Instrumento musical composto de lâminas de metal de tamanhos diferentes, percutidas com baquetas.}{vi.bra.fo.ne}{0}
\verb{vibrante}{}{}{}{}{adj.2g.}{Que vira ou em que há vibração.}{vi.bran.te}{0}
\verb{vibrante}{}{Gram.}{}{}{s.f.}{Consoante em cuja articulação há vibração, como os sons das letras \textit{r} e \textit{rr}.}{vi.bran.te}{0}
\verb{vibrar}{}{}{}{}{v.t.}{Fazer produzir som; soar.}{vi.brar}{0}
\verb{vibrar}{}{}{}{}{}{Entrar em vibração; trepidar, estremecer.}{vi.brar}{0}
\verb{vibrar}{}{}{}{}{}{Mover com rapidez; agitar, brandir.}{vi.brar}{0}
\verb{vibrar}{}{}{}{}{v.i.}{Entusiasmar"-se; empolgar"-se.}{vi.brar}{\verboinum{1}}
\verb{vibrátil}{}{}{"-eis}{}{adj.2g.}{Suscetível de vibrar.}{vi.brá.til}{0}
\verb{vibrátil}{}{}{"-eis}{}{}{Que vibra.}{vi.brá.til}{0}
\verb{vibratório}{}{}{}{}{adj.}{Que tem ou causa vibração.}{vi.bra.tó.rio}{0}
\verb{vibrião}{}{Biol.}{"-ões}{}{s.m.}{Tipo de bactérias que têm forma de bastonetes recurvados.}{vi.bri.ão}{0}
\verb{vibrissas}{}{Anat.}{}{}{s.f.pl.}{Pelos das fossas nasais.}{vi.bris.sas}{0}
\verb{viçar}{}{}{}{}{v.t.}{Produzir ou tomar viço; vicejar.}{vi.çar}{0}
\verb{viçar}{}{}{}{}{v.i.}{Aumentar, expandir"-se, desenvolver"-se.}{vi.çar}{\verboinum{3}}
\verb{viçar}{}{}{}{}{v.i.}{Estar no cio ou parir.}{vi.çar}{0}
\verb{viçar}{}{}{}{}{}{Ter vício.}{vi.çar}{\verboinum{3}}
\verb{vicariato}{}{}{}{}{s.m.}{Cargo de vigário.}{vi.ca.ri.a.to}{0}
\verb{vicariato}{}{}{}{}{}{Residência ou jurisdição de vigário.}{vi.ca.ri.a.to}{0}
\verb{vicariato}{}{Por ext.}{}{}{}{Substituição no exercício de qualquer função.}{vi.ca.ri.a.to}{0}
\verb{vicário}{}{}{}{}{adj.}{Que substitui outrem ou outra coisa.}{vi.cá.rio}{0}
\verb{vice}{}{}{}{}{s.2g.}{Redução de substantivos iniciados por \textit{vice"-}, como \textit{vice"-presidente, vice"-governador} etc.}{vi.ce}{0}
\verb{vice"-almirantado}{}{}{vice"-almirantados}{}{s.m.}{Cargo ou dignidade de vice"-almirante.}{vi.ce"-al.mi.ran.ta.do}{0}
\verb{vice"-almirante}{}{}{vice"-almirantes}{}{s.m.}{Posto de oficial da Marinha logo acima do contra"-almirante e abaixo do almirante"-de"-esquadra.}{vi.ce"-al.mi.ran.te}{0}
\verb{vice"-almirante}{}{}{vice"-almirantes}{}{}{Militar que ocupa esse posto.}{vi.ce"-al.mi.ran.te}{0}
\verb{vice"-campeão}{}{}{vice"-campeões}{}{s.m.}{Clube ou atleta que ficou com a segunda colocação em uma competição.}{vi.ce"-cam.pe.ão}{0}
\verb{vice"-cônsul}{}{}{vice"-cônsules}{}{s.m.}{Substituto do cônsul.}{vi.ce"-côn.sul}{0}
\verb{vice"-governador}{}{}{vice"-governadores}{}{s.m.}{Substituto do governador.}{vi.ce"-go.ver.na.dor}{0}
\verb{vicejante}{}{}{}{}{adj.2g.}{Que viceja, que tem viço.}{vi.ce.jan.te}{0}
\verb{vicejar}{}{}{}{}{v.i.}{Ter viço, vegetar com opulência.}{vi.ce.jar}{0}
\verb{vicejar}{}{}{}{}{}{Ostentar"-se de maneira exuberante.}{vi.ce.jar}{0}
\verb{vicejar}{}{}{}{}{v.t.}{Dar viço.}{vi.ce.jar}{0}
\verb{vicejar}{}{}{}{}{}{Produzir, lançar.}{vi.ce.jar}{\verboinum{1}}
\verb{vicejo}{ê}{}{}{}{s.m.}{Ato ou efeito de vicejar.}{vi.ce.jo}{0}
\verb{vice"-líder}{}{}{vice"-líderes}{}{s.m.}{Clube ou atleta que se encontra momentaneamente na segunda colocação em uma competição.}{vi.ce"-lí.der}{0}
\verb{vicênio}{}{}{}{}{s.m.}{Período de vinte anos.}{vi.cê.nio}{0}
\verb{vice"-prefeito}{ê}{}{vice"-prefeitos}{}{s.m.}{Substituto e auxiliar do prefeito.}{vi.ce"-pre.fei.to}{0}
\verb{vice"-presidência}{}{}{vice"-presidências}{}{s.f.}{Cargo ou dignidade de vice"-presidente.}{vi.ce"-pre.si.dên.cia}{0}
\verb{vice"-presidente}{}{}{vice"-presidentes}{}{s.m.}{Substituto e auxiliar do presidente.}{vi.ce"-pre.si.den.te}{0}
\verb{vice"-rei}{}{}{vice"-reis}{}{s.m.}{Governador de um Estado subordinado a um reino.}{vi.ce"-rei}{0}
\verb{vice"-reitor}{ô}{}{vice"-reitores ⟨ô⟩}{}{s.m.}{Substituto e auxiliar do reitor.}{vi.ce"-rei.tor}{0}
\verb{vice"-versa}{é}{}{}{}{adv.}{Em sentido oposto; ao contrário.}{vi.ce"-ver.sa}{0}
\verb{vice"-versa}{é}{}{}{}{}{Mutuamente, reciprocamente.}{vi.ce"-ver.sa}{0}
\verb{viciado}{}{}{}{}{adj.}{Que tem vício.}{vi.ci.a.do}{0}
\verb{viciado}{}{}{}{}{}{Que tem defeito ou impureza.}{vi.ci.a.do}{0}
\verb{viciado}{}{}{}{}{}{Adulterado ou falsificado.}{vi.ci.a.do}{0}
\verb{viciar}{}{}{}{}{v.t.}{Produzir vício.}{vi.ci.ar}{0}
\verb{viciar}{}{}{}{}{}{Corromper, perverter.}{vi.ci.ar}{0}
\verb{viciar}{}{}{}{}{}{Adulterar, falsificar.}{vi.ci.ar}{0}
\verb{viciar}{}{}{}{}{v.i.}{Causar dependência.}{vi.ci.ar}{0}
\verb{viciar}{}{}{}{}{v.pron.}{Adquirir vício.}{vi.ci.ar}{\verboinum{6}}
\verb{vicinal}{}{}{"-ais}{}{adj.2g.}{Que está próximo; vizinho, limítrofe.}{vi.ci.nal}{0}
\verb{vicinal}{}{}{"-ais}{}{}{Diz"-se de estrada de pequeno porte que liga localidades próximas.}{vi.ci.nal}{0}
\verb{vício}{}{}{}{}{s.m.}{Defeito grave.}{ví.cio}{0}
\verb{vício}{}{}{}{}{}{Mau hábito.}{ví.cio}{0}
\verb{vício}{}{}{}{}{}{Desregramento habitual; costume condenável; devassidão.}{ví.cio}{0}
\verb{vício}{}{}{}{}{}{Apego a entorpecentes.}{ví.cio}{0}
\verb{vicioso}{ô}{}{"-osos ⟨ó⟩}{"-osa ⟨ó⟩}{adj.}{Que tem vício.}{vi.ci.o.so}{0}
\verb{vicioso}{ô}{}{"-osos ⟨ó⟩}{"-osa ⟨ó⟩}{}{Adulterado, falsificado.}{vi.ci.o.so}{0}
\verb{vicioso}{ô}{}{"-osos ⟨ó⟩}{"-osa ⟨ó⟩}{}{Corrompido.}{vi.ci.o.so}{0}
\verb{vicioso}{ô}{}{"-osos ⟨ó⟩}{"-osa ⟨ó⟩}{}{Contrário a certas regras.}{vi.ci.o.so}{0}
\verb{vicissitude}{}{}{}{}{s.f.}{Variação de coisas que se sucedem.}{vi.cis.si.tu.de}{0}
\verb{vicissitude}{}{}{}{}{}{Eventualidade, contingência, acaso.}{vi.cis.si.tu.de}{0}
\verb{vicissitude}{}{}{}{}{}{Revés.}{vi.cis.si.tu.de}{0}
\verb{viço}{}{}{}{}{s.m.}{Exuberância, vigor, verdor, frescor.}{vi.ço}{0}
\verb{viço}{}{}{}{}{}{Crescimento opulento.}{vi.ço}{0}
\verb{viço}{}{}{}{}{}{Carinho, mimo.}{vi.ço}{0}
\verb{viço}{}{}{}{}{}{Juventude.}{vi.ço}{0}
\verb{viçoso}{ô}{}{"-osos ⟨ó⟩}{"-osa ⟨ó⟩}{adj.}{Que tem viço.}{vi.ço.so}{0}
\verb{viçoso}{ô}{Fig.}{"-osos ⟨ó⟩}{"-osa ⟨ó⟩}{}{Inexperiente.}{vi.ço.so}{0}
\verb{vicunha}{}{}{}{}{s.f.}{Mamífero quadrúpede que possui lã macia, encontrado nos Andes.}{vi.cu.nha}{0}
\verb{vicunha}{}{Por ext.}{}{}{}{A lã extraída desse animal.}{vi.cu.nha}{0}
\verb{vicunha}{}{Por ext.}{}{}{}{O tecido fabricado com essa lã.}{vi.cu.nha}{0}
\verb{vida}{}{}{}{}{s.f.}{Capacidade de animais e vegetais estarem em atividade, se alimentarem, crescerem e se reproduzirem.}{vi.da}{0}
\verb{vida}{}{}{}{}{}{Tempo que vai do nascimento até a morte, do começo ao fim; existência.}{vi.da}{0}
\verb{vida}{}{}{}{}{}{Modo, maneira de viver.}{vi.da}{0}
\verb{vida}{}{}{}{}{}{Animação, entusiasmo. (\textit{As flores deram vida ao pequeno sítio.})}{vi.da}{0}
\verb{vidão}{}{Bras.}{}{}{s.m.}{Vida de regalos e prazeres; boa vida.}{vi.dão}{0}
\verb{vide}{}{}{}{}{s.f.}{Bacelo.}{vi.de}{0}
\verb{vide}{}{}{}{}{}{Videira.}{vi.de}{0}
\verb{vide}{}{}{}{}{}{Fórmula com que o leitor se remete a outro trecho ou livro.  }{\textit{vide}}{0}
\verb{videira}{ê}{}{}{}{s.f.}{Trepadeira lenhosa, dotada de gavinhas, da família das vitáceas, cultivada em todo o mundo desde a Antiguidade por seus frutos em bagas, as uvas, muito apreciadas, e das quais se faz vinho; vide, vinha, cepa.  }{vi.dei.ra}{0}
\verb{vidência}{}{}{}{}{s.f.}{Qualidade ou faculdade de quem é vidente.}{vi.dên.cia}{0}
\verb{vidente}{}{}{}{}{adj.2g.}{Diz"-se daquele capaz de ver o passado, o futuro, ou o que se passa num lugar sem estar nele presente.}{vi.den.te}{0}
\verb{vidente}{}{}{}{}{s.2g.}{Indivíduo com essa faculdade.}{vi.den.te}{0}
\verb{vidente}{}{}{}{}{}{Indivíduo que faz profecias.}{vi.den.te}{0}
\verb{vidente}{}{}{}{}{}{Indivíduo que pode fazer uso da visão, que pode ver.}{vi.den.te}{0}
\verb{vídeo}{}{}{}{}{s.m.}{Tela de televisor ou computador, onde aparece a imagem.}{ví.deo}{0}
\verb{vídeo}{}{}{}{}{}{Forma reduzida de \textit{videocassete}.}{ví.deo}{0}
\verb{vídeo}{}{}{}{}{}{Forma reduzida de \textit{videoteipe}.}{ví.deo}{0}
\verb{videoarte}{}{Art.}{}{}{s.f.}{Uso de imagens e sons gravados em vídeo nas artes plásticas.}{vi.de.o.ar.te}{0}
\verb{videoarte}{}{}{}{}{}{Obra de videoarte.}{vi.de.o.ar.te}{0}
\verb{videocassete}{é}{}{}{}{s.m.}{Cassete com fita magnética que registra imagens e sons.}{vi.de.o.cas.se.te}{0}
\verb{videocassete}{é}{Por ext.}{}{}{}{Equipamento que reproduz os sons e imagens gravados nessa fita.}{vi.de.o.cas.se.te}{0}
\verb{videoclipe}{}{}{}{}{s.m.}{Filme de curta duração que acompanha uma música ou apresenta o trabalho de um artista.}{vi.de.o.cli.pe}{0}
\verb{videoclube}{}{}{}{}{s.m.}{Clube especializado na exibição de filmes em videocassete.}{vi.de.o.clu.be}{0}
\verb{videoclube}{}{}{}{}{}{Videolocadora.}{vi.de.o.clu.be}{0}
\verb{videoconferência}{}{}{}{}{s.f.}{Conferência realizada a distância, com transmissão de imagem e som entre os participantes, situados em lugares diferentes uns dos outros, por meio de certos dispositivos (televisão, rede de computadores etc.).}{vi.de.o.con.fe.rên.cia}{0}
\verb{videodisco}{}{Informát.}{}{}{s.m.}{Disco com gravações digitalizadas de áudio e vídeo, que é lido por meio de um dispositivo óptico que usa um feixe de \textit{laser}.}{vi.de.o.dis.co}{0}
\verb{video game}{}{}{}{}{s.m.}{Jogo eletrônico, interativo, que se disputa numa tela de televisão.}{\textit{video game}}{0}
\verb{videolaparoscopia}{}{Med.}{}{}{s.f.}{Procedimento de endoscopia que permite o exame da cavidade abdominal, e eventual realização de intervenção cirúrgica nela, por meio de uma câmera de vídeo.  }{vi.de.o.la.pa.ros.co.pi.a}{0}
\verb{videolocadora}{ô}{}{}{}{s.f.}{Estabelecimento especializado no empréstimo de fitas de videocassete, por um determinado período, e mediante pagamento; videoclube.}{vi.de.o.lo.ca.do.ra}{0}
\verb{videoteipe}{}{}{}{}{s.m.}{Fita magnética usada para gravação, edição e reprodução de imagens, geralmente acompanhadas de sons.}{vi.de.o.tei.pe}{0}
\verb{videoteipe}{}{Por ext.}{}{}{}{O processo pelo qual se registram as imagens e sons nessa fita.}{vi.de.o.tei.pe}{0}
\verb{videotexto}{ês}{}{}{}{s.m.}{Sistema de visualização de informações em monitor de vídeo, na forma de textos transmitidos por linha telefônica ou televisão a cabo.}{vi.de.o.tex.to}{0}
\verb{vidraça}{}{}{}{}{s.f.}{Lâmina de vidro.}{vi.dra.ça}{0}
\verb{vidraça}{}{}{}{}{}{Caixilho com vidro para janelas ou portas.}{vi.dra.ça}{0}
\verb{vidraçaria}{}{}{}{}{s.f.}{Estabelecimento onde se fabricam ou  se vendem vidros e vidraças; vidraria.}{vi.dra.ça.ri.a}{0}
\verb{vidraçaria}{}{}{}{}{}{Conjunto de vidraças de uma edificação (casa, apartamento etc.).}{vi.dra.ça.ri.a}{0}
\verb{vidraceiro}{ê}{}{}{}{s.m.}{Indivíduo que vende ou fabrica vidros e vidraças.}{vi.dra.cei.ro}{0}
\verb{vidraceiro}{ê}{}{}{}{}{Profissional que coloca vidros em caixilhos ou molduras.}{vi.dra.cei.ro}{0}
\verb{vidrado}{}{}{}{}{adj.}{Recoberto de matéria parecida com o vidro; vitrificado.}{vi.dra.do}{0}
\verb{vidrado}{}{Pop.}{}{}{}{Encantado, deslumbrado, gamado.}{vi.dra.do}{0}
\verb{vidrar}{}{}{}{}{v.t.}{Recobrir com matéria parecida com o vidro; vitrificar.}{vi.drar}{0}
\verb{vidrar}{}{Pop.}{}{}{}{Ficar deslumbrado, encantado com alguém ou alguma coisa; gamar.}{vi.drar}{\verboinum{1}}
\verb{vidraria}{}{}{}{}{s.f.}{Arte ou comércio de vidros.}{vi.dra.ri.a}{0}
\verb{vidraria}{}{}{}{}{}{Vidraçaria.}{vi.dra.ri.a}{0}
\verb{vidreiro}{ê}{}{}{}{adj.}{Relativo à indústria de vidros.}{vi.drei.ro}{0}
\verb{vidreiro}{ê}{}{}{}{s.m.}{Indivíduo que trabalha com vidro.}{vi.drei.ro}{0}
\verb{vidrilho}{}{}{}{}{s.m.}{Conta de vidro, ou de outro material, na forma de um pequeno cilindro oco, usada na confecção de bijuterias, ornatos, bordados etc.; miçanga.}{vi.dri.lho}{0}
\verb{vidro}{}{}{}{}{s.m.}{Matéria sólida, transparente ou translúcida, que se quebra com facilidade, de que são feitas as vidraças de janelas, garrafas, vasos etc.}{vi.dro}{0}
\verb{viela}{é}{}{}{}{s.f.}{Rua curta e estreita; beco.}{vi.e.la}{0}
\verb{viés}{}{}{}{}{s.m.}{Direção ou trajetória oblíqua, em diagonal. }{vi.és}{0}
\verb{viés}{}{}{}{}{}{Palavra usada na expressão \textit{de viés}: obliquamente, de esguelha.}{vi.és}{0}
\verb{vietnamita}{}{}{}{}{adj.2g.}{Relativo a Vietnã (Sudeste asiático).}{vi.et.na.mi.ta}{0}
\verb{vietnamita}{}{}{}{}{s.2g.}{Indivíduo natural ou habitante desse país.}{vi.et.na.mi.ta}{0}
\verb{vietnamita}{}{}{}{}{s.m.}{Uma das línguas faladas no Vietnã.}{vi.et.na.mi.ta}{0}
\verb{viga}{}{}{}{}{s.f.}{Peça comprida de madeira, concreto, ferro etc., usada na sustentação horizontal de edificações; trave.}{vi.ga}{0}
\verb{vigamento}{}{}{}{}{s.m.}{Conjunto de vigas de uma construção; travejamento.}{vi.ga.men.to}{0}
\verb{vigarice}{}{}{}{}{s.f.}{Ato próprio de vigarista; trapaça, logro, conto"-do"-vigário.}{vi.ga.ri.ce}{0}
\verb{vigário}{}{}{}{}{s.m.}{Sacerdote que dirige uma paróquia; pároco.}{vi.gá.rio}{0}
\verb{vigarista}{}{}{}{}{adj.2g.}{Que pratica o conto"-do"-vigário, que engana as pessoas para obter vantagens; trapaceiro, embusteiro, espertalhão, tratante, velhaco.}{vi.ga.ris.ta}{0}
\verb{vigarista}{}{}{}{}{s.2g.}{Pessoa que passa o conto"-do"-vigário.}{vi.ga.ris.ta}{0}
\verb{vigência}{}{}{}{}{s.f.}{Qualidade ou condição de vigente.}{vi.gên.cia}{0}
\verb{vigência}{}{}{}{}{}{Tempo durante o qual alguma coisa (lei, regulamento etc.) vigora, tem validade.}{vi.gên.cia}{0}
\verb{vigente}{}{}{}{}{adj.2g.}{Que vige ou vigora; que tem validade.}{vi.gen.te}{0}
\verb{viger}{ê}{}{}{}{v.i.}{Estar em vigor; valer, vigorar. }{vi.ger}{\verboinum{16}}
\verb{vigésimo}{}{}{}{}{num.}{Ordinal e fracionário correspondente a vinte.}{vi.gé.si.mo}{0}
\verb{vigésimo}{}{}{}{}{s.m.}{Uma das vinte partes iguais de alguma coisa.}{vi.gé.si.mo}{0}
\verb{vigia}{}{}{}{}{s.f.}{Ato de vigiar; vigilância. }{vi.gi.a}{0}
\verb{vigia}{}{}{}{}{s.2g.}{Pessoa que vigia; guarda, sentinela.}{vi.gi.a}{0}
\verb{vigiar}{}{}{}{}{v.t.}{Observar com atenção alguém ou alguma coisa; estar atento; vigilar.}{vi.gi.ar}{0}
\verb{vigiar}{}{}{}{}{}{Cuidar de alguma coisa ou de alguém com atenção; velar.}{vi.gi.ar}{\verboinum{6}}
\verb{vígil}{}{}{"-eis}{}{adj.2g.}{Que vigia ou vela.}{ví.gil}{0}
\verb{vigilância}{}{}{}{}{s.f.}{Ato ou efeito de vigiar.}{vi.gi.lân.cia}{0}
\verb{vigilante}{}{}{}{}{adj.2g.}{Que vigia; vígil.}{vi.gi.lan.te}{0}
\verb{vigilante}{}{}{}{}{s.2g.}{Guarda, vigia.}{vi.gi.lan.te}{0}
\verb{vigilar}{}{}{}{}{v.t.}{Vigiar.}{vi.gi.lar}{\verboinum{1}}
\verb{vigília}{}{}{}{}{s.f.}{Estado de quem está vigilante, acordado, desperto; insônia.}{vi.gí.lia}{0}
\verb{vigília}{}{}{}{}{}{Condição de quem passa a noite acordado; velada.}{vi.gí.lia}{0}
\verb{vigor}{ô}{}{}{}{s.m.}{Força, energia, robustez.}{vi.gor}{0}
\verb{vigor}{ô}{}{}{}{}{Capacidade de ação; vitalidade.}{vi.gor}{0}
\verb{vigor}{ô}{}{}{}{}{Tempo durante o qual uma lei, regulamento, decreto etc. tem validade.}{vi.gor}{0}
\verb{vigorante}{}{}{}{}{adj.2g.}{Que vigora; vigente.}{vi.go.ran.te}{0}
\verb{vigorante}{}{}{}{}{}{Que torna vigoroso; fortalecedor, reparador.}{vi.go.ran.te}{0}
\verb{vigorar}{}{}{}{}{v.t.}{Dar vigor; fortalecer.}{vi.go.rar}{0}
\verb{vigorar}{}{}{}{}{v.i.}{Adquirir força, robustez.}{vi.go.rar}{0}
\verb{vigorar}{}{}{}{}{}{Estar em vigor, ter vigência (lei, regulamento etc.); viger.}{vi.go.rar}{\verboinum{1}}
\verb{vigoroso}{ô}{}{"-osos ⟨ó⟩}{"-osa ⟨ó⟩}{adj.}{Que tem vigor, energia; robusto, forte.}{vi.go.ro.so}{0}
\verb{vil}{}{}{vis}{}{adj.2g.}{Que tem pouco valor; reles, ordinário.}{vil}{0}
\verb{vil}{}{}{vis}{}{}{Que inspira desprezo; abjeto, indigno, infame.}{vil}{0}
\verb{vila}{}{}{}{}{s.f.}{Povoado pequeno, menor que uma cidade e maior que uma aldeia.}{vi.la}{0}
\verb{vila}{}{}{}{}{}{Conjunto de casas dispostas de maneira a formar uma rua particular.}{vi.la}{0}
\verb{vila}{}{}{}{}{}{Casa de campo.}{vi.la}{0}
\verb{vilania}{}{}{}{}{s.f.}{Vileza.}{vi.la.ni.a}{0}
\verb{vilão}{}{Desus.}{vilões, vilãos \textit{ou} vilães}{vilã \textit{ou} viloa}{adj.}{Que vive numa vila.}{vi.lão}{0}
\verb{vilão}{}{}{vilões, vilãos \textit{ou} vilães}{vilã \textit{ou} viloa}{s.m.}{Pessoa maldosa, cruel, desprezível.}{vi.lão}{0}
\verb{vilão}{}{}{vilões, vilãos \textit{ou} vilães}{vilã \textit{ou} viloa}{}{No teatro, nas novelas e nos filmes, o personagem mau e cruel; bandido. }{vi.lão}{0}
\verb{vilarejo}{ê}{}{}{}{s.m.}{Vila ou povoado pequeno; lugarejo.}{vi.la.re.jo}{0}
\verb{vilegiatura}{}{Desus.}{}{}{s.f.}{Férias, temporada de descanso que se passa fora da cidade, na praia, no campo etc.; veraneio.}{vi.le.gi.a.tu.ra}{0}
\verb{vileza}{ê}{}{}{}{s.f.}{Qualidade de vil.}{vi.le.za}{0}
\verb{vileza}{ê}{}{}{}{}{Ato vil, degradante; indignidade, infâmia, abjeção.}{vi.le.za}{0}
\verb{vilipendiar}{}{}{}{}{v.t.}{Tratar alguém com desprezo ou desdém; desprezar, aviltar, repelir.}{vi.li.pen.di.ar}{\verboinum{6}}
\verb{vilipêndio}{}{}{}{}{s.m.}{Ofensa que faz com que a pessoa se sinta desprezada; humilhação, desprezo, menoscabo, aviltamento, desvalorização.}{vi.li.pên.dio}{0}
\verb{vime}{}{}{}{}{s.m.}{Vara de vimeiro, que se dobra com facilidade, usada para fazer cestos, móveis etc.}{vi.me}{0}
\verb{vimeiro}{ê}{Bot.}{}{}{s.m.}{Árvore ou arbusto, de folhas delgadas, longos ramos pendentes, que cresce nos terrenos úmidos ou à beira dos rios, cultivada como ornamental ou pelas madeiras; salgueiro.}{vi.mei.ro}{0}
\verb{vinagre}{}{}{}{}{s.m.}{Líquido azedo, que se obtém por meio da fermentação do ácido de certas bebidas, frutas ou cereais, muito usado como condimento em saladas; ácido acético.}{vi.na.gre}{0}
\verb{vinagreira}{ê}{}{}{}{s.f.}{Recipiente onde se faz ou se acondiciona vinagre.}{vi.na.grei.ra}{0}
\verb{vinagreiro}{ê}{}{}{}{s.m.}{Indivíduo que vende ou fabrica vinagre.}{vi.na.grei.ro}{0}
\verb{vinagrete}{é}{Cul.}{}{}{s.m.}{Tipo de molho preparado com vinagre, azeite, cebola, alho, pimenta do reino e sal, entre outros ingredientes, usado como acompanhamento de diversos pratos.}{vi.na.gre.te}{0}
\verb{vincar}{}{}{}{}{v.t.}{Fazer vincos; preguear, enrugar, dobrar.}{vin.car}{\verboinum{2}}
\verb{vincendo}{}{}{}{}{adj.}{Diz"-se de juros, dívidas etc., que estão por vencer.  }{vin.cen.do}{0}
\verb{vinco}{}{}{}{}{s.m.}{Marca ou sulco deixado por uma dobra.}{vin.co}{0}
\verb{vinculação}{}{}{"-ões}{}{s.f.}{Ato ou efeito de vincular.}{vin.cu.la.ção}{0}
\verb{vinculador}{ô}{}{}{}{adj.}{Que vincula; vinculativo, vinculatório.  }{vin.cu.la.dor}{0}
\verb{vincular}{}{}{}{}{v.t.}{Prender ou ligar com vínculos; criar ligações entre pessoas ou coisas.}{vin.cu.lar}{\verboinum{1}}
\verb{vinculatório}{}{}{}{}{adj.}{Vinculador.}{vin.cu.la.tó.rio}{0}
\verb{vínculo}{}{}{}{}{s.m.}{Aquilo que liga, une, ata; nó, liame.}{vín.cu.lo}{0}
\verb{vínculo}{}{}{}{}{}{Aquilo que liga duas ou mais pessoas; laço, elo.}{vín.cu.lo}{0}
\verb{vinda}{}{}{}{}{s.f.}{Ato ou efeito de vir; volta, regresso.}{vin.da}{0}
\verb{vindicar}{}{}{}{}{v.t.}{Exigir ou reclamar a devolução ou entrega; reivindicar.}{vin.di.car}{0}
\verb{vindicar}{}{}{}{}{}{Exigir algo em nome da lei.}{vin.di.car}{\verboinum{2}}
\verb{vindicativo}{}{}{}{}{adj.}{Que pode ou é próprio pra vindicar; punitivo.}{vin.di.ca.ti.vo}{0}
\verb{vindicativo}{}{}{}{}{}{Que (se) vinga; vingador.}{vin.di.ca.ti.vo}{0}
\verb{vindima}{}{}{}{}{s.f.}{Colheita das uvas.}{vin.di.ma}{0}
\verb{vindima}{}{}{}{}{}{As uvas colhidas, vindimadas.}{vin.di.ma}{0}
\verb{vindima}{}{Por ext.}{}{}{}{O tempo da colheita das uvas.}{vin.di.ma}{0}
\verb{vindimar}{}{}{}{}{v.t.}{Colher uvas.}{vin.di.mar}{\verboinum{1}}
\verb{vindita}{}{}{}{}{s.f.}{Pena, punição legal.}{vin.di.ta}{0}
\verb{vindita}{}{}{}{}{}{Vingança.}{vin.di.ta}{0}
\verb{vindo}{}{}{}{}{adj.}{Que veio; que chegou.     }{vin.do}{0}
\verb{vindo}{}{}{}{}{}{Proveniente, procedente.}{vin.do}{0}
\verb{vindoiro}{ô}{}{}{}{}{Var. de \textit{vindouro}.}{vin.doi.ro}{0}
\verb{vindouro}{ô}{}{}{}{adj.}{Que vem, que está por vir; futuro.}{vin.dou.ro}{0}
\verb{vingador}{ô}{}{}{}{adj.}{Que (se) vinga.}{vin.ga.dor}{0}
\verb{vingador}{ô}{}{}{}{s.m.}{Indivíduo que (se) vinga, que tem sede de vingança.}{vin.ga.dor}{0}
\verb{vingança}{}{}{}{}{s.f.}{Ato de (se) vingar; desforra.}{vin.gan.ça}{0}
\verb{vingar}{}{}{}{}{v.t.}{Tirar a desforra; castigar, punir alguém que fez algum mal.}{vin.gar}{0}
\verb{vingar}{}{}{}{}{v.i.}{Alcançar êxito; resistir (animal ou planta recém"-nascida); medrar. }{vin.gar}{\verboinum{5}}
\verb{vingativo}{}{}{}{}{adj.}{Que se vinga, que tem o desejo de vingança.}{vin.ga.ti.vo}{0}
\verb{vinha}{}{}{}{}{s.f.}{Plantação de videiras.}{vi.nha}{0}
\verb{vinhaça}{}{}{}{}{s.f.}{Vinho de má qualidade.}{vi.nha.ça}{0}
\verb{vinhaça}{}{}{}{}{}{Grande porção de vinho.}{vi.nha.ça}{0}
\verb{vinhaça}{}{Por ext.}{}{}{}{Bebedeira.}{vi.nha.ça}{0}
\verb{vinha"-d'alhos}{}{Cul.}{vinhas"-d'alhos}{}{s.f.}{Molho para tempero, constituído de alho socado e outras especiarias dissolvidas em vinagre de vinho.}{vi.nha"-d'a.lhos}{0}
\verb{vinhadeiro}{ê}{}{}{}{s.m.}{Vinheiro.}{vi.nha.dei.ro}{0}
\verb{vinhal}{}{}{"-ais}{}{s.m.}{Vinhedo.}{vi.nhal}{0}
\verb{vinhataria}{}{}{}{}{s.f.}{Cultura de vinhas.}{vi.nha.ta.ri.a}{0}
\verb{vinhataria}{}{}{}{}{}{Fabricação de vinho.}{vi.nha.ta.ri.a}{0}
\verb{vinhateiro}{ê}{}{}{}{adj.}{Relativo à vinhataria.}{vi.nha.tei.ro}{0}
\verb{vinhateiro}{ê}{}{}{}{}{Aquele que cultiva vinhas.}{vi.nha.tei.ro}{0}
\verb{vinhateiro}{ê}{}{}{}{s.m.}{Indivíduo que cultiva videiras.}{vi.nha.tei.ro}{0}
\verb{vinhateiro}{ê}{}{}{}{}{Indivíduo que fabrica vinhos.}{vi.nha.tei.ro}{0}
\verb{vinhático}{}{Bot.}{}{}{s.m.}{Designação de certas árvores aproveitadas pela madeira amarela.}{vi.nhá.ti.co}{0}
\verb{vinhedo}{ê}{}{}{}{s.m.}{Grande extensão de vinhas.}{vi.nhe.do}{0}
\verb{vinhedo}{ê}{}{}{}{}{O conjunto das vinhas de uma região, de um país.}{vi.nhe.do}{0}
\verb{vinheta}{ê}{}{}{}{s.f.}{Pequena estampa de um livro, para ornato ou para explicação do texto.}{vi.nhe.ta}{0}
\verb{vinheta}{ê}{}{}{}{}{Enfeite ou cercadura de uma só peça que serve de ornato numa composição tipográfica.}{vi.nhe.ta}{0}
\verb{vinheta}{ê}{}{}{}{}{Peça curta para televisão ou rádio, repetida várias vezes durante a programação.}{vi.nhe.ta}{0}
\verb{vinho}{}{}{}{}{s.m.}{Bebida alcoólica feita da fermentação do suco da uva.}{vi.nho}{0}
\verb{vinho}{}{Fig.}{}{}{}{Coisa que embriaga, que inebria.}{vi.nho}{0}
\verb{vinho}{}{}{}{}{adj.}{Da cor do vinho tinto.}{vi.nho}{0}
\verb{vinhoto}{ô}{}{}{}{s.m.}{Resíduo da destilação do álcool da cana"-de"-açúcar.}{vi.nho.to}{0}
\verb{vínico}{}{}{}{}{adj.}{Relativo ao vinho.}{ví.ni.co}{0}
\verb{vínico}{}{}{}{}{}{Que é extraído do vinho.}{ví.ni.co}{0}
\verb{vinícola}{}{}{}{}{adj.}{Relativo à vinicultura.}{vi.ní.co.la}{0}
\verb{vinicultor}{ô}{}{}{}{s.m.}{Indivíduo que se dedica à vinicultura.}{vi.ni.cul.tor}{0}
\verb{vinicultura}{}{}{}{}{s.f.}{Produção de vinhos.}{vi.ni.cul.tu.ra}{0}
\verb{vinil}{}{Quím.}{"-is}{}{s.m.}{Produto sintético que tem numerosas aplicações, é usado como substituto do couro, na fabricação de discos fonográficos etc.}{vi.nil}{0}
\verb{vinte}{}{}{}{}{num.}{Nome dado à quantidade expressa pelo número 20.  }{vin.te}{0}
\verb{vinte"-e"-um}{}{}{}{}{s.m.}{Jogo de cartas em que ganha quem completa 21 pontos exatos ou o mais próximo disso.  }{vin.te"-e"-um}{0}
\verb{vintém}{"-éns}{}{}{}{s.m.}{Antiga moeda de cobre, de Portugal e do Brasil, equivalente a vinte réis.}{vin.tém}{0}
\verb{vintena}{}{}{}{}{s.f.}{Grupo de vinte, pessoas ou coisas.}{vin.te.na}{0}
\verb{vintena}{}{}{}{}{}{A vigésima parte.}{vin.te.na}{0}
\verb{viola}{ó}{}{}{}{s.f.}{Instumento de cordas dedilháveis, semelhante ao violão, porém menor, com cinco ou seis cordas duplas de metal, de larga utilização na música rural brasileira e portuguesa.}{vi.o.la}{0}
\verb{viola}{ó}{}{}{}{}{Instrumento de arco e cordas friccionáveis semelhante ao violino, porém de maiores dimensões e som mais grave.}{vi.o.la}{0}
\verb{violação}{}{}{"-ões}{}{s.f.}{Ato ou efeito de violar; infração, transgressão.}{vi.o.la.ção}{0}
\verb{violação}{}{}{"-ões}{}{}{Estupro.}{vi.o.la.ção}{0}
\verb{violação}{}{}{"-ões}{}{}{Ofensa ao direito alheio.}{vi.o.la.ção}{0}
\verb{violação}{}{}{"-ões}{}{}{Qualquer transgressão a normas, leis ou obrigações contratuais.}{vi.o.la.ção}{0}
\verb{violácea}{}{Bot.}{}{}{s.f.}{Espécime das violáceas, família de ervas, trepadeiras e arbustos floríferos, que habitam principalmente áreas temperadas.}{vi.o.lá.cea}{0}
\verb{violáceo}{}{}{}{}{adj.}{Relativo ou semelhante à violeta.}{vi.o.lá.ceo}{0}
\verb{violáceo}{}{}{}{}{}{Da cor da violeta.}{vi.o.lá.ceo}{0}
\verb{violão}{}{}{"-ões}{}{s.m.}{Instrumento de cordas dedilháveis, com caixa de ressonânca em formato semelhante a um oito, com seis cordas, de diferentes materiais.}{vi.o.lão}{0}
\verb{violão}{}{Por ext.}{"-ões}{}{}{Violonista.}{vi.o.lão}{0}
\verb{violão}{}{Pop.}{"-ões}{}{}{Mulher de formas arredondadas, ancas largas e cintura fina.}{vi.o.lão}{0}
\verb{violar}{}{}{}{}{v.t.}{Ofender com violência.}{vi.o.lar}{0}
\verb{violar}{}{}{}{}{}{Infringir, transgredir.}{vi.o.lar}{0}
\verb{violar}{}{}{}{}{}{Estuprar, violentar.}{vi.o.lar}{0}
\verb{violar}{}{}{}{}{}{Profanar.}{vi.o.lar}{0}
\verb{violar}{}{}{}{}{}{Divulgar, revelar, de modo abusivo.}{vi.o.lar}{\verboinum{1}}
\verb{violeiro}{ê}{}{}{}{s.m.}{Indivíduo que fabrica instrumentos de corda.}{vi.o.lei.ro}{0}
\verb{violeiro}{ê}{}{}{}{}{Indivíduo que toca viola.}{vi.o.lei.ro}{0}
\verb{violência}{}{}{}{}{s.f.}{Qualidade de violento.}{vi.o.lên.cia}{0}
\verb{violência}{}{}{}{}{}{Ato de violento.}{vi.o.lên.cia}{0}
\verb{violência}{}{}{}{}{}{Ato ou efeito de violentar.}{vi.o.lên.cia}{0}
\verb{violentar}{}{}{}{}{v.t.}{Exercer violência; forçar, coagir.}{vi.o.len.tar}{0}
\verb{violentar}{}{}{}{}{}{Estuprar, violar.}{vi.o.len.tar}{0}
\verb{violentar}{}{}{}{}{}{Forçar, arrombar.}{vi.o.len.tar}{0}
\verb{violentar}{}{}{}{}{}{Desrespeitar.}{vi.o.len.tar}{\verboinum{1}}
\verb{violento}{}{}{}{}{adj.}{Que atua com força, com ímpeto.}{vi.o.len.to}{0}
\verb{violento}{}{}{}{}{}{Em que se usa de força física ou brutalidade.}{vi.o.len.to}{0}
\verb{violento}{}{}{}{}{}{Irritadiço, colérico.}{vi.o.len.to}{0}
\verb{violento}{}{}{}{}{}{Intenso, veemente.}{vi.o.len.to}{0}
\verb{violeta}{ê}{Bot.}{}{}{s.f.}{Erva originária da Europa, muito cultivada pelo valor decorativo e pelo perfume, de folhas arredondadas, flores pequenas, mas vistosas pela coloração. }{vi.o.le.ta}{0}
\verb{violeta}{ê}{}{}{}{}{A flor dessa planta.}{vi.o.le.ta}{0}
\verb{violeta}{ê}{}{}{}{s.m.}{A cor violeta; roxo.}{vi.o.le.ta}{0}
\verb{violeta}{ê}{}{}{}{adj.2g.}{Que tem a cor da violeta.}{vi.o.le.ta}{0}
\verb{violinista}{}{}{}{}{s.2g.}{Indivíduo que toca violino.}{vi.o.li.nis.ta}{0}
\verb{violino}{}{Mús.}{}{}{s.m.}{Instrumento musical de quatro cordas, executado com um arco.}{vi.o.li.no}{0}
\verb{violino}{}{Por ext.}{}{}{}{Violinista.}{vi.o.li.no}{0}
\verb{violoncelista}{}{}{}{}{s.2g.}{Indivíduo que toca violoncelo.}{vi.o.lon.ce.lis.ta}{0}
\verb{violoncelo}{é}{Mús.}{}{}{s.m.}{Instrumento musical com a forma do violino, de maiores dimensões, que se toca com um arco, apoiando"-o ao solo.}{vi.o.lon.ce.lo}{0}
\verb{violoncelo}{é}{Por ext.}{}{}{}{Violoncelista.}{vi.o.lon.ce.lo}{0}
\verb{violonista}{}{}{}{}{s.2g.}{Indivíduo que toca violino.}{vi.o.lo.nis.ta}{0}
\verb{VIP}{}{}{}{}{s.m.}{Indivíduo de grande prestígio.}{VIP}{0}
\verb{VIP}{}{}{}{}{adj.2g.}{Sala ou local destinado a essa pessoa.}{VIP}{0}
\verb{viperino}{}{}{}{}{adj.}{Relativo à vibora.}{vi.pe.ri.no}{0}
\verb{viperino}{}{}{}{}{}{Da natureza da víbora; venenoso, peçonhento.}{vi.pe.ri.no}{0}
\verb{viperino}{}{Fig.}{}{}{}{Perverso, maléfico.}{vi.pe.ri.no}{0}
\verb{vir}{}{}{}{}{v.t.}{Deslocar"-se para cá, na direção daquele que fala.}{vir}{0}
\verb{vir}{}{}{}{}{}{Ter origem; originar"-se, proceder, provir.}{vir}{\verboinum{56}}
\verb{vira}{}{}{}{}{s.f.}{Tira de couro que se costura entre as solas do calçado, junto à borda destas.}{vi.ra}{0}
\verb{vira}{}{}{}{}{s.m.}{Certa dança e música popular portuguesa.}{vi.ra}{0}
\verb{vira"-bosta}{ó}{Zool.}{vira"-bostas ⟨ó⟩}{}{s.m.}{Besouro grande; escaravelho.}{vi.ra"-bos.ta}{0}
\verb{vira"-bosta}{ó}{Zool.}{vira"-bostas ⟨ó⟩}{}{}{Denominação de vários pássaros pretos, como o chupim, que revolvem o esterco à procura de alimento.}{vi.ra"-bos.ta}{0}
\verb{virabrequim}{}{}{"-ins}{}{s.m.}{Peça do motor de explosão que possibilita o movimento alternado dos êmbolos.}{vi.ra.bre.quim}{0}
\verb{viração}{}{}{"-ões}{}{s.f.}{Vento brando e fresco, que, à tarde, costuma soprar do mar para terra; brisa.}{vi.ra.ção}{0}
\verb{viração}{}{}{"-ões}{}{}{Local de desova das tartarugas.}{vi.ra.ção}{0}
\verb{vira"-casaca}{}{}{vira"-casacas}{}{s.2g.}{Indivíduo que troca de partido ou de opinião, de acordo com seus interesses pessoais. }{vi.ra"-ca.sa.ca}{0}
\verb{virada}{}{}{}{}{s.f.}{Ato ou efeito de virar.}{vi.ra.da}{0}
\verb{virada}{}{}{}{}{}{Mudança, reviravolta.}{vi.ra.da}{0}
\verb{viradinho}{}{Cul.}{}{}{s.m.}{Prato paulista feito de feijão, refogado com óleos e temperos, misturado com um pouco de farinha de milho ou mandioca, sendo muitas vezes guarnecido de linguiça, costeletas de porco, torresmo e ovo.}{vi.ra.di.nho}{0}
\verb{virado}{}{}{}{}{adj.}{Que alguém virou ou que virou por si mesmo.}{vi.ra.do}{0}
\verb{virado}{}{Cul.}{}{}{s.m.}{Viradinho.}{vi.ra.do}{0}
\verb{virador}{ô}{}{}{}{s.m.}{O ponto de curso de um rio de onde os canoeiros retornam.}{vi.ra.dor}{0}
\verb{virago}{}{}{}{}{s.f.}{Mulher de aspecto, inclinações sexuais e hábitos masculinos.}{vi.ra.go}{0}
\verb{virago}{}{}{}{}{}{Cabo, corda.}{vi.ra.go}{0}
\verb{viral}{}{}{"-ais}{}{adj.2g.}{Relativo a vírus ou causado por ele.}{vi.ral}{0}
\verb{vira"-lata}{}{}{vira"-latas}{}{s.m.}{Cachorro de rua, sem raça definida, sem dono, que se alimenta do que encontra em sacos ou latas de lixo.}{vi.ra"-la.ta}{0}
\verb{vira"-lata}{}{Fig.}{vira"-latas}{}{}{Indivíduo desclassificado, sem"-vergonha.}{vi.ra"-la.ta}{0}
%\verb{}{}{}{}{}{}{}{}{0}
\verb{virar}{}{}{}{}{v.t.}{Fazer alguma coisa se mover em volta de seu eixo.}{vi.rar}{0}
\verb{virar}{}{}{}{}{}{Mudar a posição de pessoa ou coisa.}{vi.rar}{0}
\verb{virar}{}{}{}{}{}{Transformar"-se em alguma coisa; ficar.}{vi.rar}{0}
\verb{virar}{}{}{}{}{v.pron.}{Mudar de posição para ficar na direção de pessoa ou coisa; voltar"-se.}{vi.rar}{0}
\verb{virar}{}{}{}{}{}{Encontrar um jeito de sair de uma dificuldade; arranjar"-se.}{vi.rar}{0}
\verb{virar}{}{}{}{}{}{Ganhar a vida.}{vi.rar}{\verboinum{1}}
\verb{viravolta}{ó}{}{}{}{}{Mudança, reviravolta.}{vi.ra.vol.ta}{0}
\verb{viravolta}{ó}{}{}{}{s.f.}{Volta inteira.}{vi.ra.vol.ta}{0}
\verb{viravolta}{ó}{Fig.}{}{}{}{Rodeio, subterfúgio.}{vi.ra.vol.ta}{0}
\verb{virente}{}{}{}{}{adj.2g.}{Que verdeja; verdejante, viçoso.}{vi.ren.te}{0}
\verb{virente}{}{}{}{}{}{Que se desenvolve; florescente, próspero.}{vi.ren.te}{0}
\verb{virgem}{}{Relig.}{}{}{s.f.}{A mãe de Jesus Cristo; Virgem Maria.}{vir.gem}{0}
\verb{virgem}{}{Astron.}{}{}{}{Sexta constelação zodiacal.}{vir.gem}{0}
\verb{virgem}{}{Astrol.}{}{}{}{O signo do zodíaco referente a essa constelação.}{vir.gem}{0}
\verb{virgem}{}{}{"-ens}{}{s.f.}{Mulher que nunca teve relações sexuais.}{vir.gem}{0}
\verb{virgem}{}{}{"-ens}{}{adj.2g.}{Característico ou próprio de virgem; casto, inocente.}{vir.gem}{0}
\verb{virgem}{}{}{"-ens}{}{}{Intacto, intocado.}{vir.gem}{0}
\verb{virgem}{}{}{"-ens}{}{}{Que ainda não foi usado.}{vir.gem}{0}
\verb{virginal}{}{}{"-ais}{}{adj.2g.}{Relativo à virgem.}{vir.gi.nal}{0}
\verb{virgindade}{}{}{}{}{s.f.}{Estado de pessoa virgem.}{vir.gin.da.de}{0}
\verb{virgindade}{}{}{}{}{}{Estado daquilo que se encontra intacto.}{vir.gin.da.de}{0}
\verb{virgindade}{}{Fig.}{}{}{}{Pureza, castidade.}{vir.gin.da.de}{0}
\verb{virginiano}{}{Astrol.}{}{}{s.m.}{Indivíduo que nasceu sob o signo de virgem.}{vir.gi.ni.a.no}{0}
\verb{virginiano}{}{Astrol.}{}{}{adj.}{Relativo ou pertencente a esse signo.   }{vir.gi.ni.a.no}{0}
\verb{vírgula}{}{}{}{}{s.f.}{Sinal gráfico de pontuação, indicando uma pausa ligeira, usado para separar frases encadeadas entre si ou elementos dentro de uma frase.}{vír.gu.la}{0}
\verb{vírgula}{}{Pop.}{}{}{}{Expressão de negação ou restrição a algo que foi dito.}{vír.gu.la}{0}
\verb{virgular}{}{}{}{}{v.t.}{Pôr vírgulas.}{vir.gu.lar}{0}
\verb{virgular}{}{Fig.}{}{}{}{Entrecortar, interromper.}{vir.gu.lar}{\verboinum{1}}
\verb{viridente}{}{}{}{}{adj.2g.}{Virente.}{vi.ri.den.te}{0}
\verb{viril}{}{}{"-is}{}{adj.2g.}{Relativo a homem; próprio de homem; varonil.}{vi.ril}{0}
\verb{viril}{}{Por ext.}{"-is}{}{}{Que é dotado de coragem, energia, vigor; forte, destemido.}{vi.ril}{0}
\verb{virilha}{}{Anat.}{}{}{s.f.}{Região do corpo correspondente à junção da parte superior da coxa com o abdômen.}{vi.ri.lha}{0}
\verb{virilidade}{}{}{}{}{s.f.}{Qualidade de viril; masculinidade.}{vi.ri.li.da.de}{0}
\verb{virilidade}{}{}{}{}{}{Idade de homem entre a adolescência e a velhice.}{vi.ri.li.da.de}{0}
\verb{virilidade}{}{}{}{}{}{Energia, vigor.}{vi.ri.li.da.de}{0}
\verb{virilizar}{}{}{}{}{v.t.}{Tornar viril na natureza ou no aspecto.}{vi.ri.li.zar}{0}
\verb{virilizar}{}{}{}{}{}{Tornar forte; fortalecer, robustecer.}{vi.ri.li.zar}{\verboinum{1}}
\verb{virola}{ó}{}{}{}{s.f.}{Aro metálico que aperta ou reforça um objeto e às vezes serve para ornamento.}{vi.ro.la}{0}
\verb{virola}{ó}{Bras.}{}{}{}{Espécie de chibata; comumente de borracha.}{vi.ro.la}{0}
\verb{virologia}{}{}{}{}{s.f.}{Parte da biologia que estuda os vírus.}{vi.ro.lo.gi.a}{0}
\verb{virologista}{}{}{}{}{s.2g.}{Indivíduo que é especialista em virologia.}{vi.ro.lo.gis.ta}{0}
\verb{virose}{ó}{Med.}{}{}{s.f.}{Enfermidade produzida por um vírus.}{vi.ro.se}{0}
\verb{virótico}{}{}{}{}{adj.}{Relativo a vírus.}{vi.ró.ti.co}{0}
\verb{virtual}{}{}{"-ais}{}{adj.2g.}{Que existe apenas em potência ou como faculdade, não como realidade ou com efeito real.}{vir.tu.al}{0}
\verb{virtual}{}{}{"-ais}{}{}{Que é suscetível de realizar"-se, embora não se tenha realizado.}{vir.tu.al}{0}
\verb{virtual}{}{}{"-ais}{}{}{Que constitui uma simulação de algo, criada por meios eletrônicos.}{vir.tu.al}{0}
\verb{virtualidade}{}{}{}{}{s.f.}{Qualidade do que é virtual.}{vir.tu.a.li.da.de}{0}
\verb{virtude}{}{}{}{}{s.f.}{Disposição firme e constante para a prática do bem.}{vir.tu.de}{0}
\verb{virtude}{}{}{}{}{}{Boa qualidade moral; força, valor.}{vir.tu.de}{0}
\verb{virtude}{}{}{}{}{}{Ato virtuoso.}{vir.tu.de}{0}
\verb{virtude}{}{}{}{}{}{Castidade, pureza.}{vir.tu.de}{0}
\verb{virtude}{}{}{}{}{}{Qualidade própria para que se produzam certos efeitos; propriedade, característica.}{vir.tu.de}{0}
\verb{virtude}{}{}{}{}{}{Causa, motivo, razão.}{vir.tu.de}{0}
\verb{virtuose}{ô}{}{}{}{s.2g.}{Toda pessoa que domina em alto grau a técnica de uma arte.}{vir.tu.o.se}{0}
\verb{virtuose}{ô}{}{}{}{}{Músico de grande talento.}{vir.tu.o.se}{0}
\verb{virtuosismo}{}{}{}{}{s.m.}{Qualidade de virtuose.}{vir.tu.o.sis.mo}{0}
\verb{virtuoso}{ô}{}{"-osos ⟨ó⟩}{"-osa ⟨ó⟩}{adj.}{Que tem virtudes.}{vir.tu.o.so}{0}
\verb{virulência}{}{Med.}{}{}{s.f.}{Capacidade de um microrganismo de produzir doença, ou capacidade que um germe tem de infectar um organismo.}{vi.ru.lên.cia}{0}
\verb{virulento}{}{}{}{}{adj.}{Que tem vírus ou veneno.}{vi.ru.len.to}{0}
\verb{virulento}{}{}{}{}{}{Que tem a natureza do vírus.}{vi.ru.len.to}{0}
\verb{virulento}{}{}{}{}{}{Produzido por um vírus.}{vi.ru.len.to}{0}
\verb{virulento}{}{Fig.}{}{}{}{Que é rancoroso, violento.}{vi.ru.len.to}{0}
\verb{vírus}{}{Biol.}{}{}{s.m.}{Ser vivo minúsculo, menor que um micróbio, que se reproduz dentro das células e pode causar várias doenças nos homens, nos outros animais e nas plantas.}{ví.rus}{0}
\verb{vírus}{}{Informát.}{}{}{}{Programa feito para danificar o sistema ou as informações armazenadas em computadores.}{ví.rus}{0}
\verb{visada}{}{}{}{}{s.f.}{Ato ou efeito de visar.}{vi.sa.da}{0}
\verb{visagem}{}{}{"-ens}{}{s.f.}{A parte anterior da cabeça; cara, rosto.}{vi.sa.gem}{0}
\verb{visagem}{}{}{"-ens}{}{}{Trejeito ou expressão do rosto; careta.}{vi.sa.gem}{0}
\verb{visagem}{}{}{"-ens}{}{}{Assombração, fantasma.}{vi.sa.gem}{0}
\verb{visão}{}{}{"-ões}{}{s.f.}{Ato ou efeito de ver.}{vi.são}{0}
\verb{visão}{}{}{"-ões}{}{}{O sentido da vista; percepção do mundo exterior pelos órgãos da vista.}{vi.são}{0}
\verb{visão}{}{}{"-ões}{}{}{Imagem vã, que se acredita ver em sonhos, por medo, loucura, superstição etc.}{vi.são}{0}
\verb{visão}{}{}{"-ões}{}{}{Maneira de compreender, de perceber determinadas situações.}{vi.são}{0}
\verb{visar}{}{}{}{}{v.t.}{Dirigir o olhar para um ponto determinado.}{vi.sar}{0}
\verb{visar}{}{}{}{}{}{Apontar arma de fogo para um alvo.}{vi.sar}{0}
\verb{visar}{}{}{}{}{}{Pôr o sinal de visto.}{vi.sar}{0}
\verb{visar}{}{}{}{}{}{Ter por fim ou objetivo, ter em vista, ter em mira.}{vi.sar}{\verboinum{1}}
\verb{víscera}{}{Anat.}{}{}{s.f.}{Cada um dos órgãos alojados nas cavidades abdominal, torácica e craniana.}{vís.ce.ra}{0}
\verb{víscera}{}{Fig.}{}{}{}{A parte essencial de qualquer coisa; âmago.}{vís.ce.ra}{0}
\verb{visceral}{}{}{"-ais}{}{adj.2g.}{Relativo a víscera.}{vis.ce.ral}{0}
\verb{visceral}{}{Fig.}{"-ais}{}{}{Muito íntimo, profundo.}{vis.ce.ral}{0}
\verb{vísceras}{}{}{}{}{s.f.pl.}{Entranhas, intestinos.}{vís.ce.ras}{0}
\verb{vísceras}{}{Fig.}{}{}{}{A parte mais íntima de algo; âmago.}{vís.ce.ras}{0}
\verb{visco}{}{}{}{}{s.m.}{Planta parasita nativa do Hemisfério Norte.}{vis.co}{0}
\verb{visco}{}{}{}{}{}{Suco vegetal com que se fazem armadilhas para pássaros.}{vis.co}{0}
\verb{visco}{}{}{}{}{}{Isca, chamariz.}{vis.co}{0}
\verb{viscondado}{}{}{}{}{s.m.}{Título ou dignidade de visconde ou viscondessa.}{vis.con.da.do}{0}
\verb{viscondado}{}{}{}{}{}{Território sob jurisdição de visconde ou viscondessa.}{vis.con.da.do}{0}
\verb{visconde}{}{}{}{viscondessa ⟨ê⟩}{s.m.}{Título de nobreza imediatamente inferior ao de conde.}{vis.con.de}{0}
\verb{visconde}{}{}{}{viscondessa ⟨ê⟩}{}{Substituto do conde.}{vis.con.de}{0}
\verb{visconde}{}{}{}{viscondessa ⟨ê⟩}{}{Senhor de um viscondado.}{vis.con.de}{0}
\verb{viscondessa}{ê}{}{}{}{s.f.}{Mulher que tem título de viscondado.}{vis.con.des.sa}{0}
\verb{viscondessa}{ê}{}{}{}{}{Mulher ou viúva de visconde.}{vis.con.des.sa}{0}
\verb{viscose}{ó}{}{}{}{s.f.}{Celulose em forma de solução viscosa, usada na fabricação de tecidos e de celofane.}{vis.co.se}{0}
\verb{viscose}{ó}{}{}{}{}{O fio fabricado com essa substância.}{vis.co.se}{0}
\verb{viscose}{ó}{}{}{}{}{O tecido feito com esse fio.}{vis.co.se}{0}
\verb{viscosidade}{}{}{}{}{s.f.}{Qualidade de viscoso.}{vis.co.si.da.de}{0}
\verb{viscosidade}{}{Fís.}{}{}{}{Resistência que um fluido oferece ao movimento em seu interior; atrito interno de um fluido.}{vis.co.si.da.de}{0}
\verb{viscoso}{ô}{}{"-osos ⟨ó⟩}{"-osa ⟨ó⟩}{adj.}{Pegajoso como o visco.}{vis.co.so}{0}
\verb{viscoso}{ô}{}{"-osos ⟨ó⟩}{"-osa ⟨ó⟩}{}{Que tem muito visco.}{vis.co.so}{0}
\verb{viseira}{ê}{}{}{}{s.f.}{Parte anterior de capacete ou boné que protege o rosto da luz do sol.}{vi.sei.ra}{0}
\verb{viseira}{ê}{Por ext.}{}{}{}{Qualquer coisa que disfarça; máscara.}{vi.sei.ra}{0}
\verb{visgo}{}{}{}{}{s.m.}{Viscosidade.}{vis.go}{0}
\verb{visgo}{}{}{}{}{}{Arbusto nativo do norte da América do Sul com flores amarelas e que produz resina viscosa.}{vis.go}{0}
\verb{visguento}{}{}{}{}{adj.}{Viscoso.}{vis.guen.to}{0}
\verb{visibilidade}{}{}{}{}{s.f.}{Qualidade de visível.}{vi.si.bi.li.da.de}{0}
\verb{visigodo}{ô}{}{}{}{adj.}{Relativo aos visigodos, um dos povos que conquistaram partes do Império Romano a partir do século \textsc{iv}.}{vi.si.go.do}{0}
\verb{visigodo}{ô}{}{}{}{s.m.}{Indivíduo desse povo.}{vi.si.go.do}{0}
\verb{visionário}{}{}{}{}{adj.}{Relativo a visões.}{vi.si.o.ná.rio}{0}
\verb{visionário}{}{}{}{}{}{Que tem ideias grandiosas e extravagantes.}{vi.si.o.ná.rio}{0}
\verb{visionário}{}{}{}{}{s.m.}{Indivíduo utopista, sonhador, devaneador.}{vi.si.o.ná.rio}{0}
\verb{visita}{}{}{}{}{s.f.}{Ato ou efeito de visitar.}{vi.si.ta}{0}
\verb{visita}{}{}{}{}{}{Ato de comparecer à casa de alguém, por razões afetivas ou de caráter social.}{vi.si.ta}{0}
\verb{visita}{}{}{}{}{}{A pessoa que comparece à casa de outrem.}{vi.si.ta}{0}
\verb{visita}{}{}{}{}{}{Inspeção, vistoria.}{vi.si.ta}{0}
\verb{visitação}{}{}{"-ões}{}{s.f.}{Ato ou efeito de visitar; visita.}{vi.si.ta.ção}{0}
\verb{visitação}{}{}{"-ões}{}{}{Certa ordem religiosa feminina fundada por São Francisco de Sales no século \textsc{xvii}. (Usa"-se com inicial maiúscula nesta acepção.)}{vi.si.ta.ção}{0}
\verb{visitante}{}{}{}{}{adj.2g.}{Que visita.}{vi.si.tan.te}{0}
\verb{visitante}{}{}{}{}{s.2g.}{Indivíduo que comparece à casa de outrem; visita.}{vi.si.tan.te}{0}
\verb{visitante}{}{}{}{}{}{Indivíduo que comparece a uma exposição, monumento, localidade para conhecê"-la.}{vi.si.tan.te}{0}
\verb{visitar}{}{}{}{}{v.t.}{Comparecer à casa de alguém por cortesia, afeição ou obrigação social.}{vi.si.tar}{0}
\verb{visitar}{}{}{}{}{}{Ir a regiões, museus, exposições para conhecê"-los e apreciá"-los.}{vi.si.tar}{0}
\verb{visitar}{}{}{}{}{}{Vistoriar, inspecionar.}{vi.si.tar}{\verboinum{1}}
\verb{visível}{}{}{"-eis}{}{adj.2g.}{Que pode ser visto.}{vi.sí.vel}{0}
\verb{visível}{}{}{"-eis}{}{}{Que se destaca; muito evidente; patente, óbvio, manifesto.}{vi.sí.vel}{0}
\verb{vislumbrar}{}{}{}{}{v.t.}{Iluminar levemente.}{vis.lum.brar}{0}
\verb{vislumbrar}{}{}{}{}{}{Começar a aparecer; surgir, despontar.}{vis.lum.brar}{0}
\verb{vislumbrar}{}{}{}{}{}{Enxergar parcialmente; entrever.}{vis.lum.brar}{0}
\verb{vislumbrar}{}{Fig.}{}{}{}{Compreender de maneira imprecisa.}{vis.lum.brar}{\verboinum{1}}
\verb{vislumbre}{}{}{}{}{s.m.}{Luz tênue.}{vis.lum.bre}{0}
\verb{vislumbre}{}{}{}{}{}{Aparência vaga; vestígio.}{vis.lum.bre}{0}
\verb{vislumbre}{}{}{}{}{}{Semelhança.}{vis.lum.bre}{0}
\verb{vislumbre}{}{}{}{}{}{Ideia imprecisa; noção.}{vis.lum.bre}{0}
\verb{viso}{}{}{}{}{s.m.}{Fisionomia, aparência, aspecto.}{vi.so}{0}
\verb{viso}{}{}{}{}{}{Vislumbre, vestígio.}{vi.so}{0}
\verb{viso}{}{}{}{}{}{Recordação vaga.}{vi.so}{0}
\verb{viso}{}{}{}{}{}{Pequeno monte; colina.}{vi.so}{0}
\verb{visom}{}{Zool.}{"-ons}{}{s.m.}{Mamífero de hábitos semiaquáticos e pelagem macia.}{vi.som}{0}
\verb{visom}{}{}{"-ons}{}{}{A pele desse animal.}{vi.som}{0}
\verb{visom}{}{}{"-ons}{}{}{O casaco feito com essa pele.}{vi.som}{0}
\verb{visor}{ô}{}{}{}{adj.}{Que permite ou ajuda a ver.}{vi.sor}{0}
\verb{visor}{ô}{}{}{}{s.m.}{Dispositivo de aparelho eletrônico que exibe informações sobre seu funcionamento; \textit{display}, monitor, mostrador.}{vi.sor}{0}
\verb{visor}{ô}{}{}{}{}{Dispositivo de aparelhos ópticos por onde o usuário olha.  }{vi.sor}{0}
\verb{víspora}{}{Bras.}{}{}{s.f.}{Jogo com cartões numerados e peças numeradas que são sorteadas para que os participantes completem seus cartões; loto, bingo.}{vís.po.ra}{0}
\verb{vista}{}{}{}{}{s.f.}{Ato ou efeito de ver.}{vis.ta}{0}
\verb{vista}{}{}{}{}{}{Capacidade de perceber a cor, a forma e o tamanho das coisas pelos órgãos da visão; visão.}{vis.ta}{0}
\verb{vista}{}{}{}{}{}{Cada um dos olhos.}{vis.ta}{0}
\verb{vista}{}{}{}{}{}{Aquilo que se vê; cenário, cena, panorama.}{vis.ta}{0}
\verb{visto}{}{}{}{}{adj.}{Que se viu.}{vis.to}{0}
\verb{visto}{}{}{}{}{}{Considerado, reputado.}{vis.to}{0}
\verb{visto}{}{}{}{}{}{Debatido, estudado.}{vis.to}{0}
\verb{visto}{}{}{}{}{s.m.}{Assinatura ou declaração dada por alguém, geralmente autoridade, que examinou o documento em questão.}{vis.to}{0}
\verb{visto}{}{}{}{}{prep.}{Em razão de.}{vis.to}{0}
\verb{vistoria}{}{}{}{}{s.f.}{Inspeção de caráter oficial em coisas ou locais envolvidos em investigação ou litígio.}{vis.to.ri.a}{0}
\verb{vistoria}{}{Por ext.}{}{}{}{Inspeção ou exame feito por qualquer pessoa em qualquer lugar.}{vis.to.ri.a}{0}
\verb{vistoriar}{}{}{}{}{v.t.}{Fazer vistoria.}{vis.to.ri.ar}{0}
\verb{vistoriar}{}{Por ext.}{}{}{}{Examinar algo para verificar as condições em que se encontra; inspecionar.}{vis.to.ri.ar}{\verboinum{1}}
\verb{vistoso}{ô}{}{"-osos ⟨ó⟩}{"-osa ⟨ó⟩}{adj.}{Agradável de se ver.}{vis.to.so}{0}
\verb{vistoso}{ô}{}{"-osos ⟨ó⟩}{"-osa ⟨ó⟩}{}{Que chama a atenção; chamativo, berrante.}{vis.to.so}{0}
\verb{visual}{}{}{"-ais}{}{adj.2g.}{Relativo à visão.}{vi.su.al}{0}
\verb{visual}{}{Bras.}{"-ais}{}{s.m.}{Aparência exterior; aspecto.}{vi.su.al}{0}
\verb{visualização}{}{}{"-ões}{}{s.f.}{Ato ou efeito de visualizar.}{vi.su.a.li.za.ção}{0}
\verb{visualização}{}{}{"-ões}{}{}{Ato, processo ou efeito de tornar algo perceptível à vista.}{vi.su.a.li.za.ção}{0}
\verb{visualização}{}{}{"-ões}{}{}{Ato ou habilidade de formar na mente imagens visuais de coisas que não estão à vista.}{vi.su.a.li.za.ção}{0}
\verb{visualizar}{}{}{}{}{v.t.}{Tornar algo visível.}{vi.su.a.li.zar}{0}
\verb{visualizar}{}{}{}{}{}{Formar imagem mental de algo que não está visível.}{vi.su.a.li.zar}{0}
\verb{visualizar}{}{}{}{}{}{Tornar algo visível pela aplicação de determinado recurso.}{vi.su.a.li.zar}{\verboinum{1}}
\verb{vital}{}{}{"-ais}{}{adj.2g.}{Relativo a vida.}{vi.tal}{0}
\verb{vital}{}{}{"-ais}{}{}{Essencial à manutenção da vida.}{vi.tal}{0}
\verb{vital}{}{}{"-ais}{}{}{Que desempenha função essencial em um organismo.}{vi.tal}{0}
\verb{vital}{}{Fig.}{"-ais}{}{}{Que é de importância primordial; essencial.}{vi.tal}{0}
\verb{vitaliciedade}{}{}{}{}{s.f.}{Qualidade de vitalício.}{vi.ta.li.ci.e.da.de}{0}
\verb{vitalício}{}{}{}{}{adj.}{Que dura por toda a vida.}{vi.ta.lí.cio}{0}
\verb{vitalício}{}{}{}{}{}{Diz"-se de funcionário que tem garantia de não ser afastado do cargo.}{vi.ta.lí.cio}{0}
\verb{vitalidade}{}{}{}{}{s.f.}{Vigor físico ou mental.}{vi.ta.li.da.de}{0}
\verb{vitalidade}{}{}{}{}{}{Entusiasmo, ânimo.}{vi.ta.li.da.de}{0}
\verb{vitalidade}{}{}{}{}{}{Exuberância.}{vi.ta.li.da.de}{0}
\verb{vitalidade}{}{}{}{}{}{Qualidade de vital.}{vi.ta.li.da.de}{0}
\verb{vitalismo}{}{}{}{}{s.m.}{Doutrina dos séculos \textsc{xviii} e \textsc{xix} que defendia a ideia de um impulso vital de natureza não material, que age paralelamente aos fenômenos físicos e químicos conhecidos.}{vi.ta.lis.mo}{0}
\verb{vitalizar}{}{}{}{}{v.t.}{Dar vida; revigorar.}{vi.ta.li.zar}{0}
\verb{vitalizar}{}{}{}{}{}{Restaurar o ânimo, a vitalidade.}{vi.ta.li.zar}{\verboinum{1}}
\verb{vitamina}{}{}{}{}{s.f.}{Nome de diversas substâncias orgânicas que desempenham funções importantes no metabolismo e que são encontradas em diversos alimentos.}{vi.ta.mi.na}{0}
\verb{vitamina}{}{}{}{}{}{Leite batido com frutas e cereais.}{vi.ta.mi.na}{0}
\verb{vitaminar}{}{}{}{}{v.t.}{Enriquecer um alimento com vitaminas.}{vi.ta.mi.nar}{\verboinum{1}}
\verb{vitando}{}{}{}{}{adj.}{Que se deve evitar; muito mau, abominável, execrável.}{vi.tan.do}{0}
\verb{vitela}{é}{}{}{}{s.f.}{Vaca nova com menos de um ano de idade.}{vi.te.la}{0}
\verb{vitela}{é}{Por ext.}{}{}{}{A carne de vaca ou boi com menos de um ano de idade.}{vi.te.la}{0}
\verb{vitela}{é}{Por ext.}{}{}{}{Prato preparado com essa carne.}{vi.te.la}{0}
\verb{vitela}{é}{}{}{}{}{A pele daquele animal, usada na fabricação de calçados e em outras aplicações.}{vi.te.la}{0}
\verb{vitelino}{}{}{}{}{adj.}{Relativo a gema de ovo.}{vi.te.li.no}{0}
\verb{vitelino}{}{}{}{}{}{Que tem a cor amarela da gema de ovo.}{vi.te.li.no}{0}
\verb{vitelo}{é}{}{}{}{s.m.}{Boi novo com menos de um ano de idade.}{vi.te.lo}{0}
\verb{viticultor}{ô}{}{}{}{adj.}{Relativo à viticultura.}{vi.ti.cul.tor}{0}
\verb{viticultor}{ô}{}{}{}{s.m.}{Indivíduo dedicado à viticultura.}{vi.ti.cul.tor}{0}
\verb{viticultura}{}{}{}{}{s.f.}{Atividade e técnica de cultivo de vinhas ou de produção de vinho.}{vi.ti.cul.tu.ra}{0}
\verb{vitiligem}{}{Med.}{"-ens}{}{s.f.}{Vitiligo.}{vi.ti.li.gem}{0}
\verb{vitiligo}{}{Med.}{}{}{s.m.}{Alteração patológica da pele caracterizada pela perda localizada da pigmentação.}{vi.ti.li.go}{0}
\verb{vítima}{}{}{}{}{s.f.}{Indivíduo que sofre algum infortúnio em acidente, guerra, catástrofe, epidemia.}{ví.ti.ma}{0}
\verb{vítima}{}{}{}{}{}{Indivíduo arbitrariamente condenado a morte, tortura ou flagelo.}{ví.ti.ma}{0}
\verb{vítima}{}{}{}{}{}{Indivíduo sacrificado às paixões alheias.}{ví.ti.ma}{0}
\verb{vítima}{}{}{}{}{}{Homem ou animal imolado em sacrifício aos deuses.}{ví.ti.ma}{0}
\verb{vítima}{}{}{}{}{}{Tudo que sofre qualquer dano.}{ví.ti.ma}{0}
\verb{vitimar}{}{}{}{}{v.t.}{Tornar (alguém) vítima; sacrificar, prejudicar.}{vi.ti.mar}{0}
\verb{vitimar}{}{}{}{}{}{Causar a morte; matar.}{vi.ti.mar}{\verboinum{1}}
\verb{vitivinicultor}{ô}{}{}{}{s.m.}{Indivíduo que se dedica à vitivinicultura.}{vi.ti.vi.ni.cul.tor}{0}
\verb{vitivinicultura}{}{}{}{}{s.f.}{Atividade e técnica de cultivo de vinhas e fabricação de vinho.}{vi.ti.vi.ni.cul.tu.ra}{0}
\verb{vitória}{}{}{}{}{s.f.}{Ato ou efeito de vencer um inimigo em batalha ou adversário em competição; triunfo.}{vi.tó.ria}{0}
\verb{vitória}{}{}{}{}{}{Bom êxito, sucesso, vantagem.}{vi.tó.ria}{0}
\verb{vitoriar}{}{}{}{}{v.t.}{Manifestar aprovação; aplaudir, ovacionar.}{vi.to.ri.ar}{0}
\verb{vitoriar}{}{}{}{}{}{Saudar com júbilo e entusiasmo.}{vi.to.ri.ar}{\verboinum{6}}
\verb{vitória"-régia}{}{Bot.}{vitórias"-régias}{}{s.f.}{Erva aquática de grande porte, presa ao fundo por rizoma, com flores brancas ou rosadas que se abrem à noite e folhas redondas de até 1,8 m de diâmetro.}{vi.tó.ri.a"-ré.gia}{0}
\verb{vitoriense}{}{}{}{}{adj.2g.}{Relativo a Vitória, capital do Espírito Santo.}{vi.to.ri.en.se}{0}
\verb{vitoriense}{}{}{}{}{s.2g.}{Indivíduo natural ou habitante dessa cidade.}{vi.to.ri.en.se}{0}
\verb{vitorioso}{ô}{}{"-osos ⟨ó⟩}{"-osa ⟨ó⟩}{adj.}{Que obteve vitória; triunfante.}{vi.to.ri.o.so}{0}
\verb{vitral}{}{}{"-ais}{}{s.m.}{Vidraça feita com pedaços de vidro, geralmente coloridos, que formam desenhos.}{vi.tral}{0}
\verb{vítreo}{}{}{}{}{adj.}{Relativo a vidro.}{ví.treo}{0}
\verb{vítreo}{}{}{}{}{}{Feito de vidro.}{ví.treo}{0}
\verb{vítreo}{}{}{}{}{}{Transparente, límpido.}{ví.treo}{0}
\verb{vitrificar}{}{}{}{}{v.t.}{Converter em vidro.}{vi.tri.fi.car}{0}
\verb{vitrificar}{}{}{}{}{}{Dar aparência de vidro.}{vi.tri.fi.car}{\verboinum{2}}
\verb{vitrina}{}{}{}{}{s.f.}{Vidraça ou caixa com tampa envidraçada para expor objetos à venda.}{vi.tri.na}{0}
\verb{vitrine}{}{}{}{}{}{Var. de \textit{vitrina}.}{vi.tri.ne}{0}
\verb{vitrinista}{}{Bras.}{}{}{s.2g.}{Indivíduo especializado em projetar e decorar vitrinas.}{vi.tri.nis.ta}{0}
\verb{vitríolo}{}{Quím.}{}{}{s.m.}{Designação comum a vários sulfatos, especialmente o ácido sulfúrico.}{vi.trí.o.lo}{0}
\verb{vitrola}{ó}{}{}{}{s.f.}{Aparelho que reproduz sons gravados em discos de vinil, dotado de agulha que lê os sulcos do disco e, geralmente, amplificador e alto"-falante; toca"-discos.}{vi.tro.la}{0}
\verb{vitualhas}{}{}{}{}{s.f.pl.}{Gêneros alimentícios; mantimentos. (Usa"-se também no singular.)}{vi.tu.a.lhas}{0}
\verb{vituperação}{}{}{"-ões}{}{s.f.}{Ato ou efeito de vituperar; vitupério.}{vi.tu.pe.ra.ção}{0}
\verb{vituperar}{}{}{}{}{v.t.}{Insultar, afrontar, injuriar.}{vi.tu.pe.rar}{0}
\verb{vituperar}{}{}{}{}{}{Repreender duramente; censurar.}{vi.tu.pe.rar}{0}
\verb{vituperar}{}{}{}{}{}{Desprezar.}{vi.tu.pe.rar}{\verboinum{1}}
\verb{vitupério}{}{}{}{}{s.m.}{Insulto, injúria.}{vi.tu.pé.rio}{0}
\verb{vitupério}{}{}{}{}{}{Ato infame, vergonhoso ou criminoso.}{vi.tu.pé.rio}{0}
\verb{viúva}{}{}{}{}{s.f.}{Mulher cujo marido morreu e que não voltou a casar"-se.}{vi.ú.va}{0}
\verb{viúva"-negra}{ê}{Zool.}{viúvas"-negras ⟨ê⟩}{}{s.f.}{Aranha de cor negra com mancha vermelha no abdômen, que devora o macho após a cópula, e é considerada a mais peçonhenta das aranhas.}{vi.ú.va"-ne.gra}{0}
\verb{viuvar}{}{}{}{}{v.t.}{Tornar viúvo.}{vi.u.var}{0}
\verb{viuvar}{}{}{}{}{v.i.}{Ficar viúvo.}{vi.u.var}{\verboinum{8}}
\verb{viuvez}{ê}{}{}{}{s.f.}{Situação de quem é viúvo.}{vi.u.vez}{0}
\verb{viuvez}{ê}{Fig.}{}{}{}{Desamparo, solidão.}{vi.u.vez}{0}
\verb{viúvo}{}{}{}{}{s.m.}{Homem cuja mulher morreu e que não voltou a casar"-se.}{vi.ú.vo}{0}
\verb{viva}{}{}{}{}{s.m.}{Exclamação de congratulação, aprovação, alegria.}{vi.va}{0}
\verb{viva}{}{}{}{}{interj.}{Expressão que denota entusiasmo, alegria, desejo de vida longa.}{vi.va}{0}
\verb{vivace}{}{Mús.}{}{}{adj.2g.}{Que tem vivacidade, vitalidade.}{vi.va.ce}{0}
\verb{vivacidade}{}{}{}{}{s.f.}{Qualidade de vivaz.}{vi.va.ci.da.de}{0}
\verb{vivacidade}{}{}{}{}{}{Expressividade, entusiasmo.}{vi.va.ci.da.de}{0}
\verb{vivacidade}{}{}{}{}{}{Brilho, vitalidade.}{vi.va.ci.da.de}{0}
\verb{vivacidade}{}{}{}{}{}{Esperteza, agilidade, inteligência.}{vi.va.ci.da.de}{0}
\verb{vivaldino}{}{Pop.}{}{}{s.m.}{Indivíduo muito esperto, malandro; espertalhão.}{vi.val.di.no}{0}
\verb{vivalma}{}{}{}{}{s.f.}{Alma, alguma pessoa, alguém.}{vi.val.ma}{0}
\verb{vivaz}{}{}{}{}{adj.2g.}{Ativo, dinâmico, ligeiro.}{vi.vaz}{0}
\verb{vivaz}{}{}{}{}{}{Brilhante, expressivo.}{vi.vaz}{0}
\verb{vivaz}{}{}{}{}{}{Que vive por muito tempo.}{vi.vaz}{0}
\verb{vivedoiro}{ô}{}{}{}{}{Var. de \textit{vivedouro}.}{vi.ve.doi.ro}{0}
\verb{vivedouro}{ô}{}{}{}{adj.}{Que vive ou pode viver muito; vivaz, duradouro.}{vi.ve.dou.ro}{0}
\verb{viveiro}{ê}{}{}{}{s.m.}{Lugar apropriado onde se criam e reproduzem animais.}{vi.vei.ro}{0}
\verb{viveiro}{ê}{}{}{}{}{Canteiro onde se semeiam plantas que posteriormente serão transplantadas; sementeira.}{vi.vei.ro}{0}
\verb{vivência}{}{}{}{}{s.f.}{Experiência de vida; aquilo que se viveu.}{vi.vên.cia}{0}
\verb{vivência}{}{}{}{}{}{Conhecimento adquirido ao longo da vida; experiência.}{vi.vên.cia}{0}
\verb{vivência}{}{}{}{}{}{O fato de ter vida; existência.}{vi.vên.cia}{0}
\verb{vivência}{}{Bras.}{}{}{}{Hábitos de vida.}{vi.vên.cia}{0}
\verb{vivenciar}{}{}{}{}{v.t.}{Viver uma situação particular, especialmente de maneira intensa.}{vi.ven.ci.ar}{\verboinum{6}}
\verb{vivenda}{}{}{}{}{s.f.}{Lugar (geralmente imponente) em que se vive; morada.}{vi.ven.da}{0}
\verb{vivenda}{}{}{}{}{}{Subsistência, passadio.}{vi.ven.da}{0}
\verb{vivenda}{}{}{}{}{}{Modo de vida; comportamento.}{vi.ven.da}{0}
\verb{vivente}{}{}{}{}{adj.2g.}{Que vive.}{vi.ven.te}{0}
\verb{viver}{ê}{}{}{}{v.i.}{Ter vida; estar com vida; existir.}{vi.ver}{0}
\verb{viver}{ê}{}{}{}{}{Manter"-se; sustentar"-se.}{vi.ver}{0}
\verb{viver}{ê}{}{}{}{}{Habitar, residir, morar.}{vi.ver}{0}
\verb{viver}{ê}{}{}{}{v.t.}{Passar por alguma experiência; vivenciar. (\textit{Minha prima viveu momentos terríveis nas mãos do sequestrador.})}{vi.ver}{\verboinum{12}}
\verb{víveres}{}{}{}{}{s.m.pl.}{Gêneros alimentícios; mantimentos.}{ví.ve.res}{0}
\verb{viveza}{ê}{}{}{}{s.f.}{Vivacidade.}{vi.ve.za}{0}
\verb{vivido}{}{}{}{}{adj.}{Que viveu muito.}{vi.vi.do}{0}
\verb{vívido}{}{}{}{}{adj.}{Que tem vivacidade; animado, intenso.}{ví.vi.do}{0}
\verb{vívido}{}{}{}{}{}{Que tem muita luz; brilhante, luzente.}{ví.vi.do}{0}
\verb{vivido}{}{}{}{}{}{Que tem muita experiência de vida.}{vi.vi.do}{0}
\verb{vívido}{}{}{}{}{}{Que tem cores vivas; colorido.}{ví.vi.do}{0}
\verb{vivificante}{}{}{}{}{adj.2g.}{Que vivifica, reanima.}{vi.vi.fi.can.te}{0}
\verb{vivificar}{}{}{}{}{v.t.}{Dar vida; animar.}{vi.vi.fi.car}{0}
\verb{vivificar}{}{}{}{}{}{Restituir a vida; reanimar, reviver.}{vi.vi.fi.car}{0}
\verb{vivificar}{}{}{}{}{}{Dar movimento; ativar, estimular.}{vi.vi.fi.car}{\verboinum{2}}
\verb{vivíparo}{}{}{}{}{adj.}{Diz"-se de organismo animal que forma embriões dentro do organismo materno.}{vi.ví.pa.ro}{0}
\verb{vivisseção}{}{}{}{}{}{Var. de \textit{vivissecção}.}{vi.vis.se.ção}{0}
\verb{vivissecção}{}{}{"-ões}{}{s.f.}{Operação feita em animal vivo para realizar estudo ou experimento.}{vi.vis.sec.ção}{0}
\verb{vivo}{}{}{}{}{adj.}{Que vive, que não está morto.}{vi.vo}{0}
\verb{vivo}{}{}{}{}{}{Diz"-se da língua atualmente falada.}{vi.vo}{0}
\verb{vivo}{}{}{}{}{}{Animado, ativo, intenso.}{vi.vo}{0}
\verb{vizinhança}{}{}{}{}{s.f.}{Qualidade de estar próximo de algo ou alguém; proximidade.}{vi.zi.nhan.ça}{0}
\verb{vizinhança}{}{}{}{}{}{As pessoas ou famílias vizinhas.}{vi.zi.nhan.ça}{0}
\verb{vizinhança}{}{}{}{}{}{Cercanias, arrabaldes.}{vi.zi.nhan.ça}{0}
\verb{vizinho}{}{}{}{}{adj.}{Que está próximo ou perto; limítrofe, contíguo.}{vi.zi.nho}{0}
\verb{vizinho}{}{}{}{}{}{Que apresenta alguma relação de semelhança; análogo, semelhante.}{vi.zi.nho}{0}
\verb{vizinho}{}{}{}{}{}{Que mora ou se localiza perto.}{vi.zi.nho}{0}
\verb{vizir}{}{}{}{}{s.m.}{Ministro de um soberano de reino muçulmano.}{vi.zir}{0}
\verb{vó}{}{}{}{}{s.f.}{Mãe do pai ou da mãe em relação aos filhos destes; avó.}{vó}{0}
\verb{vô}{}{}{}{}{s.m.}{Pai do pai ou da mãe em relação aos filhos destes; avô.}{vô}{0}
\verb{voador}{ô}{}{}{}{adj.}{Que voa.}{vo.a.dor}{0}
\verb{voador}{ô}{Fig.}{}{}{}{Muito rápido; veloz.}{vo.a.dor}{0}
\verb{voador}{ô}{}{}{}{s.m.}{Acrobata que salta de um trapézio para outro, a certa distância.}{vo.a.dor}{0}
\verb{voar}{}{}{}{}{v.i.}{Sustentar"-se ou mover"-se no ar por meio de asas ou de máquinas.}{vo.ar}{0}
\verb{voar}{}{}{}{}{}{Ir de avião; fazer viagem aérea.}{vo.ar}{0}
\verb{voar}{}{}{}{}{}{Passar, deslocar"-se velozmente.}{vo.ar}{\verboinum{7}}
\verb{vocabular}{}{}{}{}{adj.2g.}{Relativo a vocábulo.}{vo.ca.bu.lar}{0}
\verb{vocabulário}{}{}{}{}{s.m.}{Conjunto de vocábulos de uma língua; léxico.}{vo.ca.bu.lá.rio}{0}
\verb{vocabulário}{}{}{}{}{}{Conjunto de termos característicos de determinado campo de conhecimento; glossário.}{vo.ca.bu.lá.rio}{0}
\verb{vocabulário}{}{}{}{}{}{Conjunto de vocábulos dominados por um falante.}{vo.ca.bu.lá.rio}{0}
\verb{vocábulo}{}{}{}{}{s.m.}{Palavra que faz parte de um vocabulário; lexema.}{vo.cá.bu.lo}{0}
\verb{vocábulo}{}{Gram.}{}{}{}{Palavra considerada apenas quanto à estrutura fonética, independentemente do seu significado.}{vo.cá.bu.lo}{0}
\verb{vocação}{}{}{"-ões}{}{s.f.}{Tendência ou inclinação para qualquer estado, profissão, ofício.}{vo.ca.ção}{0}
\verb{vocação}{}{Fig.}{"-ões}{}{}{Aptidão natural; pendor, talento.}{vo.ca.ção}{0}
\verb{vocal}{}{}{"-ais}{}{adj.2g.}{Relativo a voz ou aos órgãos da voz.}{vo.cal}{0}
\verb{vocal}{}{}{"-ais}{}{}{Que se expressa por meio da voz; oral.}{vo.cal}{0}
\verb{vocálico}{}{}{}{}{adj.}{Relativo a vogal.}{vo.cá.li.co}{0}
\verb{vocálico}{}{}{}{}{}{Que é formado por vogal ou vogais.}{vo.cá.li.co}{0}
\verb{vocálico}{}{}{}{}{}{Diz"-se do som ou sequência de sons que constitui a parte mais sonora de uma sílaba.}{vo.cá.li.co}{0}
\verb{vocalista}{}{}{}{}{s.2g.}{Pessoa que canta; cantor, cantora.}{vo.ca.lis.ta}{0}
\verb{vocalista}{}{}{}{}{}{Em um conjunto musical, o músico que canta. }{vo.ca.lis.ta}{0}
\verb{vocalização}{}{}{"-ões}{}{s.f.}{Ato ou efeito de vocalizar; emissão de sons da voz.}{vo.ca.li.za.ção}{0}
\verb{vocalização}{}{Gram.}{"-ões}{}{}{Passagem de um elemento consonantal para um elemento vocálico.}{vo.ca.li.za.ção}{0}
\verb{vocalização}{}{Mús.}{"-ões}{}{}{Conjunto de exercícios para trabalhar e modular a voz sobre uma vogal.}{vo.ca.li.za.ção}{0}
\verb{vocalizar}{}{Gram.}{}{}{v.t.}{Transformar um elemento consonantal em elemento vocálico.}{vo.ca.li.zar}{0}
\verb{vocalizar}{}{Mús.}{}{}{v.i.}{Exercitar e modular a voz sobre uma vogal.}{vo.ca.li.zar}{\verboinum{1}}
\verb{vocativo}{}{}{}{}{adj.}{Que chama, que serve para chamar.}{vo.ca.ti.vo}{0}
\verb{vocativo}{}{Gram.}{}{}{}{Um dos casos sintáticos morfologicamente marcados, de algumas línguas, como o latim.}{vo.ca.ti.vo}{0}
\verb{você}{}{}{}{}{pron.}{Forma de tratamento dirigida a quem se fala ou se escreve, utilizada em substituição a \textit{tu} na maior parte do Brasil.}{vo.cê}{0}
\verb{você}{}{}{}{}{}{Pessoa não identificada; qualquer pessoa; alguém. (\textit{Ei, você, de blusa rosa, o que está fazendo em cima dessa escada?})}{vo.cê}{0}
\verb{vociferação}{}{}{"-ões}{}{s.f.}{Ato ou efeito de vociferar; gritaria, berreiro.}{vo.ci.fe.ra.ção}{0}
\verb{vociferação}{}{}{"-ões}{}{}{Imprompério, insulto, censura.}{vo.ci.fe.ra.ção}{0}
\verb{vociferar}{}{}{}{}{v.t.}{Falar com raiva; dirigir censuras ou reclamações.}{vo.ci.fe.rar}{\verboinum{1}}
\verb{vociferar}{}{}{}{}{v.i.}{Falar em altos brados, em voz alta ou clamorosa; gritar, clamar.}{vo.ci.fe.rar}{0}
\verb{vodca}{ó}{}{}{}{s.f.}{Aguardente de cereais, originária da Rússia.}{vod.ca}{0}
\verb{vodu}{}{}{}{}{s.m.}{Culto dos negros antilhanos, de origem africana, semelhante ao candomblé.}{vo.du}{0}
\verb{voejar}{}{}{}{}{v.i.}{Voar intermitentemente, com voos curtos e sem direção certa; adejar, esvoaçar.}{vo.e.jar}{0}
\verb{voejar}{}{Por ext.}{}{}{}{Tocar levemente; perpassar.}{vo.e.jar}{\verboinum{1}}
\verb{voga}{ó}{}{}{}{}{Uso atual; moda.}{vo.ga}{0}
\verb{voga}{ó}{}{}{}{s.f.}{Ato ou efeito de vogar, navegar.}{vo.ga}{0}
\verb{voga}{ó}{}{}{}{}{Ritmo, estilo, cadência da remada.}{vo.ga}{0}
\verb{vogal}{}{Gram.}{"-ais}{}{s.f.}{Fonema que se produz mediante o livre escapamento do ar pela boca.}{vo.gal}{0}
\verb{vogal}{}{}{"-ais}{}{}{Cada uma das letras que representam os fonemas vocálicos da língua.}{vo.gal}{0}
\verb{vogal}{}{}{"-ais}{}{}{Pessoa que tem voto numa assembleia, comissão ou tribunal.}{vo.gal}{0}
\verb{vogar}{}{}{}{}{v.i.}{Ser impelido sobre as águas por remos ou velas; navegar.}{vo.gar}{0}
\verb{vogar}{}{}{}{}{}{Flutuar, boiar.}{vo.gar}{0}
\verb{vogar}{}{}{}{}{}{Estar em moda ou em uso.}{vo.gar}{\verboinum{5}}
\verb{volante}{}{}{}{}{adj.2g.}{Que voa ou que pode voar.}{vo.lan.te}{0}
\verb{volante}{}{}{}{}{}{Que muda; mudável, inconstante, volúvel.}{vo.lan.te}{0}
\verb{volante}{}{}{}{}{s.m.}{Roda de mão para dirigir o automóvel; direção.}{vo.lan.te}{0}
\verb{volante}{}{Bras.}{}{}{}{Impresso no qual se marcam apostas de jogos.}{vo.lan.te}{0}
\verb{volante}{}{}{}{}{}{Folha com propaganda, distribuída na rua.}{vo.lan.te}{0}
\verb{volante}{}{Esport.}{}{}{s.2g.}{No futebol, jogador de função defensiva no meio do campo (camisa 5).}{vo.lan.te}{0}
\verb{volátil}{}{Fís. e Quím.}{"-eis}{}{adj.2g.}{Que pode se reduzir a gás ou vapor à pressão ou temperatura ambientes.}{vo.lá.til}{0}
\verb{volátil}{}{}{"-eis}{}{}{Que pode voar ou voa; voador.}{vo.lá.til}{0}
\verb{volátil}{}{Fig.}{"-eis}{}{}{Que não é permanente; volúvel, inconstante.}{vo.lá.til}{0}
\verb{volatilidade}{}{}{}{}{s.f.}{Qualidade do que é volátil; facilidade de evaporar.}{vo.la.ti.li.da.de}{0}
\verb{volatilizar}{}{}{}{}{v.t.}{Reduzir a gás ou vapor; evaporar, vaporizar.}{vo.la.ti.li.zar}{0}
\verb{volatilizar}{}{}{}{}{}{Fazer desaparecer; dissipar, desfazer.}{vo.la.ti.li.zar}{\verboinum{1}}
\verb{vôlei}{}{Esport.}{}{}{s.m.}{Jogo disputado entre duas equipes de seis jogadores cada uma, que rebatem uma bola por sobre uma rede, usando as mãos ou os punhos.}{vô.lei}{0}
\verb{voleibol}{ó}{Esport.}{}{}{s.m.}{Vôlei.}{vo.lei.bol}{0}
\verb{voleio}{ê}{}{}{}{s.m.}{Em um jogo, devolução da bola ao oponente antes que ela toque no chão.}{vo.lei.o}{0}
\verb{volição}{}{}{"-ões}{}{s.f.}{Ato de manifestar a vontade; arbítrio.}{vo.li.ção}{0}
\verb{volitar}{}{}{}{}{v.i.}{Dirigir"-se voando; esvoaçar, adejar.}{vo.li.tar}{\verboinum{1}}
\verb{volitivo}{}{}{}{}{adj.}{Relativo a volição; que provém da vontade.}{vo.li.ti.vo}{0}
\verb{volt}{ô}{Fís.}{}{}{s.m.}{No Sistema Internacional, unidade de medida de diferença de potencial elétrico.}{volt}{0}
%\verb{}{ó}{}{}{}{s.f.}{Ato ou efeito de voltar; regresso, retorno.}{}{0}
%\verb{}{ó}{}{}{}{}{Ato ou efeito de virar; giro, volteio.}{}{0}
%\verb{}{ó}{}{}{}{}{Mudança de direção em um caminho; curva, sinuosidade.}{}{0}
%\verb{}{ó}{}{}{}{}{Alteração, reviravolta, revés.}{}{0}
%\verb{}{ó}{}{}{}{}{Caminhada curta; passeio.}{}{0}
\verb{voltagem}{}{Fís.}{"-ens}{}{s.f.}{Medida da diferença de potencial entre dois pontos de um circuito elétrico.}{vol.ta.gem}{0}
\verb{voltar}{}{}{}{}{v.i.}{Refazer o caminho para o ponto do qual saiu; retornar, regressar. }{vol.tar}{0}
\verb{voltar}{}{}{}{}{}{Ir ou vir pela segunda vez. (\textit{Minha cunhada voltou para França para concluir o curso de pós"-graduação.})}{vol.tar}{0}
\verb{voltar}{}{}{}{}{}{Mudar de posição ou de direção; virar, volver.}{vol.tar}{0}
\verb{voltar}{}{}{}{}{}{Recomeçar, retomar.}{vol.tar}{0}
\verb{voltar}{}{}{}{}{v.pron.}{Mostrar"-se contrário a; revoltar"-se, hostilizar.}{vol.tar}{\verboinum{1}}
\verb{voltarete}{ê}{}{}{}{s.m.}{Antigo jogo de cartas com três parceiros.}{vol.ta.re.te}{0}
\verb{voltear}{}{}{}{}{v.t.}{Andar em torno; contornar.}{vol.te.ar}{0}
\verb{voltear}{}{}{}{}{}{Fazer girar; dar voltas.}{vol.te.ar}{0}
\verb{voltear}{}{}{}{}{}{Fazer volteios na corda bamba ou de arame.}{vol.te.ar}{\verboinum{4}}
\verb{volteio}{ê}{}{}{}{s.m.}{Exercício de equilibrista na corda bamba ou no arame.}{vol.tei.o}{0}
\verb{volteio}{ê}{}{}{}{}{Rodopio ou dança ao redor de algo.}{vol.tei.o}{0}
\verb{voltímetro}{}{}{}{}{s.m.}{Aparelho utilizado para medir a voltagem de um circuito elétrico.}{vol.tí.me.tro}{0}
\verb{volubilidade}{}{}{}{}{s.f.}{Falta de estabilidade; inconstância, mutabilidade.}{vo.lu.bi.li.da.de}{0}
\verb{volubilidade}{}{}{}{}{}{Qualidade daquilo que se move facilmente; mobilidade.}{vo.lu.bi.li.da.de}{0}
%\verb{}{}{}{}{}{s.m.}{Pacote, embrulho.}{}{0}
%\verb{}{}{}{}{}{}{Intensidade do ou da voz.}{}{0}
%\verb{}{}{}{}{}{}{Tamanho, corpulência.}{}{0}
\verb{volumétrico}{}{}{}{}{adj.}{Relativo a volumetria e a determinação dos volumes.}{vo.lu.mé.tri.co}{0}
\verb{volumoso}{ô}{}{"-osos ⟨ó⟩}{"-osa ⟨ó⟩}{adj.}{Que tem grande volume, grandes proporções.}{vo.lu.mo.so}{0}
\verb{volumoso}{ô}{}{"-osos ⟨ó⟩}{"-osa ⟨ó⟩}{}{Que se constitui de muitos volumes ou tomos.}{vo.lu.mo.so}{0}
\verb{voluntariado}{}{}{}{}{s.m.}{Conjunto de voluntários.}{vo.lun.ta.ri.a.do}{0}
\verb{voluntariado}{}{}{}{}{}{Serviço prestado por voluntários.}{vo.lun.ta.ri.a.do}{0}
\verb{voluntariedade}{}{}{}{}{s.f.}{Qualidade de alguém que age por vontade própria; espontaneidade.}{vo.lun.ta.ri.e.da.de}{0}
\verb{voluntariedade}{}{}{}{}{}{Capricho, obstinação, teima.}{vo.lun.ta.ri.e.da.de}{0}
\verb{voluntário}{}{}{}{}{adj.}{Que é feito por vontade, sem ser forçado ou coagido; espontâneo.}{vo.lun.tá.rio}{0}
\verb{voluntário}{}{}{}{}{}{Que se oferece para uma tarefa a que estava obrigado.}{vo.lun.tá.rio}{0}
\verb{voluntário}{}{}{}{}{}{Que age apenas segundo sua própria vontade; voluntarioso.}{vo.lun.tá.rio}{0}
\verb{voluntarioso}{ô}{}{"-osos ⟨ó⟩}{"-osa ⟨ó⟩}{adj.}{Que age segundo a própria vontade, sem considerar a vontade do outro; caprichoso, obstinado.}{vo.lun.ta.ri.o.so}{0}
\verb{volúpia}{}{}{}{}{s.f.}{Intenso prazer dos sentidos; sensualidade.}{vo.lú.pia}{0}
\verb{volúpia}{}{}{}{}{}{Grande prazer sexual; luxúria.}{vo.lú.pia}{0}
\verb{voluptuosidade}{}{}{}{}{s.f.}{Inclinação exagerada aos prazeres sexuais; sensualidade.  }{vo.lup.tu.o.si.da.de}{0}
\verb{voluptuoso}{ô}{}{"-osos ⟨ó⟩}{"-osa ⟨ó⟩}{adj.}{Que é dado ou se entrega aos prazeres do sentido, principalmente aos sexuais; libidinoso, lascivo, sensual.}{vo.lup.tu.o.so}{0}
\verb{voluptuoso}{ô}{}{"-osos ⟨ó⟩}{"-osa ⟨ó⟩}{}{Que conduz à excitação ou a algum prazer sensual.}{vo.lup.tu.o.so}{0}
\verb{voluta}{}{}{}{}{s.f.}{Motivo decorativo enrolado em espiral.}{vo.lu.ta}{0}
\verb{voluta}{}{Mús.}{}{}{}{Espiral na parte superior dos instrumentos de arco.}{vo.lu.ta}{0}
\verb{volutear}{}{}{}{}{v.t.}{Andar em roda; rodopiar, voltear.}{vo.lu.te.ar}{\verboinum{4}}
\verb{volúvel}{}{}{"-eis}{}{adj.2g.}{Que muda de opinião com facilidade; inconstante, instável, mutável.}{vo.lú.vel}{0}
\verb{volúvel}{}{}{"-eis}{}{}{Que muda facilmente de direção.}{vo.lú.vel}{0}
\verb{volver}{ê}{}{}{}{v.t.}{Virar para um lado ou para outro.}{vol.ver}{0}
\verb{volver}{ê}{}{}{}{}{Mexer, revolver, pôr em movimento, agitar.}{vol.ver}{\verboinum{12}}
\verb{volvo}{ô}{Med.}{}{}{s.m.}{Obstrução do tráfego intestinal por torção de uma das alças do intestino; obstrução intestinal; vólvulo.}{vol.vo}{0}
\verb{vólvulo}{}{Med.}{}{}{s.m.}{Volvo.}{vól.vu.lo}{0}
\verb{vômer}{}{Anat.}{}{}{s.m.}{Pequeno osso achatado e delgado que forma a parte inferior e posterior do septo das fossas nasais.}{vô.mer}{0}
\verb{vômico}{}{}{}{}{adj.}{Que faz vomitar; emético.}{vô.mi.co}{0}
\verb{vomitar}{}{}{}{}{v.t.}{Expelir pela boca substância que estava no estômago.}{vo.mi.tar}{0}
\verb{vomitar}{}{}{}{}{}{Lançar violentamente para fora de si; jorrar.}{vo.mi.tar}{0}
\verb{vomitar}{}{}{}{}{}{Falar com violência ou com intenção injuriosa.}{vo.mi.tar}{\verboinum{1}}
\verb{vomitivo}{}{}{}{}{adj.}{Que faz vomitar; emético, vômico.}{vo.mi.ti.vo}{0}
\verb{vômito}{}{}{}{}{s.m.}{Ato ou efeito de vomitar; emissão de conteúdo do estômago pela boca.}{vô.mi.to}{0}
\verb{vômito}{}{}{}{}{}{Aquilo que se vomitou.}{vô.mi.to}{0}
\verb{vomitório}{}{}{}{}{adj.}{Diz"-se do medicamento que provoca vômito; emético, vômico, vomitivo.}{vo.mi.tó.rio}{0}
\verb{vôngole}{}{Zool.}{}{}{s.m.}{Molusco bivalve, marinho, que vive enterrado na areia, comum no litoral brasileiro, e de largo uso na alimentação;  marisco.}{vôn.go.le}{0}
\verb{vontade}{}{}{}{}{s.f.}{Aquilo que faz querer ou deixar de querer fazer algo; desejo, aspiração.}{von.ta.de}{0}
\verb{vontade}{}{}{}{}{}{Energia, ânimo, coragem e perseverança no querer ou realizar.}{von.ta.de}{0}
\verb{vontade}{}{}{}{}{}{Capricho, arbítrio. (\textit{É uma criança cheia de vontades.})}{von.ta.de}{0}
\verb{voo}{ô}{}{}{}{s.m.}{Deslocamento no ar, próprio das aves, insetos e aeronaves, sem nenhum contato com o solo.}{vo.o}{0}
\verb{voo}{ô}{}{}{}{}{Percurso que uma ave ou aeronave cobre de uma vez, voando.}{vo.o}{0}
\verb{voo}{ô}{}{}{}{}{Viagem feita por via aérea.}{vo.o}{0}
\verb{voo}{ô}{}{}{}{}{Qualquer movimento ou deslocamento muito rápido ou leve pelo ar.}{vo.o}{0}
\verb{voo}{ô}{Fig.}{}{}{}{Elevação do pensamento; êxtase, arrebatamento, arroubo.}{vo.o}{0}
\verb{voracidade}{}{}{}{}{s.f.}{Qualidade de voraz; grande apetite; sofreguidão.}{vo.ra.ci.da.de}{0}
\verb{voragem}{}{}{}{}{s.f.}{Aquilo que é capaz de tragar, devorar com violência.}{vo.ra.gem}{0}
\verb{voragem}{}{}{}{}{}{Redemoinho, turbilhão, sorvedouro.}{vo.ra.gem}{0}
\verb{voragem}{}{Fig.}{}{}{}{Aquilo que corrói, consome ou destrói com violência.}{vo.ra.gem}{0}
\verb{voraz}{}{}{}{}{adj.2g.}{Que devora ou consome com avidez. }{vo.raz}{0}
\verb{voraz}{}{}{}{}{}{Que é capaz de corroer, arruinar; destruidor.}{vo.raz}{0}
\verb{voraz}{}{}{}{}{}{Que tem muita cobiça; ávido, ambicioso.}{vo.raz}{0}
\verb{vórtice}{}{}{}{}{s.m.}{Movimento forte e giratório; turbilhão, redemoinho, voragem.}{vór.ti.ce}{0}
\verb{vos}{}{}{}{}{pron.}{Pronome pessoal de segunda pessoa do plural do caso oblíquo, empregado como complemento do verbo transitivo.}{vos}{0}
\verb{vós}{}{}{}{}{pron.}{Pronome pessoal da segunda pessoa do plural do caso reto, empregado como sujeito.}{vós}{0}
\verb{vosear}{}{}{}{}{v.t.}{Tratar alguém como \textit{vós}.}{vo.se.ar}{\verboinum{4}}
\verb{vosmecê}{}{Desus.}{}{}{pron.}{Forma contraída de (\textit{Vossa Mercê}).}{vos.me.cê}{0}
\verb{vossemecê}{}{Desus.}{}{}{pron.}{Forma contraída de (\textit{Vossa Mercê}).}{vos.se.me.cê}{0}
\verb{vosso}{ó}{}{}{}{pron.}{Que vos pertence ou diz respeito. (\textit{Deveis reclamar todos os vossos direitos.})}{vos.so}{0}
\verb{votação}{}{}{"-ões}{}{s.f.}{Ato, processo ou efeito de votar; eleição, escrutínio.}{vo.ta.ção}{0}
\verb{votação}{}{}{"-ões}{}{}{O conjunto dos votos dados ou recolhidos em uma eleição ou assembleia.}{vo.ta.ção}{0}
\verb{votante}{}{}{}{}{adj.2g.}{Que vota ou que tem direito de votar.}{vo.tan.te}{0}
\verb{votar}{}{}{}{}{v.t.}{Aprovar, decidir por meio de voto.}{vo.tar}{0}
\verb{votar}{}{}{}{}{}{Fazer voto, prometer formal e solenemente.}{vo.tar}{0}
\verb{votar}{}{}{}{}{}{Dedicar, oferecer à divindade; consagrar.}{vo.tar}{0}
\verb{votar}{}{}{}{}{}{Empregar, destinar, empenhar.}{vo.tar}{\verboinum{1}}
\verb{votivo}{}{}{}{}{adj.}{Relativo a voto.}{vo.ti.vo}{0}
\verb{votivo}{}{}{}{}{}{Prometido ou ofertado em cumprimento a voto.}{vo.ti.vo}{0}
\verb{voto}{ó}{}{}{}{s.m.}{Manifestação da vontade ou da opinião  num ato eleitoral ou numa assembleia; sufrágio}{vo.to}{0}
\verb{voto}{ó}{}{}{}{}{Cédula que se usou para votar numa eleição.}{vo.to}{0}
\verb{voto}{ó}{}{}{}{}{Em uma ordem religiosa, obrigação a que um indivíduo se compromete voluntariamente.}{vo.to}{0}
\verb{voto}{ó}{}{}{}{}{Promessa solene feita à divindade.}{vo.to}{0}
\verb{vovó}{}{}{}{}{s.f.}{Forma familiar de \textit{avó}.}{vo.vó}{0}
\verb{vovô}{}{}{}{}{s.m.}{Forma familiar de \textit{avô}.}{vo.vô}{0}
\verb{voyeur}{}{}{}{}{s.m.}{Indivíduo que experimenta prazer sexual ao ver objetos associados à sexualidade ou o próprio ato sexual praticado por outros.}{\textit{voyeur}}{0}
\verb{voyeurismo}{}{}{}{}{s.m.}{Excitação sexual que consiste em olhar um corpo nu ou a cópula entre terceiros.}{voy.eu.ris.mo}{0}
\verb{voz}{ó}{}{}{}{s.f.}{Som emitido pela vibração das cordas vocais na laringe.}{voz}{0}
\verb{voz}{ó}{}{}{}{}{Faculdade de falar; linguagem.}{voz}{0}
\verb{voz}{ó}{}{}{}{}{Direito de falar; opinião, voto.}{voz}{0}
\verb{voz}{ó}{Gram.}{}{}{}{Forma em que o verbo indica a ação como ativa, passiva ou reflexiva.}{voz}{0}
\verb{vozear}{}{}{}{}{v.t.}{Dizer aos gritos; berrar.}{vo.ze.ar}{0}
\verb{vozear}{}{}{}{}{v.i.}{Falar muito alto e de maneira insolente.}{vo.ze.ar}{\verboinum{4}}
\verb{vozearia}{}{}{}{}{s.f.}{Som de grande quantidade de vozes; gritaria, clamor, vozerio, vozeria.}{vo.ze.a.ri.a}{0}
\verb{vozeio}{ê}{}{}{}{s.m.}{Ato ou efeito de vozear; vozearia, vozerio.}{vo.zei.o}{0}
\verb{vozeirão}{}{}{"-ões}{}{s.m.}{Voz muito grave e forte.}{vo.zei.rão}{0}
\verb{vozeirão}{}{}{"-ões}{}{}{Indivíduo dotado desse timbre de voz.}{vo.zei.rão}{0}
\verb{vozeria}{}{}{}{}{s.f.}{Vozearia.}{vo.ze.ri.a}{0}
\verb{vozerio}{}{}{}{}{s.m.}{Vozearia.}{vo.ze.ri.o}{0}
\verb{vulcânico}{}{}{}{}{}{Relativo a vulcão.}{vul.câ.ni.co}{0}
\verb{vulcânico}{}{Fig.}{}{}{adj.}{Semelhante a vulcão; ardente, impetuoso.}{vul.câ.ni.co}{0}
\verb{vulcanização}{}{}{"-ões}{}{s.f.}{Tratamento da borracha natural com átomos de enxofre, para deixá"-la mais  elástica, insolúvel e resistente a temperaturas altas e baixas.}{vul.ca.ni.za.ção}{0}
\verb{vulcanizar}{}{}{}{}{v.t.}{Submeter a borracha ao processo de vulcanização.}{vul.ca.ni.zar}{0}
\verb{vulcanizar}{}{}{}{}{}{Tornar ardente; abrasar, calcinar.}{vul.ca.ni.zar}{0}
\verb{vulcanizar}{}{Fig.}{}{}{}{Inflamar, entusiasmar, exaltar.}{vul.ca.ni.zar}{\verboinum{1}}
\verb{vulcanologia}{}{}{}{}{s.f.}{Ramo da geologia que estuda os vulcões e os fenômenos vulcânicos.}{vul.ca.no.lo.gi.a}{0}
\verb{vulcão}{}{Geol.}{"-ões}{}{s.m.}{Abertura na superfície do planeta, através da qual o magma e seus gases associados, oriundos de camadas profundas, são lançados à superfície.}{vul.cão}{0}
\verb{vulcão}{}{Fig.}{"-ões}{}{}{Perigo iminente de desordem social.}{vul.cão}{0}
\verb{vulcão}{}{Fig.}{"-ões}{}{}{Pessoa de natureza explosiva, impetuosa, violenta.}{vul.cão}{0}
\verb{vulgacho}{}{}{}{}{s.m.}{Conjunto de indivíduos pertencentes à camada inferior de uma sociedade; ralé, plebe, populacho.}{vul.ga.cho}{0}
\verb{vulgar}{}{}{}{}{adj.2g.}{Relativo ao vulgo, à plebe; popular.}{vul.gar}{0}
\verb{vulgar}{}{}{}{}{}{Que não se destaca, não se distingue; banal, corriqueiro, ordinário.}{vul.gar}{0}
\verb{vulgar}{}{}{}{}{}{De qualidade inferior; chulo, reles, baixo.}{vul.gar}{0}
\verb{vulgar}{}{}{}{}{}{Que não tem caráter nenhum de nobreza ou distinção.}{vul.gar}{0}
\verb{vulgaridade}{}{}{}{}{s.f.}{Qualidade ou caráter do que é vulgar.}{vul.ga.ri.da.de}{0}
\verb{vulgaridade}{}{}{}{}{}{Ação, coisa ou dito vulgar; vulgarismo. }{vul.ga.ri.da.de}{0}
\verb{vulgarismo}{}{}{}{}{s.m.}{Palavra, dito ou procedimento vulgar; vulgaridade.}{vul.ga.ris.mo}{0}
\verb{vulgarização}{}{}{"-ões}{}{s.f.}{Ato ou efeito de vulgarizar.}{vul.ga.ri.za.ção}{0}
\verb{vulgarização}{}{}{"-ões}{}{}{Popularização, banalização.}{vul.ga.ri.za.ção}{0}
\verb{vulgarizador}{ô}{}{}{}{adj.}{Que vulgariza, populariza.}{vul.ga.ri.za.dor}{0}
\verb{vulgarizar}{}{}{}{}{v.t.}{Tornar vulgar, comum; popularizar.}{vul.ga.ri.zar}{0}
\verb{vulgarizar}{}{}{}{}{}{Tornar muito conhecido; divulgar, propalar.}{vul.ga.ri.zar}{0}
\verb{vulgarizar}{}{}{}{}{}{Fazer perder a dignidade; tornar desprezível, reles.}{vul.ga.ri.zar}{\verboinum{1}}
\verb{vulgata}{}{}{}{}{s.f.}{Tradução da Bíblia para o latim feita no século \textsc{iv} e considerada a versão oficial da Igreja Romana pelo Concílio de Trento.}{vul.ga.ta}{0}
\verb{vulgo}{}{}{}{}{s.m.}{A classe popular; a plebe, o povo.}{vul.go}{0}
\verb{vulgo}{}{}{}{}{adv.}{Vulgarmente, popularmente.}{vul.go}{0}
\verb{vulnerar}{}{}{}{}{v.t.}{Causar ferimento; machucar.}{vul.ne.rar}{0}
\verb{vulnerar}{}{}{}{}{}{Provocar ofensa; magoar, melindrar.}{vul.ne.rar}{\verboinum{1}}
\verb{vulnerável}{}{}{"-eis}{}{adj.2g.}{Que pode ser ferido, ofendido ou tocado.}{vul.ne.rá.vel}{0}
\verb{vulpino}{}{}{}{}{adj.}{Relativo a raposa.}{vul.pi.no}{0}
\verb{vulpino}{}{Fig.}{}{}{}{Astuto, manhoso, pérfido.}{vul.pi.no}{0}
\verb{vulto}{}{}{}{}{s.m.}{A parte anterior da cabeça; rosto, semblante, face.}{vul.to}{0}
\verb{vulto}{}{}{}{}{}{Figura ou imagem pouco nítida.}{vul.to}{0}
\verb{vulto}{}{}{}{}{}{Volume, massa, grandeza.}{vul.to}{0}
\verb{vulto}{}{Fig.}{}{}{}{Indivíduo notável, de grande importância.}{vul.to}{0}
\verb{vultoso}{ô}{}{"-osos ⟨ó⟩}{"-osa ⟨ó⟩}{adj.}{De grande vulto; de volume considerável; volumoso.}{vul.to.so}{0}
\verb{vultoso}{ô}{}{"-osos ⟨ó⟩}{"-osa ⟨ó⟩}{}{De grande importância; notável.}{vul.to.so}{0}
\verb{vultuosidade}{}{Med.}{}{}{s.f.}{Condição do rosto quando as faces e os lábios estão vermelhos e inchados; congestão.}{vul.tu.o.si.da.de}{0}
\verb{vultuoso}{ô}{}{"-osos ⟨ó⟩}{"-osa ⟨ó⟩}{adj.}{Acometido de vultuosidade; inchado, vermelho.}{vul.tu.o.so}{0}
\verb{vulturino}{}{}{}{}{adj.}{Relativo a abutre.}{vul.tu.ri.no}{0}
\verb{vulva}{}{Anat.}{}{}{s.f.}{Conjunto das partes externas dos órgãos genitais femininos dos mamíferos, que, na mulher, compreende os grandes e os pequenos lábios, o clitóris, o orifício da uretra e a abertura da vagina.}{vul.va}{0}
\verb{vurmo}{}{}{}{}{s.m.}{O pus das chagas e úlceras.}{vur.mo}{0}
\verb{vurmoso}{ô}{}{"-osos ⟨ó⟩}{"-osa ⟨ó⟩}{adj.}{Diz"-se do ferimento que apresenta vurmo, pus.}{vur.mo.so}{0}
