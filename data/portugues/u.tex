\verb{u}{}{}{}{}{s.m.}{Vigésima primeira letra e quinta vogal do alfabeto português.}{u}{0}
\verb{U}{}{Quím.}{}{}{}{Símb. do \textit{urânio}.}{U}{0}
\verb{uacari}{}{Zool.}{}{}{s.m.}{Cascudo.}{u.a.ca.ri}{0}
\verb{uacari}{}{Zool.}{}{}{}{Nome comum dado a vários macacos, arborícolas, de cauda curta e cabeça pelada ou quase desprovida de pelos, encontrados no noroeste da Amazônia.}{u.a.ca.ri}{0}
\verb{uai}{}{Bras.}{}{}{interj.}{Expressão que denota surpresa, espanto, susto.}{u.ai}{0}
\verb{uai}{}{Bras.}{}{}{}{Expressão usada após uma afirmação para enfatizá"-la ou para reforçar o caráter evidente dessa afirmação.}{u.ai}{0}
\verb{ubá}{}{Bras.}{}{}{s.m.}{Pequena embarcação indígena de fundo chato fabricada com um só tronco.}{u.bá}{0}
\verb{uberdade}{}{}{}{}{s.f.}{Qualidade de úbere; fartura.}{u.ber.da.de}{0}
\verb{uberdade}{}{}{}{}{}{Opulência; riqueza; fertilidade.}{u.ber.da.de}{0}
\verb{úbere}{}{}{}{}{adj.}{Farto, abundante.}{ú.be.re}{0}
\verb{úbere}{}{}{}{}{s.m.}{Teta de animal, especialmente a vaca.}{ú.be.re}{0}
%\verb{}{}{}{}{}{}{}{}{0}
%\verb{}{}{}{}{}{}{}{}{0}
\verb{ubiquidade}{}{}{}{}{s.m.}{Qualidade de ubíquo.}{u.bi.qui.da.de}{0}
\verb{ubíquo}{}{}{}{}{adj.}{Que está ou existe ao mesmo tempo em todos os lugares; onipresente.}{u.bí.quo}{0}
\verb{ubre}{}{}{}{}{}{Var. de \textit{úbere}.}{u.bre}{0}
\verb{uca}{}{Bras.}{}{}{s.f.}{Aguardente de cana; cachaça.}{u.ca}{0}
\verb{uçá}{}{Bras.}{}{}{s.m.}{Caranguejo de cor verde azulado e pernas avermelhadas encontrado nos mangues.}{u.çá}{0}
\verb{ucharia}{}{}{}{}{s.f.}{Depósito de gêneros alimentícios; despensa.}{u.cha.ri.a}{0}
\verb{ucraniano}{}{}{}{}{adj.}{Relativo à República da Ucrânia (ou Ucraína).}{u.cra.ni.a.no}{0}
\verb{ucraniano}{}{}{}{}{s.m.}{Indivíduo natural ou habitante desse país.    }{u.cra.ni.a.no}{0}
\verb{ucraniano}{}{}{}{}{}{O idioma da Ucrânia.}{u.cra.ni.a.no}{0}
\verb{ué}{}{}{}{}{interj.}{Expressão que denota surpresa, espanto, admiração.}{u.é}{0}
\verb{uê}{}{}{}{}{interj.}{Ué.}{u.ê}{0}
\verb{ufa}{}{}{}{}{interj.}{Expressão que denota cansaço, alívio, desabafo ou satisfação pelo fim de uma tarefa cansativa ou um aborrecimento. (\textit{Ufa! Até que enfim terminanos.})}{u.fa}{0}
\verb{ufanar}{}{}{}{}{v.t.}{Tornar vaidoso; envaidecer.}{u.fa.nar}{0}
\verb{ufanar}{}{}{}{}{}{Alegrar muito.}{u.fa.nar}{0}
\verb{ufanar}{}{}{}{}{v.pron.}{Mostrar"-se presunçoso; vangloriar"-se.}{u.fa.nar}{0}
\verb{ufanar}{}{}{}{}{}{Alegrar"-se muito.}{u.fa.nar}{\verboinum{1}}
\verb{ufania}{}{}{}{}{s.f.}{Qualidade de ufano.}{u.fa.ni.a}{0}
\verb{ufania}{}{}{}{}{}{Vaidade exagerada; soberba, imodéstia.}{u.fa.ni.a}{0}
\verb{ufanismo}{}{}{}{}{s.m.}{Qualidade de quem se orgulha exageradamente de algo.}{u.fa.nis.mo}{0}
\verb{ufanismo}{}{Bras.}{}{}{}{Orgulho, geralmente muito intenso, pelo país em que se nasceu; patriotismo exacerbado.}{u.fa.nis.mo}{0}
\verb{ufanista}{}{}{}{}{adj.2g.}{Que é dotado de ufanismo.}{u.fa.nis.ta}{0}
\verb{ufano}{}{}{}{}{adj.}{Que se orgulha de algo.}{u.fa.no}{0}
\verb{ufano}{}{}{}{}{}{Arrogante, jactancioso, imodesto.}{u.fa.no}{0}
\verb{ufano}{}{}{}{}{}{Vaidoso.}{u.fa.no}{0}
\verb{ufo}{}{}{}{}{s.m.}{Objeto voador não identificado (sigla do inglês \textit{unidentified flying object}); óvni.}{u.fo}{0}
\verb{ufologia}{}{}{}{}{s.f.}{Estudo dos óvnis; ovniologia.}{u.fo.lo.gi.a}{0}
\verb{ufologista}{}{}{}{}{s.2g.}{Especialista em ufologia.}{u.fo.lo.gis.ta}{0}
\verb{ugandense}{}{}{}{}{adj.2g.}{Relativo a Uganda.}{u.gan.den.se}{0}
\verb{ugandense}{}{}{}{}{s.2g.}{Indivíduo natural ou habitante desse país. }{u.gan.den.se}{0}
\verb{uh}{}{}{}{}{interj.}{Expressão que denota espanto, desdém, repugnância ou é usada para dar susto em outrem.}{uh}{0}
\verb{ui}{}{}{}{}{interj.}{Expressão que denota dor, surpresa ou repugnância.}{ui}{0}
\verb{uiara}{}{Bras.}{}{}{s.f.}{Ente mitológico amazônico meio mulher, meio peixe, semelhante a uma sereia.}{ui.a.ra}{0}
\verb{uiara}{}{Bras.}{}{}{}{Boto de corpo alongado, focinho longo e cor cinza ou rosada, encontrado em rios da Amazônia; boto"-cor"-de"-rosa.}{ui.a.ra}{0}
\verb{uirapuru}{}{Zool.}{}{}{s.m.}{Ave florestal, típica do Brasil, colorida e que emite um canto bastante melodioso ao amanhecer.}{ui.ra.pu.ru}{0}
\verb{uísque}{}{}{}{}{s.m.}{Bebida destilada feita a partir de grãos fermentados de centeio, milho, aveia ou cevada.}{u.ís.que}{0}
\verb{uisqueria}{}{Bras.}{}{}{s.f.}{Estabelecimento onde se vendem uísque e outras bebidas alcoólicas.}{u.is.que.ri.a}{0}
\verb{uivar}{}{}{}{}{v.i.}{Dar uivos; ulular.}{ui.var}{0}
\verb{uivar}{}{}{}{}{}{Emitir ruído semelhante a uivo.}{ui.var}{0}
\verb{uivar}{}{}{}{}{}{Gritar, berrar.}{ui.var}{0}
\verb{uivar}{}{}{}{}{}{Vociferar, bramir, esbravejar.}{ui.var}{\verboinum{1}}
\verb{uivo}{}{}{}{}{s.m.}{Voz aguda e lamentosa dos cães e lobos.}{ui.vo}{0}
\verb{uivo}{}{Fig.}{}{}{}{Ato de gritar ou vociferar.}{ui.vo}{0}
\verb{úlcera}{}{Med.}{}{}{s.f.}{Lesão na pele ou em mucosa, geralmente acompanhada de inflamação; ferida.}{úl.ce.ra}{0}
\verb{ulceração}{}{Med.}{"-ões}{}{s.f.}{Úlcera.}{ul.ce.ra.ção}{0}
\verb{ulcerar}{}{}{}{}{v.t.}{Causar ou produzir úlcera.}{ul.ce.rar}{0}
\verb{ulcerar}{}{}{}{}{}{Transformar"-se em úlcera.}{ul.ce.rar}{0}
\verb{ulcerar}{}{Fig.}{}{}{}{Angustiar, magoar.}{ul.ce.rar}{\verboinum{1}}
\verb{ulceroso}{ô}{}{"-osos ⟨ó⟩}{"-osa ⟨ó⟩}{adj.}{Que tem úlceras.}{ul.ce.ro.so}{0}
\verb{ulceroso}{ô}{}{"-osos ⟨ó⟩}{"-osa ⟨ó⟩}{}{Semelhante a ou com características de úlcera.}{ul.ce.ro.so}{0}
\verb{ulemá}{}{}{}{}{s.m.}{Entre os muçulmanos, indivíduo conhecedor da religião e das leis; teólogo, sábio.}{u.le.má}{0}
\verb{ulna}{}{Anat.}{}{}{s.f.}{Osso interno do antebraço; cúbito.}{ul.na}{0}
\verb{ulna}{}{}{}{}{}{Antiga medida de comprimento.}{ul.na}{0}
\verb{ulterior}{ô}{}{}{}{adj.}{Que está, ocorre ou vem depois.}{ul.te.ri.or}{0}
\verb{ultimação}{}{}{"-ões}{}{s.f.}{Ato ou efeito de ultimar; conclusão, término.}{ul.ti.ma.ção}{0}
\verb{ultimação}{}{Fig.}{"-ões}{}{}{Aperfeiçoamento, acabamento.}{ul.ti.ma.ção}{0}
\verb{ultimamente}{}{}{}{}{adv.}{Nos últimos tempos; recentemente.}{ul.ti.ma.men.te}{0}
\verb{ultimamente}{}{}{}{}{}{Por último.}{ul.ti.ma.men.te}{0}
\verb{ultimar}{}{}{}{}{v.t.}{Pôr fim a; concluir, completar.}{ul.ti.mar}{0}
\verb{ultimar}{}{}{}{}{}{Realizar de maneira definitiva (um negócio); fechar.}{ul.ti.mar}{\verboinum{1}}
\verb{últimas}{}{}{}{}{s.f.pl.}{Ponto extremo, momento decisivo ou final.}{úl.ti.mas}{0}
\verb{últimas}{}{}{}{}{}{Momento terminal da vida; agonia.}{úl.ti.mas}{0}
\verb{ultimato}{}{}{}{}{s.m.}{Exigência final e irrevogável após a qual o não cumprimento implica consequências.}{ul.ti.ma.to}{0}
\verb{ultimato}{}{}{}{}{}{Último aviso.}{ul.ti.ma.to}{0}
\verb{ultimátum}{}{}{}{}{}{Var. de \textit{ultimato}.}{ul.ti.má.tum}{0}
\verb{último}{}{}{}{}{adj. e s.m.  }{Diz"-se de ou o elemento final de uma sequência.}{úl.ti.mo}{0}
\verb{último}{}{}{}{}{}{Diz"-se de ou o evento mais recente em uma sucessão.}{úl.ti.mo}{0}
\verb{último}{}{}{}{}{adj.}{Ínfimo, desprezível, inferior.}{úl.ti.mo}{0}
\verb{ultra}{}{}{}{}{s.2g.}{Partidário de ideias muito avançadas ou extremas; radical.}{ul.tra}{0}
\verb{ultracorreção}{}{Gram.}{"-ões}{}{s.f.}{Ver \textit{hipercorreção}.}{ul.tra.cor.re.ção}{0}
\verb{ultrajante}{}{}{}{}{adj.2g.}{Que ultraja, humilha; humilhante.}{ul.tra.jan.te}{0}
\verb{ultrajar}{}{}{}{}{v.t.}{Ofender a dignidade; insultar, injuriar.}{ul.tra.jar}{0}
\verb{ultrajar}{}{}{}{}{}{Difamar.}{ul.tra.jar}{\verboinum{1}}
\verb{ultraje}{}{}{}{}{s.m.}{Ato ou efeito de ultrajar; ofensa, insulto.}{ul.tra.je}{0}
\verb{ultraje}{}{}{}{}{}{Difamação.}{ul.tra.je}{0}
\verb{ultraleve}{é}{}{}{}{s.m.}{Avião pequeno, leve e simplificado, que geralmente comporta somente um tripulante, equipado com motor a gasolina.}{ul.tra.le.ve}{0}
\verb{ultramar}{}{}{}{}{s.m.}{Região situada além do mar, do outro lado do oceano.}{ul.tra.mar}{0}
\verb{ultramar}{}{}{}{}{}{Certa tinta azul de tonalidade forte.}{ul.tra.mar}{0}
\verb{ultramar}{}{}{}{}{}{A cor dessa tinta.}{ul.tra.mar}{0}
\verb{ultramarino}{}{}{}{}{adj.}{Relativo a ultramar ou situado no ultramar.}{ul.tra.ma.ri.no}{0}
\verb{ultramicroscópio}{}{}{}{}{s.m.}{Tipo de microscópio que utiliza iluminação especial a qual permite a observação de objetos minúsculos.}{ul.tra.mi.cros.có.pio}{0}
\verb{ultrapassado}{}{}{}{}{adj.}{Que se ultrapassou.}{ul.tra.pas.sa.do}{0}
\verb{ultrapassado}{}{}{}{}{}{Superado.}{ul.tra.pas.sa.do}{0}
\verb{ultrapassado}{}{}{}{}{}{Obsoleto, antiquado.}{ul.tra.pas.sa.do}{0}
\verb{ultrapassagem}{}{}{"-ens}{}{s.f.}{Ato ou efeito de ultrapassar.}{ul.tra.pas.sa.gem}{0}
\verb{ultrapassar}{}{}{}{}{v.t.}{Passar além de; transpor.}{ul.tra.pas.sar}{0}
\verb{ultrapassar}{}{}{}{}{}{Exceder; extrapolar.}{ul.tra.pas.sar}{0}
\verb{ultrapassar}{}{}{}{}{}{Superar.}{ul.tra.pas.sar}{0}
\verb{ultrapassar}{}{}{}{}{v.i.}{Passar à frente.}{ul.tra.pas.sar}{\verboinum{1}}
\verb{ultrarromântico}{}{}{ultrarromânticos}{}{adj.}{Relativo ao ultrarromantismo.}{ul.trar.ro.mân.ti.co}{0}
\verb{ultrarromântico}{}{}{ultrarromânticos}{}{s.m.}{Indivíduo adepto do ultra"-romantismo.}{ul.trar.ro.mân.ti.co}{0}
\verb{ultrarromantismo}{}{}{}{}{s.m.}{Movimento literário que exacerbou os ideais do romantismo, caracterizado pelo sentimentalismo e pelo desespero.}{ul.trar.ro.man.tis.mo}{0}
\verb{ultrassensível}{}{}{ultrassensíveis}{}{adj.2g.}{Extremamente sensível.}{ul.tras.sen.sí.vel}{0}
\verb{ultrassom}{}{Fís.}{ultrassons}{}{s.m.}{Som de frequência superior a 20.000 Hz e, portanto, inaudível pelo ouvido humano.}{ul.tras.som}{0}
\verb{ultrassom}{}{}{ultrassons}{}{}{Redução de \textit{ultra"-sonografia}.}{ul.tras.som}{0}
\verb{ultrassônico}{}{}{ultra"-sônicos}{}{adj.}{Relativo a ultra"-som.}{ul.tras.sô.ni.co}{0}
\verb{ultrassonografia}{}{Med.}{ultrassonografias}{}{s.f.}{Técnica que permite obter imagens de órgãos internos do corpo pela emissão de ondas sonoras de alta frequência.}{ul.tras.so.no.gra.fi.a}{0}
\verb{ultravioleta}{ê}{Fís.}{}{}{adj.}{Diz"-se de radiação eletromagnética de frequência superior à da luz violeta e, portanto, invisível aos olhos humanos.}{ul.tra.vi.o.le.ta}{0}
\verb{ululante}{}{}{}{}{adj.2g.}{Que ulula.}{u.lu.lan.te}{0}
\verb{ululante}{}{Fig.}{}{}{}{Evidente, óbvio, gritante.}{u.lu.lan.te}{0}
\verb{ulular}{}{}{}{}{v.t.}{Proferir gritando; vociferar, bradar.}{u.lu.lar}{0}
\verb{ulular}{}{}{}{}{v.i.}{Soltar voz aguda e lamentosa (diz"-se de cães e algumas aves); ganir.}{u.lu.lar}{0}
\verb{ulular}{}{}{}{}{}{Gritar de maneira ruidosa denotando ira ou desespero.}{u.lu.lar}{0}
\verb{ulular}{}{}{}{}{}{Gritar de dor ou aflição.}{u.lu.lar}{\verboinum{1}}
\verb{ululo}{}{}{}{}{s.m.}{Ato ou efeito de ulular.}{u.lu.lo}{0}
\verb{um}{}{}{}{}{num.}{Nome dado à quantidade expressa pelo número 1.  }{um}{0}
\verb{um}{}{}{}{}{art.}{Designa pessoa, animal ou coisa de modo impreciso, vago, não conhecido; qualquer, algum, certo.}{um}{0}
\verb{um}{}{}{}{}{pron.}{Uma coisa.}{um}{0}
\verb{uma}{}{}{}{}{}{Feminino de \textit{um}.}{u.ma}{0}
\verb{umbanda}{}{Relig.}{}{}{s.f.}{Religião formada a partir de elementos espíritas e afro"-brasileiros.}{um.ban.da}{0}
\verb{umbandista}{}{}{}{}{adj.2g.}{Relativo a umbanda.}{um.ban.dis.ta}{0}
\verb{umbandista}{}{}{}{}{}{Praticante da umbanda. }{um.ban.dis.ta}{0}
\verb{umbaúba}{}{Bot.}{}{}{s.f.}{Árvore de tronco mais ou menos oco; imbaúba.}{um.ba.ú.ba}{0}
\verb{umbela}{é}{}{}{}{s.f.}{Pequeno guarda"-chuva; sombrinha.}{um.be.la}{0}
\verb{umbela}{é}{Por ext.}{}{}{}{Aquilo que tem forma de umbela.}{um.be.la}{0}
\verb{umbelífera}{}{Bot.}{}{}{s.f.}{Espécime das umbelíferas, família composta principalmente de herbáceas aromáticas, providas de rizoma, cujas flores, pequenas e hermafroditas, são dispostas em umbelas; seus representantes mais comuns são a salsa, a cenoura, a erva"-doce e a mandioquinha.  }{um.be.lí.fe.ra}{0}
\verb{umbigada}{}{}{}{}{s.f.}{Pancada com o umbigo ou com a barriga.}{um.bi.ga.da}{0}
\verb{umbigada}{}{}{}{}{}{A região do umbigo.}{um.bi.ga.da}{0}
\verb{umbigada}{}{Bras.}{}{}{}{Pancada que, em uma dança de roda, o dançarino dá na pessoa que vai substituí"-lo.}{um.bi.ga.da}{0}
\verb{umbigo}{}{}{}{}{s.m.}{Depressão localizada no meio do ventre, na qual fica a cicatriz deixada pelo corte do cordão umbilical.}{um.bi.go}{0}
\verb{umbilical}{}{}{"-ais}{}{adj.2g.}{Relativo a umbigo.}{um.bi.li.cal}{0}
\verb{umbral}{}{}{"-ais}{}{s.m.}{Ombreira da porta.}{um.bral}{0}
\verb{umbral}{}{}{"-ais}{}{}{Entrada, limiar, soleira.}{um.bral}{0}
\verb{umbrela}{é}{}{}{}{}{Var. de \textit{umbela}.}{um.bre.la}{0}
\verb{umbrífero}{}{Liter.}{}{}{adj.}{Sombrio, umbroso.}{um.brí.fe.ro}{0}
\verb{umbroso}{ô}{}{"-osos ⟨ó⟩}{"-osa ⟨ó⟩}{adj.}{Que tem sombra; sombrio, escuro.}{um.bro.so}{0}
\verb{umbroso}{ô}{}{"-osos ⟨ó⟩}{"-osa ⟨ó⟩}{}{Que produz sombra; copado, frondoso.}{um.bro.so}{0}
\verb{umbu}{}{Bras.}{}{}{s.m.}{Fruto do umbuzeiro, amarelo avermelhado, doce e sumarento, em forma de baga e comestível; seriguela.  }{um.bu}{0}
\verb{umbuzeiro}{ê}{Bot.}{}{}{s.m.}{Árvore bastante copada, com flores pequenas, frutos comestíveis que acumula água nas raízes, encontrada em regiões de caatinga; seriguela.  }{um.bu.zei.ro}{0}
\verb{umectante}{}{}{}{}{adj.2g.}{Que umecta, umedece.}{u.mec.tan.te}{0}
\verb{umectar}{}{}{}{}{v.t.}{Umedecer, molhar.}{u.mec.tar}{\verboinum{1}}
\verb{umedecer}{ê}{}{}{}{v.t.}{Tornar úmido, levemente molhado.}{u.me.de.cer}{\verboinum{15}}
\verb{úmero}{}{Anat.}{}{}{s.m.}{Osso do braço (entre ombro e cotovelo).}{ú.me.ro}{0}
\verb{umidade}{}{}{}{}{s.f.}{Qualidade ou estado de úmido.}{u.mi.da.de}{0}
\verb{umidade}{}{}{}{}{}{Quantidade de vapor de água na atmosfera, que pode ser medida por aparelho meteorológico.}{u.mi.da.de}{0}
\verb{umidificar}{}{}{}{}{v.t.}{Tornar úmido; umedecer, umectar.}{u.mi.di.fi.car}{\verboinum{2}}
\verb{úmido}{}{}{}{}{}{Da natureza da água; aquoso.  }{ú.mi.do}{0}
\verb{úmido}{}{}{}{}{adj.}{Um tanto molhado. }{ú.mi.do}{0}
\verb{úmido}{}{}{}{}{}{Impregnado de água ou de vapor d'água.}{ú.mi.do}{0}
%\verb{}{}{}{}{}{}{}{}{0}
%\verb{}{}{}{}{}{}{}{}{0}
\verb{unânime}{}{}{}{}{adj.}{Que está em conformidade (sentimento ou opinião) com os demais considerados.}{u.nâ.ni.me}{0}
\verb{unânime}{}{}{}{}{}{Que provém de acordo ou concordância geral.}{u.nâ.ni.me}{0}
\verb{unanimidade}{}{}{}{}{s.f.}{Qualidade de unânime.}{u.na.ni.mi.da.de}{0}
\verb{unção}{}{}{"-ões}{}{s.f.}{Ato ou efeito de ungir, de aplicar os óleos consagrados.}{un.ção}{0}
\verb{unção}{}{Fig.}{"-ões}{}{}{Sentimento de piedade religiosa.}{un.ção}{0}
\verb{unção}{}{Por ext.}{"-ões}{}{}{Doçura na maneira de se expressar que comove.}{un.ção}{0}
\verb{undécimo}{}{}{}{}{num.}{Que ocupa o décimo primeiro lugar numa sequência.}{un.dé.ci.mo}{0}
\verb{undécimo}{}{}{}{}{s.m.}{Cada uma das onze partes iguais em que se pode dividir um todo.}{un.dé.ci.mo}{0}
\verb{undécuplo}{}{}{}{}{num.}{Que é onze vezes maior.}{un.dé.cu.plo}{0}
\verb{undécuplo}{}{}{}{}{s.m.}{Quantidade onze vezes maior que a considerada.}{un.dé.cu.plo}{0}
\verb{undícola}{}{Biol.}{}{}{adj.2g.}{Diz"-se de animal que vive na água.}{un.dí.co.la}{0}
\verb{undícola}{}{}{}{}{s.2g.}{Esse animal.}{un.dí.co.la}{0}
\verb{undífero}{}{}{}{}{adj.}{Que tem ondas; undoso.}{un.dí.fe.ro}{0}
\verb{undívago}{}{}{}{}{adj.}{Que vaga sobre ondas; flutívago.}{un.dí.va.go}{0}
\verb{undoso}{ô}{}{"-osos ⟨ó⟩}{"-osa ⟨ó⟩}{adj.}{Que forma ondas.}{un.do.so}{0}
\verb{undoso}{ô}{}{"-osos ⟨ó⟩}{"-osa ⟨ó⟩}{}{Que ondula; ondulante, ondulatório.}{un.do.so}{0}
%\verb{}{}{}{}{}{}{}{}{0}
\verb{ungir}{}{}{}{}{v.t.}{Esfregar com óleo, unguento ou qualquer substância gordurosa; untar.}{un.gir}{0}
\verb{ungir}{}{}{}{}{}{Aplicar óleos consagrados em (alguém); sagrar.}{un.gir}{0}
\verb{ungir}{}{Relig.}{}{}{}{Dar a extrema"-unção a.}{un.gir}{0}
\verb{ungir}{}{Fig.}{}{}{}{Purificar, corrigir, melhorar.}{un.gir}{\verboinum{34}}
\verb{ungueal}{}{}{"-ais}{}{adj.2g.}{Relativo ou pertencente à unha.}{un.gue.al}{0}
\verb{unguento}{}{Farm.}{}{}{s.m.}{Medicamento de consistência mole, o qual tem por base um corpo gorduroso e se aplica externamente.}{un.guen.to}{0}
\verb{unguento}{}{}{}{}{}{Nome de essências com que os antigos perfumavam e embalsamavam os corpos.}{un.guen.to}{0}
\verb{unguiculado}{}{}{}{}{adj.}{Que tem forma ou é dotado de unha; ungueal. }{un.gui.cu.la.do}{0}
\verb{unguiculado}{}{Bot.}{}{}{}{Que é provido de unha (base alongada nas sépalas e pétalas).}{un.gui.cu.la.do}{0}
\verb{unguiculado}{}{Zool.}{}{}{}{Espécime dos unguiculados, antiga divisão dos mamíferos placentários dotados de unhas ou garras.}{un.gui.cu.la.do}{0}
\verb{unguífero}{}{}{}{}{adj.}{Que é provido de unhas, garras ou estruturas similares; uncinado, ungueado.}{un.guí.fe.ro}{0}
\verb{unguiforme}{ó}{}{}{}{adj.2g.}{Que tem forma de unha ou garra.}{un.gui.for.me}{0}
\verb{úngula}{}{}{}{}{s.f.}{Unha, garra ou casco de um animal.}{ún.gu.la}{0}
\verb{úngula}{}{}{}{}{}{Saliência membranosa situada no ângulo interno do olho.}{ún.gu.la}{0}
\verb{ungulado}{}{}{}{}{adj.}{Diz"-se de mamífero que possui casco.}{un.gu.la.do}{0}
\verb{ungulado}{}{Zool.}{}{}{s.m.}{Numa antiga classificação, espécime dos ungulados, divisão que compreendia os mamíferos com dedos revestidos de casco.}{un.gu.la.do}{0}
%\verb{}{}{}{}{}{}{}{}{0}
\verb{unha}{}{Anat.}{}{}{s.f.}{Lâmina dura, de queratina, semitransparente, que recobre a extremidade dorsal dos dedos.}{u.nha}{0}
\verb{unha}{}{}{}{}{}{Garra.}{u.nha}{0}
\verb{unha}{}{Zool.}{}{}{}{Casco dos ruminantes e paquidermes.}{u.nha}{0}
\verb{unha}{}{Bot.}{}{}{}{Parte inferior, alongada e estreita, de sépalas e pétalas.}{u.nha}{0}
\verb{unha}{}{}{}{}{}{Pé de caranguejo.}{u.nha}{0}
\verb{unha}{}{}{}{}{}{Qualquer parte encurvada e pontiaguda de um utensílio, objeto ou instrumento.}{u.nha}{0}
\verb{unhaca}{}{Pop.}{}{}{s.2g.}{Avarento, sovina, unha"-de"-fome. }{u.nha.ca}{0}
\verb{unhaca}{}{}{}{}{}{Pessoa com quem se tem muita amizade ou intimidade.}{u.nha.ca}{0}
\verb{unhaço}{}{}{}{}{s.m.}{Unhada.}{u.nha.ço}{0}
\verb{unhada}{}{}{}{}{s.f.}{Ferimento ou arranhão feito com unha.}{u.nha.da}{0}
\verb{unha"-de"-fome}{ó/ ou /ô}{}{unhas"-de"-fome ⟨ó/ ou /ô⟩}{}{s.2g.}{Avarento.}{u.nha"-de"-fo.me}{0}
\verb{unha"-de"-gato}{}{Bot.}{unhas"-de"-gato}{}{s.f.}{Nome comum a várias espécies de planta, da família das leguminosas, providas de espinhos semelhantes a unhas de gato.}{u.nha"-de"-ga.to}{0}
\verb{unha"-de"-gato}{}{Bras.}{unhas"-de"-gato}{}{s.f.}{Ancinho.}{u.nha"-de"-ga.to}{0}
\verb{unha"-de"-vaca}{}{Bot.}{unhas"-de"-vaca}{}{s.f.}{Trepadeira lenhosa e alta, cujos frutos são vagens; pata"-de"-vaca, cipó"-escada.}{u.nha"-de"-va.ca}{0}
\verb{unhar}{}{}{}{}{v.t.}{Produzir escoriação, ferir com as unhas; agatanhar, arranhar.}{u.nhar}{\verboinum{1}}
\verb{unheiro}{ê}{Pop.}{}{}{s.m.}{Inflamação da pele em volta da unha; panarício.}{u.nhei.ro}{0}
\verb{união}{}{}{"-ões}{}{s.f.}{Ato de unir; junção. (\textit{Os empresários fizeram a união de seus negócios.})}{u.ni.ão}{0}
\verb{união}{}{}{"-ões}{}{}{Ato de se unir; aliança. (\textit{As famílias comemoraram a união de seus filhos.})}{u.ni.ão}{0}
\verb{união}{}{}{"-ões}{}{}{Esforço feito por todos; concórdia, harmonia. (\textit{Vivemos em perfeita união.})}{u.ni.ão}{0}
\verb{unicameral}{}{}{"-ais}{}{adj.2g.}{Diz"-se de sistema de representação política que tem apenas uma câmara legislativa.}{u.ni.ca.me.ral}{0}
\verb{unicelular}{}{Biol.}{}{}{adj.2g.}{Diz"-se de organismo que possui ou é formado por uma única célula.}{u.ni.ce.lu.lar}{0}
\verb{unicidade}{}{}{}{}{s.f.}{Qualidade ou estado do que é único; singularidade.}{u.ni.ci.da.de}{0}
\verb{único}{}{}{}{}{adj.}{Que é um só, que não tem outro igual.    }{ú.ni.co}{0}
\verb{único}{}{}{}{}{}{Incomparável, singular, exclusivo.   }{ú.ni.co}{0}
\verb{unicolor}{ô}{}{}{}{adj.2g.}{Que tem apenas uma cor; monocromático.}{u.ni.co.lor}{0}
\verb{unicorne}{ó}{}{}{}{adj.2g.}{Que tem apenas um corno.}{u.ni.cor.ne}{0}
\verb{unicorne}{ó}{}{}{}{s.m.}{Unicórnio.}{u.ni.cor.ne}{0}
\verb{unicórnio}{}{Mit.}{}{}{s.m.}{Animal fabuloso com corpo de cavalo e provido de um único e longo chifre na cabeça, dotado de poderes sobrenaturais; unicorne, licorne.}{u.ni.cór.nio}{0}
\verb{unicórnio}{}{}{}{}{}{Certo rinoceronte indiano, com um corno apenas.}{u.ni.cór.nio}{0}
\verb{unicórnio}{}{Por ext.}{}{}{}{Substância ou objeto feito a partir do chifre desse animal.}{u.ni.cór.nio}{0}
\verb{unicórnio}{}{Astron.}{}{}{}{Certa constelação equatorial. (Nesta acepção, com letra maiúscula.)}{u.ni.cór.nio}{0}
\verb{unidade}{}{}{}{}{s.f.}{Grandeza tomada como termo de comparação entre grandezas da mesma espécie. }{u.ni.da.de}{0}
\verb{unidade}{}{}{}{}{}{O número um.}{u.ni.da.de}{0}
\verb{unidade}{}{}{}{}{}{Qualidade de estar unido.   }{u.ni.da.de}{0}
\verb{unidimensional}{}{}{"-ais}{}{adj.2g.}{Que tem ou envolve apenas uma dimensão.}{u.ni.di.men.si.o.nal}{0}
\verb{unidirecional}{}{}{"-ais}{}{adj.2g.}{Que se move, se propaga ou flui numa única direção.}{u.ni.di.re.ci.o.nal}{0}
\verb{unidirecional}{}{Fig.}{"-ais}{}{}{Que tem ou envolve uma única direção.}{u.ni.di.re.ci.o.nal}{0}
\verb{unidirecional}{}{}{"-ais}{}{}{Que é direcionável ou reage numa única direção.   }{u.ni.di.re.ci.o.nal}{0}
\verb{unido}{}{}{}{}{adj.}{Que se uniu; ligado, junto.}{u.ni.do}{0}
\verb{unido}{}{}{}{}{}{Que está em contato, ligado a outrem por laços de amizade.      }{u.ni.do}{0}
\verb{unificação}{}{}{"-ões}{}{s.f.}{Ato ou efeito de unificar.}{u.ni.fi.ca.ção}{0}
\verb{unificador}{ô}{}{}{}{adj.}{Que unifica.}{u.ni.fi.ca.dor}{0}
\verb{unificar}{}{}{}{}{v.t.}{Reunir em um só corpo ou em um todo coeso.}{u.ni.fi.car}{0}
\verb{unificar}{}{}{}{}{}{Fazer convergir para um só fim.}{u.ni.fi.car}{0}
\verb{unificar}{}{}{}{}{v.pron.}{Tornar"-se uno, uniforme; reunir"-se num todo, conglomerar"-se, conglobar"-se.}{u.ni.fi.car}{\verboinum{2}}
\verb{unifloro}{ó}{Bot.}{}{}{adj.}{Diz"-se da planta que possui uma única flor.}{u.ni.flo.ro}{0}
\verb{uniforme}{ó}{}{}{}{adj.2g.}{Que tem só uma forma.}{u.ni.for.me}{0}
\verb{uniforme}{ó}{}{}{}{}{Que não varia; igual.}{u.ni.for.me}{0}
\verb{uniforme}{ó}{}{}{}{s.m.}{Roupa do mesmo modelo, usada por todas as pessoas de um grupo.  }{u.ni.for.me}{0}
\verb{uniformidade}{}{}{}{}{s.f.}{Qualidade ou estado de uniforme.}{u.ni.for.mi.da.de}{0}
\verb{uniformidade}{}{}{}{}{}{Coerência, harmonia.}{u.ni.for.mi.da.de}{0}
\verb{uniformidade}{}{}{}{}{}{Falta de diversidade, de variedade; monotonia.}{u.ni.for.mi.da.de}{0}
\verb{uniformização}{}{}{"-ões}{}{s.f.}{Ato ou efeito de uniformizar.}{u.ni.for.mi.za.ção}{0}
\verb{uniformizar}{}{}{}{}{v.t.}{Dar a mesma forma a.}{u.ni.for.mi.zar}{0}
\verb{uniformizar}{}{}{}{}{}{Vestir com uniforme.}{u.ni.for.mi.zar}{\verboinum{1}}
\verb{unigênito}{}{}{}{}{adj.}{Diz"-se do filho único; único gerado.}{u.ni.gê.ni.to}{0}
\verb{unigênito}{}{}{}{}{}{Cognome de Jesus Cristo. (Nesta acepção, com maiúscula)}{u.ni.gê.ni.to}{0}
\verb{unilateral}{}{}{"-ais}{}{adj.2g.}{Disposto de um único lado.}{u.ni.la.te.ral}{0}
\verb{unilateral}{}{}{"-ais}{}{}{Que envolve ou diz respeito a um lado só; parcial.}{u.ni.la.te.ral}{0}
\verb{unilateral}{}{}{"-ais}{}{}{Diz"-se de contrato em que só uma das partes tem obrigações para com a outra.}{u.ni.la.te.ral}{0}
%\verb{}{}{}{}{}{}{}{}{0}
%\verb{}{}{}{}{}{}{}{}{0}
\verb{unilinear}{}{}{}{}{adj.2g.}{Que possui ou segue uma única linha.}{u.ni.li.ne.ar}{0}
\verb{unilinear}{}{}{}{}{}{Que leva em consideração apenas um dos ascendentes, o pai ou a mãe.}{u.ni.li.ne.ar}{0}
\verb{unilíngue}{}{}{}{}{adj.2g.}{Que está expresso ou escrito numa única língua.}{u.ni.lín.gue}{0}
\verb{unilíngue}{}{}{}{}{s.2g.}{Monolíngue.}{u.ni.lín.gue}{0}
%\verb{}{}{}{}{}{}{}{}{0}
%\verb{}{}{}{}{}{}{}{}{0}
%\verb{}{}{}{}{}{}{}{}{0}
%\verb{}{}{}{}{}{}{}{}{0}
\verb{uníparo}{}{Biol.}{}{}{adj.}{Diz"-se de fêmea que pare uma única cria por vez.}{u.ní.pa.ro}{0}
\verb{uníparo}{}{Bot.}{}{}{}{Diz"-se da cimeira que tem um único ramo lateral florífero, terminado em uma flor apenas.}{u.ní.pa.ro}{0}
\verb{unipolar}{}{Fís.}{}{}{adj.2g.}{Que só tem um polo.}{u.ni.po.lar}{0}
\verb{unir}{}{}{}{}{v.t.}{Juntar, ligar.}{u.nir}{0}
\verb{unir}{}{}{}{}{}{Estabelecer comunicação entre.}{u.nir}{0}
\verb{unir}{}{}{}{}{}{Reunir em um grupo só; associar.}{u.nir}{\verboinum{18}}
\verb{unissex}{écs}{}{}{}{adj.2g.}{Diz"-se de roupa ou corte de cabelo, entre outros, que serve ou pode ser usado tanto por homens quanto por mulheres.}{u.nis.sex}{0}
\verb{unissexuado}{cs}{}{}{}{adj.}{Que tem apenas um sexo; unissexual.}{u.nis.se.xu.a.do}{0}
\verb{unissexuado}{cs}{Bot.}{}{}{}{Diz"-se das flores que só têm pistilos ou estames.}{u.nis.se.xu.a.do}{0}
\verb{unissexual}{cs}{}{"-ais}{}{adj.2g.}{Que tem apenas um sexo; unissexuado.}{u.nis.se.xu.al}{0}
\verb{unissonância}{}{Mús.}{}{}{s.f.}{Qualidade de uníssono.}{u.nis.so.nân.cia}{0}
\verb{uníssono}{}{}{}{}{adj.}{Que tem o mesmo som que outro.}{u.nís.so.no}{0}
\verb{uníssono}{}{}{}{}{s.m.}{Conjunto de sons cuja entoação é a mesma.}{u.nís.so.no}{0}
\verb{unitário}{}{}{}{}{adj.}{Que se refere a unidade.}{u.ni.tá.rio}{0}
\verb{unitivo}{}{}{}{}{adj.}{Que tem a qualidade de se unir ou de fazer unir.}{u.ni.ti.vo}{0}
\verb{univalente}{}{Quím.}{}{}{adj.2g.}{Que tem apenas uma valência; monovalente.}{u.ni.va.len.te}{0}
\verb{univalve}{}{Zool.}{}{}{adj.2g.}{Diz"-se da concha de molusco que possui apenas uma peça. }{u.ni.val.ve}{0}
\verb{universal}{}{}{"-ais}{}{adj.2g.}{Que pertence ao universo.}{u.ni.ver.sal}{0}
\verb{universal}{}{}{"-ais}{}{}{Mundial.}{u.ni.ver.sal}{0}
\verb{universal}{}{}{"-ais}{}{}{Comum a todos.}{u.ni.ver.sal}{0}
\verb{universal}{}{}{"-ais}{}{}{Geral, total, absoluto.}{u.ni.ver.sal}{0}
\verb{universalidade}{}{}{}{}{s.f.}{Qualidade do que é universal; totalidade.}{u.ni.ver.sa.li.da.de}{0}
\verb{universalismo}{}{}{}{}{s.m.}{Tendência para universalizar uma obra, uma ideia, um sistema.}{u.ni.ver.sa.lis.mo}{0}
\verb{universalização}{}{}{"-ões}{}{s.f.}{Ato ou efeito de universalizar. }{u.ni.ver.sa.li.za.ção}{0}
\verb{universalizar}{}{}{}{}{v.t.}{Tornar universal; generalizar.}{u.ni.ver.sa.li.zar}{0}
\verb{universalizar}{}{}{}{}{}{Espalhar, difundir.}{u.ni.ver.sa.li.zar}{\verboinum{1}}
\verb{universidade}{}{}{}{}{s.f.}{Conjunto de faculdades de ensino superior.}{u.ni.ver.si.da.de}{0}
\verb{universidade}{}{}{}{}{}{Conjunto das disciplinas, dos professores e alunos dessas faculdades.}{u.ni.ver.si.da.de}{0}
\verb{universidade}{}{}{}{}{}{A sede onde funciona esse conjunto de faculdades.}{u.ni.ver.si.da.de}{0}
\verb{universitário}{}{}{}{}{adj.}{Que se refere a universidade.}{u.ni.ver.si.tá.rio}{0}
\verb{universitário}{}{}{}{}{s.m.}{Aluno de universidade, estudante de curso superior.}{u.ni.ver.si.tá.rio}{0}
\verb{universo}{é}{}{}{}{s.m.}{Conjunto dos astros do espaço; cosmo.}{u.ni.ver.so}{0}
\verb{unívoco}{}{}{}{}{adj.}{Que só permite uma interpretação.}{u.ní.vo.co}{0}
\verb{unívoco}{}{}{}{}{}{Homogêneo.}{u.ní.vo.co}{0}
%\verb{}{}{}{}{}{}{}{}{0}
\verb{uno}{}{}{}{}{adj.}{Que é único; um, singular.}{u.no}{0}
\verb{uno}{}{}{}{}{}{Que não se pode dividir; sem partes.}{u.no}{0}
\verb{uno}{}{}{}{}{}{Que não pode ser desfeito; íntegro, unido.}{u.no}{0}
%\verb{}{}{}{}{}{}{}{}{0}
%\verb{}{}{}{}{}{}{}{}{0}
%\verb{}{}{}{}{}{}{}{}{0}
\verb{unúntrio}{}{Quím.}{}{}{s.m.}{Elemento químico sintético, de compostos desconhecidos, com tempo de desintegração ínfimo e propriedades supostamente semelhantes aàs do tálio. \elemento{113}{(286)}{Uut}.}{u.nún.trio}{0}
\verb{untanha}{}{}{}{}{s.f.}{Nome comum aos anfíbios anuros, dotados de pequenos cornos carnosos sobre os olhos e que se alimentam de outros sapos e pequenos vertebrados, tornando"-se agressivos quando molestados; intanha.}{un.ta.nha}{0}
\verb{untar}{}{}{}{}{v.t.}{Aplicar qualquer substância gordurosa em alguma coisa; besuntar.}{un.tar}{\verboinum{1}}
\verb{unto}{}{}{}{}{s.m.}{Gordura de suíno; banha.}{un.to}{0}
\verb{unto}{}{}{}{}{}{Unguento.}{un.to}{0}
\verb{untuosidade}{}{}{}{}{s.f.}{Qualidade do que é untuoso.  }{un.tu.o.si.da.de}{0}
\verb{untuoso}{ô}{}{"-osos ⟨ó⟩}{"-osa ⟨ó⟩}{adj.}{Que tem unto; oleoso, gorduroso. }{un.tu.o.so}{0}
\verb{untuoso}{ô}{}{"-osos ⟨ó⟩}{"-osa ⟨ó⟩}{}{Lubrificado, escorregadiço.}{un.tu.o.so}{0}
\verb{untuoso}{ô}{Fig.}{"-osos ⟨ó⟩}{"-osa ⟨ó⟩}{}{Bajulador, melífluo, subserviente.}{un.tu.o.so}{0}
\verb{untura}{}{}{}{}{s.f.}{Ato ou efeito de untar; untadura, unção.}{un.tu.ra}{0}
\verb{untura}{}{}{}{}{}{Qualquer substância gordurosa; banha, gordura, óleo, unto.}{un.tu.ra}{0}
\verb{untura}{}{Fig.}{}{}{}{Conhecimento superficial sobre algo.}{un.tu.ra}{0}
%\verb{}{}{}{}{}{}{}{}{0}
\verb{ununóctio}{}{Quím.}{}{}{s.m.}{Nome provisório do último elemento da tabela periódica, sintetizado artificialmente e conhecido como ``elemento 118'' ou ``eka-radônio''. É gasoso e radioativo e talvez o primeiro gás com semicondutividade. \elemento{118}{(293)}{Uup}.}{u.nun.óc.tio}{0}
\verb{ununpêntio}{}{Quím.}{}{}{s.m.}{Nome provisório do elemento químico sintético, também conhecido como ``elemento 115'' ou ``eka-bismuto'', é radioativo e provavelmente metálico. Sintetizado provavelmente em 2004, sua descoberta ainda não está confirmada. \elemento{115}{(288)}{Uup}.}{u.nun.pên.tio}{0}
\verb{ununséptio}{}{Quím.}{}{}{s.m.}{Nome provisório do elemento químico artificial, também conhecido como ``eka-astato'', sintetizado em 2010. \elemento{117}{(294)}{Uus}.}{u.nun.sép.tio}{0}
\verb{upa}{}{}{}{}{interj.}{Diz"-se para encorajar, incentivar animal ou pessoa a realizar alguma atividade árdua.}{u.pa}{0}
\verb{upa}{}{}{}{}{}{Expressa surpresa, admiração, susto.}{u.pa}{0}
\verb{upgrade}{}{Informát.}{}{}{s.m.}{Atualização de componente físico ou lógico de um computador.}{\textit{upgrade}}{0}
\verb{urânio}{}{Quím.}{}{}{s.m.}{Elemento químico radioativo, do grupo dos actinídeos, obtido naturalmente, usado em pilhas e reatores nucleares. \elemento{92}{238.0289}{U}.}{u.râ.nio}{0}
\verb{uranismo}{}{}{}{}{s.m.}{Homossexualismo, especialmente o masculino.}{u.ra.nis.mo}{0}
\verb{Urano}{}{Astron.}{}{}{s.m.}{Sétimo planeta do sistema solar, pela ordem de afastamento do Sol, com diâmetro quatro vezes maior que o da Terra e 15 satélites.}{U.ra.no}{0}
\verb{uranografia}{}{Astron.}{}{}{s.f.}{Ciência cujo objetivo é o estudo e a descrição do céu; astronomia, uranologia.}{u.ra.no.gra.fi.a}{0}
\verb{uranografia}{}{Desus.}{}{}{}{Astrometria.}{u.ra.no.gra.fi.a}{0}
\verb{uranografia}{}{}{}{}{}{Fotografia astronômica.}{u.ra.no.gra.fi.a}{0}
\verb{urbanidade}{}{}{}{}{s.f.}{Qualidade ou condição de urbano.}{ur.ba.ni.da.de}{0}
\verb{urbanidade}{}{Fig.}{}{}{}{Civilidade, cortesia entre cidadãos.}{ur.ba.ni.da.de}{0}
\verb{urbanismo}{}{}{}{}{s.m.}{Técnica de construir, reformar e embelezar uma cidade.}{ur.ba.nis.mo}{0}
\verb{urbanista}{}{}{}{}{adj.2g.}{Que se refere a urbanismo.}{ur.ba.nis.ta}{0}
\verb{urbanista}{}{}{}{}{s.2g.}{Especialista ou técnico em urbanismo.}{ur.ba.nis.ta}{0}
\verb{urbanístico}{}{}{}{}{adj.}{Relativo a urbanismo.}{ur.ba.nís.ti.co}{0}
\verb{urbanização}{}{}{"-ões}{}{s.f.}{Ato ou efeito de urbanizar.}{ur.ba.ni.za.ção}{0}
\verb{urbanização}{}{}{"-ões}{}{}{Concentração cada vez maior de população em aglomerados de caráter urbano.}{ur.ba.ni.za.ção}{0}
\verb{urbanizar}{}{}{}{}{v.t.}{Tornar urbano; civilizar.}{ur.ba.ni.zar}{0}
\verb{urbanizar}{}{}{}{}{}{Realizar urbanização.}{ur.ba.ni.zar}{\verboinum{1}}
\verb{urbano}{}{}{}{}{adj.}{Que se refere à cidade.}{ur.ba.no}{0}
\verb{urbano}{}{}{}{}{}{Que tem características de cidade.}{ur.ba.no}{0}
\verb{urbe}{}{}{}{}{s.f.}{Cidade.}{ur.be}{0}
\verb{urdideira}{ê}{}{}{}{s.f.}{Mulher que urde ou tece.}{ur.di.dei.ra}{0}
\verb{urdideira}{ê}{}{}{}{}{Máquina de urdir, usada na indústria.}{ur.di.dei.ra}{0}
\verb{urdidor}{ô}{}{}{}{adj.}{Que urde.}{ur.di.dor}{0}
\verb{urdidor}{ô}{}{}{}{s.m.}{Indivíduo que urde, que tece; tecelão.}{ur.di.dor}{0}
\verb{urdidura}{}{}{}{}{s.f.}{Ato ou efeito de urdir; urdimento, urdume.}{ur.di.du.ra}{0}
\verb{urdidura}{}{Fig.}{}{}{}{Enredo, trama de uma obra de ficção.}{ur.di.du.ra}{0}
\verb{urdidura}{}{Fig.}{}{}{}{Maquinação com o objetivo de enganar alguém; tramoia, armação, cilada.}{ur.di.du.ra}{0}
\verb{urdimento}{}{}{}{}{s.m.}{Urdidura.}{ur.di.men.to}{0}
\verb{urdir}{}{}{}{}{v.t.}{Dispor (os fios) para tecer.}{ur.dir}{0}
\verb{urdir}{}{Fig.}{}{}{}{Tramar, maquinar, armar.}{ur.dir}{\verboinum{18}}
\verb{urdume}{}{}{}{}{s.m.}{Urdidura.}{ur.du.me}{0}
\verb{ureia}{é}{Quím.}{}{}{s.f.}{Substância incolor, cristalina, eliminada na urina como produto do metabolismo dos aminoácidos; obtida sinteticamente, tem vários usos na indústria farmacêutica.}{u.rei.a}{0}
\verb{uremia}{}{Med.}{}{}{s.f.}{Sintomas de intoxicação causados pelo acúmulo de substâncias tóxicas no sangue que normalmente seriam eliminadas pelos rins na urina.}{u.re.mi.a}{0}
\verb{urente}{}{}{}{}{adj.2g.}{Que arde, queima.  }{u.ren.te}{0}
\verb{urente}{}{Bot.}{}{}{}{Diz"-se da planta com algum dispositivo urticante, irritante, geralmente pelos.}{u.ren.te}{0}
\verb{ureter}{é}{Anat.}{}{}{s.m.}{Cada um dos dois canais que conduzem a urina dos rins à bexiga.}{u.re.ter}{0}
\verb{urético}{}{}{}{}{adj.}{Diurético.}{u.ré.ti.co}{0}
\verb{urético}{}{}{}{}{}{Relativo a urina. }{u.ré.ti.co}{0}
\verb{urético}{}{Med.}{}{}{}{Diz"-se de qualquer enfermidade que acometa o canal da uretra.}{u.ré.ti.co}{0}
\verb{uretra}{é}{Anat.}{}{}{s.f.}{Canal que excreta a urina da bexiga para fora do corpo e que, no homem, serve também de canal ao esperma.}{u.re.tra}{0}
\verb{uretral}{}{}{"-ais}{}{adj.2g.}{Relativo ou pertencente à uretra.}{u.re.tral}{0}
\verb{urgência}{}{}{}{}{s.f.}{Qualidade ou condição de urgente.}{ur.gên.cia}{0}
\verb{urgência}{}{}{}{}{}{Situação grave que precisa ser atendida sem demora e com prioridade; emergência.}{ur.gên.cia}{0}
\verb{urgente}{}{}{}{}{adj.2g.}{Que urge; que é preciso ser feito ou atendido imediatamente.}{ur.gen.te}{0}
\verb{urgente}{}{}{}{}{}{Imprescindível, indispensável.}{ur.gen.te}{0}
\verb{urgente}{}{}{}{}{}{Iminente, premente, impendente.}{ur.gen.te}{0}
\verb{urgir}{}{}{}{}{v.i.}{Ser necessário sem demora; ser urgente. (\textit{O tempo urge.})}{ur.gir}{\verboinum{34}\verboirregular{\emph{def.} urgimos, urgis}}
\verb{úrico}{}{}{}{}{adj.}{Relativo a urina.}{ú.ri.co}{0}
\verb{úrico}{}{Quím.}{}{}{}{Diz"-se de ácido orgânico nitrogenado, excretado por répteis, pássaros e em pequena quantidade na urina humana, produto da degradação de proteínas.}{ú.ri.co}{0}
\verb{urina}{}{}{}{}{s.f.}{Líquido orgânico formado nos rins, coletado na bexiga e daí evacuado pela uretra, que contém substâncias metabólicas que devem ser eliminadas do organismo; mijo, pipi, xixi.}{u.ri.na}{0}
\verb{urinar}{}{}{}{}{v.t.}{Expelir junto com urina.}{u.ri.nar}{0}
\verb{urinar}{}{}{}{}{v.i.}{Excretar urina; fazer xixi; mijar.}{u.ri.nar}{0}
\verb{urinar}{}{}{}{}{v.pron.}{Expelir urina involuntariamente por deficiência ou medo; molhar"-se.}{u.ri.nar}{\verboinum{1}}
\verb{urinário}{}{}{}{}{adj.}{Relativo a urina.}{u.ri.ná.rio}{0}
\verb{urinol}{ó}{}{"-óis}{}{s.m.}{Recipiente próprio para nele se defecar ou urinar; penico.}{u.ri.nol}{0}
\verb{urna}{}{}{}{}{s.f.}{Caixa ou sacola com uma pequena abertura estreita, na qual os eleitores colocam seus votos.}{ur.na}{0}
\verb{urna}{}{}{}{}{}{Caixa na qual se guardam as cinzas ou outros restos mortais de um defunto; caixão, esquife.}{ur.na}{0}
\verb{urodelo}{é}{Zool.}{}{}{adj.}{Diz"-se de animal que tem cauda visível ou persistente.}{u.ro.de.lo}{0}
\verb{urodelo}{é}{}{}{}{s.m.}{Espécime dos caudados, ordem composta pelos anfíbios de corpo alongado, cauda persistente e pernas curtas; são representados, por exemplo, pelas salamandras; caudado.}{u.ro.de.lo}{0}
\verb{urodinia}{}{Med.}{}{}{s.f.}{Dor causada pelo ato de urinar.}{u.ro.di.ni.a}{0}
\verb{urologia}{}{Med.}{}{}{s.f.}{Especialidade médica que se dedica ao estudo e ao tratamento das doenças urogenitais.}{u.ro.lo.gi.a}{0}
\verb{urológico}{}{}{}{}{adj.}{Relativo a urologia.}{u.ro.ló.gi.co}{0}
\verb{urologista}{}{}{}{}{s.2g.}{Indivíduo que é especialista em urologia.}{u.ro.lo.gis.ta}{0}
\verb{uropatágio}{}{Zool.}{}{}{s.m.}{Membrana que liga as patas traseiras dos morcegos, normalmente aprisionando a cauda.}{u.ro.pa.tá.gio}{0}
\verb{uropígio}{}{Zool.}{}{}{s.m.}{Apêndice triangular, formado pelas últimas vértebras das aves e no qual as penas da cauda se prendem.}{u.ro.pí.gio}{0}
\verb{uroscopia}{}{Med.}{}{}{s.f.}{Exame da urina.}{u.ros.co.pi.a}{0}
\verb{urrar}{}{}{}{}{v.i.}{Soltar urros; bramir, rugir.}{ur.rar}{0}
\verb{urrar}{}{Fig.}{}{}{v.t.}{Soltar, proferir à maneira de urros.}{ur.rar}{\verboinum{1}}
\verb{urro}{}{}{}{}{s.m.}{Bramido ou rugido de algumas feras.}{ur.ro}{0}
\verb{urro}{}{Fig.}{}{}{}{Berro ou grito rouco, muito forte.}{ur.ro}{0}
\verb{ursa}{}{}{}{}{s.f.}{A fêmea do urso.}{ur.sa}{0}
\verb{ursa}{}{Astron.}{}{}{}{Nome dado a duas constelações do hemisfério norte, Ursa Maior e Ursa Menor. (Nesta acepção, com maiúscula)}{ur.sa}{0}
\verb{ursada}{}{}{}{}{s.f.}{Procedimento desleal, especialmente provindo de um amigo.}{ur.sa.da}{0}
\verb{ursídeo}{}{Zool.}{}{}{s.m.}{Espécie da família dos ursídeos,  mamíferos carnívoros, constituída pelos ursos.}{ur.sí.deo}{0}
\verb{urso}{}{Zool.}{}{}{s.m.}{Mamífero carnívoro, de pelagem longa e espessa e cauda curta, comum em região de clima temperado.}{ur.so}{0}
\verb{urso}{}{Fig.}{}{}{}{Indivíduo pouco sociável.}{ur.so}{0}
\verb{ursulina}{}{Relig.}{}{}{s.f.}{Religiosa de várias ordens femininas, em particular da Ordem de Santa Úrsula, cujo objetivo é a educação.}{ur.su.li.na}{0}
\verb{urticação}{}{}{"-ões}{}{s.f.}{Ato de irritar a pele com urtiga.}{ur.ti.ca.ção}{0}
\verb{urticação}{}{Med.}{"-ões}{}{}{Formação ou desenvolvimento de urticária.}{ur.ti.ca.ção}{0}
\verb{urticante}{}{}{}{}{adj.2g.}{Que produz sensação semelhante à da irritação provocada na pele pela urtiga.}{ur.ti.can.te}{0}
\verb{urticária}{}{Med.}{}{}{s.f.}{Doença da pele semelhante a uma queimadura de urtiga, provocada pela sensibilização alérgica do organismo a plantas, animais, entre outros.}{ur.ti.cá.ria}{0}
\verb{urtiga}{}{Bot.}{}{}{s.f.}{Planta cujas folhas são cobertas de pelos finos, os quais, em contato com a pele, produzem um ardor irritante, devido à ação do ácido fórmico.}{ur.ti.ga}{0}
\verb{uru}{}{Zool.}{}{}{s.m.}{Ave galiforme, de bico curto, muito apreciada como caça.}{u.ru}{0}
\verb{uru}{}{}{}{}{}{Cesto indígena com alça, feito de palha de carnaúba.}{u.ru}{0}
\verb{urubu}{}{Zool.}{}{}{s.m.}{Ave de cor negra, que apresenta cabeça e pescoço nus e alimenta"-se de carnes em putrefação.}{u.ru.bu}{0}
%\verb{}{}{}{}{}{}{}{}{0}
\verb{urubu"-rei}{}{Zool.}{urubus"-reis \textit{ou} urubus"-rei}{}{s.m.}{Urubu que ocorre na América tropical, em regiões florestais e campestres, de cabeça e pescoço nus, pintados de vermelho, amarelo e alaranjado, asas e caudas pretas e o lado inferior branco.}{u.ru.bu"-rei}{0}
\verb{urucu}{}{}{}{}{}{Var. de \textit{urucum}.}{u.ru.cu}{0}
\verb{urucubaca}{}{Pop.}{}{}{s.f.}{Má sorte, azar, infelicidade.}{u.ru.cu.ba.ca}{0}
\verb{urucum}{}{Bot.}{"-uns}{}{s.m.}{Fruto do urucuzeiro, que apresenta cápsula grande, revestida de espinhos moles.}{u.ru.cum}{0}
\verb{urucungo}{}{}{}{}{s.m.}{Instrumento musical de percussão, composto de um fio de arame preso a um arco, com uma cabaça na parte de baixo; berimbau.}{u.ru.cun.go}{0}
\verb{urucuzeiro}{ê}{Bot.}{}{}{s.m.}{Arbusto ou pequena árvore, de flores avermelhadas, cujo fruto é o urucum, que fornece matéria corante vermelha utilizada na culinária como colorante e condimento, sob a forma de pó.}{u.ru.cu.zei.ro}{0}
%\verb{}{}{}{}{}{}{}{}{0}
%\verb{}{}{}{}{}{}{}{}{0}
\verb{uruguaio}{}{}{}{}{adj.}{Relativo ao Uruguai.}{u.ru.guai.o}{0}
\verb{uruguaio}{}{}{}{}{s.m.}{Indivíduo natural ou habitante desse país. }{u.ru.guai.o}{0}
\verb{urundeúva}{}{Bot.}{}{}{s.f.}{Árvore cuja madeira, resistente à deterioração, é empregada em obras externas.}{u.run.de.ú.va}{0}
%\verb{}{}{}{}{}{}{}{}{0}
%\verb{}{}{}{}{}{}{}{}{0}
\verb{urupê}{}{Bot.}{}{}{s.m.}{Cogumelo que cresce no tronco das árvores.}{u.ru.pê}{0}
\verb{urupema}{}{}{}{}{s.f.}{Espécie de peneira de fibra vegetal, para utilidades culinárias.}{u.ru.pe.ma}{0}
\verb{urupema}{}{Por ext.}{}{}{}{Trançado de fibra vegetal, usado para encosto de cadeiras.}{u.ru.pe.ma}{0}
\verb{urupemba}{}{}{}{}{}{Var. de \textit{urupema}.}{u.ru.pem.ba}{0}
\verb{urutau}{}{Zool.}{}{}{s.m.}{Ave de rapina de canto melancólico.}{u.ru.tau}{0}
\verb{urutu}{}{Zool.}{}{}{s.2g.}{Cobra venenosa, de cor escura, com uma mancha em forma de cruz na cabeça.}{u.ru.tu}{0}
\verb{urzal}{}{}{"-ais}{}{s.m.}{Terreno onde crescem urzes.}{ur.zal}{0}
\verb{urze}{}{Bot.}{}{}{s.m.}{Espécies de arbustos, alguns cultivados como ornamentais.}{ur.ze}{0}
\verb{usado}{}{}{}{}{adj.}{Que já teve algum uso; que não é novo.}{u.sa.do}{0}
\verb{usado}{}{}{}{}{}{Que é adaptado ou condicionado a algo; habituado, acostumado.}{u.sa.do}{0}
\verb{usado}{}{}{}{}{}{Que é habitual, frequente; usual.}{u.sa.do}{0}
\verb{usança}{}{}{}{}{s.f.}{Uso, costume, tradição, hábito antigo.}{u.san.ça}{0}
\verb{usar}{}{}{}{}{v.t.}{Fazer uso de; servir"-se de, empregar.}{u.sar}{0}
\verb{usar}{}{}{}{}{}{Empregar habitualmente; praticar.}{u.sar}{0}
\verb{usar}{}{}{}{}{}{Ter por costume.}{u.sar}{0}
\verb{usar}{}{}{}{}{}{Trajar, vestir.}{u.sar}{\verboinum{1}}
\verb{usável}{}{}{"-eis}{}{adj.2g.}{Que pode ser usado.}{u.sá.vel}{0}
\verb{useiro}{ê}{}{}{}{adj.}{Que costuma usar ou fazer certa coisa.}{u.sei.ro}{0}
\verb{usina}{}{}{}{}{s.f.}{Estabelecimento industrial equipado de máquinas, onde se transformam matérias"-primas em produtos finais.}{u.si.na}{0}
\verb{usina}{}{}{}{}{}{Engenho de açúcar.}{u.si.na}{0}
\verb{usina}{}{}{}{}{}{Conjunto de instalações destinadas à geração e aproveitamento de energia.}{u.si.na}{0}
\verb{usineiro}{ê}{}{}{}{adj.}{Relativo a usina.}{u.si.nei.ro}{0}
\verb{usineiro}{ê}{}{}{}{s.m.}{Proprietário de usina de açúcar.}{u.si.nei.ro}{0}
\verb{uso}{}{}{}{}{s.m.}{Ato ou efeito de usar.}{u.so}{0}
\verb{uso}{}{}{}{}{}{Aplicação, emprego.}{u.so}{0}
\verb{uso}{}{}{}{}{}{Prática, costume, hábito.}{u.so}{0}
\verb{uso}{}{}{}{}{}{Gasto, deterioração.}{u.so}{0}
\verb{usual}{}{}{"-ais}{}{adj.2g.}{Que se usa habitualmente; comum, frequente.}{u.su.al}{0}
\verb{usuário}{}{}{}{}{adj.}{Que utiliza algo; que tem apenas o direito de uso, mas não a propriedade.}{u.su.á.rio}{0}
\verb{usuário}{}{}{}{}{}{Que serve, que é próprio para uso.}{u.su.á.rio}{0}
\verb{usuário}{}{}{}{}{s.m.}{Indivíduo que, por direito de uso, serve"-se de algo ou desfruta de suas utilidades.}{u.su.á.rio}{0}
\verb{usucapião}{}{Jur.}{"-ões}{}{s.m.}{Conquista de uma propriedade por ter morado nela durante determinado tempo.}{u.su.ca.pi.ão}{0}
\verb{usucapir}{}{}{}{}{v.i.}{Adquirir"-se por uso.}{u.su.ca.pir}{0}
\verb{usucapir}{}{Jur.}{}{}{v.t.}{Adquirir por usucapião.}{u.su.ca.pir}{\verboinum{18}}
\verb{usufruir}{}{}{}{}{v.t.}{Ter a posse e o gozo de.}{u.su.fru.ir}{\verboinum{26}}
\verb{usufruto}{}{}{}{}{s.m.}{Ato ou efeito de usufruir.}{u.su.fru.to}{0}
\verb{usufruto}{}{}{}{}{}{Aquilo que se usufrui.}{u.su.fru.to}{0}
\verb{usufruturário}{}{}{}{}{adj.}{Relativo a usufruto.}{u.su.fru.tu.rá.rio}{0}
\verb{usufruturário}{}{}{}{}{}{Que usufrui; desfrutador.}{u.su.fru.tu.rá.rio}{0}
\verb{usura}{}{}{}{}{s.f.}{Juro de capital.}{u.su.ra}{0}
\verb{usura}{}{}{}{}{}{Juro abusivo.}{u.su.ra}{0}
\verb{usura}{}{}{}{}{}{Lucro excessivo e exagerado.}{u.su.ra}{0}
\verb{usura}{}{Bras.}{}{}{}{Mesquinhez, sovinice.}{u.su.ra}{0}
\verb{usurário}{}{}{}{}{adj.}{Que empresta com usura.}{u.su.rá.rio}{0}
\verb{usurário}{}{Pop.}{}{}{}{Agiota.}{u.su.rá.rio}{0}
\verb{usurário}{}{}{}{}{}{Sovina, avarento, mesquinho.}{u.su.rá.rio}{0}
\verb{usurpação}{}{}{"-ões}{}{s.f.}{Ato ou efeito de usurpar.}{u.sur.pa.ção}{0}
\verb{usurpador}{ô}{}{}{}{adj.}{Que usurpa, que se apossa violentamente de algo.}{u.sur.pa.dor}{0}
\verb{usurpar}{}{}{}{}{v.t.}{Apoderar"-se de algo sem ter direito, de maneira indevida e geralmente pelo uso da força.}{u.sur.par}{0}
\verb{usurpar}{}{}{}{}{}{Exercer indevidamente.}{u.sur.par}{\verboinum{1}}
\verb{utensílio}{}{}{}{}{s.m.}{Qualquer instrumento de trabalho, especialmente os que servem a uma tarefa específica; ferramenta.}{u.ten.sí.lio}{0}
\verb{uterino}{}{}{}{}{adj.}{Relativo ao útero.}{u.te.ri.no}{0}
\verb{uterino}{}{Jur.}{}{}{}{Diz"-se de irmão nascido do mesmo útero, consanguíneo por parte de mãe.}{u.te.ri.no}{0}
\verb{útero}{}{Anat.}{}{}{s.m.}{Órgão dos mamíferos onde se dá o desenvolvimento do feto.}{ú.te.ro}{0}
\verb{útil}{}{}{"-eis}{}{adj.2g.}{Que serve para algum uso.}{ú.til}{0}
\verb{útil}{}{}{"-eis}{}{}{Que traz proveito; profícuo, proveitoso, prestadio.}{ú.til}{0}
\verb{útil}{}{}{"-eis}{}{}{Diz"-se de dias em que há atividade produtiva.}{ú.til}{0}
\verb{utilidade}{}{}{}{}{s.f.}{Qualidade do que é útil.}{u.ti.li.da.de}{0}
\verb{utilidade}{}{}{}{}{}{Serventia, função.}{u.ti.li.da.de}{0}
\verb{utilitário}{}{}{}{}{adj.}{Relativo a utilidade.}{u.ti.li.tá.rio}{0}
\verb{utilitário}{}{Bras.}{}{}{s.m.}{Veículo de pequeno porte utilizado no transporte de cargas, como peruas, caminhonetes.}{u.ti.li.tá.rio}{0}
\verb{utilitarismo}{}{}{}{}{s.m.}{Modo de agir de quem tem a utilidade como finalidade principal de suas ações.}{u.ti.li.ta.ris.mo}{0}
\verb{utilitarista}{}{}{}{}{adj.2g.}{Relativo ao utilitarismo.}{u.ti.li.ta.ris.ta}{0}
\verb{utilitarista}{}{}{}{}{}{Diz"-se de pessoa que segue o utilitarismo.}{u.ti.li.ta.ris.ta}{0}
\verb{utilização}{}{}{"-ões}{}{s.f.}{Ato ou efeito de utilizar.}{u.ti.li.za.ção}{0}
\verb{utilizar}{}{}{}{}{v.t.}{Fazer uso de; empregar, usar.}{u.ti.li.zar}{\verboinum{1}}
\verb{utopia}{}{}{}{}{s.f.}{Projeto de sociedade ideal, em que o sistema político"-econômico e as leis sejam concebidos e praticados de forma a efetivamente garantir o bem"-estar de todos os indivíduos.}{u.to.pi.a}{0}
\verb{utopia}{}{}{}{}{}{Ideia impraticável; fantasia.}{u.to.pi.a}{0}
\verb{utópico}{}{}{}{}{adj.}{Relativo a utopia.}{u.tó.pi.co}{0}
\verb{utópico}{}{}{}{}{}{Fantasioso, irrealizável.}{u.tó.pi.co}{0}
\verb{utopista}{}{}{}{}{adj.2g.}{Que defende a possibilidade de uma utopia ser realmente posta em prática.}{u.to.pis.ta}{0}
\verb{utopista}{}{}{}{}{}{Que concebe projetos grandiosos, quiméricos, ideais.}{u.to.pis.ta}{0}
\verb{utrículo}{}{}{}{}{s.m.}{Pequeno saco.}{u.trí.cu.lo}{0}
%\verb{}{}{}{}{}{}{}{}{0}
\verb{uva}{}{}{}{}{s.f.}{Fruto da videira, comestível, doce e ligeiramente ácido, de cor verde, rosada, azulada, vermelha ou preta, muito utilizado na fabricação de vinho e vinagre.}{u.va}{0}
\verb{uva}{}{Desus.}{}{}{}{Mulher muito bonita.}{u.va}{0}
\verb{uva}{}{Por ext.}{}{}{}{Qualquer coisa muito bonita.}{u.va}{0}
\verb{uvaia}{}{Bot.}{}{}{s.f.}{Arbusto de folhas pequenas e compridas, cujo fruto, comestível, é amarelo e ácido.}{u.vai.a}{0}
\verb{uvaia}{}{}{}{}{}{A fruta dessa planta.}{u.vai.a}{0}
\verb{úvula}{}{Anat.}{}{}{s.f.}{Saliência existente no fundo da boca, na entrada da garganta, conhecida vulgarmente como \textit{campainha}.}{ú.vu.la}{0}
\verb{uvular}{}{}{}{}{adj.2g.}{Relativo à úvula.}{u.vu.lar}{0}
\verb{uvular}{}{Gram.}{}{}{}{Diz"-se de som da fala cuja articulação envolve a úvula.}{u.vu.lar}{0}
\verb{uvulite}{}{Med.}{}{}{s.f.}{Inflamação da úvula.}{u.vu.li.te}{0}
\verb{uxoricida}{cs}{}{}{}{s.m.}{Indivíduo que comete uxoricídio.}{u.xo.ri.ci.da}{0}
\verb{uxoricídio}{cs}{}{}{}{s.m.}{Assassinato da mulher pelo próprio marido.}{u.xo.ri.cí.dio}{0}
\verb{uxório}{cs}{}{}{}{adj.}{Relativo a mulher casada.}{u.xó.rio}{0}
