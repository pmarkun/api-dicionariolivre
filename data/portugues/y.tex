\verb{y}{}{}{}{}{s.m.}{Vigésima quinta letra do alfabeto português.}{y}{0}
\verb{Y}{}{Quím.}{}{}{}{Símb. de \textit{ítrio}.}{Y}{0}
\verb{yakisoba}{ô}{Cul.}{}{}{s.m.}{Iguaria japonesa preparada com macarrão e verduras refogadas.}{\textit{yakisoba}}{0}
\verb{yang}{}{Filos.}{}{}{s.m.}{Princípio fundamental da filosofia taoísta chinesa, que se opõe e se complementa com o \textit{yin}.}{\textit{yang}}{0}
\verb{yanomami}{}{}{}{}{adj.}{Relativo aos Yanomami.}{ya.no.ma.mi}{0}
\verb{yanomami}{}{}{}{}{s.2g.}{Indivíduo pertencente ao povo yanomami, família linguística yanomami.}{ya.no.ma.mi}{0}
\verb{yawalapiti}{}{}{}{}{adj.}{Relativo aos Yawalapiti.}{ya.wa.la.pi.ti}{0}
\verb{yawalapiti}{}{}{}{}{s.2g.}{Indivíduo pertencente ao povo yawalapiti, família linguística aruak.}{ya.wa.la.pi.ti}{0}
\verb{Yb}{}{Quím.}{}{}{}{Símb. do \textit{itérbio}.}{Yb}{0}
\verb{yd}{}{}{}{}{}{Símb. de \textit{jarda}.}{yd}{0}
\verb{yearling}{}{}{}{}{s.m.}{Cavalo puro"-sangue entre um e dois anos de idade.}{\textit{yearling}}{0}
\verb{yin}{}{Filos.}{}{}{s.m.}{Princípio fundamental do taoísmo chinês, que se opõe e se complementa com o \textit{yang}.}{\textit{yin}}{0}
\verb{yin"-yang}{}{Filos.}{}{}{s.m.}{No pensamento oriental, par de forças ou princípios fundamentais, opostos e complementares do Universo, que regem todos os aspectos e fenômenos da vida.}{\textit{yin"-yang}}{0}
\verb{yuppie}{}{}{}{}{s.2g.}{Denominação do jovem executivo bem"-sucedido, dado ao consumo de artigos caros e luxuosos.}{\textit{yuppie}}{0}
